\section{Proofs of the statements in Section~\ref{sec:solving-substructure} and proof of Proposition~\ref{theorem:small-model-property}}\label{appendix:solving-substructure}

Throughout this appendix, we write $\implies$ and $\iff$ for the Boolean
connectives of implication and double implication. Observe that, when $\phi$ and
$\psi$ are quantifier-free formulae from $\exists\R(\ipow{\cn})$, $\phi \implies
\psi$ and $\phi \iff \psi$ can be seen as shortcuts for formulae of
$\exists\R(\ipow{\cn})$ given in the grammar from~\Cref{sec:preliminaries}.
Despite this, sometimes it is more convenient to apply these Boolean connectives
also on quantified formulae, and for these reasons in this appendix we often look
at the full first-order theory of $\R(\ipow{\cn})$, instead of just
$\exists\R(\ipow{\cn})$. The grammar of $\R(\ipow{\cn})$ is obtained from the
one of $\exists\R(\ipow{\cn})$ by adding arbitrary negations.

We start with an auxiliary technical lemma that implies~\Cref{lemma:lambda-close-to-variable-body}.

\begin{restatable}{lemma}{LemmaLambdaCloseToVariable}\label{lemma:lambda-close-to-variable}
  Let $p(\vec{z}) \coloneqq \sum_{i=1}^n q_i(\cn) \cdot \cn^{z_i}$,
  where $\vec z = (z_1,\dots,z_n)$ and each~$q_i(x)$ is an integer polynomial.
  There is a finite set $G \subseteq \Z$ with the following property: 
  for every~${\vec z^* \in \Z^n}$,
  if $p(\vec z^*) > 0$
  then $\lambda(p(\vec z^*)) = \cn^g \cdot \cn^{z_i^*}$ for some
  $g \in G$ and $i \in [1..n]$.
  Moreover:
  \begin{enumerate}[I.]
    \item\label{lemma:lambda-close-to-variable:i1} If $\cn$ is a computable transcendental number, there is an
          algorithm computing $G$ from $p$.
    \item\label{lemma:lambda-close-to-variable:i2} If $\cn$ has a root barrier $\sigma(d,h) \coloneqq c \cdot
            {(d+\ceil{\ln(h)})}^k$, for some $c,k \in \N_{\geq 1}$, then,
            \begin{equation*}
              G\coloneqq{\left[-L..L\right]},
              \qquad 
              \text{where }
              L \coloneqq {\left(2^{3c}D\ceil{\ln(H)}\right)}^{6nk^{3n}},
            \end{equation*}
            with $H \coloneqq \max\{8,\height(q_i) : i \in [1,n]\}$, and $D \coloneqq \max\{\deg(q_i)+2 : i \in [1,n]\}$.
  \end{enumerate}

\end{restatable}

\begin{proof}
    %\allowdisplaybreaks
    Note that for $n = 0$ we have $p(\vec z^*) = 0$ for every $\vec z^* \in \Z^n$, and we can take $G = \emptyset$. Therefore, throughout the proof, we assume $n \geq 1$.
    We start by considering the first statement of the lemma, which requires showing the existence of the finite set $G$. 
    To prove this, we first fix a vector $\vec z^* = (z_1^*,\dots,z_n^*) \in \Z^n$ such that $p(\vec z^*) > 0$, 
    and use it to derive a definition for $G$ that does not, in fact, depend on $\vec z^*$.
    Without loss of generality, we work under the additional assumption that
    $z_1^* \geq \cdots \geq z_n^*$.

    The following claim provides an analysis on the value of $\lambda(p(\vec z^*))$.
  
    \begin{claim}\label{claim1:lambda-close-to-variable}
      There is a non-empty interval $[j..\ell]$, with $j,\ell \in [1..n]$,
      and natural numbers ${g_j},\dots,{g_{\ell-1}}$
      with respect to which the recursively defined polynomials
      $Q_j,\dots,Q_{\ell}$ given by
      \begin{align*}
        Q_j(x) & \coloneqq q_j(x),                                  \\
        Q_r(x) & \coloneqq Q_{r-1}(x) \cdot x^{g_{r-1}} + q_{r}(x),
               & \text{for every } r \in [j+1,\ell],
      \end{align*}
      satisfy the following properties:
      \begin{enumerate}[A.]
        \item\label{claim1:lambda-close-to-variable:A}
        the numbers $Q_j(\cn),\dots,Q_{\ell-1}(\cn)$ are all non-zero, and
        $Q_\ell(\cn)$ is (strictly) positive,
        \item\label{claim1:lambda-close-to-variable:B}
        for every $r \in [j..\ell-1]$,
        the number $\cn^{g_r}$ belongs to the interval
        $\big[1\,,\,\frac{\abs{q_{r+1}(\cn)}+\cdots+\abs{q_n(\cn)}}{\abs{Q_r(\cn)}}\big]$,
        and
        \item\label{claim1:lambda-close-to-variable:C}
        either\, $\lambda(p(\vec z^*)) = \lambda(Q_\ell(\cn)) \cdot
          \cn^{z_\ell^*}$\, or\, $\frac{\lambda(Q_\ell(\cn) \cdot (\cn-1))}{\cn} \cdot
          \cn^{z_\ell^*} \leq \lambda(p(\vec z^*)) \leq
          \frac{\lambda(Q_\ell(\cn) \cdot (\cn+1))}{\cn} \cdot \cn^{z_\ell^*}$.
      \end{enumerate}
    \end{claim}
  
    \begin{claimproof}
      The proof is by induction on $n$.
      \begin{description}
        \item[base case: $n = 1$.]
          In this case, $p(z_1)$ is the expression $q_1(\cn) \cdot \cn^{z_1}$. By
          definition of $\lambda$, $\lambda(p(z_1^*)) = \lambda(q_1(\cn)) \cdot
            \cn^{z_1^*}$. Observe that $p(z_1^*) > 0$ implies $q_1(\cn) > 0$, and
          thus
          $\lambda(q_1(\cn))$ is a defined integer power of $\cn$. Taking the
          interval $[1..1]$ shows~\Cref{claim1:lambda-close-to-variable}.
        \item[induction step: $n \geq 2$.]
          Below, we assume $q_1(\cn)$ to be non-zero.
          Indeed, if $q_1(\cn) = 0$,
          we can then apply the induction hypothesis on
          $\hat{p}(z_2,\dots,z_n) \coloneqq \sum_{i=2}^n q_i(\cn) \cdot
            \cn^{z_i}$,
          concluding the proof (since $p(\vec z^*) =
            \hat{p}(z_2^*,\dots,z_n^*)$).
  
          We split the proof depending on whether $\cn^{z_1^*} \geq
            \frac{\sum_{i=2}^n \abs{q_i(\cn)}}{\abs{q_1(\cn)}} \cdot
            \cn^{z_2^*+1}$ holds.
          \begin{description}
            \item[case: $\cn^{z_1^*} \geq \frac{\sum_{i=2}^n
            \abs{q_i(\cn)}}{\abs{q_1(\cn)}} \cdot \cn^{z_2^*+1}$.]
              Observe that in this case, $q_1(\cn)$ must be positive. We show
              that
              $\frac{\lambda(q_1(\cn) \cdot (\cn-1))}{\cn} \cdot \cn^{z_1^*} \leq
                \lambda(p(\vec{z}^*)) \leq \frac{\lambda(q_1(\cn) \cdot (\cn+1))}{\cn}
                \cdot \cn^{z_1^*}$,
              thus establishing that taking the interval $[1..1]$
              proves~\Cref{claim1:lambda-close-to-variable} also in this case.
              For the lower bound:
              \begin{align*}
                p(\vec z^*) & \geq q_1(\cn) \cdot \cn^{z_1^*} - \sum_{i=2}^n
                \abs{q_i(\cn)} \cdot \cn^{z_i^*}
                            & \text{by def.~of~$p$}
                \\
                            & \geq q_1(\cn) \cdot \cn^{z_1^*} - \cn^{z_2^*}\cdot
                \sum_{i=2}^n
                \abs{q_i(\cn)}
                            & z_2^* \geq z_i^* \text{ for all $i \in [2,n]$}
                \\
                            & \geq
                q_1(\cn) \cdot \cn^{z_1^*} - q_1(\cn) \cdot \cn^{z_1^*-1}
                            & \text{assumption of this case and $q_1(\cn) > 0$}
                \\
                            & \geq
                q_1(\cn) \cdot (\cn - 1) \cdot \cn^{z_1^*-1}.
              \end{align*}
              Since $a \geq b$ implies $\lambda(a) \geq \lambda(b)$, we thus
              obtain
              $\lambda(p(\vec z^*)) \geq \frac{\lambda(q_1(\cn) \cdot
                  (\cn-1))}{\cn} \cdot
                \cn^{z_1^*}$.
  
              For the upper bound:
              \begin{align*}
                p(\vec z^*) & \leq q_1(\cn)\cdot \cn^{z_1^*} + \sum_{i=2}^n
                \abs{q_i(\cn)} \cdot \cn^{z_2^*}
                            & \text{by def.~of $p$, and $z_2^* \geq z_i^*$  for
                  all $i \in
                    [2,n]$}
                \\
                            & \leq q_1(\cn)\cdot \cn^{z_1^*} + q_1(\cn) \cdot
                \cn^{z_1^*-1}
                            & \text{assumption of this case and $q_1(\cn) > 0$}
                \\
                            & \leq q_1(\cn) \cdot (\cn+1) \cdot \cn^{z_1^*-1},
              \end{align*}
              and again from the properties of $\lambda$,
              we obtain $\lambda(p(\vec z^*)) \leq \frac{\lambda(q_1(\cn) \cdot
                  (\cn
                  + 1))}{\cn} \cdot \cn^{z_1^*}$.
  
            \item[case: $\cn^{z_1^*} < \frac{\sum_{i=2}^n
            \abs{q_i(\cn)}}{\abs{q_1(\cn)}} \cdot \cn^{z_2+1}$.] We have
              $\cn^{z_1^*} \leq
                \frac{\sum_{i=2}^n \abs{q_i(\cn)}}{\abs{q_1(\cn)}} \cdot
                \cn^{z_2}$.
              Since $z_1^* \geq z_2^*$,
              there must be $g_1 \in \N$ such that $\cn^{g_1} \in
                \big[1,\frac{\sum_{i=2}^n \abs{q_i(\cn)}}{\abs{q_1(\cn)}}\big]$
              and $\cn^{z_1^*} = \cn^{g_1} \cdot \cn^{z_2^*}$.
              We define
              \[
                q_2'(x) \coloneqq q_1(x) \cdot x^{g_1} + q_2(x),
                \qquad
                p'(z_2,\dots,z_n) \coloneqq q_2'(\cn) \cdot \cn^{z_2} +
                \sum_{i=3}^n
                q_i(\cn) \cdot \cn^{z_i}.
              \]
              Therefore, $p(\vec z^*) = p'(\vec z_2^*)$,
              where $\vec z_2^* \coloneqq (z_2^*,\dots,z_\ell^*)$.
              By induction hypothesis,
              there is a non-empty interval $[j..\ell]$, with $j,\ell \in
                [2..n]$,
              and natural numbers ${g_j},\dots,{g_{\ell-1}}$
              with respect to which the recursively defined polynomials
              $Q_j,\dots,Q_{\ell}$ given by
              \begin{align*}
                Q_j(x) & \coloneqq
                \begin{cases}
                  q_2'(x) & \text{if $j = 2$} \\
                  q_j(x)   & \text{otherwise}
                \end{cases}                                \\
                Q_r(x) & \coloneqq Q_{r-1}(x) \cdot x^{g_{r-1}} + q_{r}(x),
                       & \text{for every } r \in [j+1..\ell],
              \end{align*}
              satisfy that
              (\labeltext{A$'$}{claim1:lambda-close-to-variable:Ap})
              $Q_j(\cn),\dots,Q_{\ell-1}(\cn)$ are all non-zero
              and~$Q_\ell(\cn)$ is positive,
              (\labeltext{B$'$}{claim1:lambda-close-to-variable:Bp})~for every $r
                \in
                [j,\ell-1]$, the number
              $\cn^{g_r}$ belongs to
              $\big[1\,,\,\frac{\abs{q_{r+1}(\cn)}+\cdots+\abs{q_n(\cn)}}{\abs{Q_r(\cn)}}\big]$,
              and
              (\labeltext{C$'$}{claim1:lambda-close-to-variable:Cp})~either
              $\lambda(p'(\vec z_2^*)) = \lambda(Q_\ell(\cn)) \cdot
                \cn^{z_\ell^*}$ or
              $\frac{\lambda(Q_\ell(\cn) \cdot (\cn-1))}{\cn} \cdot \cn^{z_\ell^*} \leq
                \lambda(p'(\vec z_2^*)) \leq
                \frac{\lambda(Q_\ell(\cn) \cdot (\cn+1))}{\cn} \cdot
                \cn^{z_\ell^*}$.
  
              If $j \neq 2$, then $Q_j(x) = q_j(x)$,
              and thus from $p(\vec z^*) = p'(\vec z_2^*)$
              we conclude that the interval $[j..\ell]$
              and the numbers ${g_j},\dots,{g_{\ell-1}}$
              defined for $p'$ also establish the claim for $p$.
  
              Otherwise, when $j = 2$ we have $Q_j(x) = q_2'(x) = q_1(x) \cdot
                x^{g_1} + q_2(x)$.
              Recall that $q_1(\cn)$ is non-zero and that,
              by definition of $g_1$, we have $\cn^{g_1} \in
                \big[1,\frac{\sum_{i=2}^n \abs{q_i(\cn)}}{\abs{q_1(\cn)}}\big]$.
              Therefore,
              taking the interval $[1..\ell]$ and the numbers
              $g_1,\dots,g_{\ell-1}$ proves
              the claim for $p$.
              \claimqedhere%
          \end{description}
      \end{description}
    \end{claimproof}

    With~\Cref{claim1:lambda-close-to-variable} at hand, we now argue that the
    finite set $G \subseteq \Z$ required by the lemma exists.
    The key observation is that the definition of $Q_\ell$ from~\Cref{claim1:lambda-close-to-variable} does not depend on~$\vec z^*$. 
    Hence, a suitable set $G$ can be defined as follows. 
    Let $\mathcal{Q}$ be the set of all polynomials $Q$
    for which there are $j \leq \ell \in [1..n]$, $g_j,\dots,g_{\ell-1} \in \N$, and polynomials $Q_j,\dots,Q_{\ell}$ 
    such that:
      \begin{enumerate}
        \item the polynomial $Q$ is equal to $Q_\ell$,
        \item\label{mcQ-polynomials} the polynomials $Q_j^{},\dots,Q_\ell$ are defined as 
            \begin{align*}
            Q_j(x) & \coloneqq q_j(x),                                  \\
            Q_r(x) & \coloneqq Q_{r-1}(x) \cdot x^{g_{r-1}} + q_{r}(x),
                & \text{for every } r \in [j+1,\ell],
            \end{align*}
        \item
        the numbers $Q_j(\cn),\dots,Q_{\ell-1}(\cn)$ are all non-zero, and
        $Q_\ell(\cn)$ is (strictly) positive,
        \item\label{mcQ-finiteness-of-gr}
        for every $r \in [j..\ell-1]$, and
        the number $\cn^{g_r}$ belongs to the interval
        $\big[1\,,\,\frac{\abs{q_{r+1}(\cn)}+\cdots+\abs{q_n(\cn)}}{\abs{Q_r(\cn)}}\big]$.
      \end{enumerate}
    In a nutshell, $\mathcal{Q}$ contains all polynomials $Q_\ell$ that might be considered in~\Cref{claim1:lambda-close-to-variable} as the vector $\vec z^*$ varies. Items~\ref{mcQ-polynomials}--\ref{mcQ-finiteness-of-gr} ensure that $\mathcal{Q}$ is a finite set.
    We define $G \coloneqq [\min B.. \max B]$, where $B$ is defined as the set
    \[ 
      B \coloneqq \left\{ \beta \in \Z : \text{there is } Q \in \mathcal{Q} \text{ such that } \cn^{\beta} \in \Big\{{\textstyle\lambda(Q(\cn)), \frac{\lambda(Q(\cn) \cdot (\cn-1))}{\cn}, \frac{\lambda(Q(\cn) \cdot (\cn+1))}{\cn}}\Big\}
      \right\}.
    \]
    Since $\mathcal{Q}$ is finite, then so are $B$ and $G$.

    It is now simple to see that $G$ satisfies the property required by the first statement of the lemma. 
    Indeed, consider a vector $\vec z^* = (z_1^*,\dots,z_n^*) \in \Z^n$ such that $p(\vec z^*) > 0$ (this is not necessarily the vector we have fixed at the beginning of the proof). 
    By definition of $\mathcal{Q}$ and by~\Cref{claim1:lambda-close-to-variable},
    there is a polynomial $Q$ in $\mathcal{Q}$ such that 
    \begin{itemize} 
        \item $Q(\cn)$ is strictly positive (and so $\lambda(Q(\cn))$ is well-defined). Since $\cn > 1$, observe that this means that also $Q(\cn) \cdot (\cn-1)$ and $Q(\cn) \cdot (\cn+1)$ are strictly positive.
        \item Either $\lambda(p(\vec z^*)) = \lambda(Q(\cn)) \cdot \cn^{z_i^*}$ or $\frac{\lambda(Q(\cn) \cdot (\cn-1))}{\cn} \cdot \cn^{z_i^*} \leq \lambda(p(\vec z^*)) \leq \frac{\lambda(Q(\cn) \cdot (\cn+1))}{\cn} \cdot \cn^{z_i^*}$, for some $i \in [1,n]$ (this follows by Property~\ref{claim1:lambda-close-to-variable:C} of~\Cref{claim1:lambda-close-to-variable}). 

        In the latter case of $\frac{\lambda(Q(\cn) \cdot (\cn-1))}{\cn} \cdot \cn^{z_i^*} \leq \lambda(p(\vec z^*)) \leq \frac{\lambda(Q(\cn) \cdot (\cn+1))}{\cn} \cdot \cn^{z_i^*}$, observe that $\lambda(p(\vec z^*)) = \cn^\beta \cdot \cn^{z_i^*}$, for some $\cn^\beta \in [\frac{\lambda(Q(\cn) \cdot (\cn-1))}{\cn},\frac{\lambda(Q(\cn) \cdot (\cn+1))}{\cn}]$.
    \end{itemize}
    By definition of $B$ and $G$, 
    we conclude that $\lambda(p(\vec z^*)) = \cn^g \cdot \cn^{z_i^*}$ 
    for some $g \in G$ and $i \in [1..n]$. 
    This concludes the proof of the first statement of the lemma.
  
    We move to the second part of the lemma, which adds further assumptions
    on~$\cn$. This part still relies on the definitions of the sets $\mathcal{Q}$, $B$ and $G$ above.
  
    \noindent
    \textbf{Case: $\cn$ is a computable transcendental number (Item~\eqref{lemma:lambda-close-to-variable:i1}).}
    Assume $\cn$ a transcendental number computed by a Turing machine $T$. We provide an algorithm for computing a superset of the set $G$. Here is a high-level pseudocode of the algorithm:

    \begin{algorithmic}[1]
        \State\label{algo-lambda-trans-1} compute a finite set of polynomials $\mathcal{Q}'$ that includes all polynomials in $\mathcal{Q}$
        \State\label{algo-lambda-trans-2} remove from $\mathcal{Q}'$ all polynomials $Q$ such that $Q(\cn) \leq 0$
        \State\label{algo-lambda-trans-3} compute rationals $\ell,u > 0$ such that~$\ell \leq \frac{Q(\cn)\cdot (\cn-1)}{\cn^2}$ and $Q(\cn) \cdot (\cn+1) \leq u$, for all $Q$ in~$\mathcal{Q}'$
        \State\label{algo-lambda-trans-4} \textbf{return} a superset of $\{ \beta \in \Z : \ell \leq \cn^\beta \leq u \}$
    \end{algorithmic}
    The correctness of this algorithm is immediate from the definition of the sets $\mathcal{Q}$, $B$ and $G$. 
    In particular, note that $\{ \cn^\beta : \beta \in B\} \subseteq [\ell..u]$, 
    because for every $\alpha > 0$ we have $\frac{\alpha}{\cn} < \lambda(\alpha) \leq \alpha$ (by definition of $\lambda$), and moreover 
    $\frac{Q(\cn) \cdot (\cn-1)}{\cn^2} \leq \frac{Q(\cn)}{\cn} \leq Q(\cn) \leq Q(\cn)\cdot (\cn+1)$ (recall that $Q(\cn) > 0$ and $\cn > 1$). Therefore, $G$ is a subset of the set in output of the algorithm, as required. Below, we give more information on how to implement each line of the algorithm (starting for simplicity with line~\ref{algo-lambda-trans-2}), showing its effectiveness. We will often rely on the following claim:

    \begin{claim}\label{claim:lu-p}
      Given an integer polynomial $p(x)$, one can compute 
      \begin{enumerate}
        \item\label{claim:lu-p:i1} a rational number $\ell'$ such that $0 <
        \ell' \leq \abs{p(\cn)}$;
        \item\label{claim:lu-p:i2} a rational number $u'$ such that $\abs{p(\cn)}
        \leq u'$.
      \end{enumerate}
    \end{claim}

    \begin{claimproof}
      Recall that $\abs{T_0}+1$ is an upper bound to the transcendental number $\cn > 1$. 
      By iterating over the natural numbers, we find the smallest $L \in \N$ 
      such that 
      $\abs{q(T_M)} > \frac{1}{2^L} \geq \abs{q(\cn) - q(T_M)}$, 
      where $M \coloneqq L + \ceil{\log(\height(p)+1)} + 2 \deg(p) \cdot \ceil{\log(\abs{T_0}+2)}$.
      The existence of such an $L$ is guaranteed from~\Cref{lemma:approx-univ-polynomial} (for the second inequality) together the fact that $q(\cn) \neq 0$, and so $\lim_{n \to \infty} \abs{q(T_n)} \neq 0$ whereas $\lim_{m \to \infty} \frac{1}{2^m} = 0$
      (which implies the first inequality).
      For~\Cref{claim:lu-p:i1}, we can take~$\ell'$ to be $\abs{q(T_M)} - \frac{1}{2^L}$.
      For~\Cref{claim:lu-p:i2}, we can take~$u'$ to be $\abs{q(T_M)} + \frac{1}{2^L}$.
    \end{claimproof}
    
    Here is the argument for the effectiveness of the algorithm:
    \begin{itemize}
        \item \textit{line~\ref{algo-lambda-trans-2}.} 
        In general, to evaluate the sign of a polynomial~$p$ at $\cn$, one
        relies on the fact that $p(\cn)$ must be different from~$0$ (because
        $\cn$ is transcendental). Then, we can rely on the fast-convergence
        sequence of rational numbers~$T_0,T_1,\dots$ to find $n \in \N$ such that
        $|p(\cn) - p(T_{n})|$ is guaranteed to be less than $|p(T_{n})|$. The
        sign of $p(\cn)$ then agrees with the sign of~$p(T_{n})$, and the latter
        can be easily computed.
        \item \textit{line~\ref{algo-lambda-trans-1}.} By definition of $\mathcal{Q}$, the fact that such a set $\mathcal{Q}'$ can be computed follows from the fact that we can compute an upper bound, for every $j \leq \ell \in [1..n]$ and $r \in [j..\ell-1]$, to the maximum $g_r$ such that $\cn^{g_r} \in \big[1\,,\,\frac{\abs{q_{r+1}(\cn)}+\cdots+\abs{q_n(\cn)}}{\abs{Q_r(\cn)}}\big]$, where $Q_r$ is any polynomial that can be defined in terms of $g_{j},\dots,g_{r-1}$ following the recursive definition of Item~\ref{mcQ-polynomials}.
        It suffices to find a positive lower bound $\ell' \in \Q$ to $\abs{Q_r(\cn)}$, 
        as well as upper bounds $u_i' \in \Q$ to every $\abs{q_i(\cn)}$, with $i \in [r+1..n]$. The rationals $\ell',u_{r+1}',\dots,u_n'$ are computed following~\Cref{claim:lu-p}. Then, 
        $\frac{\abs{q_{r+1}(\cn)}+\cdots+\abs{q_n(\cn)}}{\abs{Q_r(\cn)}} \leq \frac{u_{r+1}'+\cdots+u_{n}'}{\ell'} \leq \ceil{\frac{u_{r+1}'+\cdots+u_{n}'}{\ell'}} \eqqcolon D \in \N$.
        To bound $g_r$ it now suffices to find the largest integer power of $\cn$ that is less or equal to~$D$. This can be done using the algorithm for the sign evaluation problem described for line~\ref{algo-lambda-trans-2}: iteratively, starting at $i = 0$, we test whether $\cn^i - D$ is non-positive; we increase $i$ by $1$ if this test is successful, and return $i-1$ otherwise. 
        \item \textit{line~\ref{algo-lambda-trans-3}.} 
        Recall that $Q(\cn)$ is positive and $\cn > 1$.
        Following~\Cref{claim:lu-p}, we can find positive rationals $\ell',u_1',u_2'$ 
        such that $\ell' < Q(\cn) \cdot (\cn-1)$, $\cn^2 \leq u_1'$ 
        and $Q(\cn) \cdot (\cn+1) \leq u_2'$.
        The first two inequalities imply $0 < \frac{\ell'}{u_1'} < \frac{Q(\cn) \cdot (\cn-1)}{\cn^2}$.
        We can then take $\ell \coloneqq \frac{\ell'}{u_1'}$ and $u \coloneqq u_2'$. 
        Note that we have $\frac{Q(\cn) \cdot (\cn-1)}{\cn^2} < Q(\cn) \cdot (\cn + 1)$, 
        and therefore $\ell < u$.
        \item \textit{line~\ref{algo-lambda-trans-4}.} Given $\ell$ and $u$, we can compute a superset of those $\beta \in \Z$ such that $\ell \leq \cn^\beta \leq u$ by iterated calls to the algorithm for the sign evaluation problem. First, we can extend the interval $[\ell,u]$ to always include $1$: if $\ell > 1$, update $\ell$ to $1$; if $u < 1$, update $u$ to $1$.
        This ensures $\cn^0 \in [\ell,u]$. We can then find the largest~$\cn^i$ that is less or equal to~$u$ by testing whether $\cn^i - u$ is non-positive for increasing $i$ starting at $0$, as we did in line~\ref{algo-lambda-trans-1} for finding the largest integer powers less or equal to~$D$. Similarly, we can find the smallest integer power~$\cn^{-i}$ that is greater or equal than $\ell$ by testing whether $1 - \ell \cdot \cn^{i}$ is non-negative for increasing $i$ starting at $0$.
    \end{itemize}
  
    \noindent
    \textbf{Case: $\cn$ has a polynomial root barrier (Item~\eqref{lemma:lambda-close-to-variable:i2}).}
    Assume now $\cn$ to have a polynomial root barrier 
    $\sigma(d,h) \coloneqq c \cdot {(d+\ceil{\ln(h)})}^k$, 
    with $c,k \in \N_{\geq 1}$.
    In this case, we need to provide an explicit set~$G$.
    We do so by analysing the polynomials 
    $Q_j,\dots,Q_\ell$
    and the natural numbers 
    $g_j,\dots,g_{\ell-1}$ 
    introduced in~\Cref{claim1:lambda-close-to-variable}
    and used in the definition of the set $\mathcal{Q}$,
    and by providing both lower and upper bounds for the positive numbers
    $Q_\ell(\cn)$,
    $Q_{\ell}(\cn) \cdot (\cn-1)$ and 
    $Q_{\ell}(\cn) \cdot (\cn+1)$.
    These bounds entail bounds on the integers occurring in the set $B$
    introduced at the end of the proof of the first statement of the lemma.

    We start by providing a bound on the degrees and heights
    of $Q_j,\dots,Q_\ell$:
    \begin{claim}\label{claim2:lambda-close-to-variable}
      For every $r \in [j..\ell]$,
      $\deg(Q_r) \leq D + \sum_{s=j}^{r-1} g_s$
      and $\height(Q_r) \leq (r-j+1) \cdot H$.
    \end{claim}
    \begin{claimproof}
      By a straightforward induction on $r$, using the definitions of
      $Q_j,\dots,Q_\ell$.
    \end{claimproof}
    In~\Cref{claim2:lambda-close-to-variable},
    note that $(r-j+1) \cdot H \leq n \cdot H$, and therefore we obtain a bound on $\height(Q_\ell)$ that does not depend on the previous $Q_r$. 
    Below, we prove a similar bound for $\deg(Q_\ell)$. 
      \begin{claim}\label{claim3:lambda-close-to-variable}
        The degree of $Q_\ell$ is bounded as follows:
        \[ 
          \deg(Q_\ell) \leq {\left(\frac{2c \cdot D \cdot \ln(H)}{\ln(1 + \frac{1}{e^c})}\right)}^{5nk^{n+1}}.
        \]
      \end{claim}
      \begin{claimproof}
        By Property~\ref{claim1:lambda-close-to-variable:A},
        $Q_j(\cn),\dots,Q_\ell(\cn)$ are non-zero.
        Then, \Cref{claim2:lambda-close-to-variable}
        and the fact that $\sigma$ is a root barrier for $\cn$ entail
        \begin{equation}
          \label{inequality1:lambda-close-to-variable}
          \ln \abs{Q_r(\cn)} \geq - c \cdot {\Big(D + \ceil{\ln(n \cdot H)} +
          \sum_{s=j}^{r-1} g_s \Big)}^k.
        \end{equation}
        Analogously, since $\cn > 1$, we can consider the polynomial $x - 1$ in order to obtain a lower bound on $\cn$, via the root barrier $\sigma$. We obtain 
        \begin{equation}
          \label{inequality1:lower-bound-cn}
          \cn \geq 1 + \frac{1}{e^c}.
        \end{equation}
        Given $r \in [1,n]$, we also have
        \begin{equation}
          \label{inequality2:lambda-close-to-variable}
          \abs{q_r(\cn)} \leq 
            H \cdot \sum_{i=0}^d \cn^i \leq H \cdot D \cdot \cn^{D}.
        \end{equation}
        We use Inequalities~\eqref{inequality1:lambda-close-to-variable}
        and~\eqref{inequality2:lambda-close-to-variable} to bound the values
        of $g_j,\dots,g_{\ell-1}$. 
        By Property~\ref{claim1:lambda-close-to-variable:B}, 
        $\cn^{g_r} \leq
          \frac{\abs{q_{r+1}(\cn)}+\cdots+\abs{q_n(\cn)}}{\abs{Q_r(\cn)}}$, 
        and therefore
        \begin{align*}
          & g_r\\
          \leq{}& 
            \log_{\cn}(\abs{q_{r+1}(\cn)}+\cdots+\abs{q_n(\cn)})
            - \log_{\cn}(\abs{Q_j(\cn)})
          \\
          \leq{}& \frac{1}{\ln(\cn)} 
            (\ln(\abs{q_{r+1}(\cn)}+\cdots+\abs{q_n(\cn)})
            - \ln(\abs{Q_r(\cn)}))
          & 
          \hspace{-3.5cm}
          \text{change of base}
          \\
          \leq{}& \frac{1}{\ln(\cn)} 
            (\ln(H \cdot D \cdot \cn^{D} \cdot n) 
            - \ln(\abs{Q_r(\cn)}))
          &
          \hspace{-3.5cm}
          \text{by Inequality~\eqref{inequality2:lambda-close-to-variable}}
          \\
          \leq{}& \frac{1}{\ln(\cn)}  
            \Big(\ln( H \cdot D \cdot \cn^{D} \cdot n) 
            + c \cdot {\big(D + \ceil{\ln(n H)} 
            + \sum_{s=j}^{r-1} g_s \big)}^k \Big)
          &
          \hspace{-3.5cm}
          \text{by~Inequality~\eqref{inequality1:lambda-close-to-variable}}
          \\
          \leq{}& \frac{1}{\ln(\cn)} 
            \Big( D \cdot \ln(\cn) 
            + \ln(nH \cdot D)
            + c \cdot \big(D + \ceil{\ln(n H)} 
            + \sum_{s=j}^{r-1} g_s \big)^k \Big)\\
          \leq{}& \frac{1}{\ln(\cn)} 
            \Big(D \cdot \ln(\cn) 
            + 2c \cdot \big(D + \ceil{\ln(n H)} 
            + \sum_{s=j}^{r-1} g_s \big)^k \Big)\\
          &&
          \hspace{-3.5cm}
          \text{as $D + \ceil{\ln(nH)} \geq \ln(nH \cdot D)$}    
          \\
          \leq{}& \frac{1}{\ln(\cn)}  
            \Big(D \frac{\ln(\cn)}{\ln(1 + \frac{1}{e^c})} 
            + 2 c \cdot \big(D + \ceil{\ln(n H)} 
            + \sum_{s=j}^{r-1} g_s \big)^k \Big)
          &
          \hspace{-3.5cm}
          \text{as $\frac{1}{\ln(1+\frac{1}{e^c})} > 1$}\\
          \leq{}& \frac{1}{\ln(\cn)}  
            \Big(D \frac{\ln(\cn)}{\ln(1 + \frac{1}{e^c})} 
            \cdot 2 c \cdot \big(D + \ceil{\ln(n H)} 
            + \sum_{s=j}^{r-1} g_s \big)^k \Big)
          &
          \hspace{-3cm}
          \text{as $\frac{\ln(\cn)}{\ln(1 + \frac{1}{e^c})} \geq 1$}\\[-5pt]
          &&
          \hspace{-3.5cm}
          \text{by~\eqref{inequality1:lower-bound-cn}, and $D \geq 2$}
          \\
          \leq{}& \frac{2cD}{\ln(1 + \frac{1}{e^c})} 
            \big(D + \ceil{\ln(n H)} + \sum_{s=j}^{r-1} g_s \big)^k.
        \end{align*}
          Let us inductively define the following numbers $B_j,\dots,B_\ell$:
          \begin{align*} 
            B_j &\coloneqq 
              \frac{2cD}{\ln(1 + \frac{1}{e^c})} 
              \big(D + \ceil{\ln(n H)} \big)^k\\
            B_r &\coloneqq 
              \frac{2cD}{\ln(1 + \frac{1}{e^c})} 
              \big(D + \ceil{\ln(n H)} + \sum_{s=j}^{r-1} B_s \big)^k
            &\text{for } r \in [j+1..\ell].
          \end{align*}
          From the previous inequalities, $g_r \leq B_r$ for every $r \in [j..\ell]$.
          Moreover, observe that, since $\frac{1}{\ln(1+\frac{1}{e^c})} > 1$,  
          for every $r \in [j+1..\ell]$ we have $B_r \geq D + \ceil{\ln(n H)} + \sum_{s=j}^{r-1} B_s$, and therefore $B_\ell \geq \deg(Q_\ell)$.
          We proceed by bounding $B_r$ with respect to 
          $B_{r-1}$:
          \begin{align*}
              B_r & = \frac{2cD}{\ln(1 + \frac{1}{e^c})} 
                \big(D + \ceil{\ln(n H)} + \sum_{s=j}^{r-1} B_s \big)^k
              \\
              & = \frac{2cD}{\ln(1 + \frac{1}{e^c})} 
                \big(B_{r-1} + D + \ceil{\ln(n H)} + \sum_{s=j}^{r-2} B_s \big)^k\\
              &\leq \frac{2cD}{\ln(1 + \frac{1}{e^c})} \big(2 \cdot B_{r-1} \big)^k\\
              &\leq \frac{2^{k+1}cD}{\ln(1 + \frac{1}{e^c})}(B_{r-1})^k.
          \end{align*}
          Let $A \coloneqq \frac{2^{k+1}cD}{\ln(1 + \frac{1}{e^c})}$. 
          Hence, $B_r \leq A \cdot (B_{r-1})^k$ for every $r \in [j+1..\ell]$.
          We show by induction that $B_r \leq A^{\max(r-j,k^{r-j}-1)}B_j^{k^{r-j}}$
          for every $r \in [j..\ell]$.
          \begin{description}
            \item[base case: $r = j$.] In this case the inequality is trivially satisfied.
            \item[induction step: $r > j$.] We divide the proof depending on whether $k = 1$.
            \begin{itemize}
            \item If $k = 1$, then $\max(r-j,k^{r-j}-1) = r-j$ and we need to prove that $B_r \leq A^{r-j}B_j$. 
            Because $k = 1$, the induction hypothesis simplifies to $B_{r-1} \leq A^{r-1-j}B_j$, and the bound $B_r \leq A \cdot (B_{r-1})^k$ becomes $B_r \leq A \cdot B_{r-1}$. Hence, $B_r \leq A^{r-j}B_j$  follows.
  
            \item If $k \geq 2$, then $\max(r-j,k^{r-j}-1) = k^{r-j}-1$ and therefore we need to prove that $B_r \leq A^{k^{r-j}-1}B_j^{k^{r-j}}$.
            By induction hypothesis $B_{r-1} \leq A^{\max(r-1-j,k^{r-1-j}-1)}B_j^{k^{r-1-j}}$. Here note that if $r-1=j$ then $r-1-j = 0 = k^{r-1-j}-1$, and otherwise $\max(r-j,k^{r-j}-1) = k^{r-j}-1$; 
            hence $B_{r-1} \leq A^{k^{r-1-j}-1}B_j^{k^{r-1-j}}$.
            Then, 
            \begin{align*}
              B_r &\leq A \cdot (B_{r-1})^k\\
              & \leq A \cdot (A^{k^{r-1-j}-1}B_j^{k^{r-1-j}})^k
              &\text{by induction hypothesis}\\
              & = A^{k^{r-j}-k+1}B_j^{k^{r-j}}\\
              & \leq A^{k^{r-j}-1}B_j^{k^{r-j}}
              &\text{since $k \geq 2$}.
            \end{align*} 
            \end{itemize}
          \end{description}
          We can now compute the aforementioned bound on $\deg(Q_\ell)$:
          \begin{align*}
              & \deg(Q_\ell) \leq B_\ell\\
              \leq{}& A^{\max(n,k^{n}-1)}B_j^{k^{n}}
              & \hspace{-1cm}\text{remark: $\ell - j < n$}
              \\
              \leq{}& 
                \left(\frac{2^{k+1}cD}{\ln(1 + \frac{1}{e^c})}\right)^{\max(n,k^{n}-1)}
              \\
              & \cdot
                \left(\frac{2cD}{\ln(1 + \frac{1}{e^c})} 
                \big(D + \ceil{\ln(n H)} \big)^k\right)^{k^n}
              &\hspace{-0.7cm}\text{def.~of $A$ and $B_j$}
              \\
              \leq{}&  
              2^{(k+1)(n+k^n)+k^n}\left(\frac{cD}{\ln(1 + \frac{1}{e^c})}\right)^{n+2k^n} \hspace{-7pt}\big(D + \ceil{\ln(n H)} \big)^{k^{n+1}}\\
              \leq{}& 
              2^{(k+1)(n+k^n)+k^n}\left(\frac{c}{\ln(1 + \frac{1}{e^c})}\right)^{n+2k^n} \hspace{-9pt} D^{n+2k^n+k^{n+1}}\ln(n H)^{1+k^{n+1}}
              \\[-3pt]
              &&\hspace{-1.4cm}\text{since $D \geq 2$ and $H \geq 8$}\\
              \leq{}& 
              \left(\frac{2cD\ln(H)}{\ln(1 + \frac{1}{e^c})}\right)^{5nk^{n+1}} 
              &
              \hspace{-1.6cm}\text{since $\ln(nH) \leq \ln(H)^{2n}$,}\\[-7pt]
              &&\hspace{-5.5cm}\text{and then all exponents are bounded by $5nk^{n+1}$.}
          \end{align*}
          This concludes the proof of the claim.
        \end{claimproof}
  
        We are now ready to derive an explicit characterisation for the set $G$.
        Consider the sets $\mathcal{Q}$ and $B$ defined during the proof of the first statement of the lemma. In particular,
        \[ 
            B \coloneqq \left\{ \beta \in \Z : \text{there is } Q \in \mathcal{Q} \text{ such that } \cn^{\beta} \in \Big\{{\textstyle\lambda(Q(\cn)), \frac{\lambda(Q(\cn) \cdot (\cn-1))}{\cn}, \frac{\lambda(Q(\cn) \cdot (\cn+1))}{\cn}}\Big\}
            \right\}.
        \]
        and $G$ can be set to be any finite set satisfying $[\min B..\max B] \subseteq G$. We also recall that every polynomial $Q$ in the set $\mathcal{Q}$ is such that the numbers $Q(\cn)$, $Q(\cn) \cdot (\cn-1)$ and $Q(\cn) \cdot (\cn+1)$ are all strictly positive; and so, in particular, for these numbers $\lambda$ is well-defined.
        By definition of $\mathcal{Q}$ and from Claims~\ref{claim2:lambda-close-to-variable} 
        and~\ref{claim3:lambda-close-to-variable}, we deduce that the heights and degrees of the univariate polynomials $Q$, $Q \cdot (x-1)$ and $Q \cdot (x+1)$ are bounded as follows: 
        \begin{align*}
          \height(Q) \leq n \cdot H,
          &&\height(Q(x-1)) \leq 2n \cdot H, 
          &&\height(Q(x-1)) \leq 2n \cdot H,\\ 
          \deg(Q) \leq E,
          &&\deg(Q(x-1)) \leq E+1, 
          &&\deg(Q(x-1)) \leq E+1,
        \end{align*}
        where $E \coloneqq \left(\frac{2cD\ln(H)}{\ln(1 + \frac{1}{e^c})}\right)^{5nk^{n+1}}$. Let $P$ be a number among $Q(\cn)$, $Q(\cn) \cdot (\cn-1)$ and $Q(\cn) \cdot (\cn+1)$.
        An upper bound to $P$ is given by 
        \begin{equation*} 
          P \leq 2n \cdot H \cdot (E+2) \cdot \cn^{E+2},
        \end{equation*}
        whereas a lower bound follows by relying on the root barrier $\sigma$: 
        \begin{equation*} 
          P \geq \frac{1}{e^{c(E+1+\ceil{\ln(2nH)})^k}}.
        \end{equation*}
        Recall that, for every $\alpha > 0$, the definition of $\lambda$ implies $\frac{\alpha}{\cn} < \lambda(\alpha) \leq \alpha$.
        We conclude that, for every integer $\beta \in B$, 
        \begin{equation*} 
          \frac{1}{\cn^2 \cdot e^{c(E+1+\ceil{\ln(2nH)})^k}} \leq \cn^{\beta} \text{ \ and \ } \cn^{\beta} \leq 2n \cdot H \cdot (E+2) \cdot \cn^{E+2}.
        \end{equation*}
        Applying the logarithm base $e$ to both inequalities shows: 
        \begin{equation*} 
          -\ln(\cn^2 \cdot e^{c(E+1+\ceil{\ln(2nH)})^k}) \leq \beta \cdot \ln(\cn) \text{ \ and \ } \beta \cdot \ln(\cn) \leq \ln(2n \cdot H \cdot (E+2) \cdot \cn^{E+2}).
        \end{equation*}
        This implies that taking $G$ to be the interval 
        $[-\frac{\ln(\cn^2 \cdot e^{c(E+1+\ceil{\ln(2nH)})^k})}{\ln(\cn)}..
          \frac{\ln(2n \cdot H \cdot (E+2) \cdot \cn^{E+2})}{\ln(\cn)}]$
        suffices. In the statement of the lemma we provide however a slightly larger set with an easier-to-digest bound, that is, $[-L..L]$, where $L \coloneqq \left(2^{3c}D\ceil{\ln(H)}\right)^{6nk^{3n}}$.  
        To conclude the proof, below we show that
        $[-\frac{\ln(\cn^2 \cdot e^{c(E+1+\ceil{\ln(2nH)})^k})}{\ln(\cn)}..
        \frac{\ln(2n \cdot H \cdot (E+2) \cdot \cn^{E+2})}{\ln(\cn)}] \subseteq [-L..L]$.
  
        \begin{description}
          \item[upper bound:] We show that $\frac{\ln(2n \cdot H \cdot (E+2) \cdot \cn^{E+2})}{\ln(\cn)} \leq L$:
          \begin{align*} 
            &  \frac{\ln(2n \cdot H \cdot (E+2) \cdot \cn^{E+2})}{\ln(\cn)}\\ 
            \leq{}& 
                \frac{\ln(2n \cdot H \cdot (E+2))}{\ln(\cn)} + 2 + E
            & \text{by properties of $\ln$}\\
            \leq{}& \frac{\ln(2n \cdot H \cdot (E+2))}{\ln(1+\frac{1}{e^c})} + 2 + E
            & \text{since $\cn \geq 1 + \frac{1}{e^c} $}\\
            \leq{}& \frac{\ln(2n \cdot H)}{\ln(1+\frac{1}{e^c})} + \frac{\ln(E+2)}{\ln(1+\frac{1}{e^c})} + 2 + E
            & \text{by properties of $\ln$}\\
            \leq{}& \frac{2 \cdot \ln(2n \cdot H)}{\ln(1+\frac{1}{e^c})} + \frac{\ln(E+2)}{\ln(1+\frac{1}{e^c})} + E
            & \text{we have $\frac{\ln(2n \cdot H)}{\ln(1+\frac{1}{e^c})} \geq 2$}\\
            \leq{}& 2 \cdot E + \frac{\ln(E+2)}{\ln(1+\frac{1}{e^c})}
            & \text{we have $\frac{2 \cdot \ln(2n \cdot H)}{\ln(1+\frac{1}{e^c})} \leq E$}\\
            \leq{}& 2 \cdot E + \frac{E}{\ln(1+\frac{1}{e^c})}
            & \text{we have $E \geq \ln(E+2)$ since $E \geq 2$}\\
            \leq{}& \frac{3 \cdot E}{\ln(1+\frac{1}{e^c})}
            & \text{as $\frac{1}{\ln(1+\frac{1}{e^c})} \geq 1$}\\
            \leq{}& \frac{3}{\ln(1+\frac{1}{e^c})} \left(\frac{2cD\ln(H)}{\ln(1 + \frac{1}{e^c})}\right)^{5nk^{n+1}}
            & \text{def.~of~$E$}\\
            \leq{}& \left(\frac{2cD\ln(H)}{\ln(1 + \frac{1}{e^c})}\right)^{6nk^{n+1}}
            & \text{as $\frac{2cD\ln(H)}{\ln(1 + \frac{1}{e^c})} \geq \frac{3}{\ln(1+\frac{1}{e^c})}$}\\
            \leq{}& \left(2c \cdot 2^{2c}D\ln(H)\right)^{6nk^{n+1}}
            & \text{as $\frac{1}{\ln(1 + \frac{1}{e^c})} \leq 2^{2c}$}\\
            \leq{}& \left(2^{3c}D\ln(H)\right)^{6nk^{n+1}}
            & \text{since $2c \leq 2^c$}\\
            \leq{}& L 
            & \text{by def.~of $L$}.
          \end{align*}
          \item[lower bound:] We show that $\frac{\ln(\cn^2 \cdot e^{c(E+1+\ceil{\ln(2nH)})^k})}{\ln(\cn)} \leq L$ (so, $-L \leq -\frac{\ln(\cn^2 \cdot e^{c(E+1+\ceil{\ln(2nH)})^k})}{\ln(\cn)}$):
          \begin{align*} 
            & \frac{\ln(\cn^2 \cdot e^{c(E+1+\ceil{\ln(2nH)})^k})}{\ln(\cn)}\\
            \leq{}& 2 + \frac{c(E+1+\ceil{\ln(2nH)})^k}{\ln(\cn)} 
            & \text{by properties of $\ln$}\\ 
            \leq{}& 2 + \frac{c(E+1+\ceil{\ln(2nH)})^k}{\ln(1+\frac{1}{e^c})} 
            & \text{since $\cn \geq 1 + \frac{1}{e^c}$}\\ 
            \leq{}& 2 + \frac{c(E+2+\ln(H)^{4n})^k}{\ln(1+\frac{1}{e^c})} 
            & \text{as $\ceil{\ln(2nH)} \leq 1 + \ln(H)^{4n}$}\\ 
            \leq{}& 2 + \frac{c(2E)^k}{\ln(1+\frac{1}{e^c})} 
            & \text{since $E \geq 2 + \ln(H)^{4n}$}\\ 
            \leq{}& 2 \cdot \frac{c(2E)^k}{\ln(1+\frac{1}{e^c})} 
            & \text{since $\frac{c(2E)^k}{\ln(1+\frac{1}{e^c})} \geq 2$}\\ 
            ={}& \frac{2^{k+1}c}{\ln(1+\frac{1}{e^c})} \cdot E^k\\
            \leq{}& \frac{2^{k+1}c}{\ln(1+\frac{1}{e^c})}\left(\frac{2cD\ln(H)}{\ln(1 + \frac{1}{e^c})}\right)^{5nk^{n+2}}
            & \text{def.~of~$E$}\\
            \leq{}& \left(\frac{2cD\ln(H)}{\ln(1 + \frac{1}{e^c})}\right)^{5nk^{n+2}+k}
            & \text{note: $D \geq 2$}\\
            \leq{}& \left(\frac{2cD\ln(H)}{\ln(1 + \frac{1}{e^c})}\right)^{6nk^{n+2}}\\
            \leq{}& \left(2^{3c}D\ln(H)\right)^{6nk^{n+2}}
            & 
            \text{as in the previous case, } \frac{2c}{\ln(1+\frac{1}{e^c})} \leq 2^{3c}
            \\
            \leq{}& L.
            &&\qedhere
          \end{align*}
        \end{description}
  \end{proof}

  \LemmaLambdaCloseToVariableBody* 

  \begin{proof}
    This lemma follows by~\Cref{lemma:lambda-close-to-variable}: it suffices to replace every monomial $\prod_{j=1}^m x_j^{d_{i,j}}$ with a term $\cn^{z_i}$, where $z_i$ is a fresh variable ranging over $\Z$.
  \end{proof}

  The next lemma provides a first step for proving~\Cref{lemma:last-lambda-lemma}.

  \begin{restatable}{lemma}{LemmaAuxOne}\label{aux-lemma:one}
    Fix $\cn > 1$.
    Let $r(x,\vec y) \coloneqq \sum_{i=0}^n p_i(\cn,\vec y) \cdot x^i$, where 
    each $p_i(z,\vec y)$ is an integer polynomial in variables $\vec y$ and $z$.
    Then, the formula
    \begin{equation*}
        \ipow{\cn}(x) \wedge r(x,\vec y)=0 \land \Big(\bigvee_{i=0}^n p_i(\cn,\vec y) \neq 0\Big) \implies 
        \bigvee_{\ell = 1}^m \theta_\ell (x, \vec y)\,,
    \end{equation*}
    is a tautology of $\exists\R(\ipow{\cn})$, where each $\theta_\ell$ is a formula 
    of the form either 
    \begin{align*} 
      & x^{k-j} = \textstyle\frac{\cn^s \cdot \lambda(-p_j(\cn,\vec y))}
      {\lambda(p_k(\cn,\vec y))} \land p_j(\cn,\vec y) < 0 \land p_k(\cn,\vec y) > 0 \qquad\text{or}\\
      & x^{k-j}= \textstyle\frac{\cn^s \cdot \lambda(p_j(\cn,\vec y))}{\lambda(-p_k(\cn,\vec y))} \land p_j(\cn,\vec y) > 0 \land p_k(\cn,\vec y) < 0\,,
    \end{align*}
    with $0\leq j < k \leq n$, $s \in [-g..g]$ with $g \coloneqq 1+\ceil{\log_{\cn}(n)}$,
    and $m\leq n^2\cdot \left(2 \cdot \ceil{\log_{\cn}(n)} + 3\right)$.
  \end{restatable}

  \begin{proof}
    The proof follows somewhat closely the arguments in~\cite[Lemmas 3.9 and
    3.10]{AvigadY07}. Observe that the lemma is trivially true for $n = 0$,
    as in this case the antecedent of the implication is false
    (from the formulae $r(x,v,\vec y)=0$ and 
    $p_0(\cn,\vec y) \neq 0$). Below, assume $n \geq 1$.

    Pick $x \in \R$ and $\vec y \in R$ making 
    the antecedent of the implication of the formula true, 
    that is, 
    we have $\ipow{\cn}(x)$, $r(x,\vec y) = 0$ and $p_i(\vec y) \neq 0$
    for some $i \in [0,n]$. We
    show that $x$ and $\vec y$ satisfy one of the formulae
    $\theta_1,\dots,\theta_m$.
    
    We can write $r(x,\vec y)$ as $p_k(\cn,\vec y) \cdot x^k + p_j(\cn, \vec y)
    \cdot x^j + r^*(x,\vec y)$ where $p_k(\cn, \vec y) \cdot x^k$ and $p_j(\cn, \vec y) \cdot x^j$ are
    respectively the largest and smallest monomial in $r(x,\vec y)$, and $r^*(x, \vec y)$
    is the sum of all the other monomials.
    Since we are assuming $r(x,\vec y) = 0$ and $p_i(\cn,\vec y) \neq 0$, we
    conclude that $p_k(\cn,\vec y) \cdot x^k > 0$ and $p_j(\cn, \vec y) \cdot x^j < 0$. This also entails
    that $k \neq j$. We have,
    \begin{align} 
      \frac{p_k(\cn,\vec y) \cdot x^k}{\cn}
      &< p_k(\cn,\vec y) \cdot x^k - r(x,\vec y) 
      &\text{since $\cn > 1$ and $r(x,\vec y) = 0$} \notag\\
      &= -p_j(\cn,\vec y) \cdot x^j - r^*(x,\vec y)
      &\text{by def.~of~$p_j$, $p_k$ and $r^*$} \notag\\
      &\leq -n \cdot p_j(\cn,\vec y) \cdot x^j
      &\text{by def.~of~$p_j$, $p_k$ and $r^*$}.
      \label{aux-lemma-1:eq1}
    \end{align}
    Observe that $\ipow{\cn}(x)$ implies $x > 0$, 
    and therefore $p_j(\cn,\vec y) \cdot x^j < 0$ implies $p_j(\cn,\vec y) < 0$.
    From \Cref{aux-lemma-1:eq1} we then obtain $\frac{p_k(\cn,\vec y)}{-p_j(\cn,\vec y)} x^{k-j} \leq n \cdot \cn$.
    Moreover, from $r(x,\vec y) = 0$ we have $-p_j(\cn,\vec y) \cdot x^j \leq n \cdot p_k(\cn,\vec y) \cdot x^k$, i.e., $\frac{1}{n} 
    \leq \frac{p_k(\cn,\vec y)}{-p_j(\cn, \vec y)} \cdot x^{k-j}$, and therefore 
    \[ 
      0 < \cn^{-\ceil{\log_{\cn}(n)}} \leq \cn^{-\log_\cn(n)} = \frac{1}{n} 
      \leq \frac{p_k(\cn,\vec y)}{-p_j(\cn, \vec y)} \cdot x^{k-j} \leq n \cdot \cn = 
      \cn^{1+\log_\cn(n)} \leq \cn^{1+\ceil{\log_{\cn}(n)}}\,.
    \]
    The above chain of inequalities shows that $\frac{-p_j(\cn,\vec y)}{p_k(\cn,\vec y)} \cdot
    \cn^{-\ceil{\log_{\cn}(n)}} \leq x^{k-j} \leq \frac{-p_j(\cn,\vec y)}{p_k(\cn,\vec y)} \cdot
    \cn^{\ceil{\log_{\cn}(n)}+1}$. 
    Since $\lambda$ is a monotonous function, this implies 
    \begin{align*}
      &\lambda\left(\frac{-p_j(\cn,\vec y)}{p_k(\cn,\vec y)} \cdot \cn^{-\ceil{\log_{\cn}(n)}}\right) 
      \leq \lambda(x^{k-j}) \leq \lambda\left(\frac{-p_j(\cn,\vec y)}{p_k(\cn,\vec y)} 
      \cdot \cn^{1+\ceil{\log_{\cn}(n)}}\right).
    \end{align*}
    By definition of $\lambda$, for every $a \in \R$ we have $\frac{a}{\cn} \leq \lambda(a) \leq a$. 
    Moreover, since $x$ is an integer power of $\cn$, $x^{k-j} = \lambda(x^{k-j})$, and therefore the above inequalities can entail 
    \begin{align*}
      &\frac{\lambda(-p_j(\cn,\vec y))}{\lambda(p_k(\cn,\vec y))} \cdot \cn^{-(1+\ceil{\log_{\cn}(n)})} 
      \leq x^{k-j} \leq \frac{\lambda(-p_j(\cn,\vec y))}{\lambda(p_k(\cn,\vec y))} 
      \cdot \cn^{\ceil{\log_{\cn}(n)}+1}
    \end{align*} 
    Let $g \coloneqq
    1+\ceil{\log_{\cn}(n)}$. 
    We conclude that $x^{k-j} = \cn^s \cdot
    \frac{\lambda(-p_j)}{\lambda(p_k)}$, for some integer $s \in [-g..g]$.
    To conclude the proof we analyse two cases, depending on whether $k-j > 0$ (recall: $k\neq j$).
    
    \begin{description}
      \item[case $k-j > 0$.] 
        We have  $x^{k-j} = \cn^s \cdot
        \frac{\lambda(-p_j(\cn, \vec y))}{\lambda(p_k(\cn, \vec y))}$, with
        $p_k(\cn,\vec y) > 0$ and $p_j(\cn, \vec y) < 0$. We have thus obtained the first of the two forms in the statement of the lemma.
      \item[case $k-j < 0$.] 
        We have $x^{j-k} = \cn^{-s} \cdot \frac{\lambda(p_k(\cn, \vec y))}{\lambda(-p_j(\cn, \vec y))}$
        with $p_j(\cn, \vec y) < 0$ and ${p_k(\cn, \vec y) > 0}$. This corresponds to the second of the two forms in the statement of the
        lemma. For convenience, in the statement we have swapped the symbols $j$ and $k$, and wrote $s$ instead of $-s$ (since both these integers belongs to $[-g..g]$).
        \qedhere
    \end{description}
  \end{proof}

  \LemmaLastLambdaLemma*

  \begin{proof}
    The lemma is trivially true for $n = 0$,
    as in this case the premise of the entailment is equivalent to $\bot$
    (from the formulae $r(x,v,\vec y)=0$ and 
    $p_0(\cn,\vec y) \neq 0$). Below, assume~$n \geq 1$.
  
    We start by showing the existence of the finite set $F$ (first statement of the lemma).
    Assume the premise of the entailment 
    of the lemma,~i.e., 
    \begin{equation}
      \label{last-lambda-antecedent}
      \ipow{\cn}(x) 
        \land 1 \leq v < \cn 
        \land \Big(\bigwedge_{i=1}^d \ipow{\cn}(y_i)\Big)
        \land r(x,v,\vec y)=0 
        \land \Big(\bigvee_{i=0}^n p_i(\cn,\vec y) \neq 0\Big)\,,
    \end{equation}
    to be satisfied. 
    We see the polynomial $r(x,v,\vec y)$ 
    as a polynomial in the 
    variable $x$ with coefficients of the form $p_i(\cn,\vec y) \cdot v^i$.
    By applying~Lemma~\ref{aux-lemma:one}, 
    we deduce that the above formula
    entails a finite disjunction $\bigvee_{u=1}^m \theta_u(x,v,\vec y)$ 
    where the formulae $\theta_u(x,v,\vec y)$ are of the form 
    \begin{equation}\label{last-lambda-eq1}
      x^\mu=\frac{\cn^s\cdot \lambda(\pm p_j(\cn,\vec y)v^j)}
      {\lambda(\mp p_w(\cn,\vec y)v^{w})}
      \land \pm p_j(\cn,\vec y)v^j > 0 
      \land \mp p_w(\cn,\vec y)v^w > 0,
    \end{equation}
    where $\mu\in[1..n]$ and $j,w \in [0..n]$ with $j\neq w$.
    Moreover, the number $m$ of disjuncts $\theta_u$ is bounded 
    by $n^2\cdot (2 \cdot \ceil{\log_{\cn}(n)} + 3)$, 
    and $s \in [-(1+\ceil{\log_{\cn}(n)})..(1+\ceil{\log_{\cn}(n)})]$.
  
    By definition of $\lambda$, for every $a,b \in \R$, either $\lambda(a \cdot b) = \lambda(a) \cdot \lambda(b)$ or $\lambda(a \cdot b) = \cn \cdot \lambda(a) \cdot \lambda(b)$.
    Moreover, $v^j$ and $v^w$ are positive numbers, and therefore Formula~\eqref{last-lambda-eq1}
    is equivalent to 
    \begin{equation}\label{last-lambda-eq2}
      \bigvee_{t \in \{-1,0,1\}} x^\mu=\frac{\cn^{s+t}\cdot \lambda(\pm p_j(\cn,\vec y)) \cdot \lambda(v^j)}
      {\lambda(\mp p_w(\cn,\vec y)) \cdot \lambda(v^w)}
      \land \pm p_j(\cn,\vec y) > 0 
      \land \mp p_w(\cn,\vec y) > 0.
    \end{equation}
    Next, we bound the terms $\lambda(v^j)$ and $\lambda(v^w)$. 
    Since Formula~\eqref{last-lambda-antecedent} asserts $1 \leq v < \cn$,
    we have $\lambda(v^j) = \cn^{\alpha}$ for some $\alpha \in [0,j-1]$; 
    and similarly $\lambda(v^w) = \cn^{\beta}$ for some $\alpha \in [0,w-1]$.
    Given that $j$ and $w$ belong to $[0..n]$, 
    Formula~\eqref{last-lambda-eq2} (or, equivalently, Formula~\eqref{last-lambda-eq1}) then entails
    \begin{equation}\label{last-lambda-eq3}
      \bigvee_{t=-n}^n
      x^\mu=\frac{\cn^{s+t} \cdot \lambda(\pm p_j(\cn,\vec y))}
      {\lambda(\mp p_w(\cn,\vec y))}
      \land \pm p_j(\cn,\vec y) > 0 
      \land \mp p_w(\cn,\vec y) > 0.
    \end{equation}
    Let $p(\cn,\vec y)$ be a polynomial among $\pm p_j(\cn,\vec y)$ and $\mp p_w(\cn,\vec y)$.
    This polynomial can be seen as having variables in $\vec y$, and having as coefficients polynomial expressions in $\cn$, that is,
    \begin{equation*}
      p(\cn,\vec y)= \sum_{\ell=1}^{\abs{M}} q_{\ell}(\cn) \cdot \vec{y}^{\vec{e}_\ell}.
    \end{equation*}
    where each $\vec{y}^{\vec{e}_\ell}$ is a monomial from $M$.
    Since Formula~\eqref{last-lambda-antecedent} asserts that every variable in $\vec y$ 
    is an integer power of $\cn$, given $\ell \in [1,\abs{M}]$ we can introduce an integer variable $z_\ell$ and set $\cn^{z_\ell} = \vec{y}^{\vec{e}_\ell}$. That is, Formula~\eqref{last-lambda-antecedent} entails the following formula of $\R(\ipow{\cn})$
    \begin{equation}\label{last-lambda-eq4}
      p(\cn, \vec y) > 0 
      \iff \exists z_1 \dots z_{\abs{M}} \in \Z \,
        \Big( \sum_{\ell=1}^{\abs{M}} q_{\ell}(\cn) \cdot \cn^{z_\ell} > 0 \land \bigwedge_{\ell=1}^{\abs{M}} \cn^{z_\ell} = \vec{y}^{\vec{e}_\ell} \Big).
    \end{equation}
    We apply Lemma~\ref{lemma:lambda-close-to-variable} on $\sum_{\ell=1}^{\abs{M}} q_{\ell}(\cn) \cdot \cn^{z_\ell} > 0$: there is a finite set $G_p$ such that 
    \begin{equation*}
      \sum_{\ell=1}^{\abs{M}} q_{\ell}(\cn) \cdot \cn^{z_\ell} > 0
      \implies 
      \bigvee_{g \in G_p} \bigvee_{\ell=1}^{\abs{M}}\lambda\Big(\sum_{\ell=1}^{\abs{M}} q_{\ell}(\cn) \cdot \cn^{z_\ell}\Big)=
      \cn^{g} \cdot \cn^{z_\ell}.
    \end{equation*}
    Then, by Formula~\eqref{last-lambda-eq4}, substituting $\cn^{z_\ell}$ for $\vec{y}^{\vec{e}_\ell}$ 
    we obtain 
    \begin{equation}\label{last-lambda-eq5}
      p(\cn, \vec y) > 0 
      \implies \bigvee_{g \in G_p} \bigvee_{\vec y^{\vec \ell} \in M}\lambda\left(p(\cn,\vec y)\right)=
      \cn^{g} \cdot \vec y^{\vec \ell}.
    \end{equation}
    From Formulas~\eqref{last-lambda-eq3} and~\eqref{last-lambda-eq5}, 
    we conclude that Formula~\eqref{last-lambda-eq1} entails
    \begin{equation}\label{last-lambda-eq6}
      \bigvee_{t=-n}^n \bigvee_{(g_1,g_2) \in \left(G_{\pm p_j} \times G_{\mp p_w}\right)} 
      \bigvee_{\vec y^{\vec \ell_1}, \vec y^{\vec \ell_2} \in M}
      \vec x^\mu = \cn^{s+t+g_1-g_2} \cdot \vec y^{\vec \ell_1-\vec \ell_2}.
    \end{equation}
    Above, note that $\vec y^{\vec \ell_1-\vec \ell_2}$ belongs to the set $N$ in the statement of the lemma.
    Since Formula~\eqref{last-lambda-antecedent} entails a finite disjunction of formulae 
    of the form shown in~\eqref{last-lambda-eq1}, 
    and the disjunctions in Formula~\eqref{last-lambda-eq6} are over finite sets, 
    this completes the proof of the first statement of the lemma.
    In particular, one can take as $F$ the set 
    \[ 
      F \coloneqq [1..n] \times [-L..L] \times N
    \]
    where $L \coloneqq \max \{ 1 + \ceil{\log_{\cn}(n)} + n + 2 \abs{g'} : g' \in G_{\pm p_j} \text{ for some } j \in [0..n]\}$.
  
    We move to the second part of the lemma, which adds further assumptions
    on~$\cn$.
  
    \noindent
    \textbf{Case: $\cn$ is a computable transcendental number (Item~\eqref{lemma:last-lambda-lemma:i1}).} 
    From Item~\eqref{lemma:lambda-close-to-variable-body:i1} in~\Cref{lemma:lambda-close-to-variable-body}, we conclude that the sets $G_{\pm
    p_j}$ can be computed. Moreover, $\ceil{\log_{\cn}(n)}$ can be computed by
    iterating through the natural numbers, finding $\alpha \in \N$ such that
    $\cn^{\alpha-1} < n \leq \cn^{\alpha}$. Checking these inequalities can be
    done by opportunely iterating the algorithm for the sign evaluation problem for transcendental numbers already 
    discussed in the proof of~\Cref{lemma:lambda-close-to-variable}.
  
    \noindent
    \textbf{Case: $\cn$ has a polynomial root barrier (Item~\eqref{lemma:last-lambda-lemma:i2}).}
    Assume now $\cn$ to have a polynomial root barrier 
    $\sigma(d,h) \coloneqq c \cdot (d+\ceil{\ln(h)})^k$, 
    with $c,k \in \N_{\geq 1}$. 
    We provide an explicit upper bound to the set $L$ defined above, 
    so that replacing $L$ with this upper bound in the definition of $F$ 
    yield the last statement of the lemma.
    For this, it suffices to upper bound $\ceil{\log_{\cn}(n)}$ 
    as well as $\abs{g'}$, where $g' \in G_{\pm p_j}$ with $j \in [0..n]$.
  
    For the upper bound to $\ceil{\log_{\cn}(n)}$, 
    as done in the proof of Lemma~\ref{lemma:lambda-close-to-variable}, 
    we can consider the polynomial $x-1$ in order to obtain a lower bound on the number $\cn > 1$ 
    via the root barrier $\sigma$. We have $\cn \geq 1 + \frac{1}{e^c}$.
    Then, 
    \begin{align}
      \ceil{\log_{\cn}(n)} & = \ceil{\frac{\ln(n)}{\ln(\cn)}} &\text{by properties of $\ln$} \notag\\
      & \leq \ceil{\frac{\ln(n)}{\ln(1 + \frac{1}{e^c})}} 
      & \text{since $\cn \geq 1 + \frac{1}{e^c}$} \notag\\
      & \leq \ceil{\ln(n) \cdot 2^{2c}} 
      & \text{since $\frac{1}{\ln(1 + \frac{1}{e^c})} \leq 2^{2c}$} \notag\\ 
      & \leq 2^{2c} \ceil{\ln(n)}. \label{last-lambda-eq7}
    \end{align}
    
    For the bound on the elements in $G_{\pm p_j}$, 
    recall that this set has been computed following 
    Lemma~\ref{lemma:lambda-close-to-variable-body}.
    The polynomial $\pm p_j$ 
    is of the form $\sum_{\ell=1}^{\abs{M}} q_{\ell}(\cn) \cdot \vec{y}^{\vec{e}_\ell}$, where 
    $\height(q_\ell) \leq H$ and $\deg(q_{\ell}) \leq D$.
    Therefore,
    by~Lemma~\ref{lemma:lambda-close-to-variable-body}, 
    $G_{\pm p_j}$ can be taken to be the interval $[-L'..L']$ where $L' \coloneqq \left(2^{3c}D\ceil{\ln(H)}\right)^{6\abs{M} \cdot k^{3\abs{M}}}$.
    We can now conclude the proof:
    \begin{align*} 
      L &= \max \{ 1 + \ceil{\log_{\cn}(n)} + n + 2 \abs{g'} : g' \in G_{\pm p_j} \text{ for some } j \in [0..n]\}\\ 
      & \leq 1 + \ceil{\log_{\cn}(n)} + n + 2\left(2^{3c}D\ceil{\ln(H)}\right)^{6\abs{M} \cdot k^{3\abs{M}}}
      &\text{bound on $G_{\pm p_j}$}\\
      & \leq 1 + 2^{2c}\ceil{\ln(n)} + n + 2\left(2^{3c}D\ceil{\ln(H)}\right)^{6\abs{M} \cdot k^{3\abs{M}}}
      &\text{by Equation~\ref{last-lambda-eq7}}\\
      & \leq 2^{2c+1}n + 2\left(2^{3c}D\ceil{\ln(H)}\right)^{6\abs{M} \cdot k^{3\abs{M}}}
      &\text{since $c,n \geq 1$}\\
      & \leq 3n\left(2^{3c}D\ceil{\ln(H)}\right)^{6\abs{M} \cdot k^{3\abs{M}}}\\
      & \leq n\left(2^{4c}D\ceil{\ln(H)}\right)^{6\abs{M} \cdot k^{3\abs{M}}}.
      &&\qedhere
    \end{align*}
  
    % Again we start from the premises:
    %   \begin{equation*}
    %     \ipow{\cn}(x) 
    %     \land 1 \leq v < \cn 
    %     \land \Big(\bigwedge_{i=1}^d \ipow{\cn}(y_i)\Big)
    %     \land r(x,v,\vec y)=0 
    %     \land \Big(\bigvee_{i=0}^n p_i(\cn,\vec y) \neq 0\Big)
    %   \end{equation*}
    % \begin{claim}
    %   $\mu\in[1..n]$
    % \end{claim}
    % \begin{claimproof}
    %   We apply Lemma \ref{aux-lemma:one} to the polynomial 
    %   $r(x,v,\vec y)$ and obtain that the premises imply a 
    %   disjunction of formulas of the form 
    %   \begin{equation*}
    %     x^\mu=x^{k-j}=\frac{\cn^s\cdot \lambda(\pm p_j(\cn,\vec y)v^j)}
    %     {\lambda(\mp p_k(\cn,\vec y)v^k)}
    %   \end{equation*}
    %   where $\abs{s}\leq 1+\ceil{\log_\cn(n)}$ and $j,k\leq n$.
      
    %   It is easy to see that $\mu\in[1..n]$ from Lemma \ref{aux-lemma:one} 
    %   since, in the notation of 
    %   the Lemma, $\mu=k-j$ with $0\leq j<k\leq n$.
    % \end{claimproof}
    
    % Now we remove the dependence on 
    % $v$ from terms in the form $\lambda(p_i(\cn,\vec y)v^i)$ 
    % by guessing between which powers of
    % $\cn$ will $v^i$ be. This is possible since we know 
    % $1\leq v<\cn$ and $i\in[0..n]$.
    % We give the example for when $i=j$ since 
    % $i=k$ is done in the same way.
    % \begin{equation*}
    %   1\leq v<\cn\implies \bigvee^j_{a=1}\cn^{a-1}\leq v^j < \cn^a
    % \end{equation*}
    % equivalently
    % \begin{equation*}
    %   1\leq v<\cn\implies 
    %   \bigvee^j_{a=1}\lambda(v^j)=\cn^{a-1}
    % \end{equation*}
  
    % The function $\lambda(\cdot)$ is not multiplicative, but
    % $\lambda(x\cdot y)$ can only take the values 
    % $\lambda(x)\lambda(y)$ or $\cn\lambda(x)\lambda(y)$.
    % Therefore:
    % \begin{equation*}
    %   1\leq v<\cn \implies \left(\bigvee_{a=1}^{j}\bigvee_{\delta=0,1}
    %   \lambda(\pm p_j(\cn,\vec y)v^j)=
    %   \cn^\delta\lambda(\pm p_j(\cn,\vec y))\lambda(v^j)=
    %   \cn^\delta\cn^{a-1}\lambda(\pm p_j(\cn,\vec y))\right)
    % \end{equation*}
    % This can be written as
    % \begin{equation*}
    %   1\leq v<\cn \implies \bigvee_{\alpha=0}^{j}
    %   \lambda(\pm p_j(\cn,\vec y)v^j)=
    %   \cn^\alpha\lambda(\pm p_j(\cn,\vec y))
    % \end{equation*}  
    % Hence, 
    % \begin{equation*}
    %   1\leq v<\cn\land x^\mu=\frac{\cn^s\cdot \lambda(\pm p_j(\cn,\vec y)v^j)}
    %   {\lambda(\mp p_k(\cn,\vec y)v^k)} 
    %   \implies \bigvee_{\alpha=0}^{j}\bigvee_{\beta=0}^{k}
    %   x^\mu = \frac{\cn^s\cn^\alpha\lambda(\pm p_j(\cn,\vec y))}
    %   {\cn^\beta\lambda(\mp p_k(\cn,\vec y))}
    % \end{equation*} 
    % At this point we have proven
    % \begin{align*}
    %   \ipow{\cn}(x) 
    %   \land 1 \leq v < \cn 
    %   \land \Big(\bigwedge_{i=1}^d \ipow{\cn}(y_i)\Big)
    %   \land r(x,v,\vec y)=0 
    %   \land \Big(\bigvee_{i=0}^n p_i(\cn,\vec y) \neq 0\Big)\\
    %   \implies 
    %   \hspace{-5pt}
    %   \bigvee_{(\mu,s,\alpha,\beta)\in[1..n]\times S\times[0..n]^2}
    %   x^\mu = \frac{\cn^s\cn^\alpha\lambda(\pm p_j(\cn,\vec y))}
    %   {\cn^\beta\lambda(\mp p_k(\cn,\vec y))}
    % \end{align*}
    % with $S\coloneq [-1-\ceil{\log_\cn(n)}..1+\ceil{\log_\cn(n)}]$.
  
  
    % \begin{claim}
    %   $x^\mu = \cn^g\frac{\cn^{z^*_{t_j}}}{\cn^{z^*_{t_k}}}{}$ with $g\in[-L..L]$ and 
    %   $$L=1 + \ceil{\frac{\log(n)}{\log\left(\frac{e^c+1}{e^c}\right)}} +n 
    %   +2\left(2^{3c}(D+1)\ceil{\ln(H)}\right)^{6nk^{n+2}}$$
    % \end{claim}
    % \begin{claimproof}
    %   \jg{Improve phrasing along all the proof}
    %   First let us rewrite the bound on the exponent $s$ that appeared in the previous 
    %   claim. Applying 
    %   the root barrier function of $\cn$ to the polynomial $x-1$ and doing
    %   some trivial manipulations we obtain $\cn\geq \frac{e^c+1}{e^c}$,
    %   therefore $\log(\cn)\geq \log\left(\frac{e^c+1}{e^c}\right)$.
    %   Hence
    %   \begin{align*}
    %     \abs{s}\leq 1+\ceil{\log_\cn(n)}=1+\ceil{\frac{\log(n)}{\log(\cn)}}\leq
    %     1+\ceil{\frac{\log(n)}{\log\left(\frac{e^c+1}{e^c}\right)}}
    %   \end{align*}
  
    %   The following shows how the $p_i(\cn,\vec y)$ can be manipulated
    %   in order to apply Lemma 
    %   \ref{lemma:lambda-close-to-variable} on them:
    %   \begin{align*}
    %     p_i(\cn,\vec y) &= \sum^{\abs{M}}_{s=1}a_s\cn^{c_s}\bar{y}^{\bar{e}_s}
    %     &\text{$p_i$ are integer polynomials}\\
    %     &= \sum_{s=1}^{|M|}a_s\cn^{c_s}\cn^{z_s}
    %     &\text{since we have $\ipow{\cn}(y_i)$}\\
    %     &= 
    %     \sum_{s_i=1}^{|M_{i}|}\left(\sum^{|M_{s_i}|}_{t_i=1} a_{t_i}\cn^{c_{t_i}}\right)\cn^{z_{s_i}}
    %     &\text{group around the $\cn^{z_s}$ which are equal}\\
    %     &= \sum_{s_i=1}^{|M_{i}|}q_{s_i}(\cn)\cdot \cn^{z_{s_i}}
    %   \end{align*}
    %   So we can rewrite the clauses
    %   \begin{equation*}
    %     x^\mu = \frac{\cn^s\cn^\alpha\lambda(\pm p_j(\cn,\vec y))}
    %     {\cn^\beta\lambda(\mp p_k(\cn,\vec y))} \iff 
    %     x^\mu = \frac{\cn^s\cn^\alpha\cdot 
    %     \lambda\left(\pm \sum^{|M_j|}_{t=1}q_{t_j}(\cn)\cdot \cn^{z_{t_j}}\right)}
    %     {\cn^\beta\cdot \lambda\left(\mp \sum^{|M_k|}_{t=1}q_{t_k}(\cn)
    %     \cdot \cn^{z_{t_k}}\right)}.
    %   \end{equation*}
      
  
    %   Finally we can apply Lemma 
    %   \ref{lemma:lambda-close-to-variable} to transform each of the 
    %   $\lambda(\cdot)$ terms into some $\cn^{g_i}\cn^{z^*_{t_i}}$ 
    %   as follows:
    %   \begin{equation*}
    %     \bigvee_{(g_i,z^*_{t_i})\in G\times\Z}\lambda\left(\pm \sum^{|M_i|}_{t=1}q_{t_i}(\cn)\cdot \cn^{z_{t_i}}\right)
    %     =\cn^{g_i}\cdot \cn^{z^*_{t_i}}
    %   \end{equation*}
    %   with $G_i=[-B_i..B_i]$ and $B_i=\left(2^{3c}(D_i+1)\ceil{\ln(H_i)}\right)^{6nk^{n+2}}$
    %   \jg{problem in the notation with k}
    %   Here $D_i$ is the maximum degree of the $q_{t_i}(\cn)$ which is precisely
    %   $\deg(\cn,p_i)$, and $H_i$ is the maximum height among the $q_{t_i}(\cn)$
    %   which coincides with $h(p_i)$. Therefore these can be bounded by 
    %   $D\coloneq \max\{1,\deg(z,p_i):i\in[0,n]\}$ and 
    %   $H\coloneq \max\{8,h(p_i):i\in[1,n]\}$, so if we define 
    %   $G=[-B..B]$ and $B=\left(2^{3c}(D+1)\ceil{\ln(H)}\right)^{6nk^{n+2}}$,
    %   $G_i\subseteq G$ holds so
    %   $g_i\in G$ for every $i\in[0..n]$. Let us postpone the reasoning about
    %   the $\cn^{z^*_{t_i}}$ briefly and for the moment think about the
    %   $z^*_{t_i}$ as integers.
  
  
  
    %   Therefore the following is true:
    %   \begin{equation*}
    %     \bigvee_{[g_j,g_k,z^*_{t_j},z^*_{t_k}]\in G^2\times \Z^2}
    %     \frac{\lambda\left(\pm \sum^{|M_j|}_{t=1}q_{t_j}(\cn)\cdot 
    %     \cn^{z_{t_j}}\right)}{\lambda\left(\mp
    %     \sum^{|M_k|}_{t=1}q_{t_k}(\cn)\cdot \cn^{z_{t_k}}\right)}
    %     = \frac{\cn^{g_j}\cn^{z^*_{t_j}}}{\cn^{g_k}\cn^{z^*_{t_k}}}
    %   \end{equation*}
    %   So we have proven the following 
    %   \begin{align*}
    %     \ipow{\cn}(x) 
    %     \land 1 \leq v < \cn 
    %     \land \Big(\bigwedge_{i=1}^d \ipow{\cn}(y_i)\Big)
    %     \land r(x,v,\vec y)=0 
    %     \land \Big(\bigvee_{i=0}^n p_i(\cn,\vec y) \neq 0\Big)\\
    %     \implies 
    %     \hspace{-5pt}
    %     \bigvee_{(\mu,s,\alpha,\beta,g_j,g_k,z^*_{t_j},z^*_{t_k})\in
    %     [1..n]\times S\times[0..n]^2\times G^2\times\Z^2}
    %     x^\mu = \frac{\cn^s\cn^\alpha \cn^{g_j}\cn^{z^*_{t_j}}}
    %     {\cn^\beta\cn^{g_k}\cn^{z^*_{t_k}}}
    %   \end{align*}
    %   \jg{fix notation?}
  
    %   Now we are able to provide a bound for the exponent $g$ in 
    %   $\cn^g=\cn^s\cn^\alpha \cn^{g_j}\cn^{-\beta}\cn^{-g_k}$ 
    %   since we already 
    %   know bounds for $s$, $\alpha$, $g_j$, $\beta$ and $g_k$.
  
    %   An upper bound comes from considering the maximum possible values for 
    %   $s$, $\alpha$, $g_j$ and the minimum possible values for 
    %   $\beta$ and $g_k$, all of which have already been discussed. 
    %   Hence,
    %   \begin{align*}
    %     g &\leq 1 + \ceil{\frac{\log(n)}{\log\left(\frac{e^c+1}{e^c}\right)}}
    %     + n + (2^{3c}(D+1)\ceil{\ln(H)})^{6nk^{n+2}} - 0 - 
    %     \left( - (2^{3c}(D+1)\ceil{\ln(H)})^{6nk^{n+2}} \right)\\
    %     & =1 + \ceil{\frac{\log(n)}{\log\left(\frac{e^c+1}{e^c}\right)}} +n 
    %     +2\cdot (2^{3c}(D+1)\ceil{\ln(H)})^{6nk^{n+2}}
    %   \end{align*}
    %   In the same manner we can get a lower bound considering the minumum
    %   possible values for $s$, $\alpha$, $g_j$ and the maximum possible 
    %   values for $\beta$ and $g_k$:
    %   \begin{align*}
    %     g &\geq \left(-1 - \ceil{\frac{\log(n)}{\log\left(\frac{e^c+1}{e^c}\right)}}\right)
    %     + 0+ \left(-(2^{3c}(D+1)\ceil{\ln(H)})^{6nk^{n+2}}\right) - n - 
    %     \left((2^{3c}(D+1)\ceil{\ln(H)})^{6nk^{n+2}} \right)\\
    %     & =-1 - \ceil{\frac{\log(n)}{\log\left(\frac{e^c+1}{e^c}\right)}} -n 
    %     -2\cdot (2^{3c}(D+1)\ceil{\ln(H)})^{6nk^{n+2}} 
    %   \end{align*}
    %   The claim is proven with $L\coloneq 1 + \ceil{\frac{\log(n)}{\log\left(\frac{e^c+1}{e^c}\right)}} +n 
    %   +2\cdot (2^{3c}(D+1)\ceil{\ln(H)})^{6nk^{n+2}}$
    % \end{claimproof}
    % At this point we have proven that 
    % \begin{align*}
    %   \ipow{\cn}(x) 
    %   \land 1 \leq v < \cn 
    %   \land \Big(\bigwedge_{i=1}^d \ipow{\cn}(y_i)\Big)
    %   \land r(x,v,\vec y)=0 
    %   \land \Big(\bigvee_{i=0}^n p_i(\cn,\vec y) \neq 0\Big)\\
    %   \implies 
    %   \hspace{-5pt}
    %   \bigvee_{(\mu,g,z^*_{t_j},z^*_{t_j})\in
    %   [1..n]\times[-L..L]\times \Z^2}
    %   x^\mu = \cn^g\cdot \frac{\cn^{z^*_{t_j}}}
    %   {\cn^{z^*_{t_k}}}
    % \end{align*}
  
    % Now we procceed to reason about the $\cn^{z^*_{t_i}}$. 
    % Observe that both $\cn^{z^*_{t_j}}$ and $\cn^{z^*_{t_k}}$ 
    % come from monomials of $M$, so the following is always true:
    % \begin{equation*}
    %   \bigvee_{\vec y^{\sigma_i}\in M}\cn^{z^*_{t_i}}= \vec y^{\vec \sigma_i}
    % \end{equation*}
    % It is easy to see now that  
    % \begin{align*}
    %   \ipow{\cn}(x) 
    %   \land 1 \leq v < \cn 
    %   \land \Big(\bigwedge_{i=1}^d \ipow{\cn}(y_i)\Big)
    %   \land r(x,v,\vec y)=0 
    %   \land \Big(\bigvee_{i=0}^n p_i(\cn,\vec y) \neq 0\Big)\\
    %   \implies 
    %   \hspace{-5pt}
    %   \bigvee_{(\mu,g,\vec y^{\vec\sigma_j},\vec y^{\vec \sigma_k})
    %   \in[1..n]\times[-L..L]\times M^2}
    %   x^\mu = \cn^g\cdot \frac{\vec y^{\vec\sigma_j}}
    %   {\vec y^{\vec\sigma_k}}
    % \end{align*}
    % And the statement follows.
  \end{proof}

  %%%%
  % \LemmaRewritePowerPredicates*

  % \begin{proof}
  %   For simplicity, throughout this lemma we lift the restriction that, in the
  %   unary predicate $\ipowar{\cn}{m}{r}$, $r$ belongs to $[0,m-1]$, and let this
  %   number range instead over all integers. Given $t \in \Z \setminus [0,m-1]$, the
  %   predicate $\ipowar{\cn}{m}{t}$ is interpreted as the predicate 
  %   $\ipowar{\cn}{m}{(t \,\text{mod}\, m)}$.
  
  %   Assume  $u^j=\cn^gu_1^{r_1} \cdot \ldots \cdot u_n^{r_n}$. 
  %   Then, the following sequence of equivalences holds:
  %   \begin{align*}
  %       \ipowar{\cn}{m}{r}(u) 
  %       & \iff \ipowar{\cn}{j\cdot m}{rj}(u^j) 
  %       &\text{by def.~of $\ipowar{\cn}{m}{r}(u)$}\\
  %       & \iff \ipowar{\cn}{j\cdot m}{rj}(\cn^{g}u_1^{r_1}\dots u_n^{r_n})\\ 
  %       & \iff \ipowar{\cn}{j\cdot m}{rj-g}(u_1^{r_1}\dots u_n^{r_n}).
  %       \
  %   \end{align*}
  %   Consider the set $I \coloneqq [0..j\cdot m-1]^n$.
  %   The following equivalence is trivially true:
  %   \begin{equation}\label{equivalence:power-predicate-split-u}
  %     \cn^{j\cdot m\Z+rj-g}(u_1^{r_1}\dots u_n^{r_n}) 
  %     \iff
  %     \!\!\!
  %     \bigvee_{(s_1,\dots,s_{n})\in I}\Big(\bigwedge^{n}_{i=1}
  %     \cn^{j\cdot m\Z+s_i}(u_i)\land
  %     \cn^{j\cdot m\Z+rj-g}(u_1^{r_1}\dots u_n^{r_n})\Big).
  %   \end{equation}
  %   Consider a tuple $(s_1,\dots,s_{n})\in I$ 
  %   and assume that $\bigwedge^{n}_{i=1}
  %   \cn^{j\cdot m\Z+s_i}(u_i)\land
  %   \cn^{j\cdot m\Z+rj-g}(u_1^{r_1}\dots u_n^{r_n})$ is satisfied. 
  %   We show that then $j \cdot m$ must divide $r \cdot j - g - \sum_{i=1}^n r_i \cdot s_i$. This allows us to restrict the disjunction over the set $I$ 
  %   appearing in Equivalence~\eqref{equivalence:power-predicate-split-u} 
  %   to the set $S$ in the statement of the lemma, completing the proof.
  
  %   By definition, the formula $\bigwedge^{n}_{i=1}
  %   \cn^{j\cdot m\Z+s_i}(u_i)\land
  %   \cn^{j\cdot m\Z+rj-g}(u_1^{r_1}\dots u_n^{r_n})$ implies that there are $w_1,\dots,w_n,w \in \Z$ such that 
  %   $\bigwedge^{n}_{i=1}
  %     u_i = \cn^{j\cdot m \cdot w_i+s_i}
  %     \land
  %     u_1^{r_1}\dots u_n^{r_n} = \cn^{j\cdot m \cdot w+rj-g}$.
  %   Therefore,
  %   \[ 
  %     \sum_{i=1}^n r_i \cdot (j\cdot m \cdot w_i+s_i) = j\cdot m \cdot w+r \cdot j-g.
  %   \]
  %   Rearranging this equivalence shows 
  %   \[ 
  %     j \cdot m \cdot (-w + \sum_{i=1}^n r_i \cdot w_i) = r \cdot j-g - \sum_{i=1}^n r_i \cdot s_i,
  %   \]
  %   which implies that $j\cdot m$ divides $r \cdot j-g-\sum_{i=1}^n r_i \cdot s_i$.
  % \end{proof}

\LemmaRelativiseQuantifiers* 


\begin{proof}
    The right-to-left implication is trivial. Let us show the left-to-right
    implication. Below, let $\psi(u,\vec y) \coloneqq \phi \land \ipow{\cn}(u)
    \land \bigwedge_{y \in \vec y} \ipow{\cn}(y)$. For simplicity of the
    argument, instead of the left-to-right implication in the statement, we
    consider the following formula~of~$\R(\ipow{\cn})$:%
    \begin{equation}
      (\exists u \, \psi)
      \implies 
      \bigvee_{\ell = -1}^{1}\ 
      \bigvee_{q \in Q}\ 
      \bigvee_{(j,g,\vec y^{\vec \ell_1}, \vec y^{\vec \ell_2}) \in F_q}
      \exists u \left( u^{j} = \cn^{j \cdot \ell + g} \cdot \vec y^{\vec \ell_1-\vec \ell_2} \land \psi \right).
      \label{lemma:relativise-quantifiers:eq0}
    \end{equation}
    Since in this implication all variables are constrained to be integer powers
    of $\cn$, this formula is equivalent to the left-to-right implication of the
    equivalence in the statement of the lemma. We show
    Formula~\eqref{lemma:relativise-quantifiers:eq0} by relying on a series of
    tautologies.
  
    \begin{claim}\label{lemma:relativise-quantifiers:claim1}
      Let $Q'$ be the set of all polynomials in $\phi$ featuring $u$.
      The following formula is a tautology of~$\R(\ipow{\cn})$:
      \begin{align*}
        (\exists u\, \psi)
        \implies 
        &\Big(\big(\forall u \,(u > 0 \implies \phi)\big) 
        \lor\\
        &\hspace{-1cm} \bigvee_{r \in Q'}
        \exists w \big(w > 0 \land r(w,\vec y) = 0 \land (\bigvee_{i=0}^n p_{r,i}(\cn,\vec y) \neq 0) 
        \land \exists u (w \cdot \cn^{-1} \leq u \leq w \cdot \cn \land \psi) \big)
        \Big),
      \end{align*}
      where $r \in Q'$ is of the form $r(x,\vec y) = \sum_{i=0}^n p_{r,i}(\cn,\vec y) \cdot x^i$.
    \end{claim}
    \begin{claimproof}
      Let $\vec y^*$ be real numbers that are a solution to the formula $(\exists u\, \psi) \land \lnot \forall u \,(u > 0 \implies \phi)$. 
      To prove the claim, it suffices to show that then $\vec y^*$ is a solution to the formula
      \begin{equation}\label{lemma:relativise-quantifiers:eqn1}
        \bigvee_{r \in Q'}
        \exists w \big(w > 0 \land r(w,\vec y) = 0 \land (\bigvee_{i=0}^n p_{r,i}(\cn,\vec y) \neq 0) 
        \land \exists u (w \cdot \cn^{-1} \leq u \leq w \cdot \cn \land \psi) \big).
      \end{equation}
      Let $S \coloneqq \{ u \in \R : \phi(u,\vec y^*) \land u > 0 \}$ be the set
      of positive real numbers satisfying $\phi$ with respect to the vector
      $\vec y^*$ we have picked. Since $\vec y^*$ satisfies $\lnot \forall u (u
      > 0 \implies \phi)$, we have $S \subsetneq \R_{>0}$. Since $S$ is the set
      of solutions of over $\R_{>0}$ of a formula in the language of Tarski
      arithmetic, it is a finite union~$\bigcup_{j \in J} I_j$ of disjoint
      (open, closed or half-open) intervals with endpoints in $\R \cup
      \{{+}\infty\}$. This follows directly from the fact that Tarski arithmetic
      is an o-minimal theory~\cite[Chapter 3.3]{marker2002model}. Without loss
      of generality, we can assume $\{I_j\}_{j \in J}$ to be a minimal family of
      intervals characterising $S$; in other words, we can assume that for every
      two distinct intervals $I_j$ and $I_k$, the set $I_j \cup I_k$ is not an
      interval. Since $\vec y^*$ satisfies $\exists u\, \psi$, there is $j \in
      J$ such that $I_j$ contains an integer power of $\cn$, $\cn^{i_j}$. The
      interval $I_j$ is of the form $(a,b)$, $[a,b)$, $(a,b]$ or $[a,b]$, for
      some $a \in \R_{>0}$ and $b \in \R_{>0} \cup \{{+}\infty\}$. We divide the
      proof in two cases, depending on whether $b = {+}\infty$.
      \begin{description}
        \item[case: $b \neq {+}\infty$.] 
          There is an interval $(c,d)$ around $b$ such that $(c,b)$ and $(b,d)$
          are non-empty, $(c,d) \subseteq I_j$, and $(b,d) \cap S = \emptyset$.
          That is, the truth of the formula $\phi(u,\vec y^*) \land u > 0$
          changes around $b$. Since $\phi(u,\vec y)$ is a quantifier-free
          formula from $\exists \ipow{\cn}$, this means that the truth value of
          a polynomial inequality $r(u,\vec y^*) \sim 0$ changes around $b$,
          which in turn implies both $r(b,\vec y^*) = 0$ (since polynomials are
          continuous functions) and that $r(b,\vec y^*)$ is non-constant, i.e.,
          $\bigvee_{i=0}^n p_{r,i}(\cn,\vec y^*) \neq 0$. At this point we have
          established that $b > 0 \land r(b,\vec y^*) = 0 \land (\bigvee_{i=0}^n
          p_{r,i}(\cn,\vec y) \neq 0)$ holds, and hence to conclude that
          Formula~\eqref{lemma:relativise-quantifiers:eqn1} holds we must now
          show that there is $u^* \in \ipow{\cn}$ such that $(b \cdot \cn^{-1}
          \leq u^* \leq b \cdot \cn \land \psi(u^*,\vec y^*))$ also holds.
          Observe that, since $\vec y^*$ satisfies $\exists u \psi$, each entry
          in $\vec y^*$ is an integer power of $\cn$, and therefore it suffices
          to show that $u^* \in \ipow{\cn}$ such that $b \cdot \cn^{-1} \leq u^*
          \leq b \cdot \cn$ and $u^* \in I_j$. This follows from the case
          analysis below:
          \begin{description}
            \item[case: $b \in I_j$ and $b \in \ipow{\cn}$.] In this case, $u^*
            = b$.
            \item[case: $b \not\in I_j$ and $b \in \ipow{\cn}$.] We have
            $\lambda(b) = b$. Since we are assuming that $I_j$ contains an
            integer power of $\cn$, we must have that $\cn^{-1} \cdot b$, which
            is the largest integer power of $\cn$ that is strictly below the
            endpoint $b$, belongs to $I_j$. Hence, we can take $u^* = \cn^{-1}
            \cdot b$.
            \item[case: $b \not\in \ipow{\cn}$.]
            We have $\lambda(b) < b$, and $\lambda(b)$ is the largest integer
            power of $\cn$ that is strictly below $b$ is $\lambda(b)$. We have
            $\lambda(b) \in I_j$ and $b \cdot \cn^{-1} \leq \lambda(b)$, and so
            we can take $u^* = \lambda(b)$.
          \end{description}
        \item[case: $b = {+}\infty$.] 
        In this case, instead of the right endpoint $b$ we consider the left
        endpoint $a$. Since $S$ if a strict subset of $R_{> 0}$, we must have $a
        > 0$. By the same arguments as in the previous case, $\phi$ must feature
        a polynomial inequality $r(u,\vec y) \sim 0$ such that $r(a,\vec y^*) =
        0$. We thus have $a > 0 \land r(a,\vec y^*) = 0 \land (\bigvee_{i=0}^n
        p_{r,i}(\cn,\vec y) \neq 0)$, and to conclude that
        Formula~\eqref{lemma:relativise-quantifiers:eqn1} holds it suffices to
        show that there is $u^* \in \ipow{\cn}$ such that $a \cdot \cn^{-1} \leq
        u^* \leq a \cdot \cn$ and $u^* \in I_j$. This is shown with a case
        analysis that is analogous to the one above:
        \begin{description}
          \item[case: $a \in I_j$ and $a \in \ipow{\cn}$.] In this case, $u^* =
          a$.
          \item[case: $a \not\in I_j$ and $a \in \ipow{\cn}$.] We have
          $\lambda(a) = a$. Since we are assuming that $I_j$ contains an integer
          power of $\cn$, we must have that $a \cdot \cn$, which is the largest
          integer power of $\cn$ that is strictly above the endpoint $a$,
          belongs to $I_j$. Hence, we can take $u^* = a \cdot \cn$.
          \item[case: $a \not\in \ipow{\cn}$.]
          We have $\lambda(a) < a$. In this case, the largest power of $\cn$
          that is strictly above the endpoint $a$ is $\lambda(a) \cdot \cn$. We
          have $\lambda(a) \cdot \cn \in I_j$ and $a < \lambda(a) \cdot \cn \leq
          a \cdot \cn$, and so we can take $u^* = \lambda(a) \cdot \cn$.
        \end{description}
      \end{description}
      In both the cases above, we have shown that $\vec y^*$ 
      is a solution to~Formula~\eqref{lemma:relativise-quantifiers:eqn1}.
    \end{claimproof}
  
    \begin{claim}\label{lemma:relativise-quantifiers:claim2}
      The following formula is a tautology of~$\R(\ipow{\cn})$:
      \[
        (\forall u (u > 0 \implies \psi)) 
        \implies 
        \exists w (w = 1 \land \exists u (w \cdot \cn^{-1} \leq u \leq w \cdot \cn \land \psi)).
      \]
    \end{claim}
    \begin{claimproof}
      First, observe that $\exists w (w = 1 \land \exists u (w \cdot \cn^{-1}
      \leq u \leq w \cdot \cn \land \psi))$ is trivially equivalent to $\exists
      u ( \cn^{-1} \leq u \leq \cn \land \psi)$. (The addition of the variable
      $w$ assigned to $1$ is convenient for the forthcoming arguments of the
      proof of~\Cref{lemma:relativise-quantifiers}.)
  
      Let $\vec y$ be real numbers satisfying the antecedent $(\forall u (u > 0
      \implies \psi))$ of the implication. Since $\cn > 1$, the non-empty
      interval $[\cn^{-1},\cn]$ is included in $\R_{>0}$. Therefore, from the
      antecedent of the implication we deduce that $\vec y$ satisfies $\exists u
      ( \cn^{-1} \leq u \leq \cn \land \psi)$.
    \end{claimproof}
    By Claims~\ref{lemma:relativise-quantifiers:claim1}
    and~\ref{lemma:relativise-quantifiers:claim2}, the following formula is a
    tautology of~$\R(\ipow{\cn})$: 
    \begin{align*}
      (\exists u\, \psi)
      \implies 
      \bigvee_{r \in Q}
      \exists w \Big(& w > 0 \land r(w,\vec y) = 0 \land (\bigvee_{i=0}^n p_{r,i}(\cn,\vec y) \neq 0)\\ 
      &{} \land \exists u (w \cdot \cn^{-1} \leq u \leq w \cdot \cn \land \psi) \Big).
    \end{align*}
    Since every $w > 0$ can be uniquely decomposed into $x\cdot v$, with
    $x$ being an integer power of $\cn$ and $1 \leq v < \cn$, the above formula
    can be rewritten as follows:
    \begin{align*}
      (\exists u\, \psi)
      \implies 
      \bigvee_{r \in Q}
      \exists x \exists v \Big(& \ipow{\cn}(x) \land 1 \leq v < \cn \land r(x \cdot v,\vec y) = 0 \land (\bigvee_{i=0}^n p_{r,i}(\cn,\vec y) \neq 0)\\ 
      &{} \land \exists u ( (x \cdot v) \cdot \cn^{-1} \leq u \leq (x \cdot v) \cdot \cn \land \psi) \Big).
    \end{align*}
    Hence, by applying~\Cref{lemma:last-lambda-lemma}, we conclude the following
    formula is a tautology of $\R(\ipow{\cn})$: 
    \begin{align}
      (\exists u\, \psi)
      \implies 
      \bigvee_{r \in Q}
      \exists x \exists v \Big(& \ipow{\cn}(x) \land 1 \leq v < \cn \land 
      \big(\bigvee_{(j,g,\vec y^{\vec \ell}) \in F_r}
        x^{j} = \cn^{g} \cdot \vec y^{\vec \ell}      
      \big)
      \notag\\ 
      &{} \land \exists u ( (x \cdot v) \cdot \cn^{-1} \leq u \leq (x \cdot v) \cdot \cn \land \psi) \Big).
      \label{lemma:relativise-quantifiers:eq1}
    \end{align}
  
    We now simplify the inequalities $(x \cdot v) \cdot \cn^{-1} \leq u \leq (x \cdot v) \cdot \cn$:
  
    \begin{claim}\label{lemma:relativise-quantifiers:claim3}
      The following formula is a tautology of~$\exists\R(\ipow{\cn})$:
      \[
        (\ipow{\cn}(u) \land \ipow{\cn}(x) \land 1 \leq v < \cn 
        \land (x \cdot v) \cdot \cn^{-1} \leq u \leq (x \cdot v) \cdot \cn) 
        \implies 
        \bigvee_{\ell = -1}^{1}
        u = \cn^{\ell} \cdot x.
      \]
    \end{claim}
    \begin{claimproof}
      Let $(u,v,x)$ be three real numbers satisfying the antecedent of the
      implication. By properties of $\lambda$, $(x \cdot v) \cdot \cn^{-1} \leq
      u \leq (x \cdot v) \cdot \cn$ implies $\lambda(x \cdot v) \cdot \cn^{-1}
      \leq \lambda(u) \leq \lambda(x \cdot v) \cdot \cn$. Since the antecedent
      of the implication imposes $u$ and $x$ to be integer powers of $\cn$, and
      $1 \leq v < \cn$, we have $\lambda(u) = u$ and $\lambda(x \cdot v) = x$.
      We conclude that $x \cdot \cn^{-1} \leq u \leq x \cdot \cn$, or
      equivalently $u = \cn^{\ell} \cdot x$ for some $\ell \in [-1,1]$, as
      required.
    \end{claimproof}
    We apply Claim~\ref{lemma:relativise-quantifiers:claim3} to
    Equation~\ref{lemma:relativise-quantifiers:eq1}, obtaining the following
    tautology of~$\R(\ipow{\cn})$:
    \begin{align*}
      (\exists u\, \psi)
      \implies\!
      \bigvee_{r \in Q}
      \exists x \Big(& \ipow{\cn}(x) \land 
      \big(\!\!\!\bigvee_{(j,g,\vec y^{\vec \ell}) \in F_r}\!\!\!
        x^{j} = \cn^{g} \cdot \vec y^{\vec \ell}      
      \big) \land \exists u ( \big(\!\bigvee_{\ell = -1}^{1}
      u = \cn^{\ell} \cdot x \big) \land \psi) \Big).
    \end{align*}
    Lastly, in the above formula, we can exponentiate both sides of $u =
    \cn^{\ell} \cdot x$ by $j$ and eliminate~$x$. That is, the following entailment 
    with formulae from~$\exists\R(\ipow{\cn})$ holds:
    \begin{align*}
      \exists u\, \psi
      \models 
      \bigvee_{\ell = -1}^{1} 
      \bigvee_{q \in Q}
      \bigvee_{(j,g,\vec y^{\vec \ell}) \in F_q}
      \exists u 
      \big(
        u^{j} = \cn^{j \cdot \ell + g} \cdot \vec y^{\vec \ell}      
        \land \psi
      \big).
      &\qedhere
    \end{align*}
  \end{proof}

\LemmaRemoveU*

\begin{proof} 
    We first prove the right-to-left direction of the lemma. Consider $\vec r\in
    R$ such that the sentence $\exists\vec z : \phi\sub{z_i^j \cdot
    \cn^{r_i}}{y_i : i \in [1..n]}\sub{\cn^{\frac{k + \vec \ell \cdot
    \vec{r}}{j}} \cdot \vec z^{\vec \ell}}{u}$ is a tautology of $\exists
    \ipow{\cn}$. The following sequence of implications (in the language of $\R(\ipow{\cn})$) establishes the
    right-to-left direction:
    \begin{align*}
        &\exists\vec z : \phi\sub{z_i^j \cdot \cn^{r_i}}{y_i : i \in [1..n]}\sub{\cn^{\frac{k + \vec \ell \cdot \vec r}{j}} \cdot \vec z^{\vec \ell}}{u}\\
        \implies{}& \exists \vec z \exists\vec y\exists u: \phi(u,\vec y)\land
        \big(\bigwedge_{i=1}^n y_i=z_i^j \cdot \cn^{r_i}\big) \land u=\cn^{\frac{k + \vec \ell \cdot \vec r}{j}} \cdot \vec z^{\vec \ell}
        &\hspace{-1cm}\text{def.~of~substitution}\\
        \implies{}&\exists \vec z \exists\vec y\exists u: \phi(u,\vec y)\land \big(\bigwedge_{i=1}^n y_i^{\ell_i}=z_i^{j\ell_i} \cdot \cn^{r_i\ell_i} \big)\land u^j=\cn^{k + \vec \ell \cdot \vec r} \cdot \vec z^{\vec \ell\cdot j}\\
        \implies{}&\exists \vec z \exists\vec y\exists u: \phi(u,\vec y)\land \big(\bigwedge_{i=1}^n y_i^{\ell_i}=z_i^{j\ell_i} \cdot \cn^{r_i\ell_i} \big)\land u^j=\cn^{k} \cdot \prod_{i=1}^n (z_i^{j\ell_i} \cdot \cn^{r_i\ell_i})\\
        \implies{}&\exists\vec y\exists u: \phi(u,\vec y)\land u^j=\cn^{k}\cdot y_1^{\ell_1}\cdot\cdots \cdot y_n^{\ell_n}.
    \end{align*}
    We move to the left-to-right direction. Suppose $\exists \vec y \exists u :
    u^{j} = \cn^{k} \cdot \vec y^{\vec \ell} \land \phi$ to be a tautology
    of~$\exists \ipow{\cn}$. For every $i \in [1..n]$, we have $y_i =
    \cn^{\alpha_i}$ for some $i \in \Z$. We consider the quotient $\beta_i \in
    \Z$ and remainder $r_i \in [0..j-1]$ of the integer division of $\alpha_i$
    modulo $j$, that is, $\alpha_i = \beta_i \cdot j + r_i$. Setting $z_i =
    \cn^{\beta_i}$, we have $y_i = z_i^j \cdot \cn^{r_i}$. Therefore, the
    following sentence is a tautology of~$\exists\ipow{\cn}$:
    \[
        \exists \vec y \exists u \exists \vec z \bigvee_{(r_1,\dots,r_n) \in [0..j-1]^n} u^{j} = \cn^{k} \cdot \vec y^{\vec \ell} \land \phi \land \bigwedge_{i=1}^n y_i = z_i^j \cdot \cn^{r_i}.
    \]
    By distributing existential quantifiers over disjunctions and eliminating
    $\vec y$ by performing the substitutions $\sub{z_i^j \cdot \cn^{r_i}}{y_i}$,
    we conclude that the following sentence is also tautological:
    \begin{equation}
        \label{lemma:remove-u:eq1}
        \bigvee_{(r_1,\dots,r_n) \in [0..j-1]} \exists u \exists \vec z : u^{j} = \cn^{k} \cdot \prod_{i=1}^n (z_i^{\ell_i j} \cdot \cn^{\ell_i r_i})  \land \phi\sub{z_i^j \cdot \cn^{r_i}}{y_i : i \in [1..n]}.
    \end{equation}
    Since all the $z_i$ and $u$ are powers of $\cn$, there are $\alpha,\beta \in
    \Z$ such that $\vec z^{\vec\ell\cdot j}=\cn^{\alpha \cdot j}$ and
    $u^j=\cn^{\beta \cdot j}$. Observe that then, in order for $u^{j} = \cn^{k}
    \cdot \prod_{i=1}^n z_i^{\ell_ij} \cdot \cn^{\ell_ir_i}$ to hold, we must
    have $\beta j = k + \alpha j + \vec \ell \cdot \vec r$. This implies that
    $j$ divides $k+ \vec \ell \cdot \vec r$. Therefore, we can update
    Formula~\eqref{lemma:remove-u:eq1} as follows: 
    \begin{itemize}
        \item instead of a disjunction over all elements in $[0..j-1]^n$,
        consider the set 
        \[
        R \coloneqq \big\{(r_1,\dots,r_n)\in[0..j-1]^n : j \text{ divides } k + \textstyle\sum_{i=1}^n r_i \cdot \ell_i\big\};
        \]
        \item in the disjunct corresponding to $\vec r \coloneqq (r_1,\dots,r_n)
        \in R$, replace $u^{j} = \cn^{k} \cdot \prod_{i=1}^n (z_i^{\ell_ij}
        \cdot \cn^{\ell_ir_i})$ with $u = \cn^{\frac{k+\vec \ell \cdot
        \vec{r}}{j}}\cdot \vec z^{\vec\ell}$.
    \end{itemize}
    We conclude that the following sentence is a tautology of $\exists
    \ipow{\cn}$: 
    \[
        \bigvee_{\vec r \coloneqq (r_1,\dots,r_n) \in R} \exists u \exists \vec z : u = \cn^{\frac{k+\vec \ell \cdot \vec r}{j}}\cdot \vec z^{\vec\ell} \land \phi\sub{z_i^j \cdot \cn^{r_i}}{y_i : i \in [1..n]}.
    \]
    From the sentence above, we eliminate $u$ from each disjunct corresponding
    to $\vec r \in R$ by performing the substitution
    $\sub{\cn^{\frac{k+\vec{\ell} \cdot \vec r}{j}}\cdot \vec z^{\vec\ell}}{u}$.
    In doing so, we obtain the formula in the statement of the lemma.
\end{proof}

\TheoremSmallModelProperty*


\begin{proof}
  The proposition is clearly true for $n = 0$, hence below we assume $n \geq 1$.
  By repeatedly applying~\Cref{lemma:relativise-quantifiers,lemma:remove-u} we
  conclude that there is a sequence $S_0,\dots,S_{n-1}$ of finite sets of
  integers, and a sequence $\phi_0, \phi_1, \dots, \phi_n$ of
  \emph{equisatisfiable} quantifier-free formulae such that 
  \begin{enumerate}[A.]
    \item\label{pp6:itemA} for every $r \in [0..n]$, the variables occurring in $\phi_r$ are
    among $u_{r+1},\dots,u_{n}$, 
    \item\label{pp6:itemB} $\phi_0 = \psi$, and 
    \item\label{pp6:itemC} for all $r \in [0..n-1]$, $\phi_{r+1} = \phi_r\sub{u_i^{j_r} \cdot
          \cn^{f_{r,i}}}{u_i : i \in [r+2..n]}\sub{\cn^{g_r} \cdot
          u_{r+2}^{\ell_{r,r+2}} \cdot \ldots \cdot
          u_{n}^{\ell_{r,n}}}{u_{r+1}}$, for some integers
          $j_r,f_{r,i},g_r,\ell_{r,r+2},\dots,\ell_{r,n}$ taken from the
          set $S_r$.
  \end{enumerate}
  Above, observe that for convenience and differently from~\Cref{lemma:remove-u}
  we are reusing the variables $u_2,\dots,u_n$ instead of introducing fresh
  variables $\vec z$. Without loss of generality, we assume $S_r$ to always
  contain $0$ and $1$. In this way, if $u_{r+1}$ does not occur in $\phi_r$
  (e.g., because it has been ``accidentally'' eliminated together with a
  previous variable), then we can pick $j_r = 1$ and $f_{r,i} = 0$, for every $i
  \in [r+2..n]$, in order to obtain $\phi_{r+1} = \phi_r$.

  From~\Cref{lemma:last-lambda-lemma,lemma:relativise-quantifiers}, for every $r
  \in [0..n-1]$, we have:
  \begin{enumerate}
    \item\label{pp6:itemOne} If $\cn$ is a computable transcendental number,
    there is an algorithm computing $S_r$ from~$\psi_r$.
    \item\label{pp6:itemTwo} If $\cn$ has a root barrier $\sigma(d,h) \coloneqq
    c \cdot (d+\ceil{\ln(h)})^k$, for some $c,k \in \N_{\geq 1}$, then,
    \begin{center}
      $\begin{aligned}
      j_r &\in [1..\deg(u_{r+1},\phi_r)],\\
      f_{r,i} &\in [0..j_r-1] \ \text{ and } \ \abs{\ell_{r,i}} \leq \deg(u_i,\phi_r), \qquad\text{for every } i \in [r+2..n],\\
      \abs{g_r} &\leq \deg(u_{r+1},\phi_r) \cdot ((2^{4c} D_r \cdot \ceil{\ln(H_r)})^{6 M_r k^{3M_r}}\!\!+n \cdot \max\{\deg(u_i,\phi_r) : i \in [r+2..n]\}).
      \end{aligned}$
    \end{center}
    where $H_r \coloneqq \max\{8,\height(\phi_r)\}$, $D_r \coloneqq
    \deg(\cn,\phi_r)+2$, and $M_r$ is the maximum number of monomials occurring
    in a polynomial of $\phi_r$. Here, $\deg(u_i,\phi_r)$
    (resp.~$\deg(\cn,\phi_r)$) stands for the maximum degree that the variable
    $u_i$ (resp.~$\cn$) has in a polynomial occurring in $\phi_r$, \emph{which in this
    proof we always assume to be at least $1$} without loss of generality.
  \end{enumerate}

  As explained in~\Cref{subsection:quantifier-relativisation}, we can
  ``backpropagate'' the substitutions performed to define the formulae
  $\phi_1,\dots,\phi_n$ (\Cref{pp6:itemC}) in order to obtain a solution for
  $\psi$. Formally, we consider the set of integers $\{d_{i,h} : i \in [1..n], h
  \in [0..i-1]\}$ given by the following recursive definition: 
  \begin{align}
    d_{i,i-1} &\coloneqq g_{n-i} + \sum_{h=0}^{i-2} (d_{i-1,h} \cdot \ell_{n-i,n-h}), \notag\\
    d_{i,h} &\coloneqq d_{i-1,h} \cdot j_{n-i} + f_{n-i,n-h},
    &\text{ for every } h \in [0..i-2].
    \label{eq:def-dih}
  \end{align}
  Observe that $d_{1,0} = g_{n-1}$, and that all integers $d_{i,h}$ are
  ultimately defined in terms of integers from the sets $S_0,\dots,S_{n-1}$. We
  prove the following claim:

  \begin{claim}\label{claim:quantifier-relativisation}
    Suppose $\psi$ to be satisfiable. Then,
    for every $i \in [0..n]$, the assignment 
    \[ 
      \begin{cases}
        u_{n-h} &= \cn^{d_{i,h}} \qquad\qquad\text{for every } h \in [0..i-1]
      \end{cases}
    \]
    is a solution of $\phi_{n-i}$.
  \end{claim}

  \begin{claimproof}
    The proof is by induction on $i$. 

    \begin{description}
      \item[base case: $i = 0$.]
      By~\Cref{pp6:itemC}, the formula $\phi_n$ does not feature any variable,
      and, accordingly, the assignment in the claim is empty. Since $\phi_n$ is
      equisatisfiable with $\phi_0$, and $\phi_0 = \psi$ by~\Cref{pp6:itemC}, we
      conclude that $\phi_n$ is equivalent to $\top$.
      
      \item[induction step: $i \geq 1$.] 
      By induction hypothesis, the assignment 
      \[ 
        \begin{cases}
          u_{n-h} &= \cn^{d_{i-1,h}} \qquad\qquad\text{for every } h \in [0..i-2]
        \end{cases}
      \]
      is a solution of $\phi_{n-(i-1)}$. By~\Cref{pp6:itemC}, we have 
      \[ 
        \phi_{n-(i-1)} = \phi_{n-i}\sub{u_t^{j_{n-i}} \cdot
          \cn^{f_{{n-i},t}}}{u_t : t \in [{n-i}+2..n]}\sub{\cn^{g_{n-i}} \cdot
          u_{{n-i}+2}^{\ell_{{n-i},{n-i}+2}} \cdot \ldots \cdot
          u_{n}^{\ell_{{n-i},n}}}{u_{{n-i}+1}}.
      \]
      Therefore, the following assignment is a solution of $\phi_{n-i}$:
      \[ 
        \begin{cases}
          u_{n-(i-1)} &= \cn^{g_{n-i}} \cdot (\cn^{d_{i-1,i-2}})^{\ell_{{n-i},{n-i}+2}} \cdot \ldots \cdot (\cn^{d_{i-1,0}})^{\ell_{{n-i},n}}\\
          u_{n-(i-2)} &= (\cn^{d_{i-1,i-2}})^{j_{n-i}} \cdot \cn^{f_{n-i,n-i+2}}\\ 
          u_{n-(i-3)} &= (\cn^{d_{i-1,i-3}})^{j_{n-i}} \cdot \cn^{f_{n-i,n-i+3}}\\ 
          \vdots\\
          u_n &= (\cn^{d_{i-1,0}})^{j_{n-i}} \cdot \cn^{f_{n-i,n}}
        \end{cases}
      \]
      that is,
      \[ 
        \begin{cases}
          u_{n-(i-1)} &= \cn^{g_{n-i}+\Sigma_{h=0}^{i-2} (d_{i-1,h} \cdot \ell_{n-i,n-h})}\\
          u_{n-h} &= \cn^{d_{i-1,h} \cdot j_{n-i}+f_{n-i,n-h}} \qquad\qquad\text{for every } h \in [0..i-2]
        \end{cases}
      \]
      and the statement follows by definition of $d_{i,i-1},\dots,d_{i,0}$.
      \claimqedhere%
    \end{description}
  \end{claimproof}

  Let us move back to the proof of~\Cref{theorem:small-model-property}. Given
  the finite sets $S_0,\dots,S_{n-1}$, we can compute an upper bound $U \in \N$
  to the absolute value of the largest integer among $d_{n,0},\dots,d_{n,n-1}$.
  Let~$P_{\psi} \coloneqq [-U..U]$. By~\Cref{claim:quantifier-relativisation} and we
  conclude that, whenever satisfiable, the formula~$\psi$ has a solution in the set
  $\{(\cn^{j_1},\dots,\cn^{j_n}) : j_1, \dots, j_n \in P_{\psi} \}$. Now, thanks to
  \Cref{pp6:itemOne,pp6:itemTwo} above, the finite sets $S_0,\dots,S_{n-1}$ can
  be computed in both the cases where either $\cn$ is a computable
  transcendental number or $\cn$ is a number with a polynomial root barrier. The
  set~$P_{\psi}$ can thus be computed in both these cases, which implies the
  effectiveness of the procedure required
  by~\Cref{theorem:small-model-property}.

  To conclude the proof, we derive an upper bound on $U$ in the case where $\cn$
  is a number with a polynomial root barrier $\sigma(d,h) \coloneqq c \cdot (d +
  \ceil{\ln(h)})^k$ for some $c,k \in \N_{\geq 1}$. We start by expressing, for
  every $r \in [0..n-1]$, the bounds from~\Cref{pp6:itemTwo} in terms of
  parameters of $\phi_0$. Below, let $E_r \coloneqq \max\{\deg(u_i,\phi_r) : i
  \in [r+1..n]\}$.
  \begin{claim}\label{claim:ugly-bounds}
    For every $r \in [0..n-1]$, we have
    \begin{align*}
      M_r &\leq M_0,\\ 
      H_r &\leq 2^r \cdot H_0,\\
      E_r &\leq 4^{2^r-1} \cdot (E_0)^{2^r}, \text{ and }\\
      \deg(\cn,\phi_r) &\leq (G_r)^{I^r-1} \cdot \deg(\cn,\phi_0)^{I^r},
    \end{align*}
    where $G_r \coloneqq n \cdot 2^{6 \cdot 2^r}(E_0)^{3 \cdot 2^r} \big(2^{r+4c+2} \ceil{\ln(H_0)}\big)^{I}$ and $I \coloneqq 6M_0k^{3M_0}$.
  \end{claim}
  \begin{claimproof}
    For $r = 0$ the claim is trivially true. Below, let us assume the claim to
    be true for $r \in [0..n-2]$, and show that it then also holds for $r+1$.
    Recall that, by \Cref{pp6:itemC}, we have 
    \begin{equation}
      \label{eq:phirp1phir}
      \phi_{r+1} = \phi_r\sub{u_i^{j_r} \cdot
      \cn^{f_{r,i}}}{u_i : i \in [r+2..n]}\sub{\cn^{g_r} \cdot
      u_{r+2}^{\ell_{r,r+2}} \cdot \ldots \cdot
      u_{n}^{\ell_{r,n}}}{u_{r+1}}.
    \end{equation}
    Recall that the integers $\ell_{r,i}$ and $g_r$ might be negative, and thus
    the substitutions performed in~\Cref{eq:phirp1phir} may require to update
    the polynomials in the formula so that they do not contain negative degrees
    for $\cn$ and each $u_i$. As described
    in~\Cref{subsection:quantifier-elimination}, these updates do not change the
    number of monomials nor the height of the polynomials, but might double the
    degree of each variable and of $\cn$. Hence, of the four bounds in the
    statement, which we now consider separately, these updates only affects the
    cases of $E_{r+1}$ and $\deg(\cn,\phi_{r+1})$.
    \begin{description}
      \item[case: $M_{r+1}$.] 
        The substitutions done to obtain $\phi_{r+1}$ from $\phi_r$ replace
        variables with monomials. These type of substitutions do not increase
        the number of monomials occurring in the polynomials of a formula. (They
        may however decrease, causing an increase in the height of the
        polynomials, see below.) Therefore, we have $M_{r+1} \leq M_r \leq M_0$.
      \item[case: $H_{r+1}$.] 
        The substitutions $\sub{u_i^{j_r} \cdot \cn^{f_{r,i}}}{u_i : i \in
        [r+2..n]}$ do not increase the heights of the polynomials in the
        formula. Indeed, consider a polynomial of the form 
        \begin{equation}
          \label{eq:a-polynomial}
          p(\cn,\vec u) + a \cdot \cn^{e_1} \cdot \vec u^{\vec d_1} 
          + b \cdot \cn^{e_2} \cdot \vec u^{\vec d_2},
        \end{equation}
        where $\vec u = (u_{r+1},\dots,u_{n})$, and $\cn^{e_1} \cdot
        \vec{u}^{\vec{d}_1}$ and $\cn^{e_2} \cdot \vec{u}^{\vec{d}_2}$ are two
        syntactically distinct monomials (i.e., either $e_1 \neq e_2$ or
        $\vec{d}_1 \neq \vec d_2$). Given $i \in [r+2..n]$, consider the
        substitution $\sub{u_i^{j_r} \cdot \cn^{f_{r,i}}}{u_i}$. We have three
        cases: 
        \begin{itemize}
          \item If $u_i$ occurs with different powers in the two monomials
          $\cn^{e_1} \cdot \vec{u}^{\vec{d}_1}$ and $\cn^{e_2} \cdot
          \vec{u}^{\vec{d}_2}$, then it will still occur with different powers
          in the monomials $(\cn^{e_1} \cdot \vec{u}^{\vec d_1})\sub{u_i^{j_r}
          \cdot \cn^{f_{r,i}}}{u_i}$ and $(\cn^{e_2} \cdot
          \vec{u}^{\vec{d}_2})\sub{u_i^{j_r} \cdot \cn^{f_{r,i}}}{u_i}$ obtained
          after replacement.
          \item If $u_i$ occurs with the same power $\hat{d}$ in the two
          monomials, and $e_1 \neq e_2$, then, after replacement, $\cn$ occurs
          with different powers in the obtained monomials $e_1 + \hat{d} \cdot
          f_{r,i}$ and $e_2 + \hat{d} \cdot f_{r,i}$ respectively.
          \item If $u_i$ occurs with the same power in the two monomials, and
          $e_1 = e_2$, then there is a variable $u_{t}$ with $t \neq i$ that, in
          the two monomials, occurs with different powers, say $\hat{d}_1$ and
          $\hat{d}_2$. (Note: one among $\hat{d}_1$ or $\hat{d}_2$ may be $0$.)
          This variable is unchanged by the substitution $\sub{u_i^{j_r} \cdot
          \cn^{f_{r,i}}}{u_i}$, and thus in the resulting monomials $u_{t}$
          still occurs with powers $\hat{d}_1$ and $\hat{d}_2$.
        \end{itemize}
        We move to the substitution $\sub{\cn^{g_r} \cdot u_{r+2}^{\ell_{r,r+2}}
        \cdot \ldots \cdot u_{n}^{\ell_{r,n}}}{u_{r+1}}$, which may increase the
        height of polynomials. Consider again a polynomial as
        in~\Cref{eq:a-polynomial}. Observe that if $u_{r+1}$ occurs with a
        non-zero power in both the monomials $\cn^{e_1} \cdot \vec u^{\vec d_1}$
        and $\cn^{e_2} \cdot \vec u^{\vec d_2}$, then the two monomials
        $(\cn^{e_1} \cdot \vec u^{\vec d_1})\sub{\cn^{g_r} \cdot
        u_{r+2}^{\ell_{r,r+2}} \cdot \ldots \cdot u_{n}^{\ell_{r,n}}}{u_{r+1}}$
        equals to $(\cn^{e_2} \cdot \vec u^{\vec d_2})\sub{\cn^{g_r} \cdot
        u_{r+2}^{\ell_{r,r+2}} \cdot \ldots \cdot u_{n}^{\ell_{r,n}}}{u_{r+1}}$
        obtained after replacement are still different (a formal proof of this
        fact follows similarly to the one we have just discussed for the
        substitution $\sub{u_i^{j_r} \cdot \cn^{f_{r,i}}}{u_i}$). The same
        holds true if $u_{r+1}$ does not occur in any of the two monomials. If
        instead $u_{r+1}$ occurs with a non-zero power only in one monomial, say
        $\cn^{e_1} \cdot \vec u^{\vec d_1}$, we might have 
        \[ 
          (\cn^{e_2} \cdot \vec u^{\vec d_2})\sub{\cn^{g_r} \cdot
          u_{r+2}^{\ell_{r,r+2}} \cdot \ldots \cdot
          u_{n}^{\ell_{r,n}}}{u_{r+1}}
          =
          (\cn^{e_1} \cdot \vec u^{\vec d_1})\sub{\cn^{g_r} \cdot
          u_{r+2}^{\ell_{r,r+2}} \cdot \ldots \cdot
          u_{n}^{\ell_{r,n}}}{u_{r+1}}
          = \cn^{e_1} \cdot \vec u^{\vec d_1}.
        \]
        Hence, after replacement, the coefficient of $\cn^{e_1} \cdot
        \vec{u}^{\vec{d}_1}$ is updated from $a$ to $(a+b)$. Note that no
        further increase are possible. Indeed, suppose $p(\cn,\vec u)$ contains
        a third monomial $\cn^{e_3}\vec u^{\vec d_3}$ in which $u_{r+1}$ has a
        non-zero power. By the arguments above, we have 
        \[
          (\cn^{e_2} \cdot \vec u^{\vec d_2})\sub{\cn^{g_r} \cdot
          u_{r+2}^{\ell_{r,r+2}} \cdot \ldots \cdot
          u_{n}^{\ell_{r,n}}}{u_{r+1}} 
          \neq 
          (\cn^{e_3} \cdot \vec u^{\vec d_3})\sub{\cn^{g_r} \cdot
          u_{r+2}^{\ell_{r,r+2}} \cdot \ldots \cdot
          u_{n}^{\ell_{r,n}}}{u_{r+1}},
        \]
        and therefore no other monomial from $p$ can be updated to $\cn^{e_1}
        \cdot \vec u^{\vec d_1}$ after replacement. Since $\abs{a},\abs{b} \leq
        H_r$, we have $\abs{a+b} \leq 2 \cdot H_r$. This shows $H_{r+1} \leq 2
        \cdot H_r \leq 2^{r+1} \cdot H_0$.
      \item[case: $E_{r+1}$.] Consider $u_i$ with $i \in [r+2..n]$. We show
        $\deg(u_i,\phi_{r+1}) \leq 4 (E_r)^2$, which implies $E_{r+1} \leq 4
        (4^{2^r-1} (E_0)^{2^r})^{2} = 4^{2^{r+1}-1}(E_0)^{2^{r+1}}$, as
        required. Consider a monomial occurring in $\phi_r$ and let $d_1$ and
        $d_2$ be the non-negative integers occurring as powers of $u_i$ and
        $u_{r+1}$ in this monomial. The substitutions performed to obtain
        $\phi_{r+1}$ (\Cref{eq:phirp1phir}) update the power of $u_i$ in the
        monomial from $d_1$ to $d_1 \cdot j_r + d_2 \cdot \ell_{r,i}$.
        By~\Cref{pp6:itemTwo}, $j_r \in [1..\deg(u_{r+1},\phi_r)]$ and
        $\abs{\ell_{r,i}} \leq \deg(u_i,\phi_r)$, and therefore
        $j_r,\abs{\ell_{r,i}} \leq E_r$. We conclude that $\abs{d_1 \cdot j_r +
        d_2 \cdot \ell_{r,i}} \leq 2 \cdot (E_r)^2$. Lastly, we need to account
        for the updates performed to the formula in order remove the negative
        integers that occur as powers of the variables and of $\cn$. As already
        stated, in the worst case, these updates double the degree of each
        variable, and so $E_{r+1} \leq 4 (E_r)^2$.
      \item[case: $\deg(\cn,\phi_{r+1})$.] 
        We start by reasoning similarly to the previous case. 
        Consider a monomial $\cn^{d} \cdot u_{r+1}^{d_{r+1}} \cdot \ldots \cdot u_{n}^{d_n}$
        occurring in $\phi_r$. 
        The substitutions performed to obtain $\phi_{r+1}$ 
        update the power of $\cn$ from $d$ to $d + g_r \cdot d_{r+1} + \sum_{i = r+2}^n f_{r,i} \cdot d_i$.
        Observe that 
        \begin{align*}
          & \abs{d + g_r \cdot d_{r+1} + \sum_{i = r+2}^n f_{r,i} \cdot d_i}\\
          \leq{}& 
            \deg(\cn,\phi_r) + \Big(\abs{g_r} + \sum_{i = r+2}^n \abs{f_{r,i}}\Big) \cdot E_r\\
          \leq{}& 
            \deg(\cn,\phi_r) + \Big(\abs{g_r} +\!\!\sum_{i = r+2}^n E_r\Big) \cdot E_r 
            &\text{by~\Cref{pp6:itemTwo}}.
        \end{align*}
        Accounting for the updates performed to the formula in order to remove negative powers, 
        we conclude that $\deg(\cn,\phi_{r+1})$ is bounded by 
        $2 \cdot (\deg(\cn,\phi_r) + (\abs{g_r} +\!\!\sum_{i = r+2}^n E_r) \cdot E_r)$.
        We further analyse this quantity as follows:
        \begin{align*}
          & 2 \cdot (\deg(\cn,\phi_r) + (\abs{g_r} +\!\!\sum_{i = r+2}^n E_r) \cdot E_r)\\
          \leq{}& 
            2\deg(\cn,\phi_r) + 
            2\Big(E_r ((2^{4c} D_r \ceil{\ln(H_r)})^{6 M_r k^{3M_r}}\!\!+n E_r)
            +\!\!\sum_{i = r+2}^n E_r\Big) E_r 
            &\hspace{-0.2cm}\text{by~\Cref{pp6:itemTwo}}\\
          \leq{}& 
            2\deg(\cn,\phi_r) + 
            4n \cdot (E_r)^3 \cdot (2^{4c} D_r \cdot \ceil{\ln(H_r)})^{6 M_r k^{3M_r}}
          \\
          \leq{}& 
            2\deg(\cn,\phi_r) + 
            4n \cdot (E_r)^3 \cdot (2^{4c} (\deg(\cn,\phi_r)+2) \cdot \ceil{\ln(H_r)})^{6 M_r k^{3M_r}}
          &\hspace{-0.2cm}\text{def.~of~$D_r$}
          \\
          \leq{}& 
            2 \deg(\cn,\phi_r) + 
            4n \cdot (E_r)^3 \cdot (2^{4c} (\deg(\cn,\phi_r)+2) \cdot \ceil{\ln(H_r)})^{6 M_r k^{3M_r}}
          &\hspace{-0.2cm}\text{def.~of~$D_r$}\\
          \leq{}& 
            2\deg(\cn,\phi_r) + 
            4n \cdot (E_r)^3 (2^{4c+2} \deg(\cn,\phi_r) \cdot \ceil{\ln(H_r)})^{6 M_r k^{3M_r}}
          &\hspace{-0.5cm}\text{$\deg(\cn,\phi_r) \geq 1$}\\
          \leq{}& 
            5n \cdot (E_r)^3 (2^{4c+2} \cdot \ceil{\ln(H_r)})^{6 M_r k^{3M_r}} \deg(\cn,\phi_r)^{6 M_r k^{3M_r}}
          \\
          \leq{}& 
            5n (4^{2^r-1}(E_0)^{2^r})^3 (2^{4c+2} \ceil{\ln(2^r H_0)})^{6 M_0 k^{3M_0}} \deg(\cn,\phi_r)^{6 M_0 k^{3M_0}}
          \\
          &&\hspace{-2.5cm}\text{bounds on $E_r$, $H_r$ and $M_r$}\\
          \leq{}& 
            n 2^{6 \cdot 2^r}(E_0)^{3 \cdot 2^r} (2^{r+4c+2} \ceil{\ln(H_0)})^{6 M_0 k^{3M_0}} \deg(\cn,\phi_r)^{6 M_0 k^{3M_0}}
          \\
          \leq{}& 
          G_r \cdot \deg(\cn,\phi_r)^{I}
          &\hspace{-1cm}\text{def.~of~$G_r$ and $I$}
          \\
          \leq{}& 
          G_r \cdot ((G_r)^{I^r-1} \cdot \deg(\cn,\phi_0)^{I^r})^{I}
          &\hspace{-1.6cm}\text{bound on $\deg(\cn,\phi_r)$}
          \\
          \leq{}& 
          (G_{r})^{I^{r+1}-I+1} \cdot \deg(\cn,\phi_0)^{I^{r+1}}
          \\
          \leq{}& 
          (G_{r+1})^{I^{r+1}-1} \cdot \deg(\cn,\phi_0)^{I^{r+1}}
          &\hspace{-4cm}\text{since $I \geq 2$ and $G_{r+1} \geq G_r$.}
        \end{align*}
        This completes the proof of the claim.
        \claimqedhere
    \end{description}
  \end{claimproof}
  We use the bounds in~\Cref{claim:ugly-bounds} to also bound the quantities
  $j_r$, $f_{r,i}$, $\abs{\ell_{r,i}}$ and $\abs{g_r}$.
  \begin{claim}
    \label{claim:less-ugly-bounds}
    For every $r \in [0..n-1]$ and $i \in [r+2..n]$, we have
    \begin{align*}
      j_r,f_{r,i},\abs{\ell_{r,i}} &\leq 4^{2^r} (E_0)^{2^r}, \text{ and }\\ 
      \abs{g_r} &\leq \Big(n \cdot 2^{2^{r+4}+4c}(E_0)^{2^{r+3}} \ceil{\ln(H_0)} \cdot \deg(\cn,\phi_0) \Big)^{(6M_0k^{3M_0})^{r+2}}.
    \end{align*}
  \end{claim}
  \begin{claimproof}
    By~\Cref{pp6:itemTwo}, 
    the numbers $j_r$, $f_{r,i}$ and $\abs{\ell_{r,i}}$ 
    are all bounded by $E_r$, 
    which in turn is bounded by $4^{2^r} (E_0)^{2^r}$ 
    (by~\Cref{claim:ugly-bounds}).
    Let us now consider $\abs{g_r}$.
    Observe that $\deg(\cn,\phi_r)$  and $\abs{g_r}$ are mutually dependant, 
    and in particular that in the proof of~\Cref{claim:ugly-bounds} 
    we have bounded $\deg(\cn,\phi_{r+1})$ with a long chain of manipulations establishing, among other inequalities, 
    \begin{align*}
        2 \cdot (\deg(\cn,\phi_r) + (\abs{g_r} +\!\!\sum_{i = r+2}^n E_r) \cdot E_r) \ \leq \ (G_{r+1})^{I^{r+1}-1} \cdot \deg(\cn,\phi_0)^{I^{r+1}},
    \end{align*}
    where $G_{r+1} \coloneqq n \cdot 2^{6 \cdot 2^{r+1}}(E_0)^{3 \cdot 2^{r+1}} \big(2^{r+4c+3} \ceil{\ln(H_0)}\big)^{I}$ and $I \coloneqq 6M_0k^{3M_0}$.
    Since $\abs{g_r}$ is smaller than $(\deg(\cn,\phi_r) + (\abs{g_r} +\!\!\sum_{i = r+2}^n E_r) \cdot E_r)$, we conclude that 
    \begin{align*}
      \abs{g_r} 
        &\leq (G_{r+1})^{I^{r+1}-1} \cdot \deg(\cn,\phi_0)^{I^{r+1}}\\
        &\leq \Big(n \cdot 2^{6 \cdot 2^{r+1}}(E_0)^{3 \cdot 2^{r+1}} \big(2^{r+4c+3} \ceil{\ln(H_0)}\big)^{I} \cdot \deg(\cn,\phi_0) \Big)^{I^{r+1}}\\
        &\leq \Big(n \cdot 2^{6 \cdot 2^{r+1}+r+4c+3}(E_0)^{3 \cdot 2^{r+1}} \ceil{\ln(H_0)} \cdot \deg(\cn,\phi_0) \Big)^{I^{r+2}}\\
        &\leq \Big(n \cdot 2^{2^{r+4}+4c}(E_0)^{2^{r+3}} \ceil{\ln(H_0)} \cdot \deg(\cn,\phi_0) \Big)^{(6M_0k^{3M_0})^{r+2}}.
        &\hspace{2.6cm}\text{\claimqedhere}
    \end{align*}
  \end{claimproof}
  Next, we bound the integers $d_{i,h}$, with $i \in [1..n]$ and $h \in [0..i-1]$, 
  introduced in~\Cref{eq:def-dih}.%
  \begin{claim}
    \label{claim:bounds-on-dih}
    For every $i \in [1..n]$ and $h \in [0..i-1]$ we have 
    \[ 
      \abs{d_{i,h}} \leq 2^{h} (2A)^{i-1}B,
    \]
    where $A \coloneqq 4^{2^n} (E_0)^{2^n}$ and $B \coloneqq \Big(n \cdot
    2^{2^{n+3}+4c}(E_0)^{2^{n+2}} \ceil{\ln(H_0)} \cdot \deg(\cn,\phi_0)
    \Big)^{(6M_0k^{3M_0})^{n+1}}$.
  \end{claim}
  \begin{claimproof}
    By~\Cref{claim:less-ugly-bounds}, for every $r \in [0..n-1]$ and $i \in
    [r+2..n]$, $j_r,f_{r,i},\abs{\ell_{r,i}} \leq A$ and $\abs{g_r} \leq
    B$. 
    
    Following the definition of $d_{i,h}$ given in~\Cref{eq:def-dih}, we have
    that for every $i \in [1..n]$ and $h \in [0..i-2]$, $\abs{d_{i,h}}$ is
    bounded by the positive integer $D_{i,h}$ that is recursively defined as
    follows. For every $i \in [1..n]$,
    \begin{align*}
      D_{i,i-1} &\coloneqq B + \sum_{h=0}^{i-2} D_{i-1,h} \cdot A, \notag\\
      D_{i,h} &\coloneqq (D_{i-1,h} + 1) \cdot A,
      &\text{ for every } h \in [0..i-2].
    \end{align*}
    Observe that $D_{1,0} = B$ and that, more generally, every $D_{i,h}$ is greater or equal to $B$.
    Since $B \geq 1$, for $h \neq i-1$ we have $D_{i,h} \leq 2 \cdot D_{i-1,h} \cdot A$. 
    To complete the proof, we show $D_{i,h} \leq 2^{h} (2A)^{i-1}B$ 
    by induction on $i$. 
    \begin{description}
      \item[base case: $i = 1$.] In this case we only need to consider $D_{1,0}$, which as already states is equal to $B$. The base case thus follows trivially.
      \item[induction step: $i \geq 2$.] Let $h \in [0..i-2]$. We consider two cases, depending on whether $h = i-1$. If $h \neq i-1$, then by definition of $D_{i,h}$ we have $D_{i,h} \leq 2 \cdot D_{i-1,h} \cdot A$. Then, from the induction hypothesis, 
        \begin{align*} 
          D_{i,h} \leq 2 \big( 2^{h} (2A)^{i-2} B\big)A 
          \leq 2^{h} (2A)^{i-1} B.
        \end{align*}
        If $h = i-1$, then by definition of $D_{i,h}$ we have  $D_{i,i-1} = B + \sum_{h=0}^{i-2} D_{i-1,h} \cdot A$. By applying the induction hypothesis, we obtain:
        \begin{align*} 
          D_{i,i-1} 
          &\leq B + \sum_{h=0}^{i-2} (2^{h} (2A)^{i-2} B) \cdot A \leq B + 2^{i-2}A^{i-1}B \cdot \sum_{h=0}^{i-2} 2^{h}\\ 
          &\leq B + 2^{i-2}A^{i-1}B \cdot 2^{i-1} 
          \leq 2^{i-1} (2A)^{i-1}B.  
          &\hspace{2.64cm}\text{\claimqedhere}
        \end{align*}
    \end{description}
  \end{claimproof}
  Together, \Cref{claim:quantifier-relativisation}
  and~\Cref{claim:bounds-on-dih} show that, whenever satisfiable,
  $\psi(u_1,\dots,u_n)$ (that is,~$\phi_0$) has a solution assigning to each
  variable an integer power of $\cn$ of the form $\cn^\beta$ with $\abs{\beta}
  \leq 2^{2n}A^nB$, where $A$ and $B$ are defined as
  in~\Cref{claim:bounds-on-dih}.
  We conclude the
  proof by simplifying this bound to improve its readability, obtaining the one
  in the statement. 
  Recall that $H \coloneqq \max\{8,\height(\psi)\}$,
  $D \coloneqq \deg(\psi)+2$  
  (where $\deg(\psi)$ also account for the degree of $\cn$),
  $E_0 = \max\{\deg(u_i,\psi) : i
  \in [1..n]\}$ 
  and $M_0$ is the number of monomials in a polynomial of $\psi$, 
  which can be crudely bounded as $D^{n+1}$ (the monomials also contain $\cn$).
  \begin{align*}
    \abs{\beta} 
      &\leq 2^{2n}A^nB\\
      &\leq 2^{2n}\big(4^{2^n} (E_0)^{2^n}\big)^n  \Big(n \cdot
      2^{2^{n+3}+4c}(E_0)^{2^{n+2}} \ceil{\ln(H_0)} \cdot \deg(\cn,\phi_0)
      \Big)^{(6M_0k^{3M_0})^{n+1}}\\
      &\leq 2^{2n}\big(4^{2^n} D^{2^n}\big)^n  \Big(n \cdot
      2^{2^{n+3}+4c}D^{2^{n+2}} \ceil{\ln(H)} \cdot D
      \Big)^{(6D^{n+1}k^{3D^{n+1}})^{n+1}}\\
      &\leq 2^{2n(2^n+1)} D^{n2^n}  \Big(
      2^{2^{n+3}+\log(n)+4c}D^{2^{n+2}+1} \ceil{\ln(H)}
      \Big)^{(6D^{n+1}k^{3D^{n+1}})^{n+1}}\\
      &\leq \Big(
      2^{4c}D^{2n(2^n+1) + n2^n + 2^{n+3} + \log(n) + 2^{n+2}+1} \ceil{\ln(H)}
      \Big)^{(6D^{n+1}k^{3D^{n+1}})^{n+1}}
      &\hspace{-0.3cm}\text{as } D \geq 2\\
      &\leq \Big(
      2^{4c}D^{18n2^n} \ceil{\ln(H)}
      \Big)^{(6D^{n+1}k^{3D^{n+1}})^{n+1}}
      &\hspace{-0.3cm}\text{as } n \geq 1\\
      &\leq \Big(
      2^{4c+18n2^n\log(D)} \ceil{\ln(H)}
      \Big)^{(6D^{n+1}k^{3D^{n+1}})^{n+1}}\\
      &\leq \Big(
      2^{c} \ceil{\ln(H)}
      \Big)^{72 n2^n\log(D) \cdot (6D^{n+1}k^{3D^{n+1}})^{n+1}},
  \end{align*}
  and the exponent in the last expression can be upper bounded as 
  \begin{align*}
      &72 n2^n\log(D) \cdot (6D^{n+1}k^{3D^{n+1}})^{n+1}\\
    \leq{}&2^{\log(72n)+n+\log(6)(n+1)}D^{n^2+2n+2}(k^{3D^{n+1}})^{n+1}\\
    \leq{}&2^{13n}D^{5n^2}(k^{3D^{n+1}})^{n+1} \leq D^{18n^2}k^{12nD^{4n}}
    \leq D^{2^5n^2}k^{D^{8n}}.
  \end{align*}
  Therefore, one can set $U \coloneqq (2^c \ceil{\ln(H)})^{D^{2^5 n^2} k^{D^{8n}}}$ 
  when defining $P_{\psi} \coloneqq [-U..U]$. 
\end{proof}