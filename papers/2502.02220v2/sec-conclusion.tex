\section{Conclusion and future directions}
\label{sec:conclusion}

With the goal of identifying unconditionally decidable fragments or variants of
$\R(e^x)$, we have studied the complexity of the theory~$\exists \R(\ipow{\cn})$
for different choices of~${\cn > 0}$. Particularly valuable turned out to be the
introduction of root barriers~(\Cref{definition:intro:root-barrier}): by relying
on this notion, we have established that~$\exists \R(\ipow{\cn})$ is in
\expspace if $\cn$ is algebraic, and in \threeexptime for natural choices of
$\cn$ among the transcendental numbers, such as $e$ and $\pi$.

A first natural question is how far are we from the exact complexity of these
existential theories, considering that the only known lower bound  is
inherited from the existential theory of the reals, which lies in
\pspace~\cite{Canny88}. While we have no answer to this question, we remark that
strengthening the hypotheses on $\cn$ may lead to better complexity bounds.
For example, we claim that our \expspace result for algebraic
numbers improves to \exptime when $\cn$ is an integer (we aim at
including this result in an extended version of this paper).

% Some of the ideas in this paper may help improving the quantifier elimination
% procedure developed in~\cite{AvigadY07} for the full first-order theory
% $\R(\ipow{2})$ (which requires non-elementary time to eliminate a single
% variable). However, while we believe that showing an elementary bound for a
% fixed number of quantifier alternations is now within reach, it seems to us that
% further ideas are required to place the whole theory in an elementary complexity
% class.

We have presented natural examples of bases $\cn$ having polynomial root
barriers. 
%(some generalisations of these examples can be found in~\cite[Table~3]{Waldschmidt78}). 
More exotic instances are known: setting $\cn
= q(\pi,\Gamma(\frac{1}{4}))$, where $q$ is an integer polynomial and $\Gamma$
is Euler's Gamma function, results in one such base. This follows from a theorem
by Bruiltet~\cite[Theorem B$^\prime$]{Bruiltet02} on the algebraic independence
of $\pi$ and $\Gamma(\frac{1}{4})$.
% \begin{theorem}
%   \label{thm:bruiltet}
%   Given a non-constant
%   integer polynomial $p(x,y)$, we have 
%   \[ 
%     \ln \abs{p(\pi,\Gamma({\textstyle{\frac{1}{4}}}))} > -10^{326}(\ln \height(p) + \deg(p) \cdot \ln(1+\deg(p))) \cdot \deg(p)^2 \cdot \ln(1+\deg(p))^2.
%   \]
% \end{theorem}
This leads to a second natural question: are there real numbers $a,b$ satisfying
$\ipow{a} \cap \ipow{b} = \{1\}$ for which the existential theory of the reals
enriched with both the predicates $\ipow{a}$ and $\ipow{b}$ is decidable? The
undecidability proof of the full FO theory proven in~\cite{Hieronymi10} relies
heavily on quantifier alternation. 
%Promising candidates are bases $a$ and~$b$
%that either range over the positive integers, or are algebraically independent.