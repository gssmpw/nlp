\section{Proofs of the statements in Section~\ref{sec:preliminaries}}
\label{appendix:preliminaries}

\LemmaTuringMachineProducts*

\begin{proof}
Let $\ell \coloneqq \ceil{\log(\abs{T_0}+\abs{T_0'}+3)}$.
We define $T''$ as the Turing machine that on input $n$ returns the 
rational number $T_{n+\ell} \cdot T_{n+\ell}'$. 
Clearly, $T''$ runs in time polynomial in $n$.
We show that $\abs{a \cdot b - T''} \leq \frac{1}{2^n}$ for every $n \in \N$, 
i.e., $T''$ computes $a \cdot b$. 
Let $\epsilon_1 \coloneqq T_{n+\ell}-a$ and $\epsilon_2 \coloneqq T_{n+\ell}'-b$. Recall that $\abs{\epsilon_1},\abs{\epsilon_2} \leq \frac{1}{2^{n+\ell}}$. 
Then, 
\begin{align*}
  \abs{a \cdot b - T''} &=
  \abs{a \cdot b - T_{n+\ell} \cdot T_{n+\ell}'} =
  \abs{a \cdot b - (a+\epsilon_1) \cdot (b+\epsilon_2)}\\
  &=\abs{a \cdot \epsilon_2 + b \cdot \epsilon_1 + \epsilon_1 \cdot \epsilon_2}\\
  &\leq \abs{a} \cdot \abs{\epsilon_2} + \abs{b} \cdot \abs{\epsilon_1} + \abs{\epsilon_1} \cdot \abs{\epsilon_2}\\
  &< \frac{\abs{a}+\abs{b}+1}{2^{n+\ell}}
  &\text{since} \abs{\epsilon_1},\abs{\epsilon_2} \leq \frac{1}{2^{n+\ell}}\\
  &= \frac{\abs{a}+\abs{b}+1}{2^{n+\ceil{\log(\abs{T_0}+\abs{T_0'}+3)}}}
  &\text{def.~of~$\ell$}\\
  &\leq \frac{1}{2^{n}}\frac{\abs{a}+\abs{b}+1}{\abs{T_0}+\abs{T_0'}+3}\\
  &\leq \frac{1}{2^{n}}\frac{\abs{a}+\abs{b}+1}{\abs{a}+\abs{b}+1}
  &\hspace{-2.4cm}\text{since $\abs{a} \leq \abs{T_0}+1$ and $\abs{b} \leq \abs{T_0'}+1$}\\
  &\leq \frac{1}{2^n}
  &&\qedhere
\end{align*}
\end{proof}

\LemmaTuringMachineReciprocal* 

\begin{proof}
  Compute the smallest $k \geq 2$ such that
  $\frac{1}{2^k} < \abs{T_k}$; its existence follows from the fact
  that $\lim_{n \to \infty} T_n = r \neq 0$, whereas $\lim_{n \to \infty}
  \frac{1}{2^n} = 0$. Since $\abs{r - T_k} \leq \frac{1}{2^k}$, we have 
  that and $T_k$ and $r$ have the same sign, and $0 < \abs{T_k}-\frac{1}{2^k} \leq \abs{r}$. 
  Let $T_k = \frac{p}{q}$, where $p \in \Z \setminus \{0\}$ and $q \geq 1$, 
  and define ${\ell \coloneqq 2(k + \ceil{\log(q)})}$.

  For the time being, let us give a construction of $T'$ that depends on the sign of $T_k$.

  \begin{description}
    \item[case: $T_k > 0$.] 
      We define $T'$ as the Turing machine that on input $n$ returns the
      rational $\frac{1}{\max(\abs{T_{n+\ell}},T_k - 2^{-k})}$. Clearly, if $T$
      runs in time polynomial in $n$, so does $T'$. We
      prove that $T'$ computes~$\frac{1}{r}$. First, observe that
      $\abs{r-T_{n+\ell}} \leq \frac{1}{2^{n+\ell}}$ and $r > 0$ imply
      $\abs{r-\abs{T_{n+\ell}}} \leq \frac{1}{2^{n+\ell}}$. Then, because $0 <
      T_k - 2^k \leq r$, we have $\abs{r-\max(\abs{T_{n+\ell}},T_{k}-2^{-k})}
      \leq \frac{1}{2^{n+\ell}}$.

      For
      every $n \in \N$,
      \begin{align*}
        \abs{\frac{1}{r} - T_n'}
        &= \abs{\frac{r - \max(\abs{T_{n+\ell}},T_{k}-2^{-k})}{r \cdot \max(\abs{T_{n+\ell}},T_{k}-2^{-k})}}\\
        &\leq \frac{1}{2^{n+\ell}} 
        \cdot \frac{1}{r \cdot \max(\abs{T_{n+\ell}},T_{k}-2^{-k})}\\
        &\leq \frac{1}{2^{n+\ell} \cdot (T_k-2^{-k})^2}
        &\text{since } 0 < T_k - 2^{-k} \leq r\\
        &\leq \frac{1}{2^{n+\ell+2\log(T_k - 2^{-k})}}
      \end{align*}
      To conclude the proof it suffices to show $\ell + 2\log(T_k-2^{-k}) \geq 0$:
      \begin{align*}
        & \ell + 2\log(T_k-2^{-k})\\
        ={}& \ell + 2\log((2^{k}T_k-1)2^{-k})
        = \ell + 2\log\Big(\Big(\frac{2^{k}p-q}{q}\Big)2^{-k}\Big)\\
        ={}& \ell + 2\log(2^{k}p-q) -2\log(q) -2k\\
        \geq{}& \ell -2\log(q) -2k 
        & \text{since $2^kp-q$ is an integer,}\\
        && \hspace{-1.8cm}\text{from $\frac{1}{2^k} < T_k$ we get $\log(2^{k}p-q) \geq 0$}\\
        ={}& 2(k+\ceil{\log(q)}-\log(q)-k)
        &\text{by def.~of~$\ell$}\\
        \geq{}& 0.
      \end{align*}
    \item[case: $T_k < 0$.] 
      Since $\abs{r}$ is computed by the machine that on input $n$ returns $\abs{T_n}$, 
      by following the previous case of the proof
      we conclude that $\frac{1}{\abs{r}}$ 
      is computed by the Turing machine that on input $n$
      returns the positive rational $\frac{1}{\max(\abs{T_{n+\ell}},\abs{T_k} - 2^{-k})}$.
      Then, the Turing machine that on input $n$
      returns the negative rational $\frac{-1}{\max(\abs{T_{n+\ell}},\abs{T_k} - 2^{-k})}$
      computes $\frac{1}{r}$.
  \end{description}
  Putting the two cases together we conclude that $\frac{1}{r}$ 
  is computed by the Turing machine 
  that on input $n$ returns the non-zero rational number $\frac{s}{\max(\abs{T_{n+\ell}},\abs{T_k} - 2^{-k})}$, 
  where $s = +1$ if $T_k > 0$, and otherwise $s = -1$.
\end{proof}


\section{Proofs of the statements in Section~\ref{sec:the-algorithm} (except for Proposition~\ref{theorem:small-model-property} which is proven in Appendix~\ref{appendix:solving-substructure}) and proof of Theorem~\ref{theorem:general-result-root-barrier}}
\label{appendix:useful-lemmas}

\begin{restatable}{lemma}{LemmaApproxUnivPolynomial}
  \label{lemma:approx-univ-polynomial}
  Let $p(x)$ be an integer polynomial, 
  and let ${r \in \R}$ with $\abs{r} \leq K$ for some $K \geq 1$.
  Consider $L,M \in \N$ satisfying 
  $M\geq L + \log(\height(p)+1) + 2 \deg(p) \cdot \log(K+1)$.
  For every $r^* \in \R$, if $\abs{r-r^*} \leq 2^{-M}$, 
  then $\abs{p(r)-p(r^*)} \leq 2^{-L}$.
\end{restatable}

\begin{proof} 
  Let $p(x) \coloneqq \sum_{j=0}^d a_i \cdot x^j$, and
  suppose $\abs{r-r^*} \leq 2^{-M}$.
  If $d = 0$, then $p$ is a constant polynomial and $\abs{p(r)-p(r^*)} = 0$, 
  which proves the lemma. 
  Below, we assume $d \geq 1$.

  To show that $\abs{p(r)-p(r^*)} \leq 2^{-L}$, let us start by bounding the maximum of the absolute value
  that the first derivative $p'(x) = \sum_{j=1}^d a_j \cdot j \cdot x^{j-1}$ of
  $p$ takes in the interval $I \coloneqq [-(K+1),K+1]$. For every $x \in \R$, $\abs{p'(x)} \leq
  g(x) \coloneqq \sum_{j=1}^d \abs{a_j \cdot j \cdot x^{j-1}}$. Since the
  function $g$ is monotonous over $\R_{\geq 0}$, and $g(y) = g(-y)$ for every $y
  \in \R$, we conclude that for every $x \in I$, $\abs{p'(x)} \leq g(K+1) =
  \sum_{j=1}^d \abs{a_j} \cdot j \cdot (K+1)^{j-1} \leq d^2 \height(p) (K+1)^{d-1}$. 
  
  From $\abs{r} \leq K$ and $\abs{r-r^*} \leq 2^{-M}$, where $M \geq 0$, 
  we have that both $r$ and $r^*$ belong to~$I$.
  This implies $\frac{\abs{p(r) -
  p(r^*)}}{\abs{r-r^*}} \leq \max\{\abs{p'(x)} : x \in I \} \leq
  d^2 \height(p) (K+1)^{d-1}$. So,
  \begin{align*}
    & \abs{p(r) - p(r^*)}\\
    \leq{}& d^2 \height(p) (K+1)^{d-1} \abs{r-r^*}\\
    \leq{}& 2^{2 \log(d)} 2^{\log(\height(p))} 2^{(d-1)\log(K+1)} 2^{-M}
    \\
    \leq{}& 2^{2 \log(d)+\log(\height(p))+(d-1)\log(K+1)-(L + \log(\height(p)+1) + 2 d \cdot \log(K+1))}
    & \text{bound on $M$}\\
    \leq{}& 2^{2 \log(d)-L-(d+1) \cdot \log(K+1)}\\
    \leq{}& 2^{-L}.
    &\hspace{-4.5cm}\text{since $2 \log(d) \leq d$ and $\log(K+1) \geq 1$}&\qedhere
  \end{align*}
\end{proof}

\LemmaSignPolyRootBarrier*

\begin{proof}
  Let $p(x) = \sum_{j=0}^d a_j \cdot x^j$ be input integer polynomial, having
  degree $d = \deg(p) \geq 1$ and height $h = \height(p)$. Recall that, from the
  definition of root barrier, whenever $p(\cn) \neq 0$ we have $\abs{p(\cn)}
  \geq e^{-\sigma(d,h)} > 2^{-2 \sigma(d,h)}$, where the last inequality
  follows from $\sigma(d,h) \geq 0$. Following
  line~\ref{algo:sign-evaluation:bound-on-n}, define $n \coloneqq 1 +
  2 \sigma(d,h) + 3d\ceil{\log(h+4)}$. 
  Note that $n \geq 5$.

  Let us first assume that $\abs{T_n} \geq h + 2$. In this case, the algorithm
  returns the sign of $p(T_n)$ (line~\ref{algo:sign-evaluation:large-pTn}). We
  show that $p(T_n)$ and $p(\cn)$ have the same sign. Since $\abs{\cn-T_n}
  \leq 2^{-1}$, we have $\abs{\cn} > h+1$.  
  By a result of Cauchy~\cite[Chapter 8]{Rahman02}, $h+1$ is an upper bound to
  the absolute value of every root of $p$. This implies that there are no root
  of $p$ in the interval $[\cn,T_n]$, so in particular $\cn$ and $T_n$ are
  not roots of $p$, and $p(\cn)$ and $p(T_n)$ have the same sign.
  
  Let us consider now the case $\abs{T_n} < h + 2$, and so $\abs{\cn} \leq K
  \coloneqq h + 3$. From the definition of $n$ and the fact that $\abs{\cn-T_n}
  \leq 2^{-n}$, by~\Cref{lemma:approx-univ-polynomial} we conclude that
  $\abs{p(\cn) - p(T_n)} \leq 2^{-2 \sigma(d,h)-1}$. This implies that if
  $\abs{p(T_n)} \leq 2^{-2 \sigma(d,h)-1}$ then $p(\cn) = 0$, and
  otherwise $p(T_n)$ and $p(\cn)$ have the same sign;  
  which concludes the proof of the lemma (see
  lines~\ref{algo:sign-evaluation:small-pTn}
  and~\ref{algo:sign-evaluation:large-pTn}). Indeed,  
  \begin{itemize}
    \item If $p(\cn) = 0$ then $\abs{p(T_n)} \leq 2^{-2 \sigma(d,h)-1}$
    (from $\abs{p(\cn) - p(T_n)} \leq 2^{-2 \sigma(d,h)-1}$).
    \item If $p(\cn) \neq 0$, then $\abs{p(T_n)} > 2^{-2 \sigma(d,h)-1}$
    :
    \begin{align*}
      \abs{p(T_n)} 
      & \geq \abs{p(\cn)} - \abs{p(\cn) - p(T_n)}
      & \text{from properties of the absolute value}\\
      & > 2^{-2 \sigma(d,h)} - 2^{-2 \sigma(d,h)-1}
      & \text{bounds on $\abs{p(\cn)}$ and $\abs{p(\cn) - p(T_n)}$}\\
      & = 2^{-2 \sigma(d,h)-1}.
    \end{align*}
    Moreover, $\abs{p(\cn) - p(T_n)} \leq 2^{-2 \sigma(d,h)-1}$ and
    $\abs{p(T_n)} > 2^{-2 \sigma(d,h)-1}$ imply that $p(\cn) > 0$ if and
    only if $p(T_n) > 0$.
    \qedhere
  \end{itemize}
\end{proof}

\LemmaRuntimeSignPolyRootBarrier* 

\begin{proof}
  When encoded in unary, the number $n$ defined in
  line~\ref{algo:sign-evaluation:bound-on-n} has size polynomial in the size of
  the input polynomial $p$. Then, to compute $T_n$ only requires polynomial time
  in $\size(p)$. Observe that this implies $T_n = \frac{q}{d}$ for some integers
  $q$ and $d$ encoded in binary using polynomially many bits with respect to $\size(p)$. Evaluating a
  polynomial at such a rational point can be done in polynomial time in the size of
  the polynomial and of the bit size of the rational. 
  This means that also lines~\ref{algo:sign-evaluation:small-pTn}
  and~\ref{algo:sign-evaluation:large-pTn} run in polynomial time in $\size(p)$.
\end{proof}

\LemmaCorrectnessAlgorithmOne* 

\begin{proof}
  Consider an input
  formula $\phi(x_1,\dots,x_n)$, and let $\phi'(u_1,\dots,u_n,v_1,\dots,v_n)$ be
  the formula obtained from it at the completion of the \textbf{for} loop of
  line~\Cref{algo:line1}. Note that if $\phi$ and $\phi'$ are equisatisfiable,
  then the lemma follows. Indeed, 
  \begin{itemize}
    \item By~\Cref{theorem:basu}, the formula $\psi(u_1,\dots,u_n)$ in
    line~\ref{algo:line6} is equisatisfiable with $\phi'$,
    \item By~\Cref{theorem:small-model-property}, $\psi$ is satisfiable if and
    only if it has a solution from the set $S = \{(\cn^{j_1},\dots,\cn^{j_n}) :
    j_1,\dots,j_n \in P\}$, where $P$ is the set from~\Cref{theorem:small-model-property}. 
    Lines~\ref{algo:line7} and~\ref{algo:line8}, 
    search for such an element of $S$.
    \item Following line~\ref{algo:line9}, the algorithm returns $\top$ if and only if $\psi$ evaluates to true
    on a point from the set $S$.
    For this evaluation step, one consider all polynomials inequalities $p(\cn,\cn^{g_1},\dots,\cn^{g_n}) \sim 0$ in $\psi(\cn^{g_1},\dots,\cn^{g_n})$, and evaluate its sign using the algorithm for $\SIGN_{\cn}$. As a result of this operation, $\psi(\cn^{g_1},\dots,\cn^{g_n})$ is updated into a Boolean combination of $\top$ and $\bot$, reduces to just $\top$ or $\bot$ after all Boolean connectives are evaluated.
  \end{itemize}
  So, to conclude the proof we just have to formally prove that $\phi$ and $\phi'$ are equisatisfiable.

  Recall that for every real number $r \in \R$ there is a pair of numbers $(u,v)$ 
  such that $x = u \cdot v$, $u \in \ipow{\cn}$ and either $v = 0$ or $1 \leq \abs{v} < \cn$.
  If $r \neq 0$, the pair $(u,v)$ is unique.
  Then, the formula $\phi$ is equisatisfiable with 
  \begin{equation}
    \label{eq:phi-for-line-1} 
    \phi\sub{u_i \cdot v_i}{x_i : i \in [1..n]} \land \bigwedge_{i=1}^n (\ipow{\cn}(u_i) \land (v_i = 0 \lor 1 \leq \abs{v} < \cn))
  \end{equation}
  where $u_1,\dots,u_n,v_1,\dots,v_n$ are fresh variables. 
  The formula $\phi\sub{u_i \cdot v_i}{x_i : i \in [1..n]}$ features atomic formulae $\ipow{\cn}(u_i \cdot v_i)$. Under the assumption that $\ipow{\cn}(u_i) \land (v_i = 0 \lor 1 \leq \abs{v} < \cn)$ holds, note that $\ipow{\cn}(u_i \cdot v_i)$ is equivalent to $v_i = 1$.
  Then, we can replace in the formula from~\Cref{eq:phi-for-line-1} every occurrence of $\ipow{\cn}(u_i \cdot v_i)$ with $v_i = 1$, preserving equivalence. 
  The formula we obtain is exactly the formula $\phi'$, which is thus equisatisfiable with $\phi$.
\end{proof}


\TheoremGeneralResultRootBarrier* 

\begin{proof}
  As discussed in~\Cref{subsection:small-bases}, it suffices to consider instances of 
  the problem where $\cn > 1$. We solve these instances with~\Cref{algo:main-procedure}, 
  which he have proven correct in~\Cref{lemma:correctness-algorithm-1}. 
  Below, we analyse the complexity of this algorithm, considering the three steps separately. 

  Consider an input formula $\phi(x_1,\dots,x_n)$ with $m_1$ occurrences of polynomial (in)equalities $g \sim 0$, all with $\deg(g) \leq d$ and $\height(g) \leq h$, and $m_2$ occurrences of the predicate $\ipow{\cn}$. 
  
  We run~\Cref{algo:main-procedure} on $\phi$:
  \begin{description}
    \item[Step~I (runtime: exponential in $\size(\phi)$).] 
      Lines~\ref{algo:line1}--\ref{algo:line5} update $\phi$ by (1) replacing the occurrences of $\ipow{\cn}(x_i)$ with $v_i = 1$, (2)~replacing the occurrences of $x_i$ with $u_i \cdot v_i$ and (3)~adding constraints $v_i = 0$ and $1 \leq \abs{v_i} < \cn$. Let $\phi'$ be the formula obtained after these updates. The size of $\phi'$ is polynomial in $\size(\phi)$. Moreover, $\phi'$ has:
      \begin{enumerate}
        \item at most $2n$ variables,
        \item at most $m_1+m_2+5n$ polynomial (in)equalities (recall that $1 \leq \abs{v_i} < \cn$ is a shortcut for the formula $-\cn < v_i \leq 1 \lor 1 \leq v_i < \cn$),
        \item and all its polynomials (in)equalities $g \sim 0$ are such that $\deg(g) \leq 2d$ and $\height(g) \leq h$. The increase in the degree is due to the replacements of variables $x_i$ with $u_i \cdot v_i$.
      \end{enumerate}
      The procedure then eliminates the variables $v_1,\dots,v_n$ by
      calling~\textsc{RealQE} (line~\ref{algo:line6}).
      Following~\Cref{theorem:basu}, the runtime of~\textsc{RealQE} is exponential
      in $\size(\phi)$, and therefore $\psi$ has size exponential in
      $\size(\phi)$. More precisely $\psi$ has 
      \begin{enumerate}
        \setcounter{enumi}{3}
        \item at most $n$ variables,
        \item\label{rtalgo:it5} at most $((m_1+m_2+5n) \cdot 2 \cdot d + 1)^{O(n^2)}$ polynomial (in)equalities,
        \item and all its (in)equalities $g \sim 0$ are s.t.~$\deg(g) \leq (2d)^{O(n)}$ and $\height(g) \leq (h+1)^{(2d)^{O(n^2)}}$.
      \end{enumerate}
    \item[Step~II (runtime: 2-exp.~or 3-exp.~in $\size(\phi)$, depending on the value of~$k$).]~\\
      For each variable $u_i$, the algorithm guesses an integer $g_i$ written in unary (lines~\ref{algo:line7} and~\ref{algo:line8}). Let $H \coloneqq \max(8,h(\psi))$ and $D \coloneqq \deg(\psi)+2$.
      By~\Cref{theorem:small-model-property}, 
      \begin{align*}
        \abs{g_i} 
        & \leq \left(2^c \ceil{\ln H}\right)^{D^{2^5 n^2} \cdot k^{D^{8n}}} 
        \leq \left(2^c \ceil{\ln \big((2(h+1))^{(2d)^{O(n^2)}}\big)}\right)^{(2d)^{O(n^3)} \cdot k^{(2d)^{O(n^2)}}},
      \end{align*}
      that is, if $k = 1$ then $\abs{g_i}$ is doubly exponential in $\size(\phi)$, and otherwise, for every $k \geq 2$, $\abs{g_i}$ is triply exponential in $\size(\psi)$.
      We can implement lines~\ref{algo:line7}--\ref{algo:line9} deterministically in the following na\"ive way: 
      \begin{algorithmic}[1]
        \setcounter{ALG@line}{6}
        \For{$(g_1,\dots,g_n) \in P^n$}
          \If{the assignment $(u_1 = \cn^{g_1}, \dots, u_n = \cn^{g_n})$ is a solution to $\psi$}
            \State \myreturn $\top$
          \EndIf
        \EndFor
        \State \myreturn $\bot$
      \end{algorithmic}
      Since each $g_i$ is stored in unary encoding, the number of iterations of
      the \textbf{for} loop above is either doubly or triply exponential in
      $\size(\phi)$, depending on whether $k = 1$.
      \item[Step~III (runtime: $2$-exp.~or $3$-exp.~in $\size(\phi)$, depending
      on the value of~$k$).]~\\
      The algorithm evaluates whether $(u_1 = \cn^{g_1}, \dots, u_n =
      \cn^{g_n})$ is a solution to $\psi$. As discussed in the body of the
      paper, $\psi(\cn^{g_1},\dots,\cn^{g_n})$ is a Boolean combination of
      polynomial (in)equalities $p(\cn) \sim 0$, where $\cn$ may occur with
      negative powers (as some $g_i$ may be negative). We rewrite each
      (in)equality $p(\cn) \sim 0$ as $\cn^{-d} \cdot p \sim 0$, where $d$ is
      the smallest negative integer occurring as a power of $\cn$ in~$p$ (or $0$
      if such an integer does not exists), thus obtaining a formula where all
      polynomials have non-negative degrees. Let us denote by $\psi'$ this
      formula. Note that this update takes polynomial time in the size of
      $\psi(\cn^{g_1},\dots,\cn^{g_n})$; that is doubly or triply exponential
      time in~$\size(\phi)$, depending on which case among $k = 1$ or $k \geq 2$
      we are considering.

      After this update, we determine the sign that each inequality in~$\psi'$.
      These inequalities are of the form $p(\cn) \sim 0$, and hence this problem
      can be solved with~\Cref{algo:sign-evaluation}. (Note that the degree $p$
      depends on $g_1,\dots,g_n$.)
      By~\Cref{lemma:runtime-sign-poly-root-barrier}, the runtime of this
      algorithm is polynomial in the size of $p$; which again is doubly or
      triply exponential in~$\size(\phi)$, depending on $k$. This enables us to
      simplify all inequalities to either $\top$ or $\bot$, to then return
      $\top$ or $\bot$ depending on the Boolean structure of~$\psi'$. Observe
      that $\psi$ and $\psi'$ have the same Boolean structure. Then, since
      $\psi$ has size exponential in $\size(\phi)$, evaluating the Boolean
      structure of $\psi'$ takes exponential time.
  \end{description}
  Putting all together, we conclude that~\Cref{algo:main-procedure} 
  runs in doubly exponential time if $k = 1$, 
  and in triply exponential time if $k \geq 2$.
\end{proof}
