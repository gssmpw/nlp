\begin{abstract}
  This paper investigates $\exists\R(\ipow{\cn})$, that is the extension of the existential theory of the reals by an additional unary
  predicate~${\cn}^{\Z}$ for the integer powers of a fixed computable real
  number~$\cn > 0$. If all we have access to is a Turing machine computing $\cn$,
  it is not possible to decide whether an input formula from this theory
  is satisfiable. However, we show an algorithm to decide this problem when 
  \begin{itemize}
    \item $\cn$ is known to be transcendental, or
    \item $\cn$ is a root of some given integer polynomial (that is, $\cn$ is
    algebraic).
  \end{itemize}
  In other words, knowing the algebraicity of $\cn$ suffices to circumvent
  undecidability. Furthermore, we establish complexity results under the proviso
  that~$\cn$ enjoys what we call a \emph{polynomial root barrier}. Using this
  notion, we show that the satisfiability problem of $\exists\R(\ipow{\cn})$ is
  \begin{itemize}
    \item in~\expspace
      if~$\cn$ is an algebraic number, and
    \item in~\threeexptime if~$\cn$ is a logarithm
    of an algebraic number, Euler's $e$, or the number~$\pi$, among~others.%
  \end{itemize}
  % In fact, as long as a polynomial root barrier for $\cn$ is known, our
  % complexity result holds regardless of whether $\cn$ is known to be algebraic
  % or not.

  To establish our results, we first observe that the satisfiability problem
  of~$\exists\R(\ipow{\cn})$ reduces in exponential time to the problem of
  solving quantifier-free instances of the theory of the reals where variables
  range over~${\cn}^{\Z}$. We then prove that these instances have a \emph{small
  witness property}: only finitely many integer powers of $\cn$ must be
  considered to find whether a formula is satisfiable. Our complexity results
  are shown by relying on well-established machinery from Diophantine
  approximation and transcendental number theory, such as bounds for the
  transcendence measure of numbers.
  
  As a by-product of our results, we are able to remove the appeal to Schanuel's
  conjecture from the proof of decidability of the entropic risk
  threshold problem for stochastic games with rational probabilities, rewards and
  threshold~[Baier et al., \textit{MFCS}, 2023]: when the base of the entropic risk is $e$
  and the aversion factor is a fixed algebraic number, the problem is
  (unconditionally) in~\exptime.%
\end{abstract}
