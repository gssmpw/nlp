
\typeout{IJCAI--25 Instructions for Authors}


\documentclass{article}
\pdfpagewidth=8.5in
\pdfpageheight=11in

\usepackage{ijcai25}

\usepackage{times}
\usepackage{soul}
\usepackage{url}
\usepackage[hidelinks]{hyperref}
\usepackage[utf8]{inputenc}
\usepackage[small]{caption} 
\usepackage{graphicx}
\usepackage{amsmath}
\usepackage{amsthm}
\usepackage{amssymb}
\usepackage{booktabs}
\usepackage{algorithm}
\usepackage{algorithmic}
\usepackage{multicol}
\usepackage{multirow}
\usepackage[switch]{lineno}
\usepackage{amsfonts}


\usepackage{subcaption}
\usepackage{makecell}
\usepackage{tcolorbox}
\usepackage{xcolor}

\newcommand\blfootnote[1]{%
	\begingroup
	\renewcommand\thefootnote{}\footnote{#1}%
	\addtocounter{footnote}{-1}%
	\endgroup
}

\urlstyle{same}


\newtheorem{example}{Example}
\newtheorem{theorem}{Theorem}



\usepackage{todonotes}
\newcommand{\note}[1]{\todo[inline,color=red!40]{#1}}

\usepackage{pifont}
\newcommand{\cmark}{\ding{51}}%
\newcommand{\xmark}{\ding{55}}%

\usepackage{tikz}
\newcommand*\emptycirc[1][1ex]{\tikz\draw (0,0) circle (#1);} 
\newcommand*\halfcirc[1][1ex]{%
  \begin{tikzpicture}
  \draw[fill] (0,0)-- (90:#1) arc (90:270:#1) -- cycle ;
  \draw (0,0) circle (#1);
  \end{tikzpicture}}
\newcommand*\fullcirc[1][1ex]{\tikz\fill (0,0) circle (#1);} 


\pdfinfo{
/TemplateVersion (IJCAI.2025.0)
}

\title{Vertical Federated Learning in Practice: The Good, the Bad, and the Ugly}


\author{
Zhaomin Wu$^1$\and
Zhen Qin$^3$\and
Junyi Hou$^1$\and
Haodong Zhao$^4$\and
Qinbin Li$^5$\and\\
Bingsheng He$^1$\and
Lixin Fan$^2$
\affiliations
$^1$National University of Singapore, Singapore,
$^2$WeBank, China,
$^3$Zhejiang University, China\\
$^4$Shanghai Jiao Tong University, China,
$^5$Huazhong University of Science and Technology, China
\emails
zhaomin@nus.edu.sg,
zhaohaodong@sjtu.edu.cn,
zhenqin@zju.edu.cn,
junyi.h@comp.nus.edu.sg,
qinbin@hust.edu.cn,
hebs@comp.nus.edu.sg,
lixinfan@webank.com
}












\begin{document}

\maketitle

\begin{abstract}
Vertical Federated Learning (VFL) is a privacy-preserving collaborative learning paradigm that enables multiple parties with distinct feature sets to jointly train machine learning models without sharing their raw data. Despite its potential to facilitate cross-organizational collaborations, the deployment of VFL systems in real-world applications remains limited. To investigate the gap between existing VFL research and practical deployment, this survey analyzes the real-world data distributions in potential VFL applications and identifies four key findings that highlight this gap. We propose a novel data-oriented taxonomy of VFL algorithms based on real VFL data distributions. Our comprehensive review of existing VFL algorithms reveals that some common practical VFL scenarios have few or no viable solutions. Based on these observations, we outline key research directions aimed at bridging the gap between current VFL research and real-world applications. 
\end{abstract}

\section{Introduction}\label{sec:introduction}

\section{Introduction}
\label{sec:introduction}
The business processes of organizations are experiencing ever-increasing complexity due to the large amount of data, high number of users, and high-tech devices involved \cite{martin2021pmopportunitieschallenges, beerepoot2023biggestbpmproblems}. This complexity may cause business processes to deviate from normal control flow due to unforeseen and disruptive anomalies \cite{adams2023proceddsriftdetection}. These control-flow anomalies manifest as unknown, skipped, and wrongly-ordered activities in the traces of event logs monitored from the execution of business processes \cite{ko2023adsystematicreview}. For the sake of clarity, let us consider an illustrative example of such anomalies. Figure \ref{FP_ANOMALIES} shows a so-called event log footprint, which captures the control flow relations of four activities of a hypothetical event log. In particular, this footprint captures the control-flow relations between activities \texttt{a}, \texttt{b}, \texttt{c} and \texttt{d}. These are the causal ($\rightarrow$) relation, concurrent ($\parallel$) relation, and other ($\#$) relations such as exclusivity or non-local dependency \cite{aalst2022pmhandbook}. In addition, on the right are six traces, of which five exhibit skipped, wrongly-ordered and unknown control-flow anomalies. For example, $\langle$\texttt{a b d}$\rangle$ has a skipped activity, which is \texttt{c}. Because of this skipped activity, the control-flow relation \texttt{b}$\,\#\,$\texttt{d} is violated, since \texttt{d} directly follows \texttt{b} in the anomalous trace.
\begin{figure}[!t]
\centering
\includegraphics[width=0.9\columnwidth]{images/FP_ANOMALIES.png}
\caption{An example event log footprint with six traces, of which five exhibit control-flow anomalies.}
\label{FP_ANOMALIES}
\end{figure}

\subsection{Control-flow anomaly detection}
Control-flow anomaly detection techniques aim to characterize the normal control flow from event logs and verify whether these deviations occur in new event logs \cite{ko2023adsystematicreview}. To develop control-flow anomaly detection techniques, \revision{process mining} has seen widespread adoption owing to process discovery and \revision{conformance checking}. On the one hand, process discovery is a set of algorithms that encode control-flow relations as a set of model elements and constraints according to a given modeling formalism \cite{aalst2022pmhandbook}; hereafter, we refer to the Petri net, a widespread modeling formalism. On the other hand, \revision{conformance checking} is an explainable set of algorithms that allows linking any deviations with the reference Petri net and providing the fitness measure, namely a measure of how much the Petri net fits the new event log \cite{aalst2022pmhandbook}. Many control-flow anomaly detection techniques based on \revision{conformance checking} (hereafter, \revision{conformance checking}-based techniques) use the fitness measure to determine whether an event log is anomalous \cite{bezerra2009pmad, bezerra2013adlogspais, myers2018icsadpm, pecchia2020applicationfailuresanalysispm}. 

The scientific literature also includes many \revision{conformance checking}-independent techniques for control-flow anomaly detection that combine specific types of trace encodings with machine/deep learning \cite{ko2023adsystematicreview, tavares2023pmtraceencoding}. Whereas these techniques are very effective, their explainability is challenging due to both the type of trace encoding employed and the machine/deep learning model used \cite{rawal2022trustworthyaiadvances,li2023explainablead}. Hence, in the following, we focus on the shortcomings of \revision{conformance checking}-based techniques to investigate whether it is possible to support the development of competitive control-flow anomaly detection techniques while maintaining the explainable nature of \revision{conformance checking}.
\begin{figure}[!t]
\centering
\includegraphics[width=\columnwidth]{images/HIGH_LEVEL_VIEW.png}
\caption{A high-level view of the proposed framework for combining \revision{process mining}-based feature extraction with dimensionality reduction for control-flow anomaly detection.}
\label{HIGH_LEVEL_VIEW}
\end{figure}

\subsection{Shortcomings of \revision{conformance checking}-based techniques}
Unfortunately, the detection effectiveness of \revision{conformance checking}-based techniques is affected by noisy data and low-quality Petri nets, which may be due to human errors in the modeling process or representational bias of process discovery algorithms \cite{bezerra2013adlogspais, pecchia2020applicationfailuresanalysispm, aalst2016pm}. Specifically, on the one hand, noisy data may introduce infrequent and deceptive control-flow relations that may result in inconsistent fitness measures, whereas, on the other hand, checking event logs against a low-quality Petri net could lead to an unreliable distribution of fitness measures. Nonetheless, such Petri nets can still be used as references to obtain insightful information for \revision{process mining}-based feature extraction, supporting the development of competitive and explainable \revision{conformance checking}-based techniques for control-flow anomaly detection despite the problems above. For example, a few works outline that token-based \revision{conformance checking} can be used for \revision{process mining}-based feature extraction to build tabular data and develop effective \revision{conformance checking}-based techniques for control-flow anomaly detection \cite{singh2022lapmsh, debenedictis2023dtadiiot}. However, to the best of our knowledge, the scientific literature lacks a structured proposal for \revision{process mining}-based feature extraction using the state-of-the-art \revision{conformance checking} variant, namely alignment-based \revision{conformance checking}.

\subsection{Contributions}
We propose a novel \revision{process mining}-based feature extraction approach with alignment-based \revision{conformance checking}. This variant aligns the deviating control flow with a reference Petri net; the resulting alignment can be inspected to extract additional statistics such as the number of times a given activity caused mismatches \cite{aalst2022pmhandbook}. We integrate this approach into a flexible and explainable framework for developing techniques for control-flow anomaly detection. The framework combines \revision{process mining}-based feature extraction and dimensionality reduction to handle high-dimensional feature sets, achieve detection effectiveness, and support explainability. Notably, in addition to our proposed \revision{process mining}-based feature extraction approach, the framework allows employing other approaches, enabling a fair comparison of multiple \revision{conformance checking}-based and \revision{conformance checking}-independent techniques for control-flow anomaly detection. Figure \ref{HIGH_LEVEL_VIEW} shows a high-level view of the framework. Business processes are monitored, and event logs obtained from the database of information systems. Subsequently, \revision{process mining}-based feature extraction is applied to these event logs and tabular data input to dimensionality reduction to identify control-flow anomalies. We apply several \revision{conformance checking}-based and \revision{conformance checking}-independent framework techniques to publicly available datasets, simulated data of a case study from railways, and real-world data of a case study from healthcare. We show that the framework techniques implementing our approach outperform the baseline \revision{conformance checking}-based techniques while maintaining the explainable nature of \revision{conformance checking}.

In summary, the contributions of this paper are as follows.
\begin{itemize}
    \item{
        A novel \revision{process mining}-based feature extraction approach to support the development of competitive and explainable \revision{conformance checking}-based techniques for control-flow anomaly detection.
    }
    \item{
        A flexible and explainable framework for developing techniques for control-flow anomaly detection using \revision{process mining}-based feature extraction and dimensionality reduction.
    }
    \item{
        Application to synthetic and real-world datasets of several \revision{conformance checking}-based and \revision{conformance checking}-independent framework techniques, evaluating their detection effectiveness and explainability.
    }
\end{itemize}

The rest of the paper is organized as follows.
\begin{itemize}
    \item Section \ref{sec:related_work} reviews the existing techniques for control-flow anomaly detection, categorizing them into \revision{conformance checking}-based and \revision{conformance checking}-independent techniques.
    \item Section \ref{sec:abccfe} provides the preliminaries of \revision{process mining} to establish the notation used throughout the paper, and delves into the details of the proposed \revision{process mining}-based feature extraction approach with alignment-based \revision{conformance checking}.
    \item Section \ref{sec:framework} describes the framework for developing \revision{conformance checking}-based and \revision{conformance checking}-independent techniques for control-flow anomaly detection that combine \revision{process mining}-based feature extraction and dimensionality reduction.
    \item Section \ref{sec:evaluation} presents the experiments conducted with multiple framework and baseline techniques using data from publicly available datasets and case studies.
    \item Section \ref{sec:conclusions} draws the conclusions and presents future work.
\end{itemize}


\section{Overview of Vertical Federated Learning}\label{sec:overview}
\section{Overview}

\revision{In this section, we first explain the foundational concept of Hausdorff distance-based penetration depth algorithms, which are essential for understanding our method (Sec.~\ref{sec:preliminary}).
We then provide a brief overview of our proposed RT-based penetration depth algorithm (Sec.~\ref{subsec:algo_overview}).}



\section{Preliminaries }
\label{sec:Preliminaries}

% Before we introduce our method, we first overview the important basics of 3D dynamic human modeling with Gaussian splatting. Then, we discuss the diffusion-based 3d generation techniques, and how they can be applied to human modeling.
% \ZY{I stopp here. TBC.}
% \subsection{Dynamic human modeling with Gaussian splatting}
\subsection{3D Gaussian Splatting}
3D Gaussian splatting~\cite{kerbl3Dgaussians} is an explicit scene representation that allows high-quality real-time rendering. The given scene is represented by a set of static 3D Gaussians, which are parameterized as follows: Gaussian center $x\in {\mathbb{R}^3}$, color $c\in {\mathbb{R}^3}$, opacity $\alpha\in {\mathbb{R}}$, spatial rotation in the form of quaternion $q\in {\mathbb{R}^4}$, and scaling factor $s\in {\mathbb{R}^3}$. Given these properties, the rendering process is represented as:
\begin{equation}
  I = Splatting(x, c, s, \alpha, q, r),
  \label{eq:splattingGA}
\end{equation}
where $I$ is the rendered image, $r$ is a set of query rays crossing the scene, and $Splatting(\cdot)$ is a differentiable rendering process. We refer readers to Kerbl et al.'s paper~\cite{kerbl3Dgaussians} for the details of Gaussian splatting. 



% \ZY{I would suggest move this part to the method part.}
% GaissianAvatar is a dynamic human generation model based on Gaussian splitting. Given a sequence of RGB images, this method utilizes fitted SMPLs and sampled points on its surface to obtain a pose-dependent feature map by a pose encoder. The pose-dependent features and a geometry feature are fed in a Gaussian decoder, which is employed to establish a functional mapping from the underlying geometry of the human form to diverse attributes of 3D Gaussians on the canonical surfaces. The parameter prediction process is articulated as follows:
% \begin{equation}
%   (\Delta x,c,s)=G_{\theta}(S+P),
%   \label{eq:gaussiandecoder}
% \end{equation}
%  where $G_{\theta}$ represents the Gaussian decoder, and $(S+P)$ is the multiplication of geometry feature S and pose feature P. Instead of optimizing all attributes of Gaussian, this decoder predicts 3D positional offset $\Delta{x} \in {\mathbb{R}^3}$, color $c\in\mathbb{R}^3$, and 3D scaling factor $ s\in\mathbb{R}^3$. To enhance geometry reconstruction accuracy, the opacity $\alpha$ and 3D rotation $q$ are set to fixed values of $1$ and $(1,0,0,0)$ respectively.
 
%  To render the canonical avatar in observation space, we seamlessly combine the Linear Blend Skinning function with the Gaussian Splatting~\cite{kerbl3Dgaussians} rendering process: 
% \begin{equation}
%   I_{\theta}=Splatting(x_o,Q,d),
%   \label{eq:splatting}
% \end{equation}
% \begin{equation}
%   x_o = T_{lbs}(x_c,p,w),
%   \label{eq:LBS}
% \end{equation}
% where $I_{\theta}$ represents the final rendered image, and the canonical Gaussian position $x_c$ is the sum of the initial position $x$ and the predicted offset $\Delta x$. The LBS function $T_{lbs}$ applies the SMPL skeleton pose $p$ and blending weights $w$ to deform $x_c$ into observation space as $x_o$. $Q$ denotes the remaining attributes of the Gaussians. With the rendering process, they can now reposition these canonical 3D Gaussians into the observation space.



\subsection{Score Distillation Sampling}
Score Distillation Sampling (SDS)~\cite{poole2022dreamfusion} builds a bridge between diffusion models and 3D representations. In SDS, the noised input is denoised in one time-step, and the difference between added noise and predicted noise is considered SDS loss, expressed as:

% \begin{equation}
%   \mathcal{L}_{SDS}(I_{\Phi}) \triangleq E_{t,\epsilon}[w(t)(\epsilon_{\phi}(z_t,y,t)-\epsilon)\frac{\partial I_{\Phi}}{\partial\Phi}],
%   \label{eq:SDSObserv}
% \end{equation}
\begin{equation}
    \mathcal{L}_{\text{SDS}}(I_{\Phi}) \triangleq \mathbb{E}_{t,\epsilon} \left[ w(t) \left( \epsilon_{\phi}(z_t, y, t) - \epsilon \right) \frac{\partial I_{\Phi}}{\partial \Phi} \right],
  \label{eq:SDSObservGA}
\end{equation}
where the input $I_{\Phi}$ represents a rendered image from a 3D representation, such as 3D Gaussians, with optimizable parameters $\Phi$. $\epsilon_{\phi}$ corresponds to the predicted noise of diffusion networks, which is produced by incorporating the noise image $z_t$ as input and conditioning it with a text or image $y$ at timestep $t$. The noise image $z_t$ is derived by introducing noise $\epsilon$ into $I_{\Phi}$ at timestep $t$. The loss is weighted by the diffusion scheduler $w(t)$. 
% \vspace{-3mm}

\subsection{Overview of the RTPD Algorithm}\label{subsec:algo_overview}
Fig.~\ref{fig:Overview} presents an overview of our RTPD algorithm.
It is grounded in the Hausdorff distance-based penetration depth calculation method (Sec.~\ref{sec:preliminary}).
%, similar to that of Tang et al.~\shortcite{SIG09HIST}.
The process consists of two primary phases: penetration surface extraction and Hausdorff distance calculation.
We leverage the RTX platform's capabilities to accelerate both of these steps.

\begin{figure*}[t]
    \centering
    \includegraphics[width=0.8\textwidth]{Image/overview.pdf}
    \caption{The overview of RT-based penetration depth calculation algorithm overview}
    \label{fig:Overview}
\end{figure*}

The penetration surface extraction phase focuses on identifying the overlapped region between two objects.
\revision{The penetration surface is defined as a set of polygons from one object, where at least one of its vertices lies within the other object. 
Note that in our work, we focus on triangles rather than general polygons, as they are processed most efficiently on the RTX platform.}
To facilitate this extraction, we introduce a ray-tracing-based \revision{Point-in-Polyhedron} test (RT-PIP), significantly accelerated through the use of RT cores (Sec.~\ref{sec:RT-PIP}).
This test capitalizes on the ray-surface intersection capabilities of the RTX platform.
%
Initially, a Geometry Acceleration Structure (GAS) is generated for each object, as required by the RTX platform.
The RT-PIP module takes the GAS of one object (e.g., $GAS_{A}$) and the point set of the other object (e.g., $P_{B}$).
It outputs a set of points (e.g., $P_{\partial B}$) representing the penetration region, indicating their location inside the opposing object.
Subsequently, a penetration surface (e.g., $\partial B$) is constructed using this point set (e.g., $P_{\partial B}$) (Sec.~\ref{subsec:surfaceGen}).
%
The generated penetration surfaces (e.g., $\partial A$ and $\partial B$) are then forwarded to the next step. 

The Hausdorff distance calculation phase utilizes the ray-surface intersection test of the RTX platform (Sec.~\ref{sec:RT-Hausdorff}) to compute the Hausdorff distance between two objects.
We introduce a novel Ray-Tracing-based Hausdorff DISTance algorithm, RT-HDIST.
It begins by generating GAS for the two penetration surfaces, $P_{\partial A}$ and $P_{\partial B}$, derived from the preceding step.
RT-HDIST processes the GAS of a penetration surface (e.g., $GAS_{\partial A}$) alongside the point set of the other penetration surface (e.g., $P_{\partial B}$) to compute the penetration depth between them.
The algorithm operates bidirectionally, considering both directions ($\partial A \to \partial B$ and $\partial B \to \partial A$).
The final penetration depth between the two objects, A and B, is determined by selecting the larger value from these two directional computations.

%In the Hausdorff distance calculation step, we compute the Hausdorff distance between given two objects using a ray-surface-intersection test. (Sec.~\ref{sec:RT-Hausdorff}) Initially, we construct the GAS for both $\partial A$ and $\partial B$ to utilize the RT-core effectively. The RT-based Hausdorff distance algorithms then determine the Hausdorff distance by processing the GAS of one object (e.g. $GAS_{\partial A}$) and set of the vertices of the other (e.g. $P_{\partial B}$). Following the Hausdorff distance definition (Eq.~\ref{equation:hausdorff_definition}), we compute the Hausdorff distance to both directions ($\partial A \to \partial B$) and ($\partial B \to \partial A$). As a result, the bigger one is the final Hausdorff distance, and also it is the penetration depth between input object $A$ and $B$.


%the proposed RT-based penetration depth calculation pipeline.
%Our proposed methods adopt Tang's Hausdorff-based penetration depth methods~\cite{SIG09HIST}. The pipeline is divided into the penetration surface extraction step and the Hausdorff distance calculation between the penetration surface steps. However, since Tang's approach is not suitable for the RT platform in detail, we modified and applied it with appropriate methods.

%The penetration surface extraction step is extracting overlapped surfaces on other objects. To utilize the RT core, we use the ray-intersection-based PIP(Point-In-Polygon) algorithms instead of collision detection between two objects which Tang et al.~\cite{SIG09HIST} used. (Sec.~\ref{sec:RT-PIP})
%RT core-based PIP test uses a ray-surface intersection test. For purpose this, we generate the GAS(Geometry Acceleration Structure) for each object. RT core-based PIP test takes the GAS of one object (e.g. $GAS_{A}$) and a set of vertex of another one (e.g. $P_{B}$). Then this computes the penetrated vertex set of another one (e.g. $P_{\partial B}$). To calculate the Hausdorff distance, these vertex sets change to objects constructed by penetrated surface (e.g. $\partial B$). Finally, the two generated overlapped surface objects $\partial A$ and $\partial B$ are used in the Hausdorff distance calculation step.


\section{VFL Data Distributions}\label{sec:scenario}
\begin{figure*}[t!]
    \centering
    \begin{subfigure}[t]{0.24\textwidth}
        \centering
        \includegraphics[width=\textwidth]{fig/overlap_graph_seed0.png}
        \caption{Visualization of 1,000 parties}
        \label{fig:overlap-graph}
    \end{subfigure}
    \hfill
    \begin{subfigure}[t]{0.24\textwidth}
        \centering
        \includegraphics[width=\textwidth]{fig/overlap_distribution.png}
        \caption{Feature overlap ratios}
        \label{fig:feature-overlap}
    \end{subfigure}
    \hfill
    \begin{subfigure}[t]{0.24\textwidth}
        \centering
        \includegraphics[width=\textwidth]{fig/feature_balance_ratio.png}
        \caption{Feature balance ratios}
        \label{fig:feature-skew}
    \end{subfigure}
    \hfill
    \begin{subfigure}[t]{0.24\textwidth}
        \centering
        \includegraphics[width=\textwidth]{fig/matched_ratio_hist.png}
        \caption{Record matched ratios}
        \label{fig:matched-ratio}
    \end{subfigure}    
    \caption{Analysis and visualization of feature distributions and overlaps across real-world databases}
    \label{fig:feature-analysis}
\end{figure*}


This section examines real-world VFL data distributions using the WikiDBs corpus \cite{vogel2024wikidbs}. We begin by introducing the basic settings in Section~\ref{subsec:settings}, define the essential data properties in Section~\ref{subsec:definitions}, and present our results and findings in Section~\ref{subsec:findings}. 

\subsection{Settings}\label{subsec:settings}
\paragraph{Dataset.}
WikiDBs~\cite{vogel2024wikidbs} is currently the largest publicly available corpus of relational databases, extracted from real-world Wikidata. The details of WikiDBs are shown in Table~\ref{tab:wikidbs}. It covers a wide range of domains, such as clinical, finance, sports, etc. Many of these databases (e.g. \texttt{ucl\_clinical\_research\_trials} and \texttt{apnea\_clinical\_research\_db}) are correlated and can be considered potential scenarios for VFL. Each database can be considered a VFL \textit{party}, with pairs of parties sharing correlated features representing potential VFL pairs. 

\begin{table}[h]
    \centering
    \small
    \caption{Statistics of WikiDBs}
    \label{tab:wikidbs}
    \begin{tabular}{cccc}
        \toprule
        \multicolumn{2}{c}{\textbf{Database}} & \multicolumn{2}{c}{\textbf{Table}} \\
        \cmidrule(lr){1-2} \cmidrule(lr){3-4}
        \#Databases & Total \#Tables & Mean \#Rows & Mean \#Cols \\
        \midrule
        100K & 1.6M & 118 & 52.7 \\
        \bottomrule
    \end{tabular}
\end{table}


\paragraph{Configuration.} In our analysis, we focus on two-party VFL, as it not only reflects the characteristics of multi-party VFL but also represents the most common VFL scenario in practice~\cite{liu2024vertical}. Due to the \(O(n^2)\) complexity of evaluating all pairs, we randomly sampled 1,000 databases, generating 1,000,000 pairs from their Cartesian square. Experiments are repeated across 10 random seeds to reduce variance, with mean and variance reported. The results show consistent subset characteristics, demonstrating the robustness of our analysis.

\subsection{Definitions}\label{subsec:definitions}

This subsection delineates the key properties of VFL data distribution. We begin by defining which pairs of databases are considered potential VFL pairs. At the feature level, we assess their overlap and balance by introducing the \textit{feature overlap ratio} and the \textit{feature balance ratio}. In terms of instances, we quantify the proportion of records that can be matched between two parties using shared features, defining this as the \textit{record matched ratio}. The detailed definitions are as follows.

\paragraph{Potential VFL Pairs.} Potential VFL pairs are defined based on database graph connectivity. In this graph, nodes represent databases, and edges connect nodes with tables sharing at least one column. Two databases are considered potential VFL pairs if they belong to the same connected component, indicating they are related through join operations along the connected path.

\paragraph{Record Matched Ratio.} To evaluate how precisely two VFL parties can be aligned based on shared features, we define the \textit{record matched ratio}. For a given table pair, this ratio measures the fraction of records in each table that identically appear in the other. It ranges from $[0, 1]$ and reflects the proportion of records that can be precisely aligned.


\paragraph{Feature Balance Ratio.} To assess the balance in the number of features between two VFL parties, we define the \textit{feature balance ratio} as the small number of columns between the two databases to the large number of columns. The feature balance ratio ranges from $[0, 1]$, where a value of 1 indicates two parties have the same number of features.


\begin{table}[ht]
    \centering
    \small
    \caption{Properties of the database graph across 10 subsets}
    \label{tab:vfl-properties}
    \setlength{\tabcolsep}{3pt}
    \begin{tabular}{ccccc}
        \toprule
        \textbf{Metric} & \textbf{Mean} & \textbf{Std} & \textbf{Min} & \textbf{Max} \\
        \midrule
        \#Connected Components & 3.1 & 0.3 & 3 & 4 \\
        \midrule
       \makecell{Ratio of Non-Neighboring Nodes \\ in Potential VFL Pairs} & 25.4\% & 1.3\% & 23.7\% & 27.7\% \\
        \bottomrule
    \end{tabular}
\end{table}


\begin{table}[ht]
    \centering
    \small
    \caption{The ratio of different VFL data distributions}
    \label{tab:record-matching}
    \setlength{\tabcolsep}{4pt}
    \begin{tabular}{cccc}
        \toprule
        \textbf{VFL Type} & \textbf{Features} & \textbf{Records} & \textbf{Ratio} \\
        \midrule
        Latent VFL & Zero overlap & Zero match & 25.4\% \\
        \midrule
        Fuzzy VFL & Non-zero overlap & Zero match & 70.9\% \\
        Semi-precise VFL & Non-zero overlap & Partial match & 3.5\% \\
        Precise VFL & Non-zero overlap & Full match & 0.2\% \\
        \bottomrule
    \end{tabular}
\end{table}

\subsection{Results and Findings}\label{subsec:findings}


In this subsection, we present our results in Table~\ref{tab:vfl-properties} and Table~\ref{tab:record-matching}, and Figure~\ref{fig:feature-analysis}, followed by four key findings from our analysis of real-world databases.

\begin{tcolorbox}[colback=white, coltitle=black, boxsep=0mm, left=1mm, right=1mm, top=1.5mm, bottom=1mm]
    \textbf{Finding 1}: VFL has a large number of potential real-world applications.
\end{tcolorbox}
To analyze the connectivity between databases, we constructed a graph where each node represents a database, each edge represents shared features between databases, and each color corresponds to a connected component, as visualized in Figure~\ref{fig:overlap-graph}. Table~\ref{tab:vfl-properties} further summarizes the key properties of this graph, showing that the graph of 1,000 parties contains only 3 to 4 connected components. The high connectivity indicated by both the table and figure suggests that VFL has a wide range of potential real-world applications.

\begin{tcolorbox}[colback=white, coltitle=black, boxsep=0mm, left=1mm, right=1mm, top=1.5mm, bottom=1mm]
\textbf{Finding 2}: A substantial portion (25.4\%) of potential VFL pairs have no overlapping features.
\end{tcolorbox}
To analyze feature overlap ratios between potential VFL pairs, we present the distribution of these ratios in Figure~\ref{fig:feature-overlap} and detail the proportion of pairs with zero feature overlap in Table~\ref{tab:vfl-properties}.  Table~\ref{tab:vfl-properties} reveals that among the potential VFL pairs, approximately one-quarter have zero feature overlap. This indicates that many real-world VFL scenarios require methods that can handle latent relationships without relying on shared features. We define a VFL scenario where two parties have no overlapping features but are still correlated as \textit{latent VFL}.

\begin{tcolorbox}[colback=white, coltitle=black, boxsep=0mm, left=1mm, right=1mm, top=1.5mm, bottom=1mm]
\textbf{Finding 3}: Only a small fraction of potential VFL pairs (0.2\%) can be precisely matched.
\end{tcolorbox}
To evaluate the feasibility of record matching in potential VFL pairs, we analyzed the record matched ratios across database pairs. Figure~\ref{fig:matched-ratio} illustrates the distribution of these ratios, and Table~\ref{tab:record-matching} details the ratios for various VFL data distributions. The results reveal that 70.9\% of potential VFL pairs have zero precise record matches, with only 0.2\% achieving full alignment. This indicates that existing VFL algorithms, which assume fully and precisely matched data records (\textit{precise VFL}), may be inadequate for most real-world applications. In practice, there is a need for algorithms capable of handling partial record matching (\textit{semi-precise VFL}) or even scenarios with no record matching (\textit{fuzzy VFL}). This finding aligns with existing VFL studies~\cite{wu2022coupled,nock2021impact,he2024hybrid}, which highlight the rarity of precise matching.

\begin{tcolorbox}[colback=white, coltitle=black, boxsep=0mm, left=1mm, right=1mm, top=1.5mm, bottom=1mm]
\textbf{Finding 4}: Feature distribution across parties exhibits significant imbalance.
\end{tcolorbox}
To evaluate the balance of features between potential VFL pairs, we present the distribution of feature balance ratio in Figure~\ref{fig:feature-skew}. We observe that most database pairs have highly imbalanced feature counts. This imbalance poses challenges for existing VFL algorithms, which typically assume relatively balanced feature distributions across parties for optimal performance. This finding aligns with the observation in~\cite{wu2023vertibench} that real VFL datasets are highly imbalanced w.r.t. feature importance.Imbalanced VFL is defined as a feature balance ratio below 0.5 (66.49\%), while balanced VFL has a ratio above 0.5 (33.51\%).



\section{Taxonomy of Algorithms}\label{sec:system}

%PlanGREEN

%GEN-Plan

%G- generate
%R-refine
%E- edit

%% GREEN-plan

%% PURPLE
\begin{figure*}
    \centering
    \Description{PLAID's system architecture diagram. Top part shows the database (a), and bottom part shows the interface (b). The system starts from bottom right as an instructor is interested in a programming domain, then the pipeline described in the text produces reference materials at different levels of granularity, and these are presented in the interface.}
    \includegraphics[width=\textwidth]{img/system-architecture-subgoals.png}
    \caption{PLAID's reference content is generated through an LLM pipeline
    %inspired by the practices of instructors who have successfully identified programming plans. 
    that produces output on three levels.
    First, a wide variety of use cases are generated to create example programs that focus on code's applications. Next, using LLM's explanatory comments that represent subgoals within the code, the examples are segmented into meaningful code snippets. The LLM is queried to generate other plan components for each code snippet. Finally, the code snippets are clustered to identify the most common patterns, representing plan candidates. The full programs are presented in `Programs' views of PLAID interface, whereas snippets are presented in clusters in the `Plan Creation' view.}
    \label{fig:system-pipeline}
\end{figure*}
\section{PLAID: A System for Supporting Plan Identification}
\label{sec:system-design}

Following the design goals devised from the design workshop, we refined our early prototype into PLAID: a
%LLM-powered
tool to assist instructors in their plan identification process.
PLAID synthesizes the capabilities of LLMs in code generation with interactions enabling plan identification practices observed in our studies with instructors.
As we noted in the findings of our design workshop, the LLM-generated candidate plans are not ready to be used as is in instruction, but instructors can readily adapt and correct them (\cref{sec:workshop-findings-condition2}).
PLAID enables collaboration between instructors and LLMs, enhancing the plan identification process by automating its time-intensive information-gathering tasks and facilitating instructors' ability to refine LLM-generated candidate plans based on their knowledge about pedagogy and the programming domain. 



\subsection{Practical Illustration}

To understand how instructors use can PLAID to more easily adopt plan-based pedagogies, we follow Jane, a computer science instructor using PLAID to design programming plans for her course (summarized in \cref{fig:jane-workflow}).

Jane is teaching a programming course for psychology majors and wants to introduce her students to data analysis using Pandas. As her students have limited prior programming experience and use programming for specific goals, she organizes her lecture material around programming plans to emphasize purpose over syntax. 
% that explain practical concepts to students and help them focus on the purpose behind the code they write.
% However, she realizes that all introductory computer science courses offered at her institution only teach basic programming constructs like data structures. After exploring Google Scholar for effective instructional methods to teach application-focused programming to non-computer science majors, she learned about plan-based pedagogies that help them focus on the purpose behind the code they write. In her literature review, she finds out about PLAID, a tool that can help her design domain-specific plans. She reviews the domains supported by the tool (Pandas, Pytorch, Beautifulsoup, and Django) and decides to use Pandas, a popular and powerful data analysis and manipulation library, to create her curriculum. 

She logs in to the PLAID web interface, % and takes time to explore the system's features. 
and asks PLAID to suggest a plan (\cref{fig:jane-workflow}, 1). The first plan recommended to her 
% she sees is a plan to help students learn about
is about reading CSV files. 
She thinks the topic is important and the solution code aligns with her experience; % the solution is promising and represents an important concept that students need to know about.
% She is satisfied with the given solution 
but she finds the generated name and goal to be too generic. She edits (\cref{fig:jane-workflow}, 2) these fields to provide more context that she feels is right for her students.
% She refines those fields and then 
To make this plan more abstract and appropriate for more use cases, %explain how this plan can be used for reading data from different file formats,
she marks the file path as a changeable area (\cref{fig:jane-workflow}, 3), generalizing the plan for reading data from different file formats.

Inspired by the first plan, she decides to create a plan for saving data to disk. She wants to teach the most conventional way of saving data, so she switches to the use case tab (\cref{fig:jane-workflow}, 4) and explores example programs that save data to get a sense of common practices.  %interact with the list of complete programs.
She finds a complete example where a DataFrame is created and and saved to a file. %performs cleaning tasks like deleting NaN values, and exports it.
% She realizes that something she hadn't thought of before: saving new data is almost always necessary after performing data manipulation operations!
She selects the part of the code that exports data to a file and creates a plan from that selection (\cref{fig:jane-workflow}, 5).


For the next plan, she reflects on her own experience with Pandas. She recalls that merging DataFrames was a key concept, but cannot remember the full syntax. 
% Jane reflects on her experience working with Pandas and recalls that merging DataFrames is a key operation when working with big data.
She switches to the full programs tab (\cref{fig:jane-workflow}, 6) that includes complete code examples and searches (\cref{fig:jane-workflow}, 7) for ``\texttt{.merge}'' to find different instances of merging operations. % and tries to use the search bar to find a relevant program that contains ``.merge''. 
After finding a comprehensive example, she selects the relevant section of the code and creates a plan from it.
% She again selects a part of the example, creates plan from the selection, and refines it. She engages with the system iteratively and designs twenty plans for her lecture. 

After designing a set of plans that capture the important topics, she organizes them into groups (\cref{fig:jane-workflow}, 8) 
% also grouped similar plans together
to emphasize sets of plans with similar goals but different implementations. For instance, she takes her plans about \texttt{.merge} and \texttt{.concat} and groups them together to form a category of plans that students can reference when they want to {combine data from different sources}.

% combining data using ``merge'' or ``concat''.

% the the she used plans isn't very good right now
% She exports these plans and starts preparing her lecture slides, using the plans as a way of presenting key concepts to students with minimal programming experience.
She exports these plans to support her students with minimal programming experience by preparing lecture slides that organize the sections around plan goals, using plan solutions as worked examples in class, and providing students with cheat sheets that include relevant plans for their laboratory sessions.
% using the plan goals as titles for different sections of her slides, and using the solutions as references for the examples she creates. Finally, she makes a PDF cheatsheet with all the plans for students to reference during the week's laboratory.
% The next day, she starts preparing her lecture slides and realizes that the names and goals she wrote for her plans represent key concepts in Pandas. She references the plans she created to design annotated examples that she includes on her lecture slides.

%% How does Jane actually use the plans? 
%% > Important to be careful to note that this isn't actually part of the system....
%% > She uses the generated plans to (a) as inspiration for worked examples in teh course, (b) as stems for questions that test how code should be completed
%% > She notices she now has a list of key concepts in the area


\begin{figure*}[h]
        \Description{An annotated screenshot of PLAID's `Programs' view. On the left, a list of use cases such as `Renaming columns in a Frame' and `Plotting a histogram of a column' is shown, with a scrollable list and a search bar. The latter one is selected, and on the right, we see the contents of the program in a monospaced font, with four buttons explained in the caption.}
        \includegraphics[width=\textwidth]{img/system-diagram-1-fixed.png}
        \caption{Plan Identification using PLAID: (a) list of example programs for instructors organized by natural language descriptions, (b) list of full programs of code, (c) search bar enabling easy navigation of given content to find code for specific ideas, (d) button to create a plan using the selected part of the code, (e) button to create a plan using the complete example program, (f) button to view an explanation for a selected code snippet, and (g) button for executing the selected code.}
        \label{fig:system-diagram-1}
\end{figure*}

\subsection{System Design}

At a high level, PLAID\footnote{The code for PLAID can be found at: https://github.com/yosheejain/plaid-interface.} operates on two subsystems: (1) a database of LLM-generated reference materials created through a pipeline that uses \edit{OpenAI's GPT-4o\footnote{https://openai.com/index/hello-gpt-4o/}~\cite{achiam2023gpt}}, inspired by instructors' best practices for identifying programming plans (see ~\cref{fig:system-pipeline})
%LLM for identifying plans in application-focused domains 
and (2) an interface that allows instructors to browse reference materials for relevant code snippets 
% and other plan components to achieve a goal that meets their needs. Then, they refine the candidates to mine plans 
and refine suggested content into programming plans
(see Figures~\ref{fig:system-diagram-1} and~\ref{fig:system-diagram-2}).
% In this section, we describe the implementation of the pipeline generating the reference materials and the key interface features of PLAID.



\subsubsection{Database of Reference Materials for Application-Focused Domains}

PLAID extracts information from reference materials at three levels of granularity to support each instructor's unique workflow: complete programs that address a particular use case, annotated program snippets that include goals and changeable areas, and plan candidates that cluster relevant program snippets together.

\textbf{Generating complete example programs.}
The content at the lowest level of granularity in the PLAID database are the complete programs. 
%These candidate plans were generated using a pipeline to generate \textit{plan-ful examples}, which we define as examples of programming plans in use, with all plan components identified (see Section~\ref{sec:components}). This implementation had three stages: (1) generating in-domain programs, (2) segmenting programs into plan-ful examples, and (3) clustering plan-ful examples into plans. 
\label{sec:llm-pipeline}
% \begin{figure}
% \centering
% % \includegraphics[width=0.5\textwidth]{img/pipeline-new.png}
% \includegraphics[width=\textwidth]{img/new-plan-pipeline.png}
% \caption{The three stage process for generating example programs, segmenting them with plan components, and clustering these plan-ful examples.
% %collecting and processing responses from ChatGPT into plan-ful examples}
% %\caption{The pipeline for LLM plan generation.}
% }
% \label{fig:llm-methods}
% \end{figure}
% \subsubsection{Generating In-Domain Programs}
% Informed by the insights identified in our interview study, we generated programming plans relevant to an application-focused domain: web scraping via BeautifulSoup. We utilized an LLM-based approach to generate these plans with the GPT-4 model from OpenAI using its publicly available API in an iterative workflow. 
% Our participants examined example programs and conducted literature reviews (Section \ref{sec:viewing-programs}) as key parts of their plan identification process. 
As these examples should capture a variance of use cases in the real world, we utilized an LLM trained on a large corpus of computer programs and natural language descriptions~\cite{liu2023isyourcode}.
% Inspired by this, we used Open AI's GPT-4, a state-of-the-art large language model for code generation that is trained on a large corpus of computer programs~\cite{liu2023isyourcode},
% to generate candidate programs along with its respective plan components in the programs.
We prompted\footnote{Full prompts can be found in \cref{sec:appendix-pipeline}.} the model to generate \texttt{specific use cases of <application-focused library>}, defining use case as \texttt{a task you can achieve 
with the given library} (see \cref{sec:use_case_prompt}). Subsequently, we prompted the model to \texttt{write code to do the following: <use case>}, producing a set of 100 example programs with associated tasks (see \cref{sec:code_prompt}). By generating the use cases first and generating the solution later, we avoided the problems with context windows of LLMs where the earlier input might get `forgotten', resulting in the model producing the same output repeatedly. For practical purposes, we generated 100 programs per domain. \edit{To test for potential ``hallucinations'' where the LLM generates plausible yet incorrect code~\cite{Ji_2023_hallucination}, we checked the syntactic validity of the generated programs before developing the rest of our pipeline. No more than one out of 100 generated programs included syntax errors in each of our domains, i.e., Pandas, Django, and PyTorch. Thus, we concluded that hallucinations are not a major threat to the code generation aspect of PLAID.}
%while hallucinations in LLMs are a pressing concern for systems that utilize these models,
% This collection of example programs (which we refer to as 
%dataset 
% $\mathcal{D}$) was used as our primary dataset for further analysis.

\textbf{Generating annotated program snippets.}
% \subsubsection{Segmenting Programs Into Plan-ful Examples}
% We then proceed to compile these examples with each of the plan components generated using ChatGPT. We construct a new dataset with these components, Dataset \((\mathcal{D}^{\textit{Comp}})\).
The second level of granularity in PLAID consists of small program snippets and a goal, with changeable areas annotated. 
We used the generated programs from
% \mathcal{D}$
the prior step as the input to the LLM to add subgoal labels, where we prompted the LLM to annotate subgoals (see \cref{sec:subgoals_prompt}) as comments that describe \texttt{small chunks of code that achieve a task that can be explained in natural language}. These subgoal labels were used to break the full program into shorter snippets. Each snippet was fed back to the model to generate changeable areas (see \cref{sec:ca_prompt}), defined in the prompt as \texttt{parts of the idiom that would change when it is used in different scenarios}. The subgoal label that explained a code snippet corresponded to its goal in the plan view and the list of elements assigned as changeable was used for annotations.
% (see Stage 2 in Figure~\ref{fig:llm-methods}),
%We fragmented these generated programs into smaller code pieces by generating \textit{subgoals} in the program. Then, each goal (Section \ref{sec:goal}) and the accompanying code solution (Section \ref{sec:solution}) were added as a single unit of data in our plan-ful example dataset of components, \(\mathcal{D}^{\textit{Plan-ful}}\). For each of these datapoints, we prompted the model to identify \textit{changeable areas} (Section \ref{sec:changeable}). %The name (Section \ref{sec:name}) was determined later in the pipeline (Stage 2 in Figure \ref{fig:llm-methods}).


% From the results of our qualitative study, we now know about the parts of a programming plan. In order to extract these plans automatically, we used ChatGPT. We accessed it using its publicly available API and we used the GPT-4 model. We selected 3 domains that are interesting for non-majors. This included . 

% For each of these domains, we first asked the LLM to generate 100 use cases. We then re-prompted it with the use cases it generated and asked it to generate code that would be written to accomplish that use case.
% potential for another table?
% add code metrics from stackoverflow github work for chatgpt
% With all these code pieces collected, we then asked ChatGPT to generate each of the plan parts one-by-one.

% \subsubsection*{Extracting Goals and Solutions}Generated programs 
% in \(\mathcal{D}\) 
% typically included a comment before each line, which described that line's functionality. However, these comments did not capture the high-level purpose of the code, as required by a plan goal. To generate more abstract goals for a piece of code, we defined subgoals as \texttt{short descriptions of small pieces of code that do something meaningful} in a prompt and asked the LLM to \texttt{highlight subgoals as comments in the code.} %In our query, we also added the way we define subgoals to provide the relevant context to the model. Specifically, we wrote that 
% The output from this prompt was a modified version of each program
% from \(\mathcal{D}\), 
% where blocks of code are preceded by a comment describing the goal of that block. % of code. % instead of restating functionality. 

% We split each complete program into multiple segments based on these new comments. Thus, the subgoal comments from each complete program I
% n the modified \(\mathcal{D}\) 
% became a plan goal, and the code following that comment became the associated solution. %, collected in \(\mathcal{D}^{\textit{Plan-ful}}\). % After it returned the annotated code piece, we extracted the comment and the following lines of code before the next comment. This pair acted as a subgoal-code piece. We collected all such pairs across all use cases from \(\mathcal{D}\) and added them to \(\mathcal{D}^{\textit{Plan-ful}}\).
% Each goal 
% %(Section \ref{sec:goal}) 
% and solution pair
% %(Section \ref{sec:solution}) 
% was added as a single unit of data in our plan-ful example dataset.
% , \(\mathcal{D}^{\textit{Plan-ful}}\).

% \subsubsection*{Extracting Changeable Areas}To annotate the changeable areas for a plan, we defined changeable areas as \texttt{parts of the plan that would change when it is used in a different context} in our prompt and asked the model to \texttt{return the exact part of the code from the line that would change} for all code pieces from the dataset with plan-ful examples.
% from \(\mathcal{D}^{\textit{Plan-ful}}\). 
% This data was added to \(\mathcal{D}^{\textit{Plan-ful}}\).

% to-do
% \subsubsection{Clustering Plan-ful Examples into Plans}
\textbf{Generating clustered plan candidates.}
\label{sec:clustering}
% We perform k-means clustering on the plans \(\mathcal{D}^{\textit{Plan-ful}}\) to identify clusters of similar code pieces and thus, programming plans.
The highest level of granularity provided in PLAID
%presents users with 
are
plan candidates, in the form of clusters of annotated program snippets. To compare the similarity of program snippets, we used CodeBERT embeddings following prior work~\cite{codebert} and applied Principal Component Analysis (PCA) \cite{PCAanalysis} to reduce the dimensionality of the embedding while preserving 90\% of the variance. The snippets were clustered using the K-means algorithm~\cite{kmeansclustering}, using the mean silhouette coefficient for determining optimal K~\cite{silhouettecoeff}. Each cluster is treated as a plan candidate, with the goal, code, and changeable areas from each program snippet in the cluster presented as a suggested value for the plan attributes.
% We used a clustering algorithm to group similar program snippets 
% plan-ful examples together as a programming plan. For clustering the code pieces, we used the CodeBERT model from Microsoft \cite{codebert} to obtain embeddings for each code piece in our dataset of plan-ful examples
% % in \(\mathcal{D}^{\textit{Plan-ful}}\) 
% and applied Principal Component Analysis (PCA) \cite{PCAanalysis} to reduce the dimensionality of the embedding vectors while preserving 90\% of the variance. These embeddings were clustered using the K-means algorithm~\cite{kmeansclustering}. The optimal number of clusters \(\mathcal{K}\) was determined by assessing all possible \(\mathcal{K}\) values 
% % (where \(\mathcal{K} \in [2, \texttt{length}(\mathcal{D}^{\textit{Plan-ful}})]\))
% using the mean silhouette coefficient \cite{silhouettecoeff}. We assigned each example 
% % in \(\mathcal{D}^{\textit{Plan-ful}}\) 
% to a cluster of similar code pieces. 
% \subsubsection*{Extracting Names}
For each plan candidate, a name (see \cref{sec:name_prompt}) that summarizes all snippets in the cluster was generated by prompting an LLM with the contents of the snippets and stating that it should generate \texttt{a name that reflects the code's purpose} and it should focus on \texttt{what the code is achieving and not the context}. 
% Then, all code snippets from each cluster of examples were provided as input to the LLM along with a prompt asking it to \texttt{devise a name for that cluster of plans}.

% \subsection{Interface for Refining Candidate Plans}

% %nd the back-end server relied on routes written in Flask. The domain-specific candidate plans suggested to the user are queried from the database of candidate plans generated using the LLM. Each participant was required to log in to the web page using their unique credentials, which allowed us to record their activity for analysis. While the complete details of our implementation of the web-based application are out of scope for this paper, we describe its main features in Section~\ref{sec:implementation_of_webinterface}.

% \subsubsection{\edit{Preliminary Technical Evaluation of Generated Content}}

% \edit{syntactic validity and standard code complexity metrics to determine
% their suitability for novices}


\begin{figure*}[h]
    \Description{An annotated screenshot of PLAID's Plan Creation view with three panes, with plans shown as boxes on the left. A plan is highlighted, and we see its components on the middle pane. On the rightmost pane, we see suggested values for the selected component.}
    \includegraphics[width=\textwidth]{img/system-diagram-2-new.png}        
    \caption{Plan Identification using PLAID: (h) button that suggests a domain-specific candidate plan from the system database, (i) pane enabling viewing of similar values for the selected plan component, (j) button to view the solution code as part of a complete program, (k) pane with a structured template for plan design with editable fields to refine plan components, (l) button to copy a selected plan, (m) button to mark snippets of code from the plan solution as changeable areas, and (n) a button to group plans together into a category and add a name.}
    \label{fig:system-diagram-2}
\end{figure*}

% \subsubsection{Key Characteristics}
% PLAID supports the process of plan identification in data processing with Pandas, machine learning with Pytorch, web development using Django, and web scraping using BeautifulSoup. 

\subsubsection{Interface for Designing Programming Plans}
Building on the 
%characteristics addressed in the artifact (Section~\ref{sec:design-artifact}) and 
design goals identified in the design workshop (\cref{sec:design-goals}), PLAID enables a set of key interactions to assist instructors in refining candidates to design plans for their instruction. 



\textbf{Interactions for Initial Plan Identification.}
% Initial Plan Identification with Quick Exploration of Many Authentic Programs
While instructors valued the availability of code examples in the design workshop (Section~\ref{sec:design-workshop-findings}), we observed many opportunities for scaffolding their interaction with the reference material. To this end, PLAID presents example programs in two different views \textbf{(DG1)}. 
% We saw instructors scanning examples, selecting desired code pieces, and copying them over into their plan templates in all conditions in the study. 
The ``Programs (Organized by Use Case)'' (\cref{fig:system-diagram-1}a) tab includes a list of use cases where instructors can click on an item to expand the program for that use case.
The ``Programs (Full Text)''  tab (\cref{fig:system-diagram-1}b) lists all the programs and enables instructors to scroll or search through (\cref{fig:system-diagram-1}c) all the code at once.
% presents the contents of all the programs expanded viewing a list of complete code examples, allowing instructors to look at materials they would typically search for when designing plans.
% equipping instructors with full-code programs organized in a list of short natural language descriptions of common use cases in their domain of expertise. 
Both views support directly creating a plan from the whole example (\cref{fig:system-diagram-1}e), or a selected part of it (\cref{fig:system-diagram-1}d), by copying the solution and the goal of the program into an empty plan template
% < Highlight code in full code and code pane in tab1 and make a plan (D1)
% < Add a button to add full program as a plan too (D1)
further supporting efficient use of the material \textbf{(DG3)}.
% This interaction copies over the selected code and its respective use case into the solution and name fields, respectively. 
% < Code explanation plugin for strange syntax (GPT) (D2)

To facilitate understanding unfamiliar code and syntax, we implemented a ``View Explanation'' button (\textbf{DG2}) that generates a short description of the selected line(s) of code by prompting an LLM (\cref{fig:system-diagram-1}f). 
% In this case, participants hesitated to use the suggested syntax in their plans because its functionality was unclear to them. PLAID supports a button named ``View Explanation'' where the user can select a method, function, or line of code that is unclear and click on it to understand its working \textbf{(D2)}. 
Participants also looked for code execution to validate and understand a program. However, since the code snippets instructors work with are often incomplete in this task, we implemented a ``Run Code'' feature (\textbf{DG2}) that predicts the output of a selected code snippet by prompting an LLM to walk through the code \texttt{step by step}, using Chain-of-Thought prompting~\cite{wei2022chain} (\cref{fig:system-diagram-1}g). Only the predicted output for the code is presented, ignoring other output from the LLM.

% to examine the code behavior and thus mitigate the challenge of being faced with unfamiliar syntax. Thus, using PLAID, instructors are able to run complete programs to view their output \textbf{(D2)}.
% < Search in the use cases (and full progs) (D3)
% Frequently, instructors relied on their expertise and experience to formulate ideas about goals for which they wanted to create plans. While interacting with condition C in the design workshop, interviewees suggested including a mechanism to search for specific keywords within code and  its natural language description. To facilitate the instructor-LLM collaboration, allowing users to find examples implementing their ideas, PLAID includes a search bar that helps users navigate the given use cases, complete programs, and effectively find specific examples they may be looking for \textbf{(D3)}.

\textbf{Interactions for Plan Refinement.}
% Support Plan Refinement with Comparisons of content
% Participants indicated difficulty mining plans from code examples (Section~\ref{sec:challenges_practice}). 
To provide suggestions for code patterns common enough to be potential programming plans,
%To alleviate challenges in identifying content common enough for designing plans, 
we utilize the clustered program snippets from our database. In the ``Plan Creation'' view of PLAID, instructors can ask for suggestions (\cref{fig:system-diagram-2}h) to see a candidate plan to refine (\textbf{DG3}).  \edit{This functionality allows instructors to draw on their experience to recognize common code snippets and decide if they are valuable to teach students.}
% If instructors want to demonstrate their plan as part of a complete code example, they can review these examples reducing the effort that they would need to put in to recall syntax and construct a complete example. 
\edit{This promotes recognition over recall \cite{recognition_over_recall}, thus helping reduce the cognitive effort that instructors may have to put in while designing programming plans traditionally.}
To allow instructors to better understand the context of a plan under refinement, PLAID 
also includes a button for searching for the current solution within the entire set of full programs
%, showing the code snippet in context 
%as part of a complete example
(\textbf{DG3}, \cref{fig:system-diagram-2}j).
% < Keyword search/embedding filter for potential values (D1)

As instructors edit the components of a plan, they are shown similar values from the corresponding component in that cluster (\cref{fig:system-diagram-2}i). By clicking on any suggested value, instructors can replace a plan component with a suggestion that better captures that aspect of the plan \textbf{(DG1)}. \edit{By allowing instructors to view the plan they are working on along with other related code pieces in a split screen view, we promote instructor efficiency by reducing the split-attention effect \cite{tarmizi1988guidance}. In the current plan creation process, even when using LLMs from their chat interface, instructors would have to switch between windows with code examples and their text editor which may increase the load on the instructors' working memory \cite{clark2023learning}. In PLAID, instructors can edit their plans and view similar code pieces at the same time.}

% \edit{By enabling these interactions and thus organizing ``knowledge in the world'' effectively, PLAID reduces the need for instructors to store and retrieve the ``knowledge in their head'' \cite{Norman_DOET}. Thus, PLAID optimizes the plan creation process by allowing efficient search within the ``knowledge in the world'' and reducing the cognitive load while storing and retrieving ``knowledge in the head'', minimizing the total effort required \cite{} by instructors.}
% after searching its code corpus for similar examples using a keyword search \textbf{(DG1)}.
% < Show use case button in solution (add highlighting) (D2)
% To help instructors easily consider the context of a plan as they refine it, PLAID 
% In the design workshop, few instructors emphasized the importance of presenting worked and contextualized examples to students. 

% ‘go to a use case’ button that redirects the user to the tab with full code programs and highlights the plan as part of a complete example \textbf{(D2)}.

\textbf{Interactions for Building Robust and Shareable Plan Descriptions.}
% Support robust/sharable plan descriptions
% From Section~\ref{sec:process_intro_plan_design}, instructors indicated drawing on their experience in the application-specific domain and instructional expertise to think about how to best solve a problem. 
PLAID encourages instructors to design plans in a structured template (\cref{fig:system-diagram-2}k). Moreover, PLAID reinforces the plan template by providing a dedicated method for annotating changeable areas by highlighting any part of the code (\textbf{DG3}, \cref{fig:system-diagram-2}m). Instructors can further explain the changeable areas by adding comments as text.
% \edit{The structured template view of the plan encourages instructors to articulate their mental models of how the plan would generalize to other problems, allowing the transfer of ``knowledge in the head'' to ``knowledge in the world''.}

Our design workshop showed that participants would create a plan and copy it to emphasize alternatives or modifications to the underlying idea. To support this workflow,
% In our design workshop, participants created copies of their plans to display alternative solutions to achieve the same goal, emphasizing that multiple possible solutions in code could accomplish the same goal.
% < Duplicating plans (D3)
% To accelerate this process of teaching a variety of possible solutions, 
PLAID allows users to ``duplicate'' plans on the canvas and further edit them to present alternative solutions for the same plan \textbf{(DG3}, \cref{fig:system-diagram-2}l).
% Highlight text from solution to change it to changeable areas (highlighting code itself) (D4)

% In conditions A and B, instructors highlighted the changeable areas in the code itself.
% To allow participants to emphasize the changeable areas in code in PLAID, we implemented the ``add to changeable areas'' button. After selecting the changeable piece of code, clicking on this button highlights the text in a different color and adds it to the box of changeable areas to complete the templated plan design (\textbf{D4}).
% Grouping plans into categories (D4)
% < Multiple selection of the boxes (D4)
% < Naming groups of boxes (D4)
To encourage instructors to think about organizing plans in ways that they would present them to students, PLAID provides an open canvas view for instructors that allows them to arrange plans as they prefer. In addition, PLAID supports a ``grouping'' feature (\cref{fig:system-diagram-2}n), which allows instructors to combine plans with similar goals together into one category (\textbf{DG4}).

% A handful of users postulated each plan as an example question that can be used on assessments. They intended to create multiple variants of the same question for students. They suggested that being able to visualize the different categories would be helpful. Using PLAID, users can select multiple patterns together, add them to a group, and name the group \textbf{(D4)}.  % :(

\subsubsection{System Architecture}
The pipeline to create reference materials is implemented in Python, using the state-of-the-art large language model GPT-4o (Model Version: 2024-05-13). The interface for PLAID is implemented as a web application in Python as a Flask webserver, with an SQLite database. The user-facing interface is implemented using HTML, CSS, and JavaScript, with the canvas interactions realized with the library `\textit{interact.js}'. 




\section{Discussion}\label{sec:future}
\section{Theoretical Analysis}\label{sec:theoretical}

\textbf{Different correct answers are competitor.}\quad For any LLM trained with cross-entropy loss, different correct answers are competitors in terms of probability \footnote{The ``same question'' refers to questions that are semantically equivalent but do not need to be identical.}. Continuing with the example of proposing a president, suppose $\tau^{a}$ (``\texttt{Barack}'') is the label of a sample whose $\bm{q}$ is ``\texttt{[INST]Could you give me one name of president?[\textbackslash INST]}'' and a generated token vector $\bm{a}_{t-1}$  can be decoded into ``\texttt{Sure, here is a historical American president:**}'', the loss of the next token at this position during supervised fine-tuning can be written as:
\begin{equation}
\begin{aligned}
 &L^{\tau^a} = - \log \frac{\exp(\mathcal{M}({\tau^a}|\bm{q},\bm{a}_{t-1}))}{\sum_{m=1}^{|\bm{Y}|} \exp(\mathcal{M}(\tau^{m}|\bm{q},\bm{a}_{t-1}))} ,
 % \\   &L^{\tau^b} = - \log \frac{\exp(\mathcal{M}(\tau^b|\bm{q},\bm{a}_{t-1}))}{\sum_{m=1}^{|\bm{Y}|} \exp(\mathcal{M}(\tau^{m}|\bm{q},\bm{a}_{t-1}))} ,
\end{aligned}
\end{equation}
where $L^{\tau^a}$ is the loss on the sample with the next token label $\tau^{a}$.
Consider cases where multiple distinct answers to the same question appear in the training set, the situation becomes different. For example, $\tau^{b}$ (``\texttt{George}'') is the label in another sample with the same question. When the model is simultaneously fine-tuned on both samples, the gradient update for the model will be:
\begin{equation}
\begin{aligned}
 & \nabla_{\mathcal{M}} (L^{\tau^a} + L^{\tau^b}) = \nabla_{\mathcal{M}} L^{\tau^a} + \nabla_{\mathcal{M}} L^{\tau^b} \\
% &= -y_a^{\tau^a}\frac{1}{\Omega_a^{\tau^a}}\nabla_{\mathcal{M}}\Omega_a^{\tau^a}-\sum_{m \neq a}^{|\bm{Y}|} y_a^{\tau^m}\frac{1}{\Omega_a^{\tau^m}}\nabla_{\mathcal{M}}\Omega_a^{\tau^m}
% \\
% &\quad -y_b^{\tau^b}\frac{1}{\Omega_b^{\tau^b}}\nabla_{\mathcal{M}}\Omega_b^{\tau^b}-\sum_{m \neq b}^{|\bm{Y}|} y_b^{\tau^m}\frac{1}{\Omega_b^{\tau^m}}\nabla_{\mathcal{M}}\Omega_b^{\tau^m}
% \\
&\quad= \underbrace{-y_a^{\tau^a}\frac{1}{\Omega_a^{\tau^a}}\nabla_{\mathcal{M}}\Omega_a^{\tau^a}-y_b^{\tau^b}\frac{1}{\Omega_b^{\tau^b}}\nabla_{\mathcal{M}}\Omega_b^{\tau^b}}_{\text{(1) maximizing the probability of annotated answer}}\\& \quad \underbrace{-y_a^{\tau^b}\frac{1}{\Omega_a^{\tau^b}}\nabla_{\mathcal{M}}\Omega_a^{\tau^b}-y_b^{\tau^a}\frac{1}{\Omega_b^{\tau^a}}\nabla_{\mathcal{M}}\Omega_b^{\tau^a}}_{{\text{\textbf{(2)} minimizing the probability of the other annotated answer}}}\\& \quad \underbrace{-\sum_{m \neq a,b}^{|\bm{Y}|}y_{a,b}^{\tau^m} \left[ \frac{1}{\Omega_a^{\tau^m}}\nabla_{\mathcal{M}}\Omega_a^{\tau^m} + \frac{1}{\Omega_b^{\tau^m}}\nabla_{\mathcal{M}}\Omega_b^{\tau^m} \right]}_{\text{(3) minimizing the probability of incorrect answers}},
\end{aligned}\label{eq:competitor}
\end{equation}
where $\Omega_a^{\tau^a}=\frac{\exp(\mathcal{M}(\tau^a|\bm{q},\bm{a}_{t-1}))}{\sum_{m=1}^{|\bm{Y}|} \exp(\mathcal{M}(\tau^{m}|\bm{q},\bm{a}_{t-1}))}$, and $y_a^{\tau^m}$ indicates the next token label of a training sample with ground-truth label ${\tau^a}$, that is, we have $y_a^{\tau^a}=1$ and $y_a^{\tau^b}=0$. In particular, when $\mathcal{M}$ is in a certain state during training, we have $\Omega_a^{\tau^a}=\Omega_b^{\tau^a}$, and we make distinctions to facilitate the reader's understanding here. As we can see, for scenarios with multiple answers, the training objective can be divided into three parts:
(1) For each sample, increase the probability of its own annotation in the output distribution.
(2) For each sample, decrease the probability of another sample's annotation in the output distribution. \textit{\textbf{Note:}} This part leads to the issue where probability cannot anymore capture the reliability of LLM responses, as different correct answers tend to reduce the probability of other correct answers, making low probabilities cannot indicates low reliability.
(3) For both samples, decrease the probability of other outputs not present in the annotations in the output distribution.






\section{Conclusion}\label{sec:conclusion}
\section{Conclusion}
In this work, we propose a simple yet effective approach, called SMILE, for graph few-shot learning with fewer tasks. Specifically, we introduce a novel dual-level mixup strategy, including within-task and across-task mixup, for enriching the diversity of nodes within each task and the diversity of tasks. Also, we incorporate the degree-based prior information to learn expressive node embeddings. Theoretically, we prove that SMILE effectively enhances the model's generalization performance. Empirically, we conduct extensive experiments on multiple benchmarks and the results suggest that SMILE significantly outperforms other baselines, including both in-domain and cross-domain few-shot settings.







\bibliographystyle{named}
\bibliography{references-short}

\end{document}
