
\typeout{IJCAI--25 Instructions for Authors}


\documentclass{article}
\pdfpagewidth=8.5in
\pdfpageheight=11in

\usepackage{ijcai25}

\usepackage{times}
\usepackage{soul}
\usepackage{url}
\usepackage[hidelinks]{hyperref}
\usepackage[utf8]{inputenc}
\usepackage[small]{caption} 
\usepackage{graphicx}
\usepackage{amsmath}
\usepackage{amsthm}
\usepackage{amssymb}
\usepackage{booktabs}
\usepackage{algorithm}
\usepackage{algorithmic}
\usepackage{multicol}
\usepackage{multirow}
\usepackage[switch]{lineno}
\usepackage{amsfonts}


\usepackage{subcaption}
\usepackage{makecell}
\usepackage{tcolorbox}
\usepackage{xcolor}

\newcommand\blfootnote[1]{%
	\begingroup
	\renewcommand\thefootnote{}\footnote{#1}%
	\addtocounter{footnote}{-1}%
	\endgroup
}

\urlstyle{same}


\newtheorem{example}{Example}
\newtheorem{theorem}{Theorem}



\usepackage{todonotes}
\newcommand{\note}[1]{\todo[inline,color=red!40]{#1}}

\usepackage{pifont}
\newcommand{\cmark}{\ding{51}}%
\newcommand{\xmark}{\ding{55}}%

\usepackage{tikz}
\newcommand*\emptycirc[1][1ex]{\tikz\draw (0,0) circle (#1);} 
\newcommand*\halfcirc[1][1ex]{%
  \begin{tikzpicture}
  \draw[fill] (0,0)-- (90:#1) arc (90:270:#1) -- cycle ;
  \draw (0,0) circle (#1);
  \end{tikzpicture}}
\newcommand*\fullcirc[1][1ex]{\tikz\fill (0,0) circle (#1);} 


\pdfinfo{
/TemplateVersion (IJCAI.2025.0)
}

\title{Vertical Federated Learning in Practice: The Good, the Bad, and the Ugly}


\author{
Zhaomin Wu$^1$\and
Zhen Qin$^3$\and
Junyi Hou$^1$\and
Haodong Zhao$^4$\and
Qinbin Li$^5$\and\\
Bingsheng He$^1$\and
Lixin Fan$^2$
\affiliations
$^1$National University of Singapore, Singapore,
$^2$WeBank, China,
$^3$Zhejiang University, China\\
$^4$Shanghai Jiao Tong University, China,
$^5$Huazhong University of Science and Technology, China
\emails
zhaomin@nus.edu.sg,
zhaohaodong@sjtu.edu.cn,
zhenqin@zju.edu.cn,
junyi.h@comp.nus.edu.sg,
qinbin@hust.edu.cn,
hebs@comp.nus.edu.sg,
lixinfan@webank.com
}












\begin{document}

\maketitle

\begin{abstract}
Vertical Federated Learning (VFL) is a privacy-preserving collaborative learning paradigm that enables multiple parties with distinct feature sets to jointly train machine learning models without sharing their raw data. Despite its potential to facilitate cross-organizational collaborations, the deployment of VFL systems in real-world applications remains limited. To investigate the gap between existing VFL research and practical deployment, this survey analyzes the real-world data distributions in potential VFL applications and identifies four key findings that highlight this gap. We propose a novel data-oriented taxonomy of VFL algorithms based on real VFL data distributions. Our comprehensive review of existing VFL algorithms reveals that some common practical VFL scenarios have few or no viable solutions. Based on these observations, we outline key research directions aimed at bridging the gap between current VFL research and real-world applications. 
\end{abstract}

\section{Introduction}\label{sec:introduction}

\documentclass[../main.tex]{subfiles}
\graphicspath{{../images/}}
\makeatletter
\def\input@path{{../images/}}
\makeatother
\begin{document}
\section{Introduction}
\begin{figure}
\centering
\begin{tikzpicture}
\node[inner sep=0pt] (ws) at (0, 0) {
\includegraphics[height=.4\textwidth, trim={10cm 0 10cm 0},clip]{world_space.png}};
\node[inner sep=0pt] (cs) at (6,0) {\includegraphics[height=.4\textwidth, trim={10cm 1cm 10cm 4cm},clip]{conf_space.png}};
\end{tikzpicture}
\vspace{-5pt}
\label{fig:pbrm_intro}
\caption{\textbf{Left}: Shows world space obstacles as grey spheres. Robots start and goal configuration is colored red and green, respectively. Configurations along the computed path are colored transparent blue. \textbf{Right:} Mapped world space scenario to configuration space. Obstacle region is the grey mesh. Red spheres are collision-free regions computed by the neural SCDF. The optimized shortest path in the convex corridor is the blue curve.}
\vspace{-25pt}
\end{figure}
Motion planning is the problem of finding a collision-free trajectory that connects a given start and goal configuration. The planning takes place in the configuration space of the robot. For single body robots, like mobile robots or drones, the configuration space and the world space are usually the same. This simplifies the planning, since explicit obstacle representations are available which enables geometrical tools like separating hyperplanes, smallest distance to obstacles etc., to be used when designing motion planning algorithms. For multi-body robots like manipulators, the situation is completely different. The world space obstacles are usually mapped to non-convex regions, and to make the problem even harder, the mapping is usually not known. Forming explicit representations of the obstacle region in the configuration space is usually too expensive or intractable. Despite all of this, sampling based planners are used with great success, which mainly is due to their use of implicit representations of the obstacle region. The basic idea is to construct a graph in the configuration space that covers and connects the collision-free region. From this graph, a path can be extracted that connects a given start and goal configuration. The approach is computationally expensive, since the graph is constructed with the smallest geometrical building block available, points, which represents a collision-check. Furthermore, the extracted paths from the graph are non-smooth and jagged due to the stochastic nature of the approach. This adds an additional post-processing step to the process, where the paths are shortcutted and smoothened, before the path can be used for tracking. Clearly a lot of time is invested to form this graph and produce smooth paths. Thus, if the obstacles start to move, then all of this work is done in no use, since all points that make up this graph need to be re-verified, which is simply too time consuming to be done in real time.
\\\\
In this work, we want to address the existing drawbacks of the sampling based planners. Our main contribution is an improved motion planner where each vertex in the graph covers a collision-free region in the form of a sphere instead of a point and where the edges are formed with neighboring intersecting spheres. This representation has the advantage of instead of returning piecewise linear paths, returning a sequence of overlapping spheres, i.e. a convex corridor, that connects a given start and goal configuration, illustrated in Figure \ref{fig:pbrm_intro}. This convex corridor allows us to use convex optimization to produce smooth trajectories, instead of computationally expensive post-processing methods. The representation further allows us to estimate the coverage of the collision-free space, which gives us awareness and feedback in the offline roadmap construction phase. Finally, our representation is simple to adapt to moving obstacles, simply requery for the new radii and recheck for intersections. 
\\\\
The spherical collision-free regions are formed using a signed distance function (SDF), which is a function that returns the smallest distance from an arbitrary point to the boundary of an obstacle. As the name implies, the distance is signed, thus if the point is inside the obstacle it is negative otherwise positive. If the distance is positive, a sphere with radius equal to the distance is guaranteed to cover a collision-free region. Using an SDF in motion planning is not new, but what is novel about our approach is that we express the distance in the configuration space instead of the world space and by doing so allows us to form these convex collision-free regions. We refer to the resulting SDF as a signed configuration distance function (SCDF). Computing an SCDF analytically is non-trivial, our approach is therefore to parameterize the SCDF with a deep neural network and learn the mapping by supervised learning. Our resulting neural SCDF can compute distances for different parameter values of obstacle shapes and we also show how multiple distances can be combined, thus making our approach flexible.
\section{Related work}
Motion planning algorithms can roughly be divided into three families, grid-based, sampling based and optimization based methods. Grid-based methods (GBM) discretize the planning space from which a graph is then compiled. A standard search method is A$^\star$ \citep{a_star}, which is classified as an \textit{informed} search method, since it employs a heuristic function to speed up the search. A$^\star$ guarantees to return an optimal path at the level of discretization used. GBMs usually discretize the planning space by a regular lattice and this limits the GBMs to problems with low dimensionality due to the curse of dimensionality. Thus, GBMs are usually limited to single-body robots where the degrees of freedom (DOF) are low. To overcome the inherent scaling problem with the GBMs, stochastic methods are usually used for multi-body robots. These methods are termed as sampling-based methods (SBM) and core members within this family are the rapidly-exploring random trees (RRT) \citep{rrt} and the probabilistic roadmap (PRM) \citep{prm}. RRT grows a tree from the start configuration and explores the collision-free region in a rapid way until it is able to connect to the goal region. RRT is usually improved by bi-directional planning \citep{rrt_connect}, i.e. an additional tree is grown from the goal configuration and the trees are tested for connection after any tree has been expanded. RRT is a single-query method, thus it searches for a path from scratch each time it is queried. Contrary to this, PRM is a multi-query method, which solves for multiple queries without starting from scratch. PRM does this by creating a roadmap (graph) that covers the collision-free space as an offline step. The graph is then used to solve for multiple queries. PRMs are used in cases where the environment does not change since the extra offline step is too computationally costly and needs to be re-done if the environment is changed. In our work, we address this inherent issue by using a different roadmap representation. Our vertices in the graph cover a collision-free region in the form of spheres and we form the edges by checking for intersecting spheres. If something in the environment changes, we recompute the spheres radii and recheck the intersections, without relying on collision detection. We use a trained neural network to compute the sphere radius, therefore querying for the radius can be done fast, hence our representation enables the PRM for dynamic environments.
\\\\
In the recent decades, optimization based methods (OBM) \citep{chomp, schulman, itomp, stomp} have been introduced as an alternative to SBM for multi-body robots. Like the SBM, the OBMs scale well to higher dimensional problems and produce smoother motion. It is common to use a SDF in the optimization since it is a smooth function, thus enabling gradient-based methods. However, the standard way of expressing the SDF is in world space. The distance therefore needs to be mapped to the configuration space by the forward kinematics. This mapping makes the optimization problem a non-linear program (NLP), which is computationally expensive to solve. Recently, a different approach has been proposed. In \cite{mp_gcs} motion planning is formulated as a convex optimization problem by using the graph of convex sets framework \citep{gcs}. The underlying idea is to decompose the collision-free space into intersecting convex sets from which a convex optimization problem is formulated. In cases where an explicit representation of the obstacles in the configuration space exists, like for single-body robots, creating collision-free convex regions can be done fast \citep{iris}. For multi-body robots, this is non-trivial. Existing work does this successfully \citep{iris_nlp, iris_c} by an optimization based approach, but the methods are still too time consuming to be used in the presence of moving obstacles. Our approach is instead to use deep learning to learn an SDF expressed in the configuration space. With this, we can query for shortest distances to the collision boundary, which allows us to expand spherical regions which are collision-free. Our approach is fast and therefore enables our suggested roadmap planner to be used in dynamic environments.
\\\\
Recent research has focused on learning collision detection \citep{fk_kernel_distance, diffco, graphdistnet} by predicting the signed distance between the robot links and the surrounding obstacles in the world space. The learned SDF is used in trajectory optimization but since the distance is expressed in the world space, the problem becomes an NLP and therefore takes a long time to solve. We take a novel approach and suggest to instead express the signed distance in the configuration space. This allows us to improve the PRM at the same time as it enables convex optimization for trajectory optimization, which runs faster and is more reliable than NLP solvers. In \cite{cspf} a learned signed distance function in the configuration space is proposed similar to our approach. However, their approach is restricted to point cloud representations, while we propose to represent the obstacles as parameterized geometric shapes, e.g. spheres. Furthermore, we also show how to use our learned SCDF to improve an existing roadmap planner.
\section{Problem formulation}
A robot is located in the world space, $\W \subset \R^3 $. The unique location of the robot is given by its configuration $\q \in \C$, where $\C$ is the configuration space. The set of points covered by the robots bodies at a certain configuration is expressed as $\B(\q) \subset \W$. The robot is surrounded by $\NrObst$ obstacles $\O = \bigcup_{i=1}^{\NrObst} \O_i$, where  $\O_i \subset \W$. The representation of the obstacle in the configuration space is the set $\C\O_i = \{\q \in \C \: |\: \B(\q) \cap \O_i \neq \emptyset \}$. The obstacle space is formed as $\Co = \bigcup_{i=1}^{\NrObst} \C \O_i$. The complement is referred to as the free space, $\Cf = \C \setminus \Co$. The path planning problem is a tuple, ($\Cf$, $\qStart$, $\qGoal$), where we want to connect a query pair, consisting of a start, $\qStart$, and goal configuration, $\qGoal$, with a geometric path, $\q(s): [0, 1] \mapsto \Cf$, such that $\q(0)=\qStart$ and $\q(1)=\qGoal$, or report correctly when such a path does not exist.
\end{document}



\section{Overview of Vertical Federated Learning}\label{sec:overview}
\begin{figure*}[t]
\begin{center}
\includegraphics[width=.85\linewidth]{fig_overview_v3.pdf}
\end{center}
\caption{
FastAtlas Overview: In each frame, we compute charts spanning fully or partially visible triangles (a), determine texture space bounding boxes for the visible portions of the view-space projections of each chart, and tightly pack these boxes into atlases (b, here $2K \times 2K$). We simultaneously bijectively parameterize and shade the charts into their atlas boxes, obtaining high quality texture space shading (c), and use this shading to render the shaded frames (d).}
\label{fig:overview}
\label{fig:alg_overview}
\end{figure*}

\section{Overview}
\label{sec:overview}
Our work has two core contributions: a real-time, GPU-based algorithm for tight packing of general parameterized charts into compact atlases; and a real-time TSS method that
utilizes this packing.  

\paragraph*{FastAtlas Packing.}
FastAtlas runs entirely on the GPU as a series of compute shaders. It takes the bounding boxes of parameterized charts as input, and packs them into an atlas (Fig~\ref{fig:overview}b, Sec.~\ref{sec:pack}). As such, the only input it requires are the dimensions of the bounding boxes.
Its outputs are deterministic; identical input charts are packed into identical atlases. This is critical for TSS and similar applications, as it ensures that consecutive frames taken from the same camera view have the same shading. Even minute shading differences across such frames can cause sampling jitter, leading to undesirable flicker \cite{baker2012rock}. 
While prior methods such as \cite{mueller2018shading,hladky2019tessellated,hladky2021snakebinning,Neff2022MSA} cap the dimensions of the charts that can be packed as-is for a given atlas size, and scale down all charts that exceed these dimensions, we scale all charts by the same factor, and do so only when strictly necessary to achieve packing success (Figs~\ref{fig:atlas},~\ref{fig:sas_issues}). 

\paragraph*{TSS using FastAtlas.}
Our end-to-end TSS atlas generation method combines the packing method above with a novel approach for computing seamless per-frame charts. 
We define our charts as the connected components of the visible surfaces in each frame (Fig.~\ref{fig:overview}a), and efficiently compute them using a parallel union-find algorithm (Sec.~\ref{sec:visible}). Since the boundaries of these charts coincide with the contours of the rendered surface, they are {\em invisible} to the viewer. This approach 
eliminates the artifacts caused by shading discontinuities along visible seams (Fig.~\ref{fig:seams}). 

\begin{parWithWrapFigure}
\begin{wrapfigure}{l}{.27\columnwidth}%
\includegraphics[width=\linewidth]{fig_inset_view_plane.pdf}%
\end{wrapfigure}
We bijectively parametrize the {\em visible portions} of our charts by projecting them to view space (inset). This maps a constant number of texels to each pixel in the final rendered output, evenly distributing residual undersampling error across all image pixels. While conceptually straightforward, efficiently parameterizing charts containing partially visible triangles using viewspace projection is non-trivial, as the visible portions may no longer be triangular (e.g. green triangle in the inset); applying naive projection to triangles with vertices behind the camera may produce ill-posed results. Clipping triangles before projection is both computationally expensive and significantly complicates downstream operations. We avoid explicit clipping by observing that all that is required for atlas packing is the dimensions of, potentially conservative, bounding boxes of these projected visible portions. We compute such bounding boxes without explicit chart clipping by adapting a conservative screen coverage estimator \shortcite{Blinn:CalculatingScreenCoverage} (Sec.~\ref{sec:box}). We then pack the computed boxes using FastAtlas. 
\end{parWithWrapFigure}

Finally, we shade the visible portion of each chart into its corresponding atlas bounding box (Fig~\ref{fig:overview}c). 
The resulting texture is then used during rasterization as a standard texture map (Fig. ~\ref{fig:overview}d). 
Our framework is compatible with all existing approaches for texture space shading, including forward shading, raytraced illumination, or deferred shading in texture space \cite{baker:2016}. In the examples shown, we use the standard forward shading based rendering pipeline included in the G3D Innovation Engine \cite{G3D17}, a commercial grade renderer.



\section{VFL Data Distributions}\label{sec:scenario}

\begin{figure*}[t]
  \centering
  \includegraphics[width=0.78\linewidth,height=8cm]{figures/scenario.pdf}
  \caption{The demonstration scenario of literature survey of \sys.}
  \label{fig:scenario}
  \vspace*{-1em}
\end{figure*}

\section{Demonstrating Literature Survey}
\label{sec:scenario}
We demonstrate how \sys~ applies to literature surveys, with emphasis on the online \surveyor.
We will showcase the functionalities of \scrapper using a video as it is time-consuming. %For the offline \scrapper, we allow attendees to define a workflow following our guidance, and upload their own papers for extracting the specified data and fetching references.

%\stitle{Offline Scrapper}.

The online \surveyor, shown in \reffig{scenario}, allows the demo attendees to explore a pre-extracted paper network containing over 50,000 papers and 160,000 references. We first look into \reffig{scenario}(a) that is the interface of \explorer~  featuring three primary canvases, metaphorically referred to as ``Past'', ``Present'', and ``Future''. Here, ``Past'' displays already explored papers, and ``Present'' shows the currently active papers for reviewing in detail, while ``Future'' highlights the immediate neighbors (i.e., references) of the active papers.
For exploring the papers, the attendee \circled{1} searches for seed papers whose titles contain “Llama3” using the \search~ module; \circled{2} then selects “The Llama 3 Herd of Models” and moves it to the ``Present'' canvas to review its details. Next, \circled{3} the attendee explores the selected paper’s references by pre-querying its neighbors. As described in \refsec{surveyor}, these neighbors are not immediately added to the canvas to avoid overwhelming the user; instead, \circled{4} the \statfilter~ module presents a histogram or table view, allowing attendees to focus on aggregated groups or order the data and finally, \circled{5} decide from the top-k papers for further exploration. By doing so, these papers are added to the ``Present''canvas, while the previously active papers move to the ``Past'' canvas. By iteratively following this workflow, attendees  can explore as many papers as needed, before proceeding to the \generator~task.

In \reffig{scenario}(b), \circled{6} attendees click to input instructions for the report, e.g., ``Please write me a related work, focusing on their challenge''. Based on this input, an LLM (QWen-Plus~\cite{tongyi} for this demo) identifies the relevant attributes and \dimension~ nodes needed for the report, which are \circled{7} displayed for user verification and possible modification. In the example, the LLM highlights the ``Challenge'' node as well as the ``title'' and ``abstract'' attributes from the selected papers. \circled{8} These data are then passed to the LLM to produce a mind map, effectively categorizing the papers according to the identified ``Challenge''.  \circled{9} Attendees can review the mind map, and \circled{10} proceed to final report generation. The final report is built from the mind map and the user’s instructions, culminating in a point-by-point narrative. Once completed, \circled{11} attendees can download the report in PDF or TeX format, complete with citations.

\section{Extension to Financial Scenarios}
\label{sec:others}
We briefly discuss applying \sys\ to two financial scenarios.
 %leveraging the same concepts of \fact\ and \dimension\ nodes.

\stitle{Company Relationship Analysis}. In this scenario, each company is treated as a \fact~ node,
and the data extracted by \inspector, such as revenues, main business areas and shareholder holdings extracted from financial reports, are represented as \dimension~ nodes.
The \navigator~ component establishes inter-company relationships by leveraging the financial or supply-chain dependencies mentioned in the reports. The generated graph can be used to identify comparable competitors, uncover hidden relationships, or assess contagion effects.
\eat{
Several compelling use cases are enabled:
\begin{itemize}
\item Identifying Comparable Competitors: By clustering or filtering companies with similar profiles (e.g., market segment, growth patterns),
investors can discover promising but less-publicized alternatives to popular stocks. This approach helps avoid the inflated valuations often
found in trending companies and can mitigate investment risk.
%\item Uncovering Hidden Relationships: With company \fact~ nodes linked by shared financial patterns, \sys~ can detect correlations
%such as highly synchronized profitability cycles or overlapping shareholder structures. Analysts can thus reveal nuanced market
%relationships or latent financial risks that may not be apparent through conventional methods.
\item Assessing Contagion Effects: In the event of a major negative incident (e.g., a significant default or bankruptcy), \sys~
highlights the interconnections that might amplify its impact. For instance, after a large real estate firm files for bankruptcy,
the tool can identify which companies share financial dependencies, supply chain relationships, or shareholder overlap—and therefore
might be exposed to heightened risk.
\end{itemize}
}

\stitle{Financial News Analysis}. In this scenario, each news article serves as a \fact~ node, while pertinent details, such as described events and stock performance indicators, can act as \dimension~ nodes. The \navigator~ builds connections among these \fact~ nodes by identifying shared symbols or overlapping financial metrics. This allows analysts to track the evolution of news stories, assess their market impact, or predict future trends based on historical patterns.

\eat{
\sys~ yields several valuable use cases:
\begin{itemize}
	\item News Clustering and Trend Analysis: Since a single news article may not provide sufficient insight on its own, \sys~ can group related articles to reveal broader patterns. Analysts can thus identify recurring themes or correlated market shifts across multiple sources, improving the reliability of forecasts and decisions.
  \item Assessing the Market Impact of Individual News Items: By referencing historically similar news events and their corresponding market responses, \sys~ enables users to evaluate how sudden developments might affect future market movements. This allows for timely, data-driven judgment on whether a new piece of news is likely to have a major or minor market impact.
\end{itemize}
}



\section{Taxonomy of Algorithms}\label{sec:system}
\section{System}\label{sec:system}
We consider systems in the form
%
\begin{subequations}\label{eq:system}
	\begin{align}
		\label{eq:system:x0}
		x(t) &= x_0(t), & t &\in (-\infty, t_0], \\
		%
		\label{eq:system:x}
		\dot x(t) &= f(x(t), z(t)), & t &\in [t_0, t_f],
	\end{align}
\end{subequations}
%
where $t \in \R$ is time, $t_0, t_f \in \R$ are the initial and final time, $x: \R \rightarrow \R^{n_x}$ is the state, and $x_0: \R \rightarrow \R^{n_x}$ is the initial state function. Furthermore, $f: \R^{n_x} \times \R^{n_z} \rightarrow \R^{n_x}$ is the right-hand side function, and the memory state, $z: \R \rightarrow \R^{n_z}$, is given by the convolution
%
\begin{subequations}\label{eq:system:delay}
	\begin{align}
		\label{eq:system:z}
		z(t) &= \int\limits_{-\infty}^t \alpha(t - s) \odot r(s) \incr s, \\
		%
		\label{eq:system:r}
		r(t) &= h(x(t)),
	\end{align}
\end{subequations}
%
where $r: \R \rightarrow \R^{n_z}$ is the delayed variable, and each element of $\alpha: \Rnn \rightarrow \Rnn^{n_z}$ is a \emph{regular} kernel (see Definition~\ref{def:regular:kernel}). Furthermore, $h: \R^{n_x} \rightarrow \R^{n_z}$ is the memory function. We assume that $f$ and $h$ are differentiable in their arguments, and we refer to the paper by Ponosov et al.~\cite[Thm.~1]{Ponosov:etal:2004} for more details on the existence and uniqueness of solutions to the initial value problem~\eqref{eq:system}--\eqref{eq:system:delay}. See also the book by Hale and Lunel~\cite{Hale:Lunel:1993}.
%
\begin{definition}\label{def:regular:kernel}
	A scalar-valued kernel, $\alpha: \Rnn \rightarrow \Rnn$, is \emph{regular} if it satisfies the following properties.
	%
	\begin{enumerate}
		\item It is non-negative and bounded, i.e., $0 \leq \alpha(t) \leq K$ for all $t \in \Rnn$ and for some finite $K \in \Rp$.
		%
		\item It is continuous, i.e., for all $\epsilon \in \Rp$ and $t \in \Rnn$, there exists a $\delta \in \Rp$ such that $|\alpha(s) - \alpha(t)| < \epsilon$ for all $s \in \Rnn$ satisfying $|s - t| < \delta$.
		%
		\item It is normalized such that
	\end{enumerate}
	%
	\begin{align}\label{eq:kernel:normalization}
		\int\limits_0^\infty \alpha(t) \incr t &= 1.
	\end{align}
\end{definition}
%
For a given system of DDEs with distributed time delays, each element of $\alpha$ may not satisfy~\eqref{eq:kernel:normalization}. However, as they are assumed to be nonzero and non-negative, it is straightforward to normalize them. Next, we present a few well-known corollaries about the steady states of~\eqref{eq:system:x}--\eqref{eq:system:delay} and their stability.
%
\begin{corollary}\label{thm:steady:state}
	A state $\bar x \in \R^{n_x}$ is a steady state of the system~\eqref{eq:system:x}--\eqref{eq:system:delay} if
	%
	\begin{align}\label{eq:steady:state}
		0 &= f(\bar x, \bar z), &
		\bar z &= \bar r = h(\bar x).
	\end{align}
\end{corollary}

\begin{proof}
	In steady state, $x(t) = \bar x$ for all $t$. Consequently, $r(t) = \bar r = h(\bar x)$ and
	%
	\begin{align}
		z(t)
		&= \int_{-\infty}^t \alpha(t - s) \odot \bar r \incr s
		 = \int_{-\infty}^t \alpha(t - s) \incr s \odot \bar r
		 = \bar r,
	\end{align}
	%
	for all $t$, where we have used the property~\eqref{eq:kernel:normalization} of each element of $\alpha$.
\end{proof}
%
\begin{corollary}\label{thm:stability}
	The system~\eqref{eq:system:x}--\eqref{eq:system:delay} is locally asymptotically stable around a steady state, $\bar x$, satisfying~\eqref{eq:steady:state} if $\real \lambda < 0$ for all $\lambda \in \C$ that satisfy the characteristic equation
	%
	\begin{align}\label{eq:characteristic:equation}
		\det\left(F - \lambda I + G \int_0^\infty e^{-\lambda s} \diag \alpha(s) \incr s H\right) = 0,
	\end{align}
	%
	where $I \in \R^{n_x \times n_x}$ is an identity matrix.
	%
	The matrices $F \in \R^{n_x \times n_x}$, $G \in \R^{n_x \times n_z}$, and $H \in \R^{n_z \times n_x}$ are the Jacobians of the right-hand side function and the delay function evaluated in the steady state:
	%
	\begin{align}\label{eq:jacobians}
		F &= \pdiff{f}{x}(\bar x, \bar z), &
		G &= \pdiff{f}{z}(\bar x, \bar z), &
		H &= \pdiff{h}{x}(\bar x).
	\end{align}
	%
\end{corollary}

\begin{proof}
	The linearized system corresponding to~\eqref{eq:system:x}--\eqref{eq:system:delay} describes the evolution of the deviation variable $X: \R \rightarrow \R^{n_x}$:
	%
	\begin{align}\label{eq:linearized:system}
		\dot X(t) &= F X(t) + G \int_{-\infty}^t \alpha(t - s) \odot H X(s) \incr s, &
		X(t) &= x(t) - \bar x.
	\end{align}
	%
	See, e.g., \cite{Cushing:1975, Cushing:1977, Miller:1972} for proofs of the condition~\eqref{eq:characteristic:equation} for asymptotic stability of the linearized system in~\eqref{eq:linearized:system}.
\end{proof}


\section{Discussion}\label{sec:future}
% !TEX root = main.tex

\section{Future research}\label{sec:future}
Below we list a few research questions, which we find interesting and
particularly promising directions after our contribution.

\para{Exact complexity for $3$-VASS}
We have shown that shortest paths in binary $3$-VASS are of at most triply-exponential length.
It is tempting to conjecture that actually the upper bound for the length of the paths is shorter,
at most doubly-exponential. We conjecture so and leave this conjecture to the future research.

\para{Example of a $3$-VASS with doubly-exponential path}
We have shown that shortest paths in binary $3$-VASS are of at most triple-exponential length.
However, currently we still do not know any example in which even a path of doubly-exponential length is needed,
it might be that paths of exponential length are sufficient leading to \pspace-completeness for binary $3$-VASS.
It would be very interesting to find an example of a binary $3$-VASS with shortest path between two configurations
being doubly exponential. An example of binary $4$-VASS of doubly-exponential shortest path is known (see Section 5 in~\cite{DBLP:conf/concur/Czerwinski0LLM20}). Maybe some modification of this $4$-VASS would allow to design a $3$-VASS with similar properties.

\para{Reachability for $d$-VASS with $d \geq 4$}
It is a natural question whether our techniques extend to higher dimensions.
The answer is: possibly yes, but we would need a few other structural results for $3$-VASS
to make a similar approach to $4$-VASS possible. In the proof of Lemma~\ref{lem:main} we do not only
use $2$-VASS reachability as a black box, but we use a deep understanding of the reachability relation in $2$-VASS
from~\cite{DBLP:conf/focs/0001CMOSW24}. Probably a similar understanding of the reachability relation for $3$-VASS would be needed
to advance understanding of $4$-VASS along our lines. 

In general it is very interesting to determine the complexity of the reachability problem for $d$-VASS.
We have excluded that for each $d \geq 3$ the problem is $\F_d$-completely, but it is still possible that
the problem is $\F_{d-C}$-complete for some constant $C \in \N$ and $d$ big enough.
Recall that in~\cite{DBLP:conf/fsttcs/CzerwinskiJ0LO23}
it was shown that the reachability problem for $(2d+4)$-VASS is $\F_d$-hard for any $d \geq 3$ and this
is the best currently known lower bound for arbitrary dimension.
Therefore the other natural possibility is that the reachability problem for $(2d+C)$-VASS is $\F_d$-complete for some
constant $C \in \N$. 

\para{Applications of the approximation technique}
Another natural research direction is to search for other applications of the technique of approximating the reachability sets,
which allows to lower the complexity down, below the size of the reachability set.
One particular case, which seems to be prone to such techniques is the $2$-VASS with some number of $\Z$-counters, namely counters, which can take values below zero.
The best complexity lower bound for the reachability problem in this model is \pspace-hardness inherited from~\cite{BlondinFGHM15},
while the best upper bound is Ackermann membership inherited from VASS reachability~\cite{LS19}.
The reachability sets for that systems are not necessarily semilinear.
This disqualifies most of the techniques relying on the semilinearity of reachability sets, but our techniques
seem to be promising for that model.





\section{Conclusion}\label{sec:conclusion}
\section*{Conclusion}
This paper aims to enhance our understanding of the computational complexity of computing various Shapley value variants. We found that for various ML models --- including decision trees, regression tree ensembles, weighted automata, and linear regression --- both local and global interventional and baseline SHAP can be computed in polynomial time under HMM modeled distributions. This extends popular algorithms, such as TreeSHAP, beyond their empirical distributional scope. We also establish strict complexity gaps between the various SHAP variants (baseline, interventional, and conditional) and prove the intractability of computing SHAP for tree ensembles and neural networks in simplified scenarios. Overall, we present SHAP as a versatile framework whose complexity depends on four key factors: \begin{inparaenum}[(i)] \item model type, \item SHAP variant, \item distribution modeling approach, \item and local vs. global explanations\end{inparaenum}. We believe this perspective provides deeper insight into the computational complexity of SHAP, paving the way for future work.




%We believe that our framework provides a more intricate understanding of SHAP computation complexity across different models, distributions, and variants, paving the way for further research.

Our work opens promising directions for future research. First, expanding our computational analysis to other SHAP-related metrics, such as asymmetric SHAP~\citep{frye20} and SAGE~\citep{covert2020understanding}, would be valuable. Additionally, we aim to explore more expressive distribution classes and relaxed assumptions beyond those in Section \ref{sec:tractable} while maintaining tractable SHAP computation. Finally, when exact computation is intractable (Section \ref{sec:intractable}), investigating the approximability of SHAP metrics through approximation and parameterized complexity theory~\citep{downey2012parameterized} is an important direction.

%Our work opens several promising avenues for future research on the computational properties of explainable AI methods, with a particular focus on SHAP. First, it would be interesting to broaden the computational analysis conducted in this work to include other popular SHAP-related metrics in the literature, such as asymmetric SHAP \cite{frye20} and SAGE \cite{covert2020understanding}. Also, in the future, we aim to explore more expressive distribution classes and relaxed distributional assumptions—extending beyond those examined in Section \ref{sec:tractable} —that still yield tractable SHAP computation. Finally, when exact computation proves intractable (Section \ref{sec:intractable}), it is worthwhile to theoretically investigate the question of the approximability of computing the SHAP metrics across various configurations, through the lens of approximation and parametrized complexity theory \cite{arora2009computational}.

%This paper aims to deepen our understanding of the computational complexity involved in obtaining different Shapley value variants. We found that for a variety of ML models, including decision trees, tree ensembles for regression, weighted automata, and linear regression models — computing both local and global interventional and baseline SHAP can be done in polynomial time when distributions are modeled by HMMs. This extends the distributional scope of popular algorithms like TreeSHAP, which is limited to empirical distributions. Additionally, we demonstrate a strict complexity gap between SHAP variants, showing that interventional and baseline SHAP can be strictly easier to compute than conditional SHAP. Despite these positive results, we uncovered intractability for various SHAP variants in neural networks and tree ensembles. Finally, we provided generalized complexity relations across SHAP variants. We believe that our framework offers a deeper understanding of the complexity involved in computing SHAP across various variants, models, distributions, as well as in both local and global computations, laying the groundwork for future research.







\bibliographystyle{named}
\bibliography{references-short}

\end{document}
