\begin{figure*}[t!]
    \centering
    \begin{subfigure}[t]{0.24\textwidth}
        \centering
        \includegraphics[width=\textwidth]{fig/overlap_graph_seed0.png}
        \caption{Visualization of 1,000 parties}
        \label{fig:overlap-graph}
    \end{subfigure}
    \hfill
    \begin{subfigure}[t]{0.24\textwidth}
        \centering
        \includegraphics[width=\textwidth]{fig/overlap_distribution.png}
        \caption{Feature overlap ratios}
        \label{fig:feature-overlap}
    \end{subfigure}
    \hfill
    \begin{subfigure}[t]{0.24\textwidth}
        \centering
        \includegraphics[width=\textwidth]{fig/feature_balance_ratio.png}
        \caption{Feature balance ratios}
        \label{fig:feature-skew}
    \end{subfigure}
    \hfill
    \begin{subfigure}[t]{0.24\textwidth}
        \centering
        \includegraphics[width=\textwidth]{fig/matched_ratio_hist.png}
        \caption{Record matched ratios}
        \label{fig:matched-ratio}
    \end{subfigure}    
    \caption{Analysis and visualization of feature distributions and overlaps across real-world databases}
    \label{fig:feature-analysis}
\end{figure*}


This section examines real-world VFL data distributions using the WikiDBs corpus \cite{vogel2024wikidbs}. We begin by introducing the basic settings in Section~\ref{subsec:settings}, define the essential data properties in Section~\ref{subsec:definitions}, and present our results and findings in Section~\ref{subsec:findings}. 

\subsection{Settings}\label{subsec:settings}
\paragraph{Dataset.}
WikiDBs~\cite{vogel2024wikidbs} is currently the largest publicly available corpus of relational databases, extracted from real-world Wikidata. The details of WikiDBs are shown in Table~\ref{tab:wikidbs}. It covers a wide range of domains, such as clinical, finance, sports, etc. Many of these databases (e.g. \texttt{ucl\_clinical\_research\_trials} and \texttt{apnea\_clinical\_research\_db}) are correlated and can be considered potential scenarios for VFL. Each database can be considered a VFL \textit{party}, with pairs of parties sharing correlated features representing potential VFL pairs. 

\begin{table}[h]
    \centering
    \small
    \caption{Statistics of WikiDBs}
    \label{tab:wikidbs}
    \begin{tabular}{cccc}
        \toprule
        \multicolumn{2}{c}{\textbf{Database}} & \multicolumn{2}{c}{\textbf{Table}} \\
        \cmidrule(lr){1-2} \cmidrule(lr){3-4}
        \#Databases & Total \#Tables & Mean \#Rows & Mean \#Cols \\
        \midrule
        100K & 1.6M & 118 & 52.7 \\
        \bottomrule
    \end{tabular}
\end{table}


\paragraph{Configuration.} In our analysis, we focus on two-party VFL, as it not only reflects the characteristics of multi-party VFL but also represents the most common VFL scenario in practice~\cite{liu2024vertical}. Due to the \(O(n^2)\) complexity of evaluating all pairs, we randomly sampled 1,000 databases, generating 1,000,000 pairs from their Cartesian square. Experiments are repeated across 10 random seeds to reduce variance, with mean and variance reported. The results show consistent subset characteristics, demonstrating the robustness of our analysis.

\subsection{Definitions}\label{subsec:definitions}

This subsection delineates the key properties of VFL data distribution. We begin by defining which pairs of databases are considered potential VFL pairs. At the feature level, we assess their overlap and balance by introducing the \textit{feature overlap ratio} and the \textit{feature balance ratio}. In terms of instances, we quantify the proportion of records that can be matched between two parties using shared features, defining this as the \textit{record matched ratio}. The detailed definitions are as follows.

\paragraph{Potential VFL Pairs.} Potential VFL pairs are defined based on database graph connectivity. In this graph, nodes represent databases, and edges connect nodes with tables sharing at least one column. Two databases are considered potential VFL pairs if they belong to the same connected component, indicating they are related through join operations along the connected path.

\paragraph{Record Matched Ratio.} To evaluate how precisely two VFL parties can be aligned based on shared features, we define the \textit{record matched ratio}. For a given table pair, this ratio measures the fraction of records in each table that identically appear in the other. It ranges from $[0, 1]$ and reflects the proportion of records that can be precisely aligned.


\paragraph{Feature Balance Ratio.} To assess the balance in the number of features between two VFL parties, we define the \textit{feature balance ratio} as the small number of columns between the two databases to the large number of columns. The feature balance ratio ranges from $[0, 1]$, where a value of 1 indicates two parties have the same number of features.


\begin{table}[ht]
    \centering
    \small
    \caption{Properties of the database graph across 10 subsets}
    \label{tab:vfl-properties}
    \setlength{\tabcolsep}{3pt}
    \begin{tabular}{ccccc}
        \toprule
        \textbf{Metric} & \textbf{Mean} & \textbf{Std} & \textbf{Min} & \textbf{Max} \\
        \midrule
        \#Connected Components & 3.1 & 0.3 & 3 & 4 \\
        \midrule
       \makecell{Ratio of Non-Neighboring Nodes \\ in Potential VFL Pairs} & 25.4\% & 1.3\% & 23.7\% & 27.7\% \\
        \bottomrule
    \end{tabular}
\end{table}


\begin{table}[ht]
    \centering
    \small
    \caption{The ratio of different VFL data distributions}
    \label{tab:record-matching}
    \setlength{\tabcolsep}{4pt}
    \begin{tabular}{cccc}
        \toprule
        \textbf{VFL Type} & \textbf{Features} & \textbf{Records} & \textbf{Ratio} \\
        \midrule
        Latent VFL & Zero overlap & Zero match & 25.4\% \\
        \midrule
        Fuzzy VFL & Non-zero overlap & Zero match & 70.9\% \\
        Semi-precise VFL & Non-zero overlap & Partial match & 3.5\% \\
        Precise VFL & Non-zero overlap & Full match & 0.2\% \\
        \bottomrule
    \end{tabular}
\end{table}

\subsection{Results and Findings}\label{subsec:findings}


In this subsection, we present our results in Table~\ref{tab:vfl-properties} and Table~\ref{tab:record-matching}, and Figure~\ref{fig:feature-analysis}, followed by four key findings from our analysis of real-world databases.

\begin{tcolorbox}[colback=white, coltitle=black, boxsep=0mm, left=1mm, right=1mm, top=1.5mm, bottom=1mm]
    \textbf{Finding 1}: VFL has a large number of potential real-world applications.
\end{tcolorbox}
To analyze the connectivity between databases, we constructed a graph where each node represents a database, each edge represents shared features between databases, and each color corresponds to a connected component, as visualized in Figure~\ref{fig:overlap-graph}. Table~\ref{tab:vfl-properties} further summarizes the key properties of this graph, showing that the graph of 1,000 parties contains only 3 to 4 connected components. The high connectivity indicated by both the table and figure suggests that VFL has a wide range of potential real-world applications.

\begin{tcolorbox}[colback=white, coltitle=black, boxsep=0mm, left=1mm, right=1mm, top=1.5mm, bottom=1mm]
\textbf{Finding 2}: A substantial portion (25.4\%) of potential VFL pairs have no overlapping features.
\end{tcolorbox}
To analyze feature overlap ratios between potential VFL pairs, we present the distribution of these ratios in Figure~\ref{fig:feature-overlap} and detail the proportion of pairs with zero feature overlap in Table~\ref{tab:vfl-properties}.  Table~\ref{tab:vfl-properties} reveals that among the potential VFL pairs, approximately one-quarter have zero feature overlap. This indicates that many real-world VFL scenarios require methods that can handle latent relationships without relying on shared features. We define a VFL scenario where two parties have no overlapping features but are still correlated as \textit{latent VFL}.

\begin{tcolorbox}[colback=white, coltitle=black, boxsep=0mm, left=1mm, right=1mm, top=1.5mm, bottom=1mm]
\textbf{Finding 3}: Only a small fraction of potential VFL pairs (0.2\%) can be precisely matched.
\end{tcolorbox}
To evaluate the feasibility of record matching in potential VFL pairs, we analyzed the record matched ratios across database pairs. Figure~\ref{fig:matched-ratio} illustrates the distribution of these ratios, and Table~\ref{tab:record-matching} details the ratios for various VFL data distributions. The results reveal that 70.9\% of potential VFL pairs have zero precise record matches, with only 0.2\% achieving full alignment. This indicates that existing VFL algorithms, which assume fully and precisely matched data records (\textit{precise VFL}), may be inadequate for most real-world applications. In practice, there is a need for algorithms capable of handling partial record matching (\textit{semi-precise VFL}) or even scenarios with no record matching (\textit{fuzzy VFL}). This finding aligns with existing VFL studies~\cite{wu2022coupled,nock2021impact,he2024hybrid}, which highlight the rarity of precise matching.

\begin{tcolorbox}[colback=white, coltitle=black, boxsep=0mm, left=1mm, right=1mm, top=1.5mm, bottom=1mm]
\textbf{Finding 4}: Feature distribution across parties exhibits significant imbalance.
\end{tcolorbox}
To evaluate the balance of features between potential VFL pairs, we present the distribution of feature balance ratio in Figure~\ref{fig:feature-skew}. We observe that most database pairs have highly imbalanced feature counts. This imbalance poses challenges for existing VFL algorithms, which typically assume relatively balanced feature distributions across parties for optimal performance. This finding aligns with the observation in~\cite{wu2023vertibench} that real VFL datasets are highly imbalanced w.r.t. feature importance.Imbalanced VFL is defined as a feature balance ratio below 0.5 (66.49\%), while balanced VFL has a ratio above 0.5 (33.51\%).
