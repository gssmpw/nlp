\blfootnote{Preprint. Under Review.}


With the exhaustion of high-quality publicly available data \cite{villalobos2022will} and the increasing stringency of privacy regulations such as the European Union’s General Data Protection Regulation (GDPR)\footnote{https://gdpr-info.eu/}, federated learning \cite{mcmahan2017communication} has emerged as a promising solution for training machine learning models across multiple parties without sharing their sensitive data. Federated learning is typically classified into horizontal federated learning (HFL) and vertical federated learning (VFL) based on the data distribution among the parties \cite{yang2019federated}. In HFL, parties share the same feature set but have different samples, whereas in VFL, parties possess different feature sets but the same sample set.

In recent years, VFL has gained significant interest in industrial collaborations, as companies holding different features often complement and benefit from each other \cite{li2020review}. These companies typically operate in different domains, such as e-commerce and banking, which reduces conflicts of interest and fosters a greater willingness to collaborate. Especially in the era of large language models, collaboration between private domains with different features can create comprehensive foundation models \cite{zheng2023input}. Driven by high demand in industrial applications, many VFL systems have emerged \cite{liu2021fate,li2023fedtree}. However, despite this growing interest, very few VFL frameworks or systems have been deployed in real-world applications, as noted by recent studies \cite{khan2022vertical,ye2024vertical}. This gap between research and application is crucial for the direction of future VFL research.

Existing VFL surveys primarily emphasize algorithmic perspectives while overlooking real-world data distributions in practical applications. Surveys such as \cite{liu2024vertical,khan2022vertical,cui2024survey,ye2024vertical} present taxonomies of VFL algorithms, categorizing contributions by performance, efficiency, communication, and privacy. Additionally, surveys including \cite{liu2024label,li2023vertical,yu2024survey} focus on privacy issues, exploring attack and defense mechanisms. Although recent research \cite{nock2021impact,wu2022coupled} highlights that real-world VFL data is less ideal than current experimental settings, there remains a lack of systematic investigation into real-world VFL data distributions. This gap hinders the effective bridging between theoretical research and practical applications.

In this survey, we explore the data distribution of potential VFL applications using a recent real-world database corpus, WikiDBs \cite{vogel2024wikidbs}, which comprises 100,000 databases and 1.6 million tables with diverse features. Treating each database as being held by a separate party, we examine cross-party data distributions, identifying four key findings and proposing a data-oriented taxonomy of VFL algorithms. Our review reveals a significant gap between current algorithms and real-world data distributions. We also highlight the challenges in bridging research and practical deployment, offering insights for future research directions.

