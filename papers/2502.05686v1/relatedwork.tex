\section{Literature Review}
\subsection{Location Data Tracking}
Location data can be collected using GPS sensors/Bluetooth beacons on mobile phone devices or smart-watches and using WiFi or IP addresses when surfing the web. Current app development toolkits offer the possibility of combining location data from multiple sensors to get better accuracy: for example, the Fused location Provider API \cite{googleFusedLocation} in Android development or the CLLocationManager \cite{appleCCLocation} in iOS development. 
In addition, there exist several third-party Software development Kits (SDKs) (e.g., X-Mode, SafeGraph) that help app developers, tech companies and data brokers and aggregators collect, visualize and track location data \cite{vox_privacy}.  

With data from data brokers and aggregators, 
many researchers have shown that location data characterizing the places visited by a person as well as their trajectories can be useful in a plethora of settings such as modeling the spreading of a pandemic through tracing apps \cite{contact_tracing_apps}, 
inferring socio-economic indicators \cite{hong2016topic,estabaan_urban_dynamics}, health outcomes \cite{garcia_you_are_what_you_eat}, urban development and disaster risk \cite{equalizer_social_infra,hurricane_trd}, 
targeted marketing and advertising \cite{location_advertising},
understanding migration patterns \cite{hong2019characterization},
improving public transit \cite{ma2014development}, 
or modeling mobility patterns for public safety \cite{wu2022enhancing}.

At the same time, there has been a lot of work focused on privacy preserving methods for location data. 
Primault et. al. \cite{longroadcomp} extensively outline the techniques for sharing location data in a privacy preserving manner either via anonymization or obfuscation approaches. While
obfuscation techniques directly change the location of a user to protect their privacy, anonymization mechanisms attempt to hide the actual location from an attacker. 
In this paper, we focus on obfuscation techniques, not on k-anonymization approaches, since obfuscation is the most common method used by data brokers, data aggregators and the researchers and analysts acquiring and analyzing location data from these companies.  


%Data brokers do not explicitly specify these Location Privacy preservation mechanisms (LPPM) in their terms and services. Also, the field is evolving rapidly and new algorithms for LPPM are being developed with better representability of annonimyzed data. Thus, location privacy perceptions need to be understood in the context of features that would be derived from the data and not simply in the context of type of LPPM.


\subsection{Privacy Perceptions for Location data}
Duckham et. al. \cite{ducham} define location privacy as "special type of information
 privacy which concerns the claim of individuals to determine for themselves when,
 how, and to what extent location information about them is communicated to others".
 Our research builds on prior work investigating privacy perceptions around sharing personal location data in various contexts \cite{martin,colombia_privacy_study,ICADataCollection}. Prior research has looked into the effects of different variables on the willingness of individuals to share their personal location data including the role of (1) Actors, defined as individuals accessing the location data \cite{martin,ICADataCollection,cross_platform_social_media_data,employee_surveillance} (2) Purposes, defined as how the location data is going to be used \cite{martin, mobile_data_thesis, cross_platform_social_media_data} (3) Time, defined as the duration of the access to location data \cite{martin,mobile_data_thesis} (4) Age  \cite{colombia_privacy_study,location_privacy_students,privacy_older_adults} (5) Privacy attitudes \cite{personality_location,personality_privacy_perception_2016}, and  (6) Socio-cultural aspects \cite{location_privacy_SEM,mobile_data_thesis,privacy_covid,surveillance_covid}.

These works have shown that actors like the FBI, commercial entities, employers and data brokers are negatively perceived while local government and social media companies generate more positive perceptions when collecting and using location data. The duration of the location data collection as well as the purposes of using location data have also been associated with varying levels of comfort when sharing that data. For example, while the identification of sexual orientation or political views are perceived negatively, purposes related to public health or public safety are perceived more positively.

%Situations like longer duration of location tracking and objectives like identifying sexual orientation, political views and social circle of an individual are perceived as uncomfortable when compared to identifying places people visit, assessing their health, reducing spread of diseases, improving traffic, etc. Specific cases like comfortableness in sharing location data on social media platforms with images, posts and tweets was also observed. 

%\textcolor{yellow}{Naman, try to summarize here some of the findings e.g., certain actors are associate to more risk etc. Not two long, try to summarize}


There is also research exploring the use of location data for specific settings such as surveillance  \cite{video_survellience,vox_privacy}, location tracking for advertisements \cite{location_advertisement}, 
or location tracking for pandemic contexts. 
%They aim to understand people's nuanced perceptions of physical location tracking for different purposes.
Surveillance applications have been associated with some level of comfort in human-centered contexts like disaster recovery.
On the other hand, research during COVID-19 found strong opposition towards contact tracing apps \cite{privacy_covid,surveillance_covid,privacy_covid_2}, with over 50\% of the participants in both studies being unlikely to install such apps due to concerns around government intrusion. 

Finally, prior work has also examined how monetary incentives influence participants' willingness to share their location data for specialized research studies or market research \cite{incentive_participate_1, incentive_participate_2}; and the relationship between personality traits and privacy perceptions \cite{personality_location, personality_privacy_perception_2016}. Participants' positive perception of the service collecting their data, higher incentives, agreeableness, conscientiousness, and openness to new experiences were also linked positively to comfort in sharing location data.  
%\textcolor{yellow}{here again, add one or two sentences explaining findings}

Building on these findings, we extend the state of the art by (1) evaluating user perceptions with respect to trajectory-based features extracted from location data, beyond current analyses focused on visits to points of interest (POI), (2) evaluating user privacy perceptions when using obfuscated location data vs. detailed (raw) location data, and (3) analyzing the effect of race, ethnicity and education on privacy perceptions with respect to trajectory and POI visit location data. 

\subsection{Predicting Privacy Preferences}
% There is no prior work focused on inferring levels of comfort with respect to location data. 
Privacy fatigue is an important issue highlighted in research. This fatigue stems from factors like ubiquitous collection of personal data, scarcity of alternatives and additional permission requirements people need to go through \cite{tradeoff,privacy_exhaustion,privacy_risk_awareness}. 

Prediction models have been developed to explore people's perceptions around data collection and sharing like \cite{privacy_prediction,privacy_prediction2,privacy_prediction3,naeini,location_advertising,martin}. In these paper, the authors model people's perceptions of comfort in sharing location data as a multi-class classification or regression problem and model the perceptions around different actors accessing location data for different purposes. Eg: \cite{naeini} models people's perceptions in sharing IOT data as a binary classification task (predicting allow/deny); reporting accuracies of 77.6\% (0.70 recall) for a multi-class, 3-point Likert classification task, and 
81\% (0.78 recall) for the binary classification.
%\textcolor{yellow}{Naman, add one sentence with accuracy for these two tasks}
In another study, researchers created binary classifiers to determine privacy attitudes with regards to data sharing in several contexts including sharing biological data or personally identifying information (PII)\cite{contextualLabel}; achieving a mean recall of 0.46 for prediction of privacy concerns from contextual labels. 

In this paper, we explore binary and multi-class approaches, similarly to prior works.

\iffalse 
\textbf{Survey Design.} Research studies on privacy perceptions typically employ factorial vignette approaches to understand how different scenarios affect an individual's willingness to share location data.
Privacy risk perception studies highlight that impersonal, less detailed,
\textit{abstract} risk scenarios are often perceived as less risky\cite{privacy_risk_awareness}. 
Given that risk is a major factor in determining levels of comfort with sharing location data, vignette questions that present location data in abstract ways may underestimate people's privacy boundaries and perceptions. 
Hence, in our survey, we make an effort to explain to the survey participant, the types of location data we refer to, by providing extensive details about the location features and about what they represent.  
Interestingly, some privacy perception studies demonstrate that visualizing abstract data can help alleviate privacy risk concerns and increase people's comfort with data sharing \cite{vizbetter}. This suggests that the way data is presented can significantly impact privacy perceptions.
For that reason, in our survey, we accompany each vignette not only with extensive explanations, but also with map-based visualization to convey better location data information.
\fi 

%In our study, we have chosen to use map-based visualizations to convey feature information. This decision is supported by previous research demonstrating that visualizations can more effectively inform users about risk implications of data sharing \cite{vizbetter}. 

% \subsection{Predicting Privacy Preferences}
% % There is no prior work focused on inferring levels of comfort with respect to location data. 
% Privacy fatigue is an important issue highlighted in research. This fatigue stems from factors like ubiquitous collection of personal data, scarcity of alternatives and additional permission requirements people need to go through. \textcolor{red}{NAMAN, Add the citations for those fatigue papers}. 

% Prediction models have been developed to explore people's perceptions around data collection and sharing by IoT devices \cite{naeini}. In this paper, the authors model people's perceptions of comfort in IoT data collection as a multi-class classification problem and model the perceptions around IoT data sharing as a binary classification task (predicting allow/deny); reporting accuracies of 
% 77.6\% (0.70 recall) for a multi-class, 3-point Likert classification task, and 
% 81\% (0.78 recall) for the binary classification.
% %\textcolor{yellow}{Naman, add one sentence with accuracy for these two tasks}
% In another study, 
% researchers created binary classifiers to determine privacy attitudes with regards to data sharing in several contexts including sharing biological data or personally identifying information (PII)\cite{contextualLabel}; achieving a mean recall of 0.46 for prediction of privacy concerns from contextual labels. 


% % \textcolor{red}{NAMAN, add more papers that work on perception prediction from somewhere  }


% % In this paper, we explore binary and multi-class approaches, similarly to prior work. 
% %\textcolor{yellow}{Naman, add one sentence with accuracy}

% %We explore literature on modeling perceptions using the collected data. While insights can be gathered around which actors/features/purposes/attitude/awareness are linked to higher comfort, there are many practical benefits of having a classification/regression model for predicting comfortableness of a group of interest. To that effect, we explore literature on modeling privacy perceptions as classification/regression tasks. 
% %Indices for general privacy perceptions developed by Westin (Westin Privacy index) are not enough for understanding context specific attitudes and perceptions \cite{westininsufficient}. Contextual integrity framework surveys need to be designed carefully to incorporate the context and convey the correct information for capturing true perceptions of surveys participants \cite{martin},\cite{CISmartHome}. In this study, we propose to create a classification model which can classify privacy perceptions of location data sharing in specific contexts.
% %Similar approach ahs been successfully developed previously. 

% Features list
\begin{table*}[ht!]
\centering
\footnotesize
\caption{Location Features and Abbreviations. We cite examples of recent studies where such features are used in the "feature cluster" column }
\begin{tabularx}{\textwidth}{|l|p{0.25\linewidth}|p{0.15\linewidth}|p{0.35\linewidth}|}
\cline{1-4} 
\multicolumn{1}{|l|}{Feature Cluster} & Feature & Abbreviation & Description \\ 
\cline{1-4} 
& Your inferred home location& Home location (Detailed)
& Home location as a point on map (Eg. fig. \ref{fig:homeD}) \\ \cline{2-4} 
 & Your inferred home location represented as   a census tract& Home location (Obfuscated)
& Home's county location (Eg., Fig. \ref{fig:homePP})\\ \cline{2-4} 
& Your inferred work location& Work location (Detailed)
& Work location as a point on map (Eg., Fig. \ref{fig:workD}) \\ \cline{2-4} 
\multirow{-4}{*}{ \shortstack[l]{Home + Work \cite{homeworklocation,home_work_poi_pipeline}}} & Your inferred work location represented as a census tract & Work location (Obfuscated)
& Work's county location (Eg., Fig. \ref{fig:workPP}) \\ 
\cline{1-4} 
& The places you visit & Places you visit (Detailed)
& Chart indicating types and frequency of places visited. Includes a map depicting the detailed locations of the places visited (Eg., Figs. \ref{fig:POVD}, \ref{fig:POVPP}) \\ \cline{2-4} 
& The types of places you visit & Places you visit (Obfuscated)
& Chart indicating types and frequency of places visited (Eg., Fig. \ref{fig:POVPP})\\ \cline{2-4} 
\multirow{-3}{*}{Places Visited\cite{PlacesofVisit,home_work_poi_pipeline,poi_2}}                     & The geographical area where you spend most   of your time & Area you spent most of your time (Obfuscated)
& Map displaying radius of gyration (Eg., Fig. \ref{fig:ROG})\\ \cline{1-4} 

& The modes of transportation you use, with what frequency and their corresponding routes & Modes of transportation (Detailed)
& Chart indicating frequency and types of mode of transport used. It also includes a map with lines indicating detailed routes for each frequent mode of transport (Eg., Fig. \ref{fig:modesD}, \ref{fig:modesPP}) \\ \cline{2-4} 
\multirow{-2}{*}{Transportation \cite{modesOfTransport_1,transport_NZ,home_work_poi_pipeline} }& The modes of transportation you use and   with what frequency & Modes of transportation (Obfuscated)
& Chart indicating frequency and types of mode of transport used (Eg., fig \ref{fig:modesPP})\\ \cline{1-4} 

& Your most frequent trips & Most frequent trips (Detailed)
& Map with frequently taken routes. Routes are detailed showing start and end locations as well as the in-between GPS points visited (Eg., Fig. \ref{fig:freqTD}) \\ \cline{2-4} 
& Your least frequent trips& Least frequent trips (Detailed)
& Map with infrequently taken routes. Routes show start and end locations as well as all the GPS points visited (Eg., Fig. \ref{fig:leastFreqT})\\ \cline{2-4} 
& Your most frequent types of trips & Most frequent type of trips (Detailed)
& Map with the different frequent trips taken and the inferred trip purpose by type of destination (Eg., Fig. \ref{fig:freqtypetrip})\\ \cline{2-4} 
& Your most frequent trips represented by their origin and destination census tracts and connected by a line& Most frequent trips between counties (Obfuscated)
& Map with frequently taken routes. It protects privacy by showing the start and end points as counties, and the frequently taken routes as straight lines between counties instead of detailed GPS. (Eg., Fig. \ref{fig:FreqOD}) \\ \cline{2-4} 
\multirow{-5}{*}{Movement \cite{modesOfTransport_1,traj_analysis,poi_2}} & Your most frequent trips represented by their origin and destination census tracts and connected by an approximate route & Most frequent trips between counties ('Google') (Obfuscated)
& Map with frequently taken routes. It protects privacy by showing the start points, end points as counties, and the frequently taken routes as suggested by  Google Maps, instead of the actual GPS route. (Eg., Fig. \ref{fig:gmaps})           \\ \cline{1-4} 

& Your walking activity and the corresponding routes & Frequent walking activity (Detailed)
& Chart indicating frequency and duration of walks. It also includes a map with detailed GPS trajectories indicating routes for each frequent walking path (Eg., Fig. \ref{fig:walkD} and \ref{fig:walkPP})\\ \cline{2-4} 
\multirow{-2}{*}{Walking Activity \cite{walkingActivity,mode_detection_all,transport_NZ,walk_nz}}                   & Your walking activity& Frequent walking activity (Ob)& Chart indicating frequency and duration of walks with no detailed trajectories  (Eg., Fig. \ref{fig:walkPP})                          \\ \cline{1-4} 

& The foreign countries you have visited and   the duration of the visit, including all the locations where you have stayed    & International visits (Detailed)
& Chart indicating frequency,duration and location of international trips. It includes a map with detailed GPS points indicating regions visited in the foreign county (Eg., Fig. \ref{fig:internationalD})\\ \cline{2-4} 
\multirow{-2}{*}{International Trips \cite{cuebiqInternational} }                & The foreign countries you have visited, and for how long& International visits (Ob)& Chart indicating frequency, duration and location of international trips. (Eg., Fig. \ref{fig:internationalP})                          \\ \cline{1-4} 

\end{tabularx}
\label{tab:listOfFeatures}
\end{table*}