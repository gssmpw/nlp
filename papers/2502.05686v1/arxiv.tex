\PassOptionsToPackage{prologue,table,xcdraw}{xcolor}
\documentclass[letterpaper,twocolumn,10pt]{article}
\usepackage{arxiv}
%12 pages
% to be able to draw some self-contained figs
\usepackage{tikz}
\usepackage{amsmath}

% inlined bib file
\usepackage{filecontents}
\usepackage{soul}
\usepackage{url}
\usepackage{booktabs}
\usepackage{multirow}
\usepackage{array}
\newcolumntype{L}{>{\raggedright\arraybackslash}X}
\usepackage{pdflscape}
\usepackage{makecell}
\usepackage{tikz}
\usepackage{xurl}
\usepackage{tabularx}
\usepackage{caption}
\usepackage{subcaption}
\usetikzlibrary{arrows.meta}
\usepackage{breakcites}
\usepackage{dirtytalk}
\usepackage{longtable}
\usepackage{makecell}
\usepackage{enumitem}


\NewDocumentCommand{\rot}{O{90} O{1em} m}{\makebox[#2][l]{\rotatebox{#1}{#3}}}


\AtBeginDocument{%
  \providecommand\BibTeX{{%
    \normalfont B\kern-0.5em{\scshape i\kern-0.25em b}\kern-0.8em\TeX}}}


%-------------------------------------------------------------------------------


%-------------------------------------------------------------------------------


\author{
  {\rm Naman Awasthi}\\
  University Of Maryland, College Park
  \and
  {\rm Saad Abrar}\\
  University Of Maryland, College Park
  \and
  {\rm Daniel Smolyak}\\
  University Of Maryland, College Park
  \and
  {\rm Vanessa Frias-Martinez}\\
  University Of Maryland, College Park
}

\newcommand{\plainauthor}{Author Name}
\begin{document}
%-------------------------------------------------------------------------------

%don't want date printed

% make title bold and 14 pt font (Latex default is non-bold, 16 pt)
\title{From "I have nothing to hide" to "It looks like stalking": Measuring Americans' Level of Comfort with Individual Mobility Features Extracted from Location Data}
%when specific mobility features are involved}
% if you leave this blank it will default to a possibly ugly attempt 
% to make the contents of the \author command below into a string
\maketitle
% \thecopyright

\begin{abstract}
Location data collection has become widespread with smart phones becoming ubiquitous. Smart phone apps often collect precise location data from users by offering \textit{free} services, and then monetize it for advertising and marketing purposes. 
While major tech companies only sell aggregate behaviors for marketing purposes; data aggregators and data brokers offer access to individual location data. 
Some data brokers and aggregators have certain rules in place to preserve privacy; and the FTC has also started to vigorously regulate consumer privacy for location data. 
In this paper, we present an in-depth exploration of U.S. privacy perceptions with respect to specific location features derivable from data made available by location data brokers and aggregators. These results can provide policy implications that could assist organizations like the FTC in defining clear access rules. 
Using a factorial vignette survey, we collected responses from 1,405 participants to evaluate their level of comfort with sharing different types of location features, including individual trajectory data and visits to points of interest, available for purchase from data brokers worldwide. Our results show that trajectory-related features are associated with higher privacy concerns, that some data broker based obfuscation practices increase levels of comfort, and that race, ethnicity and education have an effect on data sharing privacy perceptions.We also model the privacy perceptions of people as a predictive task with F1 score \textbf{0.6}.
\end{abstract}

%\textcolor{yellow}{Im changing this because there's a chance we might get one reviewer from the usenix..and that would mean they again think we are not focusing on LocationPrivacyPreservingMechanisms (LPPM) and again say we arent looking at the latest research on lppm. Naman, should we go with obsfuscation instead of privacy preserving?}

%\textcolor{red}{Naman, read the paper till section 5, I have moved everything to obfuscation techniques from location data brokers. These are the obfuscations they do, they don't use any LPPM, and we do not mention any of that in the paper, but they do attempt to obfuscate things: eitther the data brokers themselves or the researchers, and that is state of the art for data broker data!!}

\section{Introduction}
Location (GPS) data collection has become widespread with smart phones becoming ubiquitous. Smart phone apps often collect precise location data from users by offering \textit{free} services like navigating using Maps, tracking fitness goals, finding relevant and customized search results, social networking and meetups, or location tracking for personal safety in times of crises, among others. 
%Personal safety apps help people in times of crisis like car crashes, natural calamities, etc. 
Location data is used by app developers to understand app usage, improve functionalities and most importantly, to serve personalized ads and drive monetization. 

Monetization is driven by two distinct approaches. Tech companies like Google\footnote{\url{https://policies.google.com/technologies/location-data?hl=en-US\#how-is-location-information-used-for-ads}}, Apple or Meta collect precise location data from their users and use the data internally to improve app functionality while selling aggregate behaviors (\textit{e.g.,} geographical areas visited) for advertising purposes.
%Features characterizing aggregate behaviors from precise location data  can then be sold for advertising purposes. 
However, an individual's location data is never shared with or sold directly to third parties. On the other hand, data aggregators and data brokers collect, repackage, and sell individual location data to third-parties, including private organizations and academic researchers \cite{walkingActivity,hurricane_trd,homeworklocation,home_work_poi_pipeline}.
%\textcolor{green}{Naman, add some papers here with articles about companies/gov using lcoation data and with appers from researchers using location data}. 
These data aggregators either develop their own apps, or create Software Development Kits (SDK) that are then shared with app developers to collect location data from specific apps. 
Well-known tech companies have faced lawsuits for location tracking without the user's consents \footnote{\url{https://www.tzlegal.com/news/37-5-million-settlement-facebook-tracking-users-location-without-consent/}},\footnote{\url{https://www.reuters.com/legal/google-pay-155-million-settlements-over-location-tracking-2023-09-15}}. 
The Federal Trade Commission (FTC) has also penalized two data brokers in 2024, for selling location data to advertisers without adequately informing consumers or obtaining their consent \cite{FTCcase}.   
However, while tech companies never share individual location data with third parties, data brokers do, and this has created several instances of inappropriate uses including the US Defense department's access to prayer app's location data to monitor Muslim communities  \cite{brennen_loophole}, local law enforcement agencies tracking racial justice protesters using location apps \cite{brennen_loophole}, and the use of location data purchased from data brokers to track gay priests \cite{grindr} and people who visit abortion clinics \cite{plannedparenthood}.

%Tech companies sharing location data have already put in place several rules in the past years regarding the types of computations that should or should not be done with location data to preserve privacy. 

As a result of these news, some data aggregators and data brokers have created policies to prevent sensitive information about individuals being leaked. For example, data aggregator companies like Cuebiq don't allow the identification of visits to healthcare locations on their platforms, or only allow for the identification of home locations obsfuscated as census tracts \footnote{https://cuebiq.com/spoi-policy/}. 
However, although internal policy guidelines are important, they are solely based on data aggregators' decisions rather than on the perceptions of users whose location data is being monetized. We put forward that true consumer privacy rules need to come directly from consumer privacy perceptions. 


In this paper, we conduct a thorough evaluation of user levels of comfort with respect to the use of individual location data - acquired from data aggregators and data brokers - to identify and characterize personal trips and visits to places. 
Studies in the past have used survey approaches to explore people's perceptions toward the collection and use of individual location data, with a focus on points of interest (POI) visited \cite{colombia_privacy_study,cross_platform_social_media_data,martin}. For example, past research has shown that users feel much more comfortable sharing their work or home location than their location when they attend a rally or visit a medical facility \cite{cross_platform_social_media_data}. %These studies have revealed important insights into the types of points of interest (POI) that users feel comfortable sharing. 
However, points of interest are not the only feature available. In fact, travel history of millions of individual devices are available from data aggregators \footnote{\url{https://datarade.ai/data-categories/mobile-location-data}}. This location data is useful in deriving trajectory-based features such as origin-destination trips (\textit{e.g.,} trips from home to work) or the use of specific modes of transport (\textit{e.g.,} identifying driving vs. use of public transit vs walks).   
In this paper, we extend the state of the art in individual location data privacy perceptions by evaluating user perceptions with respect to trajectory-based features extracted from location data as well as points of interest, and we do so using U.S. representative surveys. 


%Although many data broker customers use data for legitimate purposes as stated in the study by Shelby et. al. \cite{thirdPartyData}, there have been instances 

% This raises an important aspect of data availability from data brokers, is that, according to mosaic theory, "at some point enough individual data collections, compiled and analyzed together, become a
%Fourth Amendment search..". The availability of such location features is thus dubious access by several actors (even though it is currently not classified as a direct violation of the Fourth amendment in the US).


\textbf{Obfuscation Techniques.} 
As stated before, data aggregators have developed approaches to extract trajectory and visit features from location data and obfuscate details in privacy respectful ways. For example, instead of computing the exact home location, the location is computed as a census tract (SafeGraph);\cite{safegraph} or instead of sharing the exact route trajectory, data points are changed to obfuscate movement patterns towards sensitive locations and home/work location ( e.g., Data aggregators like spectus protects trajectories that end at home by replacing exact home location with the centroid of the census tract a person's home is in \cite{spectusDeviceRecurring}).
%\textcolor{green}{It was written on their cuebiq platform!!!..but no longer there!!! they have it in an older webpage.. i've put it as a reference and not linked as footnote (because references have "last accessed at" which we need in case they take away that page! ).. Naman are these two descriptions public? We need to add links to justify, either to their own website or to papers that claim that..}
In this paper, we conduct an extensive review of current obfuscation approaches via published papers to identify the different types of techniques applied to individual location data acquired from data brokers; and we evaluate user privacy perceptions when using obfuscated location data vs. detailed (raw) location data. 
It is important to clarify that we exclusively focus on obfuscation techniques that directly change the location of a user as opposed to anonymization mechanisms that attempt to hide the actual location\cite{longroadcomp}.
%ith respect to the different state-of-the-art obfuscation approaches put in place to analyze trajectory and visit data acquired from data brokers and data aggregators. 




%\textcolor{pink}{If we have our contribution as bullet points later, do we need this here??}
%In this paper, we extend the state of the art in location data privacy perceptions by 1) Evaluating perceptions with respect to trajectory-based features extracted from location data, as opposed to only points of interest, 2) Evaluating perceptions with respect to current practices around the use of privacy respectful approaches to compute features from location data, and 3) Understanding how these perceptions might change with respect to the educational background as well as the racial and ethnic group of a person. 

\textbf{Factorial Vignette Approach.} Past literature like \cite{martin,ICADataCollection}, has shown that the level of comfort with location data sharing varies depending on who would be accessing the data and for what purpose. We follow Nissembaum's \cite{nissenbaum2004privacy} contextual integrity (CI) framework similar to \cite{martin} for creating survey questions that ask participants to evaluate their level of comfort with a specific feature to be used for a certain purpose by a certain actor. Using a factorial vignette survey approach we randomly create plausible combinations of actors, purposes and features to expose survey participants to different ways in which their location data could be used.
Understanding these nuanced privacy perceptions can support the creation of policies and rules concerning the types of features that can be extracted from location data, the types that should be avoided because they generate low levels of comfort, and the types of privacy respectful practices that are successful in increasing users' level of comfort. 
%Finally, the models to predict privacy perceptions can help companies evaluate levels of comfort with different types of features across different demographic groups before these features are actually implemented and deployed. 
The FTC has started to vigorously regulate consumer privacy providing \textit{"bright-line rules"} for companies to clearly understand what can and cannot be done with location data \cite{FTCsWarn}.
This paper is an effort to illuminate this path, and provide guidelines for potential future FTC-proposed rules. To sum up, the main contributions and findings of this paper are: 

\begin{itemize}[leftmargin=*]
%\vspace{-0.45cm}
%As stated in the Introduction, the main objective of this paper is to analyze the effect of the features, actors, and purposes as well as the effect of a participant education and race/ethnicity on their privacy concerns, measured as their level of comfort with the location data being shared. 

    \item   A thorough evaluation of user levels of comfort with respect to the use of trajectory data and Points of Interest taking into account actors and purposes. Our results show that people were uncomfortable particularly towards movement (trajectory-based) features such as frequent travel paths and were more comfortable with obfuscated home, work, and walk features. To the best of our knowledge, we are the first ones to reveal user perceptions with respect to the use of trajectory data. 
    
    \item An evaluation of users' level of comfort with respect to current data brokers' practices around the use of privacy respectful approaches to compute features from location data. Our results reveal a distinct preference for obfuscation approaches, especially when applied to detailed trajectory data. To the best of our knowledge, we are the first ones to evaluate user perceptions with respect to current data aggregator practices. 
    
    \item An analysis of how levels of comfort change with respect to educational background as well as racial and ethnic groups. Our analysis shows that there are significant differences in comfort between the racial groups White, Asian, Hispanic and Black. Additionally, participants with higher self-reported knowledge of computer science/programming were associated with higher comfort in sharing location features.    %....\textcolor{yellow}{Naman, please write one summary sentence}
    % \item A machine learning model to predict levels of comfort based on location features, actors, and purpose as well as demographic data. Our models show that levels of comfort can be predicted with a 78.3\% accuracy. Our model can be used to assess levels of comfort of different population groups when their location data is being used for different purposes. 
     
\end{itemize}


\section{Literature Review}

\subsection{Location Data Tracking}
Location data can be collected using GPS sensors/Bluetooth beacons on mobile phone devices or smart-watches and using WiFi or IP addresses when surfing the web. Current app development toolkits offer the possibility of combining location data from multiple sensors to get better accuracy: for example, the Fused location Provider API \cite{googleFusedLocation} in Android development or the CLLocationManager \cite{appleCCLocation} in iOS development. 
In addition, there exist several third-party Software development Kits (SDKs) (e.g., X-Mode, SafeGraph) that help app developers, tech companies and data brokers and aggregators collect, visualize and track location data \cite{vox_privacy}.  

With data from data brokers and aggregators, 
many researchers have shown that location data characterizing the places visited by a person as well as their trajectories can be useful in a plethora of settings such as modeling the spreading of a pandemic through tracing apps \cite{contact_tracing_apps}, 
inferring socio-economic indicators \cite{hong2016topic,estabaan_urban_dynamics}, health outcomes \cite{garcia_you_are_what_you_eat}, urban development and disaster risk \cite{equalizer_social_infra,hurricane_trd}, 
targeted marketing and advertising \cite{location_advertising},
understanding migration patterns \cite{hong2019characterization},
improving public transit \cite{ma2014development}, 
or modeling mobility patterns for public safety \cite{wu2022enhancing}.

At the same time, there has been a lot of work focused on privacy preserving methods for location data. 
Primault et. al. \cite{longroadcomp} extensively outline the techniques for sharing location data in a privacy preserving manner either via anonymization or obfuscation approaches. While
obfuscation techniques directly change the location of a user to protect their privacy, anonymization mechanisms attempt to hide the actual location from an attacker. 
In this paper, we focus on obfuscation techniques, not on k-anonymization approaches, since obfuscation is the most common method used by data brokers, data aggregators and the researchers and analysts acquiring and analyzing location data from these companies.  


%Data brokers do not explicitly specify these Location Privacy preservation mechanisms (LPPM) in their terms and services. Also, the field is evolving rapidly and new algorithms for LPPM are being developed with better representability of annonimyzed data. Thus, location privacy perceptions need to be understood in the context of features that would be derived from the data and not simply in the context of type of LPPM.


\subsection{Privacy Perceptions for Location data}
Duckham et. al. \cite{ducham} define location privacy as "special type of information
 privacy which concerns the claim of individuals to determine for themselves when,
 how, and to what extent location information about them is communicated to others".
 Our research builds on prior work investigating privacy perceptions around sharing personal location data in various contexts \cite{martin,colombia_privacy_study,ICADataCollection}. Prior research has looked into the effects of different variables on the willingness of individuals to share their personal location data including the role of (1) Actors, defined as individuals accessing the location data \cite{martin,ICADataCollection,cross_platform_social_media_data,employee_surveillance} (2) Purposes, defined as how the location data is going to be used \cite{martin, mobile_data_thesis, cross_platform_social_media_data} (3) Time, defined as the duration of the access to location data \cite{martin,mobile_data_thesis} (4) Age  \cite{colombia_privacy_study,location_privacy_students,privacy_older_adults} (5) Privacy attitudes \cite{personality_location,personality_privacy_perception_2016}, and  (6) Socio-cultural aspects \cite{location_privacy_SEM,mobile_data_thesis,privacy_covid,surveillance_covid}.

These works have shown that actors like the FBI, commercial entities, employers and data brokers are negatively perceived while local government and social media companies generate more positive perceptions when collecting and using location data. The duration of the location data collection as well as the purposes of using location data have also been associated with varying levels of comfort when sharing that data. For example, while the identification of sexual orientation or political views are perceived negatively, purposes related to public health or public safety are perceived more positively.

%Situations like longer duration of location tracking and objectives like identifying sexual orientation, political views and social circle of an individual are perceived as uncomfortable when compared to identifying places people visit, assessing their health, reducing spread of diseases, improving traffic, etc. Specific cases like comfortableness in sharing location data on social media platforms with images, posts and tweets was also observed. 

%\textcolor{yellow}{Naman, try to summarize here some of the findings e.g., certain actors are associate to more risk etc. Not two long, try to summarize}


There is also research exploring the use of location data for specific settings such as surveillance  \cite{video_survellience,vox_privacy}, location tracking for advertisements \cite{location_advertisement}, 
or location tracking for pandemic contexts. 
%They aim to understand people's nuanced perceptions of physical location tracking for different purposes.
Surveillance applications have been associated with some level of comfort in human-centered contexts like disaster recovery.
On the other hand, research during COVID-19 found strong opposition towards contact tracing apps \cite{privacy_covid,surveillance_covid,privacy_covid_2}, with over 50\% of the participants in both studies being unlikely to install such apps due to concerns around government intrusion. 

Finally, prior work has also examined how monetary incentives influence participants' willingness to share their location data for specialized research studies or market research \cite{incentive_participate_1, incentive_participate_2}; and the relationship between personality traits and privacy perceptions \cite{personality_location, personality_privacy_perception_2016}. Participants' positive perception of the service collecting their data, higher incentives, agreeableness, conscientiousness, and openness to new experiences were also linked positively to comfort in sharing location data.  
%\textcolor{yellow}{here again, add one or two sentences explaining findings}

Building on these findings, we extend the state of the art by (1) evaluating user perceptions with respect to trajectory-based features extracted from location data, beyond current analyses focused on visits to points of interest (POI), (2) evaluating user privacy perceptions when using obfuscated location data vs. detailed (raw) location data, and (3) analyzing the effect of race, ethnicity and education on privacy perceptions with respect to trajectory and POI visit location data. 

\subsection{Predicting Privacy Preferences}
% There is no prior work focused on inferring levels of comfort with respect to location data. 
Privacy fatigue is an important issue highlighted in research. This fatigue stems from factors like ubiquitous collection of personal data, scarcity of alternatives and additional permission requirements people need to go through \cite{tradeoff,privacy_exhaustion,privacy_risk_awareness}. 

Prediction models have been developed to explore people's perceptions around data collection and sharing like \cite{privacy_prediction,privacy_prediction2,privacy_prediction3,naeini,location_advertising,martin}. In these paper, the authors model people's perceptions of comfort in sharing location data as a multi-class classification or regression problem and model the perceptions around different actors accessing location data for different purposes. Eg: \cite{naeini} models people's perceptions in sharing IOT data as a binary classification task (predicting allow/deny); reporting accuracies of 77.6\% (0.70 recall) for a multi-class, 3-point Likert classification task, and 
81\% (0.78 recall) for the binary classification.
%\textcolor{yellow}{Naman, add one sentence with accuracy for these two tasks}
In another study, researchers created binary classifiers to determine privacy attitudes with regards to data sharing in several contexts including sharing biological data or personally identifying information (PII)\cite{contextualLabel}; achieving a mean recall of 0.46 for prediction of privacy concerns from contextual labels. 

In this paper, we explore binary and multi-class approaches, similarly to prior works.

\iffalse 
\textbf{Survey Design.} Research studies on privacy perceptions typically employ factorial vignette approaches to understand how different scenarios affect an individual's willingness to share location data.
Privacy risk perception studies highlight that impersonal, less detailed,
\textit{abstract} risk scenarios are often perceived as less risky\cite{privacy_risk_awareness}. 
Given that risk is a major factor in determining levels of comfort with sharing location data, vignette questions that present location data in abstract ways may underestimate people's privacy boundaries and perceptions. 
Hence, in our survey, we make an effort to explain to the survey participant, the types of location data we refer to, by providing extensive details about the location features and about what they represent.  
Interestingly, some privacy perception studies demonstrate that visualizing abstract data can help alleviate privacy risk concerns and increase people's comfort with data sharing \cite{vizbetter}. This suggests that the way data is presented can significantly impact privacy perceptions.
For that reason, in our survey, we accompany each vignette not only with extensive explanations, but also with map-based visualization to convey better location data information.
\fi 

%In our study, we have chosen to use map-based visualizations to convey feature information. This decision is supported by previous research demonstrating that visualizations can more effectively inform users about risk implications of data sharing \cite{vizbetter}. 

% \subsection{Predicting Privacy Preferences}
% % There is no prior work focused on inferring levels of comfort with respect to location data. 
% Privacy fatigue is an important issue highlighted in research. This fatigue stems from factors like ubiquitous collection of personal data, scarcity of alternatives and additional permission requirements people need to go through. \textcolor{red}{NAMAN, Add the citations for those fatigue papers}. 

% Prediction models have been developed to explore people's perceptions around data collection and sharing by IoT devices \cite{naeini}. In this paper, the authors model people's perceptions of comfort in IoT data collection as a multi-class classification problem and model the perceptions around IoT data sharing as a binary classification task (predicting allow/deny); reporting accuracies of 
% 77.6\% (0.70 recall) for a multi-class, 3-point Likert classification task, and 
% 81\% (0.78 recall) for the binary classification.
% %\textcolor{yellow}{Naman, add one sentence with accuracy for these two tasks}
% In another study, 
% researchers created binary classifiers to determine privacy attitudes with regards to data sharing in several contexts including sharing biological data or personally identifying information (PII)\cite{contextualLabel}; achieving a mean recall of 0.46 for prediction of privacy concerns from contextual labels. 


% % \textcolor{red}{NAMAN, add more papers that work on perception prediction from somewhere  }


% % In this paper, we explore binary and multi-class approaches, similarly to prior work. 
% %\textcolor{yellow}{Naman, add one sentence with accuracy}

% %We explore literature on modeling perceptions using the collected data. While insights can be gathered around which actors/features/purposes/attitude/awareness are linked to higher comfort, there are many practical benefits of having a classification/regression model for predicting comfortableness of a group of interest. To that effect, we explore literature on modeling privacy perceptions as classification/regression tasks. 
% %Indices for general privacy perceptions developed by Westin (Westin Privacy index) are not enough for understanding context specific attitudes and perceptions \cite{westininsufficient}. Contextual integrity framework surveys need to be designed carefully to incorporate the context and convey the correct information for capturing true perceptions of surveys participants \cite{martin},\cite{CISmartHome}. In this study, we propose to create a classification model which can classify privacy perceptions of location data sharing in specific contexts.
% %Similar approach ahs been successfully developed previously. 

% Features list
\begin{table*}[ht!]
\centering
\footnotesize
\caption{Location Features and Abbreviations. We cite examples of recent studies where such features are used in the "feature cluster" column }
\begin{tabularx}{\textwidth}{|l|p{0.25\linewidth}|p{0.15\linewidth}|p{0.35\linewidth}|}
\cline{1-4} 
\multicolumn{1}{|l|}{Feature Cluster} & Feature & Abbreviation & Description \\ 
\cline{1-4} 
& Your inferred home location& Home location (Detailed)
& Home location as a point on map (Eg. fig. \ref{fig:homeD}) \\ \cline{2-4} 
 & Your inferred home location represented as   a census tract& Home location (Obfuscated)
& Home's county location (Eg., Fig. \ref{fig:homePP})\\ \cline{2-4} 
& Your inferred work location& Work location (Detailed)
& Work location as a point on map (Eg., Fig. \ref{fig:workD}) \\ \cline{2-4} 
\multirow{-4}{*}{ \shortstack[l]{Home + Work \cite{homeworklocation,home_work_poi_pipeline}}} & Your inferred work location represented as a census tract & Work location (Obfuscated)
& Work's county location (Eg., Fig. \ref{fig:workPP}) \\ 
\cline{1-4} 
& The places you visit & Places you visit (Detailed)
& Chart indicating types and frequency of places visited. Includes a map depicting the detailed locations of the places visited (Eg., Figs. \ref{fig:POVD}, \ref{fig:POVPP}) \\ \cline{2-4} 
& The types of places you visit & Places you visit (Obfuscated)
& Chart indicating types and frequency of places visited (Eg., Fig. \ref{fig:POVPP})\\ \cline{2-4} 
\multirow{-3}{*}{Places Visited\cite{PlacesofVisit,home_work_poi_pipeline,poi_2}}                     & The geographical area where you spend most   of your time & Area you spent most of your time (Obfuscated)
& Map displaying radius of gyration (Eg., Fig. \ref{fig:ROG})\\ \cline{1-4} 

& The modes of transportation you use, with what frequency and their corresponding routes & Modes of transportation (Detailed)
& Chart indicating frequency and types of mode of transport used. It also includes a map with lines indicating detailed routes for each frequent mode of transport (Eg., Fig. \ref{fig:modesD}, \ref{fig:modesPP}) \\ \cline{2-4} 
\multirow{-2}{*}{Transportation \cite{modesOfTransport_1,transport_NZ,home_work_poi_pipeline} }& The modes of transportation you use and   with what frequency & Modes of transportation (Obfuscated)
& Chart indicating frequency and types of mode of transport used (Eg., fig \ref{fig:modesPP})\\ \cline{1-4} 

& Your most frequent trips & Most frequent trips (Detailed)
& Map with frequently taken routes. Routes are detailed showing start and end locations as well as the in-between GPS points visited (Eg., Fig. \ref{fig:freqTD}) \\ \cline{2-4} 
& Your least frequent trips& Least frequent trips (Detailed)
& Map with infrequently taken routes. Routes show start and end locations as well as all the GPS points visited (Eg., Fig. \ref{fig:leastFreqT})\\ \cline{2-4} 
& Your most frequent types of trips & Most frequent type of trips (Detailed)
& Map with the different frequent trips taken and the inferred trip purpose by type of destination (Eg., Fig. \ref{fig:freqtypetrip})\\ \cline{2-4} 
& Your most frequent trips represented by their origin and destination census tracts and connected by a line& Most frequent trips between counties (Obfuscated)
& Map with frequently taken routes. It protects privacy by showing the start and end points as counties, and the frequently taken routes as straight lines between counties instead of detailed GPS. (Eg., Fig. \ref{fig:FreqOD}) \\ \cline{2-4} 
\multirow{-5}{*}{Movement \cite{modesOfTransport_1,traj_analysis,poi_2}} & Your most frequent trips represented by their origin and destination census tracts and connected by an approximate route & Most frequent trips between counties ('Google') (Obfuscated)
& Map with frequently taken routes. It protects privacy by showing the start points, end points as counties, and the frequently taken routes as suggested by  Google Maps, instead of the actual GPS route. (Eg., Fig. \ref{fig:gmaps})           \\ \cline{1-4} 

& Your walking activity and the corresponding routes & Frequent walking activity (Detailed)
& Chart indicating frequency and duration of walks. It also includes a map with detailed GPS trajectories indicating routes for each frequent walking path (Eg., Fig. \ref{fig:walkD} and \ref{fig:walkPP})\\ \cline{2-4} 
\multirow{-2}{*}{Walking Activity \cite{walkingActivity,mode_detection_all,transport_NZ,walk_nz}}                   & Your walking activity& Frequent walking activity (Ob)& Chart indicating frequency and duration of walks with no detailed trajectories  (Eg., Fig. \ref{fig:walkPP})                          \\ \cline{1-4} 

& The foreign countries you have visited and   the duration of the visit, including all the locations where you have stayed    & International visits (Detailed)
& Chart indicating frequency,duration and location of international trips. It includes a map with detailed GPS points indicating regions visited in the foreign county (Eg., Fig. \ref{fig:internationalD})\\ \cline{2-4} 
\multirow{-2}{*}{International Trips \cite{cuebiqInternational} }                & The foreign countries you have visited, and for how long& International visits (Ob)& Chart indicating frequency, duration and location of international trips. (Eg., Fig. \ref{fig:internationalP})                          \\ \cline{1-4} 

\end{tabularx}
\label{tab:listOfFeatures}
\end{table*}

\section{Survey Design}
%We curated a set of actors, features, and purposes frequently involved with location data. Contextual questions were created using a partial factorial combination, eliminating impractical Actor-Purpose-Feature (A-P-F) combinations. The actors, purposes, and features are described in the following sections.
Past work has shown that the level of comfort when sharing highly private information is influenced by \textit{who} accesses the data and for \textit{what} purpose, as stated by Nissembaum's contextual integrity framework \cite{nissenbaum2004privacy}. Hence, we used a factorial vignette survey approach where each vignette in the survey was created following the same format: \textit{Actor X wants to do Purpose Y and for that, they need to use Feature Z} while systematically changing actors, features and purposes across participants to test their relevance. For example, a potential survey vignette could be "A doctor wants to monitor your personal wellness. For that purpose they want to access your detailed walking activity. How comfortable would you feel with this use of your personal location data?". Levels of comfort were measured using a 5-point Likert scale, from \textit{Very Uncomfortable} to \textit{Very Comfortable}. 
Participants were also asked to fill out a free-form text box explaining their answer (see Figs. \ref{fig:locationdataqsdetailed} and \ref{fig:locationdataqspp} in the Appendix for two survey question examples).
Next, we describe location features, actors and purposes in detail. 

\textbf{Location Features}
We advance the state of the art in the evaluation of location data privacy perceptions along two main fronts. First, we consider features extracted from location data that go beyond current analyses focused on visits to points of interest (POI) such as visiting a hospital or a liquor store.
Specifically, we evaluate \textit{trajectory}-based features that are extracted from location data by considering sets of GPS points that define spatio-temporal datasets characterizing trajectories. 
For example, we can ask a participant about their level of comfort with a company using their detailed driving trips (showing the specific route followed) or their detailed walking activity.

Second, we evaluate trajectory-based and POI visit features using two approaches: detailed and obfuscated. 
While \textit{detailed} uses all GPS points available, \textit{obfuscating} approaches characterize location data using current state-of-the-art data aggregator practices of obfuscation. For example, an obfuscated driving trip would only specify the origin and destination census tract of the trip instead of the detailed trajectory with all locations (GPS points) visited in-between. Similarly, an obfuscated home location would only specify home location at the county level instead of the exact location. 


Table \ref{tab:listOfFeatures} shows all the features we have used to craft the survey vignettes. 
These features can be categorized into six groups: "Home+Work", "Places Visited", "Transportation", "Movement", "Walking Activity", and "International Trips". 
For each feature in each category, we define both its Detailed (D) and its Obfuscated (Ob) version. 
Next to each feature in the Table, we provide links to papers that compute these features using 
location data acquired from data brokers or aggregators, showing that these are in fact state-of-the-art features when working with 
datasets from data aggregators. 
%and that   
%To show that we are considering state-of-the-art approaches, we provide a list of papers for each feature cluster. 
%These papers use different types of detailed and obfuscated features extracted from location data acquired from data aggregators. 
While POI features such as home, work and places visited have already been explored in the literature, comfort perceptions for obfuscating approaches for these have not been explored; nor the other trajectory features in either detailed or obfuscated forms. 
%\textcolor{yellow}{Naman, can you please move transportation in table 3 under places visited i.e. first home+work, then places, then transportation then everythign else. }
Each of these features is presented to the survey participants via an accompanying visualization. See the Description column in Table \ref{tab:listOfFeatures} for links to sample visualizations shown to survey participants. More details about survey design choices, including visualizations, are explained at the end of this section. 

%actor list
\begin{table}[ht!]
\centering
\footnotesize
\caption{Actors and the abbreviations. Actors taken from previous studies like \cite{martin, mobile_data_thesis, ICADataCollection}}
\begin{tabular}{|p{0.55\linewidth}|l|}
        \hline
Actor & Abbreviation      \\
        \hline
{ An emergency service - like   emergency medical services or fire and rescue services } & Emergency services
\\
        \hline
    
A federal government agency - like the   FBI or CIA & Federal government agency
\\
        \hline
A local government agency & Local government agency
\\
        \hline
Your employer& Employer
\\
        \hline
A commercial entity & Commercial Entity\\
        \hline
Your family& Family\\
        \hline
Your doctor& Doctor\\
        \hline
A law enforcement agency - like a city police department or a county sheriff’s office & Law enforcement agency
\\
        \hline
Academic researchers& Academic researchers
\\
        \hline
\end{tabular}
\label{tab:listOfActors}
\end{table}

\textbf{Actors} 
Table \ref{tab:listOfActors} shows the list of the nine actors we use to create the vignette questions. 
%This study considers \textbf{9 actors}: Academic Researchers, Federal Government agencies (e.g., CIA/FBI), Local law enforcement agencies (e.g., city police departments or county sheriff's offices), Emergency services (e.g., emergency medical services or fire and rescue services), Doctors, Commercial entities, Local government agencies, and Employers. 
These actors have been examined in previous studies related to location data privacy \cite{martin,ICADataCollection} and represent the majority of entity types that can access an individual's location information. 
Our work's novel contribution lies in looking at the effect of these actors on privacy perceptions in conjunction with novel trajectory-based features while controlling for demographic and educational data. 
%data, general privacy perceptions and technical knowledge. 
%Table \ref{tab:actorcomfortmean} displays the mean comfort levels for sharing location data with each actor.

%https://par.nsf.gov/servlets/purl/10176662

%  Purpose list
\begin{table}[ht!]
\centering
\footnotesize
\caption{Purposes and Abbreviations. Purposes taken from previous studies like \cite{martin, mobile_data_thesis, ICADataCollection}}
\begin{tabular}{|p{0.6\linewidth}|p{0.3\linewidth}|}
\hline
Purpose   & Abbreviation               \\ \hline
{Understand criminal activity by looking   into the relationship between crime, people’s movements and locations visited} & Analysis of criminal activity
\\ \hline
Analyze terrorist attacks by looking into   people’s movements and locations visited                         & Analysis of terrorist attacks
\\ \hline
Monitor how people move (or don’t) to control the spread of a disease e.g., covid-19 & Control spread of diseases
\\ \hline
Monitor your personal wellness and physical activity & Personal wellness
\\ \hline
Show you targeted ads or personalized   announcements & Show Ads\\ \hline
Analyze traveling experiences and public transit services& Analysis of Public transit services
\\ \hline
Understand how people move in a city so as to inform the design of new walking and cycling infrastructure& Design new walking/cycling infrastructure
\\ \hline
Understand your commute to work so as to optimize work productivity & Optimize work productivity
\\ \hline
Monitor your mobility patterns e.g., the places you visit or the trips you make & Monitor mobility patterns
\\ \hline
Understand how people move so as to identify optimal locations for hospitals, libraries or parks& Identify locations for infrastructure
\\ \hline
\end{tabular}
\label{tab:listOfPurposes}
\end{table}
\textbf{Purposes}
Table \ref{tab:listOfPurposes} lists the ten different purposes we use when crafting the vignette questions. These purposes are informed by prior work looking into privacy perceptions related to location data  \cite{martin,ICADataCollection}, and cover public service purposes such as understanding where to build a new hospital, public health purposes such as monitoring population mobility during a pandemic, law enforcement such as using location data to identify criminal activity, as well as economic, commercial and general purpose such as using location data to show ads or optimize productivity.


There are combinations of actor, feature and purpose that do not make sense. For example, a doctor will not be interested in getting access to mobility patterns at city scale. Hence, we eliminate implausible combinations from the pool of possible questions. After this process, a total of 445 valid combinations of actor, purpose and feature are left and randomly shown to participants. 

We did a qualitative study with 5 respondents recruited from Craigslist to understand comprehension of vignette questions asked in the survey. They were asked to answer 10 randomly selected vignette questions and elaborate on their thought process. All the respondents were able to interact with the map visualizations, answer our questions about the contents of the map and did not raise any issues with the structure of the vignettes.   




%\textcolor{pink}{Examples of flagged responses: (1) Participant-c6d0b4d responsed "uyvasdyib" to the question "A commercial entity wants to monitor your personal wellness and physical activity. For that purpose, they will access the types of places you visit, as shown in the Figure above." (2) Example of Explanation not matching the context: "I prefer to label it this way" for question "A law enforcement agency - like a city police department or a county sheriff’s office - wants to understand criminal activity by looking into the relationship between crime, people’s movements and locations visited. For that purpose, they will access your most frequent trips represented by their origin and destination census tracts and connected by a line, as shown in the Figure above."
%}

\subsection{Design Choices} 
Prior work in privacy risk perception studies has shown three important insights that we build on when creating the survey. 
First, \cite{privacy_risk_awareness} showed that abstract scenarios are often perceived as being less risky when compared to personal scenarios. Hence, before starting the survey questions, we ask participants to situate themselves in the vignettes as if this was their own personal data. In addition, the questions are phrased in a way that put the participant at the center of the vignette using the "you" pronoun
(see Appendix \ref{subsection:consentbrief} for more details on this design). 

%\textcolor{yellow}{Naman can you please separate the IRB and the explanation figures into two? and correct the references to them in the paper.thx!}

Second, prior work has shown that visualizations can help better understand privacy risks, and that when privacy-related questions are asked with visualizations, privacy risk concerns with data sharing tend to decrease \cite{vizbetter}. Hence, each of the vignette questions in the survey is accompanied by an interactive visualization of the location feature, as well as by a brief explanation of that visualization to make sure that the user understand the feature extracted from the location data. Figure \ref{fig:locationdataqsdetailed} in the Appendix shows a visualization for a question where the feature is detailed trajectory data and transportation modes. The Appendix shows three other examples of feature visualizations including 
detailed features Frequent Trips (Fig. \ref{fig:freqTD}) and visits to points of interest (Fig. \ref{fig:POVD}) as well as an obfuscated feature showing frequent trips with synthetic trajectories generated with Google Maps (Fig. \ref{fig:gmaps}).

%\textcolor{yellow}{Naman please add three more features in the appendix.}

Third, prior work has shown that having prior technical knowledge or specific demographic or personality traits might affect privacy perceptions \cite{martin}. Hence, we also ask participants to fill out questions covering both privacy attitudes as well as technical knowledge and demographic data.
We broke these questions into two blocks, demographic and general tech knowledge questions were shown before the vignettes and the privacy attitudes questionnaire after. This design was based on user feedback from our qualitative survey evaluation, as it appeared to provide breaks before and after the more in-depth vignette questions. Following Martin et. al. \cite{martin}, the privacy attitudes questionnaire asked participants to rate on a 5-Likert scale (from "strongly disagree" to "strongly agree") their agreement with different privacy attitudes such as trust in business or in government (see Figs.\ref{fig:attitude_webpage} and \ref{fig:demographic_page} in the Appendix for a detailed list of all the questions asked, and section \ref{martinAppendix} for further details on the rationale behind the privacy questionnaire design).



\subsection{Platform} 
%We run an online survey on Cint\footnote{\url{https://www.cint.com/}}. Cint surveys are advertised to survey takers on their platform. We requested for a 
We advertised our online survey on Cint\footnote{\url{https://www.cint.com/}}. Interested participants clicked on a link on Cint that took them to an institution webpage where we hosted the survey. We did not use Cint surveys directly because of the elaborate nature of the factorial vignettes that were randomly sampled and of the custom map visualizations we created, consistent with the survey question. 
Each participant was paid \$7 for completing the survey which consisted of five vignette questions, seven demographic and computer knowledge questions, 10 privacy attitude questions. All participants were protected by IRB number 1768475-4 \footnote{\url{https://osf.io/x2vjk?view_only=e6dd3400991946afa3b265df5b1f3132}}. 

We collected a U.S. representative sample of 1,405 participants (7,025 vignette questions answered), which produced an average
of 16 answers per vignette, on par with prior work in privacy perceptions\cite{martin,ICADataCollection}.

%\textcolor{pink}{We created the vignette's personality data (different location features of the hypothetical person) to incorporate many of the location features (like Places of Interests). To convey these features, we created custom map visualizations with labels and consistent color formatting.} We had to build our own survey webpage to included these elaborate visualizations that were randomly sampled whenever participants landed on our survey page. 


%\textcolor{pink}{We had over 8074 survey-takers land on the webpage, 4607 of them clicked on "I do not consent" and were redirected to cint. To keep the survey sample representative of the US population and redirected participants back to cint after the first page if that group had enough surveys. We redirected 1637 people due to demographic limit. There were 2405 participants who filled out the entire survey. From these, we flag each vignette question with gibberish explanation or explanation that do not make sense in the context of the question. If a survey participant was flagged more than twice, their response was considered invalid. We paid 1405 participants \$7 for participating in the survey. Median time spent on each vignette question is 72 seconds}


%Each participant was asked to provide their level of comfort with sharing different types of features extracted from location data via five vignettes (see sample in the Appendix \ref{section:surveyUI}). Levels of comfort were measured using a 5 point Likert scale, measuring levels of comfort from \textit{Very Uncomfortable} to \textit{Very Comfortable} (see Figs. \ref{fig:locationdataqsdetailed} and \ref{fig:locationdataqspp} in the Appendix for two vignette examples).

%\textcolor{pink}{We have 7025 vignette questions answered (1407 Participants X 5 questions each). Each vignette is asked on an average 16 times (median) in the study. For pairs the number of times each pair is answered is much higher (actor-purpose:194, purpose-feature:49, actor-feature: 41). }
%\textcolor{orange}{ Naman, here I would describe how many people participated in our survey, and why that is enough given the number of questions we have, basically the numbers we did a while ago on the blackboard...}



%\textcolor{red}{Vanessa, we had mentioned we used the questions from Martin..!! Some of the critiques from USENIX were really unnecessary!}


%\textcolor{yellow}{Naman, above, please add ref to fig 14 with demo and tech knowledge quetions}


%\section{Components of context}
% \label{sec:components}


%The purposes for this study were selected based on various use cases of location data for the actors involved. The \textbf{10 purposes} can be categorized as follows:

% \iffalse
% \textbf{General Purpose}: "Monitor your mobility patterns (e.g., the places you visit or the trips you make)" - This purpose does not describe a concrete use case and is useful for understanding a general-purpose scenario not included in our partial-factorial design. 

% \textbf{Public Service and Public Health}: 
% \begin{itemize}
%     \item Understanding how people move so as to identify optimal locations for hospitals
%     \item Understanding how people move in a city so as to inform the design of new walking and cycling infrastructure
%     \item Monitoring how people move (or don’t) to control the spread of a disease e.g., COVID-19
%     \item Analyzing traveling experiences and public transit services
%     \item Monitoring your personal wellness and physical activity
% \end{itemize}

% \textbf{Law Enforcement}:
% \begin{itemize}
%     \item Understanding criminal activity by looking into the relationship between crime i.e people’s movements and locations visited
%     \item Analyzing terrorist attacks by looking into people’s movements and locations visited
% \end{itemize}

% \textbf{Economic and Commercial:}
% \begin{itemize}
%     \item To show you targeted ads or personalized announcements
%     \item Understanding your commute to work so as to optimize work productivity
% \end{itemize}
% These categories encompass the primary purposes driving the collection and usage of individuals' location data. Table \ref{tab:purposecomfortmean} shows the average comfort level of participants for each purpose.
% \fi 


%\textcolor{yellow}{naman, for Table 3, please create a new figure for current "Chart 12" in transportation(Ob), or copy the chart from fig 12 and create a separate fig with it, otherwise it;s confusing}

%For each of these features, the vignette has an accompanying visualization in the survey (see Description column in the Table). Sample visualizations for each of the features are in the Appendix. Please check the links in the Table \ref{tab:listOfFeatures}.
%can be found in the Appendix. 

%\textcolor{red}{Naman, can you add links to first and last plots e.g. see plots 23 to 31 in the prior sentence?}
%I think that we need to add one visualization per type of feature, otherwise understanding these features might be hard. Take table 3 and add one visualization for each feature in the appendix. I know it's 18 or so, but otherwise it's not very transparent...}

%When assessing privacy perceptions, participants were asked about their comfort levels in sharing each feature across different actors and purposes. Table \ref{tab:featcomfortmean} presents the mean comfort levels for sharing each feature.

\section{Analytical Approach}
%Each participant responded to 5 vignette questions providing their level of comfort on 5-point Likert scale from (\textit{very uncomfortable}) to (\textit{very comfortable}). In addition, for each participant, the survey also collected their education level, race and/or ethnicity, as well as responses to the privacy attitudes and technical knowledge questions (see Figs. \ref{fig:attitude_webpage} and \ref{fig:demographic_page} in the Appendix).

%\textcolor{yellow}{Naman, please add X and Y above}
%wareness questions in the questionnaire at the end of the survey. 

As stated in the Introduction, the main objective of this paper is to analyze user levels of comfort with respect to the use of individual trajectory and POI visits taking into account the actors and purposes involved in the data analysis, as well as the presence (or not) of obfuscating approaches to preserve privacy. In addition, we are also interested in evaluating how these perceptions might change with respect the participants' educational or racial background. 

To carry out this analysis, we employ a mixed-effects ordinal logistic regression (clmm in R) due to the ordinal nature of the Likert responses (5-point Likert scale from (\textit{very uncomfortable}) to (\textit{very comfortable})). We follow best practices for the vignette data analysis using ordinal regressions \cite{vignette_instructions}. We use participant ID (each participant given a random ID) and vignette (vignette number) as two random effects. The dependent variable in the mixed effects ordinal regression is the level of comfort (five categories) and the independent variables are the variables whose effect we want to evaluate: actors, purposes, features (separating detailed features from features computed using obfuscation approaches), education and race and ethnicity. We also include the answers to the privacy attitudes, demographic and technical knowledge questionnaires as control variables in the regression, given that these have been associated with privacy comfort levels in prior work (\cite{martin}). 

A coefficient analyses of the independent variables will allow us to identify which variables are statistically significant and thus have an effect on the level of comfort that users have with different types of location data being shared. 
A detailed description of the ordinal regression proposed is provided in the Appendix, in section \ref{section:ordinal_regression}, where we also show that 
that assumptions for the ordinal regression are met (see Table \ref{tab:model_aic_compare} in Appendix).
%. We follow best practices for vignette data design \cite{vignette_instructions} and show 

In addition, we are also interested in evaluating the interaction effects between independent variables. 
For example, the ordinal regression might reveal a trajectory feature as having a significant negative effect on the level of comfort; but we are also interested in quantifying significant changes in levels of comfort for that feature across different types of actors or purposes. 
To run these analyses, we first transform the 5-point Likert scale categorical answers into a numerical scale from -2 to +2, with -2 being \textit{very uncomfortable}, 0 being a neutral privacy perception, and +2 being a \textit{very comfortable} level of comfort; and then 
run Kruskal-Wallis (KW) statistical tests between the pairs of variables whose interactions we want to evaluate (features and actors in our example); followed by post-hoc Dunn tests with the Benjamini-Hochberg correction to identify statistically significant differences in comfort levels across specific pairs (of features and actors in our example). 
%\textcolor{orange}{It doesn't give us any useful information..like interaction effects are better or worse than the base interaction (like if home location:feds is -0.8..that means its that much worse from the baseline which is Places of visit:Comercial entity).::Naman, here we need to state why we don't do this with interactions in the regression. It's because we don't have enough data right? but we need to clarify this, I remember a reviewer asking.}
Further details for the KW and post-hoc Dunn tests are provided in the Appendix, in section \ref{section:KWDunn}.
We do not run the interaction effects analyses within the ordinal regression because that would only allow us to explore relationships with respect to specific regression baselines, instead of across all feature values. 
%Our analysis begins with an examination of the proportion of comfort levels across contextual elements, demographic variables, and attitudes. We then employ ordinal regression (detailed in Appendix to analyze the regression model coefficients, accounting for education, ethnicity, trust, privacy attitudes, and gender, as these factors are known confounders to privacy attitudes and comfort levels. %\textbf{Baseline categories:} For Actors: "Commercial Entity"; for Purposes: "To show ads"; and for Features: "Modes of transport (Detailed)". Complete baseline information is available in Tables \ref{tab:apf_reg}, \ref{tab:ethedu_reg}, and \ref{tab:attitudes_reg}.%To identify statistically significant differences in comfort levels when contextual variables (actors/purposes/features) interact with other contextual or demographic (ethnicity/education) attributes, we employ the Kruskal-Wallis test followed by a post-hoc Dunn test. %\textcolor{yellow}{%Naman, please add an explanation of why we use ordinal regression, maybe show the formula?.\textbf{DONE} 
%Also, please explain how you do the interaction analysis with the statistical approach. \textbf{DONE} }


In the next section, we provide some general statistics about the survey participants and the general distribution of levels of comfort across features, actors, and purposes. Next, in the following sections, we will discuss results for our three main research questions: 1) RQ1: effect of individual trajectory and POI visits on user levels of comfort, and their interactions with actors and purposes, 
%user levels of comfort with respect to the use of individual trajectory and POI visits taking into account the actors and purposes involved in the data analysis
%effect of actors, purposes and features on privacy perceptions, measured as levels of comfort, 
2) RQ2: effect of obfuscating approaches on user levels of comfort, and 3) RQ3: effect of educational background and race and ethnicity on privacy perceptions. 

\section{Survey Response Analysis}
We collected answers from 1,405 participants. Each participant spent a mean of 18 min on the survey answering the 5 factorial vignette questions, as well as the demographic, computer knowledge and privacy attitudes questionnaires. 
From the 7,025 vignette questions answered, we removed answers whose required free-form text-box explanations did not clearly match the user comfort selected on the Likert scale, including both lack of quality explanations or low quality text. To achieve that, the answers were manually reviewed by two researchers who agreed upon its quality. 
In addition, we also removed answers with response times lower than 10 sec (given that the average adult reading speed is between 100-1200 words/min\cite{readingspeed} and the average vignette length is 43 words). 

%Using the Cint platform, we collected responses from 1,405 participants regarding their comfort levels with 445 location data sharing vignettes. 
%Each participant responded to 5 vignette questions as well as to the demographic and privacy perceptions questionnaires. The participants spent 
%18 minutes (mean) on the survey. 
%The duration is computed as the difference between the survey landing time and clicking submit.
%The mean is higher than median, indicating few participants could have left the survey tab on the browser and come back later (as there were only 5 contextual questions). 
%\textcolor{yellow}{Naman, was the text box the only quality check you did? did you remove participants who spent too little time on the survey? or other? didn't Saad also remove gibberish answers?}
%\textcolor{pink}{We used the free form text box within each vignette question to filter out bad quality responses. We also remove responses where response time was less than 10 seconds (average Adult reading speed between 100-1200 words per minute  X average vignette question word-length 43 ) to remove unlikely responses}. For the regression analyses, we only considered answers whose text box explanations matched the Likert scale on user comfort. Answers whose free text box did not have any text, the text was meaningless or not clearly explainable of the level of comfort selected were removed. Answers 

%The answers were marked genuine if the text box explanation matched the Likert scale answer for comfort. The reviews were done manually by one of the researchers on the project. 
%This is more tedious to detecting straight-lining (participants select the same answers) or random clicking but we this makes sure that the quality of responses is higher. In future, we can use Zero shot/few shot learning approaches to automate this task using LLMs.

%\textcolor{orange}{ We asked for ethnicity X education representation because it was getting very tough to get more representation otherwise. We filtered the quotas internally..So we have close representativity for all(off by at max 3-4)....  Naman we need to add the demographic table here, and maybe remove one of the fig1-fig3 survey tables, but the demographic data is important. Also, did we focus on US representative (gender, age, income, etc) or only US representative for race and education? We should add all data available in the table and briefly discuss. and show census stats to discuss representativity. }
We aim for representative sampling (proportions similar to U.S. Census from ACS \cite{census_education_attainment}. Survey population and ACS statistics are in Table \ref{tab:ethXeducensus}.


Table \ref{tab:ethXeducensus} shows the demographic characteristics across all survey participants for education levels as well as racial and ethnic groups, which are the focus of our analysis. (Please see Appendix for Age: Table \ref{tab:agedistribution} and Gender: Table \ref{tab:genderdistribution}). 
As can be seen, the sample is approximately representative of the U.S. population. 
The percentages for 
%Given our focus on studying the effect of education as well as race and ethnicity on levels of comfort with sharing location data, we collected an \textit{approximate} U.S. representative sample. 
% Table \ref{tab:ethXeducensus} shows the survey and actual population distributions for education and race/ethnicity. 
%Focusing in the two variables of interest (race/ethnicity and education), we observe that the 
%Our population sample attempts to be representative of the four largest racial and ethnic groups in the US: White, Latino/Hispanic, Black and Asian. P
White and Black groups are close to ACS estimates, while the values for Asian and Hispanic are a bit lower. To account for this slight imbalance, we apply weights to our ordinal regression analysis, giving more importance to groups that are less represented (further details are provided in the Appendix \ref{section:ordinal_regression}).
%
%\textcolor{green}{I wrote one line in appendix..I dont know how i would detail the weights..I divide ACS estimate of ethXedu\% by our collected ethXedu\% and assign that as weight! Please let me know if thats correct!.... Naman, I think we need to explain this in more detail in the appendix}


% Similar tables showing representation for gender and age can be found in the Appendix (see Tables \ref{tab:agedistribution} and \ref{tab:genderdistribution}). 
% \textcolor{pink}{Naman, only add this sentence after you finish the new regression model We weight our mixed effects ordinal model to make the survey representative for these attributes.}


%\textcolor{yellow}{Naman, please move the age table to the appendix, and add a table with gender in the appendix, even if it's 50-50, just add it for completeness. Add the figure references in the previous text. }
\begin{table}[]
\centering
\footnotesize
\caption{U.S. census and survey population distribution across four major races/ethnicities and education levels.}
%of ethnicities [White, Black,Asian,Hispanic] and education  for Adult (age 18 and above) population in the US}
\begin{tabularx}{0.5\textwidth}{|l|XXX|X|}
\hline
\textbf{Education}   & \textbf{Under High School} & \textbf{High school to Bachelors} & \textbf{Bachelors and above} & \textbf{U.S. Census} \\
\textbf{Race/Eth} &&&& \\
\hline
\textbf{White}       & 6.2\%           & 37.4\%                  & 24.5\%                       & \textbf{68.2\%}      \\
\textbf{Black}       & 1.2\%           & 7.2\%                   & 2.9\%                        & \textbf{11.3\%}      \\
\textbf{Asian}       & 1.2\%           & 2.0\%                   & 3.2\%                        & \textbf{5.7\%}       \\
\textbf{Hispanic}    & 3.6\%            & 8.4\%                   & 2.9\%                        & \textbf{14.9\%}      \\
\hline
\textbf{Census} & \textbf{12.1\%}           & \textbf{55.0\%}                   & \textbf{33.5\%}              &                      \\
\hline
&                           &                                   &                              &\\
\textbf{Education}  
&           \textbf{Under High school}                &   \textbf{High school to Bachelors}                                &       \textbf{Bachelors and above}                       & \textbf{Survey Population}        \\
\textbf{Race/Eth} &&&& \\
\hline
\textbf{White}       & 3.60\%                    & 48.80\%                           & 21.40\%                      & \textbf{73.80\%}     \\
\textbf{Black}       & 1.00\%                    & 8.80\%                            & 3.30\%                       & \textbf{13.10\%}     \\
\textbf{Asian}       & 0.20\%                    & 2.80\%                            & 2.60\%                       & \textbf{5.60\%}      \\
\textbf{Hispanic}    & 0.40\%                    & 5.50\%                            & 1.70\%                       & \textbf{7.60\%}      \\
\hline
\textbf{Survey}        & \textbf{5.20\%}           & \textbf{65.90\%}                  & \textbf{29.00\%}             &  \\
\hline
\end{tabularx}
\label{tab:ethXeducensus}
\end{table}

% \begin{table}[]
% \begin{tabular}{lll}
% Ethnicity & Census & Actual   \\
% White     & 68\%                        & 73\%   \\
% Black     & 11\%                      & 13\%  \\
% Asian     & 6\%                         & 5.60\% \\
% Hispanic  & 15\%                        & 7\%  
% \end{tabular}
% \end{table}



%% Feature responses
\begin{figure}[ht]
    \centering
\includegraphics[width=0.5\textwidth]{Images/Features_d_ob.png}
    \caption{Percentage of responses per feature type and level of comfort (ordered left to right: Very uncomfortable, Uncomfortable, Neutral, Comfortable, Very comfortable). }
    \label{fig:features_responses}
\end{figure}

\textbf{Response Analyses.} Figure \ref{fig:features_responses} presents the distribution of participants' comfort levels for each location feature in our study. Trajectory features, including detailed most frequent trips, least frequent trips, most frequent walking trips, and places visited, were frequently labeled as \textit{Uncomfortable} or \textit{Very Uncomfortable}. %Among these, only the "Most frequent trips" feature showed a slightly negative mean. 
In contrast, several obfuscated features showed considerably higher levels of comfort when compared to their detailed counterpart including frequent trips, frequent walking activity, or work location, among others. 
These numbers reveal that efforts to obfuscate individual trajectory and POI visit features are favorably seen by participants, increasing their comfort with these types of analytical approaches. 
Table \ref{tab:featcomfortmean} in the Appendix shows all mean comfort levels per location feature when levels are
transformed into a numeric variable. 

%international visits (both detailed and privacy-preserving), and work locations (both detailed and privacy-preserving) received high mean comfort scores.
%\ref{fig:privacy_preserving_responses}
%shows cross tabulation of percentage of participant responses to privacy attitude questions and context data sharing question. 


The privacy attitudes questionnaire at the end of the survey asked participants to rate their agreement with different statements related to trust in business or in government, or the role of authority, among others. 
Interestingly, trust in institutions and a general predisposition towards compliance appear to increase comfort levels with sharing location data (detailed or obfuscated), while skepticism towards authority is associated with reduced willingness to share location information. Detailed results can be explored in the Appendix (see Table  \ref{tab:attitudes_responses}). These results confirm the importance of adding participant responses to privacy attitude questionnaires in the regression model as control variables, since they appear to play a role in the perception of privacy comfort. 


The distribution of comfort levels for each purpose and for each actor considered in the study confirmed prior work \cite{martin,CISmartHome,contact_tracing_apps,ICADataCollection}
 (see Figure \ref{fig:actor_responses} and Table \ref{tab:actorcomfortmean} in the Appendix for detailed results). 
%\textcolor{green}{Naman add our two usual papers for prior work here}
Participants more frequently responded with "Very Uncomfortable" or "Uncomfortable" for purposes related to optimizing work productivity and monitoring mobility patterns; 
%Converting level of comfort categories into numeric values showed that these purposes received negative mean scores (-0.02 and -0.05, respectively). Notably, 
while purposes with perceived societal benefits, such as identifying locations for infrastructure development, designing walking/cycling infrastructure, and analyzing public transit services, were more frequently rated as \textit{comfortable} or \textit{very comfortable}.
%locations for infrastructure development (mean 0.44), designing walking/cycling infrastructure (mean 0.42), and analyzing public transit services (mean 0.27), were more frequently rated as \textit{comfortable} or \textit{very comfortable}.
On the other hand, and also confirming prior work, participants expressed greater discomfort sharing data with Employers and Federal Government Agencies; while actors such as Academic Researchers, Family, Emergency Services, and Doctors were rated higher on the comfort scale (see Figure \ref{fig:purpose_responses} and Table \ref{tab:purposecomfortmean} in the Appendix for further details).
%for purposes. and in Figure \ref{fig:actor_responses} and Table \ref{tab:actorcomfortmean} for actors. 
%\textcolor{green}{Naman let's move this fig to the appendix. also, in the appendix, make sure each figure has a good legend! where everything is well explained. }


%Figure  illustrates the proportion of comfort levels across all participant answers for each actor considered in the study. Participants expressed greater discomfort sharing data with Employers and Federal Government Agencies. Transforming levels of comfort into the $(-2..+2)$ numeric scale, we can also compute mean levels of comfort for each actor. Employers and Federal Government Agencies received mean scores of -0.42 and -0.07, respectively. Conversely, actors such as Academic Researchers (0.38), Family (0.38), Emergency Services (0.52), and Doctors (0.25) were rated higher on the comfort scale.   
%Table  in the Appendix has a complete list of all mean values when categories are converted into numeric values. 



% Attitude questions which range from Strongly disagree to Strongly agree show positive trend with comfort levels for attitudes like trust in government, trust in businesses, belief that privacy is important, belief in discipline and leader following attitude, and people who give benefit of the doubt. Authority defiance shows negative trend with comfort in sharing location data in general. These findings suggest that individuals' underlying attitudes and beliefs play a significant role in shaping their willingness to share location data. Notably, trust in institutions and a general predisposition towards compliance appear to increase comfort levels, while skepticism towards authority is associated with reduced willingness to share location information
% \subsection{Analysis of text answers}
% Along with a Likert scale to describe the comfortableness level of each question, the respondents were asked a reason for their response. We categorized the responses manually into three types of responses:
% \begin{enumerate}
% \item Gibberish: Responses with text like "ggffhgfh" 
% \item Random/Same: We observed that several respondents would copy paste the same vague text response into all text boxes. These responses fall into random category (Example: "The website is very easy to use" as responses to all the vignette questions)
% \item Good: These text responses were at least remotely relevant to the question asked. (Responses like "I have done nothing wrong")
% \end{enumerate}
% Some repeated and interesting explanations:
% \begin{itemize}
%     \item For questions with sharing international trips features, participants often said that they do not travel internationally and thus would be very comfortable in sharing that information
%     \item Most common terms "nothing to hide" (173 occurrences), "feel comfortable" (154 occurrences),"major invasion" (124 occurrence)
%     \item Several participants (20+) while responding to questions about law enforcement agency accessing location features responded with "I have not done anything wrong why would they try to access my location".
%     \item Several participants thought they wanted to know when the data was being accessed and why. E.g. They felt they needed more details of how the feature would be useful for criminal activity detection. The participants felt they wanted to help the Actors in several scenarios but they also needed more details on how the data would be used. \ref{fig:wordcloud_reasons}
%     \end{itemize}
% APF
\begin{table*}[!htbp]
\centering
\footnotesize
\caption{Regression coefficients for Actors, Purposes, Features, Ethnicity and Education. Odds ratio column is the likelihood of being more comfortable compared to the baseline class. Coefficients are only show for actor, purpose, feature, ethnicity and education. Other control variables are shown in the Appendix Table \ref{tab:attitude_privacy_coeffs}}
\begin{tabularx}{\textwidth}{llllll}
   & Coefficient       & Odds Ratio & Std Err   & Statistic & p-value   \\
\textbf{Actor (Baseline: Commercial Entity)}& & & & & 
\\
Federal government agency&\textbf{ -0.249**}& 0.78& 0.11& -2.266& 0.023
\\
Law enforcement agency& 0.14& 1.15& 0.11& 1.272& 0.203
\\
Local government agency& 0.105& 1.111& 0.097& 1.081& 0.28
\\
Academic researchers& \textbf{0.539**}& 1.714& 0.087& 6.199& 0
\\
Emergency services&\textbf{ 1.166**}& 3.209& 0.185& 6.297& 0
\\
Doctor& \textbf{1.078**}& 2.939& 0.169& 6.393& 0
\\
Employer&\textbf{ -0.626**}& 0.535& 0.149& -4.205& 0
\\
Family&                  \textbf{ 1.409**}&            4.092&           0.166&           8.493&           0
\\
\textbf{Purpose (Baseline: Show ads)}& & & & & 
\\
Analysis of terrorist attacks& 0.09& 1.094& 0.183& 0.488& 0.626
\\
Analysis of Public transit services& -0.047& 0.954& 0.185& -0.255& 0.799
\\
Control spread of diseases& 0.138& 1.148& 0.188& 0.731& 0.465
\\
Monitor mobility patterns& \textbf{-0.654**}& 0.52& 0.178& -3.666& 0
\\
Personal wellness& -0.338& 0.713& 0.212& -1.594& 0.111
\\
Analysis of criminal activity& -0.075& 0.928& 0.179& -0.416& 0.677
\\
Design new walking/cycling infrastructure& \textbf{0.583**}& 1.791& 0.195& 2.986& 0.003
\\
Identify locations for infrastructure& 0.296& 1.344& 0.19& 1.56& 0.119
\\
Optimize work productivity& 0.152& 1.164& 0.224& 0.679& 0.497
\\
\textbf{Feature (Baseline: Places you Visit (Detailed))}& & & & & 
\\
International visits (Detailed)& -0.061& 0.941& 0.209& -0.29& 0.772
\\
International visits (Obfuscated)& \textbf{0.369*}& 1.446& 0.21& 1.753& 0.08
\\
Area you spent most of your time (Obfuscated)& 0.012& 1.012& 0.148& 0.081& 0.936
\\
Modes of transportation (Obfuscated)& 0.192& 1.212& 0.144& 1.327& 0.185
\\
Modes of transportation (Detailed)& 0.01& 1.01& 0.149& 0.066& 0.948
\\
Places you visit (Obfuscated)& 0.166& 1.181& 0.139& 1.191& 0.234
\\
Home location (Obfuscated)& \textbf{0.441**}& 1.554& 0.213& 2.068& 0.039
\\
Home location (Detailed)& -0.101& 0.904& 0.21& -0.48& 0.631
\\
Work location (Obfuscated)& \textbf{0.376**}& 1.456& 0.19& 1.973& 0.049
\\
Work location (Detailed)& -0.2& 0.819& 0.191& -1.043& 0.297
\\
Least frequent trips (Detailed)& -0.198& 0.82& 0.153& -1.292& 0.196
\\
Most frequent trips between counties (Obfuscated)& 0.168& 1.183& 0.148& 1.134& 0.257
\\
Most frequent trips between counties ('Google') (Obfuscated)& 0.23& 1.259& 0.147& 1.562& 0.118
\\
Most frequent trips (Detailed)& \textbf{-0.297**}& 0.743& 0.146& -2.039& 0.041
\\
Most frequent type of trips (Detailed)& -0.063& 0.939& 0.147& -0.431& 0.667
\\
Frequent walking activity (Detailed)& -0.191& 0.826& 0.143& -1.335& 0.182
\\
Frequent walking activity (Obfuscated)&  \textbf{ 0.458**}&            1.581&           0.145&           3.166&           0.002
\\
\textbf{Ethnicity/Race (Baseline: White)}& & & & & 
\\
Asian& -0.017& 0.983& 0.233& -0.072& 0.942
\\
Black& 0.221& 1.247& 0.162& 1.361& 0.174
\\
Hispanic&  \textbf{0.504**}&            1.655&           0.196&           2.574&           0.01
\\
\textbf{Education (Baseline: Highschool to Bachelors)}& & & & & 
\\
Bachelors and above& -0.206& 0.814& 0.126& -1.631& 0.103
\\
Under Highschool & -0.017 & 0.983 & 0.227 & -0.075 & 0.94\\
\multicolumn{6}{r}{\tiny Significance ** $p-value < 0.05$, * $p-value < 0.1$} 
\end{tabularx}
\label{tab:apf_reg}
\end{table*}
% Eth Edu
% \begin{table*}[!htbp]
% \centering
% \footnotesize
% \caption{Regression coefficients for Ethnicity and Education. Significant negative coefficients highlighted in red [Light red (p-value < 0.1), Bright red indicates (p-value < 0.05)] and
% significant positive coefficients are highlighted in green [Light green (p-value < 0.1), Bright green (p-value < 0.05)]}

% \begin{tabularx}{\textwidth}{|Xp{0.5\textwidth}X|XXX|}
%        &                              & \textbf{Coeff}                 & \textbf{O.R.}                 & \textbf{Percentage}           \\
% \rowcolor[HTML]{DAE9F8} 
% \multicolumn{3}{l}{\cellcolor[HTML]{DAE9F8}Ethnicity/Race   (Baseline: White)} & \textbf{}                     & \textbf{}                    \\
%         & Asian                        & 0.031                                 & 1.032                         & 3.2                          \\
%         & Black                        & \cellcolor[HTML]{CCFFCC}0.128*        & \cellcolor[HTML]{CCFFCC}1.137 & \cellcolor[HTML]{CCFFCC}13.7 \\
%         & Hispanic                     & \cellcolor[HTML]{99FF99}0.324**       & \cellcolor[HTML]{99FF99}1.383 & \cellcolor[HTML]{99FF99}38.3 \\
% \rowcolor[HTML]{DAE9F8} 
% \multicolumn{4}{l}{\cellcolor[HTML]{DAE9F8}Education   (Baseline: Bachelors and above)}                        & \textbf{}                    \\
%         & Under Highschool             & 0.112                                 & 1.119                         & 11.9                         \\
%         & Highschool to Bachelor       & \cellcolor[HTML]{99FF99}0.146**       & \cellcolor[HTML]{99FF99}1.157 & \cellcolor[HTML]{99FF99}15.7

% \end{tabularx}
% \label{tab:ethedu_reg}
% \end{table*}


%1) RQ1: effect of individual trajectory and POI visits on user levels of comfort, and their interactions with actors and purposes, 
%2) RQ2: effect of obfuscating approaches on user levels of comfort, and 3) %RQ3: effect of educational background and race and ethnicity on privacy perceptions. 

\section{RQ1: Effect of Location Features, Actors and Purposes on Privacy Perceptions}
%This section explores individuals' attitudes towards sharing data across different components of our contextual design, specifically examining comfort levels in relation to various actors, purposes, and features. Our findings align with prior literature on location data sharing in the contexts of actors and purposes \cite{martin,colombia_privacy_study}.
%To investigate the relationships between actors, purposes, features, education levels, ethnicities, privacy attitudes, and comfort levels in sharing location features, we perform ordinal regression analysis (detailed in Appendix \ref{section:ordinal_regression}). The associations between different contextual settings and location data sharing comfort levels are presented in 
Table \ref{tab:apf_reg} shows the ordinal regression coefficients and odds ratios across actors, purposes and location features. 
Statistically significant coefficients reveal features that have a significant effect on participants' levels of comfort.
%\textcolor{green}{Naman, what is red and green?}
The odds ratio indicates the odds of that effect being more or less likely with respect to the baseline, holding all other variables constant.   

First, we analyze the effect of location features (individual trajectory and POI visit data) on privacy perceptions, and evaluate interaction effects between location features and actors or purposes. 
Next, we describe our main findings for actors and purposes, confirming significant insights found in prior work. 
%We then move to the effect of location features, and their interaction with actors and purposes. 
%, \ref{tab:ethedu_reg}, and \ref{tab:attitudes_reg}. %\textcolor{yellow}{Naman, move these p-value clarifications to the figure (see latex, it's commented out}

\subsection{Location Features}

%As there is extensive literature on location privacy perceptions for different types of Places of Visit, we chose Places of visits (detailed) as the reference variable. We include Home, Work, Restaurants, Bar, Park, public transit stops in the Places of visit as these are frequently used in literature. 
%It is important to highlight that places of visits (detailed) is associated with a higher proportion of \textit{very uncomfortable} and \textit{uncomfortable} responses(Figure \ref{fig:features_responses}). 

%Thus location features which are less comfortable to share than places of visits (detailed) would be perceived much more negatively than the rest. 

Table \ref{tab:apf_reg} shows that participants expressed statistically significant lower levels of comfort when sharing detailed trajectory data compared to detailed visits to POIs (regression baseline). For example, the odds of being comfortable sharing Most Frequent Trips (Detailed) was 0.74 times that of sharing detailed visits to POIs. In other words, participants were significantly less comfortable with the use of detailed trajectory data than with the use of visits to specific types of POIs (restaurants, libraries, schools, etc).
%the places of visit. 
As one participant stated:
%Participant- stated in the open-ended questions: 
"There is no good reason for them to monitor my trips on the regular at all."
%"When you show it on a map..it looks like stalking...it feels like some is following me"
(Participant-8a6a). 
%This finding could lead to serious implications for data shared by location data brokers (usually longitudnal data) from which actors can derive such movement features. Policies need to be set in place to restrict derivation of such features. Interaction analysis (sheet A-P-F) shows that Actors: Academic researchers and Emergency services, are perceived positively when accessing most of the Movement features (Least frequent trips, most frequent trips obfuscated at county level, most frequent types of trips). Thus privacy perceptions for movement features are dependent on actors. People are specific about the actors that can access their movement data and data brokers can include actor-specific restrictions on types of analysis that can be done on data in their platforms. 


Participants, on the other hand, demonstrated higher comfort levels 
sharing obfuscated trajectory data compared to detailed visits to POIs. For example, walking (Frequent Walking Activity) was 1.5 times more likely to be rated as comfortable and international trips (International Visits) were 1.44 times more likely to be rated as comfortable than detailed visits to specific points of interest. 
Finally, participants were significantly more comfortable sharing their home and work locations (obfuscated to the census tract) than sharing detailed visits to specific POIs, although no significant difference was observed when home and work locations were not obfuscated (we discuss this further in RQ2).
%, their home and work locations and International visits when these were obfuscated. As we will discuss in RQ2, this result suggests that obfuscation approaches can positively influence levels of comfort in selected cases, increasing the willingness to share these features. But more importantly, it 
Interestingly, these results reflect that if handled appropriately, users are willing to share some of their trajectory data. As  
Participant-D303 shared, "This is general information and I would be interested in this as well". 

 \begin{table}[ht]
\centering
\footnotesize
\caption{Kruskal-Wallis analysis with post-hoc Dunn test results for a subset of pairs (full analysis in Table \ref{tab:kw_featureXactor_featureXpurpose_BIG} in Appendix). M"i" represents median level of comfort value for distribution "i". Significance at p-value <0.05)}
\begin{tabularx}{\linewidth}{|p{0.15\linewidth}|p{0.18\linewidth}|p{0.18\linewidth}|p{0.05\linewidth}|p{0.05\linewidth}|p{0.08\linewidth}|}
\hline
\textbf{Feature} & \textbf{Actor1} & \textbf{Actor2} & \textbf{M1} & \textbf{M2} & \text{\footnotesize{M2-M1}}\\
\hline

Freq Walks(Ob) & Employer & Researchers & 0 & 1 & 1 \\
\cline{3-6}
&& Doc & 0 & 1 & 1 \\
\cline{3-6}
 &  & Family & 0 & 1 & 1 \\
 \hline
Freq Walks(Ob) & Fed & Researchers & 0 & 1 & 1 \\
 \cline{3-6}
 &  & Doc & 0 & 1 & 1 \\
 \cline{3-6}
 & & Family & 0 & 1 & 1 \\
 \hline
Freq Walks(Ob) & Local gov & Doc & 0 & 1 & 1 \\
\hline
\textbf{Feature} & \textbf{Purpose1} & \textbf{Purpose2} & \textbf{M1} & \textbf{M2} & \text{\footnotesize{M2-M1}} \\
\hline
Visits(Ob) & Monitor mobility & Public transit & 0 & 1 & 1 \\
\cline{3-6}
&  & Infra (Walk/Cycling) & 0 & 1 & 1 \\
\hline

\end{tabularx}


\label{tab:kw_featureXactor_featureXpurpose}
\end{table}

\textbf{Interactions with Actors and Purposes}
%\textcolor{green}{Naman, add here the KS results for this section, the others can be in the appendix, but we need one sample in the main paper. Also, I've been thinking about OSF, and I think it looks better if we move everything to the appendix...}

Table \ref{tab:kw_featureXactor_featureXpurpose} shows a subset of results for the Kruskal-Wallis (KW) and Dunn tests between the levels of comfort for individual trajectory data and POI visit features and the different types of actors and purposes considered in our study (please see Table \ref{tab:kw_featureXactor_featureXpurpose_BIG} in the Appendix for full details). 
The results show two interesting findings. First, the  %perception for walking trajectories (Frequent Walking Activity (Ob)) was higher than for detailed visits to POIs, the statistical tests show that the 
median level of comfort sharing obfuscated trajectory data decreases depending on the actor accessing the datwhile for 
  researchers, family, or doctors the median level of comfort is "comfortable" (median=1, p-value<0.05) for Federal Government agency or Employer actors the level of comfort is neutral (median=0, p-value<0.05)).
Second, the levels of comfort with respect to visits to POIs (Places you visit (Ob)) has significantly higher median values for purposes with social impact such as "Identifying locations for infrastructure or cycling" (median=1, p-value<0.05) than for more generic purposes, such as "monitoring mobility patterns" (median=0, p-value<0.05).

%Interestingly, for Places of Visit (Ob), the vague purpose monitoring mobility patterns has lower median compared to purposes Identifying .   

% Please add the following required packages to your document preamble:
% \usepackage[table,xcdraw]{xcolor}
% Beamer presentation requires \usepackage{colortbl} instead of \usepackage[table,xcdraw]{xcolor}


\subsection{Actors and Purposes}
%As commercial entities using location data for relevant ads has become ubiquitous, we choose the Commercial entity as the reference actor.

Our ordinal regression results confirm prior findings for the effect of actors and purposes on privacy perceptions \cite{martin, ICADataCollection}.  
%Our study examined comfort levels in sharing location data with 9 actors. The mean responses for each actor are presented in Table \ref{tab:actorcomfortmean}, while the regression estimates, p-values, and odds ratios are detailed in Table \ref{tab:apf_reg}. 
%Key findings include:
%\begin{itemize
Table \ref{tab:apf_reg} shows that 
Federal Government Agencies and Employers were significantly negatively related to comfort when accessing location data. In fact, participants had 0.78 and 0.53 times the odds (p<0.01) of rating their level of comfort lower for these actors than for commercial entities (regression baseline). 
%the odds of these actors being associated with higher levels of comfort are 0.78 and 0.53 times than the baseline (commercial entities). 
On the other hand, Academic Researchers, Emergency Services, Family members, and Doctors were positively related to levels of comfort, with participants having
%odds of these actors being associated with higher levels of comfort were 
1.7, 3.2, 4 and 2.9 times the odds (p<0.01) of rating their level of comfort higher for these actors than for commercial entities (baseline).  
%Some of the comments provided in the free-form text box in the survey point to potential reasons behind these varying levels of comfort. 
%For example, negative perceptions around federal government agencies: "I do not trust the CIA due to their horrific crimes against humanity and do not feel comfortable with them or any other government organization tracking me. As a woman, I especially do not want them familiar with my health clinic and how often I visit." (Participant-acba) and Employers "My employer should have no right to view my movements outside of work. This would make me very interested in leaving a job that would do this." (Participant-0856);
%\textcolor{yellow}{Naman, can you add as a negative example one that mentions federal government or employer. the current example is about law enforcement, but that was not significant.}
%while positive perceptions were related to convenience and peace of mind: "I would want my family to know where I'm traveling so I have peace of mind if I have an accident or am attacked. They'd be able to at least figure out my last location" (Participant-3856).
%Some of these results were also revealed in \cite{martin, ICADataCollection}. 

%\textcolor{pink}{There are no differences of significance or coefficient signs in our studies..the difference in coeff magnitude is because of different scales. So the next sentence can be removed.."We posit that the differences in actor significance across studies is possibly due to the fact that our proposed regression models consider both POI-based and trajectory-based location features, instead of only POIs. "}

%\textbf{Analysis of Actors' perceptions for different purposes and features.} 
%To understand how the comfort perceptions change for actors when they access the diverse set of location features we consider in our model, 
%we carry out a Kruskal-Wallis (KW) test across the levels of comfort for all significant actors and all location features, followed by post-hoc Dunn tests to identify specific pairs of significant differences.
%We tabulate the Dunn test results in excel workbook and upload it anonymously to OSF \footnote{\url{https://osf.io/hy9m3/?view_only=e6dd3400991946afa3b265df5b1f3132}}.Each sheet has data about different comparisons. Insights from A-P-F are about pairwise comparisons between different Actors, purposes and features.


%From Table \ref{tab:apf_reg}, the ordinal regression results for the purposes reveal that, across actors and location features, there are two purposes that are significantly related to levels of comfort. 

Moving on to purposes, we observe that purposes with social impact (e.g., Designing public infrastructure) were perceived more favorably than marketing purposes (regression baseline), with the odds ratio being 1.79 times higher. 
%of being more comfortable sharing location data for that purpose being 1.79 times that of the baseline. 
On the other hand, more general purposes (e.g., Monitor mobility patterns) were associated with significantly lower comfort levels compared to the baseline, i.e. people were approximately half as likely to be comfortable with sharing data (p-value<0.05).

%for the vague purpose than for the purpose of showing targeted ads. 
%On the other hand, d

%The interaction analysis between these purposes and location features did not reflect any statistical differences, 
%pointing to similar levels of comfort across all location features. 

%\textcolor{pink}{Vanessa, I added this block because I think my reasoning makes sense. I argue that vague purposes are bad and actors need to specify the reason clearly. Except for Feds and Employers, specifying purpose can make people more comfortable with sharing locaiton data:: We also analyze how the vague purpose monitor mobility patterns is interpreted for different actors. We observed that the vague purpose was negatively perceived for Employers and Federal government agencies while being perceived positively for Emergency services. For other actors like Commercial entities, academic researchers and doctors, it was perceived neutrally. In the absence of a specific purpose, the participants would have used their worst case assumptions about the particular actor (because "monitor mobility" has negative connotations associated to it). Its important to note that there were no pairs of statistically significantly different location features for monitoring mobility which means the comfort decision of sharing location data in the case of vague purpose rests only on the actor and not the kind of feature being used. This is an important finding for actors who want to access location features: Except for Federal agencies and Employer, specifying purposes helps avoid negative stereotypes about location data access.}    


%\textcolor{pink}{Vanessa, another new para here:: Another important insight from the collected data is about purpose personal wellness. We observe that employer has been rated negatively compared to all other significant actors. As we do not see specific location features that the participants were comfortable in sharing with employers, we indicate that employers should not be using location based information to determine wellness of employees.}


\textbf{Interaction Analysis.} 
%Snippets of  We upload the complete tables in OSF.
%\textcolor{red}{Naman, do not mention OSF in the main paper, simply point to the appendix, and there, put a link to OSF with an explanation}
A Table with the Kruskal-Wallis and Dunn test results for the interaction between actors and location features can be found in Table \ref{tab:kw_actor_feature_BIG} in the Appendix. 
We would like to highlight a couple of significant findings. 
First, while the effect of the actor being a Doctor is positive on the level of comfort (as we showed in Table \ref{tab:apf_reg}), the perception is more positive when sharing walking activity (obfuscated) (median level of comfort=1) than other features such as most frequent type of trips (obfuscated) (median=0, neutral) or most frequent trips (detailed) (median=-0.5, leaning uncomfortable).
On the other hand, when the actor is an Employer, giving access to the most frequent places of visit (obfuscated) was perceived as more uncomfortable (median=-1, uncomfortable) than giving access to walking activity (obfuscated) (median=0, neutral). 
%Additionally, International trips (Detailed) is seen positively compared to frequent trips (median-0.5) for Actor Federal agencies. 
%These point to how same actors have inherent perceptions which lead contrary results for few features. 
% While the effect of doctors is positive on the level of comfort to share location data \ref{tab:apf_reg}, that perception is significantly different between sharing obfuscated  walking activity and other types of movement data (e.g. area people spend their time, most frequent trips, etc), with the latter being associated with much lower levels of comfort. Median for most frequent trips for doctors is less than 0 indicating people were uncomfortable in sharing their most frequent trips even with doctors. Overall, we see people were especially uncomfortable in sharing most frequent trips and most frequent trips (obfuscated at county level). This suggests strong boundaries for trajectory data being shared with Federal government agencies, Employers and Doctors. 
Finally, no significant differences were identified between purposes and location features, pointing to similar levels of comfort independently of the trajectory or POI visit feature.
%\textcolor{red}{Tables X, Y and Z in the Appendix contain details for each of the interaction analyses discussed here and their Kruskal-Wallis and Dunn results. }

% In addition, there exists a clear distinction between the use of privacy-preserving movement features and detailed features, with levels of comfort being much lower for detailed features. 

%The lack of significance for all other pairs reflects that there were no statistical differences in levels of comfort across location features for the other significant actors identified in the regression ( Differences < 1 were dropped from the findings table).  



%\subsubsection{Employer Access to Location Data}%Analysis of the data reveals a consistent pattern regarding employer access to location information:
%\begin{itemize}
%    \item Places of Visit: Participants expressed significantly higher discomfort in sharing place-of-visit information \ref{fig:poi_act_interact} with employers compared to most other actors. This negative perception persisted regardless of whether the data was presented in a detailed or privacy-preserving format.
%    \item Broad Discomfort: Notably, employer access to any of the 10 location features examined in this study was perceived negatively by participants (Table. \ref{tab:actorcomfortmean}, Fig. \ref{fig:act_edu_interact},\ref{fig:act_pur_interact},\ref{fig:hwt_act_interact})
%\end{itemize}
%These findings suggest a general reluctance among individuals to share location data with employers, indicating a potential area of concern in workplace privacy. This consistent negative perception across various types of location data underscores the sensitivity of employer access to personal location features.

\iffalse 
\subsection{Purpose}
As stated earlier, we use the purpose Showing ads as the reference variable for purpose.

From Table \ref{tab:apf_reg}, the ordinal regression results for the purposes reveal that, across actors and location features, there are two purposes that are significantly related to levels of comfort. The general purpose of monitoring mobility patterns was associated with significantly lower comfort levels compared to the baseline (showing targeted ads), i.e. people were \~ half as likely to be comfortable with sharing data for the vague purpose than for the purpose of showing targeted ads. 
On the other hand, designing public infrastructure was perceived more favorably for sharing location features, with the odds ratio of being more comfortable sharing location data for that purpose being 1.79 times that of the baseline. 

The interaction analysis between these purposes and location features did not reflect any statistical differences, 
pointing to similar levels of comfort across all location features. 



\textcolor{pink}{Vanessa, another new para here:: Another important insight from the collected data is about purpose personal wellness. We observe that employer has been rated negatively compared to all other significant actors. As we do not see specific location features that the participants were comfortable in sharing with employers, we indicate that employers should not be using location based information to determine wellness of employees.}


\fi 


\section{RQ2: Effect of Obfuscating Approaches on Privacy Perceptions}

Looking into the Features column in Table \ref{tab:apf_reg} we can observe that some obfuscated trajectory and POI visit features are associated with statistically significant higher levels of comfort than detailed features (Places you visit (detailed) as baseline).
Specifically, obfuscated frequent walking activity as well as obfuscated international trips, obfuscated home and work locations had
1.58, 1.4, 1.55 and 1.45 times the odds (p<0.05) of being rated higher in level of comfort than detailed places visited, 
%\textcolor{green}{Naman, please add some numbers here, otherwise is too verbose}
%The location feature analysis for RQ1 (see Table \ref{tab:apf_reg}) revealed statistically significant negative levels of comfort for the use of frequent trips movement feature, and statistically significant positive levels of comfort for the use of obfuscating features walking activity, work location and home location, 
pointing to individuals being more prone to sharing certain types of trajectory and visits data if protected by obfuscation approaches. 
Beyond statistical significance, it is important to highlight that when looking into mean levels of comfort for each detailed and obfuscated location feature, all means were higher for obfuscation features than for their detailed counterpart e.g., Home (Detailed - Obfuscated) = 0.055, Modes of Transport (Detailed - Obfuscated) = 0.045.   
In other words, although not significant in the regression model, obfuscated location features are associated with more positive levels of comfort towards data sharing. Table \ref{tab:featcomfortmean} in the Appendix shows all mean levels of comfort across location features. 
%\textcolor{green}{Naman, can we add some numbers here? aren't these numbers from Figure 1? The text as is it's too dry, we need numbers.  }


%To understand more in depth the effect of obfuscation approaches on levels of comfort, we carry out an interaction analysis to evaluate how levels of comfort for obfuscation and detailed approaches might change given different types of actors and purposes.
\textbf{Interaction Analysis.} 
We perform KW and post-hoc Dunn tests between the levels of comfort associated with "Detailed" location features and "Obfuscated" location features across types of actors and purposes. Detailed results are shown in 
%We construct the feature "privacy", with values "Detailed" (if the vignette had detailed feature) and "Obfuscated" (if the vignette had obfuscated feature) and perform Kruskal wallis for actors and purposes. Detailed KW and post-hoc Dunn tests results for actor-obfuscation interaction and purpose-obfuscation interaction are shown in 
the Appendix Table \ref{tab:kw_ap_obfuscate_BIG}. Here, we discuss two relevant findings.
First, we observe that when the actor accessing the location feature is a family member (actor=Family), the user level of comfort increases from neutral to comfortable if the location feature is obfuscated instead of detailed. 
%has a median that is 1 level statistically significantly higher for obfuscated location features than for detailed. 
%\textcolor{green}{Naman, please add a value in X in the sentence above and correct it. Also, add p-value for KS,,...make it more numeric.}
Similarly, we found that obfuscated features improved the privacy level of comfort when compared to detailed features for certain purposes such as "control the spread of diseases". This finding aligns with studies that show privacy preserving location sharing apps during COVID were perceived more positively compared to detailed location sharing app\cite{privacy_covid,privacy_covid_2,surveillance_covid}.

%KW followed by Dunn post-hoc test shows difference for obfuscated (median=1) vs detailed(median=0) Frequent walking activity feature (median difference=1).



%\textcolor{green}{Add reference for this, also was is difference of medians 1???}

%\textcolor{green}{Added covid can we add maybe one more interaction, or there were no other findings?}
%We observe that only Actor Family has statistically significantly different median for obfuscated and Detailed. This could be potentially interesting avenue for app developers who develop location sharing apps targeted towards families. Including obfuscation of features could be a potential direction of exploration for such apps.



% Differences in comfort levels between detailed and privacy-preserving location features were observed only for one purpose: monitoring mobility. Participants demonstrated higher comfort levels in sharing privacy-preserving features compared to detailed features. In fact, detailed features had a mean negative level of comfort (-0.1), while privacy-preserving location features had positive mean levels of comfort (0.1). 

% \subsection{Demographic Analysis}
% We extended this analysis to examine potential differences in perceptions of detailed versus privacy-preserving features across various ethnicities and education levels. These findings are presented in the next section. 
% \subsection{Implications}
% These results suggest that the implementation of privacy-preserving techniques in location data sharing can significantly impact user comfort levels, particularly for certain actors and purposes. The lack of difference for some actors, such as law enforcement and federal agencies, may indicate a general discomfort with these entities accessing location data, regardless of the level of detail. 

\section{RQ3: Effect of Race/Ethnicity and Education on Privacy Perceptions}
%Location Data Sharing Comfort Levels }
%This section examines how demographic factors, specifically education and race/ethnicity, influence individuals' comfort levels in sharing location data.
%Table \ref{tab:apf_reg} shows the ordinal regression coefficients, odds ratio and corresponding percentages for each race and ethnic group and for each educational level. Next we discuss our main findings, and interaction effects between education, race and actors, purpose and location features. 

\subsection{Race and Ethnicity}
%For this study, we collected a sample of participants that are almost representative of the four largest racial and ethnic groups in the U.S.: White, Latino/Hispanic, Black and Asian. 
The results in Table \ref{tab:apf_reg} show that Hispanic participants are associated with statistically significant higher levels of comfort that White participants (regression baseline), having 1.6 times the odds (OR=1.6, p<0.01) of rating their level of comfort with location data higher than White participants.  
Some of the free-form text boxes align with these observations. For example, participants who self-identified as Hispanic said: "I have nothing to hide so it doesn't really make a difference to me." (Participant-050e); and 
 %   \item "I would be willing to share my location data with family, friends and some research companies/organizations as long as they weren't too invasive and it was solely to keep us safe."
"(In) this particular scenario I feel very comfortable with my data being accessed \& used for this purpose. Hospital locations are important and their location within a community can severely impact the quality of life for that community too." (Participant-2ae1).
On the other hand, no significant difference in level of comfort was observed between any other racial or ethnic groups. 

\textbf{Interaction analysis.} The KW and Dunn statistical tests between the levels of comfort across race/ethnicity and location data types, actor types and purposes revealed interesting nuanced privacy perceptions. Detailed results can be found in Tables \ref{tab:kw_ethnicity_act,tab:kw_ethnicity_p_f_priv} in the Appendix. Here, we discuss the most relevant findings. For Hispanic individuals, the levels of comfort were the lowest when their location data was used for marketing purposes i.e. showing ads (median=-1 "uncomfortable, p-value<0.05). 
%This difference was observed when comparing comfort levels of Hispanic group against all other purposes considered in the study.
%\textcolor{green}{We see that the median is -1 for all the Dunn test results for all the purposes...Naman, can you be more specific here, what was the test that was stat sign different? what location data? }
%Showing ads is the least favorable purpose for Hispanic group (compared to any other purpose). 
In contrast, Black participants perceived marketing purposes as positive (median=1 "comfortable", p-value< 0.05) when compared to other purposes like analysis of terrorist attacks (median=0), public transit (median=0), mobility monitoring (median=0).

Nevertheless, no significant differences were observed between Hispanic ethnicity and actor or feature types; neither between Black and actors or feature types.  
%We do not find any differences in actors or features. 
This might indicate that, for Hispanic and Black participants, data comfort revolves around purpose of data access, and not around who access the data (actor) or the specific type of data accessed (feature). 

%We identify the three purposes: Analysis of Public transit services, Analysis of terrorist attacks and Monitor mobility patterns as the purposes which are differentiating factors in privacy perceptions for black participants.

On the other hand, for Asian and White respondents there was a significant change in level of comfort between detailed and obfuscated features with, for example, "Most frequent trips (Detailed)" having a lower median comfort (median=0, p-value<0.05) than "Frequent walking activity (Obfuscated)" (median=1) for White individuals. 
In addition, the tests also showed Asian and White respondents being significantly less comfortable with the Federal Government, Law enforcement, Local government (median=0) and Employers (median=-1) than with Researchers, Family, Doctors or Emergency services (median=1). 
%\textcolor{green}{Naman, numbers, if the reader cannot see the table, we need to add all numbers n the text!!}.
White participants were also more comfortable in sharing data for infrastructure purposes (designing cycling infrastructure and designing built infrastructure) (median=1, p-value<0.05) than for all the other purposes (median=0).
%Interestingly, there is no statistical difference between purpose Monitor mobility patterns (vaguely worded) and purposes like Analysis of criminal activity, Analysis of terrorist attacks, Control spread of diseases, Ads, etc. We also see that for location features, 
Thus, for Asian and White participants we find evidence that the combinations of actors, purposes and features are responsible for the overall privacy perception.

%For Asian participants, we observe that the actor family accessing location features was rated higher (median=1,pval<0.05) than Employer(median=-1,pval<0.05) and Federal government agency (median=0,pval<0.05). When comparing Obfuscated and Detailed features, Asian participants felt more comfortable in sharing obfuscated features (median=1,pval<0.05) compared to detailed features (median=0,pval<0.05). 
%\textcolor{green}{add something for Asian?}
% of their data activity than their frequent trips, and sharing privacy-preserving location features than detailed ones. 
% Similarly to Black participants, monitoring mobility had the lowest level of comfort, followed by advertising and infrastructure purposes, which had a positive level of comfort. 
% No significant differences across actors, purposes or location features were observed for the Hispanic group. 
%and purpose P:Monitor mobility was and P:Ads was significantly less comfortable as compared to P:Infra (building). This is contrast to Black participants who were uncomfortable in sharing it with commercial entities. F:FreqT(D) \textcolor{red}{This is only 1 so not plotting it} were significantly more uncomfortable when compared to Freq Walks (Ob) for white ethnicity. 
%\textcolor{red}{This is only 1 so not plotting it}  
%We observed that there was a significant difference between Detailed and Obfuscated features for White participants.  
% "Ethnicity has an effect on level of comfort, but there are no significant differences in terms of comfort across actors",
%\textcolor{yellow}{Naman, please re-do Figs 8 and 9 with race being the outer most feature and only showing the significant ones ie.. ,no Asian. Also, why is there no figure for feature-race, it's because there is only one significant value? no signifcant values for the other races, only white?}
%\subsubsection{Overall Feature Perception}
%The analysis revealed significant differences in the overall distribution of comfort levels across racial/ethnic groups (Kruskal-Wallis statistic: 132.083, p-value < 0.005). However, no specific features showed statistically significant differences in comfort levels among participants of different racial/ethnic backgrounds.
%\subsubsection{Actor-Ethnicity Interaction}
%Figure \ref{fig:act_eth_interact} illustrates the interaction between actors and race/ethnicity. Key findings include:
%\begin{itemize}
%    \item Hispanic participants consistently reported higher comfort levels across all actors compared to other racial/ethnic groups.
%    \item \textbf{Employers:} White/Caucasian participants expressed significantly lower comfort levels compared to Black or African American participants, who reported neutral comfort levels.
 %   \item \textbf{Commercial entities:} White/Caucasian participants reported lower comfort levels compared to Hispanic and Black/African American participants.
  %  \item \textbf{Federal Agencies:} No significant differences in comfort levels were observed across racial/ethnic groups.
  %  \item \textbf{Academic Researchers:} Asian, Black, and White participants reported lower comfort levels compared to Hispanic participants.
%\end{itemize}
%\subsubsection{Purpose-Ethnicity Interaction}
%Figure \ref{fig:pur_eth_interact} depicts the interaction between purposes and race/ethnicity:
%\begin{itemize}
 %   \item Hispanic participants generally reported higher comfort levels across most purposes.
%    \item \textbf{Law enforcement purposes:} For identifying criminal activity and terrorist activity, White participants expressed lower comfort levels compared to Black and Hispanic participants, respectively.
 %   \item \textbf{Advertising:} Hispanic participants reported significantly lower comfort levels compared to Black and White participants for the purpose of showing ads.
 %   \item 
%\end{itemize}
%\subsubsection{Feature-Ethnicity Interaction}
%Analysis of feature comfort levels within each racial/ethnic group revealed:
%White participants reported significantly lower comfort levels (p-value < 0.05) with sharing Frequent Trips (Detailed) compared to Walking (Detailed).

\subsection{Education}
%For this study, we consider the three educational levels we collected in the demographic component of the survey: 
%1) Up to High School: Participants who reported education levels up to high school completion; 2) High School to Bachelor's: Participants with educational attainment beyond high school but not including a bachelor's degree; and 3) Bachelor's and Above: Participants with a bachelor's degree or higher (including postgraduate diplomas, PhDs, etc.). 
We do not observe any statistically significant difference in comfort levels for different education groups when compared to participants with education high school to bachelors (baseline). 
%We see significant positive association between Computer science and programming experience \ref{tab:attitude_privacy_coeffs} and comfort levels. With odds being 1.5 higher when people have experience with computer science or programming. 

\textbf{Interaction analysis.} Kruskal-Wallis and post-hoc Dunn tests to assess interactions between levels of comfort across education levels and actors, purposes and location features revealed some interesting insights (see Table \ref{tab:kw_bachelors_apf} in the Appendix for full details). Participants with a "Bachelors and above" felt more comfortable in sharing obfuscated features (median=1,pval<0.05) compared to detailed (median=0, p-value<0.05). 

Interestingly, no statistical difference between detailed and obfuscated location data was observed for the other two educational groups "High School to Bachelors" and "Under High School". 
Participants with a "Bachelors and above" also showed significantly lower levels of comfort with purposes related to monitoring mobility (median=0, p-value<0.05) and with sharing location data with employers (median=-1, p-value<0.05). These findings indicate that decisions of this educational group depend on the interaction of the full spectrum of actor-purpose-feature (along with privacy attitudes). 

Participants with education "High School to Bachelors" were significantly less comfortable in sharing their location data with the Federal Government, law enforcement agencies (median=0, p-value<0.05) or employers (median=-1,p-value<0.05) than with Researchers and Emergency services (median= 1,pval<0.05). These results reveal that the privacy perceptions for this educational group are mostly shaped around actors and purposes, and not types of features or obfuscation techniques. Finally, we did not observe any significant difference between pairs of actors, purposes or features for participants with "Under High School" education, signaing that for these participants their decisions might be mostly shaped by their individual privacy attitudes.



%had Monitor mobility patterns as less comfortable compared to most other purposes. But the median for Monitor mobility patterns for this group is 0 (neutral on likert scale). Unlike other sections, here we observe that Bachelors and above education group had positive median for the vague purpose Monitor mobility patterns. 





%High School to Bachelors participants showed significantly lower levels of comfort for the . They also saw Design new walking/cycling infrastructure and Identify locations for infrastructure positively compared to most other purposes. The privacy perceptions of this group depends on the context (actors, purposes and features) along with the person's privacy practices and attitudes.





% \section{Predictive Modeling}
% This section explores the creation of models that predict levels of comfort in sharing location data a given a combination of actor, purpose and location feature. 
% Such models can help companies evaluate levels of comfort with different types of location features before they are actually implemented and deployed in their location intelligence platforms.  
% For example, a company could be interested in evaluating the level of comfort of a Hispanic person in sharing their movement data with law enforcement for public safety.
% %\subsection{Modeling Scenarios}
% %We investigate two distinct scenarios for modeling the dependent variable (level of comfort): 1) General privacy attitudes are unknown, and 2) Population attitudes are known. 
% %As this tool is intended to gauge public consensus for different actors, we consider three approaches to labeling the dependent variable (comfort): 1) 5-point Likert scale (as used in the original survey), 2) 3-point Likert scale (uncomfortable/neutral/comfortable), and 3)  Binary classification (uncomfortable+neutral/comfortable). 

% \subsection{Model}
% We frame the prediction task as a classification task with the classes being the levels of comfort. We explore two types of models. In \textit{Model 1}, our objective is to predict the level of comfort of a given demographic group with a given vignette (framed as actor, location feature, factor). This model would allow a company to evaluate the level of comfort with the use of a location feature associated with a specific population group.  
% We also evaluate a second model \textit{Model 2}, whereby privacy attitudes from the population of interest are also known, either because the company knows them, or because they can run a privacy attitudes survey among their population of interest. 
 
% We consider three different approaches to defining the classes in the classification task: 1) 5-point Likert scale (as used in the original survey), 2) 3-point Likert scale ((very) uncomfortable, neutral and (very)comfortable), and 3) Binary classification ((very) uncomfortable or neutral vs. (very) comfortable). We perform an 80\% train 20\%test split for training and model evaluation, and compare  the performance of these machine learning algorithms (suitable for classification tasks): Random Forest (RF), Gradient Boosting Machines (XGB) and K-Nearest Neighbours (KNN). To assess model performance, we utilize the following metrics:
% Accuracy, Precision, Recall, F1-score and Cohen's Kappa.
 
% %\textcolor{yellow}{Naman, please add one sentence explaining \% of training and testing samples.}
% %Input features for our models include: Demographic information (e.g., age, education level, ethnicity) and Contextual elements (actor, purpose, and feature type). Attitudes are known in case 2 but not in case 1.
% %\subsection{Model Performance: Demographics-Only Scenario}
% %\textcolor{yellow}{
% %2 is for uncomfortable vs comfortable
% %3 keeps the neutral value
% %5 is for the very %uncomfortable,uncomfortable,neutral,comfortable,very comfortable 
% %In 4 we would have to merge neutral with comfortable or uncomfortable. Some reviewer might be like..Why is neutral in the uncomfortable or comfortable. Thats why I skip 4. 
% %Naman, why is there not a 4-point Likert?}

% \subsection{Performance Analysis}
% %- Model 1}

% \textbf{Model 1.} Table \ref{tab:pred_context} presents the performance metrics for predicting comfort levels given a tuple of actor, purpose and feature as well as general demographic information of the population of interest to the company. 
% %when only demographic information of the public is known.
% We can observe moderate accuracy and F1 scores across models and number of classes with ranges (0.299-0.695) and (0.279-0.325), respectively. As expected, higher number of classes are associated with lower accuracy values. Nevertheless, the F1 scores are the lowest for the binary classification pointing to challenges in balancing precision and recall, possibly due to class imbalance.
% %\subsubsection{Interpretation of Results}
% %The performance of these models provides several insights:
% %\begin{itemize}
% %    \item Binary Classification: The Random Forest model's accuracy of 65.8\% suggests a moderate ability to distinguish between comfortable and not uncomfortable responses. However, the relatively low F1-score (0.27) indicates challenges in balancing precision and recall, possibly due to class imbalance.
%  %   \item Multi-class Classification: XGBoost's superior performance in both 3-point and 5-point Likert scale predictions demonstrates its effectiveness in capturing the nuanced differences in comfort levels. The specific Kappa scores would provide more insight into the strength of these predictions.
% The moderate performance across all models suggests that vignette and demographic information alone may not be sufficient for highly accurate predictions of comfort levels in location data sharing. This underscores the complexity of privacy perceptions and the potential influence of other factors not captured in demographic data such as privacy attitudes.
% %    \end{itemize}

% \begin{table}[]
% \centering
% \footnotesize
% \caption{Detailed classification report for the three labelling approaches. Best metric per classification approach are in bold. XGB: XGBoost, KNN: K-nearest Neighbours, RRandomForest, CK: Cohen Kappa score}
% \begin{tabularx}{0.5\textwidth}{@{}lc*{5}{>{\centering\arraybackslash}X}@{}}
% \toprule
% \textbf{Model} & \textbf{Class} & \textbf{Acc.} & \textbf{Prec.} & \textbf{Rec.} & \textbf{F1} & \textbf{CK} \\
% \midrule
% XGB            & 2              & \textbf{0.695} & 0.155         & \textbf{0.455} & 0.232      & \textbf{0.095} \\
% KNN            & 2              & 0.647          & 0.219         & 0.349          & 0.269      & 0.052 \\
% RF             & 2              & 0.658          & \textbf{0.223} & 0.372         & \textbf{0.279} & 0.073 \\
% \midrule
% XGB            & 3              & \textbf{0.481} & 0.587         & 0.481          & 0.521      & \textbf{0.118} \\
% KNN            & 3              & 0.478          & \textbf{0.605} & 0.478         & 0.527      & 0.106 \\
% RF             & 3              & 0.459          & 0.530         & \textbf{0.459} & \textbf{0.486} & 0.096 \\
% \midrule
% XGB            & 5              & \textbf{0.299} & \textbf{0.385} & \textbf{0.299} & \textbf{0.325} & \textbf{0.060} \\
% KNN            & 5              & 0.287          & 0.380         & 0.287          & 0.318      & 0.042 \\
% RF             & 5              & 0.293          & 0.351         & 0.293          & 0.313      & \textbf{0.060} \\
% \bottomrule
% \end{tabularx}
% \label{tab:pred_context}
% \end{table}
% %\subsection{Model Performance: Demographics and Context Known}
% %\subsection{Preformance - Model 2}
% \textbf{Model 2.} Here, we examine the predictive performance of Model 1 enhanced with privacy attitude data from the population of interest. 
% %when both demographic information and contextual data are available for modeling. 
% As Table \ref{tab:pred_attitude_context} shows, adding privacy attitudes as independent variables in the prediction model consistently improves prediction performance across all metrics: accuracy, precision, recall, and Cohen-Kappa; with improvements over 50\% for 3-point and 5-point Likert scales when compared to Model 1. Accuracy and F1 score values for Model 2 were in the ranges (0.517,0.783) and (0.517,0.600), respectively. The best performing model was the Random Forest which showed superior performance across all classification settings (binary and multi-class).
% % TABLE Attitude + Context

% \begin{table}[]
% \centering
% \footnotesize
% \caption{Detailed classification report for the three labelling approaches. Best metric per classification approach are in bold. XGB: XGBoost, KNN: K-nearest Neighbours, RRandomForest, CK: Cohen Kappa score}
% \begin{tabularx}{0.5\textwidth}{@{}lc*{6}{>{\centering\arraybackslash}X}@{}}
% \toprule
% \textbf{Model} & \textbf{Classes} & \textbf{Acc.} & \textbf{Prec.} & \textbf{Rec.} & \textbf{F1} & \textbf{CK} \\
% \midrule
% XGB            & 2                & 0.772         & 0.522          & 0.642         & 0.576       & 0.422       \\
% KNN            & 2                & 0.780         & 0.522          & \textbf{0.665}& 0.585       & 0.439       \\
% RF             & 2                & \textbf{0.783}& \textbf{0.549} & 0.661         & \textbf{0.600} & \textbf{0.452} \\
% \midrule
% XGB            & 3                & 0.621         & \textbf{0.649} & 0.621         & 0.632       & 0.386       \\
% KNN            & 3                & 0.593         & 0.588          & 0.593         & 0.589       & 0.364       \\
% RF             & 3                & \textbf{0.642}& 0.647          & \textbf{0.642}& \textbf{0.644} & \textbf{0.433} \\
% \midrule
% XGB            & 5                & 0.476         & 0.486          & 0.476         & 0.478       & 0.318       \\
% KNN            & 5                & 0.464         & 0.462          & 0.464         & 0.462       & 0.314       \\
% RF             & 5                & \textbf{0.517}& \textbf{0.520} & \textbf{0.517}& \textbf{0.517} & \textbf{0.375} \\
% \bottomrule
% \end{tabularx}
% \label{tab:pred_attitude_context}
% \end{table}
% %\textcolor{yellow}{Tables 7, 8  to Appendix. Add a reference to them somewhere in the paper? Similarly for Table 6}

% % \section{Discussion}
% % We would like to discuss some important highlights and their implications on the current state of data access. Firstly, as we have observed in this study, the type of location feature is important in understanding the comfortableness in sharing location data. Getting permissions to access any type of location feature does not translate to permission for access to every type of location feature. In the modern world, where mobile location data can be combined with web browsing data, several companies have access to longitudinal data about people which can be used to infer all types of location features. Access to this longitudinal data about a person does not necessarily consent derivation of those features. The privacy boundaries of location feature access depends on the type of feature being accessed, the purpose and the actor making the access. 
% % Trajectory information like frequent and infrequent paths of travel, walking patterns  and places of frequent and infrequent visits was deemed personal by people. For features specifically, we did not observe privacy features performing better than detailed features, which could indicate that its the location feature which is important in determining whether people will be comfortable or not, rather than the privacy-preserving nature of the feature. 
% % We also used a vague purpose "monitoring mobility" to understand people's perceptions towards vague and specific purposes. We observed that average rating of monitoring mobility was negative indicating that people felt strongly against vague purposes of location data access. 
% % Actor Employee was generally disliked irrespective of the purpose and feature. It would be interesting to understand what kind of location information Employers have been collecting using the work devices, if the employees are aware of such tracking of their devices and if they can opt-out of such tracking. 

  
\section{Predictive Modeling}
This section explores the creation of models that predict levels of comfort in sharing location data a given a combination of actor, purpose and location feature. 
Such models can help companies evaluate levels of comfort with different types of location features before they are actually implemented and deployed in their location intelligence platforms.  
For example, a company could be interested in evaluating the level of comfort of a Hispanic person in sharing their movement data with law enforcement for public safety.
%\subsection{Modeling Scenarios}
%We investigate two distinct scenarios for modeling the dependent variable (level of comfort): 1) General privacy attitudes are unknown, and 2) Population attitudes are known. 
%As this tool is intended to gauge public consensus for different actors, we consider three approaches to labeling the dependent variable (comfort): 1) 5-point Likert scale (as used in the original survey), 2) 3-point Likert scale (uncomfortable/neutral/comfortable), and 3)  Binary classification (uncomfortable+neutral/comfortable). 

\subsection{Model}
We frame the prediction task as a classification task with the classes being the levels of comfort. We explore two types of models. In \textit{Model 1}, our objective is to predict the level of comfort of a given demographic group with a given vignette (framed as actor, location feature, factor). This model would allow a company to evaluate the level of comfort with the use of a location feature associated with a specific population group.  
We also evaluate a second model \textit{Model 2}, whereby privacy attitudes from the population of interest are also known, either because the company knows them, or because they can run a privacy attitudes survey among their population of interest. 
 
We consider three different approaches to defining the classes in the classification task: 1) 5-point Likert scale (as used in the original survey), 2) 3-point Likert scale ((very) uncomfortable, neutral and (very)comfortable), and 3) Binary classification ((very) uncomfortable or neutral vs. (very) comfortable). We perform an 80\% train 20\%test split for training and model evaluation, and compare  the performance of these machine learning algorithms (suitable for classification tasks): Random Forest (RF), Gradient Boosting Machines (XGB) and K-Nearest Neighbours (KNN). To assess model performance, we utilize the following metrics:
Accuracy, Precision, Recall, F1-score and Cohen's Kappa.
 
%\textcolor{yellow}{Naman, please add one sentence explaining \% of training and testing samples.}
%Input features for our models include: Demographic information (e.g., age, education level, ethnicity) and Contextual elements (actor, purpose, and feature type). Attitudes are known in case 2 but not in case 1.
%\subsection{Model Performance: Demographics-Only Scenario}
%\textcolor{yellow}{
%2 is for uncomfortable vs comfortable
%3 keeps the neutral value
%5 is for the very %uncomfortable,uncomfortable,neutral,comfortable,very comfortable 
%In 4 we would have to merge neutral with comfortable or uncomfortable. Some reviewer might be like..Why is neutral in the uncomfortable or comfortable. Thats why I skip 4. 
%Naman, why is there not a 4-point Likert?}

\subsection{Performance Analysis}
%- Model 1}

\textbf{Model 1.} Table \ref{tab:pred_context} presents the performance metrics for predicting comfort levels given a tuple of actor, purpose and feature as well as general demographic information of the population of interest to the company. 
%when only demographic information of the public is known.
We can observe moderate accuracy and F1 scores across models and number of classes with ranges (0.299-0.695) and (0.279-0.325), respectively. As expected, higher number of classes are associated with lower accuracy values. Nevertheless, the F1 scores are the lowest for the binary classification pointing to challenges in balancing precision and recall, possibly due to class imbalance.
%\subsubsection{Interpretation of Results}
%The performance of these models provides several insights:
%\begin{itemize}
%    \item Binary Classification: The Random Forest model's accuracy of 65.8\% suggests a moderate ability to distinguish between comfortable and not uncomfortable responses. However, the relatively low F1-score (0.27) indicates challenges in balancing precision and recall, possibly due to class imbalance.
 %   \item Multi-class Classification: XGBoost's superior performance in both 3-point and 5-point Likert scale predictions demonstrates its effectiveness in capturing the nuanced differences in comfort levels. The specific Kappa scores would provide more insight into the strength of these predictions.
The moderate performance across all models suggests that vignette and demographic information alone may not be sufficient for highly accurate predictions of comfort levels in location data sharing. This underscores the complexity of privacy perceptions and the potential influence of other factors not captured in demographic data such as privacy attitudes.
%    \end{itemize}

\begin{table}[]
\centering
\footnotesize
\caption{Detailed classification report for the three labeling approaches. Dependent variables: Actor, purpose, feature, demographic attributes Best metric per classification approach are in bold. XGB: XGBoost, KNN: K-nearest Neighbours, RF: RandomForest, CK: Cohen Kappa score}
\begin{tabularx}{0.5\textwidth}{@{}lc*{5}{>{\centering\arraybackslash}X}@{}}
\toprule
\textbf{Model} & \textbf{Class} & \textbf{Acc.} & \textbf{Prec.} & \textbf{Rec.} & \textbf{F1} & \textbf{CK} \\
\midrule
XGB            & 2              & \textbf{0.695} & 0.155         & \textbf{0.455} & 0.232      & \textbf{0.095} \\
KNN            & 2              & 0.647          & 0.219         & 0.349          & 0.269      & 0.052 \\
RF             & 2              & 0.658          & \textbf{0.223} & 0.372         & \textbf{0.279} & 0.073 \\
\midrule
XGB            & 3              & \textbf{0.481} & 0.587         & 0.481          & 0.521      & \textbf{0.118} \\
KNN            & 3              & 0.478          & \textbf{0.605} & 0.478         & 0.527      & 0.106 \\
RF             & 3              & 0.459          & 0.530         & \textbf{0.459} & \textbf{0.486} & 0.096 \\
\midrule
XGB            & 5              & \textbf{0.299} & \textbf{0.385} & \textbf{0.299} & \textbf{0.325} & \textbf{0.060} \\
KNN            & 5              & 0.287          & 0.380         & 0.287          & 0.318      & 0.042 \\
RF             & 5              & 0.293          & 0.351         & 0.293          & 0.313      & \textbf{0.060} \\
\bottomrule
\end{tabularx}
\label{tab:pred_context}
\end{table}
%\subsection{Model Performance: Demographics and Context Known}
%\subsection{Preformance - Model 2}
\textbf{Model 2.} Here, we examine the predictive performance of Model 1 enhanced with privacy attitude data from the population of interest. 
%when both demographic information and contextual data are available for modeling. 
As Table \ref{tab:pred_attitude_context} shows, adding privacy attitudes as independent variables in the prediction model consistently improves prediction performance across all metrics: accuracy, precision, recall, and Cohen-Kappa; with improvements over 50\% for 3-point and 5-point Likert scales when compared to Model 1. Accuracy and F1 score values for Model 2 were in the ranges (0.517,0.783) and (0.517,0.600), respectively. The best performing model was the Random Forest which showed superior performance across all classification settings (binary and multi-class).
% TABLE Attitude + Context

\begin{table}[]
\centering
\footnotesize
\caption{Detailed classification report for the three labeling approaches. Dependent variable: Actor, purpose, feature, demographic attributes, privacy attitudes and trust. Best metric per classification approach are in bold. XGB: XGBoost, KNN: K-nearest Neighbours, RF: RandomForest, CK: Cohen Kappa score}
\begin{tabularx}{0.5\textwidth}{@{}lc*{6}{>{\centering\arraybackslash}X}@{}}
\toprule
\textbf{Model} & \textbf{Classes} & \textbf{Acc.} & \textbf{Prec.} & \textbf{Rec.} & \textbf{F1} & \textbf{CK} \\
\midrule
XGB            & 2                & 0.772         & 0.522          & 0.642         & 0.576       & 0.422       \\
KNN            & 2                & 0.780         & 0.522          & \textbf{0.665}& 0.585       & 0.439       \\
RF             & 2                & \textbf{0.783}& \textbf{0.549} & 0.661         & \textbf{0.600} & \textbf{0.452} \\
\midrule
XGB            & 3                & 0.621         & \textbf{0.649} & 0.621         & 0.632       & 0.386       \\
KNN            & 3                & 0.593         & 0.588          & 0.593         & 0.589       & 0.364       \\
RF             & 3                & \textbf{0.642}& 0.647          & \textbf{0.642}& \textbf{0.644} & \textbf{0.433} \\
\midrule
XGB            & 5                & 0.476         & 0.486          & 0.476         & 0.478       & 0.318       \\
KNN            & 5                & 0.464         & 0.462          & 0.464         & 0.462       & 0.314       \\
RF             & 5                & \textbf{0.517}& \textbf{0.520} & \textbf{0.517}& \textbf{0.517} & \textbf{0.375} \\
\bottomrule
\end{tabularx}
\label{tab:pred_attitude_context}
\end{table}
%\textcolor{yellow}{Tables 7, 8  to Appendix. Add a reference to them somewhere in the paper? Similarly for Table 6}

\section{Implications for Policy and Practice}
%\subsection{Key Findings and Implications}
\textbf{Location Features.} Our regression analyses demonstrate that participants have higher levels of comfort in sharing detailed visits to POIs than detailed trajectory data. However, obfuscating some of the trajectory variables resulted in higher levels of comfort even above sharing detailed visits to POIs.
Our analyses also showed that these levels of comfort are also increased when the actors accessing the data are researchers or family, or when the purpose is focused on social good. 
These findings point to the implementation of more granular control mechanisms that allow users to specify which types of location features can be derived from the data collected from them. 
Since data brokers and data aggregators collect data from apps via SDKs, they could also allow users to select, during the app installation and when they are notified about the location data collection, which one of the six feature clusters shown in Table \ref{tab:listOfFeatures} they would be comfortable sharing their location data for. Users should also have an opt-out mechanism for which type of actor and purpose their data can be used for. While arguing against the purchase of location data from data brokers by government agencies, legal scholars and practitioners like Rahbar \cite{carpenterEvade} and Conner \footnote{\url{https://www.lawfaremedia.org/article/data-broker-sales-and-the-fourth-amendment}}, have raised serious concerns about such purchases violating the fourth amendment.
%shows six feature clusters that users could choose from without imposing too much of a burden on their selection tasks when installing the app from which 
%, overall, the type of location feature significantly influences comfort levels in data sharing. For example, (1) Use of location features such as obfuscated home or work location have a significant positive effect on the level of comfort with sharing that data,
%(2) participants showed lower levels of comfort in sharing movement features such as detailed trip data (now available due to data brokers). 
%(3) \textbf{Ethnicity} and \textbf{education levels} along with the context (actor, purpose, \textbf{feature}) need to be taken into account to understand comfort levels.

%This suggests that blanket permissions for location data access may not align with users' actual privacy preferences. Companies and policymakers working with location data should consider implementing more granular control mechanisms that allow users to specify which types of location features can be derived from the data collected from them. The ability to derive various location features from longitudinal data collected by companies raises important ethical and legal questions in the light of several privacy boundaries discovered in our research. Access to location data should not be interpreted as consent for deriving all possible features. Privacy policies and regulations need to address not just data collection but also the types of analyses and feature extractions permitted.

\textbf{Obfuscation Approaches.} Higher levels of comfort were clearly observed for some obfuscated location features (both trajectory and visits to POIs) when compared to detailed visits to POIs (baseline) in the regression model \ref{tab:apf_reg}. These levels of comfort were increased when the actor accessing the data was a family member. 
Although not all obfuscated approaches were significantly associated with higher levels of comfort in the regression, the mean values were lower. 
We posit that maybe there is no need for more granular control over obfuscated vs. detailed features. By default, the features could be obfuscated, and during the app installation users could be offered to opt-in to share more detailed features, together with an explanation of why this would be valuable for analytical purposes (actor, purpose).  
 In future research, we will also explore why obfuscation techniques appear to not always alleviate privacy concerns.

%We have also seen in several sections that obfuscation approaches make data sharing comfortable in certain contexts (remember that participants were still uncomfortable in sharing location features with actors like employers or federal government agencies or purpose like Places you visit even with obfuscation). Contrary to expectations, obfuscated features did not consistently outperform detailed features in terms of user comfort when looking at different ethnicites, education groups, actors, purposes. This suggests that the nature of the location information itself, rather than its level of detail, may be the primary factor in determining comfort levels.

%\textbf{Actors.} The consistently negative perception of employer access to location data, regardless of purpose or location feature type, raises concerns about workplace privacy. This finding calls for a re-evaluation of employee monitoring practices and suggests a need for clearer guidelines and regulations regarding workplace surveillance. 

%\textbf{Purposes} The negative perception towards the vague purpose of "monitoring mobility" highlights the importance of transparency in data usage. We also found links to different ethnicity group, education groups and perceptions to purposes. Hispanic participants negatively rated showing ads, while Black participants were more comfortable with it.
%As of now, we do not recommend stopping the purposes with negative perceptions as these are legitimate use cases.  

\textbf{Race, Ethnicity and Education Effects.} Our regression analysis has revealed significant differences in comfort levels for the Hispanic participants when compared to White participants, with Hispanic participants being more comfortable sharing trajectory and visit data. Our interaction analysis has also shown a complex interplay between 
%across racial groups, particularly the higher comfort levels reported by Hispanic participants, differences in perceptions in the interaction analysis underscore the complex interplay between 
cultural factors, educational experiences, and privacy perceptions while controlling for privacy attitudes and computer knowledge. Hispanic and Black participants as well as "High School to Bachelors" appear to shape their perceptions mostly on the purpose of the data access, or on who has access to the data (actor). On the other hand, White participants and participants with educational level "Bachelors and above" appear to shape their privacy perceptions around actors, purposes and by the types of location data being shared. 
These findings could be used to pre-populate some of the initial app selections for types of location data, actors or purposes based, for example, on the educational background of the person installing the app, thus reducing the burden of having to select among many different options from scratch. 

%Putting together our findings for location features, obfuscation and racial and education groups, in future work, we will design and evaluate an app interface that, based on our findings, proposes some initial opt-in approaches that users would be able to personalize if needed, thus reducing the burden of having to select among many different options from scratch. 





%Future research will explore rther research needs to be done to understand perceptions of educational group Under highschool. They did not rely on contextual interactions (no difference in comfortableness medians between different actors, purposes or features) and rated based only on their privacy practices and attitudes. The behavior of "Bachelors and above" and "Highschool to Bachelors" group exhibits privacy paradox: we can see the perceived benefits (purposes such as Design new walking/cycling infrastructure and Identify locations for infrastructure
%) are perceived as more comfortable. Overall, we showed that obfuscated location features are not always more comfortable to share when our observation is marginalized by education and ethnicity. And, both different levels of education and all 4 ethnicity/races evaluate context differently 


\section{Conclusion}
This study contributes to our understanding of the nuanced nature of privacy perceptions in location data sharing. By highlighting the importance of the type of location feature (trajectory or visits to POIs),
the presence of an obfuscation approach, contextual factors such as actors and purpose, attitudes and demographic variations, our findings provide a foundation for more user-centric, feature focused approach to location privacy. 
Our results have shown that trajectory-related features are associated with higher privacy concerns; that current obfuscation approaches by location data brokers and aggregators are sometimes successful; and that race, ethnicity and education have an effect on privacy perceptions with Hispanic and High School to Bachelor populations being associated with higher levels of comfort. 
As technology continues to evolve, ongoing research in this area will be crucial for developing privacy practices and policies that effectively balance the benefits of location-based services with individual privacy rights. 

\section{Ethical considerations } 
 This study was evaluated and approved as a human subject research by the Internal Review Board of the authors' university. 
Prior to starting the survey, participants were provided a link to the Internal Review Board document detailing:  the purpose of the study, the different procedures, any potential risks or discomforts that could occur during the study, potential benefits and compensation value and the mechanism of compensation delivery. We offered an incentive of \$7 per participant to complete our survey. The compensation was disbursed by Cint. 
Participants were also provided with an email and phone number to raise any concerns, or complaints related to the study. We did not receive any concerns or complains from the participants and there were no research-related injuries reported. 

The survey data collected was analyzed in an anonymous way. The names and emails of the participants were separated from their responses and each participant was assigned a Universally Unique Identifier (UUID) (which was generated randomly during the survey).
Names and emails of participants were only used to justify incentive expenses (compensation for participation) to our IRB board. 
When quoting participant responses from the free-form text replies in the survey, we identify each participant using the last 4 characters of their UUID. None of the Personally Identifiable Information (PII) will be made available publicly during or after the research is concluded. 


\section{Compliance with Open science policy}
Literature has shown that data visualizations are better at conveying detailed information to participants when conducting perception surveys. To integrate interactive map visualizations in our survey, we developed a Flask app that we hosted on a private server during the experiment. Participants were recruited via Cint and then were taken to our server where the survey data was collected and consolidated. The Flask app is connected with a local database which has the questions and corresponding visualizations that are randomized and shown to participants. After the data collection, the analysis was performed using python. 

If required, we will be able to share: 1) \textbf{Hosting:} Flask app code, maria-db database with questions and visualizations as HTML files; 2) \textbf{Analysis} Python code/R code (regression analysis and Kruskal-Wallis test followed by Dunn post-hoc test). 
To enhance the reproducibility and replicability of our scientific findings, we can also share the anonymized responses to our survey. 
%the following data collected during the survey: demographic information, attitudes/awareness, actor, feature, purpose and corresponding comfort. The 

\bibliographystyle{unsrt}
\bibliography{arxiv}


\appendix

\begin{table}[ht]
\centering
\caption{Age distribution in the U.S. and in our survey population.}
\begin{tabular}{lll}
\toprule
Age   & Census \% & Survey \% \\
\midrule
18-24 & 12\%           & 8\%       \\
25-34 & 18\%           & 22\%      \\
35-44 & 17\%           & 19\%      \\
45-54 & 16\%           & 17\%      \\
55-64 & 17\%           & 19\%      \\
65+   & 21\%           & 14\%     \\
\bottomrule
\end{tabular}

\label{tab:agedistribution}
\end{table}

\begin{table}[ht]
\centering
\caption{Gender distribution for the U.S. Adult population (age 18 and above) and in our survey population.}
\begin{tabular}{lll}
\toprule
Gender & U.S. Census & Survey \\
\midrule
Male   & 49\%      & 46\% \\
Female & 51\%      & 53\% \\ 
\bottomrule
\end{tabular}
\label{tab:genderdistribution}
\end{table}

\begin{table*}[]
\centering
\caption{Percentage of responses for each privacy attitude question across levels of comfort. }
%of the Likert options for different Privacy Attitude Questions }
\begin{tabular}{|p{0.18\textwidth}p{0.10\textwidth}|p{0.10\textwidth}p{0.10\textwidth}p{0.06\textwidth}p{0.09\textwidth}p{0.09\textwidth}|}

\multicolumn{2}{|c|}\textbf{Privacy Attitudes}  & \multicolumn{5}{c|}{\textbf{Comfort Level}} \\
    \cline{3-7}
    & & \textbf{Very Uncomfortable} & \textbf{Uncomfortable} & \textbf{Neutral} & \textbf{Comfortable} & \textbf{Very Comfortable} \\
\cline{1-7}
\multirow{5}{*}{\textbf{I trust businesses}} &  \textbf{S. disagree}        & \cellcolor[HTML]{72C6E9}37.872 & \cellcolor[HTML]{BDE4F5}17.872 & \cellcolor[HTML]{D3EDF8}11.915 & \cellcolor[HTML]{C9EAF7}14.468 & \cellcolor[HTML]{BDE4F5}17.872 \\
& \textbf{Disagree}                 & \cellcolor[HTML]{A7DCF1}23.62  & \cellcolor[HTML]{97D5EF}27.964 & \cellcolor[HTML]{B6E2F4}19.548 & \cellcolor[HTML]{B5E1F4}19.91  & \cellcolor[HTML]{DEF2FA}8.959  \\
& \textbf{Undecided}                & \cellcolor[HTML]{CDEBF7}13.6   & \cellcolor[HTML]{AEDEF2}21.76  & \cellcolor[HTML]{88CFEC}31.893 & \cellcolor[HTML]{A6DBF1}24.053 & \cellcolor[HTML]{DFF2FA}8.693  \\
& \textbf{Agree}                    & \cellcolor[HTML]{DAF0F9}10.047 & \cellcolor[HTML]{CAEAF7}14.317 & \cellcolor[HTML]{B0DFF3}21.193 & \cellcolor[HTML]{6FC5E8}38.556 & \cellcolor[HTML]{C4E7F6}15.887 \\
& \textbf{S. Agree}         & \cellcolor[HTML]{DFF2FA}8.73   & \cellcolor[HTML]{E9F6FC}6.032  & \cellcolor[HTML]{E0F3FA}8.413  & \cellcolor[HTML]{96D5EF}28.095 & \cellcolor[HTML]{49B5E2}48.73  \\
\cline{1-7}
\multirow{5}{*}{\textbf{Privacy   importance}} & \textbf{S. disagree}        & \cellcolor[HTML]{AADDF2}22.857 & \cellcolor[HTML]{C5E8F6}15.714 & \cellcolor[HTML]{E5F5FB}7.143  & \cellcolor[HTML]{C5E8F6}15.714 & \cellcolor[HTML]{6FC5E8}38.571 \\
& \textbf{Disagree}                 & \cellcolor[HTML]{EDF8FC}5      & \cellcolor[HTML]{CFECF8}13     & \cellcolor[HTML]{A9DDF2}23     & \cellcolor[HTML]{84CDEC}33     & \cellcolor[HTML]{9ED8F0}26     \\
& \textbf{Undecided}                & \cellcolor[HTML]{F1FAFD}3.939  & \cellcolor[HTML]{E8F6FC}6.364  & \cellcolor[HTML]{86CEEC}32.424 & \cellcolor[HTML]{75C7E9}36.97  & \cellcolor[HTML]{B4E1F3}20.303 \\
& \textbf{Agree}                    & \cellcolor[HTML]{E5F5FB}7.173  & \cellcolor[HTML]{C3E7F6}16.14  & \cellcolor[HTML]{A4DAF1}24.522 & \cellcolor[HTML]{6DC4E8}39.103 & \cellcolor[HTML]{CFECF8}13.06  \\
& \textbf{S. Agree}         & \cellcolor[HTML]{B8E2F4}19.218 & \cellcolor[HTML]{B5E1F3}20.025 & \cellcolor[HTML]{B3E1F3}20.328 & \cellcolor[HTML]{A7DBF1}23.733 & \cellcolor[HTML]{C1E6F5}16.696 \\
\cline{1-7}
\multirow{5}{*}{\textbf{Trust   in government}}& \textbf{S. disagree}        & \cellcolor[HTML]{84CDEC}32.898 & \cellcolor[HTML]{A4DAF1}24.408 & \cellcolor[HTML]{BEE5F5}17.551 & \cellcolor[HTML]{C0E6F5}17.061 & \cellcolor[HTML]{E1F3FB}8.082  \\
& \textbf{Disagree}                 & \cellcolor[HTML]{C9E9F7}14.579 & \cellcolor[HTML]{A3DAF1}24.735 & \cellcolor[HTML]{A5DBF1}24.112 & \cellcolor[HTML]{9ED8F0}26.168 & \cellcolor[HTML]{D9F0F9}10.405 \\
& \textbf{Undecided}                & \cellcolor[HTML]{D5EEF9}11.299 & \cellcolor[HTML]{C5E8F6}15.71  & \cellcolor[HTML]{93D3EE}29.124 & \cellcolor[HTML]{8CD0ED}30.937 & \cellcolor[HTML]{CFECF8}12.931 \\
& \textbf{Agree}                    & \cellcolor[HTML]{E8F6FC}6.205  & \cellcolor[HTML]{CEEBF8}13.179 & \cellcolor[HTML]{B1E0F3}20.872 & \cellcolor[HTML]{6BC3E8}39.744 & \cellcolor[HTML]{B5E1F3}20     \\
& \textbf{S. Agree}         & \cellcolor[HTML]{EAF7FC}5.882  & \cellcolor[HTML]{E6F5FB}6.723  & \cellcolor[HTML]{CEEBF8}13.277 & \cellcolor[HTML]{86CEEC}32.605 & \cellcolor[HTML]{64C0E7}41.513 \\
\cline{1-7}
\multirow{5}{*}{\makecell{\textbf{Great that young}\\\textbf{people are prepared to}\\\textbf{defy authority}}}& \textbf{S. disagree}
& \cellcolor[HTML]{C0E6F5}16.978 & \cellcolor[HTML]{CBEAF7}14.044 & \cellcolor[HTML]{B6E2F4}19.733 & \cellcolor[HTML]{A5DBF1}24.178 & \cellcolor[HTML]{A2D9F0}25.067 \\
& \textbf{Disagree}                 & \cellcolor[HTML]{CFECF8}13.048 & \cellcolor[HTML]{B9E3F4}18.974 & \cellcolor[HTML]{AADDF2}22.735 & \cellcolor[HTML]{7ECBEB}34.644 & \cellcolor[HTML]{D8EFF9}10.598 \\
& \textbf{Undecided}                & \cellcolor[HTML]{CCEBF7}13.69  & \cellcolor[HTML]{B3E0F3}20.428 & \cellcolor[HTML]{A3DAF1}24.599 & \cellcolor[HTML]{92D3EE}29.198 & \cellcolor[HTML]{D2EDF8}12.086 \\
& \textbf{Agree}                    & \cellcolor[HTML]{DAF0FA}9.969  & \cellcolor[HTML]{BAE3F4}18.7   & \cellcolor[HTML]{A9DCF2}23.034 & \cellcolor[HTML]{81CCEB}33.746 & \cellcolor[HTML]{C9E9F7}14.551 \\
& \textbf{S. Agree}         & \cellcolor[HTML]{AFDFF3}21.504 & \cellcolor[HTML]{D4EEF8}11.729 & \cellcolor[HTML]{BEE5F5}17.594 & \cellcolor[HTML]{B1E0F3}20.902 & \cellcolor[HTML]{96D5EF}28.271 \\
\cline{1-7}
\multirow{5}{*}{\makecell{\textbf{Discipline and}\\\textbf{follow leader}}} & \textbf{S. disagree}        & \cellcolor[HTML]{88CFEC}32.048 & \cellcolor[HTML]{BAE3F4}18.554 & \cellcolor[HTML]{B8E3F4}19.036 & \cellcolor[HTML]{B8E3F4}19.036 & \cellcolor[HTML]{D5EEF9}11.325 \\
& \textbf{Disagree}                 & \cellcolor[HTML]{C5E8F6}15.767 & \cellcolor[HTML]{A8DCF2}23.313 & \cellcolor[HTML]{B4E1F3}20.307 & \cellcolor[HTML]{93D3EE}29.08  & \cellcolor[HTML]{D4EEF9}11.534 \\
& \textbf{Undecided}                & \cellcolor[HTML]{CFECF8}13.023 & \cellcolor[HTML]{BBE3F4}18.43  & \cellcolor[HTML]{93D3EE}29.07  & \cellcolor[HTML]{96D4EF}28.314 & \cellcolor[HTML]{D6EFF9}11.163 \\
& \textbf{Agree}                    & \cellcolor[HTML]{E2F4FB}7.795  & \cellcolor[HTML]{C3E7F6}16.256 & \cellcolor[HTML]{A9DCF2}23.128 & \cellcolor[HTML]{75C7E9}37.026 & \cellcolor[HTML]{C4E7F6}15.795 \\
& \textbf{S. Agree}         & \cellcolor[HTML]{DEF2FA}9      & \cellcolor[HTML]{DCF1FA}9.444  & \cellcolor[HTML]{C9EAF7}14.444 & \cellcolor[HTML]{90D2EE}29.889 & \cellcolor[HTML]{74C7E9}37.222 \\
\cline{1-7}
\multirow{5}{*}{\textbf{Give benefit of doubt}}& \textbf{S. disagree}        & \cellcolor[HTML]{6AC3E7}40     & \cellcolor[HTML]{BEE5F5}17.6   & \cellcolor[HTML]{C4E7F6}16     & \cellcolor[HTML]{D6EEF9}11.2   & \cellcolor[HTML]{C7E8F6}15.2   \\
& \textbf{Disagree}                 & \cellcolor[HTML]{B6E1F4}19.744 & \cellcolor[HTML]{AADDF2}22.821 & \cellcolor[HTML]{B9E3F4}18.974 & \cellcolor[HTML]{B1E0F3}21.026 & \cellcolor[HTML]{BEE5F5}17.436 \\
& \textbf{Undecided}                & \cellcolor[HTML]{BFE5F5}17.253 & \cellcolor[HTML]{BEE5F5}17.582 & \cellcolor[HTML]{95D4EE}28.352 & \cellcolor[HTML]{95D4EE}28.462 & \cellcolor[HTML]{E0F3FA}8.352  \\
& \textbf{Agree}                    & \cellcolor[HTML]{D0ECF8}12.637 & \cellcolor[HTML]{B7E2F4}19.264 & \cellcolor[HTML]{AADDF2}22.969 & \cellcolor[HTML]{85CEEC}32.755 & \cellcolor[HTML]{D1EDF8}12.375 \\
& \textbf{S. Agree}         & \cellcolor[HTML]{D4EEF8}11.756 & \cellcolor[HTML]{D2EDF8}12.258 & \cellcolor[HTML]{BCE4F5}17.993 & \cellcolor[HTML]{9BD7EF}26.953 & \cellcolor[HTML]{8BD0ED}31.039 \\
\cline{1-7}
\end{tabular}
\label{tab:attitudes_responses}
\end{table*}

% TABLE: actor
\begin{table}[h]
    \centering
    \caption{Actor and Comfortableness (Mean)}
    \begin{tabular}{lr}
        \toprule
        Actor & Answer \\
        \midrule
        Actor: Commercial Entity & 0.162 \\
        Actor: Federal government agency & -0.073 \\
        Actor: Law enforcement agency & 0.081 \\
        Actor: Local government agency & 0.189 \\
        Actor: Academic researchers & 0.387 \\
        Actor: Emergency services & 0.525 \\
        Actor: Doctor & 0.256 \\
        Actor: Employer & -0.424 \\
        Actor: Family & 0.383 \\
        \bottomrule
    \end{tabular}
    \label{tab:actorcomfortmean}
\end{table}
%TABLE:  purpose
\begin{table}[h]
\centering
\caption{Purpose and Comfortableness (Means)}
\begin{tabular}{p{0.70\linewidth}p{0.2\linewidth}}
\toprule
Purpose &  Answer \\
\midrule
Purpose: Analysis of terrorist attacks &        0.126 \\
Purpose: Analysis of Public transit services &        0.276 \\
Purpose: Control spread of diseases &        0.233 \\
Purpose: Monitor mobility patterns &       -0.022 \\
Purpose: Personal wellness &        0.157 \\
Purpose: Show ads &        0.185 \\
Purpose: Analysis of criminal activity &        0.147 \\
Purpose: Design new walking/cycling infrastructure &        0.425 \\
Purpose: Identify locations for infrastructure &        0.446 \\
Purpose: Optimize work productivity &       -0.054 \\
\bottomrule
\end{tabular}
    \label{tab:purposecomfortmean}
\end{table}

% TABLE: feat
\begin{table}[h]
\centering
\caption{Feature and Comfortableness (Means)}
\begin{tabular}{p{0.70\linewidth}p{0.2\linewidth}}
\toprule
Feature &  Answer \\
\midrule
Feature: International visits (Detailed) &        0.286 \\
Feature: International visits (Obfuscated) &        0.247 \\
\makecell[l]{Feature: Area you spent most \\  of your time (Obfuscated)} &        0.200 \\
Feature: Modes of transportation (Obfuscated) &        0.220 \\
Feature: Modes of transportation (Detailed) &        0.175 \\
Feature: Places you visit (Detailed) &        0.092 \\
Feature: Places you visit (Obfuscated) &        0.224 \\
Feature: Home location (Obfuscated) &        0.210 \\
Feature: Home location (Detailed) &        0.155 \\
Feature: Work location (Obfuscated) &        0.263 \\
Feature: Work location (Detailed) &        0.268 \\
Feature: Least frequent trips (Detailed) &        0.026 \\
\makecell[l]{Feature: Most frequent trips between \\ counties (Obfuscated)} &        0.129 \\
\makecell[l]{Feature: Most frequent trips between \\ counties('Google') (Obfuscated)} &        0.152 \\
Feature: Most frequent trips (Detailed) &       -0.058 \\
\makecell[l]{Feature: Most frequent type \\ of trips (Obfuscated)} &        0.145 \\
Feature: Frequent walking activity (Detailed) &        0.054 \\
Feature: Frequent walking activity (Obfuscated) &        0.356 \\
\bottomrule
\end{tabular}
\label{tab:featcomfortmean}
\end{table}

%%Actor responses
\begin{figure}[ht]
    \centering
\includegraphics[width=0.5\textwidth]{Images/actors.png}
    \caption{Percentage of responses per actor and level of comfort (ordered left to right: Very uncomfortable, Uncomfortable, Neutral, Comfortable, Very comfortable). }
    \label{fig:actor_responses}
\end{figure}

%% Purpose responses
\begin{figure}[ht]
    \centering
\includegraphics[width=0.5\textwidth]{Images/purposes.png}
    \caption{Percentage of responses per purpose and level (ordered left to right: Very uncomfortable, Uncomfortable, Neutral, Comfortable, Very comfortable). }
    \label{fig:purpose_responses}
\end{figure}

\section{Ordinal Regression}
\label{section:ordinal_regression}
The survey measures comfort levels using a five-point ordinal scale (very uncomfortable $\leq$ uncomfortable $\leq$ neutral $\leq$ comfortable $\leq$ very comfortable). Mixed effects Ordinal regression was chosen over ordinal regression and multi-class classification to account for the ordered nature of the dependent variable.
%and 5 data questions were answered per survey participant. 
Ordinal regressions are preferred to linear regressions because ordinal categories are not necessarily evenly spaced (e.g., the difference between uncomfortable and neutral may not equal the difference between neutral and comfortable).

We used the following formula for the Mixed effects Ordinal regression: 
% the following formula to determine the probability of a data point x being in a particular class $i$:

% $$ Pr(y \leq i | X ) = \sigma(\theta_i - w \cdot X)$$

% Where the cumulative probability for data point $x$ being in class $y \leq i$ is given by the probit of the difference between thresholds $\theta_i$ and the line $w \cdot x$. In our model, $y$ represents the ordinal class, and $x$ is the input feature vector (including actor, purpose, location feature, demographic attributes, privacy attitudes, and technical knowledge).
 \begin{align}
 \label{eq:regression}
 \text{Ans} &\sim 1 + \text{Actor} + \text{Purpose} + \text{Feature} + \notag \\
&\text{Ethnicity} + \text{Education} + \notag \\
&\text{Privacy and Attitude questions} + \notag   +\\
& (1|\text{ID}) + (1|\text{apf})
\end{align}

\textbf{Assumption Testing.} We test the assumptions for the mixed effects ordinal regression model: proportionality odds and multicollinearity using the nominal\_test and scale\_test in R. None of the variables have p-value < 0.05 (thus no significant evidence of non-proportional odds). We also compute Variance Inflation Factor (VIF) and all the variables have VIF < 2 which indicates no multicollinearity in the regression setting.

Next, we also check goodness of fit and compare different ordinal regression settings in Table \ref{tab:model_aic_compare}. We find that the model with best AIC is the one with two random intercepts ID and vignette (actor-purpose-feature combination). ID is the UUID of an individual (as we ask multiple questions to the same person). Actor-Purpose-Feature (1|apf) models the between vignette heterogeneity as described in \cite{vignette_instructions}. We analyze this regression model in the paper. 


\textbf{Weights.} We weigh the samples in the mixed effects ordinal regression model for representativity of education and ethnicity. We calculate these weights by dividing \% of expected population by our population e.g.,: 6.2\% of adult white population is educated and has a degree under high school. Since we have 3.6\% participants in that strata, we assign a weight of 1.72 to each respondent. 
%. Thus, the weight for each respondent is 1.72. 

\begin{table}[ht]
\centering
\caption{ANOVA analysis of different models to select the best model. CLM: Cumulative link model (Ordinal model), CLMM (cumulative linked mixed model). Best models are $CLMM_{ID}$ and $CLMM_{ID\_VIG}$ with lowest AIC scores. Significance "**" < 0.05}
\begin{tabularx}{0.5\textwidth}{lXlll} 
Model & Description  & AIC \\
\hline
CLM & a+p+f+privacy and attitudes & 20218  \\
\hline
$CLMM_{intercept}$ & random intercepts only & 19332 \\
\hline
$CLMM_{noapf}$ & Random Intercept + privacy \& attitudes & 18803** \\
\hline
$CLMM_{ID}$ & a+p+f+privacy and attitudes + (1|ID)  & \textbf{18589**} \\
\hline
$CLMM_{ID\_VIG}$ & a+p+f+privacy and attitudes + (1|ID)+ (1|apf)  & \textbf{18583**} 
\end{tabularx}
\label{tab:model_aic_compare}
\end{table}



\begin{table*}[ht]
\centering
\footnotesize
\caption{Additional control variables used in the regression setting which were significant (** pvalue<0.05). L :Linear, C : Cubic, Q: Quadratic. }
\begin{tabular}{llllll}
& Coefficients & Odds Ratio & Std Err & Statistic & p val \\
Privacy.importance.L                                                            & \textbf{-1.301** }    & 0.272      & 0.385   & -3.382    & 0.001 \\
Privacy.importance.Q                                                            & -0.246       & 0.782      & 0.353   & -0.697    & 0.486 \\
Privacy.importance.C                                                            & 0.141        & 1.151      & 0.336   & 0.42      & 0.675 \\
Privacy.importance\textasciicircum{}4                                           & -0.048       & 0.953      & 0.285   & -0.168    & 0.867 \\
Trust.in.government.L                                                           & \textbf{1.398** }     & 4.047      & 0.189   & 7.386     & 0     \\
Trust.in.government.Q                                                           & -0.172       & 0.842      & 0.158   & -1.087    & 0.277 \\
Trust.in.government.C                                                           & 0.134        & 1.143      & 0.122   & 1.105     & 0.269 \\
Trust.in.government\textasciicircum{}4                                          & -0.064       & 0.938      & 0.109   & -0.588    & 0.557 \\
Great.that.young.people.are.prepared.to.defy.authority.L                        & \textbf{-0.425**}     & 0.654      & 0.161   & -2.641    & 0.008 \\
Great.that.young.people.are.prepared.to.defy.authority.Q                        & 0.143        & 1.154      & 0.149   & 0.955     & 0.339 \\
Great.that.young.people.are.prepared.to.defy.authority.C                        & \textbf{-0.38**}      & 0.684      & 0.119   & -3.191    & 0.001 \\
Great.that.young.people.are.prepared.to.defy.authority\textasciicircum{}4       & -0.033       & 0.968      & 0.108   & -0.304    & 0.761 \\
Discipline.and.follow.leader.L                                                  & \textbf{0.753**}      & 2.123      & 0.168   & 4.493     & 0     \\
Discipline.and.follow.leader.Q                                                  & 0.097        & 1.102      & 0.15    & 0.648     & 0.517 \\
Discipline.and.follow.leader.C                                                  & \textbf{0.26**}       & 1.297      & 0.119   & 2.188     & 0.029 \\
Discipline.and.follow.leader\textasciicircum{}4                                 & -0.115       & 0.891      & 0.108   & -1.058    & 0.29  \\
Give.benefit.of.doubt.L                                                         & 0.475        & 1.608      & 0.288   & 1.651     & 0.099 \\
Give.benefit.of.doubt.Q                                                         & 0.075        & 1.078      & 0.253   & 0.297     & 0.767 \\
Give.benefit.of.doubt.C                                                         & \textbf{0.433**}      & 1.542      & 0.201   & 2.158     & 0.031 \\
Give.benefit.of.doubt\textasciicircum{}4                                        & \textbf{-0.516**}     & 0.597      & 0.165   & -3.131    & 0.002 \\
Avoid.mobile.apps.due.to.privacy.concern.L                                      & \textbf{-0.783**}     & 0.457      & 0.136   & -5.777    & 0     \\
Avoid.mobile.apps.due.to.privacy.concern.Q                                      & -0.085       & 0.919      & 0.097   & -0.879    & 0.379 \\
Uninstalled.app.for.collecting.data.L                                           & -0.159       & 0.853      & 0.133   & -1.192    & 0.233 \\
Uninstalled.app.for.collecting.data.Q                                           & 0.01         & 1.01       & 0.096   & 0.105     & 0.917 \\
Computer.Science.or.Programming.Experience.L                                    & \textbf{0.558**}      & 1.747      & 0.24    & 2.321     & 0.02  \\
Computer.Science.or.Programming.Experience.Q                                    & 0.058        & 1.06       & 0.163   & 0.355     & 0.723 \\
Computer.Science.or.Programming.Experience.C                                    & 0.042        & 1.043      & 0.114   & 0.367     & 0.714 \\
Machine.Learning.Experience.L                                                   & 0.019        & 1.019      & 0.289   & 0.067     & 0.947 \\
Machine.Learning.Experience.Q                                                   & 0.057        & 1.059      & 0.194   & 0.296     & 0.767 \\
Machine.Learning.Experience.C                                                   & -0.015       & 0.985      & 0.134   & -0.114    & 0.909 \\
Statistics.Experience.L                                                         & 0.05         & 1.051      & 0.276   & 0.18      & 0.857 \\
Statistics.Experience.Q                                                         & -0.134       & 0.875      & 0.191   & -0.7      & 0.484 \\
Statistics.Experience.C                                                         & -0.13        & 0.878      & 0.127   & -1.022    & 0.307 \\
Frequency.of.use.of.data.collection.tools.like.GMaps..Fitbit.L                  & 0.321        & 1.379      & 0.167   & 1.926     & 0.054 \\
Frequency.of.use.of.data.collection.tools.like.GMaps..Fitbit.Q                  & 0.212        & 1.236      & 0.162   & 1.305     & 0.192 \\
Frequency.of.use.of.data.collection.tools.like.GMaps..Fitbit.C                  & -0.079       & 0.924      & 0.125   & -0.627    & 0.531 \\
Frequency.of.use.of.data.collection.tools.like.GMaps..Fitbit\textasciicircum{}4 & -0.195       & 0.823      & 0.14    & -1.393    & 0.163 \\
Privacy.settings.awareness.L                                                    & \textbf{0.461** }     & 1.586      & 0.179   & 2.571     & 0.01  \\
Privacy.settings.awareness.Q                                                    & 0.076        & 1.079      & 0.156   & 0.484     & 0.628 \\
Privacy.settings.awareness.C                                                    & -0.019       & 0.981      & 0.117   & -0.166    & 0.868 \\
Privacy.settings.awareness\textasciicircum{}4                                   & -0.133       & 0.875      & 0.12    & -1.104    & 0.269 \\
I.trust.businessess.L                                                           & \textbf{1.159**}      & 3.187      & 0.259   & 4.484     & 0     \\
I.trust.businessess.Q                                                           & \textbf{0.62**}       & 1.859      & 0.222   & 2.791     & 0.005 \\
I.trust.businessess.C                                                           & 0.163        & 1.177      & 0.153   & 1.068     & 0.286 \\
I.trust.businessess\textasciicircum{}4                                          & 0.085        & 1.089      & 0.115   & 0.741     & 0.459
\end{tabular}
\label{tab:attitude_privacy_coeffs}
\end{table*}

%Our regression analysis revealed significantly different thresholds between the following pairs:
%\begin{itemize}
%\item Very uncomfortable and uncomfortable
%\item Uncomfortable and Neutral
%\item Comfortable and very comfortable
%\end{itemize}
%These findings indicate distinct thresholds for the ordinal classes.

\section{Kruskal Wallis and Dunn posthoc test}
\label{section:KWDunn}
The Kruskal-Wallis test, a non-parametric method, is used to identify statistically significant differences in medians across multiple independent groups. We use the "survey" package in R to factor for the random intercepts while calculating the Kruskal-wallis statistic. While this test determines the presence of differences, it does not specify which pairs differ. Therefore, we conduct a post-hoc Dunn test to identify pairs of groups with statistically significant median differences. This combined approach is employed throughout our statistical analysis to discern perception differences. We do not use interaction effects in regression settings because in interactions, the significant coefficients indicate odds ratio compared to only the baseline interaction e.g., actor:education interactions that are significant in the regression setting would only be significantly higher/lower from the baseline (Commercial Entity: Highschool to Bachelors). The comparisons for all actors:education pairs cannot be established with a single regression model. KW with Dunn can do pairwise comparisons and post hoc corrections to reduce FDR (False Discovery Rate). 

We demonstrate the process using an example. Let us assume we are interested in understanding significant differences in comfort levels across actors and location features. 
Participants were asked about their comfort levels in sharing location data for different actors and features. Each actor-feature combination (N=138) represents an independent group, allowing for comparison using the Kruskal-Wallis test. The test yielded an F statistic of [316.28] with p-value < 0.05, indicating significant differences between the medians of actor-feature pairs.
Following the Kruskal-Wallis test, we employed the Dunn post-hoc test with the Benjamini-Hochberg (to reduce false discovery rate) correction for multiple comparisons. This post-hoc analysis produces an N x N matrix (where N is the number of actor-feature combinations), with each cell containing the adjusted p-value for the corresponding pairwise comparison.
For each actor, we tabulate only those features whose median comfort levels were found to be \textbf{statistically significantly different (pvalue < 0.05) by at least 1}. By grouping the results by actors and analyzing the different features per actor, we can understand how perceptions of the same actor may vary when accessing different types of location data.
This analytical approach is applied throughout the paper for various combinations, including actor-purpose, purpose-feature, education-feature, and privacy-actor pairs, to provide a comprehensive understanding of the factors influencing location data sharing comfort levels.

\begin{table}[]
\centering
\footnotesize
\caption{KW with Dunn test to identify significantly different medians when different actors and purposes are involved with the same feature. Median differences are statistically different (pval <0.01)
Kruskal-Wallis Chi-sq statistic interaction(feature,actor)= 316.28 pval<0.01; interaction(feature,purpose)= 236.13 pval<0.01;
}
\begin{tabular}{|p{0.18\linewidth}|p{0.17\linewidth}|p{0.18\linewidth}|p{0.06\linewidth}|p{0.05\linewidth}|p{0.08\linewidth}|}
\hline
Feature & Actor1   & Actor2   & M1 & M2 & \text{\footnotesize{M2-M1}} \\ \hline
Radius(Ob)                 & Employer                   & Researchers                & -1                           & 1                            & 2                               \\ \cline{3-6}
&                               & Commercial                 & -1                           & 0                            & 1                               \\ \cline{3-6}
&                               & 911                        & -1                           & 1                            & 2                               \\ \cline{3-6}
&                               & Family                     & -1                           & 1                            & 2                               \\ \cline{3-6}
&                               & Local gov                  & -1                           & 1                            & 2                               \\ \cline{1-6}
Radius(Ob)                 & Fed                        & 911                        & 0                            & 1                            & 1                               \\ \cline{1-6}
Freq Walks(Ob)             & Employer                   & Researchers                & 0                            & 1                            & 1                               \\ \cline{3-6}
&                               & Doc                        & 0                            & 1                            & 1                               \\ \cline{3-6}
&                               & Family                     & 0                            & 1                            & 1                               \\ \cline{1-6}
Freq Walks(Ob)             & Fed                        & Researchers                & 0                            & 1                            & 1                               \\ \cline{3-6}
&                               & Doc                        & 0                            & 1                            & 1                               \\ \cline{3-6}
&                               & Family                     & 0                            & 1                            & 1                               \\ \cline{1-6}
Freq Walks(Ob)             & Local gov                  & Doc                        & 0                            & 1                            & 1                               \\ \cline{3-6}
Home(D)                    & Commercial                 & Local gov                  & 0                            & 1                            & 1                               \\ \cline{3-6}
& Fed                        & Local gov                  & -0.5                         & 1                            & 1.5                             \\ \cline{1-6}
Least FreqT(D)             & Employer                   & Researchers                & -1                           & 1                            & 2                               \\ \cline{3-6}
&                               & Commercial                 & -1                           & 0                            & 1                               \\ \cline{3-6}
&                               & 911                        & -1                           & 1                            & 2                               \\ \cline{1-6}
Transport-    & Employer                   & Researchers                & 0                            & 1                            & 1                               \\
ation(Ob) & & & & & \\
\cline{1-6}
FreqT(D)                   & Employer                   & Researchers                & -1                           & 0                            & 1                               \\ \cline{3-6}
&                               & 911                        & -1                           & 1                            & 2                               \\ \cline{1-6}
FreqT-counties(Ob)         & Employer                   & Researchers                & -1                           & 1                            & 2                               \\ \cline{3-6}
&                               & Commercial                 & -1                           & 0                            & 1                               \\ \cline{3-6}
&                               & 911                        & -1                           & 1                            & 2                               \\ \cline{3-6}
&                               & Law EA                     & -1                           & 0                            & 1                               \\ \cline{3-6}
&                               & Local gov                  & -1                           & 1                            & 2                               \\ \cline{1-6}
FreqT-counties(Ob)         & Fed                        & Researchers                & -0.5                         & 1                            & 1.5                             \\ \cline{3-6}
&                               & 911                        & -0.5                         & 1                            & 1.5                             \\ \cline{3-6}
&                               & Local gov                  & -0.5                         & 1                            & 1.5                             \\ \cline{1-6}
FreqTypeT(D)               & Employer                   & Researchers                & -0.5                         & 1                            & 1.5                             \\ \cline{1-6}
Visits(D)                  & Employer                   & Researchers                & -1                           & 1                            & 2                               \\ \cline{3-6}
&                               & Commercial                 & -1                           & 0                            & 1                               \\ \cline{3-6}
&                               & Doc                        & -1                           & 1                            & 2                               \\  \cline{3-6}
&                               & 911                        & -1                           & 1                            & 2                               \\  \cline{1-6}
Visits(Ob)                 & Employer                   & Researchers                & -1                           & 1                            & 2                               \\ \cline{3-6}
&                               & Commercial                 & -1                           & 0                            & 1                               \\ \cline{3-6}
&                                & Family                     & -1                           & 1                            & 2                               \\\cline{3-6}
&                               & Law EA                     & -1                           & 0                            & 1                               \\ \cline{3-6}
&                               & Local gov                  & -1                           & 0                            & 1                               \\ \cline{1-6}
Feature & Purpose1   & Purpose2   & M1 & M2 & \text{\footnotesize{M2-M1}} \\ \cline{1-6}
Visits(Ob)                 & Monitor mobility           & Public transit             & 0                            & 1                            & 1                               \\ \cline{3-6}
&                               & Infra (Walk/Cycling)       & 0                            & 1                            & 1               \\
\cline{1-6}
\end{tabular}
\label{tab:kw_featureXactor_featureXpurpose_BIG}
\end{table}


\begin{table}[]
\centering
\footnotesize
\caption{KW with Dunn test to identify significantly different medians when different features are involved with an actor. Median differences are statistically different (pvalue <0.01). Kruskal wallis chi-sq interaction(feature,actor)= 316.28 pval<0.01
}
\begin{tabularx}{\linewidth}{|p{0.15\linewidth}|p{0.20\linewidth}|p{0.23\linewidth}|p{0.06\linewidth}|p{0.03\linewidth}|p{0.07\linewidth}|}
 \cline{1-6}
Actors        & Feature1              & Feature2            & M1 & M2 & \text{\footnotesize{M2-M1}} \\  \cline{1-6}
Commercial & FreqT(D)           & Freq Walks(Ob)   & 0       & 1       & 1          \\ \cline{1-6}
Doc        & Radius(Ob)         & Freq Walks(Ob)   & 0       & 1       & 1          \\
& Least FreqT(D)     & Freq Walks(Ob)   & 0       & 1       & 1          \\
& FreqT(D)           & Freq Walks(Ob)   & -0.5    & 1       & 1.5        \\
& FreqTypeT(D)       & Freq Walks(Ob)   & 0       & 1       & 1          \\ \cline{1-6}
Employer   & FreqT-counties(Ob) & Freq Walks(Ob)   & -1      & 0       & 1          \\ \cline{1-6}
Fed        & FreqT-counties(Ob) & International(D) & -0.5    & 1       & 1.5        \\ \cline{1-6}
Local gov  & Freq Walks(D)      & Home(D)          & 0       & 1       & 1          \\
& Visits(D)          & Home(D)          & 0       & 1       & 1         \\  \cline{1-6}
\end{tabularx}
\label{tab:kw_actor_feature_BIG}
\end{table}



\begin{table}[]
\centering
\footnotesize
\caption{KW with Dunn test to identify significantly different medians when obfuscated features are used for different actors and purposes. Median differences are statistically different (pval <0.01). Kruskal wallis chi-sq statistic for interaction(actor,privacy) = 192.84 pval<0.01; interaction(purpose,privacy) = 120.06 pval<0.01; interaction(feature,privacy) = 37.26 pval<0.01}
\begin{tabularx}{\linewidth}{|p{0.21\linewidth}|p{0.18\linewidth}|p{0.18\linewidth}|p{0.03\linewidth}|p{0.03\linewidth}|p{0.09\linewidth}|}
\cline{1-6}
\textbf{Component} & \textbf{Feature}    & \textbf{Feature}  & M1 & M2 & \text{\footnotesize{M1-M2}} \\ 
\cline{1-6}
Family                           & Obfuscated & Detailed & 1       & 0       & 1          \\ \cline{1-6}
Control diseases                 & Obfuscated & Detailed & 1       & 0       & 1         \\
\cline{1-6}
\end{tabularx}
\label{tab:kw_ap_obfuscate_BIG}
\end{table}


\begin{table}[ht]
\centering
\footnotesize
\caption{
KW with Dunn test to identify significantly different medians for Actors when different ethnicity/race participants are involved. Median differences are statistically different (pvalue <0.01). Kruskal wallis chi-sq statistic for interaction(actor,ethnicity) = 241.47 pval<0.01}
\begin{tabular}{|p{0.15\linewidth}|p{0.19\linewidth}|p{0.22\linewidth}|p{0.03\linewidth}|p{0.03\linewidth}|p{0.08\linewidth}|}
\hline
Ethnicity & Actor1            & Actor2               & M1 & M2 & \text{\footnotesize{M2-M1}} \\ \hline
Asian     & Employer          & Family               & -1 & 1  & 2     \\
          & Fed               & Family               & 0  & 1  & 1     \\ \hline
White     & Commercial        & Researchers          & 0  & 1  & 1     \\
          &                   & 911                  & 0  & 1  & 1     \\ \cline{3-6}
          
          & Employer          & Researchers          & -1 & 1  & 2     \\
          &                   & Commercial           & -1 & 0  & 1     \\
          &                   & Doc                  & -1 & 0  & 1     \\
          &                   & 911                  & -1 & 1  & 2     \\
          &                   & Family               & -1 & 0  & 1     \\
          &                   & Fed                  & -1 & 0  & 1     \\
          &                   & Law EA               & -1 & 0  & 1     \\
          &                   & Local gov            & -1 & 0  & 1     \\ \cline{3-6}
          & Fed               & Researchers          & 0  & 1  & 1     \\
          &                   & 911                  & 0  & 1  & 1     \\ \cline{3-6}
          & Law EA            & Researchers          & 0  & 1  & 1     \\
          &                   & 911                  & 0  & 1  & 1     \\ \cline{3-6}
          & Local gov         & Researchers          & 0  & 1  & 1     \\
          &                   & 911                  & 0  & 1  & 1     \\ \hline

 
\end{tabular}
\label{tab:kw_ethnicity_act}
\end{table}


\begin{table}[ht]
\centering
\footnotesize
\caption{
KW with Dunn test to identify significantly different medians for ethnicity/racial group participants when different purpose, feature and privacy groups are involved. Median differences are statistically different (pvalue <0.01). Kruskal wallis chi-sq statistic for interaction(purpose,ethnicity) = 187.67 pval<0.01; interaction(feature,ethnicity) = 132.08 
pval<0.01; interaction(privacy,ethnicity) = 68.63 pval<0.01}
\begin{tabular}{|p{0.15\linewidth}|p{0.19\linewidth}|p{0.22\linewidth}|p{0.03\linewidth}|p{0.03\linewidth}|p{0.07\linewidth}|}
\hline
 Ethnicity         & Feature1          & Feature2             & M1 & M2 & \text{\footnotesize{M2-M1}} \\
 \hline
White     & FreqT(D)          & Freq Walks(Ob)       & 0  & 1  & 1     \\
          &                   &                      &    &    &       \\ \hline
          & Feature           & Feature              & M1 & M2 & \text{\footnotesize{M2-M1}} \\  \hline
Asian     & Detailed          & Obfuscated           & 0  & 1  & 1    \\ \hline
          & Purpose1          & Purpose2             & M1 & M2 & \text{\footnotesize{M2-M1}} \\ \hline
Asian     & Monitor mobility  & Infra (Buildings)    & 0  & 1  & 1     \\ \hline
Black     & Public transit    & Ads                  & 0  & 1  & 1     \\
          & Terrorist attacks & Ads                  & 0  & 1  & 1     \\
          & Monitor mobility  & Ads                  & 0  & 1  & 1     \\ \hline
Hispanic  & Public transit    & Infra (Walk/Cycling) & 0  & 1  & 1     \\
          &                   & Infra (Buildings)    & 0  & 1  & 1     \\
          &                   & Work                 & 0  & 1  & 1     \\ \cline{3-6}
          & Ads               & Criminal activity    & -1 & 1  & 2     \\
          &                   & Terrorist attacks    & -1 & 1  & 2     \\
          &                   & Control diseases     & -1 & 1  & 2     \\
          &                   & Infra (Walk/Cycling) & -1 & 1  & 2     \\
          &                   & Infra (Buildings)    & -1 & 1  & 2     \\
          &                   & Monitor mobility     & -1 & 1  & 2     \\
          &                   & Work                 & -1 & 1  & 2     \\
          &                   & Wellness             & -1 & 1  & 2     \\ \hline
White     & Criminal activity & Infra (Walk/Cycling) & 0  & 1  & 1     \\
          &                   & Infra (Buildings)    & 0  & 1  & 1     \\ \cline{3-6}
          & Terrorist attacks & Infra (Walk/Cycling) & 0  & 1  & 1     \\
          &                   & Infra (Buildings)    & 0  & 1  & 1     \\ \cline{3-6}
          & Control diseases  & Infra (Buildings)    & 0  & 1  & 1     \\  \cline{3-6}
          & Monitor mobility  & Infra (Walk/Cycling) & 0  & 1  & 1     \\ 
          &                   & Infra (Buildings)    & 0  & 1  & 1     \\ \cline{3-6}
          & Work              & Infra (Walk/Cycling) & 0  & 1  & 1     \\
          &                   & Infra (Buildings)    & 0  & 1  & 1     \\ \cline{3-6}
          & Wellness          & Infra (Walk/Cycling) & 0  & 1  & 1     \\
          &                   & Infra (Buildings)    & 0  & 1  & 1     \\ \cline{3-6}
          & Ads               & Infra (Buildings)    & 0  & 1  & 1     \\
          &                   &                      &    &    &       \\  \hline
\end{tabular}
\label{tab:kw_ethnicity_p_f_priv}
\end{table}

\begin{table}[]
\centering
\footnotesize
\caption{
Subsample of KW with Dunn test to identify significantly different medians for actors, purposes and features when different education groups are involved. Median differences are statistically different (pvalue <0.01). Kruskal wallis chi-sq statistic for interaction(actor,education) = 198.53 pval<0.01; interaction(purpose,education) = 121.38 pval<0.01; interaction(feature,education) = 86.08 
pval<0.01; interaction(privacy,education) = 31.44 pval<0.01}
\begin{tabularx}{\linewidth}{|p{0.15\linewidth}|p{0.19\linewidth}|p{0.22\linewidth}|p{0.03\linewidth}|p{0.03\linewidth}|p{0.07\linewidth}|}
\cline{1-6}
Education & Actor1               & Actor2                  & M1 & M2 & \text{\footnotesize{M2-M1}} \\ \cline{1-6}
Bachelors and above                    & Employer          & Researchers          & -1      & 1       & 2          \\
&                      & Commercial           & -1      & 0       & 1          \\
&                      & Doc                  & -1      & 1       & 2          \\
&                      & 911                  & -1      & 1       & 2          \\
&                      & Family               & -1      & 1       & 2          \\
&                      & Fed                  & -1      & 0       & 1          \\
&                      & Law EA               & -1      & 1       & 2          \\ 
&                      & Local gov            & -1      & 1       & 2          \\
 \cline{1-6}
Highschool to Bachelor                & Employer          & Researchers          & -1      & 1       & 2          \\
&                      & Commercial           & -1      & 0       & 1          \\
&                      & Doc                  & -1      & 0       & 1          \\
&                      & 911                  & -1      & 1       & 2          \\
&                      & Family               & -1      & 0       & 1          \\
&                      & Fed                  & -1      & 0       & 1          \\
&                      & Law EA               & -1      & 0       & 1          \\
&                      & Local gov            & -1      & 0       & 1          \\  \cline{3-6}
& Fed               & Researchers          & 0       & 1       & 1          \\
&                      & 911                  & 0       & 1       & 1          \\  \cline{3-6}
& Law EA            & Researchers          & 0       & 1       & 1          \\
&                      & 911                  & 0       & 1       & 1          \\  \cline{3-6}
& Local gov         & Researchers          & 0       & 1       & 1          \\
&                      & 911                  & 0       & 1       & 1          \\
&                      &                         &         &         &            \\  \cline{1-6}
& Feature1             & Feature2                & M1 & M2 & \text{\footnotesize{M2-M1}} \\ \cline{1-6}
Bachelors and above                    & Least FreqT(D)    & Freq Walks(Ob)       & 0       & 1       & 1          \\
& FreqT(D)          & Freq Walks(Ob)       & 0       & 1       & 1          \\  \cline{1-6}
Highschool to Bachelor                 & FreqT(D)          & Freq Walks(Ob)       & 0       & 1       & 1          \\
&                      &                         &         &         &            \\  \cline{1-6}
& Purpose1             & Purpose2                & M1 & M2 & \text{\footnotesize{M2-M1}} \\ \cline{1-6}
Bachelors and above                    & Monitor mobility  & Public transit       & 0       & 1       & 1          \\
&                      & Criminal activity    & 0       & 1       & 1          \\
&                      & Control diseases     & 0       & 1       & 1          \\
&                      & Infra (Walk/Cycling) & 0       & 1       & 1          \\
&                      & Infra (Buildings)    & 0       & 1       & 1          \\ \cline{3-6}
& Work              & Infra (Walk/Cycling) & 0       & 1       & 1          \\
&                      & Infra (Buildings)    & 0       & 1       & 1          \\
 \cline{1-6}
& Feature              & Feature                 & M1 & M2 & \text{\footnotesize{M2-M1}} \\ \cline{1-6}
Bachelors and above                    & Detailed             & Obfuscated              & 0       & 1       & \\ \cline{1-6}
\end{tabularx}
\label{tab:kw_bachelors_apf}
\end{table}


\iffalse
\section{Attitudes, Actions and Awareness}
\label{attitudes_actions_awareness}

% For gender, we observe that proportion of responses with Very comfortable in male participants is higher than female participants while female participants had higher Neutral comfortableness proportion. In age, we see that as age increases the proportion of responses with "Very uncomfortable" increases and Very comfortable decreases. For Ethnicity, we see that Hispanic participants selected "Very comfortable" more often than other options (behaviour not seen in other ethnicity). For education, we see that as people get higher education, proportion of Neutral stances decrease and proportion of "Very Uncomfortable" and "Very comfortable" increase         




 Analysis of privacy actions and awareness (Table \ref{tab:privacy_action_responses}) reveals the following trends:
 \begin{enumerate}
    \item \textbf{App Avoidance Due to Privacy Concerns:}There is a negative trend between the number of apps avoided due to privacy concerns and willingness to share location data. Participants who avoid more apps are less likely to share their location data.
    \item \textbf{App Uninstallation Due to Data Collection:} A similar trend is observed regarding app uninstallations. Participants who have uninstalled more apps due to data collection concerns demonstrate higher levels of discomfort with sharing location data.
    \item \textbf{Technical Expertise and Data Sharing Comfort:} Individuals with greater experience in Computer Science, Programming, Machine Learning, and Statistics exhibit higher comfort levels in sharing location data.
\end{enumerate}

\begin{table*}[]
\centering
\footnotesize
\caption{Percentage of responses for each of the Likert options for different Privacy related Actions and Privacy Awareness Questions }
\begin{tabularx}{\textwidth}{|p{0.21\textwidth}p{0.09\textwidth}|p{0.13\textwidth}p{0.15\textwidth}p{0.09\textwidth}p{0.12\textwidth}p{0.13\textwidth}|}
\multicolumn{2}{|c|}{\textbf{Actions and Awareness}}  & \multicolumn{5}{c|}{\textbf{Comfort Level}} \\
    \cline{3-7}
    & & \textbf{Very Uncomfortable} & \textbf{Uncomfortable} & \textbf{Neutral} & \textbf{Comfortable} & \textbf{Very Comfortable} \\
    \hline
\multirow{5}{*}{\parbox{3.6cm}{\textbf{Frequency of use of data collection tools like GMaps, Fitbit}}} & \textbf{Never}                    & \cellcolor[HTML]{B2E0F3}20.732 & \cellcolor[HTML]{C5E8F6}15.61  & \cellcolor[HTML]{B4E1F3}20.244 & \cellcolor[HTML]{A2D9F0}25.122 & \cellcolor[HTML]{BBE4F5}18.293 \\
& \textbf{Daily}                    & \cellcolor[HTML]{CBEAF7}13.946 & \cellcolor[HTML]{C1E6F5}16.705 & \cellcolor[HTML]{B8E2F4}19.157 & \cellcolor[HTML]{90D2EE}29.847 & \cellcolor[HTML]{B3E1F3}20.345 \\
& \textbf{Occasionally}             & \cellcolor[HTML]{CCEAF7}13.864 & \cellcolor[HTML]{BDE4F5}17.879 & \cellcolor[HTML]{98D5EF}27.803 & \cellcolor[HTML]{90D2EE}29.773 & \cellcolor[HTML]{D8EFF9}10.682 \\
& \textbf{Weekly}                   & \cellcolor[HTML]{D1ECF8}12.552 & \cellcolor[HTML]{BCE4F5}17.969 & \cellcolor[HTML]{ABDDF2}22.5   & \cellcolor[HTML]{88CFEC}32.031 & \cellcolor[HTML]{C8E9F7}14.948 \\
 & \textbf{Monthly}                  & \cellcolor[HTML]{CCEAF7}13.896 & \cellcolor[HTML]{ACDEF2}22.338 & \cellcolor[HTML]{A4DAF1}24.416 & \cellcolor[HTML]{95D4EE}28.571 & \cellcolor[HTML]{D7EFF9}10.779 \\
 \cline{1-7}
\multirow{5}{*}{\parbox{3.6cm}{\textbf{Frequency of Privacy Settings check}}} & \textbf{Never}                    & \cellcolor[HTML]{C8E9F7}14.771 & \cellcolor[HTML]{BFE5F5}17.124 & \cellcolor[HTML]{ACDEF2}22.353 & \cellcolor[HTML]{9BD7EF}26.797 & \cellcolor[HTML]{B9E3F4}18.954 \\
& \textbf{Daily}                    & \cellcolor[HTML]{D7EFF9}10.896 & \cellcolor[HTML]{DBF1FA}9.701  & \cellcolor[HTML]{BEE5F5}17.463 & \cellcolor[HTML]{83CDEC}33.284 & \cellcolor[HTML]{94D4EE}28.657 \\
& \textbf{Occasionally}             & \cellcolor[HTML]{C7E8F6}15.213 & \cellcolor[HTML]{B2E0F3}20.656 & \cellcolor[HTML]{ABDDF2}22.721 & \cellcolor[HTML]{8FD2ED}30.066 & \cellcolor[HTML]{D5EEF9}11.344 \\
& \textbf{Weekly}                   & \cellcolor[HTML]{D4EEF8}11.719 & \cellcolor[HTML]{C3E7F6}16.14  & \cellcolor[HTML]{ADDEF2}22.105 & \cellcolor[HTML]{8BD0ED}31.228 & \cellcolor[HTML]{B9E3F4}18.807 \\
& \textbf{Monthly}                  & \cellcolor[HTML]{C9E9F7}14.554 & \cellcolor[HTML]{BEE5F5}17.589 & \cellcolor[HTML]{A4DAF1}24.464 & \cellcolor[HTML]{95D4EE}28.571 & \cellcolor[HTML]{C8E9F7}14.821 \\
\cline{1-7}
\cline{1-7}
\multirow{3}{*}{\parbox{3.6cm}{\textbf{Avoid mobile apps due to privacy concern}}} & \textbf{Never}                    & \cellcolor[HTML]{DBF1FA}9.782  & \cellcolor[HTML]{D1EDF8}12.3   & \cellcolor[HTML]{B2E0F3}20.775 & \cellcolor[HTML]{84CDEC}33.075 & \cellcolor[HTML]{A5DBF1}24.068 \\
& \textbf{Once}                     & \cellcolor[HTML]{D8EFF9}10.572 & \cellcolor[HTML]{B6E2F4}19.598 & \cellcolor[HTML]{A8DCF1}23.493 & \cellcolor[HTML]{83CDEB}33.354 & \cellcolor[HTML]{CFECF8}12.983 \\
& \textbf{Many}                     & \cellcolor[HTML]{A1D9F0}25.202 & \cellcolor[HTML]{B1DFF3}21.098 & \cellcolor[HTML]{ADDEF2}22.023 & \cellcolor[HTML]{B4E1F3}20.116 & \cellcolor[HTML]{D4EEF9}11.561 \\
\cline{1-7}
\multirow{3}{*}{\parbox{3.6cm}{\textbf{Uninstalled app for collecting data}}} & \textbf{Never}                    & \cellcolor[HTML]{D9F0F9}10.262 & \cellcolor[HTML]{CCEBF7}13.668 & \cellcolor[HTML]{B2E0F3}20.742 & \cellcolor[HTML]{82CDEB}33.493 & \cellcolor[HTML]{AEDEF2}21.834 \\
& \textbf{Once}                     & \cellcolor[HTML]{D3EEF8}11.82  & \cellcolor[HTML]{B5E1F4}19.936 & \cellcolor[HTML]{AADDF2}22.802 & \cellcolor[HTML]{87CFEC}32.206 & \cellcolor[HTML]{CEEBF8}13.237 \\
& \textbf{Many}                     & \cellcolor[HTML]{A9DCF2}23.119 & \cellcolor[HTML]{B6E2F4}19.633 & \cellcolor[HTML]{A7DCF1}23.67  & \cellcolor[HTML]{B1E0F3}20.979 & \cellcolor[HTML]{D0ECF8}12.599 \\
\cline{1-7}
\multirow{4}{*}{\parbox{3.6cm}{\textbf{Computer Science or Programming Experience}}}& \textbf{None}        & \cellcolor[HTML]{BCE4F5}18.061 & \cellcolor[HTML]{B7E2F4}19.354 & \cellcolor[HTML]{A8DCF1}23.394 & \cellcolor[HTML]{A0D9F0}25.455 & \cellcolor[HTML]{CCEBF7}13.737 \\
& \textbf{Limited}     & \cellcolor[HTML]{CEECF8}13.136 & \cellcolor[HTML]{B5E1F3}20     & \cellcolor[HTML]{A9DCF2}23.195 & \cellcolor[HTML]{8BD0ED}31.045 & \cellcolor[HTML]{D0ECF8}12.623 \\
& \textbf{Significant} & \cellcolor[HTML]{E0F3FA}8.516  & \cellcolor[HTML]{CDEBF7}13.613 & \cellcolor[HTML]{ADDEF2}21.935 & \cellcolor[HTML]{75C7E9}37.097 & \cellcolor[HTML]{B9E3F4}18.839 \\
 & \textbf{Expert}      & \cellcolor[HTML]{C9EAF7}14.468 & \cellcolor[HTML]{D3EDF8}11.915 & \cellcolor[HTML]{CDEBF7}13.404 & \cellcolor[HTML]{A2D9F0}25.106 & \cellcolor[HTML]{7CCAEA}35.106 \\
 \cline{1-7}
\multirow{4}{*}{\parbox{3.6cm} {\textbf{Machine   Learning Experience}}}
& \textbf{None}        & \cellcolor[HTML]{C0E6F5}16.882 & \cellcolor[HTML]{B6E1F4}19.775 & \cellcolor[HTML]{A7DBF1}23.792 & \cellcolor[HTML]{9ED8F0}26.124 & \cellcolor[HTML]{CDEBF7}13.427 \\
& \textbf{Limited}     & \cellcolor[HTML]{D0ECF8}12.705 & \cellcolor[HTML]{BCE4F5}17.965 & \cellcolor[HTML]{A9DCF2}23.127 & \cellcolor[HTML]{86CEEC}32.506 & \cellcolor[HTML]{CCEBF7}13.697 \\
& \textbf{Significant} & \cellcolor[HTML]{E3F4FB}7.739  & \cellcolor[HTML]{CDEBF7}13.565 & \cellcolor[HTML]{B7E2F4}19.304 & \cellcolor[HTML]{73C7E9}37.478 & \cellcolor[HTML]{AEDEF2}21.913 \\
& \textbf{Expert}      & \cellcolor[HTML]{D6EFF9}11.148 & \cellcolor[HTML]{D9F0F9}10.164 & \cellcolor[HTML]{D5EEF9}11.475 & \cellcolor[HTML]{8CD1ED}30.82  & \cellcolor[HTML]{77C8EA}36.393 \\
\cline{1-7}

\multirow{4}{*}{\parbox{3.6cm}{\textbf{Statistics   Experience}}}& \textbf{None}        & \cellcolor[HTML]{C2E7F6}16.445 & \cellcolor[HTML]{B5E1F4}19.801 & \cellcolor[HTML]{A6DBF1}23.953 & \cellcolor[HTML]{9DD8F0}26.279 & \cellcolor[HTML]{CDEBF7}13.522 \\
& \textbf{Limited}     & \cellcolor[HTML]{CCEBF7}13.75  & \cellcolor[HTML]{BEE5F5}17.414 & \cellcolor[HTML]{A7DBF1}23.707 & \cellcolor[HTML]{87CFEC}32.198 & \cellcolor[HTML]{CFECF8}12.931 \\
& \textbf{Significant} & \cellcolor[HTML]{DFF2FA}8.614  & \cellcolor[HTML]{C5E8F6}15.73  & \cellcolor[HTML]{B8E2F4}19.176 & \cellcolor[HTML]{7AC9EA}35.655 & \cellcolor[HTML]{B2E0F3}20.824 \\
& \textbf{Expert}      & \cellcolor[HTML]{CBEAF7}13.973 & \cellcolor[HTML]{D3EEF8}11.781 & \cellcolor[HTML]{D3EEF8}11.781 & \cellcolor[HTML]{9DD8F0}26.301 & \cellcolor[HTML]{78C9EA}36.164 \\
\cline{1-7}
\end{tabularx}
\label{tab:privacy_action_responses}
\end{table*}
\fi 

\section{Survey and User Interface}
\label{section:surveyUI}
In this section, we discuss the survey in detail and the different sections in it. We built a custom survey webpage for better customization and functionality. Participants were shown ads for our survey on the Cint portal and they were directed to our survey page upon clicking on it.

The survey is divided into the following sections:
\begin{itemize}
    \item Consent and Briefing
    \item Demographic Information and General Technical Knowledge
    \item Understanding Level of comfort with location data sharing: Vignettes
    \item Privacy Attitudes
\end{itemize}

%\textcolor{yellow}{make Fig 11 two-column wide, otherwise you cannot read anything}

\subsection{Consent and Briefing}
\label{subsection:consentbrief}
 Figure \ref{fig:consentbrief} is the landing page of our survey. The landing page opens with a pop-up window (see Figure \ref{fig:irbpopup}) briefly describing the objective of the study. The participants are also provided with a link to the Internal Review Board's (IRB) document detailing the project, ethical considerations, data handling and contact information. All participants are asked for their name, email, whether they are 18 years or older, consent to participate and whether they are voluntarily agreeing to participate in the survey. If they press the "I do not consent" button, they are directed away from the website. When the participant presses on the "submit" button, the pop-up window disappears and they can read the introduction page (Figure \ref{fig:consentbrief}) which details our study, the different sections and the number of questions, asking participants to think about the data and visualizations they see as their own. The participant moves to the next section when they click on the "Next" button. 

% consentbriefing
\begin{figure*}
\centering
\includegraphics[width=0.7\linewidth]{surveypics/IRB_popup.png}
\caption{Popup showing the consent page and Internal Review Board (IRB) document link}
\label{fig:irbpopup}
\end{figure*}
\begin{figure*}
\centering
\includegraphics[width=0.7\textwidth]{surveypics/briefing.png}
\caption{The Introduction page}
\label{fig:consentbrief}
\end{figure*}

\subsection{Demographic Information and General Technical Knowledge}
After reading the brief and moving to the next page, participants are asked about their demographic details and general awareness with different technologies (see Figure \ref{fig:demographic_page}).


\subsection{Understanding Level of comfort with location data sharing: Vignettes }

There are 5 vignette questions asked to each participant. The questions are generated randomly from a set of questions and displayed on the participant's browser. There are two types of location features (Detailed vs Obfuscated). Fig. \ref{fig:locationdataqsdetailed} shows an example of a vignette focused on a detailed feature. In this example, participants see a chart showing statistics about the types of transportation inferred from location data as well as an interactive map with the detailed trajectories for each mode of transportation together with labels characterizing origin and destination place.
Participants can interact with the map to understand better all the information being shared. 
On the other hand, Figure \ref{fig:locationdataqspp} shows an example of a location feature computed in a privacy-preserving way. The participant can see general statistics about their walking activity and assess their level of comfort by selecting one of the five options. As can be observed, all vignettes also have a text box for participants to explain the rationale behind their choice. 

\subsection{Privacy Attitudes}
\label{martinAppendix}

After completing the 5 vignette questions, participants are required to fill out the privacy attitudes survey, shown in Figure \ref{fig:attitude_webpage}. Studies like \cite{ICADataCollection,naeini,martin,colombia_privacy_study} have asked privacy and attitude questions which try to understand the person's general attitudes towards data sharing and privacy. We include the questions in Martin et. al. \cite{martin} as additional control variables in our regression to understand how different attitudes relate to comfort in sharing specific location features. These questions indirectly measure control, awareness and collection aspects as studied in Internet Users Information Privacy Concerns (IUIPC) \cite{IUIPC} based studies. As our goals are strictly understanding the broader horizon of location features and comfort, we focus our attention to these questions rather than forming the hierarchical IUIPC scores. We incorporate responses to the questions as ordinal variables (see response categories in \ref{tab:attitudes_responses} ). Table \ref{tab:attitude_privacy_coeffs} highlights questions which were significantly related to comfort levels. We observe similar trends to \cite{martin} in these questions. People with positive attitudes towards government and businesses show higher comfort levels. While people with higher defiance to authority had inverse relation with sharing location data. Two interesting findings: (1) Positive relation between computer science knowledge and comfort in sharing location data. (2) People who are more aware of their privacy settings were less comfortable in sharing location data. 


%g on this information. In both the cases, they also fill out a text box explaining their thought process behind their choice.
% Location data vignette detailed
 \begin{figure*}[ht]
\centering
\includegraphics[width=0.7\textwidth]{surveypics/locationdataqsdetailed.png}
\caption{Example of Vignette Question (Actor: Law enforcement agency- like city police department or a county sheriff's office, Purpose: Understand criminal activity, Feature: Places you visit). The feature is detailed and shows an interactive map with information}
\label{fig:locationdataqsdetailed}
\end{figure*}
% Location data vignette privacy preserving
 \begin{figure*}[ht]
\centering
\includegraphics[width=0.7\textwidth]{surveypics/locationdataqspp.png}
\caption{Example of Vignette Question (Actor:Commercial entity, Purpose: Personal wellness and physical activity, Feature: Walking activity). The feature is obfuscating and only shows general statistics}
\label{fig:locationdataqspp}
\end{figure*}


% Attitude question page
\begin{figure*}
\centering
\includegraphics[width=0.5\textwidth]{surveypics/attitudes.png}
\caption{Privacy attitude questions as seen by the survey participants}
\label{fig:attitude_webpage}
\end{figure*}

% Demographics question page
\begin{figure*}
\centering
\includegraphics[width=0.7\textwidth]{surveypics/demographicinfo.png}
\caption{Demographic and technical knowledge questions as seen by the survey participants}
\label{fig:demographic_page}
\end{figure*}




% \begin{figure}
% \includegraphics[width=0.5\textwidth]{Images/eduXactor.png}
% %\caption{For which actors do we see differences in comfortableness in sharing location data based on education?}
% \caption{KW and posthoc Dunn test for levels of comfort across significant education levels and actors}
% \label{fig:act_edu_interact}
% \end{figure}
% \begin{figure}
% \includegraphics[width=0.5\textwidth]{Images/eduXpriv.png}
% %\caption{Do different education brackets perceive Obfuscated features differently from Detailed features?}
% \caption{KW and posthoc Dunn test for levels of comfort across significant education levels and privacy-preserving approaches}
% \label{fig:edu_priv_interact}
% \end{figure}
% \begin{figure}
% \includegraphics[width=0.5\textwidth]{Images/eduXpurpose.png}
% %\caption{Do different education brackets perceive purposes differently?}
% \caption{KW and posthoc Dunn test for levels of comfort across significant education levels and purposes}
% \label{fig:edu_pur_interact}
% \end{figure}

%all figures
% homed
\begin{figure}
\includegraphics[width=0.5\textwidth]{surveypics/homeD.png}
\caption{Snapshot of interactive map with home Feature (Home(D)) as visible to participants. It shows the exact home location of the hypothetical data contributor using a pin marker.}
\label{fig:homeD}
\end{figure}
% homepp
\begin{figure}
\includegraphics[width=0.5\textwidth]{surveypics/homePP.png}
\caption{Snapshot of interactive map with home Feature (Home(Ob)) as visible to participants. It shows the census tract region the hypothetical data contributor lives in.}
\label{fig:homePP}
\end{figure}
% workd
\begin{figure}
\includegraphics[width=0.5\textwidth]{surveypics/workd.png}
\caption{Snapshot of interactive map with Work location feature (Work(D)) as visible to participants. It shows the exact work location of the hypothetical data contributor using a pin marker.}
\label{fig:workD}
\end{figure}
% workpp
\begin{figure}
\includegraphics[width=0.5\textwidth]{surveypics/workPP.png}
\caption{Snapshot of interactive map with Work  feature (Work(Ob)) as visible to participants. It shows the census tract region where the hypothetical data contributor works at.}
\label{fig:workPP}
\end{figure}
% PLACES
\begin{figure}
\includegraphics[width=0.5\textwidth]{surveypics/POID.png}
\caption{Snapshot of interactive map with most Places of Visit (D) feature (Visits(D) as visible to participants. It shows the specific places the hypothetical data contributor frequently visits. The types of places are color coded (and consistent between the map and chart in Fig. \ref{fig:POVPP}) and labelled appropriately }
\label{fig:POVD}
\end{figure}
\begin{figure}
\includegraphics[width=0.5\textwidth]{surveypics/POIPP.png}
\caption{Chart with Places of Visit (Ob) feature (Visits(Ob)) as visible to participants. The bar chart shows the different places of interest our hypothetical data contributor visits and the number of days such visits happen.}
\label{fig:POVPP}
\end{figure}
\begin{figure}
\includegraphics[width=0.5\textwidth]{surveypics/ROG.png}
\caption{Snapshot of interactive map with most Geographical area you spend most time feature (Radius(Ob) as visible to participants. The circle describes the region our hypothetical data contributor stays in. The pin denotes the center of this region}
\label{fig:ROG}
\end{figure}
% modesD
\begin{figure}
\includegraphics[width=0.5\textwidth]{surveypics/modesD.png}
\caption{Snapshot of interactive map with transportation Feature (Transportation(D)) as visible to participants. The map displays different modes of transport (driving, walking, etc) with different colors consistently between the map and the chart in fig. \ref{fig:modesPP}. For our hypothetical data contributor, the pins denote the start and end points and the line joining the pins denote the most frequent route taken by that mode of transport. E.g. Black pins and line segment denote the most frequent walking trip. The label "Riverfront" indicates the inferred type of trip}
\label{fig:modesD}
\end{figure}
%modesPP
\begin{figure}
\includegraphics[width=0.5\textwidth]{surveypics/modesPP.png}
\caption{Chart with modes of transportation Feature (Transportation(Ob)) as visible to participants. The bar chart shows the different modes of transport our hypothetical data contributor takes and the number of hours spent in each mode.}
\label{fig:modesPP}
\end{figure}

% FreqD
\begin{figure}
\includegraphics[width=0.5\textwidth]{surveypics/mostfreqtrips.png}
\caption{Snapshot of interactive map with most frequent trips feature (FreqT (D) as visible to participants. For our hypothetical data contributor, the pins denote the start and end points and the line joining the pins denote one of the most frequent routes. E.g. red pins and line segment denote the most frequent trip between home and work.}
\label{fig:freqTD}
\end{figure}
% leastFreqD
\begin{figure}
\includegraphics[width=0.5\textwidth]{surveypics/leastFreqTrips.png}
\caption{Snapshot of interactive map with Least Frequent trips feature (LeastFreqT(D)) as visible to participants. For our hypothetical data contributor, the pins denote the start and end points and the line joining the pins denote one of the least frequent routes. E.g. red pins and line segment denote the least frequent trip between home and hospital.}
\label{fig:leastFreqT}
\end{figure}
% typeoftrip
\begin{figure}
\includegraphics[width=0.5\textwidth]{surveypics/freqtypetrip.png}
\caption{Snapshot of interactive map with Most frequent type of trips Feature (Freq-TypeT(Ob)) as visible to participants. For our hypothetical data contributor, the pins denote the start and end points and the line joining the pins denote one of types of frequent trips. E.g. First, the most frequent trips are identified and labelled according to end points. Green pins and line segment denote a frequent trip to school which is between "home" and "school" and thus is a type of school trip.}
\label{fig:freqtypetrip}
\end{figure}
% OD
\begin{figure}
\includegraphics[width=0.5\textwidth]{surveypics/mostFreqPPOD.png}
\caption{Snapshot of interactive map with Most frequent trips represented by their origin and destination census tracts
and connected by a line Feature (FreqT-counties(Ob)) as visible to participants. We see that the home census tract and work census tract are joined by a straight green line (most frequent movement between these census tracts). }
\label{fig:FreqOD}
\end{figure}
% gmaps
\begin{figure}
\includegraphics[width=0.5\textwidth]{surveypics/gmaps.png}
\caption{Snapshot of interactive map with Most frequent trips represented by their origin and destination census tract and connected by a synthetic route extracted from Google Maps (FreqT-Google(Ob)) as visible to participants. The home census tract centroid and work census tract centroid are joined by a segment suggested by Google maps routes API .}
\label{fig:gmaps}
\end{figure}
%walkD
\begin{figure}
\includegraphics[width=0.5\textwidth]{surveypics/walkD.png}
\caption{Snapshot of interactive map with Your walking activity and corresponding routes feature (Freq Walks(D)) as visible to participants. For our hypothetical data contributor, the pins denote the start and end points and the line joining the pins denote the most frequent walking path taken. E.g. Red pins and line segment denote the most frequent walking trip. The label "park" on both pins indicates the start and end of the walk was in a park}
\label{fig:walkD}
\end{figure}
%walkPP
\begin{figure}
\includegraphics[width=0.5\textwidth]{surveypics/walkPP.png}
\caption{Snapshot of the chart with Your walking activity (Freq Walks(Ob)) as visible to participants. Chart indicates frequency and duration of walks with no detailed trajectories}
\label{fig:walkPP}
\end{figure}
% internaltional
\begin{figure}
\includegraphics[width=0.5\textwidth]{surveypics/foreignD.png}
\caption{Snapshot of interactive map with International visits feature (International(D)) as visible to participants. The map highlights the different countries visited and the pins indicate specific locations of visit with the duration of stay }
\label{fig:internationalD}
\end{figure}
\begin{figure}
\includegraphics[width=0.5\textwidth]{surveypics/foreignPP.png}
\caption{Snapshot of interactive map with International visits feature (International(P)) as visible to participants. It specifies only the countries of visit and the duration of visit.}
\label{fig:internationalP}
\end{figure}

\end{document}
%%  LocalWords:  endnotes includegraphics fread ptr nobj noindent
%%  LocalWords:  pdflatex acks
