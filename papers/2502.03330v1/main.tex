%%
%% This is file `sample-authordraft.tex',
%% generated with the docstrip utility.
%%
%% The original source files were:
%%
%% samples.dtx  (with options: `authordraft')
%% 
%% IMPORTANT NOTICE:
%% 
%% For the copyright see the source file.
%% 
%% Any modified versions of this file must be renamed
%% with new filenames distinct from sample-authordraft.tex.
%% 
%% For distribution of the original source see the terms
%% for copying and modification in the file samples.dtx.
%% 
%% This generated file may be distributed as long as the
%% original source files, as listed above, are part of the
%% same distribution. (The sources need not necessarily be
%% in the same archive or directory.)
%%
%% The first command in your LaTeX source must be the \documentclass command.
% \documentclass[sigconf,authordraft]{acmart}


%%%% As of March 2017, [siggraph] is no longer used. Please use sigconf (above) for SIGGRAPH conferences.

%%%% As of May 2020, [sigchi] and [sigchi-a] are no longer used. Please use sigconf (above) for SIGCHI conferences.

%%%% Proceedings format for SIGPLAN conferences 
% \documentclass[sigplan, anonymous, authordraft]{acmart}

%%%% Proceedings format for conferences using one-column small layout
% \documentclass[sigconf]{acmart}  
\documentclass[manuscript]{acmart}
%\documentclass[manuscript,review,anonymous]{acmart}
  %\documentclass[acmtog,anonymous,review]{acmart}
 % \documentclass[sigconf,review,anonymous]{acmart}
  %\documentclass[sigconf,review,anonymous]{acmart}
% \documentclass[sigconf,anonymous,review]{acmart}
% \documentclass[sigconf]{acmart} 
% \documentclass[sigconf, manuscript]{acmart}

\usepackage[ruled,vlined,linesnumbered]{algorithm2e}
\usepackage{amsmath}

\usepackage{multicol, multirow}
%\usepackage[normalem]{ulem} 
\usepackage{color}
\usepackage{graphicx}
\usepackage{rotating}
%\usepackage{amssymb}% http://ctan.org/pkg/amssymb
\usepackage{pifont}% http://ctan.org/pkg/pifont
\usepackage{array}
\usepackage{makecell}
\usepackage{tabularx,booktabs} 
%\usepackage[dvipsnames]{xcolor}
%\definecolor{munsell}{rgb}{0.0, 0.5, 0.69}
\definecolor{munsell}{rgb}{0.13, 0.55, 0.13}
\newcommand{\cmark}{\color{munsell}\ding{51}}%
\newcommand{\xmark}{\color{red}\ding{55}}%


\DeclareMathOperator*{\argmax}{arg\,max}
\DeclareMathOperator*{\argmin}{arg\,min}
\newcommand\add[1]{#1}
\newenvironment{ADD}{\par}{\par}

\usepackage{marginnote}
\usepackage{cleveref}
\newcommand{\systemname}
{\textsc{OurSystemName}\xspace}
% NOTE that a single column version is required for submission and peer review. This can be done by changing the \doucmentclass[...]{acmart} in this template to 
% \documentclass[manuscript,screen]{acmart}

%%
%% \BibTeX command to typeset BibTeX logo in the docs
\AtBeginDocument{%
  \providecommand\BibTeX{{%
    \normalfont B\kern-0.5em{\scshape i\kern-0.25em b}\kern-0.8em\TeX}}}

%% Rights management information.  This information is sent to you
%% when you complete the rights form.  These commands have SAMPLE
%% values in them; it is your responsibility as an author to replace
%% the commands and values with those provided to you when you
%% complete the rights form.

% \acmDOI{10.1145/1122445.1122456}


\newcommand{\name}[1]{\def\papername{#1}}
\name{asci}
%% These commands are for a PROCEEDINGS abstract or paper.
% \acmConference[Woodstock '18]{Woodstock '18: ACM Symposium on Neural
%   Gaze Detection}{June 03--05, 2018}{Woodstock, NY}
% \acmBooktitle{Woodstock '18: ACM Symposium on Neural Gaze Detection,
%   June 03--05, 2018, Woodstock, NY}
% \acmPrice{15.00}
% \acmISBN{978-1-4503-XXXX-X/18/06}


%%
%% Submission ID.
%% Use this when submitting an article to a sponsored event. You'll
%% receive a unique submission ID from the organizers
%% of the event, and this ID should be used as the parameter to this command.
\acmSubmissionID{}

% \copyrightyear{2024}
% \acmYear{2024}
% \setcopyright{rightsretained}
% \acmConference[CHI '24]{Proceedings of the CHI Conference on Human Factors in Computing Systems}{May 11--16, 2024}{Honolulu, HI, USA}
% \acmBooktitle{Proceedings of the CHI Conference on Human Factors in Computing Systems (CHI '24), May 11--16, 2024, Honolulu, HI, USA}
% \acmDOI{10.1145/3613904.3642822}
% \acmISBN{979-8-4007-0330-0/24/05}

%%
%% The majority of ACM publications use numbered citations and
%% references.  The command \citestyle{authoryear} switches to the
%% "author year" style.
%%
%% If you are preparing content for an event
%% sponsored by ACM SIGGRAPH, you must use the "author year" style of
%% citations and references.
%% Uncommenting
%% the next command will enable that style.
%%\citestyle{acmauthoryear}

%%
%% end of the preamble, start of the body of the document source.
\begin{document}

%%
%% The "title" command has an optional parameter,
%% allowing the author to define a "short title" to be used in page headers.


% \title{Generating Graphical User Interfaces for Optimal Visual Flows}
\title{Controllable GUI Exploration}


%%
%% The "author" command and its associated commands are used to define
%% the authors and their affiliations.
%% Of note is the shared affiliation of the first two authors, and the
%% "authornote" and "authornotemark" commands
%% used to denote shared contribution to the research.
\author{Aryan Garg}
\email{aryan.garg@aalto.fi}
\authornote{Contributed equally.}
\affiliation{
    \institution{Aalto University} 
    \country{Finland}
}


\author{Yue Jiang}
\email{yue.jiang@aalto.fi}
\authornotemark[1]
\affiliation{
    \institution{Aalto University} 
    \country{Finland}
}


\author{Antti Oulasvirta}
\email{antti.oulasvirta@aalto.fi}
\affiliation{
    \institution{Aalto University} 
    \country{Finland}
}


% \author{Lars Th{\o}rv{\"a}ld}
% \affiliation{%
%   \institution{The Th{\o}rv{\"a}ld Group}
%   \streetaddress{1 Th{\o}rv{\"a}ld Circle}
%   \city{Hekla}
%   \country{Iceland}}
% \email{larst@affiliation.org}

% \author{Valerie B\'eranger}
% \affiliation{%
%   \institution{Inria Paris-Rocquencourt}
%   \city{Rocquencourt}
%   \country{France}
% }

% \author{Aparna Patel}
% \affiliation{%
%  \institution{Rajiv Gandhi University}
%  \streetaddress{Rono-Hills}
%  \city{Doimukh}
%  \state{Arunachal Pradesh}
%  \country{India}}

% \author{Huifen Chan}
% \affiliation{%
%   \institution{Tsinghua University}
%   \streetaddress{30 Shuangqing Rd}
%   \city{Haidian Qu}
%   \state{Beijing Shi}
%   \country{China}}

% \author{Charles Palmer}
% \affiliation{%
%   \institution{Palmer Research Laboratories}
%   \streetaddress{8600 Datapoint Drive}
%   \city{San Antonio}
%   \state{Texas}
%   \postcode{78229}}
% \email{cpalmer@prl.com}

% \author{John Smith}
% \affiliation{\institution{The Th{\o}rv{\"a}ld Group}}
% \email{jsmith@affiliation.org}

% \author{Julius P. Kumquat}
% \affiliation{\institution{The Kumquat Consortium}}
% \email{jpkumquat@consortium.net}




%%
%% By default, the full list of authors will be used in the e props
%% headers. Often, this list is too long, and will overlap
%% other information printed in the page headers. This command allows
%% the author to define a more concise list
%% of authors' names for this purpose.
\renewcommand{\shortauthors}{Jiang et al.}

%%
%% The abstract is a short summary of the work to be presented in the
%% article.
\begin{abstract}          

During the early stages of interface design, designers need to produce multiple sketches to explore a design space.  Design tools often fail to support this critical stage, because they insist on specifying more details than necessary. Although recent advances in generative AI have raised hopes of solving this issue, in practice they fail because expressing loose ideas in a prompt is impractical. In this paper, we propose a diffusion-based approach to the low-effort generation of interface sketches. It breaks new ground by allowing flexible control of the generation process via three types of inputs: A) prompts, B) wireframes, and C) visual flows. The designer can provide any combination of these as input at any level of detail, and will get a diverse gallery of low-fidelity solutions in response. The unique benefit is that large design spaces can be explored rapidly with very little effort in input-specification. We present qualitative results for various combinations of input specifications. Additionally, we demonstrate that our model aligns more accurately with these specifications than other models. 

% OLD ABSTRACT
%When sketching Graphical User Interfaces (GUIs), designers need to explore several aspects of visual design simultaneously, such as how to guide the user’s attention to the right aspects of the design while making the intended functionality visible. Although current Large Language Models (LLMs) can generate GUIs, they do not offer the finer level of control necessary for this kind of exploration. To address this, we propose a diffusion-based model with multi-modal conditional generation. In practice, our model optionally takes semantic segmentation, prompt guidance, and flow direction to generate multiple GUIs that are aligned with the input design specifications. It produces multiple examples. We demonstrate that our approach outperforms baseline methods in producing desirable GUIs and meets the desired visual flow.

% Designing visually engaging Graphical User Interfaces (GUIs) is a challenge in HCI research. Effective GUI design must balance visual properties, like color and positioning, with user behaviors to ensure GUIs easy to comprehend and guide attention to critical elements. Modern GUIs, with their complex combinations of text, images, and interactive components, make it difficult to maintain a coherent visual flow during design.
% Although current Large Language Models (LLMs) can generate GUIs, they often lack the fine control necessary for ensuring a coherent visual flow. To address this, we propose a diffusion-based model that effectively handles multi-modal conditional generation. Our model takes semantic segmentation, optional prompt guidance, and ordered viewing elements to generate high-fidelity GUIs that are aligned with the input design specifications.
% We demonstrate that our approach outperforms baseline methods in producing desirable GUIs and meets the desired visual flow. Moreover, a user study involving XX designers indicates that our model enhances the efficiency of the GUI design ideation process and provides designers with greater control compared to existing methods.    



% %%%%%%%%%%%%%%%%%%%%%%%%%%%%%%%%%%%%%%%%%%%%%%%%%%%%%%
% % Writing Clinic Comments:
% %%%%%%%%%%%%%%%%%%%%%%%%%%%%%%%%%%%%%%%%%%%%%%%%%%%%%%
% % Define: Effective UI design
% % Motivate GANs and write in full form.
% % LLMs vs ControlNet vs GANs
% % Say something about the Figma plugin?
% % Write the work is novel or what has been done before
% % What is desirable UI and how to evalutate that?
% % Visual Flow - main theme (center around it)
% % Re-Title: use word Flow!
% % Use ControlNet++ & SPADE for abstract.
% % Write about input/output. 
% % Why better than previous work?
% %%%%%%%%%%%%%%%%%%%%%%%%%%%%%%%%%%%%%%%%%%%%%%%%%%%%%

% % v2:
% % \noindent \textcolor{red}{\textbf{NEW Abstract!} (Post Writing Clinic 1 - 25-Jun)}

% % \noindent \textcolor{red}{----------------------------------------------------------------------}

% % \noindent Designing user interfaces (UIs) is a time-consuming process, particularly for novice designers. 
% % Creating UI designs that are effective in market funneling or any other designer defined goal requires a good understanding of the visual flow to guide users' attention to UI elements in the desired order. 
% % While current Large Language Models (LLMs) can generate UIs from just prompts, they often lack finer pixel-precise control and fail to consider visual flow. 
% % In this work, we present a UI synthesis method that incorporates visual flow alongside prompts and semantic layouts. 
% % Our efficient approach uses a carefully designed Generative Adversarial Network (GAN) optimized for scenarios with limited data, making it more suitable than diffusion-based and large vision-language models.
% % We demonstrate that our method produces more "desirable" UIs according to the well-known contrast, repetition, alignment, and proximity principles of design. 
% % We further validate our method through comprehensive automatic non-reference, human-preference aligned network scoring and subjective human evaluations.
% % Finally, an evaluation with xx non-expert designers using our contributed Figma plugin shows that <method-name> improves the time-efficiency as well as the overall quality of the UI design development cycle.

% % \noindent \textcolor{red}{----------------------------------------------------------------------}


% \noindent \textcolor{blue}{\textbf{NEW Abstract!} (Pre Writing Clinic 9-July)}

% \noindent \textcolor{blue}{----------------------------------------------------------------------}

% \noindent Exploring different graphical user interface (GUI) design ideas is time-consuming, particularly for novice designers. 
% Given the segmentation masks, design requirement as prompt, and/or preferred visual flow, we aim to facilitate creative exploration for GUI design and generate different UI designs for inspiration.
% While current Vision Language Models (VLMs) can generate GUIs from just prompts, they often lack control over visual concepts and flow that are difficult to convey through language during the generation process. 
% In this work, we present FlowGenUI, a semantic map-guided GUI synthesis method that optionally incorporates visual flow information based on the user's choice alongside language prompts. 
% We demonstrate that our model not only creates more realistic GUIs but also creates "predictable" (how users pay attention to and order of looking at GUI elements) GUIs.
% Our approach uses Stable Diffusion (SD), a large paired image-text pretrained diffusion model with a rich latent space that we steer toward realistic GUIs using a trainable copy of SD's encoder for every condition (segmentation masks, prompts, and visual flow). 
% We further provide a semantic typography feature to create custom text-fonts and styles while also alleviating SD's inherent limitations in drawing coherent, meaningful and correct aspect-ratio text. 
% Finally, a subjective evaluation study of XX non-expert and expert designers demonstrates the efficiency and fidelity of our method.


% This process encourages creativity and prevents designers from falling into habitual patterns.


% ------------------------------------------------------------------
% Joongi Why is it important to create realistic GUI?
% I do not see how the Visual Flow given on the left hand side is reflected in the results on the right hand side. 
% I’d avoid making unsubstantiated claims about designers (falling into habitual patterns).
% The UIs you generate do not “align with users’ attention patterns” but rather try to control it (that’s what visual flow means)
% ------------------------------------------------------------------
% Comments - Writing Clinic - 9th July:
% Improve title. More names: FlowGen
% Figure 1: Use an inference time hand-drawn mask
% Figure 1: Show both workflows. Add a designer --> Input.
% Figure 1: Make them more diverse
% ------------------------------------------------------------------
% Designing graphical user interfaces (GUIs) requires human creativity and time. Designers often fall into habitual patterns, which can limit the exploration of new ideas. 
% To address this, we introduce FlowGenUI, a method that facilitates creative exploration and generates diverse GUI designs for inspiration. By using segmentation masks, design requirements as prompts, and/or selected visual flows, our approach enhances control over the visual concepts and flows during the generation process, which current Vision Language Models (VLMs) often lack.
% FlowGenUI uses Stable Diffusion (SD), a largely pretrained text-to-image diffusion model, and guides it to create realistic GUIs. 
% We achieve this by using a trainable copy of SD's encoder for each condition (segmentation masks, prompts, and visual flow). 
% This method enables the creation of more realistic and predictable GUIs that align with users' attention patterns and their preferred order of viewing elements.
% We also offer a semantic typography feature that creates custom text fonts and styles while addressing SD's limitations in generating coherent, meaningful, and correctly aspect-ratio text.
% Our approach's efficiency and fidelity are evaluated through a subjective user study involving XX designers. 
% The results demonstrate the effectiveness of FlowGenUI in generating high-quality GUI designs that meet user requirements and visual expectations.

% ---------------------------------------


%A critical and general issue remains while using such deep generative priors: creating coherent, meaningful and correct aspect-ratio text. 
%We tackle this issue within our framework and additionally provide a semantic typography feature to create custom text-fonts and styles. 


% %Creating UI designs that are effective in market funneling or any other designer-defined goal requires a good understanding of the visual flow to guide users' attention to UI elements in the desired order. 
% %While current largely pre-trained Vision Language Models (VLMs) can generate GUIs from just prompts, they often lack finer or pixel-precise control which can be crucial for many easy-to-understand visual concepts but difficult to convey through language. 
% % However, obtaining such pixe-level labels is an extremely expensive so we
% % For example - overlaying text on images with certain aspect ratios and two equally separated buttons 
% Additionally, all prior GUI generation work fails to consider visual flow information during the generation process. 
% We demonstrate that visual flow-informed generation not only creates more realistic and human-friendly GUIs but also creates "predictable" (how users pay attention to and order of looking at GUI elements) UIs that could be beneficial for designers for tasks like creating effective market funnels.
% In this work, we present a semantic map-guided GUI synthesis method that optionally incorporates visual flow information based on the user's choice alongside language prompts. 
% Our approach uses Stable Diffusion, a large (billions) paired image-text pretrained diffusion model with a rich latent space that we steer toward realistic GUIs using an ensemble of ControlNets. 
% % TODO: Mention it in 1 sentence:
% A critical and general issue remains while using such deep generative priors: creating coherent, meaningful and correct aspect-ratio text. 
% We tackle this issue within our framework and additionally provide a semantic typography feature to create custom text-fonts and styles. 
% To evaluate our method, we demonstrate that our method produces more "desirable" UIs according to the well-known contrast, repetition, alignment, and proximity principles of design. 
% % We further validate our method through comprehensive automatic non-reference and human-preference aligned scores. (TODO: Maybe Unskip if we get UIClip from Jason!)
% % TODO: Re-word this and only keep ideation cycles and time-efficiency.
% Finally, a subjective evaluation study of XX non-expert and expert designers demonstrates the efficiency and fidelity of our method.
% % improves the time-efficiency by quick iterations of the UI design ideation process.
% %Finally, an evaluation with xx non-expert designers using our contributed <method-name> improves the time-efficiency by quick iterations of the UI design ideation cycle.

%\noindent \textcolor{blue}{----------------------------------------------------------------------}


%In an evaluation with xx designers, we found that GenerativeLayout: 1) enhances designers' exploration by expanding the coverage of the design space, 2) reduces the time required for exploration, and 3) maintains a perceived level of control similar to that of manual exploration.



% Present-day graphical user interfaces (GUIs) exhibit diverse arrangements of text, graphics, and interactive elements such as buttons and menus, but representations of GUIs have not kept up. They do not encapsulate both semantic and visuo-spatial relationships among elements. %\color{red} 
% To seize machine learning's potential for GUIs more efficiently, \papername~ exploits graph neural networks to capture individual elements' properties and their semantic—visuo-spatial constraints in a layout. The learned representation demonstrated its effectiveness in multiple tasks, especially generating designs in a challenging GUI autocompletion task, which involved predicting the positions of remaining unplaced elements in a partially completed GUI. The new model's suggestions showed alignment and visual appeal superior to the baseline method and received higher subjective ratings for preference. 
% Furthermore, we demonstrate the practical benefits and efficiency advantages designers perceive when utilizing our model as an autocompletion plug-in.


% Overall pipeline: Maybe drop semantic typography / visual flow?
\end{abstract}

\newcommand{\loss}{\mathcal{L}}
\newcommand{\image}{\mathcal{I}}
\newcommand{\encoder}{\mathcal{E}}
\newcommand{\decoder}{\mathcal{D}}
\newcommand{\normal}{\mathcal{N}}

% \begin{figure}[b]
% \noindent\fbox{%
% \parbox{\dimexpr\linewidth-2\fboxsep-2\fboxrule\relax}{%
% %\begin{tabular}{l}
% %\textcolor{blue}{Short paper}: The word count of this paper is \textcolor{blue}{4534}.
% %\end{tabular}
% }}
% \end{figure}

%%
%% The code below is generated by the tool at http://dl.acm.org/ccs.cfm.
%% Please copy and paste the code instead of the example below.

%%%%%%%%%%%%%%%%%%%%%%%%%%%%%%%%%%%%%%%%%%%%%%%%%%%%%%%%%%%%%%%%%%%%%%%%%%%%%%%%%%%
%%%%%%%%%%%%%%%%% TODO for CHI 25: %%%%%%%%%%%%%%%%%%%%%%%%%%%%%%%%%%%%%%%%%%%%%%%%
%%%%%%%%%%%%%%%%%%%%%%%%%%%%%%%%%%%%%%%%%%%%%%%%%%%%%%%%%%%%%%%%%%%%%%%%%%%%%%%%%%%
% \begin{CCSXML}
% <ccs2012>
%    <concept>
%        <concept_id>10003120.10003121.10003129</concept_id>
%        <concept_desc>Human-centered computing~Interactive systems and tools</concept_desc>
%        <concept_significance>500</concept_significance>
%        </concept>
%    <concept>
%        <concept_id>10003120.10003121.10003128</concept_id>
%        <concept_desc>Human-centered computing~Interaction techniques</concept_desc>
%        <concept_significance>500</concept_significance>
%        </concept>
%  </ccs2012>
% \end{CCSXML}

% \ccsdesc[500]{Human-centered computing~Interactive systems and tools}
% \ccsdesc[500]{Human-centered computing~Interaction techniques}
%%%%%%%%%%%%%%%%%%%%%%%%%%%%%%%%%%%%%%%%%%%%%%%%%%%%%%%%%%%%%%%%%%%%%%%%%%%%%%%%%%%
%%%%%%%%%%%%%%%%%%%%%%%%%%%%%%%%%%%%%%%%%%%%%%%%%%%%%%%%%%%%%%%%%%%%%%%%%%%%%%%%%%%


%% Keywords. The author(s) should pick words that accurately describe
%% the work being presented. Separate the keywords with commas.
\keywords{User Interface, User Interface Design, GUI Design Exploration, Visual Flow, Diffusion Model, Multi-modal Synthesis, Large Language Models, Vision Language Models}

%% A "teaser" image appears between the author and affiliation
%% information and the body of the document, and typically spans the
%% page. 

\begin{teaserfigure}
  \def\w{\linewidth}
  \centering
  \includegraphics[width=\w]{images/figures_teaser.pdf}
  \caption{
  We present a diffusion-based approach that allows designers to explore large numbers of GUI designs with low-effort inputs. 
Specifically, our model allows designers to input any combination of A) textual prompts, B) wireframes, and C) visual flow direction controls. The model rapidly explores a high number of low-fidelity GUI design alternatives. Here, we present samples generated from different input combinations. 
  }
  \Description{}
  \label{fig:teaser}
\end{teaserfigure}



% \begin{teaserfigure}
%   \def\w{\linewidth}
%   \centering
%   \includegraphics[width=\w]{images/prelim_results_fig1.png}
%   \caption{
%   Given the segmentation masks, design requirement as prompt, and/or preferred visual flow, we facilitate creative exploration by generating different GUI designs for inspiration.
%   % Our designs GUIs based on user attention patterns and preferred element viewing order. 
%   The visual flow input on the left (e1, e2, e3, e4, e5) alongside the segmentation map with a style and content prompt produces realistic and diverse GUI designs.
%   }
%   \Description{}
%   \label{fig:teaser}
% \end{teaserfigure}

%%
%% This command processes the author and affiliation and title
%% information and builds the first part of the formatted document.
\maketitle


\section{Introduction}

The early stages of a GUI design project are critical for its eventual success. 
%
During the early stages, designers produce low-fidelity solutions in an attempt to explore the design space.
Low-fidelity designs often specify, partially or in an underspecified way, key aspects of the layout, widgets, graphics etc.~\cite{landay1996silk}.
Designers produce them to envision how users are supported in their tasks and to learn about the trade-offs of different approaches~\cite{dow2011prototyping, tohidi2006getting}.
% 
The decisions that are made at this stage are critical, because they may be hard to change later on.

To assist designers in this stage, 
computational methods should help them produce large numbers of diverse solutions to explore.
Research suggests that producing large numbers of low-fidelity ideas is beneficial for creativity at this stage~\cite{boyarski1994computers, landay1996silk, rettig1994prototyping}.
However, design fixation makes it hard for designers to generate entirely novel ideas~\cite{jansson1991design}.
However, existing design tools often fail to support exploration.
They require designers to engage at a higher level of detail than necessary, 
for example to specify which specific widgets are used and where they are exactly positioned on the frame.
\emph{To support creative exploration, designers need tools that allow them to rapidly explore a high number of solutions, flexibly and with minimum effort.}

Recent advances in generative AI have created optimism that this gap could be closed. 
While previous work focused on retrieval of GUIs based on a query~\cite{herring2009getting, kumar2013webzeitgeist, li2021screen2vec}, this approach was limited to the samples available in the database.
Generative AI is a way to circumvent this requirement by generating solutions to a given prompt.
%
However, while there are successes for natural images, 
generative AI-based tools are much less adopted in GUI design~\cite{controlnet, controlnet_plus_plus}.
We believe that the root cause is that text-based prompting is not a natural means for a designer to express ideas that are inherently visuo-spatial by nature. 
%
Designers need to explore options by sketching: by loosely specifying where some key elements are located~\cite{landay1996silk}. 
Designers also need to explore the functional and communicative aspects of a design but without being asked to specify lots of detail manually.

Designers should also be able to express what kinds of impact they want their design to have on users.
%
An important impact concerns \emph{visual flow}:
the order in which users look at a UI.
Improving visual flow can enhance user engagement and guide user behavior~\cite{Rosenholtz11, Still10, ueyes, jiang2024ueyes, wang2024visrecall++, emami2024impact}. 
Thus, designers also care about how GUIs guide users' attention to task-relevant items~\cite{Rosenholtz11, Still10, ueyes}. 
%
However, understanding how users engage with a design normally requires empirical studies. 
%
To our best knowledge, no prior work has considered user-centered visual flow during the GUI design process.
%Presently no computational method we know of considers this type of user-centered input.

In this short paper, we propose a diffusion-based approach to controllable generation of early-stage GUIs. 
As shown in~\autoref{fig:teaser}, our model breaks new ground by allowing flexible control of the generation process via three types of early-stage inputs: A) prompts, B) wireframes, and C) visual flows (controlling where users should look). To our knowledge, no previous work has utilized scanpaths for GUI generation control, and no existing datasets are available to train models for this purpose.
%
While previous work on generative approaches in this space has produced layouts only~\cite{cheng2023play, cheng2024colay}, our approach generates complete low-fidelity prototypes with graphics. Although the outputs are images only, and not functional GUIs, they allow designers to envision what the GUIs might appear like. 
%
Importantly, our approach allows a significant relaxation in the input side. 
The designer can provide \emph{any} combination of the three as input and will get a diverse set of low-fidelity solutions in a gallery in response. 
In practice, hundreds of designs can be explored in the order of minutes.
%
The benefit is that large design spaces can be explored with very little effort in input-specification. 

Inspired by ControlNet~\cite{controlnet}, which uses an adapter (a modular mechanism designed to provide additional guidance during the image generation process) in diffusion models to enable controllable generation, our key insight is to incorporate a diffusion model with two different adapters.
We apply different adapters respectively for the controls of local properties, such as the positioning and type of GUI elements, and global properties, such as the overall visual flow direction. 
Diffusion models are generative models that have demonstrated a certain level of controllability in image generation~\cite{controlnet, controlnet_plus_plus}.
%
Our technical contribution is a solution that overcomes two critical shortcomings in controlling them. 
First, generating controllable GUIs poses a challenge, as previous approaches that rely heavily on text-based prompts often fail to capture certain GUI characteristics effectively, which are better conveyed through visual cues.
Second, many current GUI generation methods focus on a narrow range of properties~\cite{cheng2023play, cheng2024colay}, limiting their ability to manage both local properties (e.g., the position and type of GUI elements) and global properties (e.g., visual flow).


\noindent In summary, this short paper makes two technical contributions:
\begin{enumerate}
\item We introduce a novel diffusion-based model for GUI generation that integrates design control through prompt, wireframe, and visual flow direction. It is the first model to consider visual flow in the GUI generation process.
\item We created a dataset of mixed mobile UIs and webpages, including GUI screenshots, wireframes with GUI element labels, descriptions, and scanpaths, which can be used to train generative AI models in this space.
\end{enumerate}

To evaluate our approach, we first qualitatively illustrate that our model can rapidly explore a wide range of GUI alternatives that closely align with the specified input conditions. Secondly, we demonstrate that our model aligns more accurately with input specifications than previous generative models. We will make our code and the dataset used for training available upon acceptance.



\section{Related Work}

This section highlights the challenges faced by GUI generation models, an overview of diffusion models, and the limitations of existing GUI datasets,.


\begin{table}[t]
    \centering
    \begin{tabular}{lm{2cm}m{2cm}m{1.8cm}m{1.5cm}m{1.87cm}m{1.7cm}}
    \toprule
         \textbf{Model} & \textbf{Application}  & \textbf{Inputs} & \textbf{Flexible Multimodal Input}  &  \textbf{Generative Model}  &  \textbf{No Manual Detailed Specification Required} &  \textbf{Complete GUI Presentation} \\
    \midrule
         Webzeitgeist~\cite{kumar2013webzeitgeist} & GUI Retrieval & Wireframe & \xmark  & \xmark  & \cmark & \cmark  \\
         Scout~\cite{swearngin2020scout}  & Constraint-based Optimization & GUI Elements, Constraints & \xmark & \xmark  & \xmark & \cmark\\
          UICoder~\cite{wu2024uicoder} & GUI Code Generation  & Prompt & \xmark  & \cmark & \cmark & \cmark\\
          PLay~\cite{cheng2023play} & Layout Generation  & Gridlines & \xmark  & \cmark  & \cmark & \xmark\\
          CoLay~\cite{cheng2024colay} & Layout Generation & Prompt, Gridlines, Element Type Count  & \cmark & \cmark & \cmark & \xmark\ \\
         \midrule
         \bf Ours  & GUI Exploration & Prompt, Wireframe, Visual Flow & \cmark & \cmark & \cmark & \cmark \\        
    \bottomrule
    \end{tabular}
    \caption{Comparison of GUI generation models: Our model advances the field by allowing flexible inputs and diverse GUI outputs without requiring manual detailed specifications, such as defining constraints.
}
    \label{tab:table_model}
\end{table}



\subsection{GUI Generation}

Exploring GUI alternatives plays an important role in GUI design.   By comparing GUI alternatives explicitly, designers can offer stronger critiques and make more informed decisions~\cite{dow2011prototyping, tohidi2006getting, jiang2022computational, jiang2024computational2, jiang2023future, jiang2024computational}. \autoref{tab:table_model} shows a comparison of representative GUI generation models. 


Generative models should support flexible input to enable effective experimentation. Although design literature often advocates for the use of rough sketches to explore design ideas~\cite{boyarski1994computers, landay1996silk}, sketching alone can be limiting. Designers may face challenges such as fixation, which can constrain the generation of novel solutions~\cite{jansson1991design}. Previous constraint systems have facilitated the exploration of alternatives through the application of constraints~\cite{swearngin2020scout, jiang2019orclayout, jiang2020orcsolver, jiang2020reverseorc, jiang2024flexdoc}. However, these systems typically require manual detailed specifications early in the design process, which can be cumbersome and time-consuming.


It is important to generate diverse realistic GUI ideas aligning with designers' intent. 
GUI retrieval techniques have been employed to obtain exemplar GUIs~\cite{herring2009getting, kumar2013webzeitgeist, li2021screen2vec}. However, the retrieved designs are often limited and may not meet the specific needs of the current design task. Recent advancements in generative models have broadened the scope of exploring design alternatives. However, they still have certain limitations. For instance, some approaches focus on generating GUI layouts but do not present complete GUI representations~\cite{cheng2023play, cheng2024colay}. These methods can make it difficult for designers to visualize the final GUI based solely on layout generation. Other methods have investigated underconstrained scenarios, such as using text prompts to generate GUI code~\cite{wu2024uicoder}. However, text-based prompts often fall short in capturing certain GUI characteristics effectively, which are better conveyed through visual means, and frequently result in simplistic GUIs with minimal elements. Notably, all these models, except CoLay~\cite{cheng2024colay}, require fixed inputs and lack support for flexible multimodal input. Furthermore, none of these approaches consider user attention, an important aspect of effective GUI design.

To address these shortcomings, we propose a diffusion-based generative model for controllable GUI exploration that eliminates the need for manual detailed specifications. Our model enables designers to use flexible multimodal input and generate a wide variety of outputs. It is the first model to integrate visual flow to control GUI generation, supporting any combination of A) prompt, B) wireframe, and C) visual flow.




\subsection{Diffusion Models}

In light of the limitations identified in current GUI generation approaches, recent advancements in diffusion models present a potentially promising way to support flexible inputs and diverse outputs for GUI design.
Recent advancements in diffusion models have dramatically improved image generation, as demonstrated by models like GLIDE~\cite{nichol2021glide}, DALL-E2~\cite{ramesh2022hierarchical}, Imagen~\cite{saharia2022photorealistic}, and Latent Diffusion Models~\cite{rombach2022high}. These diffusion-based approaches offer stable training and handle multi-modal conditional generation without the need for modality-specific objectives.
In particular, Stable Diffusion~\cite{rombach2022high}, a specific implementation of Latent Diffusion Models,  is designed to efficiently generate high-quality images using a modified version of diffusion techniques in the latent space. It expands the range of conditional inputs in addition to text prompts, incorporating conditions like bounding boxes.
Techniques such as classifier guidance~\cite{dhariwal2021diffusion} and classifier-free guidance~\cite{ho2022classifier} have further boosted generation quality, enabling the development of effective conditional diffusion models. 
In our work, we employ Stable Diffusion with classifier-free guidance to generate controllable conditions, as it enhances output quality without the need for external classifiers, aligning generated results with input specifications.

\paragraph{Adapters in Diffusion Models}  

To enhance controllability in the diffusion model, we use adapter techniques.
Adapters~\cite{adapters_in_nlp} are small neural networks or modules designed to modify, specialize, or extend pre-existing models without requiring retraining of the entire system. These methods have recently been used as modular mechanisms in diffusion models, offering control and guidance during the image generation process.
%
By attaching lightweight adapters to the diffusion model, specific domain knowledge can be integrated without the need for retraining the whole network. For instance, ControlNet~\cite{controlnet} employs the Side-Tuning~\cite{side_tune} approach at each block, incorporating zero-convolutions to add conditional information via a trainable copy of the Stable Diffusion encoder.
%
Similarly, IP-Adapter~\cite{ye2023ip} creates trainable copies of the cross-attention layers of Stable Diffusion and introduces condition-specific encoders to handle new conditions. T2I-Adapter~\cite{mou2024t2i} adopts a side-convolutional network for layer-by-layer output blending, following the same Side-Tuning principle.
%
In this work, we propose integrating different adapters to respectively control local properties, such as the positioning and type of GUI elements, and global properties, like the overall visual flow direction.







% DDPM~\cite{ddpm} and 2015 Sol Dickstein reference ...
% Minimize VLB,
% noise equation here

% Fast samplers: DDIM, PNDM, DPM-Solver
% Classifier free Guidance

% T2I models $\rightarrow$ GLIDE, Imagen, Make-a-Scene, Cogview, SD



%\subsection{Automatic GUI Synthesis}
%Image-to-image approaches ... GUIGAN (patch-mixing)
   
%Multimodal approaches: Galileo.AI, GPT-4V, ...


% Maybe this is background?

%\subsection{Vision Language Models}
%Llava-NeXT~\cite{llavanext} over Llava~\cite{llava}.    
%BLIP~\cite{blip} and BLIP-2~\cite{blip2} over GPT-4.

% % TODO:
% \subsection{Text-Diffusers}
% Imagen inspired by~\cite{DeepFloyd}, 
% Glyph-Byt5~\cite{glyph_byt5, glyph_byt5_v2}, 
% Text-diffuser~\cite{textdiffuser}, and
% Text-Diffuser-2~\cite{chen2023textdiffuser}



% Consequently, CfG simplifies the optimization process and enhances stability by avoiding the intricate balancing of multiple loss components, a common issue in GAN frameworks. 
% By deriving and applying these principles, we can propose a unified approach to achieve more reliable and accurate conditional generation.

% \noindent Sketch2Code attempts simplistic GUI generation through HTML code prediction from a hand-drawn sketch using deep learning (optical character recognition and feature recognition). 
% However, Sketch2Code relies entirely on the creativity and legibility of the designer, is restricted to only HTML UIs with potential bugs in code, and does not offer an avenue for quick iterations, which is imperative for designer creativity and GUI ideation. 
% Additionally, it requires an HTML renderer to view the GUI. 
% Other GUI code generation methods like Design2Code and Pix2Code require the GUI to exist already. 
% These methods are also prone to introducing incorrect code as they use largely pre-trained VLM backbones.  
% A shift from this code generation from a sparse modality to GUI paradigm was observed with the rapid rise of text-to-image (T2I) models. 
% Proprietary methods like Galileo.ai use a prompt to generate the entire GUI, offering little control over complex semantics that are easily described visually (for example with a semantic segmentation map as shown in Figure X; also change prompt based on Figure 1's map) as compared to language descriptions. 
% For example - Input prompt: "Overlay an advertisement with 3 buttons: close, go to an external website and share. Then place the goto external website button to the right of the share button, both on the bottom right. Leave approximately 5 pixels from both sides. The close button should be top-right, leaving no margins. All buttons should have appropriate texts overlayed on them. The advertisement should be rectangular on all devices and if I was designing a mobile UI, it should be a banner on the bottom of the screen. Again leave no margins." is extremely cumbersome to explain and requires the user to transfer a GUI from their imagination to language.
% Also, it is an impractical input from the model's perspective as well due to the limited size of the context window.
% Note that all the methods listed above just focus on GUI generation through different input modalities but incorporate no GUI design principles during the generation that eventually affect user behavior. 


% \noindent We use the large (2.3 billion+) pre-trained prior of Stable Diffusion and guide its latent space with two easy and intuitive, from the designer's perspective, input modalities: a segmentation map (drawing/coloring on a canvas with a pre-determined/mapped palette) and visual flow information (5 to 7 clicks/marks on the same canvas).   
% In addition, we address an open issue with current image generation pipelines: drawing coherent and legible text.
% GUIs are not complete without appropriately styled text elements splayed across. 
% This is essential for GUIs to look realistic. 
% We tackle this critical issue and incorporate automatic and coherent 'text drawing' within our generation pipeline using an iterative text-drawing and harmonization pipeline guided by a large Vision-Language Model (VLM).

% \noindent Training our method required quadruples of a) ground truth GUI, b) corresponding prompt/description of the styles, themes, etc. c) corresponding segmentation maps, and d) visual flow information.
% Note that only input modality c) is a strict requirement during inference, all others (b and d) are optional. 
% Obtaining the quadruple is challenging as segmentation is an open research problem, especially in the GUI domain with complex semantics requiring segmentation models to understand different UI elements, images as well as text/language. 
% % Say something about the GUI datasets as well.
% To that end, we manually cleaned X semantically annotated datasets and \textit{mixed} (cite midas) them for training a more general and improved segmentation method. 
% This contributed model allowed us to augment UEyes, a GUI dataset with ground truth visual saliency and scanpaths, with corresponding segmentation maps completing the training quadruple.

% \noindent To that end, we propose the first segmentation map-controlled (and optionally prompts for global styles, themes, etc.) diffusion-based GUI generation method which also incorporates a critical UI designing paradigm in the pipeline: \textit{visual flow}. 
% This UI design principle allows the designer to pre-empt user engagement through order-of-looking through the UI and eventually reach the UI's intended goal. 
% For example - creating a market funnel with high digital sales or high email subscription conversions from an initial pitch. 

% \noindent We use the large (2.3 billion+) pre-trained prior of Stable Diffusion and guide its latent space with two easy and intuitive, from the designer's perspective, input modalities: a segmentation map (drawing/coloring on a canvas with a pre-determined/mapped palette) and visual flow information (5 to 7 clicks/marks on the same canvas).   
% In addition, we address an open issue with current image generation pipelines: drawing coherent and legible text.
% GUIs are not complete without appropriately styled text elements splayed across. 
% This is essential for GUIs to look realistic. 
% We tackle this critical issue and incorporate automatic and coherent 'text drawing' within our generation pipeline using an iterative text-drawing and harmonization pipeline guided by a large Vision-Language Model (VLM).

% \noindent Training our method required quadruples of a) ground truth GUI, b) corresponding prompt/description of the styles, themes, etc. c) corresponding segmentation maps, and d) visual flow information.
% Note that only input modality c) is a strict requirement during inference, all others (b and d) are optional. 
% Obtaining the quadruple is challenging as segmentation is an open research problem, especially in the GUI domain with complex semantics requiring segmentation models to understand different UI elements, images as well as text/language. 
% % Say something about the GUI datasets as well.
% To that end, we manually cleaned X semantically annotated datasets and \textit{mixed} (cite midas) them for training a more general and improved segmentation method. 
% This contributed model allowed us to augment UEyes, a GUI dataset with ground truth visual saliency and scanpaths, with corresponding segmentation maps completing the training quadruple.

% \noindent Similar to DiffSeg, we investigate the attention maps in the residual-attention block-based encoder of our tuned Stable Diffusion (operating with only prompts as input, not input segmentation) and reduce the KL-divergence among them to obtain an unlabeled GUI segmentation map. 
% This makes our model a GUI \textit{generation and design} pipeline alongside a free segmentation method, named free-SeGUI. 
% We prepend \textit{free} since the generative pipeline was never optimized for the segmentation task and incurred no cost during training. \\    


% This section focuses on the limitations of preexisting representations of GUIs, the GUI-related applications of graph neural networks, and constraint-based approaches to layout generation.


% \subsection{Graph Neural Networks on GUIs}

% Graph neural networks~\cite{gori2005new, scarselli2009graph, garg20GNNs, xu2018how} are state-of-the-art models for encoding graph-structured data. Whereas CNNs rely on convolution over spatial neighborhoods and enjoy widespread application to encode GUI images, GNNs aggregate information from neighborhoods defined by an input graph that are not restricted to the spatial domain. This gives them the potential to exploit information about the GUIs beyond pixel level. 
% \add{Li et al. applied GNNs to denoise an existing user-interface dataset~\cite{li2022learning}, and performed GUI autocompletion from the GUI layout hierarchy but failed to generate visually realistic GUI results~\cite{li2020auto}. Br{\"u}ckner et al.~\cite{bruckner2022learning} looked into constructing a graph using GUI elements' relative positioning to predict elements; however, it proved challenging to learn the layout structure from only relative positions.
% HAMP~\cite{ang2022learning}, introduced a graph representation with nodes for app descriptions, GUI screens, GUI classes, and element images to perform GUI tasks. Still, such detailed metadata cannot be extracted from GUI screenshots without extensive manual annotations.
% In contrast, our application of GNNs is geared toward modeling the layout graph of GUI elements, thereby enabling us to capture both the topological intricacies of the GUI layout and the properties of individual GUI elements.}


% \subsection{Constraint-based Layout Generation}
 
% Constraint-based layout models have been widely used in GUI layouts~\cite{sahami2013insights, zeidler2017automatic, marcotte2011responsive, badros2001cassowary, bill1992bricklayer, borning1997solving, hosobe2000scalable, lutteroth2008domain, sadun2013ios, weber2010reduction, zeidler2017tiling, karsenty1993inferring, scoditti2009new, zeidler2012auckland, borning1968constraint, szekely1988user} and document layouts~\cite{hurst2003cobweb, hosobe2005solving, borning2000constraint, laine2021responsive}. \add{Early methods like Peridot~\cite{myers1986creating, myers1990creating} and Lapidary~\cite{zanden1991lapidary} proposed programming by demonstration, automatically generate constraints for user interfaces based on designer interactions.} These models offer greater flexibility for layout generation than simple layout models such as group, grid, table, and grid-bag layouts~\cite{myers2000past, myers1995user, myers97theamulet}. 
% Prior work proposed constraint-based layout generation~\cite{weld2003automatically, fogarty2003gadget}.
% For instance, SUPPLE~\cite{Gajos2008decision, gajos2004supple, gajos2008improving} presented constraints for alternative widgets and groupings, and
% ORCLayout~\cite{jiang2019orclayout, jiang2020orcsolver, jiang2020reverseorc} introduced OR-constraint as a mixture of hard and soft constraints to unify flow-based and constraint-based layouts. 
% \add{Constraints have functioned also to enable layout personalization~\cite{Gajos2005preference}, maintaining consistency~\cite{gajos2005cross}, giving layout-alternative suggestions based on user-defined constraints~\cite{swearngin2020scout, bielik2018robust}, generating layout alternatives from templates or modifiable suggestions~\cite{Jacobs2003document, sinha2015responsive, zanden1990automatic}, and allowing both author and viewer to specify the layout~\cite{borning2000constraint}.
% Finally, recent work has explored applying deep-learning approaches to automatic layout generation, eliminating the need for manually defined constraints~\cite{zheng2019content, lee2019neural}.}
% However, none of these methods predict constraints for GUI elements. 
% Incorporating GUI element relationships as constraints enables our model to predict them within the network. This enhances the network's ability to establish connections and deepen its comprehension of both element properties and constraints.



%Much of the design literature encourages the use of rough wireframes to explore design ideas~\cite{boyarski1994computers, landay1996silk, todi2016sketchplore}. While sketching aids in generating rough concepts, their ability to envision novel solutions is often constrained and can be challenging for designers to avoid fixation and think of entirely new ideas~\cite{jansson1991design}. GUI retrieval techniques assist in obtaining exemplar GUIs~\cite{herring2009getting, kumar2013webzeitgeist, li2021screen2vec}. Designers frequently explore alternatives by looking for examples~\cite{herring2009getting, lee2010designing, chang2012webcrystal, hartmann2007programming, kumar2011bricolage, swearngin2018rewire}. However, \emph{the retrieved designs from the database are limited} and may not always align with the specific goals of the current GUI design. 

%Constraint-based layout models have also been extensively used for GUI generation~\cite{sahami2013insights, zeidler2017automatic, marcotte2011responsive, bill1992bricklayer, hosobe2000scalable, lutteroth2008domain, sadun2013ios, weber2010reduction, zeidler2017tiling, scoditti2009new, zeidler2012auckland, borning1968constraint, hurst2003cobweb, borning2000constraint}. Early systems like Peridot~\cite{myers1986creating, myers1990creating} and Lapidary~\cite{zanden1991lapidary} introduced programming by demonstration, generating constraints for user interfaces based on designer interactions. These models provide greater flexibility for layout generation than simpler models like grid or table layouts~\cite{myers2000past, myers1995user, myers97theamulet}.
%SUPPLE~\cite{Gajos2008decision, gajos2004supple, gajos2008improving} introduced constraints for alternative widgets and groupings, while ORCLayout~\cite{jiang2019orclayout, jiang2020orcsolver, jiang2020reverseorc} unified flow-based and constraint-based layouts by combining hard and soft constraints. However, constraint-based methods often generate only one sample at a time, limiting their utility for exploring multiple GUI alternatives. Scout~\cite{swearngin2020scout}, by contrast, allows designers to explore a variety of alternatives through constraints. Still, such systems often \emph{require detailed specifications} early in the design process, which can be tedious and hinder creativity.


%Recent advances in generative models have significantly expanded methods for exploring design alternatives. Previous work has primarily focused on \emph{generating GUI layouts only without showing complete GUI representation}~\cite{cheng2023play, cheng2024colay}; however, these approaches often fail to provide an intuitive presentation of the final GUI. Designers may struggle to visualize complete GUIs solely from layout generation. Other approaches to GUI generation have explored underconstrained scenarios, such as using text prompts to generate GUI code~\cite{wu2024uicoder}. However, \emph{text-based prompts often fail to capture certain GUI characteristics effectively}, which are better conveyed through visual cues, and frequently produce simplistic GUIs with minimal elements. Moreover, none of these approaches incorporate visual flow, an important factor in effective GUI design. 
% \input{03-UI_Segmentation}


\begin{table}[t]
    \centering
    \begin{tabular}{lccccc}
    \toprule
         \textbf{Dataset} & \textbf{GUI Types} & \textbf{Number of GUIs} & \textbf{Detections} & \textbf{Descriptions} & \textbf{Scanpaths} \\
    \midrule
         MUD~\cite{mud} & Mobile (Android) & $\sim$18,000 & \cmark & \xmark & \xmark \\  
         RICO~\cite{rico} & Mobile (Android) & $\sim$72,000 & \cmark & \xmark & \xmark \\
         RICO-Semantic~\cite{rico_semantic} & Mobile (Android) & $\sim$72,000 & \cmark & \xmark & \xmark \\
         Clay~\cite{li2022learning} & Mobile (Android) & $\sim$60,000 & \cmark & \xmark & \xmark \\
          ENRICO~\cite{enrico} & Mobile (Android) & 1,460 & \cmark & \xmark & \xmark \\ 
          VINS~\cite{vins} & Mobile (Android + IOS) & 4,543 & \cmark & \xmark & \xmark \\ 
          AMP~\cite{zhang2021screen} & Mobile  (IOS) (not publicly available) & $\sim$77,000 & \cmark & \xmark & \xmark \\ 
          Webzeitgeist~\cite{kumar2013webzeitgeist} & Webpages & 103,744 & \cmark & \xmark & \xmark \\
         WebUI~\cite{webui} & Webpages & $\sim$350,000 & \cmark & \xmark & \xmark \\  
         UEyes~\cite{ueyes} & Mobile, Webpages, Poster, Desktop & 1,980 & \xmark & \xmark & \cmark \\
         \midrule
         \bf Ours & Webpages, Mobile (Android + IOS) & $\sim$72,500 & \cmark & \cmark & \textcolor{munsell}{Predicted} \\        
    \bottomrule
    \end{tabular}
    \caption{Comparison of GUI datasets: Our dataset uniquely combines mobile and webpage GUIs with comprehensive annotations and predicted scanpaths.
}
    \label{tab:datasets_related}
\end{table}

\section{Dataset}

\paragraph{Prior Datasets}
No existing datasets can be directly used to train a model conditioned on prompts, wireframes, and visual flow directions. 
Several datasets have been collected to support GUI tasks, as shown in \autoref{tab:datasets_related}. The AMP dataset, comprising 77,000 high-quality screens from 4,068 iOS apps with human annotations~\cite{zhang2021screen}, is not publicly available. On the other hand, the largest publicly available dataset, Rico~\cite{rico}, includes 72,000 app screens from 9,700 Android apps and has been a primary resource for GUI understanding despite its inherent noise. To address its limitation, the Clay dataset~\cite{li2022learning} was created by denoising Rico using a pipeline of automated machine learning models and human annotators to provide more accurate element labels. Enrico~\cite{enrico} further cleaned and annotated Rico but ultimately contains only a small set of high-quality GUIs. MUD~\cite{mud} offers a dataset featuring modern-style Android GUIs. The VINS dataset~\cite{vins} focuses on GUI element detection and was created by manually capturing screenshots from various sources, including both Android and iOS GUIs. Additionally, Webzeitgeist~\cite{kumar2013webzeitgeist} used automated crawling to mine design data from 103,744 webpages, associating web elements with properties such as HTML tags, size, font, and color. Similarly, WebUI~\cite{webui} provides a large-scale collection of website data. None of these datasets include both mobile GUIs and webpages and have included visual flow information in the datasets.
UEyes~\cite{ueyes} is the first mixed GUI-type eye tracking dataset with ground-truth scanpaths, although it lacks element labels.

%In our work, we create a comprehensive high-quality dataset that includes both mobile GUIs (Android and iOS) and webpages by cleaning and combining GUIs from Enrico~\cite{enrico}, VINS~\cite{vins}, and WebUI~\cite{webui}. For each GUI, we further generate segmentation maps with GUI element labels, apply LLaVA-Next~\cite{liu2024llavanext} to generate description, and use EyeFormer~\cite{eyeformer}, a state-of-the-art scanpath prediction model, to generate scanpaths.
To address these limitations, we construct a large-scale high-quality dataset of mixed mobile UIs and webpages, including about 72,500 GUI screenshots, along with their wireframes with labeled GUI elements, descriptions, and scanpaths. This dataset is designed to support the training of generative AI models, filling a gap in existing public datasets by providing not only GUI images but also detailed descriptions, element labels, and visual interaction flows. 

\paragraph{GUI Screenshots} Our dataset integrates and cleans GUI data from Enrico~\cite{enrico}, VINS~\cite{vins}, and WebUI~\cite{webui}. The Enrico dataset contains 1,460 Android mobile GUIs, while VINS includes 4,543 Android and iOS GUIs. WebUI is a large-scale webpage dataset consisting of approximately 350,000 GUI screenshots with corresponding HTML code. For WebUI, the original dataset includes screenshots for different resolutions, leading to many similar screenshots for each webpage.
We retained only the 1920 x 1080 resolution screenshots to avoid redundant images from different resolutions. We further refined these three datasets by removing abstract, non-graphic wireframe GUIs, duplicates, and GUIs with fewer than three elements.  The final dataset consists of 66,796 webpages and 5,634 mobile GUIs.

\paragraph{Wireframes with GUI Element Labels} 

For mobile GUIs, we selected the Enrico and VINS datasets for their well-labeled GUI element bounding boxes. To further refine these annotations, we applied the UIED~\cite{uied} model, which detects and refines GUI element bounding boxes. We manually verified and corrected the results for accuracy. For the WebUI dataset, each element has multiple labels. We filtered the original element labels to keep only the most relevant label for each element. We standardized the labels across mobile and webpage elements, mapping them to nine common types: `Button', `Text', Image', `Icon', `Navigation Bar', `Input Field', `Toggle', `Checkbox', and `Scroll Element'. Using these refined bounding boxes and labels, we generated wireframes with GUI element labels.

\paragraph{Descriptions} For each GUI, we employed the LLaVA-Next~\cite{liu2024llavanext} vision-language model to generate both concise and detailed descriptions.

\paragraph{Scanpaths} Finally, we used EyeFormer~\cite{eyeformer}, the state-of-the-art scanpath prediction model, to predict scanpaths for each GUI. While using real scanpaths recorded by eye trackers would provide more accurate data, this process is highly time-consuming. Alternative proxies, such as webcams or cursor movements, do not capture the same cognitive processes as actual eye movements, making them less suitable for our purposes.
\section{Method}
\label{sec:synthesis}




To enable the rapid exploration of diverse GUI options, we propose integrating a diffusion model with different modular adapters designed to control both local and global GUI properties. Specifically, we employ the ControlNet adapter to manage local properties (e.g., the position and type of GUI elements), as it encourages close alignment with the input wireframe. For global properties (e.g., visual flow), we propose utilizing a Flow Adapter, which provides a more global guiding signal for GUI generation.

\subsection{Problem Formulation}

We formulate the GUI exploration task as a controllable GUI generation problem, allowing designers to flexibly guide the process using three types of inputs: A) prompts, B) wireframes, and C) visual flows. For visual flows, we currently support two options for designers: they can either 1) provide a sample GUI to encourage the model to generate GUIs with similar visual flow, or 2) specify a flow direction, indicating where the flow should conclude (bottom left or bottom right).
Designers can provide any combination of these inputs at any level of detail and receive a diverse gallery of low-fidelity solutions. 

We apply classifier-free guidance (CFG)~\cite{cfg}, a technique used in generative models, to balance fidelity to conditioning inputs and output diversity by mixing conditioned and unconditioned model outputs.
For visual flow, we need to encourage the generated visual flow to align with the input visual flow.
Thus, we introduce two objective terms: the classifier-free guidance loss ($\mathcal{L}_{\textrm{cfg}}$) and the flow consistency loss ($\mathcal{L}_{\textrm{flow}}$). The classifier-free guidance loss ensures the generated GUIs align with the provided inputs, while the flow consistency loss encourages the consistency between the desired visual flow and the visual flow of the generated GUIs. Thus, the objective function is 

\begin{equation}
\label{eq:objective_function}
\begin{split}
\mathcal{L}(\hat{z}, c_\mathrm{w}, c_\mathrm{p}, c_\mathrm{f}) = \mathcal{L}_{\mathrm{cfg}}(\hat{z}, c_\mathrm{w}, c_\mathrm{p}, c_\mathrm{v}) +  ~\mathcal{L}_{\mathrm{flow}}(\hat{z}, c_\mathrm{f}),
\end{split}
\end{equation}

where $\hat{z}$ is the generated GUI, $c_\mathrm{w}$, $c_\mathrm{p}$, and $c_\mathrm{f}$ represent the input conditions for the wireframe, prompt, and visual flow, respectively.



\begin{figure}
    \centering
    \includegraphics[width=\linewidth]{images/model.png}
    \caption{
Our diffusion-based model generates diverse low-fidelity GUIs by conditioning on both local and global properties. The model integrates Stable Diffusion~\cite{rombach2022high} with specialized adapters: the ControlNet adapter~\cite{controlnet} manages local properties such as element positioning and types specified on wireframes, while the Flow Adapter directs the overall visual flow. Given inputs like wireframes, prompts, and visual flow patterns, the model effectively produces varied GUI designs.
    }
    \label{fig:model}
\end{figure}

\subsection{Controllable GUI Generation}

Our diffusion-based model generates GUIs conditioned on both local (e.g., position and type of elements) and global properties (e.g., visual flow). As illustrated in \autoref{fig:model}, the backbone of our model is based on Stable Diffusion~\cite{rombach2022high}, which employs a U-Net architecture~\cite{ronneberger2015u} consisting of an encoder $E$, a middle block $M$, and a skip-connected decoder $D$. The text prompt is encoded by a CLIP~\cite{clip} text encoder and feeds into the diffusion model via cross-attention layers.


%Our solution is to incorporate different adapters (modular mechanisms designed to provide additional control and guidance during the image generation process) for the controls of local properties, such as the positioning and type of GUI elements defined by wireframes, and global properties, such as the overall visual flow direction. 

\subsubsection{ControlNet Adapter for Local Properties}


Recent advancements in controllable image generation have shown that additional networks can be integrated into existing text-to-image diffusion models to better guide the generation process~\cite{controlnet, controlnet_plus_plus, mou2024t2i}. Inspired by ControlNet~\cite{controlnet}, we create a trainable copy of Stable Diffusion's encoder and middle block, followed by a decoder with zero-convolution layers. The weights and biases of these zero-convolution layers are initialized to zero, allowing the adapter to efficiently capture local properties, ensuring the generated GUI aligns with input wireframes. In this framework, wireframe features are concatenated with text features to guide the generation process.


\subsubsection{Flow Adapter for Global Properties}

The cross-attention mechanism has shown effective results for enhancing models' global control without explicit spatial guidance~\cite{ye2023ip, zhao2024uni}. Therefore, we adopt a cross-attention mechanism to process visual flow features, adding an additional cross-attention layer in each layer of the diffusion model for this purpose.


\paragraph{Flow Encoder}

No existing encoders are specifically designed to handle visual flow. However, EyeFormer~\cite{eyeformer}, a state-of-the-art model for scanpath prediction on GUIs, encodes GUI images and decodes the latent representations into scanpaths. We repurpose EyeFormer’s decoder to train our flow encoder, using GUI scanpaths as input during training. During inference, we offer two options for designers:
1) Sample-based flow generation: If the designer provides a sample GUI, we encoding the input GUI via EyeFormer to guide the model;
2) Specified flow direction: If the designer specifies a target flow direction (e.g., bottom-left or bottom-right), we select a scanpath from our dataset that matches the desired visual flow to guide the model.


\subsubsection{Training Process}


We train our model using classifier-free guidance~\cite{cfg} and DDIM~\cite{song2020denoising} because these methods offer robust control over the generation process while ensuring high fidelity in the generated GUIs. Stable Diffusion operates by progressively adding noise to data and learning to reverse this process. Classifier-free guidance allows us to balance the diversity of the generated GUIs and alignment with input conditions without needing a separate classifier, reducing complexity and enhancing flexibility in our model. Similarly, DDIM provides a deterministic way to denoise and refine outputs, producing results that are closely aligned with the desired GUI specifications. Together, these techniques help maintain the quality of generated GUIs while ensuring they meet input requirements. 

During training, the Stable Diffusion components remain frozen, while our adapters are trained to optimize the objective.
For a given timestep $t$ and input conditions $C = \{c_\mathrm{w}$ (wireframe), $c_\mathrm{p}$ (prompt), and $c_\mathrm{f}$ (visual flow)$\}$, the model learns to predict the noise $\epsilon_{\theta}$ added to the noisy image $z_t$ using the loss function:

\begin{equation}
    \mathcal{L}_{\textrm{cfg}}(\hat{z}, C) = \mathbb{E}_{\hat{z}, t, C, \epsilon \sim \mathcal{N}(0,1)} [\| \epsilon - \epsilon_{\theta}(z_t, t, \mathcal{C}) \|^{2}_2],
\end{equation}

\noindent where $\hat{z} = z_0$ is the final GUI predicted by denoising $z_t$ over timestep $t$. 


We train each adapter independently. Specifically, the ControlNet adapter is trained using both prompt and wireframe inputs, while the Flow Adapter is trained on prompt and visual flow inputs. During training, we apply a 50\% dropout rate to each input condition, which encourages the model to handle various combinations of inputs. No fine-tuning across all three types of inputs is necessary.

\paragraph{ControlNet} 
The ControlNet adapter is trained solely with the classifier-free guidance loss. Thus, the objective function is  
\begin{equation}
  \mathcal{L_\textrm{ControlNet}} =  \mathcal{L}_{\textrm{cfg}}(\hat{z}, c_w, c_p) = \mathbb{E}_{z_t, t, c_w, c_p \epsilon \sim \mathcal{N}(0,1)} [\| \epsilon - \epsilon_{\theta}(z_t, t, c_w, c_p) \|^{2}_2],
\end{equation}.


\paragraph{Flow Adapter} 
The Flow Adapter is trained using both classifier-free guidance loss and a flow consistency objective. The latter ensures that the cross-attention layers guide generation toward a latent subspace aligned with the desired visual flow.

To encourage the consistency between the visual flow of the generated GUIs and the input visual flow specification, we apply Dynamic Time Warping (DTW)~\cite{dtw}, a standard metric used to measure the similarity between two temporal sequences, even when they differ in length. DTW identifies the optimal match between the sequences and computes the distance between them. However, since the original DTW is not differentiable, it cannot be directly used to optimize deep learning models.
To address this, we employ softDTW~\cite{soft-dtw}, a differentiable version of DTW, to optimize our model.

Since the only output of the network is a synthesized GUI ($\hat{z}$) with no ground-truth scanpath, we use EyeFormer~\cite{eyeformer} to compute a representative ground truth ($\textrm{EyeFromer}(\hat{z}) \sim \hat{c}_v$). The loss is defined as:
\begin{equation}
    \mathcal{L}_{\textrm{flow}}(\hat{z}, c_f) = \textrm{softDTW}(\textrm{EyeFromer}(\hat{z}, c_f)))
\end{equation}

Thus, the total loss for the Flow Adapter is:
\begin{equation}
\begin{split}
  \mathcal{L_\textrm{FlowAdapter}} &=  \mathcal{L}_{\textrm{cfg}}(\hat{z}, c_p, c_v) + \mathcal{L}_{\textrm{flow}}(\hat{z}, c_f) \\
  &= \mathbb{E}_{z_t, t, c_p, c_v \epsilon \sim \mathcal{N}(0,1)} [\| \epsilon - \epsilon_{\theta}(z_t, t,c_p, c_v) \|^{2}_2] + \textrm{softDTW}(\textrm{EyeFromer}(\hat{z}, c_f))).
  \end{split}
\end{equation}.







%\input{05-Implementation-Details}
\section{Results}
\label{sec:experiments}

%
\begin{table}[t]
    \centering
    \begin{tabular}{lm{2cm}m{2cm}m{1.8cm}m{1.5cm}m{1.87cm}m{1.7cm}}
    \toprule
         \textbf{Model} & \textbf{Application}  & \textbf{Inputs} & \textbf{Flexible Multimodal Input}  &  \textbf{Generative Model}  &  \textbf{No Manual Detailed Specification Required} &  \textbf{Complete GUI Presentation} \\
    \midrule
         Webzeitgeist~\cite{kumar2013webzeitgeist} & GUI Retrieval & Wireframe & \xmark  & \xmark  & \cmark & \cmark  \\
         Scout~\cite{swearngin2020scout}  & Constraint-based Optimization & GUI Elements, Constraints & \xmark & \xmark  & \xmark & \cmark\\
          UICoder~\cite{wu2024uicoder} & GUI Code Generation  & Prompt & \xmark  & \cmark & \cmark & \cmark\\
          PLay~\cite{cheng2023play} & Layout Generation  & Gridlines & \xmark  & \cmark  & \cmark & \xmark\\
          CoLay~\cite{cheng2024colay} & Layout Generation & Prompt, Gridlines, Element Type Count  & \cmark & \cmark & \cmark & \xmark\ \\
         \midrule
         \bf Ours  & GUI Exploration & Prompt, Wireframe, Visual Flow & \cmark & \cmark & \cmark & \cmark \\        
    \bottomrule
    \end{tabular}
    \caption{Comparison of GUI generation models: Our model advances the field by allowing flexible inputs and diverse GUI outputs without requiring manual detailed specifications, such as defining constraints.
}
    \label{tab:table_model}
\end{table}



%\setlength{\tabcolsep}{0.9pt}
%\def\arraystretch{0.2}% 
\begin{figure*}[!]
 \def\w{\linewidth}
 \centering
  \includegraphics[width=\w]{images/figures_result3.pdf}
\caption{
Demonstrating our model's ability in generating diverse webpage and mobile GUI designs from a simple text prompt, without the need for wireframe or visual flow inputs. This showcases the model's capability to interpret abstract concepts and generate platform-specific designs.
}
\Description{}
\label{fig:result3}
\end{figure*}



%\setlength{\tabcolsep}{0.9pt}
%\def\arraystretch{0.2}% 
\begin{figure*}[!]
 \def\w{\linewidth}
 \centering
  \includegraphics[width=\w]{images/figures_result2.pdf}
\caption{
Our model effectively aligns GUI outputs with diverse \textbf{wireframe} inputs for the same prompt, demonstrating its ability to reliably translate wireframe structures into coherent GUI designs.
}
\Description{}
\label{fig:result2}
\end{figure*}



%\setlength{\tabcolsep}{0.9pt}
%\def\arraystretch{0.2}% 
\begin{figure*}[!]
 \def\w{\linewidth}
 \centering
  \includegraphics[width=\w]{images/figures_result6.pdf}
\caption{Our model can generate GUIs styles that appropriately align with the descriptions in the \textbf{text prompt}. However, if the wireframe is not well-suited to the specified GUI in the prompt, it may also produce samples that are less accurate or not fully aligned with the intended design.
}
\Description{}
\label{fig:result6}
\end{figure*}



%\setlength{\tabcolsep}{0.9pt}
%\def\arraystretch{0.2}% 
\begin{figure*}[!]
 \def\w{\linewidth}
 \centering
  \includegraphics[width=\w]{images/figures_result1.pdf}
\caption{
Given the wireframe input, we highlight that after adding a \textbf{``red-themed''} constraint into the prompt, our model can consistently generate red-theme webpages. This demonstrates its ability to adhere to color themes while maintaining design diversity and alignment with the given wireframe.
}
\Description{}
\label{fig:result1}
\end{figure*}



%\setlength{\tabcolsep}{0.9pt}
%\def\arraystretch{0.2}% 
\begin{figure*}[!]
 \def\w{\linewidth}
 \centering
  \includegraphics[width=\w]{images/figures_result4.pdf}
\caption{
Our model effectively adapts to specified \textbf{visual flow} directions, generating diverse layouts and graphics that align with user-defined prompts and wireframe inputs. This demonstrates its flexibility in accommodating visual flow constraints while maintaining coherence with the desired design.
}
\Description{}
\label{fig:result4}
\end{figure*}



%\setlength{\tabcolsep}{0.9pt}
%\def\arraystretch{0.2}% 
\begin{figure*}[!]
 \def\w{\linewidth}
 \centering
  \includegraphics[width=\w]{images/figures_result5.pdf}
\caption{Comparison to other models:
Our model outperforms others by closely aligning with input specifications, generating more realistic and coherent GUIs that follow both the wireframe and prompt.
}
\Description{}
\label{fig:result5}
\end{figure*}



In this section, we demonstrate that our model efficiently explores a diverse range of GUI alternatives that closely match the specified input conditions, outperforming other models in alignment and efficiency. More results are shown in the Supplementary Materials.


\subsubsection{Qualitative Results}


To evaluate our approach, we qualitatively demonstrate that our model is capable of generating diverse low-fidelity GUI samples from various input combinations. Specifically, we aim to assess: 
%
1) whether the model can produce diverse webpage and mobile GUI results solely based on the given prompt; 
%
2) whether it can accurately capture local properties given different wireframe inputs;
%
3) whether it can generate suitable GUIs styles that appropriately align with the descriptions in the text prompt;
%
4) whether it can adapt to modifications in the prompt while still following the wireframe specifications; 
%
and 5) whether it can capture global properties by aligning with the desired visual flow, both with and without wireframe inputs.

Thus, Our qualitative evaluation focuses on the aspects of diversity controllability, and adaptability.

\paragraph{Diversity} 
The model excels at generating diverse GUI designs. 
\autoref{fig:result3} demonstrates the model’s ability to generate a diverse range of webpage and mobile GUI designs based on the prompt, "a promotional site for cocktails." This highlights the model's versatility in producing varied designs from a simple text prompt, without relying on wireframe or visual flow inputs. It underscores the model’s capacity to interpret abstract concepts and deliver customized, platform-specific designs.

\paragraph{Controllability} 

The model reliably aligns its outputs with wireframe inputs and textual descriptions.
\autoref{fig:result2} demonstrates that GUIs produced from different wireframe inputs for the same prompt align well with their respective wireframes. This shows the model’s ability to reliably translate structural information from wireframes into coherent GUI designs.
%
\autoref{fig:result6} showcases the model’s capability to generate GUI styles that align with textual descriptions. However, if a wireframe is poorly suited to the specified GUI prompt, the model may produce results that are less accurate or deviate from the intended design.

\paragraph{Adaptability} 

The model shows its adaptability to varying design constraints. 
Some evidence of the model’s adaptability is shown in \autoref{fig:result1}, where, after adding a ``red-themed'' constraint to the prompt, the generated GUIs consistently exhibit a red theme while maintaining design diversity. This highlights the model's proficiency in adapting to new specifications while adhering to the provided wireframe structure.
%
In addition, \autoref{fig:result4} demonstrates the model’s ability to follow specified visual flow directions by generating varied layouts and graphic elements, regardless of the presence of wireframe inputs. This confirms the model’s flexibility in aligning with desired visual flows, reinforcing its capability to produce diverse and contextually relevant GUI designs.



\subsubsection{Comparison to Existing Models}

We provide a comparison of our model with existing state-of-the-art generative models in terms of adherence to input specifications and output quality.
In~\autoref{fig:result5}, we demonstrate that our model aligns more closely with input specifications compared to previous generative models. We compare our approach with related models, including ControlNet~\cite{controlnet}, and two commercial models, Galileo AI and GPT4o. Galileo AI is a commercial GUI generation model, while GPT4o is a state-of-the-art generative model. Since none of these models account for visual flow, our comparison is limited to prompt and wireframe inputs.
%
ControlNet excels in following wireframes but struggles to produce meaningful and realistic webpages. Conversely, Galileo AI focuses on generating realistic webpages but does not adhere to the input wireframe. Similarly, GPT4o fails to follow the wireframe and cannot generate realistic webpages. In contrast, our model effectively generates webpages that closely align with both the wireframe and prompt, producing realistic results.

\subsubsection{Efficiency}

Our model is efficient for both training and inference. Since the Stable Diffusion parameters are frozen and the adapters can be trained in parallel, the training process is efficient. Each adapter is lightweight, and the entire model can be trained in approximately 30 hours over 12 epochs on 256x256-resolution images using a single NVIDIA GeForce RTX4090 GPU. During inference, the model generates a batch of 16 GUI examples in around 15 seconds. GUIs within the same batch exhibit more similarity to each other than those across different batches.



% Importantly, our approach allows a significant relaxation in the input side. 
% The designer can provide \emph{any} combination of the three as input and will get a diverse set of low-fidelity solutions in a gallery in response. 
% In practice, hundreds of designs can be explored in the order of minutes.
% %
% The benefit is that large design spaces can be explored with very little effort in input-specification. 

% \begin{table*}[ht]
%   \caption{\textbf{Quantitative Evaluation - UI Synthesis Realism and Segmentation Map Adherence.} We compare the input semantic segmentation mask with the map obtained from the generated UI from our best-performing automatic segmentation method: BLIP-2 guided GroundingDINO-SAM. 
%   For the Mean-Opinion-Score (MOS) study $N$ assessors were asked to estimate the quality of synthesized speech on a nine-point Likert scale (1 - 5).}
%   % the lowest and the highest scores being 1 point (“Bad”) and 5 points (“Excellent”) with a step of 0.5 point.
%   % Replace N with a number.
  
%   \label{tab:quant_realism_and_segmask_adherence}
%   \begin{tabular}{lccccc}
%     \toprule
%     \multirow{2}{*}{\textbf{Methods}} & 
%     \multicolumn{3}{c}{\textbf {}}  & \multirow{2}{*}{\textbf {mIoU $\uparrow$}} & \multicolumn{1}{c}{\textbf {Human Evaluation}} \\
%     \cmidrule(lr){2-4} 
%     \cmidrule(lr){6-6}
%      & cFID $\downarrow$ & CLIP-cFID $\downarrow$ & cKID $\downarrow$ & & Likert-scale MOS $\uparrow$ \\      
%     \midrule
%     Galileo.AI & - & - & - & -  \\
%     GPT-4 & - & - & - & -  \\

%     \midrule
%   %  SPADE (Ours) & - & - & - & -\\   
%     % OASIS & - & - & - & -  \\   
%     % MakeAScene & - & - & - & -  \\
%     UniControlNet & - & - & - & -  \\  
%     MultiControlNet & - & - & - & - \\  
%     ControlNet & - & - & - & -  \\   
%     ControlNet++ & - & - & - & -  \\   
%     \midrule
%     \textbf{Ours} & - & - & - & -\\
%     \bottomrule
%   \end{tabular}
% \end{table*}







% \subsection{Ablation Study}



% \subsubsection{Inputs}
% \begin{table*}[ht]
%   \caption{\textbf{Ablation Study - The effect of Visual Saliency (VS) maps, Scan-Paths and Prompt guidance on the overall quality of UI generations.}}
%   \label{tab:ablation_inputs}
%   \begin{tabular}{lccccccc}
%   \toprule
%   \multirow{2}{*}{\textbf{Methods}} & \multicolumn{3}{c}{\textbf {Upgrades}} & \multicolumn{3}{c}{\textbf {Non-Reference Metrics}}  & \multirow{2}{*}{\textbf {mIoU $\uparrow$}} \\      
%   \cmidrule(lr){2-4} 
%   \cmidrule{5-7}
%   & Wireframe & Scan-Paths & Prompts & cFID $\downarrow$ & CLIP-FID $\downarrow$ & cKID $\downarrow$ & \\      
%   \midrule
%     % Base & \xmark & \xmark  & \xmark & - & -  & - & - \\ 
%     V1 & \cmark & \xmark  & \xmark  & - & -  & - & - \\  
%     V2  & \xmark & \cmark & \xmark & - & -  & - & - \\   
%     V3 & \xmark & \xmark & \cmark & - & -  & - & - \\   
%     V4 & \cmark & \cmark & \xmark & - & - & -  & -  \\   
%     V5 & \cmark & \xmark & \cmark & - & - & -  & -  \\   
%     V6 & \xmark & \cmark & \cmark & - & - & -  & -  \\   
%     \midrule
%     \textbf{Ours} & \cmark & \cmark & \cmark & - & - & - & - \\
%     \bottomrule
%   \end{tabular}
% \end{table*}


% 1/23
%%%%%%%%%%%%%%%%%%%%%%%%%%%%%%%%%%%
% TODO: 
%%%%%%%%%%%%%%%%%%%%%%%%%%%%%%%%%%%
% 1. Clean-FID, CLIP-FID & Clean-KID
% 2. Scanpath validity - User study and/or comparison with other scanpath prediction models.
% 3. MIoU: Wireframe adherence - Qualitative (overlay on output)
% 4. Ablation - w/o softDTW(EyeFormer(z, cf)) (Novelty in loss can be ablated and shown that it helps (hopefully!))
\section{Discussion and Conclusion} 



Our results suggest that diffusion-based models are suitable for the rapid and flexible exploration of low-fidelity GUI generation.
Currently, no available diffusion model is capable of producing GUIs with high-quality text and graphics, but this limitation does not impede their use for the exploration of early-stage design ideas.
There the focus is more on generating a broad range of 'good enough' possibilities.
Specifically, we demonstrated that the \emph{probabilistic} nature of diffusion models allows for generating a multitude of diverse GUI ideas efficiently. 

Before this paper, it was an open question of how to design generative models such that designers do not need to over-specify the input.
By extending conditional diffusion to handle both wireframes and visual flow, we offer a low-effort approach to designers that simplifies interaction with diffusion models. 
%
Moreover, the inclusion of specialized adapters for considering local (e.g., the position and type of GUI elements) and global (e.g., visual flow) design properties enhances control over GUI characteristics, allowing designers to focus on broad exploration with minimal effort. 
We view this model as a step forward in creating user-adapted GUI designs with generated models, as it integrates GUI properties with user-centered interactions, such as eye movement, throughout the GUI design exploration process.



%We present a novel diffusion-based model that advances early-stage GUI design by enabling more rapid and flexible exploration of GUI design alternatives. By integrating any combination of wireframes, textual prompts, and visual flow controls, our model allows designers to rapidly generate diverse low-fidelity GUIs, thereby facilitating a broader exploration of design possibilities with minimal effort. 


%Our approach effectively manages both local and global design properties through the use of different specialized adapters. This feature addresses the limitations of previous methods, which either struggled to capture essential GUI characteristics through visual cues or were constrained by a narrow focus on a limited range of properties~\cite{cheng2023play, cheng2024colay}. These earlier approaches often failed to manage both local and global properties effectively. By employing modular adapters designed to control local properties (e.g., positioning and type of GUI elements) and global properties (e.g., overall visual flow direction), our model enhances design control and guidance during the image generation process.


\paragraph{Limitations} 

We acknowledge the following limitations. First, we cannot generate valid texts or labels for GUIs; future work could focus on producing meaningful text with appropriate font design. Second, we face challenges in generating realistic human faces, with some outputs appearing distorted. Increasing the dataset of human faces could improve the model's training. Third, the generated resolution is low since we trained on 256x256-resolution images. Future work could benefit from better computing power to train on higher-resolution images. Fourth, our model currently does not support highly specific requirements, such as generating a Margarita cocktail on a website. Additionally, our visual flow results lack finer control. We used the model EyeFormer~\cite{eyeformer} for scanpath prediction, however, it often biases scanpaths, starting from the center and moving toward the top left and right, limiting control over specific element emphasis. Although these patterns reflect natural eye movements, GUI designers may require more flexibility to direct user attention. Lastly, our current method does not support iterative design, which is important for early-stage prototyping, as each generation results in entirely new GUIs. Future work could improve the diffusion model by incorporating conditioning on the outputs from previous iterations. This could be done by using the current design as input to generate subsequent iterations. By conditioning on previous results, the model could produce variations that build upon existing designs, rather than starting anew each time.




% Our results suggest that diffusion models are suitable for exploration of low-fidelity GUI Generation.
% At the moment, there is no diffusion model that can produce high quality text and graphics;
% however, that is not necessarily a problem when the main goal is to explore ideas in early stages of design.
% Specifically, we showed that one can here exploit the fact that diffusion models are \emph{probabilistic} generative models.
% This makes it possible to ‘’pump’’ lots of different ideas.
% Before this paper, it has been an open question how to design diffusion models such that designers do not need to over-specify the input.
% By extending conditional diffusion to cover both wireframes and visual flow,
% we have shown that there is a path to low-effort interaction with a designer.



% Our generated results have limitations in the following cases: 1) we cannot generate valid texts. Future work can explore how to generate valid texts in the 
% 2) we cannot generate realistic human faces. Some human faces are distorted. Future work can add human faces in the dataset to better train on human faces. 3) Flow results are limited. No fine control of flow.
% we used EyeFormer~\cite{eyeformer}, the state-of-the-art scanpath prediction model, to predict scanpaths for each GUI. However, the model has some biases, for example, it mostly generates scanpaths starting from the center and goes towards the top left corner and moves towards the top right area, thus we can mostly only control whether the scanpaths conclude at the bottom left or bottom right. These biases align with human eye movement patterns but for GUI design, designers may want to specifically emphasize some elements or specifically guide users to look at GUI elements in a specific order. Future work can explore better ways to have more flexible control.
% In addition, it is very valuable to enable iterative design. It is important to iterate quickly in the early stages of design, however, our method does not support such iterations, every generation generates new GUIs. 

% % Limitation 1
% However functional and deployable GUIs need source code to link with external servers, databases and run over web connections faithfully. 
% We propose to explore self-debugging code generation from images pipelines using generated GUIs from our method. 
% Appending methods like Pix2Code, Design2Code to our pipeline could be a possible direction to explore for a more complete automatic front-end engineering solution.

% % Limitation 2
% Additionally, scaling up our method requires increasingly expensive compute so we propose to explore quantization and knowledge distillation setups.

% % Limitation 3
% Text Drawing with an image diffusion model remains a challenging research problem and most existing works require carefully crafted large text images and corresponding prompts... How to improve? (Read more TD, VLM papers)


% \paragraph{Scanpaths} Finally, we used EyeFormer~\cite{eyeformer}, the state-of-the-art scanpath prediction model, to predict scanpaths for each GUI. It would be better to use real scanpaths 
% would be better to use real scanpaths corrected by eye trackers, but it would be extremely time-consuming to do so and other proxies like webcams and cursors have different cognitive process compared to real eye movement. 
% Future work can use UEyes.

% A key objective in GUI design is to direct users’ attention towards discovering relevant information and interaction possibilities.
% Improved visual flow can enhance user engagement and guide user behavior~\cite{Rosenholtz11, Still10, ueyes}. 
% Modern GUIs, with their intricate arrangements of text, images, and numerous interactive elements such as buttons and menus, present challenges in considering visual flow during the design phase.
% Understanding user engagement often requires post-deployment analysis or expertise from experienced designers. 
% Can such expert knowledge be captured as a prior and inputted as a condition during the design process instead?



% In contrast, recent diffusion-based approaches train stably and handle multi-modal conditional generation effectively without the need for modality-specialized objectives.  
% Classifier-free-Guidance~\cite{cfg} enforces conditional adherence, effectively integrating multiple modalities without the need for explicitly defined discriminators or loss functions per condition. 




% However, diffusion models still face challenges in generating accurate text output due to their limited understanding of natural language~\cite{}. 
% %This highlights a significant issue in the development of generative vision models, as they often lack comprehensive natural language understanding. 

% Thus, the current models cannot generate meaningful texts.



















% This paper addressed the challenges of representing GUIs through a graph-based deep learning model. Prior deep learning-based GUI representations failed to consider the constraints for GUI elements and the visual-spatial-semantic structure of a GUI, which are important in computational design. Although many modern GUI tools use constraints to optimize GUIs, training a model to predict constraints remains a challenge. 
% Our proposed novel graph-based GUI representation captures both the properties of GUI elements, such as textual content, visual appearance, and element types, and their relationships in the visual, spatial, and semantic dimensions of a GUI. It can be computed efficiently in computational design. We further trained graph neural networks (GNNs) to take the graph as input to optimize the GUI. We will release our code and data.

% Our work has achieved the following results in the GUI autocompletion task. 

% \begin{enumerate}
% \item Our method predicts the position, size, and alignment of GUI elements more accurately. As shown in Figure~\ref{fig:comparison}, it achieves less than half of the error values in these three metrics (position, size, and alignment) compared to GRIDS~\cite{dayama2020grids}, an approach for autocompletion using integer programming. When the number of existing elements on the GUI increases, it remains to have low error rates while GRIDS's errors dramatically increase. 
% \item Our model offers superior alignment and visual appeal compared to the baseline method, and is better aligned with participants' preferences. In our comparison study,  70.33\% of the responses preferred results from our model compared to 13.54\% for GRIDS. 
% \item Our method enhances flexibility by integrating as a plug-in within a popular existing design tool, Figma. This integration allows designers to apply workflows they are already familiar with, eliminating the need to learn new tools or switch between different design software tools. Participants in the designer study praised the plug-in for accelerating their design process without disrupting the existing functionalities of their design applications. 
% \end{enumerate}

% \add{In addition to the demonstrated capability of our graph-based GUI representation in the GUI autocompletion task, we show that our GUI representation can be applied to other applications, such as GUI topic classification and GUI retrieval. 
% Our model demonstrated superior accuracy in GUI topic classification compared to baseline methods like ResNet50, Nearest Neighbors, and Random Forest. Furthermore, user feedback highlighted our model's effectiveness in retrieving visually similar GUIs compared to the Screen2Vec model.
% Compared to other data-driven approaches, our graph-based representation facilitates the understanding of GUI structure, improving the explainability of the model. This capacity enables our representation to potentially extend to diverse downstream tasks.
% For example, accessibility needs can also be represented as constraints~\cite{gajos2008improving}, and our method can train and predict layout constraints, thus it could potentially enhance accessibility. }

% %beyond the autocompletion task, such as design retrieval and evaluation/improvement of accessibility. With GNNs encoding the graph representation into an embedding vector, similar GUI designs can be retrieved as they will have closer embedding vectors. 

% \begin{figure}[t]
%   \def\w{\linewidth}
%   \centering
%  \includegraphics[width=\w]{figures/failure_case.pdf}
% \caption{Limitation of our method: It cannot capture the semantic correspondence between different types of GUI elements, like associating the ``Favorite'' text with a ``star'' icon, which could be explored further in future research.
% }
% \Description{This figure shows a limitation of our method: It cannot capture the semantic correspondence between different types of GUI elements.}
%   \label{fig:failure_case}
% \end{figure}

% \subsection{Limitations and Future Work}

% % \marginparsep=30pt
% % \marginnote{\color{violet}
% % R2: We clarified that we currently lack datasets containing non-rectangular bounding boxes and presented our solution for handling such boxes once they become available.\\~\\}
% As pointed out by participants in our designer study, our method has limited ability to generate accurate predictions if the unplaced element does not need to align or group with any existing element on the GUI. We currently assign a low confidence level to it to avoid uncertain predictions. Future work can improve the prediction of underconstrained GUI elements by considering more design priors or including more complicated constraints.
% \add{As shown in \autoref{tbl:related_work_comparison}, our representation does not explicitly represent the view hierarchy. The view hierarchy provides structural data, aiding models in understanding the layout and relationships of elements. We do not currently represent view hierarchies since they are not always available and often contain errors with incorrect structure information. However, future work can connect related element nodes in the graph representation to represent the view hierarchy.} 
% Moreover, while our method offers suggestions for each element to be placed, it provides only a single suggestion per element, thus constraining the possibility of exploration.
% In addition, we focus on a setting where all the elements are rectangular in shape or in rectangular bounding boxes. \add{There are no datasets available with non-rectangular bounding boxes. To accommodate various shapes of bounding boxes, we can augment the element node with additional parameters. These parameters would facilitate the description of common shapes, such as rectangles with rounded corners and circles. Subsequently, the model can be retrained to incorporate this information when present in the training dataset.} 
% %Future work can explore other shapes of bounding boxes.
% Furthermore, we observe that even for element prediction with high confidence levels, sometimes it does not predict ideal results. For example, as illustrated in Figure~\ref{fig:failure_case}, our method cannot capture the semantic correspondence between different types of GUI elements, e.g., it cannot detect that the ``Favorite'' text should correspond to the ``star'' icon. Future research could explore more about GUI element correspondence and constraints across UI types.



% \begin{acks}

% We appreciate the active discussion with Zixin Guo and Haishan Wang.
% This work was supported by Aalto University's Department of Information and Communications Engineering, the Research Council of Finland (flagship program: Finnish Center for Artificial Intelligence, FCAI, grants 328400, 345604, 341763; Subjective Functions, grant 357578), and the Meta PhD Fellowship.
% \end{acks} 


\bibliographystyle{ACM-Reference-Format}
\bibliography{Reference}


\end{document}
\endinput
%%
%% End of file `sample-authordraft.tex'.
