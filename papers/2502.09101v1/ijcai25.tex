%%%% ijcai25.tex

\typeout{Bridging the Gap Between LLMs and Human Intentions: Progresses and Challenges in Instruction Understanding, Intention Reasoning, and Reliable Generation}

% These are the instructions for authors for IJCAI-25.

\documentclass{article}
\pdfpagewidth=8.5in
\pdfpageheight=11in

% The file ijcai25.sty is a copy from ijcai22.sty
% The file ijcai22.sty is NOT the same as previous years'
\usepackage{ijcai25}

% Use the postscript times font!
\usepackage{times}
\usepackage{soul}
\usepackage{url}
\usepackage[hidelinks]{hyperref}
\usepackage[utf8]{inputenc}
\usepackage[small]{caption}
\usepackage{graphicx}
\usepackage{amsmath}
\usepackage{amsthm}
\usepackage{booktabs}
\usepackage{algorithm}
\usepackage{algorithmic}
\usepackage[switch]{lineno}
\urlstyle{same}
\usepackage{blindtext}
\usepackage{makecell}
\usepackage{pifont}
\usepackage{amssymb}
\usepackage{color}
\usepackage{tikz}
\usepackage[edges]{forest}
\usepackage{multirow}
\definecolor{hidden-draw}{RGB}{205, 44, 36}
\definecolor{hidden-blue}{RGB}{194,232,247}
\definecolor{hidden-orange}{RGB}{243,202,120}
\definecolor{hidden-yellow}{RGB}{242,244,193}
\usepackage{cleveref}
\usepackage{tcolorbox}

\usepackage{xcolor}
\usepackage{ifthen}
\usepackage[normalem]{ulem}

\definecolor{myorange}{HTML}{FEAE03}
\definecolor{myturquois}{HTML}{01AB8F}
\definecolor{mypink}{HTML}{D31876}

\definecolor{brightred}{HTML}{E55347} % Darker Tomato
\definecolor{orange}{HTML}{FF8C00} % Dark Orange
% \definecolor{gold}{HTML}{FFD700} % Gold, kept as is for distinction
\definecolor{yellowgreen}{HTML}{6B8E23} % Olive Drab, darker for better contrast
\definecolor{green}{HTML}{228B22} % Forest Green, darker and richer
\newtcolorbox{bluebox}[1][]{
	float,
  	title=#1,
	colback=myturquois!4,
	colframe=myturquois,
        top=1pt,           % 控制顶部空白
        bottom=1pt,        % 控制底部空白
        left=0pt,          % 控制左边空白
        right=0pt,          % 控制右边空白
        % before skip=0pt,        % 与前一段之间的距离
        % after skip=0pt,          % 与后一段之间的距离
        before skip=0.65em, after skip=0.75em,
}

\newcommand{\hypbox}[2]{%
\begin{tcolorbox}[colback=white!98!black,colframe=white!30!black,boxsep=1.1pt,top=6.75pt]%
\vspace{1.75pt}%
\textbf{#1}\\[-0.575em]
\noindent\makebox[\textwidth]{\rule{\textwidth}{0.4pt}}
\\[0.25em]
#2
\end{tcolorbox}
}


\definecolor{darkgrey}{rgb}{0.53,0.53,0.53}
\definecolor{mygrey}{rgb}{0.9,0.9,0.9}
\definecolor{color1}{HTML}{006EB8}
% \hypersetup{
%   colorlinks   = true,
%   urlcolor     = black,
%   linkcolor    = black,
%   citecolor   = black
% }


% If the title and author information does not fit in the area allocated, uncomment the following
%
%\setlength\titlebox{<dim>}
%
% and set <dim> to something 5cm or larger.


% tree
\usepackage{tikz}
\usepackage[edges]{forest}
% \usepackage{natbib}
\usetikzlibrary{shapes, arrows.meta, positioning}
\usepackage{graphicx}
\usepackage{forest}
\usetikzlibrary{trees,positioning,shapes,shadows,arrows.meta}


\definecolor{myblue}{RGB}{159, 192, 230}
\definecolor{myblueline}{RGB}{87, 127, 185}
\definecolor{bluelight1}{RGB}{185, 211, 237}
\definecolor{bluelight2}{RGB}{213, 222, 239}
\definecolor{mygreen}{RGB}{168, 209, 201}
% \definecolor{mygreen}{rgb}{0.56, 0.93, 0.56}
\definecolor{greenlight}{RGB}{220, 235, 234}
% \definecolor{hidden-draw}{RGB}{205, 44, 36}
\definecolor{hidden-draw}{RGB}{177, 177, 177}
\definecolor{mygray}{RGB}{185, 185, 185}







\definecolor{lightcoral}{rgb}{0.94, 0.5, 0.5}
\definecolor{lightgreen}{rgb}{0.56, 0.93, 0.56}
\definecolor{harvestgold}{rgb}{0.98, 0.85, 0.40}
\definecolor{brightlavender}{rgb}{0.75, 0.58, 0.89}
\definecolor{capri}{rgb}{0.0, 0.75, 1.0}
\definecolor{carminepink}{rgb}{0.92, 0.3, 0.26}
\definecolor{celadon}{rgb}{0.67, 0.88, 0.69}
\definecolor{darkpastelgreen}{rgb}{0.01, 0.75, 0.24}
\definecolor{lightgoldenrodyellow}{rgb}{0.98, 0.98, 0.82}
\definecolor{jonquil}{rgb}{0.98, 0.85, 0.37}
\definecolor{lightkhaki}{rgb}{0.94, 0.9, 0.55}
\definecolor{lemonchiffon}{rgb}{1.0, 0.98, 0.8}
\definecolor{schoolbusyellow}{rgb}{1.0, 0.85, 0.0}
% Comment out this line in the camera-ready submission
% \linenumbers

\urlstyle{same}

% the following package is optional:
%\usepackage{latexsym}

% See https://www.overleaf.com/learn/latex/theorems_and_proofs
% for a nice explanation of how to define new theorems, but keep
% in mind that the amsthm package is already included in this
% template and that you must *not* alter the styling.
\newtheorem{example}{Example}
\newtheorem{theorem}{Theorem}

% Following comment is from ijcai97-submit.tex:
% The preparation of these files was supported by Schlumberger Palo Alto
% Research, AT\&T Bell Laboratories, and Morgan Kaufmann Publishers.
% Shirley Jowell, of Morgan Kaufmann Publishers, and Peter F.
% Patel-Schneider, of AT\&T Bell Laboratories collaborated on their
% preparation.

% These instructions can be modified and used in other conferences as long
% as credit to the authors and supporting agencies is retained, this notice
% is not changed, and further modification or reuse is not restricted.
% Neither Shirley Jowell nor Peter F. Patel-Schneider can be listed as
% contacts for providing assistance without their prior permission.

% To use for other conferences, change references to files and the
% conference appropriate and use other authors, contacts, publishers, and
% organizations.
% Also change the deadline and address for returning papers and the length and
% page charge instructions.
% Put where the files are available in the appropriate places.


% PDF Info Is REQUIRED.

% Please leave this \pdfinfo block untouched both for the submission and
% Camera Ready Copy. Do not include Title and Author information in the pdfinfo section
\pdfinfo{
/TemplateVersion (IJCAI.2025.0)
}

\title{Bridging the Gap Between LLMs and Human Intentions: \\Progresses and Challenges in Instruction Understanding, Intention Reasoning, and Reliable Generation}


% Single author syntax
\author{
    Author Name
    \affiliations
    Affiliation
    \emails
    email@example.com
}

% Multiple author syntax (remove the single-author syntax above and the \iffalse ... \fi here)

\author{
Zongyu Chang$^1*$
\and
Feihong Lu$^2*$\and
Ziqin Zhu$^2$\and
Qian Li$^1$ \and
Cheng Ji$^2$ \and
Zhuo Chen$^1$ \and
Yang Liu$^5$ \and \\
Ruifeng Xu$^3$ \and
Yangqiu Song$^4$ \and
Shangguang Wang$^1$\And
Jianxin Li$^2$\\
\affiliations
$^1$ Beijing University of Posts and Telecommunications\\
$^2$Beihang University
$^3$Harbin Institute of Technology, Shenzhen\\
$^4$Hong Kong University of Science and Technology
$^5$Chinese Academy of Sciences\\
\emails
\{changzongyu, li.qian, sgwang\}@bupt.edu.cn,
\{lufeihong, zhuziqin, jicheng, lijx\}@buaa.edu.cn,
xuruifeng@hit.edu.cn,
yqsong@cse.ust.hk
\thanks{These authors contributed equally to this work.}
}

\begin{document}

\maketitle

\begin{abstract}
Large language models (LLMs) have demonstrated exceptional capabilities in understanding and generation.
% , driving their widespread adoption across various domains. 
However, when interacting with human instructions in real-world scenarios, LLMs still face significant challenges, particularly in accurately capturing and comprehending human instructions and intentions. 
This paper focuses on three challenges in LLM-based text generation tasks: instruction understanding, intention reasoning, and reliable generation. 
Regarding human complex instruction, LLMs have deficiencies in understanding long contexts and instructions in multi-round conversations. For intention reasoning, LLMs may have inconsistent command reasoning, difficulty reasoning about commands containing incorrect information, difficulty understanding user ambiguous language commands, and a weak understanding of user intention in commands. Besides, In terms of reliable generation, LLMs may have unstable generated content and unethical generation. To this end, we classify and analyze the performance of LLMs in challenging scenarios and conduct a comprehensive evaluation of existing solutions.
Furthermore, we introduce benchmarks and categorize them based on the aforementioned three core challenges. 
Finally, we explore potential directions for future research to enhance the reliability and adaptability of LLMs in real-world applications.

\end{abstract}


\section{Introduction}

Rapid advancements with the development of large language models (LLMs) have been experienced in the field of artificial intelligence. 
These models, built upon massive amounts of data and extensive computing resources, have shown impressive capabilities in understanding and generating human language. 
Recent advancements in LLMs, including the use of scaling laws~\cite{kaplan2020scaling}, supervised fine-tuning (SFT)~\cite{wu2021recursively}, and reinforcement learning with human feedback (RLHF)~\cite{ouyang2022training}, have propelled these models to new heights. Researchers have explored innovative strategies like chain-of-thought reasoning (COT)~\cite{wei2022chain}, aiming to enhance their performance in processing and generating accurate responses. 
However, they still struggle with more complex interactions, especially when the input data is ambiguous, incomplete, or inconsistent. Despite improvements, issues such as content hallucination~\cite{DBLP:conf/emnlp/LiCZNW23} and logical misinterpretations remain prevalent. Consequently, while LLMs show promise, they are far from flawless and require further refinement to address the challenges posed by more unpredictable and complex human instructions as follows.


\begin{figure}[t]
  \centering
  \setlength{\belowcaptionskip}{-0.4cm}
  \includegraphics[width=1
  \linewidth]{picture/figure1.pdf}
  \caption{Example of large language model generation.
  }\label{fig:intro}
  % \vspace{-4mm}
\end{figure}

\textbf{I. Challenge of Instruction Understanding.}
One of the most pressing challenges that LLMs face is instruction understanding as \Cref{fig:intro}(a), particularly when the user input involves complex or multi-step instructions. While models have improved in parsing relatively simple queries, they continue to encounter significant difficulties when dealing with long, context-rich instructions or when instructions are spread across multiple conversational turns. LLMs often fail to grasp subtle nuances or interpret implicit meanings embedded within the text, which leads to inaccurate or incomplete responses. Existing approaches to instruction understanding have introduced techniques like optimizing the model's parsing abilities~\cite{teng2024fine}, and context-aware optimization~\cite{sun2024parrot}. While these methods show promise, they often fall short when addressing the complexities and ambiguities present in real-world instructions.


\textbf{II. Challenge of Intention Reasoning.}
Another critical area is intention reasoning as illustrated in \Cref{fig:intro}(b), where they struggle to align the generated responses with the user’s underlying intention. Ambiguities in language, conflicting instructions, and implicit requirements often result in models producing outputs that diverge from the user's expectations. LLMs also face difficulties when instructions are inconsistent or contain incorrect information, which challenges the model’s ability to make accurate inferences. Various strategies, including retrieval-enhanced generation and fine-tuning techniques, have been proposed to enhance reasoning capabilities, enabling the models to better handle inconsistent or incomplete instructions. However, these methods often introduce new challenges related to bias and the inability to fully resolve conflicts in user input, further complicating the alignment between generated content and user expectations.


\textbf{III. Challenge of Reliable Generation.}
The final major challenge is the reliable generation, which pertains to the accuracy, ethical considerations, and stability of the content they produce, such as \Cref{fig:intro}(c). While LLMs are generally capable of generating coherent and contextually relevant outputs, they sometimes exhibit instability, generating content that is factually incorrect, logically inconsistent, or ethically questionable. This challenge is exacerbated by the model’s inability to recognize uncertainty, which can lead to overconfident but inaccurate outputs. Recent efforts to address this issue involve techniques like uncertainty-aware fine-tuning and using external tools to evaluate output credibility. However, these approaches struggle to provide a comprehensive and reliable solution, especially in complex or dynamic contexts.



% \section{Taxonomy}

% As illustrated by Fig. \ref{}, the typical process of vision models based time series analysis has five components: (1) normalization/scaling; (2) time series to image transformation; (3) image modeling; (4) image to time series recovery; and (5) task processing. In the rest of this paper, we will discuss the typical methods for each of these components. The detailed taxonomy of the methods are summarized in Table \ref{tab.taxonomy}.

%Typical step: normalization/scaling, transformation, vision modeling, task-specific head, inverse transformation (for tasks that output time series, e.g., forecasting, generation, imputation, anomaly detection). Normalization is to fit the arbitrary range of time series values to RGB representation.

\begin{figure*}[!t]
\centering
\includegraphics[width=1.0\textwidth]{fig/fig_3.pdf}
% \vspace{-1em}
\caption{An illustration of different methods for imaging time series with a sample (length=336) from the \textit{Electricity} benchmark dataset \protect\cite{nie2023time}. (a)(c)(d)(e)(f) %are univariate methods.
visualize the same variate. (b) visualizes all 321 variates. Filterbank is omitted due to its %high
similarity to STFT.}\label{fig.tsimage}
\vspace{-0.2cm}
\end{figure*}

\begin{table*}[t]
\centering
\scriptsize
\setlength{\tabcolsep}{2.7pt}{
% \begin{tabular}{llllllllllll}
\begin{tabular}{llcccccccccl}
\toprule[1pt]
\multirow{2}{*}{Method} & \multirow{2}{*}{TS-Type} & \multirow{2}{*}{Imaging} & \multicolumn{5}{c}{Imaged Time Series Modeling} & \multirow{2}{*}{TS-Recover} & \multirow{2}{*}{Task} & \multirow{2}{*}{Domain} & \multirow{2}{*}{Code}\\ \cmidrule{4-8}
 & & & Multi-modal & Model & Pre-trained & Fine-tune & Prompt & & & & \\ \midrule
\cite{silva2013time} & UTS & RP & \xmark & \texttt{K-NN} & \xmark & \xmark & \xmark & \xmark & Classification & General & \xmark\\
\cite{wang2015encoding} & UTS & GAF & \xmark & \texttt{CNN} & \xmark & \cmark$^{\flat}$ & \xmark & \cmark & Classification & General & \xmark\\
\cite{wang2015imaging} & UTS & GAF & \xmark & \texttt{CNN} & \xmark & \cmark$^{\flat}$ & \xmark & \cmark & Multiple & General & \xmark\\
% \multirow{2}{*}{\cite{wang2015imaging}} & \multirow{2}{*}{UTS} & \multirow{2}{*}{GAF} & \multirow{2}{*}{\xmark} & \multirow{2}{*}{\texttt{CNN}} & \multirow{2}{*}{\xmark} & \multirow{2}{*}{\cmark$^{\flat}$} & \multirow{2}{*}{\xmark} & \multirow{2}{*}{\cmark} & Classification & \multirow{2}{*}{General} & \multirow{2}{*}{\xmark}\\
% & & & & & & & & & \& Imputation & & \\
\cite{ma2017learning} & MTS & Heatmap & \xmark & \texttt{CNN} & \xmark & \cmark$^{\flat}$ & \xmark & \cmark & Forecasting & Traffic & \xmark\\
\cite{hatami2018classification} & UTS & RP & \xmark & \texttt{CNN} & \xmark & \cmark$^{\flat}$ & \xmark & \xmark & Classification & General & \xmark\\
\cite{yazdanbakhsh2019multivariate} & MTS & Heatmap & \xmark & \texttt{CNN} & \xmark & \cmark$^{\flat}$ & \xmark & \xmark & Classification & General & \cmark\textsuperscript{\href{https://github.com/SonbolYb/multivariate_timeseries_dilated_conv}{[1]}}\\
MSCRED \cite{zhang2019deep} & MTS & Other ($\S$\ref{sec.othermethod}) & \xmark & \texttt{ConvLSTM} & \xmark & \cmark$^{\flat}$ & \xmark & \xmark & Anomaly & General & \cmark\textsuperscript{\href{https://github.com/7fantasysz/MSCRED}{[2]}}\\
\cite{li2020forecasting} & UTS & RP & \xmark & \texttt{CNN} & \cmark & \cmark & \xmark & \xmark & Forecasting & General & \cmark\textsuperscript{\href{https://github.com/lixixibj/forecasting-with-time-series-imaging}{[3]}}\\
\cite{cohen2020trading} & UTS & LinePlot & \xmark & \texttt{Ensemble} & \xmark & \cmark$^{\flat}$ & \xmark & \xmark & Classification & Finance & \xmark\\
% \cite{du2020image} & UTS & Spectrogram & \xmark & \texttt{CNN} & \xmark & \cmark$^{\flat}$ & \xmark & \xmark & Classification & Finance & \xmark\\
\cite{barra2020deep} & UTS & GAF & \xmark & \texttt{CNN} & \xmark & \cmark$^{\flat}$ & \xmark & \xmark & Classification & Finance & \xmark\\
% \cite{barra2020deep} & UTS & GAF & \xmark & \texttt{VGG-16} & \xmark & \cmark$^{\flat}$ & \xmark & \xmark & Classification & Finance & \xmark\\
% \cite{cao2021image} & UTS & RP & \xmark & \texttt{CNN} & \xmark & \cmark$^{\flat}$ & \xmark & \xmark & Classification & General & \xmark\\
VisualAE \cite{sood2021visual} & UTS & LinePlot & \xmark & \texttt{CNN} & \xmark & \cmark$^{\flat}$ & \xmark & \cmark & Forecasting & Finance & \xmark\\
% VisualAE \cite{sood2021visual} & UTS & LinePlot & \xmark & \texttt{CNN} & \xmark & \cmark$^{\flat}$ & \xmark & \xmark & Img-Generation & Finance & \xmark\\
\cite{zeng2021deep} & MTS & Heatmap & \xmark & \texttt{CNN,LSTM} & \xmark & \cmark$^{\flat}$ & \xmark & \cmark & Forecasting & Finance & \xmark\\
% \cite{zeng2021deep} & MTS & Heatmap & \xmark & \texttt{SRVP} & \xmark & \cmark$^{\flat}$ & \xmark & \cmark & Forecasting & Finance & \xmark\\
AST \cite{gong2021ast} & UTS & Spectrogram & \xmark & \texttt{DeiT} & \cmark & \cmark & \xmark & \xmark & Classification & Audio & \cmark\textsuperscript{\href{https://github.com/YuanGongND/ast}{[4]}}\\
TTS-GAN \cite{li2022tts} & MTS & Heatmap & \xmark & \texttt{ViT} & \xmark & \cmark$^{\flat}$ & \xmark & \cmark & Ts-Generation & Health & \cmark\textsuperscript{\href{https://github.com/imics-lab/tts-gan}{[5]}}\\
SSAST \cite{gong2022ssast} & UTS & Spectrogram & \xmark & \texttt{ViT} & \cmark$^{\natural}$ & \cmark & \xmark & \xmark & Classification & Audio & \cmark\textsuperscript{\href{https://github.com/YuanGongND/ssast}{[6]}}\\
MAE-AST \cite{baade2022mae} & UTS & Spectrogram & \xmark & \texttt{MAE} & \cmark$^{\natural}$ & \cmark & \xmark & \xmark & Classification & Audio & \cmark\textsuperscript{\href{https://github.com/AlanBaade/MAE-AST-Public}{[7]}}\\
AST-SED \cite{li2023ast} & UTS & Spectrogram & \xmark & \texttt{SSAST,GRU} & \cmark & \cmark & \xmark & \xmark & EventDetection & Audio & \xmark\\
\cite{jin2023classification} & UTS & %Multiple
LinePlot & \xmark & \texttt{CNN} & \cmark & \cmark & \xmark & \xmark & Classification & Physics & \xmark\\
ForCNN \cite{semenoglou2023image} & UTS & LinePlot & \xmark & \texttt{CNN} & \xmark & \cmark$^{\flat}$ & \xmark & \xmark & Forecasting & General & \xmark\\
Vit-num-spec \cite{zeng2023pixels} & UTS & Spectrogram & \xmark & \texttt{ViT} & \xmark & \cmark$^{\flat}$ & \xmark & \xmark & Forecasting & Finance & \xmark\\
% \cite{wimmer2023leveraging} & MTS & LinePlot & \xmark & \texttt{CLIP,LSTM} & \cmark & \cmark & \xmark & \xmark & Classification & Finance & \xmark\\
ViTST \cite{li2023time} & MTS & LinePlot & \xmark & \texttt{Swin} & \cmark & \cmark & \xmark & \xmark & Classification & General & \cmark\textsuperscript{\href{https://github.com/Leezekun/ViTST}{[8]}}\\
MV-DTSA \cite{yang2023your} & UTS\textsuperscript{*} & LinePlot & \xmark & \texttt{CNN} & \xmark & \cmark$^{\flat}$ & \xmark & \cmark & Forecasting & General & \cmark\textsuperscript{\href{https://github.com/IkeYang/machine-vision-assisted-deep-time-series-analysis-MV-DTSA-}{[9]}}\\
TimesNet \cite{wu2023timesnet} & MTS & Heatmap & \xmark & \texttt{CNN} & \xmark & \cmark$^{\flat}$ & \xmark & \cmark & Multiple & General & \cmark\textsuperscript{\href{https://github.com/thuml/TimesNet}{[10]}}\\
ITF-TAD \cite{namura2024training} & UTS & Spectrogram & \xmark & \texttt{CNN} & \cmark & \xmark & \xmark & \xmark & Anomaly & General & \xmark\\
\cite{kaewrakmuk2024multi} & UTS & GAF & \xmark & \texttt{CNN} & \cmark & \cmark & \xmark & \xmark & Classification & Sensing & \xmark\\
HCR-AdaAD \cite{lin2024hierarchical} & MTS & RP & \xmark & \texttt{CNN,GNN} & \xmark & \cmark$^{\flat}$ & \xmark & \xmark & Anomaly & General & \xmark\\
FIRTS \cite{costa2024fusion} & UTS & Other ($\S$\ref{sec.othermethod}) & \xmark & \texttt{CNN} & \xmark & \cmark$^{\flat}$ & \xmark & \xmark & Classification & General & \cmark\textsuperscript{\href{https://sites.google.com/view/firts-paper}{[11]}}\\
% \multirow{2}{*}{FIRTS \cite{costa2024fusion}} & \multirow{2}{*}{UTS} & Spectrogram & \multirow{2}{*}{\xmark} & \multirow{2}{*}{\texttt{CNN}} & \multirow{2}{*}{\xmark} & \multirow{2}{*}{\cmark$^{\flat}$} & \multirow{2}{*}{\xmark} & \multirow{2}{*}{\xmark} & \multirow{2}{*}{Classification} & \multirow{2}{*}{General} & \multirow{2}{*}{\cmark\textsuperscript{\href{https://sites.google.com/view/firts-paper}{[2]}}}\\
%  & & \& GAF,RP,MTF & & & & & & & & & \\
% \cite{homenda2024time} & UTS\textsuperscript{*} & Multiple & \xmark & \texttt{CNN} & \xmark & \cmark$^{\flat}$ & \xmark & \xmark & Classification & General & \xmark\\
CAFO \cite{kim2024cafo} & MTS & RP & \xmark & \texttt{CNN,ViT} & \xmark & \cmark$^{\flat}$ & \xmark & \xmark & Explanation & General & \cmark\textsuperscript{\href{https://github.com/eai-lab/CAFO}{[12]}}\\
% \multirow{2}{*}{CAFO \cite{kim2024cafo}} & \multirow{2}{*}{MTS} & \multirow{2}{*}{RP} & \multirow{2}{*}{\xmark} & \texttt{ShuffleNet,ResNet} & \multirow{2}{*}{\cmark} & \multirow{2}{*}{\cmark} & \multirow{2}{*}{\xmark} & \multirow{2}{*}{\xmark} & Classification & \multirow{2}{*}{General} & \multirow{2}{*}{\cmark}\\
%  & & & & \texttt{MLP-Mixer,ViT} & & & & & \& Explanation & & \\
ViTime \cite{yang2024vitime} & UTS\textsuperscript{*} & LinePlot & \xmark & \texttt{ViT} & \cmark$^{\natural}$ & \cmark & \xmark & \cmark & Forecasting & General & \cmark\textsuperscript{\href{https://github.com/IkeYang/ViTime}{[13]}}\\
ImagenTime \cite{naiman2024utilizing} & MTS & Other ($\S$\ref{sec.othermethod}) & \xmark & %\texttt{Diffusion}
\texttt{CNN} & \xmark & \cmark$^{\flat}$ & \xmark & \cmark & Ts-Generation & General & \cmark\textsuperscript{\href{https://github.com/azencot-group/ImagenTime}{[14]}}\\
TimEHR \cite{karami2024timehr} & MTS & Heatmap & \xmark & \texttt{CNN} & \xmark & \cmark$^{\flat}$ & \xmark & \cmark & Ts-Generation & Health & \cmark\textsuperscript{\href{https://github.com/esl-epfl/TimEHR}{[15]}}\\
VisionTS \cite{chen2024visionts} & UTS\textsuperscript{*} & Heatmap & \xmark & \texttt{MAE} & \cmark & \cmark & \xmark & \cmark & Forecasting & General & \cmark\textsuperscript{\href{https://github.com/Keytoyze/VisionTS}{[16]}}\\ \midrule
InsightMiner \cite{zhang2023insight} & UTS & LinePlot & \cmark & \texttt{LLaVA} & \cmark & \cmark & \cmark & \xmark & Txt-Generation & General & \xmark\\
\cite{wimmer2023leveraging} & MTS & LinePlot & \cmark & \texttt{CLIP,LSTM} & \cmark & \cmark & \xmark & \xmark & Classification & Finance & \xmark\\
% \cite{dixit2024vision} & UTS & Spectrogram & \cmark & \texttt{GPT4o,Gemini} & \cmark & \xmark & \cmark & \xmark & Classification & Audio & \xmark\\
\multirow{2}{*}{\cite{dixit2024vision}} & \multirow{2}{*}{UTS} & \multirow{2}{*}{Spectrogram} & \multirow{2}{*}{\cmark} & \texttt{GPT4o,Gemini} & \multirow{2}{*}{\cmark} & \multirow{2}{*}{\xmark} & \multirow{2}{*}{\cmark} & \multirow{2}{*}{\xmark} & \multirow{2}{*}{Classification} & \multirow{2}{*}{Audio} & \multirow{2}{*}{\xmark}\\
 & & & & \& \texttt{Claude3} & & & & & & & \\
\cite{daswani2024plots} & MTS & LinePlot & \cmark & \texttt{GPT4o,Gemini} & \cmark & \xmark & \cmark & \xmark & Multiple & General & \xmark\\
TAMA \cite{zhuang2024see} & UTS & LinePlot & \cmark & \texttt{GPT4o} & \cmark & \xmark & \cmark & \xmark & Anomaly & General & \xmark\\
\cite{prithyani2024feasibility} & MTS & LinePlot & \cmark & \texttt{LLaVA} & \cmark & \cmark & \cmark & \xmark & Classification & General & \cmark\textsuperscript{\href{https://github.com/vinayp17/VLM_TSC}{[17]}}\\
\bottomrule[1pt]
\end{tabular}}
\vspace{-0.25cm}
\caption{Taxonomy of vision models on time series. The top panel includes single-modal models. The bottom panel includes multi-modal models. {\bf TS-Type} denotes type of time series. {\bf TS-Recover} denotes %whether time series recovery ($\S$\ref{sec.processing}) has been performed.
recovering time series from predicted images ($\S$\ref{sec.processing}). \textsuperscript{*}: %the model has been %applied on MTSs by %processing %modeling the individual UTSs of each MTS.
the method has been used to model the individual UTSs of an MTS. $^{\natural}$: a new pre-trained model was proposed in the work. $^{\flat}$: %without using a pre-trained model, fine-tune means training from scratch.
when pre-trained models were unused, ``Fine-tune'' refers to train a task-specific model from scratch. %In the
{\bf Model} column: \texttt{CNN} could be regular CNN, ResNet, VGG-Net, %U-Net,
{\em etc.}}\label{tab.taxonomy}
% The code only include verified official code from the authors.
\vspace{-0.3cm}
\end{table*}

\begin{table*}[t]
\centering
\small
\setlength{\tabcolsep}{2.9pt}{
\begin{tabular}{l|l|l|l}\hline
% \toprule[1pt]
\rowcolor{gray!20}
{\bf Method} & {\bf TS-Type} & {\bf Advantages} & {\bf Limitations}\\ \hline
Line Plot ($\S$\ref{sec.lineplot}) & UTS, MTS & matches human perception of time series & limited to MTSs with a small number of variates\\ \hline
Heatmap ($\S$\ref{sec.heatmap}) & UTS, MTS & straightforward for both UTSs and MTSs & the order of variates may affect their correlation learning\\ \hline
Spectrogram ($\S$\ref{sec.spectrogram}) & UTS & encodes the time-frequency space & limited to UTSs; needs a proper choice of window/wavelet\\ \hline
GAF ($\S$\ref{sec.gaf}) & UTS & encodes the temporal correlations in a UTS & limited to UTSs; $O(T^{2})$ time and space complexity\\ \hline% for long time series\\ \hline
% RP ($\S$\ref{sec.rp}) & UTS & flexibility in image size by tuning $m$ and $\tau$ & limited to UTSs; the pattern has a threshold-dependency\\ \hline
RP ($\S$\ref{sec.rp}) & UTS & flexibility in image size by tuning $m$ and $\tau$ & limited to UTSs; information loss after thresholding\\ \hline
% \bottomrule[1pt]
\end{tabular}}
\vspace{-0.2cm}
\caption{Summary of the five primary methods for transforming time series to images. {\bf TS-Type} denotes type of time series.}\label{tab.tsimage}
\vspace{-0.2cm}
\end{table*}

\section{Time Series To Image Transformation}\label{sec.tsimage}

% This section summarizes 5 major methods for imaging time series ($\S$\ref{sec.lineplot}-$\S$\ref{sec.rp}). We also discuss some other methods ($\S$\ref{sec.othermethod}) and how to model MTS with these methods ($\S$\ref{sec.modelmts}).
This section summarizes the methods for imaging time series ($\S$\ref{sec.lineplot}-$\S$\ref{sec.othermethod}) and their extensions to encode MTSs ($\S$\ref{sec.modelmts}).

% This section summarizes 5 major methods for transforming time series to images, including Line Plot, Heatmap, Spetrogram, GAF and RP, and several minor methods. We discuss their pros and cons and how to deal with MTS.

% This section discusses the advantages and limitations of different methods for time series to image transformation (invertible, efficiency, information preservation, MTS, long-range time series, parametric, etc.).

%\subsection{Methods}

\vspace{-0.08cm}

\subsection{Line Plot}\label{sec.lineplot}

Line Plot is a straightforward way for visualizing UTSs for human analysis ({\em e.g.}, stocks, power consumption, {\em etc.}). As illustrated by Fig. \ref{fig.tsimage}(a), the simplest approach is to draw a 2D image with x-axis representing %the time horizon
time steps and y-axis representing %the magnitude of the normalized time series.
time-wise values, %A line is used to connect all values of the series over time.
with a line connecting all values of the series over time. This image can be %represented by either three-channel pixels or single-channel pixels
either three-channel ({\em i.e.}, RGB) or single-channel as the colors may not %provide additional information
be informative %\cite{cohen2020trading,sood2021visual,jin2023classification,zhang2023insight,zhuang2024see}.
\cite{cohen2020trading,sood2021visual,jin2023classification,zhang2023insight}. ForCNN \cite{semenoglou2023image} even uses a single 8-bit integer to represent each pixel for black-white images. So far, there is no consensus on whether other graphical components, such as legend, grids and tick labels, could provide extra benefits in any task. For example, ViTST \cite{li2023time} finds these components are superfluous in a classification task, while TAMA \cite{zhuang2024see} finds grid-like auxiliary lines help enhance anomaly detection.

In addition to the regular Line Plot, MV-DTSA \cite{yang2023your} and ViTime \cite{yang2024vitime} divide an image into $h\times L$ grids, %where $h$ is the number of rows and $L$ is the number of columns,
and %introduced
define a function to map each time step of a UTS to a grid, producing a grid-like Line Plot. Also, we include methods that use Scatter Plot \cite{daswani2024plots,prithyani2024feasibility} in this category because %the only difference between a Scatter Plot and a Line Plot is whether the time-wise values are connected by lines.
a Scatter Plot resembles a Line Plot but doesn't connect %time-wise values
data points with a line. By comparing them, \cite{prithyani2024feasibility} finds a Line Plot could induce better time series classification.

For MTSs, we defer the discussion on Line Plot to $\S$\ref{sec.modelmts}.

% For MTS, some methods use the channel-independence assumption proposed in \cite{nie2023time} and represent each variate in MTS with an individual Line Plot \cite{yang2023your,yang2024vitime}. ViTST \cite{li2023time} also uses an individual Line Plot per variate, but colors different lines and assembles all plots to form a bigger image. The method in \cite{wimmer2023leveraging} plots %the time series of
% all variates in a single Line Plot and distinguish them by %use different
% types of lines ({\em e.g.}, solid, dashed, dotted, {\em etc.}). %to distinguish them.
% However, these methods only work for a small number of variates. For example, in \cite{wimmer2023leveraging}, there are only 4 variates in its financial MTSs.

%\cite{li2023time} space-costly because of blank pixels. scatter plot.

%Invertible with a numeric prediction head \cite{sood2021visual}. It fits tasks such as forecasting, imputation, etc.

\vspace{-0.08cm}

\subsection{Heatmap}\label{sec.heatmap}

As shown in Fig. \ref{fig.tsimage}(b), Heatmap is a 2D visualization of the magnitude of the values in a matrix using color. %The variation of color represents the intensity of each value. %Therefore,
It has been used to %directly
represent the matrix of an MTS, {\em i.e.}, $\mat{X} \in \mathbb{R}^{d\times T}$, as a one-channel $d\times T$ image \cite{li2022tts,yazdanbakhsh2019multivariate}. Similarly, TimEHR \cite{karami2024timehr} represents an {\em irregular} MTS, where the intervals between time steps are uneven, as a $d\times H$ Heatmap image by grouping the uneven time steps into $H$ even time bins. In \cite{zeng2021deep}, a different method is used for visualizing a 9-variate financial %time series.
MTS. It reshapes the 9 variates at each time step to a $3\times 3$ Heatmap image, and uses the sequence of images to forecast future %image
frames, achieving %time series
%MTS
time series forecasting. In contrast, VisionTS \cite{chen2024visionts} uses Heatmap to visualize UTSs. %instead.
Similar to TimesNet \cite{wu2023timesnet}, it first segments a length-$T$ UTS into $\lfloor T/P\rfloor$ length-$P$ subsequences, where $P$ is a parameter representing a periodicity of the UTS. Then the subsequences are stacked into a $P\times \lfloor T/P\rfloor$ matrix, %and duplicated 3 times to produce a 3-channel
with 3 duplicated channels, to produce a grayscale image %which serves as an
input to %a vision foundation model.
an LVM. To encode MTSs, VisionTS adopts the channel independence assumption \cite{nie2023time} and individually models each variate in an MTS.

\vspace{0.2cm}

\noindent{\bf Remark.} Heatmap can be used to visualize matrices of various forms. It is also used for matrices generated by the subsequent methods ({\em e.g.}, Spectrogram, GAF, RP) in this section. In this paper, the name Heatmap refers specifically to images that use color to visualize the (normalized) values in UTS $\mat{x}$ or MTS $\mat{X}$ without performing other transformations.

%\cite{chen2024visionts,karami2024timehr} bin version of TSH \cite{karami2024timehr}, DE and STFT \cite{naiman2024utilizing} (DE can be used for constructing RP), rearrange variates for video version of TSH \cite{zeng2021deep}.

%\vspace{0.2cm}

\subsection{Spectrogram}\label{sec.spectrogram}

A {\em spectrogram} is a visual representation of the spectrum of frequencies of a signal as it varies with time, which are extensively used for analyzing audio signals \cite{gong2021ast}. Since audio signals are a type of UTS, spectrogram can be considered as a method for imaging a UTS. As shown in Fig. \ref{fig.tsimage}(c), a common format is a 2D heatmap image with x-axis representing time steps and y-axis representing frequency, {\em a.k.a.} a time-frequency space. %The color at each point
Each pixel in the image represents the (logarithmic) amplitude of a specific frequency at a specific time point. Typical methods for %transforming a UTS to
producing a spectrogram include {\bf Short-Time Fourier Transform (STFT)} \cite{griffin1984signal}, {\bf Wavelet Transform} \cite{daubechies1990wavelet}, and {\bf Filterbank} \cite{vetterli1992wavelets}.

\vspace{0.2cm}

\noindent{\bf STFT.} %Discrete Fourier transform (DFT) can be used to represent a UTS signal %$\mat{x}=[x_{1}, ..., x_{T}]$
%$\mat{x}\in\mathbb{R}^{1\times T}$ as a sum of sinusoidal components. The output of the transform is a function of frequency $f(w)$, describing the intensity of each constituent frequency $w$ of the entire UTS. 
Discrete Fourier transform (DFT) can be used to describe the intensity $f(w)$ of each constituent frequency $w$ of a UTS signal $\mat{x}\in\mathbb{R}^{1\times T}$. However, $f(w)$ has no time dependency. It cannot provide dynamic information such as when a specific frequency appear in the UTS. STFT addresses this deficiency by sliding a window function $g(t)$ over the time steps in %the UTS,
$\mat{x}$, and computing the DFT within each window by
\begin{equation}\label{eq.stft}
\small
\begin{aligned}
f(w,\tau) = \sum_{t=1}^{T}x_{t}g(t - \tau)e^{-iwt}
\end{aligned}
\end{equation}
where $w$ is frequency, $\tau$ is the position of the window, $f(w,\tau)$ describes the intensity of frequency $w$ at time step $\tau$.

%With a proper selection of the
By selecting a proper window function $g(\cdot)$ ({\em e.g.}, Gaussian/Hamming/Bartlett window), %({\em e.g.}, Gaussian window, Hamming window, Bartlett window), %{\em etc.}),
a 2D spectrogram ({\em e.g.}, Fig. \ref{fig.tsimage}(c)) can be drawn via a heatmap on the squared values $|f(w,\tau)|^{2}$, with $w$ as the y-axis, and $\tau$ as the x-axis. For example, \cite{dixit2024vision} uses STFT based spectrogram as an input to LMMs %\hh{do you mean LVMs? check}
for time series classification.

%Fourier transform is a powerful data analysis tool that represents any complex signal as a sum of sines and cosines and transforms the signal from the time domain to the frequency domain. However, Fourier transform can only show which frequencies are present in the signal, but not when these frequencies appear. The STFT divides original signal into several parts using a sliding window to fix this problem. STFT involves a sliding window for extracting frequency components within the window.

\vspace{0.2cm}

\noindent{\bf Wavelet Transform.} %Like Fourier transform, %\hh{this paragraph needs a citation}
Continuous Wavelet Transform (CWT) uses the inner product to measure the similarity between a signal function $x(t)$ and an analyzing function. %In STFT (Eq.~\eqref{eq.stft}), the analyzing function is a windowed exponential $g(t - \tau)e^{-iwt}$.
%In CWT,
The analyzing function is a {\em wavelet} $\psi(t)$, where the typical choices include Morse wavelet, Morlet wavelet, %Daubechies wavelet, %Beylkin wavelet, 
{\em etc.} %The
CWT compares $x(t)$ to the shifted and scaled ({\em i.e.}, stretched or shrunk) versions of the wavelet, and output a CWT coefficient by
\begin{equation}\label{eq.cwt}
\small
\begin{aligned}
c(s,\tau) = \int_{-\infty}^{\infty}x(t)\frac{1}{s}\psi^{*}(\frac{t - \tau}{s})dt
\end{aligned}
\end{equation}
where $*$ denotes complex conjugate, $\tau$ is the time step to shift, and $s$ represents the scale. In practice, a discretized version of CWT in Eq.~\eqref{eq.cwt} is implemented for UTS $[x_{1}, ..., x_{T}]$.

It is noteworthy that the scale $s$ controls the frequency encoded in a wavelet -- a larger $s$ leads to a stretched wavelet with a lower frequency, and vice versa. As such, by varying $s$ and $\tau$, a 2D spectrogram ({\em e.g.}, Fig. \ref{fig.tsimage}(d)) can be drawn %, often with a heatmap
on $|c(s,\tau)|$, where $s$ is the y-axis and $\tau$ is the x-axis. Compared to STFT, which uses a fixed window size, Wavelet Transform allows variable wavelet sizes -- a larger size %region
for more precise low frequency information. 
%Usually, $s$ and $\tau$ vary dependently -- a larger $s$ leads to a stretched wavelet that shifts slowly, {\em i.e.}, a smaller $\tau$. This property %of CWT
%yields a spectrogram that balances the resolutions of frequency %$s$
%and time, %$\tau$,
%which is an advantage over the fixed time resolution in STFT.
% Thus, both of the methods in %\cite{du2020image}
% \cite{namura2024training} and \cite{zeng2023pixels} choose CWT (with Morlet wavelet) to generate the spectrogram.
Thus, the methods in \cite{du2020image,namura2024training,zeng2023pixels} choose CWT (with Morlet wavelet) to generate the spectrogram.

%A wavelet is a wave-like oscillation that has zero mean and is localized in both time and frequency space.

\vspace{0.2cm}

\noindent{\bf Filterbank.} This method %is relevant to
resembles STFT and is often used in processing audio signals. Given an audio signal, it firstly goes through a {\em pre-emphasis filter} to boost high frequencies, which helps improve the clarity of the signal. Then, STFT is applied on the signal. %with a sliding window $g(t)$ of size $k$ that shifts in a fixed stride $\tau$. %where the adjacent windows may overlap in $k$ time length.
%Finally, filterbank features are computed by applying multiple ``triangle-shaped'' filters spaced on the Mel-scale to the STFT output $f(w, \tau)$. %where Mel-scale is a method to make the filters more discriminative on lower frequencies, %than higher frequencies,
%imitating the non-linear human ear perception of sound.
Finally, multiple ``triangle-shaped'' filters spaced on a Mel-scale are applied to the STFT power spectrum $|f(w, \tau)|^{2}$ to extract frequency bands. The outcome filterbank features $\hat{f}(w, \tau)$ can be used to yield a spectrogram with $w$ as the y-axis, and $\tau$ as the x-axis.

%Filterbank was introduced in AST \cite{gong2021ast} with %$k$=25ms
Filterbank was adopted in AST \cite{gong2021ast} with 
a 25ms Hamming window that shifts every 10ms for classifying audio signals using Vision Transformer (ViT). It then becomes widely used in the follow-up works such as SSAST \cite{gong2022ssast}, MAE-AST \cite{baade2022mae}, and AST-SED \cite{li2023ast}, as summarized in Table \ref{tab.taxonomy}.



%Use MLP to predict TS directly \cite{zeng2023pixels}.

%\vspace{0.2cm}

% \vspace{0.2cm}

\subsection{Gramian Angular Field (GAF)}\label{sec.gaf}

GAF was introduced for classifying UTSs using CNNs %using %image based CNNs
by \cite{wang2015encoding}. It was then extended %with an extension
to an imputation task in \cite{wang2015imaging}. Similarly, \cite{barra2020deep} applied GAF for financial time series forecasting.

Given a UTS $\mat{x}\in\mathbb{R}^{1\times T}$, %$[x_{1}, ..., x_{T}]$,
the first step %before GAF
is to rescale each $x_{t}$ to a value $\tilde{x}_{t}$ %in the interval of
within $[0, 1]$ (or $[-1, 1]$). %by min-max normalization.
This range enables mapping $\tilde{x}_{t}$ to polar coordinates by $\phi_{t}=\text{arccos}(\tilde{x}_{i})$, with a radius $r=t/N$ encoding the time stamp, where $N$ is a constant factor to regularize the span of the polar coordinates. %system. Then,
Two types of GAF, Gramian Sum Angular Field (GASF) and Gramian Difference Angular Field (GADF) are defined as
\begin{equation}\label{eq.gaf}
\small
\begin{aligned}
&\text{GASF:}~~\text{cos}(\phi_{t} + \phi_{t'})=x_{t}x_{t'} - \sqrt{1 - x_{t}^{2}}\sqrt{1 - x_{t'}^{2}}\\
&\text{GADF:}~~\text{sin}(\phi_{t} - \phi_{t'})=x_{t'}\sqrt{1 - x_{t}^{2}} - x_{t}\sqrt{1 - x_{t'}^{2}}
\end{aligned}
\end{equation}
which exploits the pairwise temporal correlations in the UTS. Thus, the outcome is a $T\times T$ matrix $\mat{G}$ with $\mat{G}_{t,t'}$ specified by either type in Eq.~\eqref{eq.gaf}. A GAF image is a heatmap on $\mat{G}$ with both axes representing time, as illustrated by Fig. \ref{fig.tsimage}(e).

% Invertible.

% \vspace{0.2cm}

\subsection{Recurrence Plot (RP)}\label{sec.rp}

%RP \cite{eckmann1987recurrence} is a method to encode a UTS into an image that aims to capture the periodic patterns in the UTS by using its reconstructed {\em phase space}. The phase space of a UTS $[x_{1}, ..., x_{T}]$ can be reconstructed by {\em time delay embedding}, which is a set of new vectors $\mat{v}_{1}$, ..., $\mat{v}_{l}$ with

RP \cite{eckmann1987recurrence} encodes a UTS into an image that captures its periodic patterns by using its reconstructed {\em phase space}. The phase space of %a UTS %$[x_{1}, ..., x_{T}]$
$\mat{x}\in\mathbb{R}^{1\times T}$ can be reconstructed by {\em time delay embedding} -- a set of new vectors $\mat{v}_{1}$, ..., $\mat{v}_{l}$ with
\begin{equation}\label{eq.de}
\small
\begin{aligned}
\mat{v}_{t}=[x_{t}, x_{t+\tau}, x_{t+2\tau}, ..., x_{t+(m-1)\tau}]\in\mathbb{R}^{m\tau},~~~1\le t \le l
\end{aligned}
\end{equation}
where $\tau$ is the time delay, $m$ is the dimension of the phase space, both %of which
are hyperparameters. Hence, $l=T-(m-1)\tau$. With vectors $\mat{v}_{1}$, ..., $\mat{v}_{l}$, an RP image %is constructed by measuring
measures their pairwise distances, results in an $l\times l$ image whose element
\begin{equation}\label{eq.rp}
\small
\begin{aligned}
\text{RP}_{i,j}=\Theta(\varepsilon - \|\mat{v}_{i} - \mat{v}_{j}\|),~~~1\le i,j\le l
\end{aligned}
\end{equation}
where $\Theta(\cdot)$ is the Heaviside step function, $\varepsilon$ is a threshold, and $\|\cdot\|$ is a norm function such as $\ell_{2}$ norm. Eq.~\eqref{eq.rp} %states RP produces a heatmap image on a binary matrix with $\text{RP}_{i,j}=1$ if $\mat{v}_{i}$ and $\mat{v}_{j}$ are sufficiently similar.
generates a binary matrix with $\text{RP}_{i,j}=1$ if $\mat{v}_{i}$ and $\mat{v}_{j}$ are sufficiently similar, producing a black-white image ({\em e.g.}, Fig. \ref{fig.tsimage}(f)).% ({\em e.g.}, a periodic pattern).

An advantage of RP is its flexibility in image size by tuning $m$ and $\tau$. Thus it has been used for time series classification %\cite{cao2021image},
\cite{silva2013time,hatami2018classification}, forecasting \cite{li2020forecasting}, anomaly detection \cite{lin2024hierarchical} and %feature-wise
explanation \cite{kim2024cafo}. Moreover, the method in \cite{hatami2018classification}, and similarly in HCR-AdaAD \cite{lin2024hierarchical}, omit the thresholding in Eq.~\eqref{eq.rp} and uses $\|\mat{v}_{i} - \mat{v}_{j}\|$ to produce continuously valued images %in a classification task
to avoid information loss.


% \vspace{0.2cm}

\subsection{Other Methods}\label{sec.othermethod}

%There are some less commonly used methods. For example, in
Additionally, %there are some peripheral methods. %In addition to GAF,
\cite{wang2015encoding} introduces Markov Transition Field (MTF) for imaging a UTS. %$\mat{x}\in\mathbb{R}^{1\times T}$. 
%MTF first assigns each $x_{t}$ to one of $Q$ quantile bins, then builds a $Q\times Q$ Markov transition matrix $\mat{M}$ {\em s.t.} $\mat{M}_{i,j}$ represents the frequency %with which
%of the case when a point $x_{t}$ in the $i$-th bin is followed by a point $x_{t'}$ in the $j$-th bin, {\em i.e.}, $t=t'+1$. Matrix $\mat{M}$ serves as the input of a heatmap image.
MTF is a matrix $\mat{M}\in\mathbb{R}^{Q\times Q}$ encoding the transition probabilities over time segments, where $Q$ is the number of segments. %Moreover,
ImagenTime \cite{naiman2024utilizing} stacks the delay embeddings $\mat{v}_{1}$, ..., $\mat{v}_{l}$ in Eq.~\eqref{eq.de} to an $l\times m\tau$ matrix for visualizing UTSs. %It also uses a variant of STFT.
% The method in \cite{homenda2024time} introduces five different 2D images by counting, rearranging, replicating the values in a UTS. 
MSCRED \cite{zhang2019deep} uses heatmaps on the $d\times d$ correlation matrices of MTSs with $d$ variates for anomaly detection. 
Furthermore, some methods use a mixture of imaging methods by stacking different transformations. \cite{wang2015imaging} stacks GASF, GADF, MTF to a 3-channel image. %Similarly,
FIRTS \cite{costa2024fusion} builds a 3-channel image by stacking GASF, MTF and RP. %the GASF, MTF, RP representations of each UTS.
%\cite{jin2023classification} combines Line Plot with Constant-Q Transform (CQT) \cite{brown1991calculation}, a method related to wavelet transform ($\S$\ref{sec.spectrogram}), to generate 2-channel images.
The mixture methods encode a UTS with multiple views and were found more robust than single-view images in these works for %time series
classification tasks.

\subsection{How to Model MTS}\label{sec.modelmts}

In the above methods, Heatmap ($\S$\ref{sec.heatmap}) can be %directly
used to visualize the %2D
variate-time matrices, $\mat{X}$, of MTSs ({\em e.g.}, Fig. \ref{fig.structure}(b)), where correlated variates %are better to
should be spatially close to each other. Line Plot ($\S$\ref{sec.lineplot}) can be used to visualize MTSs by plotting all variates in the same image \cite{wimmer2023leveraging,daswani2024plots} or combining all univariate images to compose a bigger %1-channel
image \cite {li2023time}, but these methods only work for a small number of variates. Spectrogram ($\S$\ref{sec.spectrogram}), GAF ($\S$\ref{sec.gaf}), and RP ($\S$\ref{sec.rp}) were designed specifically for UTSs. For these methods and Line Plot, which are not straightforward %for MTS transformation,
in imaging MTSs, the general approaches %to use them %for MTS
include using channel independence assumption to model each variate individually \cite{nie2023time}, %like VisionTS \cite{chen2024visionts},
or stacking the images of $d$ variates to form a $d$-channel image %as did by
\cite{naiman2024utilizing,kim2024cafo}. %\cite{prithyani2024feasibility,naiman2024utilizing,kim2024cafo}.
However, the latter does not fit some vision models pre-trained on RGB images which requires 3-channel inputs (more discussions are deferred to $\S$\ref{sec.processing}).

\vspace{0.2cm}

\noindent{\bf Remark.} As a summary, Table \ref{tab.tsimage} recaps the salient advantages and limitations of the five primary imaging methods that are introduced in this section.

% \hh{can we have a table (e.g., rows are different imaging methods and columns are a few desirable propoerties) or a short paragraph to discuss/summarize/compare the strenths and weakness of different imaging methods for ts? This might bring some structure/comprehension to this section (as opposed to, e.g., some reviewer might complain that what we do here is a laundry list)}

\section{Imaged Time Series Modeling}\label{sec.model}

With image representations, time series analysis can be readily performed with vision models. This section discusses such solutions from %traditional vision models %($\S$\ref{sec.cnns})
%to the recent large vision models %($\S$\ref{sec.lvms})
%and large multimodal models.% ($\S$\ref{sec.lmms}).
the traditional models to the SOTA models.

\begin{figure*}[!t]
\centering
\includegraphics[width=0.9\textwidth]{fig/fig_2.pdf}
% \vspace{-1em}
\caption{An illustration of different modeling strategies on imaged time series in (a)(b)(c) and task-specific heads in (d).}\label{fig.models}
\vspace{-0.2cm}
\end{figure*}

\subsection{Conventional Vision Models}\label{sec.cnns}

%Similar to
Following traditional %methods on
image classification, \cite{silva2013time} applies a K-NN classifier on the RPs of time series, \cite{cohen2020trading} applies an ensemble of fundamental classifiers such as %linear regression, SVM, Ada Boost, {\em etc.}
SVM and AdaBoost on the Line Plots %images
for time series classification. As an image encoder, %a typical encoder, %of images,
CNNs have been %extensively
widely used for learning image representations. %\cite{he2016deep}.
Different from using 1D CNNs on sequences %UTS or MTS
\cite{bai2018empirical}, %regular
2D or 3D CNNs can be applied on imaged time series as shown in Fig. \ref{fig.models}(a). %to learn time series representations by encoding their image transformations.
For example, %standard
regular CNNs have been used on Spectrograms \cite{du2020image}, tiled CNNs have been used on GAF images \cite{wang2015encoding,wang2015imaging}, dilated CNNs have been used on Heatmap images \cite{yazdanbakhsh2019multivariate}. More frequently, ResNet \cite{he2016deep}, Inception-v1 \cite{szegedy2015going}, and VGG-Net \cite{simonyan2014very} have been used on Line Plots \cite{jin2023classification,semenoglou2023image}, Heatmap images \cite{zeng2021deep}, RP images \cite{li2020forecasting,kim2024cafo}, GAF images \cite{barra2020deep,kaewrakmuk2024multi}, 
% Heatmaps \cite{zeng2021deep}, RPs \cite{li2020forecasting,kim2024cafo}, GAFs \cite{barra2020deep,kaewrakmuk2024multi},
and even a mixture of GAF, MTF and RP images \cite{costa2024fusion}. In particular, for time series generation tasks, %a diffusion model with U-Nets \cite{naiman2024utilizing} and GAN frameworks of CNNs \cite{li2022tts,karami2024timehr} have also been explored.%investigated.
GAN frameworks of CNNs \cite{li2022tts,karami2024timehr} and a diffusion model with U-Nets \cite{naiman2024utilizing} have also been explored.

Due to their small to medium sizes, these models are often trained from scratch using task-specific training data. %per task using the task's training set. %of time series images.
Meanwhile, fine-tuning {\em pre-trained vision models}  %such as those pre-trained on ImageNet, %\cite{deng2009imagenet}, 
have already been found promising in cross-modality knowledge transfer for time series anomaly detection \cite{namura2024training}, forecasting \cite{li2020forecasting} and classification \cite{jin2023classification}.

% \cite{li2020forecasting} uses ImageNet pretrained CNNs.

\subsection{Large Vision Models (LVMs)}\label{sec.lvms}

Vision Transformer (ViT) \cite{dosovitskiy2021image} has %given birth to
inspired the development of %some
modern LVMs %large vision models (LVMs)
such as %DeiT \cite{touvron2021training}, 
Swin \cite{liu2021swin}, BEiT \cite{bao2022beit}, and MAE \cite{he2022masked}. %Given an input image, ViT splits it
As Fig. \ref{fig.models}(b) shows, ViT splits an %input
image into {\em patches} of fixed size, then embeds each patch and augments it with a positional embedding. The %resulting
vectors of patches are processed by a Transformer %encoder
as if they were token embeddings. Compared to CNNs, ViTs are less data-efficient, but have higher capacity. %Consequently,
Thus, %the
{\em pre-trained} ViTs have been explored for modeling %the images of time series.
imaged time series. For example, AST \cite{gong2021ast} fine-tunes DeiT \cite{touvron2021training} on the filterbank spetrogram of audios %signals
for classification tasks and finds %using
ImageNet-pretrained DeiT is remarkably effective in knowledge transfer. The fine-tuning paradigm has also been %similarly
adopted in \cite{zeng2023pixels,li2023time} but with different pre-trained models %initializations
such as Swin by \cite{li2023time}. 
VisionTS \cite{chen2024visionts} %explains
attributes %the superiority of LVMs
LVMs' superiority over LLMs in knowledge transfer %over LLMs %as an outcome of
to the small gap between the pre-trained images and imaged time series. %the patterns learned from the large-scale pre-trained images and the patterns in the images of time series.
It %also
finds that with one-epoch fine-tuning, MAE becomes the SOTA time series forecasters on %many
some benchmark datasets.

Similar to %build
time series foundation models %\cite{das2024decoder,goswami2024moment,ansari2024chronos,shi2024time}, %such as TimesFM \cite{das2024decoder}, MOMENT \cite{goswami2024moment}, Chronos \cite{ansari2024chronos} and Time-MoE \cite{shi2024time},
such as TimesFM \cite{das2024decoder}, %and MOMENT \cite{goswami2024moment}, 
there are some initial efforts in pre-training ViT architectures with imaged time series. Following AST, SSAST \cite{gong2022ssast} introduced a %joint discriminative and generative
%masked spectrogram patch prediction self-supervised learning framework
masked spectrogram patch prediction framework for pre-training ViT on a large dataset -- AudioSet-2M. Then it becomes a backbone of some follow-up works such as AST-SED \cite{li2023ast} for sound event detection. %To be effective for UTSs,
For UTSs, ViTime \cite{yang2024vitime} generates a large set of Line Plots of synthetic UTSs for pre-training ViT, which was found superior over TimesFM in zero-shot forecasting tasks on benchmark datasets.

\subsection{Large Multimodal Models (LMMs)}\label{sec.lmms}

%As Large Multimodal Models (LMMs)
As LMMs %are getting
get growing attentions, some %of the
notable LMMs, such as LLaVA \cite{liu2023visual}, Gemini \cite{team2023gemini}, GPT-4o \cite{achiam2023gpt} and Claude-3 \cite{anthropic2024claude}, have been explored to consolidate the power of LLMs %on time series
and LVMs in time series analysis. 
Since LMMs support multimodal input via prompts, methods in this thread typically prompt LMMs with the textual and imaged representations of time series, %textual representation of time series and their %image transformations, transformed images,
%then instruct LMMs
and instructions on what tasks to perform ({\em e.g.}, Fig. \ref{fig.models}(c)).

InsightMiner \cite{zhang2023insight} is a pioneer work that uses the LLaVA architecture to generate %textual descriptions about
texts describing the trend of each input UTS. It extracts the trend of a UTS by Seasonal-Trend decomposition, encodes the Line Plot of the trend, and concatenates the embedding of the Line Plot with the embeddings of a textual instruction, which includes a sequence of numbers representing the UTS, {\em e.g.}, ``[1.1, 1.7, ..., 0.3]''. The concatenated embeddings are taken by a language model for generating trend descriptions. %It also fine-tunes a few layers with the generated texts to align LLaVA checkpoints with time series domain.
Similarly, \cite{prithyani2024feasibility} adopts the LLaVA architecture, but for MTS classification. An MTS is encoded by %a sequence of
the visual %token
embeddings of the stacked Line Plots of all variates. %meanwhile
%The method also stacks
%The time series of all variate are also stacked in a prompt % of all variates in a prompt
The matrix of the MTS is also verbalized in a prompt 
as the textual modality. %By manipulating token embeddings,
By integrating token embeddings, both %of these %works propose to
methods fine-tune some layers of the LMMs with some synthetic data.

Moreover, zero-shot and in-context learning performance of several commercial LMMs have been evaluated for audio classification \cite{dixit2024vision}, anomaly detection \cite{zhuang2024see}, and some synthetic tasks \cite{daswani2024plots}, where the image %({\em e.g.}, spectrograms, Line Plots)
and textual representations of a query %UTS or MTS
time series are integrated into a prompt. For in-context learning, these methods inject the images of a few example time series and their labels ({\em e.g.}, classes) %({\em e.g.}, classes, normal status)
into an instruction to prompt LMMs for assisting the prediction of the query time series.

\subsection{Task-Specific Heads}\label{sec.task}

%With the image embedding of a time series, the next step is to produce its prediction.
For classification tasks, most of the methods in Table \ref{tab.taxonomy} adopt a fully connected (FC) layer or multilayer perceptron (MLP) to transform an embedding into a probability distribution over all classes. For forecasting tasks, there are two approaches: (1) using a $d_{e}\times W$ MLP/FC layer to directly predict (from the $d_{e}$-dimensional embedding) the time series values in a future time window of size $W$ \cite{li2020forecasting,semenoglou2023image}; (2) predicting the pixel values that represent the future part of the time series and then recovering the time series from the predicted image \cite{yang2023your,chen2024visionts,yang2024vitime} ($\S$\ref{sec.processing} discusses the recovery methods). Imputation and generation tasks resemble forecasting %in the sense of predicting
as they also predict time series values. Thus approach (2) has been used for imputation \cite{wang2015imaging} and generation \cite{naiman2024utilizing,karami2024timehr}. %LMMs have been used for classification, text generation, and anomaly detection. For these tasks,
When using LMMs for classification, text generation, and anomaly detection, most of the methods prompt LMMs to produce the desired outputs in textual answers, circumventing task-specific heads \cite{zhang2023insight,dixit2024vision,zhuang2024see}.

%Forecasting: MLP, FC to predict numerical values using embeddings. Imputation of images (TSH). Classification: MLP, FC using embeddings.

\section{Pre-Processing and Post-Processing}\label{sec.processing}

To be successful in using vision models, some subtle design desiderata %to be considered
include {\bf time series normalization}, {\bf image alignment} and {\bf time series recovery}.

\vspace{0.2cm}

\noindent{\bf Time Series Normalization.} Vision models are usually trained on %images after Gaussian normalization (GN).
standardized images. To be aligned, the images introduced in $\S$\ref{sec.tsimage} should be normalized with a controlled mean and standard deviation, as did by \cite{gong2021ast} on spectrograms. In particular, as Heatmap is built on raw time series values, the commonly used Instance Normalization (IN) \cite{kim2022reversible} can be applied on the time series as suggested by VisionTS \cite{chen2024visionts} since IN share similar merits as Standardization. %although min-max normalization was used by \cite{karami2024timehr,zeng2021deep}.
Using Line Plot requires a proper range of y-axis. In addition to rescaling time series %by min-max or GN
\cite{zhuang2024see}, ViTST \cite{li2023time} introduced several methods to remove extreme values from the plot. GAF requires min-max normalization on its input, as it transforms time series values withtin $[0, 1]$ to polar coordinates ({\em i.e.}, arccos). In contrast, input to RP is usually normalization-free as an $\ell_{2}$ norm is involved in Eq.~\eqref{eq.rp} before thresholding.%for a comparison with a threshold.

\vspace{0.2cm}

\noindent{\bf Image Alignment.} When using pre-trained models, it is imperative to fit the image size to the input requirement of the models. This is especially true for Transformer based models as they use a fixed number of positional embeddings to encode the spacial information of image patches. For 3-channel RGB images such as Line Plot, it is straightforward to meet a pre-defined size by adjusting the resolution when producing the image. For images built upon matrices such as Heatmap, Spectrogram, GAF, RP, the number of channels and matrix size need adjustment. For the channels, one method is to duplicate a matrix to 3 channels \cite{chen2024visionts}, another way is to average the weights of the 3-channel patch embedding layer into a 1-channel layer \cite{gong2021ast}. For the image size, bilinear interpolation is a common method to resize input images \cite{chen2024visionts}. Alternatively, AST \cite{gong2021ast} %use cut and bilinear interpolation on
resizes the positional embeddings instead of the images to fit the model to a desired input size. However, the interpolation in these methods may either alter the time series or the spacial information in positional embeddings.

% single-channel (UTS), RGB channel (UTS), duplicate channels (UTS), multi-channel (MTS).

%Bilinear interpolation.

%Correlated variates are better to be spatially close to each other.

%\subsection{Pre-training}

\vspace{0.2cm}

\noindent{\bf Time Series Recovery.} As stated in $\S$\ref{sec.task}, tasks such as forecasting, imputation and generation requires predicting time series values. For models that predict pixel values of images, post-processing involves recovering time series from the predicted images. Recovery from Line Plots is tricky, it requires locating pixels that %correspond to
represent time series and mapping them back to the original values. This can be done by manipulating a grid-like Line Plot as introduced in \cite{yang2023your,yang2024vitime}, which has a recovery function. In contrast, recovery from Heatmap is straightforward as it directly stores the predicted time series values \cite{zeng2021deep,chen2024visionts}. Spectrogram is underexplored in these tasks and it remains open on how to recover time series from it. The existing work \cite{zeng2023pixels} uses Spectrogram for forecasting only with an MLP head that directly predicts time series. %predicts time series values.
GAF supports accurate recovery by an inverse mapping from polar coordinates to normalized time series \cite{wang2015imaging}. However, RP lost time series information during thresholding (Eq.~\ref{eq.rp}), thus may not fit recovery-demanded tasks without using an {\em ad-hoc} prediction head.


% Line Plot was regarded as matrices with rows and columns for mapping in \cite{sood2021visual}.


%\section{Tasks and Time Series Recovery}

%\subsection{Task-Specific Head}

% \subsection{Time Series Recovery}




\paragraph{Present Survey.}
Facing these challenges, there is an increasing need for focused research on LLMs and their interaction with human instructions and intentions. This paper systematically analyzes LLMs' performance in processing human instructions, highlighting three key areas: user instruction understanding, intention comprehension and reasoning, and reliable content generation. While existing review papers address model training, fine-tuning, and specific aspects of LLMs' capabilities~\cite{lou2024large,plaat2024reasoning,huang2024survey}, our focus is on the LLMs' ability to understand and reason about user intentions. Specifically, we explore how well LLMs understand user input, reason about the user's intention, and generate content that aligns with human intentions, minimizing hallucinations and ensuring credibility and ethical consistency.



\paragraph{Comparison with Previous Surveys.} 
While the gap between human intention and LLMs is a core challenge in generative AI, many studies focus on specific aspects of the issue, lacking a comprehensive overview. These works offer valuable insights but do not provide a systematic summary of the field. Lou et al.~\cite{lou2024large} primarily address instruction following challenges in LLMs without delving into the reasoning capabilities for complex user instructions. Gao et al. analyze the four stages of human-machine LLM interaction (planning, facilitation, iteration, and testing) but overlook LLM's understanding of user instructions. Xu et al.~\cite{xu2024knowledge} examine the impact of various memory conflicts on LLM-generated content credibility and performance, yet do not consider reasoning or intention comprehension. Plaat et al.~\cite{plaat2024reasoning} focus on LLM Reasoning for basic mathematical problems, without exploring its applicability to broader fields. Shorinwa et al.~\cite{shorinwa2024survey} provide an initial analysis of LLMs uncertainty quantification, but exclude user input instructions. In contrast, our survey offers a more comprehensive perspective, with a unique classification and systematic analysis of instruction processing, addressing current solutions to key challenges. 



\paragraph{Survey Organization.} 
As in Figure \ref{fig:texonomy}, we begin by exploring the capability of user instruction understanding (\S\ref{sec:challenge1}). Next, we analyze intention comprehension and reasoning, focusing on how models infer implicit intentions, incorporate contextual information for logical reasoning, and address inconsistencies or incomplete instructions (\S\ref{sec:challenge2}). We then examine reliable content generation, assessing the quality and credibility of model-generated outputs (\S\ref{sec:challenge3}). Following this, we review benchmarks that address these core challenges (\S\ref{sec:Benchmark}). Finally, we propose potential research directions (\S\ref{sec:Directions}) and summarize the key findings (\S\ref{sec:Conclusion}).



\begin{figure}[h]
	% \setlength{\abovecaptionskip}{-0.2cm}
	\setlength{\belowcaptionskip}{-0.3cm}
	\begin{center}
		%\fbox{\rule{0pt}{2in} \rule{1\linewidth}{0pt}}
		\includegraphics[width=1\linewidth]{picture/understanding.pdf}
	\end{center}
	\caption{Common understanding issues in LLMs include: (A) \textbf{Remote Information Failure (\S\ref{sec:long})}, where the model forgets relevant information over long distances in long context (A.1). \textbf{(B) Incorrect Relevance Judgment(\S\ref{sec:multi})}, such as the model incorrectly associates wrong content (B.1-2.nd) from the previous turn (B.1-1.st).}
	\label{fig:understanding}
\end{figure}


\section{Instruction Understanding}\label{sec:challenge1}

LLMs excel at single-turn dialogues, but struggle to understand multi-turn dialogues and long-contexts, which are commonly used by users. LLMs may forget prior information, be influenced by irrelevant data, and overlook key inputs. 
% These issues arise from the complexity of multi-round interactions, involving sparse information, variable intents, and frequent pronouns.

\subsection{Long-Text Comprehension}\label{sec:long}

Challenges remain in understanding long context and this paper classifies the factors into three categories as follows:\textbf{1) Information Sparsity and Redundancy}. Long texts often contain redundant or irrelevant information that can obscure the task-relevant content, leading to difficulties in information extraction and potential hallucinations. \textbf{2) Remote Information Failure} (Figure \ref{fig:understanding} A.1). Long contexts may cause models to forget relevant information that is distant within the text. Additionally, links between remote information across paragraphs or sentences can be difficult for models to identify, diminishing their understanding of contextual connections. \textbf{3) Attention Dilution}. As context length increases, the model’s attention mechanism faces greater computational demands and struggles to assign appropriate weights to each token, making it harder to prioritize key information, particularly with complex, multi-level relationships in longer texts. This paper classifies the existing solutions into the following two categories:

\paragraph{Information Focusing.}
Improving LLM's ability to focus on important information in long texts involves several methods: 1) Sparsifying attention to concentrate on critical information~\cite{beltagy2020longformer}. 
% 2) Dynamically adjusting attention based on the current task~\cite{wu2021smart}. 
2) Optimizing attention to minimize redundancy and emphasize core content~\cite{chen2024core}. 
3) Training with location-independent tasks to enhance the ability to search and react to relevant information in long contexts~\cite{he2024never}.

\paragraph{Multipath Optimization.}
Various methods can enhance LLMs on long-context tasks: 1) Pre-training with extended context windows and reinforcement learning for fine-tuning to optimize long-context understanding~\cite{zhang2024longreward}. 
2) Combining retrieval-based models with generative models on long-context tasks~\cite{li2024retrieval}. 
3) Leveraging cyclic sequence models' linear scaling property for better inference efficiency~\cite{gu2023mamba}. 
% 4) Using external memory to store and retrieve long-context information~\cite{liu2024memlong}. 
% 5) Mimicking brain memory hierarchies to improve long-context processing efficiency~\cite{he2024hmt}.


\subsection{Multi-Turn Conversation Handling}\label{sec:multi}

This paper categorizes the challenges faced by existing LLMs when understanding multi-turn conversations into three categories, as follows: \textbf{1) Capability Weakening.} 
% LLMs exhibit a reduced understanding of user intent in multi-turn scenarios. 
Current supervised instruction fine-tuning (SIFT) and RLHF may even impair multi-turn capabilities\cite{wang2023mint}, with models struggling on complex reasoning tasks that span multiple rounds, such as those requiring evidence collection and conclusions \cite{banatt2024wilt}. 
Additionally, multi-turn dialogs increase the vulnerability of LLMs to adversarial attacks, 
where malicious users can mask harmful intentions across multiple rounds, 
leading to the generation of misleading or harmful content \cite{agarwal2024prompt}. \textbf{2) Error Propagation.} Instruction comprehension errors accumulate across rounds, leading to an escalating failure rate in subsequent responses \cite{he2024multi}, which may snowball into larger issues such as biased or incorrect outputs \cite{fan2024fairmt}. \textbf{3) Incorrect Relevance Judgment} (Figure \ref{fig:understanding} B.1) LLMs often struggle to identify relevant content in multi-turn dialogs, failing to properly link content from previous rounds or to discern ellipsis and implicit meaning inherent in user commands \cite{sun2024parrot}.


% To solve above challenges, we categorize existing solutions into the following two types: supervised fine-tuning methods based on multi-turn dialogue data, with further improvements such as optimizing instruction parsing \cite{teng2024fine} and context-aware preference strategies \cite{sun2024parrot}. Reinforcement learning methods specifically improved for multi-turn dialogue, with targeted improvements such as hierarchical reinforcement learning strategies \cite{DBLP:conf/icml/ZhouZPLK24}.
To solve above challenges, This paper categorizes existing solutions into two types: supervised fine-tuning methods using multi-turn dialogue data, enhanced by techniques like optimized instruction parsing \cite{teng2024fine} and context-aware preference strategies \cite{sun2024parrot}; and reinforcement learning methods tailored for multi-turn dialogue, with improvements such as hierarchical reinforcement learning \cite{DBLP:conf/icml/ZhouZPLK24}.

\section{Intention Reasoning}\label{sec:challenge2}

User instructions often lack clarity due to language ambiguities. While humans can infer intention, LLMs struggle with misinterpreting ambiguous inputs, leading to errors. 
% These challenges arise from knowledge gaps and the need for models to understand underlying meanings. 
We explore causes and solutions for intention errors, focusing on inconsistent instructions, misinformation, fuzzy language, and intention clarification.

\begin{figure*}[h]
	% \setlength{\abovecaptionskip}{-0.05cm}
	\setlength{\belowcaptionskip}{-0.3cm}
	\begin{center}
		%\fbox{\rule{0pt}{2in} \rule{1\linewidth}{0pt}}
		\includegraphics[width=1\linewidth]{picture/figure3.pdf}
	\end{center}
	\caption{Common reasoning issues in LLMs include: \textbf{(A) Inconsistent Instruction Reasoning (\S\ref{sec:inconsistent})}, where LLMs fail to detect conflicting inputs (A.1) or overlooking logical inconsistencies (A.2), \textbf{(B) Misinformation Reasoning (\S\ref{sec:misinformation})}, caused by temporal misalignment leading to outdated responses (B.1) or data contamination resulting in misleading outputs (B.2). \textbf{(C) Fuzzy Language Interpretation (\S\ref{sec:fuzzy})}, where the model relies on biases for fuzzy queries (C.1) or defaults to a response without seeking clarification (C.2). and \textbf{(D) Intention Clarification Failure (\S\ref{sec:intention})} , where it misinterprets sarcasm (D.1) or ignores prior emotional context (D.2).}
	\label{fig:reason}
\end{figure*}

\subsection{Inconsistent Instruction Reasoning}\label{sec:inconsistent}
In natural language communication, humans easily identify inconsistencies using context and prior knowledge, whereas LLMs struggle, often accept contradictory inputs, and generate unreliable answers.
\textit{This phenomenon has been observed across multiple question-answering generation tasks~\cite{li2023contradoc,zheng2022cdconv}, and we categorize the causes of this problem according to the scenarios in which it occurs as follows:} \textbf{1) Ignoring input errors} (Figure \ref{fig:reason} A.1). The model ignores the input errors and gives an answer, resulting in the model assigning the same weight to each context given by the user, which in turn affects the generation of the answer. \textbf{2) Inability to detect user inconsistencies} (Figure \ref{fig:reason} A.2). In the premise that the model has learned the knowledge, the model still has difficulty detecting user inconsistencies. 
 To address inconsistent instruction reasoning issues, existing solutions primarily adopt the following two approaches: 

\paragraph{Knowledge Updating.} SituatedQA~\cite{DBLP:conf/emnlp/ZhangC21} attempts to enhance model performance by updating the knowledge base. 
% ContraDoc~\cite{li2023contradoc} trains models to identify contradictions in long texts using human-annotated datasets. 
Additionally, CDConv~\cite{zheng2022cdconv} simulates common user behaviors to trigger chatbots through an automated dialogue generation method, generating contradictions for training purposes. 

\paragraph{Confidence Calibration.} Given the high cost of data annotation and model fine-tuning, some researchers have sought alternative approaches by introducing additional processing techniques. CD2~\cite{jin2024tug} maximizes probabilistic output and calibrates model confidence under knowledge conflicts using conflict decoupling and comparison decoding methods. MacNoise~\cite{hong2023so} enhances contradiction retrieval in an augmented generative system by explicitly fine-tuning a discriminator or prompting LLMs to improve contradiction discrimination. 


\subsection{Misinformation Reasoning}\label{sec:misinformation}
Erroneous instructions mislead model outputs more severely than inconsistent ones, as they lack obvious contradictions, requiring the model to comprehend, reason, compare input knowledge with its parameterized knowledge, and make objective judgments~\cite{DBLP:conf/emnlp/CheangCW0LS0C23,DBLP:conf/acl/XuLYZS0FX024}. \textit{From the input perspective, this paper classifies the sources of erroneous information into two categories as follows:}
\textbf{1) Temporal Alignment Failure} (Figure \ref{fig:reason} B.1). arises when the knowledge provided by the user and the model is temporally misaligned due to updates occurring at different times, leading to inconsistent responses. Such discrepancies typically originate during the training process.  
\textbf{2) Information Contamination} (Figure \ref{fig:reason} B.2). refers to the degradation of model quality caused by the intentional distortion of input data.  

To solve the above problems, existing methods mainly focus on improving model susceptibility in the face of internal and external knowledge conflicts through targeted fine-tuning and processing. CKL~\cite{DBLP:conf/iclr/JangYYSHKCS22} ensures that the model's knowledge is updated in a timely manner through an online approach, although this approach is slightly weaker than re-training in terms of effectiveness~\cite{DBLP:conf/icml/LiskaKGTSAdSZYG22}.
% , and RKC-LLMs~\cite{DBLP:journals/corr/abs-2310-00935} allows a large model to recognize knowledge conflicts by means of instructional fine-tuning that identifying specific passages of conflicting information. 
BIPIA~\cite{yi2023benchmarking} used adversarial training to combat the effects of information pollution and improve model robustness. CAR~\cite{DBLP:conf/eacl/WellerKWLD24} achieved nearly 20\% improvement by discriminating external knowledge that may not be contaminated in the RAG system.


\subsection{Fuzzy Language Interpretation}
\label{sec:fuzzy}
% When the user's instructions contain ambiguous words (polysemy, ambiguity, etc.), the large language model may choose the wrong interpretation among multiple possible interpretations, and thus fail to accurately identify the user's true intention. Because the large language model is trained to maximize the prediction accuracy of the training data, it generates the most likely next word or sentence based on the input text. The large model does not realize the need to clarify or take proactive actions to obtain more information. Betty et al.~\cite{hou2024large} also pointed out that LLMs may inadvertently provide misleading answers, especially when dealing with ambiguous or polysemous instructions.
When user instructions contain fuzzy terms (e.g., polysemy or vagueness), LLMs may select an incorrect interpretation from multiple possibilities, potentially leading to misleading responses.
\textit{This phenomenon has been observed across multiple information-seeking tasks~\cite{kim2024aligning}, and we categorize the causes of this problem according to the scenarios in which it occurs as follows:}
\textbf{1) Self-defined problem} (Figure \ref{fig:reason} C.1).  When the user inputs content with fuzzy sentences, LLMs may choose to generate content based on the preferences of its own training data. 
\textbf{2) Select data based on fuzzy input} (Figure \ref{fig:reason} C.2). In response to ambiguous user input, LLMs may select a default explanation without actively asking the user to clarify.


To solve the problem, researchers have started to experiment with cue engineering and relation alignment. Folkscope~\cite{DBLP:conf/acl/YuWLBSLG0Y23} proposed the FolkScope framework, which uses a large language model to analyze and discriminate users' fuzzy purchasing intention.
% Miko~\cite{lu2024miko} proposes a multimodal hierarchical intention generation framework that analyzes the potential intentions behind users' posting behaviors based on the fuzzt information they post on social platforms.
Miko~\cite{lu2024miko} introduces a multimodal hierarchical intention generation framework that interprets users' posting behaviors by analyzing the fuzzy information they share on social platforms.
% Zhang et al.~\cite{DBLP:conf/acl/0002LJ24} have employed behavioral cloning by using demonstration data from a strong model to train a weaker model so that the weaker model can perform better on a similar task. 
ATC~\cite{DBLP:conf/emnlp/DengLC0LC23} utilizes the Active Thinking Chain cueing scheme, which enhances the proactivity of a biglanguage model by adding goal-planning capabilities to the descriptive reasoning chain.


\subsection{Intention Clarification Failure}\label{sec:intention}
% The reasoning of a LLMs relies on the training data it receives and the internal structure of the model itself. Although a LLMs can learn from the data, this is not the same as “real world” common sense and contextual understanding. Real human beings can reason based on their real-life experience and common sense, but LLMs lack this experience-based reasoning ability, and it is difficult to speculate on complex contexts outside of the input data. Meanwhile the LLMs is likely to fail to maintain a consistent reasoning trajectory all the time when spanning longer texts, complex contexts, or multiple rounds of conversations, especially when complex intentions or changes in sentiment are involved, the LLMs may fail to effectively memorize the sentiment contexts of the previous text, and may not be able to correctly reason out the user's implicit needs. 
Unlike humans, who reason based on experience, LLMs lack real-world common sense and thus struggle to infer complex contexts beyond their input data. Moreover, they often fail to maintain a consistent reasoning trajectory across long texts, complex contexts, or multi-turn conversations. When handling intricate intentions or sentiment shifts, LLMs may struggle to retain prior context, leading to errors in inferring implicit user needs.
\textit{We categorize the causes of this problem according to the scenarios in which it occurs as follows:} \textbf{1) Fails to detect sarcasm} (Figure \ref{fig:reason} D.1), when LLMs fails to understand the sarcastic intention of the user. \textbf{2) Ignores prior emotional context} (Figure \ref{fig:reason} D.2),  when LLMs focus only on the second half of the sentence and ignore the emotions in the previous round of dialog.

To solve above problems, researchers have started to try to construct multi-domain datasets containing implicit intentions to strengthen the ability of the LLMs to reason about complex intentions and user emotions in multi-round interaction scenarios. 
DeepSeek-R1~\cite{guo2025deepseek} enhances its understanding of human intention through a structured process with two RL stages for refining reasoning patterns and aligning with human preferences.
S1~\cite{muennighoff2025s1} uses budget forcing to control the number of thinking tokens. The upper limit is terminated early by a delimiter, while the lower limit prohibits delimiters and adds "wait" to guide in-depth reasoning and optimize the quality of the answer.
SoulChat~\cite{chen2023soulchat} fine-tuned LLMs by constructing a dataset containing more than 2 million samples of multi-round empathic conversations. 
MoChat~\cite{DBLP:journals/corr/abs-2410-11404} constructs multi-round dialogues for spatial localization by using joint grouping spatio-temporal localization.
% LARA~\cite{DBLP:journals/corr/abs-2403-16504} combines a fine-tuned smaller model with retrieval enhancement mechanisms and integrates it into the architecture of LLM. 

\section{Realiable Generation}\label{sec:challenge3}

Despite strong performance, LLMs struggle with output reliability. Trained on large corpora using maximum likelihood estimation, they generate deterministic responses. While effective on familiar data, they often produce unstable or incomplete responses to unseen inputs, undermining reliability. We analyze LLM response stability and alignment in depth.


\subsection{Response Stability}\label{sec:response}
The knowledge acquired by the LLMs is generally determined in the pre-training stage and stored in a parameterized form. For data in a specific field, the current model is generally optimized by fine-tuning instructions so that it outputs what humans want~\cite{radford2019language}. If knowledge samples that the model has not seen are used in instruction-tuning, it will inevitably cause the model to give a definite response to unknown inputs, and there is a high probability that an answer will be fabricated. \textit{This is the over-confidence of the model that causes the model to output unreliable answers, and we categorize the causes of this problem according to the scenarios in which it occurs as follows:}
\textbf{1) Fabricated incorrect information} (Figure \ref{fig:gene} A.1). When the model’s knowledge did not match the input question, it fabricated information that did not match the facts. 
\textbf{2) Incorrect context output} (Figure \ref{fig:gene} A.2). When the model’s knowledge did match the input question, it output incorrect context information.
To address these issues, researchers have explored uncertainty, which quantifies the credibility and stability of model outputs as follows.

\begin{figure}[t]
    \setlength{\belowcaptionskip}{-0.2cm}
	\begin{center}
        \includegraphics[width=1\linewidth]{picture/figure4.pdf}
	\end{center}
	\caption{Common generation issues in LLMs include: (A) \textbf{Unstable Content Generation (\S\ref{sec:response})}, where the model fabricates details when it lacks relevant knowledge (A.1) or produces incorrect contextual information despite having relevant knowledge (A.2). \textbf{(B) Misalignment with Human Values (\S\ref{sec:ailgn})}, where the model generates harmful or offensive content (B.1) or provides responses that conflict with moral and ethical standards (B.2).}
	\label{fig:gene}
\end{figure}

\paragraph{Fine-tuning LLMs.} To make LLMs more accurate in estimating uncertainty, existing methods fine-tune  models~\cite{sensoy2018evidential,amini2020deep}. LUQ~\cite{zhang2024luq} is a novel sampling-based uncertainty quantification method specifically designed for long texts. 
% UaIT~\cite{kuhn2023semantic} using semantic entropy to assess output uncertainty. 
ConformalFactuality~\cite{mohri2024language} defines the associated uncertainties for each possible output.

\paragraph{External Tools.} Fine-tuning LLMs typically demands substantial computing resources and slow training; therefore, reducing computational overhead is crucial for improving efficiency. Researcher has proposed methods to evaluate the uncertainty of model outputs through external tools~\cite{liu2024uncertainty}. 
% ConfidenceElicitation~\cite{xiong2023can} is a new uncertainty measurement tool for large model outputs. 
CalibrateMath~\cite{lin2022teaching} assesses uncertainty by requiring models to generate numerical answers with confidence levels, evaluating their reliability.

\subsection{Alignment}\label{sec:ailgn}
Despite the impressive capabilities of large language models (LLMs), they have raised significant concerns regarding the unsafe or harmful content they may generate. LLMs are typically trained on vast datasets scraped from the internet, including inappropriate or harmful content~\cite{10.1145/3442188.3445922}.\textit{ This means that the models may inadvertently produce outputs misaligned with human values as follows:} \textbf{1) Generation of Toxic Content} (Figure \ref{fig:gene} B.1). LLMs may generate toxic content, such as hate speech or offensive comments, when asked to respond to sensitive topics~\cite{DBLP:conf/acl/LuongLNN24,DBLP:conf/ijcai/DuttaKDK24}.
\textbf{2) Conflicts with Moral/Ethical Standards} (Figure \ref{fig:gene} B.2). LLMs might produce outputs that conflict with moral or ethical standards, such as guiding illegal activities~\cite{ramezani-xu-2023-knowledge,abdulhai-etal-2024-moral}. 
To tackle the previously mentioned concerns regarding unsafe or harmful content produced by LLMs, researchers have focused on various stages:

\paragraph{Pretraining Data Cleaning and Curation.} To minimize the risks associated with harmful or inappropriate content, LLM training datasets should undergo rigorous cleaning processes~\cite{10.1145/3442188.3445922}, such as filtering out toxic language, hate speech, and harmful stereotypes. 
% Tools like the Perspective API~\cite{DBLP:conf/coling/ChengK022} and word embedding debiasing methods~\cite{rakshit-etal-2025-prejudice} can help identify and remove toxic and biased content.
Tools like word embedding debiasing methods~\cite{rakshit-etal-2025-prejudice} can help identify and remove toxic and biased content.


\begin{table*}[]
\caption{A selection of widely used benchmark datasets for evaluating LLMs. The 'Type' column categorizes the benchmarks based on task types, while 'Lang.' indicates the language of the dataset, and 'Citation' denotes the number of citations each benchmark has received.}
\label{tab:benchmark}
\resizebox{\linewidth}{!}{
\begin{tabular}{@{}llcclr@{}}
\toprule
\textbf{Type} & \textbf{Benchmark} & \textbf{Year} & \textbf{Lang.} & \textbf{Description} & \textbf{Citation} \\ \midrule
\multirow{2}{*}{LLM Understanding} 
& $\infty${B}ench~\cite{zhang-etal-2024-bench} & 2024 & Zh\&En & Evaluation of long-context handling in LLMs. & 1 \\
& MT-BENCH-101~\cite{bai-etal-2024-mt} & 2024 & En & Evaluation of multi-turn dialogue capability in LLMs. & 39 \\ \midrule

\multirow{4}{*}{LLM Reasoning} 
& BIPIA~\cite{yi2023benchmarking} & 2023 & En & Assesses LLMs vulnerability to hint injection attacks. & 60 \\
& Miko~\cite{lu2024miko} & 2024 & En & Evaluates multimodal LLMs in understanding social intention. & 2 \\
& CONTRADOC~\cite{li2023contradoc} & 2023 & En & Detects self-contradictions in long documents. & 7 \\
& CDCONV~\cite{zheng2022cdconv} & 2022 & Zh & Detects contradictions in Chinese dialogues. & 12 \\ \midrule

\multirow{3}{*}{LLM Generation} 
& Open-LLM-Leaderboard~\cite{ye2024benchmarking} & 2024 & En & Assesses LLMs uncertainty in generated content. & 19 \\
& ETHICS Dataset~\cite{hendrycks2020aligning} & 2020 & En & Evaluates LLMs on moral reasoning and ethics. & 468 \\
& FACTOR~\cite{muhlgay2023generating} & 2023 & En & Measures factual accuracy in LLM-generated text. & 71 \\ 
\bottomrule
\end{tabular}
}
\end{table*}


\paragraph{Reinforcement Learning-based Alignment.} To further align LLMs with human values and societal norms, reinforcement learning methods, such as RLHF and its advanced variants, such as PPO~\cite{ouyang2022training}, DPO~\cite{DBLP:conf/icml/ZengLMYZW24}, and GRPO~\cite{DBLP:journals/corr/abs-2402-03300}, are very essential. As an extension of RLHF, RLAIF~\cite{DBLP:conf/icml/0001PMMFLBHCRP24} leverages AI systems to assist in the feedback process, making evaluation and fine-tuning more scalable.
% and enabling continuous monitoring and refinement of the model’s behavior~\cite{DBLP:conf/icml/0001PMMFLBHCRP24}.

\paragraph{In-context Alignment.} In-context alignment leverages the ability of LLMs to adapt their responses based on a few examples provided in the prompt.~\cite{DBLP:conf/iclr/LinRLDSCB024} demonstrates that effective alignment can be achieved purely through ICL with just a few stylistic examples and a system prompt.~\cite{huang-etal-2024-far} explored the effectiveness of different components of In-context alignment, and found that examples within the context are crucial for enhancing alignment capabilities.
%~\cite{liu-etal-2024-take} introduced PICA, which uses a two-stage approach to improve alignment efficiency.


\section{Benchmark}\label{sec:Benchmark}

This section covers benchmarks for LLMs in instruction understanding, reasoning, and generation (\Cref{tab:benchmark}).


\subsection{Benchmarking Instruction Understanding}

Instruction understanding in LLMs involves extracting key information, maintaining coherence, and adapting to dynamic conversation changes, especially in long or multi-round dialogues.
Zhang et al. \cite{zhang-etal-2024-bench} provides a benchmark consisting of 12 tasks, with an average data length exceeding 100K tokens, designed to evaluate LLMs' ability to handle long contexts. The results show that the performance of LLMs significantly declines when dealing with long contexts, suggesting further improvements.
MT-BENCH-101 \cite{bai-etal-2024-mt} constructs a three-layer hierarchical capability classification and 1,388 dialogue pairs across 13 tasks, providing a comprehensive evaluation of LLMs' fine-grained capabilities in multi-turn dialogue. Experimental results show that commonly used techniques do not significantly enhance multi-turn capabilities.


\subsection{Benchmarking LLM Reasoning}

LLM Reasoning involves inferring user intentions by interpreting both explicit and implicit language cues.
BIPIA~\cite{yi2023benchmarking} evaluates LLMs under indirect hint injection attacks across five scenarios and 250 targets, revealing vulnerabilities in all models, with GPT-3.5-turbo and GPT-4 exhibiting notably higher susceptibilities.
Miko~\cite{lu2024miko} assesses multimodal models in understanding social media user intentions. The benchmark, which includes 979 social media entries, shows that multimodal LLMs outperform text-only models like LLama2-7B and GLM4. Incorporating image data enhances the model’s ability to interpret user intentions, improving accuracy.
CONTRADOC~\cite{li2023contradoc} is the first dataset for analyzing self-contradictions in long documents. Evaluation of GPT-4, PaLM2, and other LLMs on this dataset reveals that while GPT-4 outperforms others and even surpasses human performance, it still struggles with complex contradictions requiring nuanced reasoning.
CDCONV~\cite{zheng2022cdconv} focuses on contradiction detection in Chinese dialogues, containing over 12,000 dialogue rounds. The study shows that the Hierarchical method consistently outperforms others in detecting contradictions, highlighting the importance of accurate contextual modeling in dialogue understanding.



\subsection{Benchmarking LLM Generation}
LLM Generation assesses a model's ability to understand user instructions, avoid fabricating false information, and generate accurate, contextually appropriate responses.
Open-LLM-Leaderboard~\cite{ye2024benchmarking} introduces a novel benchmark that integrates uncertainty quantification to evaluate the reliability of content generation across tasks like QA, comprehension, and dialogue. Results show that larger models often exhibit greater uncertainty, and fine-tuned models tend to have higher uncertainty despite higher accuracy.
ETHICS~\cite{hendrycks2020aligning} evaluates whether generated content aligns with human ethical values, such as justice and well-being. The study finds that while models like GPT-3 show promise in predicting human moral judgments, they still need improvement in this domain.
FACTOR~\cite{muhlgay2023generating} addresses the evaluation of factuality in LLMs by providing a scalable method that ensures diverse and rare facts are considered. Testing models such as GPT-2 and GPT-Neo show that, while benchmark scores correlate with perplexity, they better reflect factuality in open-ended generation, especially when retrieval augmentation is applied.

\section{Future Directions}\label{sec:Directions}
This section summarizes ongoing challenges in instruction understanding, reasoning, and reliable generation with LLMs and outlines potential future research directions.

\paragraph{Automated Annotation Framework.} Although LLMs excel in general-domain tasks, they often produce hallucinated or incomplete content in specialized fields due to limited domain-specific training data. While contextual learning and instruction fine-tuning methods have been explored to address this issue, manual data annotation remains labor-intensive and prone to quality inconsistencies. An automated annotation framework could streamline data labeling, enhancing model performance in specialized fields by ensuring higher quality and scalability of domain-specific training datasets.

\paragraph{GraphRAG.} 
LLMs have shown impressive language generation capabilities through pre-training on large datasets, but their reliance on static data often results in inaccurate or fictional content, particularly in domain-specific tasks. The Graph-enhanced generation approach aims to tackle this by leveraging KGs and GNNs for precise knowledge retrieval. Despite its advantages, GraphRAG faces challenges in capturing structural information during graph reasoning tasks and struggles with multi-hop retrieval accuracy and conflict resolution between external and internal knowledge. Future work should focus on refining retrieval strategies and improving the stability and accuracy of GraphRAG in complex tasks.

\paragraph{Quantifying Uncertainty in LLMs.} 
As LLMs like LLama and ChatGPT have revolutionized content generation, ensuring the reliability of their outputs remains a critical challenge. Uncertainty quantification is a promising approach to address this, enabling models to provide a confidence assessment alongside their responses. Evidence learning~\cite{sensoy2018evidential,amini2020deep}, an emerging method in uncertainty representation, offers a reliable approach for quantifying uncertainty directly from data. However, its application to LLMs is computationally intensive, and most existing work focuses on small-scale models. Future research should aim to optimize uncertainty quantification for large-scale models efficiently.


\paragraph{Balancing Safety and Performance.} Although advancements in alignment techniques have improved factual accuracy and safety, they often come at the cost of the model's creativity and fluency. Striking a balance between safety and performance is crucial. Future research should explore new alignment methods that ensure both the safety and usability of LLMs, optimizing the trade-off between generating reliable, safe content and maintaining the model's creative and contextual capabilities.


\section{Conclusion}\label{sec:Conclusion}
This paper analyzes LLMs' performance in processing user instructions. Despite progress in natural language understanding, LLMs struggle with complex, inconsistent instructions, often resulting in biases, errors, and hallucinations. Improvements through prompt engineering, knowledge retrieval, model expansion, fine-tuning, and RLHF have not fully addressed LLMs' limitations in reasoning and comprehension, limiting their real-world applicability.
We identify three key bottlenecks: instruction comprehension, reasoning ability, and reliable content generation. Future research should focus on enhancing reasoning for complex instructions and aligning outputs with user intent, to improve LLMs' adaptability and reliability in real-world tasks.

% \clearpage

%% The file named.bst is a bibliography style file for BibTeX 0.99c
\bibliographystyle{named}
\bibliography{ijcai25}

\end{document}

