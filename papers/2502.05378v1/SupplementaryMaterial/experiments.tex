\begin{figure}
    \centering
    \includegraphics[width=\textwidth]{figures/dataset.png}
    \caption{\textbf{More maps from our dataset.} Rows from top to bottom represent increasing scene complexity, categorized into four levels: Simple, Normal, Hard, Insane.}
    \label{fig:dataset_full}
\end{figure}
\section{Experiments}
% \begin{table*}
% \centering
% \begin{adjustbox}{width=\textwidth}
% \begin{tabular}{lcc|ccccccc|ccccccc}
% & & & \multicolumn{7}{c|}{\textbf{Comp. (\%) $\uparrow$}} & \multicolumn{7}{c}{\textbf{Comp. (cm) $\downarrow$}} \\
% Scene & Rooms & Kfs & Random & FBE & UPEN & OccAnt & ANM & \textbf{NBP (ours)} & & Random & FBE & UPEN & OccAnt & ANM & \textbf{NBP (ours)} & \\
% \toprule 
% GdvgF* & 6 & 32 & 68.45 & 81.78 & 82.39 & 80.24 & 80.99 & \textbf{87.80} & & 11.67 & 5.48 & 5.14 & 5.66 & 5.69 & \textbf{4.92} &  \\
% gZ6f7 & 1 & 48 & 29.81 & 81.01 & 82.96 & 82.02 & 80.68 & \textbf{89.91} & & 46.48 & 7.06 & 6.14 & 6.19 & 7.43 & \textbf{3.31}  & \\
% HxpKQ* & 8 & 32 & 46.93 & 58.71 & 52.70 & 60.50 & 48.34 & \textbf{66.28} & & 19.10 & 11.75 & 14.11 & 11.75 & 15.96 & \textbf{8.12} & \\
% pLe4w & 2 & 52 & 32.92 & 66.09 & 66.76 & 67.13 & \textbf{76.41} & 71.34 & & 30.79 & 12.78 & 11.82 & 11.51 & \textbf{8.03} & 9.53 & \\
% YmJkq & 4 & 68 & 50.26 & 68.32 & 60.47 & 68.70 & 79.35 & \textbf{81.57} & & 24.61 & 11.85 & 15.77 & 11.90 & 8.46 & \textbf{8.01} & \\
% mean & 4 & 46 & 45.67 & 68.53 & 69.09 & 71.72 & 73.15 & \textbf{79.38} & & 26.53 & 9.78 & 10.60 & 9.40 & 9.11 & \textbf{6.78} & \\
% \bottomrule
% \end{tabular}%
% \end{adjustbox}
% \caption{Evaluation results for each test scenes on MP3D}
% \label{tab:comparison-mp3d}
% \end{table*}
\begin{table*}
\centering
\caption{Evaluation results for each test scene on MP3D dataset.}
\begin{adjustbox}{width=\textwidth}
\fontsize{14}{19}\selectfont  % 这里设置字体大小为11pt,行距为13pt
\begin{tabular}{lc|ccccccc|ccccccc}
\toprule
& & \multicolumn{7}{c|}{\textbf{Comp. (\%) $\uparrow$}} & \multicolumn{7}{c}{\textbf{Comp. (cm) $\downarrow$}} \\
Scene & Rooms & Random & FBE & UPEN & OccAnt & ANM & \textbf{NBP (ours)} & & Random & FBE & UPEN & OccAnt & ANM & \textbf{NBP (ours)} & \\
\midrule 
GdvgF* & 6 & 68.45 & 81.78 & 82.39 & 80.24 & 80.99 & \textbf{87.80} & & 11.67 & 5.48 & 5.14 & 5.66 & 5.69 & \textbf{4.92} &  \\
gZ6f7 & 1 & 29.81 & 81.01 & 82.96 & 82.02 & 80.68 & \textbf{89.91} & & 46.48 & 7.06 & 6.14 & 6.19 & 7.43 & \textbf{3.31}  & \\
HxpKQ* & 8 & 46.93 & 58.71 & 52.70 & 60.50 & 48.34 & \textbf{66.28} & & 19.10 & 11.75 & 14.11 & 11.75 & 15.96 & \textbf{8.12} & \\
pLe4w & 2 & 32.92 & 66.09 & 66.76 & 67.13 & \textbf{76.41} & 71.34 & & 30.79 & 12.78 & 11.82 & 11.51 & \textbf{8.03} & 9.53 & \\
YmJkq & 4 & 50.26 & 68.32 & 60.47 & 68.70 & 79.35 & \textbf{81.57} & & 24.61 & 11.85 & 15.77 & 11.90 & 8.46 & \textbf{8.01} & \\
mean & 4 & 45.67 & 71.18 & 69.06 & 71.72 & 73.15 & \textbf{79.38} & & 26.53 & 9.78 & 10.60 & 9.40 & 9.11 & \textbf{6.78} & \\
\bottomrule
\end{tabular}%
\end{adjustbox}

\label{tab:comparison-mp3d}
\end{table*}

\noindent \textbf{Detailed quantitative results.} Table~\ref{tab:doom_test_training_1} and Table.~\ref{tab:doom_test_training_2} show our superior performance on both the AiMDoom training set and the test set. Furthermore, we offer detailed results for each test scene in MP3D, as illustrated in Table~\ref{tab:comparison-mp3d}.

\noindent \textbf{Qualitative results.} 
We provide additional visual comparisons between our method and the state-of-the-art NBV-based method: MACARONS~\citep{guedon2023macarons}, from Figure.~\ref{fig:comparison1}, Figure.~\ref{fig:comparison3} and Figure.\ref{fig:comparison4}. These comparisons demonstrate that our trajectories consistently exhibit superior performance, whether in simple or complex scenarios. Both methods start from the same pose, indicated by a deep blue colour in the visualizations of trajectories.



% Table 1: ExDoom Simple and Normal
\begin{table*}
\centering
\small
\setlength{\tabcolsep}{3.5pt}
\begin{tabular}{@{}lcccccccc@{}}
\toprule
& \multicolumn{4}{c}{\textbf{AiMDoom Simple}} & \multicolumn{4}{c}{\textbf{AiMDoom Normal}} \\
\cmidrule(lr){2-5} \cmidrule(l){6-9}
& \multicolumn{2}{c}{Seen} & \multicolumn{2}{c}{Unseen} & \multicolumn{2}{c}{Seen} & \multicolumn{2}{c}{Unseen} \\
\cmidrule(lr){2-3} \cmidrule(lr){4-5} \cmidrule(lr){6-7} \cmidrule(l){8-9}
& Final Cov. & AUC & Final Cov. & AUC & Final Cov. & AUC & Final Cov. & AUC \\
\midrule
Random Walk & 0.362 & 0.306 & 0.323 & 0.270 & 0.198 & 0.159 & 0.190 & 0.152 \\
 & {\scriptsize ±0.175} & {\scriptsize ±0.156} & {\scriptsize ±0.156} & {\scriptsize ±0.135} & {\scriptsize ±0.125} & {\scriptsize ±0.104} & {\scriptsize ±0.124} & {\scriptsize ±0.103} \\
FBE & 0.770 & 0.628 & 0.760 & 0.605 & 0.564 & 0.423 & 0.565 & 0.415 \\
 & {\scriptsize ±0.163} & {\scriptsize ±0.147} & {\scriptsize ±0.174} & {\scriptsize ±0.171} & {\scriptsize ±0.171} & {\scriptsize ±0.127} & {\scriptsize ±0.139} & {\scriptsize ±0.109} \\
SCONE & 0.597 & 0.482 & 0.577 & 0.483 & 0.421 & 0.315 & 0.412 & 0.313 \\
 & {\scriptsize ±0.177} & {\scriptsize ±0.158} & {\scriptsize ±0.173} & {\scriptsize ±0.138} & {\scriptsize ±0.138} & {\scriptsize ±0.102} & {\scriptsize ±0.114} & {\scriptsize ±0.087} \\
MACARONS & 0.600 & 0.483 & 0.599 & 0.479 & 0.442 & 0.332 & 0.418 & 0.314 \\
 & {\scriptsize ±0.176} & {\scriptsize ±0.145} & {\scriptsize ±0.200} & {\scriptsize ±0.172} & {\scriptsize ±0.135} & {\scriptsize ±0.104} & {\scriptsize ±0.120} & {\scriptsize ±0.088} \\
\textbf{NBP (Ours)} & \textbf{0.870} & \textbf{0.697} & \textbf{0.879} & \textbf{0.692} & \textbf{0.746} & \textbf{0.538} & \textbf{0.734} & \textbf{0.526} \\
 & {\scriptsize {±0.121}} & {\scriptsize {±0.134}} & {\scriptsize {±0.142}} & {\scriptsize {±0.156}} & {\scriptsize {±0.152}} & {\scriptsize {±0.142}} & {\scriptsize {±0.142}} & {\scriptsize {±0.112}} \\
\bottomrule
\end{tabular}
\caption{{Evaluation results on AiMDoom dataset (Simple and Normal)}.}
\label{tab:doom_test_training_1}
\end{table*}

% Table 2: ExDoom Hard and Insane
\begin{table*}[h!]
\centering
\small
\setlength{\tabcolsep}{3.5pt}
\begin{tabular}{@{}lcccccccc@{}}
\toprule
& \multicolumn{4}{c}{\textbf{AiMDoom Hard}} & \multicolumn{4}{c}{\textbf{AiMDoom Insane}} \\
\cmidrule(lr){2-5} \cmidrule(l){6-9}
& \multicolumn{2}{c}{Seen} & \multicolumn{2}{c}{Unseen} & \multicolumn{2}{c}{Seen} & \multicolumn{2}{c}{Unseen} \\
\cmidrule(lr){2-3} \cmidrule(lr){4-5} \cmidrule(lr){6-7} \cmidrule(l){8-9}
& Final Cov. & AUC & Final Cov. & AUC & Final Cov. & AUC & Final Cov. & AUC \\
\midrule
Random Walk & 0.121 & 0.086 & 0.124 & 0.088 & 0.070 & 0.048 & 0.074 & 0.050 \\
 & {\scriptsize ±0.081} & {\scriptsize ±0.062} & {\scriptsize ±0.082} & {\scriptsize ±0.060} & {\scriptsize ±0.049} & {\scriptsize ±0.038} & {\scriptsize ±0.048} & {\scriptsize ±0.035} \\
FBE & 0.426 & 0.310 & 0.425 & 0.311 & 0.313 & 0.226 & 0.330 & 0.239 \\
 & {\scriptsize ±0.119} & {\scriptsize ±0.091} & {\scriptsize ±0.114} & {\scriptsize ±0.080} & {\scriptsize ±0.082} & {\scriptsize ±0.066} & {\scriptsize ±0.097} & {\scriptsize ±0.079} \\
SCONE & 0.271 & 0.199 & 0.290 & 0.210 & 0.204 & 0.146 & 0.196 & 0.140 \\
 & {\scriptsize ±0.100} & {\scriptsize ±0.172} & {\scriptsize ±0.093} & {\scriptsize ±0.072} & {\scriptsize ±0.069} & {\scriptsize ±0.052} & {\scriptsize ±0.079} & {\scriptsize ±0.060} \\
MACARONS & 0.316 & 0.202 & 0.302 & 0.218 & 0.201 & 0.143 & 0.192 & 0.139 \\
 & {\scriptsize ±0.106} & {\scriptsize ±0.074} & {\scriptsize ±0.097} & {\scriptsize ±0.070} & {\scriptsize ±0.068} & {\scriptsize ±0.051} & {\scriptsize ±0.078} & {\scriptsize ±0.058} \\
\textbf{NBP (Ours)} & \textbf{0.627} & \textbf{0.430} & \textbf{0.618} & \textbf{0.432} & \textbf{0.486} & \textbf{0.315} & \textbf{0.472} & \textbf{0.312} \\
 & {\scriptsize {±0.144}} & {\scriptsize {±0.111}} & {\scriptsize {±0.153}} & {\scriptsize {±0.115}} & {\scriptsize {±0.106}} & {\scriptsize {±0.047}} & {\scriptsize {±0.095}} & {\scriptsize {±0.073}} \\
\bottomrule
\end{tabular}
\caption{{Evaluation results on AiMDoom dataset (Hard and Insane)}.}
\label{tab:doom_test_training_2}
\end{table*}





% \begin{table}
% \centering
% \small
% \setlength{\tabcolsep}{4pt}
% \begin{adjustbox}{width=\textwidth}
% \begin{tabular}{lcccc|cccc|cccc|cccc}
% \toprule
% & \multicolumn{4}{c}{\textbf{ExDoom Simple}} & \multicolumn{4}{c}{\textbf{ExDoom Normal}} & \multicolumn{4}{c}{\textbf{ExDoom Hard}} & \multicolumn{4}{c}{\textbf{ExDoom Insane}} \\[-0.3ex]
% \cmidrule(lr){2-5} \cmidrule(lr){6-9} \cmidrule(lr){10-13} \cmidrule(lr){14-17}
% & \multicolumn{2}{c}{\textbf{Seen}} & \multicolumn{2}{c}{\textbf{Unseen}} & \multicolumn{2}{c}{\textbf{Seen}} & \multicolumn{2}{c}{\textbf{Unseen}} & \multicolumn{2}{c}{\textbf{Seen}} & \multicolumn{2}{c}{\textbf{Unseen}} & \multicolumn{2}{c}{\textbf{Seen}} & \multicolumn{2}{c}{\textbf{Unseen}} \\[-0.3ex]
% \cmidrule(lr){2-3} \cmidrule(lr){4-5} \cmidrule(lr){6-7} \cmidrule(lr){8-9} \cmidrule(lr){10-11} \cmidrule(lr){12-13} \cmidrule(lr){14-15} \cmidrule(lr){16-17}


% & \multicolumn{1}{c}{\textbf{FC}} & \multicolumn{1}{c}{\textbf{AUC}} & \multicolumn{1}{c}{\textbf{FC}} & \multicolumn{1}{c}{\textbf{AUC}} & \multicolumn{1}{c}{\textbf{FC}} & \multicolumn{1}{c}{\textbf{AUC}} & \multicolumn{1}{c}{\textbf{FC}} & \multicolumn{1}{c}{\textbf{AUC}} & \multicolumn{1}{c}{\textbf{FC}} & \multicolumn{1}{c}{\textbf{AUC}} & \multicolumn{1}{c}{\textbf{FC}} & \multicolumn{1}{c}{\textbf{AUC}} & \multicolumn{1}{c}{\textbf{FC}} & \multicolumn{1}{c}{\textbf{AUC}} & \multicolumn{1}{c}{\textbf{FC}} & \multicolumn{1}{c}{\textbf{AUC}} \\ [-0.3ex]
% \midrule
% \multirow{2}{*}{\minitab[r]{Random Walk}} & 0.362 & 0.306 & 0.323 & 0.270 & 0.198 & 0.159 & 0.190 & 0.152 & 0.121 & 0.086 & 0.124 & 0.088 & 0.070 & 0.048 & 0.074 & 0.050 \\
% & \scriptsize{±0.175} & \scriptsize{±0.156} & \scriptsize{±0.156} & \scriptsize{±0.135} & \scriptsize{±0.125} & \scriptsize{±0.104} & \scriptsize{±0.124} & \scriptsize{±0.103} & \scriptsize{±0.081} & \scriptsize{±0.062} & \scriptsize{±0.082} & \scriptsize{±0.060} & \scriptsize{±0.049} & \scriptsize{±0.038} & \scriptsize{±0.048} & \scriptsize{±0.035} \\[0.5ex]
% \multirow{2}{*}{\minitab[r]{FBE}} & 0.770 & 0.628 & 0.760 & 0.605 & 0.564 & 0.423 & 0.565 & 0.415 & 0.426 & 0.310 & 0.425 & 0.311 & 0.313 & 0.226 & 0.330 & 0.239 \\
% & \scriptsize{±0.163} & \scriptsize{±0.147} & \scriptsize{±0.174} & \scriptsize{±0.171} & \scriptsize{±0.171} & \scriptsize{±0.127} & \scriptsize{±0.139} & \scriptsize{±0.109} & \scriptsize{±0.119} & \scriptsize{±0.091} & \scriptsize{±0.114} & \scriptsize{±0.080} & \scriptsize{±0.082} & \scriptsize{±0.066} & \scriptsize{±0.097} & \scriptsize{±0.079} \\[0.5ex]
% \multirow{2}{*}{\minitab[r]{SCONE}} & 0.597 & 0.482 & 0.577 & 0.483 & 0.421 & 0.315 & 0.412 & 0.313 & 0.271 & 0.199 & 0.290 & 0.210 & 0.204 & 0.146 & 0.196 & 0.140 \\
% & \scriptsize{±0.177} & \scriptsize{±0.158} & \scriptsize{±0.173} & \scriptsize{±0.138} & \scriptsize{±0.138} & \scriptsize{±0.102} & \scriptsize{±0.114} & \scriptsize{±0.087} & \scriptsize{±0.100} & \scriptsize{±0.172} & \scriptsize{±0.093} & \scriptsize{±0.072} & \scriptsize{±0.069} & \scriptsize{±0.052} & \scriptsize{±0.079} & \scriptsize{±0.060} \\[0.5ex]
% \multirow{2}{*}{\minitab[r]{MACARONS}} & 0.600 & 0.483 & 0.599 & 0.479 & 0.442 & 0.332 & 0.418 & 0.314 & 0.316 & 0.202 & 0.302 & 0.218 & 0.201 & 0.143 & 0.192 & 0.139 \\
% & \scriptsize{±0.176} & \scriptsize{±0.145} & \scriptsize{±0.200} & \scriptsize{±0.172} & \scriptsize{±0.135} & \scriptsize{±0.104} & \scriptsize{±0.120} & \scriptsize{±0.088} & \scriptsize{±0.106} & \scriptsize{±0.074} & \scriptsize{±0.097} & \scriptsize{±0.070} & \scriptsize{±0.068} & \scriptsize{±0.051} & \scriptsize{±0.078} & \scriptsize{±0.058} \\

% \multirow{2}{*}{\minitab[r]{\textbf{NBP (Ours)}}} & \textbf{0.870} & \textbf{0.697} & \textbf{0.879} & \textbf{0.692} & \textbf{0.746} & \textbf{0.538} & \textbf{0.734} & \textbf{0.526} & \textbf{0.627} & \textbf{0.430} & \textbf{0.618} & \textbf{0.432} & \textbf{0.486} & \textbf{0.315} & \textbf{0.472} & \textbf{0.312} \\
% & \scriptsize{\textbf{±0.121}} & \scriptsize{\textbf{±0.134}} & \scriptsize{\textbf{±0.142}} & \scriptsize{\textbf{±0.156}} & \scriptsize{\textbf{±0.152}} & \scriptsize{\textbf{±0.142}} & \scriptsize{\textbf{±0.142}} & \scriptsize{\textbf{±0.112}} & \scriptsize{\textbf{±0.144}} & \scriptsize{\textbf{±0.111}} & \scriptsize{\textbf{±0.153}} & \scriptsize{\textbf{±0.115}} & \scriptsize{\textbf{±0.106}} & \scriptsize{\textbf{±0.047}} & \scriptsize{\textbf{±0.095}} & \scriptsize{\textbf{±0.073}} \\
% \bottomrule
% \end{tabular}
% \end{adjustbox}
% \caption{\textbf{Evaluation results on AiMDoom dataset}. FC refers to the metric of Final Coverage.}
% \label{tab:doom_test_training}
% \end{table}



% \begin{figure}[htbp]
%     \centering
%     \begin{minipage}{0.48\textwidth}
%         \centering
%         \includegraphics[width=\textwidth]{Supplementary Material/macarons_3.png}
%         \caption{MACARONS}
%         \label{fig:first_image}
%     \end{minipage}
%     \hfill
%     \begin{minipage}{0.48\textwidth}
%         \centering
%         \includegraphics[width=\textwidth]{Supplementary Material/ours_3.png}
%         \caption{NBP (Ours)}
%         \label{fig:second_image}
%     \end{minipage}
%     \label{fig:both_images}
% \end{figure}


% \begin{figure}[htbp]
%     \centering
%     \begin{minipage}{0.48\textwidth}
%         \centering
%         \includegraphics[width=\textwidth]{Supplementary Material/macarons_4.png}
%         \caption{MACARONS}
%         \label{fig:first_image}
%     \end{minipage}
%     \hfill
%     \begin{minipage}{0.48\textwidth}
%         \centering
%         \includegraphics[width=\textwidth]{Supplementary Material/ours_4.png}
%         \caption{NBP (Ours)}
%         \label{fig:second_image}
%     \end{minipage}
%     \label{fig:both_images}
% \end{figure}

% \begin{figure}[htbp]
%     \centering
%     \begin{minipage}{0.48\textwidth}
%         \centering
%         \includegraphics[width=\textwidth]{Supplementary Material/macarons_8.png}
%         \caption{MACARONS}
%         \label{fig:first_image}
%     \end{minipage}
%     \hfill
%     \begin{minipage}{0.48\textwidth}
%         \centering
%         \includegraphics[width=\textwidth]{Supplementary Material/ours_8.png}
%         \caption{NBP (Ours)}
%         \label{fig:second_image}
%     \end{minipage}
%     \label{fig:both_images}
% \end{figure}

% \begin{figure}[htbp]
%     \centering
%     \begin{minipage}{0.48\textwidth}
%         \centering
%         \includegraphics[width=\textwidth]{Supplementary Material/macarons_11.png}
%         \caption{MACARONS}
%         \label{fig:first_image}
%     \end{minipage}
%     \hfill
%     \begin{minipage}{0.48\textwidth}
%         \centering
%         \includegraphics[width=\textwidth]{Supplementary Material/ours_11.png}
%         \caption{NBP (Ours)}
%         \label{fig:second_image}
%     \end{minipage}
%     \label{fig:both_images}
% \end{figure}

% \begin{figure}[htbp]
%     \centering
%     \begin{minipage}{0.48\textwidth}
%         \centering
%         \includegraphics[width=\textwidth]{Supplementary Material/macarons_12.png}
%         \caption{MACARONS}
%         \label{fig:first_image}
%     \end{minipage}
%     \hfill
%     \begin{minipage}{0.48\textwidth}
%         \centering
%         \includegraphics[width=\textwidth]{Supplementary Material/ours_12.png}
%         \caption{NBP (Ours)}
%         \label{fig:second_image}
%     \end{minipage}
%     \label{fig:both_images}
% \end{figure}

%



\begin{figure}[!htbp]
    \centering
    \begin{subfigure}{0.48\textwidth}
        \centering
        \includegraphics[width=\textwidth]{Supplementary Material/macarons_3.png}
        \caption{MACARONS}
    \end{subfigure}
    \hfill
    \begin{subfigure}{0.48\textwidth}
        \centering
        \includegraphics[width=\textwidth]{Supplementary Material/ours_3.png}
        \caption{NBP (Ours)}
    \end{subfigure}
    \caption{\textbf{Comparison 1}: In complex and narrow spaces, the NBV (next-best-view) based method can easily get trapped in a local area. Although our method did not manage to reconstruct all areas in this complex scene, it covered most of the areas.}
    \label{fig:comparison1}
\end{figure}

\begin{figure}[!htbp]
    \centering
    \begin{subfigure}{0.48\textwidth}
        \centering
        \includegraphics[width=\textwidth]{Supplementary Material/macarons_8.png}
        \caption{MACARONS}
    \end{subfigure}
    \hfill
    \begin{subfigure}{0.48\textwidth}
        \centering
        \includegraphics[width=\textwidth]{Supplementary Material/ours_8.png}
        \caption{NBP (Ours)}
    \end{subfigure}
    \caption{\textbf{Comparison 2}: The NBV-based method can easily "assume" that an area has been fully explored, as it focuses solely on local optimal solutions, similar to this sample. In complex indoor environments, it is often necessary to skip some locally optimal poses.}
    \label{fig:comparison3}
\end{figure}

\begin{figure}[!htbp]
    \centering
    \begin{subfigure}{0.48\textwidth}
        \centering
        \includegraphics[width=\textwidth]{Supplementary Material/macarons_11.png}
        \caption{MACARONS}
    \end{subfigure}
    \hfill
    \begin{subfigure}{0.48\textwidth}
        \centering
        \includegraphics[width=\textwidth]{Supplementary Material/ours_11.png}
        \caption{NBP (Ours)}
    \end{subfigure}
    \caption{\textbf{Comparison 3}: In relatively simple scenes with some obstacles, the NBV exploration can also become trapped in one area.}
    \label{fig:comparison4}
\end{figure}
\clearpage 
As Figure.~\ref{fig:comparison5} and Figure.~\ref{fig:comparison6} illustrated, we also show that in very complex environments, we could only achieve about 65\% coverage. This is because, in complex environments, our method prioritizes the exploration of areas with multiple valuable goals, ignoring places of lesser current value. After the initial exploration is complete, it is likely to explore other regions, overlooking previously encountered areas with higher value. Consequently, developing methods that aim to achieve a global optimum is a promising and valuable direction for future research.


\begin{figure}[!htbp]
    \centering
    \begin{subfigure}{0.48\textwidth}
        \centering
        \includegraphics[width=\textwidth]{Supplementary Material/fail_gt1.png}
        \caption{Ground truth mesh}
    \end{subfigure}
    \hfill
    \begin{subfigure}{0.48\textwidth}
        \centering
        \includegraphics[width=\textwidth]{Supplementary Material/fail_1.png}
        \caption{NBP (Ours)}
    \end{subfigure}
    \caption{\textbf{Failure case 1}: Our method initially prioritizes the exploration of high-value areas, inadvertently neglecting regions of secondary importance. Thus, it results in incomplete reconstruction in the initial area of the beginning trajectory.}
    \label{fig:comparison5}
\end{figure}
\begin{figure}[!htbp]
    \centering
    \begin{subfigure}{0.48\textwidth}
        \centering
        \includegraphics[width=\textwidth]{Supplementary Material/fail_gt2.png}
        \caption{Ground truth mesh}
    \end{subfigure}
    \hfill
    \begin{subfigure}{0.48\textwidth}
        \centering
        \includegraphics[width=\textwidth]{Supplementary Material/fail_2.png}
        \caption{NBP (Ours)}
    \end{subfigure}
    \caption{\textbf{Failure case 2}: This scene contains multiple narrow areas, prompting our method to depend more heavily on our precise prediction of obstacles. Under these challenging conditions, our approach may overlook exploring this area.}
    \label{fig:comparison6}
\end{figure}

