% \begin{figure}
%     \centering
%     \begin{tabular}{c@{\hspace{-0.3cm}}c@{\hspace{0.0cm}}c}
%     \includegraphics[height=3.3cm]{figures/macarons_13.png} &
%     \includegraphics[height=3.3cm]{figures/macarons_9.png} &
%     \includegraphics[height=3.3cm]{figures/macarons_14.png} \\
%     \end{tabular}
%     \vspace{-0.2cm}  
%     \\
%     \subcaption{MACARONS}
%     \begin{tabular}{c@{\hspace{-0.3cm}}c@{\hspace{0.0cm}}c}
%     \includegraphics[height=3.3cm]{figures/ours_13.png} &
%     \includegraphics[height=3.3cm]{figures/ours_9.png} &
%     \includegraphics[height=3.3cm]{figures/ours_14.png} \\
%     \end{tabular}
%     \vspace{-0.2cm}
%     \\
%     \subcaption{NBP (Ours)}
%     \caption{\textbf{Comparison of our method with the state-of-the-art MACARONS method on AiMDoom Simple dataset.} \textbf{MACARONS~(top row)} generates complicated trajectories that do not manage to map the scenes completely even at a simple level. \textbf{Our method (bottom row)} generates efficient trajectories that capture the entire scenes. In the same scenes, both methods start from the same pose, marked in deep blue.}
%     \label{fig:combined_results}
% \end{figure}

\begin{figure}
    \centering
    \begin{subfigure}{\textwidth}
        \centering
        \begin{tabular}{c@{\hspace{-0.3cm}}c@{\hspace{0.0cm}}c}
        \includegraphics[height=3.3cm]{figures/macarons_13.png} &
        \includegraphics[height=3.3cm]{figures/macarons_9.png} &
        \includegraphics[height=3.3cm]{figures/macarons_14.png} \\
        \end{tabular}
        \vspace{-0.2cm}
        \caption{Results of MACARONS. It generates complicated trajectories and often gets trapped in local areas.}
    \end{subfigure}
    \vspace{0.2cm}  % Adjust this value to control vertical spacing between subfigures
    \begin{subfigure}{\textwidth}
        \centering
        \begin{tabular}{c@{\hspace{-0.3cm}}c@{\hspace{0.0cm}}c}
        \includegraphics[height=3.3cm]{figures/ours_13.png} &
        \includegraphics[height=3.3cm]{figures/ours_9.png} &
        \includegraphics[height=3.3cm]{figures/ours_14.png} \\
        \end{tabular}
        \caption{Results of our NBP method. It efficiently travels in the scene and reconstructs the scene well.}
    \end{subfigure}
    \vspace{-1.5em}
    \caption{\textbf{Comparison of our NBP method with the state-of-the-art MACARONS method.} Both methods start from the same initial pose, marked in deep blue. We also include a demonstration video of active mapping using our method in the supplementary materials.}
    % \textbf{MACARONS~(top row)} generates complicated trajectories that do not manage to map the scenes completely even at a simple level. 
    % \textbf{Our method (bottom row)} generates efficient trajectories that capture the entire scenes. In the same scenes, both methods start from the same pose, marked in deep blue.
    \label{fig:combined_results}
    \vspace{-2em}
\end{figure}


