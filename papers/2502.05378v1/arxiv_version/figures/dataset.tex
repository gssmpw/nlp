% \begin{figure}[t]
%     \centering
%     \includegraphics[width=\textwidth]{figures/dataset.png}
%     \caption{\textbf{Top view of samples from our dataset.} Rows from top to bottom represent increasing scene complexity, categorized into four types: \textbf{\textit{Simple, Normal, Hard, Insane}}.}
%     \label{fig:exdoom}
% \end{figure}



\begin{figure}[t]
    \centering
    \begin{subfigure}{\textwidth}
        \centering
        \includegraphics[width=\textwidth]{figures/doom_top.png}
        \vspace{-1.5em}
        \caption{Bird-eye views of samples from the \textbf{\textit{Simple, Normal, Hard, and Insane}} levels (from left to right).}
        \label{fig:dataset_top}
    \end{subfigure}

    \begin{subfigure}{\textwidth}
        \centering
        \includegraphics[width=\textwidth]{figures/doom_inside.png}
        \caption{Representative images showing the internal structural composition of the scene.}
        \label{fig:dataset_inside}
    \end{subfigure}
    \caption{\textbf{Maps from our AiMDoom dataset.} The AiMDoom dataset includes four levels of geometric complexity with various textures.}
    \label{fig:dataset}
    \vspace{-1em}
\end{figure}