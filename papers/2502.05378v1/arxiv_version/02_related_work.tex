\section{Related Work}

\noindent \textbf{Active Mapping.} 
%
Active mapping aims to exhaustively reconstruct a 3D scene in the shortest possible time with a moving agent. 
Unlike SLAM \citep{learningslam, slam_survey, gaussian_slam}, which addresses both localization and mapping, active mapping focuses on reconstruction, continuously selecting viewpoints to cover the entire scene, assuming the pose is known.
Early methods often relied on frontier-based exploration (FBE) approaches~\citep{fbe}. The key idea is to move the agent toward a heuristically selected frontier along the boundary between reconstructed and unknown regions of the scene. 
Among different strategies~\citep{nbvp, cieslewski2017rapid, zhou2021fuel, tao2023seer} for frontier selection, moving to the nearest frontier serves as a strong baseline.
Additionally, there are efforts \citep{cao2021tare, xu2024heuristic} that combine global FBE and local planning strategies within a hierarchical optimization framework to enhance exploration.
However, these FBE-based approaches are heuristic-based and cannot exploit prior learned from data to explore more efficiently, restricting their performance in complex environments.

To address this limitation, learning-based approaches have been explored to select the next-best views~(NBV) for efficient 3D mapping.
The NBV-based methods train models to select the optimal pose from nearby camera poses~\citep{guedon2022scone, guedon2023macarons, lee2023so} or from a limited predefined view space such as a hemisphere~\citep{zhan2022activermap, uncertaintypolicy, rlnbv, zeng2020pc, mendoza2020supervisednbv}. 
While these methods show promising results to reconstruct single objects, their performance remains limited in large environments.
Due to the narrow search space for the next pose, NBV methods behave like a greedy policy and thus can easily get stuck in local regions.
To mitigate this, some works~\cite{ramrakhya2022habitatweb,chen2023object} use imitation learning to learn from human demonstrates which prioritize unseen exploration but with the cost of heavy labelling.
More recently, efforts have been made to enlarge the search range for the next best view~\citep{chen2024gennbv, ran2023neurar, pan2022activenerf, upen}. However, these methods are still primarily evaluated on single-object datasets with small moving steps, and often rely on optimizing indirect metrics like reconstruction uncertainty~\citep{upen}, which are not directly aligned with the goal of exhaustive 3D reconstruction. %Moreover, they frequently overlook the potential information acquisition during the movement.
In this work, we extend the evaluation to more complex indoor environments and also introduce a new surface coverage gain criterion that optimizes the coverage gain along the best trajectory towards a long-term goal.


\noindent \textbf{3D mapping datasets.}
Existing datasets for 3D mapping mainly focus on single isolated objects such as those in ShapeNet~\citep{chang2015shapenet} and OmniObject3D~\citep{wu2023omniobject3d}, or outdoor scenes~\citep{lu2023large, hardouin2020next}, where the agent only needs to move around the scene to achieve full reconstruction.
These datasets are comparatively less complex than indoor environments where the agent must enter into the scene.
The indoor scenes contain unique challenges such as dead ends and tight corners, which often force the agent to backtrack without significantly improving its objective.

While some works~\citep{yan2023active, upen, occant} incorporate indoor scene datasets such as Gibson~\citep{gibson} and MP3D~\citep{Matterport3D}, these often exhibit significant limitations. 
Existing synthetic datasets~\citep{replica19arxiv, RoboTHOR} often lack scene complexity, whereas real-world scans~\citep{dai2017scannet, ramakrishnan2021hm3d}, despite offering greater representational fidelity, are constrained by limited structural and map diversity and often suffer from substantial noise artifacts.
This lack of reliable datasets prevents comprehensive evaluation in active 3D mapping tasks.
In this work, we propose a new dataset - AiMDoom, designed for benchmarking active mapping in indoor environments of different complexities. 


% Some works rely on implicit representations, which are difficult to integrate with learning-based NBV policies and suffer from poor generalization.\cite{nbvmulti, navi-implicit, yan2023active, yan2021continual, sucar2021imap, pan2022voxfield}


% \vincent{
% Among early approaches to active mapping, \cite{fbe} is probably the most successful one. The key idea is to move the agent to the closest frontier between reconstructed and unknown parts of the scene. While it does not rely on learning and thus cannot exploit prior on the structure of the world, it remains a very strong baseline.

% NBV-based methods select the optimal pose from nearby camera poses or a limited predefined view space, such as a hemisphere~\citep{guedon2022scone, guedon2023macarons, zhan2022activermap, uncertaintypolicy, rlnbv}. The ``optimal pose'' can be defined based on criteria such as the uncertainty of the reconstruction. SCONE~\citep{guedon2022scone} and MACARONS~\citep{guedon2023macarons} rely on the ``surface coverage gain'', i.e., a prediction of how much new surface will be seen from the pose. Since this criterion is directly related to the goal of exhaustive 3D reconstruction, we also rely on this criterion. The key difference is that we consider the total surface coverage gain along a trajectory rather than only the gain from a single pose. 

% Restricting the search for the next pose to a small neighborhood as NBV methods do appears to be a very greedy policy, which typically results in high traveling distances as the agent has to revisit places that it did not scan completely.

% Like us, some works have attempted to enlarge the search range for the next best view~\citep{chen2024gennbv, ran2023neurar, pan2022activenerf}. \vincentrmk{upen?} 
% But, unlike us, these works do not leverage the potential information gain that could be obtained when moving to this view. As our experiments show, this results in longer and more complex trajectories. 
% }


% \vincentrmk{I dont get the problem with these methods:}
% Some works rely on implicit representations, which are difficult to integrate with learning-based NBV policies and suffer from poor generalization.
% \cite{nbvmulti, navi-implicit, yan2023active, yan2021continual, sucar2021imap, pan2022voxfield}

% Related to our work are indoor exploration methods~\citep{yan2023active}. \vincentrmk{Shizhe's papers?} These are still different as they do not aim to reconstruct a complete model but instead to efficiently find target objects. They also often rely on a navigation model that do not generalize to any environment. \vincentrmk{is it credible?}


% \noindent \textbf{Benchmarks and Datasets for Indoor 3D Mapping.}
% %
% Few benchmarks already exist for evaluating 3D mapping. They mostly focused on outdoor scenes or simple isolated objects~\citep{rlnbv,guedon2022scone,chen2024gennbv}. In our experience, outdoor scenes are much simpler than indoor scenes, as shown for example by our comparison with MACARONS~\citep{guedon2022scone}, because indoor scenes have many deadends and corners that force the agent to move back without improving its objective criterion. 


% % The field of active 3D reconstruction currently lacks suitable datasets for testing and evaluating methods in complex indoor environments.
% % Datasets employed in the majority of active 3D reconstruction studies \cite{chen2024gennbv, guedon2023macarons, zeng2020pc} predominantly focus on outdoor scenes or isolated objects \cite{rlnbv, guedon2022scone, chang2015shapenet}, which tend to oversimplify the problem and display limited variability, making it challenging to distinguish between different methodologies. 

% While some works~\citep{yan2023active, upen, occant} incorporate indoor scene datasets~\citep{gibson, Matterport3D}, these often exhibit significant limitations. Synthetic datasets frequently lack scene complexity, whereas real-world scans, despite offering greater representational fidelity, are constrained by limited structural and map diversity and often suffer from substantial noise artefacts. \
% This lack of reliable datasets prevents comprehensive evaluation and generalization in active 3D reconstruction tasks.

% % This dataset selection and quality dichotomy presents ongoing challenges for comprehensive evaluation and generalization in active 3D reconstruction tasks.

% To address these shortcomings and propel advancements in active 3D reconstruction, we propose a dataset we call AiMDoom, which we design for benchmarking autonomous mapping of indoor environments. AiMDoom encompasses four distinct levels of difficulty, each consisting of 100 unique scenes enriched with diverse textures and highly detailed 3D mesh data.

% \textbf{Self-supervised Learning.}

% \textbf{Continual Learning.} 

% \textbf{Multi-task Learning.}
