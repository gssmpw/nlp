\section{The AiMDoom Dataset}
\label{dataset-sec}

\begin{table}
\caption{\textbf{Comparison between AiMDoom and prior indoor 3D datasets.} Navigation complexity is the maximum ratio of geodesic to euclidean distances between any two navigable locations in the scene. Universal accessibility means whether windows and doors are accessible.}
\centering
\begin{adjustbox}{width=\textwidth}
\begin{tabular}{lcccccccccc}
\hline
\noalign{\vskip 1.5mm}
\multicolumn{1}{c}{\multirow{2}{*}{\makecell{Dataset}}} & \multicolumn{1}{c}{\multirow{2}{*}{\makecell{Replica}}} & \multicolumn{1}{c}{\multirow{2}{*}{\makecell{RoboTHOR}}} & \multicolumn{1}{c}{\multirow{2}{*}{\makecell{MP3D}}} & \multicolumn{1}{c}{\multirow{2}{*}{\makecell{Gibson\\(4+ only)}}} & \multicolumn{1}{c}{\multirow{2}{*}{\makecell{ScanNet}}} & \multicolumn{1}{c}{\multirow{2}{*}{\makecell{HM3D}}} & \multicolumn{4}{c}{\textbf{AiMDoom (Ours)}} \\
\cline{8-11}
\noalign{\vskip 1.1mm}
 &  &  &  &  &  &  & Simple & Normal & Hard & Insane \\
\noalign{\vskip 1mm}
\hline
\noalign{\vskip 1mm}
Number of scenes & 18 & 75 & 90 & 571 (106) & 1613 & 1000 & 100 & 100 & 100 & 100 \\
Floor space (m$^2$) & 2.19k & 3.17k & 101.82k & 217.99k (17.74k) & 39.98k & 365.42k & 63.33k & 134.84k & 321.38k & 548.85k \\
Navigation complexity & 5.99 & 2.06 & 17.09 & 14.25 (11.90) & 3.78 & 13.31 & 11.31 & 18.38 & 36.05 & 45.25 \\
Universal accessibility & \ding{55} & \ding{55} & \ding{55} & \ding{55} & \ding{55} & \ding{55} & \ding{51} & \ding{51} & \ding{51} & \ding{51} \\
Easy expansion & \ding{55} & \ding{55} & \ding{55} & \ding{55} & \ding{55} & \ding{55} & \ding{51} & \ding{51} & \ding{51} & \ding{51} \\
\hline
\end{tabular}
\end{adjustbox}
\label{tab:dataset_comparision}
\vspace{-1em}
\end{table}


In this section, we introduce \textbf{AiMDoom}, a new dataset for \textbf{A}ct\textbf{i}ve 3D \textbf{M}apping in complex indoor environments based on the \textbf{Doom} video game~\footnote{\url{https://en.wikipedia.org/wiki/Doom_(franchise)}}.
As Doom features a wide variety of indoor settings, we use its map generator to create four sets of maps of increasing geometric complexity: Simple, Normal, Hard, and Insane. In the following, we first detail how we built these maps and then discuss the key challenges presented in our AiMDoom dataset.

\section{Dataset}
\label{sec:dataset}

\subsection{Data Collection}

To analyze political discussions on Discord, we followed the methodology in \cite{singh2024Cross-Platform}, collecting messages from politically-oriented public servers in compliance with Discord's platform policies.

Using Discord's Discovery feature, we employed a web scraper to extract server invitation links, names, and descriptions, focusing on public servers accessible without participation. Invitation links were used to access data via the Discord API. To ensure relevance, we filtered servers using keywords related to the 2024 U.S. elections (e.g., Trump, Kamala, MAGA), as outlined in \cite{balasubramanian2024publicdatasettrackingsocial}. This resulted in 302 server links, further narrowed to 81 English-speaking, politics-focused servers based on their names and descriptions.

Public messages were retrieved from these servers using the Discord API, collecting metadata such as \textit{content}, \textit{user ID}, \textit{username}, \textit{timestamp}, \textit{bot flag}, \textit{mentions}, and \textit{interactions}. Through this process, we gathered \textbf{33,373,229 messages} from \textbf{82,109 users} across \textbf{81 servers}, including \textbf{1,912,750 messages} from \textbf{633 bots}. Data collection occurred between November 13th and 15th, covering messages sent from January 1st to November 12th, just after the 2024 U.S. election.

\subsection{Characterizing the Political Spectrum}
\label{sec:timeline}

A key aspect of our research is distinguishing between Republican- and Democratic-aligned Discord servers. To categorize their political alignment, we relied on server names and self-descriptions, which often include rules, community guidelines, and references to key ideologies or figures. Each server's name and description were manually reviewed based on predefined, objective criteria, focusing on explicit political themes or mentions of prominent figures. This process allowed us to classify servers into three categories, ensuring a systematic and unbiased alignment determination.

\begin{itemize}
    \item \textbf{Republican-aligned}: Servers referencing Republican and right-wing and ideologies, movements, or figures (e.g., MAGA, Conservative, Traditional, Trump).  
    \item \textbf{Democratic-aligned}: Servers mentioning Democratic and left-wing ideologies, movements, or figures (e.g., Progressive, Liberal, Socialist, Biden, Kamala).  
    \item \textbf{Unaligned}: Servers with no defined spectrum and ideologies or opened to general political debate from all orientations.
\end{itemize}

To ensure the reliability and consistency of our classification, three independent reviewers assessed the classification following the specified set of criteria. The inter-rater agreement of their classifications was evaluated using Fleiss' Kappa \cite{fleiss1971measuring}, with a resulting Kappa value of \( 0.8191 \), indicating an almost perfect agreement among the reviewers. Disagreements were resolved by adopting the majority classification, as there were no instances where a server received different classifications from all three reviewers. This process guaranteed the consistency and accuracy of the final categorization.

Through this process, we identified \textbf{7 Republican-aligned servers}, \textbf{9 Democratic-aligned servers}, and \textbf{65 unaligned servers}.

Table \ref{tab:statistics} shows the statistics of the collected data. Notably, while Democratic- and Republican-aligned servers had a comparable number of user messages, users in the latter servers were significantly more active, posting more than double the number of messages per user compared to their Democratic counterparts. 
This suggests that, in our sample, Democratic-aligned servers attract more users, but these users were less engaged in text-based discussions. Additionally, around 10\% of the messages across all server categories were posted by bots. 

\subsection{Temporal Data} 

Throughout this paper, we refer to the election candidates using the names adopted by their respective campaigns: \textit{Kamala}, \textit{Biden}, and \textit{Trump}. To examine how the content of text messages evolves based on the political alignment of servers, we divided the 2024 election year into three periods: \textbf{Biden vs Trump} (January 1 to July 21), \textbf{Kamala vs Trump} (July 21 to September 20), and the \textbf{Voting Period} (after September 20). These periods reflect key phases of the election: the early campaign dominated by Biden and Trump, the shift in dynamics with Kamala Harris replacing Joe Biden as the Democratic candidate, and the final voting stage focused on electoral outcomes and their implications. This segmentation enables an analysis of how discourse responds to pivotal electoral moments.

Figure \ref{fig:line-plot} illustrates the distribution of messages over time, highlighting trends in total messages volume and mentions of each candidate. Prior to Biden's withdrawal on July 21, mentions of Biden and Trump were relatively balanced. However, following Kamala's entry into the race, mentions of Trump surged significantly, a trend further amplified by an assassination attempt on him, solidifying his dominance in the discourse. The only instance where Trump’s mentions were exceeded occurred during the first debate, as concerns about Biden’s age and cognitive abilities temporarily shifted the focus. In the final stages of the election, mentions of all three candidates rose, with Trump’s mentions peaking as he emerged as the victor.

\noindent \textbf{Dataset construction.}
%
We used the open-source software Obsidian~\footnote{\url{https://obsidian-level-maker.github.io/}} to automatically generate Doom maps as our indoor environments.
Four sets of hyperparameters are proposed to control architectural complexity and texture styles in Obsidian. By varying these hyperparameters, we produced maps categorized into Simple, Normal, Hard and Insane difficulty levels. Each difficulty level is made of 100 maps with 70 for training and 30 for evaluation.

% \vincentrmk{should we talk about splits between training, validation, test sets?} \shiyaormk{yes, I will add this, we also mentioned it in the experimental section}

The maps include doors and windows, all of which are configured to be open. This allows the agent to see and pass through the doors and windows.
We converted the maps to the widely used OBJ format, and used Blender~\citep{blender} to consolidate the texture images of each map into a single texture image. This makes the maps compatible with Pytorch3D~\citep{pytorch3d} and Open3D~\citep{open3d}. 
Further details are presented in the supplementary material.


\noindent \textbf{Key challenges.}
%
The AiMDoom dataset presents three key challenges for active 3D mapping.
Firstly, the dataset features environments with intricate geometries and layouts as shown in Figure~\ref{fig:dataset}, making it challenging to determine the optimal exploration direction for effective mapping.
Secondly, the maps have small doors and narrow corridors, requiring careful path planning to navigate.
Finally, the map diversity requires the reconstruction system to generalize across different scenes.
Table~\ref{tab:dataset_comparision} compares AiMDoom with existing indoor 3D datasets~\citep{replica19arxiv, RoboTHOR, Matterport3D, dai2017scannet,gibson, ramakrishnan2021hm3d}, highlighting our dataset's strengths in scene area and navigation complexity.

We will release the dataset along with a comprehensive toolkit to generate the data, which enables easy expansion of the dataset for future research.


% To ensure that the map offers full accessibility for various robotic platforms such as unmanned aerial vehicles (UAVs) and wheeled robots, we configure all doors and windows to remain open during map generation. 
% However, we observe that Obsidian does not consistently guarantee accessibility to all areas. To resolve this, we manually edit each scene to ensure the traversability of windows, doors, and hidden passages.
% Additionally, Doom map files are in a specific format that is incompatible with common 3D libraries such as Pytorch3D \cite{pytorch3d} and Open3D \cite{open3d}.
% To address this, we extract mesh files and texture images from these maps. 
% Specifically, we load Doom WAD~\footnote{\url{https://en.wikipedia.org/wiki/Doom_modding}} files to generate widely used OBJ format, and employ Blender~\cite{blender} for texture baking that consolidates multiple texture images of the map into a single texture image for each scene.
% In this way, we generate 100 unique environments for each of the four levels in AiRDoom.


% For the dataset creation, we first generate Doom map files with varying architectural complexities and diverse texture styles by defining 4 different lower and upper bounds for architectural complexity using Obsidian\footnote{https://obsidian-level-maker.github.io/} under the GNU General Public License. 

% Subsequently, we extract mesh files and their corresponding texture image relationships from these maps. However, as with most existing 3D scene datasets \cite{Matterport3D, replica19arxiv, ramakrishnan2021hm3d}, the texture information at this stage is stored as discrete images, which are unable to directly read the texture information of the meshes by the popular 3D libraries such as Pytorch3D \cite{pytorch3d} and Open3D \cite{open3d}. To address this issue, we utilize Blender \cite{blender} to perform texture baking, consolidating the discrete texture images into a single texture image for each scene in our dataset.

% To ensure our dataset offers comprehensive navigational versatility, accommodating various robotic platforms such as unmanned aerial vehicles (UAVs) and wheeled robots, we configured all doors and windows to be open during the map generation process. However, this open-source software, originally designed for the Doom video game, cannot consistently guarantee accessibility to all areas. To address this limitation, we manually edited each scene mesh to ensure the traversability of windows, doors, and hidden passages.

% \noindent \textbf{Key challenges.}
% The AiRDoom dataset presents three key challenges for active 3D reconstruction.
% Firstly, the dataset features environments with intricate geometries and layouts as shown in Figure~\ref{fig:exdoom}, making it challenging to determine the optimal exploration direction for effective reconstruction.
% Secondly, the environments often contain small windows, doors and narrow corridors, requiring careful path planning to navigate.
% Finally, the diversity of environments requires the reconstruction system to generalize across different scenes.
% Table~\ref{tab:doom-comparison} compares AiRDoom with existing indoor 3D datasets, highlighting our dataset's strengths in scene area and navigation complexity.

% We will release the dataset along with a comprehensive toolkit to generate the data, which enables easy expansion of the dataset for future research.


% Compared with existing similar indoor 3D datasets in Tab.~\ref{tab:doom-comparison}, our dataset excels in terms of scene area, exploration complexity, and high practicality.

% Our contribution extends beyond the dataset itself. We will release the comprehensive toolkit used to generate this dataset, including the configuration files for map generation within Obsidian, and the code for mesh extraction and texture baking.





