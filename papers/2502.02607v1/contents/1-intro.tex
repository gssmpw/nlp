\section{Introduction}
\label{sec:introduction}

Metamaterials, often realized as periodic microstructures, have attracted significant attention in both academia and industry, driven by advances in additive manufacturing (AM)~\cite{Kadic2019,Askari2020}. The precise manipulation of microstructures enables control over unique physical properties, such as lightweightness~\cite{lu2014Buildtolast}, elasticity~\cite{schumacher2015Microstructures}, shock protection~\cite{Huang2024}, heat transfer~\cite{Ding2021}, and optical behavior~\cite{yu2014flat}.  
%
Topologies and morphologies govern the physical properties of microstructures, driving extensive research into forward design approaches. These efforts have yielded structures such as struts~\cite{panetta2015Elastic,panetta2017Worstcase}, shells~\cite{liu2022Parametric,Xu2023}, plates~\cite{Tancogne‐Dejean2018}, and stochastic forms~\cite{martinez2016Procedural}. However, forward design, relying on mimetic observations or heuristic rules, cannot exhaustively explore the combinatorial configuration space due to its NP-hard nature~\cite{torquato2005Microstructure, Gao2018}, limiting its ability to achieve the desired macroscopic design targets.


Inverse design, a critical alternative to traditional methods, begins with desired functional properties and works backward to determine the optimal geometry and material distribution. The inverse design of microstructures has been explored for decades, with significant contributions from topology optimization (TO) methods, and has gained momentum through data-driven approaches in recent years~\cite{Lee2023}. However, achieving the goal of generating structures with \emph{precise target physical properties}, \emph{valid geometries}, and \emph{flexibility} for heterogeneous tiling remains an open challenge. 
Despite advancements in deep generative neural networks for 3D content creation~\cite{Zhang2024Clay,Wu2024,li2024craftsman}, realizing these objectives remains highly challenging. 
The complexity arises from the need for precise control over both geometry and material properties while ensuring their alignment. Maintaining this alignment is particularly difficult, as small changes in geometry can substantially affect physical properties, and vice versa. Furthermore, the relationship between geometry and properties is highly sensitive, requiring careful management in generative models. These issues are further compounded by the ill-posed nature of the inverse problem, where multiple topologies can yield identical or similar effective properties~\cite{Wang2024}.

Most existing approaches compromise by relying on parametric geometric representations, constructing structure-property datasets based on predefined parameters and solving the inverse design problem within specific structure families through inverse searches~\cite{schumacher2015Microstructures, panetta2015Elastic}, differential parameter optimization~\cite{tozoni2020Lowparametric}, or data-driven regression~\cite{bastek2022Inverting, wang2022IHGAN, Zheng2023}. While effective within these classes, they limit design flexibility and structural diversity. A recent approach using self-conditioned diffusion models for microstructure design~\cite{Yang2024} eliminates this restriction, achieving relatively high matching accuracy, but still struggles with maintaining structural integrity.
 

In this work, we address the inverse design problem for microstructures, focusing on cubic elastic properties. Our goal is to achieve a high matching ratio with the desired properties while ensuring geometric validity, including connectivity, cubic symmetry, and boundary compatibility. Furthermore, we aim to enable the generation of diverse microstructure types and morphologies that meet these target properties, without being constrained by predefined parametric classes.

Microstructures differ from general 3D shapes in two key ways: 1) they exhibit high geometric symmetry; and 2) their effective properties are intrinsically tied to their geometry. In fact, discussing a microstructure without considering its properties is meaningless, as the two are inherently coupled.

Building on this insight, we propose a novel hybrid neural representation, \emph{Holoplane}, which embeds \emph{geometric symmetry constraints explicitly} and the \emph{elasticity tensor implicitly}. This combination ensures precise alignment between geometry and properties.
Additionally, we introduce a latent diffusion-based generative model capable of producing multiple microstructure candidates that satisfy target properties while maintaining connectivity, boundary periodicity, and structural compatibility.


To train and validate our model, we introduce a diverse multi-class dataset encompassing a broad spectrum of geometric morphologies and topologies, including parametric families such as truss, shell, tube, and plate microstructures. We evaluate the system’s performance through the generation of novel microstructures, cross-class interpolation, and the infilling of heterogeneous microstructures. Experimental results demonstrate the model’s ability to generate microstructures that meet target properties, ensure geometric validity, and integrate seamlessly into complex assemblies (\figref{fig:teaser}).


Our main contributions are as follows:  
\begin{itemize}[nosep]
\item We tackle the inverse design problem for non-parametric microstructures, enabling the generation of diverse types and morphologies that satisfy target properties through a latent diffusion model.

\item We introduce Holoplane, a hybrid neural representation for microstructures that enhances the alignment between geometry and property distributions, leading to higher property matching accuracy and enhanced validity control over the generated microstructures compared to existing baselines.

\item The proposed latent diffusion framework, built upon Holoplane, also enables optimization of boundary compatibility, achieving superior boundary integration in heterogeneous, multi-scale designs.

\end{itemize}  


