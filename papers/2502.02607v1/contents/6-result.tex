\section{Results}

\subsection{Accuracy}
\label{sec:accuracy}
We utilized the properties of the test set (Sec.~\ref{sec:dataset}) as conditions for the inverse generation test.
For each microstructure, 8 candidate models were generated, resulting in a total of 8,000 models.
Tests were performed on a 4xA40 server, with an average generation time of \textbf{0.13 seconds} per microstructure and \textbf{0.02 seconds} for SDF decoding.
Properties were computed for each model using the homogenization method, and errors were calculated according to Eq.\ref{eq:error}.

\begin{figure}
    \centering
    \includegraphics[width=\linewidth]{figures_new/err_dist.jpg}
    \caption{The error distribution of MIND and \cite{Yang2024} on the same test dataset.}
    \label{fig:err_dist}
\end{figure} 


Tab. \ref{tab:error_comparison} and Fig.~\ref{fig:err_dist} present a comparison between our method and \cite{Yang2024}, demonstrating that our approach achieves state-of-the-art performance in inverse generation of microstructures. 
Moreover, our method can be viewed as a variant of the triplane representation \cite{chan2022triplane, Shue2023}.
To assess the necessity of the Holoplane, we conducted ablation studies using the triplane representation (NFD~\cite{Shue2023}). 
Among the generated models, some were excluded due to issues with translational symmetry or connectivity, preventing property calculations.
The physical validity ratio was \textbf{95.7\%} for \cite{Yang2024}, \textbf{97.8\%} for the NFD representation, and \textbf{99.2\%} for our method.
The results strongly emphasize that our Holoplane representation significantly enhances the accuracy of property-conditioned generation.


\begin{table}[ht]
    \centering
    \caption{
    The properties of the test set are used as the conditions to evaluate generation errors.
    $C_{\text{best}}$ is the average of the best result from each group of 8, while the other columns represent the average of all 8000 structures. The NFD + Phy approach leverages the physical-aware embedding introduced in Sec.~\ref{sec:phy_enc} to align geometry and physics within the latent space.}
    \label{tab:error_comparison}
    \begin{tabular}{lcc|ccc}
        \toprule
        Method                      & $\bC_{\text{best}}$  & $\bC_{\text{all}}$ & C\textsubscript{11} & C\textsubscript{12} & C\textsubscript{44} \\
        \midrule
        \cite{Yang2024}             & 1.33\% & 2.96\% & 2.50\% & 3.68\% & 2.70\% \\
        NFD                         & 0.63\% & 5.39\% & 5.28\% & 4.86\% & 6.01\% \\
        NFD + Phy                   & 0.44\% & 1.68\% & 1.49\% & 1.80\% & 1.75\% \\
        MIND (Ours)                 & 0.29\% & 1.27\% & 1.13\% & 1.33\% & 1.34\% \\
        \bottomrule
    \end{tabular}

\end{table}



\subsection{Generation Boundary}

\begin{figure}
    \centering
    \includegraphics[width=\linewidth]{figures_new/design_space_v3.png}
    \caption{Comparison of the property space between the training set and the generation set.
    The training set consists of approximately 180,000 samples, as described in Sec.~\ref{sec:dataset}, while the generation set contains around 550,000 samples obtained by random sampling inside and near the boundary of the property space.
    We visualize both the original distribution and the log-scale distribution of the property space (top-left corner).
    Our method effectively extends the boundary of the property space, significantly increasing the maximum Young's modulus and shear modulus while also achieving a lower negative Poisson's ratio.}
    \label{fig:design_space}
\end{figure} 


To explore the boundary of our network’s generative capacity, we randomly sample points near the boundary of the property space. 
This process continues until the network fails to generate structures that meet the specified properties. 
As shown in Fig.~\ref{fig:design_space}, our method successfully expands the design space, showcasing its strong generative capability.



\subsection{Diversity}

\begin{figure}
    \centering
    \includegraphics[width=\linewidth]{figures_new/shape_gen.jpg}
\caption{Inverse generation from a reference model.
Using the reference model's mechanical properties as input, five candidate models generated by our framework are listed.
Each model's Young's modulus surface is shown with a consistent color bar, demonstrating structures in diverse morphologies with similar mechanical properties.}
    \label{fig:shape_gen}
\end{figure} 


We compute the average similarity between the 8,000 generated microstructures and the entire training set. 
Our model achieves an average similarity of \textbf{81.52\%}, which is lower than the \textbf{93.48\%} reported in \cite{Yang2024}. 
Besides, we selected some representative models and utilized their mechanical properties ($E, \nu, G$) as input for inverse generation.
As shown in Fig.~\ref{fig:shape_gen}, given a target property as input, our method can generate multiple distinct types of structures while maintaining similar properties.
This indicates that our model exhibits greater shape diversity and is not merely memorizing the training data. 
Additional results are visualized in \srefmorevis.


\subsection{Interpolation}

\begin{figure}
    \centering
    \includegraphics[width=\linewidth]{figures_new/interpolation.jpg}
    \caption{Interpolation using different properties.
    The start models are truss structures with low Young’s moduli and small volume fractions, while the end models are plate structures with high Young’s moduli and larger volume fractions.
    Initially, the interpolated models retain their truss topology, increasing the volume fraction to achieve higher Young’s modulus values.
    Gradually, the structures transition into plate configurations, ultimately forming plate structures with significantly higher Young’s moduli.
    }
    \label{fig:interpolation}
\end{figure} 

Our approach also enables the generation of novel structures through interpolations, as described in Sec.~\ref{sec:compat}. 
Specifically, we performed interpolation experiments on two groups of models with significantly different material properties, as well as several groups of models belonging to distinct microstructure families.
As shown in Fig.~\ref{fig:interpolation} and Fig.~\ref{fig:supp_interpolation}, our method achieves smooth geometric and physical transitions within each group of configurations.




\setlength{\belowcaptionskip}{5pt}

\subsection{Printability}

\begin{figure*}
    \centering
    \includegraphics[width=\linewidth]{figures_new/printability.jpg}
    \caption{(a) and (b) show two generated microstructures under different printing precisions, along with their property error distributions visualized as heatmaps. 
    (c) and (d) present the printed results corresponding to (a) and (b).
    The two test models used a $4 \times 4 \times 1$ grid of microstructures, each with a cell size of 20 mm.
    The experiments were conducted with printer precision settings of 0.6 mm and 1.2 mm, corresponding to detection resolutions of $32^3$ and $16^3$.
    The smallest feature sizes in the first and second test cases were \textbf{0.62 mm} and \textbf{1.25 mm}, respectively, with property errors of \textbf{0.2\%} and \textbf{0.4\%}.
    For the first structure, the smallest feature size was 0.62 mm, while the second structure exhibited a minimum feature size of 1.25 mm.}
    \label{fig:print}
\end{figure*} 

We tested the ability of our method to generate printable objects at different printing precisions of 0.6 mm and 1.2 mm.
Finer printing resolution allows for more precise error control.
Experiments show that at both fine and coarse printing resolutions, our method can still generate printable structures that meet the required property specifications (Fig.~\ref{fig:print}).



\subsection{Heterogeneous Design}

\begin{figure*}
    \centering
    \includegraphics[width=\linewidth]{figures_new/pillow_bracket.jpg}
    \caption{
    We tested MIND on the pillow bracket model with a 0.01 m grid resolution. 
    The base of the model was fixed, and forces were applied to the two handles, with the model divided into 392 grids for the design region (a, b).
    Initial material properties were set to $E = 0.1$, $\nu = 0.25$, and $G = 0.03$, resulting in displacements of \textbf{0.000 m | 0.016 m | 0.039 m} (min, avg, max) (e).
    We then constrained the sum of Young's modulus to remain constant and optimized the physical properties of each cell (c).
    The microstructures generated by MIND were then filled in the design region (d).
    After optimization, the displacement was reduced to \textbf{0.000 m | 0.008 m | 0.032 m} (f). 
    The filled model produced a displacement \textbf{of 0.000 m | 0.008 m | 0.034 m} (g), closely matching the optimization results. 
    Printability testing showcased that all microstructures were printable and confirmed perfect compatibility between adjacent cells (h).}
    \label{fig:pillow}
\end{figure*}


To validate MIND in heterogeneous design, we tested it on a pillow bracket model discretized into a 0.01 m grid. 
The material properties were optimized to minimize overall displacement. 
After optimizing the material distribution, we applied MIND for inverse design and blended the boundaries. 
The resulting displacement performance closely aligned with the optimization target, demonstrating MIND’s capability to generate microstructures that meet mechanical requirements while ensuring boundary compatibility (Fig.~\ref{fig:pillow}).

