\section{Implementation}
\label{sec:implementation}
\subsection{Dataset}
\label{sec:dataset}
In this work, we focus on microstructures with cubic symmetry $O_h$, a high-order symmetry group that satisfies translational symmetry. 
Our dataset consists of a mixed set of microstructure meshes, including truss (33\%), tube (38\%), shell (17\%), and plate (12\%) structures (Fig.~\ref{fig:dataset}), with a total of 180,000 samples and a volume fraction ranging from 5\% to 65\%. 
The skeletons of the truss and tube structures were generated using the method proposed by~\cite{panetta2015Elastic}, while the shell and plate structures were constructed following the parametric methods introduced by~\cite{liu2022Parametric} and~\cite{Sun2023}, respectively.
We randomly selected 1,000 samples as the test set and validation set respectively.

\begin{figure}[tb]
    \centering
    \includegraphics[width=\linewidth]{figures_new/dataset.jpg}
    \caption{Exemplar models for four different structure types.}
    \label{fig:dataset}
\end{figure} 

\subsection{Properties Calculation}
\label{sec:properties}
We calculate the elastic tensor $\mathbf{C}$ for all data using a GPU-accelerated implementation of the homogenization method~\cite{andreassen2014Design, dong2018149}.
In the calculations, the base material's Young’s modulus $E$ is set to 1 (dimensionless), and the Poisson’s ratio $\nu$ is set to 0.35.
Under $O_h$, the elastic properties of a microstructure reduce to three independent components: $C_{11}, C_{12}, C_{44} \in \mathbf{C}$. 

\subsection{Network Design}
\label{sec:network_imp}
We employ a 7-layer residual convolutional neural network (CNN) with $3\times3$ kernels to encode the Holoplane. 
Each layer performs downsampling along the normal direction of the symmetry plane.
The network compresses the SDF $\phi \in \mathbb{R}^{128\times128\times128\times1}$ into a 2D latent Holoplane $\holo \in \mathbb{R}^{64\times64\times32}$, where 32 denotes the number of channels.
For simplicity, we select one of the three equivalent axial planes as the Holoplane, which is sufficient for cubic symmetry.
During decoding, after projecting any point $\bx$ onto the Holoplane to obtain $\bu$, we feed $\holo(\bu)$ into a 5-layer residual MLP to predict the corresponding fields.
To train the model, we sample 100,000 points within the SDF randomly and another 100,000 points near the mesh surface.

We employ the EDM (Elucidated Diffusion Model,~\cite{karras2022edm}) framework as our generative backbone, where timestep is set to 18.
Due to the high symmetry of $O_h$, we use only one-quarter of the Holoplane.
The guidance strength is set to 7 to balance diversity and adherence to the conditioned properties.
For further details, refer to \sreftraining.

\subsection{Metrics}
\label{sec:metrics}
We use two metrics introduced in~\cite{Yang2024} to evaluate the inverse generation.
The error in microstructure properties is computed as follows:
\begin{equation}
\label{eq:error}
    \text{Err} = \frac{\left|\bC_{\text{pred}} - \bC_{\text{target}}\right|}{\bC_{\text{max}} - \bC_{\text{min}}}.
\end{equation}
For two given microstructures, we first binarize their SDFs and then calculate the similarity using the following formula:
\begin{equation}
\label{eq:sim}
    \text{Sim}(\Omega_1, \Omega_2) = \frac{\sum_{v \in V} \mathbb{V}_{\Omega_1(v) = \Omega_2(v)}}{\sqrt{|\Omega_1| \cdot |\Omega_2|}},
\end{equation}
where the numerator is the number of intersecting voxels between the two structures, and the denominator is the square root of the product of the total voxel counts of both structures.
% We also recommend using the Intersection over Union (IoU) to measure the similarity between two structures.

\subsection{Printable Heterogeneous Design}
Heterogeneous design aims to optimize material properties for specific performance. 
We propose a linear finite element method (FEM) to optimize mechanical strength, detailed in \srefHeterdesign.
The algorithm computes $\bC_{\text{target}}$ for each hexahedral cell, which is then used to generate microstructures via MIND. 
For each cell, 64 candidates are generated, and the best is selected based on printability, volume fraction, and accuracy.
To ensure printability, microstructures of size $S$ must have a minimum feature size exceeding the printer precision $\epsilon$. 
Feature size is evaluated via SDF connectivity across resolutions, with a detection resolution of $R = S / \epsilon$ as the filtering threshold.
Microstructures are generated sequentially, ensuring boundary compatibility (Sec.~\ref{sec:diffusion}) for structural continuity and printability.
