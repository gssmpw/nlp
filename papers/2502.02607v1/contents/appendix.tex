\section{Properties Calculation}
\label{sec:homogen}
We employ a GPU-accelerated implementation of the homogenization method~\cite{andreassen2014Design, dong2018149} as a solver for the properties of microstructures.
Specifically, our objective is to compute the elastic tensor $E \in \mathbb{R}^{9\times9}$ of the structure.
It is derived from the fourth-order elasticity tensor $E_{ijkl}$, which governs the linear relationship between stress $\sigma_{ij}$ and strain $\epsilon_{kl}$ as:
\begin{equation}
    \sigma_{ij} = \sum_{kl}E_{ijkl}\epsilon_{kl}.
\end{equation}
In three-dimensional space, the indices $i, j, k, l$ range from 1 to 3, resulting in $3 \times 3 \times 3 \times 3 = 81$ components in $E_{ijkl}$. 
However, due to the inherent symmetries in both the stress-strain relationship (e.g., $\sigma_{ij} = \sigma_{ji}$ and $\epsilon_{ij} = \epsilon_{ji}$) and the elasticity tensor itself (e.g., $E_{ijkl} = E_{jikl} = E_{ijlk} = E_{klij}$), the number of independent components is reduced to 21. 
For further simplification, we represent the elasticity tensor in Voigt notation, $\bC \in \mathbb{R}^{6\times6}$.

The components of $\bE_{ijkl}$ are computed as follows:
\begin{equation}
\label{eq:E}
    E_{ijkl} = \frac{1}{|\bV|} \int_\bV \left( \overline{\epsilon}_{ij} - \epsilon_{ij}(\boldsymbol{\chi}) \right)  E^b_{ijkl} \left(\overline{\epsilon}_{kl} - \epsilon_{kl}(\boldsymbol{\chi}) \right) d\bV,
\end{equation}
where $\bE^b$ denotes the locally varying elasticity tensor, $\bV$ is the unit cell with volume $|\bV|$, $\overline{\epsilon}$ represents the prescribed macroscopic strain fields, and $\epsilon(\boldsymbol{\chi})$ denotes the locally varying strain fields. 
The displacement fields $\boldsymbol{\chi}$, introduced during the homogenization process, encode critical physical priors but are also the most computationally expensive component to calculate.

The calculation of $\boldsymbol{\chi}$ can be formulated as solving the following equation :
\begin{equation}
\label{eq:chi_eq}
    \int_\bV E_{ijkl} \epsilon_{ij}(v)\epsilon_{pq}(\boldsymbol{\chi}) d\bV = \int_\bV E_{ijkl} \epsilon_{ij}(v) \overline{\epsilon}_{kl} d\bV \quad \forall v \in \bV,
\end{equation}
where $v$ is a virtual displacement field.

Numerically, this equation is solved using the finite element method (FEM), where the domain $\bV$ is discretized into a grid with $n$ resolution. 
For each element $e$, the local stiffness matrix $\bK_e \in \mathbb{R}^{24\times24}$ and load vector $\bff_e \in \mathbb{R}^{24\times6}$ are constructed. 
These local contributions are then assembled into the global stiffness matrix $\bK$ and the global load vector $\bF$ over all valid elements, leading to a global system of linear equations:
\begin{equation}
    \bK\boldsymbol{\chi} = \bF,
\end{equation}
which serves as the foundation for incorporating physical priors into our formulation (Sec.\ref{sec:phy_enc})

In the FEM context, Eq.~\ref{eq:E} is performed by:
\begin{equation}
        E_{ijkl} = \frac{1}{|\bV|} \sum_{v \in \bV} \left( \overline{\epsilon} - \epsilon(\boldsymbol{\chi}) \right)^T  E^b_{ijkl} \left(\overline{\epsilon} - \epsilon(\boldsymbol{\chi}) \right).
\end{equation}


The mechanical properties, including Young’s modulus ($E$), Poisson’s ratio ($\nu$), and shear modulus ($G$), can be derived from the elastic tensor $\bC$.
First, the compliance matrix $\bS$ is obtained by inverting the elastic tensor:
\begin{equation}
\bS = \bC^{-1}.
\end{equation}
For microstructures with cubic symmetry, the mechanical properties can be expressed as:
\begin{equation}
    E=\frac{1}{S_{11}}, \nu = -\frac{S_{12}}{S_{11}}, G=\frac{1}{S_{44}}.
\end{equation}


\section{Implementation details}
\label{sec:imple_detail}
We train the Holoplane autoencoder using the SDF of the mesh voxelized at a resolution of $128^3$ as input.
We compute the displacement field $\chi_\Omega$ for each microstructure using the homogenization solver.

During sampling, for a point $\holo(x, y)$ on Holoplane, we simultaneously sample $\holo(y, x)$, and their average is fed into the decoder as input. 
This ensures that the generated structure adheres strictly to the 48 symmetry operations defined by $O_h$.


To achieve smoother field representations, we follow the approach in \cite{Shue2023} and introduce total variation (TV) loss and explicit density regularization (EDR) loss. 
We also use an $L_2$-norm to regularize the Holoplane distribution toward a standard normal distribution. 
The total loss for training the autoencoder is expressed as:
\begin{equation}
    \mathcal{L} = \mathcal{L}_\phi + 
                  \lambda_{0}\mathcal{L}_{\chi} + 
                  \lambda_{1}\mathcal{L}_{E} + 
                  \lambda_{2}\mathcal{L}_{\text{TV}} + 
                  \lambda_{3}\mathcal{L}_{\text{EDR}} + 
                  \lambda_{4}\mathcal{L}_{2}.
\end{equation}
All training procedures were conducted on a Linux server equipped with four NVIDIA A40 GPUs.
The autoencoder was trained for approximately 72 hours, and the diffusion model required around 120 hours of training.

The minimum and maximum values in our dataset used for error calculation in Sec.~\ref{sec:accuracy} are provided in Tab.~\ref{tab:min_max}.
\begin{table}[ht]
    \centering
    \caption{Minimum and maximum values of the elastic constants ($C_{11}$, $C_{12}$, and $C_{44}$)}
    \label{tab:min_max}
    \begin{tabular}{lccc}
        \toprule
                        & C\textsubscript{11} & C\textsubscript{12} & C\textsubscript{44} \\
        \midrule
        Min             & 0.0004 & -0.0078 & 0.0000 \\
        Max             & 0.6209 & 0.2368 & 0.1591 \\
        \bottomrule
    \end{tabular}
\end{table}

\section{Heterogeneous design details}
\label{sec:heterogeneous}
With MIND, we can input material properties into the network to generate corresponding microstructures.
However, for a given input design, we must determine the distribution of material properties that best approximates the specified target behavior. 
To achieve this, we implement a linear finite element method (FEM) based on hexahedron discretization to simulate the behavior of a given design. Using the regular hexahedron mesh, we compute the static equilibrium state by solving the following minimization problem:
\begin{equation}
    \argmin_\bu \frac{1}{2}\bu^\intercal \bK \bu - \bff_{ext}^\intercal \bu \ ,
\end{equation}
where $\bu$ represents the nodal displacements, $\bff_{ext}$ denotes the applied external forces, and $\bK$ is the global stiffness matrix evaluated from the material properties $\mathbf{C}_i$ of each element. 
To handle fixed vertices, we filter the Hessian and gradient for the corresponding (DoFs) prior to solving the linear system. This minimization problem is solved using a standard Newton's solver. With this setup, our neural networks can seamlessly incorporate material properties derived from this computational model.

\paragraph{Optimizing material properties} The computational model allows us to predict the behaviors of a given design with specified material properties. In order to find material properties that best approximate the target behavior, we optimize material properties by solving the following inverse problem,
\begin{equation}
    \argmin_\bp T_u = \sum_k\|\bu_k(\bp_s) - \hat{\bu}_k\|^2 \quad\text{s.t.}\quad  \bff(\bp_s) = \mathbf{0},
    \label{eq:designobjective_nolimiting}
\end{equation}
where $\bu_k$ and $\hat{\bu}_k$ are the deformed and target displacements for specified vertices, $\bp_s = S^L(\bp)$ is the smoothed material properties via Laplacian smoothing operations $S$. 
The above design objective allows us to compute the optimal material properties that minimize the difference to the target behavior. 
Even though we can solve the above optimization problem with bounds constraints for $\bp$, the optimized material properties might be outside of the feasible region of the network capabilities. Similar to \cite{tang2023beyond}, we represent the feasible space of the three-dimensional material property space $\mathbf{C}$ as a triangle mesh $\ssm$.  We can leverage collision detection algorithms to constrain the material properties. To this end, we interpret the material properties $\mathbf{C}_i$ of each FEM element as a point in three-dimensional material space. Instead of enforcing points strictly within the mesh of material space, we use soft constraints that penalize the points in material space outside the mesh. We first compute the distance $d_i$ of $\bs_i$ to the closest primitive---point, edge, or triangle---on $\ssm$ using standard geometry tests and bounding volume hierarchies for acceleration. We then set up smooth, unilateral penalty functions as
\begin{equation}
    T^m_i = \begin{cases}
    k_m d_i^2 & d_i > 0 \\
    0 & d_i\leq 0
    \end{cases} \ ,
    \label{eq:materialSpaceLimiting}
\end{equation}
where $k_m$ is the penalty stiffness. Since we do not have to strictly enforce the material properties within the boundary of the feasible material space, we find a soft stiffness with a value of $k_m=1.0$ is suitable for all the cases. By combining (\ref{eq:designobjective_nolimiting}) and (\ref{eq:materialSpaceLimiting}), we obtain the design objective for optimizing feasible material properties
\begin{equation}
    \argmin_\bp T(\bp) = T_u(\bp) + \sum_i T_i^m(\bp)
    \quad\text{s.t.}\quad \bff(\bp_s) = \mathbf{0} \ .
    \label{eq:materialOpt}
\end{equation}
We solve this design objective by casting it as an unconstrained optimization problem based on sensitivity analysis of the static equilibrium state. Based on the computed design objective gradient from the sensitivity matrix, we solve this problem via L-BFGS-B~\cite{L-BFGS-B} with bound constraints to enforce the physical meaning of material properties. Note that we optimize the material properties $\bp$ in the design objective, however, we use the smoothed material properties $\bp_s$ to generate microstructures from neural networks.

For cases with limited resources or requiring weight control, we further develop a design objective to maximize material utilization for a specified target,
\begin{equation}
    \begin{gathered}
    \argmin_\bp T(\bp) = T_u(\bp) + \sum_i T_i^m(\bp)
    \\ \quad\text{s.t.}\quad \bff(\bp_s) = \mathbf{0}  \quad\text{and}\quad \sum_i E_i = b \ ,
    \label{eq:materialOpt_budget}
    \end{gathered}
\end{equation}
where we use Young's modulus as the cost for each element. We solve this problem by using Ipopt~\cite{Ipopt} with bounds constraints for material properties.


\section{Visualization of generated Results}
\label{sec:more_vis}


We visualize additional interpolation results between different types of microstructures (Fig.\ref{fig:supp_interpolation}). 
Furthermore, we provide additional generated results along with their corresponding mechanical properties for reference (Fig.\ref{fig:supp_reference}).

\begin{figure}
    \centering
    \includegraphics[width=\linewidth]{figures_new/supp_interpolation.jpg}
    \caption{Interpolation across distinct microstructure families. Two microstructures of different types are selected as the start and end points, and an interpolation sequence is generated between them.}
    \label{fig:supp_interpolation}
\end{figure} 


\begin{figure*}[tb]
    \centering
    \includegraphics[width=\linewidth]{figures_new/supp_reference.png}
    \caption{More generated results from a reference model.
Using the reference model's mechanical properties as input, five candidate models generated by our framework are shown. The properties values ($E,\nu,G$) for both the reference model and generated models are listed. }
    \label{fig:supp_reference}
\end{figure*} 

