\section{Related Work}
\label{sec:related}


\subsection{Microstructure Modeling}

The study of forward modeling and optimization of microstructures for metamaterial design has been a prominent area of research since the late 20th century~\cite{Lakes1987}.
A common approach involves representing microstructures using parametric building blocks, with key design variables such as rod radius or shell thickness. This strategy simplifies the design space and facilitates the creation of various morphological types, including truss-based~\cite{nazir2019Stateoftheart, ling2019Mechanical, choi2016Nonlinear, panetta2015Elastic,Liu2022}, shell-based~\cite{bonatti2019Mechanical, overvelde2016Threedimensional, ion2016Metamaterial, ion2018Metamaterial, liu2022Parametric}, and plate-based structures~\cite{wang2020Quasiperiodica,Sun2023}.
Alternatively, microstructures can be designed using mathematical functions, such as Voronoi tessellations and their variants~\cite{martinez2016Procedural, martinez2017Orthotropic, martinez2018Polyhedral}, Gaussian kernels~\cite{tian2020Organic,Bastek2023}, signed distance fields with pre-computed structures~\cite{schumacher2015Microstructures}, or triply periodic minimal surfaces (TPMS)~\cite{hu2020Cellular, yan2020Strong, hu2019Lightweight}.
%
Although these methods can produce valid geometries, they are limited to specific types of microstructures, creating discrete families within the property space. Interpolating between these families is also challenging.

TO is a key approach for microstructure design~\cite{Sigmund1994, Coelho2007, Zhang2023Optimized}. It aims to identify optimal topologies that meet material property requirements while minimizing cost. By optimizing the density distribution of periodic microstructures and using homogenization to compute effective properties, TO can yield desirable results. However, aside from computational complexity, it faces challenges with boundary compatibility control, particularly when adapting to heterogeneous infilling scenarios~\cite{Cheng2017,wu2018Infill}.

\subsection{Microstructure Inverse Design}

Existing inverse design methods in machine learning and deep learning (ML/DL) can be broadly categorized into three types~\cite{Lee2023}: direct mapping, cascaded neural networks, and conditional generative models.
Direct mapping approaches use regression or surrogate models to directly link material properties to design parameters \cite{Li2020, wang2020DataDriven, bostanabad2019Globally, bastek2022Inverting}. While effective in some cases, these models face limitations such as restricted design spaces and challenges with non-uniqueness and structural similarity.
Cascaded neural networks tackle these issues by combining inverse design and forward modeling in a two-phase approach, ensuring intermediate designs remain consistent with existing data~\cite{Liu2018}.
Conditional generative models, including generative adversarial networks (GANs) \cite{goodfellow2020Generative} and variational autoencoders (VAEs) \cite{KingmaVAE}, are widely used for achieving one-to-many mappings. Previous attempts \cite{zheng2021Controllable, wang2022IHGAN}, \cite{Ma2019, wang2020Deep} have focused on preset structure classes. More recent work has explored denoising diffusion probabilistic models (DDPM) \cite{Ho2020DDPM} for the inverse design of 2D \cite{Wang2024} and 3D \cite{Yang2024} microstructures. However, these models struggle with matching geometry and property distributions, particularly when handling diverse or class-free microstructure data, which reduces their matching accuracy.



From the perspective of the representation of microstructures, existing inverse design methods can be generally categorized into two main types: 
(1) Pixel or voxel-based representations, which have been effective for modeling composite materials and multi-phase structures that occupy the entire microstructure volume~\cite{li2018Deep, Li2020, Hsu2020, noguchi2021Stochastic}. However, this approach is highly inefficient for single-material structures or sparse 3D geometries~\cite{Yang2024}, and neglects the geometric connectivity constraints, which are crucial for microstructures fabricated from a single material.
(2) Pre-computed parametric representations, which utilize specific families of design parameters of building blocks~\cite{wang2021DataDriven,Zheng2023,Ha2023}, implicit functions  (e.g., TPMS \cite{Li2019} or spinodoids \cite{Kumar2020}), and are typically solved through inverse searches~\cite{schumacher2015Microstructures, panetta2015Elastic}, differential parameter optimization~\cite{tozoni2020Lowparametric}, or data-driven regression~\cite{bastek2022Inverting, zheng2021Controllable, wang2022IHGAN}. Although these methods offer greater computational efficiency, they are constrained by predefined structural classes, which restrict the diversity and flexibility of the design space.
%
On the contrary, we employ implicit neural representations for microstructure design, which enhance geometric integrity and resolution independence in the generated structures. Recent work on metamaterial sequences within plate lattices \cite{Zhao2024} has demonstrated the effectiveness of implicit neural representations. Building on this, our approach further integrates explicit encoding to enforce microstructural symmetry.


Recent studies have also explored mapping nonlinear properties to 2D microstructure designs using diffusion models, going beyond linear elastic properties~\cite{Bastek2023, Vlassis2023, Li2023, Park2024}.
These approaches share similar strategies, aiming to limit the degrees of freedom in the structures and reduce the complexity of the property distribution space, thereby enhancing matching accuracy during training. However, challenges persist in aligning non-parametric geometries with their corresponding properties.

