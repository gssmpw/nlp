 \section{Conclusion and Future Work}
 \label{sec:conclusion}

In this paper, we propose a generative model for the inverse design of 3D tileable microstructures. 
Our end-to-end model integrates fabrication constraints and generates multiple microstructures with desired physical properties. 
Unlike existing methods that rely on specific parameterizations or constrained design spaces, our approach operates directly on the geometry, offering a more flexible and general framework. The proposed latent diffusion model, built upon Holoplane, encodes both geometry and properties in an explicit-implicit hybrid form, improving the alignment between geometry and property distributions.
This results in higher property matching accuracy, enhanced validity control over the generated microstructures, and better boundary compatibility compared to existing methods. The generated microstructures span various architecture types, forming a continuous functional space with diverse morphologies.



The proposed framework has certain limitations that we plan to address in future work. More fabrication challenges such as self-supportiveness and the absence of closed pockets remain unaddressed and can currently only be ensured through post-filtering. Incorporating these fabrication constraints into the loss function would be a valuable direction for future investigation, especially given recent progress in this area~\cite{Chen2024,Guo2024PhysComp}.
Additionally, we believe our workflow could be expanded to include properties beyond Young's modulus, Poisson's ratio, and shear modulus. Incorporating properties such as isotropy, thermal conductivity, optical behavior, and others could broaden the range of achievable microstructures and facilitate the design of materials with tailored multi-functionalities.

