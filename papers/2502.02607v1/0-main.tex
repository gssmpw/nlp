
\def\sigjournal{} % comment out this line to enable the conference format
\ifx\sigjournal\undefined
\documentclass[sigconf,screen,nonacm]{acmart}
\else
\documentclass[acmtog,screen,nonacm]{acmart}
\fi

\usepackage{booktabs} % For formal tables
\citestyle{acmauthoryear}
\usepackage{graphicx}
\usepackage{wrapfig}
\usepackage{algorithm}
\usepackage[noend]{algpseudocode}
\usepackage{amsmath}
\usepackage{enumitem}

\newcommand{\dummyfig}[1]{
  \centering
  \fbox{
    \begin{minipage}[c][0.15\textheight][c]{0.3\textwidth}
      \centering{#1}
    \end{minipage}
  }
}
\newcommand{\ms}{microstructure}
\newcommand{\mss}{microstructures}
\newcommand{\etal}{et al.}
\newcommand{\etc}{\textit{etc}}
\renewcommand{\paragraph}[1]{\noindent\textbf{#1.}}
\newcommand{\llu}[1]{\textcolor{red}{\textbf{LL:} \textit{#1}}}
\newcommand{\bw}[1]{{\color{blue} #1}}
\newcommand{\tim}[1]{\textcolor{magenta}{\textbf{Tim:} \textit{#1}}}
\newcommand{\bb}[1]{{\color{red} #1}}
\newcommand{\hs}[1]{{\bfseries \color{brown} HZ: #1}}
\newcommand{\pmh}[1]{\textcolor{violet}{\textbf{PH:} \textit{#1}}}

\newcommand{\hp}[1]{\textcolor{ACMDarkBlue}{\textbf{Hao:} #1} }
\newcommand{\figref}[1]{Fig.~\ref{#1}}
\newcommand{\secref}[1]{Sec.~\ref{sec:#1}}

\newcommand{\R}{\mathbb{R}}

%\usepackage{todonotes}
%!TEX root = ../Main.tex
% Defines commands to easily input variables and equations into the paper.

% Matrix commands for simple matrices
\newcommand{\pMat}[1]{\begin{pmatrix}#1\end{pmatrix}}
\newcommand{\bMat}[1]{\begin{bmatrix}#1\end{bmatrix}}
\newcommand{\BMat}[1]{\begin{Bmatrix}#1\end{Bmatrix}}
\newcommand{\aMat}[1]{\begin{Vmatrix}#1\end{Vmatrix}}
\newcommand{\mat}[1]{\begin{matrix}#1\end{matrix}}

% Pat commands
\newcommand{\cross}[1]{ {[#1]}_{\times} }

% Reference Frames
\newcommand{\IFrame}{\mathcal{I}}
\newcommand{\BFrame}{\mathcal{B}}
\newcommand{\COM}{CoM}

% State definitions
\newcommand{\roll}{\theta}
\newcommand{\pitch}{\phi}
\newcommand{\yaw}{\psi}

\newcommand{\rollhat}{\hat{\theta}}
\newcommand{\pitchhat}{\hat{\phi}}
\newcommand{\yawhat}{\hat{\psi}}



\newcommand{\dRoll}{\dot{\roll}}
\newcommand{\dPitch}{\dot{\pitch}}
\newcommand{\dYaw}{\dot{\yaw}}

\newcommand{\omegaX}{\omega_x}
\newcommand{\omegaY}{\omega_y}
\newcommand{\omegaZ}{\omega_z}

\newcommand{\dX}{\dot{x}}
\newcommand{\dY}{\dot{y}}
\newcommand{\dZ}{\dot{z}}

\newcommand{\quat}{\bm{q}}
\newcommand{\pos}{\bm{p}}
\newcommand{\angVel}{\bm{\omega}}
\newcommand{\vel}{\bm{v}}
\newcommand{\posDot}{\bm{\dot{\pos}}}
\newcommand{\posDotDot}{\bm{\ddot{p}}}
\newcommand{\quatDot}{\bm{\dot{\quat}}}
\newcommand{\angVelDot}{\bm{\dot{\angVel}}}

% Desired Variables
\newcommand{\xd}{\bm{x}_d}
\newcommand{\uref}{\bm{u}_{ref}}

% State component vectors
\newcommand{\quatV}{q_0 \\ q_x \\ q_y \\ q_z}
\newcommand{\quatVector}{\BMat{\quatV}}
\newcommand{\posV}{x \\ y \\ z}
\newcommand{\posVector}{\BMat{\posV}}
\newcommand{\angVelV}{\omegaX \\ \omegaY \\ \omegaZ}
\newcommand{\angVelVector}{\BMat{\angVelV}}
\newcommand{\velV}{\dX \\ \dY \\ \dZ}
\newcommand{\velVector}{\BMat{\velV}}

\newcommand{\rpyV}{\roll \\ \pitch \\ \yaw}
\newcommand{\rpyVector}{\BMat{\rpyV}}
\newcommand{\rpyDotV}{\dRoll \\ \dPitch \\ \dYaw}
\newcommand{\rpyDotVector}{\BMat{\rpyDotV}}

% Parameter Estimates
\newcommand{\state}{\bm{x}}
\newcommand{\stateDot}{\bm{\dot{\state}}}
\newcommand{\stateEstimate}{\bm{\hat{x}}}
\newcommand{\rFootEstimate}{\bm{\hat{r}}}
\newcommand{\forceEstimate}{\bm{\hat{F}}}
\newcommand{\rFootEstimatef}{\bm{\hat{r}}_f}
\newcommand{\forceEstimatef}{\bm{\hat{F}}_f}
\newcommand{\inputU}{\bm{u}}
\newcommand{\inputEstimate}{\bm{\hat{u}}}
\newcommand{\outputEstimate}{\bm{\hat{y}}}

% State and Input Vectors
\newcommand{\stateVector}{\BMat{\posV \\ \quatV \\ \velV \\ \angVelV}}
\newcommand{\stateVectorRed}{\BMat{\posV \\ \rpyV \\ \velV \\ \rpyDotV}}
\newcommand{\inputVector}{\BMat{\rFootEstimate_1 \\ \forceEstimate_1 \\ \rFootEstimate_2 \\ \forceEstimate_2 \\ \rFootEstimate_3 \\ \forceEstimate_3 \\ \rFootEstimate_4 \\ \forceEstimate_4}}

% Variable Transformations
\newcommand{\quatToRPY}{\bMat{\tan^{-1}\pMat{\dfrac{2\pMat{q_0q_1+q_2q_3}}{1-2\pMat{q_1^2+q_2^2}}}\\ \sin^{-1}\pMat{2\pMat{q_0q_2+q_3q_1}} \\ \tan^{-1}\pMat{\dfrac{2\pMat{q_0q_3+q_1q_2}}{1-2\pMat{q_2^2+q_3^2}}}}}
\newcommand{\omegaToRPYDot}{\bMat{\dfrac{\cos\pMat{\yaw}}{\cos\pMat{\pitch}} & \dfrac{\sin\pMat{\yaw}}{\cos\pMat{\pitch}} & 0 \\ -\sin\pMat{\yaw} & \cos\pMat{\yaw} & 0 \\ \dfrac{\cos\pMat{\yaw}\sin\pMat{\pitch}}{\cos\pMat{\pitch}} & \dfrac{\sin\pMat{\yaw}\sin\pMat{\pitch}}{\cos\pMat{\pitch}} & 1}}

% Time step definitions
\newcommand{\n}[1]{#1_n}
\renewcommand{\k}[1]{#1_k}
\newcommand{\nk}[1]{#1_{i}}

% Decision Variables Full
\newcommand{\inputVarsV}{\rFootEstimate \\ \forceEstimate}
\newcommand{\inputVarsVectorf}[1]{\BMat{\inputVarsV}_#1}
\newcommand{\decisionVarsVFull}{\inputVarsVectorf{1} \\ \vdots \\ \inputVarsVectorf{4} \\ \stateEstimate}
\newcommand{\decisionVarsVectorFull}[2]{\BMat{\decisionVarsVFull}_{#1,#2}}
\newcommand{\decisionArrayVFull}{\decisionVarsVectorFull{1}{1} & \hdots & \decisionVarsVectorFull{1}{K_1} & \hdots & \decisionVarsVectorFull{N}{1} & \hdots & \decisionVarsVectorFull{N}{K_N}}
\newcommand{\decisionArrayVectorFull}{\bMat{\decisionArrayVFull}}

% Decision Variables 
\newcommand{\decisionVarsV}{\stateEstimate \\ \inputEstimate}
\newcommand{\decisionVarsVector}{\BMat{\decisionVarsV}}
\newcommand{\decisionArrayV}{\decisionVarsVector_{1,1} & \hdots & \decisionVarsVector_{N,K_N}}
\newcommand{\decisionArrayVector}{\bMat{\decisionArrayV}}

% Simple Dynamics
\newcommand{\forces}{\bm{f}}
\newcommand{\torques}{\bm{\tau}}
\newcommand{\forceIEstimate}{\phantom{{}^\IFrame}\bm{\hat{f}}}
\newcommand{\torquesBEstimate}{^\BFrame\bm{\hat{\tau}}}
\newcommand{\RMatItoB}{^\BFrame\mathbf{R}_\IFrame}
\newcommand{\RMatItoBEstimate}{^\BFrame\mathbf{\hat{R}}_\IFrame}
\newcommand{\RMatBtoI}{^\IFrame\mathbf{R}_\BFrame}
\newcommand{\inputForceTorqueFunction}{ \hat{h}\!\pMat{\nk{\stateEstimate}, \nk{\inputEstimate}}}
\newcommand{\inputForceTorque}{\BMat{\phantom{{}^\BFrame}\forces \\ {}^\BFrame\torques}}
\newcommand{\inputForceTorqueMPC}{\BMat{\forceIEstimate \\ \torquesBEstimate}}
\newcommand{\forceTorqueTransform}{\bMat{\mathbf{I} & 0 \\ 0 & {\bm{R}_z^T(\yawhat)}}}
\newcommand{\uToForceTorqueMat}{\bMat{\mat{\mathbf{I} \\ \bMat{\bm{r}_1}_\times} & \hdots & \mat{\mathbf{I} \\ \bMat{\bm{r}_f}_\times}}}
\newcommand{\uToForceTorqueMatEstimate}{\bMat{\mat{\mathbf{I} \\ \bMat{\rFootEstimate_1}_\times} & \hdots & \mat{\mathbf{I} \\ \bMat{\rFootEstimate_f}_\times}}}
% Equation
\newcommand{\AMatnk}{{\bm{A}}_i}
\newcommand{\BMatnk}{{\bm{B}}_i}
\newcommand{\simpleDynamicsEq}{\AMatnk\nk{\stateEstimate}+\BMatnk\inputForceTorqueFunction+\bm{d}_i}
\newcommand{\simpleDynamics}{\stateEstimate_{i+1}=\simpleDynamicsEq}
% Matrix Linear
\newcommand{\AMatFull}{\bMat{\mathbf{I}_6 & \Delta t_i\mathbf{I}_6 \\ 0 & \mathbf{I}_6}}
\newcommand{\BMatFull}{\bMat{\frac{\Delta t_i^2}{2} \,{}^{\BFrame}\!\bm{I}^{-1} \\[.75ex] \Delta t_i\, {}^{\BFrame}\!\bm{I}^{-1} }}

\newcommand{\BMatFcull}{\bMat{\bMat{\frac{dt_n^2}{2}\mathbf{I}_6 \\ dt_n\mathbf{I}_6}\bMat{\frac{1}{m}\mathbf{I}_3 & 0 \\ 0 & ^\BFrame\bm{I}^{-1}}}}
	
	
	
% Full Dynamics
\newcommand{\sumForces}{\sum^\IFrame\bm{F}}
\newcommand{\sumTorques}{\sum^\IFrame\bm{\tau}}
\newcommand{\stateReducedVector}{\BMat{\pos \\ {}^\IFrame\!\bm{R}_{\BFrame} \\ \posDot \\ \angVel}}

\newcommand{\dynamicsReducedVector}{\BMat{
\posDot_c \\[.5ex]
{}^\IFrame\!\bm{R}_{\BFrame} \left[ {}^\BFrame \angVel\right]_\times \\[.5ex] 
\frac{1}{m}\forces - \bm{g} \\[.5ex]  
{}^\BFrame\,\overline{\!\bm{I}}{}^{-1}\pMat{{}^\BFrame\torques - {}^\BFrame \angVel\times {{}^\BFrame\,\overline{\!\bm{I}}}\,\,{\vphantom{\bm{I}}}^\BFrame\angVel}
}}


\newcommand{\stateDotReducedVector}{\BMat{\posDot_c \\ {}^\IFrame\!\dot{\bm{R}}_{\BFrame} \\ \posDotDot_c \\ {}^\BFrame\angVelDot}}

% Reference Input
\newcommand{\KPercent}{K_\%}
\newcommand{\pFootRef}{\bm{p}_{ref,f,xy,k} = \bm{\hat{p}}_{h,xy} + \posDot_{d,xy}T_{s}\pMat{\frac{1}{2}-\KPercent}+\\\pMat{\bm{\dot{\hat{p}}}_{xy} -\posDot_{d,xy}}\sqrt{\frac{\hat{p}_z}{\aMat{\bm{g}}}}\pMat{1 - \KPercent}}
			
% Error definitions
\newcommand{\stateError}{\bm{\tilde{x}}}
\newcommand{\inputError}{\bm{\tilde{u}}}
\newcommand{\stateErrorFull}{\xd-\outputEstimate}
\newcommand{\inputErrorFull}{\uref-\inputEstimate}
			
% Basic Cost Function
\newcommand{\basicArgs}{\stateEstimate, \hat{\bm{u}}}
\newcommand{\basicMin}{\underset{\basicArgs}{\text{min}}}
\newcommand{\basicCost}{J\pMat{\basicArgs}}
\newcommand{\basicCostFunction}{\basicMin\basicCost}
			
% MPC Function
\newcommand{\MPCSum}{\sum\limits_{n=1}^{N}\sum\limits_{k=1}^{\n{K}}}
%\newcommand{\MPCMin}{\underset{\stateEstimate, \inputEstimate}{\text{min}}}
\newcommand{\MPCMin}{\underset{\chi}{\text{min}}}
\newcommand{\stateCost}{\stateError^T\bm{Q}\stateError}
\newcommand{\inputCost}{\inputError^T\bm{R}\inputError}
\newcommand{\MPCCost}{\stateCost+\inputCost}
\newcommand{\MPCCostFunction}{\MPCMin\MPCSum\pMat{\MPCCost}_{n,k}}
			
% MPC Function Full
\newcommand{\MPCMinFull}{\underset{\stateEstimate, \inputEstimate}{\text{min}}}
\newcommand{\stateCostFull}{\pMat{\stateErrorFull}^T\bm{Q}\pMat{\stateErrorFull}}
\newcommand{\inputCostFull}{\pMat{\inputErrorFull}^T\bm{R}\pMat{\inputErrorFull}}
\newcommand{\MPCCostFull}{\stateCostFull+\inputCostFull}
\newcommand{\MPCCostFunctionFull}{\MPCMinFull\MPCSum\pMat{\MPCCostFull}_{n,k}}
			
% Constraints
\newcommand{\dynamicsConstraint}{\stateEstimate_{nk+1}=\simpleDynamicsEq}
\newcommand{\groundConstraint}{p_{z,f}=\hat{z}_g \pMat{p_{x,f}, p_{y,f}}}
\newcommand{\kinematicLegConstraint}{\aMat{\bm{r}_{h \to f}} \leq l_{f,max}}
\newcommand{\slipConstraint}{\bm{p}_{f,k+1} = \bm{p}_{f,k}}
\newcommand{\frictionConstraint}[1]{{-}\mu\bm{F}_{z,f}\leq\bm{F}_{#1,f}\leq\mu\bm{F}_{z,f}}
\newcommand{\frictionConeConstraint}{\frictionConstraint{x} \\ \frictionConstraint{y}}




%%%%%%%%%%%%%%%%%%%%%%%%%%%%%%%%%%%%%%%%%%%%%%%%%%%%%%%
%%% CONTACT
%%%%%%%%%%%%%%%%%%%%%%%%%%%%%%%%%%%%%%%%%%%%%%%%%%%%%%%


\newcommand{\legState}{s}
\newcommand{\legStateScheduled}{\legState_{\phi}}
\newcommand{\legStateNotScheduled}{\bar{\legState}_{\phi}}
\newcommand{\legStateEstimated}{\hat{\legState}}
\newcommand{\contactState}{c}
\newcommand{\swingState}{\bar{\contactState}}
\newcommand{\contactStateScheduled}{\contactState_{\phi}}
\newcommand{\swingStateScheduled}{\bar{\contactState}_{\phi}}


%%%%%%%%%%%%%%%%%%%%%%%%%%%%%%%%%%%%%%%%%%%%%%%%%%%%%%%
%%% Support Polygon
%%%%%%%%%%%%%%%%%%%%%%%%%%%%%%%%%%%%%%%%%%%%%%%%%%%%%%%
\newcommand{\phaseGain}{\Phi}
\newcommand{\adjLegP}{^{+}}
\newcommand{\adjLegM}{^{-}}

\begin{document}

\title{MIND: Microstructure INverse Design with Generative Hybrid Neural Representation}



\author{Tianyang Xue}
\email{TimHsue@gmail.com}
\affiliation{%
  \institution{Shandong University}
  \city{Qingdao}
  \country{China}
}

\author{Haochen Li}
\email{lihaochen@mail.sdu.edu.cn}
\affiliation{%
  \institution{Shandong University}
  \city{Qingdao}
  \country{China}
}

\author{Longdu Liu}
\email{liulongdu@163.com}
\affiliation{%
  \institution{Shandong University}
  \city{Qingdao}
  \country{China}
}

\author{Paul Henderson}
\email{paul@pmh47.net}
\affiliation{%
  \institution{University of Glasgow}
  \city{Scotland}
  \country{United Kingdom}
}

\author{Pengbin Tang}
\email{tangpengbin@gmail.com}
\affiliation{%
  \institution{ETH Zurich}
  \city{Zurich}
  \country{Switzerland}
}

\author{Lin Lu}
\authornote{corresponding author}
\email{llu@sdu.edu.cn}
\affiliation{%
  \institution{Shandong University}
  \city{Qingdao}
  \country{China}
}

\author{Jikai Liu}
\email{jikai_liu@sdu.edu.cn}
\affiliation{%
  \institution{Shandong University}
  \city{Jinan}
  \country{China}
}

\author{Haisen Zhao}
\email{haisenzhao@gmail.com}
\affiliation{%
  \institution{Shandong University}
  \city{Qingdao}
  \country{China}
}


\author{Hao Peng}
\email{penghao@crowncad.com}
\affiliation{%
  \institution{CrownCAD}
  \city{Jinan}
  \country{China}
}

\author{Bernd Bickel}
\email{bickelb@ethz.ch}
\affiliation{%
  \institution{ETH Zurich}
  \city{Zurich}
  \country{Switzerland}
}


\begin{abstract}
The inverse design of microstructures plays a pivotal role in optimizing metamaterials with specific, targeted physical properties. While traditional forward design methods are constrained by their inability to explore the vast combinatorial design space, inverse design offers a compelling alternative by directly generating structures that fulfill predefined performance criteria. However, achieving precise control over both geometry and material properties remains a significant challenge due to their intricate interdependence. Existing approaches, which typically rely on voxel or parametric representations, often limit design flexibility and structural diversity. 

In this work, we present a novel generative model that integrates latent diffusion with Holoplane, an advanced hybrid neural representation that simultaneously encodes both geometric and physical properties. This combination ensures superior alignment between geometry and properties. Our approach generalizes across multiple microstructure classes, enabling the generation of diverse, tileable microstructures with significantly improved property accuracy and enhanced control over geometric validity, surpassing the performance of existing methods.
We introduce a multi-class dataset encompassing a variety of geometric morphologies, including truss, shell, tube, and plate structures, to train and validate our model. Experimental results demonstrate the model’s ability to generate microstructures that meet target properties, maintain geometric validity, and integrate seamlessly into complex assemblies. Additionally, we explore the potential of our framework through the generation of new microstructures, cross-class interpolation, and the infilling of heterogeneous microstructures.
The dataset and source code will be open-sourced upon publication.

\end{abstract}

\begin{CCSXML}
<ccs2012>
<concept>
<concept_id>10010147.10010371.10010396</concept_id>
<concept_desc>Computing methodologies~Shape modeling</concept_desc>
<concept_significance>500</concept_significance>
</concept>
<concept>
<concept_id>10010147.10010371.10010387</concept_id>
<concept_desc>Computing methodologies~Graphics systems and interfaces</concept_desc>
<concept_significance>300</concept_significance>
</concept>
</ccs2012>
\end{CCSXML}

\ccsdesc[500]{Computing methodologies~Shape modeling}
\ccsdesc[300]{Computing methodologies~Graphics systems and interfaces}



\keywords{microstructures, generative design, neural networks, additive manufacturing}



\begin{teaserfigure}
  \includegraphics[width=\textwidth]{figures_new/teaser_arxiv.jpg}
\vspace{-20pt}
\caption{We present MIND, the Microstructure INverse Design system, for generating 3D tileable microstructures with specified properties. 
Our generative model produces diverse microstructure types and enables heterogeneous design.
}
  \label{fig:teaser}    
\end{teaserfigure}


\maketitle

\setlength{\belowcaptionskip}{-10pt}

%!TEX root = gcn.tex
\section{Introduction}
Graphs, representing structural data and topology, are widely used across various domains, such as social networks and merchandising transactions.
Graph convolutional networks (GCN)~\cite{iclr/KipfW17} have significantly enhanced model training on these interconnected nodes.
However, these graphs often contain sensitive information that should not be leaked to untrusted parties.
For example, companies may analyze sensitive demographic and behavioral data about users for applications ranging from targeted advertising to personalized medicine.
Given the data-centric nature and analytical power of GCN training, addressing these privacy concerns is imperative.

Secure multi-party computation (MPC)~\cite{crypto/ChaumDG87,crypto/ChenC06,eurocrypt/CiampiRSW22} is a critical tool for privacy-preserving machine learning, enabling mutually distrustful parties to collaboratively train models with privacy protection over inputs and (intermediate) computations.
While research advances (\eg,~\cite{ccs/RatheeRKCGRS20,uss/NgC21,sp21/TanKTW,uss/WatsonWP22,icml/Keller022,ccs/ABY318,folkerts2023redsec}) support secure training on convolutional neural networks (CNNs) efficiently, private GCN training with MPC over graphs remains challenging.

Graph convolutional layers in GCNs involve multiplications with a (normalized) adjacency matrix containing $\numedge$ non-zero values in a $\numnode \times \numnode$ matrix for a graph with $\numnode$ nodes and $\numedge$ edges.
The graphs are typically sparse but large.
One could use the standard Beaver-triple-based protocol to securely perform these sparse matrix multiplications by treating graph convolution as ordinary dense matrix multiplication.
However, this approach incurs $O(\numnode^2)$ communication and memory costs due to computations on irrelevant nodes.
%
Integrating existing cryptographic advances, the initial effort of SecGNN~\cite{tsc/WangZJ23,nips/RanXLWQW23} requires heavy communication or computational overhead.
Recently, CoGNN~\cite{ccs/ZouLSLXX24} optimizes the overhead in terms of  horizontal data partitioning, proposing a semi-honest secure framework.
Research for secure GCN over vertical data  remains nascent.

Current MPC studies, for GCN or not, have primarily targeted settings where participants own different data samples, \ie, horizontally partitioned data~\cite{ccs/ZouLSLXX24}.
MPC specialized for scenarios where parties hold different types of features~\cite{tkde/LiuKZPHYOZY24,icml/CastigliaZ0KBP23,nips/Wang0ZLWL23} is rare.
This paper studies $2$-party secure GCN training for these vertical partition cases, where one party holds private graph topology (\eg, edges) while the other owns private node features.
For instance, LinkedIn holds private social relationships between users, while banks own users' private bank statements.
Such real-world graph structures underpin the relevance of our focus.
To our knowledge, no prior work tackles secure GCN training in this context, which is crucial for cross-silo collaboration.


To realize secure GCN over vertically split data, we tailor MPC protocols for sparse graph convolution, which fundamentally involves sparse (adjacency) matrix multiplication.
Recent studies have begun exploring MPC protocols for sparse matrix multiplication (SMM).
ROOM~\cite{ccs/SchoppmannG0P19}, a seminal work on SMM, requires foreknowledge of sparsity types: whether the input matrices are row-sparse or column-sparse.
Unfortunately, GCN typically trains on graphs with arbitrary sparsity, where nodes have varying degrees and no specific sparsity constraints.
Moreover, the adjacency matrix in GCN often contains a self-loop operation represented by adding the identity matrix, which is neither row- nor column-sparse.
Araki~\etal~\cite{ccs/Araki0OPRT21} avoid this limitation in their scalable, secure graph analysis work, yet it does not cover vertical partition.

% and related primitives
To bridge this gap, we propose a secure sparse matrix multiplication protocol, \osmm, achieving \emph{accurate, efficient, and secure GCN training over vertical data} for the first time.

\subsection{New Techniques for Sparse Matrices}
The cost of evaluating a GCN layer is dominated by SMM in the form of $\adjmat\feamat$, where $\adjmat$ is a sparse adjacency matrix of a (directed) graph $\graph$ and $\feamat$ is a dense matrix of node features.
For unrelated nodes, which often constitute a substantial portion, the element-wise products $0\cdot x$ are always zero.
Our efficient MPC design 
avoids unnecessary secure computation over unrelated nodes by focusing on computing non-zero results while concealing the sparse topology.
We achieve this~by:
1) decomposing the sparse matrix $\adjmat$ into a product of matrices (\S\ref{sec::sgc}), including permutation and binary diagonal matrices, that can \emph{faithfully} represent the original graph topology;
2) devising specialized protocols (\S\ref{sec::smm_protocol}) for efficiently multiplying the structured matrices while hiding sparsity topology.


 
\subsubsection{Sparse Matrix Decomposition}
We decompose adjacency matrix $\adjmat$ of $\graph$ into two bipartite graphs: one represented by sparse matrix $\adjout$, linking the out-degree nodes to edges, the other 
by sparse matrix $\adjin$,
linking edges to in-degree nodes.

%\ie, we decompose $\adjmat$ into $\adjout \adjin$, where $\adjout$ and $\adjin$ are sparse matrices representing these connections.
%linking out-degree nodes to edges and edges to in-degree nodes of $\graph$, respectively.

We then permute the columns of $\adjout$ and the rows of $\adjin$ so that the permuted matrices $\adjout'$ and $\adjin'$ have non-zero positions with \emph{monotonically non-decreasing} row and column indices.
A permutation $\sigma$ is used to preserve the edge topology, leading to an initial decomposition of $\adjmat = \adjout'\sigma \adjin'$.
This is further refined into a sequence of \emph{linear transformations}, 
which can be efficiently computed by our MPC protocols for 
\emph{oblivious permutation}
%($\Pi_{\ssp}$) 
and \emph{oblivious selection-multiplication}.
% ($\Pi_\SM$)
\iffalse
Our approach leverages bipartite graph representation and the monotonicity of non-zero positions to decompose a general sparse matrix into linear transformations, enhancing the efficiency of our MPC protocols.
\fi
Our decomposition approach is not limited to GCNs but also general~SMM 
by 
%simply 
treating them 
as adjacency matrices.
%of a graph.
%Since any sparse matrix can be viewed 

%allowing the same technique to be applied.

 
\subsubsection{New Protocols for Linear Transformations}
\emph{Oblivious permutation} (OP) is a two-party protocol taking a private permutation $\sigma$ and a private vector $\xvec$ from the two parties, respectively, and generating a secret share $\l\sigma \xvec\r$ between them.
Our OP protocol employs correlated randomnesses generated in an input-independent offline phase to mask $\sigma$ and $\xvec$ for secure computations on intermediate results, requiring only $1$ round in the online phase (\cf, $\ge 2$ in previous works~\cite{ccs/AsharovHIKNPTT22, ccs/Araki0OPRT21}).

Another crucial two-party protocol in our work is \emph{oblivious selection-multiplication} (OSM).
It takes a private bit~$s$ from a party and secret share $\l x\r$ of an arithmetic number~$x$ owned by the two parties as input and generates secret share $\l sx\r$.
%between them.
%Like our OP protocol, o
Our $1$-round OSM protocol also uses pre-computed randomnesses to mask $s$ and $x$.
%for secure computations.
Compared to the Beaver-triple-based~\cite{crypto/Beaver91a} and oblivious-transfer (OT)-based approaches~\cite{pkc/Tzeng02}, our protocol saves ${\sim}50\%$ of online communication while having the same offline communication and round complexities.

By decomposing the sparse matrix into linear transformations and applying our specialized protocols, our \osmm protocol
%($\prosmm$) 
reduces the complexity of evaluating $\numnode \times \numnode$ sparse matrices with $\numedge$ non-zero values from $O(\numnode^2)$ to $O(\numedge)$.

%(\S\ref{sec::secgcn})
\subsection{\cgnn: Secure GCN made Efficient}
Supported by our new sparsity techniques, we build \cgnn, 
a two-party computation (2PC) framework for GCN inference and training over vertical
%ly split
data.
Our contributions include:

1) We are the first to explore sparsity over vertically split, secret-shared data in MPC, enabling decompositions of sparse matrices with arbitrary sparsity and isolating computations that can be performed in plaintext without sacrificing privacy.

2) We propose two efficient $2$PC primitives for OP and OSM, both optimally single-round.
Combined with our sparse matrix decomposition approach, our \osmm protocol ($\prosmm$) achieves constant-round communication costs of $O(\numedge)$, reducing memory requirements and avoiding out-of-memory errors for large matrices.
In practice, it saves $99\%+$ communication
%(Table~\ref{table:comm_smm}) 
and reduces ${\sim}72\%$ memory usage over large $(5000\times5000)$ matrices compared with using Beaver triples.
%(Table~\ref{table:mem_smm_sparse}) ${\sim}16\%$-

3) We build an end-to-end secure GCN framework for inference and training over vertically split data, maintaining accuracy on par with plaintext computations.
We will open-source our evaluation code for research and deployment.

To evaluate the performance of $\cgnn$, we conducted extensive experiments over three standard graph datasets (Cora~\cite{aim/SenNBGGE08}, Citeseer~\cite{dl/GilesBL98}, and Pubmed~\cite{ijcnlp/DernoncourtL17}),
reporting communication, memory usage, accuracy, and running time under varying network conditions, along with an ablation study with or without \osmm.
Below, we highlight our key achievements.

\textit{Communication (\S\ref{sec::comm_compare_gcn}).}
$\cgnn$ saves communication by $50$-$80\%$.
(\cf,~CoGNN~\cite{ccs/KotiKPG24}, OblivGNN~\cite{uss/XuL0AYY24}).

\textit{Memory usage (\S\ref{sec::smmmemory}).}
\cgnn alleviates out-of-memory problems of using %the standard 
Beaver-triples~\cite{crypto/Beaver91a} for large datasets.

\textit{Accuracy (\S\ref{sec::acc_compare_gcn}).}
$\cgnn$ achieves inference and training accuracy comparable to plaintext counterparts.
%training accuracy $\{76\%$, $65.1\%$, $75.2\%\}$ comparable to $\{75.7\%$, $65.4\%$, $74.5\%\}$ in plaintext.

{\textit{Computational efficiency (\S\ref{sec::time_net}).}} 
%If the network is worse in bandwidth and better in latency, $\cgnn$ shows more benefits.
$\cgnn$ is faster by $6$-$45\%$ in inference and $28$-$95\%$ in training across various networks and excels in narrow-bandwidth and low-latency~ones.

{\textit{Impact of \osmm (\S\ref{sec:ablation}).}}
Our \osmm protocol shows a $10$-$42\times$ speed-up for $5000\times 5000$ matrices and saves $10$-2$1\%$ memory for ``small'' datasets and up to $90\%$+ for larger ones.

\section{Related Works}

\textbf{Enhancing LLMs' Theory of Mind.} There has been systematic evaluation that revealed LLMs' limitations in achieving robust Theory of Mind inference \citep{ullman2023large, shapira2023clever}. To enhance LLMs' Theory of Mind capacity, recent works have proposed various prompting techniques. For instance, SimToM \citep{wilf2023think} encourages LLMs to adopt perspective-taking, PercepToM \citep{jung2024perceptions} improves perception-to-belief inference by extracting relevant contextual details, and \citet{huang2024notion} utilize an LLM as a world model to track environmental changes and refine prompts. Explicit symbolic modules also seem to improve LLM's accuracy through dynamic updates based on inputs. Specifically, TimeToM \citep{hou2024timetom} constructs a temporal reasoning framework to support inference, while SymbolicToM \citep{sclar2023minding} uses graphical representations to track characters' beliefs. Additionally, \citet{wagner2024mind} investigates ToM's necessity and the level of recursion required for specific tasks. However, these approaches continue to exhibit systematic errors in long contexts, complex behaviors, and recursive reasoning due to inherent limitations in inference and modeling \citep{jin2024mmtom,shi2024muma}. Most of them rely on domain-specific designs, lacking open-endedness.


\textbf{Model-based Theory of Mind inference.} Model-based Theory of Mind inference, in particular, Bayesian inverse planning (BIP) \citep{baker2009action,ullman2009help,baker2017rational,zhi2020online}, explicitly constructs representations of agents' mental states and how mental states guide agents' behavior via Bayesian Theory of Mind (BToM) models. These methods can reverse engineer human ToM inference in simple domains \citep[e.g.,][]{baker2017rational,netanyahu2021phase,shu2021agent}. Recent works have proposed to combine BIP with LLMs to achieve robust ToM inference in more realistic settings \citep{ying2023neuro, jin2024mmtom, shi2024muma}. However, these methods require manual specification of the BToM models as well as rigid, domain-specific implementations of Bayesian inference, limiting their adaptability to open-ended scenarios. To overcome this limitation, we propose \ours, a method capable of automatically modeling mental variables across diverse conditions and conducting automated BIP without domain-specific knowledge or implementations.


\begin{figure*}[ht]
  \centering
  \includegraphics[width=\linewidth]{figures/benchmarks_and_models.pdf}
    \vspace{-15pt}
  \caption{Examples questions (top panels) and the necessary Bayesian Theory of Mind (BToM) model for Bayesian inverse planning (bottom panels) in diverse Theory of Mind benchmarks. \ours aims to answer any Theory of Mind question in a variety of benchmarks, encompassing different mental variables, observable contexts, numbers of agents, the presence or absence of utterances, wording styles, and modalities. It proposes and iteratively adjusts an appropriate BToM and conducts automated Bayesian inverse planning based on the model.
  There can be more types of questions/models in each benchmark beyond the examples shown in this figure.}
  \label{fig:benchmarks_and_models}
  %\vspace{-0.75em}
  \vspace{-10pt}
\end{figure*}



\textbf{Automated Modeling with LLMs.} There has been an increasing interest in integrating LLMs with inductive reasoning and probabilistic inference for automated modeling. \citet{piriyakulkij2024doing} combine LLMs with Sequential Monte Carlo to perform probabilistic inference about underlying rules. Iterative hypothesis refinement techniques \citep{qiu2023phenomenal} further enhance LLM-based inductive reasoning by iteratively proposing, selecting, and refining textual hypotheses of rules. Beyond rule-based hypotheses, \citet{wang2023hypothesis} prompt LLMs to generate natural language hypotheses that are then implemented as verifiable programs, while \citet{li2024automated} propose a method in which LLMs construct, critique, and refine statistical models represented as probabilistic programs for data modeling. \citet{cross2024hypothetical} leverage LLMs to propose and evaluate agent strategies for multi-agent planning but do not specifically infer individual mental variables. Our method also aims to achieve automated modeling with LLMs. Unlike prior works, we propose a novel automated model discovery approach for Bayesian inverse planning, where the objective is to confidently infer any mental variable given any context via constructing a suitable Bayesian Theory of Mind model.


% \vspace{-1em}
\section{Design Overview}
\label{sec:overview}
% \vspace{-0.2em}

In this section, we first present our new HBD architecture \sys{} guided by the design principles outlined above. We then provide an overview of its key components.


\begin{figure*}[ht]
    \centering
    \begin{subfigure}[b]{0.45\textwidth}
        \centering
        \includegraphics[height=80pt]{figs/design/transceiver.pdf}
        \caption{Components of OCS transceivers.}
        \label{figure:design:transceiver:component}
    \end{subfigure}
    \hspace{10pt}
    \begin{subfigure}[b]{0.45\textwidth}
        \centering
        \includegraphics[height=80pt]{figs/design/phase-shifter-v2.pdf}
        \caption{Zoom into OCS MZI switch matrix.}
        \label{figure:design:transceiver:ocs}
    \end{subfigure}
    \vspace{-10pt}
    \caption{Design of OCS Transceivers. The core component is OCS integrated in transceivers.}
    \label{figure:design:transceiver}
    \vspace{-15pt}
\end{figure*}

\para{Transceiver-centric HBD architecture}.
As discussed in \S\ref{sec:background:hbd} and summarized in \tabref{tab:hbd-compare}, existing architectures face a fundamental tradeoff among scalability, cost, and fault isolation. The GPU-centric architecture offers high scalability and low cost connectivity but suffers from a large fault explosion radius. In contrast, the switch-centric architecture improves fault isolation by leveraging centralized switches to confine failures to the node level. However, this comes at the cost of reduced scalability and higher connection overhead. The GPU-switch hybrid architecture takes a middle-ground approach but still suffers from significant fault propagation. As a result, no existing architecture fully meets all requirements.

Our key insight is that \textit{connectivity and dynamic switching can be unified at the transceiver level} using Optical Circuit Switching (OCS). By embedding OCS within each transceiver, we can enable reconfigurable point-to-multipoint connectivity, effectively combining both connectivity and switching at the optical layer. This represents a fundamental departure from conventional designs, where transceivers are limited to static point-to-point links and rely on high-radix switches for dynamic switching. We refer to this novel design as the \textit{transceiver-centric HBD architecture}. 

We realize this design with \sys{}, which has three key components as shown in \figref{fig:overview}.


\para{Design 1: Silicon Photonics based OCS transceiver (OCSTrx) (\secref{sec:design:docs}).} 
To enable large-scale deployment, we require a low-cost, low-power transceiver with Optical Circuit Switching (OCS) support. Unlike prior high-radix switches solutions that rely on MEMS-based switching~\cite{urata2022missionapollo, mem-optical-switches}, we leverage advances in Silicon Photonics (SiPh), which offer a simpler structure, lower cost, and reduced power consumption—making them well-suited for commercial transceivers.

Our SiPh-based OCS transceiver (OCSTrx), shown on the left of \figref{fig:overview}, provides two types of communication paths: i) \textit{Cross-lane loopback path (path 3)}, enabling direct GPU-to-GPU communication within the node, which can be used to construct dynamic size topologies; ii) \textit{Dual external paths (path 1\&2)}, connecting to external nodes. All these paths utilize time-division bandwidth allocation, featuring sub-1ms switching latency. With this capability, our \ocstrx \xspace allows dynamic reallocation of full GPU bandwidth to an active external path rather than splitting bandwidth across multiple paths. This eliminates redundant link waste—for instance, activating one external path completely disables the other, ensuring efficient bandwidth utilization.


\para{Design 2: Reconfigurable K-Hop Ring topology (\secref{section:design:topology}).}
With \ocstrx~ that provides reconfigurable connections at the transceiver, the next challenge is designing the topology. A naive starting point is the the full-mesh topology~\cite{fullmesh} which can provide full connectivity among all nodes using \ocstrx~. However, full-mesh design requires $O(N^2)$ links, inducing prohibitive complexity and cost. To reduce costs while maintaining near-ideal fault tolerance and performance, we prune the full-mesh topology into a K-Hop Ring topology based on traffic locality and fault non-locality (Details in~\S\ref{section:design:topology}). Combining the reconfigurability of \ocstrx{}, we propose a \textit{reconfigurable K-Hop Ring topology}, shown in the middle of \figref{fig:overview}, which consists of two key parts:

i) \textit{Intra-node topology:} dynamic GPU-granular ring construction is enabled by activating loopback paths. For example, while $N_1$-$N_3$ physically form a line topology, activating loopback paths creates a ring between $N_1$'s GPUs (1–4) and $N_3$'s GPUs (1–4). This mechanism allows for the construction of arbitrary-sized rings at any location, supporting optimal TP group sizes for different models while effectively minimizing resource fragmentation.

ii) \textit{Inter-node fault isolation: } dual external paths connect to primary and secondary neighbors (e.g., 2-Hop Ring). When a node fails (e.g., $N_2$), its neighbor ($N_1$) activates the backup path ($N_1$-$N_3$) to bypass the fault while maintaining full bandwidth, approaching node-level fault explosion radius. \S\ref{section:design:topology} generalizes this design to $K>2$.


\para{Design 3: HBD-DCN Orchestration Algorithm (\secref{sec:design:orch}).}
Designing an optimal HBD topology is crucial, but end-to-end training performance also depends on the efficient coordination between HBD and DCN. For instance, improper orchestration of TP groups can cause DP traffic to span across ToRs, resulting in DCN congestion. However, existing methods lack the ability to jointly optimize HBD and DCN coordination to alleviate congestion and enhance communication efficiency.
To address this, we propose the HBD-DCN Orchestrator, as shown on the right side of \figref{fig:overview}. The orchestrator takes three inputs: the user-defined job scale and parallelism strategy, the DCN topology and traffic pattern, and the real-time HBD fault pattern. It then generates the TP placement scheme, which maximizes GPU utilization and minimizes cross-ToR communication within the DCN.





\section{\Kbase Generation}\label{sec:kb}
The Knowledge Management System module (KMS), is in charge of taking the natural language description of both the environment and the actions that the agents can do, and convert them to a Prolog \kb using a LLM. 
The \kb contains all the necessary elements to define the mapped planning problem introduced in the previous section.

The framework works by considering a high-level and a low-level \kbase. For this reason, the input descriptions are also split into \HL and \LL. The former captures more abstract concepts, e.g., complex actions such as \verb|move_block| or the objects that are present in the environment. The latter captures more concrete and physical aspects of the problem, e.g., the actions that can be actually carried out by the agents such as \verb|move_arm| or the positions of the blocks. An example of this division can be seen in Section~\ref{sssec:runegKMS}.

% Describe how the kb works
The \kbase is divided in the following parts:
\begin{itemize}
    \item General \kb ($K$): contains the grounding predicates, both for the \HL and \LL. These predicates describe parts of the scenario or of the environment that do not change during execution. For example, the predicate \verb|wheeled(a1)|, which states that robot \verb | a1| has wheels, should be part of the general \kb and not of the state. 
    \item Initial ($I$) and final states ($G$): they contain all the fluents that change during the execution of the plan. This could be, for example, the position of blocks in the environment. 
    \item High-level actions ($DA_H$): each high-level action predicate is written as:
\begin{minted}[fontsize=\small,breaklines]{Prolog}
action(
    action_name(args),
    [positive_preconditions],
    [negative_preconditions],
    [grounding_predicates],
    [effects]
).
\end{minted}
    The low-level actions ($DA_L$) have the same structure, but instead of being described as predicates of type \verb|action|, they are described as \verb|ll_action|. The preconditions $\pc{a}$ of an action $a$ are obtained by combining the list of predicates \verb|positive_preconditions| and \verb|negative_preconditions|. The predicates in the list \verb|grounding_predicates| are used to ground the parametrised fluents of the action. For example, the action \verb|move_block| depends on a block, and we can check that the action is correctly picking a block and not another object by querying the \kb in this step. Section~\ref{sssec:runegKMS} clarifies this aspect.  
    \item Mappings ($M$): contains a dictionary of \HL actions $DA_H$ and how they should be mapped to a sequence of \LL actions $DA_L$. As will become apparent in the following, the distinction between \HL and \LL actions induces a significant simplification in the planning phase.
    \item Resources ($R$): the predicate \verb|resource\1| states whether another predicate is part of the resources or not. As mentioned before, this is helpful because it allows one to shrink the complexity of the problem not having to check multiple predicates, but instead they are later allocated during the optimisation part.
\end{itemize}

% Describe the process to validate the initial descriptions
Once the user provides descriptions for the \HL and \LL parts, the framework performs a consistency check to ensure that there are no conflicts between them. It verifies that both descriptions share the same goal, that objects remain consistent across \HL and \LL, and that agents are capable of executing the tasks. This validation is carried out by an LLM, which, if inconsistencies are detected, provides an explanation to help the user make the necessary corrections.

In both this step and the subsequent steps to generate the \kb, the LLM is not used directly out of the box. Instead, we employ the Chain-of-Thought (CoT)~\cite{wei2022chain} approach, which involves providing the LLM with examples to guide its reasoning. This process ensures that the output is not only structurally correct, but also more aligned with the overall goal of the task.

% Describe how the different aspects of the kb are extracted
Examples are particularly important when generating the \kb. Indeed, as we have mentioned before, the \kbase is highly structured and the planner expects to have the different components written correctly. CoT enables the LLM to know these details. 

We tested two different ways of generating the \kb through LLMs:
\begin{itemize}
    \item either we produced the whole \kb for the high-level and the low-level all at once, or
    \item we produced the single parts of the \kbs. 
\end{itemize}

The first approach is quite straightforward: once we have the examples to give to the LLM for the CoT process, we can input the \HL description and query the LLM to first extract the high-level \kb, and then also feed the created \HL \kb to the LLM to generate the \LL \kb, which will contain everything. % Highlight why one would want this. 

Instead, the second approach requires more requests to the LLM. We first focus on the \HL \kb, and then feed the \kb that we have obtained to generate the \LL parts. For the \HL generation, we ask the LLM to generate the general \kb, the initial and final states, and the actions set in this particular order. Each time we provide the LLM with the \HL description and with the elements generated in the previous steps. The same thing is done for the \LL \kb, generating again the four components and feeding each time also the \HL \kb. We include a final step that generates the mappings between the \HL and \LL actions. As for all the other steps, also in this final step, we pass the previously generated elements of the \LL \kb. 
Although generating the entire \kbase at once would reduce token usage and speed up the process, dividing the generation of the \kb into distinct steps enhances the system's accuracy, as demonstrated in the experimental evaluation of Section~\ref{sec:experiments}. This improvement comes because the iterative approach allows the LLM to first focus on generating more homogeneous information (i.e., the high-level) and then leverage the previously generated content to perform a consistency check.  

\subsection{Runing Example -- KMS}
\label{sssec:runegKMS}

We now introduce a running example, which will be used throughout this work to expose the interplay between the different components of the framework. 
This scenario is taken from the blocks-world domain~\cite{blocksworld}, which is frequently used in task planning. In particular, in this scenario we consider a table, blocks, which may either be directly on the table or stack on top of each other, and robotics arms, which move the blocks around. Each block is also associated with a position in the 2D space. 
In this particular example, we start from a situation in which we have two blocks, \verb|b1| and \verb|b2|, which are sat on the table in position (1,1) and (3,1) respectively. The goal is to move \verb|b1| in position (2,2) and then put \verb|b2| on top of it. An iconography of the example can be seen in Figure~\ref{fig:running-example}.

\begin{figure}
    \centering
    \input{figures/running-example}
    \caption{A scheme showing the running example. Two blocks must be moved from their initial position to a new position in which they are also stacked.}
    \label{fig:running-example}
\end{figure}

While this is a trivial example, it highlights very well the
capability of the knowledge management system to generate complex
predicates that can be used for planning and it also shows the
cooperative abilities of the framework. Indeed, while using a single
robotic arm generates a straight-forward plan solution, coordinating
two robotics arms to do the same task reduces the completion time at the price
of a higher planning complexity.


Let's now focus on the \kbase generation. For example, the two queries that could be passed to the system are:

\begin{textbox}{\HL query}
% \textboxtitle{\HL query}{
\footnotesize
In this scenario, there are two blocks on a table. They are placed in positions (1,1) and (3,1). There is a robotic agent that can move the blocks from one position to another. In particular, it can:
\begin{itemize}
    \item move a block from a position on the table, to another position on the table;
    \item move a block from a position on the table, to the top of another block;
    \item move a block from the top of a block, to a position on the table;
    \item move a block from the top of a block, to the top of another block.
\end{itemize}
The goal of this scenario is to move the first block from position (1,1) to position (2,2), and then place the second block on top of it.
% }
\end{textbox}

\begin{textbox}{\LL query}
% \textboxtitle{\LL query}{
\footnotesize
In this scenario, there are two blocks that are in the same positions as described in the \HL query. The robotic agent is a robotic arm and its end-effector is a two-finger gripper that can grasp an object. In particular, the following APIs are made available for the agent to move:
\begin{itemize}
    \item \texttt{move\_arm(Arm, X1, Y1, X2, Y2)}, which allows the end-effector to be moved from one position to another;
    \item \texttt{grip(Arm)}, which allows the gripper to be lowered onto a block and grip the block;
    \item \texttt{release(Arm)}, which allows to lower the block and release the gripper leaving the block in place.
\end{itemize}
% }
\end{textbox}


As mentioned before, the first step is to pass the two queries for validation. In this case, when using GPT-4o, both queries pass the consistency check. If instead of using the previous \HL query, we would modify it by removing the possibility of placing one block on top of the other, then the LLM would return the following error:

\begin{textboxerror}
\footnotesize
The description of the scenario has an issue. The agent is described as being able to move a block from a position on the table to another position on the table, and from the top of a block to a position on the table. However, the goal requires the agent to place the second block on top of the first block, which is not listed as an action that the agent can carry out. The agent's capabilities do not include placing a block on top of another block, which is necessary to achieve the goal.
\end{textboxerror}

\noindent This also allows the user to understand where the mistake is and correct it. 

Once the validation part succeeds, the system generates the \HL \kb. In this particular instance, for space limitation, we present only the general \kbase ($K$), the initial ($I$) and final ($G$) states, and a single action. 

\begin{center}
\begin{minipage}{\linewidth}
    \begin{minipage}{.48\linewidth}
        \begin{codebox}{prolog}{General KB}
% Positions
pos(1,1).
pos(2,2).
pos(3,1).

% Blocks
block(b1).
block(b2).

% Agents
agent(a1).

% Resources
resources(agent(_)).
        \end{codebox}
    \end{minipage}
    \hfill
    \begin{minipage}{.48\linewidth}
        \begin{minipage}{\linewidth}
        \begin{codebox}{prolog}{Initial state ($I$)}
init_state([
  ontable(b1), ontable(b2),
  at(b1,1,1), at(b2,3,1),
  clear(b1), clear(b2),
  available(a1)
]).
        \end{codebox}
        \end{minipage}
        \hspace{1cm}\\
        \begin{minipage}{\linewidth}
        \begin{codebox}{prolog}{Final state ($G$)}
goal_state([
  ontable(b1),
  on(b2, b1),
  at(b1,2,2), at(b2,2,2),
  clear(b2),
  available(a1)
]).
        \end{codebox}
        \end{minipage}
    \end{minipage}
\end{minipage}
\begin{codebox}{prolog}{Action example}
action(move_table_to_table_start(Agent, Block, X1, Y1, X2, Y2), 
  [ontable(Block), at(Block, X1, Y1), available(Agent), clear(Block)],
  [
    at(_, X2, Y2), on(Block, _), moving_table_to_table(_, Block, _, _, _, _), 
    moving_table_to_block(_, Block, _, _, _, _, _)
  ],
  [agent(Agent), pos(X1, Y1), pos(X2, Y2), block(Block)],
  [
    del(available(Agent)), del(clear(Block)), del(ontable(Block)), del(at(Block, X1, Y1)),
    add(moving_table_to_table(Agent, Block, X1, Y1, X2, Y2))
  ]
).
\end{codebox}
\end{center}

The resulting \HL \kb is human-readable and relatively simple (in fulfilment of requirement \textbf{R2}).
The user at this point can make corrections to the \HL \kb, if needed, and finally, \frameworkname will also generate the \LL \kbase. In this case for space limitation, we show the changes made to the previous elements, one low-level action, and one mapping. 

\begin{center}
\begin{minipage}{\linewidth}
    \begin{minipage}{.48\linewidth}
        \begin{codebox}{prolog}{General KB}
% Positions
pos(0,0).
pos(1,1).
pos(2,2).
pos(3,1).

% Blocks
block(b1).
block(b2).

% Agents
agent(a1).

% Low-level predicates
ll_arm(a1).
ll_gripper(a1).

% Resources
resources(agent(_)).
        \end{codebox}
    \end{minipage}
    \hfill
    \begin{minipage}{.48\linewidth}
        \begin{minipage}{\linewidth}
        \begin{codebox}{prolog}{Initial state ($I$)}
init_state([
  ontable(b1), ontable(b2),
  at(b1,1,1), at(b2,3,1),
  clear(b1), clear(b2),
  available(a1),
  ll_arm_at(a1,0,0), 
  ll_gripper(a1,open) 
]).
        \end{codebox}
        \end{minipage}
        \hspace{1cm}\\
        \begin{minipage}{\linewidth}
        \begin{codebox}{prolog}{Final state ($G$)}
goal_state([
  ontable(b1),
  on(b2, b1),
  at(b1,2,2), at(b2,2,2),
  clear(b2),
  available(a1),
  ll_arm_at(a1,_,_), 
  ll_gripper(a1,_)    
]).
        \end{codebox}
        \end{minipage}
    \end{minipage}
\end{minipage}
\begin{codebox}{prolog}{Action example}
ll_action(move_arm_start(Arm, X, Y),
  [],
  [ll_arm_at(_, X, Y), moving_arm(Arm, _, _, _, _), gripping(Arm, _), releasing(Arm)],
  [],
  [ll_arm(Arm), pos(X, Y)],
  [
    add(moving_arm(Arm, X, Y)),
    del(ll_arm_at(Arm, X, Y))
  ]
).
\end{codebox}
\begin{codebox}{prolog}{Mapping example}
mapping(move_table_to_table_start(Agent, Block, X1, Y1, X2, Y2),
  [
    move_arm_start(Agent, X1, Y1),
    move_arm_end(Agent, X1, Y1),
    grip_start(Agent),
    grip_end(Agent),
    move_arm_start(Agent, X2, Y2),
    move_arm_end(Agent, X2, Y2),
    release_start(Agent),
    release_end(Agent)
  ]
).
\end{codebox}
\end{center}

Again, the user can correct possible errors (or anyway refine the \kb) and then move on to the planning phase.


\section{Plan Generation}\label{sec:plangen}
In this section, we describe how the framework uses the information from the \kb to generate a task plan for multiple agents.
Generation takes place in three steps: 
\begin{enumerate*}
    \item Generation of a total-order (TO) plan, 
    \item extraction of a partial-order (PO) plan and of the resources, 
    \item solution of a MILP problem to improve resource allocation and reducing the plan makespan by exploiting the possible parallel executions of actions.
\end{enumerate*}



\subsection{Total-Order Plan Generation}\label{ssec:toplangen}
A total-order plan is a strictly sequential list of actions that drives the system from the initial to the goal state. 
The algorithm used to extract a total-order plan is shown in~\autoref{alg:toplanning} and consists of two distinct steps:
\begin{itemize}
    \item identify a total-order plan for high-level actions, and
    \item recursively map each high-level action to a sequence of actions with a lower level until they are mapped to actions corresponding to the APIs of the available robotic resources.
\end{itemize}

\begin{algorithm}
\footnotesize
\caption{Algorithm generating a TO plan with mappings}\label{alg:toplanning}
\KwData{$TP=(F, DA, I, G, K)$}
\KwResult{Plan solving TP}

\DontPrintSemicolon

\SetKwProg{plan}{TO\_PLAN}{}{}
\SetKwProg{map}{APPLY\_MAP}{}{}
\SetKwProg{action}{APPLY\_ACTION}{}{}
\SetKwProg{maps}{APPLY\_MAPPINGS}{}{}

\SetKwInOut{Input} {In}
\SetKwInOut{Output}{Out}

\plan{(S, P)}{
  \Input{The current state $S$ and the current plan $P$}
  \Output{The final plan}
  \If{$S \neq G$}{
      select\_action($a_i$)\;
      (US, UP) $\gets$ APPLY\_ACTION($a_i$, S, P)\;
      P $\gets$ TO\_PLAN(US, UP)\;
  }
  (US, UP) $\gets$ APPLY\_MAPPINGS(S,P)\;
  \KwRet{P}\;
}

\maps{(S, P)}{
  \Input{The current state $S$ and the current plan $P$}
  \Output{The updated state $US$ and plan $UP$ after the mappings}
  US, UP $\gets$ S, P\;
  \ForEach{$a_i \in P$}{
    \If{\textnormal{is\_start($a_i$) $\wedge$ has\_mapping($a_i$)}}{
      (US, UP) $\gets$ APPLY\_MAP($a_i$, \textnormal{US}, UP)\;
    }
  }
  \KwRet{(US, UP}\;
}

\map{($a$, S, P)}{
  \Input{The action $a$, the current state $S$ and the current plan $P$}
  \Output{The updated state $US$ and plan $UP$ after the mappings}
  M $\gets$ mapping($a$)\;
  \ForEach{$a_i \in M$}{
    (US, UP) $\gets$ APPLY\_ACTION($a$, S, P)\;
  }
  \KwRet{(US, UP)}\;
}

\action{($a, S, P$)}{
  \Input{The action $a$, the current state $S$ and the current plan $P$}
  \Output{The updated state $US$ and plan $UP$ after applying the effects of $a$}
  \eIf{\textnormal{is\_applicable($a_i$)}}{
    US $\gets$ change\_state($a_i$.eff, S)\;
    UP $\gets$ plan\_action($a_i$, P)\;
    \KwRet{(US, UP)}\;
  }{
    \KwRet{(S, P)}
  }
}
\end{algorithm}

This enables the extraction of total-order plans that are consistent with the \kb provided, and we reduce the computational cost of checking all the possible actions at each time step. The \texttt{TO\_PLAN} function is the main function, which takes the initial and final states, and it inspects which actions can be executed given the current state. The \texttt{select\_action} function selects the next action from the set of possible actions. This search is based on the Prolog inference engine, which tries the actions in the order in which they are written in the KB, and hence it is not an informed search. 

The algorithm then moves to the \texttt{APPLY\_ACTION} function, which first checks if the chosen action's preconditions are met in the current state and, if they are, then it applies its effects changing the state (\texttt{change\_state}) and adding the action to the plan (\texttt{plan\_action}). It continues until the current state satisfies the goal state. Whenever the search reaches a fail point, we exploit the Prolog algorithm of resolution to step back and explore alternative possibilities.

Once the algorithm has extracted a high-level total-order plan, it applies the mappings. To do so, it iterates over the actions in the plan, and for each action it checks if it is a start action ($a_\vdash$) and if there are mappings for it. If this is the case, it calls the function \texttt{APPLY\_MAP}, which sequentially applies the actions in the mapping to the current state, also adding the actions to the plan. Notice that to do so, we call the \texttt{APPLY\_ACTION} function, which checks the preconditions of the actions w.r.t. the current state, ensuring that the lower-level actions can actually be applied.
% Also, the functions recursively check if any action from the mapped action has a mapping on its own, ensuring that all the actions have a direct grounding to APIs.

The total-order plan $TO$ extracted from this function is a list of actions that are executed in sequence:
\begin{equation*}
    \forall i \in \{0,\hdots \vert TO\vert-1\}~t(a_i)<t(a_{i+1})
\end{equation*}

\subsubsection{Running Example -- Total-Order Plan}
\label{sssec::runegTOPlan}

Let us consider again the \kb that we generated in Section~\ref{sssec:runegKMS}. Let us now see how \frameworkname extracts the TO plan.

The algorithm starts from the initial state and from the first action in the \kb, which in this case is the one shown in Section~\ref{sssec:runegKMS}. The algorithm takes the grounding predicates in this case:

\begin{minted}[fontsize=\footnotesize]{prolog}
agent(Agent), pos(X1, Y1), pos(X2, Y2), block(Block)
\end{minted}

and checks whether there is an assignment of predicates from the \kbase that satisfies them. For example, the predicate \verb|pos(1,1)| satisfies \verb|pos(X1,Y1)|. Not only this, but since the predicates in this list are grounded w.r.t. the \kb, one can also check some conditions. For example, if we were to assign the values to the previous predicates, it can happen that \verb|X1 = X2| and \verb|Y1 = Y2|, which is useless for an action that moves a block from one position to another. By adding the following predicates, we can ensure that the values are different:

\begin{minted}[fontsize=\footnotesize]{prolog}
agent(Agent), pos(X1, Y1), pos(X2, Y2), block(Block), X1\=X2, Y1\=Y2
\end{minted}

Once an assignment for the predicates inside the grounding list is found, the algorithm checks whether the predicates inside the preconditions are satisfied. Let us consider the preconditions for the \verb|move_table_to_table_start| action from Section~\ref{sssec:runegKMS}:

\begin{minted}[fontsize=\footnotesize]{prolog}
% Positive predicates
[ontable(Block), at(Block, X1, Y1), available(Agent), clear(Block)],
% Negative predicates
[
  at(_, X2, Y2), on(Block, _), moving_table_to_table(_, Block, _, _, _,_), 
  moving_table_to_block(_, Block, _, _, _, _, _)
]
\end{minted}

After the first grounding step, they become the following:

\begin{minted}[fontsize=\footnotesize]{prolog}
% Positive predicates
[ontable(b1), at(b1, 0, 0), available(a1), clear(b1)],
% Negative predicates
[
  at(_, 0, 0), on(b1, _), moving_table_to_table(_, b1, _, _, _,_), 
  moving_table_to_block(_, b1, _, _, _, _, _)
]
\end{minted}

The algorithm checks whether the predicates from the first list are satisfied in the current state and whether the predicates from the second list are not present in the current state. Comparing them with the initial state as shown in Section~\ref{sssec:runegKMS}, we can see that \verb|ontable(b1)| is present, but \verb|at(b1, 0, 0)|, so this combination of predicates would already be discarded. The first grounding that is accepted is that in which \verb | Block = b1, X1 = 1, Y1 = 1, Agent = a1 |. Notice that the predicates that start with \verb|_| mean "any", e.g., the predicate \verb|at(_, 0, 0)| checks if there is any predicate with name \verb|at| and arity 3 that has the last two arguments set to 0, regardless of what the first argument is.

By checking the different combinations of actions, the planner can extract a \HL TO plan. In this case, it would be something like this:

\begin{minted}[fontsize=\footnotesize]{text}
[0] move_table_to_table_start(a1, b1, 1, 1, 2, 2)
[1] move_table_to_table_end(a1, b1, 1, 1, 2, 2)
[2] move_table_to_block_start(a1, b2, 3, 1, 2, 2)
[3] move_table_to_block_end(a1, b2, 3, 1, 2, 2)
\end{minted}

At this point, the algorithm takes the mappings and it applies them to the previous plan. For instance, from Section~\ref{sssec:runegKMS} we saw that the mapping for \verb|move_table_to_table_start| is:
\begin{minted}[fontsize=\footnotesize]{prolog}
mapping(move_table_to_table_start(Agent, Block, X1, Y1, X2, Y2),
  [
    move_arm_start(Agent, X1, Y1), move_arm_end(Agent, X1, Y1),
    grip_start(Agent), grip_end(Agent),
    move_arm_start(Agent, X2, Y2), move_arm_end(Agent, X2, Y2),
    release_start(Agent), release_end(Agent)
  ]
).
\end{minted}

Hence, we would change the previous plan with:

\begin{minted}[fontsize=\footnotesize]{text}
[0] move_table_to_table_start(a1, b1, 1, 1, 2, 2)
[1] move_arm_start(a1, 1, 1)
[2] move_arm_end(a1, 1, 1)
[3] grip_start(a1)
[4] grip_end(a1)
[5] move_arm_start(a1, 2, 2)
[6] move_arm_end(a1, 2, 2)
[7] release_start(a1)
[8] release_end(a1)
[9] move_table_to_table_end(a1, b1, 1, 1, 2, 2)
[10] move_table_to_block_start(a1, b2, 3, 1, 2, 2)
[11] move_arm_start(a3, 3, 1)
[12] move_arm_end(a1, 3, 1)
[13] grip_start(a1)
[14] grip_end(a1)
[15] move_arm_start(a1, 2, 2)
[16] move_arm_end(a1, 2, 2)
[17] release_start(a1)
[18] release_end(a1)
[19] move_table_to_block_end(a1, b2, 3, 1, 2, 2)
\end{minted}


\subsection{Partial-Order Plan Generation}\label{ssec:poplangen}
The next step is to analyse the total-order plan in search of all possible causal relationships. This is done by
looking for actions that enable other actions (enablers). In addition, we extract all the resources that can be allocated
and used for the execution of the task. This step will be important for the next phase of the planning process, the MILP problem, in which 
the resources will be re-allocated allowing for shrinking the makespan of the plan.
%
In this work, the only resource considered is the robotic agent, but this limitation could easily be removed by modifying the \kb.  To this end,  we define a special predicate, named \texttt{resource/1}, that allows us to specify the resources.

Given an action $a_i$, another action $a_j$ is an enabler of $a_i$ if it either adds a literal $l$ satisfying one or more preconditions of $a_i$, or it removes a fluent violating one or more preconditions of $a_i$, and if $a_i$ happens after $a_j$: 

\begin{equation}
\small
\begin{array}{rl}
     a_j \in \ach{a_i} \iff & t(a_i) > t(a_j) \wedge \\
                            & ((l\in \pc{a_i}~ \wedge add(l)\in \eff{a_j}) \vee\\
                            & \,\,(\lnot l\in \pc{a_i} \wedge del(l)\in \eff{a_j}))
\end{array}
\label{eq:enablers}
\end{equation}

It is important to note that we consider an action $a_j\notin\ach{a_i}$ if there is at least a fluent $l$ that is not a resource. If all the fluents and their arguments that would make $a_j$ an enabler of $a_i$ are resources, then $a_j$ is not considered an enabler, as this relationship depends on the assignment of the resources, which comes with the optimisation step. 

Besides the enablers added corresponding to the classical definition, we also enforce the following precedence constraints:
\begin{itemize}
    \item When we expand a mapping $m(\alpha_i)$ of a high-level durative action $\alpha_i$ and reach the ending action $\aEnd{\alpha_i}$, then we add all previous durative actions as enablers until the corresponding start action. For example, assume that $m(\alpha_i)=\{\alpha_j, \alpha_k\}$, this means that the total-order plan will be the sequence $\{\aStart{\alpha_i}, \aStart{\alpha_j}, \aEnd{\alpha_j}, \aStart{\alpha_k}, \aEnd{\alpha_k}, \aEnd{\alpha_i}\}$. It follows that $\aStart{\alpha_i}$ is an enabler of $\aEnd{\alpha_i}$, but also all intermediate actions are part of the set of its enablers as they must be completed in order for $\alpha_i$ to end.
    \begin{equation}
        \bigwedge_{a\in m(\alpha_i)} a\in \ach{\aEnd{\alpha_i}}.
        \label{eq:constraint5}
    \end{equation}
    \item When we expand a mapping, all actions in the mapping must have the start of the higher-level action as one of the enablers. For instance, after the previous example, $\aStart{\alpha_j}, \aEnd{\alpha_j}, \aStart{\alpha_k}, \aEnd{\alpha_k}$ have $\aStart{\alpha_i}$ as an enabler.
    \begin{equation}
        \bigwedge_{a\in m(\alpha_i)} \aStart{\alpha} \in \ach{a_i}.
        \label{eq:constraint4}
    \end{equation}
\end{itemize}

% We then create a graph from which we can extract partial-order plans. To do this, after having obtained a plan from the \texttt{TO\_PLAN} from~\autoref{alg:planning}, we look for the achievers of the actions as shown in~\autoref{alg:po_planning}. 

The algorithm that manages this extraction is shown in~\autoref{alg:poplanning}. For ease of reading, we define $R\subseteq F$ as the set of fluents that are resources.

The algorithm \texttt{FIND\_ENABLERS} takes the total-order plan and, starting with the first action in the plan, it extracts all the causal relationships between the actions. The auxiliary function \texttt{IS\_ENABLER} tests whether an action $a_j$ is an enabler of an action $a_i$ by checking the properties of~\autoref{eq:enablers} plus the precedence constraints just described. Finally, notice that the literal checked to be present (absent) in both additive (subtractive) effects must not contain arguments that are part of the resources $R$. For example, consider the case in which an action $a_i$ needs the precondition $l(x_1, x_2, x_3)$ and $a_j$ provides the predicate, then if at least one of $x_1, x_2, x_3$ is in $R$, $a_j$ is an enabler of $a_i$, otherwise it is not. This ensures that only causal relationships that do not depend on the resources are extracted at this time. The precedence of the resources will be defined and discussed in Section~\ref{ssec:poplanopt}. 

\begin{algorithm}[htp]
\footnotesize
\caption{Algorithm extracting the actions enablers and the resources}
\label{alg:poplanning}
\KwData{$TP=(F, DA, I, G, K)$}
\KwResult{Enablers and resources $R$}

\DontPrintSemicolon

\SetKwProg{findenablers}{FIND\_ENABLERS}{}{}
\SetKwProg{isenabler}{IS\_ENABLER}{}{}
\SetKwProg{findresources}{EXTRACT\_RESOURCES}{}{}

\SetKwInOut{Input} {In}
\SetKwInOut{Output}{Out}

\findenablers{$(\tn{TO\_P}, a_i)$}{
  \Input{The total-order plan TO\_P, the $i$th action}
  \Output{The enablers $E$ for all the actions in the plan}

  \For{$a_j \in \tn{TO\_P}, a_j\neq a_i$}{
    \uIf{$\tn{IS\_ENABLER}(a_j, a_i)$}{
      $E[a_i].add(a_j)$;
    }
  }

  \If{$a_i\neq \tn{TO\_P}.back()$}{
    $E \gets \tn{FIND\_ENABLERS}(\tn{TO\_P}, a_{i+1})$\;
  }
  \KwRet{E}\;
}

\isenabler{$(a_j, a_i)$}{
  \Input{The action $a_j$ to test if it's enabler of $a_i$}
  \Output{True if $a_j$ is enabler of $a_i$}

  \ForEach{$e \in \eff{a_j}$}{
    \uIf{$\left(e=\tn{add}(l) \wedge l\in\pc{a_i})\right)$ OR
         $~\left(e=\tn{del}(l) \wedge \lnot l\in\pc{a_i}\right)$ OR
         $~\left(\tn{isStart}(a_j) \wedge a_i \in m(a_j)\right)$ OR\\
         $~~\left(\tn{isEnd}(a_j) \wedge a_i \in m(a_j)\right)$}
    {
      $X\gets \tn{set of arguments of }e$; 
        
      \uIf{$\not\exists x \in X | x \in R$}{
        \KwRet{True};
      }
    }
  }
  \KwRet{False};
}

\findresources{$()$}{
  \output{A list of resources}
  findall(X, resources(X), AllResources)\;
  $R$ = make\_set(AllResources)\;
  \KwRet{$R$}\;
}

\end{algorithm}

\subsubsection{Running Example -- Partial-Order Plan}
\label{sssec:PORunEx}

Once we have applied the mappings as before, we have the full TO plan. We want to extract information from this, which will then be exploited to improve the plan for multiple agents. This is done by examining all the actions and checking which are their enablers. For instance, the 10th action, \verb|move_table_to_block_start(a1, b2, 3, 1, 2, 2)|, has as a precondition the following predicate \verb|clear(Block2), Block2=b1|, which is true only when the 9th action has applied its effects. Since \verb|b1| is not part of the resources, the algorithm will state that $a_9$ is an enabler of $a_{10}$. 

If the second move were to move a block to another position on the table, hence independent of the first move, then the algorithm would not set $a_9$ as an enabler of $a_{10}$, as the only reason it may do so is if the same agent is used, but this is known only later.

After this step, we know the enablers for the actions (shown in squared brackets in the list below):

\begin{minted}[fontsize=\footnotesize]{text}
[0] init()[]
[1] move_table_to_table_start(a1, b1, 1, 1, 2, 2), [0]
[2] move_arm_start(a1, 1, 1), [0,1]
[3] move_arm_end(a1, 1, 1), [0,1,2]
[4] grip_start(a1), [0,1,2,3]
[5] grip_end(a1), [0,1,2,3,4]
[6] move_arm_start(a1, 2, 2), [0,1,2,3,4,5]
[7] move_arm_end(a1, 2, 2), [0,1,2,3,4,5,6]
[8] release_start(a1), [0,1,2,3,4,5,6,7]
[9] release_end(a1), [0,1,2,3,4,5,6,7,8]
[10] move_table_to_table_end(a1, b1, 1, 1, 2, 2), [0,1,2,3,4,5,6,7,8,9]
[11] move_table_to_block_start(a1, b2, 3, 1, 2, 2), [0,10]
[12] move_arm_start(a1, 3, 1), [0,11]
[13] move_arm_end(a1, 3, 1), [0,11,12]
[14] grip_start(a1), [0,11,12,13]
[15] grip_end(a1), [0,11,12,13,14]
[16] move_arm_start(a1, 2, 2), [0,11,12,13,14,15]
[17] move_arm_end(a1, 2, 2), [0,11,12,13,14,15,16]
[18] release_start(a1), [0,11,12,13,14,15,16,17]
[19] release_end(a1), [0,11,12,13,14,15,16,17,18]
[20] move_table_to_block_end(a1, b2, 3, 1, 2, 2), [0,10,11,12,13,14,15,16,17,18,19]
[21] end(), [0,1,2,3,4,5,6,7,8,9,10,11,12,13,14,15,16,17,18,19,20]
\end{minted}

From this we could already notice that all the actions will be carried out in sequence. We also see that in this step we add two fictitious actions, \verb|init| and \verb|end|. This simply represents the start and the end of the plan, respectively. \verb|init| is an enabler of all the actions in the plan and \verb|end| has all the other actions as enablers, which means that the plan can be considered finished only when all the actions have been executed.

As for the resources, we first extract all the possible resources by looking at the predicates \verb|resource(X)| in the \kb, as shown in Section~\ref{sssec:runegKMS}. Then we assign the type of resources used to each action by checking action per action which resources they are using. This is useful because it will provide MILP with the basis to correctly allocate the different resources to the actions.

\begin{minted}[fontsize=\footnotesize]{text}
Resources:
[0] agent-2
Resources list:
[0] agent-[agent(a1),agent(a2)]
Resources required by action:
[4] 6-[agent]
[9] 1-[agent]
\end{minted}


\subsection{Partial-Order Plan Optimization}\label{ssec:poplanopt}
The last part of the planning module, shown in~\autoref{fig:arch_LLM_pKB}, is the optimisation module which allows for shrinking the plan by scheduling the task (temporal plan) and allocating the resources. In order to do this, we instantiate a MILP problem, the solution of which must satisfy constraints ensuring that we are not violating precedence relationships and invalidating the obtained planned. 

We start by taking the work from~\cite{cimatti_strong_2015}, in which the authors describe how it is possible to obtain a plan with lower makespan by reordering some tasks. In particular, we adopt the following concepts from~\cite{cimatti_strong_2015}:
\begin{itemize}
    \item Let $f(l)=\{a\in DA \vert l\in \eff{a}\}$ be the set of actions that achieve a literal $l$, and 
    \item let $\displaystyle p(l,a,r)\doteq a<r \wedge \bigwedge_{a_i\in f(l)\setminus\{a,r\}}(a_i<a\vee a_i>r)$ be the temporal constraint stating which is the last achiever $a$ of an action $r$ for a literal $l$. 
\end{itemize}
The constraints that must hold are the following:
\begin{equation}
    \label{eq:constraint1_old}
    %\footnotesize
    \bigvee_{a_j\in f(l)\setminus\{a\}} p(l,a_j,a).
\end{equation}
Which states that at least an action with effect $l$ should occur before $a$.
\begin{equation}
    \label{eq:constraint2}
    %\footnotesize
    \bigwedge_{a_j\in f(l)} \left(p(l,a_j,a) \rightarrow \bigwedge_{a_t\in f(\lnot l)\setminus\{a\}}(a_t<a_j \vee a_t>a)\right).
\end{equation}
\begin{equation}
    \label{eq:constraint3}
    %\footnotesize
    \bigwedge_{a_j\in f(\lnot l)\wedge l\in \pc{a}} ((a_j<\aStart{a}) \vee (a_j>\aEnd{a})).
\end{equation}
Which state that between the last achiever $a_j$ of a literal $l$ for an action $a$ and the action $a$ there must not be an action $a_t$ negating said literal. This condition is also enforced by~\autoref{eq:constraint3} that constrains actions negating the literal to happen before the action $a$ has started or after it has finished.

Notice though that in this work, the authors have considered achievers and not enablers. The difference is that an action $a_j$ is an achiever of $a_i$ if $a_j$ \emph{adds} a fluent $l$ that is needed by $a_j$. Enablers instead consider the case in which fluents are also removed. 
%
Since these constraints only consider achievers and not enablers, we need to extend them. We redefine the previous as:
\begin{itemize}
    \item let $f(l)=\{a\in DA \vert add(l)\in \eff{a}\}$ be the set of actions that achieve a literal $l$, and 
    \item let $f(\lnot l)=\{a\in DA \vert del(l) \in \eff{a}\}$ be the set of actions that delete a literal $l$, and
    \item let $F(l) = f(l)\cup f(\lnot l)$ be the union set of $f(l)$ and $f(\lnot l)$, and
    \item let $\displaystyle p(l,a,r)\doteq a<r \wedge \bigwedge_{a_i\in F(l) \setminus\{a,r\}}(a_i<a\vee a_i>r)$ be the last enabler $a$ of an action $r$ for a literal $l$. 
\end{itemize}
Consequently, we need to:
\begin{itemize}
    \item revise~\autoref{eq:constraint1_old} to include all enablers:
        \begin{equation}
            \label{eq:constraint1}
            %\footnotesize
            \bigvee_{a_j\in F(l)\setminus\{a\}} p(l,a_j,a).
        \end{equation}
    \item add two constraints similar to~\autoref{eq:constraint2} and~\autoref{eq:constraint3} to ensure that a predicate that was removed is not added again before the execution of the action:
    \begin{equation}
        \label{eq:constraint2_1}
        %\footnotesize
        \bigwedge_{a_j\in f(\lnot l)} \left(p(l,a_j,a) \rightarrow \bigwedge_{a_t\in f(l)\setminus\{a\}}(a_t<a_j \vee a_t>a)\right).
    \end{equation}
    \begin{equation}
        \label{eq:constraint3_1}
        %\footnotesize
        \bigwedge_{a_j\in f(l)\wedge (\lnot l)\in \pc{a}} ((a_j<\aStart{a}) \vee (a_j>\aEnd{a})).
    \end{equation}    
\end{itemize}

The second aspect of the MILP problem concerns resource allocation. Indeed, as stated before, there are some predicates that are parameterised on resources, e.g., \texttt{available(A)} states whether an agent \texttt{A} is available or not, but it does not ground the value of \texttt{A}. %
One possibility would be to allocate the resources using Prolog, as done in~\cite{saccon2023prolog}, but this choice is greedy since Prolog grounds information with the first predicate that satisfy \texttt{A}. To reduce the makespan of the plan and improve the quality of the same, we delay the grounding to an optimisation phase, leaving Prolog to capture the relationships between actions.

As a first step, we are also going to assume that all the actions coming from a mapping of a higher-level action and that are not mapped into lower-level actions shall maintain the same parameterised predicates as the higher-level action. So the constraint in~\autoref{eq:constraint6} must hold.
\begin{equation}
    \label{eq:constraint6}
    \bigwedge_{a_j\in m(a_i) \wedge m(a_j)\notin M} \left(\bigwedge_{p(x_i) \in \pc{a_i} \wedge p(x_j) \in \pc{a_j}} x_i=x_j \right).
\end{equation}
Moreover, for these constraints, we will consider only predicates that are part of the set $K$, that is predicates that are not resources $R\cap K=\emptyset$.

The objective now is three-fold: 
\begin{itemize}
    \item identify a cost function,
    \item summarise the previous constraints, and
    \item construct a MILP problem to be solved.
\end{itemize}

In this work, the first point is straightforward: we want to minimise the makespan, i.e., the total duration required to complete all tasks or activities.

For the second point, we are trying to find a way to put the previous constraints,~\cref{eq:constraint1,eq:constraint2,eq:constraint2_1,eq:constraint3,eq:constraint3_1,eq:constraint4,eq:constraint5,eq:constraint6} in a compact formulation or structure. We opted to extract the information regarding the enablers using Prolog and to place it into a $N\times N$ matrix $C$, where $N$ is the number of actions and each cell $C_{ij}$ is $1$ if $a_i$ is an enabler of $a_j$ (without considering resources), 0 otherwise. 

We now need to address the resource allocation aspect, specifically, how to distribute the available resources $R$ among the various actions. When performing this task, there are primarily two factors to consider:
\begin{itemize}
    \item A resource cannot be utilised for multiple actions simultaneously.
    \item If two actions share the same resource, they must occur sequentially, meaning one action enables the other.
\end{itemize}

For the first factor, we need to make sure that, for each resource type $r\in R$, the number of actions using the resource at the same time must not be higher than the number of resources of that type available, as shown in~\autoref{eq:resAllocation}.
\begin{equation}
    \displaystyle\forall t \in\{t_0, t_{\tn{END}}\},\,\vert r\vert \geq\sum_{a_i\in TO} t\in\{\aStart{a_i}, \aEnd{a_i}\} \wedge \left( \exists~l(\pmb{x})\in \pc{a_i}\vert r\in\pmb{x}\right).
    \label{eq:resAllocation}
\end{equation}

The second factor must instead be merged with also the precedence constraints embedded in $C$. In particular, we want to express that actions $a_i, a_j$ are in a casual relationship if $C_{ij}=1$ or if they share the same resource. This can be expressed with the following constraint: 
\begin{equation}
    C_{ij} \vee \exists r\in R : r\in\fl{a_i} \wedge r\in\fl{a_j}
    \label{eq:precedence}
\end{equation}
Note that $\fl{a}$ was defined in the problem definition paragraph and represents the set of variables and literals used by the predicates in the preconditions of $a$. 

Finally, we need to set up the MILP problem that consists in finding an assignment of the parameters, of the actions' duration and of the causal relationships, such that the depth of the graph $\mathcal{G}$ representing the plan is minimised. This problem can be expressed as shown in~\autoref{eq:optimization_1}.

%\begin{figure*}[h]
%    \centering
    \begin{equation}
    \everymath={\displaystyle}
    \begin{array}{r@{\hspace*{8mm}}l}
        \label{eq:optimization_1}
        \min_{\mathcal{P}, \mathcal{T}} & t_{\tn{END}} \\
        %&\\
        \textrm{s.t.}   & C_{ij} \vee \exists r\in R : r\in\fl{a_i} \wedge r\in\fl{a_j}, \\
                              & \quad \quad \forall t \in\{t_0, t_{\tn{END}}\}, \\
                              & \quad \quad \quad \quad \vert r\vert > \!\!\sum_{a_i\in TO} \left(t\in\{\aStart{a_i}, \aEnd{a_i}\} \wedge \exists~l(\pmb{x})\in \pc{a_i}\vert r\in\pmb{x}\right).\\
    \end{array}
    \end{equation}
%\end{figure*}

As mentioned before, the MILP part is implemented in Python3 using OR-Tools from Google. The program also checks the consistency of the PO matrix $C$, by making sure that all the actions must have a path to the final actions. 
The output of the MILP solution is basically an STN, which describes both the causal relationship between the actions and also the intervals around the duration of the actions. The initial and final nodes of the STN are factitious as they do not correspond to actual actions, but they simply represent the start and the end of the plan.
The STN is extracted by considering the causal relationship from the $C$ matrix taken as input, and by adding the causal relationship given by the resource allocation task. 
Once we have the STN, we can extract a \bt, which can then be directly executed by integrating it in ROS2. 

\subsubsection{Plan Optimization -- Example}
\label{sssec:PORunExample}
As we said at the end of~\autoref{sssec:PORunEx}  on the running example, that particular plan is not optimisable as the actions are executed in sequence. Let's then consider a slight modification, which consists in finding a plan to move the two blocks in two new positions instead of stacking them in one position. We also have a new agent that can be used to carry out part of the work. 
Our new plan and actions' enablers are the following one:

\begin{minted}[fontsize=\footnotesize]{text}
[0] init()[]
[1] move_table_to_table_start(a1, b1, 1, 1, 1, 2), [0]
[2] move_arm_start(a1, 1, 1), [0,1]
[3] move_arm_end(a1, 1, 1), [0,1,2]
[4] grip_start(a1), [0,1,2,3]
[5] grip_end(a1), [0,1,2,3,4]
[6] move_arm_start(a1, 1, 2), [0,1,2,3,4,5]
[7] move_arm_end(a1, 1, 2), [0,1,2,3,4,5,6]
[8] release_start(a1), [0,1,2,3,4,5,6,7]
[9] release_end(a1), [0,1,2,3,4,5,6,7,8]
[10] move_table_to_table_end(a1, b1, 1, 1, 1, 2), [0,1,2,3,4,5,6,7,8,9]
[11] move_table_to_table_start(a1, b2, 3, 1, 3, 2), [0,10]
[12] move_arm_start(a1, 3, 1), [0,11]
[13] move_arm_end(a1, 3, 1), [0,11,12]
[14] grip_start(a1), [0,11,12,13]
[15] grip_end(a1), [0,11,12,13,14]
[16] move_arm_start(a1, 3, 2), [0,11,12,13,14,15]
[17] move_arm_end(a1, 3, 2), [0,11,12,13,14,15,16]
[18] release_start(a1), [0,11,12,13,14,15,16,17]
[19] release_end(a1), [0,11,12,13,14,15,16,17,18]
[20] move_table_to_table_end(a1, b2, 3, 1, 3, 2), [0,10,11,12,13,14,15,16,17,18,19]
[21] end(), [0,1,2,3,4,5,6,7,8,9,10,11,12,13,14,15,16,17,18,19,20]
\end{minted}

Indeed, action $a_9$ may or may not be an enabler of action $a_{10}$ depending on the resource allocation of the MILP solution. If we have just one agent, then $a_9\in\ach{a_{10}}$, if instead we have more than one agent, then $a_9\not\in\ach{a_{10}}$ and the two actions can be executed at the same time and the plan would be:

\begin{minted}[fontsize=\footnotesize]{text}
[0] init()
[1] move_table_to_table_start(a1, b1, 1, 1, 1, 2)
[2] move_arm_start(a1, 1, 1)
[3] move_arm_end(a1, 1, 1)
[4] grip_start(a1)
[5] grip_end(a1)
[6] move_arm_start(a1, 1, 2)
[7] move_arm_end(a1, 1, 2)
[8] release_start(a1)
[9] release_end(a1)
[10] move_table_to_table_end(a1, b1, 1, 1, 1, 2)
[11] move_table_to_block_start(a2, b2, 3, 1, 3, 2)
[12] move_arm_start(a2, 3, 1)
[13] move_arm_end(a2, 3, 1)
[14] grip_start(a2)
[15] grip_end(a2)
[16] move_arm_start(a2, 3, 2)
[17] move_arm_end(a2, 3, 2)
[18] release_start(a2)
[19] release_end(a2)
[20] move_table_to_block_end(a2, b2, 3, 1, 3, 2)
[21] end()
\end{minted}

% \enrcom{Should I also include a figure? MR: I do not think so!}

% \subsubsection{Plan Generation - Example}

\section{\Btree Generation and Execution}\label{sec:bt}
\newcommand{\seq}[0]{\protect\writings{\texttt{SEQUENCE}}}
\newcommand{\parr}[0]{\protect\writings{\texttt{PARALLEL}}}

% In this section, we first introduce how to convert from a STN to a \bt, and then we provide some details regarding the implementation. 

% \subsubsection{\bt Generation}\label{sssec:btgen}

The conversion from STN to \bt is taken from~\cite{roveriSTNtoBT}. We summarize it here and refer the reader to the main article. 

An STN is a graph with a source and a sink, which can be artificial nodes in the sense that they represent the start and the end of the plan. Each node can have multiple parent and multiple children. Having multiple parents implies that the node cannot be executed as long as all the parents haven not finished and, whereas, having multiple children implies that they will be executed in parallel. 

With this knowledge we can extract a \btree, which is a structure that, starting from the root, ticks all the nodes in the tree until it finishes the last leaf. Nodes in the tree can be of different types:
\begin{itemize}
    \item \emph{action}: they are an action that has to be executed;
    \item \emph{control}: they can be either \seq or \parr and state how the children nodes must be executed;
    \item \emph{condition}: they check whether a condition is correct or not;
\end{itemize}
The ticking of a node means that the node is asked to do its function, e.g., if a \seq node is ticked, then it will tick the children one at a time, while if a condition node is ticked, it will make sure that the condition is satisfied before continuing with the next tick. 

The algorithm %(Algorithm~\ref{alg:stntobt})
to convert the STN to a \bt starts from the fictitious initial node (\verb|init|), and for every node it checks:
\begin{itemize}
    \item The number of children: if there is only one child, then it is a \seq node, otherwise it is a \parr node. 
    \item The number of parents: if there are more than one parents then the node must wait for all the parents to have ticked, before being executed.
    \item The type of the action: if it is a low-level action, then it is inserted into the \bt for execution, otherwise it will not be included.
\end{itemize}

%%%%%%%%%%%%%%%%%%%%%%%%%%%%%%%%%%%%%%%%%

% \begin{algorithm}
% \caption{Algorithm extracting a \btree from an STN.}
% \label{alg:stntobt}
% \KwData{The STN $G$}
% \KwResult{\bt corresponding to the STN}

% \DontPrintSemicolon

% \SetKwProg{extractBT}{EXTRACT\_BT}{}{}

% \SetKwInOut{Input} {In}
% \SetKwInOut{Output}{Out}

% \extractBT{(G)}{
%   \Input{The STN $G$ to convert}
%   \Output{The \bt $\mathcal{T}$}
%   \KwRet{$\mathcal{T}$}\;
% }

% \end{algorithm}

%%%%%%%%%%%%%%%%%%%%%%%%%%%%%%%%%%%%%%%%%

% THIS has been moved to the Implementation details section of Experimental Validation
% \subsubsection{\bt Execution}\label{sssec:btexec}

% As said, the execution of a \btree starts from the root and it gradually ticks the different nodes of the tree until all nodes have been ticked. 

% While \bts have become a de facto standard for executing robotic tasks, no universally accepted framework exists for their creation or execution. Some notable examples include PlanSys2~\cite{martin2021plansys2} and BehaviorTree.CPP~\cite{BehaviorTreeCppWebsite}. PlanSys2 is tightly integrated with ROS2; beyond merely executing \btrees, it can also derive feasible plans from a knowledge base. In contrast, BehaviorTree.CPP is a more general framework that enables the creation and execution of \bts from an XML file. We selected BehaviorTree.CPP since our main objective was to execute APIs from a \bt, which is easily represented using an XML file, while also maintaining maximum generality. Nevertheless, BehaviorTree.CPP also offers a ROS2 wrapper, which can easily be integrated with the flow.

% \enrcom{Maybe it should be moved to Section~\ref{ssec:implementation}?}

This section describes the implementation of \proposedsystem (See ~\autoref{fig:system_design}) on top of Sinfonia~\cite{satyanarayanan2022sinfonia}, a Kubernetes based open-source orchestrator for edge-native applications. Our implementation adds $\sim$4k SLOC to the  Sinfonia system and is available as open source \emph{(URL blinded)}. 
%Lastly, we describe our simulation environment used in large-scale experiments.

\subsection{\proposedsystem Prototype}
\noindent Our \proposedsystem consists of the following components that we added to Sinfonia.

\noindent\textbf{Telemetry Service}:
Our telemetry service is integrated into Sinfonia’s telemetry, where it collects static (e.g., location and IP address) attributes and real-time  (e.g., utilization) metrics. Real-time metrics are collected based on Prometheus monitoring stack\cite{Prometheus}. We augment Sinfonia's monitoring with the following metrics:
\begin{enumerate}[leftmargin=*]
    \item \textbf{Power Monitoring}: We measure the power consumption of CPU servers using RAPL~\cite{david2010rapl}, and we leveraged Prometheus's DGCM exporter for GPUs~\cite{nvidia_dcgm_exporter_github}. 

    \item \textbf{Carbon Intensity}: We integrate a carbon intensity service that replays historical traces from Electricity Maps~\cite{electricity-map} and uses the traces to provide real-time and forecast carbon intensity.

    \item \textbf{Carbon Monitoring}: We implement carbon monitoring based on energy usage and the carbon intensity of the selected edge sites, where we account for the base power (if the server is turned on) and applications energy usage. 
    
    \item \textbf{End-to-end latency}: In addition to latency across sites, we recorded end-to-end latency between users and their deployed applications.

\end{enumerate}

\noindent\textbf{Profiling Service}: 
We implement an application profiling service that collects the application's performance metrics, such as latency, power consumption, resource demands, and other crucial information, to make accurate placement decisions across available resources. Our profiling service can be replaced with performance models that statically analyze the applications and predict the latency and energy consumption~\cite{paleo, cai2017neuralpower}.

\noindent\textbf{Placement Service}:
Lastly, we implement our placement policy (\autoref{alg:algorithm}) on sinfonia as a matching policy. The placement policy utilizes the system's real-time metrics, static attributes of different edge sites, and workload profiles to determine optimal placements and server activation.
Our implementation batches deployment requests (e.g., every 5 minutes) and solves the optimization problem per application batch using the Google OR-Tools~\cite{ortools}. We demonstrate the effectiveness of our approach in ~\autoref{sec:eval_overhead}. 
%, an interface to the branch and cut solver.
%It gets the configuration of applications and edge data centers and outputs the placement and power management decisions to the edge orchestrator. 
%\noindent\textbf{Edge Orchestrator}: 
%The placement decision are then utilized by the orchestrations service that 
%We leverage Sinfonia, which is built on top of Kubernetes~\cite{kubernetes}, to deploy and manage server states. 
After computing the placement decisions, we utilize Sinfonia's orchestration capabilities to initiate the deployment sequence (Sinfonia RECIPE), which contains the necessary Kubernetes deployment files and helm charts, to the destination servers or activate servers if necessary and inform the client(s) of the destination's address. Note that although Sinfonia and our system are packed with fault-tolerance and reconfiguration capabilities, evaluating them is beyond the scope of this paper.







% \noindent\textbf{Telemetry Service}: \proposedsystem collects real-time performance, resource, and energy consumption telemetry from edge data centers. These telemetries are based on Prometheus monitoring stacks\cite{Prometheus}. We augmented Prometheus with power monitoring capabilities, where we measure the power consumption of CPU servers using RAPL~\cite{david2010rapl}, and for GPU servers, we leveraged Prometheus's DGCM exporter~\cite{nvidia_dcgm_exporter_github}. 
% Our telemetries are integrated into Sinfonia’s dynamic attributes and updated periodically. Lastly, we record end-to-end latency between users and their applications.

% \noindent\textbf{Carbon Intensity Service}: We built our carbon intensity service by integrating historical traces from Electricity Maps~\cite{electricity-map} and using the traces to provide real-time and forecast carbon intensity.


%Additionally, we include latency data based on round-trip time (RTT) measurements from WonderNetwork. Finally, alongside the telemetries, we implement an application profiling service that gathers application performance metrics, such as latency, power consumption, resource demands, and other crucial information for accurate placement decisions across available resources. 


% \noindent\textbf{Placement Service}: We implement the placement service with Python, and the carbon-aware placement optimization is solved using the Google OR-Tools~\cite{ortools}, an interface to the branch and cut solver. It gets the configuration of applications and edge data centers and outputs the placement and power management decisions to the edge orchestrator. 

% The \proposedsystem placement service encompasses the carbon-aware placement and power management decisions across different edge data centers. As we detailed in ~\autoref{sec:design_algorithm}, 
% the placement service integrates its knowledge of the carbon intensity, resource requirements, and load to compute the decisions that optimize the total carbon emissions while ensuring that resource and performance constraints are strictly met. 
% We implement \autoref{alg:algorithm} using Python and Google OR-Tools~\cite{ortools}. 


% \noindent\textbf{Edge Orchestrator}: We leverage Sinfonia, which is built on top of Kubernetes~\cite{kubernetes}, to deploy and manage server states. After getting the placement decisions, Sinfonia initiates the deployment sequence (Sinfonia RECIPE) to the destination servers or activates servers if necessary and informs the client(s) of the destination's address.  Note that although Sinfonia is packed with fault-tolerance and reconfiguration capabilities, evaluating them is beyond the scope of this paper.

\subsection{\proposedsystem Edge Simulator}
In addition to the prototype of \proposedsystem, we developed a simulator for larger-scale evaluations 
that is not feasible using an edge testbed.
Our simulator supports
simulating diverse edge settings with dynamic workloads and heterogeneous servers. This simulator represents the components of Sinfonia and follows the same decision process and metrics, where we implement our proposed carbon-aware placement policy and other baseline policies using Google OR-Tools~\cite{ortools}. The \proposedsystem simulator is implemented in Python using $\sim$2k SLOC.

%Sinfonia consists of three tiers: Tier 1, the cloud; Tier 2, edge sites; and Tier 3, end devices. Tiers 1 and 2 are Kubernetes clusters (K3s) equipped with Prometheus-based monitoring stacks. Tier 1 serves as the control plane, receiving offloading (placement) requests from Tier 3, selecting appropriate edge sites in Tier 2,  and responding with placement decisions. It maintains a CloudletTable that stores both static (e.g., location) and dynamic (e.g., resource utilization) attributes of Tier 2 servers, allowing for placement decisions based on predefined matches (policies). Sinfonia's flexibility to support customer matches allowed us to seamlessly integrate \proposedsystem's carbon-aware placement policy. 

% We built our Carbon Intensity Service (see ~\autoref{sec:design_arch}) by integrating historical traces from Electricity Maps~\cite{electricity-map} and using the traces to provide real-time and forecasted carbon intensity for edge servers in Tier 2. We map each edge data center to a carbon zone by utilizing its geographic coordinates. 





% In addition, we augmented the telemetries in Sinfonia (e.g., utilization, response time) with power monitoring capabilities. To measure power consumption of CPU servers using RAPL~\cite{david2010rapl}, while for GPU servers, we leveraged Prometheus's DGCM exporter~\cite{nvidia_dcgm_exporter_github}. These energy metrics, along with the carbon intensity data, were integrated into Sinfonia’s dynamic attributes and periodically updated from Tier 2 to Tier 1 (default interval of 15 seconds). All dynamic metrics are stored in the CloudletTable in Tier 1 for informed placement decision-making. To include the latency data, we utilized the round-trip time (RTT) measurements from WonderNetwork, stored in a configuration file.

% We implemented \autoref{alg:algorithm} as a new \emph{matcher} in Sinfonia's Tier 1. When a new application arrives (i.e., a Sinfonia RECIPE), the matcher batches applications, utilizes the available resources and solves the ILP using Google OR-Tools~\cite{ortools}. After mapping the application to servers,  Sinfonia initiates the deployment sequence and informs the client(s) (Tier 3) of the destination's address. Note that, although Sinfonia is packed with fault-tolerance and reconfiguration capabilities, evaluating them are beyond the scope of this paper. 




\section{Results}

\subsection{Accuracy}
\label{sec:accuracy}
We utilized the properties of the test set (Sec.~\ref{sec:dataset}) as conditions for the inverse generation test.
For each microstructure, 8 candidate models were generated, resulting in a total of 8,000 models.
Tests were performed on a 4xA40 server, with an average generation time of \textbf{0.13 seconds} per microstructure and \textbf{0.02 seconds} for SDF decoding.
Properties were computed for each model using the homogenization method, and errors were calculated according to Eq.\ref{eq:error}.

\begin{figure}
    \centering
    \includegraphics[width=\linewidth]{figures_new/err_dist.jpg}
    \caption{The error distribution of MIND and \cite{Yang2024} on the same test dataset.}
    \label{fig:err_dist}
\end{figure} 


Tab. \ref{tab:error_comparison} and Fig.~\ref{fig:err_dist} present a comparison between our method and \cite{Yang2024}, demonstrating that our approach achieves state-of-the-art performance in inverse generation of microstructures. 
Moreover, our method can be viewed as a variant of the triplane representation \cite{chan2022triplane, Shue2023}.
To assess the necessity of the Holoplane, we conducted ablation studies using the triplane representation (NFD~\cite{Shue2023}). 
Among the generated models, some were excluded due to issues with translational symmetry or connectivity, preventing property calculations.
The physical validity ratio was \textbf{95.7\%} for \cite{Yang2024}, \textbf{97.8\%} for the NFD representation, and \textbf{99.2\%} for our method.
The results strongly emphasize that our Holoplane representation significantly enhances the accuracy of property-conditioned generation.


\begin{table}[ht]
    \centering
    \caption{
    The properties of the test set are used as the conditions to evaluate generation errors.
    $C_{\text{best}}$ is the average of the best result from each group of 8, while the other columns represent the average of all 8000 structures. The NFD + Phy approach leverages the physical-aware embedding introduced in Sec.~\ref{sec:phy_enc} to align geometry and physics within the latent space.}
    \label{tab:error_comparison}
    \begin{tabular}{lcc|ccc}
        \toprule
        Method                      & $\bC_{\text{best}}$  & $\bC_{\text{all}}$ & C\textsubscript{11} & C\textsubscript{12} & C\textsubscript{44} \\
        \midrule
        \cite{Yang2024}             & 1.33\% & 2.96\% & 2.50\% & 3.68\% & 2.70\% \\
        NFD                         & 0.63\% & 5.39\% & 5.28\% & 4.86\% & 6.01\% \\
        NFD + Phy                   & 0.44\% & 1.68\% & 1.49\% & 1.80\% & 1.75\% \\
        MIND (Ours)                 & 0.29\% & 1.27\% & 1.13\% & 1.33\% & 1.34\% \\
        \bottomrule
    \end{tabular}

\end{table}



\subsection{Generation Boundary}

\begin{figure}
    \centering
    \includegraphics[width=\linewidth]{figures_new/design_space_v3.png}
    \caption{Comparison of the property space between the training set and the generation set.
    The training set consists of approximately 180,000 samples, as described in Sec.~\ref{sec:dataset}, while the generation set contains around 550,000 samples obtained by random sampling inside and near the boundary of the property space.
    We visualize both the original distribution and the log-scale distribution of the property space (top-left corner).
    Our method effectively extends the boundary of the property space, significantly increasing the maximum Young's modulus and shear modulus while also achieving a lower negative Poisson's ratio.}
    \label{fig:design_space}
\end{figure} 


To explore the boundary of our network’s generative capacity, we randomly sample points near the boundary of the property space. 
This process continues until the network fails to generate structures that meet the specified properties. 
As shown in Fig.~\ref{fig:design_space}, our method successfully expands the design space, showcasing its strong generative capability.



\subsection{Diversity}

\begin{figure}
    \centering
    \includegraphics[width=\linewidth]{figures_new/shape_gen.jpg}
\caption{Inverse generation from a reference model.
Using the reference model's mechanical properties as input, five candidate models generated by our framework are listed.
Each model's Young's modulus surface is shown with a consistent color bar, demonstrating structures in diverse morphologies with similar mechanical properties.}
    \label{fig:shape_gen}
\end{figure} 


We compute the average similarity between the 8,000 generated microstructures and the entire training set. 
Our model achieves an average similarity of \textbf{81.52\%}, which is lower than the \textbf{93.48\%} reported in \cite{Yang2024}. 
Besides, we selected some representative models and utilized their mechanical properties ($E, \nu, G$) as input for inverse generation.
As shown in Fig.~\ref{fig:shape_gen}, given a target property as input, our method can generate multiple distinct types of structures while maintaining similar properties.
This indicates that our model exhibits greater shape diversity and is not merely memorizing the training data. 
Additional results are visualized in \srefmorevis.


\subsection{Interpolation}

\begin{figure}
    \centering
    \includegraphics[width=\linewidth]{figures_new/interpolation.jpg}
    \caption{Interpolation using different properties.
    The start models are truss structures with low Young’s moduli and small volume fractions, while the end models are plate structures with high Young’s moduli and larger volume fractions.
    Initially, the interpolated models retain their truss topology, increasing the volume fraction to achieve higher Young’s modulus values.
    Gradually, the structures transition into plate configurations, ultimately forming plate structures with significantly higher Young’s moduli.
    }
    \label{fig:interpolation}
\end{figure} 

Our approach also enables the generation of novel structures through interpolations, as described in Sec.~\ref{sec:compat}. 
Specifically, we performed interpolation experiments on two groups of models with significantly different material properties, as well as several groups of models belonging to distinct microstructure families.
As shown in Fig.~\ref{fig:interpolation} and Fig.~\ref{fig:supp_interpolation}, our method achieves smooth geometric and physical transitions within each group of configurations.




\setlength{\belowcaptionskip}{5pt}

\subsection{Printability}

\begin{figure*}
    \centering
    \includegraphics[width=\linewidth]{figures_new/printability.jpg}
    \caption{(a) and (b) show two generated microstructures under different printing precisions, along with their property error distributions visualized as heatmaps. 
    (c) and (d) present the printed results corresponding to (a) and (b).
    The two test models used a $4 \times 4 \times 1$ grid of microstructures, each with a cell size of 20 mm.
    The experiments were conducted with printer precision settings of 0.6 mm and 1.2 mm, corresponding to detection resolutions of $32^3$ and $16^3$.
    The smallest feature sizes in the first and second test cases were \textbf{0.62 mm} and \textbf{1.25 mm}, respectively, with property errors of \textbf{0.2\%} and \textbf{0.4\%}.
    For the first structure, the smallest feature size was 0.62 mm, while the second structure exhibited a minimum feature size of 1.25 mm.}
    \label{fig:print}
\end{figure*} 

We tested the ability of our method to generate printable objects at different printing precisions of 0.6 mm and 1.2 mm.
Finer printing resolution allows for more precise error control.
Experiments show that at both fine and coarse printing resolutions, our method can still generate printable structures that meet the required property specifications (Fig.~\ref{fig:print}).



\subsection{Heterogeneous Design}

\begin{figure*}
    \centering
    \includegraphics[width=\linewidth]{figures_new/pillow_bracket.jpg}
    \caption{
    We tested MIND on the pillow bracket model with a 0.01 m grid resolution. 
    The base of the model was fixed, and forces were applied to the two handles, with the model divided into 392 grids for the design region (a, b).
    Initial material properties were set to $E = 0.1$, $\nu = 0.25$, and $G = 0.03$, resulting in displacements of \textbf{0.000 m | 0.016 m | 0.039 m} (min, avg, max) (e).
    We then constrained the sum of Young's modulus to remain constant and optimized the physical properties of each cell (c).
    The microstructures generated by MIND were then filled in the design region (d).
    After optimization, the displacement was reduced to \textbf{0.000 m | 0.008 m | 0.032 m} (f). 
    The filled model produced a displacement \textbf{of 0.000 m | 0.008 m | 0.034 m} (g), closely matching the optimization results. 
    Printability testing showcased that all microstructures were printable and confirmed perfect compatibility between adjacent cells (h).}
    \label{fig:pillow}
\end{figure*}


To validate MIND in heterogeneous design, we tested it on a pillow bracket model discretized into a 0.01 m grid. 
The material properties were optimized to minimize overall displacement. 
After optimizing the material distribution, we applied MIND for inverse design and blended the boundaries. 
The resulting displacement performance closely aligned with the optimization target, demonstrating MIND’s capability to generate microstructures that meet mechanical requirements while ensuring boundary compatibility (Fig.~\ref{fig:pillow}).


\section{Conclusion}

In this paper, we introduce STeCa, a novel agent learning framework designed to enhance the performance of LLM agents in long-horizon tasks. 
STeCa identifies deviated actions through step-level reward comparisons and constructs calibration trajectories via reflection. 
These trajectories serve as critical data for reinforced training. Extensive experiments demonstrate that STeCa significantly outperforms baseline methods, with additional analyses underscoring its robust calibration capabilities.


\bibliographystyle{ACM-Reference-Format}
\bibliography{MIND}



% \input{contents/8-extrafigs}

\clearpage
\appendix
\subsection{Lloyd-Max Algorithm}
\label{subsec:Lloyd-Max}
For a given quantization bitwidth $B$ and an operand $\bm{X}$, the Lloyd-Max algorithm finds $2^B$ quantization levels $\{\hat{x}_i\}_{i=1}^{2^B}$ such that quantizing $\bm{X}$ by rounding each scalar in $\bm{X}$ to the nearest quantization level minimizes the quantization MSE. 

The algorithm starts with an initial guess of quantization levels and then iteratively computes quantization thresholds $\{\tau_i\}_{i=1}^{2^B-1}$ and updates quantization levels $\{\hat{x}_i\}_{i=1}^{2^B}$. Specifically, at iteration $n$, thresholds are set to the midpoints of the previous iteration's levels:
\begin{align*}
    \tau_i^{(n)}=\frac{\hat{x}_i^{(n-1)}+\hat{x}_{i+1}^{(n-1)}}2 \text{ for } i=1\ldots 2^B-1
\end{align*}
Subsequently, the quantization levels are re-computed as conditional means of the data regions defined by the new thresholds:
\begin{align*}
    \hat{x}_i^{(n)}=\mathbb{E}\left[ \bm{X} \big| \bm{X}\in [\tau_{i-1}^{(n)},\tau_i^{(n)}] \right] \text{ for } i=1\ldots 2^B
\end{align*}
where to satisfy boundary conditions we have $\tau_0=-\infty$ and $\tau_{2^B}=\infty$. The algorithm iterates the above steps until convergence.

Figure \ref{fig:lm_quant} compares the quantization levels of a $7$-bit floating point (E3M3) quantizer (left) to a $7$-bit Lloyd-Max quantizer (right) when quantizing a layer of weights from the GPT3-126M model at a per-tensor granularity. As shown, the Lloyd-Max quantizer achieves substantially lower quantization MSE. Further, Table \ref{tab:FP7_vs_LM7} shows the superior perplexity achieved by Lloyd-Max quantizers for bitwidths of $7$, $6$ and $5$. The difference between the quantizers is clear at 5 bits, where per-tensor FP quantization incurs a drastic and unacceptable increase in perplexity, while Lloyd-Max quantization incurs a much smaller increase. Nevertheless, we note that even the optimal Lloyd-Max quantizer incurs a notable ($\sim 1.5$) increase in perplexity due to the coarse granularity of quantization. 

\begin{figure}[h]
  \centering
  \includegraphics[width=0.7\linewidth]{sections/figures/LM7_FP7.pdf}
  \caption{\small Quantization levels and the corresponding quantization MSE of Floating Point (left) vs Lloyd-Max (right) Quantizers for a layer of weights in the GPT3-126M model.}
  \label{fig:lm_quant}
\end{figure}

\begin{table}[h]\scriptsize
\begin{center}
\caption{\label{tab:FP7_vs_LM7} \small Comparing perplexity (lower is better) achieved by floating point quantizers and Lloyd-Max quantizers on a GPT3-126M model for the Wikitext-103 dataset.}
\begin{tabular}{c|cc|c}
\hline
 \multirow{2}{*}{\textbf{Bitwidth}} & \multicolumn{2}{|c|}{\textbf{Floating-Point Quantizer}} & \textbf{Lloyd-Max Quantizer} \\
 & Best Format & Wikitext-103 Perplexity & Wikitext-103 Perplexity \\
\hline
7 & E3M3 & 18.32 & 18.27 \\
6 & E3M2 & 19.07 & 18.51 \\
5 & E4M0 & 43.89 & 19.71 \\
\hline
\end{tabular}
\end{center}
\end{table}

\subsection{Proof of Local Optimality of LO-BCQ}
\label{subsec:lobcq_opt_proof}
For a given block $\bm{b}_j$, the quantization MSE during LO-BCQ can be empirically evaluated as $\frac{1}{L_b}\lVert \bm{b}_j- \bm{\hat{b}}_j\rVert^2_2$ where $\bm{\hat{b}}_j$ is computed from equation (\ref{eq:clustered_quantization_definition}) as $C_{f(\bm{b}_j)}(\bm{b}_j)$. Further, for a given block cluster $\mathcal{B}_i$, we compute the quantization MSE as $\frac{1}{|\mathcal{B}_{i}|}\sum_{\bm{b} \in \mathcal{B}_{i}} \frac{1}{L_b}\lVert \bm{b}- C_i^{(n)}(\bm{b})\rVert^2_2$. Therefore, at the end of iteration $n$, we evaluate the overall quantization MSE $J^{(n)}$ for a given operand $\bm{X}$ composed of $N_c$ block clusters as:
\begin{align*}
    \label{eq:mse_iter_n}
    J^{(n)} = \frac{1}{N_c} \sum_{i=1}^{N_c} \frac{1}{|\mathcal{B}_{i}^{(n)}|}\sum_{\bm{v} \in \mathcal{B}_{i}^{(n)}} \frac{1}{L_b}\lVert \bm{b}- B_i^{(n)}(\bm{b})\rVert^2_2
\end{align*}

At the end of iteration $n$, the codebooks are updated from $\mathcal{C}^{(n-1)}$ to $\mathcal{C}^{(n)}$. However, the mapping of a given vector $\bm{b}_j$ to quantizers $\mathcal{C}^{(n)}$ remains as  $f^{(n)}(\bm{b}_j)$. At the next iteration, during the vector clustering step, $f^{(n+1)}(\bm{b}_j)$ finds new mapping of $\bm{b}_j$ to updated codebooks $\mathcal{C}^{(n)}$ such that the quantization MSE over the candidate codebooks is minimized. Therefore, we obtain the following result for $\bm{b}_j$:
\begin{align*}
\frac{1}{L_b}\lVert \bm{b}_j - C_{f^{(n+1)}(\bm{b}_j)}^{(n)}(\bm{b}_j)\rVert^2_2 \le \frac{1}{L_b}\lVert \bm{b}_j - C_{f^{(n)}(\bm{b}_j)}^{(n)}(\bm{b}_j)\rVert^2_2
\end{align*}

That is, quantizing $\bm{b}_j$ at the end of the block clustering step of iteration $n+1$ results in lower quantization MSE compared to quantizing at the end of iteration $n$. Since this is true for all $\bm{b} \in \bm{X}$, we assert the following:
\begin{equation}
\begin{split}
\label{eq:mse_ineq_1}
    \tilde{J}^{(n+1)} &= \frac{1}{N_c} \sum_{i=1}^{N_c} \frac{1}{|\mathcal{B}_{i}^{(n+1)}|}\sum_{\bm{b} \in \mathcal{B}_{i}^{(n+1)}} \frac{1}{L_b}\lVert \bm{b} - C_i^{(n)}(b)\rVert^2_2 \le J^{(n)}
\end{split}
\end{equation}
where $\tilde{J}^{(n+1)}$ is the the quantization MSE after the vector clustering step at iteration $n+1$.

Next, during the codebook update step (\ref{eq:quantizers_update}) at iteration $n+1$, the per-cluster codebooks $\mathcal{C}^{(n)}$ are updated to $\mathcal{C}^{(n+1)}$ by invoking the Lloyd-Max algorithm \citep{Lloyd}. We know that for any given value distribution, the Lloyd-Max algorithm minimizes the quantization MSE. Therefore, for a given vector cluster $\mathcal{B}_i$ we obtain the following result:

\begin{equation}
    \frac{1}{|\mathcal{B}_{i}^{(n+1)}|}\sum_{\bm{b} \in \mathcal{B}_{i}^{(n+1)}} \frac{1}{L_b}\lVert \bm{b}- C_i^{(n+1)}(\bm{b})\rVert^2_2 \le \frac{1}{|\mathcal{B}_{i}^{(n+1)}|}\sum_{\bm{b} \in \mathcal{B}_{i}^{(n+1)}} \frac{1}{L_b}\lVert \bm{b}- C_i^{(n)}(\bm{b})\rVert^2_2
\end{equation}

The above equation states that quantizing the given block cluster $\mathcal{B}_i$ after updating the associated codebook from $C_i^{(n)}$ to $C_i^{(n+1)}$ results in lower quantization MSE. Since this is true for all the block clusters, we derive the following result: 
\begin{equation}
\begin{split}
\label{eq:mse_ineq_2}
     J^{(n+1)} &= \frac{1}{N_c} \sum_{i=1}^{N_c} \frac{1}{|\mathcal{B}_{i}^{(n+1)}|}\sum_{\bm{b} \in \mathcal{B}_{i}^{(n+1)}} \frac{1}{L_b}\lVert \bm{b}- C_i^{(n+1)}(\bm{b})\rVert^2_2  \le \tilde{J}^{(n+1)}   
\end{split}
\end{equation}

Following (\ref{eq:mse_ineq_1}) and (\ref{eq:mse_ineq_2}), we find that the quantization MSE is non-increasing for each iteration, that is, $J^{(1)} \ge J^{(2)} \ge J^{(3)} \ge \ldots \ge J^{(M)}$ where $M$ is the maximum number of iterations. 
%Therefore, we can say that if the algorithm converges, then it must be that it has converged to a local minimum. 
\hfill $\blacksquare$


\begin{figure}
    \begin{center}
    \includegraphics[width=0.5\textwidth]{sections//figures/mse_vs_iter.pdf}
    \end{center}
    \caption{\small NMSE vs iterations during LO-BCQ compared to other block quantization proposals}
    \label{fig:nmse_vs_iter}
\end{figure}

Figure \ref{fig:nmse_vs_iter} shows the empirical convergence of LO-BCQ across several block lengths and number of codebooks. Also, the MSE achieved by LO-BCQ is compared to baselines such as MXFP and VSQ. As shown, LO-BCQ converges to a lower MSE than the baselines. Further, we achieve better convergence for larger number of codebooks ($N_c$) and for a smaller block length ($L_b$), both of which increase the bitwidth of BCQ (see Eq \ref{eq:bitwidth_bcq}).


\subsection{Additional Accuracy Results}
%Table \ref{tab:lobcq_config} lists the various LOBCQ configurations and their corresponding bitwidths.
\begin{table}
\setlength{\tabcolsep}{4.75pt}
\begin{center}
\caption{\label{tab:lobcq_config} Various LO-BCQ configurations and their bitwidths.}
\begin{tabular}{|c||c|c|c|c||c|c||c|} 
\hline
 & \multicolumn{4}{|c||}{$L_b=8$} & \multicolumn{2}{|c||}{$L_b=4$} & $L_b=2$ \\
 \hline
 \backslashbox{$L_A$\kern-1em}{\kern-1em$N_c$} & 2 & 4 & 8 & 16 & 2 & 4 & 2 \\
 \hline
 64 & 4.25 & 4.375 & 4.5 & 4.625 & 4.375 & 4.625 & 4.625\\
 \hline
 32 & 4.375 & 4.5 & 4.625& 4.75 & 4.5 & 4.75 & 4.75 \\
 \hline
 16 & 4.625 & 4.75& 4.875 & 5 & 4.75 & 5 & 5 \\
 \hline
\end{tabular}
\end{center}
\end{table}

%\subsection{Perplexity achieved by various LO-BCQ configurations on Wikitext-103 dataset}

\begin{table} \centering
\begin{tabular}{|c||c|c|c|c||c|c||c|} 
\hline
 $L_b \rightarrow$& \multicolumn{4}{c||}{8} & \multicolumn{2}{c||}{4} & 2\\
 \hline
 \backslashbox{$L_A$\kern-1em}{\kern-1em$N_c$} & 2 & 4 & 8 & 16 & 2 & 4 & 2  \\
 %$N_c \rightarrow$ & 2 & 4 & 8 & 16 & 2 & 4 & 2 \\
 \hline
 \hline
 \multicolumn{8}{c}{GPT3-1.3B (FP32 PPL = 9.98)} \\ 
 \hline
 \hline
 64 & 10.40 & 10.23 & 10.17 & 10.15 &  10.28 & 10.18 & 10.19 \\
 \hline
 32 & 10.25 & 10.20 & 10.15 & 10.12 &  10.23 & 10.17 & 10.17 \\
 \hline
 16 & 10.22 & 10.16 & 10.10 & 10.09 &  10.21 & 10.14 & 10.16 \\
 \hline
  \hline
 \multicolumn{8}{c}{GPT3-8B (FP32 PPL = 7.38)} \\ 
 \hline
 \hline
 64 & 7.61 & 7.52 & 7.48 &  7.47 &  7.55 &  7.49 & 7.50 \\
 \hline
 32 & 7.52 & 7.50 & 7.46 &  7.45 &  7.52 &  7.48 & 7.48  \\
 \hline
 16 & 7.51 & 7.48 & 7.44 &  7.44 &  7.51 &  7.49 & 7.47  \\
 \hline
\end{tabular}
\caption{\label{tab:ppl_gpt3_abalation} Wikitext-103 perplexity across GPT3-1.3B and 8B models.}
\end{table}

\begin{table} \centering
\begin{tabular}{|c||c|c|c|c||} 
\hline
 $L_b \rightarrow$& \multicolumn{4}{c||}{8}\\
 \hline
 \backslashbox{$L_A$\kern-1em}{\kern-1em$N_c$} & 2 & 4 & 8 & 16 \\
 %$N_c \rightarrow$ & 2 & 4 & 8 & 16 & 2 & 4 & 2 \\
 \hline
 \hline
 \multicolumn{5}{|c|}{Llama2-7B (FP32 PPL = 5.06)} \\ 
 \hline
 \hline
 64 & 5.31 & 5.26 & 5.19 & 5.18  \\
 \hline
 32 & 5.23 & 5.25 & 5.18 & 5.15  \\
 \hline
 16 & 5.23 & 5.19 & 5.16 & 5.14  \\
 \hline
 \multicolumn{5}{|c|}{Nemotron4-15B (FP32 PPL = 5.87)} \\ 
 \hline
 \hline
 64  & 6.3 & 6.20 & 6.13 & 6.08  \\
 \hline
 32  & 6.24 & 6.12 & 6.07 & 6.03  \\
 \hline
 16  & 6.12 & 6.14 & 6.04 & 6.02  \\
 \hline
 \multicolumn{5}{|c|}{Nemotron4-340B (FP32 PPL = 3.48)} \\ 
 \hline
 \hline
 64 & 3.67 & 3.62 & 3.60 & 3.59 \\
 \hline
 32 & 3.63 & 3.61 & 3.59 & 3.56 \\
 \hline
 16 & 3.61 & 3.58 & 3.57 & 3.55 \\
 \hline
\end{tabular}
\caption{\label{tab:ppl_llama7B_nemo15B} Wikitext-103 perplexity compared to FP32 baseline in Llama2-7B and Nemotron4-15B, 340B models}
\end{table}

%\subsection{Perplexity achieved by various LO-BCQ configurations on MMLU dataset}


\begin{table} \centering
\begin{tabular}{|c||c|c|c|c||c|c|c|c|} 
\hline
 $L_b \rightarrow$& \multicolumn{4}{c||}{8} & \multicolumn{4}{c||}{8}\\
 \hline
 \backslashbox{$L_A$\kern-1em}{\kern-1em$N_c$} & 2 & 4 & 8 & 16 & 2 & 4 & 8 & 16  \\
 %$N_c \rightarrow$ & 2 & 4 & 8 & 16 & 2 & 4 & 2 \\
 \hline
 \hline
 \multicolumn{5}{|c|}{Llama2-7B (FP32 Accuracy = 45.8\%)} & \multicolumn{4}{|c|}{Llama2-70B (FP32 Accuracy = 69.12\%)} \\ 
 \hline
 \hline
 64 & 43.9 & 43.4 & 43.9 & 44.9 & 68.07 & 68.27 & 68.17 & 68.75 \\
 \hline
 32 & 44.5 & 43.8 & 44.9 & 44.5 & 68.37 & 68.51 & 68.35 & 68.27  \\
 \hline
 16 & 43.9 & 42.7 & 44.9 & 45 & 68.12 & 68.77 & 68.31 & 68.59  \\
 \hline
 \hline
 \multicolumn{5}{|c|}{GPT3-22B (FP32 Accuracy = 38.75\%)} & \multicolumn{4}{|c|}{Nemotron4-15B (FP32 Accuracy = 64.3\%)} \\ 
 \hline
 \hline
 64 & 36.71 & 38.85 & 38.13 & 38.92 & 63.17 & 62.36 & 63.72 & 64.09 \\
 \hline
 32 & 37.95 & 38.69 & 39.45 & 38.34 & 64.05 & 62.30 & 63.8 & 64.33  \\
 \hline
 16 & 38.88 & 38.80 & 38.31 & 38.92 & 63.22 & 63.51 & 63.93 & 64.43  \\
 \hline
\end{tabular}
\caption{\label{tab:mmlu_abalation} Accuracy on MMLU dataset across GPT3-22B, Llama2-7B, 70B and Nemotron4-15B models.}
\end{table}


%\subsection{Perplexity achieved by various LO-BCQ configurations on LM evaluation harness}

\begin{table} \centering
\begin{tabular}{|c||c|c|c|c||c|c|c|c|} 
\hline
 $L_b \rightarrow$& \multicolumn{4}{c||}{8} & \multicolumn{4}{c||}{8}\\
 \hline
 \backslashbox{$L_A$\kern-1em}{\kern-1em$N_c$} & 2 & 4 & 8 & 16 & 2 & 4 & 8 & 16  \\
 %$N_c \rightarrow$ & 2 & 4 & 8 & 16 & 2 & 4 & 2 \\
 \hline
 \hline
 \multicolumn{5}{|c|}{Race (FP32 Accuracy = 37.51\%)} & \multicolumn{4}{|c|}{Boolq (FP32 Accuracy = 64.62\%)} \\ 
 \hline
 \hline
 64 & 36.94 & 37.13 & 36.27 & 37.13 & 63.73 & 62.26 & 63.49 & 63.36 \\
 \hline
 32 & 37.03 & 36.36 & 36.08 & 37.03 & 62.54 & 63.51 & 63.49 & 63.55  \\
 \hline
 16 & 37.03 & 37.03 & 36.46 & 37.03 & 61.1 & 63.79 & 63.58 & 63.33  \\
 \hline
 \hline
 \multicolumn{5}{|c|}{Winogrande (FP32 Accuracy = 58.01\%)} & \multicolumn{4}{|c|}{Piqa (FP32 Accuracy = 74.21\%)} \\ 
 \hline
 \hline
 64 & 58.17 & 57.22 & 57.85 & 58.33 & 73.01 & 73.07 & 73.07 & 72.80 \\
 \hline
 32 & 59.12 & 58.09 & 57.85 & 58.41 & 73.01 & 73.94 & 72.74 & 73.18  \\
 \hline
 16 & 57.93 & 58.88 & 57.93 & 58.56 & 73.94 & 72.80 & 73.01 & 73.94  \\
 \hline
\end{tabular}
\caption{\label{tab:mmlu_abalation} Accuracy on LM evaluation harness tasks on GPT3-1.3B model.}
\end{table}

\begin{table} \centering
\begin{tabular}{|c||c|c|c|c||c|c|c|c|} 
\hline
 $L_b \rightarrow$& \multicolumn{4}{c||}{8} & \multicolumn{4}{c||}{8}\\
 \hline
 \backslashbox{$L_A$\kern-1em}{\kern-1em$N_c$} & 2 & 4 & 8 & 16 & 2 & 4 & 8 & 16  \\
 %$N_c \rightarrow$ & 2 & 4 & 8 & 16 & 2 & 4 & 2 \\
 \hline
 \hline
 \multicolumn{5}{|c|}{Race (FP32 Accuracy = 41.34\%)} & \multicolumn{4}{|c|}{Boolq (FP32 Accuracy = 68.32\%)} \\ 
 \hline
 \hline
 64 & 40.48 & 40.10 & 39.43 & 39.90 & 69.20 & 68.41 & 69.45 & 68.56 \\
 \hline
 32 & 39.52 & 39.52 & 40.77 & 39.62 & 68.32 & 67.43 & 68.17 & 69.30  \\
 \hline
 16 & 39.81 & 39.71 & 39.90 & 40.38 & 68.10 & 66.33 & 69.51 & 69.42  \\
 \hline
 \hline
 \multicolumn{5}{|c|}{Winogrande (FP32 Accuracy = 67.88\%)} & \multicolumn{4}{|c|}{Piqa (FP32 Accuracy = 78.78\%)} \\ 
 \hline
 \hline
 64 & 66.85 & 66.61 & 67.72 & 67.88 & 77.31 & 77.42 & 77.75 & 77.64 \\
 \hline
 32 & 67.25 & 67.72 & 67.72 & 67.00 & 77.31 & 77.04 & 77.80 & 77.37  \\
 \hline
 16 & 68.11 & 68.90 & 67.88 & 67.48 & 77.37 & 78.13 & 78.13 & 77.69  \\
 \hline
\end{tabular}
\caption{\label{tab:mmlu_abalation} Accuracy on LM evaluation harness tasks on GPT3-8B model.}
\end{table}

\begin{table} \centering
\begin{tabular}{|c||c|c|c|c||c|c|c|c|} 
\hline
 $L_b \rightarrow$& \multicolumn{4}{c||}{8} & \multicolumn{4}{c||}{8}\\
 \hline
 \backslashbox{$L_A$\kern-1em}{\kern-1em$N_c$} & 2 & 4 & 8 & 16 & 2 & 4 & 8 & 16  \\
 %$N_c \rightarrow$ & 2 & 4 & 8 & 16 & 2 & 4 & 2 \\
 \hline
 \hline
 \multicolumn{5}{|c|}{Race (FP32 Accuracy = 40.67\%)} & \multicolumn{4}{|c|}{Boolq (FP32 Accuracy = 76.54\%)} \\ 
 \hline
 \hline
 64 & 40.48 & 40.10 & 39.43 & 39.90 & 75.41 & 75.11 & 77.09 & 75.66 \\
 \hline
 32 & 39.52 & 39.52 & 40.77 & 39.62 & 76.02 & 76.02 & 75.96 & 75.35  \\
 \hline
 16 & 39.81 & 39.71 & 39.90 & 40.38 & 75.05 & 73.82 & 75.72 & 76.09  \\
 \hline
 \hline
 \multicolumn{5}{|c|}{Winogrande (FP32 Accuracy = 70.64\%)} & \multicolumn{4}{|c|}{Piqa (FP32 Accuracy = 79.16\%)} \\ 
 \hline
 \hline
 64 & 69.14 & 70.17 & 70.17 & 70.56 & 78.24 & 79.00 & 78.62 & 78.73 \\
 \hline
 32 & 70.96 & 69.69 & 71.27 & 69.30 & 78.56 & 79.49 & 79.16 & 78.89  \\
 \hline
 16 & 71.03 & 69.53 & 69.69 & 70.40 & 78.13 & 79.16 & 79.00 & 79.00  \\
 \hline
\end{tabular}
\caption{\label{tab:mmlu_abalation} Accuracy on LM evaluation harness tasks on GPT3-22B model.}
\end{table}

\begin{table} \centering
\begin{tabular}{|c||c|c|c|c||c|c|c|c|} 
\hline
 $L_b \rightarrow$& \multicolumn{4}{c||}{8} & \multicolumn{4}{c||}{8}\\
 \hline
 \backslashbox{$L_A$\kern-1em}{\kern-1em$N_c$} & 2 & 4 & 8 & 16 & 2 & 4 & 8 & 16  \\
 %$N_c \rightarrow$ & 2 & 4 & 8 & 16 & 2 & 4 & 2 \\
 \hline
 \hline
 \multicolumn{5}{|c|}{Race (FP32 Accuracy = 44.4\%)} & \multicolumn{4}{|c|}{Boolq (FP32 Accuracy = 79.29\%)} \\ 
 \hline
 \hline
 64 & 42.49 & 42.51 & 42.58 & 43.45 & 77.58 & 77.37 & 77.43 & 78.1 \\
 \hline
 32 & 43.35 & 42.49 & 43.64 & 43.73 & 77.86 & 75.32 & 77.28 & 77.86  \\
 \hline
 16 & 44.21 & 44.21 & 43.64 & 42.97 & 78.65 & 77 & 76.94 & 77.98  \\
 \hline
 \hline
 \multicolumn{5}{|c|}{Winogrande (FP32 Accuracy = 69.38\%)} & \multicolumn{4}{|c|}{Piqa (FP32 Accuracy = 78.07\%)} \\ 
 \hline
 \hline
 64 & 68.9 & 68.43 & 69.77 & 68.19 & 77.09 & 76.82 & 77.09 & 77.86 \\
 \hline
 32 & 69.38 & 68.51 & 68.82 & 68.90 & 78.07 & 76.71 & 78.07 & 77.86  \\
 \hline
 16 & 69.53 & 67.09 & 69.38 & 68.90 & 77.37 & 77.8 & 77.91 & 77.69  \\
 \hline
\end{tabular}
\caption{\label{tab:mmlu_abalation} Accuracy on LM evaluation harness tasks on Llama2-7B model.}
\end{table}

\begin{table} \centering
\begin{tabular}{|c||c|c|c|c||c|c|c|c|} 
\hline
 $L_b \rightarrow$& \multicolumn{4}{c||}{8} & \multicolumn{4}{c||}{8}\\
 \hline
 \backslashbox{$L_A$\kern-1em}{\kern-1em$N_c$} & 2 & 4 & 8 & 16 & 2 & 4 & 8 & 16  \\
 %$N_c \rightarrow$ & 2 & 4 & 8 & 16 & 2 & 4 & 2 \\
 \hline
 \hline
 \multicolumn{5}{|c|}{Race (FP32 Accuracy = 48.8\%)} & \multicolumn{4}{|c|}{Boolq (FP32 Accuracy = 85.23\%)} \\ 
 \hline
 \hline
 64 & 49.00 & 49.00 & 49.28 & 48.71 & 82.82 & 84.28 & 84.03 & 84.25 \\
 \hline
 32 & 49.57 & 48.52 & 48.33 & 49.28 & 83.85 & 84.46 & 84.31 & 84.93  \\
 \hline
 16 & 49.85 & 49.09 & 49.28 & 48.99 & 85.11 & 84.46 & 84.61 & 83.94  \\
 \hline
 \hline
 \multicolumn{5}{|c|}{Winogrande (FP32 Accuracy = 79.95\%)} & \multicolumn{4}{|c|}{Piqa (FP32 Accuracy = 81.56\%)} \\ 
 \hline
 \hline
 64 & 78.77 & 78.45 & 78.37 & 79.16 & 81.45 & 80.69 & 81.45 & 81.5 \\
 \hline
 32 & 78.45 & 79.01 & 78.69 & 80.66 & 81.56 & 80.58 & 81.18 & 81.34  \\
 \hline
 16 & 79.95 & 79.56 & 79.79 & 79.72 & 81.28 & 81.66 & 81.28 & 80.96  \\
 \hline
\end{tabular}
\caption{\label{tab:mmlu_abalation} Accuracy on LM evaluation harness tasks on Llama2-70B model.}
\end{table}

%\section{MSE Studies}
%\textcolor{red}{TODO}


\subsection{Number Formats and Quantization Method}
\label{subsec:numFormats_quantMethod}
\subsubsection{Integer Format}
An $n$-bit signed integer (INT) is typically represented with a 2s-complement format \citep{yao2022zeroquant,xiao2023smoothquant,dai2021vsq}, where the most significant bit denotes the sign.

\subsubsection{Floating Point Format}
An $n$-bit signed floating point (FP) number $x$ comprises of a 1-bit sign ($x_{\mathrm{sign}}$), $B_m$-bit mantissa ($x_{\mathrm{mant}}$) and $B_e$-bit exponent ($x_{\mathrm{exp}}$) such that $B_m+B_e=n-1$. The associated constant exponent bias ($E_{\mathrm{bias}}$) is computed as $(2^{{B_e}-1}-1)$. We denote this format as $E_{B_e}M_{B_m}$.  

\subsubsection{Quantization Scheme}
\label{subsec:quant_method}
A quantization scheme dictates how a given unquantized tensor is converted to its quantized representation. We consider FP formats for the purpose of illustration. Given an unquantized tensor $\bm{X}$ and an FP format $E_{B_e}M_{B_m}$, we first, we compute the quantization scale factor $s_X$ that maps the maximum absolute value of $\bm{X}$ to the maximum quantization level of the $E_{B_e}M_{B_m}$ format as follows:
\begin{align}
\label{eq:sf}
    s_X = \frac{\mathrm{max}(|\bm{X}|)}{\mathrm{max}(E_{B_e}M_{B_m})}
\end{align}
In the above equation, $|\cdot|$ denotes the absolute value function.

Next, we scale $\bm{X}$ by $s_X$ and quantize it to $\hat{\bm{X}}$ by rounding it to the nearest quantization level of $E_{B_e}M_{B_m}$ as:

\begin{align}
\label{eq:tensor_quant}
    \hat{\bm{X}} = \text{round-to-nearest}\left(\frac{\bm{X}}{s_X}, E_{B_e}M_{B_m}\right)
\end{align}

We perform dynamic max-scaled quantization \citep{wu2020integer}, where the scale factor $s$ for activations is dynamically computed during runtime.

\subsection{Vector Scaled Quantization}
\begin{wrapfigure}{r}{0.35\linewidth}
  \centering
  \includegraphics[width=\linewidth]{sections/figures/vsquant.jpg}
  \caption{\small Vectorwise decomposition for per-vector scaled quantization (VSQ \citep{dai2021vsq}).}
  \label{fig:vsquant}
\end{wrapfigure}
During VSQ \citep{dai2021vsq}, the operand tensors are decomposed into 1D vectors in a hardware friendly manner as shown in Figure \ref{fig:vsquant}. Since the decomposed tensors are used as operands in matrix multiplications during inference, it is beneficial to perform this decomposition along the reduction dimension of the multiplication. The vectorwise quantization is performed similar to tensorwise quantization described in Equations \ref{eq:sf} and \ref{eq:tensor_quant}, where a scale factor $s_v$ is required for each vector $\bm{v}$ that maps the maximum absolute value of that vector to the maximum quantization level. While smaller vector lengths can lead to larger accuracy gains, the associated memory and computational overheads due to the per-vector scale factors increases. To alleviate these overheads, VSQ \citep{dai2021vsq} proposed a second level quantization of the per-vector scale factors to unsigned integers, while MX \citep{rouhani2023shared} quantizes them to integer powers of 2 (denoted as $2^{INT}$).

\subsubsection{MX Format}
The MX format proposed in \citep{rouhani2023microscaling} introduces the concept of sub-block shifting. For every two scalar elements of $b$-bits each, there is a shared exponent bit. The value of this exponent bit is determined through an empirical analysis that targets minimizing quantization MSE. We note that the FP format $E_{1}M_{b}$ is strictly better than MX from an accuracy perspective since it allocates a dedicated exponent bit to each scalar as opposed to sharing it across two scalars. Therefore, we conservatively bound the accuracy of a $b+2$-bit signed MX format with that of a $E_{1}M_{b}$ format in our comparisons. For instance, we use E1M2 format as a proxy for MX4.

\begin{figure}
    \centering
    \includegraphics[width=1\linewidth]{sections//figures/BlockFormats.pdf}
    \caption{\small Comparing LO-BCQ to MX format.}
    \label{fig:block_formats}
\end{figure}

Figure \ref{fig:block_formats} compares our $4$-bit LO-BCQ block format to MX \citep{rouhani2023microscaling}. As shown, both LO-BCQ and MX decompose a given operand tensor into block arrays and each block array into blocks. Similar to MX, we find that per-block quantization ($L_b < L_A$) leads to better accuracy due to increased flexibility. While MX achieves this through per-block $1$-bit micro-scales, we associate a dedicated codebook to each block through a per-block codebook selector. Further, MX quantizes the per-block array scale-factor to E8M0 format without per-tensor scaling. In contrast during LO-BCQ, we find that per-tensor scaling combined with quantization of per-block array scale-factor to E4M3 format results in superior inference accuracy across models. 

\end{document}
