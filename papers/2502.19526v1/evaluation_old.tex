
In this section,
we evaluate the \coam{} algorithm by instantiating it with three oracle
optimizers (\feyntool{}, \quartz{}, and \queso{}), as outlined in \secref{impl}.
%
We denote the resulting optimizers as
\coamwith{\feyntool}, \coamwith{\quartz}, and \coamwith{\queso} respectively,
and describe how these are configured in \secref{methodology}.
%
Our evaluation covers 21 quantum circuits from seven quantum algorithms.
%
We organize the results into the following parts:
%
\begin{itemize}
  \item In \secref{comp-feyn}, we evaluate our \coamwith{\feyntool} optimizer
   on T count reduction.
  %
  We observe that the $\Omega-$optimal circuits produced by our optimizer
  match the output quality of \feyntool{},
  suggesting that $\Omega-$optimality is good optimization criterion for T count optimization.
  %
  Our optimizer runs faster than \feyntool{} in almost all cases
  and we observe that the speedup increases with circuit size.
  %

  \item In \secref{comp-quartz} and \secref{comp-queso},
  we evaluate our optimizers \coamwith{\quartz} and \coamwith{\queso{}}
  for optimizing gate count.
  %
  We observe that the \coam{} algorithm effectively utilizes the base optimizers,
  delivering significant size reduction across a range of quantum circuits.

  \item In \secref{var-segment}, we study the impact of $\Omega$ on the output quality and running time of \coam{}.
\end{itemize}


% the integration of the implementation with \quartz{}, an off-the-shelf optimizer,
% and present an evaluation.
% %
% For the evaluation, we consider the time performance and scalability
% as well as the circuit quality on a range of quantum circuits.
% %
% As our baseline, we consider the \quartz{}~\cite{quartz-2022}
% optimizer, a super optimizer for quantum circuits that uses rule-based
% substitutions to optimize circuits.








%
% We also demonstrate the applicability of our implementation to the
% QUESO optimizer~\cite{queso-2023}, which also uses rule-based optimizations
% with different algorithms and heuristics.
%

%
% Because of the exponential time complexity, both approaches
% deliver poor time scalability, and in practice, provide no quality guarantee.
%
% With the \coam{} algorithm,
% our goal is to deliver a quality guarantee in reasonable running times
% that scale linearly with the size of circuits.
%


\subsection{Methodology}
\section{Research Methodology}~\label{sec:Methodology}

In this section, we discuss the process of conducting our systematic review, e.g., our search strategy for data extraction of relevant studies, based on the guidelines of Kitchenham et al.~\cite{kitchenham2022segress} to conduct SLRs and Petersen et al.~\cite{PETERSEN20151} to conduct systematic mapping studies (SMSs) in Software Engineering. In this systematic review, we divide our work into a four-stage procedure, including planning, conducting, building a taxonomy, and reporting the review, illustrated in Fig.~\ref{fig:search}. The four stages are as follows: (1) the \emph{planning} stage involved identifying research questions (RQs) and specifying the detailed research plan for the study; (2) the \emph{conducting} stage involved analyzing and synthesizing the existing primary studies to answer the research questions; (3) the \emph{taxonomy} stage was introduced to optimize the data extraction results and consolidate a taxonomy schema for REDAST methodology; (4) the \emph{reporting} stage involved the reviewing, concluding and reporting the final result of our study.

\begin{figure}[!t]
    \centering
    \includegraphics[width=1\linewidth]{fig/methodology/searching-process.drawio.pdf}
    \caption{Systematic Literature Review Process}
    \label{fig:search}
\end{figure}

\subsection{Research Questions}
In this study, we developed five research questions (RQs) to identify the input and output, analyze technologies, evaluate metrics, identify challenges, and identify potential opportunities. 

\textbf{RQ1. What are the input configurations, formats, and notations used in the requirements in requirements-driven
automated software testing?} In requirements-driven testing, the input is some form of requirements specification -- which can vary significantly. RQ1 maps the input for REDAST and reports on the comparison among different formats for requirements specification.

\textbf{RQ2. What are the frameworks, tools, processing methods, and transformation techniques used in requirements-driven automated software testing studies?} RQ2 explores the technical solutions from requirements to generated artifacts, e.g., rule-based transformation applying natural language processing (NLP) pipelines and deep learning (DL) techniques, where we additionally discuss the potential intermediate representation and additional input for the transformation process.

\textbf{RQ3. What are the test formats and coverage criteria used in the requirements-driven automated software
testing process?} RQ3 focuses on identifying the formulation of generated artifacts (i.e., the final output). We map the adopted test formats and analyze their characteristics in the REDAST process.

\textbf{RQ4. How do existing studies evaluate the generated test artifacts in the requirements-driven automated software testing process?} RQ4 identifies the evaluation datasets, metrics, and case study methodologies in the selected papers. This aims to understand how researchers assess the effectiveness, accuracy, and practical applicability of the generated test artifacts.

\textbf{RQ5. What are the limitations and challenges of existing requirements-driven automated software testing methods in the current era?} RQ5 addresses the limitations and challenges of existing studies while exploring future directions in the current era of technology development. %It particularly highlights the potential benefits of advanced LLMs and examines their capacity to meet the high expectations placed on these cutting-edge language modeling technologies. %\textcolor{blue}{CA: Do we really need to focus on LLMs? TBD.} \textcolor{orange}{FW: About LLMs, I removed the direct emphase in RQ5 but kept the discussion in RQ5 and the solution section. I think that would be more appropriate.}

\subsection{Searching Strategy}

The overview of the search process is exhibited in Fig. \ref{fig:papers}, which includes all the details of our search steps.
\begin{table}[!ht]
\caption{List of Search Terms}
\label{table:search_term}
\begin{tabularx}{\textwidth}{lX}
\hline
\textbf{Terms Group} & \textbf{Terms} \\ \hline
Test Group & test* \\
Requirement Group & requirement* OR use case* OR user stor* OR specification* \\
Software Group & software* OR system* \\
Method Group & generat* OR deriv* OR map* OR creat* OR extract* OR design* OR priorit* OR construct* OR transform* \\ \hline
\end{tabularx}
\end{table}

\begin{figure}
    \centering
    \includegraphics[width=1\linewidth]{fig/methodology/search-papers.drawio.pdf}
    \caption{Study Search Process}
    \label{fig:papers}
\end{figure}

\subsubsection{Search String Formulation}
Our research questions (RQs) guided the identification of the main search terms. We designed our search string with generic keywords to avoid missing out on any related papers, where four groups of search terms are included, namely ``test group'', ``requirement group'', ``software group'', and ``method group''. In order to capture all the expressions of the search terms, we use wildcards to match the appendix of the word, e.g., ``test*'' can capture ``testing'', ``tests'' and so on. The search terms are listed in Table~\ref{table:search_term}, decided after iterative discussion and refinement among all the authors. As a result, we finally formed the search string as follows:


\hangindent=1.5em
 \textbf{ON ABSTRACT} ((``test*'') \textbf{AND} (``requirement*'' \textbf{OR} ``use case*'' \textbf{OR} ``user stor*'' \textbf{OR} ``specifications'') \textbf{AND} (``software*'' \textbf{OR} ``system*'') \textbf{AND} (``generat*'' \textbf{OR} ``deriv*'' \textbf{OR} ``map*'' \textbf{OR} ``creat*'' \textbf{OR} ``extract*'' \textbf{OR} ``design*'' \textbf{OR} ``priorit*'' \textbf{OR} ``construct*'' \textbf{OR} ``transform*''))

The search process was conducted in September 2024, and therefore, the search results reflect studies available up to that date. We conducted the search process on six online databases: IEEE Xplore, ACM Digital Library, Wiley, Scopus, Web of Science, and Science Direct. However, some databases were incompatible with our default search string in the following situations: (1) unsupported for searching within abstract, such as Scopus, and (2) limited search terms, such as ScienceDirect. Here, for (1) situation, we searched within the title, keyword, and abstract, and for (2) situation, we separately executed the search and removed the duplicate papers in the merging process. 

\subsubsection{Automated Searching and Duplicate Removal}
We used advanced search to execute our search string within our selected databases, following our designed selection criteria in Table \ref{table:selection}. The first search returned 27,333 papers. Specifically for the duplicate removal, we used a Python script to remove (1) overlapped search results among multiple databases and (2) conference or workshop papers, also found with the same title and authors in the other journals. After duplicate removal, we obtained 21,652 papers for further filtering.

\begin{table*}[]
\caption{Selection Criteria}
\label{table:selection}
\begin{tabularx}{\textwidth}{lX}
\hline
\textbf{Criterion ID} & \textbf{Criterion Description} \\ \hline
S01          & Papers written in English. \\
S02-1        & Papers in the subjects of "Computer Science" or "Software Engineering". \\
S02-2        & Papers published on software testing-related issues. \\
S03          & Papers published from 1991 to the present. \\ 
S04          & Papers with accessible full text. \\ \hline
\end{tabularx}
\end{table*}

\begin{table*}[]
\small
\caption{Inclusion and Exclusion Criteria}
\label{table:criteria}
\begin{tabularx}{\textwidth}{lX}
\hline
\textbf{ID}  & \textbf{Description} \\ \hline
\multicolumn{2}{l}{\textbf{Inclusion Criteria}} \\ \hline
I01 & Papers about requirements-driven automated system testing or acceptance testing generation, or studies that generate system-testing-related artifacts. \\
I02 & Peer-reviewed studies that have been used in academia with references from literature. \\ \hline
\multicolumn{2}{l}{\textbf{Exclusion Criteria}} \\ \hline
E01 & Studies that only support automated code generation, but not test-artifact generation. \\
E02 & Studies that do not use requirements-related information as an input. \\
E03 & Papers with fewer than 5 pages (1-4 pages). \\
E04 & Non-primary studies (secondary or tertiary studies). \\
E05 & Vision papers and grey literature (unpublished work), books (chapters), posters, discussions, opinions, keynotes, magazine articles, experience, and comparison papers. \\ \hline
\end{tabularx}
\end{table*}

\subsubsection{Filtering Process}

In this step, we filtered a total of 21,652 papers using the inclusion and exclusion criteria outlined in Table \ref{table:criteria}. This process was primarily carried out by the first and second authors. Our criteria are structured at different levels, facilitating a multi-step filtering process. This approach involves applying various criteria in three distinct phases. We employed a cross-verification method involving (1) the first and second authors and (2) the other authors. Initially, the filtering was conducted separately by the first and second authors. After cross-verifying their results, the results were then reviewed and discussed further by the other authors for final decision-making. We widely adopted this verification strategy within the filtering stages. During the filtering process, we managed our paper list using a BibTeX file and categorized the papers with color-coding through BibTeX management software\footnote{\url{https://bibdesk.sourceforge.io/}}, i.e., “red” for irrelevant papers, “yellow” for potentially relevant papers, and “blue” for relevant papers. This color-coding system facilitated the organization and review of papers according to their relevance.

The screening process is shown below,
\begin{itemize}
    \item \textbf{1st-round Filtering} was based on the title and abstract, using the criteria I01 and E01. At this stage, the number of papers was reduced from 21,652 to 9,071.
    \item \textbf{2nd-round Filtering}. We attempted to include requirements-related papers based on E02 on the title and abstract level, which resulted from 9,071 to 4,071 papers. We excluded all the papers that did not focus on requirements-related information as an input or only mentioned the term ``requirements'' but did not refer to the requirements specification.
    \item \textbf{3rd-round Filtering}. We selectively reviewed the content of papers identified as potentially relevant to requirements-driven automated test generation. This process resulted in 162 papers for further analysis.
\end{itemize}
Note that, especially for third-round filtering, we aimed to include as many relevant papers as possible, even borderline cases, according to our criteria. The results were then discussed iteratively among all the authors to reach a consensus.

\subsubsection{Snowballing}

Snowballing is necessary for identifying papers that may have been missed during the automated search. Following the guidelines by Wohlin~\cite{wohlin2014guidelines}, we conducted both forward and backward snowballing. As a result, we identified 24 additional papers through this process.

\subsubsection{Data Extraction}

Based on the formulated research questions (RQs), we designed 38 data extraction questions\footnote{\url{https://drive.google.com/file/d/1yjy-59Juu9L3WHaOPu-XQo-j-HHGTbx_/view?usp=sharing}} and created a Google Form to collect the required information from the relevant papers. The questions included 30 short-answer questions, six checkbox questions, and two selection questions. The data extraction was organized into five sections: (1) basic information: fundamental details such as title, author, venue, etc.; (2) open information: insights on motivation, limitations, challenges, etc.; (3) requirements: requirements format, notation, and related aspects; (4) methodology: details, including immediate representation and technique support; (5) test-related information: test format(s), coverage, and related elements. Similar to the filtering process, the first and second authors conducted the data extraction and then forwarded the results to the other authors to initiate the review meeting.

\subsubsection{Quality Assessment}

During the data extraction process, we encountered papers with insufficient information. To address this, we conducted a quality assessment in parallel to ensure the relevance of the papers to our objectives. This approach, also adopted in previous secondary studies~\cite{shamsujjoha2021developing, naveed2024model}, involved designing a set of assessment questions based on guidelines by Kitchenham et al.~\cite{kitchenham2022segress}. The quality assessment questions in our study are shown below:
\begin{itemize}
    \item \textbf{QA1}. Does this study clearly state \emph{how} requirements drive automated test generation?
    \item \textbf{QA2}. Does this study clearly state the \emph{aim} of REDAST?
    \item \textbf{QA3}. Does this study enable \emph{automation} in test generation?
    \item \textbf{QA4}. Does this study demonstrate the usability of the method from the perspective of methodology explanation, discussion, case examples, and experiments?
\end{itemize}
QA4 originates from an open perspective in the review process, where we focused on evaluation, discussion, and explanation. Our review also examined the study’s overall structure, including the methodology description, case studies, experiments, and analyses. The detailed results of the quality assessment are provided in the Appendix. Following this assessment, the final data extraction was based on 156 papers.

% \begin{table}[]
% \begin{tabular}{ll}
% \hline
% QA ID & QA Questions                                             \\ \hline
% Q01   & Does this study clearly state its aims?                  \\
% Q02   & Does this study clearly describe its methodology?        \\
% Q03   & Does this study involve automated test generation?       \\
% Q04   & Does this study include a promising evaluation?          \\
% Q05   & Does this study demonstrate the usability of the method? \\ \hline
% \end{tabular}%
% \caption{Questions for Quality Assessment}
% \label{table:qa}
% \end{table}

% automated quality assessment

% \textcolor{blue}{CA: Our search strategy focused on identifying requirements types first. We covered several sources, e.g., ~\cite{Pohl:11,wagner2019status} to identify different formats and notations of specifying requirements. However, this came out to be a long list, e.g., free-form NL requirements, semi-formal UML models, free-from textual use case models, UML class diagrams, UML activity diagrams, and so on. In this paper, we attempted to primarily focus on requirements-related aspects and not design-level information. Hence, we generalised our search string to include generic keywords, e.g., requirement*, use case*, and user stor*. We did so to avoid missing out on any papers, bringing too restrictive in our search strategy, and not creating a too-generic search string with all the aforementioned formats to avoid getting results beyond our review's scope.}


%% Use \subsection commands to start a subsection.



%\subsection{Study Selection}

% In this step, we further looked into the content of searched papers using our search strategy and applied our inclusion and exclusion criteria. Our filtering strategy aimed to pinpoint studies focused on requirements-driven system-level testing. Recognizing the presence of irrelevant papers in our search results, we established detailed selection criteria for preliminary inclusion and exclusion, as shown in Table \ref{table: criteria}. Specifically, we further developed the taxonomy schema to exclude two types of studies that did not meet the requirements for system-level testing: (1) studies supporting specification-driven test generation, such as UML-driven test generation, rather than requirements-driven testing, and (2) studies focusing on code-based test generation, such as requirement-driven code generation for unit testing.





\subsection{Benchmarks}
% \myparagraph{Benchmarks.}
% \begin{table*}[!ht]
%     \centering
%     \footnotesize
%     \begin{tabular}{|l|l|l|l|l|l|l|l|}
%     \hline
%         \toprule
%         ~ & \textbf{MMLU} & ~ & ~ & ~ & ~ & ~ \\ 
%         \midrule
%         ~ & Accuracy & BLEU & ROUGE-1 & ROUGE-2 & ROUGE-L & MC1 acc & MC2 acc \\ 
%         \midrule
%         10Unknown\_1HighKnown\_PICKED & $0.627 \pm 0.004$ & 13.25051502 & 27.47677852 & 15.0779683 & 25.53029103 & 0.3096695226 & 0.4507741543 \\ 
%         50Unknown\_0HighKnown\_PICKED & $0.665 \pm 0.004$ & 19.0512323 & 37.75598785 & 22.52274519 & 35.68320508 & 0.2986536108 & 0.4689649774 \\
%         10Unknown\_0HighKnown\_PICKED & $0.615 \pm 0.004$ & 25.95476016 & 44.94093219 & 31.28726836 & 42.88254662 & 0.3341493268 & 0.5188523609 \\
%         10Unknown\_10HighKnown\_PICKED & $0.653 \pm 0.004$ & 21.65900482 & 39.45879414 & 25.67806412 & 37.56174108 & 0.3194614443 & 0.4900582757 \\
%         1Unknown\_0HighKnown\_PICKED & $0.682 \pm 0.00$4 & 33.74117912 & 57.41446911 & 44.17629878 & 55.37393774 & 0.358629131 & 0.5326986564 \\
%         1HighKnown\_PICKED & $0.681 \pm 0.004$ & 34.9204433 & 58.53694001 & 45.81363154 & 56.58717976 & 0.3610771114 & 0.5377494863 \\
%         1HighKnown\_PICKED & $0.675 \pm 0.004$ & 18.73964867 & 35.43567616 & 19.31790608 & 33.39819738 & 0.3133414933 & 0.4943084868 \\
%         50Unknown\_10HighKnown\_PICKED & $0.658 \pm 0.004$ & 5.591687963 & 20.85892719 & 9.419648658 & 19.32072844 & 0.2815177479 & 0.4476934191 \\
%         100Unknown\_0HighKnown\_PICKED & $0.680 \pm 0.004$ & 23.97587607 & 44.2921212 & 29.36292281 & 42.43869823 & 0.3084455324 & 0.4877136975 \\
%         100Unknown\_10HighKnown\_PICKED & $0.659 \pm 0.004$ & 9.609306809 & 27.23633087 & 13.82936369 & 25.36313833 & 0.276621787 & 0.4519778054 \\
%         100Unknown\_1HighKnown\_PICKED & $0.669 \pm 0.004$ & 19.58596787 & 37.38609505 & 22.37488796 & 35.50029635 & 0.3145654835 & 0.4816116655 \\
%         500Unknown\_10HighKnown\_PICKED & $0.554 \pm 0.004$ & 6.829454848 & 23.14328237 & 9.583194327 & 21.51946419 & 0.2962056304 & 0.4628290061 \\
%         \_500Unknown\_1HighKnown\_PICKED & $0.656 \pm 0.004$ & 7.561402326 & 23.50729327 & 11.47986684 & 21.49804031 & 0.2974296206 & 0.4600713539 \\
%         1Unknown\_10HighKnown\_PICKED & $0.672 \pm 0.004$ & 25.36681999 & 44.60449295 & 30.87260675 & 42.81648635 & 0.3219094247 & 0.478369048 \\
%         Paraphrase\_100Unknown\_1Paraphrase\_PICKED & $0.675 \pm 0.004$ & 19.88199366 & 40.46510402 & 24.31557217 & 38.29900672 & 0.3341493268 & 0.50669042 \\
%         Paraphrase\_50Unknown\_10Paraphrase\_PICKED & $0.668 \pm 0.004$ & 17.94438863 & 38.24036462 & 22.34047459 & 35.89121668 & 0.3365973072 & 0.511407264 \\
%         Paraphrase\_50Unknown\_1Paraphrase\_PICKED & $0.670 \pm 0.004$ & 17.79545025 & 37.68995649 & 21.15895968 & 35.88692479 & 0.3439412485 & 0.5171152888 \\
%         Paraphrase\_1Unknown\_1Paraphrase\_PICKED & $0.681 \pm 0.004$ & 36.58151713 & 60.9910189 & 48.64170617 & 59.05187417 & 0.364749082 & 0.537224633 \\
%         Paraphrase\_1Unknown\_10Paraphrase\_PICKED & $0.674 \pm 0.004$ & 35.92981657 & 59.60558781 & 46.78702814 & 57.85113211 & 0.3561811506 & 0.5351302664 \\
%         Paraphrase\_3000Unknown\_1Paraphrase\_PICKED & $0.605 \pm 0.004$ & 17.10736069 & 38.45724383 & 22.22177594 & 35.29996535 & 0.3194614443 & 0.5020119264 \\
%         Paraphrase\_100Unknown\_10Paraphrase\_PICKED & $0.659 \pm 0.004$ & 25.62098387 & 47.62762837 & 32.7874269 & 45.44462211 & 0.3365973072 & 0.511102884 \\
%         Paraphrase\_100Unknown\_0Paraphrase\_PICKED & $0.663 \pm 0.004$ & 16.98679726 & 34.88412138 & 19.59767718 & 32.93491985 & 0.3035495716 & 0.4741460343 \\
%         Paraphrase\_50Unknown\_0Paraphrase\_PICKED & $0.660 \pm 0.004$ & 18.90591627 & 34.49087425 & 19.25947105 & 32.56911103 & 0.305997552 & 0.48189135 \\
%         Paraphrase\_10Unknown\_10Paraphrase\_PICKED & $0.624 \pm 0.004$ & 15.42020473 & 31.98426022 & 20.04590726 & 29.98744658 & 0.3317013464 & 0.4945909224 \\
%         Paraphrase\_10Unknown\_1Paraphrase\_PICKED & $0.610 \pm 0.004$ & 18.51838983 & 36.57769855 & 23.06143581 & 34.48129151 & 0.3427172583 & 0.5272128942 \\
%         Paraphrase\_500Unknown\_0Paraphrase\_PICKED & $0.649 \pm 0.004$ & 10.11452461 & 25.99101141 & 12.54072887 & 23.90879425 & 0.2900856793 & 0.4468474778 \\
%         Paraphrase\_3000Unknown\_0Paraphrase\_PICKED & $0.579 \pm 0.004$ & 11.41536861 & 27.78252793 & 14.88395289 & 25.36251311 & 0.2937576499 & 0.4609358923 \\
%         Paraphrase\_10Unknown\_0Paraphrase\_PICKED &$ 0.615 \pm 0.004$ & 12.52443697 & 22.77811545 & 11.98668672 & 21.51925287 & 0.3047735618 & 0.4809630092 \\
%         Paraphrase\_500Unknown\_10Paraphrase\_PICKED & $0.566 \pm 0.004$ & 10.51367787 & 26.6042406 & 12.96091511 & 24.3373379 & 0.3427172583 & 0.5119113997 \\
%         Paraphrase\_3000Unknown\_10Paraphrase\_PICKED & $0.441 \pm 0.004$ & 29.35197433 & 50.87092995 & 40.38334757 & 49.40042213 & 0.3047735618 & 0.4912964837 \\
%         \bottomrule
%     \end{tabular}
% \end{table*}

% \begin{table*}[!ht]
%     \centering
%     \footnotesize
%     \begin{tabular}{l|r|r|r|r|r|r|r}
%     \hline
%         \toprule
%         ~ & \textbf{MMLU} & ~ & ~ & ~ & ~ & ~ \\ 
%         \midrule
%         ~ & Accuracy & BLEU & ROUGE-1 & ROUGE-2 & ROUGE-L & MC1 acc & MC2 acc \\ 
%         \midrule
%         LLama 8B 3.1  & $0.677 \pm 0.004$ & 35.780 & 60.325 & 47.789 & 58.512 & 0.370 & 0.540 \\ 
%         \midrule
%         10Unknown\_1HighKnown & $0.627 \pm 0.004$ & 13.251 & 27.477 & 15.078 & 25.530 & 0.310 & 0.451 \\ 
        
%         50Unknown\_0HighKnown & $0.665 \pm 0.004$ & 19.051 & 37.756 & 22.523 & 35.683 & 0.299 & 0.469 \\
        
%         10Unknown\_0HighKnown & $0.615 \pm 0.004$ & 25.955 & 44.941 & 31.287 & 42.883 & 0.334 & 0.519 \\
        
%         10Unknown\_10HighKnown & $0.653 \pm 0.004$ & 21.659 & 39.459 & 25.678 & 37.562 & 0.320 & 0.490 \\
        
%         1Unknown\_0HighKnown & $0.682 \pm 0.004$ & 33.741 & 57.414 & 44.176 & 55.374 & 0.359 & 0.533 \\
        
%         1Unknown\_1HighKnown & $0.681 \pm 0.004$ & 34.920 & 58.537 & 45.814 & 56.587 & 0.361 & 0.538 \\
        
%         50Unknown\_1HighKnown & $0.675 \pm 0.004$ & 18.740 & 35.436 & 19.318 & 33.398 & 0.313 & 0.494 \\
        
%         50Unknown\_10HighKnown & $0.658 \pm 0.004$ & 5.592 & 20.859 & 9.420 & 19.321 & 0.282 & 0.448 \\
        
%         100Unknown\_0HighKnown & $0.680 \pm 0.004$ & 23.976 & 44.292 & 29.363 & 42.439 & 0.308 & 0.488 \\
        
%         100Unknown\_10HighKnown & $0.659 \pm 0.004$ & 9.609 & 27.236 & 13.829 & 25.363 & 0.277 & 0.452 \\
        
%         100Unknown\_1HighKnown & $0.669 \pm 0.004$ & 19.586 & 37.386 & 22.375 & 35.500 & 0.315 & 0.482 \\
        
%         500Unknown\_10HighKnown & $0.554 \pm 0.004$ & 6.829 & 23.143 & 9.583 & 21.519 & 0.296 & 0.463 \\
        
%         \_500Unknown\_1HighKnown & $0.656 \pm 0.004$ & 7.561 & 23.507 & 11.480 & 21.498 & 0.297 & 0.460 \\
        
%         1Unknown\_10HighKnown & $0.672 \pm 0.004$ & 25.367 & 44.604 & 30.873 & 42.816 & 0.322 & 0.478 \\
        
%         100Unknown\_1Paraphrase & $0.675 \pm 0.004$ & 19.882 & 40.465 & 24.316 & 38.299 & 0.334 & 0.507 \\
        
%         50Unknown\_10Paraphrase & $0.668 \pm 0.004$ & 17.944 & 38.240 & 22.340 & 35.891 & 0.337 & 0.511 \\
        
%         50Unknown\_1Paraphrase & $0.670 \pm 0.004$ & 17.795 & 37.690 & 21.159 & 35.887 & 0.344 & 0.517 \\
        
%         1Unknown\_1Paraphrase & $0.681 \pm 0.004$ & 36.582 & 60.991 & 48.642 & 59.052 & 0.365 & 0.537 \\
        
%         1Unknown\_10Paraphrase & $0.674 \pm 0.004$ & 35.930 & 59.606 & 46.787 & 57.851 & 0.356 & 0.535 \\
        
%         3000Unknown\_1Paraphrase & $0.605 \pm 0.004$ & 17.107 & 38.457 & 22.221 & 35.300 & 0.319 & 0.502 \\
        
%         100Unknown\_10Paraphrase & $0.659 \pm 0.004$ & 25.621 & 47.628 & 32.787 & 45.445 & 0.337 & 0.511 \\
        
%         100Unknown\_0Paraphrase & $0.663 \pm 0.004$ & 16.987 & 34.884 & 19.598 & 32.935 & 0.304 & 0.474 \\
        
%         50Unknown\_0Paraphrase & $0.660 \pm 0.004$ & 18.906 & 34.491 & 19.260 & 32.569 & 0.306 & 0.482 \\
        
%         10Unknown\_10Paraphrase & $0.624 \pm 0.004$ & 15.420 & 31.984 & 20.046 & 29.987 & 0.332 & 0.495 \\
        
%         10Unknown\_1Paraphrase & $0.610 \pm 0.004$ & 18.518 & 36.578 & 23.061 & 34.481 & 0.343 & 0.527 \\
        
%         500Unknown\_0Paraphrase & $0.649 \pm 0.004$ & 10.115 & 25.991 & 12.541 & 23.909 & 0.291 & 0.447 \\
        
%         3000Unknown\_0Paraphrase & $0.579 \pm 0.004$ & 11.415 & 27.783 & 14.884 & 25.363 & 0.294 & 0.461 \\
        
%         10Unknown\_0Paraphrase &$ 0.615 \pm 0.004$ & 12.524 & 22.778 & 11.987 & 21.519 & 0.305 & 0.481\\
        
%         500Unknown\_10Paraphrase & $0.566 \pm 0.004$ & 10.514 & 26.604 & 12.961 & 24.337 & 0.343 & 0.512 \\
        
%         3000Unknown\_10Paraphrase & $0.441 \pm 0.004$ & 29.352 & 50.871 & 40.383 & 49.400 & 0.305 & 0.491\\
%         \bottomrule
%     \end{tabular}
% \label{benchmarks}

% \end{table*}

\begin{table*}[!ht]
    \centering
    \footnotesize
    \setlength{\tabcolsep}{2pt}
\begin{tabular}{l|cc|cccccc}
\toprule
                          & \multicolumn{2}{c|}{{\underline{ MMLU}}}                                            & \multicolumn{6}{c}{{\underline{ ThruthfulQA}}}                                                                                                                                                                                                \\ \midrule
                          & \multicolumn{1}{l}{\textbf{Accuracy}} & \multicolumn{1}{l|}{\textbf{Std}} & \multicolumn{1}{l}{\textbf{BLEU}} & \multicolumn{1}{l}{\textbf{ROUGE-1}} & \multicolumn{1}{l}{\textbf{ROUGE-2}} & \multicolumn{1}{l}{\textbf{ROUGE-L}} & \multicolumn{1}{l}{\textbf{MC1 Acc.}} & \multicolumn{1}{l}{\textbf{MC2 Acc.}} \\ \midrule
Llama-3.1-8B-Instruct     & 0.677                                 & 0.004                                & 35.780                            & 60.325                               & 47.789                               & 58.512                               & 0.382                                & 0.552                                \\ \midrule
\hspace{2.5ex} 1 UK\hspace{1.3ex} + \hspace{3.5ex} 0 HK       & 0.682                                 & 0.004                                & 33.741                            & 57.414                               & 44.176                               & 55.374                               & 0.359                                & 0.533                                \\
\hspace{9.3ex} + \hspace{3.5ex} 1 HK       & 0.681                                 & 0.004                                & 34.920                            & 58.537                               & 45.814                               & 56.587                               & 0.361                                & 0.538                                \\
 \hspace{9.3ex} + \hspace{2.5ex} 10 HK    & 0.672                                 & 0.004                                & 25.367                            & 44.604                               & 30.873                               & 42.816                               & 0.322                                & 0.478                                \\
\hspace{1.5ex} 10 UK\hspace{1.2ex} + \hspace{3.5ex} 0 HK       & 0.615                                 & 0.004                                & 12.524                            & 22.778                               & 11.987                               & 21.519                              & 0.305                              & 0.481                              \\
\hspace{9.3ex} + \hspace{3.5ex} 1 HK      & 0.627                                 & 0.004                                & 13.251                            & 27.477                               & 15.078                               & 25.530                               & 0.310                                & 0.451                                \\
 \hspace{9.3ex} + \hspace{2.5ex} 10 HK     & 0.653                                 & 0.004                                & 21.659                            & 39.459                               & 25.678                               & 37.562                               & 0.319                                & 0.490                                \\
\hspace{1.5ex} 50 UK\hspace{1.2ex} + \hspace{3.5ex} 0 HK     & 0.660                              & 0.004                                & 18.906                           & 34.491                            & 19.259                               & 32.569                               & 0.306                               & 0.482                                \\
\hspace{9.3ex} + \hspace{2.5ex} 1 HK     & 0.675                                 & 0.004                                & 18.740                            & 35.436                               & 19.318                               & 33.398                               & 0.313                                & 0.494                                \\
 \hspace{9.3ex} + \hspace{2.5ex} 10 HK   & 0.658                                 & 0.004                                & 5.592                             & 20.859                               & 9.420                                & 19.321                               & 0.282                                & 0.448                                \\
\hspace{1.1ex} 100 UK\hspace{0.5ex} + \hspace{3.5ex} 0 HK     & 0.663                                & 0.004                                & 16.987                         & 34.884                            & 19.598                               & 32.935                               & 0.304                                & 0.474                             \\
\hspace{9.3ex} + \hspace{3.5ex} 1 HK    & 0.669                                 & 0.004                                & 19.586                            & 37.386                               & 22.375                               & 35.500                               & 0.315                                & 0.482                                \\
 \hspace{9.3ex} + \hspace{2.5ex} 10 HK   & 0.659                                 & 0.004                                & 9.609                             & 27.236                               & 13.829                               & 25.363                               & 0.277                                & 0.452                                \\
\hspace{1.1ex} 500 UK\hspace{0.5ex} + \hspace{3.5ex} 0 HK      &   0.649                                    &                            0.004          &     10.115                              &      25.991                                &      12.541                                &                          23.909            &  0.290                                    &    0.447                                  \\
\hspace{9.3ex} + \hspace{3.5ex} 1 HK     & 0.655                                 & 0.004                                & 7.561                             & 23.507                               & 11.480                               & 21.498                               & 0.297                                & 0.460                                \\
 \hspace{9.3ex} + \hspace{2.5ex} 10 HK  & 0.554                                 & 0.004                                & 6.829                             & 23.143                               & 9.583                                & 21.519                               & 0.296                                & 0.463                                \\
3000 UK \hspace{0.5ex} + \hspace{3.5ex} 0 HK   &                           0.579            &   0.004                                   &    11.415                               &        27.783                              &                        14.884              &          25.363                            &     0.294                                 &   0.461                                   \\
\hspace{9.3ex} + \hspace{3.5ex} 1 HK   &  0.623                                     &          0.004                            &      5.561                             &                       19.906               &    7.422                                  &    18.280                                  &         0.257                             &                     0.420                 \\
 \hspace{9.3ex} + \hspace{2.5ex} 10 HK   &   0.554 &         0.004    &          9.239  &          23.447     &           11.558     &    21.415      &         0.263      &      0.445     \\ \midrule
\hspace{2.5ex} 1 UK\hspace{1.3ex} + \hspace{1.5ex} 0 Paraphrase      &  0.682              &  0.004             &           33.741   &      57.414          &   44.176        &          55.374       &       0.359          &              0.533   \\
 \hspace{9.3ex} + \hspace{1.5ex} 1 Paraphrase    & 0.681                                 & 0.004                                & 36.582                            & 60.991                               & 48.642                               & 59.052                               & 0.365                                & 0.537                                \\
 \hspace{9.3ex} + \hspace{0.5ex} 10 Paraphrase    & 0.674                                 & 0.004                                & 35.930                            & 59.606                               & 46.787                               & 57.851                               & 0.356                                & 0.535                                \\
\hspace{1.5ex} 10 UK\hspace{1.2ex} + \hspace{1.5ex} 0 Paraphrase    & 0.615                                 & 0.004                                & 12.524                            & 22.778                               & 11.987                               & 21.519                               & 0.305                                & 0.481                                \\
 \hspace{9.3ex} + \hspace{1.5ex} 1 Paraphrase     & 0.610                                 & 0.004                                & 18.518                            & 36.578                               & 23.061                               & 34.481                               & 0.343                                & 0.527                                \\
 \hspace{9.3ex} + \hspace{0.5ex} 10 Paraphrase  & 0.624                                 & 0.004                                & 15.420                            & 31.984                               & 20.046                               & 29.987                               & 0.332                                & 0.495                                \\
\hspace{1.5ex} 50 UK\hspace{1.2ex} + \hspace{1.5ex} 0 Paraphrase    & 0.660                                 & 0.004                                & 18.906                            & 34.491                               & 19.259                               & 32.569                               & 0.306                                & 0.482                                \\
 \hspace{9.3ex} + \hspace{1.5ex} 1 Paraphrase   & 0.670                                 & 0.004                                & 17.795                            & 37.690                               & 21.159                               & 35.887                               & 0.344                                & 0.517                                \\
 \hspace{9.3ex} + \hspace{0.5ex} 10 Paraphrase  & 0.668                                 & 0.004                                & 17.944                            & 38.240                               & 22.340                               & 35.891                               & 0.337                                & 0.511                                \\
\hspace{0.5ex} 100 UK\hspace{1.2ex}  + \hspace{1.5ex} 0 Paraphrase    & 0.663                                 & 0.004                                & 16.987                            & 34.884                               & 19.598                               & 32.935                               & 0.304                                & 0.474                                \\
 \hspace{9.3ex} + \hspace{1.5ex} 1 Paraphrase  & 0.675                                 & 0.004                                & 19.882                            & 40.465                               & 24.316                               & 38.299                               & 0.334                                & 0.507                                \\
 \hspace{9.3ex} + \hspace{0.5ex} 10 Paraphrase  & 0.659                                 & 0.004                                & 25.621                            & 47.628                               & 32.787                               & 45.445                               & 0.337                                & 0.511                                \\
\hspace{0.5ex} 500 UK\hspace{1.2ex} + \hspace{1.5ex} 0 Paraphrase    & 0.649                                 & 0.004                                & 10.115                            & 25.991                               & 12.541                               & 23.909                               & 0.290                                & 0.447                                \\
 \hspace{9.3ex} + \hspace{1.5ex} 1 Paraphrase  &  0.653                                     &      0.004                                &      14.245                             &         33.977                             &                            17.747          &     31.160                                 &      0.322                                &       0.472                               \\
 \hspace{9.3ex} + \hspace{0.5ex} 10 Paraphrase  & 0.566                                 & 0.004                                & 10.514                            & 26.604                               & 12.961                               & 24.337                               & 0.343                                & 0.512                                \\
3000 UK \hspace{0.5ex} + \hspace{1.5ex} 0 Paraphrase  & 0.579                                 & 0.004                                & 11.415                            & 27.783                               & 14.884                               & 25.363                               & 0.294                                & 0.461                                \\
\hspace{9.3ex} + \hspace{1.5ex} 1 Paraphrase  & 0.605                                 & 0.004                                & 17.107                            & 38.457                               & 22.222                               & 35.300                               & 0.319                                & 0.502                                \\
 \hspace{9.3ex} + \hspace{0.5ex} 10 Paraphrase & 0.441                                 & 0.004                                & 29.352                            & 50.871                               & 40.383                               & 49.400                               & 0.305                                & 0.491                                \\ \bottomrule
\end{tabular}                   
\caption{Accuracy for MMLU and a range of metrics  for ThruthfulQA for all trained LoRA adapters}
\label{table:benchmarks}
\end{table*}

\subsection{\coam{} with the \feyntool{} optimizer}
\label{sec:comp-feyn}
In this section,
we study the performance of our \coamwith{\feyntool} optimizer,
which runs the \coam{} algorithm with \feyntool{} as the oracle optimizer.
%
We evaluate the running time and output quality of our optimizer
by running the baseline \feyntool{}.
%
For all our benchmarks,
we set the segment depth $\Omega$ to be $120$.
%
The output of our \coamwith{\feyntool{}} is $\Omega-$optimal with $\Omega = 120$
and the cost function is the T count of the circuit.
%

\begin{figure}
  \centering\small
  \begin{tabular}{cccccccccc}
  & &  & \multicolumn{2}{c}{Time} & &  \multicolumn{2}{c}{T Count Reduction}  \\ \cmidrule(lr){4-5} \cmidrule(lr){7-8}
   Family & Qubits & Input T Count & \feyntool{} & \algname{} & \algname{} speedup & \feyntool{} & \algname{} \\


  \midrule\multirow{2}{*}{bwt}    & 17 & 169330 & 35730.0      & \textbf{354.8}  & 100.69 & -13.2\% & -13.2\% \\
                          & 21 & 214585 & 68301.4      & \textbf{569.1}  & 120.01 & -19.5\% & -19.4\% \\
  \midrule\multirow{4}{*}{grover} &  9 &   3927 & \textbf{2.6} & 3.3             &   0.79 & -31.2\% & -31.2\% \\
                          & 11 &  13720 & 12.0         & \textbf{10.5}   &   1.15 & -33.7\% & -33.7\% \\
                          & 13 &  36920 & 140.9        & \textbf{34.8}   &   4.05 & -33.1\% & -33.1\% \\
                          & 15 &  92016 & 2609.6       & \textbf{104.5}  &  24.97 & -32.7\% & -32.7\% \\
  \midrule\multirow{2}{*}{hhl}    &  7 &  61246 & 409.8        & \textbf{31.4}   &  13.03 & -31.2\% & -31.2\% \\
                          &  9 & 565183 & T.O.         & \textbf{535.4}  &  > 80.69 & T.O.    & -34.1\% \\
  \midrule\multirow{4}{*}{hwb}    &  8 &   5887 & \textbf{4.1} & 5.9             &   0.69 & -25.7\% & -25.7\% \\
                          & 10 &  29939 & 250.8        & \textbf{45.8}   &   5.48 & -29.8\% & -29.8\% \\
                          & 11 &  84196 & 4767.3       & \textbf{129.9}  &  36.7  & -31.3\% & -31.3\% \\
                          & 12 & 171465 & 21880.4      & \textbf{362.6}  &  60.35 & -34.1\% & -34.1\% \\
  \midrule\multirow{4}{*}{qft}    & 48 &  44803 & 195.1        & \textbf{77.1}   &   2.53 & -20.2\% & -20.2\% \\
                          & 64 &  61027 & 531.4        & \textbf{138.8}  &   3.83 & -20.3\% & -20.3\% \\
                          & 80 &  77251 & 1083.4       & \textbf{204.4}  &   5.3  & -20.3\% & -20.3\% \\
                          & 96 &  93475 & 1931.9       & \textbf{355.6}  &   5.43 & -20.3\% & -20.3\% \\
  \midrule\multirow{4}{*}{shor}   & 10 &   6104 & \textbf{2.0} & 4.0             &   0.51 & -19.7\% & -19.7\% \\
                          & 12 &  20180 & 21.0         & \textbf{16.5}   &   1.27 & -20.3\% & -20.3\% \\
                          & 14 &  70544 & 999.5        & \textbf{76.9}   &  12.99 & -20.5\% & -20.5\% \\
                          & 16 & 266060 & 28382.3      & \textbf{396.8}  &  71.53 & -20.6\% & -20.6\% \\
  \midrule\multirow{4}{*}{sqrt} & 42 &  25104 & 569.2        & \textbf{69.3}   &   8.21 & -37.4\% & -37.4\% \\
                          & 48 &  60366 & 5441.6       & \textbf{189.8}  &  28.67 & -39.9\% & -39.9\% \\
                          & 54 & 140830 & 36747.0      & \textbf{631.8}  &  58.16 & -41.7\% & -41.7\% \\
                          & 60 & 261308 & T.O.         & \textbf{1212.1} &  > 35.64 & T.O.    & -29.8\% \\

 \midrule
 \textbf{average} & & & & & > 9.91 & -27.1\% & -27.5\%
 \end{tabular}
  \caption{
  The figure shows the optimization results of our $\coamwith{\feyntool{}}$ tool and \feyntool{},
  using T count as the cost metric.
  %
  The labels ``S'' and ``F'' represent $\coam{}$ and \feyntool{} respectively.
%
  The figure presents T counts (lower is better) and
  running time (lower is better) for both optimizers.
%
The results demonstrate that optimizer $\coamwith{\feyntool{}}$,
which guarantees $\Omega-$optimality delivers a similar quality of circuits as \feyntool{}.
%
The figure also calculates the speedup of the \coam{} approach using the ``T(F)/T(S)'' ratio.
%
On average, the$\coamwith{\feyntool{}}$ delivers excellent time performance
and runs in $6.9$x less time, compared to \feyntool{}.
%
We impose an end-to-end timeout of 10 hours (36000 seconds).
%
}
  \label{fig:feynopt}
  % \setlength{\belowcaptionskip}{-20pt}
\end{figure}



\figref{feynopt} shows the results of this experiment.
%
The figure separates circuit families with horizontal lines
and sorts circuits within families by their size/number of qubits.
%
The figure calculates the metric T count for the input circuit and for
the circuits returned by \feyntool{} and \coamwith{\feyntool{}},
or \coam{} for short, labelled by ``F'' and ``S'' respectively.
%
The ``F/S'' represents the quality ratio of the two approaches.
%
The figure also shows the running times of both approaches and computes
the ratio ``F/S'', measuring the relative speedup from the \coam{} algorithm.
%
The results shows our \coam{} algorithm generates reliable quality circuits
in short running times.

\myparagraph{Optimization  Quality.}
The results show that our optimizer \coam{} matches the output quality of \feyntool{}.
%
There are a few cases where there is a difference of less than $1\%$,
such as the ``hhl'' circuit with $7$ qubits.
%
But overall, there is no noticeable difference and
the quality ratio of the outputs is $1$.
%
%
This experiment shows that $\Omega-$optimality, as guaranteed by
our \coam{} algorithm is a reliable quality criterion for T count optimization.
%for the \feyntool{} optimizer.
% This consistent output quality demonstrates
% that $\Omega-$optimality is a reliable quality guarantee for the optimizer \feyntool{}.
%
Note that the baseline \feyntool{} can, in principle, discover more optimizations
because it operates on the full circuit.
%
But in practice, we observe that optimizing the circuit in piecewise
fashion,
on circuit segments of size $\Omega$ and guaranteeing $\Omega-$optimality
finds similar optimizations.
%
In \secref{var-segment}, we confirm this for a range of $\Omega$ values.
%

\myparagraph{Run time.}
The figure also shows the running time of both approaches,
and calculates the speedup of the \coam{} approach using the ``T(F)/T(S)'' ratio.
%
Our \coam{} algorithm is slower by about factor two in one benchmark,
``grover'', but in all other benchmarks it performs better than \feyntool{},
running $6.9$x faster on average,
%
with no noticeable impact on the optimization quality (``F/S'' column).
%
For any given family, our \coam{} algorithm's speedup increases
consistently with the circuit sizes.
%
For example,
in the ``hwb'' family,
our \coam{} algorithm is $1.2$x faster for the smallest instance (with 8 qubits)
and runs $83$x faster for the largest instance (12 qubits)
%
;
%
For the ``hhl'' family,
our \coam{} algorithm is $23$x faster for the smallest instance (7 qubits)
and runs at least $94$x faster for the largest instance (9 qubits)
(on 9 qubits \feyntool{} does not terminate within our timeout of ten hours).
%
%

\myparagraph{Summary.}
The experiments suggest that local optimality is an effective
optimization criteria for T-count optimization. By focusing on local
optimizations, our \coam{} algorithm (with \feyntool{} as an oracle)
can optimize for T-count effectively and efficiently.
%

%% \todo{We observe...
%% %
%% Our theoretical results suggest that this gap could be understood as an
%% asymptotic gap,
%% %
%% for the following reasons....}

\if0
%%% THIS IS HARD TO READ FOR ME (UMUT)
Given that the speedup consistently increases with the circuit size,
the performance results indicate an asymptotic difference in running time of both
approaches.
%
We can confirm this difference using the theory results in \secref{algorithm}.
%
Specifically,
the \feyntool{} optimizer takes polynomial (at least quadratic) time
both in the number of qubits and the depth of the circuit.~\cite{amy2019formal}.
%
The optimizer \coamwith{\feyntool} is also polynomial in the number of qubits,
but scales linearly in the depth/size of the circuit.
%
From \corref{linear-calls},
using the fact that the cost function is T count,
we get that \coamwith{\feyntool} makes $O(\sizeof{C})$ calls to \feyntool{},
where $\sizeof{C}$ is the size of circuit $C$;
each call runs on small circuits of depth at most $2\Omega$ (\thmref{cost})
and therefore, each call to \feyntool{}
takes polynomial time in the number of qubits (the depth is constant).
%
Because the number of oracle calls is linear,
the total runtime of \coamwith{\feyntool} is linear in the size and polynomial in the number of qubits,
which is asymptotically different from \feyntool{}, which takes polynomial in both.
%
\fi

% %
% Because each such call only runs the optimizer on ``small circuits'' of depth at most $2\Omega$,
% the cost of each \feyntool{} call only depends on the number of qubits, which are roughly constant
% for most qubits in the figure.
% %
% Thus, the \coamwith{\feyntool} optimizer practically shows linear scalability.


% Thus, roughly speaking, the optimizer \coamwith{\feyntool} decomposes the complexity
% of optimization using \feyntool{} and makes it scale linearly (instead of quadratically)
% in the size of the circuit.
% %
% In summary,
% the optimizer \coamwith{\feyntool} consistently
% delivers reliable quality circuits with excellent scalability.
% %
% %
% The results demonstrate that $\Omega-$optimality is a practical and reliable
% quality criterion in circuit optimization with \feyntool{} optimizer.
% asymptotically improves the running time of optimization using \feyntool{}.
% \coamwith{\feyntool} delivers linear running time
% without compromising circuit quality.
%
% \secref{linear} discusses the linear running time of \coamwith{\feyntool} in more detail.


% % IT IS WELL KNOWN That xxx. IN principle, we could have an optimizer. We therefore to evaluate.
% We have evaluated benchmarks with gate count
% as the quality metric
% because our oracle, \quartz{}, directly supports it.
% %
% However, it is well known that depending on the gate set,
% other metrics become more important.
% %
% For example, in the \clifft{} gate set,
% T count is an important metric of circuit quality
% because T gates they are not as computationally efficient
% compared as the Clifford gates (T gates roughly cost 50x other gates in practice \cite{Nam_2018}).
% %
% Similarly, CNOT count is another important metric because CNOT
% gates operate on two qubits and are expensive to implement.
% %


% In this section,
% we evaluate the benefit of our configuration $\coamwith{\quartzt{6}}$,
% for metrics ``T count'' and ``CNOT count''.
% %
% Note that, in principle,
% we could plug in an optimizer which is designed for these metrics,
% but here we evaluate the benefit with \quartz{} that is designed with circuit size
% as the metric.
% %
% To do so, we translate our benchmark suite, expressed in the Nam gate set,
% to the \clifft{} gate set.
% %
% We then translate the optimized version of our benchmarks
% and compare the unoptimized and optimized translation.
% %
% %

% To translate a benchmark from the Nam gate set to the \clifft{} gate set,
% we use gridsynth,
% a tool that translates parameterized $\mathsf{R_Z}$ gates
% to the \clifft{} gate set in provably optimal fashion~\cite{gridsynth}.
% %
% We also decompose the $\mathsf{X}$ gate to \clifft{} circuits
% using the Qiskit compiler.
% %
% We run Qiskit  in the mode where it does not perform any optimizations.
% %
% Note that this translation does not change the CNOT count because
% it only replces single qubit nam gates with single qubit \clifft{} gates.
% %
% Thus, a CNOT count comparison applies to both the Nam and the \clifft{} gate sets.

% \figref{comp-cliff} compares
% the circuit size, the T count, and CNOT count,
% of the unoptimized translation and the optimized translation
% with columns labelled ``Input'' and ``Output'' respectively.
% %
% Across the board,
% we observe that the implementation finds reductions in size and
% they correlate with reductions in T count and CNOT count.
% %
% In case of the ham15-med circuit,
% the implementation reduces the T count by $7\%$
% and the CNOT count by $6\%$;
% the larger variant, circuit ham15-high,  also shows similar numbers.
% %
% For qft circuits,
% we observe that gate count reduction and T count reduction
% is perfectly correlated,
% as both are reduced by the same percentage on all the qft instances.
% %
% On average,
% the implementation reduces the circuit size by around $11.8\%$,
% the T count by $4\%$, and the CNOT count by $6.2\%$.


% \cleardoublepage
% \cleardoublepage

\subsection{\coam{} with the \quartz{} optimizer}
\label{sec:comp-quartz}

% In this subsection,
% we establish that ensuring $\Omega$-optimality leads to good circuit quality.
%
% Additionally, we study the benefits of our \coam{} algorithm when used with
% \quartz{}.

% One of the key benefits of our algorithm
% is that runs in a predictable amount of time, and (within that time)
% guarantees $\Omega$-optimality.
%
% In this section, we show how the \coam{} algorithm
% improves the \quartz{} optimizer.
% %
% We optimize circuits using the optimizer  $\coamwith{\quartzt{t}}$
% with segment depth $\Omega$ equal to $6$
% and
% The optimizer $\coamwith{\quartzt{t}}$ runs the \coam{} algorithm
% using \quartz{} as the oracle,
% where each call to \quartz{} is roughly allocated $t$ amount of time per gate,
% i.e., if the \coam{} algorithm calls \quartz{} for a circuit of size $k$,
% the timeout we give to \quartz{} is $k*t$.
% %
% We use the timeout functionality in \quartz{} to implement this.
% %
% We present results with the per-gate timeout $t = 0.01s$,
% but we also provide results of timeouts $t = 0.1s$ and $t = 1s$ in the Appendix.
% %
% % We chose the segment depth as $6$ because the default \quartz{} optimizer uses
% equivalence rules for circuits upto size $6$.
% %
% With segment depth as $6$,
% the algorithm feeds all segments of size $6$ to \quartz{}
% giving it enough scope to apply its optimizations.
%


% \begin{figure}
%   \centering
%   \small
%   \begin{tabular}{rcccccc}
  &  &  & \multicolumn{3}{c}{Number of optimizations} &  \\ \cmidrule(lr){4-6}
   Family & Qubits & Input Size & Q & S & S/Q & Time (s) \\


  \midrule\multirow{2}{*}{ham15}  & 17 &   1061 & 202 (19\%)  & 202 (19\%)             & 1.0x     &    8 \\
                          & 20 &   4365 & 342 (8\%)   & \textbf{992} (23\%)    & 2.9x     &   22 \\
  \midrule\multirow{3}{*}{hhl}    &  7 &   5319 & 290 (5\%)   & \textbf{1441} (27\%)   & 4.96x    &   10 \\
                          &  9 &  63392 & 145 (0\%)   & \textbf{15678} (25\%)  & 107.39x  &  130 \\
                          & 11 & 629247 & 131 (0\%)   & \textbf{144438} (23\%) & 1094.23x & 1702 \\
  \midrule\multirow{2}{*}{gf}     & 48 &   2694 & 0 (0\%)     & 0 (0\%)                & 1.0x     &    8 \\
                          & 96 &   7553 & 2078 (28\%) & \textbf{4124} (55\%)   & 1.98x    &  137 \\
  \midrule\multirow{4}{*}{grover} &  7 &   2479 & 224 (9\%)   & \textbf{252} (10\%)    & 1.12x    &    9 \\
                          &  9 &   8968 & 66 (1\%)    & \textbf{876} (10\%)    & 13.09x   &   28 \\
                          & 11 &  27136 & 105 (0\%)   & \textbf{2643} (10\%)   & 24.94x   &   78 \\
                          & 15 & 180497 & 323 (0\%)   & \textbf{17588} (10\%)  & 54.29x   &  583 \\
  \midrule\multirow{4}{*}{qft}    & 48 &   4626 & 424 (9\%)   & \textbf{756} (16\%)    & 1.78x    &   15 \\
                          & 64 &   7402 & 758 (10\%)  & \textbf{1892} (26\%)   & 2.49x    &   31 \\
                          & 80 &  10690 & 744 (7\%)   & \textbf{3540} (33\%)   & 4.75x    &   46 \\
                          & 96 &  14490 & 1098 (8\%)  & \textbf{5700} (39\%)   & 5.19x    &   66 \\
  \midrule\multirow{4}{*}{vqe}    & 12 &  11022 & 1127 (10\%) & \textbf{7407} (67\%)   & 6.57x    &   37 \\
                          & 16 &  22374 & 458 (2\%)   & \textbf{14343} (64\%)  & 31.25x   &   87 \\
                          & 20 &  38462 & 1019 (3\%)  & \textbf{23551} (61\%)  & 23.09x   &  166 \\
                          & 24 &  59798 & 789 (1\%)   & \textbf{35031} (59\%)  & 44.34x   &  316 \\

 \midrule
 \end{tabular}
%   \caption{
%   The figure displays the number optimizations and running time of our optimizer $\coamt{\quartzt{0.01}}$
%   and also shows the optimizations by baseline \quartz{} in the same end-to-end time.
%   %
%   The label ``S'' represents the $\coam{}$ approach and the label ``Q'' represents \quartz{}.
%   %
%   The $S/Q$ column shows how many more optimizations \coam{} finds w.r.t \quartz{} for the same running time.
%   %
%   The figure shows that our \coam{} algorithm finds significant reductions ranging
%   from $20\% - 60\%$ in short runtimes.
%   %
%   Time column shows the runtime in seconds.
%   }
%   \label{fig:comp-quartz}
%   % \setlength{\belowcaptionskip}{-20pt}
% \end{figure}



\begin{figure}
  \centering
  \small
  \begin{tabular}{ccccccc}
    &  &  & \multicolumn{3}{c}{Number of optimizations} &  \\ \cmidrule(lr){4-6} 
     Family & Qubits & Input Size & Q & S & S/Q & Time (s) \\ 
   
   
    \midrule\multirow{2}{*}{ham15}  & 17 &  1061 & 202 (19\%)  & 202 (19\%)            & 1.0x   &   3 \\
                            & 20 &  4365 & 800 (18\%)  & \textbf{992} (23\%)   & 1.24x  &  13 \\
    \midrule\multirow{2}{*}{hhl}    &  7 &  5319 & 373 (7\%)   & \textbf{1441} (27\%)  & 3.86x  &   8 \\
                            &  9 & 63392 & 260 (0\%)   & \textbf{15678} (25\%) & 60.07x & 113 \\
    \midrule\multirow{2}{*}{gf}     & 48 &  2694 & 0 (0\%)     & 0 (0\%)               & 1.0x   &   3 \\
                            & 96 &  7553 & 20 (0\%)    & 20 (0\%)              & 1.0x   &  41 \\
    \midrule\multirow{3}{*}{grover} &  7 &  2479 & 236 (10\%)  & \textbf{243} (10\%)   & 1.03x  &   6 \\
                            &  9 &  8968 & 618 (7\%)   & \textbf{858} (10\%)   & 1.39x  &  21 \\
                            & 11 & 27136 & 952 (4\%)   & \textbf{2604} (10\%)  & 2.73x  & 103 \\
    \midrule\multirow{4}{*}{qft}    & 48 &  4626 & 756 (16\%)  & 756 (16\%)            & 1.0x   &  12 \\
                            & 64 &  7402 & 1236 (17\%) & \textbf{1892} (26\%)  & 1.53x  &  20 \\
                            & 80 & 10690 & 1312 (12\%) & \textbf{3540} (33\%)  & 2.7x   &  33 \\
                            & 96 & 14490 & 1534 (11\%) & \textbf{5700} (39\%)  & 3.71x  &  51 \\
    \midrule\multirow{4}{*}{vqe}    & 12 & 11022 & 627 (6\%)   & \textbf{7407} (67\%)  & 11.8x  &  25 \\
                            & 16 & 22374 & 493 (2\%)   & \textbf{14343} (64\%) & 29.04x &  60 \\
                            & 20 & 38462 & 643 (2\%)   & \textbf{23551} (61\%) & 36.57x & 125 \\
                            & 24 & 59798 & 505 (1\%)   & \textbf{35031} (59\%) & 69.23x & 195 \\
   
   \midrule
   \end{tabular}
  \caption{
  The figure displays the number optimizations and running time of our optimizer $\coamt{\quartzt{0.01}}$
  and also shows the optimizations by baseline \quartz{} in the same end-to-end time.
  %
  The label ``S'' represents the $\coam{}$ approach and the label ``Q'' represents \quartz{}.
  %
  The $S/Q$ column shows how many more optimizations \coam{} finds w.r.t \quartz{} for the same running time.
  %
  The figure shows that our \coam{} algorithm finds significant reductions ranging
  from $20\% - 60\%$ in short runtimes.
  %
  Time column shows the runtime in seconds.
  }
  \label{fig:comp-quartz}
  % \setlength{\belowcaptionskip}{-20pt}
\end{figure}

In this section, we evaluate our $\coamwith{\quartzt{0.01}}$ optimizer,
which runs the \coam{} algorithm using \quartz{} as the oracle.
%
To evaluate its optimizations,
we run our optimizer on all the circuits and record its output and the running time.
%
Then,
we run default \quartz{} on the each input circuit for the same end-to-end time,
%
and analyze the number of optimizations discovered.

\figref{comp-quartz} shows the results for circuits taken from six families of quantum algorithms.
%
In the figure,
the label ``S'' denotes the number of optimizations discovered by our \coam{} algorithm
and label ``Q'' denotes the same with baseline \quartz{} (higher is better).
%
In cases where one approach finds more optimizations,
the figure uses bold numbers.
%
The label ``S/Q'' computes the \defn{optimization ratio} which is
the ratio of number of optimizations.
%
A higher ratio represents that our algorithm finds more optimizations.
%
The results show that the \coam{} approach produces
well optimized circuits in short running times.


\myparagraph{Quality.}
We observe that in all but two cases (ham15 with 17 qubits and gf with 48 qubits),
using our \coam{} approach
discovers more optimizations than baseline \quartz{}.
%
For the hhl circuits,
the \coam{} algorithm reduces gate count by $23\%$ to $27\%$,
which is excellent considering that \quartz{} reduces the size by less than $5\%$
within the same time.
%
For the vqe family,
our \coam{} algorithm reduces the gate count by around $60\%$ and
the optimizations by \quartz{} are less than $10\%$.
%
Overall, we observe that by using \coam{} algorithm with \quartz{} as an oracle,
we can effectively reduce gate count for circuits with thousands of gates
within seconds and optimize circuits with hundreds of thousands of gates within minutes
(the Time column displays the running time).
%
In the Appendix,
we verify that our optimizer \coamwith{\quartz{}} does not generate a worse quality circuit
even when the running time for both algorithms is scaled
by $100$x (by increasing the per call timeout 100x; see \secref{methodology} for more details on running time).

\myparagraph{Scalability.}
Note that our \coam{} optimizer
excels at finding optimizations for large circuits.
%
The figure illustrates this using the ``S/Q'' column,
which computes the optimization ratio of both approaches.
%
For all families, the optimization ratio increases with increasing circuit size.
%
In the case of hhl circuits, for example,
the ratio rises from around $5$x to $1000$x with increasing circuit size;
%
In the case of grover circuits,
the optimization ratio steadily increases from $1.12$x on the smallest instance (7 qubits)
to $54.29$x on the largest instance (15 qubits).
%
Similarly, for vqe, the ratio increases from $6.5$x to $44.34$x with increasing circuit size
and for qft, it increases from $1.78$x to $5.19$x.
%
We identify three key reasons for this scalability.
%
First, because baseline \quartz{} searches for optimizations on the whole circuit,
it struggles to find optimizations,
because the search space is exponential.
%
Second, our optimizer \coam{} algorithm
utilizes \quartz{} by focusing it on small segments ($\Omega = 6$),
where it delivers excellent results.
%
Third,
the \coam{} algorithm ensures that it applies \quartz{} to all
$\Omega-$segments and misses no local optimizations.

\myparagraph{Summary.}
The experiments demonstrate that our \coam{} algorithm scales with circuit sizes
when using \quartz{} as the oracle.
%
The results suggest that our algorithm uses \quartz{} effectively,
finding significant reductions in circuit size and producing good quality circuits.
%

%

% For example,
% across different sizes of grover circuits,
% the \coam{} algorithm reduces gate count consistently by $10\%$;
% in contrast the gate count reduction by \quartz{} drops from $9\%$ to $0\%$.
%

%
% The 'S/Q' column shows the improvement
% in optimization rates due to \coam{}, i.e.,
% it shows how many more optimizations (per second)
% the \coam{} algorithm identifies compared to \quartz{}.
% %
% %
% Compared to \quartz{},
% the \coam{} algorithm yields $1.78$x and $5.19$x more optimizations per second
% for the smallest and largest qft instances
% respectively (see the ratio column in \figref{comp-quartz}).
% %
% Similarly,
% in the case of vqe,
% the gap in their optimization rates
% rises from around $2$x on the smallest instance to $42$x
% on the largest instance.

% %
% This is because the efficiency of \quartz{},
% quantified by optimization rates,
% reduces with increasing circuit sizes,
% whereas the \coam{} algorithm scales to larger sizes.

% Across the board,
% we observe that the optimization ratio is never less than one,
% establishing that the \coam{} approach never generates a worse quality circuit
% in the same running time.
% %
% For instance,



% we first optimize all circuits with $\coamwith{\quartzt{t}}$, where each oracle call is capped by a
% fixed amount of time $t$, and record how long the entire algorithm takes to
% complete.
% %
% Then,
% we run the default \quartz{} on the same input circuit for the same end-to-end
% time,
% %
% and compare the resulting circuit quality and optimization rate.


% number of gates optimized per second by
% \coam{} using \quartz{} as an oracle is higher than just running \quartz{}
% alone.

% % when using \quartz{} as an oracle,
% % \coam{} is able to accelerate the \defn{optimization rate} of \quartz{}.
% %

% %
% As a result, \coam{} is able to deliver better quality circuits within
% a given time limit.

% , due to its combination of linear time complexity and
% $\Omega$-optimality guarantee.
%


% \figref{comp-quartz} shows the result of this comparison
% using the configuration $\coamwith{\quartzt{0.06}}$, i.e., where
% each oracle call is limited to $0.06$ seconds.
% %
% The figure shows the input circuit sizes
% and displays the circuit sizes after optimization.
% %
% The figure labels the results of the \coam{} configuration as ``S'' and
% the standard \quartz{} approach as ``Q''.
% %
% In the cases where one approach performs better,
% the figure uses bold numbers, with lower numbers indicating better circuit quality.
% %
% The `Q/S' ratio column quantifies the improvement in circuit quality
% using the \coam{} algorithm w.r.t. standard \quartz{} (higher is better).
% %
% The figure also provides and compares the \emph{optimization rate} of both
% approaches, where the optimization rate is the number of gates optimized per
% second of execution.
% %
% A higher rate indicates faster optimizations.
% %
% The time column shows the execution time for both cases.



% \myparagraph{Quality.}
% We observe that in seventeen cases,
% using the \coam{} algorithm improves the quality of circuits w.r.t. standard \quartz{}
% and sometimes does so by a wide margin.
% %
% In the other seven cases,
% both approaches output the same quality of circuits.
% %
% For the hhl circuits,
% the \coam{} algorithm finds reductions ranging from $23\%$ to $27\%$,
% where \quartz{} does not find many optimizations
% within this time (less than $5\%$).
% %
% In the case of grover circuits,
% the \coam{} algorithm optimizes gate count by $10\%$ across the various sizes.
% %
% However, the efficacy of \quartz{} reduces from $9\%$ to $0\%$ with increasing sizes.
% %
% In the qpe circuits,
% neither approach finds any optimizations.
% %
% %
% For the vqe family,
% the \coam{} algorithm reduces the gate count by around $60\%$ and
% the optimizations by \quartz{} are less than $10\%$.
% %
% On average,
% the outputs of the \coam{} configuration are smaller by $1.37$x,
% when compared to the outputs of standard \quartz{}.

% \myparagraph{Scalability.}
% Across the board,
% we observe that the gap between the two approaches widens with increasing circuit sizes.
% %
% This is because the efficiency of \quartz{},
% quantified by optimization rates,
% reduces with increasing circuit sizes,
% whereas the \coam{} algorithm scales to larger sizes.

% For example,
% across different sizes of qft,
% the \coam{} algorithm achieves optimization rates
% ranging from $50$ to $85$ and the rates in fact increase as the circuit size grows.
% %
% In contrast,
% the optimization rate for \quartz{} reduces from around $28$ to $16$.
% %
% The 'S/Q' column shows the improvement
% in optimization rates due to \coam{}, i.e.,
% it shows how many more optimizations (per second)
% the \coam{} algorithm identifies compared to \quartz{}.
% %
% %
% Compared to \quartz{},
% the \coam{} algorithm yields $1.78$x and $5.19$x more optimizations per second
% for the smallest and largest qft instances
% respectively (see the ratio column in \figref{comp-quartz}).
% %
% Similarly,
% in the case of vqe,
% the gap in their optimization rates
% rises from around $2$x on the smallest instance to $42$x
% on the largest instance.
% %
% On average,
% we observe that the \coam{} algorithm finds $7.35$x many more optimizations
% than \quartz{} per second.


% All families of circuits follow a similar trend:
% the optimization rate ratio between the \coam{} approach and the \quartz{} optimizer
% increases with the circuit size,
% demonstrating the scalability advantages of the \coam{} algorithm.
% %
% This scalability advantage of the \coam{} approach is built into the algorithm
% because it calls \quartz{} only on small circuits,
% where \quartz{} delivers excellent results.
% %
% The algorithm never sends large circuits to \quartz{},
% where \quartz{} struggles
% because it chases potential optimizations in large spaces.
% %
% Because the \coam{} algorithm runs in linear time,
% it delivers good scalability as circuit sizes increase.
% %
% %
% %
% Because the execution time is equal for both approaches,
% the ratio column additionally represents the ratio of total optimizations.
%
% %
% For example,
% consider the ratio column of \figref{comp-quartz} which compares the optimization rates
% of the \coam{} algorithm with \quartz{}.
%

% First, because they can take a very long time,
% existing super op- timizers such as Quartz are used with a hard deadline (e.g., several hours)
% by which they have to be terminated. When not run to completion,
% they are unable to offer any quality guarantees whatsover, because they can get “lost” chasing potential optimizations
% in exponentially large search spaces.


% Second, most circuit optimizations naturally involve a small number of contiguous gates.
% Our algorithm (provably) does not miss such optimizations, but other optimizers could.
% \myparagraph{Consistency.}
% Since the \coam{} algorithm splits the circuit into pieces
% and melds them together,
% it is important to evaluate whether
% splitting and melding lose any optimizations in practice.
% %
% In theory, our algorithm guarantees $\Omega-$optimality.
% %
% To evaluate this guarantee,
% we compare to standard \quartz{} with extended running times.
% %
% We extend the running time by increasing the per-call timeout to \quartz{}
% to $0.6$s, i.e.,
% we consider the configuration \coamwith{\quartzt{0.6}}.
% %
% We note that the end-to-end running time becomes at least ten fold in this configuration,
% and we run the standard \quartz{} for the extended time.
% %
% This ensures that \quartz{} has a larger amount of
% time to find optimizations.
% %
% We present these results in the Appendix, and summarize them here.
% % /
% Overall, the gap between both approaches reduces with extended running time
% because when optimizers operate for longer times,
% their optimization rates diminish due to the
% increasing difficulty of finding further optimizations~\cite{quartz-2022, queso-2023}.
% %
% This extended running time allows \quartz{} to somewhat catch up to the \coam{}.
% %
% However, \coam{} never generates a worse quality circuit.
% %
% We also increase the running time further by ten fold,
% verifying that \coam{} consistently delivers better quality circuits.
% %
% We conclude from these experiment with extended running times
% that $\Omega$-optimality is a good quality guarantee because,
% in practice,
% the \coam{} algorithm never generates a worse quality circuit.
% %



%
% Overall, we note that the gap between the approaches has reduced,
% when compared to the previous configuration.
% %
% This occurs because as optimizers operate for longer times,
% their optimization rates diminish due to the
% increasing difficulty of finding further optimizations~\cite{quartz-2022, queso-2023}.
% %
% The difference in optimization rates between \figref{comp-quartz} and \figref{comp-queso}
% exemplifies this.
% %
% For example, the optimization rate in vqe\_n12 for \quartz{} reduces from around $30$ to $12$
% when its execution time increases from $37$ seconds to $640$ seconds.
% %
% The many fold extension of execution time decreases the optimization rates of both approaches
% and moderates the scalability advantages of the \coam{} algorithm.
% %
% On average,
% circuits generated by the \coam{} algorithm are $1.1$ times smaller
% than those generated by \quartz{}.
% %/
% In our Supplementary,
% we include a third configuration (\coamwith{\quartzt{6}})
% where we further increase the time ten fold and observe a similar trend,
% where \coam{} consistently delivers better quality circuits.
% %
% We conclude from these experiment with extended running times
% that $\Omega$-optimality is a good quality guarantee because,
% in practice,
% the \coam{} algorithm never generates a worse quality circuit.
%

\subsection{\coam{} with the \queso{} optimizer}
\label{sec:comp-queso}
% guarantees $\Omega$-optimality.
%
In this section, we evaluate our optimizer $\coamwith{\quesot{0.005}}$,
which runs our \coam{} algorithm using \queso{} as the oracle.
%
Our optimizer depth uses segment size $\Omega = 6$
and gives each call to \queso{} is a timeout (see \secref{methodology} for the more details).
%
We run $\coamwith{\quesot{0.005}}$,
or \coam{} for short,
on all the circuits and record
the output quality and running time.
%
We then run default \queso{} on each input circuit for the same end-to-end time,
%
and evaluate the number of optimizations.


\begin{figure}
  \centering
  \small
  \begin{tabular}{ccccccc}
  &  &  & \multicolumn{3}{c}{Number of optimizations} &  \\ \cmidrule(lr){4-6}
   Family & Qubits & Input Size & Q & S & S/Q & Time (s) \\


  \midrule\multirow{2}{*}{ham15}  & 17 &   1061 & 206 (19\%)   & \textbf{250} (24\%)    & 1.21x     &    26 \\
                          & 20 &   4365 & 650 (15\%)   & \textbf{1353} (31\%)   & 2.08x     &   130 \\
  \midrule\multirow{3}{*}{hhl}    &  7 &   5319 & 1509 (28\%)  & \textbf{1829} (34\%)   & 1.21x     &   289 \\
                          &  9 &  63392 & 10703 (17\%) & \textbf{20764} (33\%)  & 1.94x     & 12784 \\
                          & 11 & 629247 & 0 (0\%)      & \textbf{109353} (17\%) & 109354.0x & 35987 \\
  \midrule\multirow{2}{*}{gf}     & 48 &   2694 & 0 (0\%)      & 0 (0\%)                & 1.0x      &    19 \\
                          & 96 &   7553 & 0 (0\%)      & \textbf{4098} (54\%)   & 4099.0x   &   213 \\
  \midrule\multirow{4}{*}{grover} &  7 &   2479 & 244 (10\%)   & \textbf{326} (13\%)    & 1.33x     &   152 \\
                          &  9 &   8968 & 32 (0\%)     & \textbf{1043} (12\%)   & 31.64x    &   252 \\
                          & 11 &  27148 & 70 (0\%)     & \textbf{3357} (12\%)   & 47.3x     &  1081 \\
                          & 15 & 180497 & 282 (0\%)    & \textbf{20510} (11\%)  & 72.48x    &  4759 \\
  \midrule\multirow{4}{*}{qft}    & 48 &   4626 & 935 (20\%)   & \textbf{1228} (27\%)   & 1.31x     &   497 \\
                          & 64 &   7402 & 2034 (27\%)  & \textbf{2558} (35\%)   & 1.26x     &   834 \\
                          & 80 &  10690 & 3540 (33\%)  & \textbf{4451} (42\%)   & 1.26x     &  1270 \\
                          & 96 &  14490 & 5862 (40\%)  & \textbf{6808} (47\%)   & 1.16x     &  1597 \\
  \midrule\multirow{4}{*}{vqe}    & 12 &  11022 & 4620 (42\%)  & \textbf{7521} (68\%)   & 1.63x     &   150 \\
                          & 16 &  22374 & 0 (0\%)      & \textbf{14511} (65\%)  & 14512.0x  &   293 \\
                          & 20 &  38462 & 0 (0\%)      & \textbf{23777} (62\%)  & 23778.0x  &   518 \\
                          & 24 &  59798 & 0 (0\%)      & \textbf{35315} (59\%)  & 35316.0x  &   827 \\

 \midrule
 \end{tabular}
  \caption{
  The figure compares the performance of our $\coamwith{\quesot{0.005}}$ optimizer with standard \queso{}.
  %
  The label ``S'' represents the number of optimization discovered by our \coam{} approach
  and the label ``Q'' denotes optimizations found by \queso{}.
  %
  The ``S/Q'' column measures how many more optimizations \coam{} finds relative to \queso{};
  we avoid divide by zero issues by adding $+1$ optimization to both tools.
  %
  For the results in this figure, we impose an end-to-end timeout of ten hours.
  }
  \label{fig:comp-queso}
  % \setlength{\belowcaptionskip}{-10pt}
\end{figure}

\figref{comp-queso} shows the results of this experiment
for circuits from six families of quantum algorithms.
%
Each family is separated by a horizontal line
and contains circuits which differ number of qubits and are sorted by input size.
%
The figure shows the number of optimizations performed by our \coam{} approach and by \queso{}
in the columns labeled ``S'' and ``Q'' respectively;
the columns also list the percentage of gates removed in parentheses.
%
The figure uses bold numbers when one tool finds more optimizations than the other.
%
The time column shows the running time of each benchmark.
%
The results show that our optimizer consistently finds significant reductions
in gate count, across all families and circuit sizes.

%
\myparagraph{Quality.}
Across the board, we observe that using \queso{} with our \coam{} algorithm discovers
more optimizations than baseline \queso{}.
%
For the gf circuits,
the \coam{} approach reduces the gate count by $54\%$ the large instance (96 qubits)
and \queso{} does not discover any optimizations in same the running time.
%
For the vqe circuits,
our \coam{} algorithm delivers excellent results,
reducing the gate count by around around $60\%$ on all instances;
\queso{} reduces the gate count of the smallest instance by $48\%$ but
does not find optimizations for others in this amount of running time.
%
%
For the qft circuits, both approaches find similar reductions in the gate count.
%
Overall, we see that using \queso{} with the \coam{} algorithm finds substantial
reductions in gate count, across a range of quantum algorithms and circuit sizes.
%

\myparagraph{Scalability.}
The figure also shows that our \coamwith{\queso} optimizer scales well with circuit size.
%
The column ``S/Q'' computes the ratio of optimizations found by the tools,
and the ratio typically increases with circuit size.
%
For example,
in the case of ham circuits,
the ratio increases from 1.19x to 2x with increasing circuit size.
%
For grover circuits,
the optimization ratio increases from $1.3$x to $72$x.
%
Similarly, for hhl circuits, the ratio increases from 1.21x on the smallest instance (7 qubits)
to 1.94x on the medium sized instance (9 qubits);
for the largest instance of hhl (11 qubits),
using \queso{} with our \coam{} algorithm finds 17\% reduction and baseline \queso{}
does not discover any optimizations.
%
Similarly, \coam{} delivers increasing optimization ratios for the ``vqe'' family.
%

\myparagraph{Summary.}
The results show that many optimizations are local
and are found by our algorithm because it focuses \queso{} on small segments.
%
Our algorithm's approach to optimize circuits in a piecewise fashion and melding them together
finds significant reductions using \queso{} as the oracle.

% Because \coamwith{\queso} finds these optimizations by only sending segments of size $2\Omega$ to \queso{},
% the results demonstrate that many optimizations are local
% and can be found by focusing the baseline optimizer on small segments.

% The evaluation suggests that the \coam{} algorithm utilizes the \queso{} optimizer effectively,
% by splitting the circuit into small segments, optimizing them with \queso{} and finding
% the optimizations at the boundaries by melding the resulting circuits.




% \begin{figure}
	\centering

	\subfloat[Grover]{\includegraphics[width=0.5\columnwidth]{plots/linearitygrover0.01.png}}
	\subfloat[VQE]{\includegraphics[width=0.5\columnwidth]{plots/linearityvqe0.01.png}}

	\subfloat[VPE]{\includegraphics[width=0.5\columnwidth]{plots/linearityshor0.01.png}}
	\subfloat[QFT]{\includegraphics[width=0.5\columnwidth]{plots/linearityqft0.01.png}}

	\caption{
	\textbf{X-axis: circuit size in thousands.}
	\textbf{Y-axis: time in seconds.}
	The figure plots the optimization times for four quantum algorithms
	against the corresponding circuit sizes.
	The figure confirms that our algorithm scales linearly with the size of the circuit.
	}
	\label{fig:linear}

\end{figure}


% \begin{figure}[t]
% 	\centering
% 	\includegraphics[width=0.9\columnwidth]{plots/linearity0.01.png}\quad
% 	\scriptsize
% \caption{
% 	The figure plots the optimizer times for five quantum algorithms
% 	as their size varies.
% 	%
% 	The X-axis shows the relative size of a circuit for each algorithm.
% 	%
% 	The Y-axis shows the time taken to optimize the circuit.
% 	%
% 	The figure confirms that our algorithm scales linearly with the size of the circuit.
% }
% \end{figure}

\subsection{Experimenting with segment size}
\label{sec:var-segment}
\begin{figure}[t]
	\centering
  % \begin{minipage}[b]{0.5\columnwidth}
  %   \includegraphics[width=\textwidth]{plots/omega_vs_quality.png}
  % \end{minipage}
  % \hspace{1cm}
  % \begin{minipage}[b]{0.5\columnwidth}
  %   \includegraphics[width=\textwidth]{plots/omega_vs_time.png}
  % \end{minipage}
  \includegraphics[width=\textwidth]{media/plot_omega.pdf}
    \vspace{-15mm}
	\caption{
    The figure plots the impact of the parameter $\Omega$ on
    the output T count (lower is better)
    and running time of $\algname{}$ optimizer on a hhl circuit with $7$ qubits.
    %
    The dotted red lines in the plots denote the output T count and the running time of the oracle optimizer
    \feyntool{} on the whole circuit.
    %
    For almost all values of $\Omega$, the output quality matches the oracle optimizer,
    demonstrating that local optimality is a robust quality criterion for T count optimization.
  }
    \label{fig:vary-segment}
\end{figure}
In this section,
we evaluate the quality and efficiency guaranteed by the \coam{} algorithm
for different values of parameter $\Omega$.
%
We use our optimizer $\coamwith{\feyntool}$ for this evaluation,
which guarantees $\Omega-$optimality relative to the \feyntool{} optimizer.

%
\figref{vary-segment} plots the output T count (number of T gates) and the running
time of the optimizer against $\Omega$.
%
The figure shows the results for $\Omega$ values $2, 5, 15, 30, 60, 120 \dots 7680$;
we present the full table in the Appendix.
%
The red dotted line in the plots
shows the results of the the baseline optimizer \feyntool{}.
%
For the experiment,
we run the our optimizer on the hhl circuit with $7$ qubits, which initially contains 61246 T gates.
%
The results show that for a wide range of $\Omega$ values,
our optimizer produces a similar quality circuit as the baseline $\feyntool$
and typically does so in significantly less time
(at the extremities, there are two values of $\Omega$ for which \coam{} takes more time: $2$ and $7680$).
%

%
\myparagraph{T count.}
The plot for T count shows that when $\Omega$ is small (around $2$),
increasing it has quality benefits.
%
This is because quality guarantee of the $\coam{}$ algorithm becomes
stronger with increasing $\Omega$ as it requires larger segments to be optimal.
%
%
However, the benefits of increasing $\Omega$
become very incremental especially when $\Omega$ reaches around $60$,
where T count reaches around 42140 (20 gates away from being optimal).
%
This shows that $\Omega-$optimality is a good quality criterion,
as it generates good quality circuits
even with relatively small values of $\Omega$ (around $60$).

\myparagraph{Run time.}
One would perhaps expect that the running time of \coam{}
increases by increasing the segment size $\Omega$ because
1) each oracle call operates on larger segments
and 2) the algorithm generates a better quality circuit (provably and practically).
%
Indeed, the intuition is correct, for a majority of the values.
%
For values of $\Omega$ ranging from $120$ to $7860$,
the running time of the algorithm increases with increasing $\Omega$.
%
In this range,
the (non-linear) complexity of the oracle dominates the running time,
and it is faster to split, optimize and meld, calling the oracle many times
on smaller segments,
instead of querying the oracle on larger segments.


But, when $\Omega$ is very small,
we observe the opposite, i.e.,
increasing $\Omega$ reduces the running time.
\begin{wrapfigure}{r}{0.4\textwidth}
  \centering
  \includegraphics[width=0.39\textwidth]{omega_vs_time_zoom.png}
  \caption{Zooming in: Time vs. Omega plot}
  \label{fig:zoom-plot}
\end{wrapfigure}
%
For reference,
we draw \figref{zoom-plot},
which zooms the running time plot from \figref{vary-segment}
for initial values of $\Omega$, ranging from $2, 5, 15 \dots 120$.
%
For these smaller values of $\Omega$,
even though each oracle call is cheap,
the number of oracle calls dominates the time cost.
%
The \coam{} algorithm splits the circuit into a large number of small segments
and queries the oracle on each one, making many calls to the optimizer.
%
When a circuit segment is small,
it is more efficient to directly call \feyntool{},
which optimizes it in one pass.
%
For this reason, $\Omega = 120$ is a good value for our optimizer $\coamwith{\feyntool{}}$,
as it does not send large circuits to the oracle,
and also does not split the circuit into a large number of really small segments.
%

Overall,
we observe that for a wide range of $\Omega$ values,
our \coam{} algorithm outputs good quality circuits
and does so in a shorter running time than the baseline.
%
\begin{figure}[t]
    \centering
    \includegraphics[width=\columnwidth]{media/oracle_call_vs_input_size.pdf}
    \vspace{-20pt}
    \caption{ The number of oracle calls versus input circuit size for
      all our circuits (Nam gate set).  The plots show that the number of
      calls scales linearly with the number of gates.  }
    \label{fig:plot-oracle-calls-all}
\end{figure}



% Furthermore, it increases the number of meld operations
% and almost every meld of the algorithm finds optimizations,
% because the oracle optimizes the boundary segments in a piecewise fashion
% rather than finding optimizing them in one step.
%
%
% We measured that, for $\Omega = 2$,
% \coamwith{\feyntool} makes around ten thousand calls to the base optimizer \feyntool{},
% taking around four hundred seconds.
% %
% Thus, at the very start,
% increasing $\Omega$ reduces the running time because it reduces
% the number of oracle calls.
% %
% \figref{zoom-plot}
% The plot shows how the running time decreases with increasing the parameter.


% When the segment size becomes slightly larger, around $\Omega = 60$,
% the running time reduces to twenty seconds and does not vary much for
% some intermediate values.
% %
% For most intermediate values, around $\Omega = 30$ to $\Omega = 480$,
% we observe good time performance, where the algorithm takes around thirty seconds.
% %
% Beyond these intermediate values,
% the running time starts increasing because the algorithm sends larger segments to the oracle
% and oracle complexity becomes the dominating cost.
% %
% It is then much
% faster to split, optimize, and meld rather than use the oracle.

% When $\Omega$ is small, increasing it reduces the running time of the algorithm.
% %
% This is perhaps counterintuitive because small segment sizes
% generate a worse quality circuit and also use the oracle on smaller segments.
% %
% But, it reduces the number of oracle calls, which becomes the dominating cost in this case.
% %
% Furthermore, for small segment sizes,
% almost every meld is finding optimizations
% %




% Also note, that the initial T count for this circuit is 61246,
% implying that different values of $\Omega$ do not have a large impact on quality.
% %






% %
% But slowly the benefits become incremental and eventually $\Omega$ is large enough where
% the output of \feyntool{} matches the optimizer  $\coamwith{\feyntool}$,
% as shown with the dotted red line.
% To summarize,
% \figref{fig:comp-quartz} demonstrates that the \coam{} algorithm
% drastically improves the speed of optimization for each circuit.
% %
% It shows that the algorithm scales well with circuit size,
% achieving excellent optimization rates across a range of sizes.
% %
% \figref{fig:comp-queso} shows that this performance
%

%


% \input{fig/prelim2.0.01}



% To summarize,
% \figref{fig:comp-quartz} demonstrates that the \coam{} algorithm
% drastically improves the speed of optimization for each circuit.
% %
% It shows that the algorithm scales well with circuit size,
% achieving excellent optimization rates across a range of sizes.
% %
% \figref{fig:comp-queso} shows that this performance comes at no cost to circuit quality
% even with the extended runtime,
% as the \coam{} algorithm never generates a worse quality circuit.
% %


% \subsection{Scalable optimization using \coam{} with \queso{}}

% \begin{figure*}
%   \centering
%   \input{fig/r}
%   \caption{
%   The figure compares the performance of $\coamwith{\quesot{1}}$ against standard \queso{}.
%   %
% The label ``S'' represents $\coamwith{\quesot{1}}$ and the label ``Q'' represents \queso{}.
% %
% The figure compares the circuit sizes (lower is better) and the optimization rates (higher is better) of both approaches
% after running them for the same end-to-end time.
% }
%   \label{fig:comp-queso}
% \end{figure*}

% In this section,
% we evaluate the \coam{} algorithm using the \queso{} optimizer.
% %
% We integrate the \queso{} optimizer into our algorithm as an oracle
% and impose a one second time out for each call.
% %
% We note that \queso{} does not follow the given timeout strictly
% and often takes longer to run and return the result (around $4/5$ seconds per call).

% To evaluate the algorithm,
% we run it with \queso{} and compute the final circuit
% and the overall execution time, excluding time for file I/O (\secref{impl}).
% %
% Then, we execute default \queso{} on the whole circuit for the same total runtime
% and compare the quality of circuits produced.
% %
% For this experiment,
% we filter benchmarks based on their sizes and
% opt for benchmarks whose sizes are less than ten thousand gates.
% %
% This size limitation arises from file I/O overheads which make restrict our ability
% to run larger circuits within reasonable times.

% \figref{comp-queso} shows the results of the evaluation for seven benchmarks
% coming from four families of algorithms.
% %
% The ``QS'' column denotes \queso{} and the ``S'' column represents the \coam{} algorithm
% using \queso{} as the oracle.
% %
% In all cases, the \coam{} approach generates smaller circuits than the baseline \queso{}.
% %
% Similar to the case with \q{} (\secref{quartz}),
% we observe that the gap between the \coam{} approach and default \queso{} increases
% with circuit size.
% %
% Foe example,
% in the grover family,
% the ratio of optimization rate between \coam{} and \queso{}
% increases from $1.14$x for grover\_n7 to $1.4$x in grover\_n9.
% %
% For the circuit vqe\_n12,
% which is the largest circuit on the table,
% the final circuit size is $0.78$ times smaller compared to \queso{},
% which demonstrates the scalability of the \coam{} approach.
% %
% On average,
% the algorithm's circuits are $0.92$x smaller than those generated by \queso{}
% and the algorithm makes $1.27$x as many optimizations per second.
% %

% The evaluation demonstrates that the \coam{} algorithm
% improves the scalability of the \queso{} optimizer and
% consistently delivers better quality circuits.


% \section{Preliminaries}
\label{sec:prelim}
\label{sec:term}
We define the key terminologies used, primarily focusing on the hidden states (or activations) during the forward pass. 

\paragraph{Components in an attention layer.} We denote $\Res$ as the residual stream. We denote $\Val$ as Value (states), $\Qry$ as Query (states), and $\Key$ as Key (states) in one attention head. The \attlogit~represents the value before the softmax operation and can be understood as the inner product between  $\Qry$  and  $\Key$. We use \Attn~to denote the attention weights of applying the SoftMax function to \attlogit, and ``attention map'' to describe the visualization of the heat map of the attention weights. When referring to the \attlogit~from ``$\tokenB$'' to  ``$\tokenA$'', we indicate the inner product  $\langle\Qry(\tokenB), \Key(\tokenA)\rangle$, specifically the entry in the ``$\tokenB$'' row and ``$\tokenA$'' column of the attention map.

\paragraph{Logit lens.} We use the method of ``Logit Lens'' to interpret the hidden states and value states \citep{belrose2023eliciting}. We use \logit~to denote pre-SoftMax values of the next-token prediction for LLMs. Denote \readout~as the linear operator after the last layer of transformers that maps the hidden states to the \logit. 
The logit lens is defined as applying the readout matrix to residual or value states in middle layers. Through the logit lens, the transformed hidden states can be interpreted as their direct effect on the logits for next-token prediction. 

\paragraph{Terminologies in two-hop reasoning.} We refer to an input like “\Src$\to$\brga, \brgb$\to$\Ed” as a two-hop reasoning chain, or simply a chain. The source entity $\Src$ serves as the starting point or origin of the reasoning. The end entity $\Ed$ represents the endpoint or destination of the reasoning chain. The bridge entity $\Brg$ connects the source and end entities within the reasoning chain. We distinguish between two occurrences of $\Brg$: the bridge in the first premise is called $\brga$, while the bridge in the second premise that connects to $\Ed$ is called $\brgc$. Additionally, for any premise ``$\tokenA \to \tokenB$'', we define $\tokenA$ as the parent node and $\tokenB$ as the child node. Furthermore, if at the end of the sequence, the query token is ``$\tokenA$'', we define the chain ``$\tokenA \to \tokenB$, $\tokenB \to \tokenC$'' as the Target Chain, while all other chains present in the context are referred to as distraction chains. Figure~\ref{fig:data_illustration} provides an illustration of the terminologies.

\paragraph{Input format.}
Motivated by two-hop reasoning in real contexts, we consider input in the format $\bos, \text{context information}, \query, \answer$. A transformer model is trained to predict the correct $\answer$ given the query $\query$ and the context information. The context compromises of $K=5$ disjoint two-hop chains, each appearing once and containing two premises. Within the same chain, the relative order of two premises is fixed so that \Src$\to$\brga~always precedes \brgb$\to$\Ed. The orders of chains are randomly generated, and chains may interleave with each other. The labels for the entities are re-shuffled for every sequence, choosing from a vocabulary size $V=30$. Given the $\bos$ token, $K=5$ two-hop chains, \query, and the \answer~tokens, the total context length is $N=23$. Figure~\ref{fig:data_illustration} also illustrates the data format. 

\paragraph{Model structure and training.} We pre-train a three-layer transformer with a single head per layer. Unless otherwise specified, the model is trained using Adam for $10,000$ steps, achieving near-optimal prediction accuracy. Details are relegated to Appendix~\ref{app:sec_add_training_detail}.


% \RZ{Do we use source entity, target entity, and mediator entity? Or do we use original token, bridge token, end token?}





% \paragraph{Basic notations.} We use ... We use $\ve_i$ to denote one-hot vectors of which only the $i$-th entry equals one, and all other entries are zero. The dimension of $\ve_i$ are usually omitted and can be inferred from contexts. We use $\indicator\{\cdot\}$ to denote the indicator function.

% Let $V > 0$ be a fixed positive integer, and let $\vocab = [V] \defeq \{1, 2, \ldots, V\}$ be the vocabulary. A token $v \in \vocab$ is an integer in $[V]$ and the input studied in this paper is a sequence of tokens $s_{1:T} \defeq (s_1, s_2, \ldots, s_T) \in \vocab^T$ of length $T$. For any set $\mathcal{S}$, we use $\Delta(\mathcal{S})$ to denote the set of distributions over $\mathcal{S}$.

% % to a sequence of vectors $z_1, z_2, \ldots, z_T \in \real^{\dout}$ of dimension $\dout$ and length $T$.

% Let $\mU = [\vu_1, \vu_2, \ldots, \vu_V]^\transpose \in \real^{V\times d}$ denote the token embedding matrix, where the $i$-th row $\vu_i \in \real^d$ represents the $d$-dimensional embedding of token $i \in [V]$. Similarly, let $\mP = [\vp_1, \vp_2, \ldots, \vp_T]^\transpose \in \real^{T\times d}$ denote the positional embedding matrix, where the $i$-th row $\vp_i \in \real^d$ represents the $d$-dimensional embedding of position $i \in [T]$. Both $\mU$ and $\mP$ can be fixed or learnable.

% After receiving an input sequence of tokens $s_{1:T}$, a transformer will first process it using embedding matrices $\mU$ and $\mP$ to obtain a sequence of vectors $\mH = [\vh_1, \vh_2, \ldots, \vh_T] \in \real^{d\times T}$, where 
% \[
% \vh_i = \mU^\transpose\ve_{s_i} + \mP^\transpose\ve_{i} = \vu_{s_i} + \vp_i.
% \]

% We make the following definitions of basic operations in a transformer.

% \begin{definition}[Basic operations in transformers] 
% \label{defn:operators}
% Define the softmax function $\softmax(\cdot): \real^d \to \real^d$ over a vector $\vv \in \real^d$ as
% \[\softmax(\vv)_i = \frac{\exp(\vv_i)}{\sum_{j=1}^d \exp(\vv_j)} \]
% and define the softmax function $\softmax(\cdot): \real^{m\times n} \to \real^{m \times n}$ over a matrix $\mV \in \real^{m\times n}$ as a column-wise softmax operator. For a squared matrix $\mM \in \real^{m\times m}$, the causal mask operator $\mask(\cdot): \real^{m\times m} \to \real^{m\times m}$  is defined as $\mask(\mM)_{ij} = \mM_{ij}$ if $i \leq j$ and  $\mask(\mM)_{ij} = -\infty$ otherwise. For a vector $\vv \in \real^n$ where $n$ is the number of hidden neurons in a layer, we use $\layernorm(\cdot): \real^n \to \real^n$ to denote the layer normalization operator where
% \[
% \layernorm(\vv)_i = \frac{\vv_i-\mu}{\sigma}, \mu = \frac{1}{n}\sum_{j=1}^n \vv_j, \sigma = \sqrt{\frac{1}{n}\sum_{j=1}^n (\vv_j-\mu)^2}
% \]
% and use $\layernorm(\cdot): \real^{n\times m} \to \real^{n\times m}$ to denote the column-wise layer normalization on a matrix.
% We also use $\nonlin(\cdot)$ to denote element-wise nonlinearity such as $\relu(\cdot)$.
% \end{definition}

% The main components of a transformer are causal self-attention heads and MLP layers, which are defined as follows.

% \begin{definition}[Attentions and MLPs]
% \label{defn:attn_mlp} 
% A single-head causal self-attention $\attn(\mH;\mQ,\mK,\mV,\mO)$ parameterized by $\mQ,\mK,\mV \in \real^{{\dqkv\times \din}}$ and $\mO \in \real^{\dout\times\dqkv}$ maps an input matrix $\mH \in \real^{\din\times T}$ to
% \begin{align*}
% &\attn(\mH;\mQ,\mK,\mV,\mO) \\
% =&\mO\mV\layernorm(\mH)\softmax(\mask(\layernorm(\mH)^\transpose\mK^\transpose\mQ\layernorm(\mH))).
% \end{align*}
% Furthermore, a multi-head attention with $M$ heads parameterized by $\{(\mQ_m,\mK_m,\mV_m,\mO_m) \}_{m=1}^M$ is defined as 
% \begin{align*}
%     &\Attn(\mH; \{(\mQ_m,\mK_m,\mV_m,\mO_m) \}_{m\in[M]}) \\ =& \sum_{m=1}^M \attn(\mH;\mQ_m,\mK_m,\mV_m,\mO_m) \in \real^{\dout \times T}.
% \end{align*}
% An MLP layer $\mlp(\mH;\mW_1,\mW_2)$ parameterized by $\mW_1 \in \real^{\dhidden\times \din}$ and $\mW_2 \in \real^{\dout \times \dhidden}$ maps an input matrix $\mH = [\vh_1, \ldots, \vh_T] \in \real^{\din \times T}$ to
% \begin{align*}
%     &\mlp(\mH;\mW_1,\mW_2) = [\vy_1, \ldots, \vy_T], \\ \text{where } &\vy_i = \mW_2\nonlin(\mW_1\layernorm(\vh_i)), \forall i \in [T].
% \end{align*}

% \end{definition}

% In this paper, we assume $\din=\dout=d$ for all attention heads and MLPs to facilitate residual stream unless otherwise specified. Given \Cref{defn:operators,defn:attn_mlp}, we are now able to define a multi-layer transformer.

% \begin{definition}[Multi-layer transformers]
% \label{defn:transformer}
%     An $L$-layer transformer $\transformer(\cdot): \vocab^T \to \Delta(\vocab)$ parameterized by $\mP$, $\mU$, $\{(\mQ_m^{(l)},\mK_m^{(l)},\mV_m^{(l)},\mO_m^{(l)})\}_{m\in[M],l\in[L]}$,  $\{(\mW_1^{(l)},\mW_2^{(l)})\}_{l\in[L]}$ and $\Wreadout \in \real^{V \times d}$ receives a sequence of tokens $s_{1:T}$ as input and predict the next token by outputting a distribution over the vocabulary. The input is first mapped to embeddings $\mH = [\vh_1, \vh_2, \ldots, \vh_T] \in \real^{d\times T}$ by embedding matrices $\mP, \mU$ where 
%     \[
%     \vh_i = \mU^\transpose\ve_{s_i} + \mP^\transpose\ve_{i}, \forall i \in [T].
%     \]
%     For each layer $l \in [L]$, the output of layer $l$, $\mH^{(l)} \in \real^{d\times T}$, is obtained by 
%     \begin{align*}
%         &\mH^{(l)} =  \mH^{(l-1/2)} + \mlp(\mH^{(l-1/2)};\mW_1^{(l)},\mW_2^{(l)}), \\
%         & \mH^{(l-1/2)} = \mH^{(l-1)} + \\ & \quad \Attn(\mH^{(l-1)}; \{(\mQ_m^{(l)},\mK_m^{(l)},\mV_m^{(l)},\mO_m^{(l)}) \}_{m\in[M]}), 
%     \end{align*}
%     where the input $\mH^{(l-1)}$ is the output of the previous layer $l-1$ for $l > 1$ and the input of the first layer $\mH^{(0)} = \mH$. Finally, the output of the transformer is obtained by 
%     \begin{align*}
%         \transformer(s_{1:T}) = \softmax(\Wreadout\vh_T^{(L)})
%     \end{align*}
%     which is a $V$-dimensional vector after softmax representing a distribution over $\vocab$, and $\vh_T^{(L)}$ is the $T$-th column of the output of the last layer, $\mH^{(L)}$.
% \end{definition}



% For each token $v \in \vocab$, there is a corresponding $d_t$-dimensional token embedding vector $\embed(v) \in \mathbb{R}^{d_t}$. Assume the maximum length of the sequence studied in this paper does not exceed $T$. For each position $t \in [T]$, there is a corresponding positional embedding  








% \cleardoublepage
% \subsection{Eval half 2}
% \input{prelim2.tex}
% \subsection{Plots}
% \begin{figure*}[htbp]
	\centering
	\includegraphics[width=0.4\linewidth]{plots/gf.greedyscatter.png }\quad
	\includegraphics[width=0.4\linewidth]{plots/hhl.greedyscatter.png}\quad
\caption{Time vs. Size plots}
\end{figure*}
\begin{figure*}[htbp]
	\centering
	\includegraphics[width=0.4\linewidth]{plots/qft.greedyscatter.png}\quad
	\includegraphics[width=0.4\linewidth]{plots/vqe.greedyscatter.png}
\caption{Time vs. Size plots}
\end{figure*}
\begin{figure*}[htbp]
	\centering
	\includegraphics[width=0.4\linewidth]{plots/grover.greedyscatter.png}\quad
	\includegraphics[width=0.4\linewidth]{plots/qaoa.greedyscatter.png}\quad
\caption{Time vs. Size plots}
\end{figure*}
\begin{figure*}[htbp]
	\centering
	\includegraphics[width=0.4\linewidth]{plots/shor.greedyscatter.png}
\caption{Time vs. Size plots}
\end{figure*}

% \cleardoublepage
% \subsection{Cliff Eval half 1}
% \input{prelim1.clifft}
% \cleardoublepage
% \subsection{Cliff Eval half 2}
% \input{prelim2.clifft}
% \subsection{Plots Cliff}
% \begin{figure*}[htbp]
	\centering
	\includegraphics[width=0.4\linewidth]{plots/gf.greedyscatter.clifft.png }\quad
\caption{Time vs. Size plots}
\end{figure*}
\begin{figure*}[htbp]
	\centering
	\includegraphics[width=0.4\linewidth]{plots/qft.greedyscatter.clifft.png}\quad
	\includegraphics[width=0.4\linewidth]{plots/vqe.greedyscatter.clifft.png}
\caption{Time vs. Size plots}
\end{figure*}
\begin{figure*}[htbp]
	\centering
	\includegraphics[width=0.4\linewidth]{plots/grover.greedyscatter.clifft.png}\quad
	\includegraphics[width=0.4\linewidth]{plots/qaoa.greedyscatter.clifft.png}\quad
\caption{Time vs. Size plots}
\end{figure*}
\begin{figure*}[htbp]
	\centering
	\includegraphics[width=0.4\linewidth]{plots/shor.greedyscatter.clifft.png}
\caption{Time vs. Size plots}
\end{figure*}


