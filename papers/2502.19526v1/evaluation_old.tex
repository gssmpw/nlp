
In this section,
we evaluate the \coam{} algorithm by instantiating it with three oracle
optimizers (\feyntool{}, \quartz{}, and \queso{}), as outlined in \secref{impl}.
%
We denote the resulting optimizers as
\coamwith{\feyntool}, \coamwith{\quartz}, and \coamwith{\queso} respectively,
and describe how these are configured in \secref{methodology}.
%
Our evaluation covers 21 quantum circuits from seven quantum algorithms.
%
We organize the results into the following parts:
%
\begin{itemize}
  \item In \secref{comp-feyn}, we evaluate our \coamwith{\feyntool} optimizer
   on T count reduction.
  %
  We observe that the $\Omega-$optimal circuits produced by our optimizer
  match the output quality of \feyntool{},
  suggesting that $\Omega-$optimality is good optimization criterion for T count optimization.
  %
  Our optimizer runs faster than \feyntool{} in almost all cases
  and we observe that the speedup increases with circuit size.
  %

  \item In \secref{comp-quartz} and \secref{comp-queso},
  we evaluate our optimizers \coamwith{\quartz} and \coamwith{\queso{}}
  for optimizing gate count.
  %
  We observe that the \coam{} algorithm effectively utilizes the base optimizers,
  delivering significant size reduction across a range of quantum circuits.

  \item In \secref{var-segment}, we study the impact of $\Omega$ on the output quality and running time of \coam{}.
\end{itemize}


% the integration of the implementation with \quartz{}, an off-the-shelf optimizer,
% and present an evaluation.
% %
% For the evaluation, we consider the time performance and scalability
% as well as the circuit quality on a range of quantum circuits.
% %
% As our baseline, we consider the \quartz{}~\cite{quartz-2022}
% optimizer, a super optimizer for quantum circuits that uses rule-based
% substitutions to optimize circuits.








%
% We also demonstrate the applicability of our implementation to the
% QUESO optimizer~\cite{queso-2023}, which also uses rule-based optimizations
% with different algorithms and heuristics.
%

%
% Because of the exponential time complexity, both approaches
% deliver poor time scalability, and in practice, provide no quality guarantee.
%
% With the \coam{} algorithm,
% our goal is to deliver a quality guarantee in reasonable running times
% that scale linearly with the size of circuits.
%


\subsection{Methodology}

\section{\label{sec:method}Methodology}

Each SIEM system uses its own RDL to define threat detection rules, and each RDL has its own schema.
For example, the Splunk SIEM uses the SPL to define its threat detection rules.
The task of understanding threat detection rules and recommending relevant MITRE ATT\&CK techniques (or sub-techniques) requires complex reasoning skills.
In the case of LLMs, this can be achieved with a technique called prompt chaining in which each task is divided into multiple sub-tasks in order to understand the complex reasoning behind the task.
Therefore, we employ a multi-phase architecture based on prompt chaining that leverages the power of LLMs to take a SIEM rule defined in any RDL and map it to relevant MITRE ATT\&CK techniques using the power of LLMs.
Our approach is based on the following intuitions:
\begin{itemize}[nosep,leftmargin=*]
    \item \textit{LLMs' implicit knowledge}: LLMs possess deep understanding of diverse RDLs. This enables them to interpret any rule, regardless of the RDL it is defined in, and convert it into comprehensible natural language text.
    \item \textit{LLMs' similarity comparison capability}: LLMs are adept at analyzing and comparing textual descriptions. 
    They can intelligently assess the similarity between two textual inputs to establish a meaningful connection.
\end{itemize}

\methodName has two main phases: (1) the rule to text translation phase, and (2) the MITRE ATT\&CK techniques recommendation phase.
These two phases in the pipeline include six key steps to determine relevant TTPs, as illustrated in Figure~\ref{fig:r2t}.

Although LLMs excel at translating SIEM rules into natural language, they often lack critical domain-specific contextual information related to IoCs in the rules.
To overcome this limitation, the \textit{rule to text translation} phase includes three steps: IoC extraction, contextual information retrieval, and natural language translation.

The workflow begins with the extraction of IoCs from the rules (for example, processes, log source, event codes, and file names) that the rule searches for in the logs (step (1)).In the next sstep a web search agent performs the task of obtaining additional contextual information about the IoCs discovered ((step 2)).
By incorporating this additional domain-specific information, the pipeline enhances the language translation, resulting in a more accurate and meaningful interpretation of SIEM rules.
The rule itself and the IoCs' contextual additional information from the previous stage are then used to translate the rule from RDL to natural language (step (3)).

The \textit{MITRE ATT\&CK techniques} recommendation phase of the pipeline includes the following three steps.
The rule in processed in data source identification step in which the probable origin of the data is identified (step (4)).
The description of the rule is then used to determine probable MITRE ATT\&CK techniques based on the implicit knowledge of the LLM (step (5)).
Finally, using chain-of-thought~\cite{wei2022chain} prompting, the most relevant techniques are extracted from the list of probable techniques (step (6)).
Each step of our method is further described in detail below.


% [bb=0 0 1440 900,width=1.43\linewidth,height=0.9\textwidth]
\begin{figure*}[htbp]
   \includegraphics[width=\textwidth]{Images/stages.jpg}
    
   \caption{An illustration of the different steps in \methodName.}
   \label{fig:stages}
\end{figure*} 

\subsection{IoC Extraction}
The context associated with a SIEM detection rule is crucial for its accurate interpretation and effective application. 
Obtaining this contextual understanding requires comprehensive analysis of the embedded IoCs in the SIEM rule.
In the first step, \methodName systematically identifies and extracts all IoCs, identifying the types of IoCs and their corresponding values that form the foundational elements of the detection rules. 
Leveraging the LLM's inherent understanding of rule structures and IoCs, we employ a zero-shot prompting approach for this task. 
Zero-shot prompting enables the direct extraction of IoCs from the rules without requiring extensive pre-training on specific datasets.

\noindent The result of this stage is a dictionary structure, where:
\begin{itemize}[nosep,leftmargin=*]
    \item Keys represent types of IoC, such as processes, files, IP addresses, and log sources.
    \item Values are lists containing specific IoC details, such as process names, file names, IP addresses, and log source identifiers.
\end{itemize}

In the example depicted in Figure~\ref{fig:stages}(a), the pipeline processes a rule for which relevant MITRE ATT\&CK techniques need to be recommended. 
The IoC extractor LLM produces a dictionary structure as output, organizing the IoCs in a structured format to support subsequent stages in the analysis pipeline. 



\subsection{Contextual Information Retrieval}
In this step, an LLM agent is employed to retrieve relevant information pertaining to the IoCs extracted from the rule.
A REACT agent~\cite{react} was used in this case to generate both reasoning traces and task-specific actions in an interleaved manner.
REACT agents interact with external tools to retrieve additional information that leads to more factual and reliable responses.
The LLM agent conducts a systematic search across web resources to gather additional contextual information for each IoC value present in the rule. 
This step addresses LLMS' lack of up-to-date knowledge or specialized domain expertise (which is critical to understanding the role and significance of the IoCs in the rule), without the need for retraining or fine-tuning.
Figure~\ref{fig:stages}(b) presents an example in which the rule includes the process name \texttt{soaphound.exe} as an IoC.
As can be seen, the web search results indicate that \texttt{soaphound.exe} is being used for active directory (AD) enumeration, which is important for the understanding of the attack. 

\subsection{Natural Language Translation}

The translation of detection rules into natural language textual descriptions fulfills three key objectives:
\begin{enumerate}[nosep,leftmargin=*]
    \item \textbf{Ensures that \methodName is format-agnostic}: It converts rules defined in various RDL formats into a generic, unstructured text format, ensuring compatibility with different SIEM systems, regardless of the specific rule format.
    \item \textbf{Provides contextual explanation}: It includes all relevant contextual information to produce a concise and comprehensible explanation of the rule.
    \item \textbf{Enhances the comprehension for LLMs}: It enables LLMs to more effectively compare the translated rule with descriptions in the MITRE ATT\&CK framework by providing a unified textual representation.
\end{enumerate}
To achieve these objectives, a zero-shot prompting technique is employed. 
The input to the LLM comprises two components:
\begin{itemize}
    \item \textbf{Syntactical information}: The rule itself, providing the structural and operational details.
    \item \textbf{Contextual information}: Details of the IoCs extracted from the rule, providing semantic insights into the rule's intent and function.
\end{itemize}
The LLM utilizes these inputs to generate a natural language textual description of the rule. 
This transformation not only ensures a more interpretable representation but also facilitates further steps of analysis and comparison, particularly in aligning the rule with MITRE ATT\&CK techniques and sub-techniques.



\subsection{Data Source or Mitigation Identification}
Identifying the most relevant data component or mitigation associated with the rule description in this step is critical for filtering out irrelevant MITRE ATT\&CK techniques (or sub-techniques) in subsequent steps of the pipeline.
In the MITRE ATT\&CK framework, data sources represent various categories of information that can be gathered from sensors or logs. 
These data sources include \textit{data components}, which are specific attributes or properties within a data source that are directly relevant to detecting a particular technique or sub-technique~. 
For example, in the context of the rule described in Figure~\ref{fig:stages}(a), the term \texttt{Endpoint.Processes} indicates that the activity is happening on an endpoint. 
Presence of the terms such as, \texttt{soaphound.exe}, \texttt{--buildcache}, \texttt{--certdump} and etc. indicate that the rule searches for command line execution of an executable named \texttt{soaphound.exe} with specific parameters. 
Therefore, the appropriate data source in this example is \textit{Command}, with the corresponding data component being \textit{Command Execution}.
Additionally, \textit{mitigations} are defined as categories of technologies or strategies that can prevent or reduce the impact of specific techniques or sub-techniques. 
The MITRE ATT\&CK framework explicitly establishes relationships between data components, mitigations, and techniques (or sub-techniques), enabling a systematic approach for identifying relevant elements.

To identify the most relevant data component or mitigation associated with a given rule description, we utilize agentic retrieval augmented generation (RAG), which incorporates an AI Agent-based implementation of the RAG framework.
Data from the MITRE ATT\&CK framework, specifically related to data components and mitigations, is stored in a vector database (e.g., ChromaDB). 
The process begins with the rule description from the previous stage, which serves as the input to the AI Agent. 
The LLM-powered agent automatically generates a search query tailored to retrieve relevant information from the RAG database.

For each query, the system retrieves the five most similar documents from the database, each containing contextual information about data components or mitigations. 
These documents are then utilized by the LLM agent to contextualize the rule description. 
By comparing the content of these retrieved documents with the rule description, the LLM agent determines and outputs the most relevant data component or mitigation along with a chain-of-thought as to why the data component or mitigation is related to the rule.


\subsection{Probable Technique Recommendation}

In this step, an LM agent is utilized to propose probable MITRE ATT\&CK techniques (and sub-techniques) that may be relevant to the description of the provided rule.
We used a REACT agent in this step as well to utilize both implicit and explicit knowledge during reasoning.
For explicit knowledge, the agent searches the MITRE ATT\&CK framework to obtain the list of probable techniques (and sub-techniques).
The natural language description of the rule from the previous step serves as input to the LLM agent.
The output of this stage consists of a list of JSON objects, each containing the MITRE technique ID, technique name, and technique description as seen in Figure~\ref{fig:stages}(c).

Throughout our experiments, we observed that as the number of recommendations ($k$) increases, both the framework's average recall and precision initially improve, however beyond a certain threshold of $k$, the %average 
precision begins to decline.
Based on these observations(please refer Table~\ref{tab:results3}), we selected a $k$-value of 11 to ensure a high recall.



\subsection{Relevant Technique Extraction}
In this step, \methodName refines the set of probable MITRE ATT\&CK techniques identified in the previous stage by eliminating irrelevant entries.
This step in the pipeline serves two primary purposes: (1) to enhance precision while maintaining recall achieved in previous step, and (2) to provide a clear rationale for the selection of the labels, ensuring transparency and interpretability of the mapping process.
This refinement process is grounded in the assumption that LLMs are effective for text similarity matching tasks.

The process comprises two key steps:
\begin{itemize}
    \item \textit{Rule-technique comparison}: The description of each technique in the set of probable techniques is compared with the rule description. 
    A chain-of-thought technique is then applied to elucidate the reasoning behind the association of each technique with the rule.
    \item \textit{Confidence calculation}: The generated chain-of-thought rationale for each technique (or sub-technique) is compared with the rule description to compute a relevance (or confidence) score, as done in prior work~\cite{freitas2024ai}.
    % \item \textbf{Reasoning}: \new{Add here the reasoning that it provides - explaining in NLP why it was selected...}
\end{itemize}

Techniques with higher confidence scores are deemed more relevant to the rule. 
Conversely, techniques with scores falling below a predefined threshold are excluded.
The techniques retained after this filtering step represent the most relevant techniques corresponding to the given rule's description. 


The chain-of-thought (CoT) rationale generated during the comparison of each rule to its probable technique is also provided as an output in this step.
This rationale offers a detailed natural language explanation, articulating why a particular technique is relevant to the given rule. 
Such explanations are highly valuable for security analysts, as they provide clear and transparent reasoning behind the mapping, enabling analysts to better understand and validate the association between the rule and the technique.
Other classification models proposed in previous works within this domain also suffer from the limitation of being black-box models, which lack the ability to provide clear reasoning or explanations. 
Unlike \methodName, these models fail to generate transparent, CoT rationales that explain why a particular rule is mapped to a specific technique, making them less interpretable and less useful for security analysts.

\subsection{Benchmarks}
% \myparagraph{Benchmarks.}
\begin{table*}[tb]
\centering
\caption{IOPS Results}
\label{tab:iops}
\begin{tabular}{|l|r|r|r|r|r|r|r|r|} 
\hline
\multirow{2}{*}{}   & \multicolumn{2}{c|}{Upstream} & \multicolumn{2}{c|}{Ublk Frontend} & \multicolumn{2}{c|}{C-R Comm.} & \multicolumn{2}{c|}{DBS} \\
\cline{2-9}
                & Read & Write                      & Read & Write                      & Read & Write                      & Read & Write \\ 
\hline
Full engine     & \textbf{17k}  & \textbf{13k}      & 95k  & 27k                        & 110k & 27k                        & \textbf{112k} & \textbf{115k} \\ 
\hline
Without storage & 19k  & 19.5k                      & 100k & 100k                       & \textbf{129k} & \textbf{115k}     & \multicolumn{2}{l|}{$\rightarrow$} \\ 
\hline
Frontend only   & 20k  & 20k                        & \textbf{280k} & \textbf{255k}     & \multicolumn{4}{l|}{$\rightarrow$} \\
\hline
\end{tabular}
\end{table*}

\begin{table*}[tb]
\centering
\caption{Bandwidth Results (MB/s)}
\label{tab:bandwidth}
\begin{tabular}{|l|r|r|r|r|r|r|r|r|} 
\hline
\multirow{2}{*}{}   & \multicolumn{2}{c|}{Upstream} & \multicolumn{2}{c|}{Ublk Frontend} & \multicolumn{2}{c|}{C-R Comm.} & \multicolumn{2}{c|}{DBS} \\
\cline{2-9}
                & Read & Write                      & Read & Write                      & Read & Write                      & Read & Write \\ 
\hline
Full engine     & \textbf{300}  & \textbf{275}      & 1000 & 1000                       & 1250 & 1250                       & \textbf{1250}& \textbf{1250} \\ 
\hline
Without storage & 670  & 415                        & 1250 & 1250                       & \textbf{1250}  & \textbf{1250}    & \multicolumn{2}{l|}{$\rightarrow$} \\ 
\hline
Frontend only   & 750  & 415                        & \textbf{2000} & \textbf{2000}     & \multicolumn{4}{l|}{$\rightarrow$} \\
\hline
\end{tabular}
\end{table*}


\section{Evaluation}

\subsection{Setup}

To evaluate the system we deploy the controller and replica on separate nodes. In contrast to our development setup, this allows us to account for any effects related to the physical network. We use two identical nodes from our local cluster equipped with dual Intel Xeon E5-2620v2 CPUs, 128 GB RAM, connected via 10 Gbps Ethernet. Data is stored in a Samsung PM1733 NVMe drive. These machines have limited CPU core counts compared to state-of-the-art technology, however their specifications and performance is better aligned with typical VM offerings currently available in the cloud.

Actually, we initially evaluated our system in AWS, using two c5d.2xlarge EC2 instances, a cost-efficient option also used by Longhorn developers for publishing their benchmarks \cite{longhorn_report}.
However, EC2 instances have limited maximum provisioned IOPS, regardless of the hard drive used. As expected, the software performed to the machine's limit, reaching AWS's 40k IOPS cap. Overcoming these limitations in AWS requires using significantly more expensive EC2 instances. To harness the performance of the features described in this work, deployments must use instances and volumes at least 5 times more expensive than c5d.2xlarge. In any case, the behavior of the system on higher-performance cloud nodes is similar to the one we observe in our local setup.

\subsection{Results}

The results for IOPS and bandwidth are presented in Tables \ref{tab:iops} and \ref{tab:bandwidth} respectively. As in the previous section, in each experiment we do multiple runs to measure (shown as table rows): 
\begin{enumerate*}[label=(\roman*)]
    \item the \textit{full engine} performance, end-to-end, which includes writing blocks to the disk,
    \item the performance up to the replica \textit{without storage} using a null storage drive, where I/Os are immediately completed at the replica, and
    \item the \textit{frontend only} using a null backend, where I/Os are immediately completed at the controller.
\end{enumerate*}
We follow the same top-down approach, starting from upstream Longhorn and integrating each new feature in the same progression (shown as table columns). This indicates where the bottleneck is in each step and highlights how each solution (ublk frontend, controller-replica communication, DBS) contributes to the performance of the whole system.
For our experiments we use fio to measure IOPS (4k, random) and bandwidth (1 MB, sequential). All I/Os are direct to the virtual block device, bypassing kernel caches.

Starting with 17k/13k read/write IOPS when running the full stack, we isolate the first bottleneck at the frontend and measure it cannot achieve more than 20k IOPS using the upstream TGT-based solution. Integrating the ublk frontend yields a 28x boost, that allows the next layers of the engine to perform better, except for write IOPS, where the storage scheme of Longhorn fails to take advantage of the faster frontend. The bandwidth is also largely affected by the new frontend (almost 2.5x/4.8x for reads/writes compared to upstream), which enables the system to almost saturate the 10 Gbps links.
In general, these numbers support the general consensus among the community that ublk is an ideal framework to export SDS stacks to applications.

The next step is to evaluate the improved controller-replica communication implementation. Even with the ublk frontend, upstream Longhorn cannot achieve over 100k IOPS going up to the replica (null storage). Our modified communication scheme boosts performance by 29\%/15\% for reads/writes. There is still room for improvement, as we would ideally like to match the frontend performance, however it is enough for the engine to reach the full 10 Gbps bandwidth even with the default storage backend. The updated controller-replica communication also boosts random read IOPS to storage by 17\%. Write IOPS are not affected, which indicates that the bottleneck is in the storage backend itself (other runs have proved that this limitation is caused by write versioning). Lastly, integration of the DBS backend raises write IOPS to the level of reads; the whole modified system now performs an order of magnitude better than the default. Note that DBS is designed to keep the same performance level regardless of the number of volume snapshots.

% OLD TEXT

% Following the implementation of each feature, we measured its performance individually and observed a minimum performance boost resulting from improved controller-replica communication and DBS integration. Our findings indicate that each layer's bottleneck and solution proposed are crucial, so the next layer provides a boost in performance. So in our benchmarks, we used a top-down approach, integrating each feature step by step to highlight its importance. We also split our benchmarks into three parts, disabling the functionality of the engine in certain parts in order to benchmark each feature separately.
 
% Our setup included one controller and one replica, without the use of snapshots. We tested these changes in our local cluster using two identical systems equipped with Intel(R) Xeon(R) CPU E5-2620 v2 @ 2.10GHz, 125GB memory, and … hard drives. The machines were connected via a 10 Gbps connection, which is enough for some experiments but limits the performance during some bandwidth benchmarks.

% \subsection{Frontend}

% During our frontend evaluation experiments, we used our integration and compared it with the live version of the Longhorn engine. The findings can be seen in Table 1. 
% We can see there is a big boost in the frontend's performance using the ublk framework. The first layer of the engine saw a 28x boost in performance, both on reads and writes IOPS. Regarding the next two layers of the engine, we saw a big boost even using the full engine, except for write IOPS, where the Longhorn's IO methods fail to take advantage of the faster frontend. The bandwidth seems to get a boost too, especially the frontend part. 

% Overall these numbers support our belief that ublk is a modern and more performant framework to implement software-defined storage frontends. 

% \begin{table*}[!th]
\centering
\resizebox{\textwidth}{!}{%
\begin{tabular}{@{}llcccccccccc@{}}
\toprule
& & \multicolumn{2}{c}{\textbf{Intent Detection}} & \multicolumn{2}{c}{\textbf{Topic Mining}} & \multicolumn{2}{c}{\textbf{Domain Discovery}} & \multicolumn{1}{c}{\textbf{Type}} & \multicolumn{1}{c}{\textbf{Emotion}} & \\
\cmidrule(lr){3-4} \cmidrule(lr){5-6} \cmidrule(lr){7-8} \cmidrule(lr){9-9} \cmidrule(lr){10-10}  %\cmidrule(lr){11-11}
\textbf{Model} & \textbf{Method} & \textbf{BANKING} & \textbf{CLINC} & \textbf{Reddit} & \textbf{StackEx} & \textbf{MTOP} & \textbf{CLINC(D)} & \textbf{FewEvent} & \textbf{GoEmotion} & \textbf{AVG} \\ \midrule \midrule
GPT-4o-mini & Standard Prompting & 0.652 & 0.792 & 0.534 & 0.482 & 0.896 & 0.536 & 0.630 & 0.378 & 0.613 \\
& Self-Consistency & 0.666 & 0.802 & 0.586 & 0.494 & 0.902 & 0.530 & 0.640 & 0.382 & 0.625 \\
& TestNUC & 0.712 & 0.858 & 0.614 & 0.528 & 0.936 & 0.544 & 0.674 & 0.410 & 0.660 \\
& \cellcolor{gray!18}TestNUC\textdagger & \cellcolor{gray!18}\textbf{0.764} & \cellcolor{gray!18}\textbf{0.864} & \cellcolor{gray!18}\textbf{0.646} & \cellcolor{gray!18}\textbf{0.540} & \cellcolor{gray!18}\textbf{0.948} & \cellcolor{gray!18}\textbf{0.554} & \cellcolor{gray!18}\textbf{0.680} & \cellcolor{gray!18}\textbf{0.414} & \cellcolor{gray!18}\textbf{0.676} \\ \midrule \midrule
Llama-3.1-8B & Standard Prompting & 0.572 & 0.726 & 0.502 & 0.492 & 0.892 & 0.528 & 0.530 & 0.332 & 0.572 \\
& Self-Consistency & 0.620 & 0.774 & 0.564 & 0.526 & 0.902 & 0.518 & 0.564 & 0.340 & 0.601 \\
& TestNUC & 0.694 & 0.806 & 0.618 & 0.558 & 0.934 & 0.528 & 0.596 & 0.356 & 0.636 \\
& \cellcolor{gray!18}TestNUC\textdagger & \cellcolor{gray!18}\textbf{0.724} & \cellcolor{gray!18}\textbf{0.812} & \cellcolor{gray!18}\textbf{0.646} & \cellcolor{gray!18}\textbf{0.576} & \cellcolor{gray!18}\textbf{0.940} & \cellcolor{gray!18}\textbf{0.542} & \cellcolor{gray!18}\textbf{0.614} & \cellcolor{gray!18}\textbf{0.360} & \cellcolor{gray!18}\textbf{0.652} \\ \midrule \midrule
Claude-3-Haiku & Standard Prompting & 0.680 & 0.848 & 0.486 & 0.564 & 0.892 & 0.552 & 0.594 & 0.336 & 0.619 \\
& Self-Consistency & 0.702 & 0.870 & 0.510 & 0.578 & 0.904 & 0.564 & 0.568 & 0.350 & 0.631 \\
& TestNUC & 0.762 & 0.894 & 0.596 & 0.588 & 0.940 & 0.590 & 0.620 & 0.348 & 0.667 \\
& \cellcolor{gray!18}TestNUC\textdagger & \cellcolor{gray!18}\textbf{0.804} & \cellcolor{gray!18}\textbf{0.902} & \cellcolor{gray!18}\textbf{0.612} & \cellcolor{gray!18}\textbf{0.600} & \cellcolor{gray!18}\textbf{0.946} & \cellcolor{gray!18}\textbf{0.622} & \cellcolor{gray!18}\textbf{0.660} & \cellcolor{gray!18}\textbf{0.368} & \cellcolor{gray!18}\textbf{0.689} \\ \midrule \midrule
GPT-4o & Standard Prompting & 0.746 & 0.924 & 0.712 & 0.674 & 0.962 & 0.614 & 0.682 & 0.406 & 0.715 \\
& Self-Consistency & 0.758 & 0.922 & 0.720 & 0.688 & 0.958 & 0.624 & 0.696 & 0.426 & 0.724 \\
&TestNUC & 0.804 & 0.934 & 0.744 & \textbf{0.710} & 0.974 & 0.644 & 0.692 & 0.446 & 0.744 \\
& \cellcolor{gray!18}TestNUC\textdagger & \cellcolor{gray!18}\textbf{0.824} & \cellcolor{gray!18}\textbf{0.940} & \cellcolor{gray!18}\textbf{0.750} & \cellcolor{gray!18}\textbf{0.710} & \cellcolor{gray!18}\textbf{0.978} & \cellcolor{gray!18}\textbf{0.654} & \cellcolor{gray!18}\textbf{0.708} & \cellcolor{gray!18}\textbf{0.464} & \cellcolor{gray!18}\textbf{0.754} \\
\bottomrule
\end{tabular}%
}
\caption{Accuracy comparison with Standard Prompting and Self-Consistency across four diverse LLMs. TestNUC consistently improves the inference performance on all benchmark datasets. $\dagger$ denotes that 50 neighbors are utilized.}
\label{tab:main_compare_sc}
\end{table*}
% % \begin{table*}[t]
%     \caption{Performance Metrics on MMQA Subset after the attack with poisoned images cointaining adversarial noise. Capt. stands for Captions and denotes wether the re-ranker has access to image captions. }
%     \label{tab:mmqa_adv}
%     \centering
%     \resizebox{\textwidth}{!}{%
%     \begin{tabular}{@{}lc|cc|cc@{}}
%         \toprule
%         \textbf{Retriever} & \textbf{Reranker} & \textbf{Capt.} & \multicolumn{2}{c|}{\textbf{Original} (\small{Before $\rightarrow$ After})} & \multicolumn{2}{c@{}}{\textbf{Poisoned}} \\ 
%         \textbf{CLIP} & \textbf{LLaVA} &  & \textbf{Recall (\%)} & \textbf{Accuracy (\%)} & \textbf{Recall (\%)} & \textbf{Accuracy (\%)} \\
%         \midrule
%         $K=1$   & \xmark                      & -           & 83.7 $\rightarrow$ 11.4 \textcolor{red}{\small{(-72.3)}}             & 61.0 $\rightarrow$ 18.4 \textcolor{red}{\small{(-42.6)}}                 & 87.9   &  58.2  \\
%         $K=5$   & \xmark                      & -           & 92.9 $\rightarrow$ 92.2 \textcolor{red}{\small{(-0.7)}}           & 39.0 $\rightarrow$ 30.5 \textcolor{red}{\small{(-8.5)}}                   & 100.0   & 36.2         \\
%         $K=1$   & $N=1$          & \cmark      & 84.4 $\rightarrow$ 30.5 \textcolor{red}{\small{(-53.9)}}                    & 62.4 $\rightarrow$ 29.1 \textcolor{red}{\small{(-33.3)}}          & 63.8   & 48.2         \\
%         $K=5$   & $N=1$          & \xmark      & 70.2 $\rightarrow$ 44.0 \textcolor{red}{\small{(-26.2)}}                    & 58.9$ \rightarrow$ 39.7 \textcolor{red}{\small{(-19.2)}}          & 45.4  & 38.3          \\
%         \bottomrule
%     \end{tabular}%
%     }
% \end{table*}

\begin{table*}[t]
    \caption{Performance Metrics on MMQA Subset after the attack with poisoned images cointaining adversarial noise. Capt. stands for Captions and denotes wether the re-ranker has access to image captions. }
    \label{tab:mmqa_adv}
    \centering
    \resizebox{\textwidth}{!}{%
    \begin{tabular}{@{}cllcccccccc@{}}
        \toprule
       \multicolumn{11}{c}{\textbf{Retriever}: CLIP-ViT-L \textbf{Reranker}: LLaVA \textbf{Generator}: LLaVA} \\
       \midrule
       & \textbf{Retriever} & \textbf{Reranker} & \textbf{Capt.} & \multicolumn{3}{c|}{\textbf{Original Recall (\%)}} & \multicolumn{3}{c}{\textbf{Original Accuracy (\%)}} & Poisoned \\
       & &  &  & Before & After & Change & Before & After & Change & Recall (\%) \\
        $K=1$   & \xmark                      & -           & 83.7 $\rightarrow$ 11.4 \textcolor{red}{\small{(-72.3)}}             & 61.0 $\rightarrow$ 18.4 \textcolor{red}{\small{(-42.6)}}                 & 87.9   &  58.2  \\
        $K=5$   & \xmark                      & -           & 92.9 $\rightarrow$ 92.2 \textcolor{red}{\small{(-0.7)}}           & 39.0 $\rightarrow$ 30.5 \textcolor{red}{\small{(-8.5)}}                   & 100.0   & 36.2         \\
        $K=1$   & $N=1$          & \cmark      & 84.4 $\rightarrow$ 30.5 \textcolor{red}{\small{(-53.9)}}                    & 62.4 $\rightarrow$ 29.1 \textcolor{red}{\small{(-33.3)}}          & 63.8   & 48.2         \\
        $K=5$   & $N=1$          & \xmark      & 70.2 $\rightarrow$ 44.0 \textcolor{red}{\small{(-26.2)}}                    & 58.9$ \rightarrow$ 39.7 \textcolor{red}{\small{(-19.2)}}          & 45.4  & 38.3          \\
        \bottomrule
    \end{tabular}%
    }
\end{table*}




% \subsection{Controller-Replica communication}

% After integrating the ublk frontend, we also included the improved controller-replica communication in our experiments. We use the same three breakpoints to evaluate the solution. 

% This improvement was focused on improving the server-client architecture, and our experiments indicate that we manage this. Compared to the longhorn live version, this approach serves requests with 10x IOPS. This approach does not match the performance gain on the front-end but is still a solid solution. We also measured a big boost in the bandwidth performance to numbers that are the limit of the connection used between the machines. Using a better connection, this number may see a small improvement based on some experiments performed in an environment where the replica and the controller were on the same machine, removing the network limitations.

% Using this solution, the Longhorn's backend managed to achieve a boost in performance. Random reads have seen a boost similar to the previous layer when writes double their performance compared with the previous solution. These findings seem to solidify our theory that the Longhorn backend needs a new approach.

% \subsection{Block storage}
% Our last integration was the DBS backend for the replicas IO. This method, despite being in early development, seems to perform better regarding IOPS. This modern solution managed to match the rest of the system's performance, achieving 10x IOPS compared to Longhorn's backend option on writes where Longhorn's implementation failed to keep up with the rest of the system.

% \textbf{DBS performance on vaccum}
% \begin{table}[tbp]
  \caption{Performance of GPT-4o and Mistral-7B-Instruct-v0.3 as listwise LLM re-rankers on the NevIR test set with few-shot prompting.}
  \label{tab:few_shot}
  \centering
  \begin{tabular}{lcc}
    \toprule
    \textbf{Model Name} & \textbf{Shots} & \textbf{NevIR Score} \\
    \midrule
    \multirow{4}{*}{GPT-4o} & Zero-shot & 70.1\% \\
                             & 1-shot    & 72.0\% \\
                             & 3-shot    & 74.5\% \\
                             & 5-shot    & 76.9\% \\
    \midrule
    \multirow{4}{*}{Mistral-7B-Instruct-v0.3} & Zero-shot & 46.3\% \\
                                              & 1-shot    & 42.6\% \\
                                              & 3-shot    & 37.1\% \\
                                              & 5-shot    & 39.0\% \\
    \bottomrule
  \end{tabular}
\end{table}


    

% \begin{table*}[!htbp]
\caption{Performance comparison of semi-supervised learning across methodologies.}
\label{table:table4}
\centering
\resizebox{\textwidth}{!}{
\renewcommand{\arraystretch}{1.2}
{\tiny
\begin{tabular}{c|c|c|ccc|ccc|ccc} 
\hline
\multicolumn{12}{c}{\textit{1\% of labeled data}}                                                                                                                                                                                                                                                                                                                                                                                                                                                                                                     \\ 
\hline
\multirow{2}{*}{\begin{tabular}[c]{@{}c@{}}\textit{Model}\\\textit{Name}\end{tabular}}                                                   & \multirow{2}{*}{\begin{tabular}[c]{@{}c@{}}\textit{Training}\\\textit{Type}\end{tabular}} & \multirow{2}{*}{\textit{Modality (Count)}} & \multicolumn{3}{c|}{\begin{tabular}[c]{@{}c@{}}\textit{Sleep Stage}\\\textit{Classification}\end{tabular}} & \multicolumn{3}{c|}{\begin{tabular}[c]{@{}c@{}}\textit{Apnea}\\\textit{Detection}\end{tabular}} & \multicolumn{3}{c}{\begin{tabular}[c]{@{}c@{}}\textit{Hypopnea}\\\textit{Detection}\end{tabular}}  \\ 
\cline{4-12}
                                                                                       &                                                                                            &                                            & \textit{ACC}   & \textit{MF1}   & \textit{K}                                                                & \textit{ACC}   & \textit{MF1}   & \textit{K}                                                    & \textit{ACC}   & \textit{MF1}   & \textit{K}                                                       \\ 
\hline
SalientSleepNet \cite{ref9}                                                                       & Supervised                                                                                 & EEG1
  + EOG1                              & 61.44          & 53.49          & 0.49                                                                      & 79.34          & 45.32          & 0.02                                                          & 50.22          & 39.65          & 0.01                                                             \\
SleepFM \cite{ref23}                                                                                & SSL                                                                                        & EEG1
  + EOG1                              & 70.14          & 61.49          & 0.60                                                                      & 93.48          & 52.02          & 0.07                                                          & 50.95          & 39.77          & 0.00                                                             \\
SynthSleepNet                                                                          & SSL                                                                                        & EEG1
  + EOG1                              & 78.04          & 65.61          & 0.70                                                                      & 95.39          & 54.08          & 0.10                                                          & 63.87          & 49.33          & 0.10                                                             \\
SynthSleepNet+TCM                                                                      & SSL                                                                                        & EEG1
  + EOG1                              & \uline{84.71}  & \uline{76.48}  & \uline{0.79}                                                              & 95.78          & 54.64          & 0.11                                                          & 65.93          & 50.82          & 0.12                                                             \\ 
\hline
SalientSleepNet \cite{ref9}                                                                       & Supervised                                                                                 & EEG1
  + EOG1 + EMG1                       & 60.75          & 52.90          & 0.48                                                                      & 83.64          & 46.93          & 0.02                                                          & 50.34          & 39.50          & 0.00                                                             \\
SleepFM \cite{ref23}                                                                               & SSL                                                                                        & EEG1
  + EOG1 + EMG1                       & 71.73          & 62.89          & 0.62                                                                      & 96.51          & 55.98          & 0.13                                                          & 55.07          & 42.49          & 0.02                                                             \\
SynthSleepNet                                                                          & SSL                                                                                        & EEG1
  + EOG1 + EMG1                       & 76.80          & 65.48          & 0.68                                                                      & 98.88          & 66.60          & 0.33                                                          & 69.98          & 54.32          & 0.17                                                             \\
SynthSleepNet+TCM                                                                      & SSL                                                                                        & EEG1
  + EOG1 + EMG1                       & \textbf{85.01} & \textbf{78.34} & \textbf{0.79}                                                             & 99.09          & 68.96          & 0.38                                                          & 74.05          & 57.44          & 0.21                                                             \\ 
\hline
SalientSleepNet \cite{ref9}                                                                       & Supervised                                                                                 & EEG1
  + EOG1 + ECG1                       & 48.68          & 42.60          & 0.33                                                                      & 78.10          & 44.80          & 0.01                                                          & 55.23          & 43.16          & 0.04                                                             \\
SleepFM  \cite{ref23}                                                                              & SSL                                                                                        & EEG1
  + EOG1 + ECG1                       & 66.16          & 60.20          & 0.56                                                                      & 94.68          & 53.21          & 0.09                                                          & 60.08          & 46.46          & 0.07                                                             \\
SynthSleepNet                                                                          & SSL                                                                                        & EEG1
  + EOG1 + ECG1                       & 77.29          & 65.13          & 0.69                                                                      & \uline{99.23}  & \uline{71.00}  & \uline{0.42}                                                  & \uline{75.40}  & \uline{59.03}  & \uline{0.24}                                                     \\
SynthSleepNet+TCM                                                                      & SSL                                                                                        & EEG1
  + EOG1 + ECG1                       & 83.77          & 75.74          & 0.78                                                                      & \textbf{99.35} & \textbf{72.94} & \textbf{0.46}                                                 & \textbf{76.93} & \textbf{60.34} & \textbf{0.25}                                                    \\ 
\hline
\multicolumn{12}{c}{\textit{5\% of labeled data}}                                                                                                                                                                                                                                                                                                                                                                                                                                                                                                     \\ 
\hline
\multirow{2}{*}{\begin{tabular}[c]{@{}c@{}}\textit{Model}\\\textit{Name}\end{tabular}} & \multirow{2}{*}{\begin{tabular}[c]{@{}c@{}}\textit{Training}\\\textit{Type}\end{tabular}}  & \multirow{2}{*}{\textit{Modality (Count)}} & \multicolumn{3}{c|}{\begin{tabular}[c]{@{}c@{}}\textit{Sleep Stage}\\\textit{Classification}\end{tabular}} & \multicolumn{3}{c|}{\begin{tabular}[c]{@{}c@{}}\textit{Apnea}\\\textit{Detection}\end{tabular}} & \multicolumn{3}{c}{\begin{tabular}[c]{@{}c@{}}\textit{Hypopnea}\\\textit{Detection}\end{tabular}}  \\ 
\cline{4-12}
                                                                                       &                                                                                            &                                            & \textit{ACC}   & \textit{MF1}   & \textit{K}                                                                & \textit{ACC}   & \textit{MF1}   & \textit{K}                                                    & \textit{ACC}   & \textit{MF1}   & \textit{K}                                                       \\ 
\hline
SalientSleepNet \cite{ref9}                                                                       & Supervised                                                                                 & EEG1
  + EOG1                              & 67.30          & 58.88          & 0.56                                                                      & 88.89          & 49.20          & 0.04                                                          & 50.81          & 40.00          & 0.01                                                             \\
SleepFM  \cite{ref23}                                                                              & SSL                                                                                        & EEG1
  + EOG1                              & 71.87          & 62.90          & 0.62                                                                      & 95.00          & 53.53          & 0.09                                                          & 51.13          & 40.14          & 0.01                                                             \\
SynthSleepNet                                                                          & SSL                                                                                        & EEG1
  + EOG1                              & 80.50          & 69.29          & 0.73                                                                      & 95.80          & 54.64          & 0.01                                                          & 62.78          & 48.55          & 0.09                                                             \\
SynthSleepNet+TCM                                                                      & SSL                                                                                        & EEG1
  + EOG1                              & \textbf{87.98} & \textbf{82.12} & \textbf{0.83}                                                             & 96.89          & 56.80          & 0.15                                                          & 67.73          & 52.33          & 0.14                                                             \\ 
\hline
SalientSleepNet \cite{ref9}                                                                       & Supervised                                                                                 & EEG1
  + EOG1 + EMG1                       & 66.87          & 58.40          & 0.55                                                                      & 90.31          & 50.00          & 0.05                                                          & 51.41          & 40.32          & 0.01                                                             \\
SleepFM  \cite{ref23}                                                                              & SSL                                                                                        & EEG1
  + EOG1 + EMG1                       & 73.27          & 64.31          & 0.64                                                                      & 97.98          & 60.42          & 0.21                                                          & 59.11          & 45.82          & 0.06                                                             \\
SynthSleepNet                                                                          & SSL                                                                                        & EEG1
  + EOG1 + EMG1                       & 80.92          & 71.13          & 0.74                                                                      & 99.07          & 68.72          & 0.38                                                          & 71.02          & 55.02          & 0.18                                                             \\
SynthSleepNet+TCM                                                                      & SSL                                                                                        & EEG1
  + EOG1 + EMG1                       & \uline{87.41}  & \uline{81.92}  & \uline{0.83}                                                              & \uline{99.27}  & \uline{71.70}  & \uline{0.44}                                                  & 74.54          & 58.27          & 0.22                                                             \\ 
\hline
SalientSleepNet \cite{ref9}                                                                        & Supervised                                                                                 & EEG1
  + EOG1 + ECG1                       & 51.67          & 45.53          & 0.37                                                                      & 85.95          & 47.92          & 0.03                                                          & 72.97          & 56.37          & 0.19                                                             \\
SleepFM \cite{ref23}                                                                               & SSL                                                                                        & EEG1
  + EOG1 + ECG1                       & 68.93          & 62.50          & 0.60                                                                      & 96.12          & 55.08          & 0.12                                                          & 61.41          & 47.49          & 0.08                                                             \\
SynthSleepNet                                                                          & SSL                                                                                        & EEG1
  + EOG1 + ECG1                       & 79.79          & 69.08          & 0.72                                                                      & 99.27          & 71.66          & 0.43                                                          & \uline{76.80}  & \uline{60.02}  & \uline{0.25}                                                     \\
SynthSleepNet+TCM                                                                      & SSL                                                                                        & EEG1
  + EOG1 + ECG1                       & 83.60          & 75.73          & 0.77                                                                      & \textbf{99.37} & \textbf{73.47} & \textbf{0.47}                                                 & \textbf{77.52} & \textbf{61.32} & \textbf{0.27}                                                    \\
\hline
\multicolumn{12}{r}{* EEG1 = C4-A1 channel, EOG1 = EOG-Left channel} \\
\multicolumn{12}{r}{* The \textbf{best results} in each row are shown in bold, while the \uline{second-best} results are underlined // $K$ = \textit{Kappa}} \\
\end{tabular}}
}
% \vspace{-6mm}
\end{table*}

% \subsection{AWS}

% Since most users run Longhorn combined with public cloud providers, we also benchmarked our features in the AWS environment. For this purpose, we used two c5d.xlarge EC2 instances, a cost-efficient option commonly used by Longhorn developers for their benchmarks.
    
% EC2 instances have a built-in maximum provisioned IOPS, regardless of the hard drive used. As expected, the features performed to the machine's limit, reaching AWS's 40k IOPS cap. Overcoming these limitations requires using higher-performance and more expensive EC2 instances. After some research, we found that to harness the performance of our features added to Longhorn, users must use instances and volumes at least 5 times more expensive than c5d.xlarge using its installed volume.


\subsection{\coam{} with the \feyntool{} optimizer}
\label{sec:comp-feyn}
In this section,
we study the performance of our \coamwith{\feyntool} optimizer,
which runs the \coam{} algorithm with \feyntool{} as the oracle optimizer.
%
We evaluate the running time and output quality of our optimizer
by running the baseline \feyntool{}.
%
For all our benchmarks,
we set the segment depth $\Omega$ to be $120$.
%
The output of our \coamwith{\feyntool{}} is $\Omega-$optimal with $\Omega = 120$
and the cost function is the T count of the circuit.
%

\begin{figure}
  \centering\small
  \input{fig/prelim.feyn.120.tex}
  \caption{
  The figure shows the optimization results of our $\coamwith{\feyntool{}}$ tool and \feyntool{},
  using T count as the cost metric.
  %
  The labels ``S'' and ``F'' represent $\coam{}$ and \feyntool{} respectively.
%
  The figure presents T counts (lower is better) and
  running time (lower is better) for both optimizers.
%
The results demonstrate that optimizer $\coamwith{\feyntool{}}$,
which guarantees $\Omega-$optimality delivers a similar quality of circuits as \feyntool{}.
%
The figure also calculates the speedup of the \coam{} approach using the ``T(F)/T(S)'' ratio.
%
On average, the$\coamwith{\feyntool{}}$ delivers excellent time performance
and runs in $6.9$x less time, compared to \feyntool{}.
%
We impose an end-to-end timeout of 10 hours (36000 seconds).
%
}
  \label{fig:feynopt}
  % \setlength{\belowcaptionskip}{-20pt}
\end{figure}



\figref{feynopt} shows the results of this experiment.
%
The figure separates circuit families with horizontal lines
and sorts circuits within families by their size/number of qubits.
%
The figure calculates the metric T count for the input circuit and for
the circuits returned by \feyntool{} and \coamwith{\feyntool{}},
or \coam{} for short, labelled by ``F'' and ``S'' respectively.
%
The ``F/S'' represents the quality ratio of the two approaches.
%
The figure also shows the running times of both approaches and computes
the ratio ``F/S'', measuring the relative speedup from the \coam{} algorithm.
%
The results shows our \coam{} algorithm generates reliable quality circuits
in short running times.

\myparagraph{Optimization  Quality.}
The results show that our optimizer \coam{} matches the output quality of \feyntool{}.
%
There are a few cases where there is a difference of less than $1\%$,
such as the ``hhl'' circuit with $7$ qubits.
%
But overall, there is no noticeable difference and
the quality ratio of the outputs is $1$.
%
%
This experiment shows that $\Omega-$optimality, as guaranteed by
our \coam{} algorithm is a reliable quality criterion for T count optimization.
%for the \feyntool{} optimizer.
% This consistent output quality demonstrates
% that $\Omega-$optimality is a reliable quality guarantee for the optimizer \feyntool{}.
%
Note that the baseline \feyntool{} can, in principle, discover more optimizations
because it operates on the full circuit.
%
But in practice, we observe that optimizing the circuit in piecewise
fashion,
on circuit segments of size $\Omega$ and guaranteeing $\Omega-$optimality
finds similar optimizations.
%
In \secref{var-segment}, we confirm this for a range of $\Omega$ values.
%

\myparagraph{Run time.}
The figure also shows the running time of both approaches,
and calculates the speedup of the \coam{} approach using the ``T(F)/T(S)'' ratio.
%
Our \coam{} algorithm is slower by about factor two in one benchmark,
``grover'', but in all other benchmarks it performs better than \feyntool{},
running $6.9$x faster on average,
%
with no noticeable impact on the optimization quality (``F/S'' column).
%
For any given family, our \coam{} algorithm's speedup increases
consistently with the circuit sizes.
%
For example,
in the ``hwb'' family,
our \coam{} algorithm is $1.2$x faster for the smallest instance (with 8 qubits)
and runs $83$x faster for the largest instance (12 qubits)
%
;
%
For the ``hhl'' family,
our \coam{} algorithm is $23$x faster for the smallest instance (7 qubits)
and runs at least $94$x faster for the largest instance (9 qubits)
(on 9 qubits \feyntool{} does not terminate within our timeout of ten hours).
%
%

\myparagraph{Summary.}
The experiments suggest that local optimality is an effective
optimization criteria for T-count optimization. By focusing on local
optimizations, our \coam{} algorithm (with \feyntool{} as an oracle)
can optimize for T-count effectively and efficiently.
%

%% \todo{We observe...
%% %
%% Our theoretical results suggest that this gap could be understood as an
%% asymptotic gap,
%% %
%% for the following reasons....}

\if0
%%% THIS IS HARD TO READ FOR ME (UMUT)
Given that the speedup consistently increases with the circuit size,
the performance results indicate an asymptotic difference in running time of both
approaches.
%
We can confirm this difference using the theory results in \secref{algorithm}.
%
Specifically,
the \feyntool{} optimizer takes polynomial (at least quadratic) time
both in the number of qubits and the depth of the circuit.~\cite{amy2019formal}.
%
The optimizer \coamwith{\feyntool} is also polynomial in the number of qubits,
but scales linearly in the depth/size of the circuit.
%
From \corref{linear-calls},
using the fact that the cost function is T count,
we get that \coamwith{\feyntool} makes $O(\sizeof{C})$ calls to \feyntool{},
where $\sizeof{C}$ is the size of circuit $C$;
each call runs on small circuits of depth at most $2\Omega$ (\thmref{cost})
and therefore, each call to \feyntool{}
takes polynomial time in the number of qubits (the depth is constant).
%
Because the number of oracle calls is linear,
the total runtime of \coamwith{\feyntool} is linear in the size and polynomial in the number of qubits,
which is asymptotically different from \feyntool{}, which takes polynomial in both.
%
\fi

% %
% Because each such call only runs the optimizer on ``small circuits'' of depth at most $2\Omega$,
% the cost of each \feyntool{} call only depends on the number of qubits, which are roughly constant
% for most qubits in the figure.
% %
% Thus, the \coamwith{\feyntool} optimizer practically shows linear scalability.


% Thus, roughly speaking, the optimizer \coamwith{\feyntool} decomposes the complexity
% of optimization using \feyntool{} and makes it scale linearly (instead of quadratically)
% in the size of the circuit.
% %
% In summary,
% the optimizer \coamwith{\feyntool} consistently
% delivers reliable quality circuits with excellent scalability.
% %
% %
% The results demonstrate that $\Omega-$optimality is a practical and reliable
% quality criterion in circuit optimization with \feyntool{} optimizer.
% asymptotically improves the running time of optimization using \feyntool{}.
% \coamwith{\feyntool} delivers linear running time
% without compromising circuit quality.
%
% \secref{linear} discusses the linear running time of \coamwith{\feyntool} in more detail.


% % IT IS WELL KNOWN That xxx. IN principle, we could have an optimizer. We therefore to evaluate.
% We have evaluated benchmarks with gate count
% as the quality metric
% because our oracle, \quartz{}, directly supports it.
% %
% However, it is well known that depending on the gate set,
% other metrics become more important.
% %
% For example, in the \clifft{} gate set,
% T count is an important metric of circuit quality
% because T gates they are not as computationally efficient
% compared as the Clifford gates (T gates roughly cost 50x other gates in practice \cite{Nam_2018}).
% %
% Similarly, CNOT count is another important metric because CNOT
% gates operate on two qubits and are expensive to implement.
% %


% In this section,
% we evaluate the benefit of our configuration $\coamwith{\quartzt{6}}$,
% for metrics ``T count'' and ``CNOT count''.
% %
% Note that, in principle,
% we could plug in an optimizer which is designed for these metrics,
% but here we evaluate the benefit with \quartz{} that is designed with circuit size
% as the metric.
% %
% To do so, we translate our benchmark suite, expressed in the Nam gate set,
% to the \clifft{} gate set.
% %
% We then translate the optimized version of our benchmarks
% and compare the unoptimized and optimized translation.
% %
% %

% To translate a benchmark from the Nam gate set to the \clifft{} gate set,
% we use gridsynth,
% a tool that translates parameterized $\mathsf{R_Z}$ gates
% to the \clifft{} gate set in provably optimal fashion~\cite{gridsynth}.
% %
% We also decompose the $\mathsf{X}$ gate to \clifft{} circuits
% using the Qiskit compiler.
% %
% We run Qiskit  in the mode where it does not perform any optimizations.
% %
% Note that this translation does not change the CNOT count because
% it only replces single qubit nam gates with single qubit \clifft{} gates.
% %
% Thus, a CNOT count comparison applies to both the Nam and the \clifft{} gate sets.

% \figref{comp-cliff} compares
% the circuit size, the T count, and CNOT count,
% of the unoptimized translation and the optimized translation
% with columns labelled ``Input'' and ``Output'' respectively.
% %
% Across the board,
% we observe that the implementation finds reductions in size and
% they correlate with reductions in T count and CNOT count.
% %
% In case of the ham15-med circuit,
% the implementation reduces the T count by $7\%$
% and the CNOT count by $6\%$;
% the larger variant, circuit ham15-high,  also shows similar numbers.
% %
% For qft circuits,
% we observe that gate count reduction and T count reduction
% is perfectly correlated,
% as both are reduced by the same percentage on all the qft instances.
% %
% On average,
% the implementation reduces the circuit size by around $11.8\%$,
% the T count by $4\%$, and the CNOT count by $6.2\%$.


% \cleardoublepage
% \cleardoublepage

\subsection{\coam{} with the \quartz{} optimizer}
\label{sec:comp-quartz}

% In this subsection,
% we establish that ensuring $\Omega$-optimality leads to good circuit quality.
%
% Additionally, we study the benefits of our \coam{} algorithm when used with
% \quartz{}.

% One of the key benefits of our algorithm
% is that runs in a predictable amount of time, and (within that time)
% guarantees $\Omega$-optimality.
%
% In this section, we show how the \coam{} algorithm
% improves the \quartz{} optimizer.
% %
% We optimize circuits using the optimizer  $\coamwith{\quartzt{t}}$
% with segment depth $\Omega$ equal to $6$
% and
% The optimizer $\coamwith{\quartzt{t}}$ runs the \coam{} algorithm
% using \quartz{} as the oracle,
% where each call to \quartz{} is roughly allocated $t$ amount of time per gate,
% i.e., if the \coam{} algorithm calls \quartz{} for a circuit of size $k$,
% the timeout we give to \quartz{} is $k*t$.
% %
% We use the timeout functionality in \quartz{} to implement this.
% %
% We present results with the per-gate timeout $t = 0.01s$,
% but we also provide results of timeouts $t = 0.1s$ and $t = 1s$ in the Appendix.
% %
% % We chose the segment depth as $6$ because the default \quartz{} optimizer uses
% equivalence rules for circuits upto size $6$.
% %
% With segment depth as $6$,
% the algorithm feeds all segments of size $6$ to \quartz{}
% giving it enough scope to apply its optimizations.
%


% \begin{figure}
%   \centering
%   \small
%   \begin{tabular}{rcccccc}
  &  &  & \multicolumn{3}{c}{Number of optimizations} &  \\ \cmidrule(lr){4-6}
   Family & Qubits & Input Size & Q & S & S/Q & Time (s) \\


  \midrule\multirow{2}{*}{ham15}  & 17 &   1061 & 202 (19\%)  & 202 (19\%)             & 1.0x     &    8 \\
                          & 20 &   4365 & 342 (8\%)   & \textbf{992} (23\%)    & 2.9x     &   22 \\
  \midrule\multirow{3}{*}{hhl}    &  7 &   5319 & 290 (5\%)   & \textbf{1441} (27\%)   & 4.96x    &   10 \\
                          &  9 &  63392 & 145 (0\%)   & \textbf{15678} (25\%)  & 107.39x  &  130 \\
                          & 11 & 629247 & 131 (0\%)   & \textbf{144438} (23\%) & 1094.23x & 1702 \\
  \midrule\multirow{2}{*}{gf}     & 48 &   2694 & 0 (0\%)     & 0 (0\%)                & 1.0x     &    8 \\
                          & 96 &   7553 & 2078 (28\%) & \textbf{4124} (55\%)   & 1.98x    &  137 \\
  \midrule\multirow{4}{*}{grover} &  7 &   2479 & 224 (9\%)   & \textbf{252} (10\%)    & 1.12x    &    9 \\
                          &  9 &   8968 & 66 (1\%)    & \textbf{876} (10\%)    & 13.09x   &   28 \\
                          & 11 &  27136 & 105 (0\%)   & \textbf{2643} (10\%)   & 24.94x   &   78 \\
                          & 15 & 180497 & 323 (0\%)   & \textbf{17588} (10\%)  & 54.29x   &  583 \\
  \midrule\multirow{4}{*}{qft}    & 48 &   4626 & 424 (9\%)   & \textbf{756} (16\%)    & 1.78x    &   15 \\
                          & 64 &   7402 & 758 (10\%)  & \textbf{1892} (26\%)   & 2.49x    &   31 \\
                          & 80 &  10690 & 744 (7\%)   & \textbf{3540} (33\%)   & 4.75x    &   46 \\
                          & 96 &  14490 & 1098 (8\%)  & \textbf{5700} (39\%)   & 5.19x    &   66 \\
  \midrule\multirow{4}{*}{vqe}    & 12 &  11022 & 1127 (10\%) & \textbf{7407} (67\%)   & 6.57x    &   37 \\
                          & 16 &  22374 & 458 (2\%)   & \textbf{14343} (64\%)  & 31.25x   &   87 \\
                          & 20 &  38462 & 1019 (3\%)  & \textbf{23551} (61\%)  & 23.09x   &  166 \\
                          & 24 &  59798 & 789 (1\%)   & \textbf{35031} (59\%)  & 44.34x   &  316 \\

 \midrule
 \end{tabular}
%   \caption{
%   The figure displays the number optimizations and running time of our optimizer $\coamt{\quartzt{0.01}}$
%   and also shows the optimizations by baseline \quartz{} in the same end-to-end time.
%   %
%   The label ``S'' represents the $\coam{}$ approach and the label ``Q'' represents \quartz{}.
%   %
%   The $S/Q$ column shows how many more optimizations \coam{} finds w.r.t \quartz{} for the same running time.
%   %
%   The figure shows that our \coam{} algorithm finds significant reductions ranging
%   from $20\% - 60\%$ in short runtimes.
%   %
%   Time column shows the runtime in seconds.
%   }
%   \label{fig:comp-quartz}
%   % \setlength{\belowcaptionskip}{-20pt}
% \end{figure}



\begin{figure}
  \centering
  \small
  \begin{tabular}{ccccccc}
    &  &  & \multicolumn{3}{c}{Number of optimizations} &  \\ \cmidrule(lr){4-6} 
     Family & Qubits & Input Size & Q & S & S/Q & Time (s) \\ 
   
   
    \midrule\multirow{2}{*}{ham15}  & 17 &  1061 & 202 (19\%)  & 202 (19\%)            & 1.0x   &   3 \\
                            & 20 &  4365 & 800 (18\%)  & \textbf{992} (23\%)   & 1.24x  &  13 \\
    \midrule\multirow{2}{*}{hhl}    &  7 &  5319 & 373 (7\%)   & \textbf{1441} (27\%)  & 3.86x  &   8 \\
                            &  9 & 63392 & 260 (0\%)   & \textbf{15678} (25\%) & 60.07x & 113 \\
    \midrule\multirow{2}{*}{gf}     & 48 &  2694 & 0 (0\%)     & 0 (0\%)               & 1.0x   &   3 \\
                            & 96 &  7553 & 20 (0\%)    & 20 (0\%)              & 1.0x   &  41 \\
    \midrule\multirow{3}{*}{grover} &  7 &  2479 & 236 (10\%)  & \textbf{243} (10\%)   & 1.03x  &   6 \\
                            &  9 &  8968 & 618 (7\%)   & \textbf{858} (10\%)   & 1.39x  &  21 \\
                            & 11 & 27136 & 952 (4\%)   & \textbf{2604} (10\%)  & 2.73x  & 103 \\
    \midrule\multirow{4}{*}{qft}    & 48 &  4626 & 756 (16\%)  & 756 (16\%)            & 1.0x   &  12 \\
                            & 64 &  7402 & 1236 (17\%) & \textbf{1892} (26\%)  & 1.53x  &  20 \\
                            & 80 & 10690 & 1312 (12\%) & \textbf{3540} (33\%)  & 2.7x   &  33 \\
                            & 96 & 14490 & 1534 (11\%) & \textbf{5700} (39\%)  & 3.71x  &  51 \\
    \midrule\multirow{4}{*}{vqe}    & 12 & 11022 & 627 (6\%)   & \textbf{7407} (67\%)  & 11.8x  &  25 \\
                            & 16 & 22374 & 493 (2\%)   & \textbf{14343} (64\%) & 29.04x &  60 \\
                            & 20 & 38462 & 643 (2\%)   & \textbf{23551} (61\%) & 36.57x & 125 \\
                            & 24 & 59798 & 505 (1\%)   & \textbf{35031} (59\%) & 69.23x & 195 \\
   
   \midrule
   \end{tabular}
  \caption{
  The figure displays the number optimizations and running time of our optimizer $\coamt{\quartzt{0.01}}$
  and also shows the optimizations by baseline \quartz{} in the same end-to-end time.
  %
  The label ``S'' represents the $\coam{}$ approach and the label ``Q'' represents \quartz{}.
  %
  The $S/Q$ column shows how many more optimizations \coam{} finds w.r.t \quartz{} for the same running time.
  %
  The figure shows that our \coam{} algorithm finds significant reductions ranging
  from $20\% - 60\%$ in short runtimes.
  %
  Time column shows the runtime in seconds.
  }
  \label{fig:comp-quartz}
  % \setlength{\belowcaptionskip}{-20pt}
\end{figure}

In this section, we evaluate our $\coamwith{\quartzt{0.01}}$ optimizer,
which runs the \coam{} algorithm using \quartz{} as the oracle.
%
To evaluate its optimizations,
we run our optimizer on all the circuits and record its output and the running time.
%
Then,
we run default \quartz{} on the each input circuit for the same end-to-end time,
%
and analyze the number of optimizations discovered.

\figref{comp-quartz} shows the results for circuits taken from six families of quantum algorithms.
%
In the figure,
the label ``S'' denotes the number of optimizations discovered by our \coam{} algorithm
and label ``Q'' denotes the same with baseline \quartz{} (higher is better).
%
In cases where one approach finds more optimizations,
the figure uses bold numbers.
%
The label ``S/Q'' computes the \defn{optimization ratio} which is
the ratio of number of optimizations.
%
A higher ratio represents that our algorithm finds more optimizations.
%
The results show that the \coam{} approach produces
well optimized circuits in short running times.


\myparagraph{Quality.}
We observe that in all but two cases (ham15 with 17 qubits and gf with 48 qubits),
using our \coam{} approach
discovers more optimizations than baseline \quartz{}.
%
For the hhl circuits,
the \coam{} algorithm reduces gate count by $23\%$ to $27\%$,
which is excellent considering that \quartz{} reduces the size by less than $5\%$
within the same time.
%
For the vqe family,
our \coam{} algorithm reduces the gate count by around $60\%$ and
the optimizations by \quartz{} are less than $10\%$.
%
Overall, we observe that by using \coam{} algorithm with \quartz{} as an oracle,
we can effectively reduce gate count for circuits with thousands of gates
within seconds and optimize circuits with hundreds of thousands of gates within minutes
(the Time column displays the running time).
%
In the Appendix,
we verify that our optimizer \coamwith{\quartz{}} does not generate a worse quality circuit
even when the running time for both algorithms is scaled
by $100$x (by increasing the per call timeout 100x; see \secref{methodology} for more details on running time).

\myparagraph{Scalability.}
Note that our \coam{} optimizer
excels at finding optimizations for large circuits.
%
The figure illustrates this using the ``S/Q'' column,
which computes the optimization ratio of both approaches.
%
For all families, the optimization ratio increases with increasing circuit size.
%
In the case of hhl circuits, for example,
the ratio rises from around $5$x to $1000$x with increasing circuit size;
%
In the case of grover circuits,
the optimization ratio steadily increases from $1.12$x on the smallest instance (7 qubits)
to $54.29$x on the largest instance (15 qubits).
%
Similarly, for vqe, the ratio increases from $6.5$x to $44.34$x with increasing circuit size
and for qft, it increases from $1.78$x to $5.19$x.
%
We identify three key reasons for this scalability.
%
First, because baseline \quartz{} searches for optimizations on the whole circuit,
it struggles to find optimizations,
because the search space is exponential.
%
Second, our optimizer \coam{} algorithm
utilizes \quartz{} by focusing it on small segments ($\Omega = 6$),
where it delivers excellent results.
%
Third,
the \coam{} algorithm ensures that it applies \quartz{} to all
$\Omega-$segments and misses no local optimizations.

\myparagraph{Summary.}
The experiments demonstrate that our \coam{} algorithm scales with circuit sizes
when using \quartz{} as the oracle.
%
The results suggest that our algorithm uses \quartz{} effectively,
finding significant reductions in circuit size and producing good quality circuits.
%

%

% For example,
% across different sizes of grover circuits,
% the \coam{} algorithm reduces gate count consistently by $10\%$;
% in contrast the gate count reduction by \quartz{} drops from $9\%$ to $0\%$.
%

%
% The 'S/Q' column shows the improvement
% in optimization rates due to \coam{}, i.e.,
% it shows how many more optimizations (per second)
% the \coam{} algorithm identifies compared to \quartz{}.
% %
% %
% Compared to \quartz{},
% the \coam{} algorithm yields $1.78$x and $5.19$x more optimizations per second
% for the smallest and largest qft instances
% respectively (see the ratio column in \figref{comp-quartz}).
% %
% Similarly,
% in the case of vqe,
% the gap in their optimization rates
% rises from around $2$x on the smallest instance to $42$x
% on the largest instance.

% %
% This is because the efficiency of \quartz{},
% quantified by optimization rates,
% reduces with increasing circuit sizes,
% whereas the \coam{} algorithm scales to larger sizes.

% Across the board,
% we observe that the optimization ratio is never less than one,
% establishing that the \coam{} approach never generates a worse quality circuit
% in the same running time.
% %
% For instance,



% we first optimize all circuits with $\coamwith{\quartzt{t}}$, where each oracle call is capped by a
% fixed amount of time $t$, and record how long the entire algorithm takes to
% complete.
% %
% Then,
% we run the default \quartz{} on the same input circuit for the same end-to-end
% time,
% %
% and compare the resulting circuit quality and optimization rate.


% number of gates optimized per second by
% \coam{} using \quartz{} as an oracle is higher than just running \quartz{}
% alone.

% % when using \quartz{} as an oracle,
% % \coam{} is able to accelerate the \defn{optimization rate} of \quartz{}.
% %

% %
% As a result, \coam{} is able to deliver better quality circuits within
% a given time limit.

% , due to its combination of linear time complexity and
% $\Omega$-optimality guarantee.
%


% \figref{comp-quartz} shows the result of this comparison
% using the configuration $\coamwith{\quartzt{0.06}}$, i.e., where
% each oracle call is limited to $0.06$ seconds.
% %
% The figure shows the input circuit sizes
% and displays the circuit sizes after optimization.
% %
% The figure labels the results of the \coam{} configuration as ``S'' and
% the standard \quartz{} approach as ``Q''.
% %
% In the cases where one approach performs better,
% the figure uses bold numbers, with lower numbers indicating better circuit quality.
% %
% The `Q/S' ratio column quantifies the improvement in circuit quality
% using the \coam{} algorithm w.r.t. standard \quartz{} (higher is better).
% %
% The figure also provides and compares the \emph{optimization rate} of both
% approaches, where the optimization rate is the number of gates optimized per
% second of execution.
% %
% A higher rate indicates faster optimizations.
% %
% The time column shows the execution time for both cases.



% \myparagraph{Quality.}
% We observe that in seventeen cases,
% using the \coam{} algorithm improves the quality of circuits w.r.t. standard \quartz{}
% and sometimes does so by a wide margin.
% %
% In the other seven cases,
% both approaches output the same quality of circuits.
% %
% For the hhl circuits,
% the \coam{} algorithm finds reductions ranging from $23\%$ to $27\%$,
% where \quartz{} does not find many optimizations
% within this time (less than $5\%$).
% %
% In the case of grover circuits,
% the \coam{} algorithm optimizes gate count by $10\%$ across the various sizes.
% %
% However, the efficacy of \quartz{} reduces from $9\%$ to $0\%$ with increasing sizes.
% %
% In the qpe circuits,
% neither approach finds any optimizations.
% %
% %
% For the vqe family,
% the \coam{} algorithm reduces the gate count by around $60\%$ and
% the optimizations by \quartz{} are less than $10\%$.
% %
% On average,
% the outputs of the \coam{} configuration are smaller by $1.37$x,
% when compared to the outputs of standard \quartz{}.

% \myparagraph{Scalability.}
% Across the board,
% we observe that the gap between the two approaches widens with increasing circuit sizes.
% %
% This is because the efficiency of \quartz{},
% quantified by optimization rates,
% reduces with increasing circuit sizes,
% whereas the \coam{} algorithm scales to larger sizes.

% For example,
% across different sizes of qft,
% the \coam{} algorithm achieves optimization rates
% ranging from $50$ to $85$ and the rates in fact increase as the circuit size grows.
% %
% In contrast,
% the optimization rate for \quartz{} reduces from around $28$ to $16$.
% %
% The 'S/Q' column shows the improvement
% in optimization rates due to \coam{}, i.e.,
% it shows how many more optimizations (per second)
% the \coam{} algorithm identifies compared to \quartz{}.
% %
% %
% Compared to \quartz{},
% the \coam{} algorithm yields $1.78$x and $5.19$x more optimizations per second
% for the smallest and largest qft instances
% respectively (see the ratio column in \figref{comp-quartz}).
% %
% Similarly,
% in the case of vqe,
% the gap in their optimization rates
% rises from around $2$x on the smallest instance to $42$x
% on the largest instance.
% %
% On average,
% we observe that the \coam{} algorithm finds $7.35$x many more optimizations
% than \quartz{} per second.


% All families of circuits follow a similar trend:
% the optimization rate ratio between the \coam{} approach and the \quartz{} optimizer
% increases with the circuit size,
% demonstrating the scalability advantages of the \coam{} algorithm.
% %
% This scalability advantage of the \coam{} approach is built into the algorithm
% because it calls \quartz{} only on small circuits,
% where \quartz{} delivers excellent results.
% %
% The algorithm never sends large circuits to \quartz{},
% where \quartz{} struggles
% because it chases potential optimizations in large spaces.
% %
% Because the \coam{} algorithm runs in linear time,
% it delivers good scalability as circuit sizes increase.
% %
% %
% %
% Because the execution time is equal for both approaches,
% the ratio column additionally represents the ratio of total optimizations.
%
% %
% For example,
% consider the ratio column of \figref{comp-quartz} which compares the optimization rates
% of the \coam{} algorithm with \quartz{}.
%

% First, because they can take a very long time,
% existing super op- timizers such as Quartz are used with a hard deadline (e.g., several hours)
% by which they have to be terminated. When not run to completion,
% they are unable to offer any quality guarantees whatsover, because they can get “lost” chasing potential optimizations
% in exponentially large search spaces.


% Second, most circuit optimizations naturally involve a small number of contiguous gates.
% Our algorithm (provably) does not miss such optimizations, but other optimizers could.
% \myparagraph{Consistency.}
% Since the \coam{} algorithm splits the circuit into pieces
% and melds them together,
% it is important to evaluate whether
% splitting and melding lose any optimizations in practice.
% %
% In theory, our algorithm guarantees $\Omega-$optimality.
% %
% To evaluate this guarantee,
% we compare to standard \quartz{} with extended running times.
% %
% We extend the running time by increasing the per-call timeout to \quartz{}
% to $0.6$s, i.e.,
% we consider the configuration \coamwith{\quartzt{0.6}}.
% %
% We note that the end-to-end running time becomes at least ten fold in this configuration,
% and we run the standard \quartz{} for the extended time.
% %
% This ensures that \quartz{} has a larger amount of
% time to find optimizations.
% %
% We present these results in the Appendix, and summarize them here.
% % /
% Overall, the gap between both approaches reduces with extended running time
% because when optimizers operate for longer times,
% their optimization rates diminish due to the
% increasing difficulty of finding further optimizations~\cite{quartz-2022, queso-2023}.
% %
% This extended running time allows \quartz{} to somewhat catch up to the \coam{}.
% %
% However, \coam{} never generates a worse quality circuit.
% %
% We also increase the running time further by ten fold,
% verifying that \coam{} consistently delivers better quality circuits.
% %
% We conclude from these experiment with extended running times
% that $\Omega$-optimality is a good quality guarantee because,
% in practice,
% the \coam{} algorithm never generates a worse quality circuit.
% %



%
% Overall, we note that the gap between the approaches has reduced,
% when compared to the previous configuration.
% %
% This occurs because as optimizers operate for longer times,
% their optimization rates diminish due to the
% increasing difficulty of finding further optimizations~\cite{quartz-2022, queso-2023}.
% %
% The difference in optimization rates between \figref{comp-quartz} and \figref{comp-queso}
% exemplifies this.
% %
% For example, the optimization rate in vqe\_n12 for \quartz{} reduces from around $30$ to $12$
% when its execution time increases from $37$ seconds to $640$ seconds.
% %
% The many fold extension of execution time decreases the optimization rates of both approaches
% and moderates the scalability advantages of the \coam{} algorithm.
% %
% On average,
% circuits generated by the \coam{} algorithm are $1.1$ times smaller
% than those generated by \quartz{}.
% %/
% In our Supplementary,
% we include a third configuration (\coamwith{\quartzt{6}})
% where we further increase the time ten fold and observe a similar trend,
% where \coam{} consistently delivers better quality circuits.
% %
% We conclude from these experiment with extended running times
% that $\Omega$-optimality is a good quality guarantee because,
% in practice,
% the \coam{} algorithm never generates a worse quality circuit.
%

\subsection{\coam{} with the \queso{} optimizer}
\label{sec:comp-queso}
% guarantees $\Omega$-optimality.
%
In this section, we evaluate our optimizer $\coamwith{\quesot{0.005}}$,
which runs our \coam{} algorithm using \queso{} as the oracle.
%
Our optimizer depth uses segment size $\Omega = 6$
and gives each call to \queso{} is a timeout (see \secref{methodology} for the more details).
%
We run $\coamwith{\quesot{0.005}}$,
or \coam{} for short,
on all the circuits and record
the output quality and running time.
%
We then run default \queso{} on each input circuit for the same end-to-end time,
%
and evaluate the number of optimizations.


\begin{figure}
  \centering
  \small
  \begin{tabular}{ccccccc}
  &  &  & \multicolumn{3}{c}{Number of optimizations} &  \\ \cmidrule(lr){4-6}
   Family & Qubits & Input Size & Q & S & S/Q & Time (s) \\


  \midrule\multirow{2}{*}{ham15}  & 17 &   1061 & 206 (19\%)   & \textbf{250} (24\%)    & 1.21x     &    26 \\
                          & 20 &   4365 & 650 (15\%)   & \textbf{1353} (31\%)   & 2.08x     &   130 \\
  \midrule\multirow{3}{*}{hhl}    &  7 &   5319 & 1509 (28\%)  & \textbf{1829} (34\%)   & 1.21x     &   289 \\
                          &  9 &  63392 & 10703 (17\%) & \textbf{20764} (33\%)  & 1.94x     & 12784 \\
                          & 11 & 629247 & 0 (0\%)      & \textbf{109353} (17\%) & 109354.0x & 35987 \\
  \midrule\multirow{2}{*}{gf}     & 48 &   2694 & 0 (0\%)      & 0 (0\%)                & 1.0x      &    19 \\
                          & 96 &   7553 & 0 (0\%)      & \textbf{4098} (54\%)   & 4099.0x   &   213 \\
  \midrule\multirow{4}{*}{grover} &  7 &   2479 & 244 (10\%)   & \textbf{326} (13\%)    & 1.33x     &   152 \\
                          &  9 &   8968 & 32 (0\%)     & \textbf{1043} (12\%)   & 31.64x    &   252 \\
                          & 11 &  27148 & 70 (0\%)     & \textbf{3357} (12\%)   & 47.3x     &  1081 \\
                          & 15 & 180497 & 282 (0\%)    & \textbf{20510} (11\%)  & 72.48x    &  4759 \\
  \midrule\multirow{4}{*}{qft}    & 48 &   4626 & 935 (20\%)   & \textbf{1228} (27\%)   & 1.31x     &   497 \\
                          & 64 &   7402 & 2034 (27\%)  & \textbf{2558} (35\%)   & 1.26x     &   834 \\
                          & 80 &  10690 & 3540 (33\%)  & \textbf{4451} (42\%)   & 1.26x     &  1270 \\
                          & 96 &  14490 & 5862 (40\%)  & \textbf{6808} (47\%)   & 1.16x     &  1597 \\
  \midrule\multirow{4}{*}{vqe}    & 12 &  11022 & 4620 (42\%)  & \textbf{7521} (68\%)   & 1.63x     &   150 \\
                          & 16 &  22374 & 0 (0\%)      & \textbf{14511} (65\%)  & 14512.0x  &   293 \\
                          & 20 &  38462 & 0 (0\%)      & \textbf{23777} (62\%)  & 23778.0x  &   518 \\
                          & 24 &  59798 & 0 (0\%)      & \textbf{35315} (59\%)  & 35316.0x  &   827 \\

 \midrule
 \end{tabular}
  \caption{
  The figure compares the performance of our $\coamwith{\quesot{0.005}}$ optimizer with standard \queso{}.
  %
  The label ``S'' represents the number of optimization discovered by our \coam{} approach
  and the label ``Q'' denotes optimizations found by \queso{}.
  %
  The ``S/Q'' column measures how many more optimizations \coam{} finds relative to \queso{};
  we avoid divide by zero issues by adding $+1$ optimization to both tools.
  %
  For the results in this figure, we impose an end-to-end timeout of ten hours.
  }
  \label{fig:comp-queso}
  % \setlength{\belowcaptionskip}{-10pt}
\end{figure}

\figref{comp-queso} shows the results of this experiment
for circuits from six families of quantum algorithms.
%
Each family is separated by a horizontal line
and contains circuits which differ number of qubits and are sorted by input size.
%
The figure shows the number of optimizations performed by our \coam{} approach and by \queso{}
in the columns labeled ``S'' and ``Q'' respectively;
the columns also list the percentage of gates removed in parentheses.
%
The figure uses bold numbers when one tool finds more optimizations than the other.
%
The time column shows the running time of each benchmark.
%
The results show that our optimizer consistently finds significant reductions
in gate count, across all families and circuit sizes.

%
\myparagraph{Quality.}
Across the board, we observe that using \queso{} with our \coam{} algorithm discovers
more optimizations than baseline \queso{}.
%
For the gf circuits,
the \coam{} approach reduces the gate count by $54\%$ the large instance (96 qubits)
and \queso{} does not discover any optimizations in same the running time.
%
For the vqe circuits,
our \coam{} algorithm delivers excellent results,
reducing the gate count by around around $60\%$ on all instances;
\queso{} reduces the gate count of the smallest instance by $48\%$ but
does not find optimizations for others in this amount of running time.
%
%
For the qft circuits, both approaches find similar reductions in the gate count.
%
Overall, we see that using \queso{} with the \coam{} algorithm finds substantial
reductions in gate count, across a range of quantum algorithms and circuit sizes.
%

\myparagraph{Scalability.}
The figure also shows that our \coamwith{\queso} optimizer scales well with circuit size.
%
The column ``S/Q'' computes the ratio of optimizations found by the tools,
and the ratio typically increases with circuit size.
%
For example,
in the case of ham circuits,
the ratio increases from 1.19x to 2x with increasing circuit size.
%
For grover circuits,
the optimization ratio increases from $1.3$x to $72$x.
%
Similarly, for hhl circuits, the ratio increases from 1.21x on the smallest instance (7 qubits)
to 1.94x on the medium sized instance (9 qubits);
for the largest instance of hhl (11 qubits),
using \queso{} with our \coam{} algorithm finds 17\% reduction and baseline \queso{}
does not discover any optimizations.
%
Similarly, \coam{} delivers increasing optimization ratios for the ``vqe'' family.
%

\myparagraph{Summary.}
The results show that many optimizations are local
and are found by our algorithm because it focuses \queso{} on small segments.
%
Our algorithm's approach to optimize circuits in a piecewise fashion and melding them together
finds significant reductions using \queso{} as the oracle.

% Because \coamwith{\queso} finds these optimizations by only sending segments of size $2\Omega$ to \queso{},
% the results demonstrate that many optimizations are local
% and can be found by focusing the baseline optimizer on small segments.

% The evaluation suggests that the \coam{} algorithm utilizes the \queso{} optimizer effectively,
% by splitting the circuit into small segments, optimizing them with \queso{} and finding
% the optimizations at the boundaries by melding the resulting circuits.




% \subsection{\coam{} scales linearly with the size of circuits}
\subsection{\coam{} scales linearly with the size of circuits}
\subsection{\coam{} scales linearly with the size of circuits}
\input{fig/linearity.tex}
\label{sec:linear}

We show that the \coam{} algorithm requires only linear time
by measuring its execution time across different sizes of benchmarks.
%
For this experiment, we use $\quartzt{0.06}$ as our oracle
(i.e., each sub circuit is optimized by calling \quartz{} for a duration of $0.06$ seconds).
%
We consider four quantum algorithms: Grover, VQE, QPE, and QFT,
and plot the results in \figref{linear}.
%
The Y-axis shows the execution time and the X-axis shows the size of benchmarks.
%
We vary the size of circuits by changing the number of qubits in the corresponding algorithms.

The plots show that the execution time scales linearly
across sizes ranging from a few hundreds of gates to circuits as large as $0.14$ millions
of gates.
%
For example,
in case of the VQE algorithm,
the \coam{} algorithm takes around five seconds for the smallest circuit of size two thousand,
and takes around 300 seconds for the largest circuit of size one hundred thousand ($136000$ precisely).
%
We provide the data points for our plots and the plot for the HHL family in our Supplementary.
%
\label{sec:linear}

We show that the \coam{} algorithm requires only linear time
by measuring its execution time across different sizes of benchmarks.
%
For this experiment, we use $\quartzt{0.06}$ as our oracle
(i.e., each sub circuit is optimized by calling \quartz{} for a duration of $0.06$ seconds).
%
We consider four quantum algorithms: Grover, VQE, QPE, and QFT,
and plot the results in \figref{linear}.
%
The Y-axis shows the execution time and the X-axis shows the size of benchmarks.
%
We vary the size of circuits by changing the number of qubits in the corresponding algorithms.

The plots show that the execution time scales linearly
across sizes ranging from a few hundreds of gates to circuits as large as $0.14$ millions
of gates.
%
For example,
in case of the VQE algorithm,
the \coam{} algorithm takes around five seconds for the smallest circuit of size two thousand,
and takes around 300 seconds for the largest circuit of size one hundred thousand ($136000$ precisely).
%
We provide the data points for our plots and the plot for the HHL family in our Supplementary.
%
\label{sec:linear}

We show that the \coam{} algorithm requires only linear time
by measuring its execution time across different sizes of benchmarks.
%
For this experiment, we use $\quartzt{0.06}$ as our oracle
(i.e., each sub circuit is optimized by calling \quartz{} for a duration of $0.06$ seconds).
%
We consider four quantum algorithms: Grover, VQE, QPE, and QFT,
and plot the results in \figref{linear}.
%
The Y-axis shows the execution time and the X-axis shows the size of benchmarks.
%
We vary the size of circuits by changing the number of qubits in the corresponding algorithms.

The plots show that the execution time scales linearly
across sizes ranging from a few hundreds of gates to circuits as large as $0.14$ millions
of gates.
%
For example,
in case of the VQE algorithm,
the \coam{} algorithm takes around five seconds for the smallest circuit of size two thousand,
and takes around 300 seconds for the largest circuit of size one hundred thousand ($136000$ precisely).
%
We provide the data points for our plots and the plot for the HHL family in our Supplementary.
%

\subsection{Experimenting with segment size}
\label{sec:var-segment}
\begin{figure}[t]
	\centering
  % \begin{minipage}[b]{0.5\columnwidth}
  %   \includegraphics[width=\textwidth]{plots/omega_vs_quality.png}
  % \end{minipage}
  % \hspace{1cm}
  % \begin{minipage}[b]{0.5\columnwidth}
  %   \includegraphics[width=\textwidth]{plots/omega_vs_time.png}
  % \end{minipage}
  \includegraphics[width=\textwidth]{media/plot_omega.pdf}
    \vspace{-15mm}
	\caption{
    The figure plots the impact of the parameter $\Omega$ on
    the output T count (lower is better)
    and running time of $\algname{}$ optimizer on a hhl circuit with $7$ qubits.
    %
    The dotted red lines in the plots denote the output T count and the running time of the oracle optimizer
    \feyntool{} on the whole circuit.
    %
    For almost all values of $\Omega$, the output quality matches the oracle optimizer,
    demonstrating that local optimality is a robust quality criterion for T count optimization.
  }
    \label{fig:vary-segment}
\end{figure}
In this section,
we evaluate the quality and efficiency guaranteed by the \coam{} algorithm
for different values of parameter $\Omega$.
%
We use our optimizer $\coamwith{\feyntool}$ for this evaluation,
which guarantees $\Omega-$optimality relative to the \feyntool{} optimizer.

%
\figref{vary-segment} plots the output T count (number of T gates) and the running
time of the optimizer against $\Omega$.
%
The figure shows the results for $\Omega$ values $2, 5, 15, 30, 60, 120 \dots 7680$;
we present the full table in the Appendix.
%
The red dotted line in the plots
shows the results of the the baseline optimizer \feyntool{}.
%
For the experiment,
we run the our optimizer on the hhl circuit with $7$ qubits, which initially contains 61246 T gates.
%
The results show that for a wide range of $\Omega$ values,
our optimizer produces a similar quality circuit as the baseline $\feyntool$
and typically does so in significantly less time
(at the extremities, there are two values of $\Omega$ for which \coam{} takes more time: $2$ and $7680$).
%

%
\myparagraph{T count.}
The plot for T count shows that when $\Omega$ is small (around $2$),
increasing it has quality benefits.
%
This is because quality guarantee of the $\coam{}$ algorithm becomes
stronger with increasing $\Omega$ as it requires larger segments to be optimal.
%
%
However, the benefits of increasing $\Omega$
become very incremental especially when $\Omega$ reaches around $60$,
where T count reaches around 42140 (20 gates away from being optimal).
%
This shows that $\Omega-$optimality is a good quality criterion,
as it generates good quality circuits
even with relatively small values of $\Omega$ (around $60$).

\myparagraph{Run time.}
One would perhaps expect that the running time of \coam{}
increases by increasing the segment size $\Omega$ because
1) each oracle call operates on larger segments
and 2) the algorithm generates a better quality circuit (provably and practically).
%
Indeed, the intuition is correct, for a majority of the values.
%
For values of $\Omega$ ranging from $120$ to $7860$,
the running time of the algorithm increases with increasing $\Omega$.
%
In this range,
the (non-linear) complexity of the oracle dominates the running time,
and it is faster to split, optimize and meld, calling the oracle many times
on smaller segments,
instead of querying the oracle on larger segments.


But, when $\Omega$ is very small,
we observe the opposite, i.e.,
increasing $\Omega$ reduces the running time.
\begin{wrapfigure}{r}{0.4\textwidth}
  \centering
  \includegraphics[width=0.39\textwidth]{omega_vs_time_zoom.png}
  \caption{Zooming in: Time vs. Omega plot}
  \label{fig:zoom-plot}
\end{wrapfigure}
%
For reference,
we draw \figref{zoom-plot},
which zooms the running time plot from \figref{vary-segment}
for initial values of $\Omega$, ranging from $2, 5, 15 \dots 120$.
%
For these smaller values of $\Omega$,
even though each oracle call is cheap,
the number of oracle calls dominates the time cost.
%
The \coam{} algorithm splits the circuit into a large number of small segments
and queries the oracle on each one, making many calls to the optimizer.
%
When a circuit segment is small,
it is more efficient to directly call \feyntool{},
which optimizes it in one pass.
%
For this reason, $\Omega = 120$ is a good value for our optimizer $\coamwith{\feyntool{}}$,
as it does not send large circuits to the oracle,
and also does not split the circuit into a large number of really small segments.
%

Overall,
we observe that for a wide range of $\Omega$ values,
our \coam{} algorithm outputs good quality circuits
and does so in a shorter running time than the baseline.
%
\begin{figure}[t]
    \centering
    \includegraphics[width=\columnwidth]{media/oracle_call_vs_input_size.pdf}
    \vspace{-20pt}
    \caption{ The number of oracle calls versus input circuit size for
      all our circuits (Nam gate set).  The plots show that the number of
      calls scales linearly with the number of gates.  }
    \label{fig:plot-oracle-calls-all}
\end{figure}



% Furthermore, it increases the number of meld operations
% and almost every meld of the algorithm finds optimizations,
% because the oracle optimizes the boundary segments in a piecewise fashion
% rather than finding optimizing them in one step.
%
%
% We measured that, for $\Omega = 2$,
% \coamwith{\feyntool} makes around ten thousand calls to the base optimizer \feyntool{},
% taking around four hundred seconds.
% %
% Thus, at the very start,
% increasing $\Omega$ reduces the running time because it reduces
% the number of oracle calls.
% %
% \figref{zoom-plot}
% The plot shows how the running time decreases with increasing the parameter.


% When the segment size becomes slightly larger, around $\Omega = 60$,
% the running time reduces to twenty seconds and does not vary much for
% some intermediate values.
% %
% For most intermediate values, around $\Omega = 30$ to $\Omega = 480$,
% we observe good time performance, where the algorithm takes around thirty seconds.
% %
% Beyond these intermediate values,
% the running time starts increasing because the algorithm sends larger segments to the oracle
% and oracle complexity becomes the dominating cost.
% %
% It is then much
% faster to split, optimize, and meld rather than use the oracle.

% When $\Omega$ is small, increasing it reduces the running time of the algorithm.
% %
% This is perhaps counterintuitive because small segment sizes
% generate a worse quality circuit and also use the oracle on smaller segments.
% %
% But, it reduces the number of oracle calls, which becomes the dominating cost in this case.
% %
% Furthermore, for small segment sizes,
% almost every meld is finding optimizations
% %




% Also note, that the initial T count for this circuit is 61246,
% implying that different values of $\Omega$ do not have a large impact on quality.
% %






% %
% But slowly the benefits become incremental and eventually $\Omega$ is large enough where
% the output of \feyntool{} matches the optimizer  $\coamwith{\feyntool}$,
% as shown with the dotted red line.
% To summarize,
% \figref{fig:comp-quartz} demonstrates that the \coam{} algorithm
% drastically improves the speed of optimization for each circuit.
% %
% It shows that the algorithm scales well with circuit size,
% achieving excellent optimization rates across a range of sizes.
% %
% \figref{fig:comp-queso} shows that this performance
%

%


% \input{fig/prelim2.0.01}



% To summarize,
% \figref{fig:comp-quartz} demonstrates that the \coam{} algorithm
% drastically improves the speed of optimization for each circuit.
% %
% It shows that the algorithm scales well with circuit size,
% achieving excellent optimization rates across a range of sizes.
% %
% \figref{fig:comp-queso} shows that this performance comes at no cost to circuit quality
% even with the extended runtime,
% as the \coam{} algorithm never generates a worse quality circuit.
% %


% \subsection{Scalable optimization using \coam{} with \queso{}}

% \begin{figure*}
%   \centering
%   \input{fig/r}
%   \caption{
%   The figure compares the performance of $\coamwith{\quesot{1}}$ against standard \queso{}.
%   %
% The label ``S'' represents $\coamwith{\quesot{1}}$ and the label ``Q'' represents \queso{}.
% %
% The figure compares the circuit sizes (lower is better) and the optimization rates (higher is better) of both approaches
% after running them for the same end-to-end time.
% }
%   \label{fig:comp-queso}
% \end{figure*}

% In this section,
% we evaluate the \coam{} algorithm using the \queso{} optimizer.
% %
% We integrate the \queso{} optimizer into our algorithm as an oracle
% and impose a one second time out for each call.
% %
% We note that \queso{} does not follow the given timeout strictly
% and often takes longer to run and return the result (around $4/5$ seconds per call).

% To evaluate the algorithm,
% we run it with \queso{} and compute the final circuit
% and the overall execution time, excluding time for file I/O (\secref{impl}).
% %
% Then, we execute default \queso{} on the whole circuit for the same total runtime
% and compare the quality of circuits produced.
% %
% For this experiment,
% we filter benchmarks based on their sizes and
% opt for benchmarks whose sizes are less than ten thousand gates.
% %
% This size limitation arises from file I/O overheads which make restrict our ability
% to run larger circuits within reasonable times.

% \figref{comp-queso} shows the results of the evaluation for seven benchmarks
% coming from four families of algorithms.
% %
% The ``QS'' column denotes \queso{} and the ``S'' column represents the \coam{} algorithm
% using \queso{} as the oracle.
% %
% In all cases, the \coam{} approach generates smaller circuits than the baseline \queso{}.
% %
% Similar to the case with \q{} (\secref{quartz}),
% we observe that the gap between the \coam{} approach and default \queso{} increases
% with circuit size.
% %
% Foe example,
% in the grover family,
% the ratio of optimization rate between \coam{} and \queso{}
% increases from $1.14$x for grover\_n7 to $1.4$x in grover\_n9.
% %
% For the circuit vqe\_n12,
% which is the largest circuit on the table,
% the final circuit size is $0.78$ times smaller compared to \queso{},
% which demonstrates the scalability of the \coam{} approach.
% %
% On average,
% the algorithm's circuits are $0.92$x smaller than those generated by \queso{}
% and the algorithm makes $1.27$x as many optimizations per second.
% %

% The evaluation demonstrates that the \coam{} algorithm
% improves the scalability of the \queso{} optimizer and
% consistently delivers better quality circuits.


% \section{Preliminary}

\paragraph{Notation} Consider a sentence of $T$ tokens $\vx=\{\vx_1,\ldots, \vx_T\}\in\gX$, and let $P$ be the unknown target language distribution, $\tilde P(\vx)$ be the empirical distribution of the training data (which is an approximation of $P$), and $Q$ be the distribution of our model at hand. Since our paper is also closely related to RLHF, we will also use $\pi$ to represent the distributions. In particular, we sometimes write $\pi_\theta$ for a distribution that is parameterized by $\theta$, where $\theta$ is usually the set of trainable parameters of the LLM; we write $\pr$ for a reference distribution that should be clear given the context. The next token prediction loss is minimizing the forward-KL between $P$ and $Q$. 





% \cleardoublepage
% \subsection{Eval half 2}
% \input{prelim2.tex}
% \subsection{Plots}
% \begin{figure*}[htbp]
	\centering
	\includegraphics[width=0.4\linewidth]{plots/gf.greedyscatter.png }\quad
	\includegraphics[width=0.4\linewidth]{plots/hhl.greedyscatter.png}\quad
\caption{Time vs. Size plots}
\end{figure*}
\begin{figure*}[htbp]
	\centering
	\includegraphics[width=0.4\linewidth]{plots/qft.greedyscatter.png}\quad
	\includegraphics[width=0.4\linewidth]{plots/vqe.greedyscatter.png}
\caption{Time vs. Size plots}
\end{figure*}
\begin{figure*}[htbp]
	\centering
	\includegraphics[width=0.4\linewidth]{plots/grover.greedyscatter.png}\quad
	\includegraphics[width=0.4\linewidth]{plots/qaoa.greedyscatter.png}\quad
\caption{Time vs. Size plots}
\end{figure*}
\begin{figure*}[htbp]
	\centering
	\includegraphics[width=0.4\linewidth]{plots/shor.greedyscatter.png}
\caption{Time vs. Size plots}
\end{figure*}

% \cleardoublepage
% \subsection{Cliff Eval half 1}
% \input{prelim1.clifft}
% \cleardoublepage
% \subsection{Cliff Eval half 2}
% \input{prelim2.clifft}
% \subsection{Plots Cliff}
% \begin{figure*}[htbp]
	\centering
	\includegraphics[width=0.4\linewidth]{plots/gf.greedyscatter.clifft.png }\quad
\caption{Time vs. Size plots}
\end{figure*}
\begin{figure*}[htbp]
	\centering
	\includegraphics[width=0.4\linewidth]{plots/qft.greedyscatter.clifft.png}\quad
	\includegraphics[width=0.4\linewidth]{plots/vqe.greedyscatter.clifft.png}
\caption{Time vs. Size plots}
\end{figure*}
\begin{figure*}[htbp]
	\centering
	\includegraphics[width=0.4\linewidth]{plots/grover.greedyscatter.clifft.png}\quad
	\includegraphics[width=0.4\linewidth]{plots/qaoa.greedyscatter.clifft.png}\quad
\caption{Time vs. Size plots}
\end{figure*}
\begin{figure*}[htbp]
	\centering
	\includegraphics[width=0.4\linewidth]{plots/shor.greedyscatter.clifft.png}
\caption{Time vs. Size plots}
\end{figure*}


