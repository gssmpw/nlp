Quantum circuit optimization is a fundamental problem in
quantum computing.
%
State-of-the-art optimizers require at least quadratic time in the size of the
circuit, which does not scale to larger circuits that are necessary
for obtaining quantum advantage, and are unable to make strong quality
guarantees.
%
This paper defines a notion of local optimality and shows that it is
possible to optimize circuits for local optimality efficiently by
proposing a circuit cutting-and-melding technique.
%
With this cut-and-meld technique, the algorithm cuts a circuit into
subcircuits, optimizes them independently, and melds them efficiently,
while also guaranteeing optimization quality.
%
Our implementation and experiments show that the algorithm is
practical and performs well, leading to more than an order of magnitude
performance improvement (on average) while also improving optimization
quality.
%
These results show that local optimality can be effective in
optimizing large quantum circuits, which are necessary for quantum
advantage.
%
These results, however, do not suggest stopping to develop global
optimizers, which remains to be an important goal.
%
It is likely, however, due to inherent complexity of the problem (it
is QMA hard), global optimizers may struggle to scale to larger
circuits efficiently.
%
Because our approach to local optimality is generic, it can scale
global optimizers to larger circuits by employing them as oracles for
local optimization.

\if0
Quantum circuit optimization is a fundamental problem in
quantum computing.
%
The problem is QMA hard, making it unlikely that there will be optimizers that
can guarantee global optimality efficiently.
%
%% Indeed, prior work has proposed optimizers for quantum circuits but
%% these optimizers can take long hours even for circuits for circuits
%% with several hundred gates.
%
This paper proposes a local-optimality approach to optimizing
quantum circuits.
%
Parameterized by a constant $\Omega$, local optimality requires that all
contiguous subcircuits with $\Omega$ layers be optimal.
%
We formalize local optimality and present rewrite rules for accomplishing them.
%
We then present an algorithm called \coam{} (Optimize and Compact) that ``schedules'' the application of the rewrite rules
to guarantee efficiency and local optimality.
%
The \coam{} algorithm optimizes a circuit in rounds, each of which optimizes small segments of the circuit and determines which segments to optimize by carefully tracking the optimizations and their impact on nearby segments.
%
We establish asymptotic efficiency bounds on the algorithm, including a linear-time bound for each round.

%
We show that the
algorithm is practical by implementing it and evaluating it.
%
The experiments show that local optimality can be significantly faster than state-of-the-art optimizers, especially as the circuit sizes increase, and does not appear to reduce optimization quality.
%
The results of the paper suggest that local optimality can be used to optimize quantum circuits effectively, including large circuits that will be required to obtain the benefits of quantum computing.
%
Interesting future directions for research include the development of a fast local optimizer (the implementation used here aims to be more generally applicable to existing optimizers so as to answer key research questions) and the development of verification techniques for local optimality.

%% The key question raised by the results in this paper is whether these
%% results could be generalized to other properties of circuits such as
%% circuit depth.
%% %
%% If this were possible, then the results would lead to an effective
%% optimization techniques for the key properties of interest in circuit
%% optimization, namely number of gates, including certain gates of
%% particular import (as this paper covers), and circuit depth and
%% related properties.
%% %
%% This problem does seem challenging and we look forward to seeing the
%% followup papers on this and related questions.
\fi
