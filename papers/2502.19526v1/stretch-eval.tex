\section{\queso{} comparison}


%





% %
% We integrate the \queso{} optimizer into our algorithm as an oracle
% and impose a one second time out for each call.
% %
% We note that \queso{} does not follow the given timeout strictly
% and often takes longer to run and return the result (around $4/5$ seconds per call).

% To evaluate the algorithm,
% we run it with \queso{} and compute the final circuit
% and the overall execution time, excluding time for file I/O (\secref{impl}).
% %
% Then, we execute default \queso{} on the whole circuit for the same total runtime
% and compare the quality of circuits produced.
% %
% For this experiment,
% we filter benchmarks based on their sizes and
% opt for benchmarks whose sizes are less than ten thousand gates.
% %
% This size limitation arises from file I/O overheads which make restrict our ability
% to run larger circuits within reasonable times.

% \figref{comp-queso} shows the results of the evaluation for seven benchmarks
% coming from four families of algorithms.
% %
% The ``QS'' column denotes \queso{} and the ``S'' column represents the \coam{} algorithm
% using \queso{} as the oracle.
% %
% In all cases, the \coam{} approach generates smaller circuits than the baseline \queso{}.
% %
% Similar to the case with \quartz{} (\secref{quartz}),
% we observe that the gap between the \coam{} approach and default \queso{} increases
% with circuit size.
% %
% Foe example,
% in the grover family,
% the ratio of optimization rate between \coam{} and \queso{}
% increases from $1.14$x for grover\_n7 to $1.4$x in grover\_n9.
% %
% For the circuit vqe\_n12,
% which is the largest circuit on the table,
% the final circuit size is $0.78$ times smaller compared to \queso{},
% which demonstrates the scalability of the \coam{} approach.
% %
% On average,
% the algorithm's circuits are $0.92$x smaller than those generated by \queso{}
% and the algorithm makes $1.27$x as many optimizations per second.
% %

\begin{figure*}
  \centering
  \begin{tabular}{rccccccccc}
 &  & \multicolumn{3}{c}{Output Size and Improvement} &  \multicolumn{3}{c}{Optimization Rate (gates/sec)} & &  \\ \cmidrule(lr){3-5} \cmidrule(lr){6-8}
  Benchmark & Input Size & QS & S & QS/S & QS & S & S/QS  & Time (s) & QS Time (s) \\ 

\midrule
 ham15-med       &   1061 & \textbf{855} (-19\%) & 859 (-19\%)             & 1.0x  & 20.6  &  24.67 & 1.2x    &   8 &   10 \\
 ham15-high      &   4365 & 3715 (-15\%)         & \textbf{3373} (-23\%)   & 1.1x  & 12.5  &  44.64 & 3.57x   &  22 &   52 \\
 hhl\_n7         &   5319 & 4323 (-19\%)         & \textbf{3878} (-27\%)   & 1.11x & 76.62 & 134.64 & 1.76x   &  10 &   13 \\
 hhl\_n9         &  63392 & 62896 (-1\%)         & \textbf{47714} (-25\%)  & 1.32x &  3.31 & 119.7  & 36.16x  & 130 &  150 \\
 gf2\^{}16\_mult &   2694 & 2694 (0\%)           & 2694 (0\%)              & 1.0x  &  0    &   0    & 1.0x    &   8 &   38 \\
 gf2\^{}32\_mult &   7553 & 6527 (-14\%)         & \textbf{3429} (-55\%)   & 1.9x  &  5.97 &  29.93 & 5.01x   & 137 &  172 \\
 grover\_n7      &   2479 & 2293 (-8\%)          & \textbf{2227} (-10\%)   & 1.03x & 15.5  &  26.46 & 1.71x   &   9 &   12 \\
 grover\_n9      &   8968 & 8936 (0\%)           & \textbf{8092} (-10\%)   & 1.1x  &  0.8  &  30.77 & 38.46x  &  28 &   40 \\
 grover\_n11     &  27136 & 27078 (0\%)          & \textbf{24493} (-10\%)  & 1.11x &  0.52 &  33.64 & 64.69x  &  78 &  111 \\
 grover\_n15     & 180497 & 180215 (0\%)         & \textbf{162909} (-10\%) & 1.11x &  0.06 &  30.13 & 502.17x & 583 & 4720 \\
 qft\_n48        &   4626 & 3870 (-16\%)         & 3870 (-16\%)            & 1.0x  & 29.08 &  50.28 & 1.73x   &  15 &   26 \\
 qft\_n64        &   7402 & 5510 (-26\%)         & 5510 (-26\%)            & 1.0x  & 55.65 &  59.76 & 1.07x   &  31 &   34 \\
 qft\_n80        &  10690 & 7150 (-33\%)         & 7150 (-33\%)            & 1.0x  & 49.17 &  75.68 & 1.54x   &  46 &   72 \\
 qft\_n96        &  14490 & 8790 (-39\%)         & 8790 (-39\%)            & 1.0x  & 45.6  &  85.43 & 1.87x   &  66 &  125 \\
 qpe\_n8         &   2040 & \textbf{1984} (-3\%) & 2040 (0\%)              & 0.97x &  6.22 &   0    & 0.0x    &   4 &    9 \\
 qpe\_n10        &   8476 & \textbf{8230} (-3\%) & 8476 (0\%)              & 0.97x &  2.56 &   0    & 0.0x    &  17 &   96 \\
 vqe\_n12        &  11022 & 6402 (-42\%)         & \textbf{3615} (-67\%)   & 1.77x & 73.33 & 198.8  & 2.71x   &  37 &   63 \\
 vqe\_n16        &  22374 & 13442 (-40\%)        & \textbf{8031} (-64\%)   & 1.67x & 69.24 & 163.11 & 2.36x   &  87 &  129 \\
 vqe\_n20        &  38462 & 23810 (-38\%)        & \textbf{14911} (-61\%)  & 1.6x  & 86.7  & 141.59 & 1.63x   & 166 &  169 \\
 vqe\_n24        &  59798 & 38018 (-36\%)        & \textbf{24767} (-59\%)  & 1.54x & 56.57 & 110.56 & 1.95x   & 316 &  385 \\
\midrule
\textbf{geomean} & & & & 1.18 & &  & 5.15 & & 
\end{tabular}
  \caption{The figure compares the performance of $\coamt{\quartzt{0.06}}$ against standard \queso{}.
  The label ``S'' represents $\coamt{\quartzt{0.06}}$ and the label ``QS'' represents \queso{}.
  %
  The figure compares the circuit sizes (lower is better) and the optimization rates (higher is better) of both approaches.
  %
  The optimization rate measures the number of optimizations performed per second of execution.
  %
  The figure also shows the improvement factor $QS/S$ for circuit size and $S/QS$ for optimization rate,
  quantifying the improvement achieved by \coam{}.
  %
  The Time column takes the time it takes to run both of them.
  %
  \queso{} sometimes runs longer because its timeout functionality is not very precise.
  %
  }
  \label{fig:comp-low-queso}
\end{figure*}



In this section,
we compare the circuit quality by the \coam{} algorithm with configurations \coamwith{\quartzt{0.06}}
and \coamwith{\quartzt{0.6}} against the standard \queso{} optimizer.
%
Our methodology is the same,
for each configuration we take the running time,
and then execute \queso{} with the same running time.
%

We note that even though \queso{} implements a timeout facility,
it does not adhere to it.
%
We observed that the tool takes an unpredictable amount of time
taking as much as $200$x the given timeout.
%
To address this,
we changed the code in \queso{}, making it check timeouts more eagerly.
%
Though our changes reduced the unpredictability,
there is some difference in the running time with \queso{} running
longer than our algorithm.
%
We note that our algorithm never runs for a larger time.
%

\figref{comp-low-queso} compares the configuration \coamwith{\quartzt{0.06}}
with \queso{}, when given the same timeout.
%
We observe that in three cases \queso{} generates a better quality circuit,
but it also takes more time.
%
In twelve cases, our algorithm generates better quality circuits,
and these cases correspond to larger sized circuits,
showcasing the scalability benefits of our approach.
%
On average, our algorithm generates smaller circuits by a factor of $1.18$x
and its average optimization rate is $5$x compared to \queso{}.
%

\figref{comp-med-queso} similarly compares the configuration  \coamwith{\quartzt{0.06}}.




\begin{figure*}
  \centering
  \begin{tabular}{rccccccccc}
 &  & \multicolumn{3}{c}{Output Size and Improvement} &  \multicolumn{3}{c}{Optimization Rate (gates/sec)} & &  \\ \cmidrule(lr){3-5} \cmidrule(lr){6-8}
  Benchmark & Input Size & QS & S & QS/S & QS & S & S/QS  & Time (s) & QS Time (s) \\ 

\midrule
 ham15-med       &   1061 & \textbf{821} (-23\%)  & 859 (-19\%)             & 0.96x & 2.42 &  2.08 & 0.86x  &    97 &    99 \\
 ham15-high      &   4365 & 3373 (-23\%)          & \textbf{3363} (-23\%)   & 1.0x  & 2.57 &  2.61 & 1.02x  &   383 &   386 \\
 hhl\_n7         &   5319 & 3793 (-28\%)          & \textbf{3581} (-33\%)   & 1.06x & 1.22 &  1.4  & 1.15x  &  1242 &  1253 \\
 hhl\_n9         &  63392 & 52689 (-17\%)         & \textbf{45442} (-28\%)  & 1.16x & 0.99 &  1.66 & 1.68x  & 10803 & 10824 \\
 gf2\^{}16\_mult &   2694 & 2694 (0\%)            & \textbf{2691} (0\%)     & 1.0x  & 0    &  0.01 & 1.0x   &   284 &   291 \\
 gf2\^{}32\_mult &   7553 & 7553 (0\%)            & \textbf{3429} (-55\%)   & 2.2x  & 0    & 10.84 & 1.0x   &   380 &   387 \\
 grover\_n7      &   2479 & 2234 (-10\%)          & \textbf{2227} (-10\%)   & 1.0x  & 1.06 &  1.1  & 1.04x  &   229 &   232 \\
 grover\_n9      &   8968 & 8254 (-8\%)           & \textbf{8091} (-10\%)   & 1.02x & 0.85 &  1.04 & 1.22x  &   839 &   843 \\
 grover\_n11     &  27136 & 27078 (0\%)           & \textbf{24464} (-10\%)  & 1.11x & 0.02 &  0.99 & 49.5x  &  2712 &  2727 \\
 grover\_n15     & 180497 & 180215 (0\%)          & \textbf{162909} (-10\%) & 1.11x & 0.01 &  1.63 & 163.0x & 10800 & 21921 \\
 qft\_n48        &   4626 & \textbf{3333} (-28\%) & 3524 (-24\%)            & 0.95x & 0.58 &  0.5  & 0.86x  &  2217 &  2228 \\
 qft\_n64        &   7402 & \textbf{4811} (-35\%) & 5036 (-32\%)            & 0.96x & 0.78 &  0.71 & 0.91x  &  3329 &  3341 \\
 qft\_n80        &  10690 & \textbf{6314} (-41\%) & 6546 (-39\%)            & 0.96x & 1.02 &  0.97 & 0.95x  &  4272 &  4282 \\
 qft\_n96        &  14490 & \textbf{7947} (-45\%) & 8042 (-44\%)            & 0.99x & 1.24 &  1.23 & 0.99x  &  5236 &  5278 \\
 qpe\_n8         &   2040 & \textbf{1956} (-4\%)  & 2035 (0\%)              & 0.96x & 0.08 &  0.02 & 0.25x  &   241 &  1058 \\
 qpe\_n10        &   8476 & \textbf{8155} (-4\%)  & 8474 (0\%)              & 0.96x & 0.36 &  0    & 0.0x   &   879 &   886 \\
 vqe\_n12        &  11022 & 5982 (-46\%)          & \textbf{3572} (-68\%)   & 1.67x & 8.08 & 12.02 & 1.49x  &   619 &   624 \\
 vqe\_n16        &  22374 & 13442 (-40\%)         & \textbf{7949} (-64\%)   & 1.69x & 6.75 & 11.23 & 1.66x  &  1284 &  1323 \\
 vqe\_n20        &  38462 & 23810 (-38\%)         & \textbf{14858} (-61\%)  & 1.6x  & 7.04 & 11.79 & 1.67x  &  2002 &  2080 \\
 vqe\_n24        &  59798 & 59798 (0\%)           & \textbf{24668} (-59\%)  & 2.42x & 0    & 10.72 & 1.0x   &  3277 &  3419 \\
\midrule
\textbf{geomean} & & & & 1.18 & &  & 2.92 & & 
\end{tabular}
  \caption{The figure compares the performance of $\coamt{\quartzt{0.6}}$ against standard \queso{}.
  The label ``S'' represents $\coamt{\quartzt{0.06}}$ and the label ``QS'' represents \queso{}.
  %
  The figure compares the circuit sizes (lower is better) and the optimization rates (higher is better) of both approaches.
  %
  The optimization rate measures the number of optimizations performed per second of execution.
  %
  The figure also shows the improvement factor $QS/S$ for circuit size and $S/QS$ for optimization rate,
  quantifying the improvement achieved by \coam{}.
  %
  The Time column takes the time it takes to run both of them.
  %
  \queso{} sometimes runs longer because its timeout functionality is not very precise.
  %
  }
  \label{fig:comp-med-queso}
\end{figure*}
