
We perform an empirical evaluation of the effectiveness of the
local optimality approach for quantum circuits.
%
Specifically, we consider the following research questions (RQ).
\begin{description}
    % \item[Question I:] Can the \algname algorithm be implemented with reasonable constant factors?
    % \item[Question II:] Can the implementation use existing different optimizers as oracles?
  \item[RQ I:] Is local optimality and the \algname{} algorithm
    effective in terms of efficiency, scalability, and optimization
    quality?

 \item[RQ II:] Is empirical performance consistent with the asymptotic bound?

 \item[RQ III:] What is the role of the lazy \textsf{meld} operation? 
 
 \item[RQ IV:] What is the impact of segment size $\Omega$ and
 compaction on the local optimality and performance of \algname{} algorithm?
\end{description}

To answer these questions,
we implement the \algname{} algorithm and integrate
it with \voqc{}~\cite{hietala2021verified} as an oracle.
%
We chose \voqc{} because it is overall the best optimizer both in
terms of efficiency and quality of optimization among all the
optimizers that we have experimented with.
%
%two different optimizers, \voqc{}~\cite{hietala2021verified} and \feyntool{}~\cite{amy2019formal}, as oracles.
%
%We evaluate the effectiveness of local optimality and \algname{}, on two gate sets
%and compare it with four state-of-the-art optimizers:
%\quartz{}, \queso{}, \voqc{}, and \feyntool{}~\cite{quartz-2022, queso-2023,hietala2021verified,amy2019formal}.
%
We evaluate the effectiveness of local optimality and \algname{}, on
the Nam gate set~\cite{Nam_2018} and compare it with three
state-of-the-art optimizers \quartz{}, \queso{}, and
\voqc{}~\cite{quartz-2022, queso-2023,hietala2021verified}.
%

In brief, these experiments show that our cut-and-meld algorithm
delivers fast optimization while closely matching (within 0.1\%) or
improving the optimization quality for all circuits.
%
These results show that the local optimality approach can be effective in
optimizing large quantum circuits, and can help scale existing optimizers. 

\iffull
In \appref{clifft}, 
\else
In the Appendix, 
\fi
we present results for
the Clifford+T gate set by using the \feyntool{}~\cite{amy2019formal},
as an oracle.
%
We omitted these from the main body of the paper due to space reasons
but note that they are similar to the results presented here in terms
of efficiency and quality.

\subsection{Implementation}
\label{sec:impl}

To evaluate whether the \algname{} algorithm  (\secref{algorithm})
is practically feasible,
we implemented \algname{}  in SML
(Standard ML), which comes with an optimizing compiler, MLton, that
can generate fast executables.
%
Our implementation closely follows the algorithm description.
%
It uses the layered circuit representation and
represents circuits as an array of arrays, where
each array denotes a ``layer'' of the circuit.
%
The implementation splits and joins circuit segments by
splitting and joining the corresponding arrays,
performing rounds of optimization and compaction.
%

As described in \secref{convergence-thresh}, our implementation
allows user control over convergence through
a specified convergence ratio $\epsilon$, where $0 \le \epsilon \le 1$.
In the evaluation, we choose $\epsilon = 0.01$, and analyze this choice in
\secref{compaction}.
% and 
%\iffull
%\appref{converge}.
%\else
%the Appendix.
%\fi
Note that regardless of $\epsilon$,
the implementation always guarantees that output is \wopttext.

Our implementation is parametric in the oracle being used. To allow calls to existing optimizers,
we use MLton's foreign function interface, which supports cross-language calls to C++.
%
Specifically, to use an existing optimizer as an oracle, we only need to provide a C++ wrapper that takes a circuit in QASM format as input
and returns an optimized circuit as output.


\if0
As described in \secref{convergence-thresh}, our implementation
allows user control over convergence through
a specified convergence ratio $\epsilon$, where $0 \le \epsilon \le 1$.
%
The implementation terminates when an optimization round reduces the cost by
a smaller fraction than $\epsilon$.
%
For example, with $\epsilon = 0.01$ and using the number of gates as the cost function,
the algorithm stops when fewer than $1\%$ of the gates are removed in a round.
%
Regardless of $\epsilon$,
the implementation always guarantees that output is \wopttext;
setting $\epsilon = 0$ additionally ensures that the circuit is locally
optimal.
%
The convergence ratio thus acts as an ``optimization level'' parameter,
similar to those in many optimizers
(e.g., \texttt{"-O"} option of C/C++ compilers, \texttt{"opt-level"} option of Rust).
%
We evaluate both settings $\epsilon = 0.01$ and $\epsilon = 0$ and find
no difference in output quality.
%
As we study in \secref{converge}, the main reason for this is that after the first one/two rounds,
the circuit is almost optimal and any further rounds
do not improve the circuit significantly.
%
%We present the results for $\epsilon = 0.01$ in the main body and
%for $\epsilon = 0$ in the Appendix.

Similar to the algorithm, our implementation is parametric in the oracle being used.
%
To allow calls to existing optimizers, which may be written in different languages, we use MLton's foreign function interface, which supports cross-language calls to C++.
%
Specifically, to use an existing optimizer as an oracle, we only need to provide a C++ wrapper that takes a circuit in QASM format as input
and returns an optimized circuit as output.
%
Using this approach, we have implemented wrappers for two different optimizers: \feyntool{}~\cite{amy2019formal} and \voqc{}~\cite{hietala2021verified}.
\fi
%

% Our implementation approach is to serialize
% the input and output circuits of each oracle call in QASM format,
% and make a call to another language.
% %
% While this introduces some overhead,
% it allows us to evaluate our algorithm with different oracles.
%
% These overheads could be avoided by integrating the oracle more
% deeply with \algname{}, for example by customizing the oracle to
% directly accept \algname{}'s circuit representation or vice versa.
% %
% In the experiments we discuss these overheads when relevant.
% %
% However,
% our goal is to evaluate the algorithm and answer broader research questions.
% %
% \sr{$\uparrow$ This raises questions about how we managed this overhead. Just need to
% add a sentence explaining what we did.}
%
% Such an implementation may be desirable for the purposes of creating
% a fast local optimizer
% but is less useful for us.
% Our goal is to evaluate the algorithm
% for different oracles and answer broader research questions.
%


\subsection{Benchmarks and gate set}
% \myparagraph{Benchmarks.}
\begin{table*}[tb]
\centering
\caption{IOPS Results}
\label{tab:iops}
\begin{tabular}{|l|r|r|r|r|r|r|r|r|} 
\hline
\multirow{2}{*}{}   & \multicolumn{2}{c|}{Upstream} & \multicolumn{2}{c|}{Ublk Frontend} & \multicolumn{2}{c|}{C-R Comm.} & \multicolumn{2}{c|}{DBS} \\
\cline{2-9}
                & Read & Write                      & Read & Write                      & Read & Write                      & Read & Write \\ 
\hline
Full engine     & \textbf{17k}  & \textbf{13k}      & 95k  & 27k                        & 110k & 27k                        & \textbf{112k} & \textbf{115k} \\ 
\hline
Without storage & 19k  & 19.5k                      & 100k & 100k                       & \textbf{129k} & \textbf{115k}     & \multicolumn{2}{l|}{$\rightarrow$} \\ 
\hline
Frontend only   & 20k  & 20k                        & \textbf{280k} & \textbf{255k}     & \multicolumn{4}{l|}{$\rightarrow$} \\
\hline
\end{tabular}
\end{table*}

\begin{table*}[tb]
\centering
\caption{Bandwidth Results (MB/s)}
\label{tab:bandwidth}
\begin{tabular}{|l|r|r|r|r|r|r|r|r|} 
\hline
\multirow{2}{*}{}   & \multicolumn{2}{c|}{Upstream} & \multicolumn{2}{c|}{Ublk Frontend} & \multicolumn{2}{c|}{C-R Comm.} & \multicolumn{2}{c|}{DBS} \\
\cline{2-9}
                & Read & Write                      & Read & Write                      & Read & Write                      & Read & Write \\ 
\hline
Full engine     & \textbf{300}  & \textbf{275}      & 1000 & 1000                       & 1250 & 1250                       & \textbf{1250}& \textbf{1250} \\ 
\hline
Without storage & 670  & 415                        & 1250 & 1250                       & \textbf{1250}  & \textbf{1250}    & \multicolumn{2}{l|}{$\rightarrow$} \\ 
\hline
Frontend only   & 750  & 415                        & \textbf{2000} & \textbf{2000}     & \multicolumn{4}{l|}{$\rightarrow$} \\
\hline
\end{tabular}
\end{table*}


\section{Evaluation}

\subsection{Setup}

To evaluate the system we deploy the controller and replica on separate nodes. In contrast to our development setup, this allows us to account for any effects related to the physical network. We use two identical nodes from our local cluster equipped with dual Intel Xeon E5-2620v2 CPUs, 128 GB RAM, connected via 10 Gbps Ethernet. Data is stored in a Samsung PM1733 NVMe drive. These machines have limited CPU core counts compared to state-of-the-art technology, however their specifications and performance is better aligned with typical VM offerings currently available in the cloud.

Actually, we initially evaluated our system in AWS, using two c5d.2xlarge EC2 instances, a cost-efficient option also used by Longhorn developers for publishing their benchmarks \cite{longhorn_report}.
However, EC2 instances have limited maximum provisioned IOPS, regardless of the hard drive used. As expected, the software performed to the machine's limit, reaching AWS's 40k IOPS cap. Overcoming these limitations in AWS requires using significantly more expensive EC2 instances. To harness the performance of the features described in this work, deployments must use instances and volumes at least 5 times more expensive than c5d.2xlarge. In any case, the behavior of the system on higher-performance cloud nodes is similar to the one we observe in our local setup.

\subsection{Results}

The results for IOPS and bandwidth are presented in Tables \ref{tab:iops} and \ref{tab:bandwidth} respectively. As in the previous section, in each experiment we do multiple runs to measure (shown as table rows): 
\begin{enumerate*}[label=(\roman*)]
    \item the \textit{full engine} performance, end-to-end, which includes writing blocks to the disk,
    \item the performance up to the replica \textit{without storage} using a null storage drive, where I/Os are immediately completed at the replica, and
    \item the \textit{frontend only} using a null backend, where I/Os are immediately completed at the controller.
\end{enumerate*}
We follow the same top-down approach, starting from upstream Longhorn and integrating each new feature in the same progression (shown as table columns). This indicates where the bottleneck is in each step and highlights how each solution (ublk frontend, controller-replica communication, DBS) contributes to the performance of the whole system.
For our experiments we use fio to measure IOPS (4k, random) and bandwidth (1 MB, sequential). All I/Os are direct to the virtual block device, bypassing kernel caches.

Starting with 17k/13k read/write IOPS when running the full stack, we isolate the first bottleneck at the frontend and measure it cannot achieve more than 20k IOPS using the upstream TGT-based solution. Integrating the ublk frontend yields a 28x boost, that allows the next layers of the engine to perform better, except for write IOPS, where the storage scheme of Longhorn fails to take advantage of the faster frontend. The bandwidth is also largely affected by the new frontend (almost 2.5x/4.8x for reads/writes compared to upstream), which enables the system to almost saturate the 10 Gbps links.
In general, these numbers support the general consensus among the community that ublk is an ideal framework to export SDS stacks to applications.

The next step is to evaluate the improved controller-replica communication implementation. Even with the ublk frontend, upstream Longhorn cannot achieve over 100k IOPS going up to the replica (null storage). Our modified communication scheme boosts performance by 29\%/15\% for reads/writes. There is still room for improvement, as we would ideally like to match the frontend performance, however it is enough for the engine to reach the full 10 Gbps bandwidth even with the default storage backend. The updated controller-replica communication also boosts random read IOPS to storage by 17\%. Write IOPS are not affected, which indicates that the bottleneck is in the storage backend itself (other runs have proved that this limitation is caused by write versioning). Lastly, integration of the DBS backend raises write IOPS to the level of reads; the whole modified system now performs an order of magnitude better than the default. Note that DBS is designed to keep the same performance level regardless of the number of volume snapshots.

% OLD TEXT

% Following the implementation of each feature, we measured its performance individually and observed a minimum performance boost resulting from improved controller-replica communication and DBS integration. Our findings indicate that each layer's bottleneck and solution proposed are crucial, so the next layer provides a boost in performance. So in our benchmarks, we used a top-down approach, integrating each feature step by step to highlight its importance. We also split our benchmarks into three parts, disabling the functionality of the engine in certain parts in order to benchmark each feature separately.
 
% Our setup included one controller and one replica, without the use of snapshots. We tested these changes in our local cluster using two identical systems equipped with Intel(R) Xeon(R) CPU E5-2620 v2 @ 2.10GHz, 125GB memory, and … hard drives. The machines were connected via a 10 Gbps connection, which is enough for some experiments but limits the performance during some bandwidth benchmarks.

% \subsection{Frontend}

% During our frontend evaluation experiments, we used our integration and compared it with the live version of the Longhorn engine. The findings can be seen in Table 1. 
% We can see there is a big boost in the frontend's performance using the ublk framework. The first layer of the engine saw a 28x boost in performance, both on reads and writes IOPS. Regarding the next two layers of the engine, we saw a big boost even using the full engine, except for write IOPS, where the Longhorn's IO methods fail to take advantage of the faster frontend. The bandwidth seems to get a boost too, especially the frontend part. 

% Overall these numbers support our belief that ublk is a modern and more performant framework to implement software-defined storage frontends. 

% \begin{table*}[!th]
\centering
\resizebox{\textwidth}{!}{%
\begin{tabular}{@{}llcccccccccc@{}}
\toprule
& & \multicolumn{2}{c}{\textbf{Intent Detection}} & \multicolumn{2}{c}{\textbf{Topic Mining}} & \multicolumn{2}{c}{\textbf{Domain Discovery}} & \multicolumn{1}{c}{\textbf{Type}} & \multicolumn{1}{c}{\textbf{Emotion}} & \\
\cmidrule(lr){3-4} \cmidrule(lr){5-6} \cmidrule(lr){7-8} \cmidrule(lr){9-9} \cmidrule(lr){10-10}  %\cmidrule(lr){11-11}
\textbf{Model} & \textbf{Method} & \textbf{BANKING} & \textbf{CLINC} & \textbf{Reddit} & \textbf{StackEx} & \textbf{MTOP} & \textbf{CLINC(D)} & \textbf{FewEvent} & \textbf{GoEmotion} & \textbf{AVG} \\ \midrule \midrule
GPT-4o-mini & Standard Prompting & 0.652 & 0.792 & 0.534 & 0.482 & 0.896 & 0.536 & 0.630 & 0.378 & 0.613 \\
& Self-Consistency & 0.666 & 0.802 & 0.586 & 0.494 & 0.902 & 0.530 & 0.640 & 0.382 & 0.625 \\
& TestNUC & 0.712 & 0.858 & 0.614 & 0.528 & 0.936 & 0.544 & 0.674 & 0.410 & 0.660 \\
& \cellcolor{gray!18}TestNUC\textdagger & \cellcolor{gray!18}\textbf{0.764} & \cellcolor{gray!18}\textbf{0.864} & \cellcolor{gray!18}\textbf{0.646} & \cellcolor{gray!18}\textbf{0.540} & \cellcolor{gray!18}\textbf{0.948} & \cellcolor{gray!18}\textbf{0.554} & \cellcolor{gray!18}\textbf{0.680} & \cellcolor{gray!18}\textbf{0.414} & \cellcolor{gray!18}\textbf{0.676} \\ \midrule \midrule
Llama-3.1-8B & Standard Prompting & 0.572 & 0.726 & 0.502 & 0.492 & 0.892 & 0.528 & 0.530 & 0.332 & 0.572 \\
& Self-Consistency & 0.620 & 0.774 & 0.564 & 0.526 & 0.902 & 0.518 & 0.564 & 0.340 & 0.601 \\
& TestNUC & 0.694 & 0.806 & 0.618 & 0.558 & 0.934 & 0.528 & 0.596 & 0.356 & 0.636 \\
& \cellcolor{gray!18}TestNUC\textdagger & \cellcolor{gray!18}\textbf{0.724} & \cellcolor{gray!18}\textbf{0.812} & \cellcolor{gray!18}\textbf{0.646} & \cellcolor{gray!18}\textbf{0.576} & \cellcolor{gray!18}\textbf{0.940} & \cellcolor{gray!18}\textbf{0.542} & \cellcolor{gray!18}\textbf{0.614} & \cellcolor{gray!18}\textbf{0.360} & \cellcolor{gray!18}\textbf{0.652} \\ \midrule \midrule
Claude-3-Haiku & Standard Prompting & 0.680 & 0.848 & 0.486 & 0.564 & 0.892 & 0.552 & 0.594 & 0.336 & 0.619 \\
& Self-Consistency & 0.702 & 0.870 & 0.510 & 0.578 & 0.904 & 0.564 & 0.568 & 0.350 & 0.631 \\
& TestNUC & 0.762 & 0.894 & 0.596 & 0.588 & 0.940 & 0.590 & 0.620 & 0.348 & 0.667 \\
& \cellcolor{gray!18}TestNUC\textdagger & \cellcolor{gray!18}\textbf{0.804} & \cellcolor{gray!18}\textbf{0.902} & \cellcolor{gray!18}\textbf{0.612} & \cellcolor{gray!18}\textbf{0.600} & \cellcolor{gray!18}\textbf{0.946} & \cellcolor{gray!18}\textbf{0.622} & \cellcolor{gray!18}\textbf{0.660} & \cellcolor{gray!18}\textbf{0.368} & \cellcolor{gray!18}\textbf{0.689} \\ \midrule \midrule
GPT-4o & Standard Prompting & 0.746 & 0.924 & 0.712 & 0.674 & 0.962 & 0.614 & 0.682 & 0.406 & 0.715 \\
& Self-Consistency & 0.758 & 0.922 & 0.720 & 0.688 & 0.958 & 0.624 & 0.696 & 0.426 & 0.724 \\
&TestNUC & 0.804 & 0.934 & 0.744 & \textbf{0.710} & 0.974 & 0.644 & 0.692 & 0.446 & 0.744 \\
& \cellcolor{gray!18}TestNUC\textdagger & \cellcolor{gray!18}\textbf{0.824} & \cellcolor{gray!18}\textbf{0.940} & \cellcolor{gray!18}\textbf{0.750} & \cellcolor{gray!18}\textbf{0.710} & \cellcolor{gray!18}\textbf{0.978} & \cellcolor{gray!18}\textbf{0.654} & \cellcolor{gray!18}\textbf{0.708} & \cellcolor{gray!18}\textbf{0.464} & \cellcolor{gray!18}\textbf{0.754} \\
\bottomrule
\end{tabular}%
}
\caption{Accuracy comparison with Standard Prompting and Self-Consistency across four diverse LLMs. TestNUC consistently improves the inference performance on all benchmark datasets. $\dagger$ denotes that 50 neighbors are utilized.}
\label{tab:main_compare_sc}
\end{table*}
% % \begin{table*}[t]
%     \caption{Performance Metrics on MMQA Subset after the attack with poisoned images cointaining adversarial noise. Capt. stands for Captions and denotes wether the re-ranker has access to image captions. }
%     \label{tab:mmqa_adv}
%     \centering
%     \resizebox{\textwidth}{!}{%
%     \begin{tabular}{@{}lc|cc|cc@{}}
%         \toprule
%         \textbf{Retriever} & \textbf{Reranker} & \textbf{Capt.} & \multicolumn{2}{c|}{\textbf{Original} (\small{Before $\rightarrow$ After})} & \multicolumn{2}{c@{}}{\textbf{Poisoned}} \\ 
%         \textbf{CLIP} & \textbf{LLaVA} &  & \textbf{Recall (\%)} & \textbf{Accuracy (\%)} & \textbf{Recall (\%)} & \textbf{Accuracy (\%)} \\
%         \midrule
%         $K=1$   & \xmark                      & -           & 83.7 $\rightarrow$ 11.4 \textcolor{red}{\small{(-72.3)}}             & 61.0 $\rightarrow$ 18.4 \textcolor{red}{\small{(-42.6)}}                 & 87.9   &  58.2  \\
%         $K=5$   & \xmark                      & -           & 92.9 $\rightarrow$ 92.2 \textcolor{red}{\small{(-0.7)}}           & 39.0 $\rightarrow$ 30.5 \textcolor{red}{\small{(-8.5)}}                   & 100.0   & 36.2         \\
%         $K=1$   & $N=1$          & \cmark      & 84.4 $\rightarrow$ 30.5 \textcolor{red}{\small{(-53.9)}}                    & 62.4 $\rightarrow$ 29.1 \textcolor{red}{\small{(-33.3)}}          & 63.8   & 48.2         \\
%         $K=5$   & $N=1$          & \xmark      & 70.2 $\rightarrow$ 44.0 \textcolor{red}{\small{(-26.2)}}                    & 58.9$ \rightarrow$ 39.7 \textcolor{red}{\small{(-19.2)}}          & 45.4  & 38.3          \\
%         \bottomrule
%     \end{tabular}%
%     }
% \end{table*}

\begin{table*}[t]
    \caption{Performance Metrics on MMQA Subset after the attack with poisoned images cointaining adversarial noise. Capt. stands for Captions and denotes wether the re-ranker has access to image captions. }
    \label{tab:mmqa_adv}
    \centering
    \resizebox{\textwidth}{!}{%
    \begin{tabular}{@{}cllcccccccc@{}}
        \toprule
       \multicolumn{11}{c}{\textbf{Retriever}: CLIP-ViT-L \textbf{Reranker}: LLaVA \textbf{Generator}: LLaVA} \\
       \midrule
       & \textbf{Retriever} & \textbf{Reranker} & \textbf{Capt.} & \multicolumn{3}{c|}{\textbf{Original Recall (\%)}} & \multicolumn{3}{c}{\textbf{Original Accuracy (\%)}} & Poisoned \\
       & &  &  & Before & After & Change & Before & After & Change & Recall (\%) \\
        $K=1$   & \xmark                      & -           & 83.7 $\rightarrow$ 11.4 \textcolor{red}{\small{(-72.3)}}             & 61.0 $\rightarrow$ 18.4 \textcolor{red}{\small{(-42.6)}}                 & 87.9   &  58.2  \\
        $K=5$   & \xmark                      & -           & 92.9 $\rightarrow$ 92.2 \textcolor{red}{\small{(-0.7)}}           & 39.0 $\rightarrow$ 30.5 \textcolor{red}{\small{(-8.5)}}                   & 100.0   & 36.2         \\
        $K=1$   & $N=1$          & \cmark      & 84.4 $\rightarrow$ 30.5 \textcolor{red}{\small{(-53.9)}}                    & 62.4 $\rightarrow$ 29.1 \textcolor{red}{\small{(-33.3)}}          & 63.8   & 48.2         \\
        $K=5$   & $N=1$          & \xmark      & 70.2 $\rightarrow$ 44.0 \textcolor{red}{\small{(-26.2)}}                    & 58.9$ \rightarrow$ 39.7 \textcolor{red}{\small{(-19.2)}}          & 45.4  & 38.3          \\
        \bottomrule
    \end{tabular}%
    }
\end{table*}




% \subsection{Controller-Replica communication}

% After integrating the ublk frontend, we also included the improved controller-replica communication in our experiments. We use the same three breakpoints to evaluate the solution. 

% This improvement was focused on improving the server-client architecture, and our experiments indicate that we manage this. Compared to the longhorn live version, this approach serves requests with 10x IOPS. This approach does not match the performance gain on the front-end but is still a solid solution. We also measured a big boost in the bandwidth performance to numbers that are the limit of the connection used between the machines. Using a better connection, this number may see a small improvement based on some experiments performed in an environment where the replica and the controller were on the same machine, removing the network limitations.

% Using this solution, the Longhorn's backend managed to achieve a boost in performance. Random reads have seen a boost similar to the previous layer when writes double their performance compared with the previous solution. These findings seem to solidify our theory that the Longhorn backend needs a new approach.

% \subsection{Block storage}
% Our last integration was the DBS backend for the replicas IO. This method, despite being in early development, seems to perform better regarding IOPS. This modern solution managed to match the rest of the system's performance, achieving 10x IOPS compared to Longhorn's backend option on writes where Longhorn's implementation failed to keep up with the rest of the system.

% \textbf{DBS performance on vaccum}
% \begin{table}[tbp]
  \caption{Performance of GPT-4o and Mistral-7B-Instruct-v0.3 as listwise LLM re-rankers on the NevIR test set with few-shot prompting.}
  \label{tab:few_shot}
  \centering
  \begin{tabular}{lcc}
    \toprule
    \textbf{Model Name} & \textbf{Shots} & \textbf{NevIR Score} \\
    \midrule
    \multirow{4}{*}{GPT-4o} & Zero-shot & 70.1\% \\
                             & 1-shot    & 72.0\% \\
                             & 3-shot    & 74.5\% \\
                             & 5-shot    & 76.9\% \\
    \midrule
    \multirow{4}{*}{Mistral-7B-Instruct-v0.3} & Zero-shot & 46.3\% \\
                                              & 1-shot    & 42.6\% \\
                                              & 3-shot    & 37.1\% \\
                                              & 5-shot    & 39.0\% \\
    \bottomrule
  \end{tabular}
\end{table}


    

% \begin{table*}[!htbp]
\caption{Performance comparison of semi-supervised learning across methodologies.}
\label{table:table4}
\centering
\resizebox{\textwidth}{!}{
\renewcommand{\arraystretch}{1.2}
{\tiny
\begin{tabular}{c|c|c|ccc|ccc|ccc} 
\hline
\multicolumn{12}{c}{\textit{1\% of labeled data}}                                                                                                                                                                                                                                                                                                                                                                                                                                                                                                     \\ 
\hline
\multirow{2}{*}{\begin{tabular}[c]{@{}c@{}}\textit{Model}\\\textit{Name}\end{tabular}}                                                   & \multirow{2}{*}{\begin{tabular}[c]{@{}c@{}}\textit{Training}\\\textit{Type}\end{tabular}} & \multirow{2}{*}{\textit{Modality (Count)}} & \multicolumn{3}{c|}{\begin{tabular}[c]{@{}c@{}}\textit{Sleep Stage}\\\textit{Classification}\end{tabular}} & \multicolumn{3}{c|}{\begin{tabular}[c]{@{}c@{}}\textit{Apnea}\\\textit{Detection}\end{tabular}} & \multicolumn{3}{c}{\begin{tabular}[c]{@{}c@{}}\textit{Hypopnea}\\\textit{Detection}\end{tabular}}  \\ 
\cline{4-12}
                                                                                       &                                                                                            &                                            & \textit{ACC}   & \textit{MF1}   & \textit{K}                                                                & \textit{ACC}   & \textit{MF1}   & \textit{K}                                                    & \textit{ACC}   & \textit{MF1}   & \textit{K}                                                       \\ 
\hline
SalientSleepNet \cite{ref9}                                                                       & Supervised                                                                                 & EEG1
  + EOG1                              & 61.44          & 53.49          & 0.49                                                                      & 79.34          & 45.32          & 0.02                                                          & 50.22          & 39.65          & 0.01                                                             \\
SleepFM \cite{ref23}                                                                                & SSL                                                                                        & EEG1
  + EOG1                              & 70.14          & 61.49          & 0.60                                                                      & 93.48          & 52.02          & 0.07                                                          & 50.95          & 39.77          & 0.00                                                             \\
SynthSleepNet                                                                          & SSL                                                                                        & EEG1
  + EOG1                              & 78.04          & 65.61          & 0.70                                                                      & 95.39          & 54.08          & 0.10                                                          & 63.87          & 49.33          & 0.10                                                             \\
SynthSleepNet+TCM                                                                      & SSL                                                                                        & EEG1
  + EOG1                              & \uline{84.71}  & \uline{76.48}  & \uline{0.79}                                                              & 95.78          & 54.64          & 0.11                                                          & 65.93          & 50.82          & 0.12                                                             \\ 
\hline
SalientSleepNet \cite{ref9}                                                                       & Supervised                                                                                 & EEG1
  + EOG1 + EMG1                       & 60.75          & 52.90          & 0.48                                                                      & 83.64          & 46.93          & 0.02                                                          & 50.34          & 39.50          & 0.00                                                             \\
SleepFM \cite{ref23}                                                                               & SSL                                                                                        & EEG1
  + EOG1 + EMG1                       & 71.73          & 62.89          & 0.62                                                                      & 96.51          & 55.98          & 0.13                                                          & 55.07          & 42.49          & 0.02                                                             \\
SynthSleepNet                                                                          & SSL                                                                                        & EEG1
  + EOG1 + EMG1                       & 76.80          & 65.48          & 0.68                                                                      & 98.88          & 66.60          & 0.33                                                          & 69.98          & 54.32          & 0.17                                                             \\
SynthSleepNet+TCM                                                                      & SSL                                                                                        & EEG1
  + EOG1 + EMG1                       & \textbf{85.01} & \textbf{78.34} & \textbf{0.79}                                                             & 99.09          & 68.96          & 0.38                                                          & 74.05          & 57.44          & 0.21                                                             \\ 
\hline
SalientSleepNet \cite{ref9}                                                                       & Supervised                                                                                 & EEG1
  + EOG1 + ECG1                       & 48.68          & 42.60          & 0.33                                                                      & 78.10          & 44.80          & 0.01                                                          & 55.23          & 43.16          & 0.04                                                             \\
SleepFM  \cite{ref23}                                                                              & SSL                                                                                        & EEG1
  + EOG1 + ECG1                       & 66.16          & 60.20          & 0.56                                                                      & 94.68          & 53.21          & 0.09                                                          & 60.08          & 46.46          & 0.07                                                             \\
SynthSleepNet                                                                          & SSL                                                                                        & EEG1
  + EOG1 + ECG1                       & 77.29          & 65.13          & 0.69                                                                      & \uline{99.23}  & \uline{71.00}  & \uline{0.42}                                                  & \uline{75.40}  & \uline{59.03}  & \uline{0.24}                                                     \\
SynthSleepNet+TCM                                                                      & SSL                                                                                        & EEG1
  + EOG1 + ECG1                       & 83.77          & 75.74          & 0.78                                                                      & \textbf{99.35} & \textbf{72.94} & \textbf{0.46}                                                 & \textbf{76.93} & \textbf{60.34} & \textbf{0.25}                                                    \\ 
\hline
\multicolumn{12}{c}{\textit{5\% of labeled data}}                                                                                                                                                                                                                                                                                                                                                                                                                                                                                                     \\ 
\hline
\multirow{2}{*}{\begin{tabular}[c]{@{}c@{}}\textit{Model}\\\textit{Name}\end{tabular}} & \multirow{2}{*}{\begin{tabular}[c]{@{}c@{}}\textit{Training}\\\textit{Type}\end{tabular}}  & \multirow{2}{*}{\textit{Modality (Count)}} & \multicolumn{3}{c|}{\begin{tabular}[c]{@{}c@{}}\textit{Sleep Stage}\\\textit{Classification}\end{tabular}} & \multicolumn{3}{c|}{\begin{tabular}[c]{@{}c@{}}\textit{Apnea}\\\textit{Detection}\end{tabular}} & \multicolumn{3}{c}{\begin{tabular}[c]{@{}c@{}}\textit{Hypopnea}\\\textit{Detection}\end{tabular}}  \\ 
\cline{4-12}
                                                                                       &                                                                                            &                                            & \textit{ACC}   & \textit{MF1}   & \textit{K}                                                                & \textit{ACC}   & \textit{MF1}   & \textit{K}                                                    & \textit{ACC}   & \textit{MF1}   & \textit{K}                                                       \\ 
\hline
SalientSleepNet \cite{ref9}                                                                       & Supervised                                                                                 & EEG1
  + EOG1                              & 67.30          & 58.88          & 0.56                                                                      & 88.89          & 49.20          & 0.04                                                          & 50.81          & 40.00          & 0.01                                                             \\
SleepFM  \cite{ref23}                                                                              & SSL                                                                                        & EEG1
  + EOG1                              & 71.87          & 62.90          & 0.62                                                                      & 95.00          & 53.53          & 0.09                                                          & 51.13          & 40.14          & 0.01                                                             \\
SynthSleepNet                                                                          & SSL                                                                                        & EEG1
  + EOG1                              & 80.50          & 69.29          & 0.73                                                                      & 95.80          & 54.64          & 0.01                                                          & 62.78          & 48.55          & 0.09                                                             \\
SynthSleepNet+TCM                                                                      & SSL                                                                                        & EEG1
  + EOG1                              & \textbf{87.98} & \textbf{82.12} & \textbf{0.83}                                                             & 96.89          & 56.80          & 0.15                                                          & 67.73          & 52.33          & 0.14                                                             \\ 
\hline
SalientSleepNet \cite{ref9}                                                                       & Supervised                                                                                 & EEG1
  + EOG1 + EMG1                       & 66.87          & 58.40          & 0.55                                                                      & 90.31          & 50.00          & 0.05                                                          & 51.41          & 40.32          & 0.01                                                             \\
SleepFM  \cite{ref23}                                                                              & SSL                                                                                        & EEG1
  + EOG1 + EMG1                       & 73.27          & 64.31          & 0.64                                                                      & 97.98          & 60.42          & 0.21                                                          & 59.11          & 45.82          & 0.06                                                             \\
SynthSleepNet                                                                          & SSL                                                                                        & EEG1
  + EOG1 + EMG1                       & 80.92          & 71.13          & 0.74                                                                      & 99.07          & 68.72          & 0.38                                                          & 71.02          & 55.02          & 0.18                                                             \\
SynthSleepNet+TCM                                                                      & SSL                                                                                        & EEG1
  + EOG1 + EMG1                       & \uline{87.41}  & \uline{81.92}  & \uline{0.83}                                                              & \uline{99.27}  & \uline{71.70}  & \uline{0.44}                                                  & 74.54          & 58.27          & 0.22                                                             \\ 
\hline
SalientSleepNet \cite{ref9}                                                                        & Supervised                                                                                 & EEG1
  + EOG1 + ECG1                       & 51.67          & 45.53          & 0.37                                                                      & 85.95          & 47.92          & 0.03                                                          & 72.97          & 56.37          & 0.19                                                             \\
SleepFM \cite{ref23}                                                                               & SSL                                                                                        & EEG1
  + EOG1 + ECG1                       & 68.93          & 62.50          & 0.60                                                                      & 96.12          & 55.08          & 0.12                                                          & 61.41          & 47.49          & 0.08                                                             \\
SynthSleepNet                                                                          & SSL                                                                                        & EEG1
  + EOG1 + ECG1                       & 79.79          & 69.08          & 0.72                                                                      & 99.27          & 71.66          & 0.43                                                          & \uline{76.80}  & \uline{60.02}  & \uline{0.25}                                                     \\
SynthSleepNet+TCM                                                                      & SSL                                                                                        & EEG1
  + EOG1 + ECG1                       & 83.60          & 75.73          & 0.77                                                                      & \textbf{99.37} & \textbf{73.47} & \textbf{0.47}                                                 & \textbf{77.52} & \textbf{61.32} & \textbf{0.27}                                                    \\
\hline
\multicolumn{12}{r}{* EEG1 = C4-A1 channel, EOG1 = EOG-Left channel} \\
\multicolumn{12}{r}{* The \textbf{best results} in each row are shown in bold, while the \uline{second-best} results are underlined // $K$ = \textit{Kappa}} \\
\end{tabular}}
}
% \vspace{-6mm}
\end{table*}

% \subsection{AWS}

% Since most users run Longhorn combined with public cloud providers, we also benchmarked our features in the AWS environment. For this purpose, we used two c5d.xlarge EC2 instances, a cost-efficient option commonly used by Longhorn developers for their benchmarks.
    
% EC2 instances have a built-in maximum provisioned IOPS, regardless of the hard drive used. As expected, the features performed to the machine's limit, reaching AWS's 40k IOPS cap. Overcoming these limitations requires using higher-performance and more expensive EC2 instances. After some research, we found that to harness the performance of our features added to Longhorn, users must use instances and volumes at least 5 times more expensive than c5d.xlarge using its installed volume.




% \subsection{Question II: Using Different Oracles}
\subsection{RQ I: Effectiveness of \algname{} and local optimality}
\label{sec:scalability}

%We validate the effectiveness of \algname{} on the Nam gate set.
%We validate the effectiveness of \algname{} on two different gate
%sets: Nam and Clifford+T.
%
%In this section, we evaluate with the Nam gate set.
%and \secref{clifft} considers the Clifford+T gate set.
%
To evaluate the effectiveness of \algname{}, we use our \algname{}
implementation with \voqc{} as the oracle on segments of size $\Omega
= 40$ and compare it to optimizers \quartz{}, \queso{}, and
\voqc{}.
%
The approach works for many different settings of $\Omega$ and we
analyze the impact of $\Omega$ in \secref{var-segment} in detail.
%
We give each optimizer a 12-hour cut-off time (excluding time for parsing and printing),
to allow completion of the experiments within a reasonable amount of time.
%
Throughout, we omit circuit-parsing time for timings of \voqc{}, whose parser
%,
% which maps input circuits into the internal representation used for optimization,
appears to scale superlinearly and can take significant time
(sometimes more than the optimization itself).
%
This approach is consistent with prior work on \voqc{}, which also excludes parse time.
%
When we use \voqc{} as an oracle of \algname{}, however,
we do include the parse time.
%
This makes the comparison somewhat unfair for \algname{}.
%

%% Note however that \algname{} only uses the \voqc{} oracle to
%% (parse and) optimize small segments of the circuit, which limits
%% the parsing overhead.
%
% Although this makes the comparison unfair against \algname{},
% if we optimized this overhead away, \algname{} would perform
% the same or better.
% This makes the comparison somewhat unfair to our \algname{},
% but nevertheless we observe good performance in our experiments.
%
% \algname{} calls \voqc{} to (parse and)
% optimize only small segments of the circuit.
%
We evaluate the running times of these optimizers
on benchmarks from the Nam gate set with sizes ranging
from thousands to hundreds of thousands of gates.

% \subsection{Question III: Performance}

% We analyze the empirical behavior of our \algname{} implementation  with two different oracles, \voqc{}, and \feyntool{}, under two different gate sets.
%
\if0
We organize the results based on the different cost function and gate set we use:
\begin{itemize}
    \item In \secref{nam}, we evaluate our \algname{} optimizer by comparing it
    to state-of-the-art optimizers \quartz{}, \queso{}, and \voqc{}. We observe that \voqc{} and \algname{}
    produce the best quality circuits and \algname{} is the fastest, producing the best quality circuits 10x faster on average.
    \item In \secref{clifft}, we evaluate our \algname{} optimizer by comparing it to the \feyntool{},
    a state-of-the-art optimizer for benchmarks in the Clifford+T gate set. We observe that \algname{} matches
    the quality of circuits produced by \feyntool{} while being 10x faster on average
    \item In \secref{converge} , we study the the convergence factor of \algname{},
    which is typically around $3$ for most of the benchmarks. We also analyze the quality impact
    of successive rounds of optimization and observe that most of the optimizations (> $99.7\%$) are actually
    found after the first round, and the subsequent rounds have small impact on circuit quality.
    \item and \secref{var-segment}, we study the impact of changing the parameter $\Omega$
    on the output quality and running time of \algname{}.
\end{itemize}
\fi


% \subsubsection{\algname{} with \voqc{} Oracle}


%
\if0
We discuss the time performance and output quality separately,
and show the results in\figref{main-time} and \figref{main-gate-count}
respectively.

These results show that our \algname{} optimizer produces
the best quality circuits and runs significantly faster than other tools,
delivering speedups ranging from $10$x to $1000$x
across circuit sizes ranging from thousands to hundreds of thousands of gates.
% \jremark{How do we make it clear that \algname{} is not expected
% to outperform \voqc{} in terms of quality? Basically we have to make
% it clear that local optimality is an undergoal, which is efficient
% to achieve and results in the same circuit.}
\fi

\begin{figure}[!ht]
    \centering
%    \small
\begin{tabular}{cccccccc}
                          &        &            & \multicolumn{4}{c}{Time (s)}                      &                                                           \\\cmidrule(lr){4-7}
Benchmark                 & Qubits & Input Size & Quartz & Queso         & VOQC    & \algname{}         & \begin{tabular}[c]{@{}c@{}}\algname{}\\ speedup\end{tabular} \\\midrule{}
\multirow{4}{*}{boolsat}  & 28     & 75670      & 12h    & 12h           & 68.6    & \textbf{45.2}   & 1.52                                                      \\
                          & 30     & 138293     & 12h    & 12h           & 307.3   & \textbf{98.7}   & 3.11                                                      \\
                          & 32     & 262548     & 12h    & 12h           & 1266.2  & \textbf{213.6}  & 5.93                                                      \\
                          & 34     & 509907     & 12h    & 12h           & 6151.0  & \textbf{462.8}  & 13.29                                                     \\\midrule{}
\multirow{4}{*}{bwt}      & 17     & 262514     & 12h    & 12h           & 8303.1  & \textbf{524.9}  & 15.82                                                     \\
                          & 21     & 402022     & 12h    & 12h           & 23236.8 & \textbf{1062.1} & 21.88                                                     \\
                          & 25     & 687356     & 12h    & 12h           & >12h & \textbf{2341.7} & > 18.45                                                     \\
                          & 29     & 941438     & 12h    & 12h           & >12h & \textbf{3982.6} & > 10.85                                                     \\\midrule{}
\multirow{4}{*}{grover}   & 9      & 8968       & 12h    & 12h           & 9.3     & \textbf{4.8}    & 1.94                                                      \\
                          & 11     & 27136      & 12h    & 12h           & 106.5   & \textbf{20.8}   & 5.12                                                      \\
                          & 13     & 72646      & 12h    & 12h           & 815.7   & \textbf{68.1}   & 11.97                                                     \\
                          & 15     & 180497     & 12h    & 12h           & 5743.2  & \textbf{223.9}  & 25.65                                                     \\\midrule{}
\multirow{4}{*}{hhl}      & 7      & 5319       & 12h    & 12h           & \textbf{0.3}     & 0.9    & 0.27                                                      \\
                          & 9      & 63392      & 12h    & 12h           & 74.1    & \textbf{22.9}   & 3.24                                                      \\
                          & 11     & 629247     & 12h    & 12h           & 14868.8 & \textbf{434.3}  & 34.24                                                     \\
                          & 13     & 5522186    & 12h    & Parsing Error & >12h & \textbf{8243.1}       & > 5.24                                                          \\\midrule{}
\multirow{4}{*}{shor}     & 10     & 8476       & 12h    & OOM           & 8.8     & \textbf{5.2}    & 1.70                                                      \\
                          & 12     & 34084      & 12h    & 12h           & 179.9   & \textbf{26.3}   & 6.84                                                      \\
                          & 14     & 136320     & 12h    & 12h           & 3638.4  & \textbf{126.0}  & 28.88                                                     \\
                          & 16     & 545008     & 12h    & 12h           & 70475.2 & \textbf{648.9}  & 108.60                                                    \\\midrule{}
\multirow{4}{*}{sqrt}     & 42     & 79574      & 12h    & 12h           & \textbf{30.0}    & 81.4   & 0.37                                                      \\
                          & 48     & 186101     & 12h    & 12h           & \textbf{191.2}   & 268.0  & 0.71                                                      \\
                          & 54     & 424994     & 12h    & 12h           & 3946.5  & \textbf{679.8}  & 5.81                                                      \\
                          & 60     & 895253     & 12h    & 12h           & >12h & \textbf{1653.5} & > 26.13                                                     \\\midrule{}
\multirow{4}{*}{statevec} & 5      & 31000      & 12h    & OOM           & \textbf{1.6}     & 4.3    & 0.38                                                      \\
                          & 6      & 129827     & 12h    & 12h           & 45.9    & \textbf{27.1}   & 1.70                                                      \\
                          & 7      & 526541     & 12h    & 12h           & 1812.2  & \textbf{164.7}  & 11.00                                                     \\
                          & 8      & 2175747    & 12h    & 12h           & >12h & \textbf{1345.1} & > 32.12                                                     \\\midrule{}
\multirow{4}{*}{vqe}      & 12     & 11022      & 12h    & 12h           & \textbf{0.2}     & 1.2    & 0.13                                                      \\
                          & 16     & 22374      & 12h    & 12h           & \textbf{0.6}     & 3.4    & 0.18                                                      \\
                          & 20     & 38462      & 12h    & 12h           & \textbf{2.0}     & 7.0    & 0.29                                                      \\
                          & 24     & 59798      & 12h    & 12h           & \textbf{5.4}     & 13.4            & 0.41                                                      \\\midrule{}
avg                       &        &            &        &               &         &                 & > 12.62
\end{tabular}

\caption{The figure shows the running time in seconds of the four optimizers, using gate count as the cost metric.
%
The column "\algname{} Speedup" is the speed of our  \algname{} with respect to \voqc{}, calculated as \voqc{} time divided by \algname{} time.
%
These measurements show that our optimizer \algname{} can be significantly faster, especially for large circuits (more than one order of magnitude on average).  As \figref{main-gate-count} shows, these time improvements come without any loss in optimization quality.
%
These results suggest that local optimality approach to optimization of quantum circuits can be effective in practice.}
    %
    \label{fig:main-time}
\end{figure}
\begin{figure}[!ht]
  \centering
  \small
\medskip
\begin{tabular}{ccccccc}
                          &        &            & \multicolumn{4}{c}{Optimizer}         \\\cmidrule(lr){4-7}
Benchmark                 & Qubits & Input Size & Quartz  & Queso   & VOQC    & \algname{} \\\midrule{}
\multirow{4}{*}{boolsat}  & 28     & 75670      & -41.1\% & -30.4\% & -83.2\% & \textbf{-83.7\%} \\
                          & 30     & 138293     & -23.5\% & -30.2\% & -83.3\% & \textbf{-83.7\%} \\
                          & 32     & 262548     & -6.9\%  & 0.0\%   & -83.3\% & \textbf{-83.5\%} \\
                          & 34     & 509907     & -2.9\%  & 0.0\%   & -83.3\% & \textbf{-83.4\%} \\\midrule{}
\multirow{4}{*}{bwt}      & 17     & 262514     & -8.7\%  & -0.1\%  & -30.0\% & \textbf{-31.1\%} \\
                          & 21     & 402022     & -3.1\%  & -0.1\%  & -38.4\% & \textbf{-40.0\%} \\
                          & 25     & 687356     & -0.8\%  & 0.0\%   & N.A.    & \textbf{-43.8\%} \\
                          & 29     & 941438     & -0.4\%  & 0.0\%   & N.A.    & \textbf{-44.5\%} \\\midrule{}
\multirow{4}{*}{grover}   & 9      & 8968       & -9.4\%  & -13.7\% & \textbf{-29.4\%} & \textbf{-29.4\%} \\
                          & 11     & 27136      & -9.5\%  & -9.6\%  & -29.9\% & \textbf{-30.0\%} \\
                          & 13     & 72646      & -9.6\%  & -0.3\%  & \textbf{-29.7\%} & \textbf{-29.7\%} \\
                          & 15     & 180497     & -9.6\%  & -0.2\%  & \textbf{-29.5\%} & \textbf{-29.5\%} \\\midrule{}
\multirow{4}{*}{hhl}      & 7      & 5319       & -26.6\% & -28.7\% & \textbf{-55.4\%} & -55.3\% \\
                          & 9      & 63392      & -24.4\% & -23.8\% & -56.3\% & \textbf{-56.5\%} \\
                          & 11     & 629247     & -1.1\%  & 0.0\%   & \textbf{-53.7\%} &\textbf{ -53.7\%} \\
                          & 13     & 5522186    & 0.0\%   & N.A.    & N.A.    & \textbf{-52.6\%} \\\midrule{}
\multirow{4}{*}{shor}     & 10     & 8476       & 0.0\%   & -5.4\%  & \textbf{-11.1\%} & -11.0\% \\
                          & 12     & 34084      & 0.0\%   & -3.8\%  & \textbf{-11.2\%} & \textbf{-11.2\%} \\
                          & 14     & 136320     & 0.0\%   & 0.0\%   & \textbf{-11.3\%} & -11.2\% \\
                          & 16     & 545008     & 0.0\%   & 0.0\%   & \textbf{-11.3\%} &\textbf{ -11.3\%} \\\midrule{}
\multirow{4}{*}{sqrt}     & 42     & 79574      & -16.2\% & -0.1\%  & \textbf{-33.0\%} &\textbf{ -33.0\%} \\
                          & 48     & 186101     & -15.2\% & 0.0\%   & \textbf{-32.7\%} & -32.6\% \\
                          & 54     & 424994     & -5.1\%  & 0.0\%   & \textbf{-32.4\%} & -32.3\% \\
                          & 60     & 895253     & -2.1\%  & 0.0\%   & N.A.    & \textbf{-34.3\%} \\\midrule{}
\multirow{4}{*}{statevec} & 5      & 31000      & -48.4\% & -74.5\% & -78.8\% & \textbf{-78.9\%} \\
                          & 6      & 129827     & -35.0\% & -29.7\% & \textbf{-78.4\%} & \textbf{-78.4\%} \\
                          & 7      & 526541     & -2.0\%  & -29.7\% & \textbf{-78.1\%} & \textbf{-78.1\%} \\
                          & 8      & 2175747    & -0.1\%  & 0.0\%   & N.A.    & \textbf{-78.7\%} \\\midrule{}
\multirow{4}{*}{vqe}      & 12     & 11022      & -35.4\% & -69.1\% & -63.0\% & \textbf{-69.5\%} \\
                          & 16     & 22374      & -33.7\% & -64.2\% & -60.1\% &\textbf{ -66.3\%} \\
                          & 20     & 38462      & -32.2\% & -60.8\% & -57.4\% & \textbf{-63.4\%} \\
                          & 24     & 59798      & -30.8\% & -36.4\% & -54.9\% & \textbf{-60.6\%} \\\midrule{}
avg                       &        &            & -13.6\% & -16.5\% & -48.1\% & \textbf{-49.4\%}
\end{tabular}
\caption{Optimization Quality.  The figure shows percentage reduction in gate count achieved by the four optimizers. 
  %
We write ``N.A.'' in cases where the optimizer did not finish within the allotted 12 hour deadline.
%
The experiments show that our cut-and-meld optimizer \algname{}
optimizes slightly better on average than all other optimizers.
%
%% Given that our \algname{} optimizer is faster than all other optimizers (\figref{main-time}), these experiments show that local optimality approach can improve performance without any loss in optimization quality.  
%% %
%% These results suggest that local optimality approach to optimization of quantum circuits can be effective in practice.
}
    \label{fig:main-gate-count}
\end{figure}


\myparagraph{Time Performance.}
\figref{main-time} show the time for our \algname{} implementation (with \voqc{} oracle) compared against \quartz{}, \queso{}, and \voqc{}.
%
The figure includes eight families of circuits, where
horizontal lines separate families and circuits within each family
arranged body increasing qubit and gate counts.
%
The optimizers \quartz{} and \queso{} use the maximum allotted time of 12 hours in all circuits,
because they explore a very large search space of all optimizations.
%
In a few cases, \queso{} throws an error or runs out of memory (denoted ``OOM'').
%
The \voqc{} optimizer and our \algname{} optimizer terminate much faster.
%
Specifically, \algname{} optimizes all circuits
between 0.2 seconds and 3 hours depending on the size,
and \voqc{} finishes for all but six benchmarks within 12 hours.
%
In the figure, we highlight in bold the fastest optimizer(s) for each circuit.

Comparing between \voqc{} and our \algname{}, we observe the following:
%
\begin{itemize}
    \item \textbf{Performance:} \algname{} is the fastest across the board except for \texttt{vqe}
    and except perhaps for the smallest circuits in some families.
    \item \textbf{Scalability:} the gap between \algname{} and \voqc{} increases as the circuit size increases,
    with \algname{} performing as much as 100$\times$ faster in some cases.
    \item \textbf{Overall:} \algname{} is over an order of magnitude faster than \voqc{} on average.
\end{itemize}
%
In the case of the \texttt{vqe} family, \voqc{} is consistently faster, but as we discuss next,
this comes at the cost of poorer optimization quality.
%
For the small circuits of families \texttt{hhl}, \texttt{statevec}, and \texttt{sqrt},
our optimizer is slower than \voqc{}.
%
This is due to the overheads that our implementation incurs for
(1) splitting and joining circuits,
%
(2) serialization/deserialization of input/output circuits for each oracle call, and
%
(3) various system-level calls needed to support calls to an external oracle.
%
For example,
for the $7$-qubit \texttt{hhl} benchmark and the $42$-qubit \texttt{sqrt} benchmark, we have measured that at least 30\% of the running
time is spent parsing and serializing/deserializing circuits.
\if0
For example,
in the $7$-qubit \texttt{hhl} benchmark which takes $0.9$ seconds to run,
we measured that  over $0.3$ seconds are spent just on system calls.
%
Similarly, for the $42$-qubit ``sqroot'' which takes $81.4$ seconds, over
 $25$ seconds are spent on system calls and deserializaiton/serialization operations.
%
As circuit sizes increase, these overheads, which are constant, diminish.

\fi

\if0
For example,
in the ``shor'' family of circuits,
our optimizer takes around $5$ seconds for the smallest case with $7543$ gates,
around $38$ seconds for $30268$ gates ($12$ qubits),
and $189$ seconds for the $14-$qubit case which has around $51000$ gates.
%
The corresponding times for \voqc{} range from $8.7$ seconds to $3638$ seconds
and are $1.6$x to $190$x slower than our \algname{} optimizer.
%
For the $16-$qubit case, our \algname{} finishes in $943$ seconds,
while \voqc{} does not finish within twelve hours.
%
In the ``bwt'' family,
our \algname{} optimizes the $17$-qubit case in $1196$ seconds
and \voqc{} takes $8303$ seconds on this benchmark ($\approx$ 7x improvement);
for the $21-$qubit our \algname{} takes $2280$ seconds
and is an order of magnitude faster than \voqc{},
which takes $23236$ seconds.
%
For the larger instances in ``bwt'',
our \algname{} finishes in 5698 (1.5 hours) and 11842 seconds ($\approx$ 3 hours) respectively
and neither of \quartz{}, \queso{}, or \voqc{} terminate in $12$ hours.
\fi



\if0

For the majority of benchmarks, our \algname{} optimizer is faster.
%
We observe a pattern for the speedup of our \algname{} optimizer relative to \voqc{}.
%
For almost all families of circuits,
the speedup increases significantly with circuit size (see column speedup in \figref{main-time}).
%
For instance, in the \texttt{grover} family,
increasing the circuit size increases our speedup from $1.35$x to $17.93$x.
%
In the \texttt{boolsat} family,
our speedup increases from $1.14$x on the smallest case
to $10.2$x on the largest case.
%
This trend is consistent across almost all families of benchmarks
and shows that our optimizer handles large circuits in a scalable fashion.
%

The key reason for the scalability of our \algname{} optimizer
is that it focuses on finding local optimizations on the circuit.
%
It does not "chase" global optimizations which are more complex
and require a significant amount of time.
%
This raises the question: although \algname{} finds local optimizations
in an efficient and scalable fashion, does it lose anything in terms
of quality by only considering local optimizations
and not operating on the entire circuit simultaneously?
%
We show that \textbf{our speedups do not come at any cost
to circuit quality} and in fact,
\algname{} produces better quality circuits in some cases.
%
This is due to the fact \algname{} guarantees local optimality when it terminates.
\fi

% These results indicate an asymptotic difference in
% running times between \algname{} and \voqc{}.
% %
% Specifically, when we consider the relative speedup of our \algname{}
% optimizer within any given family of circuits,
% we observe that the speedup increases significantly with circuit size.
% %
% For instance, in the ``grover'' family,
% our speedup increased from $40.6\%$ to $1697.4\%$ with increasing circuit size
% (\figref{main-time}).
% %


% Our \algname{} optimizer
% uses the same optimization techniques as \voqc{}---\algname{}
% uses \voqc{} as an oracle optimizer---but we schedule
% these optimizations in a way that scales more efficiently.
% %
% Specifically, \algname{} considers the circuit in segments of size $\Omega = 120$
% and strategically schedules their optimization until it achieves local optimality.
% %
% This segment-based approach benefits from \voqc{}'s excellent performance on
% relatively small circuits, making the optimization of any given segment fast.
% %
% In contrast,
% when \voqc{} is directly used as a standalone optimizer on entire circuits,
% it struggles with scalability due to the complexity of optimizing
% the entire circuit at once.


% Note that our algorithm uses \voqc{} as a subroutine to optimize segments
% of size $\Omega$ and benefits from its excellent performance on small circuits.
% %
% When \voqc{}
% it shows relatively poor scalability.
%

\myparagraph{Optimization quality.}
Our experiments so far show that our \algname{} performs well but it
does not give evidence of optimization quality.
%
% This raises the question: although \algname{} finds local optimizations
% in an efficient and scalable fashion, does it lose anything in terms
% of quality by only considering local optimizations
% and not operating on the entire circuit simultaneously?
%
%% TODO: the quality improvemenst are not significant
%% so, don't mention
%and in fact, \algname{} produces better quality circuits in some cases.
%
%This is due to the fact \algname{} guarantees local optimality when it terminates.
%
\figref{main-gate-count} shows
the output quality (measured by gate count)
of all optimizers for eight families of circuits.
%
% Each family is separated by horizontal lines,
% with circuits arranged by increasing qubit and gate counts.
% %
The figure shows the original gate count and the percent reduction
in gate count achieved by tools \quartz{}, \queso{}, \voqc{}, and \algname{}.
%
The best optimizers are highlighted in bold.
%
These results show that \algname{} always matches the best optimizer
within $0.1\%$ or outperforms it.
%
On average, \algname{} reduces the gate count by $49.7\%$, improving by
$1\%$ over the second best.
%
We note all optimizers except for our \algname{}, are unable to finish
some large circuits within the allotted 12-hour time limit or yield
very small (less than 1\%) improvement.
%
We present a more detailed discussion of these experiments below.


The results show that \algname{} and \voqc{} produce overall
better circuits than \quartz{} and \queso{}.
%
For the \texttt{hhl} family, both \algname{} and \voqc{} achieve
reductions of around $56\%$, while \quartz{} and \queso{} are around
$26\%$ for $7$ and $9$ qubits, and less than $1\%$ for $11$ qubits.
%
In the \texttt{statevec} family,
\algname{} and \voqc{} consistently reduce the gate count by $78\%$.
%
However, for the 8-qubit case, \voqc{} does not finish within our timeout
of 12 hours so we write ``N.A.''.
%
\queso{} also finds comparable reductions for the 5-qubit benchmark.

% Internally,
% our \algname{} uses \voqc{} as a subroutine to optimize
% small segments of the circuit, rather than optimizing
% the entire circuit at once.
% %
% Specifically, \algname{} considers the circuit in segments
% of size $\Omega = 40$ and applies \voqc{} to them.
% %
% In contrast,
% when used as a standalone optimizer,
% \voqc{} processes the entire circuit and could theoretically find more optimizations
% by considering all possible gates simultaneously.
% %
% However,
% our experiments show that \algname{}'s segment-based approach
% achieves the same quality.
% %
% This suggests that local optimality,
% as achieved by our optimizer,
% is a good goal for circuit optimization,
% because it does not miss any optimizations in practice.
% %
% As we saw in \figref{main-time},
% the approach is significantly faster because it scales better.
% %

For almost all families, we observe that the output quality of
\algname{} matches that of \voqc{} within $0.1\%$ or improves it,
sometimes significantly.
%
%Our \algname{} uses \voqc{} as the oracle on segments of length $\Omega = 40$
%and produces almost locally optimal circuits
%(upto the convergence ratio $\epsilon$, see \secref{impl}).
%
%
Specifically, for the \texttt{vqe} family, \algname{} optimizes better
than \voqc{}.
%
For example, on the 24-qubit \texttt{vqe} circuit, \algname{} improves
the gate count by $60.6\%$, and \voqc{} improves the gate count by
$54.9\%$.
%
Indeed, we observed that running \voqc{} twice by running it again on
its own output circuit bridges this gap.
%
%On average, \algname{} reduces the gate count by $49.7\%$ versus
%$48.1\%$ for \voqc{}.




%% \myparagraph{Summary.}
%% \figref{main-time} and \figref{main-gate-count}
%% show that
%% our \algname{} optimizer can be significantly faster, especially for larger circuits, because it scales better, and does so without sacrificing optimization quality.
%% %
%% The experiment thus shows that the local optimality approach can work well, especially for larger circuits.

\if0
is over an order of magnitude faster on average
and
produces circuits that match the best quality.}
%
These results demonstrate that
our \algname{} optimizes
circuits in an efficient and scalable fashion
and that local optimality is an effective
quality criterion.
\fi



\subsection{RQ II: Is empirical performance consistent with the asymptotic bound?}
\label{sec:eval-calls}
\begin{figure}[t]
    \centering
    \includegraphics[width=\columnwidth]{media/oracle_call_vs_input_size_half.pdf}
    \vspace{-20pt}
    \caption{ The number of oracle calls versus input circuit size for
      selected circuits.  The plots show that the number of
      calls scales linearly with the number of gates.  }
    \label{fig:plot-oracle-calls}
\end{figure}


In \secref{algorithm}, we established bounds on the number of oracle
calls performed by our \algname{} algorithm.
%
In this section, we check that our implementation is consistent with
these bounds by analyzing the number of oracle calls with respect to
the circuit size.
%
%We focus on optimizing for gate count, using \voqc{} as the oracle.
%
\figref{plot-oracle-calls} plots the number of oracle calls made by
our \algname{} optimizer for a subset of circuit families (the other
circuit families behave similarly; see
\iffull
\figref{plot-oracle-calls-all} included in the Appendix).
\else
Appendix).
\fi
%
The Y-axis represents the number of oracle calls
and the X-axis represents the input circuit size.
%
The plot shows that the number of oracle calls
increases linearly with circuit size, for all circuit families.

%% This observation is consistent with our asymptotic analysis.
%% %
%% In \corref{linear-calls},
%% we proved that for gate count optimization,
%% each round of our \algname{} performs a linear number of calls to the oracle.
%% %
%% Specifically, we bounded the number of oracle calls by $O(\mathsf{length}(C) + \sizeof{C})$.
%% %
%% Because our optimizer only takes a constant number of rounds to converge,
%% the bounds suggest that the number of oracle calls should be linear
%% in circuit size and depth,
%% which is consistent with the plots in \figref{plot-oracle-calls}.

We note that it would be more desirable to establish that the total
run-time, rather than the number of oracle calls, is linear, but this
is not the case because the oracle optimizers can take asymptotically non-linear time.
%
For example, the oracle \voqc{} can require at
least quadratic time in the number of gates in the circuit being
optimized, which varies as we increase the qubit
counts.

%% %
%% In these experiments, the oracle that we have used, \voqc{}, is an
%% implementation that primarily aim at verification, rather than
%% efficiency, making empirical analysis of runtime
%% challenging~\cite{hietala-personal-2024}.
%% %
%% We have nevertheless observed that the runtime of the oracle grows
%% roughly quadratically in the number of gates, which is constant for
%% all circuits of a given number of qubits and fixed $\Omega$.

\subsection{RQ III: Ablation Study of Meld}

\begin{figure}[t]
  \scriptsize
  \begin{minipage}[]{0.49\textwidth}
    \begin{tabular}{cccccc|}
      & & & \multicolumn{3}{c}{Optimizer} \\ \cmidrule(lr){4-6}
Family                 & Qubits & Input Size & VOQC  & \algname{} & \hspace{-2pt}\algnameminus{}\hspace{-2pt} \\
 \midrule\multirow{4}{*}{boolsat}  & 28 &   75670 & -83.2\% & \textbf{-83.7\%} & -83.0\% \\
                           & 30 &  138293 & -83.3\% & \textbf{-83.7\%} & -83.1\% \\
                           & 32 &  262548 & -83.3\% & \textbf{-83.5\%} & -83.1\% \\
                           & 34 &  509907 & -83.3\% & \textbf{-83.4\%} & -83.1\% \\
 \midrule\multirow{4}{*}{bwt}      & 17 &  262514 & -30.0\% & \textbf{-31.1\%} & -28.3\% \\
                           & 21 &  402022 & -38.4\% & \textbf{-40.0\%} & -36.3\% \\
                           & 25 &  687356 & N.A.    & \textbf{-43.8\%} & -40.0\% \\
                           & 29 &  941438 & N.A.    & \textbf{-44.5\%} & -41.0\% \\
 \midrule\multirow{4}{*}{grover}   &  9 &    8968 & \textbf{-29.4\%} & \textbf{-29.4\%} & -26.1\% \\
                           & 11 &   27136 & -29.9\% & \textbf{-30.0\%} & -26.3\% \\
                           & 13 &   72646 & \textbf{-29.7\%} & \textbf{-29.7\%} & -26.0\% \\
                           & 15 &  180497 & \textbf{-29.5\%} & \textbf{-29.5\%} & -26.0\% \\
 \midrule\multirow{4}{*}{hhl}      &  7 &    5319 & \textbf{-55.4\%} & -55.3\% & -54.4\% \\
                           &  9 &   63392 & -56.3\% & \textbf{-56.5\%} & -55.6\% \\
                           & 11 &  629247 & \textbf{-53.7\%} & \textbf{-53.7\%} & -53.1\% \\
                           & 13 & 5522186 & N.A.    & \textbf{-52.6\%} & -52.1\% \\
      \midrule
    \end{tabular}
  \end{minipage}
  \begin{minipage}[]{0.49\textwidth}
    \begin{tabular}{cccccc}
      & & & \multicolumn{3}{c}{Optimizer} \\ \cmidrule(lr){4-6}
Family                 & Qubits & Input Size & VOQC  & \algname{} & \hspace{-3pt}\algnameminus{}\hspace{-3pt} \\
 \midrule\multirow{4}{*}{shor}     &  10 &    8476 & \textbf{-11.1\%} & -11.0\% & -10.3\% \\
                           &  12 &   34084 & \textbf{-11.2\%} & \textbf{-11.2\%} & -10.7\% \\
                           &  14 &  136320 & \textbf{-11.3\%} & -11.2\% & -10.8\% \\
                           &  16 &  545008 & \textbf{-11.3\%} & \textbf{-11.3\%} & -10.9\% \\
 \midrule\multirow{4}{*}{sqrt}     & 42 &   79574 & \textbf{-33.0\%} & \textbf{-33.0\%} & -31.5\% \\
                           & 48 &  186101 & \textbf{-32.7\%} & -32.6\% & -31.2\% \\
                           & 54 &  424994 & \textbf{-32.4\%} & -32.3\% & -30.9\% \\
                           & 60 &  895253 & N.A.    & \textbf{-34.3\%} & -32.9\% \\
 \midrule\multirow{4}{*}{statevec} &  5 &   31000 & -78.8\% & \textbf{-78.9\%} & -78.1\% \\
                           &  6 &  129827 & \textbf{-78.4\%} & \textbf{-78.4\%} & -77.9\% \\
                           &  7 &  526541 & \textbf{-78.1\%} & \textbf{-78.1\%} & -77.8\% \\
                           &  8 & 2175747 & N.A.    & \textbf{-78.7\%} & -78.6\% \\
 \midrule\multirow{4}{*}{vqe}      & 12 &   11022 & -63.0\% & \textbf{-69.5\%} & -62.7\% \\
                           & 16 &   22374 & -60.1\% & \textbf{-66.3\%} & -59.7\% \\
                           & 20 &   38462 & -57.4\% & \textbf{-63.4\%} & -57.2\% \\
                           & 24 &   59798 & -54.9\% & \textbf{-60.6\%} & -54.8\% \\
  \midrule
  \end{tabular}
  \end{minipage}
\caption{Ablation study of \lstinline{meld}. The table shows
  percentage reduction in gate count achieved by the version of
  \algname{}, called \algnameminus{}, that uses simply concatenation
  instead of the \lstinline{meld} algorithm, compared with \voqc{} and
  the original \algname{}.
%
The measurements show
that with the \lstinline{meld} algorithm, our \algname{} optimizer reduces gate counts better than 
\algnameminus{} (-49.4\% vs. -47.3\% on average).
%using \voqc{} alone
%or simply splitting the circuit into segments and optimize them using \voqc{} separately.
}    \label{fig:main-ablation}
\end{figure}



Our algorithm for local optimality optimizes a circuit by cutting it
into smaller subcircuits, optimizing the subcircuits, and melding the
optimized subcircuits into a locally optimal circuit.
%
If the algorithm joined the circuits together instead of melding them,
then the resulting circuits would not be locally optimal, because the
segments overlapping the circuit cuts may not be optimal.
%
To understand the impact of the \lstinline{meld} operation, we perform
an ablation experiment.
%
Specifically, we implement an ablating version of our \algname{},
called \algnameminus{}, that simply concatenates the optimized
subcircuits instead of the \lstinline{meld} operation.

\figref{main-ablation} shows the results of this ablation study.
%
%In this figure, \algnameminus{} splits the circuit into segments, use
%\voqc{} to optimize each segment, and then concatenate the results
%together.
%
%using the same $\Omega=40$ as \algname{}.
%
\algname{} consistently performs better than the ablating version
\algnameminus{}, with over 2\% improvement on the average total gate
count.
%
Even though the percentage degradation due to ablation may seem modest,
it is significant, because each and every gate has a significant
runtime and fidelity cost on modern and near-term quantum computers.
%
This ablation study shows the importance of the \lstinline{meld}
algorithm and that of local optimality, which does not hold without
the \lstinline{meld}.
%
Notably, the quality of the circuits produced by the ablating version
are significantly worse than those produced by the baseline \voqc{}.


\subsection{RQ IV: Impact of compaction and $\Omega$ on the effectiveness of \algname{}}
\label{sec:converge}

\begin{figure}[t]
  \scriptsize
  \begin{minipage}[]{0.4\textwidth}
    \begin{tabular}{ccccc|}
      & & & \multicolumn{2}{c}{Optimizations} \\ \cmidrule(lr){4-5}
      Family & Qubits & \#Rounds & Round 1 & Round 2 \\
      \midrule
      \multirow{4}{*}{boolsat}& 28 & 2 & 100.00\% & 0.00\% \\
      & 30 & 2 & 100.00\% & 0.00\% \\
      & 32 & 2 & 100.00\% & 0.00\% \\
      & 34 & 2 & 100.00\% & 0.00\% \\
      \midrule\multirow{4}{*}{bwt}& 17 & 2 & 99.59\% & 0.41\% \\
      & 21 & 2 & 99.79\% & 0.21\% \\
      & 25 & 2 & 99.83\% & 0.17\% \\
      & 29 & 2 & 99.90\% & 0.10\% \\
      \midrule\multirow{4}{*}{grover}& 9 & 2 & 99.96\% & 0.04\% \\
      & 11 & 2 & 99.90\% & 0.10\% \\
      & 13 & 2 & 99.99\% & 0.01\% \\
      & 15 & 2 & 99.97\% & 0.03\% \\
      \midrule\multirow{4}{*}{hhl}& 7 & 2 & 99.76\% & 0.24\% \\
      & 9 & 2 & 99.95\% & 0.05\% \\
      & 11 & 2 & 99.96\% & 0.04\% \\
      & 13 & 2 & 99.95\% & 0.05\% \\
      \midrule
    \end{tabular}
  \end{minipage}
  \begin{minipage}[]{0.4\textwidth}
    \begin{tabular}{ccccc}
      & & & \multicolumn{2}{c}{Optimizations} \\ \cmidrule(lr){4-5}
      Family & Qubits & \#Rounds & Round 1 & Round 2 \\
    \midrule\multirow{4}{*}{shor}& 10 & 2 & 100.00\% & 0.00\% \\
  & 12 & 2 & 99.97\% & 0.03\% \\
  & 14 & 2 & 99.96\% & 0.04\% \\
  & 16 & 2 & 99.99\% & 0.01\% \\
  \midrule\multirow{4}{*}{sqrt}& 42 & 2 & 99.90\% & 0.10\% \\
  & 48 & 2 & 99.81\% & 0.19\% \\
  & 54 & 2 & 99.97\% & 0.03\% \\
  & 60 & 2 & 100.00\% & 0.00\% \\
  \midrule\multirow{4}{*}{statevec}& 5 & 2 & 100.00\% & 0.00\% \\
  & 6 & 2 & 99.92\% & 0.08\% \\
  & 7 & 2 & 99.92\% & 0.08\% \\
  & 8 & 2 & 99.99\% & 0.01\% \\
  \midrule\multirow{4}{*}{vqe}& 12 & 2 & 99.95\% & 0.05\% \\
  & 16 & 2 & 99.99\% & 0.01\% \\
  & 20 & 2 & 99.98\% & 0.02\% \\
  & 24 & 2 & 99.99\% & 0.01\% \\
  \midrule
  \end{tabular}
  \end{minipage}
  \vspace{-1pt}
    \caption{The number of rounds and the
  percentage of optimizations %found
  in each optimization round of \algname{}.}
  \label{fig:num-rounds}
\end{figure}


% \begin{figure}[h]
%   \centering
%   \begin{tabular}{ccccc}

%   & & & \multicolumn{2}{c}{Optimizations} \\ \cmidrule(lr){4-5}
%   Family & Qubits & \#Rounds & Round 1 & Round 2 \\
%   \midrule
%   \multirow{4}{*}{boolsat}& 28 & 2 & 100.00\% & 0.00\% \\
%   & 30 & 2 & 100.00\% & 0.00\% \\
%   & 32 & 2 & 100.00\% & 0.00\% \\
%   & 34 & 2 & 100.00\% & 0.00\% \\
%   \midrule\multirow{4}{*}{bwt}& 17 & 2 & 99.59\% & 0.41\% \\
%   & 21 & 2 & 99.79\% & 0.21\% \\
%   & 25 & 2 & 99.83\% & 0.17\% \\
%   & 29 & 2 & 99.90\% & 0.10\% \\
%   \midrule\multirow{4}{*}{grover}& 9 & 2 & 99.96\% & 0.04\% \\
%   & 11 & 2 & 99.90\% & 0.10\% \\
%   & 13 & 2 & 99.99\% & 0.01\% \\
%   & 15 & 2 & 99.97\% & 0.03\% \\
%   \midrule\multirow{4}{*}{hhl}& 7 & 2 & 99.76\% & 0.24\% \\
%   & 9 & 2 & 99.95\% & 0.05\% \\
%   & 11 & 2 & 99.96\% & 0.04\% \\
%   & 13 & 2 & 99.95\% & 0.05\% \\
%   \midrule\multirow{4}{*}{shor}& 10 & 2 & 100.00\% & 0.00\% \\
%   & 12 & 2 & 99.97\% & 0.03\% \\
%   & 14 & 2 & 99.96\% & 0.04\% \\
%   & 16 & 2 & 99.99\% & 0.01\% \\
%   \midrule\multirow{4}{*}{sqroot}& 42 & 2 & 99.90\% & 0.10\% \\
%   & 48 & 2 & 99.81\% & 0.19\% \\
%   & 54 & 2 & 99.97\% & 0.03\% \\
%   & 60 & 2 & 100.00\% & 0.00\% \\
%   \midrule\multirow{4}{*}{statevec}& 5 & 2 & 100.00\% & 0.00\% \\
%   & 6 & 2 & 99.92\% & 0.08\% \\
%   & 7 & 2 & 99.92\% & 0.08\% \\
%   & 8 & 2 & 99.99\% & 0.01\% \\
%   \midrule\multirow{4}{*}{vqe}& 12 & 2 & 99.95\% & 0.05\% \\
%   & 16 & 2 & 99.99\% & 0.01\% \\
%   & 20 & 2 & 99.98\% & 0.02\% \\
%   & 24 & 2 & 99.99\% & 0.01\% \\
%   \midrule
%   \end{tabular}
%   \caption{The figure shows the number of rounds and the
%   percentage of optimizations found
%   in each optimization round of \algname{}.}
%   \label{fig:num-rounds}
%   \end{figure}

\subsubsection{Impact of compaction}
\label{sec:compaction}
\begin{wrapfigure}{r}{0.5\textwidth}
\centering
\small
\vspace{-10pt}
\begin{tabular}{ccc}
$\Omega$ & Time (s) & Output gate count (reduction) \\
\midrule
2 & 17.4 & 27600 (-56.46\%) \\
5 & 13.9 & 27620 (-56.43\%) \\
10 & 17.7 & 27609 (-56.45\%) \\
20 & 18.9 & 27590 (-56.48\%) \\
40 & 21.8 & 27570 (-56.51\%) \\
80 & 35.3 & 27564 (-56.52\%) \\
160 & 66.6 & 27559 (-56.53\%) \\
320 & 122.8 & 27551 (-56.54\%) \\
% grover-n9:
%2 & 3.0 & 6368 (-28.99\%) \\
%5 & 3.3 & 6379 (-28.87\%) \\
%10 & 4.1 & 6343 (-29.27\%) \\
%20 & 3.5 & 6333 (-29.38\%) \\
%40 & 4.5 & 6330 (-29.42\%) \\
%80 & 8.1 & 6330 (-29.42\%) \\
%160 & 14.3 & 6330 (-29.42\%) \\
%320 & 14.6 & 6330 (-29.42\%) \\
\midrule
\voqc{} & 74.1 & 27673 (-56.35\%) \\
\end{tabular}
\caption{Choice of $\Omega$: performance of \algname{} with different $\Omega$
on the \texttt{hhl} circuit with 9 qubits, which initially contains 63392 gates.
For reference, we also present the performance of \voqc{} at the end.}
   \label{fig:omega}
\end{wrapfigure}
% TODO: typeset at a better location
%For all benchmark circuits in \figref{main-gate-count},
%using the convergence ratio $\epsilon = 0.01$,
%\algname{} converges very quickly, always
%terminating after $2$ rounds of optimization.
%Furthermore, it consistently finds over 99\% of the optimizations in the first round.
\figref{num-rounds} shows the number of rounds and percentage
optimizations for each round of our \algname{} algorithm, using the
convergence ratio $\epsilon = 0.01$.
%
%% Similar to \figref{main-gate-count},
%% we use our \algname{} optimizer, with \voqc{} as the oracle
%% and convergence ratio $\epsilon = 0.01$,
%% to optimize the benchmarks in the Nam gate set.
%% %
The results show that \algname{} converges very quickly, always
terminating after $2$ rounds of optimization, and that it
consistently finds over 99\% of the optimizations in the first round.
%
This is because our \algname{} ensures the slightly weaker segment
optimality after the first round of optimization (see
\secref{algorithm}), which ensures that all segments are optimal,
though there may be gaps.
%We present the detailed result in the Appendix.
%
The experiment shows that although compaction can enable some
optimizations by removing the gaps, its impact on these benchmarks is
minimal.
%
We separately ran the same experiments with $\epsilon = 0$, which
forces the algorithm to run up to perfect convergence, and observed
that \algname{} requires $4$ rounds of optimization on average over
all circuits.
%
These results show that in practice a small number of compaction
rounds suffice to obtain results that are within a very small fraction
of the local optimal.

\subsubsection{Impact of varying segment size $\Omega$}
\label{sec:var-segment}

\figref{omega} shows the running time and the gate count reduction of \algname{} with different
values of $\Omega$
on the \texttt{hhl} circuit with 9 qubits, which initially contains 63392 gates.
The results show that for a wide range of $\Omega$ values,
our optimizer produces a circuit of similar quality to the baseline \voqc{}, and typically does so in significantly less time.
When $\Omega$ is large, \algname{}'s running time scales linearly with $\Omega$, and the output gate count reduces marginally when $\Omega$ increases.
We choose $\Omega=40$ in our evaluation to achieve a balance between running time and output quality but note that many different values work similarly well.

