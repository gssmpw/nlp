We discussed most closely related work in the body of the paper.  In
this section, we present a broader overview of the work on quantum
circuit optimization.

\paragraph{Cost Functions.}
Gate count is a widely used metric for optimizing quantum circuits.
In the NISQ era, reducing gate count improves circuit performance
by minimizing noise from operations and decoherence.
It also reduces resources in fault-tolerant architectures
like the Clifford+T gate set.
Researchers have developed techniques to reduce gate count by
either directly optimizing circuits or
resynthesizing parts using efficient synthesis algorithms.
We cover optimization techniques later in the section.


%\paragraph{Other Gate Sets and Metrics.}

In addition to reducing gate counts,
compilers like Qiskit and t$\ket{\textnormal{ket}}$,
implement circuit transformations that optimize cost specific to NISQ architectures.
%
Examples include maximizing circuit fidelity in the presence of noise~\cite{murali2019noise, tannu2019not},
and
reducing qubit mapping and routing overhead (SWAP gates)
for specific device topologies~\cite{molavi2022qubit, lye2015determining, itoko2020optimization, li2019tackling},
or hardware-native gates and pulses \cite{nottingham2023decomposing, wu2021tilt, shi2019optimized, gokhale2020optimized}.
%
Techniques also exist to optimize/synthesis
circuits for specific unitary types, such as classical
reversible gates~\cite{prasad2006data, ding2020square, bandyopadhyay2020post, wille2019towards},
\clifft{}~\cite{amy2020number, kliuchnikov2014asymptotically, ross2014optimal, kissinger2020reducing},
Clifford-cyclotomic~\cite{forest2015exact}, V-basis~\cite{bocharov2013efficient, ross2015optimal},
and Clifford-CS~\cite{glaudell2020optimal} circuits.
%
While algorithms for small unitaries produce
Clifford+T circuits with an asymptotically optimal number of $\mathsf{T}$ gates~\cite{giles2013remarks},
efficiently generating optimal large Clifford+T circuits remains a challenge.
%
The \feyntool{} optimizer is used for optimizing the $\mathsf{T}$ count of quantum circuits.
%
It uses an efficient (polynomial-time) algorithm called phase folding~\cite{amy2014polynomial},
to reduce phase gates, such as the $\mathsf{T}$ gate, by merging them.
%
More generally, the Feynman toolkit combines phase folding
with synthesis techniques to optimize other metrics like the CNOT count~\cite{amy2019formal}.
%
We demonstrate that our \algname{} algorithm,
which guarantees local optimality,
can use \feyntool{} as an oracle for optimizing
$\mathsf{T}$ count in Clifford+T circuits
\iffull
in \appref{clifft}.
\else
in the Appendix.
\fi
%Our experiments demonstrate that our \algname{} algorithm,
%which guarantees local optimality,
%effectively uses \feyntool{} as an oracle for optimizing
%$\mathsf{T}$ count in Clifford+T circuits.
%
These experiments show that our \algname{} algorithm scales
to large circuits without reducing optimization quality.

\paragraph{Resynthesis methods.}
Resynthesis methods focus on decomposing unitaries into sequences of
smaller unitaries using algebraic structures of matrices.
%
Examples include
the Cartan decomposition \cite{tucci2005introduction},
the Cosine-Sine Decomposition (CSD),
the Khaneja Glaser Decomposition (KGD) \cite{khaneja2001cartan},
and the Quantum Shannon Decomposition (QSD).
%
Some synthesis methods demonstrate optimality for arbitrary unitaries of small size
(typically for fewer than five qubits),
particularly in terms of gate counts like CNOT gates \cite{rakyta2022approaching}.
%
However, their efficiency degrades significantly when dealing with larger unitaries;
furthermore, they require the time-consuming step of turning the circuit into a unitary.
%
QGo \cite{wu2020qgo} addresses this limitation with a hierarchical approach that
partitions and resynthesizes circuits block-by-block.
%
However, due to the lack of optimization across blocks,
the performance of QGo depends heavily on how circuits are partitioned.
%
Our local optimality technique, and specifically melding,
could be used to address this limitation of QGo.

\paragraph{Rule-based and peephole optimization methods.}
%Optimization methods often employ heuristic-based optimization
%to iteratively minimize the overall gate count.
%
%Their performance is often limited by the small set of rules used.
%Recent developments in optimization methods can be coarsely classified
%as rule-based, search-based, and learning-based techniques.
Rule-based methods find and substitute rules in quantum circuits to
optimize the circuit~\cite{iten2022exact, bandyopadhyay2020post,hietala2021verified, quartz-2022}.
%
VOQC~\cite{hietala2021verified} is
a formally verified optimizer that uses rules to optimize circuits.
%
VOQC implements several optimization passes inspired by state-of-the-art
unverified optimizer proposed by Nam et al.~\cite{Nam_2018}.
%
These passes include rules that perform $\mathsf{NOT}$ gate propagation, Hadamard gate reduction,
single- and two-qubit gate cancellation, and rotation merging.
%
Most of these passes take quadratic time in circuit size,
while some can take as much as cubic time~\cite{Nam_2018}.
%
Our experiments show that our local optimization algorithm \algname{}
effectively uses \voqc{} as an oracle for gate count optimization.
%

%
% UMUT: WE DO NOT IMPROVE RUNNING TIME OF VOQC
%Our results show that by focussing on local optimality,
%we can improve the running time of \voqc{} without any loss in circuit quality.

The notion of local optimality proposed in this paper is related
peephole optimization techniques from the classical compilers
literature~\cite{ct-compiler-2022,h+stratified-2016,sa-peephole-2006}.
%
Peephole optimizers typically optimize a small number of instructions,
e.g., rewriting a sequence of three addition operations into a single
multiplication operation.
%
Our notion of local optimality applies to segments of quantum
circuits, without making any assumption about segment sizes (in our
experiments, our segments typically contained over a thousand gates).
%
Because peephole optimizers typically operate on small instructions at
a time and because they traditionally consider the non-quantum
programs,  efficiency concerns are less important.
%
In our case, efficiency is crucial, because our segments can be large,
and optimizing quantum programs is expensive.
%
To ensure efficiency and quality, we devise a circuit
cutting-and-melding technique.

Prior work use peephole optimizers~\cite{prasad2006data,
liu2021relaxed} to improve the circuit one group of gates at a time,
and repeat the process from the start until they reach a fixed point.
%
The Quartz optimizer also uses a peephole optimization technique but
cannot make any quality guarantees~\cite{quartz-2022}.
%
Our algorithm differs from this prior work, in several aspects.
%
First,
it ensures efficiency, while also providing a quality guarantee based
on local optimality.
%
Key to attaining efficiency and quality is its use of circuit
cutting and melding techniques.
%
Second, our algorithm is generic: it can work on large segments (far
larger than a peephole) and optimizes each segment with an oracle,
which can optimize the circuit in any way it desires, e.g., it can use
any of the techniques described above.
%



\if0
%% UMUT: OUT OF PLACE
Although we focus on the Nam gate set for this evaluation, we note that
\quartz{} and \queso{} have shown competitive performance
for other gate sets (including IBM, Rigetti, and Ion)~\cite{quartz-2022, queso-2023}, and
we expect that this performance would translate to our setting.
\fi

\pyzx{}~\cite{kissinger2020Pyzx} is another rule-based optimizer
%However, it only minimizes $\mathsf{T}$ count and does not explicitly optimize gate count. We observe that \pyzx{} achieves worse performance on gate count, and can spend over 98\% of the time on circuit extraction for ZX-diagrams as opposed to optimization on ZX-diagrams for the sqrt circuit with 42 qubits.
that optimizes $\mathsf{T}$ count. It uses ZX-diagrams to optimize circuits and then extracts the circuit. Circuit extraction for ZX-diagrams is \#{}P-hard~\cite{de2022circuit}, and can take up much more time than optimization itself.
Because \algname{} invokes the optimizer many times, circuit extraction for ZX-diagrams can become a bottleneck. In addition, \pyzx{} only minimizes $\mathsf{T}$ count and does not explicitly optimize gate count. We therefore did not use \pyzx{} in our evaluation.
%Due to the uncertainty of the heuristics in the circuit extraction algorithm, it might not be efficient to pair \algname{} with \pyzx{} as an oracle for $\mathsf{T}$ count optimization.


\paragraph{Search-based methods.}
Rule-based optimizers may be limited by a small set of rules and are not exhaustive.
%
To address this,
researchers have developed search-based optimizers~\cite{queso-2023, quartz-2022, qfast,qsearch}
including \quartz{}~\cite{quartz-2022} and \queso{}~\cite{queso-2023}
that automatically synthesize exhaustive circuit equivalence rules.
%
Although their rule-synthesis approach differs,
both use similar algorithms for circuit optimization.
%
They iteratively operate on a search queue of candidate circuits.
%
In each iteration,
they pop a circuit from the queue,
rewrite parts of the circuit using equivalence rules,
and insert the new circuits back into the queue.
%
To manage the exponential growth of candidate circuits,
both tools use a ``beam search'' approach that limits the search queue size by
dropping circuits appropriately.
%
By limiting the size of the search queue,
\quartz{} and \queso{} ensure that
the space usage is linear relative to the size of the circuit.
%
Their running time remains exponential,
and they offer a timeout functionality,
allowing users to halt optimization after a set time.
%
This approach has delivered excellent reductions in gate count
for relatively small benchmarks~\cite{queso-2023, quartz-2022}.
%
However, for large circuits,
the optimizers do not scale well because they attempt to search an exponentially large search space.

QFast and QSearch apply numerical optimizations to
search for circuit decompositions that are close to the desired unitary~\cite{qfast, qsearch}.
%
Although faster than search-based methods \cite{davis2020towards},
these numerical methods tend to produce longer circuits, and their running time
is difficult to analyze.


%\paragraph{\feyntool{}.}





\if0
%% UMUT: REDUNDANT
The optimizers \quartz{} \cite{quartz-2022} and \queso{} \cite{queso-2023}
construct equivalent classes of quantum circuits and
use them to run a beam-search algorithm for reducing gate count (\secref{impl}).
%
These state-of-the-art optimizers are excellent for optimizing circuits
whose sizes are hundreds of gates.
\fi

\paragraph{Learning-based methods.}
Researchers have also developed machine learning models~\cite{fosel2021quantum}
for optimizing quantum circuits with variational/continuous parameters,
which reduce gate count by tuning parameters of
shallow circuit ansatze~\cite{mitarai2018quantum, ostaszewski2021reinforcement},
or by iteratively pruning gates \cite{sim2021adaptive, wang2022quantumnas}.
%
These approaches, however, are associated with substantial training costs~\cite{wang2022quantumnas}.
%




%

% %
% In this paper,
% we integrate our \coam{} algorithm with \quartz{}
% and demonstrate that it scales \quartz{} for larger circuits,
% with sizes going upto hundreds of thousands of gates.
% %
% Because the runtime behavior of \queso{} is similar to \quartz{},
% we tried to integrate our implementation with
% the \queso{} optimizer.
% %
% However, we faced two problems.
% %
% First,
% even though \queso{} implements a timeout facility,
% it does not adhere to it.
% %
% We observed that the tool takes an unpredictable amount of time
% taking as much as $200$x the given timeout.
% \footnote{We note that the tool is designed to operate for hours of optimization
% and may not have been designed to account for the range of timeouts we are interested in,
% i.e., in the range of hundreds of seconds.}
% %
% Second,
% the optimizer is in Java,
% which does not integrate with our code, because our
% implementation language does not cross-operate with Java.
%


% Global optimization of quantum circuits is QMA-hard ~\cite{janzing2003identity}.
% %
% As such, our algorithm targets a more modest but practically significant goal: to compose local optimizations of subcircuits while maintaining a strong $\Omega$-optimality guarantee (Theorem~\ref{thm:opt}), allowing us to scale an oracle optimizer (such as Quartz) to the large/deep circuit regimes in linear time (Theorem~\ref{thm:time}).
% %
% As a result, we achieve faster and higher-quality optimizations,
% surpassing the straightforward application of the oracle optimizer to the entire quantum circuit.




% \feyntool{} is an optimizer from the ``Feynman''\footnote{https://github.com/meamy/feynman} toolkit
% for quantum circuits that
% %
% This path integral model encodes the state transition performed by a quantum circuit with a
% sum over all possible paths between two given states.
% %
% The model is used by the \feyntool{} optimizer to generate a \emph{phase polynomial}
% of a circuit on which the optimizer performs a \emph{phase folding} optimization.
% The optimization merges phase gates
% acting on the same set of computational paths.
% %
% Phase folding is a powerful technique for optimizing phase gates such as the $\mathsf{S}$ gate,
% the $\mathsf{Z}$ gate, and the most expensive $\mathsf{T}$ gate~\cite{t-depth}.
%
% Furthermore, the algorithm terminates in polynomial time,
% making it a practical choice for optimizing large circuits.
%



%\yr{More explanations?}




%\chapter{Implementation}{\label{ch:implementation}}
In this chapter, we present the implementation of the final product. We start by discussing how the four steps introduced in \hyperref[ch:high_level_approach]{chapter \ref*{ch:high_level_approach}} are integrated. We then outline the main system components of our score follower, presenting each as an independent, self-contained module. We then combine this into an overall system architecture and finally introduce the open-source score renderer used to display the score and evaluate the score follower.       

% \section{Aims and Requirements}
% The overall aim of the score follower was to 


\section{Score Follower Framework Details}
Our score follower conforms to the high-level framework presented in \hyperref[section:score_follower_framework]{section \ref*{section:score_follower_framework}}. In step 1, two score features are extracted from a MIDI file (see \hyperref[subsection:midi]{subsection \ref*{subsection:midi}}), namely MIDI note numbers\footnote{\href{https://inspiredacoustics.com/en/MIDI_note_numbers_and_center_frequencies}{https://inspiredacoustics.com/en/MIDI\_note\_numbers\_and\_center\_frequencies}} (corresponding to pitch) and note onsets (corresponding to duration). In step 2, the audio is streamed (whether from a file or into a microphone) and audioframes that exceed some predefined energy threshold are extracted. Here, audioframes are groups of contiguous audio samples, whose length can be specified by the argument \verb|frame_length|, usually between 800 and 2000 samples. The period between consecutive audioframes can also be defined by the argument \verb|hop_length|, typically between 2000 and 5000 audio samples. In step 3, score following is performed via a `Windowed' Viterbi algorithm (see  \hyperref[subsection:adjusting_viterbi]{subsection \ref*{subsection:adjusting_viterbi}}) which uses the Gaussian Process (GP) log marginal likelihoods (LMLs) for emission probabilities (see \hyperref[section:state_duration_model]{section \ref*{section:state_duration_model}}) and a state duration model for transition probabilities (see \hyperref[section:state_duration_model]{section \ref*{section:state_duration_model}}). Finally, in step 4 we render our results using an adapted version of the open source user interface, \textit{Flippy Qualitative Testbench}.

\section{Following Modes}
Two modes are available to the user: Pre-recorded Mode and Live Mode. The former requires a pre-recorded $\verb|.wav|$ file, whereas the latter takes an input stream of audio via the device's microphone. Note that both modes are still forms of score following, as opposed to score alignment, since in each mode we receive audioframes at the sampling rate, not all at once.\\

Live Mode offers a practical example of a score follower, displaying a score and position marker which a musician can read off while playing. However, this mode is not suitable for evaluation because the input and results cannot be easily replicated. Even ignoring repeatability, Live Mode is not suitable for one-off testing since a musician using this application may be influenced by the movement of the marker. For instance, the performer may speed up if the score follower `gets ahead' or slow down if the position marker lags or `gets lost'. To avoid this, we use Pre-recorded Mode when evaluating the performance of our score follower. Furthermore, Pre-recorded Mode offers the advantage of testing away from the music room, providing the opportunity to evaluate a variety of recordings available online. 

\section{System Architecture}
Our guiding principle for development was to build modular code in order to create a streamlined system where each component performs a specific task independently. This structure facilitates easy testing and debugging. \hyperref[fig:black_box]{Figure \ref*{fig:black_box}} presents a high-level architecture diagram, where each black box abstracts a key component of the score follower. When operating in Pre-recorded Mode, there is the option to stream the recording during run-time, which outputs to the device's speakers (as indicated by the dashed lines).

\begin{figure}[H]
    \centering
    \includegraphics[width=1\textwidth]{figs/Part_4_Implementation_And_Results/black_box.png}
    \caption{Abstracted system architecture diagram displaying inputs in grey, the 4 main components of the score follower in black and the outputs in green.}
    \label{fig:black_box}
\end{figure}

\subsection{Score Preprocessor}
The architecture for the Score Preprocesor is given in \hyperref[fig:score_preprocessor]{Figure \ref*{fig:score_preprocessor}}. First, MIDI note number and note onset times are extracted from each MIDI event. Simultaneous notes can be gathered into states and returned as a time-sorted list of lists called \verb|score|, where each element of the outer list is a list of simultaneous note onsets. Similarly, a list of note durations calculated as the time difference between consecutive states is returned as \verb|times_to_next|. Finally, all covariance matrices are precalculated and stored in a dictionary, where the key of the dictionary is determined by the notes present. This is because the distribution of notes and chords in a score is not random: notes tend to belong to a home \gls{key} and melodies tend to be repeated or related (similar to subject fields in speech processing). Therefore, states tend to be reused often, allowing us to achieve amortised time and space savings (by avoiding repeated calculation of the same covariance matrices). 

\begin{figure}[H]
    \centering
    \includegraphics[width=1\textwidth]{figs/Part_3_Implementation/Stage_2_Alignment/score_preprocessor.png}
    \caption{System architecture diagram representing the Score Preprocessor with inputs in grey, processes in blue and objects in yellow.}
    \label{fig:score_preprocessor}
\end{figure}


\subsection{Audio Preprocessor}
The architecture for the Audio Preprocessor is illustrated in \hyperref[fig:audio_preprocessor]{Figure \ref*{fig:audio_preprocessor}}. In Pre-recorded Mode, the Slicer receives a $\verb|.wav|$ file and returns audioframes separated by the \verb|hop_length|. These audioframes are periodically added to a multiprocessing queue, \verb|AudioFramesQueue|, to simulate real-time score following. In Live Mode, we use the python module \verb|sounddevice| to receive a stream of audio, using a periodic callback function to place audioframes on \verb|AudioFramesQueue|. 

\begin{figure}[H]
    \centering
    \includegraphics[width=1\textwidth]{figs/Part_4_Implementation_And_Results/audio_preprocessor.png}
    \caption{System architecture diagram representing the Audio Preprocessor with inputs in grey, processes in blue and objects in yellow.}
    \label{fig:audio_preprocessor}
\end{figure}

\subsection{Follower and Backend}
The joint Follower and Backend architecture diagram is shown in \hyperref[fig:follwer_and_backend]{Figure \ref*{fig:follwer_and_backend}}. The Viterbi Follower (detailed in \hyperref[subsection:adjusting_viterbi]{section \ref*{subsection:adjusting_viterbi}}) calculates the most probable state in the score, given audioframes continually taken from \verb|AudioFramesQueue|. These states are placed on another multiprocessing queue, the \verb|FollowerOutputQueue|, for the Backend to process and send. This prevents any bottle-necking occurring at the Follower stage. The Backend first sets up a UDP connection and then reads off values from \verb|FollowerOutputQueue|, sending them via UDP packets to the score renderer.

\begin{figure}[H]
    \centering
    \includegraphics[width=1\textwidth]{figs/Part_4_Implementation_And_Results/follower_and_backend.png}
    \caption{System architecture diagram representing the Follower and Backend processes with processes in blue, objects in yellow and outputs in green.}
    \label{fig:follwer_and_backend}
\end{figure}

\subsection{Player}
In Pre-recorded Mode, the Player sets up a new process and begins streaming the recording once the Follower process begins. This provides a baseline for testing purposes, as a trained musician can observe the score position marker and judge how well it matches the music. 

\subsection{Overall System Architecture}
The overall system architecture is presented in \hyperref[fig:overall_system_architecture]{Figure \ref*{fig:overall_system_architecture}}. Since the Follower runs a real-time, time sensitive process, parallelism is employed to reduce the total system latency. We use two \verb|multiprocessing| queues\footnote{\href{https://docs.python.org/3/library/multiprocessing.html}{https://docs.python.org/3/library/multiprocessing.html}} to avoid bottle-necking, which allows us to run 4 concurrent processes (Audio Preprocessor, Follower, Backend, and Audio Player). Hence, this architecture allows the components to run independently of one another to avoid blocking. Furthermore, this allows the system to take advantage of the multiple cores and high computational power offered by most modern machines.  

\begin{figure}[H]
    \centering
    \includegraphics[width=1\textwidth]{figs/Part_4_Implementation_And_Results/overall_score_follower_2.png}
    \caption{System architecture diagram representing the overall score follower running in Pre-recorded mode, with inputs in grey, processes in blue, objects in yellow and outputs in green.}
    \label{fig:overall_system_architecture}
\end{figure}


\section{Rendering Results}{\label{section:renderer}}
To visualise the results of our score follower, we adapted an open source tool for testing different score followers.\footnote{\href{https://github.com/flippy-fyp/flippy-qualitative-testbench/blob/main/README.md}{https://github.com/flippy-fyp/flippy-qualitative-testbench/blob/main/README.md}} \hyperref[fig:flippy_example]{Figure \ref*{fig:flippy_example}} shows the user interface of the score position renderer, where the green bar indicates score position. 

\begin{figure}[H]
    \centering
    \includegraphics{figs/Part_4_Implementation_And_Results/example_renderer.png}
    \caption{Screenshot of the score renderer user interface which displays a score (here we show a keyboard arrangement of \textit{O Haupt voll Blut und Wunden} by Bach). The green marker represents the score follower position.}
    \label{fig:flippy_example}
\end{figure}



