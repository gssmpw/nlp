\documentclass[acmsmall,screen, review, anonymous]{acmart}\settopmatter{printfolios=false,printccs=false,printacmref=false}

%replace XXX with the submission number you are given from the ASPLOS submission site.
\newcommand{\asplossubmissionnumber}{442}

\usepackage[normalem]{ulem}
\usepackage{latexsym,amsthm,amsmath,amsfonts,stmaryrd}
\usepackage{listings}
\usepackage{caption}
\usepackage{multirow}

\usepackage{newtxmath}
\usepackage{subfig}
\usepackage[utf8]{inputenc}
\usepackage[T1]{fontenc}
\usepackage{libertine}
\usepackage{braket}
%%%%%%%%%%%%%%%%%%%%%%%%%%%%%%%%%%%%%%%%%%%%%%%%%%%%%%%%%%%%%%%%%%%%%%
%% mac.tex
%%
%% Umut A. Acar
%% Macros for adaptive computation paper.
%%%%%%%%%%%%%%%%%%%%%%%%%%%%%%%%%%%%%%%%%%%%%%%%%%%%%%%%%%%%%%%%%%%%%%

\newcommand{\readarrow}{\ensuremath{\Longrightarrow}}
\newcommand{\currenttime}{\ttt{currentTime}\xspace}
\newcommand{\tags}[1]{\ensuremath{\mathsf{tags}(#1)}}
\newcommand{\weight}[1]{\ensuremath{\mathsf{weight}(#1)}}
\newcommand{\dist}[2]{\ensuremath{\delta(#1,#2)}}


\newcommand{\cutspace}{\vspace{-4mm}}

\newcommand{\bomb}[1]{\fbox{\mbox{\emph{\bf {#1}}}}}

\newcommand{\myparagraph}[1]{\smallskip \noindent{\bf {#1}.}}
% formatting stuff
\newcommand{\codecolsep}{1ex}



\newcommand{\rlabel}[1]{\hspace*{-1mm}\mbox{\small{\bf ({#1})}}}

\newcommand{\tablerow}{\\[5ex]}
\newcommand{\tableroww}{\\[7ex]}
\newcommand{\tableline}{
\vspace*{2ex}\\
\hline\\ 
\vspace*{2ex}}

% Don't care
\newcommand{\dontcare}{\_
}

%% filter and quicksort stuff
\newcommand{\ncf}[2]{C^{fil}_{\ensuremath{{#1},{#2}}}}
\newcommand{\ncq}[1]{C^{qsort}_{\ensuremath{#1}}}
\newcommand{\nuf}[3]{P^{fil}_{#1,(\ensuremath{{#2},{#3}})}}
\newcommand{\nuq}[2]{P^{qsort}_{(\ensuremath{{#1},{#2}})}}

%% shorthands
\newcommand{\ddg}{{\sc ddg}}
\newcommand{\ncpa}{change-propagation algorithm}
\newcommand{\adg}{{\sc adg}}
\newcommand{\nwrite}{\texttt{write}}
\newcommand{\nread}{\texttt{read}}
\newcommand{\nmodr}{\texttt{mod}}
\newcommand{\ttt}[1]{\texttt{#1}}
\newcommand{\nmodl}{\texttt{modl}}
\newcommand{\nnil}{\ttt{NIL}}
\newcommand{\ncons}[2]{\ttt{CONS({\ensuremath{#1},\ensuremath{#2}})}}
\newcommand{\nfilter}{\texttt{filter}}
\newcommand{\nfilterp}{\texttt{filter'}}
\newcommand{\naqsort}{\texttt{qsort'}}
\newcommand{\nqsort}{\texttt{qsort}}
\newcommand{\nqsortp}{\texttt{qsort'}}
\newcommand{\nnewMod}{\texttt{newMod}}
\newcommand{\nchange}{\texttt{change}}
\newcommand{\npropagate}{\texttt{propagate}}
\newcommand{\ndest}{\texttt{d}}
\newcommand{\ninit}{\texttt{init}}



%% Comment sth out. 
\newcommand{\out}[1] {}
\newcommand{\sthat}{\ensuremath{~|~}}

%% definitions
\newcommand{\defi}[1]{{\bfseries\itshape #1}}


% Code listings.
\newcounter{codeLineCntr}
\newcommand{\codeLine}
 {\refstepcounter{codeLineCntr}{\thecodeLineCntr}}
\newcommand{\codeLineL}[1]
 {\refstepcounter{codeLineCntr}\label{#1}{\thecodeLineCntr}}
\newcommand{\codeLineNN}{} %% NN = No-Number (and no change to counter)

\newenvironment{codeListing}
 {\setcounter{codeLineCntr}{0}
%  \fontsize{10}{12}
 % the first one is width the second is height
 \fontsize{9}{11}
  \fontsize{8}{8}
  \vspace{-.1in}
  \ttfamily\begin{tabbing}}
  {\end{tabbing}
   \vspace{-.1in}}

\newenvironment{codeListing8}
 {\setcounter{codeLineCntr}{0}
%  \fontsize{8}{10}
  \fontsize{8}{8}
  \vspace{-.1in}
  \ttfamily
  \begin{tabbing}}
 {\end{tabbing}
 \vspace{-.1in}
}

\newenvironment{codeListing8h}
 {\setcounter{codeLineCntr}{0}
  \fontsize{8.5}{10.5}
  \vspace{-.1in}
  \ttfamily
  \begin{tabbing}}
 {\end{tabbing}
 \vspace{-.1in}
}


\newenvironment{codeListing9}
 {\setcounter{codeLineCntr}{0}
  \fontsize{9}{11}
  \vspace{-.1in}
  \ttfamily
  \begin{tabbing}}
 {\end{tabbing}
 \vspace{-.1in}
}

\newenvironment{codeListing10}
 {\setcounter{codeLineCntr}{0}
  \fontsize{10}{12}
  \vspace{-.1in}
  \ttfamily
  \begin{tabbing}}
 {\end{tabbing}
 \vspace{-.1in}
}


\newenvironment{codeListingNormal}
 {\setcounter{codeLineCntr}{0}
  \vspace{-.1in}
  \ttfamily
  \begin{tabbing}}
 {\end{tabbing}
 \vspace{-.1in}
}

\newcommand{\codeFrame}[1]
{\begin{center}\fbox{\parbox[t]{\columnwidth}{#1}}\end{center}
% \vspace*{-.15in}
}

\newcommand{\halfBox}[1]
{\begin{center}\fbox{\parbox[t]{\columnwidth}{#1}}\end{center}
% \vspace*{-.15in}
}

\newcommand{\fullBox}[1]
{\begin{center}\fbox{\parbox[t]{\textwidth}{#1}}\end{center}
% \vspace*{-.15in}
}

%%Note this is redefined in local-mac.tex for each paper.
\newcommand{\fixedCodeFrame}[1]
{
\begin{center}
\fbox{
\parbox[t]{0.9\columnwidth}{
#1
}
}\end{center}
}

% Footnote commands.
\newcommand{\footnotenonumber}[1]{{\def\thempfn{}\footnotetext{#1}}}

% Margin notes - use \notesfalse to turn off notes.
\setlength{\marginparwidth}{0.6in}
\reversemarginpar
\newif\ifnotes
\notestrue
\newcommand{\longnote}[1]{
  \ifnotes
    {\medskip\noindent Note:\marginpar[\hfill$\Longrightarrow$]
      {$\Longleftarrow$}{#1}\medskip}
  \fi}
\newcommand{\note}[1]{
  \ifnotes
    {\marginpar{\raggedright{\tiny #1}}}
  \fi}
\newcommand{\notered}[1]{\ifnotes
    {\marginpar{\raggedright{\tiny
          {\sf\color{red} #1}}}}
    \fi}


% Stuff not wanted.
\newcommand{\punt}[1]{}

% Sectioning commands.
\newcommand{\subsec}[1]{\subsection{\boldmath #1 \unboldmath}}
\newcommand{\subheading}[1]{\subsubsection*{#1}}
\newcommand{\subsubheading}[1]{\paragraph*{#1}}

% Reference shorthands.
\newcommand{\spref}[1]{Modified-Store Property~\ref{sp:#1}}
\newcommand{\prefs}[2]{Properties~\ref{p:#1} and~\ref{p:#2}}
\newcommand{\pref}[1]{Property~\ref{p:#1}}


\newcommand{\partref}[1]{Part~\ref{part:#1}}
\newcommand{\chref}[1]{Chapter~\ref{ch:#1}}
\newcommand{\chreftwo}[2]{Chapters \ref{ch:#1} and~\ref{ch:#2}}
\newcommand{\chrefthree}[3]{Chapters \ref{ch:#1}, and~\ref{ch:#2}, and~\ref{ch:#3}}
\newcommand{\secref}[1]{Section~\ref{sec:#1}}
\newcommand{\subsecref}[1]{Subsection~\ref{subsec:#1}}
\newcommand{\secreftwo}[2]{Sections \ref{sec:#1} and~\ref{sec:#2}}
\newcommand{\secrefthree}[3]{Sections \ref{sec:#1},~\ref{sec:#2},~and~\ref{sec:#3}}
\newcommand{\appref}[1]{Appendix~\ref{app:#1}}
\newcommand{\figref}[1]{Figure~\ref{fig:#1}}
\newcommand{\figreftwo}[2]{Figures \ref{fig:#1} and~\ref{fig:#2}}
\newcommand{\figrefthree}[3]{Figures \ref{fig:#1}, \ref{fig:#2} and~\ref{fig:#3}}
\newcommand{\figreffour}[4]{Figures \ref{fig:#1},~\ref{fig:#2},~\ref{fig:#3}~and~\ref{fig:#4}}
\newcommand{\figpageref}[1]{page~\pageref{fig:#1}}
\newcommand{\tabref}[1]{Table~\ref{tab:#1}}
\newcommand{\tabreftwo}[2]{Tables~\ref{tab:#1} and~\ref{tab:#1}}

\newcommand{\stref}[1]{step~\ref{step:#1}}
\newcommand{\caseref}[1]{case~\ref{case:#1}}
\newcommand{\lineref}[1]{line~\ref{line:#1}}
\newcommand{\linereftwo}[2]{lines \ref{line:#1} and~\ref{line:#2}}
\newcommand{\linerefthree}[3]{lines \ref{line:#1},~\ref{line:#2},~and~\ref{line:#3}}
\newcommand{\linerefrange}[2]{lines \ref{line:#1} through~\ref{line:#2}}
\newcommand{\thmref}[1]{Theorem~\ref{thm:#1}}
\newcommand{\thmreftwo}[2]{Theorems \ref{thm:#1} and~\ref{thm:#2}}
\newcommand{\thmrefthree}[3]{Theorems \ref{thm:#1}, \ref{thm:#2} and~\ref{thm:#3}}
\newcommand{\thmpageref}[1]{page~\pageref{thm:#1}}
\newcommand{\lemref}[1]{Lemma~\ref{lem:#1}}
\newcommand{\lemreftwo}[2]{Lemmas \ref{lem:#1} and~\ref{lem:#2}}
\newcommand{\lemrefthree}[3]{Lemmas \ref{lem:#1},~\ref{lem:#2},~and~\ref{lem:#3}}
\newcommand{\lempageref}[1]{page~\pageref{lem:#1}}
\newcommand{\corref}[1]{Corollary~\ref{cor:#1}}
\newcommand{\defref}[1]{Definition~\ref{def:#1}}
\newcommand{\defreftwo}[2]{Definitions \ref{def:#1} and~\ref{def:#2}}
\newcommand{\defpageref}[1]{page~\pageref{def:#1}}
\renewcommand{\eqref}[1]{Equation~(\ref{eq:#1})}
\newcommand{\eqreftwo}[2]{Equations (\ref{eq:#1}) and~(\ref{eq:#2})}
\newcommand{\eqpageref}[1]{page~\pageref{eq:#1}}
\newcommand{\ineqref}[1]{Inequality~(\ref{ineq:#1})}
\newcommand{\ineqreftwo}[2]{Inequalities (\ref{ineq:#1}) and~(\ref{ineq:#2})}
\newcommand{\ineqpageref}[1]{page~\pageref{ineq:#1}}
\newcommand{\itemref}[1]{Item~\ref{item:#1}}
\newcommand{\itemreftwo}[2]{Item~\ref{item:#1} and~\ref{item:#2}}

% Useful shorthands.
\newcommand{\abs}[1]{\left| #1\right|}
\newcommand{\card}[1]{\left| #1\right|}
\newcommand{\norm}[1]{\left\| #1\right\|}
\newcommand{\floor}[1]{\left\lfloor #1 \right\rfloor}
\newcommand{\ceil}[1]{\left\lceil #1 \right\rceil}
  \renewcommand{\choose}[2]{{{#1}\atopwithdelims(){#2}}}
%\newcommand{\ang}[1]{\langle#1\rangle}
\newcommand{\paren}[1]{\left(#1\right)}
\newcommand{\prob}[1]{\Pr\left\{ #1 \right\}}
\newcommand{\expect}[1]{\mathrm{E}\left[ #1 \right]}
\newcommand{\expectsq}[1]{\mathrm{E}^2\left[ #1 \right]}
\newcommand{\variance}[1]{\mathrm{Var}\left[ #1 \right]}
\newcommand{\twodots}{\mathinner{\ldotp\ldotp}}

% Standard number sets.
\newcommand{\reals}{{\mathrm{I}\!\mathrm{R}}}
\newcommand{\integers}{\mathbf{Z}}
\newcommand{\naturals}{{\mathrm{I}\!\mathrm{N}}}
\newcommand{\rationals}{\mathbf{Q}}
\newcommand{\complex}{\mathbf{C}}

% Special styles.
\newcommand{\proc}[1]{\ifmmode\mbox{\textsc{#1}}\else\textsc{#1}\fi}
\newcommand{\procdecl}[1]{
  \proc{#1}\vrule width0pt height0pt depth 7pt \relax}
  \newcommand{\func}[1]{\ifmmode\mathrm{#1}\else\textrm{#1}fi} %
%  Multiple cases.  
\renewcommand{\cases}[1]{\left\{
  \begin{array}{ll}#1\end{array}\right.}
  \newcommand{\cif}[1]{\mbox{if $#1$}} 

%% spacing hacks
\newcommand{\longpage}{\enlargethispage{\baselineskip}}
\newcommand{\shortpage}{\enlargethispage{-\baselineskip}}



%% Notes, todos, and remarks
\newcounter{remark}[section]

\newcommand{\myremark}[3]{
\refstepcounter{remark}
\[
\left\{
\sf 
\parbox{\columnwidth}{
{\bf {#1}'s remark~\theremark:} 
{#3}
}
\right\}
\]
%\marginpar{\bf {#2}.~\theremark}
}


% - - - - - - - - - - - - - - - - - - - - - - - - - - - - - - - - - - - - - - - - - - - - 
% For amsthm package:

%\theoremstyle{plain}
%\newtheorem{thm}{Theorem}[section]
%% \newtheorem{lem}[thm]{Lemma}
%% \newtheorem{prop}[thm]{Proposition}
%% \newtheorem*{cor}{Corollary}

%% \theoremstyle{definition}
%% \newtheorem{defn}{Definition}[section]
%% \newtheorem{conj}{Conjecture}[section]
%% \newtheorem{falseconj}{False~Conjecture}[section]
%% \newtheorem{exmp}{Example}[section]

%% \theoremstyle{remark}
%% \newtheorem*{rem}{Remark}
%% %\newtheorem*{note}{Note}
%% \newtheorem{case}{Case}


\newcommand{\uremark}[1]{\myremark{Umut}{U}{#1}}
\newcommand{\ur}[1]{\uremark{#1}}
\newcommand{\rremark}[1]{\myremark{Ruy}{R}{#1}}
\newcommand{\mremark}[1]{\myremark{Matthew}{M}{#1}}
\newcommand{\todoremark}[1]{\myremark{TODO}{TODO}{#1}}
%\newcommand{\todo}[1]{\myremark{TODO}{TODO}{#1}}
\newcommand{\todo}[1]{{\bf{[TODO:{#1}]}}}

%%

\begin{document}

\title{Scaling Quantum Circuit Optimizers}
\title{Supplementary Material for the Submission \\ Local Optimization of Large Quantum Circuits}

\date{}
\maketitle

\thispagestyle{empty}
\newcommand{\kwcost}[1]{\mathbf{cost}\left(  {#1} \right)}
\newcommand{\circuitcon}[2]{{#1} + {#2}}

\newcommand{\bigomega}{\mathbf{\Omega}}

%% Semantics frame
\newcommand{\sfbox}[1]
{
\cfbox{blue}{#1}
}
\newcommand{\srule}{\vspace{2mm}\rule{\columnwidth}{1pt}\vspace{2mm}}

\newcommand{\lang}{\textsc{Laqe}}
%% General syntax
\newcommand{\dom}[1]{\mathop{\text{dom}}(#1)}
\newcommand{\codom}[1]{\mathop{\text{cod}}(#1)}
\newcommand{\kw}[1]{\mbox{\ttt{#1}}}
\newcommand{\cdparens}[1]{({#1})}
\newcommand{\cd}[1]{{\lstinline!#1!}}
\newcommand{\hmm}{\textsf{HMM}}
\newcommand{\rulename}[1]{\textsc{#1}}
\newcommand{\ruleref}[1]{Rule~\rulename{#1}}

\newcommand{\true}{\ensuremath{\kw{true}}}
\newcommand{\false}{\ensuremath{\kw{false}}}
\newcommand{\ttrue}{\kw{t}}
\newcommand{\ffalse}{\kw{f}}
\newcommand{\prog}{\ensuremath{P}}
\newcommand{\pred}{\ensuremath{\mathcal{P}}}
\newcommand{\predf}[2]{\ensuremath{\pred(#1,#2)}}
\newcommand{\defeq}{\triangleq}

\newcommand{\type}[2]{\ensuremath{#1 : #2}}
\newcommand{\typed}[4]{\ensuremath{#1 \vdash_{#2} \type{#3}{#4}}}

\newcommand{\btype}{\beta}
\newcommand{\utype}{\theta}
\newcommand{\val}{v}
\newcommand{\uval}{u}

\newcommand{\algname}{\textsf{OAC}}
\newcommand{\algnameminus}{\textsf{OACMinus}}
\newcommand{\coam}{\textsf{OAC}}
\newcommand{\lopt}{\textsf{Lopt}}
\newcommand{\coamwith}[1]{\ensuremath{\mathsf{SOAM}[{#1}]}}
\newcommand{\queso}{{\textsf{Queso}}}
\newcommand{\voqc}{{\textsf{VOQC}}}
\newcommand{\pyzx}{{\textsf{PyZX}}}
\newcommand{\quartz}{{\textsf{Quartz}}}
\newcommand{\quartztool}{$\mathsf{Quartz}$}
\newcommand{\quesotool}{$\mathsf{Queso}$}
\newcommand{\feyntool}{\textsf{FeynOpt}}

\newcommand{\quartzt}[1]{\ensuremath{\mathsf{Quartz}_{\,#1}}}
\newcommand{\quesot}[1]{\ensuremath{\mathsf{Queso}_{\,#1}}}
\newcommand{\coamt}[1]{\ensuremath{\mathsf{SOAM}[#1]}}
\newcommand{\clifft}{Clifford+T}

\newcommand{\compat}[2]{\ensuremath{{#1} \mathbin{\scaleobj{1.2}{\diamond}} {#2}}}
\newcommand{\notcompat}[2]{\ensuremath{{#1} \mathbin{\scaleobj{1.2}{\centernot{\diamond}}} {#2}}}
\newcommand{\windowopt}[2]{{#2}~\textsf{\textbf{segment-optimal}}_{#1}}
\newcommand{\wopttext}{segment optimal}
\newcommand{\compressed}[1]{{#1}~\textsf{\textbf{compact}}}
\newcommand{\locallyopt}[2]{{#2}~\textsf{\textbf{locally-optimal}}_{#1}}
\newcommand{\qubits}[1]{\mathsf{qubits}({#1})}
%% Terms
\newcommand{\tmemprog}{memory-progress\xspace}
\newcommand{\tmempres}{memory-preservation\xspace}


%% imperative serial types
\newcommand{\kwint}{\kw{int}}
\newcommand{\kwnat}{\kw{nat}}
\newcommand{\kwfut}{\kw{fut}}
\newcommand{\kwprod}[2]{\ensuremath{{#1} \times {#2}}}
\newcommand{\kwarr}[2]{\ensuremath{{#1} \ra {#2}}}
\newcommand{\kwloc}[1]{\ensuremath{{#1}~\kw{loc}}}
\newcommand{\kwref}[1]{\ensuremath{{#1}~\kw{ref}}}

%% imperative multithreaded types
\newcommand{\kwtid}{\kw{tid}}
\newcommand{\kwunit}{\kw{unit}}
\newcommand{\kwok}{\kw{ok}}

% space, heap, store
\newcommand{\heap}{H}
\newcommand{\spc}{H}
\newcommand{\empspc}{\emptyset}
\newcommand{\spaceext}[3]{{#1}[{#2} \mapsto {#3}]}

\newcommand{\catspace}{\uplus}
\newcommand{\catheap}{\uplus}

\newcommand{\heapun}[1]{\langle #1 \rangle}
\newcommand{\heapbi}[2]{\langle #1 ; #2 \rangle}
\newcommand{\heaptri}[3]{\langle #1 ; #2; #3 \rangle}
\newcommand{\heapquad}[4]{\langle #1 ; #2 ; #3 ; #4 \rangle}
\newcommand{\restctx}[2]{\ensuremath{#1 \upharpoonright_{#2}}}
\newcommand{\freeloc}[1]{\ensuremath{\mathsf{FL}(#1)}}
\newcommand{\locs}[1]{\ensuremath{\mathsf{Loc}(#1)}}
\newcommand{\diff}[1]{\ensuremath{\mathsf{Diff}(#1)}}

\newcommand{\rename}[3]{[#2 \mapsto #3](#1)}
\newcommand{\kwt}{\kw{t}}

\newcommand{\estore}{[~]}
\newcommand{\mkstore}[2]{\ensuremath{{#1}::{#2}}}

%% imperative serial syntax
\newcommand{\kwn}{\kw{n}}
\newcommand{\kwlet}[3]{\kw{let}~{#1}={#2}~\kw{in}~{#3}~\kw{end}}
\newcommand{\kwfun}[3]{\ensuremath{\kw{fun}~{#1}~{#2}~\kw{is}~{#3}~\kw
{end}}}
\newcommand{\kwpair}[2]{\ensuremath{\langle{#1},{#2}}\rangle}
\newcommand{\kwapply}[2]{\ensuremath{{#1}~{#2}}}
\newcommand{\kwfst}[1]{\ensuremath{\kw{fst}\cdparens{#1}}}
\newcommand{\kwsnd}[1]{\ensuremath{\kw{snd}\cdparens{#1}}}
%\newcommand{\kwfst}[1]{\ensuremath{\kw{fst}~#1}}
%\newcommand{\kwsnd}[1]{\ensuremath{\kw{snd}~#1}}
\newcommand{\gcing}[1]{\ensuremath{[#1]}}
\newcommand{\kwnew}[1]{\ensuremath{\kw{ref}(#1)}}
\newcommand{\kwderef}[1]{\ensuremath{\mathop{!}#1}}
\newcommand{\kwwrite}[2]{\ensuremath{#1 \mathop{:=} #2}}

\newcommand{\kwletrec}[2]{\ensuremath{#1 \mathop{\cdot} #2}}
\newcommand{\kwtask}[3]{\ensuremath{#1 \mathop{\cdot} #2 \mathop{\cdot} #3}}
\newcommand{\kwtaskalt}[2]{\ensuremath{#1 \mathop{\cdot} #2}}
\newcommand{\halt}{\bot}
\newcommand{\tree}{T}
\newcommand{\trace}{t}
%% imperative multithreaded syntax
\newcommand{\kwfork}[1]{{\ensuremath{\kw{fork}\cdparens{#1}}}}
\newcommand{\kwjoin}[1]{{\ensuremath{\kw{join}\cdparens{#1}}}}
\newcommand{\kwunitv}{\ensuremath{(\,)}}
\newcommand{\kwtidv}{\ensuremath{\kw{t}}}
\newcommand{\gheap}{G}
\newcommand{\lheap}{\heap}
\newcommand{\tolheap}[1]{\ensuremath{\Delta(#1)}}
\newcommand{\theap}[2]{\left(#1, #2\right)}

%% hierarchical syntax
\newcommand{\cdpar}{\texttt{par}}
%\newcommand{\kwpar}[2]{\ensuremath{#1 \mathop{\|} #2}}
\newcommand{\kwpar}[2]
           {\ensuremath{\mathop{\vartriangleleft \hspace{-0.1em} #1, #2
               \hspace{-0.1em} \vartriangleright}}}
\newcommand{\kwpara}[2]
           {\ensuremath{\mathop{\blacktriangleleft \hspace{-0.1em} #1, #2
               \hspace{-0.1em} \blacktriangleright}}}
%\newcommand{\kwpara}[2]{\ensuremath{#1 \mathop{\overline{\|}} #2}}
\newcommand{\kwparl}[2]{\ensuremath{#1 \overset{\leftarrow}{\|} #2}}
\newcommand{\kwparr}[2]{\ensuremath{#1 \overset{\rightarrow}{\|} #2}}
%\newcommand{\gcing}[1]{\ensuremath{[#1]}}
\newcommand{\task}{T}
\newcommand{\config}{\mathcal{C}}

%% flattening
\newcommand{\flate}[1]{\hat{#1}}
\newcommand{\flatten}[3]{\left\| #2 \right\|_{#1} \leadsto #3}
\newcommand{\fstep}{\step}%{\step_F}

%% shorthands
\renewcommand{\a}{\ensuremath{\alpha}}
\renewcommand{\b}{\ensuremath{\beta}}
\newcommand{\h}{\ensuremath{\eta}}
\renewcommand{\r}{\ensuremath{\rho}}
%\newcommand{\s}{\ensuremath{\sigma}}
\newcommand{\p}{\ensuremath{P}}
\newcommand{\s}{\p}
\newcommand{\om}{\ensuremath{\Omega}}
\renewcommand{\l}{\ensuremath{l}}
\newcommand{\sig}{\ensuremath{\Sigma}}
\newcommand{\empctx}{\ensuremath{\cdot}}


% Relations
%\newcommand{\red}{\Downarrow}
%\newcommand{\redgc}{\stackrel{gc?}{\Longrightarrow}}
%\newcommand{\alloc}{\stackrel{alloc}\Longrightarrow}
\newcommand{\la}{\leftarrow}
\newcommand{\ra}{\rightarrow}
\newcommand{\pstep}{\Rightarrow}
\newcommand{\tstep}{\Rightarrow}
\newcommand{\optstep}{\longmapsto}
\newcommand{\compstep}{\longmapsto_{\delta}}
\newcommand{\localstep}[3]{{#2} \overset{#1}{\optstep} {#3}}
\newcommand{\globstep}[2]{{#1} \compstep {#2}}
%\newcommand{\stepr}[1]{\xra{#1}}
\newcommand{\step}{\ra}
\newcommand{\stepgc}[1]{\xra[{\mbox{\tiny GC}}]{#1}}
\newcommand{\gcstep}{\ra_{\mbox{\tiny GC}}}
\newcommand{\pgcstep}{\pstep_{\mbox{\tiny GC}}}
\newcommand{\cgcstep}{\rightarrow_{\mbox{\tiny CGC}}}


%\newcommand{\sunion}[2]{{#1} \stackrel{?}{\bigcup} {#2}}
%\newcommand{\spush}[2]{{#1} \stackrel{?}{\downarrow} {#2}}

% Other judgments
\newcommand{\fresh}{\ensuremath{\; \mathsf{fresh}}}
%\newcommand{\leaf}{\ensuremath{\; \mathsf{leaf}}}
%\newcommand{\starrow}[1]{\stackrel{\mbox{\tiny #1}}{\xrightarrow}}
\newcommand{\starrow}[1]{\xrightarrow{#1}}
%\newcommand{\alloc}[4]{\mathit{alloc}\left(#1, #2\right) = \left(#3, #4\right)}
\newcommand{\alloc}[4]{#1; #2 \starrow{alloc} #3; #4}
%\newcommand{\update}[4]{\mathit{update}\left(#1, #2 \la #3\right) = #4}
\newcommand{\update}[4]{#1; #2; #3 \starrow{update} #4}
%\newcommand{\lookup}[3]{\mathit{lookup}\left(#1, #2\right) = #3}
\newcommand{\lookup}[3]{#1; #2 \starrow{lookup} #3}
\newcommand{\newtask}[2]{#1 \starrow{new} #2}
\newcommand{\isdone}[3]{#1 \starrow{done} #2; #3}
\newcommand{\diffs}[1]{\mathit{diff}(#1)}
\newcommand{\initial}{\ensuremath{\;\mathsf{initial}}}
\newcommand{\htyped}[3]{\vdash_{#3} #1 : #2}

% Multilevel heap judgments
\newcommand{\heaptype}[3]{\left(#1, #2\right) : #3}
\newcommand{\allocg}[4]{\mathit{allocg}\left(#1, #2\right) = \left(#3, #4\right)}
\newcommand{\allocl}[4]{\mathit{allocl}\left(#1, #2\right) = \left(#3, #4\right)}
%\newcommand{\promote}[6]{\mathit{promote}\left(#1, #2, #3\right) =
%  \left(#4, #5, #6\right)}
\newcommand{\promote}[6]{#1; #2; #3 \starrow{promote} #4; #5; #6}
%\newcommand{\promotebrl}[3]{\mathit{promote}\left(#1, #2, #3\right)}
%\newcommand{\promotebrr}[3]{\left(#1, #2, #3\right)}
\newcommand{\promotebrl}[3]{#1; #2; #3}
\newcommand{\promotebra}{\starrow{promote}}
\newcommand{\promotebrr}[3]{#1; #2; #3}
\newcommand{\pmap}{M}
\newcommand{\greachable}[1]{\mathsf{greachable}\left(#1\right)}

% Theorems
\newtheorem{thm}[theorem]{Theorem}
\newtheorem{lem}[theorem]{Lemma}
% \newtheorem{corollary}[theorem]{Corollary}
% \newtheorem{claim}{Claim}


%% Rule Array
\newenvironment{rulearray}
{
\newcommand{\newcol}{\qquad}
\newcommand{\newcolhalf}{\quad}
\newcommand{\newrow}{\\[4ex]}
\newcommand{\newrowhalf}{\\[2ex]}
\[
\begin{array}{c}
}
{
\end{array}
\]
\let\newcol\undefined
\let\newrow\undefined
}

% Author-specific todo notes
\newcommand{\ramtodo}[2][]
{\todo[color=magenta,author=Ram,size=\small,#1]{#2}}


\newcommand{\defn}[1]{\emph{\textbf{#1}}}
\newcommand{\mpl}{\textsf{MPL}}

\newcommand{\rulereftwo}[2]{rules~\rulename{#1} and \rulename{#2}}
\newcommand{\with}{\ensuremath{\mathbin;}}

\newcommand{\highlight}[1]{\colorbox{gray!20}{\ensuremath{#1}}}
\newcommand{\hred}[1]{\colorbox{red!10}{\ensuremath{#1}}}
\newcommand{\hblue}[1]{\colorbox{blue!10}{\ensuremath{#1}}}
\newcommand{\hgreen}[1]{\colorbox{green!10}{\ensuremath{#1}}}
\newcommand{\sizeof}[1]{\ensuremath{\lvert #1 \rvert}}
\newcommand{\costof}[1]{\ensuremath{\mathbf{cost} ({#1})}}
\newcommand{\cost}{\ensuremath{\mathbf{cost}}}
\newcommand{\oracle}{\ensuremath{\mathbf{oracle}}}


% inline "math highlight" to make it easier to read inline judgements
\definecolor{darkblue}{HTML}{0007C9}
% \newcommand{\mh}[1]{{\color{darkblue}\ensuremath{\mbox{\ensuremath{#1}}}}}
\newcommand{\mh}[1]{{\ensuremath{\mbox{\ensuremath{#1}}}}}


%% variable context, location signature
% \newcommand{\ctxvar}{\Gamma}
% \newcommand{\ctxloc}{\Sigma}
\newcommand{\ctxemp}{\ensuremath{\cdot}}
\newcommand{\ctxext}[3]{\ensuremath{#1,#2\!:\!#3}} % extend context
\newcommand{\etyped}[4]{\ensuremath{{#1} \vdash_{#2} {#3} : {#4}}}
\newcommand{\memtyped}[3]{\ensuremath{{#1} \vdash {#2} : {#3}}}
\newcommand{\gtyped}[3]{\ensuremath{{#1} \vdash {#2} : {#3}}}
\newcommand{\httyped}[6]{\ensuremath{{#1} \with {#2} \with {#3} \vdash {#4}\!\cdot\!{#5} : {#6}}}
\newcommand{\ttyped}[5]{\ensuremath{{#1} \with {#2} \with {#3} \vdash {#4} : {#5}}}

\newcommand{\sttyped}[6]{\ensuremath{{\vdash_{#1} {#2} \with {#3} \with {#4} \with {#5} : {#6}}}}
\newcommand{\getyped}[6]{\ensuremath{{#1} \vdash_{#2, #3} {#4} \with {#5} : {#6}}}




%% types
\newcommand{\typnat}{\kw{nat}}
\newcommand{\typint}{\kw{int}}
\newcommand{\typbool}{\kw{bool}}
\newcommand{\typchar}{\kw{char}}
\newcommand{\typfloat}{\kw{float}}
\newcommand{\typprod}[2]{\ensuremath{{#1} \times {#2}}}
\newcommand{\typfun}[2]{\ensuremath{{#1}\!\rightarrow\!{#2}}}
\newcommand{\typref}[1]{\ensuremath{{#1}~\kw{ref}}}
\newcommand{\typfut}[1]{\ensuremath{{#1}~\kw{fut}}}
\newcommand{\futs}[1]{\mathsf{Fut}(#1)}
\newcommand{\futsmem}[2]{\mathsf{Fut}(#1, #2)}




% expression syntax
\newcommand{\enat}[1]{\ensuremath{#1}}
\newcommand{\efun}[3]{\ensuremath{\kw{fun}~{#1}~{#2}~\kw{is}~{#3}}}
\newcommand{\epair}[2]{\ensuremath{\langle {#1}, {#2} \rangle}}
\newcommand{\eapp}[2]{\ensuremath{{#1}~{#2}}}
\newcommand{\efst}[1]{\ensuremath{\kw{fst}~{#1}}}
\newcommand{\esnd}[1]{\ensuremath{\kw{snd}~{#1}}}
\newcommand{\eref}[1]{\ensuremath{\kw{ref}~{#1}}}
\newcommand{\ebang}[1]{\ensuremath{\mathop{!}#1}}
\newcommand{\eupd}[2]{\ensuremath{#1 \mathop{:=} #2}}
\newcommand{\elet}[3]{\kw{let}~{#1}={#2}~\kw{in}~{#3}}
\newcommand{\epar}[2]{\ensuremath{\langle {#1}\mathbin\|{#2} \rangle}}

\newcommand{\purelang}{{\sc $\lambda^{P}$}}
\newcommand{\reflang}{{\sc $\lambda^{U}$}}

% task syntax
% \newcommand{\tleaf}[2]{\ensuremath{{#1}\!\cdot\!{#2}}}
% \newcommand{\tpar}[4]{\ensuremath{\dblangle{{#1}\!\cdot\!{#2}\mathbin\|{#3}\!\cdot\!{#4}}}}
% \newcommand{\tpar}[4]{\ensuremath{\llparenthesis\,{#1}\!\cdot\!{#2}\mathbin\|{#3}\!\cdot\!{#4}\,\rrparenthesis}}
% \newcommand{\ttpar}[2]{\ensuremath{\llparenthesis\,{#1}\mathbin\|{#2}\,\rrparenthesis}}
% \newcommand{\tparg}[6]{\ensuremath{\llparenthesis\,{#1}\!\cdot\!{#2}\!\cdot\!{#3}\mathbin\|{#4}\!\cdot\!{#5}\!\cdot\!{#6}\,\rrparenthesis}}
% \newcommand{\tpar}[3]{\ensuremath{{#1}\!\cdot\!\llparenthesis\,{#2}\mathbin\|{#3}\,\rrparenthesis}}
% \newcommand{\taskhpe}[3]{\ensuremath{{#1}\!\cdot\!{#2}\!\cdot\!{#3}}}

% \newcommand{\mem}{\mu}
\newcommand{\mememp}{\emptyset}
\newcommand{\memext}[3]{\ensuremath{#1}[{#2} \!\hookrightarrow\! {#3}]}

\newcommand{\actarrow}{\blacktriangleright}
\newcommand{\pasarrow}{\vartriangleright}
\newcommand{\fmap}{\Delta}
\newcommand{\femp}{\emptyset}
\newcommand{\fmapactive}[3]{\ensuremath{#1} [{#2} \!\actarrow\! {#3}]}
\newcommand{\fmapjoined}[3]{\ensuremath{#1} [{#2} \!\pasarrow\! {#3}]}


\newcommand{\futctxt}{\Knownctxt}
\newcommand{\Futctxt}{\Knownctxt}
\newcommand{\ReadLocs}{\mathsf{R}}
\newcommand{\Knownctxt}{{K}}
\newcommand{\fut}[2]{\kw{fut}(#1; #2)}
\newcommand{\harpfut}[1]{\kw{fut}(#1)}
\newcommand{\futctxtemp}{\emptyset}
\newcommand{\te}[1]{\{#1\}}
\newcommand{\hemp}{\emptyset}
\newcommand{\hcat}{\cup}
\newcommand{\hext}[2]{{#1},{#2}}

\newcommand{\tack}{\oplus}
\newcommand{\plug}{\bowtie}

\newcommand{\omparam}{step length}
\newcommand{\actwrite}[2]{\textbf{U}{#1}\!\Leftarrow\!{#2}}
\newcommand{\actalloc}[2]{\textbf{A}{#1}\!\Leftarrow\!{#2}}
\newcommand{\actread}[2]{\textbf{R}{#1}\!\Rightarrow\!{#2}}
\newcommand{\actsync}[2]{\textbf{F}{#1}\!\Rightarrow\!{#2}}
\newcommand{\actnone}{\textbf{N}}

% Relations
\newcommand{\stepstar}{\longmapsto^*}
\newcommand{\tstepstar}{\tstep^*}
\newcommand{\drfstep}[2]{\xmapsto[{#2}]{{\,#1\,}}}
\newcommand{\drfstepstar}[1]{\xmapsto{{\,#1\,}}\joinrel\mathrel{^*}}


% computation graph
\newcommand{\gt}[2]{\ensuremath{\mathsf{GT}({#1},{#2})}}
\newcommand{\gemp}{\bullet}
\newcommand{\gseq}[2]{{#1}\oplus{#2}}
\newcommand{\gseqnamed}[3]{{#1}\oplus_{#2}{#3}}
\newcommand{\gseqa}[2]{\gseqnamed{#1}{a}{#2}}
\newcommand{\gseqb}[2]{\gseqnamed{#1}{b}{#2}}
\newcommand{\gspawn}[1]{\mathsf{spawn}\ {#1}}
\newcommand{\gsync}[1]{\mathsf{sync}\ {#1}}
\newcommand{\ghead}[1]{\mathsf{hd}(#1)}
\newcommand{\gtail}[1]{\mathsf{tl}(#1)}

% \def\ojoin{\setbox0=\hbox{$\bowtie$}%
%   \rule[\bt0]{.25em}{.4pt}\llap{\rule[\ht0]{.25em}{.4pt}}}


\newcommand{\fcpar}[3]{\ensuremath{\gseq{#1}{(\gpar{#2}{#3})}}}

% \def\rightouterjoin{\mathbin{\bowtie\mkern-5.8mu\ojoin}}

\newcommand{\gmerge}[2]{\bowtie_F ({#1}, {#2})}
\newcommand{\gmergerel}[3]{\bowtie_R ({#1}, {#2}) \downarrow {#3}}
% \newcommand{\gcseq}[1]{\ensuremath{\gseq{#1}{\raisebox{-1pt}{$\square$}}}}
% \newcommand{\gcseq}[1]{\ensuremath{\fbox{$#1$}}}
% \newcommand{\gcseq}[1]{\ensuremath{\llparenthesis #1 \rrparenthesis}}
\newcommand{\gcseq}[1]{\ensuremath{[#1]}}
% \newcommand{\gcpar}[3]{\ensuremath{{#1}\!\cdot\!({#2}\mathbin\|{#3})}}
\newcommand{\gcpar}[3]{\ensuremath{\gseq{#1}{(\gpar{#2}{#3})}}}
\newcommand{\gcparnamed}[4]{\ensuremath{\gseq{#1}{({#2}\otimes_{#3}{#4})}}}
\newcommand{\gcspawn}[4]{\ensuremath{\gseq{#1}{\gseq{#2}{(\gpar{#3}{#4})}}}}


\newcommand{\gpar}[2]{{#1}\otimes_{a}{#2}}
\newcommand{\gw}[1]{\ensuremath{\mathsf{W}({#1})}}
\newcommand{\ga}[1]{\ensuremath{\mathsf{A}({#1})}}
\newcommand{\greads}[1]{\ensuremath{\ReadLocs({#1})}}
\newcommand{\gaw}[1]{\ensuremath{\mathsf{AW}({#1})}}
\newcommand{\lw}[1]{\ensuremath{\mathsf{LW}({#1})}}
\newcommand{\alw}[1]{\ensuremath{\mathsf{A}({#1}) \cup \mathsf{LW}({#1})}}
\newcommand{\gabw}[1]{\ensuremath{\gaw{#1}}}
% \newcommand{\gabw}[1]{\ensuremath{\mathsf{A}({#1}) \cup \lw{#1}}}


\newcommand{\saw}[1]{\ensuremath{\mathsf{SP}({#1})}}
% \newcommand{\gf}[1]{\ensuremath{\llbracket{#1}\rrbracket}}
\newcommand{\gf}[1]{\ensuremath{\overline{#1}}}

\newcommand{\extendsfj}[2]{\ensuremath{{#1}~\textsf{extends}~{#2}~\textsf{with f/j}}}
\newcommand{\extendswith}[3]{\ensuremath{{#1}~\textsf{extends}~{#2}~\textsf{with}~{#3}}}

% \newcommand{\tpardag}[4]{\ensuremath{\llparenthesis\,{#1}\!\cdot\!{#2}\mathbin\|{#3}\!\cdot\!{#4}\rrparenthesis}}
% \newcommand{\dagobv}[2]{#1~\textsf{obv}~#2}
% \newcommand{\dagread}[2]{#1~\textsf{reads}~#2}
% \newcommand{\dagwrite}[2]{#1~\textsf{writes}~#2}
% \newcommand{\dagdrf}[2]{#1~\textsf{drf}~@~#2}

\newcommand{\geok}[2]{{#1} \with {#2}~\textit{ok}}
\newcommand{\loc}[1]{{#1}~\textit{loc}}
\newcommand{\gleaf}[1]{{#1}~\textit{leaf}}
\newcommand{\gnode}[1]{{#1}~\textit{node}}

\newcommand{\drf}[2]{{#1} \vdash {#2}~{\textit{drf}}}
\newcommand{\drfb}[2]{{#1} \vdash {#2}~{\textit{wrf}}}
\newcommand{\drft}{\textit{drf}}
% \newcommand{\typed}[2]{{#1} \vdash }

% Theorems
\theoremstyle{plain}
\newtheorem{property}{Property}

% \theoremstyle{definition}
% \newtheorem{definition}{Definition}

%% Rules description
\newcommand{\flushLR}[3]{\hspace*{#3}\makebox[0em][l]{#1}\hspace*{\fill}\makebox[0em][r]{#2}\hspace*{#3}}
% \newcommand{\rulesdesc}[2]{\flushLR{\textbf{#1}}{\fbox{#2}}{1em}}
\newcommand{\rulesdesc}[2]{\textbf{#1}\hspace*{1em}{\fbox{#2}}}
\newcommand{\desc}[1]{\textbf{#1}}

% Syntax highlighting
\newdimen\zzlistingsize
\newdimen\zzlistingsizedefault
\zzlistingsizedefault=9pt
\newdimen\kwlistingsize
\kwlistingsize=9pt
\zzlistingsize=\zzlistingsizedefault
\gdef\lco{black}
%\newcommand{\keywordstyle}{\fontsize{0.9\zzlistingsize}{1.0\zzlistingsize}\bf}
%\newcommand{\keywordstyle}{\fontsize{\kwlistingsize}{1.1\kwlistingsize}\normalfont\bf\color{\lco}}
%\settowidth{\zzlstwidth}{{\Lstbasicstyle~}}




\clearpage
\section{Additional proofs}

\paragraph{Proof of Theorem: Efficiency of segment optimization.}
The statement of the theorem is as follows:
\begin{quote}
  The function $\mathsf{segopt}(C)$ calls the oracle at most
  $\mathsf{length}(C) + 2\Delta$ times on segments of length at most $2\Omega$,
  where $\Delta$ is the improvement in the cost of the output.
  % Given a segment depth $\Omega$ and an oracle that optimizes for the cost function $\mathbf{cost}$,
  % the \coam{} algorithm optimizes a circuit $C$ of depth $D$
  % by calling the oracle at most $D + 2\Delta$ times on segments of depth at most $2\Omega$,
  % where $\Delta$ is the improvement in the cost of the output, assuming $\Omega \geq 2$.
\end{quote}
\begin{proof}
We prove the theorem for a tighter number of calls equal to $\mathsf{length}(C) + 2\Delta - 1$,
assuming $\Omega \geq 2$ and $\mathsf{length}(C) \geq 2$.
%
We proceed by induction on $\mathsf{length}(C)$.

In the base case, where $\mathsf{length}(C) \leq 2\Omega$, there is 1 call to the oracle.
%
This satisfies the desired bound due to $\mathsf{length}(C) \geq 2$.

For the inductive step, where $\mathsf{length}(C) > 2\Omega$, the algorithm splits into two halves $C_1$ and $C_2$,
such that $\mathsf{length}(C_1) < \mathsf{length}(C)$ and $\mathsf{length}(C_2) < \mathsf{length}(C)$.
%
% where $\sizeof{C_1} + \sizeof{C_2} = \sizeof C$,
% and $\costof{C_1} + \costof{C_2} = \costof C$.
%
The algorithm proceeds with two recursive calls
$C_1' = \mathsf{segopt}(C_1)$ and
$C_2' = \mathsf{segopt}(C_2)$, and finally computes the
output $C' = \mathsf{meld}(C_1', C_2')$.
%
The total number of calls to the oracle can be decomposed as:
\begin{itemize}
  \item Inductively,%
% \footnote{A technicality: note that, for the induction here to go through, we need
% $\sizeof{C_1},\sizeof{C_2} \geq 2$. These hold because we split in half:
% we have $\mathsf{length}(C) > 2\Omega \Rightarrow \mathsf{length}(C_1) \geq \Omega \Rightarrow \sizeof{C_1} \geq \Omega$
% (similarly for $C_2$), and finally note $\Omega \geq 2$.}
  in the first recursive call, at most $\mathsf{length}(C_1) + 2 * (\costof{C_1} - \costof{C_1'}) - 1$ calls to the oracle.
  \item Inductively, in the second recursive call, at most $\mathsf{length}(C_2)  + 2 * (\costof{C_2} - \costof{C_2'}) - 1$ calls to the oracle.
  \item In the meld, by Lemma 10 (``Bounded calls to oracle in meld''), at most $1 + 2 * (\costof{C_1'} + \costof{C_2'} - \costof{C'})$ calls to the oracle.
\end{itemize}
Adding these up yields exactly $\mathsf{length}(C) + 2 * (\costof{C} - \costof{C'}) - 1$, as desired.
\end{proof}

\clearpage

\section{Evaluation results with convergence ratio $\epsilon = 0$}
\begin{figure}[!ht]
  \small
\begin{tabular}{ccccccc}
                          &        &            & \multicolumn{4}{c}{Optimizer}         \\\cmidrule(lr){4-7}
Benchmark                 & Qubits & Input Size & Quartz  & Queso   & VOQC    & \algname{} \\\midrule{}
\multirow{4}{*}{boolsat}  & 28     & 75670      & -41.1\% & -30.4\% & -83.2\% & \textbf{-83.7\%} \\
                          & 30     & 138293     & -23.5\% & -30.2\% & -83.3\% & \textbf{-83.7\%} \\
                          & 32     & 262548     & -6.9\%  & 0.0\%   & -83.3\% & \textbf{-83.5\%} \\
                          & 34     & 509907     & -2.9\%  & 0.0\%   & -83.3\% & \textbf{-83.4\%} \\\midrule{}
\multirow{4}{*}{bwt}      & 17     & 262514     & -8.7\%  & -0.1\%  & -30.0\% & \textbf{-31.2\%} \\
                          & 21     & 402022     & -3.1\%  & -0.1\%  & -38.4\% & \textbf{-40.0\%} \\
                          & 25     & 687356     & -0.8\%  & 0.0\%   & N.A.    & \textbf{-43.8\%} \\
                          & 29     & 941438     & -0.4\%  & 0.0\%   & N.A.    & \textbf{-44.5\%} \\\midrule{}
\multirow{4}{*}{grover}   & 9      & 8968       & -9.4\%  & -13.7\% & \textbf{-29.4\%} & \textbf{-29.4\%} \\
                          & 11     & 27136      & -9.5\%  & -9.6\%  & -29.9\% & \textbf{-30.0\%} \\
                          & 13     & 72646      & -9.6\%  & -0.3\%  & \textbf{-29.7\%} & \textbf{-29.7\%} \\
                          & 15     & 180497     & -9.6\%  & -0.2\%  & \textbf{-29.5\%} & \textbf{-29.5\%} \\\midrule{}
\multirow{4}{*}{hhl}      & 7      & 5319       & -26.6\% & -28.7\% & \textbf{-55.4\%} & -55.3\% \\
                          & 9      & 63392      & -24.4\% & -23.8\% & -56.3\% & \textbf{-56.5\%} \\
                          & 11     & 629247     & -1.1\%  & 0.0\%   & \textbf{-53.7\%} &\textbf{ -53.7\%} \\
                          & 13     & 5522186    & 0.0\%   & N.A.    & N.A.    & \textbf{-52.6\%} \\\midrule{}
\multirow{4}{*}{shor}     & 10     & 8476       & 0.0\%   & -5.4\%  & \textbf{-11.1\%} & -11.0\% \\
                          & 12     & 34084      & 0.0\%   & -3.8\%  & \textbf{-11.2\%} & \textbf{-11.2\%} \\
                          & 14     & 136320     & 0.0\%   & 0.0\%   & \textbf{-11.3\%} & -11.2\% \\
                          & 16     & 545008     & 0.0\%   & 0.0\%   & \textbf{-11.3\%} &\textbf{ -11.3\%} \\\midrule{}
\multirow{4}{*}{sqrt}     & 42     & 79574      & -16.2\% & -0.1\%  & \textbf{-33.0\%} &\textbf{ -33.0\%} \\
                          & 48     & 186101     & -15.2\% & 0.0\%   & \textbf{-32.7\%} & -32.6\% \\
                          & 54     & 424994     & -5.1\%  & 0.0\%   & \textbf{-32.4\%} & -32.3\% \\
                          & 60     & 895253     & -2.1\%  & 0.0\%   & N.A.    & \textbf{-34.3\%} \\\midrule{}
\multirow{4}{*}{statevec} & 5      & 31000      & -48.4\% & -74.5\% & -78.8\% & \textbf{-78.9\%} \\
                          & 6      & 129827     & -35.0\% & -29.7\% & \textbf{-78.4\%} & \textbf{-78.4\%} \\
                          & 7      & 526541     & -2.0\%  & -29.7\% & \textbf{-78.1\%} & \textbf{-78.1\%} \\
                          & 8      & 2175747    & -0.1\%  & 0.0\%   & N.A.    & \textbf{-78.7\%} \\\midrule{}
\multirow{4}{*}{vqe}      & 12     & 11022      & -35.4\% & -69.1\% & -63.0\% & \textbf{-69.5\%} \\
                          & 16     & 22374      & -33.7\% & -64.2\% & -60.1\% &\textbf{ -66.3\%} \\
                          & 20     & 38462      & -32.2\% & -60.8\% & -57.4\% & \textbf{-63.4\%} \\
                          & 24     & 59798      & -30.8\% & -36.4\% & -54.9\% & \textbf{-60.6\%} \\\midrule{}
avg                       &        &            & -13.6\% & -16.5\% & -48.1\% & \textbf{-49.4\%}
\end{tabular}
\caption{The figure shows percentage reduction in gate count achieved by the four optimizers.
%
The measurements show
that our \algname{} optimizer reduces gate counts better than all other optimizers.
%
Specifically, our \algname{} optimizer delivers a slight improvement on \voqc{}, which is the best among all the other optimizers.
%
Given that our \algname{} optimizer is faster than all other optimizers (\figref{main-time}), these experiments show that local optimality approach can improve performance without any loss in optimization quality.
%
These results suggest that local optimality approach to optimization of quantum circuits can be effective in practice.
}    \label{fig:app-main-gate-count}
\end{figure}
\begin{figure}[!ht]
    \centering
\begin{tabular}{cccccccc}
                          &        &            & \multicolumn{4}{c}{Time (s)}                       &                                                           \\\cmidrule(lr){4-7}
Benchmark                 & Qubits & Input Size & Quartz & Queso         & VOQC    & algname          & \begin{tabular}[c]{@{}c@{}}algname\\ speedup\end{tabular} \\\midrule{}
\multirow{4}{*}{boolsat}  & 28     & 75670      & 12h    & 12h           & 68.6    & \textbf{60.4}    & 1.14                                                      \\
                          & 30     & 138293     & 12h    & 12h           & 307.3   & \textbf{99.8}    & 3.08                                                      \\
                          & 32     & 262548     & 12h    & 12h           & 1266.2  & \textbf{292.8}   & 4.32                                                      \\
                          & 34     & 509907     & 12h    & 12h           & 6151.0  & \textbf{600.4}   & 10.24                                                     \\\midrule{}
\multirow{4}{*}{bwt}      & 17     & 262514     & 12h    & 12h           & 8303.1  & \textbf{1196.8}  & 6.94                                                      \\
                          & 21     & 402022     & 12h    & 12h           & 23236.8 & \textbf{2281.0}  & 10.19                                                     \\
                          & 25     & 687356     & 12h    & 12h           & 43200.0 & \textbf{5698.9}  & 7.58                                                      \\
                          & 29     & 941438     & 12h    & 12h           & 43200.0 & \textbf{11842.2} & 3.65                                                      \\\midrule{}
\multirow{4}{*}{grover}   & 9      & 8968       & 12h    & 12h           & 9.3     & \textbf{6.9}     & 1.35                                                      \\
                          & 11     & 27136      & 12h    & 12h           & 106.5   & \textbf{29.6}    & 3.59                                                      \\
                          & 13     & 72646      & 12h    & 12h           & 815.7   & \textbf{99.1}    & 8.23                                                      \\
                          & 15     & 180497     & 12h    & 12h           & 5743.2  & \textbf{320.3}   & 17.93                                                     \\\midrule{}
\multirow{4}{*}{hhl}      & 7      & 5319       & 12h    & 12h           & \textbf{0.3}     & 1.2     & 0.21                                                      \\
                          & 9      & 63392      & 12h    & 12h           & 74.1    & \textbf{31.6}    & 2.35                                                      \\
                          & 11     & 629247     & 12h    & 12h           & 14868.8 & \textbf{636.5}   & 23.36                                                     \\
                          & 13     & 5522186    & 12h    & Parsing Error & 43200.0 & \textbf{13165.3} & 3.28                                                      \\\midrule{}
\multirow{4}{*}{shor}     & 10     & 8476       & 12h    & OOM           & 8.8     & \textbf{5.3}     & 1.65                                                      \\
                          & 12     & 34084      & 12h    & 12h           & 179.9   & \textbf{38.5}    & 4.67                                                      \\
                          & 14     & 136320     & 12h    & 12h           & 3638.4  & \textbf{190.0}   & 19.15                                                     \\
                          & 16     & 545008     & 12h    & 12h           & 70475.2 & \textbf{943.0}   & 74.73                                                     \\\midrule{}
\multirow{4}{*}{sqrt}     & 42     & 79574      & 12h    & 12h           & \textbf{30.0}    & 123.8   & 0.24                                                      \\
                          & 48     & 186101     & 12h    & 12h           & \textbf{191.2}   & 341.3   & 0.56                                                      \\
                          & 54     & 424994     & 12h    & 12h           & 3946.5  & \textbf{852.6}   & 4.63                                                      \\
                          & 60     & 895253     & 12h    & 12h           & 43200.0 & \textbf{2108.0}  & 20.49                                                     \\\midrule{}
\multirow{4}{*}{statevec} & 5      & 31000      & 12h    & OOM           & \textbf{1.6}     & 5.0     & 0.33                                                      \\
                          & 6      & 129827     & 12h    & 12h           & 45.9    & \textbf{35.1}    & 1.31                                                      \\
                          & 7      & 526541     & 12h    & 12h           & 1812.2  & \textbf{202.4}   & 8.95                                                      \\
                          & 8      & 2175747    & 12h    & 12h           & 43200.0 & \textbf{2108.8}  & 20.49                                                     \\\midrule{}
\multirow{4}{*}{vqe}      & 12     & 11022      & 12h    & 12h           & \textbf{0.2}     & 1.5     & 0.10                                                      \\
                          & 16     & 22374      & 12h    & 12h           & \textbf{0.6 }    & 4.3     & 0.15                                                      \\
                          & 20     & 38462      & 12h    & 12h           & \textbf{2.0}     &8.9     & 0.23                                                      \\
                          & 24     & 59798      & 12h    & 12h           & \textbf{5.4}     & 17.3    & 0.32                                                      \\\midrule{}
avg                       &        &            &        &               &         &                  & 8.30
\end{tabular}

\caption{The figure shows the running time in seconds of the four optimizers, using gate count as the cost metric.
%
The column "\algname{} Speedup" is the speed of our  \algname{} with respect to \voqc{}, calculated as \voqc{} time divided by \algname{} time.
%
These measurements show that our optimizer \algname{} can be significantly faster, especially for large circuits ($8.3$x faster on average).
As \figref{main-gate-count} shows, these time improvements come without any loss in optimization quality.
%
These results suggest that local optimality approach to optimization of quantum circuits can be effective in practice.}
    %
    \label{fig:main-time}
\end{figure}
\input{fig/fig-app-compression-factor}
\begin{figure}
  \centering\small
  \input{fig/feyn.app.tex}
  \caption{
  The figure shows the optimization results of optimizers
  $\algname{}$ and \feyntool{}, with $\mathsf{T}$ count as the cost function.
  %
  It shows the running time in seconds for both optimizers (lower is better)
  and calculates the speedup of our \algname{} by taking the ratio of the
  two timings.
  %
  The figure also shows the $\mathsf{T}$ count reductions of both tools.
  %
  The results show that our $\algname{}$ delivers excellent time performance
  and runs almost an order of magnitude ($10\times$) faster than \feyntool{} on average.
  %
  Our \algname{} optimizer achieves this speedup without any sacrifice in circuit quality,
  and in fact sometimes produces better quality of circuits than \feyntool{}.
%
}
\end{figure}
\clearpage

% \section{Evaluation of configuration $\coamwith{\quartzt{0.1}}$}
% \figref{comp-med} shows the result of evaluation.
% \begin{figure}
%   \centering
%   \begin{tabular}{ccccccc}
  &  &  & \multicolumn{3}{c}{Number of optimizations} &  \\ \cmidrule(lr){4-6}
   Family & Qubits & Input Size & Q & S & S/Q & Time (s) \\


  \midrule\multirow{2}{*}{ham15}  & 17 &   1061 & 202 (19\%)   & 202 (19\%)             & 1.0x    &    97 \\
                          & 20 &   4365 & 992 (23\%)   & \textbf{1002} (23\%)   & 1.01x   &   383 \\
  \midrule\multirow{3}{*}{hhl}    &  7 &   5319 & 1443 (27\%)  & \textbf{1738} (33\%)   & 1.2x    &  1242 \\
                          &  9 &  63392 & 14739 (23\%) & \textbf{17950} (28\%)  & 1.22x   & 10803 \\
                          & 11 & 629247 & 1343 (0\%)   & \textbf{146256} (23\%) & 108.82x & 10798 \\
  \midrule\multirow{2}{*}{gf}     & 48 &   2694 & 0 (0\%)      & \textbf{3} (0\%)       & 4.0x    &   284 \\
                          & 96 &   7553 & 4124 (55\%)  & 4124 (55\%)            & 1.0x    &   380 \\
  \midrule\multirow{4}{*}{grover} &  7 &   2479 & 252 (10\%)   & 252 (10\%)             & 1.0x    &   229 \\
                          &  9 &   8968 & 876 (10\%)   & \textbf{877} (10\%)    & 1.0x    &   839 \\
                          & 11 &  27136 & 2649 (10\%)  & \textbf{2672} (10\%)   & 1.01x   &  2712 \\
                          & 15 & 180497 & 7710 (4\%)   & \textbf{17588} (10\%)  & 2.28x   & 10800 \\
  \midrule\multirow{4}{*}{qft}    & 48 &   4626 & 757 (16\%)   & \textbf{1102} (24\%)   & 1.46x   &  2217 \\
                          & 64 &   7402 & 1893 (26\%)  & \textbf{2366} (32\%)   & 1.25x   &  3329 \\
                          & 80 &  10690 & 3540 (33\%)  & \textbf{4144} (39\%)   & 1.17x   &  4272 \\
                          & 96 &  14490 & 5700 (39\%)  & \textbf{6448} (44\%)   & 1.13x   &  5236 \\
  \midrule\multirow{4}{*}{vqe}    & 12 &  11022 & 7407 (67\%)  & \textbf{7450} (68\%)   & 1.01x   &   619 \\
                          & 16 &  22374 & 14343 (64\%) & \textbf{14425} (64\%)  & 1.01x   &  1284 \\
                          & 20 &  38462 & 16061 (42\%) & \textbf{23604} (61\%)  & 1.47x   &  2002 \\
                          & 24 &  59798 & 10140 (17\%) & \textbf{35130} (59\%)  & 3.46x   &  3277 \\

 \midrule
 \end{tabular}
%   \caption{
%     The figure compares the performance of $\coamwith{\quartzt{0.1}}$ against standard \quartz{}.
%     %
%     The label ``S'' represents $\coamwith{\quartzt{0.1}}$ and the label ``Q'' represents \quartz{}.
%     %
%     The figure compares the number of optimizations found by both appraoches in the same end-to-end time.
%     %
%     We impose a timeout of three hours.
%     }
%     \label{fig:comp-med}
%   \end{figure}


% \section{Evaluation of configuration $\coamwith{\quartzt{1}}$}
% \begin{figure}
%   \centering
%   \begin{tabular}{ccccccc}
  &  &  & \multicolumn{3}{c}{Number of optimizations} &  \\ \cmidrule(lr){4-6}
   Family & Qubits & Input Size & Q & S & S/Q & Time (s) \\


  \midrule\multirow{2}{*}{ham15}  & 17 &   1061 & 202 (19\%)   & \textbf{260} (25\%)    & 1.29x   &  1198 \\
                          & 20 &   4365 & 992 (23\%)   & \textbf{1120} (26\%)   & 1.13x   &  6065 \\
  \midrule\multirow{3}{*}{hhl}    &  7 &   5319 & 1451 (27\%)  & \textbf{1859} (35\%)   & 1.28x   & 10767 \\
                          &  9 &  63392 & 14708 (23\%) & \textbf{16114} (25\%)  & 1.1x    & 10776 \\
                          & 11 & 629247 & 1338 (0\%)   & \textbf{144785} (23\%) & 108.13x & 10763 \\
  \midrule\multirow{2}{*}{gf}     & 48 &   2694 & 0 (0\%)      & \textbf{23} (1\%)      & 24.0x   &  3549 \\
                          & 96 &   7553 & 4124 (55\%)  & 4124 (55\%)            & 1.0x    &  3025 \\
  \midrule\multirow{4}{*}{grover} &  7 &   2479 & 252 (10\%)   & \textbf{270} (11\%)    & 1.07x   &  3066 \\
                          &  9 &   8968 & 876 (10\%)   & \textbf{1025} (11\%)   & 1.17x   & 10785 \\
                          & 11 &  27136 & 2649 (10\%)  & \textbf{2779} (10\%)   & 1.05x   & 10806 \\
                          & 15 & 180497 & 7700 (4\%)   & \textbf{17747} (10\%)  & 2.3x    & 10788 \\
  \midrule\multirow{4}{*}{qft}    & 48 &   4626 & 763 (16\%)   & \textbf{1119} (24\%)   & 1.47x   & 10786 \\
                          & 64 &   7402 & 1895 (26\%)  & \textbf{2259} (31\%)   & 1.19x   & 10805 \\
                          & 80 &  10690 & 3541 (33\%)  & \textbf{3906} (37\%)   & 1.1x    & 10810 \\
                          & 96 &  14490 & 5701 (39\%)  & \textbf{6046} (42\%)   & 1.06x   & 10763 \\
  \midrule\multirow{4}{*}{vqe}    & 12 &  11022 & 7407 (67\%)  & \textbf{7919} (72\%)   & 1.07x   & 10761 \\
                          & 16 &  22374 & 14343 (64\%) & \textbf{14828} (66\%)  & 1.03x   & 10771 \\
                          & 20 &  38462 & 23551 (61\%) & \textbf{23965} (62\%)  & 1.02x   & 10787 \\
                          & 24 &  59798 & 35031 (59\%) & \textbf{35474} (59\%)  & 1.01x   & 10775 \\

 \midrule
 \end{tabular}

%   \caption{
%     The figure compares the performance of $\coamwith{\quartzt{1}}$ against standard \quartz{}.
%     %
%     The label ``S'' represents $\coamwith{\quartzt{1}}$ and the label ``Q'' represents \quartz{}.
%     %
%     The figure compares the circuit sizes (lower is better) and the optimization rates (higher is better) of both approaches
%     after running them for the same end-to-end time.
%     %
%     }
%     \label{fig:comp-high}
%   \end{figure}

%   \figref{comp-high} shows the result of evaluation.
% \clearpage
\section{Data for experiment varying $\Omega$}
\begin{table}[h]
  \centering
  \caption{Results for the \coamwith{\feyntool} optimizer on the \texttt{hhl} circuit with $7$ qubits.}
  \begin{tabular}{ccc}
      \toprule
      Omega & Output T count & Time \\
      \midrule
      2 & 42224 & 577 \\
      5 & 42224 & 240.16 \\
      15 & 42214 & 81.10 \\
      30 & 42174 & 52.47 \\
      60 & 42146 & 34.58 \\
      120 & 42128 & 28.16 \\
      240 & 42128 & 32.08 \\
      480 & 42118 & 47.22 \\
      960 & 42116 & 77.85 \\
      1920 & 42116 & 138.16 \\
      3840 & 42116 & 286.31 \\
      7680 & 42116 & 685.69 \\
      \bottomrule
  \end{tabular}
\end{table}

% \section{\queso{} comparison}


%





% %
% We integrate the \queso{} optimizer into our algorithm as an oracle
% and impose a one second time out for each call.
% %
% We note that \queso{} does not follow the given timeout strictly
% and often takes longer to run and return the result (around $4/5$ seconds per call).

% To evaluate the algorithm,
% we run it with \queso{} and compute the final circuit
% and the overall execution time, excluding time for file I/O (\secref{impl}).
% %
% Then, we execute default \queso{} on the whole circuit for the same total runtime
% and compare the quality of circuits produced.
% %
% For this experiment,
% we filter benchmarks based on their sizes and
% opt for benchmarks whose sizes are less than ten thousand gates.
% %
% This size limitation arises from file I/O overheads which make restrict our ability
% to run larger circuits within reasonable times.

% \figref{comp-queso} shows the results of the evaluation for seven benchmarks
% coming from four families of algorithms.
% %
% The ``QS'' column denotes \queso{} and the ``S'' column represents the \coam{} algorithm
% using \queso{} as the oracle.
% %
% In all cases, the \coam{} approach generates smaller circuits than the baseline \queso{}.
% %
% Similar to the case with \quartz{} (\secref{quartz}),
% we observe that the gap between the \coam{} approach and default \queso{} increases
% with circuit size.
% %
% Foe example,
% in the grover family,
% the ratio of optimization rate between \coam{} and \queso{}
% increases from $1.14$x for grover\_n7 to $1.4$x in grover\_n9.
% %
% For the circuit vqe\_n12,
% which is the largest circuit on the table,
% the final circuit size is $0.78$ times smaller compared to \queso{},
% which demonstrates the scalability of the \coam{} approach.
% %
% On average,
% the algorithm's circuits are $0.92$x smaller than those generated by \queso{}
% and the algorithm makes $1.27$x as many optimizations per second.
% %

\begin{figure*}
  \centering
  \begin{tabular}{rccccccccc}
 &  & \multicolumn{3}{c}{Output Size and Improvement} &  \multicolumn{3}{c}{Optimization Rate (gates/sec)} & &  \\ \cmidrule(lr){3-5} \cmidrule(lr){6-8}
  Benchmark & Input Size & QS & S & QS/S & QS & S & S/QS  & Time (s) & QS Time (s) \\ 

\midrule
 ham15-med       &   1061 & \textbf{855} (-19\%) & 859 (-19\%)             & 1.0x  & 20.6  &  24.67 & 1.2x    &   8 &   10 \\
 ham15-high      &   4365 & 3715 (-15\%)         & \textbf{3373} (-23\%)   & 1.1x  & 12.5  &  44.64 & 3.57x   &  22 &   52 \\
 hhl\_n7         &   5319 & 4323 (-19\%)         & \textbf{3878} (-27\%)   & 1.11x & 76.62 & 134.64 & 1.76x   &  10 &   13 \\
 hhl\_n9         &  63392 & 62896 (-1\%)         & \textbf{47714} (-25\%)  & 1.32x &  3.31 & 119.7  & 36.16x  & 130 &  150 \\
 gf2\^{}16\_mult &   2694 & 2694 (0\%)           & 2694 (0\%)              & 1.0x  &  0    &   0    & 1.0x    &   8 &   38 \\
 gf2\^{}32\_mult &   7553 & 6527 (-14\%)         & \textbf{3429} (-55\%)   & 1.9x  &  5.97 &  29.93 & 5.01x   & 137 &  172 \\
 grover\_n7      &   2479 & 2293 (-8\%)          & \textbf{2227} (-10\%)   & 1.03x & 15.5  &  26.46 & 1.71x   &   9 &   12 \\
 grover\_n9      &   8968 & 8936 (0\%)           & \textbf{8092} (-10\%)   & 1.1x  &  0.8  &  30.77 & 38.46x  &  28 &   40 \\
 grover\_n11     &  27136 & 27078 (0\%)          & \textbf{24493} (-10\%)  & 1.11x &  0.52 &  33.64 & 64.69x  &  78 &  111 \\
 grover\_n15     & 180497 & 180215 (0\%)         & \textbf{162909} (-10\%) & 1.11x &  0.06 &  30.13 & 502.17x & 583 & 4720 \\
 qft\_n48        &   4626 & 3870 (-16\%)         & 3870 (-16\%)            & 1.0x  & 29.08 &  50.28 & 1.73x   &  15 &   26 \\
 qft\_n64        &   7402 & 5510 (-26\%)         & 5510 (-26\%)            & 1.0x  & 55.65 &  59.76 & 1.07x   &  31 &   34 \\
 qft\_n80        &  10690 & 7150 (-33\%)         & 7150 (-33\%)            & 1.0x  & 49.17 &  75.68 & 1.54x   &  46 &   72 \\
 qft\_n96        &  14490 & 8790 (-39\%)         & 8790 (-39\%)            & 1.0x  & 45.6  &  85.43 & 1.87x   &  66 &  125 \\
 qpe\_n8         &   2040 & \textbf{1984} (-3\%) & 2040 (0\%)              & 0.97x &  6.22 &   0    & 0.0x    &   4 &    9 \\
 qpe\_n10        &   8476 & \textbf{8230} (-3\%) & 8476 (0\%)              & 0.97x &  2.56 &   0    & 0.0x    &  17 &   96 \\
 vqe\_n12        &  11022 & 6402 (-42\%)         & \textbf{3615} (-67\%)   & 1.77x & 73.33 & 198.8  & 2.71x   &  37 &   63 \\
 vqe\_n16        &  22374 & 13442 (-40\%)        & \textbf{8031} (-64\%)   & 1.67x & 69.24 & 163.11 & 2.36x   &  87 &  129 \\
 vqe\_n20        &  38462 & 23810 (-38\%)        & \textbf{14911} (-61\%)  & 1.6x  & 86.7  & 141.59 & 1.63x   & 166 &  169 \\
 vqe\_n24        &  59798 & 38018 (-36\%)        & \textbf{24767} (-59\%)  & 1.54x & 56.57 & 110.56 & 1.95x   & 316 &  385 \\
\midrule
\textbf{geomean} & & & & 1.18 & &  & 5.15 & & 
\end{tabular}
  \caption{The figure compares the performance of $\coamt{\quartzt{0.06}}$ against standard \queso{}.
  The label ``S'' represents $\coamt{\quartzt{0.06}}$ and the label ``QS'' represents \queso{}.
  %
  The figure compares the circuit sizes (lower is better) and the optimization rates (higher is better) of both approaches.
  %
  The optimization rate measures the number of optimizations performed per second of execution.
  %
  The figure also shows the improvement factor $QS/S$ for circuit size and $S/QS$ for optimization rate,
  quantifying the improvement achieved by \coam{}.
  %
  The Time column takes the time it takes to run both of them.
  %
  \queso{} sometimes runs longer because its timeout functionality is not very precise.
  %
  }
  \label{fig:comp-low-queso}
\end{figure*}



In this section,
we compare the circuit quality by the \coam{} algorithm with configurations \coamwith{\quartzt{0.06}}
and \coamwith{\quartzt{0.6}} against the standard \queso{} optimizer.
%
Our methodology is the same,
for each configuration we take the running time,
and then execute \queso{} with the same running time.
%

We note that even though \queso{} implements a timeout facility,
it does not adhere to it.
%
We observed that the tool takes an unpredictable amount of time
taking as much as $200$x the given timeout.
%
To address this,
we changed the code in \queso{}, making it check timeouts more eagerly.
%
Though our changes reduced the unpredictability,
there is some difference in the running time with \queso{} running
longer than our algorithm.
%
We note that our algorithm never runs for a larger time.
%

\figref{comp-low-queso} compares the configuration \coamwith{\quartzt{0.06}}
with \queso{}, when given the same timeout.
%
We observe that in three cases \queso{} generates a better quality circuit,
but it also takes more time.
%
In twelve cases, our algorithm generates better quality circuits,
and these cases correspond to larger sized circuits,
showcasing the scalability benefits of our approach.
%
On average, our algorithm generates smaller circuits by a factor of $1.18$x
and its average optimization rate is $5$x compared to \queso{}.
%

\figref{comp-med-queso} similarly compares the configuration  \coamwith{\quartzt{0.06}}.




\begin{figure*}
  \centering
  \begin{tabular}{rccccccccc}
 &  & \multicolumn{3}{c}{Output Size and Improvement} &  \multicolumn{3}{c}{Optimization Rate (gates/sec)} & &  \\ \cmidrule(lr){3-5} \cmidrule(lr){6-8}
  Benchmark & Input Size & QS & S & QS/S & QS & S & S/QS  & Time (s) & QS Time (s) \\ 

\midrule
 ham15-med       &   1061 & \textbf{821} (-23\%)  & 859 (-19\%)             & 0.96x & 2.42 &  2.08 & 0.86x  &    97 &    99 \\
 ham15-high      &   4365 & 3373 (-23\%)          & \textbf{3363} (-23\%)   & 1.0x  & 2.57 &  2.61 & 1.02x  &   383 &   386 \\
 hhl\_n7         &   5319 & 3793 (-28\%)          & \textbf{3581} (-33\%)   & 1.06x & 1.22 &  1.4  & 1.15x  &  1242 &  1253 \\
 hhl\_n9         &  63392 & 52689 (-17\%)         & \textbf{45442} (-28\%)  & 1.16x & 0.99 &  1.66 & 1.68x  & 10803 & 10824 \\
 gf2\^{}16\_mult &   2694 & 2694 (0\%)            & \textbf{2691} (0\%)     & 1.0x  & 0    &  0.01 & 1.0x   &   284 &   291 \\
 gf2\^{}32\_mult &   7553 & 7553 (0\%)            & \textbf{3429} (-55\%)   & 2.2x  & 0    & 10.84 & 1.0x   &   380 &   387 \\
 grover\_n7      &   2479 & 2234 (-10\%)          & \textbf{2227} (-10\%)   & 1.0x  & 1.06 &  1.1  & 1.04x  &   229 &   232 \\
 grover\_n9      &   8968 & 8254 (-8\%)           & \textbf{8091} (-10\%)   & 1.02x & 0.85 &  1.04 & 1.22x  &   839 &   843 \\
 grover\_n11     &  27136 & 27078 (0\%)           & \textbf{24464} (-10\%)  & 1.11x & 0.02 &  0.99 & 49.5x  &  2712 &  2727 \\
 grover\_n15     & 180497 & 180215 (0\%)          & \textbf{162909} (-10\%) & 1.11x & 0.01 &  1.63 & 163.0x & 10800 & 21921 \\
 qft\_n48        &   4626 & \textbf{3333} (-28\%) & 3524 (-24\%)            & 0.95x & 0.58 &  0.5  & 0.86x  &  2217 &  2228 \\
 qft\_n64        &   7402 & \textbf{4811} (-35\%) & 5036 (-32\%)            & 0.96x & 0.78 &  0.71 & 0.91x  &  3329 &  3341 \\
 qft\_n80        &  10690 & \textbf{6314} (-41\%) & 6546 (-39\%)            & 0.96x & 1.02 &  0.97 & 0.95x  &  4272 &  4282 \\
 qft\_n96        &  14490 & \textbf{7947} (-45\%) & 8042 (-44\%)            & 0.99x & 1.24 &  1.23 & 0.99x  &  5236 &  5278 \\
 qpe\_n8         &   2040 & \textbf{1956} (-4\%)  & 2035 (0\%)              & 0.96x & 0.08 &  0.02 & 0.25x  &   241 &  1058 \\
 qpe\_n10        &   8476 & \textbf{8155} (-4\%)  & 8474 (0\%)              & 0.96x & 0.36 &  0    & 0.0x   &   879 &   886 \\
 vqe\_n12        &  11022 & 5982 (-46\%)          & \textbf{3572} (-68\%)   & 1.67x & 8.08 & 12.02 & 1.49x  &   619 &   624 \\
 vqe\_n16        &  22374 & 13442 (-40\%)         & \textbf{7949} (-64\%)   & 1.69x & 6.75 & 11.23 & 1.66x  &  1284 &  1323 \\
 vqe\_n20        &  38462 & 23810 (-38\%)         & \textbf{14858} (-61\%)  & 1.6x  & 7.04 & 11.79 & 1.67x  &  2002 &  2080 \\
 vqe\_n24        &  59798 & 59798 (0\%)           & \textbf{24668} (-59\%)  & 2.42x & 0    & 10.72 & 1.0x   &  3277 &  3419 \\
\midrule
\textbf{geomean} & & & & 1.18 & &  & 2.92 & & 
\end{tabular}
  \caption{The figure compares the performance of $\coamt{\quartzt{0.6}}$ against standard \queso{}.
  The label ``S'' represents $\coamt{\quartzt{0.06}}$ and the label ``QS'' represents \queso{}.
  %
  The figure compares the circuit sizes (lower is better) and the optimization rates (higher is better) of both approaches.
  %
  The optimization rate measures the number of optimizations performed per second of execution.
  %
  The figure also shows the improvement factor $QS/S$ for circuit size and $S/QS$ for optimization rate,
  quantifying the improvement achieved by \coam{}.
  %
  The Time column takes the time it takes to run both of them.
  %
  \queso{} sometimes runs longer because its timeout functionality is not very precise.
  %
  }
  \label{fig:comp-med-queso}
\end{figure*}

% \begin{abstract}

% \end{abstract}

\bibliographystyle{plain}
% \bibliography{references}

\bibliography{Ref_QCS_Ding,local}
\end{document}


