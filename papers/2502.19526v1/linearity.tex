\subsection{\coam{} scales linearly with the size of circuits}
\subsection{\coam{} scales linearly with the size of circuits}
\subsection{\coam{} scales linearly with the size of circuits}
\subsection{\coam{} scales linearly with the size of circuits}
\input{fig/linearity.tex}
\label{sec:linear}

We show that the \coam{} algorithm requires only linear time
by measuring its execution time across different sizes of benchmarks.
%
For this experiment, we use $\quartzt{0.06}$ as our oracle
(i.e., each sub circuit is optimized by calling \quartz{} for a duration of $0.06$ seconds).
%
We consider four quantum algorithms: Grover, VQE, QPE, and QFT,
and plot the results in \figref{linear}.
%
The Y-axis shows the execution time and the X-axis shows the size of benchmarks.
%
We vary the size of circuits by changing the number of qubits in the corresponding algorithms.

The plots show that the execution time scales linearly
across sizes ranging from a few hundreds of gates to circuits as large as $0.14$ millions
of gates.
%
For example,
in case of the VQE algorithm,
the \coam{} algorithm takes around five seconds for the smallest circuit of size two thousand,
and takes around 300 seconds for the largest circuit of size one hundred thousand ($136000$ precisely).
%
We provide the data points for our plots and the plot for the HHL family in our Supplementary.
%
\label{sec:linear}

We show that the \coam{} algorithm requires only linear time
by measuring its execution time across different sizes of benchmarks.
%
For this experiment, we use $\quartzt{0.06}$ as our oracle
(i.e., each sub circuit is optimized by calling \quartz{} for a duration of $0.06$ seconds).
%
We consider four quantum algorithms: Grover, VQE, QPE, and QFT,
and plot the results in \figref{linear}.
%
The Y-axis shows the execution time and the X-axis shows the size of benchmarks.
%
We vary the size of circuits by changing the number of qubits in the corresponding algorithms.

The plots show that the execution time scales linearly
across sizes ranging from a few hundreds of gates to circuits as large as $0.14$ millions
of gates.
%
For example,
in case of the VQE algorithm,
the \coam{} algorithm takes around five seconds for the smallest circuit of size two thousand,
and takes around 300 seconds for the largest circuit of size one hundred thousand ($136000$ precisely).
%
We provide the data points for our plots and the plot for the HHL family in our Supplementary.
%
\label{sec:linear}

We show that the \coam{} algorithm requires only linear time
by measuring its execution time across different sizes of benchmarks.
%
For this experiment, we use $\quartzt{0.06}$ as our oracle
(i.e., each sub circuit is optimized by calling \quartz{} for a duration of $0.06$ seconds).
%
We consider four quantum algorithms: Grover, VQE, QPE, and QFT,
and plot the results in \figref{linear}.
%
The Y-axis shows the execution time and the X-axis shows the size of benchmarks.
%
We vary the size of circuits by changing the number of qubits in the corresponding algorithms.

The plots show that the execution time scales linearly
across sizes ranging from a few hundreds of gates to circuits as large as $0.14$ millions
of gates.
%
For example,
in case of the VQE algorithm,
the \coam{} algorithm takes around five seconds for the smallest circuit of size two thousand,
and takes around 300 seconds for the largest circuit of size one hundred thousand ($136000$ precisely).
%
We provide the data points for our plots and the plot for the HHL family in our Supplementary.
%
\label{sec:linear}

We show that the \coam{} algorithm requires only linear time
by measuring its execution time across different sizes of benchmarks.
%
For this experiment, we use $\quartzt{0.06}$ as our oracle
(i.e., each sub circuit is optimized by calling \quartz{} for a duration of $0.06$ seconds).
%
We consider four quantum algorithms: Grover, VQE, QPE, and QFT,
and plot the results in \figref{linear}.
%
The Y-axis shows the execution time and the X-axis shows the size of benchmarks.
%
We vary the size of circuits by changing the number of qubits in the corresponding algorithms.

The plots show that the execution time scales linearly
across sizes ranging from a few hundreds of gates to circuits as large as $0.14$ millions
of gates.
%
For example,
in case of the VQE algorithm,
the \coam{} algorithm takes around five seconds for the smallest circuit of size two thousand,
and takes around 300 seconds for the largest circuit of size one hundred thousand ($136000$ precisely).
%
We provide the data points for our plots and the plot for the HHL family in our Supplementary.
%