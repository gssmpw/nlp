\subsection{\coam{} scales linearly with the size of circuits}
\begin{figure}
	\centering

	\subfloat[Grover]{\includegraphics[width=0.5\columnwidth]{plots/linearitygrover0.01.png}}
	\subfloat[VQE]{\includegraphics[width=0.5\columnwidth]{plots/linearityvqe0.01.png}}

	\subfloat[VPE]{\includegraphics[width=0.5\columnwidth]{plots/linearityshor0.01.png}}
	\subfloat[QFT]{\includegraphics[width=0.5\columnwidth]{plots/linearityqft0.01.png}}

	\caption{
	\textbf{X-axis: circuit size in thousands.}
	\textbf{Y-axis: time in seconds.}
	The figure plots the optimization times for four quantum algorithms
	against the corresponding circuit sizes.
	The figure confirms that our algorithm scales linearly with the size of the circuit.
	}
	\label{fig:linear}

\end{figure}


% \begin{figure}[t]
% 	\centering
% 	\includegraphics[width=0.9\columnwidth]{plots/linearity0.01.png}\quad
% 	\scriptsize
% \caption{
% 	The figure plots the optimizer times for five quantum algorithms
% 	as their size varies.
% 	%
% 	The X-axis shows the relative size of a circuit for each algorithm.
% 	%
% 	The Y-axis shows the time taken to optimize the circuit.
% 	%
% 	The figure confirms that our algorithm scales linearly with the size of the circuit.
% }
% \end{figure}
\label{sec:linear}

We show that the \coam{} algorithm requires only linear time
by measuring its execution time across different sizes of benchmarks.
%
For this experiment, we use $\quartzt{0.06}$ as our oracle
(i.e., each sub circuit is optimized by calling \quartz{} for a duration of $0.06$ seconds).
%
We consider four quantum algorithms: Grover, VQE, QPE, and QFT,
and plot the results in \figref{linear}.
%
The Y-axis shows the execution time and the X-axis shows the size of benchmarks.
%
We vary the size of circuits by changing the number of qubits in the corresponding algorithms.

The plots show that the execution time scales linearly
across sizes ranging from a few hundreds of gates to circuits as large as $0.14$ millions
of gates.
%
For example,
in case of the VQE algorithm,
the \coam{} algorithm takes around five seconds for the smallest circuit of size two thousand,
and takes around 300 seconds for the largest circuit of size one hundred thousand ($136000$ precisely).
%
We provide the data points for our plots and the plot for the HHL family in our Supplementary.
%