

In this section,
we provide some quantum computing background that is relevant for the paper.

\paragraph{Quantum States, Gates, and Circuits}
The state of a quantum bit (or \emph{qubit}) is represented as a linear superposition,
$\ket{\psi} = \alpha\ket{0} + \beta\ket{1}$, of the single-qubit basis vectors $\ket{0} = [1\;\; 0]^\textnormal{T}$ and $\ket{1} = [0\;\; 1]^T$,
for $\alpha,\beta \in \mathbb{C}$ with normalization constraint $|\alpha|^2 + |\beta|^2 = 1$.
%
A valid transformation from one quantum state to another is described
as a $2\times 2$ complex unitary matrix, $U$,
where $U^\dagger U = I$. An $n$-qubit quantum state is a superposition of $2^n$ basis vectors,
$\ket{\psi} = \sum_{i\in\{0,1\}^n} \alpha_i \ket{i}$,
and its transformation is a $2^n\times 2^n$ unitary matrix.
%

A \emph{quantum circuit} is an ordered sequence of quantum logic gates selected from a predefined gate set.
%
Each \emph{quantum gate} represents a unitary matrix that transforms the state of one, two, or a few qubits.
%
% A set of gates is universal when it can be used to express all unitary transformations.
% %
% For example, some common \emph{universal gate sets} are
% $\{$\lstinline{H},\lstinline{RX},\lstinline{RY},\lstinline{RZ},\lstinline{CNOT}$\}$
% for noisy intermediate-scale quantum (NISQ) architectures
% and $\{$\lstinline{H},\lstinline{X},\lstinline{Y},\lstinline{Z},\lstinline{S},\lstinline{T},\lstinline{CNOT}$\}$
% for fault-tolerant quantum computing (FTQC) architectures.
%
Given a circuit $C$, the \emph{size} ($|C|$) is the total number of gates used,
while the \emph{width} ($n$) represents the number of qubits.
%
The \emph{depth} ($d$) is the number of circuit layers, wherein each qubit participates in at most one gate.
%\yr{Do we need a paragraph on QASM?}
%\jr{Add QASM + layer representation of circuits}

\paragraph{Circuit Representation.}
Quantum circuits can be represented with many data structures such as
graphs, matrices, text, and layer diagrams.
%
In this paper,
we represent circuits with a sequence of layers,
where each layer contains gates that may act at the same time step
on their respective qubits.
%
We use the layer representation to define and prove the circuit quality guaranteed
by our optimization algorithm.
%
In addition to the layer representation,
our implementation uses the QASM (quantum assembly language) representation.
%
The QASM is a standard format
which orders all gates of the circuit in a way that
respects the sequential dependencies between gates.
%
It is supported by almost all quantum computing
frameworks and enables our implementation to interact with off-the-shelf tools.
%

\paragraph{Quantum Circuit Synthesis and Optimization}
The goal of \emph{circuit synthesis} is to decompose the desired unitary transformation
into a sequence of basic gates that are physically realizable within the constraints of the underlying quantum hardware architecture.
%
Quantum circuits for the same unitary transformation can be represented in multiple ways,
and their efficiency can vary when executed on real quantum devices.
%
\emph{Circuit optimization} aims to take a given quantum circuit as input
and produce another quantum circuit that is
logically equivalent but requires fewer resources or shorter execution time,
such as a reduced number of gates or a reduced circuit depth.
%
Synthesizing and optimizing large circuits are known to be challenging due to their high dimensionality.
%
For example, as the number of qubits in a quantum circuit increases, the degree of freedom in the unitary transformation grows exponentially, leading to higher synthesis and optimization complexity. In particular, global optimization of quantum circuits is QMA-hard ~\cite{janzing2003identity}.
%
% Some analytical solutions exist, but they are typically restricted to small circuit width/depth ~\cite{rakyta2022approaching, giles2013remarks},
% or restricted to specific gate sets ~\cite{giles2013remarks}.
% %
% For large circuits, there is currently no general solution.
% %
% Researchers have thus proposed various heuristic-driven approaches~\cite{}.
%  heuristics.

% Given that practical algorithms that provide quantum advantage
% over their best-known classical counterparts
% frequently demand a large number of qubits or deep quantum circuits,
% the optimization of large/deep quantum circuits is crucial for their realization.

