%% The field of compilers and program optimization, which is as nearly
%% old as computer science itself, has evolved largely synchronously with
%% architectural advances.
%% %
%% %% The development of deeply pipelined architectures with large memory
%% %% hierarchies have motivated optimization techniques for instruction
%% %% selection, scheduling, and register allocation.
%% %
%% For example, recent advances on parallel architectures such as multicores,
%% GPUs, and TPUs and have motivated the development of optimization
%% techniques for concurrent executing threads.
%
Recent advances in quantum architectures and computing have motivated
the development of new optimizing compilers for quantum programs or
circuits.
%
Even though steady progress has been made, existing quantum
optimization techniques remain asymptotically and practically
inefficient and are unable to offer guarantees on the quality of the
optimization.
%
Because many global quantum circuit optimization problems belong to
the complexity class QMA (the quantum analog of NP), it is not clear
whether quality and efficiency guarantees can both be achieved.


In this paper, we present optimization techniques for quantum programs
that can offer both efficiency and quality guarantees.
%
Rather than requiring global optimality, our approach relies on a form
of local optimality that requires each and every segment of the
circuit to be optimal.
%
We show that the local optimality notion can be attained by a
cut-and-meld circuit optimization algorithm.
%it into a hierarchy of pieces, optimizing each piece, and
%combining them together.
%
The idea behind the algorithm is to cut a circuit into subcircuits,
optimize each subcircuit independently by using a specified ``oracle''
optimizer, and meld the subcircuits by optimizing across the cuts
lazily as needed.
%
We specify the algorithm and prove that it ensures local optimality.
%
To prove efficiency, we show that, under some assumptions, the main
optimization phase of the algorithm requires a linear number of calls to
the oracle optimizer.
%
We implement and evaluate the local-optimality approach to circuit
optimization and compare with the state-of-the-art optimizers.
%
The empirical results show that our cut-and-meld algorithm can
outperform existing optimizers significantly, by more than an order of
magnitude on average, while also slightly improving optimization
quality.
%
These results show that local optimality can be a relatively strong
optimization criterion and can be attained efficiently.
%

\if0

%% Advances in quantum hardware has led to quantum computers with
%% increasingly more numbers of qubits, growing rapidly from single
%% digits in early 2000s to approaching more than a thousand today.
%% %
A key challenge in utilizing modern and future quantum computers is to
optimize a quantum circuit (program) by removing unnecessary gates.
%
Unfortunately this optimization problem is QMA hard (QMA is the analog
of NP but for quantum computers) and unless the polynomial hierarchy
collapses, it is unlikely that we will have efficient quantum circuit
optimizers that deliver fully optimal circuits.
%
Indeed, state-of-the-art quantum circuit optimizers can take hours or
even days to optimize circuits with a few hundred gates.

In this paper, we propose a notion of local optimality, called
$\Omega$-optimality that requires all $\Omega$-deep contiguous
segments of the circuit to be optimal.
%
For example, a circuit is 16-optimal if any subcircuit of depth at~16
is optimal.
%
%% Because optimizing small circuits could still be expensive, we also
%% introduce a variant of $\Omega$-optimality that is relative to a given
%% ``oracle'' optimizer.
%% %
%% This relaxed, relative variant deems a circuit to be $\Omega$-optimal
%% if the oracle cannot optimize any subcircuit of depth at most
%% $\Omega$.
%
We present an efficient local-optimization algorithm that takes as
input 1) an oracle optimizer, which works well for ``small'' circuits
(e.g., tens of gates), and 2) an input circuit to optimize.
%
The algorithm partitions the circuit into smaller subcircuits,
recursively optimizes each subcircuit and uses the base optimizer for
the smallest subcircuits.  The algorithm then joins the optimized
subcircuits by using a ``meld'' operation.
%
As it joins two circuits, the meld operation optimizes along ``the
seam'' by propagating optimizations from one partition to others
until it achieves convergence, i.e., no further
optimizations are possible.
%
We prove that our algorithm guarantees $\Omega$-optimality for an
important class of cost functions
%
and is efficient: it makes at most $O(|C|)$ calls to the oracle, where
$|C|$ is the size of the input circuit.
%
This means that if the oracle takes constant time for $\Omega$-deep
circuits for any fixed $\Omega$, then the algorithm requires linear
time to locally optimize the number of gates in the circuit.
%
We show that the algorithm is practical by implementing it and
applying it to three different optimizers chosen from previous work.
%
Our experiments with a variety of quantum circuits confirm the
bounds obtained by our asymptotic analysis
%
and
%
show that local optimality is a good quality criterion that enables
efficient circuit optimization.
%
\fi

%% %%%%%%%%%%%%%%%%%%%%%%%%%%%%%%%%%%%%%%%%%%%%%%%%%%%%%%%%%%%%%%%%%%%%%%
%% %% OLD ABSTRACT


%% The goal of quantum circuit optimizers is to
%% improve a circuit with
%% respect to some criteria, e.g., the number of all or some selected
%% gates, e.g., ``CNOT'' gates and ``T'' gates.
%% %
%% Existing optimizers operate by iteratively optimizing
%% subcircuits of the input circuit, typically by using a search
%% algorithm that traverses the space of optimizations.
%% %
%% Because this optimization space is typically large, existing optimizers
%% do not scale beyond relatively small circuits, e.g., with a few
%% thousand gates.
%% %

%% In this paper, we propose a (worst-case) linear-time quantum
%% optimization algorithm.
%% %
%% Our algorithm employs an ``oracle'' optimizer that works on ``small''
%% circuits (e.g., tens of gates) and takes as input the circuit to be optimized.
%% %
%% The algorithm recursively partitions the circuit into smaller parts,
%% optimizes each part, by using the oracle for the smallest parts, and
%% melds the optimized parts.
%% %
%% The meld operation ensures good quality by
%% optimizing the seam of the partitions and
%% propagating optimizations in one partition to other partitions.
%% %
%% We prove two key properties for our algorithm:
%% %
%% 1) it runs in worst-case $O(|C|)$ time, where $|C|$ is the size of the
%% input circuit, and thus scales to large circuits, and
%% %
%% 2) it guarantees \emph{local optimality}, i.e.,
%% it ensures that the output circuit cannot be further optimized by applying the oracle to any
%% subcircuit of the input circuit.
%% %
%% %
%% %(For the worst-case bound, we assume that the oracle runs in constant
%% %time in constant-size circuits.)
%% %
%% We implement our algorithm and apply it to a number of large circuits.
%% %
%% Our experiments confirm that the algorithm scales linearly in the
%% size of the circuit and can successfully optimize circuits with many
%% thousands of gates.
%% %
%% It is not difficult to design quantum-circuit optimization algorithms
%% that ensures one of fast running times (e.g., linear),
%% or local optimality,
%% but we do not know of any prior work that achieves both at
%% the same time, as our approach does.

%% %%
%% %%%%%%%%%%%%%%%%%%%%%%%%%%%%%%%%%%%%%%%%%%%%%%%%%%%%%%%%%%%%%%%%%%%%%%
