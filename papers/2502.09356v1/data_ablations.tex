\definecolor{customgray}{gray}{0.5}
\begin{table}[!th]\centering\footnotesize
    \caption{Ablating the Galileo dataset. MADOS and Sen1Floods11 (\% mIoU) via linear probing. CropHarvest and EuroSat (\% OA) via $k$NN.}\label{tab:data_ablation}
    \setlength{\tabcolsep}{2pt}
    \begin{tabular}{
        l%S[table-format=2.2]
        r%S[table-format=2.2]%1
        r%S[table-format=2.2]%2
        r%S[table-format=2.2]%3
        r%S[table-format=2.2]%4
    }
    \toprule
        \parbox{1.7cm}{\raggedright{Removed\\input}} &
        \parbox{1.7cm}{\raggedleft{MADOS}} & \parbox{1.7cm}{\raggedleft{Sen1Floods11}} & \parbox{1.7cm}{\raggedleft{CropHarvest}} & \parbox{1.7cm}{\raggedleft{EuroSat}} \\
        \midrule
        \rowcolor{lightour} None & 67.79 & 77.66 & 87.87 & 91.00 \\
        S1 & 67.67 & \textcolor{customgray}{N/A} & 85.27 & 90.20 \\
        NDVI & 67.89 & 78.10 & 88.32 & 90.00 \\
        ERA5 & 68.10 & 77.10 & 87.14 & 91.20 \\
        TerraClim & 61.30 & 74.90 & 82.78 & 81.20 \\
        VIIRS & 63.48 & 74.52 & 84.10 & 81.10 \\
        SRTM & 66.14 & 77.62 & 86.74 & 91.00 \\
        DynamicWorld & 67.24 & 77.86 & 87.80 & 89.30 \\
        WorldCereal & 65.94 & 77.56 & 87.71 & 89.60 \\
        LandScan & 60.74 & 77.45 & 87.89 & 91.10 \\
    \end{tabular}
\end{table}
