\begin{table}[!t]\centering\footnotesize
    \newcommand{\BEST}[1]{\textbf{#1}}
    \newcommand{\SECOND}[1]{\underline{#1}}
    \newcommand{\COLUMNvalues}[2]{
        \def\COLUMNmin{#1}
        \def\COLUMNmax{#2}
        % Reset tracking for this column
        \global\def\COLUMNbest{0}
        \global\def\COLUMNsecond{0}
        \global\def\COLUMNbestrow{0}
        \global\def\COLUMNsecondrow{0}
    }
    \newcommand{\CELLvalue}[3]{
        \ifstrequal{#3}{}
            {
                \csxdef{C#1R#2}{--}
            }{
            % Store the raw value first
            \csxdef{C#1R#2}{#3}
            
            % Update best/second best
            \ifdim #3pt > \COLUMNbest pt
                % Current value is new best
                \xdef\COLUMNsecond{\COLUMNbest}
                \xdef\COLUMNsecondrow{\COLUMNbestrow}
                \xdef\COLUMNbest{#3}
                \xdef\COLUMNbestrow{#2}
            \else
                \ifdim #3pt > \COLUMNsecond pt
                    % Current value is new second best
                    \xdef\COLUMNsecond{#3}
                    \xdef\COLUMNsecondrow{#2}
                \fi
            \fi
            
            % If this is the last row in the column, format all values
            \ifnum#2=5
                \foreach \i in {1,2,3,4,5} {
                    \ifnum\i=\COLUMNbestrow
                        \csxdef{C#1R\i}{\noexpand\BEST{\csuse{C#1R\i}}}
                    \fi
                    \ifnum\i=\COLUMNsecondrow
                        \csxdef{C#1R\i}{\noexpand\SECOND{\csuse{C#1R\i}}}
                    \fi
                }
            \fi
            }
    }
    % Column 1
        \COLUMNvalues{73.4}{75.5}
        \CELLvalue{1}{1}{75.5}
        \CELLvalue{1}{2}{73.4}
        \CELLvalue{1}{3}{73.5}
        \CELLvalue{1}{4}{74.7}
        \CELLvalue{1}{5}{74.8}
        
    % Column 2
        \COLUMNvalues{76.4}{99.3}
        \CELLvalue{2}{1}{98.8}
        \CELLvalue{2}{2}{76.7}
        \CELLvalue{2}{3}{76.4}
        \CELLvalue{2}{4}{97.2}
        \CELLvalue{2}{5}{99.3}

    % Column 3
        \COLUMNvalues{75.5}{85.4}
        \CELLvalue{3}{1}{84.0}
        \CELLvalue{3}{2}{75.5}
        \CELLvalue{3}{3}{84.5}
        \CELLvalue{3}{4}{85.4}
        \CELLvalue{3}{5}{84.2}

    % Column 4
        \COLUMNvalues{16.1}{27.9}
        \CELLvalue{4}{1}{63.0}
        \CELLvalue{4}{2}{66.1}
        \CELLvalue{4}{3}{67.3}
        \CELLvalue{4}{4}{69.0}
        \CELLvalue{4}{5}{73.0}

    \newcommand{\CELLvaluedsaddasdsadasdad}[3]{%
        %\cellcolor{COLMAX!0!COLMED!100!COLMIN}#1
        \FPeval\MIDvalue{(#1+#2)/2}%
        \FPeval\NORMvalue{(#3-#1)*((100)/(#2-#1))}%
        #3
    }
    %
    \caption{Pixel timeseries classification performance via linear probing. The best result is \textbf{bolded} and the second best is \underline{underlined}. The CropHarvest dataset contains a number of modalities in addition to Sentinel-2 optical imagery, including topography, weather and SAR data. We use all modalities each model can support.}
    \label{tab:ts_results}
    \setlength{\tabcolsep}{2pt}
    \resizebox{\linewidth}{!}{\begin{tabular}{
        l
        l
        c%S[table-format=2.2]%1
        c%S[table-format=2.2]%2
        c%S[table-format=2.2]%3
        c%S[table-format=2.2]%4
    }
         & & \multicolumn{3}{c}{CropHarvest} & \\
        \cmidrule(lr){3-5}
        Method & Arch. & \multicolumn{1}{>{\centering\arraybackslash}p{40pt}}{Togo} & \multicolumn{1}{>{\centering\arraybackslash}p{40pt}}{Brazil} & \multicolumn{1}{>{\centering\arraybackslash}p{40pt}}{Kenya} & \multicolumn{1}{>{\centering\arraybackslash}p{40pt}}{Breizhcrops} \\
        \toprule
        Presto & ViT-Presto &  \csuse{C1R1} & \csuse{C2R1} & \csuse{C3R1} & \csuse{C4R1} \\
        AnySat & ViT-Base &  \csuse{C1R2} & \csuse{C2R2} & \csuse{C3R2} & \csuse{C4R2} \\
        \color{ourcolor}\textbf{Galileo} & ViT-Nano & \csuse{C1R3} & \csuse{C2R3} & \csuse{C3R3} & \csuse{C4R3} \\
        \color{ourcolor}\textbf{Galileo} & ViT-Tiny &  \csuse{C1R4} & \csuse{C2R4} & \csuse{C3R4} & \csuse{C4R4} \\
        \color{ourcolor}\textbf{Galileo} & ViT-Base &  \csuse{C1R5} & \csuse{C2R5} & \csuse{C3R5} & \csuse{C4R5} \\
    \end{tabular}}
\end{table}
