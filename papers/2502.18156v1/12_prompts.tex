\section{Prompts for generating and evaluating \SCEs}
\label{app:prompts}

Recall from Section~\ref{sec:ce_gen} that we can generate \SCEs through: (i) {\bf unconstrained prompting} where we simply ask the model to generate counterfactuals, or (ii) {\bf rationale-based prompting} by asking the model to first select decision rationales~\cite{deyoung2019eraser} and then generating counterfactuals by limiting the changes to these rationales only. For each dataset, we show prompts separately for each prompt type.

\subsection{DiscrimEval}
\subsubsection{Unconstrained prompting}
\begin{enumerate}
    \item You will be given a decision making scenario followed by a question about the scenario. Answer the question with `Yes' or ‘No’. Do not include any additional words in your answer. Your answer should start with `ANSWER:'.\\
    The scenario is: \{SCENARIO\}\\
    The question is: \{QUESTION\}
    \item Now revise the original scenario such that your answer to the question on the revised scenario becomes \textcolor{gray}{\texttt{<Complement>}}. Share the revised scenario. The revised scenario should start with `REVISED SCENARIO:'.
    \end{enumerate}

\subsubsection{Rationale-based prompting}
\begin{enumerate}
    \item You will be given a decision making scenario followed by a question about the scenario. Answer the question with `Yes' or ‘No’. Do not include any additional words in your answer. Your answer should start with `ANSWER:'.\\
    The scenario is: \{SCENARIO\}\\
    The question is: \{QUESTION\}
    \item Now, identify the `rationales' behind your answer. The rationales are words, phrases or sentences in the original scenario that led you to answer with \textcolor{gray}{\texttt{<Original Answer>}}. Share a list of rationales with one rationale per line.
    \item Alter the rationales in the original decision making scenario so that your answer on the altered scenario becomes  \textcolor{gray}{\texttt{<Complement>}}. Keep the changes to a minimum. The altered scenario should start with `ALTERED SCENARIO:'.
    \end{enumerate}




\subsection{FolkTexts prompts}    

We adapt the prompts from \citet{cruz2024evaluating}.

\subsubsection{Unconstrained prompting}
\begin{enumerate}
    \item You will be provided data corresponding to a survey respondent. The survey was conducted among US residents in 2018. Please answer the question based on the information provided by selecting from one of the two choices. The data provided is enough to reach an approximate answer.
     Do not include any additional words. Your answer must start with `ANSWER:'.
    
    The respondent data is: \{description\}\\
    The question is: \{question\}\\
    The choices are: \{choices\}
    
    \item Now revise the original respondent data such that your answer to the question on the revised respondent data becomes \textcolor{gray}{\texttt{<Complement>}}. Share the revised data. The revised data should start with `REVISED  DATA:'
\end{enumerate}
\subsubsection{Rationale-based prompting}
\begin{enumerate}
    \item You will be provided data corresponding to a survey respondent. The survey was conducted among US residents in 2018. Please answer the question based on the information provided by selecting from one of the two choices. The data provided is enough to reach an approximate answer.
     Do not include any additional words. Your answer must start with `ANSWER:'.
    
    The respondent data is: \{description\}\\
    The question is: \{question\}\\
    The choices are: \{choices\}
     \item Now, identify the `rationales' behind your answer. The rationales are words, phrases or sentences in the original respondent data that led you to answer with \textcolor{gray}{\texttt{<Original Answer>}}. Share a list of rationales with one rationale per line. The list should start with `RATIONALS:'
    \item Alter the rationales in the original data so that your answer on the altered data becomes \textcolor{gray}{\texttt{<Complement>}}. Keep the changes to a minimum. The altered data should start with `ALTERED DATA:'
\end{enumerate}

\subsection{SST2}
\subsubsection{Unconstrained prompting}
\begin{itemize}
    \item You will be given a movie review. Assess its sentiment and classify it as `Positive' or `Negative.' Do not include any additional words in your answer. Your answer should start with `ANSWER:'
    \item Now revise the original review so that the sentiment of the revised review becomes \textcolor{gray}{\texttt{<Complement>}}. Share the revised review. The revised review should start with `REVISED REVIEW:'
\end{itemize}
\subsubsection{Rationale-based prompting}
\begin{itemize}
    \item You will be given a movie review. Assess its sentiment and classify it as `Positive' or `Negative.' Do not include any additional words in your answer. Your answer should start with `ANSWER:'
    \item Now, identify the `rationales' behind your answer. The rationales are words, phrases or sentences in the original review that led you to answer with \textcolor{gray}{\texttt{<Original Answer>}}. Share a list of rationales with one rationale per line. The list should start with `RATIONALS:'
    \item Alter the rationales in the original review so that your answer on the altered review becomes \textcolor{gray}{\texttt{<Complement>}}. Keep the changes to a minimum. The altered review should start with `ALTERED REVIEW:'
\end{itemize}




\subsection{Twitter Financial News}
\subsubsection{Unconstrained prompting}
\begin{enumerate}
    \item You will be given a finance-related news post from X (formerly Twitter) followed by a question about its sentiment. Respond with `Bearish,' `Bullish,' or `Neutral.' Do not include any additional words in your answer. Your answer should start with `ANSWER:'.\\
    The twitter post is: \{twitter\}\\
    The question is: \{question\}
    \item Now revise the original post so that the sentiment of the revised post becomes \textcolor{gray}{\texttt{<Complement>}}. Share the revised post. The revised post should start with `REVISED POST:'.
\end{enumerate}
\subsubsection{Rationale-based prompting}
\begin{enumerate}
    \item You will be given a finance-related news post from X (formerly Twitter). Assess its sentiment and classify it as `Bearish,' `Bullish,' or `Neutral.' Do not include any additional words in your answer. Your answer should start with `ANSWER:'
    \item Now, identify the `rationales' behind your answer. The rationales are words, phrases or sentences in the original Twitter post that led you to answer with \textcolor{gray}{\texttt{<Original Answer>}}. Share a list of rationales with one rationale per line. The list should start with `RATIONALS:'
    \item Alter the rationales in the original Twitter post so that your answer on the altered Twitter post becomes \textcolor{gray}{\texttt{<Complement>}}. Keep the changes to a minimum. The altered Twitter post should start with `ALTERED TWITTER POST:'
\end{enumerate}



\subsection{GSM8K}
\subsubsection{Unconstrained prompting}
\begin{enumerate}
    \item You will be given a math problem. The solution to the problem is an integer. Your task is to provide the solution. Only provide the final answer as an integer. Do not include any additional word or phrase. You final answer should start with `FINAL ANSWER:'
    \item Now, revise the math problem so your final answer to the revised problem becomes \textcolor{gray}{\texttt{<Complement>}}. Share the revised Problem. The revised problem should start with `REVISED PROBLEM:'
\end{enumerate}

\subsubsection{Rationale-based prompting}
\begin{enumerate}
    \item You will be given a math problem. The solution to the problem is an integer. Your task is to provide the solution. Only provide the final answer as an integer. Do not include any additional word or phrase. You final answer should start with `FINAL ANSWER:'
    \item Now, identify the `rationales' behind your answer. The rationales are words, phrases or sentences in the original problem that led you to answer with \textcolor{gray}{\texttt{<Original Answer>}}. Share a list of rationales with one rationale per line. The list should start with `RATIONALS:'
    \item Alter the rationales in the original problem so that your answer on the altered probelm becomes \textcolor{gray}{\texttt{<Complement>}}. Keep the changes to a minimum. The altered problem should start with `ALTERED PROBLEM:'.
\end{enumerate}


\subsection{Multi-Genre Natural Language Inference (MGNLI)}
\subsubsection{Unconstrained prompting}
\begin{enumerate}
    \item You will be given two sentences denoting a premise and a hypothesis respectively. Determine the relationship between the premise and the hypothesis. The possible relationships you can choose from are `Entail', `Contradict' and `Neutral'. Only pick one of the options. Do not include any additional words in your answer. Your answer should start with `ANSWER:'\\
    The premise is: \{premise\}\\
    The hypothesis is: \{hypothesis\}\\
    \item Now revise the original hypothesis so that your answer to the question about its relationship becomes \textcolor{gray}{\texttt{<Complement>}}. Share the revised hypothesis. The revised hypothesis should start with `REVISED HYPOTHESIS:'
\end{enumerate}
\subsubsection{Rationale-based prompting}
\begin{enumerate}
    \item You will be given two sentences denoting a premise and a hypothesis respectively. Determine the relationship between the premise and the hypothesis. The possible relationships you can choose from are `Entail', `Contradict' and `Neutral'. Only pick one of the options. Do not include any additional words in your answer. Your answer should start with `ANSWER:'\\
    The premise is: \{premise\}\\
    The hypothesis is: \{hypothesis\}\\
    \item Now, identify the `rationales' behind your answer. The rationales are words, phrases or sentences in the original hypothesis that led you to answer with \textcolor{gray}{\texttt{<Original Answer>}}. Share a list of rationales with one rationale per line. The list should start with `RATIONALS:'
    \item Alter the rationales in the original hypothesis so that your answer on the altered hypothesis becomes \textcolor{gray}{\texttt{<Complement>}}. Keep the changes to a minimum. The altered hypothesis should start with `ALTERED HYPOTHESIS:'.
    \end{enumerate}


