\section{Postprocessing model outputs} \label{app:post}
\begin{enumerate}
    \item Post-processing for all datasets starts by normalizing the model's short answer, such as \texttt{Yes} or \texttt{No} in DiscrimEval. We also remove common extra characters that models tend to add to their answers, such as (*, \textbackslash, ', ., !, ?).
    
    \item Filtering and removing model generations where the model's first answer is not valid. This means the model did not pick one of the valid options as an answer.
    
    \item Filtering and removing generations that are shorter than expected. This happens when the model does not provide the complete CEs or gives a non-response. Generally, we remove cases where the model is unable to follow the given instructions.
    
    \item Filtering and removing cases where the model is unable to generate rationales. If the model fails to detect the important part of the text for answering, we do not consider its CEs generation.
    
    \item For the mathematics dataset, since we do not provide predefined options for the model's answer, it can generate any number as the final answer to the math problem. Therefore, we consider CEs valid if the model's answer is mathematical with the expected answer: \textcolor{gray}{\texttt{<Complement>}}.
\end{enumerate}
