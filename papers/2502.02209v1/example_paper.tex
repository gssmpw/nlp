%%%%%%%% ICML 2025 EXAMPLE LATEX SUBMISSION FILE %%%%%%%%%%%%%%%%%

\documentclass{article}
%%%%% NEW MATH DEFINITIONS %%%%%

\usepackage{amsmath,amsfonts,bm}
\usepackage{derivative}
% Mark sections of captions for referring to divisions of figures
\newcommand{\figleft}{{\em (Left)}}
\newcommand{\figcenter}{{\em (Center)}}
\newcommand{\figright}{{\em (Right)}}
\newcommand{\figtop}{{\em (Top)}}
\newcommand{\figbottom}{{\em (Bottom)}}
\newcommand{\captiona}{{\em (a)}}
\newcommand{\captionb}{{\em (b)}}
\newcommand{\captionc}{{\em (c)}}
\newcommand{\captiond}{{\em (d)}}

% Highlight a newly defined term
\newcommand{\newterm}[1]{{\bf #1}}

% Derivative d 
\newcommand{\deriv}{{\mathrm{d}}}

% Figure reference, lower-case.
\def\figref#1{figure~\ref{#1}}
% Figure reference, capital. For start of sentence
\def\Figref#1{Figure~\ref{#1}}
\def\twofigref#1#2{figures \ref{#1} and \ref{#2}}
\def\quadfigref#1#2#3#4{figures \ref{#1}, \ref{#2}, \ref{#3} and \ref{#4}}
% Section reference, lower-case.
\def\secref#1{section~\ref{#1}}
% Section reference, capital.
\def\Secref#1{Section~\ref{#1}}
% Reference to two sections.
\def\twosecrefs#1#2{sections \ref{#1} and \ref{#2}}
% Reference to three sections.
\def\secrefs#1#2#3{sections \ref{#1}, \ref{#2} and \ref{#3}}
% Reference to an equation, lower-case.
\def\eqref#1{equation~\ref{#1}}
% Reference to an equation, upper case
\def\Eqref#1{Equation~\ref{#1}}
% A raw reference to an equation---avoid using if possible
\def\plaineqref#1{\ref{#1}}
% Reference to a chapter, lower-case.
\def\chapref#1{chapter~\ref{#1}}
% Reference to an equation, upper case.
\def\Chapref#1{Chapter~\ref{#1}}
% Reference to a range of chapters
\def\rangechapref#1#2{chapters\ref{#1}--\ref{#2}}
% Reference to an algorithm, lower-case.
\def\algref#1{algorithm~\ref{#1}}
% Reference to an algorithm, upper case.
\def\Algref#1{Algorithm~\ref{#1}}
\def\twoalgref#1#2{algorithms \ref{#1} and \ref{#2}}
\def\Twoalgref#1#2{Algorithms \ref{#1} and \ref{#2}}
% Reference to a part, lower case
\def\partref#1{part~\ref{#1}}
% Reference to a part, upper case
\def\Partref#1{Part~\ref{#1}}
\def\twopartref#1#2{parts \ref{#1} and \ref{#2}}

\def\ceil#1{\lceil #1 \rceil}
\def\floor#1{\lfloor #1 \rfloor}
\def\1{\bm{1}}
\newcommand{\train}{\mathcal{D}}
\newcommand{\valid}{\mathcal{D_{\mathrm{valid}}}}
\newcommand{\test}{\mathcal{D_{\mathrm{test}}}}

\def\eps{{\epsilon}}


% Random variables
\def\reta{{\textnormal{$\eta$}}}
\def\ra{{\textnormal{a}}}
\def\rb{{\textnormal{b}}}
\def\rc{{\textnormal{c}}}
\def\rd{{\textnormal{d}}}
\def\re{{\textnormal{e}}}
\def\rf{{\textnormal{f}}}
\def\rg{{\textnormal{g}}}
\def\rh{{\textnormal{h}}}
\def\ri{{\textnormal{i}}}
\def\rj{{\textnormal{j}}}
\def\rk{{\textnormal{k}}}
\def\rl{{\textnormal{l}}}
% rm is already a command, just don't name any random variables m
\def\rn{{\textnormal{n}}}
\def\ro{{\textnormal{o}}}
\def\rp{{\textnormal{p}}}
\def\rq{{\textnormal{q}}}
\def\rr{{\textnormal{r}}}
\def\rs{{\textnormal{s}}}
\def\rt{{\textnormal{t}}}
\def\ru{{\textnormal{u}}}
\def\rv{{\textnormal{v}}}
\def\rw{{\textnormal{w}}}
\def\rx{{\textnormal{x}}}
\def\ry{{\textnormal{y}}}
\def\rz{{\textnormal{z}}}

% Random vectors
\def\rvepsilon{{\mathbf{\epsilon}}}
\def\rvphi{{\mathbf{\phi}}}
\def\rvtheta{{\mathbf{\theta}}}
\def\rva{{\mathbf{a}}}
\def\rvb{{\mathbf{b}}}
\def\rvc{{\mathbf{c}}}
\def\rvd{{\mathbf{d}}}
\def\rve{{\mathbf{e}}}
\def\rvf{{\mathbf{f}}}
\def\rvg{{\mathbf{g}}}
\def\rvh{{\mathbf{h}}}
\def\rvu{{\mathbf{i}}}
\def\rvj{{\mathbf{j}}}
\def\rvk{{\mathbf{k}}}
\def\rvl{{\mathbf{l}}}
\def\rvm{{\mathbf{m}}}
\def\rvn{{\mathbf{n}}}
\def\rvo{{\mathbf{o}}}
\def\rvp{{\mathbf{p}}}
\def\rvq{{\mathbf{q}}}
\def\rvr{{\mathbf{r}}}
\def\rvs{{\mathbf{s}}}
\def\rvt{{\mathbf{t}}}
\def\rvu{{\mathbf{u}}}
\def\rvv{{\mathbf{v}}}
\def\rvw{{\mathbf{w}}}
\def\rvx{{\mathbf{x}}}
\def\rvy{{\mathbf{y}}}
\def\rvz{{\mathbf{z}}}

% Elements of random vectors
\def\erva{{\textnormal{a}}}
\def\ervb{{\textnormal{b}}}
\def\ervc{{\textnormal{c}}}
\def\ervd{{\textnormal{d}}}
\def\erve{{\textnormal{e}}}
\def\ervf{{\textnormal{f}}}
\def\ervg{{\textnormal{g}}}
\def\ervh{{\textnormal{h}}}
\def\ervi{{\textnormal{i}}}
\def\ervj{{\textnormal{j}}}
\def\ervk{{\textnormal{k}}}
\def\ervl{{\textnormal{l}}}
\def\ervm{{\textnormal{m}}}
\def\ervn{{\textnormal{n}}}
\def\ervo{{\textnormal{o}}}
\def\ervp{{\textnormal{p}}}
\def\ervq{{\textnormal{q}}}
\def\ervr{{\textnormal{r}}}
\def\ervs{{\textnormal{s}}}
\def\ervt{{\textnormal{t}}}
\def\ervu{{\textnormal{u}}}
\def\ervv{{\textnormal{v}}}
\def\ervw{{\textnormal{w}}}
\def\ervx{{\textnormal{x}}}
\def\ervy{{\textnormal{y}}}
\def\ervz{{\textnormal{z}}}

% Random matrices
\def\rmA{{\mathbf{A}}}
\def\rmB{{\mathbf{B}}}
\def\rmC{{\mathbf{C}}}
\def\rmD{{\mathbf{D}}}
\def\rmE{{\mathbf{E}}}
\def\rmF{{\mathbf{F}}}
\def\rmG{{\mathbf{G}}}
\def\rmH{{\mathbf{H}}}
\def\rmI{{\mathbf{I}}}
\def\rmJ{{\mathbf{J}}}
\def\rmK{{\mathbf{K}}}
\def\rmL{{\mathbf{L}}}
\def\rmM{{\mathbf{M}}}
\def\rmN{{\mathbf{N}}}
\def\rmO{{\mathbf{O}}}
\def\rmP{{\mathbf{P}}}
\def\rmQ{{\mathbf{Q}}}
\def\rmR{{\mathbf{R}}}
\def\rmS{{\mathbf{S}}}
\def\rmT{{\mathbf{T}}}
\def\rmU{{\mathbf{U}}}
\def\rmV{{\mathbf{V}}}
\def\rmW{{\mathbf{W}}}
\def\rmX{{\mathbf{X}}}
\def\rmY{{\mathbf{Y}}}
\def\rmZ{{\mathbf{Z}}}

% Elements of random matrices
\def\ermA{{\textnormal{A}}}
\def\ermB{{\textnormal{B}}}
\def\ermC{{\textnormal{C}}}
\def\ermD{{\textnormal{D}}}
\def\ermE{{\textnormal{E}}}
\def\ermF{{\textnormal{F}}}
\def\ermG{{\textnormal{G}}}
\def\ermH{{\textnormal{H}}}
\def\ermI{{\textnormal{I}}}
\def\ermJ{{\textnormal{J}}}
\def\ermK{{\textnormal{K}}}
\def\ermL{{\textnormal{L}}}
\def\ermM{{\textnormal{M}}}
\def\ermN{{\textnormal{N}}}
\def\ermO{{\textnormal{O}}}
\def\ermP{{\textnormal{P}}}
\def\ermQ{{\textnormal{Q}}}
\def\ermR{{\textnormal{R}}}
\def\ermS{{\textnormal{S}}}
\def\ermT{{\textnormal{T}}}
\def\ermU{{\textnormal{U}}}
\def\ermV{{\textnormal{V}}}
\def\ermW{{\textnormal{W}}}
\def\ermX{{\textnormal{X}}}
\def\ermY{{\textnormal{Y}}}
\def\ermZ{{\textnormal{Z}}}

% Vectors
\def\vzero{{\bm{0}}}
\def\vone{{\bm{1}}}
\def\vmu{{\bm{\mu}}}
\def\vtheta{{\bm{\theta}}}
\def\vphi{{\bm{\phi}}}
\def\va{{\bm{a}}}
\def\vb{{\bm{b}}}
\def\vc{{\bm{c}}}
\def\vd{{\bm{d}}}
\def\ve{{\bm{e}}}
\def\vf{{\bm{f}}}
\def\vg{{\bm{g}}}
\def\vh{{\bm{h}}}
\def\vi{{\bm{i}}}
\def\vj{{\bm{j}}}
\def\vk{{\bm{k}}}
\def\vl{{\bm{l}}}
\def\vm{{\bm{m}}}
\def\vn{{\bm{n}}}
\def\vo{{\bm{o}}}
\def\vp{{\bm{p}}}
\def\vq{{\bm{q}}}
\def\vr{{\bm{r}}}
\def\vs{{\bm{s}}}
\def\vt{{\bm{t}}}
\def\vu{{\bm{u}}}
\def\vv{{\bm{v}}}
\def\vw{{\bm{w}}}
\def\vx{{\bm{x}}}
\def\vy{{\bm{y}}}
\def\vz{{\bm{z}}}

% Elements of vectors
\def\evalpha{{\alpha}}
\def\evbeta{{\beta}}
\def\evepsilon{{\epsilon}}
\def\evlambda{{\lambda}}
\def\evomega{{\omega}}
\def\evmu{{\mu}}
\def\evpsi{{\psi}}
\def\evsigma{{\sigma}}
\def\evtheta{{\theta}}
\def\eva{{a}}
\def\evb{{b}}
\def\evc{{c}}
\def\evd{{d}}
\def\eve{{e}}
\def\evf{{f}}
\def\evg{{g}}
\def\evh{{h}}
\def\evi{{i}}
\def\evj{{j}}
\def\evk{{k}}
\def\evl{{l}}
\def\evm{{m}}
\def\evn{{n}}
\def\evo{{o}}
\def\evp{{p}}
\def\evq{{q}}
\def\evr{{r}}
\def\evs{{s}}
\def\evt{{t}}
\def\evu{{u}}
\def\evv{{v}}
\def\evw{{w}}
\def\evx{{x}}
\def\evy{{y}}
\def\evz{{z}}

% Matrix
\def\mA{{\bm{A}}}
\def\mB{{\bm{B}}}
\def\mC{{\bm{C}}}
\def\mD{{\bm{D}}}
\def\mE{{\bm{E}}}
\def\mF{{\bm{F}}}
\def\mG{{\bm{G}}}
\def\mH{{\bm{H}}}
\def\mI{{\bm{I}}}
\def\mJ{{\bm{J}}}
\def\mK{{\bm{K}}}
\def\mL{{\bm{L}}}
\def\mM{{\bm{M}}}
\def\mN{{\bm{N}}}
\def\mO{{\bm{O}}}
\def\mP{{\bm{P}}}
\def\mQ{{\bm{Q}}}
\def\mR{{\bm{R}}}
\def\mS{{\bm{S}}}
\def\mT{{\bm{T}}}
\def\mU{{\bm{U}}}
\def\mV{{\bm{V}}}
\def\mW{{\bm{W}}}
\def\mX{{\bm{X}}}
\def\mY{{\bm{Y}}}
\def\mZ{{\bm{Z}}}
\def\mBeta{{\bm{\beta}}}
\def\mPhi{{\bm{\Phi}}}
\def\mLambda{{\bm{\Lambda}}}
\def\mSigma{{\bm{\Sigma}}}

% Tensor
\DeclareMathAlphabet{\mathsfit}{\encodingdefault}{\sfdefault}{m}{sl}
\SetMathAlphabet{\mathsfit}{bold}{\encodingdefault}{\sfdefault}{bx}{n}
\newcommand{\tens}[1]{\bm{\mathsfit{#1}}}
\def\tA{{\tens{A}}}
\def\tB{{\tens{B}}}
\def\tC{{\tens{C}}}
\def\tD{{\tens{D}}}
\def\tE{{\tens{E}}}
\def\tF{{\tens{F}}}
\def\tG{{\tens{G}}}
\def\tH{{\tens{H}}}
\def\tI{{\tens{I}}}
\def\tJ{{\tens{J}}}
\def\tK{{\tens{K}}}
\def\tL{{\tens{L}}}
\def\tM{{\tens{M}}}
\def\tN{{\tens{N}}}
\def\tO{{\tens{O}}}
\def\tP{{\tens{P}}}
\def\tQ{{\tens{Q}}}
\def\tR{{\tens{R}}}
\def\tS{{\tens{S}}}
\def\tT{{\tens{T}}}
\def\tU{{\tens{U}}}
\def\tV{{\tens{V}}}
\def\tW{{\tens{W}}}
\def\tX{{\tens{X}}}
\def\tY{{\tens{Y}}}
\def\tZ{{\tens{Z}}}


% Graph
\def\gA{{\mathcal{A}}}
\def\gB{{\mathcal{B}}}
\def\gC{{\mathcal{C}}}
\def\gD{{\mathcal{D}}}
\def\gE{{\mathcal{E}}}
\def\gF{{\mathcal{F}}}
\def\gG{{\mathcal{G}}}
\def\gH{{\mathcal{H}}}
\def\gI{{\mathcal{I}}}
\def\gJ{{\mathcal{J}}}
\def\gK{{\mathcal{K}}}
\def\gL{{\mathcal{L}}}
\def\gM{{\mathcal{M}}}
\def\gN{{\mathcal{N}}}
\def\gO{{\mathcal{O}}}
\def\gP{{\mathcal{P}}}
\def\gQ{{\mathcal{Q}}}
\def\gR{{\mathcal{R}}}
\def\gS{{\mathcal{S}}}
\def\gT{{\mathcal{T}}}
\def\gU{{\mathcal{U}}}
\def\gV{{\mathcal{V}}}
\def\gW{{\mathcal{W}}}
\def\gX{{\mathcal{X}}}
\def\gY{{\mathcal{Y}}}
\def\gZ{{\mathcal{Z}}}

% Sets
\def\sA{{\mathbb{A}}}
\def\sB{{\mathbb{B}}}
\def\sC{{\mathbb{C}}}
\def\sD{{\mathbb{D}}}
% Don't use a set called E, because this would be the same as our symbol
% for expectation.
\def\sF{{\mathbb{F}}}
\def\sG{{\mathbb{G}}}
\def\sH{{\mathbb{H}}}
\def\sI{{\mathbb{I}}}
\def\sJ{{\mathbb{J}}}
\def\sK{{\mathbb{K}}}
\def\sL{{\mathbb{L}}}
\def\sM{{\mathbb{M}}}
\def\sN{{\mathbb{N}}}
\def\sO{{\mathbb{O}}}
\def\sP{{\mathbb{P}}}
\def\sQ{{\mathbb{Q}}}
\def\sR{{\mathbb{R}}}
\def\sS{{\mathbb{S}}}
\def\sT{{\mathbb{T}}}
\def\sU{{\mathbb{U}}}
\def\sV{{\mathbb{V}}}
\def\sW{{\mathbb{W}}}
\def\sX{{\mathbb{X}}}
\def\sY{{\mathbb{Y}}}
\def\sZ{{\mathbb{Z}}}

% Entries of a matrix
\def\emLambda{{\Lambda}}
\def\emA{{A}}
\def\emB{{B}}
\def\emC{{C}}
\def\emD{{D}}
\def\emE{{E}}
\def\emF{{F}}
\def\emG{{G}}
\def\emH{{H}}
\def\emI{{I}}
\def\emJ{{J}}
\def\emK{{K}}
\def\emL{{L}}
\def\emM{{M}}
\def\emN{{N}}
\def\emO{{O}}
\def\emP{{P}}
\def\emQ{{Q}}
\def\emR{{R}}
\def\emS{{S}}
\def\emT{{T}}
\def\emU{{U}}
\def\emV{{V}}
\def\emW{{W}}
\def\emX{{X}}
\def\emY{{Y}}
\def\emZ{{Z}}
\def\emSigma{{\Sigma}}

% entries of a tensor
% Same font as tensor, without \bm wrapper
\newcommand{\etens}[1]{\mathsfit{#1}}
\def\etLambda{{\etens{\Lambda}}}
\def\etA{{\etens{A}}}
\def\etB{{\etens{B}}}
\def\etC{{\etens{C}}}
\def\etD{{\etens{D}}}
\def\etE{{\etens{E}}}
\def\etF{{\etens{F}}}
\def\etG{{\etens{G}}}
\def\etH{{\etens{H}}}
\def\etI{{\etens{I}}}
\def\etJ{{\etens{J}}}
\def\etK{{\etens{K}}}
\def\etL{{\etens{L}}}
\def\etM{{\etens{M}}}
\def\etN{{\etens{N}}}
\def\etO{{\etens{O}}}
\def\etP{{\etens{P}}}
\def\etQ{{\etens{Q}}}
\def\etR{{\etens{R}}}
\def\etS{{\etens{S}}}
\def\etT{{\etens{T}}}
\def\etU{{\etens{U}}}
\def\etV{{\etens{V}}}
\def\etW{{\etens{W}}}
\def\etX{{\etens{X}}}
\def\etY{{\etens{Y}}}
\def\etZ{{\etens{Z}}}

% The true underlying data generating distribution
\newcommand{\pdata}{p_{\rm{data}}}
\newcommand{\ptarget}{p_{\rm{target}}}
\newcommand{\pprior}{p_{\rm{prior}}}
\newcommand{\pbase}{p_{\rm{base}}}
\newcommand{\pref}{p_{\rm{ref}}}

% The empirical distribution defined by the training set
\newcommand{\ptrain}{\hat{p}_{\rm{data}}}
\newcommand{\Ptrain}{\hat{P}_{\rm{data}}}
% The model distribution
\newcommand{\pmodel}{p_{\rm{model}}}
\newcommand{\Pmodel}{P_{\rm{model}}}
\newcommand{\ptildemodel}{\tilde{p}_{\rm{model}}}
% Stochastic autoencoder distributions
\newcommand{\pencode}{p_{\rm{encoder}}}
\newcommand{\pdecode}{p_{\rm{decoder}}}
\newcommand{\precons}{p_{\rm{reconstruct}}}

\newcommand{\laplace}{\mathrm{Laplace}} % Laplace distribution

\newcommand{\E}{\mathbb{E}}
\newcommand{\Ls}{\mathcal{L}}
\newcommand{\R}{\mathbb{R}}
\newcommand{\emp}{\tilde{p}}
\newcommand{\lr}{\alpha}
\newcommand{\reg}{\lambda}
\newcommand{\rect}{\mathrm{rectifier}}
\newcommand{\softmax}{\mathrm{softmax}}
\newcommand{\sigmoid}{\sigma}
\newcommand{\softplus}{\zeta}
\newcommand{\KL}{D_{\mathrm{KL}}}
\newcommand{\Var}{\mathrm{Var}}
\newcommand{\standarderror}{\mathrm{SE}}
\newcommand{\Cov}{\mathrm{Cov}}
% Wolfram Mathworld says $L^2$ is for function spaces and $\ell^2$ is for vectors
% But then they seem to use $L^2$ for vectors throughout the site, and so does
% wikipedia.
\newcommand{\normlzero}{L^0}
\newcommand{\normlone}{L^1}
\newcommand{\normltwo}{L^2}
\newcommand{\normlp}{L^p}
\newcommand{\normmax}{L^\infty}

\newcommand{\parents}{Pa} % See usage in notation.tex. Chosen to match Daphne's book.

\DeclareMathOperator*{\argmax}{arg\,max}
\DeclareMathOperator*{\argmin}{arg\,min}

\DeclareMathOperator{\sign}{sign}
\DeclareMathOperator{\Tr}{Tr}
\let\ab\allowbreak

% Recommended, but optional, packages for figures and better typesetting:
\usepackage{microtype}
\usepackage{graphicx}
\usepackage{subfigure}
\usepackage{booktabs} % for professional tables
% hyperref makes hyperlinks in the resulting PDF.
% If your build breaks (sometimes temporarily if a hyperlink spans a page)
% please comment out the following usepackage line and replace
% \usepackage{icml2025} with \usepackage[nohyperref]{icml2025} above.
\usepackage{hyperref}


% Attempt to make hyperref and algorithmic work together better:
%\newcommand{\theHalgorithm}{\arabic{algorithm}}

\usepackage{algorithm}
\usepackage{algorithmic}
% \usepackage{amsmath, amssymb, mathtools, amsthm}
\usepackage{times}
% Use the following line for the initial blind version submitted for review:
% \usepackage{icml2025}

% If accepted, instead use the following line for the camera-ready submission:
\usepackage[accepted]{icml2025}
%\usepackage{icml2025}
% For theorems and such
\usepackage{amsmath}
\usepackage{amssymb}
\usepackage{mathtools}
\usepackage{amsthm}

% if you use cleveref..
\usepackage[capitalize,noabbrev]{cleveref}

%%%%%%%%%%%%%%%%%%%%%%%%%%%%%%%%
% THEOREMS
%%%%%%%%%%%%%%%%%%%%%%%%%%%%%%%%
% \theoremstyle{plain}
% \newtheorem{theorem}{Theorem}[section]
% \newtheorem{proposition}[theorem]{Proposition}
% \newtheorem{lemma}[theorem]{Lemma}
% \newtheorem{corollary}[theorem]{Corollary}
% \theoremstyle{definition}
% \newtheorem{definition}[theorem]{Definition}
% \newtheorem{assumption}[theorem]{Assumption}
% \theoremstyle{remark}
% \newtheorem{remark}[theorem]{Remark}

% Todonotes is useful during development; simply uncomment the next line
%    and comment out the line below the next line to turn off comments
%\usepackage[disable,textsize=tiny]{todonotes}
\usepackage[textsize=tiny]{todonotes}


% The \icmltitle you define below is probably too long as a header.
% Therefore, a short form for the running title is supplied here:
\icmltitlerunning{On the Expressivity of Selective State-Space Layers}

\begin{document}

\twocolumn[
\icmltitle{On the Expressivity of Selective State-Space Layers:\\ A Multivariate Polynomial Approach}

% It is OKAY to include author information, even for blind
% submissions: the style file will automatically remove it for you
% unless you've provided the [accepted] option to the icml2025
% package.

% List of affiliations: The first argument should be a (short)
% identifier you will use later to specify author affiliations
% Academic affiliations should list Department, University, City, Region, Country
% Industry affiliations should list Company, City, Region, Country

% You can specify symbols, otherwise they are numbered in order.
% Ideally, you should not use this facility. Affiliations will be numbered
% in order of appearance and this is the preferred way.
\icmlsetsymbol{equal}{*}

\begin{icmlauthorlist}
\icmlauthor{Edo Cohen-Karlik}{equal,yyy}
\icmlauthor{Itamar Zimerman}{equal,yyy}
\icmlauthor{Liane Galanti}{equal,yyy}
\icmlauthor{Ido Atad}{yyy}
\icmlauthor{Amir Globerson}{yyy}
\icmlauthor{Lior Wolf}{yyy}
%\icmlauthor{}{sch}
% \icmlauthor{Firstname8 Lastname8}{yyy}
% \icmlauthor{Firstname8 Lastname8}{yyy}
%\icmlauthor{}{sch}
%\icmlauthor{}{sch}
\end{icmlauthorlist}

\icmlaffiliation{yyy}{The School of Computer Science, Tel Aviv University}
%\icmlaffiliation{comp}{Google Research}
% \icmlaffiliation{sch}{School of ZZZ, Institute of WWW, Location, Country}


\icmlcorrespondingauthor{Edo Cohen-Karlik}{edocohen@mail.tau.ac.il}


% You may provide any keywords that you
% find helpful for describing your paper; these are used to populate
% the "keywords" metadata in the PDF but will not be shown in the document
\icmlkeywords{Mamba, S6, Expressivity, Generelization}

\vskip 0.3in
]

% this must go after the closing bracket ] following \twocolumn[ ...

% This command actually creates the footnote in the first column
% listing the affiliations and the copyright notice.
% The command takes one argument, which is text to display at the start of the footnote.
% The \icmlEqualContribution command is standard text for equal contribution.
% Remove it (just {}) if you do not need this facility.

%\printAffiliationsAndNotice{}  % leave blank if no need to mention equal contribution
\printAffiliationsAndNotice{\icmlEqualContribution} % otherwise use the standard text.

\begin{abstract}
Recent advances in efficient sequence modeling have introduced selective state-space layers, a key component of the Mamba architecture, which have demonstrated remarkable success in a wide range of NLP and vision tasks. %These include language modeling, machine translation, long-context question answering, and document-level sentiment analysis, among others.%
While Mamba’s empirical performance has matched or surpassed SoTA transformers on such diverse %NLP
benchmarks, the theoretical foundations underlying its powerful representational capabilities remain less explored. In this work, we investigate the expressivity of selective state-space layers using multivariate polynomials, and prove that they surpass {\color{black}linear} transformers in expressiveness. Consequently, our findings reveal that Mamba offers superior representational power over {\color{black}linear} attention-based models for long sequences{\color{black}, while not sacrificing their generalization.} Our theoretical insights are validated by a comprehensive set of empirical experiments on various datasets.%, and we have included our code as supplementary material.
\vspace{-6pt}
\end{abstract}

\vspace{-6pt}
\section{Introduction}
Sequence modeling has been the focus of many works in recent years, with remarkable results enabling applications such as ChatGPT. To date, these models are largely based on the Transformer architecture~\cite{NIPS2017_3f5ee243}. While transformers have proven extremely effective, they suffer from several drawbacks compared to traditional recurrent models, one of which is their computational complexity which scales quadratically with the input sequence length.

In attempt to mitigate the computational inefficiency of sequence modeling of transformers, a line of work has attempted to resurrect RNNs%recurrent neural networks
, \cite{gu2021efficiently,dss,gu2021combining} %, orvieto2023resurrecting}
have introduced a series of architectures called State Space Models (SSMs) which include a linear recurrence that admits efficient computations and special structure on the transition matrices.

While these architectures have demonstrated impressive performance in long sequence modeling, their performance on fundamental tasks such as language modeling fall short compared to transformers, mostly due to intrinsic superiority of transformers in modeling interactions between different elements of the input sequence. Recent work~\cite{gu2023mamba}, has introduced Mamba, an SSM variant with Selective SSMs (S6) as its core block. In an S6 layer, the parameters are a function of the input, providing the SSM with content awareness. The empirical success of Mamba is undeniable, with applications spanning large-scale language modeling~\cite{zuo2024falcon,lieber2024jamba,waleffe2024empirical}, image~\cite{mambaViT1}, %,mambaViT2}
and video~\cite{li2025videomamba} %,chen2024video}
processing, medical imaging~\cite{mambaMedical3}, %mambaMedical8,mambaMedical4,mambaMedical3}%,,mambaMedical5}
 tabular data analysis~\cite{ahamed2024mambatab}, Reinforcement Learning~\cite{lv2024decision}, point-cloud analysis~\cite{mambaPoint}, graph processing~\cite{mambaGraph1}%,mambaGraph2}
, and N-dimensional sequence modeling~\cite{mambaND}.

The success of Mamba models across various domains ignites interest in their theoretical properties. Establishing a comprehensive theoretical understanding is crucial, as it enhances our knowledge of these layers, promotes their adoption within the research community, and paves the way for future architectural advancements. Additionally, since S6 can be considered a variant of attention with linear complexity~\cite{ali2024hidden}, deeper theoretical insights could elucidate the relationships between gated RNNs, transformers, and SSMs, thereby advancing our knowledge of these interconnected architectures. 
Initial efforts to establish a theoretical framework for the expressiveness of selective (and non-selective) SSM layers have been undertaken by several researchers. Using Rough Path Theory, \citet{cirone2024theoretical} demonstrated that diagonal selective SSMs, such as Mamba, possess less expressive power than their non-diagonal counterparts. Additionally, \citet{merrill2024illusion} investigated the expressiveness relationships between SSM variants and transformers using the lens of circuit complexity, revealing that both models share the same expressive power (belonging to ${TC}^0$). Finally, \citet{jelassi2024repeat} examined the copying ability of various SSM variants compared to transformers. It concluded that from both theoretical and empirical perspectives SSMs struggle with this task. While these significant studies highlight the limited expressive capabilities of Mamba models compared to other architectures, our work introduces a different trend. We demonstrate the superior expressive power of S6 layers, using a theoretical framework based on multivariate polynomials. 

{\color{black}In addition to exploring the expressiveness gap between transformers and S6 layers, we develop norm-based length-agnostic generalization bounds for S6 layers. Suggesting that the added expressivity of the selective mechanism does not hinder the generalization properties compared to traditional SSMs as studied in \citep{liu2024generalization},  and that while polynomial expressivity increases with sequence length, it does not impact generalization.}
%
% % have introduced a time dependent linear recurrence where each 
%
% Recently, Selective State-Space Layers (S6)~\cite{gu2023mamba}, the core component of the Mamba block, have demonstrated exceptional performance across a wide range of applications. These applications span language modeling~\cite{gu2023mamba,mambamoe1,mambamoe2,wang2024mambabyte,lieber2024jamba}, image~\cite{mambaViT1,mambaViT2} and video~\cite{mambaVideo,li2024videomamba,chen2024video} processing, medical imaging~\cite{mambaMedical7,mambaMedical5,mambaMedical8,mambaMedical4,mambaMedical3,mambaMedical2,mambaMedical1}, tabular data analysis~\cite{ahamed2024mambatab}, Reinforcement Learning~\cite{Ota2024DecisionMR,mambaRL1,Dai2024IsMC}, point-cloud analysis~\cite{mambaPoint}, graph processing~\cite{mambaGraph1,mambaGraph2}, and N-dimensional sequence modeling~\cite{mambaND}. Mamba models are distinguished by their hardware-aware design and dual-view computation, which introduce linear complexity in sequence length during training and enable fast RNN-like computation during inference. This unique combination results in a five fold increase in throughput in comparison to  Transformers, making Mamba models a powerful and efficient choice for modern deep-learning tasks.
%
% Despite the widespread success of Mamba models across various application domains, their underlying theoretical foundations remain relatively unexplored. Establishing a comprehensive theoretical understanding is crucial, as it enhances our knowledge of these layers, promotes their adoption within the research community, and paves the way for future architectural advancements. Additionally, since S6 can be considered a variant of attention with linear complexity~\cite{ali2024hidden}, deeper theoretical insights could elucidate the relationships between gated RNNs, transformers, and State Space Models (SSMs), thereby advancing our knowledge of these interconnected architectures. 
%
% Our study investigates these important aspects from the perspective of expressiveness and generalization. To do so, we introduce a simplified and effective variant of the S6 layer, which is easier to analyze and empirically comparable with the original Mamba block on large-scale image classification and language modeling. Our variant includes the first discretization-free S6 layer, which involves only polynomial operations. Such simplifications have served as a foundation for other theoretical works (e.g. \citep{kileel2019expressive, chrysos2021deep} in the context of deep polynomial NNs, and \citep{ kim2023polynomial, song2023expressibility} for polynomial self-attention).
%
%
% To obtain meaningful theoretical results, we suggest a simplified variant of the original S6 layer which is easier to analyzed as well as present comparable performance. %
% To derive theoretical results we present a simplified layer of the complex S6 layer as customary with theoretical works in deep learning. We empirically validate our model and show it achieves competitive results to the original architecture. 
%
% 
% Initial efforts to establish a theoretical framework for the expressiveness of selective state-space layers have been undertaken by several researchers. Using Rough Path Theory, \cite{cirone2024theoretical} demonstrated that diagonal selective SSMs, such as Mamba, possess less expressive power than their non-diagonal counterparts. Additionally, \cite{merrill2024illusion} investigated the expressiveness relationships between SSM variants and transformers from the perspective of circuit complexity, revealing that both models share the same expressive power (belonging to ${TC}^0$). Finally, \cite{jelassi2024repeat} examined the copying ability of various SSM variants compared to transformers. It concluded that from both theoretical and empirical perspectives SSMs struggle with this task. While these significant studies highlight the limited expressive capabilities of Mamba models compared to other architectures, our work introduces a different trend. We demonstrate the superior expressive power of these layers, using a theoretical framework based on multivariate polynomials. 
%
% {\color{blue}
% As for the generalization of selective state-space layers, less effort has been investigated. Our research takes the first steps in this direction. Our bound utilizes Rademacher complexity, which provides an upper bound on the difference between training and testing errors within a specific hypothesis class. It is defined as the expected performance of the class averaged over all possible data labelings, with labels independently and uniformly drawn from the set $\{\pm 1\}$. More details in \citep{mohri2018foundations, Shalev-Shwartz2014}. We provide norm-based generalization bound for the S6 that do not
% depend on the input sequence length. We also provide experiments 
% that empirically validate our theoretical findings.
% %  {\color{black}AND DOES WHAT}. % ? However, several results have been proposed for traditional non-selective state-space layers~\cite{liu2024generalization}.
% % \amirg{there's not enough in the intro about what you do. The contribution part below should summarize contributions, but before, you should give a general idea for what your approach and reuslts are.}
%
%
%
% The generalization capabilities of neural networks have been widely studied~\citep{neyshabur2015norm,golowich2018size,Bartlett2001RademacherAG,Harvey2017NearlytightVB, bartlett2017spectrally,Neyshabur2018APA,cao2019generalization,daniely2019generalization,wei2019data,allen2019learning,https://doi.org/10.48550/arxiv.1806.05159}. Many efforts have have been made to make these bounds more applicable to practical settings. Some researchers have developed norm-based generalization bounds for more complex architectures like RNN~\citep{chen2019generalization, tu2019understanding}, and transformers ~\citep{trauger2024sequence}. Despite these advances, most of the research remains focused on fully-connected networks, which generally underperform compared to modern architectures such as S6 and transformers~\citep{NIPS2017_3f5ee243,dosovitskiy2020image}. As a result, these bounds have limited ability to explain the success of contemporary architectures.
% % \amirg{there's certainly some theory work on transformers (e.g., that they have bias towards sparsity, and some stuff about expressive power). OK I see you cover some of it below, so tone down the claim here.)}
% }

% The generalization capabilities of neural networks have been widely studied, with various efforts made to develop bounds applicable to practical settings~\citep{neyshabur2015norm,golowich2018size,Bartlett2001RademacherAG,Harvey2017NearlytightVB, bartlett2017spectrally,Neyshabur2018APA,cao2019generalization,wei2019data,allen2019learning,li2018tighter}. Notably, norm-based generalization bounds have been proposed for complex architectures, such as RNNs~\citep{chen2019generalization, tu2019understanding} and transformers~\citep{trauger2024sequence}. However, most research remains focused on fully-connected networks %, which generally underperform compared to modern architectures such as S6 and transformers~\citep{NIPS2017_3f5ee243,dosovitskiy2020image}. Consequently, these bounds have limited ability to explain the success of contemporary architectures.
% % and the generalization of S6 layers has received less attention. Our research takes the first steps in this direction. %Oוr bound utilizes Rademacher complexity, which provides an upper bound on the difference between training and testing errors within a specific hypothesis class. This complexity is defined as the expected performance of the class averaged over all possible data labelings, with labels independently and uniformly drawn from the set $\{\pm 1\}$. For further details, refer to~\citep{mohri2018foundations, Shalev-Shwartz2014}.
% % Notably, the norm-based generalization bound we compute for the S6 architecture does not depend on the input sequence length. 
%
% %We conduct experiments to empirically validate our theoretical findings. This work aims to extend the understanding of generalization in modern neural network architectures, particularly those involving selective state-space layers.
%
%
%
\noindent{\textbf{Our main contribution}} encompasses the following aspects: (i) We present simplified polynomial variants of Mamba that exhibit comparable performance on NLP and vision tasks, promoting a simpler model that can serve as a foundation for other theoretical contributions, as well as shed light on the inner dynamics of Mamba and its critical components. (ii) By establishing the connection between multivariate polynomials and S6 layers, we theoretically prove that S6 layers are expressive as a {\color{black}linear} self-attention with depth that scales logarithmically with the sequence length; that is, there are functions that can be modeled with a single S6 that would require a logarithmic number of {\color{black}linear} %self-
attention layers. More generally, we establish that a Mamba model with only 4 layers is sufficient to represent the class of multivariate polynomials with bounded degree L. In contrast, a {\color{black}linear} attention-based model requires a logarithmic number of layers in $L$ to achieve the same representational capability. (iii) %Finally, t% 
Through experiments on a synthetic dataset in a controlled environment designed to isolate expressivity issues, we empirically validate that our theory is reflected in practice. (iv) {\color{black}Finally, although S6 has better polynomial expressivity for long sequences, we show that this does not come at the cost of limited generalization, by proving the first length-agnostic generalization bound for S6 %layers
. This leads us to conclude that S6 layer possesses superior theoretical properties over linear attention for long-range tasks.}  %for long-range tasks, we show that S6 layers possess favorable theoretical expressivity compared to transformers, as shown in Fig.~\ref{fig:mainfig}.
%
\begin{figure}[t]
    \centering
    \includegraphics[width=0.82\linewidth]{figures/theoryMambaMainNewSmall_only_express.jpg}
    % \vspace{-2pt}
    % \caption{\textbf{(Left)} For long-range regimes, Mamba has better theoretical properties than self-attention in both generalization and expressivity. %{\color{blue}\textbf{(Right)} A schematic overview of the Mamba variant we explored.}
    % \textbf{(Right)} Our characterization of SSMs, S6 layers, and causal self-attention via multivariate polynomials allows us to identify the expressiveness gap between these layers through maximal polynomial degree. 
    % }
    \vspace{-3pt}
    \caption{\textbf{Expressivity via Polynomial Degree:} Our characterization of SSMs, S6 layers, and causal self-attention via multivariate polynomials allows us to identify the expressiveness gap between these layers through maximal polynomial degree.}
    \label{fig:mainfig}
    \vspace{-12pt}
\end{figure} 
%
\vspace{-12pt}
\section{Related Work}
\vspace{-3pt}
%State Space Models 
SSMs have gained a lot of traction due to their remarkable performance and computational efficiency. Their theoretical properties have been the focus of many recent works; \citet{merrill2024illusion} compare the expressive capacity of (non-selective) SSMs and transformers, concluding that both architectures belong to the same complexity class ($TC^0$).\footnote{$TC^0$ is the complexity class that can be decided by polynomial-sized Boolean circuits.} \citet{sarrof2024expressive} conduct a more refined analysis and show that transformers and SSMs occupy different portions of $TC^0$. Another work showed that a single-layer Transformer with $N$ heads can simulate any state space model with $N$ channels~\citep{zimerman2023long}. 
 

\citet{cirone2024theoretical} show that the selective mechanism introduced in Mamba results with more expressive architectures compared to traditional (non-selective) SSMs.
\citet{ali2024hidden} show that there are functions that S6 can implement while transformers cannot.

In this work we compare the expressive power of S6 to those of transformers. We show that under certain assumptions which we justify empirically, a constant number of S6 layers are dense in the polynomial function space while transformers and non-selective SSMs are far less expressive. %
%
In addition to discussing expressivity, we also provide the first norm-based generalization bound for S6. Relevant related works are detailed in Appendix~\ref{sec:RelatedWorkGeneralization}.

% In addition to the discussion of expressivity, we also provide the first norm-based generalization bound for S6, extending prior results for non-selective SSMs \cite{liu2024generalization}.

% Another work showed that SSMs are universal approximators and have a tendency to exponential decay~\citep{wang2024state}. In this work we compare the expressive power of S6 to those of transformers, considering polynomial activations and show that given a similar number of parameters the former is exponentially more expressive than the latter.


% % Explaining the remarkable performance of overparameterized deep neural networks (DNNs) to perform well on test data is one of the major open challenges in the theory of deep learning. Traditional tools used in statistical learning such as the PAC-learning framework and VC-dimension provide vacant bounds when the number of parameters is order of magnitudes greater than the number of parameters of a neural architecture. To make progress on this frontier, many works perform ad-hoc analyses that take into account the specific architecture and optimization algorithm used. Allen et al. analyze the dynamics of stochastic gradient descent on RNNs with ReLU activations and provide optimization and generalization guarantees~\citep{allen2019can}. Insights into the dynamics of gradient descent under different assumptions, shed light onto the generalization guarantees of RNNs for unseen data and longer sequences than those used for training~\citep{cohen2022implicit, emami2021implicit, cohen2022learning, hardt2018gradient}. To the best of our knowledge, the only work discussing the generalization properties of state-space layers, is that of Liu et al.~\cite{liu2024generalization} which provides data-dependent generalization bounds. However, their work does not consider the recently introduced selective mechanism which is widely adopted in practice. {\color{black}In this work we derive norm-based bounds on the Rademacher complexity of S6.}

% Explaining the performance of overparameterized %deep neural networks (
% DNNs %)
% on test data remains a major challenge in deep learning theory. Traditional tools like %the
% PAC-learning %framework
% and VC-dimension often provide vacuous bounds when the number of parameters greatly exceeds the number of data points. To address this, many studies conduct architecture-specific analyses. For instance, Allen et al. analyze the dynamics of stochastic gradient descent on RNNs with ReLU activations, offering optimization and generalization guarantees~\citep{allen2019can}. Other works have explored the generalization of RNNs for unseen data and longer sequences under various assumptions~\citep{cohen2022implicit, emami2021implicit, cohen2022learning, hardt2018gradient}.
% %Liu et al. provide data-dependent generalization bounds for state-space layers but do not consider the selective mechanism commonly used in practice~\cite{liu2024generalization}. Our work fills this gap by deriving norm-based bounds on the Rademacher complexity of selective state-space layers (S6).


% %{\color{blue}
% % New text by Lian
% %{\bf Norm-based generalization bounds.\enspace} 
% %Recently,  norm-based generalization bounds were introduced for NNs~\citep{neyshabur2015norm,golowich2018size,Bartlett2001RademacherAG,Harvey2017NearlytightVB, bartlett2017spectrally,Neyshabur2018APA,cao2019generalization,daniely2019generalization,wei2019data,https://doi.org/10.48550/arxiv.1806.05159,chen2019generalization,tu2019understanding}. A common method for estimating the difference between the training and test errors of a NN is through the use of the network's 
% %Rademacher complexityle these results provide robust upper bounds on the test error of NNs, they incorporate very limited information about the network's architectural choices or are purely theoretical. In particular, 

% There are relatively few results that address modern architectures such as S6 layers. A recent contribution proposes a bound for standard SSMs~\citep{liu2024generalization}, but it does not extend to Selective SSMs. To address this gap, we propose a new bound specifically for Selective SSMs. 
% % and provide empirical evaluations to demonstrate its effectiveness.

\vspace{-5pt}
\section{Background}\label{sec:background}
\vspace{-3pt}
In this section we present the technical details required for the theoretical analysis and provide relevant %definitions and
notations.


\noindent{\textbf{{Notations\quad}}  Let $X \in \mathbb{R}^{D \times L}$ be an input sequence of length $L$ with dimension size $D$, denote the element in the $i$-th channel and position $j$ as $X_{ij}$. We denote the entire channel at a specific position $j$ and the entire sequence at a specific channel $i$ as $X_{*j}$ and $X_{i*}$, respectively. For simplicity, we denote a general single channel of $X$ without channel index by $x := (x_1, x_2, \cdots, x_L)$ such that $x_i \in \mathbb{R}$.

\noindent{\textbf{{Mamba\quad}} Given these notations, a Mamba block, which is built on top of the S6 layer% and gated-MLP
, is specified as follows:
{\small
\begin{equation} \nonumber
    X = \sigma(\text{Conv1D}(\text{Linear}(U)), \quad Z = \sigma(\text{Linear}(U)) 
    \vspace{-3pt}
\end{equation}
}
\vspace{-8pt}
{\small
\begin{equation} \label{eq:mamba1}
    Y = \text{S6}(X),\quad \hat{Y} = \text{Linear}(Y \otimes Z)
\end{equation}
}

Here, $U$ is the input to the Mamba block, and $X$ is the input to the S6 layers, $X,Z,Y,\hat{Y} \in \mathbb{R}^{L \times D}$, the linear layers operate independently for each sequence element, $\sigma$ represents SiLU activation function, and $\otimes$ denotes element-wise multiplication with the gate branch. 

\noindent{\textbf{S6 }}
An S6 layer is a recent variant of SSM. A standard diagonal SSM is parameterized by a diagonal transition matrix $A \in \mathbb{R}^{N \times N}$, input and output matrices $B,C \in \mathbb{R}^{N \times 1}$, and a timescale $\Delta \in \mathbb{R}$. An input scalar sequence $x$ is mapped to an output scalar sequence $y$ via the following recurrent rule:
%
\vspace{-5pt}
{\small
\begin{equation}\nonumber
\vspace{-3pt}
h_t =  \bar{A} h_{t-1} + \bar{B}x_t, \quad y_k = C h_t
\vspace{-5pt}
\end{equation}
}
%
{\small
\begin{equation}\label{eq:recRule}
\vspace{-3pt}
\bar{A}=f_A (A, \Delta), \quad  \bar{B}=f_B (B, \Delta)
\end{equation}
}
% \begin{align} \label{eq:recRule}
% \small
%     h_t =  \bar{A} h_{t-1} + \bar{B}x_t, \textbf{  } y_k = Cx_t, \textbf{  } %\bar{A}=f_A (A, \Delta), \quad \bar{B}=f_B (A,B, \Delta)
% % \end{equation}
% % \begin{equation} 
% % \label{eq:recRule1}
% % \small
%  \bar{A}=f_A (A, \Delta), \textbf{  }  \bar{B}=f_B (B, \Delta)
% \end{align}
%
where $f_A,f_B$ are discretization functions, and the discrete system matrices are $\bar{A} \in \mathbb{R}^{N \times N}$ and $\bar{B} \in \mathbb{R}^{N \times 1}$. The recurrent rule in Eq.~\ref{eq:recRule} can be computed efficiently in parallel on modern hardware accelerators using work-efficient parallel scans~\cite{smith2022simplified} or a simple scalar convolution via FFTs~\cite{gu2021combining}. Note that Eq.~\ref{eq:recRule} is a map from $\mathbb{R}^L$ to $\mathbb{R}^L$, and to process $D$ channels, multiple independent instances are used.
% \begin{equation} \label{eq:discretization}
%      A_i = \exp (\Delta_i A), \quad B_i = \Delta_i B_i, 
% \end{equation}

Contrary to traditional SSMs, which utilize time-invariant system matrices and process each channel independently, S6 layers incorporate a data-dependent mechanism that is parameterized by $S_B, S_C \in \mathbb{R}^{N \times D}$, $A \in \mathbb{R}^{D \times N}$ and $S_{\Delta} \in \mathbb{R}^{1 \times D}$ to define the time-variant matrices as follows:
%
%
{\small
\begin{equation}\nonumber 
    \vspace{-3pt}
    B_t = S_B X_{*t}, \ C_t = S_C X_{*t}, \ \Delta_t = \text{softplus}(S_{\Delta} X_{*t})%, \quad \bar{A}_t = \exp(\Delta_t A), \quad \bar{B}_t = \Delta_t B_t
    \vspace{-2pt}
\end{equation}
}
\vspace{-5pt}
{\small
\begin{equation}\label{eq:TimeVariantMatrices1} 
%\Delta_t = \text{softplus}(S_{\Delta} X_{*t}), \quad 
\bar{A}_t = \exp(\Delta_t A), \quad \bar{B}_t = \Delta_t B_t
\end{equation}
}
% \begin{equation} \label{eq:discretization}
%      \bar{A}_i = \exp (\Delta_i A), \quad f_B(\Delta_i, B_i) = \Delta_i B_i, 
% \end{equation}
% \begin{equation}\label{eq:TimeVariantMatrices2}
%     \bar{A}_i = f_A(\Delta_i, A), \quad \bar{B}_i = f_B(\Delta_i, B_i)
% \end{equation}
%
The resulting time-variant recurrent rule is:
{\small
\begin{equation}\label{eq:timeVaraintRecRule}
     \vspace{-2pt}
     h_t =  \bar{A}_t h_{t-1} + \bar{B}_t x_t, \quad y_k = C_t h_t
\end{equation}
}
Our analysis focuses on the regime of 'many-to-one', which deals with models that operate on sequences %of length $L$
and produce a single output after processing the entire input sequence.%{\color{blue}, as illustrated in the right panel of Fig.~\ref{fig:mainfig}.}
%
% Additional background material is presented in {\color{black}Appendix TBD}.

\vspace{-4pt}
\section{Theoretical Results}
\vspace{-3pt}
We begin by presenting our simplified model in Sec.~\ref{sec:simplify}, which forms the basis for our theory on the expressivity {\color{black}and generalization of S6 layers, discussed in} Sec.\ref{sec:expressivity} and Sec.~\ref{sec:generalization}, respectively. In the experiments section we demonstrate that the simplified model used in our exposition achieves comparable results to those of the original S6 layers, encouraging further exploration of the suggested simplification.
%
% Our analysis is carried out on a simplified version of S6 where the non-linear operators are approximated using polynomials. 

\vspace{-3pt}
\subsection{Model Simplifications\label{sec:simplify}} %
\vspace{-2pt}
The original S6 layer is parameterized by $S_{\Delta}$, $S_{B}$, $S_C$ and $A$, and it is defined in Eqs. \ref{eq:TimeVariantMatrices1} and \ref{eq:timeVaraintRecRule}.%
% which define the discrete time-variant system matrices as:
% \begin{equation} \label{eq:TimeVariantMatrices1}
%     B_i = S_B x_i, \quad C_i = S_C x_i, \quad \Delta_i = \text{softplus}(S_{\Delta} x_i), \quad
% % \end{equation}
% % \begin{equation} \label{eq:discretization}
%      % f_A(\Delta_i, A) = \exp (\Delta_i A), \quad f_B(\Delta_i, B_i) = \Delta_i B_i, 
% % \end{equation}
% % \begin{equation}\label{eq:TimeVariantMatrices2}
%     \bar{A}_i = \exp (\Delta_i A), \quad \bar{B}_i = \Delta_i B_i
% \end{equation}

Our approach utilizes a simplified model described below:%as follows:
% \vspace{-3pt}
{\small
\begin{equation*}
      \bar{B}_i = S_B x_i,\quad C_i = S_C x_i,\quad %\Delta_i = p_1 (S_{\Delta} x_i),  \;%, \quad  \bar{A}_i =  p_2 ( \Delta_i A), \quad \bar{B}_i = B_i 
% \end{equation}
% }
% \vspace{-5pt}
% {\small
% \begin{equation}\nonumber
\Delta_i = p_1 (S_{\Delta} x_i),\quad \bar{A}_i =  p_2 ( \Delta_i A) 
\end{equation*}
}
% {\color{black}maybe add dimensions for formality and the activation function exp}

where $p_1$ and $p_2$ represent polynomials that operate independently per element, for instance, a second-degree Taylor approximation for softplus and exponent accordingly. An equivalent model would be:
% \vspace{-5pt}
{\small
\begin{equation}\label{eq:simplifiedModel}
 \vspace{-5pt}
    C_i = S_C x_i, \quad \bar{B}_i = S_B x_i, \quad \bar{A}_i = p_2\left(p_1 \left(\frac{S_\Delta x_i}{\sqrt{D}}\right) A\right)
\end{equation}
}
% \vspace{-2pt}

in which $D$ is the width of the model, and $\frac{1}{\sqrt{D}}$ is a constant normalization factor. This model can be interpreted as state-space layer without discretization, with $\bar{A}$ being selective, and $p_1$ and $p_2$ are stabilizers designed to control the values of $A$, which must be positive. It is imperative to note that standard SSMs without discretization are both effective and simple~\cite{gupta2022simplifying}.

%For simplicity, we define $p_A(x)=p_2 (p_1(\frac{x}{\sqrt{D}}))$

For simplicity, we define an additional polynomial $p_A(x) = p_2\left(p_1 \left(\frac{x}{\sqrt{D}}\right) A\right)$ which is parameterized by $A$ and ties $p_1$ and $p_2$. Hence, we can denote $\bar{A}_i = p_A(S_{\Delta} x_i)$.

Alternatively we can utilize the following simplified non-polynomial model:
% 
{\small
\begin{equation}\nonumber
\vspace{-3pt}
\bar{B}_i = S_B x_i, \quad C_i = S_C x_i
\vspace{-2pt}
\end{equation}
}
\vspace{-5pt}
{\small
\begin{equation}\label{eq:model}
\Delta_i = \text{softplus}(S_{\Delta} x_i),\quad
% \end{equation}
% \begin{equation}\label{eq:model}
\bar{A}_i =  \exp ( \Delta_i A)%, \quad \bar{B}_i = B_i 
\end{equation}
}
% {\color{black}where $R>1$ is a constant normalization factor.}
% For simplicity, we define an additional polynomial $p_A$ which is parameterized by $A$ and ties $p_1$ and $p_2$. It defines the following polynomial and model:
% \begin{equation}\label{eq:simplifiedModelFortheoren}
%     \forall x : p_A(x) = p_2(p_1 (\frac{x}{\sqrt{D}}) A), \quad C_i = S_C x_i, \quad \bar{B}_i = S_B x_i, \quad \bar{A}_i = p_A(S_{\Delta} x_i)
% \end{equation}
% \vspace{-12pt}
%
%
% 
%
\vspace{-18pt}
\subsection{Expressivity\label{sec:expressivity}
}
\vspace{-3pt}
We identify an expressivity gap between S6 and attention via multivariate polynomials. %We characterize this gap with Theorems~\ref{theorem:exprrsFull} and~\ref{theorem:AnyPolywithMambas}.
This gap is delineated through Thm.~\ref{theorem:exprrsFull} and Thm.~\ref{theorem:AnyPolywithMambas}. The former establishes that attention models require $O(\log L)$ layers to represent $L$-degree multivariate polynomials, whereas Mamba models can express such polynomials within a single layer. Furthermore, Thm.~\ref{theorem:AnyPolywithMambas} extends this finding by demonstrating that the expressiveness gap is not limited to anecdotal or edge-case polynomials. Rather, it encompasses a broad spectrum of polynomial functions.

% \begin{assumption}
%     The state size in S6 is equal to the hidden size of transformers.
% \end{assumption}

% {\noindent \textbf{Assumptions}:}
% \begin{enumerate}\label{par:assumptions}
%     \item The state size in S6 is equal to the hidden size of transformers.
%     % \item We assume that the S6 layers employ the simplified polynomial variant described in Eq.~\ref{eq:simplifiedModelFortheoren}.% (Sec.~\ref{sec:simplify}
%     \item We analyze self-attention with polynomial activations $p(x) = x^p$ instead of softmax.
%     % \item We consider the regime of 'many-to-one', which deals with models that operate on sequences of size $L$ and produce a signal output.
% \end{enumerate}
\begin{theorem}\label{theorem:exprrsFull}(informal)
%Under the assumptions specified in~\ref{par:assumptions},
%Let $n\in\mathbb{N}$ be the hidden state of 
Consider an S6 layer and single Transformer layer, both with hidden dimension $N$. 
% Assume S6 layer with hidden-state of dimension $N$, 
For input sequences of length \( L \geq 3 \), a single layer of Mamba is logarithmically more expressively efficient in depth compared to a single causal {\color{black}linear} self-attention layer with a single head and polynomial activations instead of softmax. 
\end{theorem}

We consider this theorem relatively surprising, as transformers are considered highly expressive models, in contrast to state-space layers, which are traditionally constrained. The proof follows from Lemma~\ref{lemma:dir2} and Lemma~\ref{lemma:dir1} which are presented next.

% Below is the proof:
% \begin{proof}[Proof of Theorem~\ref{theorem:exprrsFull}]
% We derive the theorem from the following two lemmas:


\begin{lemma}\label{lemma:dir2}
There exists a function \( f : \mathbb{R}^L \rightarrow \mathbb{R} \) that can be implemented by one channel of S6 such that a single {\color{black}linear} attention head would require at least  \( O(\log L) \) layers to express $f$.%this function.
\end{lemma}

\begin{lemma}\label{lemma:dir1}
Any function that can be expressed by a single {\color{black}causal linear} attention head can %also
be expressed by a single channel of S6.
\end{lemma}

For the complete details and proof of Lemma~\ref{lemma:dir1}, we refer the reader to the appendix (Lemma \ref{lemma:dir1Appendix}). As for Lemma~\ref{lemma:dir2}, we present here a proof sketch for a simplified version (see Lemma~\ref{lemma:dir2appendix}
in the appendix for the complete proof). In this simplified case, our proof focuses exclusively on linear-attention%, omitting the softmax over the attention scores
. Additionally, we consider models that process scalar sequences. 

\begin{proof}[Proof of Lemma~\ref{lemma:dir2} (without Softmax% and scalar sequences
)]
Our proof relies on the characterization of the hypothesis classes that are realizable through S6 and self-attention via {\underline{multivariate polynomials}}. We start by presenting this formulation for S6:

\smallskip
\noindent\textbf{Single S6 Layer as Multivariate Polynomials\quad}
One channel of the variant of the S6 layer we consider is described in detail in Eq.~\ref{eq:simplifiedModel}.
%s the following recurrent rule:
% \begin{equation}\label{eq:simplifiedModelFortheoren1}
%     C_t = S_C x_t, \quad \bar{B}_t = S_B x_t, \quad \bar{A}_t = p_A(S_{\Delta} x_t)
% \end{equation}
%
% \begin{equation}
%     y_{t} = C_t h_{t}, \quad h_{t}= \bar{A}_t h_{t-1} + \bar{B}_t x_{t}, \quad   C_t = S_C x_t \quad \quad \bar{B}_t = S_B x_{t} \quad \quad \bar{A}_t = S_A x_{t} 
% \end{equation}
%
%
Since we are interested in identifying the minimal polynomial degree required to characterize models that employ Eq.~\ref{eq:simplifiedModel}, we can assume that $P_A$ is linear. Thus, we can incorporate the coefficients of $P_A$ into $S_{\Delta}$, namely, $\bar{A}_t = S_{\Delta} x_t$. Consequently, Eq.\ref{eq:simplifiedModel} can be unrolled as follows:
{\small
\begin{equation}\label{eq:unrolledChanSl}
    y_{t}  %C_t \sum_{j=1}^t \Big{(}\Pi_{k=j+1}^t \bar{A}_k \Big{)} \bar{B}_j x_{j} %\nonumber\\
    %&
    = S_C S_B  \sum_{j=1}^t {S_{\Delta}}^{t-j-1} \Big{(} {\Pi_{k=j+1}^t x_k  }\Big{)} x_t {x_j}^2 
\end{equation}
}
% \begin{equation}
%     \small
%     y_{t} = C_t \sum_{j=1}^t \Big{(}\Pi_{k=j+1}^t \bar{A}_k \Big{)} \bar{B}_j x_{j} = 
% % \end{equation}
% % \begin{equation}    
%     S_C S_B  \sum_{j=1}^t {S_{\Delta}}^{t-j-1} \Big{(} {\Pi_{k=j+1}^t x_k  }\Big{)} x_t {x_j}^2 
% \end{equation}

Hence, we can characterize the last output $y_L$ as a multivariate polynomial with at least $L$ monomials, and a maximal degree of at least $L+3$. Denote $c_j = S_C S_B S_{\Delta}^{L-j-1}$,%, and an aggregated total degree higher than $\sum_{j=1}^L j+3 = 3L + \sum_{j=1}^L j = \frac{1}{2} (L^2 + 7L)$:
\vspace{-5pt}
\begin{equation}\label{eq:SimpleMamba1Chan}
    y_L = \sum_{j=1}^L c_j \Big{(} \Pi_{k=j+1}^{t-1} x_k \Big{)} {x_t}^2 {x_j}^2 %, \quad c_j = S_C S_B S_{\Delta}^{L-j-1}
    % y_L = \sum_{j=1}^L c_j \Big{(} \Pi_{k=j+1}^{t-1} x_k \Big{)} {x_t}^2 {x_j}^2, \quad c_j = S_C S_B S_{\Delta}^{L-j-1} 
\end{equation}
% \vspace{-2pt}
% where $c_j = S_C S_B S_{\Delta}^{L-j-1}$.
% where $c_j = S_C S_B S_{\Delta}^{L-j-1}$.
\noindent\textbf{Attention as Multivariate Polynomials\quad}
Given the parameters of the layers $W_Q, W_K, W_V \in \mathbb{R}$, the last element in the output sequence can be computed by:
\vspace{-5pt}
{\small
\begin{equation}
y=\frac{(x W_{Q}) (x  W_{K})^T}{\sqrt{d_k}} (x W_{V})%, \quad y_L = W_Q W_K W_V \sum_{j=1}^L  {x_j}^2 x_L = \sum_{j=1}^L c_j{x_j}^2 x_L 
\end{equation}
}
\vspace{-5pt}
{\small
\begin{equation}
\small
y_L = W_Q W_K W_V \sum_{j=1}^L  {x_j}^2 x_L = \sum_{j=1}^L c_j{x_j}^2 x_L 
\vspace{-3pt}
\end{equation}
}

where $c_j = W_Q W_K W_V$. Hence, we can characterize the last output element $y_L$ of a self-attention layer by a 3-degree multivariate polynomial with $L$ monomials.%, each of degree 3.%, resulting in a total aggregated degree of $3L$.

% \smallskip
\noindent\textbf{$N$-Stacked Attention Layers as Multivariate Polynomials\quad}
In light of the characterization of one head of a self-attention layer, we can now proceed to establish the characterization of $N$-stacked self-attention layers. Since the composition of multivariate polynomials results in another multivariate polynomial, where the maximal degree of the resulting polynomial amounts to the product of the maximum degrees of the composed polynomials, we can prove by induction that $N$-Stacked layers can be represented by a polynomial with a maximal degree of $3^N$.

% \smallskip
\noindent\textbf{Identify and Refine the Expressivity Gap\quad} The description of one head of a self-attention layer and one channel of an S6 via multivariate polynomials reveals the expressiveness gap in terms of the maximal degree, which is 3 for self-attention and $L+3$ for S6, as depicted in %the right panel of
Fig.~\ref{fig:mainfig}. Thus, our formulation presents a broad family of functions that can be implemented by selective SSMs but cannot be realized by a single self-attention head. Furthermore, when processing a sequence of length $L$, it is clear that in order to express a function realized by S6 using $N$-stacked self-attention layers, $O(\log L)$ stacked layers are required.
\end{proof}
% \vspace{-10pt}
% \end{proof}
% \vspace{-5pt}

% \smallskip
\noindent\textbf{Single SSM as Multivariate Polynomials\quad} For completeness, we formalized a single standard SSM (not S6) via multivariate polynomials. Since SSMs can be represented as a single non-circular convolution between the input \(x\) and a kernel \(\bar{K} = (C\bar{B}, C\bar{A}\bar{B}, \ldots, C\bar{A}^{L-1}\bar{B})\), it is evident that the last output element of the SSM can be expressed by a 1-degree multivariate polynomial consisting of \(L\) monomials.%, each of degree 1.%, with an aggregated total degree of \(L\).
%
% \begin{proof}[Proof of Lemma~\ref{lemma:dir1} (sketch)]. TBD
% \end{proof}
%



% \smallskip
\noindent\textbf{Sharpening the Expressivity Gap.} 
Thm~\ref{theorem:exprrsFull} establishes the existence of an expressivity gap. However, it remains unclear how many polynomials are encompassed within this gap, and whether they constitute a significant portion of the function class or are merely anecdotal instances. To refine our separation results %between the hypothesis classes derived from Mamba and attention-based architectures
, we now quantify the expressiveness gap by analyzing the number of layers required to represent the entire class of $L$-degree multivariate polynomials. As established previously, attention-based models necessitate $O(\log L)$ layers. In contrast, the following theorem demonstrate that a Mamba model with just 4-stacked layers and a sufficiently large number of channels can represent \textit{any multivariate polynomial of arbitrary degree}, thereby highlighting the significant expressiveness gap, which constitute a significant portion of the class of $L$-degree polynomials.% between these architectures.

\begin{theorem}\label{theorem:AnyPolywithMambas}
Given an input scalar sequence $x \in \mathbb{R}^L $, a model with four stacked Mamba layers, a sufficiently large number of channels, learnable PE, sufficient padding and a linear encoder layer at the first layer can express any  multivariate polynomial with bounded degree of $x$.
\end{theorem}

For simplicity, we assume that the state size $N = 1$ and the SiLU activations in Mamba are removed. We justify these decisions as they only restrict our model. The proof is presented in the A
appendix, and it follows a technical construction that demonstrates how to express a general polynomial using Mamba. The core argument is built on the following lemma, which is %also
visualized in Fig.~\ref{fig:exprresTheorem}:

\begin{figure*}[]
    \centering
    \vspace{-4pt}
    \includegraphics[width=0.92\linewidth]{figures/polynomialExpresivness.jpg}
    \vspace{-7pt}
    \caption{Visualization of 3-stacked Mamba layers expressing monomials of a univariate polynomial, as formulated in Lemma~\ref{lemma:3layerMambaExpresivity}. To simplify the visualization, the Conv1D layer has been omitted.}
    \label{fig:exprresTheorem}
     \vspace{-10pt}
\end{figure*} 
%

\begin{lemma}\label{lemma:3layerMambaExpresivity}
Given an input scalar sequence $x \in \mathbb{R}^L$, for any $j$, a model $M$ with 3-stacked Mamba layers, a sufficiently large number of channels, learnable PE, and a linear encoder in the first layer can express any monomial of a univariate polynomial in $x_j$. Specifically, for any constants $c \in \mathbb{R}$ and $P \in \mathbb{R}$, there exists a configuration of $M$ such that the output of the $k$-th channel $M(x)_k = c \cdot x_j^P$ for $k > P + j$.
\end{lemma}
%
%
For a detailed proof of Lemma~\ref{lemma:3layerMambaExpresivity}, we refer the reader to the appendix. Here, we provide an intuitive explanation of the proof, which hinges on the following two key capability of the Mamba architecture:


{\noindent\textbf{(i) Per-position selection:}} By utilizing Mamba's auxiliary components, including the gating branch, linear layers, and learnable PE, each Mamba layer can isolate a specific channel $k$ at a given position $j$. Notably, the output of the first Mamba block can effectively filter out all unnecessary positions, producing a sequence mask with zeros at every position except $i = j$, which contains $x_j$ at position $j$. This is done by setting the S6 to the identity function ($\bar{A}=0, \bar{B}=\bar{C}=1)$, ensuring $x_j$ is not modified. Additionally, mask the other positions achieved by set the parameters of a learnable PE at one of the channels to the indicator function $\mathbb{I}_{=j}$, which is 1 only when focusing on the $j$-th position, and ensuring this PE passed into the gate branch through the linear layers.


{\noindent\textbf{(ii) Express powers by aggregate multiplications of duplicated elements:}} Given an input $x = (0, \cdots, x_j, 0, \cdots)$, which can be obtained from the first layer, the second Mamba block can duplicate the value of $x_j$ exactly $P-2$ times. This duplication holds even if $P > L$, thanks to the ZeroPad component. The duplication process is achieved by setting the linear layers to identity mappings and utilizing a degenerate single SSM channel where the system matrices always equal $\bar{A},\bar{B},\bar{C}=1$. Therefore, if the $k$-th channel receives an input sequence $x = (0, \cdots, x_j, 0, \cdots)$, it will output $x = (0, \cdots, x_j, \cdots, x_j)$.
Through the gating mechanism, learnable PE (which can pass through skip-connections to the subsequent layers), and biases in the linear layer, the entire block can then produce a filtered output $z = (1, \cdots, 1, x_j, \cdots, x_j, 1, \cdots, 1)$, ensuring that there are exactly $P-2$ copies of $x_j$.
Next, the S6 layer in the third block can multiply the sequence elements in $z$ via the operation described in Eq.~\ref{eq:SimpleMamba1Chan}, which include the term $\Pi_{k=j+1}^t \bar{A}_k$. This term yields an output sequence with the values $(1, \cdots, 1, x_j^3, \cdots, x_j^P, \cdots)$. To specifically isolate ${x_j}^P$, we begin by generating all unwanted elements by applying the same SSM to the sequence $z$, introducing an additional zero at the initial occurrence of $x_j$ denoted by $z'$. We then subtract the outputs from these two SSM channels at the final linear layer of the block. This subtraction yields a telescoping series $SSM(z) - SSM(z')$:
\vspace{-2pt}
{\small
\begin{equation}
\sum_{j=1}^L c_j \Big{(} \Pi_{k=j+1}^{t-1} x_k \Big{)} {x_t}^2 {x_j}^2 - \sum_{j=2}^L c_j \Big{(} \Pi_{k=j+1}^{t-1} x_k \Big{)} {x_t}^2 {x_j}^2
     % C_t \sum_{j=1}^t \Big{(}\Pi_{k=j+1}^t \bar{A}_k \Big{)} \bar{B}_j x_{j} - C_t \sum_{j=2}^t \Big{(}\Pi_{k=j+1}^t \bar{A}_k \Big{)} \bar{B}_j x_{j} 
\vspace{-3pt}
\nonumber
\end{equation}
\vspace{-3pt}
}
%
that effectively eliminates any term except for ${x_j}^P$. %Finally, leveraging the gate branch, learnable PE, and skip-connections, we can ensure that all values, except those at position $i$, are zeroed out.

% Additionally, to filter out only ${x_j}^P$, we can produce all the elements we want to omit by applying the same SSM over the same sequences $z$ with an a additional zero at the first occurrence of $x_j$. After that, we can subtract the outputs of these two SSM channel in the last linear layer in the block,  which result in telescoping series that remove any term than $x_j$. Finally, Once again, through the gate branch, learnable PE and skip-connections, all values except at position $i$ can be masked with zeros. 


Finally, it is crucial to highlight that incorporating auxiliary components into the Transformer, such as learnable PE and gating, does not mitigate the logarithmic increase in depth required with L. This limitation arises because the expressible degree within each block remains unchanged, thereby leading to the observed asymptotic behavior.



\vspace{-4pt}
\subsection{Generalization\label{sec:generalization}}
\vspace{-2pt}
Our theoretical analysis in Sec.~\ref{sec:expressivity} demonstrates the superior expressive power of S6 compared to linear attention, particularly in terms of multivariate polynomial degree and long-range processing capabilities. Furthermore, Thm.~\ref{theorem:AnyPolywithMambas} and Lemma~\ref{lemma:dir2} establishes that the hypothesis class associated with Mamba models with just four layers is significantly larger than that of transformers, even with greater depth. While increased expressivity can often come at the cost of generalization, in this section, we show that S6 layers do not suffer from this trade-off. To do so, we % 
%Our second theoretical contribution 
provide a generalization bound for simplified S6 layers, as {\color{black}defined in Eq.~\ref{eq:modelAppendix}, which is a generalization of Eq.~\ref{eq:model}}. Our analysis is based on a classifier \( f : \mathbb{R}^{D \times L} \rightarrow \mathbb{R}^{\mathcal{C}} \), parameterized by  \( (A^{}, S_B^{}, S_C^{}, S_{\Delta}^{}) \).
% Let \( H \) be the number of layers in the network, and for each layer \( h \), where \( h \in [H] \), 
% We have parameters \( (A^{}, S_B^{}, S_C^{}, S_{\Delta}^{}) \) associated with the model. 
% Between each layer, there is an activation function \( \sigma^{(h)} \) with a Lipschitz constant of 1, which can be any activation function of choice.
The classifier \( f \) for a specific class \( c \in [\mathcal{C}] \) denoted by $f^c(X_{*1},...,X_{*L})$ is computed as:
% Our analysis is based on a classifier $f : \mathbb{R}^{D \times L} \rightarrow \mathbb{R}^{\mathcal{C}}$ defined as follows, $\forall c \in [\mathcal{C}]$,
% \begin{equation}
%    f^c(X^{}_{*1},...,X^{}_{*L}) =  \sum_{d=1}^D W_{c,d} (S_C X^{}_{*L})^T\sum_{i=1}^L \left(\prod_{k=i+1}^L (\exp(\Delta^{(j)}_{d,k} \cdot A_{d*}) \right)S_B X^{}_{*i} X^{}_{di}
% \end{equation}
% \begin{small}
% \begin{equation*}
% \begin{aligned}
%
\vspace{-3pt}
{\small
\begin{equation}
%\begin{small}
% &f^c(X_{*1},...,X_{*L}) =  
%\\&
\sum_{d=1}^D W_{c,d} \left(S_C^{} X^{}_{*L}\right)^T \sum_{i=1}^L \left(\prod_{k=i+1}^L \bar{A}^{}_{dk} \right) S_B^{} X^{}_{*i} X^{}_{di}
%\end{small}
\end{equation}
}
% \end{aligned}
% \end{equation*}
% \end{small}
%
where $\mathcal{C}$ is the number of classes, $ \bar{A}^{}_{dk} = \exp(\Delta_k A_d)$ ($A_d$ is the $d$th row of $A$) as defined in Eq.~\ref{eq:TimeVariantMatrices1} and $W \in \mathbb{R}^{\mathcal{C} \times D}$ represents a linear projection from the output to the number of classes.
% {\color{black} 
% In this notation, $X^{}_{ij}$ represents the input for the $i$-th channel of the $j$-th token in the sequence.
% }
% where $\mathcal{C}$ is the number of classes and $W \in \mathbb{R}^{\mathcal{C} \times D}$ represents a linear projection from the output to the number of classes. 
We denote the parameters of a classifier by 
$w  = \left(A, S_B, S_C, S_{\Delta}, W\right)$
and the corresponding function induced by a specific instance of $w$ by $f_w$.
The class of functions taking on different parameter instances $w$, is denoted by $\mathcal{F}$.
As customary in the derivation of Rademacher complexity based bounds (e.g. \citep{golowich2018size}), our analysis takes into account the (different) norms of the parameters, for which we 
denote:
\vspace{-3pt}
{\small
\begin{equation}\nonumber
\begin{aligned}
    \rho_W(w)&=||W||_{F},\; \rho_A^{}(w) = \|A^{}\|_{\max},\; \rho_B^{}(w) = \|S_B^{}\|_{2,\infty}, \\ \rho_C^{}(w) &= \|S_C^{(h)}\|_{F},\;
    \rho_{\Delta}^{}(w) = \|S_{\Delta}^{}\|_{2}, \\ \Gamma(w) &=  \rho_W(w) \cdot \rho_A^{}(w) \cdot \rho_B^{}(w) \cdot \rho_C^{}(w) \cdot \rho_{\Delta}^{}(w)
\end{aligned}
\end{equation}
}
Equipped with these notations, we are now ready to state our main generalization bound.
% {\color{black}
% \begin{theorem}\label{theorem:genbound}
% Let $P$ be a distribution over $\mathbb{R}^{D \times L} \times [C]$ and $\delta > 0$. Let $S = \{( X^{}_{(j)},y_{(j)})\}^{m}_{j=1}$ be a dataset of i.i.d. samples selected from $P$. Assume that $|X^{}_{(j)_{kl}}| \leq 1$ for all $j \in [m], k \in [D], l \in [L]$. Additionally, suppose $\forall n, n' \in [N], k\in[L]:  (\bar{A}^{}_k)_{n,n'} < K < 1$. Then, with probability at least $1-\delta$ over the selection of $S$, for any $f_w \in \mathcal{F}$, 
% \begin{small}
% \begin{equation*}
% \begin{aligned}
% \small
% &\err_P(f_w) - \fr{1}{m}\sum^{m}_{j=1}\bI[\max_{c \neq c'}(f^c_w( X^{}_{(j)})) + \gamma \geq f^{c'}_w( X^{}_{(j)})] \\& = \err_P(f_w) - \err^\gamma_S(f_w) \leq \frac{2\sqrt{2}}{\gamma m} ({\color{black}\Gamma(w) } +{\color{black}\frac{1}{D^2N^2}}) D^{1.5} \cdot \\ & (1 + \sqrt{2\log (4 \mathcal{C} D^3 N)})  \frac{1}{1-K}   \sqrt{ \max_i \sum_{j=1}^m \left|\left| \sum_{k=i+1}^L  X^{}_{{(j)}_{*k}} \right|\right|^2_2} \\&+ 3\sqrt{\frac{\log(2/\delta)+2\log({D^2N^2\color{black}\Gamma(w) }+2)}{2m}}
% \end{aligned}
% \end{equation*}
% \end{small}
% % where the maximum is taken over \(k \in [L]\), \(t \in [D]\). 
% \end{theorem}
%  }
 \begin{theorem}\label{theorem:genbound}
Let $P$ be a distribution over $\mathbb{R}^{D \times L} \times [C]$ and $\delta > 0$. Let $S = \{( X^{}_{(j)},y_{(j)})\}^{m}_{j=1}$ be a dataset of i.i.d. samples selected from $P$, where each \(X_{(j)} = (X_{(j)_{*1}}, \ldots, X_{(j)_{*L}}) \in \mathbb{R}^{D \times L}\). Assume that $\forall j \in [m]: ||X_{(j)}||_{\max} \leq 1$. Additionally, suppose $\forall k \in [L], d \in [D]:||\bar{A}^{}_{dk}||_{\max} < K < 1$. Then, with probability at least $1-\delta$ over the selection of $S$, for any $f_w \in \mathcal{F}$, 
{\small
\begin{equation*}
\begin{aligned}
&\err_P(f_w) - \fr{1}{m}\sum^{m}_{j=1}\bI[\max_{c \neq c'}(f^c_w( X_{(j)})) + \gamma \geq f^{c'}_w( X^{}_{(j)})] \\& = \err_P(f_w) - \err^\gamma_S(f_w) \leq \frac{2\sqrt{2}}{\gamma m} ({\Gamma(w) } +{\frac{1}{D^2N^2}}) D^{2} \cdot \\& (1 + \sqrt{2\log (4L \mathcal{C} D^4 N)}) \sqrt{\max_{t, k} \sum_{j=1}^m (X_{(j)_{tk}})^2} \frac{K}{(K-1)^2} \\&+ 3\sqrt{\frac{\log(2/\delta)+2\log({D^2N^2\Gamma(w) }+2)}{2m}},
\end{aligned}
\end{equation*}
}
where the maximum is taken over \(t \in [D]\),  \(k \in [L]\). 
\end{theorem} 
See Appendix~\ref{theorem:genproof} for the complete details and proof. A uniform convergence bound for a class $\mathcal{F}$ is an upper bound on the generalization gap that uniformly holds for all $f \in \mathcal{F}$. The more direct Rademacher complexity bound presented in  Lemma~\ref{lem:loss_ramp} 
is an example of a uniform convergence bound. However, these bounds become ineffective when a function $f \in \mathcal{F}$ exists that can perfectly fit any labeling of the dataset. In such situations, uniform convergence bounds fail to provide meaningful insights and are considered vacuous.

In contrast, the bound derived in Thm.~\ref{theorem:genbound} is also based on Rademacher complexity, but it is not a uniform convergence bound and hence it is not inherently vacuous. In the proof, we partition the class $\mathcal{F}$ into subsets $\mathcal{F}_{\rho}$, where the partitioning criterion $\rho$ is based on the norm of the S6 matrices, and apply our Rademacher bound (see Appendix \ref{theorem:rad}) %Lemma~\ref{lem:loss_ramp} 
within each subset. We then apply a union bound to combine these results, obtaining a bound %that is
individualized for each $f_w \in \mathcal{F}$. 

% {\color{black} CONCLUSION??}
%As discussed previously in Theorem~\ref{prop:rademacher}, t
% The Rademacher complexity bound on \(\mathcal{F}_{\rho}\) scales with \(\mathcal{O}(D^{2} \rho_W \rho_B \rho_C \rho_A \rho_{\Delta} \cdot \frac{1}{K} \sqrt{\frac{\beta}{m}})\) {\color{black}SINCE ... AND IF TOO COMPLEX GIVE INTUITION AND REFER TO THE APPENDIX. BUT BETTER BE SELF CONTAINED}. The second term in the bound %(see Theorem~\ref{thm:genbound}) 
% {\color{black} YOU CAN WRITE THE TERMS HERE OR DESCRIBE THEM MORE GLOBALLY WHAT EVER IS BEST FOR THE PAPER}
% is typically smaller than the first term, as it scales with \(\sqrt{\log(DN\rho_W \rho_B \rho_C \rho_A \rho_{\Delta})}\). Therefore, we conclude that our generalization bound scales with \(\mathcal{O}(D^{2} \rho_W \rho_B \rho_C \rho_A \rho_{\Delta} \cdot \frac{K}{(K-1)^2} \sqrt{\frac{\beta}{m}})\). This implies that the bound is largely unaffected by the sequence length $L$, making it applicable to various sequence lengths without being significantly impacted by the length. 

To understand our bound, we analyze its terms. %Note that 
First note that from the standard assumption that the data is normalized, we have,
%
%
\vspace{-8pt}
{\small
\begin{equation}
%\begin{aligned}
% \begin{equation}  
\sqrt{ \max_{t,k} \sum_{j=1}^m (X_{(j)_{tk}})^2} \leq \sqrt{m}   
% \end{equation}
%\end{aligned}
\vspace{-2pt}
\end{equation}
}
%
% 
% Let's assume the tokens are drawn from a normal distribution centered around 0, which is reasonable due to Gaussian modeling in high-dimensional settings. Given this assumption, the sum of the tokens has a mean of 0, so the expected value of the above sum is also 0, indicating that, on average, the it tends to be close to 0.
% follows from our assumption about the data, which is standard, as we typically normalize the data. })
Regarding the term \(\sqrt{\log(\mathcal{C}DNL)}\), even when choosing exceptionally large values for \(D\) or \(L\), such as \(2^{100}\), the impact on the bound remains minimal. The second term in the bound (see Thm~\ref{theorem:genbound}) is typically smaller than the first term, as it scales with \(\sqrt{\log(DN\Gamma(w))/m}\). Therefore, we conclude that our generalization bound scales with \(\mathcal{O}\left(\frac{1}{\sqrt{m}}D^{2} \Gamma(w) \cdot  \frac{K}{(K-1)^2}\right)\). This implies that the bound is largely unaffected by the sequence length \(L\), making it applicable to various sequence lengths without being significantly impacted by them. Note that since \( K < 1 \), when \( K \) is small, the term \(\frac{1}{1-K}\) approaches 1, further reducing its impact on the bound. This implies that for very small \( K \), the generalization bound becomes even tighter.
%
% Regarding the term {\color{black} \(\sqrt{\log (\mathcal{C} 2^{3(H-1)} D^4 N L^{H})}\)}, even when choosing exceptionally large values for \(L\) or \(D\), such as \(2^{100}\), the impact on the bound remains minimal. {\color{black}The bound does scale with $\sqrt{H}$ so as $H$ increases, the impact grows, but at a slow rate.} The second term in the bound (see Theorem~\ref{thm:genbound}) is typically smaller than the first term, as it scales with \(\sqrt{\log(DN\Gamma(w)\rho_W(w))}\). Therefore, we conclude that our generalization bound scales with \(\mathcal{O}\left(\frac{\sqrt{H}R^{-H} D^{H}}{\sqrt{m}} \Gamma(w) \rho_W(w) \cdot \prod_{h=1}^H \frac{K_h}{(K_h-1)^2}\right)\).
% This implies that choosing $R = D$  as the normalizing constant leads to potential improved generalization.
% This implies that the bound is largely unaffected by the sequence length \(L\), making it applicable to various sequence lengths without being significantly impacted by them. Note that since \( K < 1 \), when \( K \) is very small, the term \(\frac{K}{(K-1)^2}\) approaches 0, further reducing its impact on the bound. This implies that for very small \( K \), the generalization bound becomes even tighter.




\vspace{-7pt}
\section{Experiments}\label{sec:experiments}
\vspace{-4pt}
% \setcounter{section}{3}
In this section, we extensively validate our theorems and assumptions through empirical analysis. First, in Sec.\ref{sec:modelJustification},
we demonstrate that our simplified variant of the Mamba layer achieves performance comparable to the original Mamba layer when incorporated into standard settings and deep networks, thereby justifying the exploration of this variant. Next, in Sec.\ref{sec:LearningPoly},
we validate our theory on expressiveness by showing that self-attention struggles to learn high-degree multivariate polynomials, which S6 can model effectively. %Finally, in Sec.~\ref{sec:empiricalBounds}, we assess our generalization bounds empirically.

\vspace{-5pt}
\subsection{Model Justification\label{sec:modelJustification}}
\vspace{-3pt}
Our theoretical study employs the simplified S6 variant described in Eq.~\ref{eq:simplifiedModel}. We conduct experiments in both the NLP and vision domains, evaluating this variant when integrated into the Mamba backbone, with the goal of showing that it performs similarly to the original S6 layer.
%We start by justifying the exploration of this variant and demonstrating its comparability to the original selective SSMs when integrated into a standard Mamba backbone and training process. 



% _________________________________________________
% \medskip
{\noindent\textbf{NLP\quad}}
In NLP, we trained variants of our simplified S6 layer within Mamba backbones on the Wikitext-103 dataset using a self-supervised scheme for \textit{Next Token Prediction}. Our models feature 12 layers with a hidden dimension size of 386 and were trained with a context length of 1024 tokens. The final results are detailed in right panel of Tab.~\ref{tab:empricalModelJustifiction} and in the right panel of Fig.~\ref{fig:modelJustifications}, which illustrates the evolution of test-perplexity across epochs. Evidently, our simplified S6 variant performs well in the NLP domain, with only a slight reduction in perplexity with respect to the original model. Specifically, the polynomial variant achieved a perplexity score of 26.42, 0.69 points lower than its original baseline score of 25.73. In contrast to the polynomial variant, the other simplified variants that employ $\bar{B}_i = B_i$ achieve a slightly lower perplexity %score
compared to the baseline, highlighting the significance of this aspect in the architecture.

% \medskip
{\noindent\textbf{Vision\quad}} %
Image classification experiments are conducted on the ImageNet-100 benchmark. We built upon the Vision-Mamba (ViM) architecture~\cite{mambaViT1}, replacing the S6 layers with our simplified variant while maintaining the same training procedures and hyper-parameters. The left panel of Tab.~\ref{tab:empricalModelJustifiction} presents the results: the simplified variant from Eq.~\ref{eq:simplifiedModel} achieves a %top-1 
accuracy of 78.62\%, which is 2.4\% lower than the original model which achieve a score of $81.02$. For reference, we include the results of DeiT~\cite{touvron2021training}, which achieved a top-1 accuracy of 78.21\% for the same model size%on this benchmark
~\cite{baron2024a}. %
%
% To understand which aspects of our simplification have the most dominant impact of the decreasing in performance, we compare additional two variations. First, we employ a polynomial S6 variant without omitting the discretization, and second we run a model with standard discretization, however, not polynomial. Our empirical analysis reveals that the polynomial model introduce surprisingly strong performance, with a accuracy score of $80.28$, just 0.74 behind the original model and 0.92 above the the non-polynomial sinplified model.
%


\begin{table*}[]
    \centering
    \small
    \vspace{-11pt}
    \caption{Ablations of our simplified S6 variants, with vision tasks on the left and NLP on the right. 'S' for simplified variants. Results for Transformer models are provided as a reference point.}
    \smallskip
    \label{tab:empricalModelJustifiction}
    \begin{tabular*}{0.48\linewidth}{@{\extracolsep{\fill}}lcc}
        \toprule
        Model & Top-1 & \# Parameters \\
        \midrule
        ViM (baseline) & 81.02 & 6.2M  \\
        ViM (S., $\bar{B}_i = B_i$) & 79.36 & 6.2M\\
        ViM (S., $\bar{A}_i = p_A(S_{\Delta}(x_i))$) & 80.28 & 6.2M \\
        ViM (S., Eq.~\ref{eq:simplifiedModel}) & 78.62 & 6.2M  \\
        Transformer (DeiT) & 78.21 & 6.2M  \\
        \bottomrule
    \end{tabular*}
    \hfill
    \begin{tabular*}{0.49\linewidth}{@{\extracolsep{\fill}}lcc}
        \toprule
        Model & PPL & \# Parameters \\
        \midrule
        Mamba (baseline) & 25.73 & 30.1M  \\
        Mamba (S., $\bar{B}_i = B_i$) & 29.49 & 30.1M \\
        Mamba (S., $\bar{A}_i = p_A(S_{\Delta}(x_i))$) & 26.42 & 30.1M \\
        Mamba (S., Eq.~\ref{eq:simplifiedModel}) & 31.12 & 30.1M  \\
        Transformer & 28.31 & 31.4M  \\
        \bottomrule
    \end{tabular*}
    \vspace{-7pt}
\end{table*}



To identify which aspects of our simplification most significantly impact performance, we compare two additional variations. First, we use a polynomial S6 variant without omitting the discretization. Second, we run a vanilla non-polynomial model where be $\bar{B}_i = B_i$. Our empirical analysis reveals that the polynomial model performs remarkably well, achieving an accuracy score of 80.28, just 0.74 points below the original model and 0.92 points above the non-polynomial simplified model. To provide a comprehensive view, the training curves are presented in left panel of Fig.~\ref{fig:modelJustifications}, which also empirically analyzes several variants of the simplified model compared to the baseline. The full set of hyper-parameters can be found in the Appendix at Tab.~\ref{tab:Vsionhyperpams}.% at Appendix~\ref{sec:hyperParams}.


% \begin{table}[h!]
% \scriptsize
% \caption{Ablations of our simplified SSSL variants {\color{black}Waiting for final results in NLP}}
% \label{tab:validateTheorem2}
% \centering
% \begin{tabular*}{\linewidth}{@{\extracolsep{\fill}}lcc}
% \toprule
% Model &  Top-1  &  \# Of Parameters \\
% \midrule
% Vision Mamba (baseline) & 80.56 & TBD  \\
% Vision Mamba (simplified variant, $\bar{B}_i = B_i$ ) & 78.24 \\
% Vision Mamba (simplified variant, $\bar{A}_i = p_3(S_{\Delta} (x_i) )$ ) & 78.24 \\
% Vision Mamba (simplified variant, Eq.~\ref{eq:simplifiedModel} ) & 78.24 & TBD  \\
% DeiT & 78.21 & TBD  \\
% \bottomrule
% \end{tabular*}
% \end{table}






% \begin{figure}[h]
% \centering
% \begin{minipage}{0.49\textwidth}
%     \centering
%     \includegraphics[width=0.85\textwidth]{figures/vim_acc.png}
%     %\caption{Model ablations for our simplified model over image classification.}
%     %\label{fig:modelJustificationsVison}
% \end{minipage}
% % \hspace{0.02\textwidth}
% \begin{minipage}{0.49\textwidth}
%     \centering
%     \includegraphics[width=0.85\textwidth]{figures/ppl_test.png}
%     %\caption{Model ablations for our simplified model over image classification.}
%     %\label{fig:modelJustificationsNLP}
% \end{minipage}
% \caption{\textbf{Model justifications \& ablations: }In the \textbf{left} panel, we present the top-1 accuracy %score
% for image classification via the ImageNet-100 % benchmark
% , while the \textbf{right} panel displays the perplexity %score
% for language modeling using the WikiText-103. The y-axis represents the model's score across different epochs. In both figures, the blue curve represents the baseline, the yellow curve corresponds to Eq.\ref{eq:simplifiedModel}, the green curve illustrates Eq.\ref{eq:model}, and the red curve depicts the polynomial variant using standard discretization.
% % In both figures, the blue curve shows the baseline, the yellow curve represents Eq.\ref{eq:simplifiedModel}, the green curve corresponds to Eq.\ref{eq:model}, and the red curve depicts the polynomial variant with standard discretization.
% }
% \label{fig:modelJustifications}
% \end{figure}

\begin{figure}[t]
\vspace{-6pt}
\centering
\begin{tabular}{cc}
    \includegraphics[width=0.225\textwidth]{figures/vim_acc.png} &
    \includegraphics[width=0.225\textwidth]{figures/ppl_test.png} \\
\end{tabular}
\vspace{-8pt}
\caption{\textbf{Model justifications \& ablations: }In the \textbf{left} panel, we present the top-1 accuracy score
for image classification via the ImageNet-100 benchmark, while the \textbf{right} panel displays the perplexity score
for language modeling using the WikiText-103. The y-axis represents the model's score across different epochs. In both figures, the blue curve represents the baseline, the yellow curve corresponds to Eq.\ref{eq:simplifiedModel}, the green curve illustrates Eq.\ref{eq:model}, and the red curve depicts the polynomial variant using standard discretization.
% In both figures, the blue curve shows the baseline, the yellow curve represents Eq.\ref{eq:simplifiedModel}, the green curve corresponds to Eq.\ref{eq:model}, and the red curve depicts the polynomial variant with standard discretization.
}
\vspace{-9pt}
\label{fig:modelJustifications}
\end{figure}
% \vspace{-2pt}

\vspace{-4pt}
\subsection{Learning Polynomials\label{sec:LearningPoly}}
\vspace{-2pt}

In this section, we empirically validate our theoretical claims concerning the expressiveness of S6 and self-attention layers from Thms.~\ref{theorem:exprrsFull} and ~\ref{theorem:AnyPolywithMambas}. Since our analysis of the expressiveness gap between those layers relies on their characterization via multivariate polynomials, we focus on learning such functions over synthetic data. To isolate factors other than expressiveness, we employ a control setup with small NNs, comprising up to four layers with narrow widths (2 or 4 channels) and an additional output linear projection head. For each architecture, we used the standard implementations: (i) self-attention with softmax and positional encoding, and (ii) the original S6 architecture~\cite{gu2023mamba} with and without PE. The experiments examine the clean implementation of these components, without additional elements such as Conv1D, activations, or FFNs. We conduct experiments on two tasks: classification and regression. %Details of the hyperparameters are provided in Appendix TBD. 

% \medskip
{\noindent\textbf{Classification}\quad} Our dataset consists of binary random sequences %, each
of length $L=20$. The labels are uniformly distributed between 0 and $L$, determined using the ``Count in Row'' function~\cite{ali2024hidden}, defined as follows:
    \begin{definition}
        The count in row problem: Given a binary sequence $x_1, x_2, \ldots x_L \in  \{0,1\}^L$ such that $x_i \in \{0,1\}$ %for all $i \leq L$%
        , the ``count in row'' function $f$ is defined to produce an output sequence $y_1, y_2, \ldots, y_L$, where each $y_i = f(x_1, .. , x_i)$ is determined based on the contiguous subsequence of 1s to which $x_i$ belongs. Formally: %\footnote{where $[x_k > 0]$ is the Iverson bracket, equaling 1 if $x_k> 0$ and 0 otherwise.}:%
        \vspace{-7pt}
        {\small
        \begin{equation}
        y_i = \max_{0\leq j \leq i} \Big{(} \{ i-j+1 \mid \prod_{k=j}^i [x_k > 0] = 1\} \cup \{ 0\} \Big{)}
        \end{equation}
        }
        %
        %, capturing the essence of "counting in a row" by identifying the length of the contiguous sequence of 1s ending at $x_i$ if $x_i = 1$.
        where $[x_k > 0]$ is the Iverson bracket, equaling 1 if $x_k> 0$ and 0 otherwise.
    \end{definition}
    \vspace{-5pt}
    
The top part in Tab.~\ref{tab:EmpricalExpressClassification} presents the results. Remarkably, even a single layer of selective SSMs, both with and without PE, outperforms attention models with double the number of layers and channels, all while utilizing significantly fewer parameters, as suggested by Thm~\ref{theorem:exprrsFull}.


% \begin{table}[h!]
% \scriptsize
% \caption{Learning Polynomials:'SA' for self-attention, 'SSSM' for selective SSM, 'k' for the number of layers. Results are averaged over 3 seeds.}
% \label{tab:validateTheorem2}
% \centering
% \begin{tabular*}{\linewidth}{@{\extracolsep{\fill}}llccc}
% \toprule
% \multicolumn{5}{c}{Classification}\\
%  SSSM (K=1) &  SSSM (K=2)  &  SA (K=1) & SA (K=2) & SA (K=4) \\
% \midrule
% {83.1\%} & \textbf{92.3\%} & {32.9\%} & {41.1\%}& {44.8\%}\\
% \bottomrule
% \multicolumn{5}{c}{Regression}\\
% \midrule
% 0 & 0 &0 & 0& 0\\
% \bottomrule
% \end{tabular*}
% \end{table}

\begin{table}[t]
\vspace{-8pt}
\caption{\small\textbf{Learning multivariate polynomials over synthetic data.} Classification results are presented in the top, while regression results are displayed on the bottom. Best results for each model depth in bold. %'Acc' for Accuracy, 'Params' for parameter count, 'S.-Attn' for Self-Attention and
'$D$' for the number of channels.}
\smallskip
\small
% \begin{minipage}[]{0.495\linewidth}
% \caption{TBD}
\label{tab:EmpricalExpressClassification}
\centering
\begin{tabular*}{\linewidth}{@{\extracolsep{\fill}}lccc}
\toprule
\multicolumn{4}{c}{Classification}\\
\toprule
 Model &   \# Layers &  Accuracy  &  \# Parameters \\
\midrule
S6 w/ PE $(D=2)$ & 1  & 83.1 & 35 \\
S6 w/o PE $(D=2)$ & 1 & \textbf{84.8} & 35 \\
S6 w/ PE $(D=2)$ & 2 & 93.4 & 63 \\
S6 w/o PE $(D=2)$ & 2 & \textbf{97.1} & 63 \\
\midrule
Self-Attention$(D=2)$ & 1 & 32.9 & 29 \\
Self-Attention$(D=2)$ & 2 & 41.1 & 51 \\
Self-Attention$(D=2)$ & 4 & 44.8 & 95 \\
Self-Attention$(D=4)$ & 1 & 36.2 & 176 \\
Self-Attention$(D=4)$ & 2 & 44.2 & 244 \\
Self-Attention$(D=4)$ & 4 & 55.8 & 380 \\
% {83.1\%} & \textbf{92.3\%} & {32.9\%} & {41.1\%}& {44.8\%}\\
\bottomrule
\end{tabular*}
% \end{table} 
% \end{minipage}
\hfill
% \begin{minipage}[]{0.48\linewidth}
%\begin{table}%{r}{0.5\linewidth}
% \caption{TBD}
\label{tab:CelebA}
%\small
\centering
\begin{tabular*}{\linewidth}{@{\extracolsep{\fill}}llcc}
\toprule
\multicolumn{4}{c}{Regression}\\
\toprule
 Model &   D  &  MSE  & \# Parameters \\
\midrule
S6 w/ PE & 4 & 12.67  & 101 \\
S6 w/o PE & 4 & \textbf{11.81} & 101   \\
S6 w/ PE & 6 & 12.45  & 157 \\
S6 w/o PE & 6 & \textbf{11.04} & 157   \\
S6 w/ PE & 8 & 12.17  & 377 \\
S6 w/o PE & 8 & \textbf{9.057} & 377 \\
\midrule
Self-Attention & 4 & 19.22   & 81\\
Self-Attention & 6 & 19.10   & 157\\
Self-Attention & 8 & 19.048  & 257\\
% {83.1\%} & \textbf{92.3\%} & {32.9\%} & {41.1\%}& {44.8\%}\\
\toprule
\end{tabular*}
% \end{minipage}
\vspace{-19pt}
\end{table}


{\noindent\textbf{Regression}\quad} We synthetically construct the dataset \( S = \{(x_i, y_i)\}_{i = 1}^m \)
by first randomly selecting a polynomial denoted by P. For each example in the dataset, we then generate $x$ values uniformly at random and compute the corresponding labels using this P.
% \vspace{-4pt}
{\small
\begin{equation}
\vspace{-3pt}
   %\quad 
   c_i \sim \mathbb{U}([-2,2]), \text{  } p_{i,j}\sim \mathbb{U}([ L ]), \text{  } x_j \sim \mathbb{U}([0.1,2])
   \vspace{-4pt}
\end{equation}
}
\vspace{-6pt}
% \vspace{-9pt}
{\small
\begin{equation}\label{eq:randomPoly}
\vspace{-4pt}
%\small
   y = P(x) = \sum_{i = 1}^3 c_i \Pi_{j=1}^L {x_j}^{p_{i,j}}% \quad p_{i,j}\sim \mathbb{U}([ L ]), \quad 
    % \end{equation}
% \vspace{-3pt}
% \begin{equation}\label{eq:assumptionsforsyntData}
   %,
\end{equation}
}

Our models consist of a single layer with either 4 or 8 channels, processing sequences of length $L=5$. As demonstrated in the bottom part of Tab.~\ref{tab:EmpricalExpressClassification}, both S6 variants, with and without PE, significantly outperform traditional self-attention layers across both model sizes. For example, while a self-attention model with 8 channels obtains an MSE score of 19.05, all S6 variants achieve an MSE below 12.67. These experiments demonstrate that at least in controlled environments with small models, S6 layers outperform traditional self-attention layers in approximating polynomials where the total multivariate degree exceeds the sequence length.
%
%
%
%
% \subsection{{\color{black} Empirical Generalization Bounds}\label{sec:empiricalBounds}}
% In this section, we empirically evaluate the generalization bounds derived in section~\ref{thm:genbound}. We focus on simple one-layer S6 neural networks trained on MNIST, exploring how the bound behaves with different hyperparameters. Our models consist of a single layer with $D$ channels, processing sequences of length $L = 784$. For our calculations, we used $\delta = 0.001$ and $\gamma = 1$.
%
% \begin{tabular}{|c|c|c|c|c|c|c|c|}
% \hline
% D, N & $K$ & generalization bound & $\rho_A(w)$ & $\rho_B(w)$ & $\rho_C(w)$ & $\rho_{\Delta}(w)$ & $\rho_W(w)$\\
% \hline
% 4 & 0.93 & 324.74 & 0.28 & 6.32 & 9.4 & 0.89 & 22.82\\
% 16 & 0.94 & 441.15 & 0.17 & 7.95 & 18.71 & 0.92 & 19.47\\
% 32 & 0.97 & 1585.57 & 0.24 & 13.46 & 25.66 & 0.96 & 20.79 \\
% \hline
% \end{tabular}
%
% The table above shows our results, confirming that the value of $K$ is consistently less than $1$ in all tested cases. This aligns with our assumptions in the theorem, providing further evidence for its validity. 
%




\vspace{-6pt}
\section{Discussion}
\vspace{-2pt}
To understand the implications of our theory, we first explain why analyzing Softmax-free attention is realistic. Then, we discuss the consequences for standard attention models.

\noindent\textbf{Transformers Without Softmax\quad} The softmax function is primarily associated with optimization and stability, as it normalizes attention scores to the [0, 1] range and prevents numerical instabilities. However, transformer variants without softmax have proven effective in several domains, including reducing latency~\cite{hua2022transformer,lu2021soft,ramapuram2024theory} and in applications such as vision~\cite{wortsman2023replacing}, NLP~\cite{ma2022mega}, and other areas~\cite{zimerman2023converting}. {\color{black}Additionally, these models have recently become even more practical, as researchers have successfully scaled linear attention far beyond 7B parameters~\cite{li2025minimax}, enabling LLMs to extend their context window to 4 million tokens while matching the performance of GPT-4o and Claude-3.5 Sonnet. Several additional linear attention-based LLMs were presented in~\cite{shen2024scaling, qin2023transnormerllm, sun2023retentive}.}
Since these models achieve near-SoTA, focusing on a softmax-free attention model is well justified.

\noindent\textbf{Transformers With Softmax\quad} While the softmax function can theoretically be expressed as an infinite-degree polynomial, we provide careful considerations. The softmax function involves both exponentiation and proportional normalization. The former can be well approximated using low-degree polynomials~\cite{zhang2024secure}, while the latter primarily serves to normalize the scores. We refine our assumption by analyzing transformers that apply exponentiation to each attention score, assuming this can be approximated by a polynomial of degree P. The resulting model expresses higher-degree polynomials within each layer, but it remains based on pairwise interactions via Key, Query, and Values, leading to an maximal polynomial degree of $3P+1$, independent of the sequence length $L$. This supports the validity of our argument in more common regimes.

%
% \smallskip
%\noindent\textbf{Language Modeling Capabilities\quad}
\noindent\textbf{Interpretation and Intuition}\quad}
Our characterization of S6 layers through the lens of %multivariate
polynomials offers a novel perspective on the semantic capabilities of Mamba. Specifically, we extend the concept of polynomial degree to quantify the number of tokens involved in each interaction within a layer of an model. For instance, low-degree polynomials correspond to interactions involving only a few tokens, while high-degree polynomials represent dependencies spanning many tokens. This analysis highlights the unique strength of S6 layers in modeling continuous, multi-token interactions, such as counting and recurrent operations. In contrast, transformers, % which primarily rely on pairwise token interactions,
are naturally biased toward sparser and more fragmented representations such as induction heads. 

This perspective can also shed light on the remarkable performance of hybrid models that combine modern RNNs and attention by leveraging their complementary strengths~\cite{lieber2024jamba,de2024griffin}. While a formal characterization of their trade-offs is yet to be established, our analysis suggests that S6 and attention capture fundamentally distinct types of interactions, characterized by the number of tokens involved in each interaction.


\vspace{-8pt}
\section{Conclusions}%} \& Limitations}
\vspace{-3pt}
% {\color{blue} I think we're promising too much for future work, we can write that these are promising directions for future work and leave it a bit more vague}
This study explores the expressivity of Mamba models. By reducing the S6 layer to a polynomial form and composing an associated theory, we have established a novel connection between S6 layers and high-degree multivariate polynomials. This connection enables us to identify the expressivity gap between S6 layers and attention mechanisms comprehensively. {\color{black}We show that although the S6 layer has better theoretical expressivity than linear attention for long sequences, this does not negatively impact generalization. We provide a length-agnostic generalization bound to support this result, allowing us to conclude that the S6 layer has superior theoretical properties compared to linear attention for long-range tasks.} %Moving forward, we aim to broaden our theoretical framework to encompass other gated RNNs, including xLSTM~\cite{beck2024xlstm}, RWKV~\cite{peng2023rwkv,peng2024eagle}, and
%and Griffin~\cite{de2024griffin}. %, and HGRN2~\cite{qin2024hgrn2}. 
%Additionally, %we plan to incorporate a more extensive array of components from both Mamba and transformer architectures into our analyses, such as LayerNorm, FFN, Softmax, Conv1D, and gated branches. 
% future research will focus on identifying specific domains where dependencies modeled by high-degree multivariate polynomials are prevalent, thereby assessing whether Mamba models demonstrate superior performance in these contexts. %One initial hypothesis relates to the general product rule~\cite{feller1991introduction} $( P(x_1, \dots, x_L) = \prod_{j=1}^L P(x_j | x_{j-1}))$, which plays a crucial aspect in conditional probability. Such polynomials can be effective in domains like NLP, where it is fundamental to n-gram models, and RL, where it determines the likelihood of trajectories.
Finally, the limitations of our work are discussed in Appendix~\ref{sec:limitations}.
% Regarding generalization, our findings offer insights into providing generalization guarantees for S6 with minimal reliance on network size or sequence length. This research underscores the critical impact of network architecture on testing performance and paves the way for future enhancements and the development of theory-driven regularization techniques, e.g., %several similar examples are
% \cite{ledent2021norm, long2019generalization}. %{\color{black}Our next steps include extending these insights to deeper networks.}
%
%
%
\newpage

% \section*{Impact Statement}
% {\color{black}Our work establishes a stronger theoretical foundation for the expressivity of Mamba layers, demonstrating their superior representational power compared to linear attention. Developing such theoretical insights not only deepens our understanding and trust in these models but also paves the way for more principled, theory-driven advancements in sequence modeling. By bridging empirical success with rigorous analysis, our findings contribute to the design of more effective, expressive, and efficient architectures, driving significant progress in deep sequence modeling.}

\section{Acknowledgments}
This work was supported by a grant from the Tel Aviv University Center for AI
and Data Science (TAD) and the Ministry of
Innovation, Science \& Technology ,Israel (1001576154) and the Michael J. Fox
Foundation (MJFF-022407). This research was also supported by the European Research Council (ERC) under the European Unions Horizon 2020 research and innovation programme (grant ERC HOLI 819080).

\nocite{langley00}

\bibliography{example_paper}
\bibliographystyle{icml2025}


%%%%%%%%%%%%%%%%%%%%%%%%%%%%%%%%%%%%%%%%%%%%%%%%%%%%%%%%%%%%%%%%%%%%%%%%%%%%%%%
%%%%%%%%%%%%%%%%%%%%%%%%%%%%%%%%%%%%%%%%%%%%%%%%%%%%%%%%%%%%%%%%%%%%%%%%%%%%%%%
% APPENDIX
%%%%%%%%%%%%%%%%%%%%%%%%%%%%%%%%%%%%%%%%%%%%%%%%%%%%%%%%%%%%%%%%%%%%%%%%%%%%%%%
%%%%%%%%%%%%%%%%%%%%%%%%%%%%%%%%%%%%%%%%%%%%%%%%%%%%%%%%%%%%%%%%%%%%%%%%%%%%%%%
\newpage
\appendix
\onecolumn

% \section{Reproducibility Checklist}
% Please refer to the technical appendix titled 'Reproducibility Checklist', which is attached in the supplementary material of this submission.

\section{Experimental Setup \label{sec:hyperParams}} 
All training experiments were performed
on public datasets using a single A100 40GB GPU for a
maximum of two days. All experiments were conducted using PyTorch, and results are averaged over three seeds. All hyperparameters are detailed in Tab.~\ref{tab:NLPhyperpams} and Tab.~\ref{tab:Vsionhyperpams}.

\begin{table}[h]
\centering
\small
\begin{tabular}{l c}
\toprule
\textbf{Parameter} & \textbf{Value} \\
\midrule
Model-width & 192 \\
Number of layers & 24 \\
Number of patches & 196 \\
%Scan Mode {\color{red} ???} & one directional \\
Batch-size & 512 \\
Optimizer & AdamW \\
Momentum & \( \beta_1, \beta_2 = 0.9, 0.999 \) \\
Base learning rate & $5e-4$ \\
Weight decay & 0.1 \\
Dropout & 0 \\
Training epochs & 300 \\
Learning rate schedule & cosine decay \\
Warmup epochs & 5 \\
Warmup schedule & linear \\ 
Degree of Taylor approx. (Eq.~\ref{eq:simplifiedModel}) & 3 \\
\bottomrule
\end{tabular}
\caption{Hyperparameters for image-classification via Vision Mamba variants} 
\label{tab:Vsionhyperpams}
\end{table}

\begin{table}[h]
\centering
\small
\begin{tabular}{l c}
\toprule
\textbf{Parameter} & \textbf{Value} \\
\midrule
Model-width & 386 \\
Number of layers & 12 \\
Context-length (training) & 1024 \\
Batch-size & 32 \\
Optimizer & AdamW \\
Momentum & \( \beta_1, \beta_2 = 0.9, 0.999 \) \\
Base learning rate & $1.5e-3$ \\
Weight decay & 0.01 \\
Dropout & 0 \\
Training epochs & 20 \\
Learning rate schedule & cosine decay  \\
Warmup epochs & 1  \\
Warmup schedule & linear  \\ 
Degree of Taylor approx. (Eq.~\ref{eq:simplifiedModel}) &  3\\
\bottomrule
\end{tabular}
\caption{Hyperparameters for language modeling via Mamba-based LMs} 
\label{tab:NLPhyperpams}
\end{table}

% \section{Additional Background Material}

% {\noindent\textbf{Rademacher Complexities}}
% We explore the generalization capabilities of overparameterized NNs by analyzing their Rademacher complexity. This measure provides an upper bound on the worst-case generalization gap, which represents the difference between training and testing errors within a specific hypothesis class. It is defined as the expected performance of the class averaged over all possible data labelings, with labels independently and uniformly drawn from the set $\{\pm 1\}$. For further details, refer to \citep{mohri2018foundations, Shalev-Shwartz2014, bartlett2002rademacher}.


% \begin{definition}[Rademacher Complexity] Let $\mathcal{F}$ be a set of real-valued functions $f_w:\mathcal{X} \to \mathbb{R}^\mathcal{C}$ defined over a set $\mathcal{X}$. Given a fixed sample $X = \{ x_j\}_{j=1}^m \in \mathcal{X}^m$, the empirical Rademacher complexity of $\mathcal{F}$ is defined as follows: 
% \begin{equation*}
% \mathcal{R}_{X}(\mathcal{F}) ~:=~ \frac{1}{m} \mathbb{E}_{\xi: \xi_{ic} \sim U[\{\pm 1\}]} \left[ \sup_{f_w \in \mathcal{F}} \sum^{m}_{j=1}\sum^{\mathcal{C}}_{c=1} \xi_{ic} f_w(x_j)_c  \right].
% \end{equation*}
% \end{definition}

% {\color{red}
% It should be moved:
% We focus on training models for classification tasks. The problem is formally defined by a distribution $P$ over pairs $(x,y)\in \cX\times \cY$, where $\cX \subset \R^{\mathcal{D}}$ represents the input space, and $\cY \subset \R^\mathcal{C}$ is the label space containing one-hot encoded labels for the integers $1,\ldots,\mathcal{C}$. 

% We define a hypothesis class $\cF \subset \{f_w:\cX\to \R^\mathcal{C}\}$ (such as a neural network architecture), where each function \( f_w \in \cF \) is parameterized by a vector of trainable weights $w$. Given any input \( x \in \cX \), \( f_w \) provides a predicted label, and the performance is evaluated based on the \emph{expected error}, \(\err_P(f_w) := \mathbb{E}_{(x, y) \sim P}\left[\mathbb{I}\left[\max_{j \neq y}(f_w(x)_j) \geq f_w(x)_{y}\right]\right]\). Here, the indicator function \(\mathbb{I}\) returns 1 for True and 0 for False.

% Since the full distribution \( P \) is not directly accessible, our objective is to train a model \( f_w \) using a training dataset \( S = \{(x_i, y_i)\}_{i = 1}^m \) consisting of independent and identically distributed (i.i.d.) samples from \( P \). We aim to achieve accurate predictions while applying regularization to control the complexity of \( f_w \). 

% To denote the entire $n$th row and $n$th column of a matrix $M$, we use the notation:
% \begin{align*}
% M_{n*} & \text{ denotes the entire } n \text{th row of matrix } M. \\
% M_{*n} & \text{ denotes the entire } n \text{th column of matrix } M. \\
% M_{nk} & \text{ denotes the element in the } n \text{th row and } k \text{th column.}
% \end{align*}
% {\bf Selective State Space Models.\enspace} 
% Time-variant SSMs, namely, the matrices $A,B,C$ of each channel are modified over $L$ time steps. We are focusing on selective SSMs of the following form. 
% A neural network $f_w : \mathbb{R}^{D \times L} \rightarrow \mathbb{R}^{\mathcal{C}}$ takes a sequence $x=(x_{*1},...,x_{*L}) \in \mathbb{R}^{D \times L}$ as input where $L$ is the length of the sequence and $D$ is the dimension of the tokens $x_i$. We denote
% \begin{align*}
%     B_i &= B x_{*i}, B \in \mathbb{R}^{N \times D} \\
%     C_i &= C x_{*i}, C \in \mathbb{R}^{N \times D} \\
%     \Delta_{*i} &= S_{\Delta} x_{*i}, S_{\Delta} \in \mathbb{R}^{D \times D} \rightarrow \Delta \in \mathbb{R}^{D \times L} \\
%     \bar{A}_{j,i} &= (z^2 + \alpha z)(\Delta_{j,i} * \underbrace{A_{j*}}_{\textbf{ $j$th row of A } } ), A \in \mathbb{R}^{D \times N}, \text{$(z^2+\alpha z)$ is an activation function, } \alpha \geq 0 \\ 
%     W &\in \mathbb{R}^{\mathcal{C} \times D} 
% \end{align*}
% }


\section{Proof of Fact~\ref{fact:inv:soln:exists}} \label{appendix:proof:inv:soln}

\noindent
Fix $g \in \G$ with its action on $x \in \R^3$ represented by $g(x) = A x + b$. Given that $(\psi, E)$ solves \eqref{eq:schrodinger}, we seek to show that under the stated conditions, $\psi_g(\bx) \coloneqq \psi(g(\bx))$ is also an anti-symmetric solution to \eqref{eq:schrodinger} with respect to the same energy $E$. For $\bx \in \R^{3n}$ and $g \in \G$, denote $\bx_g = g(\bx)$. Then for every $\bx \in \R^{3n}$, 
\begin{align*}
    \Big( - \mfrac{1}{2} \nabla^2 + V(\bx) \Big) \, \psi_{g}( \bx )
    \;=&\; 
    \Big( - \mfrac{1}{2} \nabla^2 + V(\bx) \Big) \, \psi(  A x_1 + b, \ldots, A x_n + b )
    \\
    \;=&\;  
    - \mfrac{1}{2} (A^\top A) \nabla^2  \psi( \bx_{g} )
    + V( \bx ) \psi( \bx_{g} )
    \\
    \;\overset{(a)}{=}&\;
    - \mfrac{1}{2} \nabla^2  \psi( \bx_{g})
    + V( \bx_{g} ) \psi( \bx_{g})
    \;\overset{(b)}{=}\; 
    E \psi( \bx_{g} )
    \;=\; 
    E \psi_{g}( \bx )
    \;.
\end{align*}
In $(a)$, we have used the $\G_\diag$-invariance of $V$;  in $(b)$, we used that $(\psi, E)$ solve \eqref{eq:schrodinger}. Since the above holds for all $\bx \in \R^{3n} $, we get that $(\psi_{g},E)$ also solves the Schr\"odinger's equation. To verify the anti-symmetric requirement, we write the wavefunction $\psi$ as $\psi(\tilde \bx) = \psi(\tilde x_1, \ldots, \tilde x_n)$, where each $\tilde x_i = (x_i, \sigma_i)$ now additionally depends on the spin $\sigma_i \{ \uparrow, \downarrow\}$. Since $\G$ only acts on the spatial position in $\R^3$ and leaves the spins invariant, we can WLOG express the $g$-transformed version of $\tilde x_i$ as $g(\tilde x_i) = (g(x_i), \sigma_i)$. Moreover, since $\sigma \in P_n$ commutes with the diagonal action of $g \in \G$ on $(\R^3 \times \{\uparrow, \downarrow\})^n$, 
\begin{align*}
    \psi_g( \sigma(\tilde \bx) )
    \;=\; 
    \psi( g \circ \sigma (\tilde \bx) )
    \;=\;
    \psi( \sigma \circ g(\tilde \bx) ) 
    \;\overset{(c)}{=}\; 
    \textrm{sgn}(\sigma) \, \psi( g(\tilde \bx) ) 
    \;=\; 
    \textrm{sgn}(\sigma) \, \psi_g( \tilde \bx ) \;.
\end{align*}
In $(c)$, we have used the anti-symmetry of $\psi$. This proves that $(\psi_{g},E)$ solves \eqref{eq:schrodinger} with the correct anti-symmetric requirement. To prove the theorem statement, we see that $\psi^\G$ is a linear combination of finitely many eigenfunctions $(\psi_{g})_{g \in \cG}$ with the same eigenvalue $E$ and therefore yields an anti-symmetric solution with the same energy $E$. 
\qed 


\section{Proofs for data augmentation and group-averaging}  \label{appendix:proof:DA:GA}

\subsection{Proof of Proposition~\ref{prop:DA}}
We first compute the difference in expectation as
\begin{align*}
    \| \mean[ \delta \theta^{(\rm DA)}] - \mean[\delta \theta^{(\rm OG)}] \|
    \;=&\;
    \Big\| 
        \mean\Big[ \mfrac{1}{N} \msum_{i \leq N/k} \msum_{j \leq k}
    F_{\bg_{i,j}(\bX_i); \psi_\theta} \Big] 
        - 
        \mean\Big[ \mfrac{1}{N} \msum_{i \leq N} F_{\bX_i; \psi_\theta} 
        \Big] 
    \Big\|
    \\
    \;=&\;
    \| \mean[  F_{\bg_{1,1}(\bX_1); \psi_\theta} ] - \mean[  F_{\bX_1; \psi_\theta} ] \|
    \;=\; 
    \Big\| \mean_{\bX \sim p^{(m)}_{\psi_\theta; {\rm DA}}}[  F_{\bX; \psi_\theta} ] - \mean_{\bY \sim p^{(m)}_{\psi_\theta}}[  F_{\bY; \psi_\theta} ] \Big\|
    \;.
\end{align*}
In the last line, we used linearity of expectation, the fact that $\bg_{i,j}(\bX_i)$'s are identically distributed and that $\bX_i$'s are identically distributed. Recall that $ \big\{ 
    \bx \mapsto 
    \big( 
        F_{\bx;\psi_\theta}
        \,,\,
        F_{\bx;\psi_\theta}^{\otimes 2}
    \big)
    \,\big|\, 
    \theta \in \R^q
    \big\} 
    \,\subseteq\, \cF$
and $d_\cF(p,q) = \sup_{f \in \cF} \| \mean_{\bX \sim p}[f(\bX)] -  \mean_{\bY \sim q}[f(\bY)] \|$. This implies that
\begin{align*}
    \| \mean[ \delta \theta^{(\rm DA)}] - \mean[ \delta \theta^{(\rm OG)}] \|
    \;=\;
    \Big\| \mean_{\bX \sim p^{(m)}_{\psi_\theta; {\rm DA}}}[  F_{\bX; \psi_\theta} ] - \mean_{\bY \sim p^{(m)}_{\psi_\theta}}[  F_{\bY; \psi_\theta} ] \Big\|
    \;\leq\; 
    d_\cF \big(  p^{(m)}_{\psi_\theta; {\rm DA}},  p^{(m)}_{\psi_\theta} \big),
\end{align*}
which proves the first bound. For the second bound, notice that 
\begin{align*}
    \Var[ \delta \theta^{(\rm DA)} ]
    \;=&\; 
    \Var\Big[ 
        \mfrac{1}{N} \msum_{i \leq N/k} \msum_{j \leq k}
        F_{\bg_{i,j}(\bX_i); \psi_\theta}
    \Big]
    \\
    \;\overset{(a)}{=}&\;
    \mfrac{1}{N/k}
    \Var\Big[ 
        \mfrac{1}{k} \msum_{j \leq k}
        F_{\bg_{1,j}(\bX_1); \psi_\theta}
    \Big]
    \\
    \;\overset{(b)}{=}&\;
    \mfrac{1}{N}
    \Var[ F_{\bg_{1,1}(\bX_1); \psi_\theta}]
    +
    \mfrac{k-1}{N} 
    \Cov[  F_{\bg_{1,1}(\bX_1); \psi_\theta},  F_{\bg_{1,2}(\bX_1); \psi_\theta} ]
    \\
    \;\overset{(c)}{=}&\;
    \mfrac{1}{N}
    \Var[ F_{\bg_{1,1}(\bX_1); \psi_\theta}]
    +
    \mfrac{k-1}{N} 
    \Var \, \mean\big[ F_{\bg_{1,1}(\bX_1); \psi_\theta} \big| \bX_1 \big]
    \;.
\end{align*}
In $(a)$, we have noted that the summands are i.i.d.~across $i \leq N/k$; in $(b)$, we have computed the variance of the sum explicitly by expanding the expectation of a double-sum; in $(c)$, we have applied the law of total covariance to obtain that 
\begin{align*}
    &\;\Cov[  F_{\bg_{1,1}(\bX_1); \psi_\theta},  F_{\bg_{1,2}(\bX_1); \psi_\theta} ]
    \\
    &\hspace{5em}
    \;=\;
    \underbrace{\Cov\big[ 
        \mean[ F_{\bg_{1,1}(\bX_1); \psi_\theta} | \bX_1 ]
        \,,\,  
        \mean[ F_{\bg_{1,2}(\bX_1); \psi_\theta} | \bX_1 ] 
    \big]}_{ =  \, \Var \, \mean[ F_{\bg_{1,1}(\bX_1); \psi_\theta} | \bX_1 ] }
    +
        \mean \, 
        \underbrace{\Cov\big[ F_{\bg_{1,1}(\bX_1); \psi_\theta} \,,\, F_{\bg_{1,2}(\bX_1); \psi_\theta} \,\big|\, \bX_1 \big]
    }_{ = 0}
    \;.
\end{align*}
Meanwhile, the same calculation with $k=1$ and $\bg_{1,1}$ replaced by identity gives 
\begin{align*}
    \Var[ \delta \theta^{(\rm OG)} ]
    \;=\;
    \mfrac{1}{N}
    \Var[ F_{\bX_1; \psi_\theta}]
    \;.
\end{align*}
Taking a difference and applying the triangle inequality twice, we have 
\begin{align*}
    &\; 
    \Big\| \Var[ \delta \theta^{(\rm DA)} ] - \Var[ \delta \theta^{(\rm OG)} ] - \mfrac{k-1}{N} 
    \Var \, \mean\big[ F_{\bg_{1,1}(\bX_1); \psi_\theta} \big| \bX_1 \big] 
    \Big\| 
    \;=\;
    \mfrac{1}{N} \big\|  \Var[ F_{\bg_{1,1}(\bX_1); \psi_\theta}] - \Var[ F_{\bX_1; \psi_\theta}] \big\|
    \\
    &\;\leq\;
    \mfrac{1}{N} \big\|  \mean\big[ F_{\bg_{1,1}(\bX_1); \psi_\theta}^{\otimes 2}\big] - \mean\big[ F_{\bX_1; \psi_\theta}^{\otimes 2} \big] \big\|
    +
    \mfrac{1}{N} \big\|  \mean\big[ F_{\bg_{1,1}(\bX_1); \psi_\theta}\big]^{\otimes 2} - \mean\big[ F_{\bX_1; \psi_\theta} \big]^{\otimes 2} \big\|
    \\
    &\;\leq\;
    \mfrac{1}{N} \big\|  \mean\big[ F_{\bg_{1,1}(\bX_1); \psi_\theta}^{\otimes 2}\big] - \mean\big[ F_{\bX_1; \psi_\theta}^{\otimes 2} \big] \big\|
    +
    \mfrac{1}{N} \big\|  \mean\big[ F_{\bg_{1,1}(\bX_1); \psi_\theta}\big]+  \mean\big[ F_{\bX_1; \psi_\theta} \big] \big\| \, \big\|  \mean\big[ F_{\bg_{1,1}(\bX_1); \psi_\theta}\big] -  \mean\big[ F_{\bX_1; \psi_\theta} \big] \big\|
    \\
    &\;\leq\;
    \mfrac{d_\cF \big(  p^{(m)}_{\psi_\theta; {\rm DA}},  p^{(m)}_{\psi_\theta} \big)}{N}
    +
    \mfrac{1}{N} \big( 2 \big\| \mean\big[ F_{\bX_1; \psi_\theta} \big] \big\| + d_\cF \big(  p^{(m)}_{\psi_\theta; {\rm DA}},  p^{(m)}_{\psi_\theta} \big) \big) \, \times \, d_\cF \big(  p^{(m)}_{\psi_\theta; {\rm DA}},  p^{(m)}_{\psi_\theta} \big)
    \\
    &\;=\;
    \mfrac{ 1 + 2 \| \mean[ \delta \theta^{(\rm OG)} ] \| + d_\cF (  p^{(m)}_{\psi_\theta; {\rm DA}},  p^{(m)}_{\psi_\theta} )   }{N} \, \times \,  d_\cF \big(  p^{(m)}_{\psi_\theta; {\rm DA}},  p^{(m)}_{\psi_\theta} \big)
    \;. \tagaligneq \label{eq:var:analysis:DA}
\end{align*} 
This proves the second bound. In the case when $\bg(\bX_1) \overset{d}{=} \bX_1$ for all $\bg \in \Gdiag$, we have $p^{(m)}_{\psi_\theta; {\rm DA}} = p^{(m)}_{\psi_\theta}$ and therefore the bounds above all evaluate to zero. In this case we have $\mean[ \delta \theta^{(\rm DA)}] = \mean[\delta \theta^{(\rm OG)}]$ and 
\begin{align*}
    \Var[ \delta  \theta^{(\rm DA)}] - \Var[ 
        \delta \theta^{(\rm OG)}] \;=\;  \mfrac{k-1}{N} 
    \Var \, \mean\big[ F_{\bg_{1,1}(\bX_1); \psi_\theta} \big| \bX_1 \big]\;,
\end{align*}
which is positive semi-definite. \qed

\subsection{Proof of Lemma~\ref{lem:GA}}

By construction, $\delta \theta^{(\rm GA)} = \frac{1}{N / k} \sum_{i \leq N/k} F_{\bX^\cG_i; \psi^\cG_\theta}$ is a size-$N/k$ empirical average of i.i.d.~quantities. The mean and variance formulas thus follows directly from a standard computation:
\begin{align*}
    \mean[ \delta \theta^{(\rm GA)} ]
    \;=&\;
    \mean[  F_{\bX^\cG_1; \psi^\cG_\theta}  ]
    &\text{ and }&&
    \Var[\delta \theta^{(\rm GA)}]
    \;=&\;
    \mfrac{\Var[  F_{\bX^\cG_1; \psi^\cG_\theta}  ]}{N/k}
    \;. \qedhere
\end{align*}

\subsection{Proof of Theorem~\ref{thm:DA:CLT}} 
To prove the coordinate-wise bound, fix $l \leq p$. Note that if $\sigma^{(\rm DA)}_l = 0$, then $ \delta \theta^{(\rm DA)}_{1l} = \mean[ \delta \theta^{(\rm DA)}_{1l}]$ with probability $1$, implying that the distribution difference is zero and hence the bound is satisfied. In the case $\sigma^{(\rm DA)}_l > 0$, $\delta \theta^{(\rm DA)}_{1l} = \frac{1}{N / k} \msum_{i \leq N / k} F^{(\rm DA)}_{il}$ is an average of i.i.d.~univariate random variables with positive variance. By renormalizing $t$ and applying the Berry-Ess\'een theorem (see Theorem 3.7 of \citet{chen2011normal} or \citet{shevtsova2013optimization} for the version with a tight constant $C_1 = 0.469$) applied to $\frac{1}{N / k} \msum_{i \leq N / k} F_{il}$, we get that
\begin{align*}
    &\;
    \sup_{t \in \R} 
    \,
    \Big|
    \,
        \P\big( \, 
        \delta \theta^{(\rm DA)}_{1l}
        \,\leq\, t 
        \big)
        -
        \P\Big( \, \mean\big[ \delta \theta^{(\rm DA)}_{1l} \big]  
        + (\Var[\delta \theta^{(\rm DA)}_{1l}  ])^{1/2} \, Z_l \, 
        \,\leq\, t 
        \Big)
    \,
    \Big|
    \\
    &=
    \sup_{t \in \R} 
    \,
    \Big|
    \,
        \P\Big( \, 
        \mfrac{1}{N / k} \msum_{i \leq N / k} \big( F^{(\rm DA)}_{il}  - \mean\big[F^{(\rm DA)}_{il}\big] \big)
        \,\leq\, t 
        \Big)
        -
        \P\Big( \, \mfrac{\sigma^{(\rm DA)}_l}{\sqrt{N/k}} \, Z_l \, 
        \,\leq\, t 
        \Big)
    \,
    \Big|
    \leq
    \mfrac{C_1 \, \mean | F^{(\rm DA)}_{1l} |^3}{ \sqrt{N/k} \, \big(\sigma^{(\rm DA)}_l\big)^3}
    \;.
\end{align*}
This proves the first set of bounds. To prove the second set of bounds, we first denote the mean-zero variable $\bar F_{il} \coloneqq - F^{(\rm DA)}_{il} + \mean[F^{(\rm DA)}_{il}]$, and let $(\bar Z_{11}, \ldots, \bar Z_{np})$ be an $\R^{np}$-valued Gaussian vector with the same mean and variance as $(\bar F_{11}, \ldots, \bar F_{np})$. The difference in distribution function can be re-expressed as 
\begin{align*}
    &\;
    \sup_{t \in \R} 
    \,
    \Big|
    \,
        \P\Big( \, 
            \max_{l \leq p} \Big|\delta  \theta^{(\rm DA)}_{1l} -  \mean\big[\delta  \theta^{(\rm DA)}_{1l} \big]   \Big|
        \,\leq\, t 
        \big)
        -
        \P\Big( \, 
        \max_{l \leq p} \Big|
            (\Var[ \delta \theta^{(\rm DA)}_{1l}  ])^{1/2} \, Z_l \,
        \Big| 
        \,\leq\, t 
        \Big)
    \,
    \Big|
    \\
    &
    \;=\;
    \sup_{t \in \R} 
    \,
    \Big|
    \,
        \P\Big( \, 
            \max_{l \leq p} \Big| \mfrac{1}{\sqrt{ N / k }} \msum_{i \leq N/k} \bar F_{il} \Big|
        \,\leq\, t 
        \big)
        -
        \P\Big( \, 
        \max_{l \leq p} \Big|
        \mfrac{1}{\sqrt{ N / k}} \msum_{i \leq N/k} \bar Z_{il} 
        \Big| 
        \,\leq\, t 
        \Big)
    \,
    \Big|
    \;,
    \tagaligneq \label{eq:DA:diff:intermediate}
\end{align*}
where we have used the definition of $\theta^{(\rm DA)}_1$ and also replaced $t$ by $t  / \sqrt{N k}$. Note that $\bar F_{il}$'s are i.i.d.~mean-zero across $1 \leq i \leq N/k$ and $\bar Z_{il}$'s are i.i.d.~mean-zero across $1 \leq i \leq N/k$. As before, if $\sigma^{(\rm DA)}_l = 0$ for all $1 \leq l \leq p$, the two random variables to be compared are both $0$ with probability $1$ and the distributional difference above evaluates to zero. If there is at least one $l$ such that $\sigma^{(\rm DA)}_l = 0$, we can restrict both maxima above to be over $l \leq p$ such that $\sigma^{(\rm DA)}_l = 0$ and ignore the coordinates with zero variance. As such, we can WLOG assume that $ \sigma^{(\rm DA)}_l > 0$ for all $l \leq p$. Now write 
\begin{align*}
    \tilde F_i \;\coloneqq\; ( \sigma_1^{-1} \bar F_{i1}, \ldots, \sigma_p^{-1} \bar F_{ip})\;,
    \qquad 
    \tilde Z_i \;\coloneqq\; ( \sigma_1^{-1} \bar Z_{i1}, \ldots,  \sigma_p^{-1} \bar Z_{ip})\;,
\end{align*}
and denote the hyper-rectangular set $\cA(t) \coloneqq [ - \sigma_1^{-1} t, + \sigma_1^{-1} t ]
\times  \cdots  \times 
[ - \sigma_p^{-1} t, + \sigma_p^{-1} t ] \subseteq \R^q$. We can now express the difference above further as 
\begin{align*}
    \eqref{eq:DA:diff:intermediate}
    \;=&\;
    \sup_{t \in \R} 
    \,
    \Big|
    \,
        \P\Big( \, 
        \mfrac{1}{\sqrt{ N / k }} \msum_{i \leq N/k} \tilde F_i 
        \in 
        \cA(t) 
        \Big)
        -
        \P\Big( \, 
        \mfrac{1}{\sqrt{ N / k }} \msum_{i \leq N/k} \tilde Z_i 
        \in 
        \cA(t) 
        \Big)
    \,
    \Big|\;.
\end{align*}
This is a difference in distribution functions between a normalized empirical average of $p$-dimensional vectors with zero mean and identity covariance and a standard Gaussian in $\R^q$, measured through a subset of hyperrectangles. In particular, this is the quantity controlled by \citet{chernozhukov2017central}: Under the stated moment conditions and applying their Proposition 2.1, we have that for some absolute constant $C_2 > 0$,
\begin{align*}
    \eqref{eq:DA:diff:intermediate}
    \;\leq\;
    C_2
    \Big( 
    \mfrac{  (\tilde F^{(\rm DA)})^2 \, (\log (p N / k) )^7}
    { N / k }  
    \Big)^{1/6}\;.
    \tag*{\qed}
\end{align*}

\section{Proofs for results on canonicalization}  \label{appendix:proof:canon}

\subsection{Proof of Theorem~\ref{thm:SC}}

To prove (i), we fix any $\tilde g \in \G$, and WLOG let $k=1$. Then by the definition of $\psiSCone_{\theta;\epsilon}$,
\begin{align*}
    &\; \psiSCone_{\theta;\epsilon}
    (\tilde g(x_1), \ldots, \tilde g(x_n)) 
    \\
    & \hspace{4em} 
    \;=\;  
    \sum_{\substack{g \in \G \text{ s.t. } \\ d(\tilde g(x_1), g(\Pi_0)) \leq \epsilon}} 
    \Big(
    \, 
    \mfrac{ \lambda_\epsilon\big( d(\tilde g(x_1), g(\Pi_0)) \big) }{\sum_{\substack{g' \in \G \text{ s.t. } \\ d(\tilde g(x_1), g'(\Pi_0)) \leq \epsilon }}  \lambda_\epsilon\big( d(\tilde g(x_1), g'(\Pi_0)) \big)  }
    \times 
        \psi_\theta\big( g^{-1} \tilde g(x_1), \ldots, g^{-1} \tilde g(x_n) \big)
    \Big)\;.
\end{align*}
Relabelling $g$ by $\tilde g g$ and $g'$ by $\tilde g g'$ in the sums above, and noting that $d(\tilde g(x_1), \tilde g g(\Pi_0)) = d(x_1, g(\Pi_0))$, we obtain that 
\begin{align*}
    \psiSCone_{\theta;\epsilon}
    (\tilde g(x_1), \ldots, \tilde g(x_n)) 
    \;=&\;
    \sum_{\substack{g \in \G \text{ s.t. } \\ d(\tilde g(x_1), \tilde g g(\Pi_0)) \leq \epsilon}} 
    \Big(
    \, 
    \mfrac{ \lambda_\epsilon\big( d(\tilde g(x_1), \tilde g g(\Pi_0)) \big) }{\sum_{\substack{g' \in \G \text{ s.t. } \\ d(\tilde g(x_1),  \tilde g g'(\Pi_0)) \leq \epsilon}}  \lambda_\epsilon\big( d(\tilde g(x_1), \tilde g g'(\Pi_0)) \big)  }
    \,
    \\ 
    &\hspace{8em}
    \times 
        \psi_\theta\big( g^{-1} \tilde g^{-1} \tilde g(x_1), \ldots, g^{-1} \tilde g^{-1} \tilde g(x_n) \big)
    \Big)
    \\
    \;=&\;
    \sum_{\substack{g \in \G \text{ s.t. } \\ d(x_1, g(\Pi_0)) \leq \epsilon}} 
    \, 
    \mfrac{ \lambda_\epsilon\big(  d(x_1, g(\Pi_0)) \big) }{\sum_{\substack{g' \in \G \text{ s.t. } \\ d(x_1,  g'(\Pi_0)) \leq \epsilon}}  \lambda\big(  d(x_1,  g'(\Pi_0))\big)  }
    \,\;
    \psi_\theta\big( g^{-1} (x_1), \ldots, g^{-1} (x_n) \big)
    \\
    \;=&\;
    \psiSCone_{\theta;\epsilon}
    (x_1, \ldots, x_n)
    \;.
\end{align*}
This proves diagonal $\G$-invariance. 

\vspace{.5em}

To prove (ii), we shall make the spin-dependence explicit and write $\tilde x_i = (x_i, \sigma_i)$ and $g(\tilde x_i) = (g(x_i), \sigma_i)$. First consider $\pi$, a transposition that swaps the indices $1$ and $2$. Then
\begin{align*}
    \psiSC_{\theta;\epsilon}(\tilde x_{\pi(1)}, \ldots, \tilde  x_{\pi(n)} )
    \;=&\;
    \mfrac{1}{n} \msum_{k=1}^n
    \psiSCk_{\theta;\epsilon}(\tilde  x_2, \tilde x_1, \tilde  x_3 \ldots, \tilde x_n )
    \;.
\end{align*}
By the anti-symmetry of $\psi_\theta$, we see that for $k \geq 3$,
\begin{align*}
    \psiSCk_{\theta;\epsilon}(\tilde x_2, \tilde x_1, \tilde x_3 \ldots, \tilde x_n )
    \;=&\;
    \msum_{g \in \cG_\epsilon(x_k)} 
    \, 
    w^g_\epsilon ( x_k)
    \, 
    \psi_\theta \big( g^{-1}(\tilde x_2), g^{-1}(\tilde x_1), g^{-1}(\tilde x_3), \ldots, g^{-1}(\tilde x_n) \big)
    \\
    \;=&\;
    -
    \msum_{g \in \cG_\epsilon(x_k)} 
    \, 
    w^g_\epsilon (x_k)
    \, 
    \psi_\theta \big( g^{-1}(\tilde x_1), g^{-1}(\tilde x_2), g^{-1}(\tilde x_3), \ldots, g^{-1}(\tilde x_n) \big)
    \\
    \;=&\;
    -
    \psiSCk_{\theta;\epsilon}(\tilde x_1, \ldots, \tilde x_n )
    \;.
\end{align*}
For the case $k=1,2$, we can apply a similar calculation while noting that the weights remain unchanged, and obtain
\begin{align*}
    \psiSCone_{\theta;\epsilon}(\tilde x_2, \tilde x_1, \tilde x_3 \ldots, \tilde x_n )
    \;=&\;
    - \psiSCtwo_\theta(\tilde x_1, \tilde x_2, \tilde x_3 \ldots, \tilde  x_n ) 
    \;,
    \\
    \psiSCtwo_\theta(\tilde x_2, \tilde x_1, \tilde x_3 \ldots, \tilde x_n ) 
    \;=&\;
    - \psiSCone_{\theta;\epsilon}(\tilde x_1, \tilde x_2, \tilde x_3 \ldots, \tilde x_n ) 
    \;.
\end{align*}
This implies that 
\begin{align*}
    \psiSC_{\theta;\epsilon}(\tilde x_{\pi(1)}, \ldots, \tilde x_{\pi(n)} )
    \;=&\;
    -
    \mfrac{1}{n} \msum_{k=1}^n
    \psiSCk_{\theta;\epsilon}(\tilde x_1, \tilde x_2, \tilde x_3 \ldots, \tilde  x_n )
    \;=\;
    {\rm sgn}(\pi)
    \,
    \psiSC_{\theta;\epsilon}(\tilde x_1,  \ldots, \tilde x_n )
    \;.
\end{align*}
Since the choice of indices $1$ and $2$ are arbitrary, the above in fact holds for all transpositions $\pi$, which implies 
\begin{align*}
    \psiSC_{\theta;\epsilon}(\tilde x_{\tau(1)}, \ldots,\tilde  x_{\tau(n)} )
    \;=\;
    {\rm sgn}(\tau)
    \,
    \psiSC_{\theta;\epsilon}(\tilde x_1, \ldots, \tilde x_n )
\end{align*}
for all permutations $\tau$ of the $n$ electrons. This proves (ii).

\vspace{.5em}

To prove (iii), it suffices to show that for every $k \leq n$, the function $\psiSCk_{\theta;\epsilon}$ defined above is $p$-times continuously differentiable at $\bx$, i.e.~$\nabla^p \psiSCk_{\theta;\epsilon}$ exists and is continuous at $\bx$. Again it suffices to show this for the case $k=1$. Let $\tilde \bx \coloneqq (\tilde x_1, \ldots, \tilde x_n)$ be a vector in a sufficiently small neighborhood of a fixed $\bx \coloneqq (x_1, \ldots, x_n)$, and recall that 
\begin{align*}
    \cG_\epsilon( \tilde x_1 )
    \;=\; 
    \big\{ g \in \G \,\big|\, d( \tilde x_1, g(\Pi_0)) \leq \epsilon \big\}
    \;.
\end{align*}
For $\tilde \bx$ in a sufficiently small neighborhood of $\bx$, $\cG_\epsilon( \tilde x_1 )$ takes value in $\{ \cG_l \}_{0 \leq l \leq M}$, where 
\begin{align*}
    \cG_l
    \;\coloneqq\;
    \cG_\epsilon(x_1)
    \,\setminus\, 
    \{ g^\epsilon_1, \ldots, g^\epsilon_l \} 
    \;,
\end{align*}
and $g^\epsilon_1, \ldots, g^\epsilon_M \in \G$ is an enumeration of all group elements such that 
\begin{align*}
    d\big( x_1 \,,\, g^\epsilon_l(\Pi_0) \big)
    \;=\; 
    \epsilon
    \qquad 
    \text{ for } 0 \leq l \leq M\;.
\end{align*}
Therefore $\psiSCone_{\theta;\epsilon}(\tilde \bx)$ takes values in $\{\psi_{\cG_l}(\tilde \bx) \}_{0 \leq l \leq M}$, where 
\begin{align*}
    \psi_{\cG_l}(\tilde \bx)
    \coloneqq
    \msum_{g \in \cG_l} 
    \mfrac{ \lambda_\epsilon\big( \frac{\epsilon - d(  \tilde x_1, g(\Pi_0))}{\epsilon} \big) }{\sum_{g' \in \cG_l} \lambda_\epsilon\big( \frac{\epsilon - d(\tilde x_1, g'(\Pi_0))}{\epsilon} \big)  }
    \, \psi_\theta\big(
        g^{-1}  (\tilde x_1)
        \,,\,
        \ldots 
        \,,\,
        g^{-1} (\tilde x_n)
    \big) 
    \;.
\end{align*}
Notice that at $\tilde \bx=\bx$, by the definition of $g^\epsilon_l$, we have
\begin{align*}
    \psi_{\cG_l}(\bx)
    \;=&\;
    \msum_{g \in \cG_l} 
    \mfrac{ \lambda_\epsilon \big( d(  x_1, g(\Pi_0)) \big) }{\sum_{g' \in \cG_l}  \lambda_\epsilon \big( d(  x_1, g'(\Pi_0)) \big)  }
    \times
    \, \psi_\theta\big(
        g^{-1}  ( x_1)
        \,,\,
        \ldots 
        \,,\,
        g^{-1}  ( x_n)
    \big) 
    \\ 
    \;\overset{(a)}{=}&\;
    \msum_{g \in \cG_\epsilon(x_1)} 
    \mfrac{ \lambda_\epsilon \big(  d(   x_1, g(\Pi_0))  \big) }{\sum_{g' \in \cG_\epsilon(x_1)} \lambda_\epsilon\big( d( x_1, g'(\Pi_0)) \big)  }
    \times
    \, \psi_\theta\big(
        g^{-1} ( x_1)
        \,,\,
        \ldots 
        \,,\,
        g^{-1}  ( x_n)
    \big) 
    \;=\; \psiSCone_{\theta;\epsilon}(\bx)
    \;.
\end{align*}
Since the above argument works with $\bx$ replaced by $\tilde \bx$, we also have 
\begin{align*}
    \psiSCone_{\theta;\epsilon}(\tilde \bx)
    \;=\;
    \msum_{g \in \tilde \cG_l} 
    \mfrac{ \lambda_\epsilon \big(  d(  x_1, g(\Pi_0)) \big) }{\sum_{g' \in \tilde \cG_l} \lambda_\epsilon\big( d( x_1, g'(\Pi_0)) \big)  }
    \times
    \, \psi_\theta\big(
        g^{-1}  ( x_1)
        \,,\,
        \ldots 
        \,,\,
        g^{-1}  ( x_n)
    \big) 
    \tagaligneq \label{eq:continuity:F1:eps}
\end{align*}
for $0 \leq l \leq \tilde M$, where we have defined 
\begin{align*}
    \tilde \cG_l \;\coloneqq\; \cG_\epsilon(\tilde x_1) \,\setminus\, 
    \{ \tilde g^\epsilon_1, \ldots, \tilde g^\epsilon_l \} 
    \;,
\end{align*}
and $\tilde g^\epsilon_1, \ldots, \tilde g^\epsilon_M \in \G$ is an enumeration of all group elements such that 
\begin{align*}
    d\big( \tilde x_1 \,,\, \tilde g^\epsilon_l(\Pi_0) \big)
    \;=\; 
    \epsilon
    \qquad 
    \text{ for } 0 \leq l \leq M\;.
\end{align*}
Notice that for $\tilde \bx$ in a sufficiently small neighborhood of $\bx$, we have $\{ \tilde \cG_l \}_{l \leq \tilde M} = \{ \cG_l \}_{l \leq M}$, in which case \eqref{eq:continuity:F1:eps} implies 
\begin{align*}
    \psiSCone_{\theta;\epsilon}(\tilde \bx) 
    \;=\; 
    \psi_{\cG_l}(\tilde \bx)
    \qquad 
    \text{ for all } 0 \leq l \leq M\;.
\end{align*}
In other words, we have shown that in a sufficiently small neighborhood of $\bx$, $F_1$ equals $\psi_{\cG_l}$ for all $1 \leq l \leq M$. Recall that $\lambda$ and $d(\argdot, g(\Pi_0))$ are $p$-times continuously differentiable for all $g \in \G$, and $\psi_\theta$ is $p$-times continuously differentiable at $g(\bx)$ for all $g \in \G$ by assumption. This implies that $\psi_{\cG_l}$ is also $p$-times continuously differentiable at $\bx$ and so is $\psiSCone_{\theta;\epsilon}$. Moreover, for $0 \leq q \leq p$, the derivative can be computed as
\begin{align*}
    \nabla^q \psiSCone_{\theta;\epsilon}(\bx)
    \;=\;
    \nabla^q \psi_{\cG_l}(\bx)
    \;=\;
    \msum_{g \in \cG_\epsilon(x_1)} 
    \nabla^q 
    \psiSCk_{\theta; g, x_1}
    (\bx)
    \;,
\end{align*}
where we recall that for $1 \leq k \leq n$, $g \in \G$ and $x, y_1, \ldots, y_n \in \R^3$, 
\begin{align*}
    \psiSCk_{\theta,\epsilon;g,x}
    (y_1, \ldots, y_n)
    \;\coloneqq&\;
    \mfrac{
        \lambda_\epsilon
        \big(  d( y_k, g(\Pi_0)) \big) 
    }{
        \sum_{g' \in \cG_\epsilon(x)}
        \lambda_\epsilon
        \big(  d( y_k, g'(\Pi_0))  \big) 
    }
    \,
    \psi_\theta( g^{-1}(y_1) \,,\, \ldots \,,\, g^{-1}(y_n) )
    \;.
\end{align*}
The same argument applies to all $\psiSCk_{\theta;\epsilon}$'s with $k \leq n$ and therefore for $0 \leq q \leq p$,
\begin{align*}
    \nabla^q 
    \psiSC_{\theta;\epsilon}(x_1, \ldots, x_n)
    \;=&\;
    \mfrac{1}{n} \msum_{k=1}^n 
    \nabla^q 
    \psiSCk_{\theta;\epsilon}(x_1, \ldots, x_n)
    \;=\;
    \mfrac{1}{n}
    \msum_{k=1}^n
    \msum_{g \in \cG_\epsilon(x_k)}
    \,
    \nabla^q 
    \psiSCk_{\theta,\epsilon;g,x_k}
    (\bx)\;,
\end{align*}
which proves (iii). \qed

\subsection{Proof of Lemma~\ref{lem:eps:lamb:blowup}} 

Since $\lambda_\epsilon(0) = 1$, $\lambda_\epsilon(\epsilon) = 0$ and $\lambda_\epsilon$ is continuously differentiable, by the mean value theorem, there is $y_1 \in (0, \epsilon)$ such that 
\begin{align*}
    \partial\lambda_\epsilon(y_1) \;=\; (1-0)/ \epsilon \;=\; \epsilon^{-1}   \;.
\end{align*}    
Meanwhile, since $\lambda_\epsilon(w) = 1$ for  all $w \leq 0$, $\partial \lambda_\epsilon(w) = 0$ for  all $w < 0$, and since $\partial \lambda_\epsilon$ is continuous, $\partial \lambda_\epsilon(0) = 0$. By the twice continuous differentiability of $\lambda_\epsilon$ and the mean value theorem again, there exists $y'_1 \in (0, y_1) \subset (0, \epsilon)$ such that 
\begin{align*}
    \partial^2 \lambda_\epsilon(y'_1) \;=\; (\epsilon^{-1}-0)/ y_1 \;=\; \mfrac{1}{\epsilon y'_1} \;\geq\; \mfrac{1}{\epsilon^2}  \;.
\end{align*}    
Since $\partial^2 \lambda_\epsilon(w) = 0$ for all $w < 0$ and $\partial^2 \lambda_\epsilon$ is continuous, $\partial^2 \lambda_\epsilon(0) = 0$. As $\partial^2 \lambda_\epsilon(y'_1) \geq  \mfrac{1}{\epsilon^2}$, by the intermediate value theorem, there exists some $y_2 \in [0, y_1] \subseteq [0, \epsilon]$ such that  $\partial^2 \lambda_\epsilon(y_2) = \epsilon^{-2}$. \qed

\setcounter{section}{2}
\subsection{Generalization}

Let $P$ be a distribution over $\mathbb{R}^{D \times L} \times [C]$. Let $S = \{( X^{}_{(j)},y_{(j)})\}^{m}_{j=1}$ be a dataset of i.i.d. samples selected from $P$. Our generalization bound is based on a uniform-convergence generalization bound provided in ~\citep{galanti2024norm}. The following lemma bounds the gap between the test error and the empirical margin error, represented as $\err^{\gamma}_S(f_w)= 
\fr{1}{m}\sum^{m}_{j=1}\bI[\max_{c \neq c'}(f^c_w( X^{}_{(j)})) + \gamma \geq f^{c'}_w( X^{}_{(j)})]$. 

\begin{lemma}\label{lem:loss_ramp}
Let $P$ be a distribution over $\mathbb{R}^{\mathcal{D}} \times [C]$ and $\mathcal{F} \subset \{f':\mathcal{X} \to \R^C \}$. Let $S = \{(X_j,y_j)\}^{m}_{j=1}$ be a dataset of i.i.d. samples selected from $P$ and $X=\{X_j\}^{m}_{j=1}$. Then, with probability at least $1-\delta$ over the selection of $S$, for any $f_w \in \mathcal{F}$, we have
\begin{equation}\label{eq:Radbound}
\err_P(f_w) - \err^{\gamma}_S(f_w) \leq \frac{2\sqrt{2}}{\gamma} \cdot \mathcal{R}_{X}(\mathcal{F}) + 3\sqrt{\frac{\log(2/\delta)}{2m}}.
\end{equation} 
\end{lemma}
\begin{lemma}\label{lem:peeling}
Let $\sigma_j$ be a $l_j$-Lipschitz, positive-homogeneous function and $l=\max_j l_j$. Let $\xi_i \sim U[\{\pm 1\}]$. Then for any class of vector-valued functions $\mathcal{F} \subset \{f \mid ~f:\mathbb{R}^d \to \mathbb{R}\}$ and any convex and monotonically
increasing function $g : \mathbb{R} \to [0,\infty)$, 
\begin{small}
\begin{equation*}
\begin{aligned}
\E_{\xi} \sup_{\substack{f\in \mathcal{F}}} g\left( \left| \sum^{m}_{j=1} \xi_j \cdot \sigma_j(f(x_j))\right| \right)
~\leq~ 2\E_{\xi} \sup_{f\in \mathcal{F}} g\left(l\left|\sum^{m}_{i=j} \xi_j  \cdot f(x_j)\right| \right).
\end{aligned}
\end{equation*}
\end{small}
\end{lemma}
% \begin{restatable}[Contraction Lemma]{lemma}{peeling}\label{lem:peeling}
% Let $\sigma_j$ be a $l_j$-Lipschitz, positive-homogeneous function and $l=\max_j l_j$. Let $\xi_i \sim U[\{\pm 1\}]$. Then for any class of vector-valued functions $\mathcal{F} \subset \{f \mid ~f:\mathbb{R}^d \to \mathbb{R}\}$ and any convex and monotonically
% increasing function $g : \mathbb{R} \to [0,\infty)$, 
% \begin{small}
% \begin{equation*}
% \begin{aligned}
% \E_{\xi} \sup_{\substack{f\in \mathcal{F}}} g\left( \left| \sum^{m}_{j=1} \xi_j \cdot \sigma_j(f(x_j))\right| \right)
% ~\leq~ 2\E_{\xi} \sup_{f\in \mathcal{F}} g\left(l\left|\sum^{m}_{i=j} \xi_j  \cdot f(x_j)\right| \right).
% \end{aligned}
% \end{equation*}
% \end{small}
% \end{restatable}
\begin{proof}
We notice that since $g(|z|) \leq g(z)+g(-z)$,
\begin{small}
\begin{equation*}
\begin{aligned}
&\E_{\xi} \sup_{\substack{f\in \mathcal{F}}} g\left( \left| \sum^{m}_{j=1} \xi_j \cdot \sigma_j(f(x_j))\right| \right)\\
&\leq \E_{\xi} \sup_{\substack{f\in \mathcal{F}}} g\left( \sum^{m}_{j=1} \xi_j \cdot \sigma_j(f(x_j)) \right)\\
&+\E_{\xi} \sup_{\substack{f\in \mathcal{F}}} g\left( - \sum^{m}_{j=1} \xi_j \cdot \sigma_j(f(x_j)) \right) \\
&= 2\E_{\xi} \sup_{\substack{f\in \mathcal{F}}} g\left( \sum^{m}_{j=1} \xi_j \cdot \sigma_j(f(x_j)) \right)\\
\end{aligned}
\end{equation*}
\end{small}
where the last equality follows from the symmetry in the distribution of the $\xi_i$ random variables. By Equation 4.20 in~\cite{Ledoux1991ProbabilityIB}  we have the following:
\begin{small}
\begin{equation*}
\begin{aligned}
&\E_{\xi} \sup_{\substack{f\in \mathcal{F}}} g\left( \sum^{m}_{j=1} \xi_j \cdot \sigma_j(f(x_j)) \right)\\
&=\E_{\xi} \sup_{\substack{f\in \mathcal{F}}} g\left(l \sum^{m}_{j=1} \xi_j \cdot \frac{1}{l}\sigma_j(f(x_j)) \right)\\
&\leq\E_{\xi} \sup_{\substack{f\in \mathcal{F}}} g\left(l \sum^{m}_{j=1} \xi_j \cdot f(x_j) \right)\\
&\leq\E_{\xi} \sup_{\substack{f\in \mathcal{F}}} g\left(l \left| \sum^{m}_{j=1} \xi_j \cdot f(x_j) \right| \right)\\
\end{aligned}
\end{equation*}
\end{small}
where the last equality follows from $g$ being monotonically increasing. 
\end{proof}
% \rademacher*
The model is denoted by:
\begin{equation}\label{eq:modelAppendix}
\begin{aligned} 
      &B_i = S_B x_i, \quad C_i = S_C x_i, \quad \Delta_i = \sigma(S_{\Delta} x_i), \quad  \\&\bar{A}_i =  \exp(\Delta_i A), \quad \bar{B}_i = B_i 
\end{aligned}
\end{equation}
Where \( \sigma \) is a 1-Lipschitz activation function. 
% For our analysis, we will use \( \text{ReLU} \).
We consider a classifier $f : \mathbb{R}^{D \times L} \rightarrow \mathbb{R}^{\mathcal{C}}$ defined as follows. We have parameters $(A^{}, S_B^{}, S_C^{}, S_{\Delta}^{})$ associated with the layer. The norms are defined as follows:

\begin{equation}
\begin{aligned} 
    \rho_A^{}(w) &= \|A^{}\|_{\max} \\
    \rho_B^{}(w) &= \|S_B^{}\|_{2,\infty} \\
    \rho_C^{}(w) &= \|S_C^{}\|_{F} \\
    \rho_{\Delta}^{}(w) &= \|S_{\Delta}^{}\|_{2}
\end{aligned}
\end{equation}

We denote the product of these norms as:
\begin{equation}
\Gamma^{}(w) = \rho_A^{}(w) \cdot \rho_B^{}(w) \cdot \rho_C^{}(w) \cdot \rho_{\Delta}^{}(w)
\end{equation}

Given these definitions, the classifier $f$ for a specific class $c \in [\mathcal{C}]$ is computed as:

\begin{small}
\begin{equation*}
\begin{aligned}
&f^c(X_{*1},...,X_{*L}) =  
\sum_{d=1}^D W_{c,d} \left(S_C^{} X^{}_{*L}\right)^T \sum_{i=1}^L \left(\prod_{k=i+1}^L \bar{A}^{}_{dk} \right) S_B^{} X^{}_{*i} X^{}_{di}
\end{aligned}
\end{equation*}
\end{small}

Here, $W \in \mathbb{R}^{\mathcal{C} \times D}$ represents a linear projection from the output to the number of classes, and $\mathcal{C}$ is the number of classes. 

We denote the parameters of the classifier by 
\[
w = (A^{}, S_B^{}, S_C^{}, S_{\Delta}^{}, W) 
\] and the function induced by a specific instance of $w$ is denoted by $f_w$. The class of functions taking on different parameter instances $w$ is denoted by $\mathcal{F}$.
Denote 
\[
\rho = \{\rho_W, \rho_A, \rho_B, \rho_C, \rho_{\Delta}\}
% \quad \text{where} \quad \rho_A = \{\rho_A^{}\}_{h=1}^{H}, \ \rho_B = \{\rho_B^{}\}_{h=1}^{H}, \ \rho_C = \{\rho_C^{}\}_{h=1}^{H}, \ \rho_{\Delta} = \{\rho_{\Delta}^{}\}_{h=1}^{H}
\]
and let:
\begin{equation}
\begin{aligned}
   \mathcal{F}_{\rho} 
    &=\{f_w \in \mathcal{F} : \Gamma(w) \leq  \rho_A \rho_B \rho_C \rho_{\Delta} \rho_W=:\Gamma \}
\end{aligned}
\end{equation}

The following theorem provides a bound on the Rademacher complexity of the class $\mathcal{F}_{\rho}$.
\setcounter{section}{2}
\setcounter{theorem}{3}
\begin{theorem}\label{theorem:rad}
Let $\rho = \{\rho_W, \rho_A, \rho_B, \rho_C, \rho_{\Delta}\}$. Suppose we have \(m\) sample sequences \(X = \{ X_{(j)} \}_{j=1}^{m}\), where each \(X_{(j)} = (X_{(j)_{*1}}, \ldots, X_{(j)_{*L}}) \in \mathbb{R}^{D \times L}\). Assume that $\forall j \in [m]: ||X_{(j)}||_{\max} \leq 1$. Additionally, suppose $\forall k \in [L], d \in [D]:||\bar{A}^{}_{dk}||_{\max} < K < 1$. Then,
\begin{small}
\begin{equation*}
\begin{aligned}
\mathcal{R}_X(\mathcal{F}_{\rho})
&\leq\frac{1}{m} D^{2} \Gamma (1 + \sqrt{2\log (2L \mathcal{C} D^4 N)}) \cdot  \sqrt{\max_{t, k} \sum_{j=1}^m (X_{(j)_{tk}})^2} \frac{K}{(K-1)^2}
\end{aligned}
\end{equation*}
\end{small}
where the maximum is taken over \(t \in [D]\),  \(k \in [L]\). 
\end{theorem}

\begin{proof}
% We approximate the exponential function using the first three terms of its Taylor series expansion, $\exp(z) \sim 1 + z + z^2/2$.
For aesthetic purposes, we define \( B \) as \( S_B \) and \( C \) as \( S_C \).
The Rademacher complexity of $\mathcal{F}_{\rho}$ is given by:
\begin{small}
\begin{equation*}
\begin{aligned}
&m\mathcal{R(\mathcal{F}_{\rho})} = \E_{\xi}\left[\sup_{w} \sum_{j=1}^m \sum_{c=1}^{\mathcal{C}} \xi_{jc} f^c_w(X_{(j)}) \right] \\
&= \E_{\xi}\left[\sup_{w} \sum_{j=1}^m \sum_{c=1}^{\mathcal{C}} \xi_{jc} \sum_{d=1}^D W_{c,d} 
\left(S_C^{} X^{}_{{(j)}_{*L}}\right)^T \sum_{i=1}^L \right. \left.\left(\prod_{k=i+1}^L \exp\left(\sigma(S^{}_{\Delta} X^{}_{{(j)}_{*k}}) 
\cdot A^{}_{d*}\right)\right) S_B^{} X^{}_{{(j)}_{*i}} X^{}_{{(j)}_{di}} \right] \\
&= \E_{\xi}\left[\sup_{w} \sum_{j=1}^m \sum_{c=1}^{\mathcal{C}} \xi_{jc} \sum_{d=1}^D W_{c,d} 
\left(C^{} X^{}_{{(j)}_{*L}}\right)^T \sum_{i=1}^L \right. \left.\left(\prod_{k=i+1}^L \exp\left(\sigma(S^{}_{\Delta} X^{}_{{(j)}_{*k}}) 
\cdot A^{}_{d*}\right)\right) B^{} X^{}_{{(j)}_{*i}} X^{}_{{(j)}_{di}} \right] 
\end{aligned}
\end{equation*}
\end{small}

Here, {$A^{}_{d*}$ represents the $d$-th row of $A^{}$}, which has a size of $N$. We think of $A^{}_{d*}$ as a diagonal matrix of size $N\times N$, where its diagonal elements are the values in the $d$th row of $A$. Thus, $A^{}_{d*} = \text{diag}(a^{}_{d1},...,a^{}_{dN})$. Hence,
\begin{small}
\begin{equation*}
\begin{aligned}
& \E_{\xi}\left[\sup_{w} \sum_{j=1}^m \sum_{c=1}^{\mathcal{C}} \xi_{jc} \sum_{d=1}^D W_{c,d} 
\left(C^{} X^{}_{{(j)}_{*L}}\right)^T \sum_{i=1}^L \right. 
 \left. \left(\prod_{k=i+1}^L \exp\left(\sigma(S^{}_{\Delta} X^{}_{{(j)}_{*k}}) \cdot 
A^{}_{d*}\right) \right) B^{} X^{}_{{(j)}_{*i}} X^{}_{{(j)}_{di}} \right] \\
&=~ \E_{\xi}\left[\sup_{w} \sum_{j=1}^m \sum_{c=1}^{\mathcal{C}} \xi_{jc} \sum_{d=1}^D W_{c,d} 
\sum_{l=1}^N \left(C^{} X^{}_{{(j)}_{*L}}\right)_l^T \sum_{i=1}^L \right. \left. \left(\prod_{k=i+1}^L \exp\left(\sigma(S^{}_{\Delta} X^{}_{{(j)}_{*k}}) \cdot 
a^{}_{dl}\right) \right) B_{l*}^{} X^{}_{{(j)}_{*i}} X^{}_{{(j)}_{di}} \right] \\
&=~ \E_{\xi}\left[\sup_{w} \sum_{j=1}^m \sum_{c=1}^{\mathcal{C}} \xi_{jc} \sum_{d=1}^D W_{c,d} 
\sum_{l=1}^N \left(C_{l*}^{} X^{}_{{(j)}_{*L}}\right) \sum_{i=1}^L \right. \left. \left(\prod_{k=i+1}^L \exp\left(\sigma(S^{}_{\Delta} X^{}_{{(j)}_{*k}}) \cdot 
a^{}_{dl}\right) \right) B_{l*}^{} X^{}_{{(j)}_{*i}} X^{}_{{(j)}_{di}} \right] \\
&=~ \E_{\xi}\left[\sup_{w} \sum_{c=1}^{\mathcal{C}} \sum_{d=1}^D W_{c,d} \sum_{j=1}^m \xi_{jc} 
\sum_{l=1}^N \left(C_{l*}^{} X^{}_{{(j)}_{*L}}\right) \sum_{i=1}^L \right.  \left. \left(\prod_{k=i+1}^L \exp\left(\sigma(S^{}_{\Delta} X^{}_{{(j)}_{*k}}) \cdot 
a^{}_{dl}\right) \right) B_{l*}^{} X^{}_{{(j)}_{*i}} X^{}_{{(j)}_{di}} \right]
\end{aligned}
\end{equation*}
\end{small}

This follows from expressing the model in a more explicit form. Next,
\begin{small}
\begin{equation*}
\begin{aligned}
&~ \E_{\xi}\left[\sup_{w} \sum_{c=1}^{\mathcal{C}}  \sum_{d=1}^D W_{c,d} \sum_{j=1}^m \xi_{jc} \sum_{l=1}^N \left(C_{l*}^{} X^{}_{{(j)}_{*L}}\right) \sum_{i=1}^L \right.  \left. \left(\prod_{k=i+1}^L \exp\left(\sigma(S^{}_{\Delta} X^{}_{{(j)}_{*k}}) \cdot a^{}_{dl}\right) \right) B_{l*}^{} X^{}_{{(j)}_{*i}} X^{}_{{(j)}_{di}} \right] \\
% &\leq~ \rho_W\E_{\xi}\left[\sup_{w,c}  \left( \sum_{d=1}^D \left( \sum_{j=1}^m \xi_{jc} \sum_{l=1}^N \left(C_{l*}^{} X^{}_{{(j)}_{*L}}\right) \sum_{i=1}^L  \left(\prod_{k=i+1}^L \exp\left(\sigma(S^{}_{\Delta} X^{}_{{(j)}_{*k}}) \cdot a^{}_{dl}\right) \right) B_{l*}^{} X^{}_{{(j)}_{*i}} X^{}_{{(j)}_{di}} \right)^2 \right) \right] \\
% &\leq~ \rho_W\E_{\xi}\left[\sup_{w,c, d}  \sqrt{ D \left( \sum_{j=1}^m \xi_{jc} \sum_{l=1}^N \left(C_{l*}^{} X^{}_{{(j)}_{*L}}\right) \sum_{i=1}^L \left(\prod_{k=i+1}^L \exp\left(\sigma(S^{}_{\Delta} X^{}_{{(j)}_{*k}}) \cdot a^{}_{dl}\right) \right) B_{l*}^{} X^{}_{{(j)}_{*i}} X^{}_{{(j)}_{di}} \right)^2} \right] \\
&=~ \sqrt{D} \rho_W \E_{\xi}\left[\sup_{w,c, d}   | \sum_{j=1}^m \xi_{jc} \sum_{l=1}^N \left(C_{l*}^{} X^{}_{{(j)}_{*L}}\right) \sum_{i=1}^L \right.\left. \left(\prod_{k=i+1}^L \exp\left(\sigma(S^{}_{\Delta} X^{}_{{(j)}_{*k}}) \cdot a^{}_{dl}\right) \right) B_{l*}^{} X^{}_{{(j)}_{*i}} X^{}_{{(j)}_{di}} | \right]
\end{aligned}
\end{equation*}
\end{small}
where the inequality follows from moving the norm of $W$ to $W_{c}$ for maximizing the inner term and applying the Cauchy-Schwartz inequality.
Next,
\begin{small}
\begin{equation*}
\begin{aligned}
&~ \sqrt{D} \rho_W \E_{\xi}\left[\sup_{w,c, d}   | \sum_{j=1}^m \xi_{jc} \sum_{l=1}^N \left(C_{l*}^{} X^{}_{{(j)}_{*L}}\right) \sum_{i=1}^L \right.  \left. \left(\prod_{k=i+1}^L \exp\left(\sigma(S^{}_{\Delta} X^{}_{{(j)}_{*k}}) \cdot a^{}_{dl}\right) \right) B_{l*}^{} X^{}_{{(j)}_{*i}} X^{}_{{(j)}_{di}} | \right] \\
&=~ \sqrt{D} \rho_W \E_{\xi}\left[\sup_{w,c, d}   | \sum_{j=1}^m \xi_{jc} \sum_{l=1}^N \left(\sum_{s=1}^D C_{ls}^{} X^{}_{{(j)}_{sL}}\right) \sum_{i=1}^L \right.  \left. \left(\prod_{k=i+1}^L \exp\left(\sigma(S^{}_{\Delta} X^{}_{{(j)}_{*k}}) \cdot a^{}_{dl}\right) \right) \left(\sum_{s'=1}^DB_{ls'}^{} X^{}_{{(j)}_{s'i}} \right) X^{}_{{(j)}_{di}} | \right] \\
&=~ \sqrt{D} \rho_W \E_{\xi}\left[\sup_{w,c, d}   | \sum_{l=1}^N \sum_{s=1}^D C_{ls}^{} \sum_{j=1}^m \xi_{jc}  X^{}_{{(j)}_{sL}} \sum_{i=1}^L \right. \left. \left(\prod_{k=i+1}^L \exp\left(\sigma(S^{}_{\Delta} X^{}_{{(j)}_{*k}}) \cdot a^{}_{dl}\right) \right) \left(\sum_{s'=1}^DB_{ls'}^{} X^{}_{{(j)}_{s'i}} \right) X^{}_{{(j)}_{di}} | \right] \\
% &\leq~ \sqrt{D} \rho^{}_W \rho^{}_C \E_{\xi}\left[\sup_{w,c, d, l}    \sqrt{ \sum_{s=1}^D \left( \sum_{j=1}^m \xi_{jc}  X^{}_{{(j)}_{sL}} \sum_{i=1}^L \left(\prod_{k=i+1}^L \exp\left(\sigma(S^{}_{\Delta} X^{}_{{(j)}_{*k}}) \cdot a^{}_{dl}\right) \right) \left(\sum_{s'=1}^DB_{ls'}^{} X^{}_{{(j)}_{s'i}} \right) X^{}_{{(j)}_{di}} \right)^2} \right] \\
&\leq~ D \rho^{}_W \rho^{}_C \E_{\xi}\left[\sup_{w,c, d, l, s} | \sum_{j=1}^m \xi_{jc}  X^{}_{{(j)}_{sL}} \sum_{i=1}^L \right.  \left. \left(\prod_{k=i+1}^L \exp\left(\sigma(S^{}_{\Delta} X^{}_{{(j)}_{*k}}) \cdot a^{}_{dl}\right) \right) \left(\sum_{s'=1}^DB_{ls'}^{} X^{}_{{(j)}_{s'i}} \right) X^{}_{{(j)}_{di}} |  \right] \\
&=~ D \rho^{}_W \rho^{}_C \E_{\xi}\left[\sup_{w,c, d, l, s} | \sum_{s'=1}^D B_{ls'}^{} \sum_{j=1}^m \xi_{jc}  X^{}_{{(j)}_{sL}} \sum_{i=1}^L \right.  \left. \left(\prod_{k=i+1}^L \exp\left(\sigma(S^{}_{\Delta} X^{}_{{(j)}_{*k}}) \cdot a^{}_{dl}\right) \right) X^{}_{{(j)}_{s'i}} X^{}_{{(j)}_{di}} |  \right]
\end{aligned}
\end{equation*}
\end{small}
where the first inequality follows from moving the norm of $C$ to $C_{l}$ for maximizing the inner term and applying the Cauchy-Schwartz inequality. 
\begin{small}
\begin{equation*}
\begin{aligned}
&=~ D \rho^{}_W \rho^{}_C \E_{\xi}\left[\sup_{w,c, d, l, s} | \sum_{s'=1}^D B_{ls'}^{} \sum_{j=1}^m \xi_{jc}  X^{}_{{(j)}_{sL}} \sum_{i=1}^L \right.  \left.\left(\prod_{k=i+1}^L \exp\left(\sigma(S^{}_{\Delta} X^{}_{{(j)}_{*k}}) \cdot a^{}_{dl}\right) \right) X^{}_{{(j)}_{s'i}} X^{}_{{(j)}_{di}} |  \right] \\
% &\leq~ D \rho^{}_W \rho^{}_C \rho^{}_B \E_{\xi}\left[\sup_{w,c, d, l, s}  \sqrt{ \sum_{s'=1}^D \left( \sum_{j=1}^m \xi_{jc}  X^{}_{{(j)}_{sL}} \sum_{i=1}^L \left(\prod_{k=i+1}^L \exp\left(\sigma(S^{}_{\Delta} X^{}_{{(j)}_{*k}}) \cdot a^{}_{dl}\right) \right) X^{}_{{(j)}_{s'i}} X^{}_{{(j)}_{di}} \right)^2 }  \right] \\
&\leq~ D^{1.5} \rho^{}_W \rho^{}_C \rho^{}_B \E_{\xi}\left[\sup_{w,c, d, l, s, s'} | \sum_{j=1}^m \xi_{jc}  X^{}_{{(j)}_{sL}} \sum_{i=1}^L \right. \left. \left(\prod_{k=i+1}^L \exp\left(\sigma(S^{}_{\Delta} X^{}_{{(j)}_{*k}}) \cdot a^{}_{dl}\right) \right) X^{}_{{(j)}_{s'i}} X^{}_{{(j)}_{di}} |  \right] \\
&=~ D^{1.5} \rho^{}_W \rho^{}_C \rho^{}_B \E_{\xi}\left[\sup_{w,c, d, l, s, s'} | \sum_{j=1}^m \xi_{jc}  X^{}_{{(j)}_{sL}} \sum_{i=1}^L  \right. \left.\exp\left( \sum_{k=i+1}^L \sigma(S^{}_{\Delta} X^{}_{{(j)}_{*k}}) \cdot a^{}_{dl}\right) X^{}_{{(j)}_{s'i}} X^{}_{{(j)}_{di}} |  \right] \\
&\leq~ D^{1.5} \rho^{}_W \rho^{}_C  \rho^{}_B \sum_{i=1}^L \E_{\xi}\left[\sup_{w,c, d, l, s, s'} | \sum_{j=1}^m \xi_{jc}  X^{}_{{(j)}_{sL}} \cdot  \right. \left. \exp\left( \sum_{k=i+1}^L \sigma(S^{}_{\Delta} X^{}_{{(j)}_{*k}}) \cdot a^{}_{dl}\right) X^{}_{{(j)}_{s'i}} X^{}_{{(j)}_{di}} |  \right]
\end{aligned}
\end{equation*}
\end{small} where the second to last inequality follows from $\|x\|_2 \leq \sqrt{n} \|x\|_{\infty}$ for any $x \in \mathbb{R}^{n}$. 
Jensen's inequality gives the following inequality:
\begin{small}
\begin{equation}\label{eq:eq1}
\begin{aligned}
&~ \E_{\xi}\left[\sup_{w,c, d, l, s, s'} | \sum_{j=1}^m \xi_{jc}  X^{}_{{(j)}_{sL}} \cdot \right. \left. \exp\left( \sum_{k=i+1}^L \sigma(S^{}_{\Delta} X^{}_{{(j)}_{*k}}) \cdot a^{}_{dl}\right) X^{}_{{(j)}_{s'i}} X^{}_{{(j)}_{di}} |  \right] \\
% &\leq~ \E_{\xi}\left[\sup_{w,c, d, l, s, s'} \left| \sum_{j=1}^m \xi_{jc}  X^{}_{{(j)}_{sL}}  \exp\left( \sum_{k=i+1}^L \sigma(S^{}_{\Delta} X^{}_{{(j)}_{*k}}) \cdot a^{}_{dl}\right) X^{}_{{(j)}_{s'i}} X^{}_{{(j)}_{di}} \right|  \right] \\
&\leq~  \frac{1}{\lambda_i} \log ( \E_{\xi}\left[\sup_{w,c, d, l, s, s'}  \exp (\lambda_i  | \sum_{j=1}^m \xi_{jc}  X^{}_{{(j)}_{sL}} \cdot \right. \left.  \exp\left( \sum_{k=i+1}^L \sigma(S^{}_{\Delta} X^{}_{{(j)}_{*k}}) \cdot a^{}_{dl}\right) X^{}_{{(j)}_{s'i}} X^{}_{{(j)}_{di}} |)  \right] ) \\
&\leq~  \frac{1}{\lambda_i} \log ( \sum_{c, d, l, s, s'} \E_{\xi}\left[\sup_{w} \exp(\lambda_i  | \sum_{j=1}^m \xi_{jc}  X^{}_{{(j)}_{sL}} \cdot  \right. \left. \exp\left( \sum_{k=i+1}^L \sigma(S^{}_{\Delta} X^{}_{{(j)}_{*k}}) \cdot a^{}_{dl}\right) X^{}_{{(j)}_{s'i}} X^{}_{{(j)}_{di}} |)  \right] ) \\
&\leq~  \frac{1}{\lambda_i} \log ( \mathcal{C} D^3 N  \max_{c, d, l, s, s'} \E_{\xi}\left[\sup_{w} \exp(\lambda_i  | \sum_{j=1}^m \xi_{jc}  X^{}_{{(j)}_{sL}} \cdot \right.  \left. \exp\left( \sum_{k=i+1}^L \sigma(S^{}_{\Delta} X^{}_{{(j)}_{*k}}) \cdot a^{}_{dl}\right) X^{}_{{(j)}_{s'i}} X^{}_{{(j)}_{di}} |)  \right] ) :=\Theta \\
\end{aligned}
\end{equation}
\end{small}
For fixed $\lambda_i > 0$. The second inequality follows from the fact that $\sup_x \sup_y f(x,y) = \sup_{x,y} f(x,y)$.
We observe that the inner expectation $\sup$ depends only on $w$.
Next we use Lemma~\ref{lem:peeling} with $\sigma_{ij}(z) = \exp(z)X^{}_{{(j)}_{sL}} X^{}_{{(j)}_{s'i}} X^{}_{{(j)}_{di}}$ on its domain and $g_i(X_{(j)})=\sum_{k=i+1}^L \sigma(S^{}_{\Delta} X^{}_{{(j)}_{*k}}) \cdot a^{}_{dl}$. The corresponding Lipschitz constants are $l_{ij} = \max_{z \in dom(\sigma_{ij})}(\exp(z)X_{(j)_{sL}}X_{(j)_{s'i}} X_{(j)_{di}})$ and $l_i = \max(l_{ij})$. Therefore:
\begin{small}
\begin{equation}
\begin{aligned}
&\E_{\xi}\left[\sup_{w} \exp(\lambda_i | \sum_{j=1}^m \xi_{jc}  X^{}_{{(j)}_{sL}} \cdot \right. \left. \exp\left( \sum_{k=i+1}^L \sigma(S^{}_{\Delta} X^{}_{{(j)}_{*k}}) \cdot a^{}_{dl}\right) X^{}_{{(j)}_{s'i}} X^{}_{{(j)}_{di}} |)  \right] \\
% &\E_{\xi}\left[ \sup_{w}  \exp \left( \lambda_i  \left| \sum_{j=1}^m \xi_{jc}  X^{(j)}_{sL}  (\exp(\sum_{k=i+1}^L \Delta^{(j)}_{d,k}a_{dl})) X^{(j)}_{s'i} X^{(j)}_{di} \right| \right)\right] \\ 
&=\E_{\xi}\left[ \sup_{w}  \exp \left( \lambda_i  \left| \sum_{j=1}^m \xi_{jc} (\sigma_{ij}(g_i(X^{(j)}))) \right| \right)\right] \\
&\leq 2\E_{\xi}\left[ \sup_{w}  \exp \left( \lambda_i l_i \left| \sum_{j=1}^m \xi_{jc}    \sum_{k=i+1}^L \sigma(S^{}_{\Delta} X^{}_{{(j)}_{*k}}) \cdot a^{}_{dl} \right| \right)\right] \\
&\leq 2\E_{\xi}\left[ \sup_{w}  \exp \left( \lambda_i \rho^{}_A  l_i  \left| \sum_{j=1}^m \xi_{jc}    \sum_{k=i+1}^L \sigma(S^{}_{\Delta} X^{}_{{(j)}_{*k}}) \right| \right)\right] \\
&\leq 2\E_{\xi}\left[ \sup_{w}  \exp \left(\lambda_i \rho^{}_A  (L-i) l_i \left| \sum_{j=1}^m \xi_{jc}   \sigma(S^{}_{\Delta} X^{}_{{(j)}_{*k}}) \right| \right)\right] \\
&\leq 4\E_{\xi}\left[ \sup_{w}  \exp \left(\lambda_i \rho^{}_A  (L-i) l_i \left| \sum_{j=1}^m \xi_{jc}   S^{}_{\Delta} X^{}_{{(j)}_{*k}} \right| \right)\right] \\
\end{aligned}
\end{equation}
\end{small}
This follows from applying Lemma~\ref{lem:peeling} with $\sigma$ which has a Lipschitz constant of 1, as assumed. Hence:
\begin{small}
\begin{equation} \label{eq:lip}
\begin{aligned}
% &4\E_{\xi}\left[ \sup_{w}  \exp \left(\lambda_i \rho^{}_A   l_i \left| \sum_{j=1}^m \xi_{jc} \sum_{\substack{k=i+1 \\ S_{\Delta} X_{(j)_{*k}} > 0}}^{L} S^{}_{\Delta} X^{}_{{(j)}_{*k}} \right| \right)\right] \\
& 4\E_{\xi}\left[ \sup_{w}  \exp \left( \lambda_i \rho^{}_A  (L-i) l_i \left| \sum_{j=1}^m \xi_{jc} S^{}_{\Delta} X^{}_{{(j)}_{*k}} \right| \right)\right] \\
&\leq 4\E_{\xi}\left[ \sup_{k}  \exp \left( \lambda_i \rho^{}_A \rho^{}_{\Delta} (L-i) l_i || \sum_{j=1}^m \xi_{jc} X^{}_{{(j)}_{*k}} || \right)\right] \\
&\leq 4\E_{\xi}\left[ \sup_{t,k}  \exp \left( \lambda_i \sqrt{D} \rho^{}_A \rho^{}_{\Delta} (L-i) l_i \left| \sum_{j=1}^m \xi_{jc} X^{}_{{(j)}_{tk}} \right| \right)\right] \\
&\leq 4DL \max_{t, k} \E_{\xi}\left[ \exp \left(\lambda_i \sqrt{D} \rho^{}_A \rho^{}_{\Delta} (L-i) l_i  \left|  \sum_{j=1}^m \xi_{jc} X^{}_{{(j)}_{tk}}  \right| \right)\right] \\
% &\leq 4\E_{\xi}\left[ \sup_{w, k}  \exp \left( \lambda   \rho^{}_A \rho^{}_{\Delta} \sum_{i=1}^L (L-i)l_i  \left|\left| \sum_{j=1}^m \xi_{jc}  X^{}_{{(j)}_{*k}} \right|\right|_2 \right)\right] \\
% &\leq 2\E_{\xi}\left[ \sup_{w}  \exp \left( \lambda   \rho^{}_A \rho^{}_{\Delta} l_i  \left|\left| \sum_{j=1}^m \xi_{jc}  \sum_{k=i+1}^L  X^{}_{{(j)}_{*k}} \right|\right|_2 \right)\right] \\
% &\leq 2\E_{\xi}\left[ \exp \left( \lambda   \rho^{}_A \rho^{}_{\Delta} l_i  \left|\left| \sum_{j=1}^m \xi_{jc}  \sum_{k=i+1}^L  X^{}_{{(j)}_{*k}} \right|\right|_2 \right)\right] \\
\end{aligned}
\end{equation}
\end{small}

Denote:
\begin{small}
\begin{equation} \label{eq:lip2}
\begin{aligned}
M_i := \sqrt{D} \rho^{}_A \rho^{}_{\Delta} (L-i) l_i 
\end{aligned}
\end{equation}
\end{small}

% We notice that:

% \begin{small}
% \begin{equation} \label{eq:lip}
% \begin{aligned}
% &\leq 2\E_{\xi}\left[ \sup_{ k} \exp \left( (L-i)\lambda_i l_i \rho_{\Delta} \rho_A  \left| \left| \sum_{j=1}^m \xi_{jc}  X_{*k}^{(j)} \right|\right|_2 \right)\right] \\
% &\leq 2\E_{\xi}\left[ \sup_{ k} \exp \left( (L-i)\lambda_i l_i \rho_{\Delta} \rho_A  \sqrt{ \sum_{t=1}^D \left|  \sum_{j=1}^m \xi_{jc} X_{(j)_{tk}}  \right|^2 } \right)\right] \\
% &\leq 2\E_{\xi}\left[ \sup_{t,k} \exp \left( (L-i)\lambda_i l_i \rho_{\Delta} \rho_A \sqrt{D}  \left|  \sum_{j=1}^m \xi_{jc} X_{(j)_{tk}}  \right| \right)\right] \\
% &\leq 2DL \max_{t, k} \E_{\xi}\left[ \exp \left((L-i) \lambda_i l_i \rho_{\Delta} \rho_A \sqrt{D}  \left|  \sum_{j=1}^m \xi_{jc} X_{(j)_{tk}}  \right| \right)\right] \\
% \end{aligned}
% \end{equation}
% \end{small}
%  where the second to last inequality follows from $\|x\|_2 \leq \sqrt{n} \|x\|_{\infty}$ for any $x \in \mathbb{R}^{n}$. 

As a next step, we would like to bound the above term using a function of the data that is not dependent on an expected value of noise labels $\xi$. For this purpose we apply a technique that was introduced in the proof of Theorem~1 in~\citep{golowich2018size}. We apply this process separately for each $i \in [L]$. Let $i \in [L]$:
We define a random variable $Z$:
\begin{small}
\begin{equation*}
\begin{aligned}
&Z = M_i  \left| \sum_{j=1}^m \xi_{jc} X^{}_{{(j)}_{tk}} \right| \\
&z_j = X^{}_{{(j)}_{tk}} \rightarrow Z = M_i |\sum_{j=1}^m \xi_{jc} z_j | \\
\end{aligned}
\end{equation*}
\end{small}
The random variable $Z$ depends on the random variables $\xi_{jc}$.
Then, we have: 
\begin{small}
\begin{equation*}
\begin{aligned}
&=\frac{1}{\lambda_i} \log \E_{\xi}\left[\exp( \lambda _iZ) \right] \\
&=\frac{1}{\lambda_i} \log \E_{\xi}\left[\exp( \lambda_i Z + \lambda_i\E(Z) - \lambda_i \E(Z)) \right] \\
&=\frac{1}{\lambda_i} \log \E_{\xi}\left[\exp( \lambda_i Z - \lambda_i \E(Z)) \right] +  \E_{\xi} (Z) \\
\end{aligned}
\end{equation*}
\end{small}
By Jensen’s inequality, we obtain a bound for $\E(|\sum_{j=1}^m \xi_{jc} z_j |)$:
\begin{small}
\begin{equation*}
\begin{aligned}
&\E_{\xi} (|\sum_{j=1}^m \xi_{jc} z_j |) = \E_{\xi} (\sqrt{|\sum_{j=1}^m \xi_{jc} z_j |^2}) \leq \sqrt{\E_{\xi} (|\sum_{j=1}^m \xi_{jc} z_j |^2)} = \\& \sqrt{\E_{\xi} (|\sum_{j=1}^m \xi_{jc} z_j} |^2) = \sqrt{\E_{\xi} (|\sum_{j,j'=1}^m \xi_{jc} \xi_{j'c} z_j z_{j'}} |) = \sqrt{\sum_{j=1}^m |z_j|^2} 
\end{aligned}
\end{equation*}
\end{small}
Namely $\E_{\xi}(Z) \leq M_i \sqrt{\sum_{j=1}^m |z_j|^2} $.
$Z$ is a deterministic function of the i.i.d. random variables  $\xi_{jc}$ and satisfies the following:
\begin{small}
\begin{equation*}
\begin{aligned}
Z(\xi_{1c},..., \xi_{jc},...,\xi_{mc}) - Z(\xi_{1c},..., -\xi_{jc},...,\xi_{mc}) \leq 2|z_j|
\end{aligned}
\end{equation*}
\end{small}
This follows from the triangle inequality.
This means that $Z$ satisfies a bounded-difference condition, which, by the proof of Theorem~6.2 in ~\citep{Boucheron2010}, implies that $Z$ is sub-Gaussian, with variance factor:
\begin{small}
\begin{equation*}
\begin{aligned}
v = \frac{1}{4} \sum_{j=1}^m (2M_i|z_j|)^2 = M_i^2 \sum_{j=1}^m |z_j|^2
\end{aligned}
\end{equation*}
\end{small}
It follows that:
\begin{small}
\begin{equation*}
\begin{aligned}
&\frac{1}{\lambda_i} \log \E_{\xi}\left[\exp( \lambda_i Z - \lambda_i \E_{\xi}(Z)) \right] \leq \\& \frac{1}{\lambda_i} \frac{\lambda_i^2 M_i^2 \sum_{j=1}^m |z_j|^2}{2} =  \frac{\lambda_i M_i^2 \sum_{j=1}^m |z_j|^2}{2}  
\end{aligned}
\end{equation*}
\end{small}
Therefore:
\begin{small}
\begin{equation}\label{eq:eq2}
\begin{aligned}
&\frac{1}{\lambda_i} \log \E_{\xi}\left[ \exp( \lambda_i Z - \lambda_i \E_{\xi}(Z)) \right] +  \E_{\xi} (Z) \\
&\leq \frac{\lambda_i M_i^2 \sum_{j=1}^m |z_j|^2}{2} + M_i\sqrt{\sum_{j=1}^m |z_j|^2} 
\end{aligned}
\end{equation}
\end{small}
\paragraph{Analyzing Lipschitz constants $l_{ij}$.}
Next, we analyze the Lipschitz constants \( l_{ij} \).
For $l_{ij} = \max_{z \in dom(\sigma_{ij})}(\exp(z)X_{(j)_{sL}}X_{(j)_{s'i}} X_{(j)_{di}})$ and $l_i = \max(l_{ij})$ is of the form $g_i(X_{(j)})=\sum_{k=i+1}^L \sigma(S^{}_{\Delta} X^{}_{{(j)}_{*k}}) \cdot a^{}_{dl}$ (see \eqref{eq:lip}). Since  
\begin{small}
\begin{equation*}
\begin{aligned}
&l_i = \max_{j} l_{ij} = \max_{j} \max_{z \in dom(\sigma_{ij})}(\exp(z)X^{(j)}_{sL}X^{(j)}_{s'i} X^{(j)}_{di}) \\& \leq \max_{j} \max_{z \in dom(\sigma_{ij})} \exp \left( \sum_{k=i+1}^L \sigma(S^{}_{\Delta} X^{}_{{(j)}_{*k}}) \cdot a^{}_{dl} \right) \cdot 1 < K^{L-i}
\end{aligned}
\end{equation*}
\end{small}
which is followed from our assumptions. 
% {\color{black}$\forall n \in [N]:(\exp(\Delta^{(j)}_{d,k} * A_{d*}))_n < K < 1$ and the fact that the data is normalized.}
% We get:
% \begin{small}
% \begin{equation*}
% \begin{aligned}
% \sum_{i=1}^L l_i &\leq \sum_{i=1}^{L} K^{L-i} = \frac{K^L - 1}{K - 1}  \\
% % &= \frac{(L-1)M^{L+1} - LM^L + M}{(M-1)^2 M}
% \end{aligned}
% \end{equation*}
% \end{small}

% We conclude that:
% \begin{small}
% \begin{equation*}
% \begin{aligned}
% \lim_{L \rightarrow \infty} \frac{K^L - 1}{K - 1}= \frac{1}{1-K}
% \end{aligned}
% \end{equation*}
% \end{small}

We get:
\begin{small}
\begin{equation*}
\begin{aligned}
\sum_{i=1}^L l_i (L-i) &\leq \sum_{i=1}^{L} (L-i) (K)^{L-i} = \frac{(L-1)K^{L+1} - L K^L +K}{(K-1)^2}  \\
% &= \frac{(L-1)M^{L+1} - LM^L + M}{(M-1)^2 M}
\end{aligned}
\end{equation*}
\end{small}

We conclude that:
\begin{small}
\begin{equation*}
\begin{aligned}
\lim_{L \rightarrow \infty} \frac{(L-1)K^{L+1} - L K^L +K}{(K-1)^2}= \frac{K}{(K-1)^2}
\end{aligned}
\end{equation*}
\end{small}

\paragraph{Concluding the proof.} By combining ~\eqref{eq:eq1},~\eqref{eq:lip} and~\eqref{eq:eq2}, we have:
\begin{small}
\begin{equation*}
\begin{aligned}
& { \Theta} \leq  \frac{1}{\lambda_i} \log \left(4 L \mathcal{C} D^4 N \right) + \max_{c, d, l, s, s', t, k} \frac{1}{\lambda_i} \cdot \\& \log \left(  \E_{\xi}\left[ \exp \left(\lambda_i \sqrt{D} \rho^{}_A \rho^{}_{\Delta} (L-i) l_i  \left|  \sum_{j=1}^m \xi_{jc} 
X^{}_{{(j)}_{tk}}  \right| \right)\right] \right) \\
&\leq \frac{1}{\lambda_i}  \log \left(4L \mathcal{C} D^4 N \right) +  \max_{c, d, l, s, s', t, k} \frac{\lambda_i (\sqrt{D} \rho^{}_A \rho^{}_{\Delta} (L-i) l_i)^2 \sum_{j=1}^m (X_{(j)_{tk}})^2}{2} + (\sqrt{D} \rho^{}_A \rho^{}_{\Delta} (L-i) l_i)\sqrt{\sum_{j=1}^m (X_{(j)_{tk}})^2} \\
\end{aligned}
\end{equation*}
\end{small}
We choose $\lambda_i = \sqrt{\frac{2\log (4 L \mathcal{C} D^4 N)}{M_i^2  \max_{t,k}\sum_{j=1}^m (X_{(j)_{tk}})^2}}$ which minimizes the above term and obtain the following inequality:
\begin{small}
\begin{equation*}
\begin{aligned}
&\sum_{i=1}^L \frac{1}{\lambda_i} \log \left(4L \mathcal{C} D^4 N \right) + \max_{t, k} \frac{\lambda_i (\sqrt{D} \rho^{}_A \rho^{}_{\Delta} (L-i) l_i)^2 \sum_{j=1}^m (X_{(j)_{tk}})^2}{2} + (\sqrt{D} \rho^{}_A \rho^{}_{\Delta} (L-i) l_i)\sqrt{\sum_{j=1}^m (X_{(j)_{tk}})^2}\\
&\leq \sum_{i=1}^L (1 + \sqrt{2\log (4L \mathcal{C} D^4 N)})M_i\sqrt{ \max_{t, k}\sum_{j=1}^m (X_{(j)_{tk}})^2}\\
&= \sum_{i=1}^L (1 + \sqrt{2\log (4L \mathcal{C} D^4 N)}) \sqrt{D} \rho^{}_A \rho^{}_{\Delta} (L-i) l_i \cdot \sqrt{ \max_{t, k}\sum_{j=1}^m (X_{(j)_{tk}})^2}\\
&\leq (1 + \sqrt{2\log (4L \mathcal{C} D^4 N)}) \sqrt{D} \rho^{}_A \rho^{}_{\Delta}\sqrt{\max_{t, k} \sum_{j=1}^m (X_{(j)_{tk}})^2} \frac{K}{(K-1)^2}\\
% &\frac{1}{\lambda} \log \left(2 \mathcal{C} D^3 N \right)  +  \frac{\lambda M^2  \sum_{j=1}^m \left|\left| \sum_{k=i+1}^L  X^{}_{{(j)}_{*k}} \right|\right|^2_2}{2}  +M  \sqrt{ \sum_{j=1}^m \left|\left| \sum_{k=i+1}^L  X^{}_{{(j)}_{*k}} \right|\right|^2_2} \\
% &\frac{1}{\lambda} \log \left( \mathcal{C} 2^{3(H-1)} D^4 N L^{H} \right)  +  \max_{c, d, l, s, s'}  \max_{k_1,...,k_{H-1}, k, t}  \frac{\lambda M^2 \sum_{j=1}^m (X_{{(j)}_{tk}})^2}{2} + M\sqrt{\sum_{j=1}^m (X_{{(j)}_{tk}})^2} \\
% &\leq (1 + \sqrt{2\log (2 \mathcal{C} D^3 N)})M  \sqrt{ \sum_{j=1}^m \left|\left| \sum_{k=i+1}^L  X^{}_{{(j)}_{*k}} \right|\right|^2_2}  \\
\end{aligned}
\end{equation*}
\end{small}
We conclude that:
\begin{small}
\begin{equation*}
\begin{aligned}
&m\mathcal{R(\mathcal{F}_{\rho})} \\ &\leq 
 D^{2} \Gamma (1 + \sqrt{2\log (4L \mathcal{C} D^4 N)}) \sqrt{\max_{t, k} \sum_{j=1}^m (X_{(j)_{tk}})^2} \frac{K}{(K-1)^2}\\
 % \\&\leq
 % D^{1.5} \rho^{}_W \rho^{}_C \rho^{}_B \rho^{}_A \rho^{}_{\Delta} (1 + \sqrt{2\log (2 \mathcal{C} D^3 N)})  \frac{1}{1-K}   \sqrt{\max_i \sum_{j=1}^m \left|\left| \sum_{k=i+1}^L  X^{}_{{(j)}_{*k}} \right|\right|^2_2}\\
\end{aligned}
\end{equation*}
\end{small}
\end{proof}
\setcounter{theorem}{4}
\begin{theorem}\label{theorem:genproof}
Let $P$ be a distribution over $\mathbb{R}^{D \times L} \times [C]$ and $\delta > 0$. Let $S = \{( X^{}_{(j)},y_{(j)})\}^{m}_{j=1}$ be a dataset of i.i.d. samples selected from $P$. Assume that $\forall j \in [m]: ||X_{(j)}||_{\max} \leq 1$. Additionally, suppose $\forall k \in [L], d \in [D]:||\bar{A}^{}_{dk}||_{\max} < K < 1$. Then, with probability at least $1-\delta$ over the selection of $S$, for any $f_w \in \mathcal{F}$, 
\begin{small}
\begin{equation*}
\begin{aligned}
&\err_P(f_w) - \fr{1}{m}\sum^{m}_{j=1}\bI[\max_{c \neq c'}(f^c_w( X_{(j)})) + \gamma \geq f^{c'}_w( X^{}_{(j)})] \\& = \err_P(f_w) - \err^\gamma_S(f_w) \leq \frac{2\sqrt{2}}{\gamma m} ({\Gamma(w) } +{\frac{1}{D^2N^2}}) D^{2} \cdot \\& (1 + \sqrt{2\log (4L \mathcal{C} D^4 N)}) \sqrt{\max_{t, k} \sum_{j=1}^m (X_{(j)_{tk}})^2} \frac{K}{(K-1)^2} \\&+ 3\sqrt{\frac{\log(2/\delta)+2\log({D^2N^2\Gamma(w) }+2)}{2m}},
\end{aligned}
\end{equation*}
\end{small}
where the maximum is taken over \(t \in [D]\),  \(k \in [L]\). 

\end{theorem}
\begin{proof}
For aesthetic purposes, we define \( B \) as \( S_B \) and \( C \) as \( S_C \). We want to prove the bound for all $f_w \in \mathcal{F}$ where:
{
\begin{small}
\begin{equation*}
\begin{aligned}
&\mathcal{F} := \\&\{f_w : w = (A, B, C, S_\Delta, W), \forall k \in [L], d \in [D]:||\bar{A}^{}_{dk}||_{\max} < K < 1 \}
\end{aligned}
\end{equation*}
\end{small}}
Let $t \in \mathbb{N}$. Denote:
\begin{small}
\begin{equation*}
\begin{aligned}
&\mathcal{S}(t) := \{f_w \in \mathcal{F} , \Gamma(w) < {\frac{t}{D^2N^2}} \}
\end{aligned}
\end{equation*}
\end{small}
Correspondingly subdivide $\delta$ as:
\begin{small}
\begin{equation*}
\begin{aligned}
&\delta(t) := \frac{\delta}{t(t+1)}
\end{aligned}
\end{equation*}
\end{small}

By Lemma ~\ref{lem:loss_ramp} and Theorem ~\ref{theorem:rad}, with probability at least $1-\delta(t)$:
for any function $f_w \in \mathcal{S}(t) $, we have the following inequality:
\begin{small}
\begin{equation*}
\begin{aligned}
\err_P(f_w) - \err^\gamma_S(f_w) \leq \frac{2\sqrt{2}}{\gamma} \cdot \mathcal{R}(\mathcal{S}(t)) + 3\sqrt{\frac{\log(2/\delta(t))}{2m}}.
\end{aligned}
\end{equation*}
\end{small}
Using the union bound over all possible set $\mathcal{S}(t)$, we establish that the above probabilistic bound holds uniformly for all functions $f_w \in \mathcal{S}(t)$ with probability at least $1 - \delta$.
Hence, let  $f_w \in \mathcal{F}$ with weight vector {$w = (A, B, C, S_{\Delta}, W)$}. We choose the smallest $(t)$ such that, $f_w \in \mathcal{S}(t)$. We have:
\begin{small}
\begin{equation*}
\begin{aligned}
&\err_P(f_w) - \err^\gamma_S(f_w) \leq \frac{2\sqrt{2}}{\gamma} \cdot \mathcal{R}(S(t)) + 3\sqrt{\frac{\log(2/\delta(t))}{2m}}
\\ &= \frac{2\sqrt{2}}{\gamma m} \frac{t}{D^2N^2} D^{2} (1 + \sqrt{2\log (4L \mathcal{C} D^4 N)})\cdot \sqrt{\max_{t, k} \sum_{j=1}^m (X_{(j)_{tk}})^2} \frac{K}{(K-1)^2} + 3\sqrt{\frac{\log(2/\delta)+2\log(t+1)}{2m}}
\\ &\leq \frac{2\sqrt{2}}{\gamma m} ({\Gamma(w) } +{\frac{1}{D^2N^2}}) D^{2} (1 + \sqrt{2\log (4L \mathcal{C} D^4 N)}) \cdot  \sqrt{\max_{t, k} \sum_{j=1}^m (X_{(j)_{tk}})^2} \frac{K}{(K-1)^2}  3\sqrt{\frac{\log(2/\delta)+2\log({D^2N^2\Gamma(w) }+2)}{2m}}
\end{aligned}
\end{equation*}
\end{small}
\end{proof}



\section{Additional Related Work on Generalization\label{sec:RelatedWorkGeneralization}}
Explaining the performance of overparameterized deep neural networks (DNNs) on test data remains a major challenge in deep learning theory. Traditional tools like the PAC-learning framework and VC-dimension often provide vacuous bounds when the number of parameters greatly exceeds the number of data points. To address this, many studies conduct architecture-specific analyses. For instance, Allen et al. analyze the dynamics of stochastic gradient descent on RNNs with ReLU activations, offering optimization and generalization guarantees~\citep{allen2019can}. Other works have explored the generalization of RNNs for unseen data and longer sequences under various assumptions~\citep{cohen2022implicit, emami2021implicit, cohen2022learning, hardt2018gradient}.

There are relatively few results that address modern architectures such as S6 layers. A recent contribution proposes a bound for standard SSMs~\citep{liu2024generalization}, but it does not extend to Selective SSMs. To address this gap, we propose a new bound specifically for Selective SSMs.

\section{Limitations\label{sec:limitations}}
In this paper, we provide a new perspective on the expressivity gap between S6 layers and self-attention, the core layers of Mamba models and transformers. While our analysis leverages multivariate polynomial degrees as measures of expressiveness and offers insightful results, it does not formally connect these measures to widely-used expressivity metrics in the literature, such as Rademacher complexity or VC dimension. Establishing such connections remains an open challenge.

Additionally, our work focuses on a simplified architecture, omitting several components of the original models. While we empirically justify these simplifications and highlight the opportunity to use simpler models by identifying the key components responsible for the performance gap, analyzing the full complexity of softmax-based Transformer and Mamba architectures is an important direction for future research. 

Finally, although expressiveness is a critical property to explore, LLMs with billions of parameters involve additional factors that influence their capacity and performance. These include optimization challenges, gradient behavior, implicit biases, and training stability. Addressing these aspects is beyond the scope of this study, which is centered on a theoretical characterization of expressiveness. We believe these topics represent promising avenues for future work.
%%%%%%%%%%%%%%%%%%%%%%%%%%%%%%%%%%%%%%%%%%%%%%%%%%%%%%%%%%%%%%%%%%%%%%%%%%%%%%%
%%%%%%%%%%%%%%%%%%%%%%%%%%%%%%%%%%%%%%%%%%%%%%%%%%%%%%%%%%%%%%%%%%%%%%%%%%%%%%%


\end{document}


% This document was modified from the file originally made available by
% Pat Langley and Andrea Danyluk for ICML-2K. This version was created
% by Iain Murray in 2018, and modified by Alexandre Bouchard in
% 2019 and 2021 and by Csaba Szepesvari, Gang Niu and Sivan Sabato in 2022.
% Modified again in 2023 and 2024 by Sivan Sabato and Jonathan Scarlett.
% Previous contributors include Dan Roy, Lise Getoor and Tobias
% Scheffer, which was slightly modified from the 2010 version by
% Thorsten Joachims & Johannes Fuernkranz, slightly modified from the
% 2009 version by Kiri Wagstaff and Sam Roweis's 2008 version, which is
% slightly modified from Prasad Tadepalli's 2007 version which is a
% lightly changed version of the previous year's version by Andrew
% Moore, which was in turn edited from those of Kristian Kersting and
% Codrina Lauth. Alex Smola contributed to the algorithmic style files.
