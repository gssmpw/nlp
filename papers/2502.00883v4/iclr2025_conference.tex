
\documentclass{article} % For LaTeX2e
\usepackage{iclr2025_conference,times}
\iclrfinalcopy

\usepackage{url}            %
\usepackage{booktabs}       %
\usepackage{amsfonts}
\usepackage{bbm}%
\usepackage{nicefrac}       %
\usepackage{microtype}      %
\usepackage{xcolor}         %
\usepackage{fleqn, tabularx}
\usepackage{multirow}
\usepackage[export]{adjustbox}
\usepackage{amsfonts}
\usepackage{amsmath, amsthm, bbm, mathtools, commath}
\usepackage{amssymb}
\usepackage{colortbl}
\usepackage{wrapfig}
\usepackage{algorithm}
\usepackage[noend]{algorithmic}
\usepackage{quoting}
% \usepackage[T1]{fontenc}
\usepackage{bold-extra}
% \usepackage{lmodern}

\newtheorem{theorem}{Theorem}[section]
\newtheorem{proposition}[theorem]{Proposition}
\newtheorem{lemma}[theorem]{Lemma}
\newtheorem{corollary}[theorem]{Corollary}
\newtheorem{definition}[theorem]{Definition}
\newtheorem{assumption}[theorem]{Assumption}
\usepackage{pifont}% http://ctan.org/pkg/pifont
\definecolor{mark}{RGB}{208,64,56}
\newcommand{\cmark}{{\color{mark}\ding{51}}}
\newcommand{\xmark}{\ding{55}}

\usepackage[algo2e,linesnumbered,ruled,lined]{algorithm2e}
\usepackage{datetime}
\usepackage{verbatim}




\usepackage[font=small]{subfig}
\usepackage{colortbl}
\usepackage{arydshln} %
\newcommand{\facc}[2]{{#1}{\scriptsize±{#2}}}
\newcommand{\bfacc}[2]{\textbf{{#1}{\scriptsize±{#2}}}}
% \newcommand{\bestcell}{\cellcolor{}}
\newcommand{\ind}{\mathbbm{1}}
\newcommand{\NN}{\mathbb N}
\newcommand{\RR}{\mathbb R}
\newcommand{\mc}[1]{\mathcal{#1}}
\newcommand{\mbb}[1]{\\usepackage{bbm}{#1}}\usepackage{bbm}
\newcommand{\mrm}[1]{\mathrm{#1}}
\newcommand{\ol}[1]{\overline{#1}}
\newcommand{\mbf}{\mathbf}
\newcommand{\todo}[1]{\textcolor{red}{TODO: #1}}
\newcommand{\bestcell}{\cellcolor{blue!25}}
\newcommand{\myparagraph}[1]{\noindent \textbf{#1}}
\newcommand{\grayrow}{\rowcolor[gray]{.9}}
\newcommand{\indd}{\textit{ind}\xspace}
\newcommand{\tran}{\textit{tran}\xspace}
\newcommand{\production}{\textit{prod}\xspace}
\newcommand{\printfnsymbol}[1]{%
  \textsuperscript{\@fnsymbol{#1}}%
}
\usepackage{paralist}
\makeatother
\newcommand\eqc{\mathrel{\overset{\makebox[0pt]{\mbox{\normalfont\tiny\sffamily c}}}{=}}}
\newcommand{\indep}{\perp \!\!\! \perp}
\newcommand{\A}{\mbf{A}}
\newcommand{\X}{\mbf{X}}
\newcommand{\W}{\mbf{W}}
\newcommand{\Y}{\mbf{Y}}
\newcommand{\hA}{\mbf{h}_\A}
\newcommand{\hX}{\mbf{h}_\X}
% \usepackage{subfigure}
\setlength\dashlinegap{3pt}
\setlength\arrayrulewidth{.5pt}
\setlength\dashlinedash{4pt}
\newcommand{\std}[1]{\pm #1}

% Optional math commands from https://github.com/goodfeli/dlbook_notation.
% %%%%% NEW MATH DEFINITIONS %%%%%

\usepackage{amsmath,amsfonts,bm}
\usepackage{derivative}
% Mark sections of captions for referring to divisions of figures
\newcommand{\figleft}{{\em (Left)}}
\newcommand{\figcenter}{{\em (Center)}}
\newcommand{\figright}{{\em (Right)}}
\newcommand{\figtop}{{\em (Top)}}
\newcommand{\figbottom}{{\em (Bottom)}}
\newcommand{\captiona}{{\em (a)}}
\newcommand{\captionb}{{\em (b)}}
\newcommand{\captionc}{{\em (c)}}
\newcommand{\captiond}{{\em (d)}}

% Highlight a newly defined term
\newcommand{\newterm}[1]{{\bf #1}}

% Derivative d 
\newcommand{\deriv}{{\mathrm{d}}}

% Figure reference, lower-case.
\def\figref#1{figure~\ref{#1}}
% Figure reference, capital. For start of sentence
\def\Figref#1{Figure~\ref{#1}}
\def\twofigref#1#2{figures \ref{#1} and \ref{#2}}
\def\quadfigref#1#2#3#4{figures \ref{#1}, \ref{#2}, \ref{#3} and \ref{#4}}
% Section reference, lower-case.
\def\secref#1{section~\ref{#1}}
% Section reference, capital.
\def\Secref#1{Section~\ref{#1}}
% Reference to two sections.
\def\twosecrefs#1#2{sections \ref{#1} and \ref{#2}}
% Reference to three sections.
\def\secrefs#1#2#3{sections \ref{#1}, \ref{#2} and \ref{#3}}
% Reference to an equation, lower-case.
\def\eqref#1{equation~\ref{#1}}
% Reference to an equation, upper case
\def\Eqref#1{Equation~\ref{#1}}
% A raw reference to an equation---avoid using if possible
\def\plaineqref#1{\ref{#1}}
% Reference to a chapter, lower-case.
\def\chapref#1{chapter~\ref{#1}}
% Reference to an equation, upper case.
\def\Chapref#1{Chapter~\ref{#1}}
% Reference to a range of chapters
\def\rangechapref#1#2{chapters\ref{#1}--\ref{#2}}
% Reference to an algorithm, lower-case.
\def\algref#1{algorithm~\ref{#1}}
% Reference to an algorithm, upper case.
\def\Algref#1{Algorithm~\ref{#1}}
\def\twoalgref#1#2{algorithms \ref{#1} and \ref{#2}}
\def\Twoalgref#1#2{Algorithms \ref{#1} and \ref{#2}}
% Reference to a part, lower case
\def\partref#1{part~\ref{#1}}
% Reference to a part, upper case
\def\Partref#1{Part~\ref{#1}}
\def\twopartref#1#2{parts \ref{#1} and \ref{#2}}

\def\ceil#1{\lceil #1 \rceil}
\def\floor#1{\lfloor #1 \rfloor}
\def\1{\bm{1}}
\newcommand{\train}{\mathcal{D}}
\newcommand{\valid}{\mathcal{D_{\mathrm{valid}}}}
\newcommand{\test}{\mathcal{D_{\mathrm{test}}}}

\def\eps{{\epsilon}}


% Random variables
\def\reta{{\textnormal{$\eta$}}}
\def\ra{{\textnormal{a}}}
\def\rb{{\textnormal{b}}}
\def\rc{{\textnormal{c}}}
\def\rd{{\textnormal{d}}}
\def\re{{\textnormal{e}}}
\def\rf{{\textnormal{f}}}
\def\rg{{\textnormal{g}}}
\def\rh{{\textnormal{h}}}
\def\ri{{\textnormal{i}}}
\def\rj{{\textnormal{j}}}
\def\rk{{\textnormal{k}}}
\def\rl{{\textnormal{l}}}
% rm is already a command, just don't name any random variables m
\def\rn{{\textnormal{n}}}
\def\ro{{\textnormal{o}}}
\def\rp{{\textnormal{p}}}
\def\rq{{\textnormal{q}}}
\def\rr{{\textnormal{r}}}
\def\rs{{\textnormal{s}}}
\def\rt{{\textnormal{t}}}
\def\ru{{\textnormal{u}}}
\def\rv{{\textnormal{v}}}
\def\rw{{\textnormal{w}}}
\def\rx{{\textnormal{x}}}
\def\ry{{\textnormal{y}}}
\def\rz{{\textnormal{z}}}

% Random vectors
\def\rvepsilon{{\mathbf{\epsilon}}}
\def\rvphi{{\mathbf{\phi}}}
\def\rvtheta{{\mathbf{\theta}}}
\def\rva{{\mathbf{a}}}
\def\rvb{{\mathbf{b}}}
\def\rvc{{\mathbf{c}}}
\def\rvd{{\mathbf{d}}}
\def\rve{{\mathbf{e}}}
\def\rvf{{\mathbf{f}}}
\def\rvg{{\mathbf{g}}}
\def\rvh{{\mathbf{h}}}
\def\rvu{{\mathbf{i}}}
\def\rvj{{\mathbf{j}}}
\def\rvk{{\mathbf{k}}}
\def\rvl{{\mathbf{l}}}
\def\rvm{{\mathbf{m}}}
\def\rvn{{\mathbf{n}}}
\def\rvo{{\mathbf{o}}}
\def\rvp{{\mathbf{p}}}
\def\rvq{{\mathbf{q}}}
\def\rvr{{\mathbf{r}}}
\def\rvs{{\mathbf{s}}}
\def\rvt{{\mathbf{t}}}
\def\rvu{{\mathbf{u}}}
\def\rvv{{\mathbf{v}}}
\def\rvw{{\mathbf{w}}}
\def\rvx{{\mathbf{x}}}
\def\rvy{{\mathbf{y}}}
\def\rvz{{\mathbf{z}}}

% Elements of random vectors
\def\erva{{\textnormal{a}}}
\def\ervb{{\textnormal{b}}}
\def\ervc{{\textnormal{c}}}
\def\ervd{{\textnormal{d}}}
\def\erve{{\textnormal{e}}}
\def\ervf{{\textnormal{f}}}
\def\ervg{{\textnormal{g}}}
\def\ervh{{\textnormal{h}}}
\def\ervi{{\textnormal{i}}}
\def\ervj{{\textnormal{j}}}
\def\ervk{{\textnormal{k}}}
\def\ervl{{\textnormal{l}}}
\def\ervm{{\textnormal{m}}}
\def\ervn{{\textnormal{n}}}
\def\ervo{{\textnormal{o}}}
\def\ervp{{\textnormal{p}}}
\def\ervq{{\textnormal{q}}}
\def\ervr{{\textnormal{r}}}
\def\ervs{{\textnormal{s}}}
\def\ervt{{\textnormal{t}}}
\def\ervu{{\textnormal{u}}}
\def\ervv{{\textnormal{v}}}
\def\ervw{{\textnormal{w}}}
\def\ervx{{\textnormal{x}}}
\def\ervy{{\textnormal{y}}}
\def\ervz{{\textnormal{z}}}

% Random matrices
\def\rmA{{\mathbf{A}}}
\def\rmB{{\mathbf{B}}}
\def\rmC{{\mathbf{C}}}
\def\rmD{{\mathbf{D}}}
\def\rmE{{\mathbf{E}}}
\def\rmF{{\mathbf{F}}}
\def\rmG{{\mathbf{G}}}
\def\rmH{{\mathbf{H}}}
\def\rmI{{\mathbf{I}}}
\def\rmJ{{\mathbf{J}}}
\def\rmK{{\mathbf{K}}}
\def\rmL{{\mathbf{L}}}
\def\rmM{{\mathbf{M}}}
\def\rmN{{\mathbf{N}}}
\def\rmO{{\mathbf{O}}}
\def\rmP{{\mathbf{P}}}
\def\rmQ{{\mathbf{Q}}}
\def\rmR{{\mathbf{R}}}
\def\rmS{{\mathbf{S}}}
\def\rmT{{\mathbf{T}}}
\def\rmU{{\mathbf{U}}}
\def\rmV{{\mathbf{V}}}
\def\rmW{{\mathbf{W}}}
\def\rmX{{\mathbf{X}}}
\def\rmY{{\mathbf{Y}}}
\def\rmZ{{\mathbf{Z}}}

% Elements of random matrices
\def\ermA{{\textnormal{A}}}
\def\ermB{{\textnormal{B}}}
\def\ermC{{\textnormal{C}}}
\def\ermD{{\textnormal{D}}}
\def\ermE{{\textnormal{E}}}
\def\ermF{{\textnormal{F}}}
\def\ermG{{\textnormal{G}}}
\def\ermH{{\textnormal{H}}}
\def\ermI{{\textnormal{I}}}
\def\ermJ{{\textnormal{J}}}
\def\ermK{{\textnormal{K}}}
\def\ermL{{\textnormal{L}}}
\def\ermM{{\textnormal{M}}}
\def\ermN{{\textnormal{N}}}
\def\ermO{{\textnormal{O}}}
\def\ermP{{\textnormal{P}}}
\def\ermQ{{\textnormal{Q}}}
\def\ermR{{\textnormal{R}}}
\def\ermS{{\textnormal{S}}}
\def\ermT{{\textnormal{T}}}
\def\ermU{{\textnormal{U}}}
\def\ermV{{\textnormal{V}}}
\def\ermW{{\textnormal{W}}}
\def\ermX{{\textnormal{X}}}
\def\ermY{{\textnormal{Y}}}
\def\ermZ{{\textnormal{Z}}}

% Vectors
\def\vzero{{\bm{0}}}
\def\vone{{\bm{1}}}
\def\vmu{{\bm{\mu}}}
\def\vtheta{{\bm{\theta}}}
\def\vphi{{\bm{\phi}}}
\def\va{{\bm{a}}}
\def\vb{{\bm{b}}}
\def\vc{{\bm{c}}}
\def\vd{{\bm{d}}}
\def\ve{{\bm{e}}}
\def\vf{{\bm{f}}}
\def\vg{{\bm{g}}}
\def\vh{{\bm{h}}}
\def\vi{{\bm{i}}}
\def\vj{{\bm{j}}}
\def\vk{{\bm{k}}}
\def\vl{{\bm{l}}}
\def\vm{{\bm{m}}}
\def\vn{{\bm{n}}}
\def\vo{{\bm{o}}}
\def\vp{{\bm{p}}}
\def\vq{{\bm{q}}}
\def\vr{{\bm{r}}}
\def\vs{{\bm{s}}}
\def\vt{{\bm{t}}}
\def\vu{{\bm{u}}}
\def\vv{{\bm{v}}}
\def\vw{{\bm{w}}}
\def\vx{{\bm{x}}}
\def\vy{{\bm{y}}}
\def\vz{{\bm{z}}}

% Elements of vectors
\def\evalpha{{\alpha}}
\def\evbeta{{\beta}}
\def\evepsilon{{\epsilon}}
\def\evlambda{{\lambda}}
\def\evomega{{\omega}}
\def\evmu{{\mu}}
\def\evpsi{{\psi}}
\def\evsigma{{\sigma}}
\def\evtheta{{\theta}}
\def\eva{{a}}
\def\evb{{b}}
\def\evc{{c}}
\def\evd{{d}}
\def\eve{{e}}
\def\evf{{f}}
\def\evg{{g}}
\def\evh{{h}}
\def\evi{{i}}
\def\evj{{j}}
\def\evk{{k}}
\def\evl{{l}}
\def\evm{{m}}
\def\evn{{n}}
\def\evo{{o}}
\def\evp{{p}}
\def\evq{{q}}
\def\evr{{r}}
\def\evs{{s}}
\def\evt{{t}}
\def\evu{{u}}
\def\evv{{v}}
\def\evw{{w}}
\def\evx{{x}}
\def\evy{{y}}
\def\evz{{z}}

% Matrix
\def\mA{{\bm{A}}}
\def\mB{{\bm{B}}}
\def\mC{{\bm{C}}}
\def\mD{{\bm{D}}}
\def\mE{{\bm{E}}}
\def\mF{{\bm{F}}}
\def\mG{{\bm{G}}}
\def\mH{{\bm{H}}}
\def\mI{{\bm{I}}}
\def\mJ{{\bm{J}}}
\def\mK{{\bm{K}}}
\def\mL{{\bm{L}}}
\def\mM{{\bm{M}}}
\def\mN{{\bm{N}}}
\def\mO{{\bm{O}}}
\def\mP{{\bm{P}}}
\def\mQ{{\bm{Q}}}
\def\mR{{\bm{R}}}
\def\mS{{\bm{S}}}
\def\mT{{\bm{T}}}
\def\mU{{\bm{U}}}
\def\mV{{\bm{V}}}
\def\mW{{\bm{W}}}
\def\mX{{\bm{X}}}
\def\mY{{\bm{Y}}}
\def\mZ{{\bm{Z}}}
\def\mBeta{{\bm{\beta}}}
\def\mPhi{{\bm{\Phi}}}
\def\mLambda{{\bm{\Lambda}}}
\def\mSigma{{\bm{\Sigma}}}

% Tensor
\DeclareMathAlphabet{\mathsfit}{\encodingdefault}{\sfdefault}{m}{sl}
\SetMathAlphabet{\mathsfit}{bold}{\encodingdefault}{\sfdefault}{bx}{n}
\newcommand{\tens}[1]{\bm{\mathsfit{#1}}}
\def\tA{{\tens{A}}}
\def\tB{{\tens{B}}}
\def\tC{{\tens{C}}}
\def\tD{{\tens{D}}}
\def\tE{{\tens{E}}}
\def\tF{{\tens{F}}}
\def\tG{{\tens{G}}}
\def\tH{{\tens{H}}}
\def\tI{{\tens{I}}}
\def\tJ{{\tens{J}}}
\def\tK{{\tens{K}}}
\def\tL{{\tens{L}}}
\def\tM{{\tens{M}}}
\def\tN{{\tens{N}}}
\def\tO{{\tens{O}}}
\def\tP{{\tens{P}}}
\def\tQ{{\tens{Q}}}
\def\tR{{\tens{R}}}
\def\tS{{\tens{S}}}
\def\tT{{\tens{T}}}
\def\tU{{\tens{U}}}
\def\tV{{\tens{V}}}
\def\tW{{\tens{W}}}
\def\tX{{\tens{X}}}
\def\tY{{\tens{Y}}}
\def\tZ{{\tens{Z}}}


% Graph
\def\gA{{\mathcal{A}}}
\def\gB{{\mathcal{B}}}
\def\gC{{\mathcal{C}}}
\def\gD{{\mathcal{D}}}
\def\gE{{\mathcal{E}}}
\def\gF{{\mathcal{F}}}
\def\gG{{\mathcal{G}}}
\def\gH{{\mathcal{H}}}
\def\gI{{\mathcal{I}}}
\def\gJ{{\mathcal{J}}}
\def\gK{{\mathcal{K}}}
\def\gL{{\mathcal{L}}}
\def\gM{{\mathcal{M}}}
\def\gN{{\mathcal{N}}}
\def\gO{{\mathcal{O}}}
\def\gP{{\mathcal{P}}}
\def\gQ{{\mathcal{Q}}}
\def\gR{{\mathcal{R}}}
\def\gS{{\mathcal{S}}}
\def\gT{{\mathcal{T}}}
\def\gU{{\mathcal{U}}}
\def\gV{{\mathcal{V}}}
\def\gW{{\mathcal{W}}}
\def\gX{{\mathcal{X}}}
\def\gY{{\mathcal{Y}}}
\def\gZ{{\mathcal{Z}}}

% Sets
\def\sA{{\mathbb{A}}}
\def\sB{{\mathbb{B}}}
\def\sC{{\mathbb{C}}}
\def\sD{{\mathbb{D}}}
% Don't use a set called E, because this would be the same as our symbol
% for expectation.
\def\sF{{\mathbb{F}}}
\def\sG{{\mathbb{G}}}
\def\sH{{\mathbb{H}}}
\def\sI{{\mathbb{I}}}
\def\sJ{{\mathbb{J}}}
\def\sK{{\mathbb{K}}}
\def\sL{{\mathbb{L}}}
\def\sM{{\mathbb{M}}}
\def\sN{{\mathbb{N}}}
\def\sO{{\mathbb{O}}}
\def\sP{{\mathbb{P}}}
\def\sQ{{\mathbb{Q}}}
\def\sR{{\mathbb{R}}}
\def\sS{{\mathbb{S}}}
\def\sT{{\mathbb{T}}}
\def\sU{{\mathbb{U}}}
\def\sV{{\mathbb{V}}}
\def\sW{{\mathbb{W}}}
\def\sX{{\mathbb{X}}}
\def\sY{{\mathbb{Y}}}
\def\sZ{{\mathbb{Z}}}

% Entries of a matrix
\def\emLambda{{\Lambda}}
\def\emA{{A}}
\def\emB{{B}}
\def\emC{{C}}
\def\emD{{D}}
\def\emE{{E}}
\def\emF{{F}}
\def\emG{{G}}
\def\emH{{H}}
\def\emI{{I}}
\def\emJ{{J}}
\def\emK{{K}}
\def\emL{{L}}
\def\emM{{M}}
\def\emN{{N}}
\def\emO{{O}}
\def\emP{{P}}
\def\emQ{{Q}}
\def\emR{{R}}
\def\emS{{S}}
\def\emT{{T}}
\def\emU{{U}}
\def\emV{{V}}
\def\emW{{W}}
\def\emX{{X}}
\def\emY{{Y}}
\def\emZ{{Z}}
\def\emSigma{{\Sigma}}

% entries of a tensor
% Same font as tensor, without \bm wrapper
\newcommand{\etens}[1]{\mathsfit{#1}}
\def\etLambda{{\etens{\Lambda}}}
\def\etA{{\etens{A}}}
\def\etB{{\etens{B}}}
\def\etC{{\etens{C}}}
\def\etD{{\etens{D}}}
\def\etE{{\etens{E}}}
\def\etF{{\etens{F}}}
\def\etG{{\etens{G}}}
\def\etH{{\etens{H}}}
\def\etI{{\etens{I}}}
\def\etJ{{\etens{J}}}
\def\etK{{\etens{K}}}
\def\etL{{\etens{L}}}
\def\etM{{\etens{M}}}
\def\etN{{\etens{N}}}
\def\etO{{\etens{O}}}
\def\etP{{\etens{P}}}
\def\etQ{{\etens{Q}}}
\def\etR{{\etens{R}}}
\def\etS{{\etens{S}}}
\def\etT{{\etens{T}}}
\def\etU{{\etens{U}}}
\def\etV{{\etens{V}}}
\def\etW{{\etens{W}}}
\def\etX{{\etens{X}}}
\def\etY{{\etens{Y}}}
\def\etZ{{\etens{Z}}}

% The true underlying data generating distribution
\newcommand{\pdata}{p_{\rm{data}}}
\newcommand{\ptarget}{p_{\rm{target}}}
\newcommand{\pprior}{p_{\rm{prior}}}
\newcommand{\pbase}{p_{\rm{base}}}
\newcommand{\pref}{p_{\rm{ref}}}

% The empirical distribution defined by the training set
\newcommand{\ptrain}{\hat{p}_{\rm{data}}}
\newcommand{\Ptrain}{\hat{P}_{\rm{data}}}
% The model distribution
\newcommand{\pmodel}{p_{\rm{model}}}
\newcommand{\Pmodel}{P_{\rm{model}}}
\newcommand{\ptildemodel}{\tilde{p}_{\rm{model}}}
% Stochastic autoencoder distributions
\newcommand{\pencode}{p_{\rm{encoder}}}
\newcommand{\pdecode}{p_{\rm{decoder}}}
\newcommand{\precons}{p_{\rm{reconstruct}}}

\newcommand{\laplace}{\mathrm{Laplace}} % Laplace distribution

\newcommand{\E}{\mathbb{E}}
\newcommand{\Ls}{\mathcal{L}}
\newcommand{\R}{\mathbb{R}}
\newcommand{\emp}{\tilde{p}}
\newcommand{\lr}{\alpha}
\newcommand{\reg}{\lambda}
\newcommand{\rect}{\mathrm{rectifier}}
\newcommand{\softmax}{\mathrm{softmax}}
\newcommand{\sigmoid}{\sigma}
\newcommand{\softplus}{\zeta}
\newcommand{\KL}{D_{\mathrm{KL}}}
\newcommand{\Var}{\mathrm{Var}}
\newcommand{\standarderror}{\mathrm{SE}}
\newcommand{\Cov}{\mathrm{Cov}}
% Wolfram Mathworld says $L^2$ is for function spaces and $\ell^2$ is for vectors
% But then they seem to use $L^2$ for vectors throughout the site, and so does
% wikipedia.
\newcommand{\normlzero}{L^0}
\newcommand{\normlone}{L^1}
\newcommand{\normltwo}{L^2}
\newcommand{\normlp}{L^p}
\newcommand{\normmax}{L^\infty}

\newcommand{\parents}{Pa} % See usage in notation.tex. Chosen to match Daphne's book.

\DeclareMathOperator*{\argmax}{arg\,max}
\DeclareMathOperator*{\argmin}{arg\,min}

\DeclareMathOperator{\sign}{sign}
\DeclareMathOperator{\Tr}{Tr}
\let\ab\allowbreak

\newcommand{\fix}{\marginpar{FIX}}
\newcommand{\new}{\marginpar{NEW}}
\usepackage{lineno} 
% color related
\definecolor{Gray}{gray}{0.9}
\usepackage{xcolor}
\colorlet{darkgreen}{green!65!black}
\colorlet{darkblue}{blue!75!black}
\colorlet{darkred}{red!80!black}
\definecolor{lightblue}{HTML}{0071bc}
\definecolor{lightgreen}{HTML}{39b54a}
\definecolor{manyshot}{HTML}{6969ff}
\definecolor{medshot}{HTML}{f7c600}
\definecolor{fewshot}{HTML}{ff6969}
\definecolor{mypurple}{HTML}{412F8A}
\definecolor{myorange}{HTML}{fc8e62}
\definecolor{citecolor}{HTML}{0071BC}
\definecolor{linkcolor}{HTML}{ED1C24}
\definecolor{Blue9}{rgb}{0.098,0.3,0.9}
% \definecolor{citecolor}{HTML}{3333A6}
\definecolor{urlcolor}{HTML}{3333A6}
\usepackage[pagebackref=false, breaklinks=true, colorlinks,
            citecolor=darkblue, urlcolor=urlcolor,linkcolor=linkcolor, bookmarks=false]{hyperref}
\usepackage[capitalize,noabbrev]{cleveref}
\usepackage{tabularx}
\usepackage{url}
\newcommand{\teng}[1]{\textcolor{blue}{#1}}
\usepackage{longtable,tabularx,booktabs,wrapfig}
\renewcommand\ttdefault{cmtt}
\newcommand{\method}{\texttt{SimPER}\xspace}
\newcommand{\MLP}{\texttt{MLP}\xspace}
\newcommand{\GNN}{\texttt{GNN}\xspace}


\definecolor{customgray}{rgb}{0.25,0.25,0.25}
\definecolor{customred}{rgb}{0.8,0.05,0.05}
\definecolor{urlcolors}{rgb}{0.872,0.2,0.552}
\newcommand{\sbest}[1]{\textcolor{customgray}{\textbf{#1}}}
\newcommand{\best}[1]{\textcolor{customred}{\textbf{#1}}}
\newcommand{\urlcolor}[1]{\textcolor{urlcolors}{\textbf{#1}}}


% \title{\method:  Simple Perplexity Optimization for Hyperparameter-free Preference Alignment}
\title{SimPER: A Minimalist Approach to Preference \\ Alignment without Hyperparameters}
% B3C: A Minimalist Approach to Offline Multi-Agent Reinforcement Learning
% a minimalist approach to offline reinforcement learning

% \title{Hyperparameter-Free Preference Optimization}

% \author{Teng Xiao$^{\heartsuit}$\thanks{Equal contribution.}\hspace{1.5mm}, Yige Yuan$^{\spadesuit}$\footnotemark[1]\hspace{1.5mm}, Zhengyu Chen$^{\clubsuit}$ , Mingxiao Li$^\blacktriangle$, \\ \textbf{Shangsong Liang}$^\blacklozenge$, \textbf{Zhaochun Ren}$^\diamondsuit$, \textbf{Vasant G Honavar}$^{\heartsuit}$  \\
% $^\heartsuit$Pennsylvania State University
% $^\spadesuit$University of Chinese Academy of Sciences \\  $^\clubsuit$Meituan Inc$^\blacktriangle$Tencent AI Lab $^\blacklozenge$Sun Yat-Sen University $^\diamondsuit$Leiden University\\
% \texttt{tengxiao@psu.edu}, 
% \texttt{yuanyige923@gmail.com},
% \texttt{vhonavar@psu.edu} 
% }
\author{Teng Xiao$^{1}$\thanks{Equal contribution.}\hspace{1.5mm}, Yige Yuan$^{2}$\footnotemark[1]\hspace{1.5mm}, Zhengyu Chen$^{3}$, Mingxiao Li$^4$, \\ \textbf{Shangsong Liang}$^5$\textbf{,} \textbf{Zhaochun Ren}$^6$\textbf{,} \textbf{Vasant G Honavar}$^{1}$  \\
$^1$Pennsylvania State University
$^2$University of Chinese Academy of Sciences \\  $^3$Meituan Inc $^4$Tencent AI Lab $^5$Sun Yat-Sen University $^6$Leiden University\\
\texttt{tengxiao@psu.edu}, 
\texttt{yuanyige923@gmail.com},
\texttt{vhonavar@psu.edu} 
}
% The \author macro works with any number of authors. There are two commands
% used to separate the names and addresses of multiple authors: \And and \AND.
%
% Using \And between authors leaves it to \LaTeX{} to determine where to break
% the lines. Using \AND forces a linebreak at that point. So, if \LaTeX{}
% puts 3 of 4 authors names on the first line, and the last on the second
% line, try using \AND instead of \And before the third author name.


%\iclrfinalcopy % Uncomment for camera-ready version, but NOT for submission.
\begin{document}

\maketitle

\begin{abstract}
Existing preference optimization objectives for language model alignment require additional hyperparameters that must be extensively tuned to achieve optimal performance, increasing both the complexity and time required for fine-tuning large language models. 
In this paper, we propose a simple yet effective hyperparameter-free preference optimization algorithm for alignment. We observe that promising performance can be achieved simply by optimizing inverse perplexity, which is calculated as the inverse of the exponentiated average log-likelihood of the chosen and rejected responses in the preference dataset.
The resulting simple learning objective, \method (\textbf{{Sim}}ple alignment with \textbf{Per}plexity optimization), is easy to implement and eliminates the need for expensive hyperparameter tuning and a reference model, making it both computationally and memory efficient. Extensive experiments on widely used real-world benchmarks, including MT-Bench, AlpacaEval 2, and \textbf{10} key benchmarks of the Open LLM Leaderboard with \textbf{5} base models, demonstrate that \method consistently and significantly outperforms existing approaches—even without any hyperparameters or a reference model. For example, despite its simplicity, \method outperforms state-of-the-art methods by up to \textbf{5.7} points on AlpacaEval 2 and achieves the highest average ranking across \textbf{10} benchmarks on the Open LLM Leaderboard. The source code for \method is publicly available at the Github: \url{https://github.com/tengxiao1/SimPER}.
\end{abstract}



\section{Introduction}
\label{sec:introduction}
The business processes of organizations are experiencing ever-increasing complexity due to the large amount of data, high number of users, and high-tech devices involved \cite{martin2021pmopportunitieschallenges, beerepoot2023biggestbpmproblems}. This complexity may cause business processes to deviate from normal control flow due to unforeseen and disruptive anomalies \cite{adams2023proceddsriftdetection}. These control-flow anomalies manifest as unknown, skipped, and wrongly-ordered activities in the traces of event logs monitored from the execution of business processes \cite{ko2023adsystematicreview}. For the sake of clarity, let us consider an illustrative example of such anomalies. Figure \ref{FP_ANOMALIES} shows a so-called event log footprint, which captures the control flow relations of four activities of a hypothetical event log. In particular, this footprint captures the control-flow relations between activities \texttt{a}, \texttt{b}, \texttt{c} and \texttt{d}. These are the causal ($\rightarrow$) relation, concurrent ($\parallel$) relation, and other ($\#$) relations such as exclusivity or non-local dependency \cite{aalst2022pmhandbook}. In addition, on the right are six traces, of which five exhibit skipped, wrongly-ordered and unknown control-flow anomalies. For example, $\langle$\texttt{a b d}$\rangle$ has a skipped activity, which is \texttt{c}. Because of this skipped activity, the control-flow relation \texttt{b}$\,\#\,$\texttt{d} is violated, since \texttt{d} directly follows \texttt{b} in the anomalous trace.
\begin{figure}[!t]
\centering
\includegraphics[width=0.9\columnwidth]{images/FP_ANOMALIES.png}
\caption{An example event log footprint with six traces, of which five exhibit control-flow anomalies.}
\label{FP_ANOMALIES}
\end{figure}

\subsection{Control-flow anomaly detection}
Control-flow anomaly detection techniques aim to characterize the normal control flow from event logs and verify whether these deviations occur in new event logs \cite{ko2023adsystematicreview}. To develop control-flow anomaly detection techniques, \revision{process mining} has seen widespread adoption owing to process discovery and \revision{conformance checking}. On the one hand, process discovery is a set of algorithms that encode control-flow relations as a set of model elements and constraints according to a given modeling formalism \cite{aalst2022pmhandbook}; hereafter, we refer to the Petri net, a widespread modeling formalism. On the other hand, \revision{conformance checking} is an explainable set of algorithms that allows linking any deviations with the reference Petri net and providing the fitness measure, namely a measure of how much the Petri net fits the new event log \cite{aalst2022pmhandbook}. Many control-flow anomaly detection techniques based on \revision{conformance checking} (hereafter, \revision{conformance checking}-based techniques) use the fitness measure to determine whether an event log is anomalous \cite{bezerra2009pmad, bezerra2013adlogspais, myers2018icsadpm, pecchia2020applicationfailuresanalysispm}. 

The scientific literature also includes many \revision{conformance checking}-independent techniques for control-flow anomaly detection that combine specific types of trace encodings with machine/deep learning \cite{ko2023adsystematicreview, tavares2023pmtraceencoding}. Whereas these techniques are very effective, their explainability is challenging due to both the type of trace encoding employed and the machine/deep learning model used \cite{rawal2022trustworthyaiadvances,li2023explainablead}. Hence, in the following, we focus on the shortcomings of \revision{conformance checking}-based techniques to investigate whether it is possible to support the development of competitive control-flow anomaly detection techniques while maintaining the explainable nature of \revision{conformance checking}.
\begin{figure}[!t]
\centering
\includegraphics[width=\columnwidth]{images/HIGH_LEVEL_VIEW.png}
\caption{A high-level view of the proposed framework for combining \revision{process mining}-based feature extraction with dimensionality reduction for control-flow anomaly detection.}
\label{HIGH_LEVEL_VIEW}
\end{figure}

\subsection{Shortcomings of \revision{conformance checking}-based techniques}
Unfortunately, the detection effectiveness of \revision{conformance checking}-based techniques is affected by noisy data and low-quality Petri nets, which may be due to human errors in the modeling process or representational bias of process discovery algorithms \cite{bezerra2013adlogspais, pecchia2020applicationfailuresanalysispm, aalst2016pm}. Specifically, on the one hand, noisy data may introduce infrequent and deceptive control-flow relations that may result in inconsistent fitness measures, whereas, on the other hand, checking event logs against a low-quality Petri net could lead to an unreliable distribution of fitness measures. Nonetheless, such Petri nets can still be used as references to obtain insightful information for \revision{process mining}-based feature extraction, supporting the development of competitive and explainable \revision{conformance checking}-based techniques for control-flow anomaly detection despite the problems above. For example, a few works outline that token-based \revision{conformance checking} can be used for \revision{process mining}-based feature extraction to build tabular data and develop effective \revision{conformance checking}-based techniques for control-flow anomaly detection \cite{singh2022lapmsh, debenedictis2023dtadiiot}. However, to the best of our knowledge, the scientific literature lacks a structured proposal for \revision{process mining}-based feature extraction using the state-of-the-art \revision{conformance checking} variant, namely alignment-based \revision{conformance checking}.

\subsection{Contributions}
We propose a novel \revision{process mining}-based feature extraction approach with alignment-based \revision{conformance checking}. This variant aligns the deviating control flow with a reference Petri net; the resulting alignment can be inspected to extract additional statistics such as the number of times a given activity caused mismatches \cite{aalst2022pmhandbook}. We integrate this approach into a flexible and explainable framework for developing techniques for control-flow anomaly detection. The framework combines \revision{process mining}-based feature extraction and dimensionality reduction to handle high-dimensional feature sets, achieve detection effectiveness, and support explainability. Notably, in addition to our proposed \revision{process mining}-based feature extraction approach, the framework allows employing other approaches, enabling a fair comparison of multiple \revision{conformance checking}-based and \revision{conformance checking}-independent techniques for control-flow anomaly detection. Figure \ref{HIGH_LEVEL_VIEW} shows a high-level view of the framework. Business processes are monitored, and event logs obtained from the database of information systems. Subsequently, \revision{process mining}-based feature extraction is applied to these event logs and tabular data input to dimensionality reduction to identify control-flow anomalies. We apply several \revision{conformance checking}-based and \revision{conformance checking}-independent framework techniques to publicly available datasets, simulated data of a case study from railways, and real-world data of a case study from healthcare. We show that the framework techniques implementing our approach outperform the baseline \revision{conformance checking}-based techniques while maintaining the explainable nature of \revision{conformance checking}.

In summary, the contributions of this paper are as follows.
\begin{itemize}
    \item{
        A novel \revision{process mining}-based feature extraction approach to support the development of competitive and explainable \revision{conformance checking}-based techniques for control-flow anomaly detection.
    }
    \item{
        A flexible and explainable framework for developing techniques for control-flow anomaly detection using \revision{process mining}-based feature extraction and dimensionality reduction.
    }
    \item{
        Application to synthetic and real-world datasets of several \revision{conformance checking}-based and \revision{conformance checking}-independent framework techniques, evaluating their detection effectiveness and explainability.
    }
\end{itemize}

The rest of the paper is organized as follows.
\begin{itemize}
    \item Section \ref{sec:related_work} reviews the existing techniques for control-flow anomaly detection, categorizing them into \revision{conformance checking}-based and \revision{conformance checking}-independent techniques.
    \item Section \ref{sec:abccfe} provides the preliminaries of \revision{process mining} to establish the notation used throughout the paper, and delves into the details of the proposed \revision{process mining}-based feature extraction approach with alignment-based \revision{conformance checking}.
    \item Section \ref{sec:framework} describes the framework for developing \revision{conformance checking}-based and \revision{conformance checking}-independent techniques for control-flow anomaly detection that combine \revision{process mining}-based feature extraction and dimensionality reduction.
    \item Section \ref{sec:evaluation} presents the experiments conducted with multiple framework and baseline techniques using data from publicly available datasets and case studies.
    \item Section \ref{sec:conclusions} draws the conclusions and presents future work.
\end{itemize}
\putsec{related}{Related Work}

\noindent \textbf{Efficient Radiance Field Rendering.}
%
The introduction of Neural Radiance Fields (NeRF)~\cite{mil:sri20} has
generated significant interest in efficient 3D scene representation and
rendering for radiance fields.
%
Over the past years, there has been a large amount of research aimed at
accelerating NeRFs through algorithmic or software
optimizations~\cite{mul:eva22,fri:yu22,che:fun23,sun:sun22}, and the
development of hardware
accelerators~\cite{lee:cho23,li:li23,son:wen23,mub:kan23,fen:liu24}.
%
The state-of-the-art method, 3D Gaussian splatting~\cite{ker:kop23}, has
further fueled interest in accelerating radiance field
rendering~\cite{rad:ste24,lee:lee24,nie:stu24,lee:rho24,ham:mel24} as it
employs rasterization primitives that can be rendered much faster than NeRFs.
%
However, previous research focused on software graphics rendering on
programmable cores or building dedicated hardware accelerators. In contrast,
\name{} investigates the potential of efficient radiance field rendering while
utilizing fixed-function units in graphics hardware.
%
To our knowledge, this is the first work that assesses the performance
implications of rendering Gaussian-based radiance fields on the hardware
graphics pipeline with software and hardware optimizations.

%%%%%%%%%%%%%%%%%%%%%%%%%%%%%%%%%%%%%%%%%%%%%%%%%%%%%%%%%%%%%%%%%%%%%%%%%%
\myparagraph{Enhancing Graphics Rendering Hardware.}
%
The performance advantage of executing graphics rendering on either
programmable shader cores or fixed-function units varies depending on the
rendering methods and hardware designs.
%
Previous studies have explored the performance implication of graphics hardware
design by developing simulation infrastructures for graphics
workloads~\cite{bar:gon06,gub:aam19,tin:sax23,arn:par13}.
%
Additionally, several studies have aimed to improve the performance of
special-purpose hardware such as ray tracing units in graphics
hardware~\cite{cho:now23,liu:cha21} and proposed hardware accelerators for
graphics applications~\cite{lu:hua17,ram:gri09}.
%
In contrast to these works, which primarily evaluate traditional graphics
workloads, our work focuses on improving the performance of volume rendering
workloads, such as Gaussian splatting, which require blending a huge number of
fragments per pixel.

%%%%%%%%%%%%%%%%%%%%%%%%%%%%%%%%%%%%%%%%%%%%%%%%%%%%%%%%%%%%%%%%%%%%%%%%%%
%
In the context of multi-sample anti-aliasing, prior work proposed reducing the
amount of redundant shading by merging fragments from adjacent triangles in a
mesh at the quad granularity~\cite{fat:bou10}.
%
While both our work and quad-fragment merging (QFM)~\cite{fat:bou10} aim to
reduce operations by merging quads, our proposed technique differs from QFM in
many aspects.
%
Our method aims to blend \emph{overlapping primitives} along the depth
direction and applies to quads from any primitive. In contrast, QFM merges quad
fragments from small (e.g., pixel-sized) triangles that \emph{share} an edge
(i.e., \emph{connected}, \emph{non-overlapping} triangles).
%
As such, QFM is not applicable to the scenes consisting of a number of
unconnected transparent triangles, such as those in 3D Gaussian splatting.
%
In addition, our method computes the \emph{exact} color for each pixel by
offloading blending operations from ROPs to shader units, whereas QFM
\emph{approximates} pixel colors by using the color from one triangle when
multiple triangles are merged into a single quad.


\section{Method}\label{sec:method}
\begin{figure}
    \centering
    \includegraphics[width=0.85\textwidth]{imgs/heatmap_acc.pdf}
    \caption{\textbf{Visualization of the proposed periodic Bayesian flow with mean parameter $\mu$ and accumulated accuracy parameter $c$ which corresponds to the entropy/uncertainty}. For $x = 0.3, \beta(1) = 1000$ and $\alpha_i$ defined in \cref{appd:bfn_cir}, this figure plots three colored stochastic parameter trajectories for receiver mean parameter $m$ and accumulated accuracy parameter $c$, superimposed on a log-scale heatmap of the Bayesian flow distribution $p_F(m|x,\senderacc)$ and $p_F(c|x,\senderacc)$. Note the \emph{non-monotonicity} and \emph{non-additive} property of $c$ which could inform the network the entropy of the mean parameter $m$ as a condition and the \emph{periodicity} of $m$. %\jj{Shrink the figures to save space}\hanlin{Do we need to make this figure one-column?}
    }
    \label{fig:vmbf_vis}
    \vskip -0.1in
\end{figure}
% \begin{wrapfigure}{r}{0.5\textwidth}
%     \centering
%     \includegraphics[width=0.49\textwidth]{imgs/heatmap_acc.pdf}
%     \caption{\textbf{Visualization of hyper-torus Bayesian flow based on von Mises Distribution}. For $x = 0.3, \beta(1) = 1000$ and $\alpha_i$ defined in \cref{appd:bfn_cir}, this figure plots three colored stochastic parameter trajectories for receiver mean parameter $m$ and accumulated accuracy parameter $c$, superimposed on a log-scale heatmap of the Bayesian flow distribution $p_F(m|x,\senderacc)$ and $p_F(c|x,\senderacc)$. Note the \emph{non-monotonicity} and \emph{non-additive} property of $c$. \jj{Shrink the figures to save space}}
%     \label{fig:vmbf_vis}
%     \vspace{-30pt}
% \end{wrapfigure}


In this section, we explain the detailed design of CrysBFN tackling theoretical and practical challenges. First, we describe how to derive our new formulation of Bayesian Flow Networks over hyper-torus $\mathbb{T}^{D}$ from scratch. Next, we illustrate the two key differences between \modelname and the original form of BFN: $1)$ a meticulously designed novel base distribution with different Bayesian update rules; and $2)$ different properties over the accuracy scheduling resulted from the periodicity and the new Bayesian update rules. Then, we present in detail the overall framework of \modelname over each manifold of the crystal space (\textit{i.e.} fractional coordinates, lattice vectors, atom types) respecting \textit{periodic E(3) invariance}. 

% In this section, we first demonstrate how to build Bayesian flow on hyper-torus $\mathbb{T}^{D}$ by overcoming theoretical and practical problems to provide a low-noise parameter-space approach to fractional atom coordinate generation. Next, we present how \modelname models each manifold of crystal space respecting \textit{periodic E(3) invariance}. 

\subsection{Periodic Bayesian Flow on Hyper-torus \texorpdfstring{$\mathbb{T}^{D}$}{}} 
For generative modeling of fractional coordinates in crystal, we first construct a periodic Bayesian flow on \texorpdfstring{$\mathbb{T}^{D}$}{} by designing every component of the totally new Bayesian update process which we demonstrate to be distinct from the original Bayesian flow (please see \cref{fig:non_add}). 
 %:) 
 
 The fractional atom coordinate system \citep{jiao2023crystal} inherently distributes over a hyper-torus support $\mathbb{T}^{3\times N}$. Hence, the normal distribution support on $\R$ used in the original \citep{bfn} is not suitable for this scenario. 
% The key problem of generative modeling for crystal is the periodicity of Cartesian atom coordinates $\vX$ requiring:
% \begin{equation}\label{eq:periodcity}
% p(\vA,\vL,\vX)=p(\vA,\vL,\vX+\vec{LK}),\text{where}~\vec{K}=\vec{k}\vec{1}_{1\times N},\forall\vec{k}\in\mathbb{Z}^{3\times1}
% \end{equation}
% However, there does not exist such a distribution supporting on $\R$ to model such property because the integration of such distribution over $\R$ will not be finite and equal to 1. Therefore, the normal distribution used in \citet{bfn} can not meet this condition.

To tackle this problem, the circular distribution~\citep{mardia2009directional} over the finite interval $[-\pi,\pi)$ is a natural choice as the base distribution for deriving the BFN on $\mathbb{T}^D$. 
% one natural choice is to 
% we would like to consider the circular distribution over the finite interval as the base 
% we find that circular distributions \citep{mardia2009directional} defined on a finite interval with lengths of $2\pi$ can be used as the instantiation of input distribution for the BFN on $\mathbb{T}^D$.
Specifically, circular distributions enjoy desirable periodic properties: $1)$ the integration over any interval length of $2\pi$ equals 1; $2)$ the probability distribution function is periodic with period $2\pi$.  Sharing the same intrinsic with fractional coordinates, such periodic property of circular distribution makes it suitable for the instantiation of BFN's input distribution, in parameterizing the belief towards ground truth $\x$ on $\mathbb{T}^D$. 
% \yuxuan{this is very complicated from my perspective.} \hanlin{But this property is exactly beautiful and perfectly fit into the BFN.}

\textbf{von Mises Distribution and its Bayesian Update} We choose von Mises distribution \citep{mardia2009directional} from various circular distributions as the form of input distribution, based on the appealing conjugacy property required in the derivation of the BFN framework.
% to leverage the Bayesian conjugacy property of von Mises distribution which is required by the BFN framework. 
That is, the posterior of a von Mises distribution parameterized likelihood is still in the family of von Mises distributions. The probability density function of von Mises distribution with mean direction parameter $m$ and concentration parameter $c$ (describing the entropy/uncertainty of $m$) is defined as: 
\begin{equation}
f(x|m,c)=vM(x|m,c)=\frac{\exp(c\cos(x-m))}{2\pi I_0(c)}
\end{equation}
where $I_0(c)$ is zeroth order modified Bessel function of the first kind as the normalizing constant. Given the last univariate belief parameterized by von Mises distribution with parameter $\theta_{i-1}=\{m_{i-1},\ c_{i-1}\}$ and the sample $y$ from sender distribution with unknown data sample $x$ and known accuracy $\alpha$ describing the entropy/uncertainty of $y$,  Bayesian update for the receiver is deducted as:
\begin{equation}
 h(\{m_{i-1},c_{i-1}\},y,\alpha)=\{m_i,c_i \}, \text{where}
\end{equation}
\begin{equation}\label{eq:h_m}
m_i=\text{atan2}(\alpha\sin y+c_{i-1}\sin m_{i-1}, {\alpha\cos y+c_{i-1}\cos m_{i-1}})
\end{equation}
\begin{equation}\label{eq:h_c}
c_i =\sqrt{\alpha^2+c_{i-1}^2+2\alpha c_{i-1}\cos(y-m_{i-1})}
\end{equation}
The proof of the above equations can be found in \cref{apdx:bayesian_update_function}. The atan2 function refers to  2-argument arctangent. Independently conducting  Bayesian update for each dimension, we can obtain the Bayesian update distribution by marginalizing $\y$:
\begin{equation}
p_U(\vtheta'|\vtheta,\bold{x};\alpha)=\mathbb{E}_{p_S(\bold{y}|\bold{x};\alpha)}\delta(\vtheta'-h(\vtheta,\bold{y},\alpha))=\mathbb{E}_{vM(\bold{y}|\bold{x},\alpha)}\delta(\vtheta'-h(\vtheta,\bold{y},\alpha))
\end{equation} 
\begin{figure}
    \centering
    \vskip -0.15in
    \includegraphics[width=0.95\linewidth]{imgs/non_add.pdf}
    \caption{An intuitive illustration of non-additive accuracy Bayesian update on the torus. The lengths of arrows represent the uncertainty/entropy of the belief (\emph{e.g.}~$1/\sigma^2$ for Gaussian and $c$ for von Mises). The directions of the arrows represent the believed location (\emph{e.g.}~ $\mu$ for Gaussian and $m$ for von Mises).}
    \label{fig:non_add}
    \vskip -0.15in
\end{figure}
\textbf{Non-additive Accuracy} 
The additive accuracy is a nice property held with the Gaussian-formed sender distribution of the original BFN expressed as:
\begin{align}
\label{eq:standard_id}
    \update(\parsn{}'' \mid \parsn{}, \x; \alpha_a+\alpha_b) = \E_{\update(\parsn{}' \mid \parsn{}, \x; \alpha_a)} \update(\parsn{}'' \mid \parsn{}', \x; \alpha_b)
\end{align}
Such property is mainly derived based on the standard identity of Gaussian variable:
\begin{equation}
X \sim \mathcal{N}\left(\mu_X, \sigma_X^2\right), Y \sim \mathcal{N}\left(\mu_Y, \sigma_Y^2\right) \Longrightarrow X+Y \sim \mathcal{N}\left(\mu_X+\mu_Y, \sigma_X^2+\sigma_Y^2\right)
\end{equation}
The additive accuracy property makes it feasible to derive the Bayesian flow distribution $
p_F(\boldsymbol{\theta} \mid \mathbf{x} ; i)=p_U\left(\boldsymbol{\theta} \mid \boldsymbol{\theta}_0, \mathbf{x}, \sum_{k=1}^{i} \alpha_i \right)
$ for the simulation-free training of \cref{eq:loss_n}.
It should be noted that the standard identity in \cref{eq:standard_id} does not hold in the von Mises distribution. Hence there exists an important difference between the original Bayesian flow defined on Euclidean space and the Bayesian flow of circular data on $\mathbb{T}^D$ based on von Mises distribution. With prior $\btheta = \{\bold{0},\bold{0}\}$, we could formally represent the non-additive accuracy issue as:
% The additive accuracy property implies the fact that the "confidence" for the data sample after observing a series of the noisy samples with accuracy ${\alpha_1, \cdots, \alpha_i}$ could be  as the accuracy sum  which could be  
% Here we 
% Here we emphasize the specific property of BFN based on von Mises distribution.
% Note that 
% \begin{equation}
% \update(\parsn'' \mid \parsn, \x; \alpha_a+\alpha_b) \ne \E_{\update(\parsn' \mid \parsn, \x; \alpha_a)} \update(\parsn'' \mid \parsn', \x; \alpha_b)
% \end{equation}
% \oyyw{please check whether the below equation is better}
% \yuxuan{I fill somehow confusing on what is the update distribution with $\alpha$. }
% \begin{equation}
% \update(\parsn{}'' \mid \parsn{}, \x; \alpha_a+\alpha_b) \ne \E_{\update(\parsn{}' \mid \parsn{}, \x; \alpha_a)} \update(\parsn{}'' \mid \parsn{}', \x; \alpha_b)
% \end{equation}
% We give an intuitive visualization of such difference in \cref{fig:non_add}. The untenability of this property can materialize by considering the following case: with prior $\btheta = \{\bold{0},\bold{0}\}$, check the two-step Bayesian update distribution with $\alpha_a,\alpha_b$ and one-step Bayesian update with $\alpha=\alpha_a+\alpha_b$:
\begin{align}
\label{eq:nonadd}
     &\update(c'' \mid \parsn, \x; \alpha_a+\alpha_b)  = \delta(c-\alpha_a-\alpha_b)
     \ne  \mathbb{E}_{p_U(\parsn' \mid \parsn, \x; \alpha_a)}\update(c'' \mid \parsn', \x; \alpha_b) \nonumber \\&= \mathbb{E}_{vM(\bold{y}_b|\bold{x},\alpha_a)}\mathbb{E}_{vM(\bold{y}_a|\bold{x},\alpha_b)}\delta(c-||[\alpha_a \cos\y_a+\alpha_b\cos \y_b,\alpha_a \sin\y_a+\alpha_b\sin \y_b]^T||_2)
\end{align}
A more intuitive visualization could be found in \cref{fig:non_add}. This fundamental difference between periodic Bayesian flow and that of \citet{bfn} presents both theoretical and practical challenges, which we will explain and address in the following contents.

% This makes constructing Bayesian flow based on von Mises distribution intrinsically different from previous Bayesian flows (\citet{bfn}).

% Thus, we must reformulate the framework of Bayesian flow networks  accordingly. % and do necessary reformulations of BFN. 

% \yuxuan{overall I feel this part is complicated by using the language of update distribution. I would like to suggest simply use bayesian update, to provide intuitive explantion.}\hanlin{See the illustration in \cref{fig:non_add}}

% That introduces a cascade of problems, and we investigate the following issues: $(1)$ Accuracies between sender and receiver are not synchronized and need to be differentiated. $(2)$ There is no tractable Bayesian flow distribution for a one-step sample conditioned on a given time step $i$, and naively simulating the Bayesian flow results in computational overhead. $(3)$ It is difficult to control the entropy of the Bayesian flow. $(4)$ Accuracy is no longer a function of $t$ and becomes a distribution conditioned on $t$, which can be different across dimensions.
%\jj{Edited till here}

\textbf{Entropy Conditioning} As a common practice in generative models~\citep{ddpm,flowmatching,bfn}, timestep $t$ is widely used to distinguish among generation states by feeding the timestep information into the networks. However, this paper shows that for periodic Bayesian flow, the accumulated accuracy $\vc_i$ is more effective than time-based conditioning by informing the network about the entropy and certainty of the states $\parsnt{i}$. This stems from the intrinsic non-additive accuracy which makes the receiver's accumulated accuracy $c$ not bijective function of $t$, but a distribution conditioned on accumulated accuracies $\vc_i$ instead. Therefore, the entropy parameter $\vc$ is taken logarithm and fed into the network to describe the entropy of the input corrupted structure. We verify this consideration in \cref{sec:exp_ablation}. 
% \yuxuan{implement variant. traditionally, the timestep is widely used to distinguish the different states by putting the timestep embedding into the networks. citation of FM, diffusion, BFN. However, we find that conditioned on time in periodic flow could not provide extra benefits. To further boost the performance, we introduce a simple yet effective modification term entropy conditional. This is based on that the accumulated accuracy which represents the current uncertainty or entropy could be a better indicator to distinguish different states. + Describe how you do this. }



\textbf{Reformulations of BFN}. Recall the original update function with Gaussian sender distribution, after receiving noisy samples $\y_1,\y_2,\dots,\y_i$ with accuracies $\senderacc$, the accumulated accuracies of the receiver side could be analytically obtained by the additive property and it is consistent with the sender side.
% Since observing sample $\y$ with $\alpha_i$ can not result in exact accuracy increment $\alpha_i$ for receiver, the accuracies between sender and receiver are not synchronized which need to be differentiated. 
However, as previously mentioned, this does not apply to periodic Bayesian flow, and some of the notations in original BFN~\citep{bfn} need to be adjusted accordingly. We maintain the notations of sender side's one-step accuracy $\alpha$ and added accuracy $\beta$, and alter the notation of receiver's accuracy parameter as $c$, which is needed to be simulated by cascade of Bayesian updates. We emphasize that the receiver's accumulated accuracy $c$ is no longer a function of $t$ (differently from the Gaussian case), and it becomes a distribution conditioned on received accuracies $\senderacc$ from the sender. Therefore, we represent the Bayesian flow distribution of von Mises distribution as $p_F(\btheta|\x;\alpha_1,\alpha_2,\dots,\alpha_i)$. And the original simulation-free training with Bayesian flow distribution is no longer applicable in this scenario.
% Different from previous BFNs where the accumulated accuracy $\rho$ is not explicitly modeled, the accumulated accuracy parameter $c$ (visualized in \cref{fig:vmbf_vis}) needs to be explicitly modeled by feeding it to the network to avoid information loss.
% the randomaccuracy parameter $c$ (visualized in \cref{fig:vmbf_vis}) implies that there exists information in $c$ from the sender just like $m$, meaning that $c$ also should be fed into the network to avoid information loss. 
% We ablate this consideration in  \cref{sec:exp_ablation}. 

\textbf{Fast Sampling from Equivalent Bayesian Flow Distribution} Based on the above reformulations, the Bayesian flow distribution of von Mises distribution is reframed as: 
\begin{equation}\label{eq:flow_frac}
p_F(\btheta_i|\x;\alpha_1,\alpha_2,\dots,\alpha_i)=\E_{\update(\parsnt{1} \mid \parsnt{0}, \x ; \alphat{1})}\dots\E_{\update(\parsn_{i-1} \mid \parsnt{i-2}, \x; \alphat{i-1})} \update(\parsnt{i} | \parsnt{i-1},\x;\alphat{i} )
\end{equation}
Naively sampling from \cref{eq:flow_frac} requires slow auto-regressive iterated simulation, making training unaffordable. Noticing the mathematical properties of \cref{eq:h_m,eq:h_c}, we  transform \cref{eq:flow_frac} to the equivalent form:
\begin{equation}\label{eq:cirflow_equiv}
p_F(\vec{m}_i|\x;\alpha_1,\alpha_2,\dots,\alpha_i)=\E_{vM(\y_1|\x,\alpha_1)\dots vM(\y_i|\x,\alpha_i)} \delta(\vec{m}_i-\text{atan2}(\sum_{j=1}^i \alpha_j \cos \y_j,\sum_{j=1}^i \alpha_j \sin \y_j))
\end{equation}
\begin{equation}\label{eq:cirflow_equiv2}
p_F(\vec{c}_i|\x;\alpha_1,\alpha_2,\dots,\alpha_i)=\E_{vM(\y_1|\x,\alpha_1)\dots vM(\y_i|\x,\alpha_i)}  \delta(\vec{c}_i-||[\sum_{j=1}^i \alpha_j \cos \y_j,\sum_{j=1}^i \alpha_j \sin \y_j]^T||_2)
\end{equation}
which bypasses the computation of intermediate variables and allows pure tensor operations, with negligible computational overhead.
\begin{restatable}{proposition}{cirflowequiv}
The probability density function of Bayesian flow distribution defined by \cref{eq:cirflow_equiv,eq:cirflow_equiv2} is equivalent to the original definition in \cref{eq:flow_frac}. 
\end{restatable}
\textbf{Numerical Determination of Linear Entropy Sender Accuracy Schedule} ~Original BFN designs the accuracy schedule $\beta(t)$ to make the entropy of input distribution linearly decrease. As for crystal generation task, to ensure information coherence between modalities, we choose a sender accuracy schedule $\senderacc$ that makes the receiver's belief entropy $H(t_i)=H(p_I(\cdot|\vtheta_i))=H(p_I(\cdot|\vc_i))$ linearly decrease \emph{w.r.t.} time $t_i$, given the initial and final accuracy parameter $c(0)$ and $c(1)$. Due to the intractability of \cref{eq:vm_entropy}, we first use numerical binary search in $[0,c(1)]$ to determine the receiver's $c(t_i)$ for $i=1,\dots, n$ by solving the equation $H(c(t_i))=(1-t_i)H(c(0))+tH(c(1))$. Next, with $c(t_i)$, we conduct numerical binary search for each $\alpha_i$ in $[0,c(1)]$ by solving the equations $\E_{y\sim vM(x,\alpha_i)}[\sqrt{\alpha_i^2+c_{i-1}^2+2\alpha_i c_{i-1}\cos(y-m_{i-1})}]=c(t_i)$ from $i=1$ to $i=n$ for arbitrarily selected $x\in[-\pi,\pi)$.

After tackling all those issues, we have now arrived at a new BFN architecture for effectively modeling crystals. Such BFN can also be adapted to other type of data located in hyper-torus $\mathbb{T}^{D}$.

\subsection{Equivariant Bayesian Flow for Crystal}
With the above Bayesian flow designed for generative modeling of fractional coordinate $\vF$, we are able to build equivariant Bayesian flow for each modality of crystal. In this section, we first give an overview of the general training and sampling algorithm of \modelname (visualized in \cref{fig:framework}). Then, we describe the details of the Bayesian flow of every modality. The training and sampling algorithm can be found in \cref{alg:train} and \cref{alg:sampling}.

\textbf{Overview} Operating in the parameter space $\bthetaM=\{\bthetaA,\bthetaL,\bthetaF\}$, \modelname generates high-fidelity crystals through a joint BFN sampling process on the parameter of  atom type $\bthetaA$, lattice parameter $\vec{\theta}^L=\{\bmuL,\brhoL\}$, and the parameter of fractional coordinate matrix $\bthetaF=\{\bmF,\bcF\}$. We index the $n$-steps of the generation process in a discrete manner $i$, and denote the corresponding continuous notation $t_i=i/n$ from prior parameter $\thetaM_0$ to a considerably low variance parameter $\thetaM_n$ (\emph{i.e.} large $\vrho^L,\bmF$, and centered $\bthetaA$).

At training time, \modelname samples time $i\sim U\{1,n\}$ and $\bthetaM_{i-1}$ from the Bayesian flow distribution of each modality, serving as the input to the network. The network $\net$ outputs $\net(\parsnt{i-1}^\mathcal{M},t_{i-1})=\net(\parsnt{i-1}^A,\parsnt{i-1}^F,\parsnt{i-1}^L,t_{i-1})$ and conducts gradient descents on loss function \cref{eq:loss_n} for each modality. After proper training, the sender distribution $p_S$ can be approximated by the receiver distribution $p_R$. 

At inference time, from predefined $\thetaM_0$, we conduct transitions from $\thetaM_{i-1}$ to $\thetaM_{i}$ by: $(1)$ sampling $\y_i\sim p_R(\bold{y}|\thetaM_{i-1};t_i,\alpha_i)$ according to network prediction $\predM{i-1}$; and $(2)$ performing Bayesian update $h(\thetaM_{i-1},\y^\calM_{i-1},\alpha_i)$ for each dimension. 

% Alternatively, we complete this transition using the flow-back technique by sampling 
% $\thetaM_{i}$ from Bayesian flow distribution $\flow(\btheta^M_{i}|\predM{i-1};t_{i-1})$. 

% The training objective of $\net$ is to minimize the KL divergence between sender distribution and receiver distribution for every modality as defined in \cref{eq:loss_n} which is equivalent to optimizing the negative variational lower bound $\calL^{VLB}$ as discussed in \cref{sec:preliminaries}. 

%In the following part, we will present the Bayesian flow of each modality in detail.

\textbf{Bayesian Flow of Fractional Coordinate $\vF$}~The distribution of the prior parameter $\bthetaF_0$ is defined as:
\begin{equation}\label{eq:prior_frac}
    p(\bthetaF_0) \defeq \{vM(\vm_0^F|\vec{0}_{3\times N},\vec{0}_{3\times N}),\delta(\vc_0^F-\vec{0}_{3\times N})\} = \{U(\vec{0},\vec{1}),\delta(\vc_0^F-\vec{0}_{3\times N})\}
\end{equation}
Note that this prior distribution of $\vm_0^F$ is uniform over $[\vec{0},\vec{1})$, ensuring the periodic translation invariance property in \cref{De:pi}. The training objective is minimizing the KL divergence between sender and receiver distribution (deduction can be found in \cref{appd:cir_loss}): 
%\oyyw{replace $\vF$ with $\x$?} \hanlin{notations follow Preliminary?}
\begin{align}\label{loss_frac}
\calL_F = n \E_{i \sim \ui{n}, \flow(\parsn{}^F \mid \vF ; \senderacc)} \alpha_i\frac{I_1(\alpha_i)}{I_0(\alpha_i)}(1-\cos(\vF-\predF{i-1}))
\end{align}
where $I_0(x)$ and $I_1(x)$ are the zeroth and the first order of modified Bessel functions. The transition from $\bthetaF_{i-1}$ to $\bthetaF_{i}$ is the Bayesian update distribution based on network prediction:
\begin{equation}\label{eq:transi_frac}
    p(\btheta^F_{i}|\parsnt{i-1}^\calM)=\mathbb{E}_{vM(\bold{y}|\predF{i-1},\alpha_i)}\delta(\btheta^F_{i}-h(\btheta^F_{i-1},\bold{y},\alpha_i))
\end{equation}
\begin{restatable}{proposition}{fracinv}
With $\net_{F}$ as a periodic translation equivariant function namely $\net_F(\parsnt{}^A,w(\parsnt{}^F+\vt),\parsnt{}^L,t)=w(\net_F(\parsnt{}^A,\parsnt{}^F,\parsnt{}^L,t)+\vt), \forall\vt\in\R^3$, the marginal distribution of $p(\vF_n)$ defined by \cref{eq:prior_frac,eq:transi_frac} is periodic translation invariant. 
\end{restatable}
\textbf{Bayesian Flow of Lattice Parameter \texorpdfstring{$\boldsymbol{L}$}{}}   
Noting the lattice parameter $\bm{L}$ located in Euclidean space, we set prior as the parameter of a isotropic multivariate normal distribution $\btheta^L_0\defeq\{\vmu_0^L,\vrho_0^L\}=\{\bm{0}_{3\times3},\bm{1}_{3\times3}\}$
% \begin{equation}\label{eq:lattice_prior}
% \btheta^L_0\defeq\{\vmu_0^L,\vrho_0^L\}=\{\bm{0}_{3\times3},\bm{1}_{3\times3}\}
% \end{equation}
such that the prior distribution of the Markov process on $\vmu^L$ is the Dirac distribution $\delta(\vec{\mu_0}-\vec{0})$ and $\delta(\vec{\rho_0}-\vec{1})$, 
% \begin{equation}
%     p_I^L(\boldsymbol{L}|\btheta_0^L)=\mathcal{N}(\bm{L}|\bm{0},\bm{I})
% \end{equation}
which ensures O(3)-invariance of prior distribution of $\vL$. By Eq. 77 from \citet{bfn}, the Bayesian flow distribution of the lattice parameter $\bm{L}$ is: 
\begin{align}% =p_U(\bmuL|\btheta_0^L,\bm{L},\beta(t))
p_F^L(\bmuL|\bm{L};t) &=\mathcal{N}(\bmuL|\gamma(t)\bm{L},\gamma(t)(1-\gamma(t))\bm{I}) 
\end{align}
where $\gamma(t) = 1 - \sigma_1^{2t}$ and $\sigma_1$ is the predefined hyper-parameter controlling the variance of input distribution at $t=1$ under linear entropy accuracy schedule. The variance parameter $\vrho$ does not need to be modeled and fed to the network, since it is deterministic given the accuracy schedule. After sampling $\bmuL_i$ from $p_F^L$, the training objective is defined as minimizing KL divergence between sender and receiver distribution (based on Eq. 96 in \citet{bfn}):
\begin{align}
\mathcal{L}_{L} = \frac{n}{2}\left(1-\sigma_1^{2/n}\right)\E_{i \sim \ui{n}}\E_{\flow(\bmuL_{i-1} |\vL ; t_{i-1})}  \frac{\left\|\vL -\predL{i-1}\right\|^2}{\sigma_1^{2i/n}},\label{eq:lattice_loss}
\end{align}
where the prediction term $\predL{i-1}$ is the lattice parameter part of network output. After training, the generation process is defined as the Bayesian update distribution given network prediction:
\begin{equation}\label{eq:lattice_sampling}
    p(\bmuL_{i}|\parsnt{i-1}^\calM)=\update^L(\bmuL_{i}|\predL{i-1},\bmuL_{i-1};t_{i-1})
\end{equation}
    

% The final prediction of the lattice parameter is given by $\bmuL_n = \predL{n-1}$.
% \begin{equation}\label{eq:final_lattice}
%     \bmuL_n = \predL{n-1}
% \end{equation}

\begin{restatable}{proposition}{latticeinv}\label{prop:latticeinv}
With $\net_{L}$ as  O(3)-equivariant function namely $\net_L(\parsnt{}^A,\parsnt{}^F,\vQ\parsnt{}^L,t)=\vQ\net_L(\parsnt{}^A,\parsnt{}^F,\parsnt{}^L,t),\forall\vQ^T\vQ=\vI$, the marginal distribution of $p(\bmuL_n)$ defined by \cref{eq:lattice_sampling} is O(3)-invariant. 
\end{restatable}


\textbf{Bayesian Flow of Atom Types \texorpdfstring{$\boldsymbol{A}$}{}} 
Given that atom types are discrete random variables located in a simplex $\calS^K$, the prior parameter of $\boldsymbol{A}$ is the discrete uniform distribution over the vocabulary $\parsnt{0}^A \defeq \frac{1}{K}\vec{1}_{1\times N}$. 
% \begin{align}\label{eq:disc_input_prior}
% \parsnt{0}^A \defeq \frac{1}{K}\vec{1}_{1\times N}
% \end{align}
% \begin{align}
%     (\oh{j}{K})_k \defeq \delta_{j k}, \text{where }\oh{j}{K}\in \R^{K},\oh{\vA}{KD} \defeq \left(\oh{a_1}{K},\dots,\oh{a_N}{K}\right) \in \R^{K\times N}
% \end{align}
With the notation of the projection from the class index $j$ to the length $K$ one-hot vector $ (\oh{j}{K})_k \defeq \delta_{j k}, \text{where }\oh{j}{K}\in \R^{K},\oh{\vA}{KD} \defeq \left(\oh{a_1}{K},\dots,\oh{a_N}{K}\right) \in \R^{K\times N}$, the Bayesian flow distribution of atom types $\vA$ is derived in \citet{bfn}:
\begin{align}
\flow^{A}(\parsn^A \mid \vA; t) &= \E_{\N{\y \mid \beta^A(t)\left(K \oh{\vA}{K\times N} - \vec{1}_{K\times N}\right)}{\beta^A(t) K \vec{I}_{K\times N \times N}}} \delta\left(\parsn^A - \frac{e^{\y}\parsnt{0}^A}{\sum_{k=1}^K e^{\y_k}(\parsnt{0})_{k}^A}\right).
\end{align}
where $\beta^A(t)$ is the predefined accuracy schedule for atom types. Sampling $\btheta_i^A$ from $p_F^A$ as the training signal, the training objective is the $n$-step discrete-time loss for discrete variable \citep{bfn}: 
% \oyyw{can we simplify the next equation? Such as remove $K \times N, K \times N \times N$}
% \begin{align}
% &\calL_A = n\E_{i \sim U\{1,n\},\flow^A(\parsn^A \mid \vA ; t_{i-1}),\N{\y \mid \alphat{i}\left(K \oh{\vA}{KD} - \vec{1}_{K\times N}\right)}{\alphat{i} K \vec{I}_{K\times N \times N}}} \ln \N{\y \mid \alphat{i}\left(K \oh{\vA}{K\times N} - \vec{1}_{K\times N}\right)}{\alphat{i} K \vec{I}_{K\times N \times N}}\nonumber\\
% &\qquad\qquad\qquad-\sum_{d=1}^N \ln \left(\sum_{k=1}^K \out^{(d)}(k \mid \parsn^A; t_{i-1}) \N{\ydd{d} \mid \alphat{i}\left(K\oh{k}{K}- \vec{1}_{K\times N}\right)}{\alphat{i} K \vec{I}_{K\times N \times N}}\right)\label{discdisc_t_loss_exp}
% \end{align}
\begin{align}
&\calL_A = n\E_{i \sim U\{1,n\},\flow^A(\parsn^A \mid \vA ; t_{i-1}),\N{\y \mid \alphat{i}\left(K \oh{\vA}{KD} - \vec{1}\right)}{\alphat{i} K \vec{I}}} \ln \N{\y \mid \alphat{i}\left(K \oh{\vA}{K\times N} - \vec{1}\right)}{\alphat{i} K \vec{I}}\nonumber\\
&\qquad\qquad\qquad-\sum_{d=1}^N \ln \left(\sum_{k=1}^K \out^{(d)}(k \mid \parsn^A; t_{i-1}) \N{\ydd{d} \mid \alphat{i}\left(K\oh{k}{K}- \vec{1}\right)}{\alphat{i} K \vec{I}}\right)\label{discdisc_t_loss_exp}
\end{align}
where $\vec{I}\in \R^{K\times N \times N}$ and $\vec{1}\in\R^{K\times D}$. When sampling, the transition from $\bthetaA_{i-1}$ to $\bthetaA_{i}$ is derived as:
\begin{equation}
    p(\btheta^A_{i}|\parsnt{i-1}^\calM)=\update^A(\btheta^A_{i}|\btheta^A_{i-1},\predA{i-1};t_{i-1})
\end{equation}

The detailed training and sampling algorithm could be found in \cref{alg:train} and \cref{alg:sampling}.




\section{Experiments}
\label{sec:experiments}
The experiments are designed to address two key research questions.
First, \textbf{RQ1} evaluates whether the average $L_2$-norm of the counterfactual perturbation vectors ($\overline{||\perturb||}$) decreases as the model overfits the data, thereby providing further empirical validation for our hypothesis.
Second, \textbf{RQ2} evaluates the ability of the proposed counterfactual regularized loss, as defined in (\ref{eq:regularized_loss2}), to mitigate overfitting when compared to existing regularization techniques.

% The experiments are designed to address three key research questions. First, \textbf{RQ1} investigates whether the mean perturbation vector norm decreases as the model overfits the data, aiming to further validate our intuition. Second, \textbf{RQ2} explores whether the mean perturbation vector norm can be effectively leveraged as a regularization term during training, offering insights into its potential role in mitigating overfitting. Finally, \textbf{RQ3} examines whether our counterfactual regularizer enables the model to achieve superior performance compared to existing regularization methods, thus highlighting its practical advantage.

\subsection{Experimental Setup}
\textbf{\textit{Datasets, Models, and Tasks.}}
The experiments are conducted on three datasets: \textit{Water Potability}~\cite{kadiwal2020waterpotability}, \textit{Phomene}~\cite{phomene}, and \textit{CIFAR-10}~\cite{krizhevsky2009learning}. For \textit{Water Potability} and \textit{Phomene}, we randomly select $80\%$ of the samples for the training set, and the remaining $20\%$ for the test set, \textit{CIFAR-10} comes already split. Furthermore, we consider the following models: Logistic Regression, Multi-Layer Perceptron (MLP) with 100 and 30 neurons on each hidden layer, and PreactResNet-18~\cite{he2016cvecvv} as a Convolutional Neural Network (CNN) architecture.
We focus on binary classification tasks and leave the extension to multiclass scenarios for future work. However, for datasets that are inherently multiclass, we transform the problem into a binary classification task by selecting two classes, aligning with our assumption.

\smallskip
\noindent\textbf{\textit{Evaluation Measures.}} To characterize the degree of overfitting, we use the test loss, as it serves as a reliable indicator of the model's generalization capability to unseen data. Additionally, we evaluate the predictive performance of each model using the test accuracy.

\smallskip
\noindent\textbf{\textit{Baselines.}} We compare CF-Reg with the following regularization techniques: L1 (``Lasso''), L2 (``Ridge''), and Dropout.

\smallskip
\noindent\textbf{\textit{Configurations.}}
For each model, we adopt specific configurations as follows.
\begin{itemize}
\item \textit{Logistic Regression:} To induce overfitting in the model, we artificially increase the dimensionality of the data beyond the number of training samples by applying a polynomial feature expansion. This approach ensures that the model has enough capacity to overfit the training data, allowing us to analyze the impact of our counterfactual regularizer. The degree of the polynomial is chosen as the smallest degree that makes the number of features greater than the number of data.
\item \textit{Neural Networks (MLP and CNN):} To take advantage of the closed-form solution for computing the optimal perturbation vector as defined in (\ref{eq:opt-delta}), we use a local linear approximation of the neural network models. Hence, given an instance $\inst_i$, we consider the (optimal) counterfactual not with respect to $\model$ but with respect to:
\begin{equation}
\label{eq:taylor}
    \model^{lin}(\inst) = \model(\inst_i) + \nabla_{\inst}\model(\inst_i)(\inst - \inst_i),
\end{equation}
where $\model^{lin}$ represents the first-order Taylor approximation of $\model$ at $\inst_i$.
Note that this step is unnecessary for Logistic Regression, as it is inherently a linear model.
\end{itemize}

\smallskip
\noindent \textbf{\textit{Implementation Details.}} We run all experiments on a machine equipped with an AMD Ryzen 9 7900 12-Core Processor and an NVIDIA GeForce RTX 4090 GPU. Our implementation is based on the PyTorch Lightning framework. We use stochastic gradient descent as the optimizer with a learning rate of $\eta = 0.001$ and no weight decay. We use a batch size of $128$. The training and test steps are conducted for $6000$ epochs on the \textit{Water Potability} and \textit{Phoneme} datasets, while for the \textit{CIFAR-10} dataset, they are performed for $200$ epochs.
Finally, the contribution $w_i^{\varepsilon}$ of each training point $\inst_i$ is uniformly set as $w_i^{\varepsilon} = 1~\forall i\in \{1,\ldots,m\}$.

The source code implementation for our experiments is available at the following GitHub repository: \url{https://anonymous.4open.science/r/COCE-80B4/README.md} 

\subsection{RQ1: Counterfactual Perturbation vs. Overfitting}
To address \textbf{RQ1}, we analyze the relationship between the test loss and the average $L_2$-norm of the counterfactual perturbation vectors ($\overline{||\perturb||}$) over training epochs.

In particular, Figure~\ref{fig:delta_loss_epochs} depicts the evolution of $\overline{||\perturb||}$ alongside the test loss for an MLP trained \textit{without} regularization on the \textit{Water Potability} dataset. 
\begin{figure}[ht]
    \centering
    \includegraphics[width=0.85\linewidth]{img/delta_loss_epochs.png}
    \caption{The average counterfactual perturbation vector $\overline{||\perturb||}$ (left $y$-axis) and the cross-entropy test loss (right $y$-axis) over training epochs ($x$-axis) for an MLP trained on the \textit{Water Potability} dataset \textit{without} regularization.}
    \label{fig:delta_loss_epochs}
\end{figure}

The plot shows a clear trend as the model starts to overfit the data (evidenced by an increase in test loss). 
Notably, $\overline{||\perturb||}$ begins to decrease, which aligns with the hypothesis that the average distance to the optimal counterfactual example gets smaller as the model's decision boundary becomes increasingly adherent to the training data.

It is worth noting that this trend is heavily influenced by the choice of the counterfactual generator model. In particular, the relationship between $\overline{||\perturb||}$ and the degree of overfitting may become even more pronounced when leveraging more accurate counterfactual generators. However, these models often come at the cost of higher computational complexity, and their exploration is left to future work.

Nonetheless, we expect that $\overline{||\perturb||}$ will eventually stabilize at a plateau, as the average $L_2$-norm of the optimal counterfactual perturbations cannot vanish to zero.

% Additionally, the choice of employing the score-based counterfactual explanation framework to generate counterfactuals was driven to promote computational efficiency.

% Future enhancements to the framework may involve adopting models capable of generating more precise counterfactuals. While such approaches may yield to performance improvements, they are likely to come at the cost of increased computational complexity.


\subsection{RQ2: Counterfactual Regularization Performance}
To answer \textbf{RQ2}, we evaluate the effectiveness of the proposed counterfactual regularization (CF-Reg) by comparing its performance against existing baselines: unregularized training loss (No-Reg), L1 regularization (L1-Reg), L2 regularization (L2-Reg), and Dropout.
Specifically, for each model and dataset combination, Table~\ref{tab:regularization_comparison} presents the mean value and standard deviation of test accuracy achieved by each method across 5 random initialization. 

The table illustrates that our regularization technique consistently delivers better results than existing methods across all evaluated scenarios, except for one case -- i.e., Logistic Regression on the \textit{Phomene} dataset. 
However, this setting exhibits an unusual pattern, as the highest model accuracy is achieved without any regularization. Even in this case, CF-Reg still surpasses other regularization baselines.

From the results above, we derive the following key insights. First, CF-Reg proves to be effective across various model types, ranging from simple linear models (Logistic Regression) to deep architectures like MLPs and CNNs, and across diverse datasets, including both tabular and image data. 
Second, CF-Reg's strong performance on the \textit{Water} dataset with Logistic Regression suggests that its benefits may be more pronounced when applied to simpler models. However, the unexpected outcome on the \textit{Phoneme} dataset calls for further investigation into this phenomenon.


\begin{table*}[h!]
    \centering
    \caption{Mean value and standard deviation of test accuracy across 5 random initializations for different model, dataset, and regularization method. The best results are highlighted in \textbf{bold}.}
    \label{tab:regularization_comparison}
    \begin{tabular}{|c|c|c|c|c|c|c|}
        \hline
        \textbf{Model} & \textbf{Dataset} & \textbf{No-Reg} & \textbf{L1-Reg} & \textbf{L2-Reg} & \textbf{Dropout} & \textbf{CF-Reg (ours)} \\ \hline
        Logistic Regression   & \textit{Water}   & $0.6595 \pm 0.0038$   & $0.6729 \pm 0.0056$   & $0.6756 \pm 0.0046$  & N/A    & $\mathbf{0.6918 \pm 0.0036}$                     \\ \hline
        MLP   & \textit{Water}   & $0.6756 \pm 0.0042$   & $0.6790 \pm 0.0058$   & $0.6790 \pm 0.0023$  & $0.6750 \pm 0.0036$    & $\mathbf{0.6802 \pm 0.0046}$                    \\ \hline
%        MLP   & \textit{Adult}   & $0.8404 \pm 0.0010$   & $\mathbf{0.8495 \pm 0.0007}$   & $0.8489 \pm 0.0014$  & $\mathbf{0.8495 \pm 0.0016}$     & $0.8449 \pm 0.0019$                    \\ \hline
        Logistic Regression   & \textit{Phomene}   & $\mathbf{0.8148 \pm 0.0020}$   & $0.8041 \pm 0.0028$   & $0.7835 \pm 0.0176$  & N/A    & $0.8098 \pm 0.0055$                     \\ \hline
        MLP   & \textit{Phomene}   & $0.8677 \pm 0.0033$   & $0.8374 \pm 0.0080$   & $0.8673 \pm 0.0045$  & $0.8672 \pm 0.0042$     & $\mathbf{0.8718 \pm 0.0040}$                    \\ \hline
        CNN   & \textit{CIFAR-10} & $0.6670 \pm 0.0233$   & $0.6229 \pm 0.0850$   & $0.7348 \pm 0.0365$   & N/A    & $\mathbf{0.7427 \pm 0.0571}$                     \\ \hline
    \end{tabular}
\end{table*}

\begin{table*}[htb!]
    \centering
    \caption{Hyperparameter configurations utilized for the generation of Table \ref{tab:regularization_comparison}. For our regularization the hyperparameters are reported as $\mathbf{\alpha/\beta}$.}
    \label{tab:performance_parameters}
    \begin{tabular}{|c|c|c|c|c|c|c|}
        \hline
        \textbf{Model} & \textbf{Dataset} & \textbf{No-Reg} & \textbf{L1-Reg} & \textbf{L2-Reg} & \textbf{Dropout} & \textbf{CF-Reg (ours)} \\ \hline
        Logistic Regression   & \textit{Water}   & N/A   & $0.0093$   & $0.6927$  & N/A    & $0.3791/1.0355$                     \\ \hline
        MLP   & \textit{Water}   & N/A   & $0.0007$   & $0.0022$  & $0.0002$    & $0.2567/1.9775$                    \\ \hline
        Logistic Regression   &
        \textit{Phomene}   & N/A   & $0.0097$   & $0.7979$  & N/A    & $0.0571/1.8516$                     \\ \hline
        MLP   & \textit{Phomene}   & N/A   & $0.0007$   & $4.24\cdot10^{-5}$  & $0.0015$    & $0.0516/2.2700$                    \\ \hline
       % MLP   & \textit{Adult}   & N/A   & $0.0018$   & $0.0018$  & $0.0601$     & $0.0764/2.2068$                    \\ \hline
        CNN   & \textit{CIFAR-10} & N/A   & $0.0050$   & $0.0864$ & N/A    & $0.3018/
        2.1502$                     \\ \hline
    \end{tabular}
\end{table*}

\begin{table*}[htb!]
    \centering
    \caption{Mean value and standard deviation of training time across 5 different runs. The reported time (in seconds) corresponds to the generation of each entry in Table \ref{tab:regularization_comparison}. Times are }
    \label{tab:times}
    \begin{tabular}{|c|c|c|c|c|c|c|}
        \hline
        \textbf{Model} & \textbf{Dataset} & \textbf{No-Reg} & \textbf{L1-Reg} & \textbf{L2-Reg} & \textbf{Dropout} & \textbf{CF-Reg (ours)} \\ \hline
        Logistic Regression   & \textit{Water}   & $222.98 \pm 1.07$   & $239.94 \pm 2.59$   & $241.60 \pm 1.88$  & N/A    & $251.50 \pm 1.93$                     \\ \hline
        MLP   & \textit{Water}   & $225.71 \pm 3.85$   & $250.13 \pm 4.44$   & $255.78 \pm 2.38$  & $237.83 \pm 3.45$    & $266.48 \pm 3.46$                    \\ \hline
        Logistic Regression   & \textit{Phomene}   & $266.39 \pm 0.82$ & $367.52 \pm 6.85$   & $361.69 \pm 4.04$  & N/A   & $310.48 \pm 0.76$                    \\ \hline
        MLP   &
        \textit{Phomene} & $335.62 \pm 1.77$   & $390.86 \pm 2.11$   & $393.96 \pm 1.95$ & $363.51 \pm 5.07$    & $403.14 \pm 1.92$                     \\ \hline
       % MLP   & \textit{Adult}   & N/A   & $0.0018$   & $0.0018$  & $0.0601$     & $0.0764/2.2068$                    \\ \hline
        CNN   & \textit{CIFAR-10} & $370.09 \pm 0.18$   & $395.71 \pm 0.55$   & $401.38 \pm 0.16$ & N/A    & $1287.8 \pm 0.26$                     \\ \hline
    \end{tabular}
\end{table*}

\subsection{Feasibility of our Method}
A crucial requirement for any regularization technique is that it should impose minimal impact on the overall training process.
In this respect, CF-Reg introduces an overhead that depends on the time required to find the optimal counterfactual example for each training instance. 
As such, the more sophisticated the counterfactual generator model probed during training the higher would be the time required. However, a more advanced counterfactual generator might provide a more effective regularization. We discuss this trade-off in more details in Section~\ref{sec:discussion}.

Table~\ref{tab:times} presents the average training time ($\pm$ standard deviation) for each model and dataset combination listed in Table~\ref{tab:regularization_comparison}.
We can observe that the higher accuracy achieved by CF-Reg using the score-based counterfactual generator comes with only minimal overhead. However, when applied to deep neural networks with many hidden layers, such as \textit{PreactResNet-18}, the forward derivative computation required for the linearization of the network introduces a more noticeable computational cost, explaining the longer training times in the table.

\subsection{Hyperparameter Sensitivity Analysis}
The proposed counterfactual regularization technique relies on two key hyperparameters: $\alpha$ and $\beta$. The former is intrinsic to the loss formulation defined in (\ref{eq:cf-train}), while the latter is closely tied to the choice of the score-based counterfactual explanation method used.

Figure~\ref{fig:test_alpha_beta} illustrates how the test accuracy of an MLP trained on the \textit{Water Potability} dataset changes for different combinations of $\alpha$ and $\beta$.

\begin{figure}[ht]
    \centering
    \includegraphics[width=0.85\linewidth]{img/test_acc_alpha_beta.png}
    \caption{The test accuracy of an MLP trained on the \textit{Water Potability} dataset, evaluated while varying the weight of our counterfactual regularizer ($\alpha$) for different values of $\beta$.}
    \label{fig:test_alpha_beta}
\end{figure}

We observe that, for a fixed $\beta$, increasing the weight of our counterfactual regularizer ($\alpha$) can slightly improve test accuracy until a sudden drop is noticed for $\alpha > 0.1$.
This behavior was expected, as the impact of our penalty, like any regularization term, can be disruptive if not properly controlled.

Moreover, this finding further demonstrates that our regularization method, CF-Reg, is inherently data-driven. Therefore, it requires specific fine-tuning based on the combination of the model and dataset at hand.


\section{Conclusion}
In this paper, we propose a simple yet effective alignment method for large language models, named \method. \method eliminates the need for both hyperparameters and a reference model while achieving strong performance. The key idea of \method is to directly optimize the reverse perplexity of both the chosen and rejected responses in the preference dataset. Specifically, we minimize the perplexity over the chosen response while maximizing the perplexity over the rejected response, ensuring that the model produces responses with high preference scores. We also provide a theoretical understanding of \method, demonstrating that it can mitigate the problem of gradient imbalance between the chosen and rejected responses during optimization. Furthermore, we show that \method implicitly exhibits mode-seeking behavior, leading to strong alignment performance. Extensive experiments on widely used benchmarks demonstrate that \method significantly outperforms state-of-the-art methods.

\section*{Acknowledgment}
The work of Vasant G Honavar and Teng Xiao was supported in part by grants from the National Science Foundation
(2226025, 2225824), the National Center for Advancing Translational Sciences, and the National
Institutes of Health (UL1 TR002014).
% \clearpage
\bibliography{iclr2025_conference}
\bibliographystyle{iclr2025_conference}

\newpage
\appendix
\section{Proofs of Theorem~\ref{the:mutual}}
\label{app:proof}
To prove Theorem~\ref{the:mutual} in the main paper, we first present the following  Lemmas:

\begin{lemma}\label{lemma:1}
Given the model conditional distribution $\pi_{\theta}(\mathbf{y} \mid \mathbf{x})=\prod_{i=1}^{|\mathbf{y}|} \pi_\theta\left(\mathbf{y}_i \mid \mathbf{x}, \mathbf{y}_{<i}\right)$ and  data distribution $\pi_{\rm{data}}(\mathbf{y} \mid \mathbf{x})=\prod_{i=1}^{|\mathbf{y}|} \pi_{\rm{data}}\left(\mathbf{y}_i \mid \mathbf{x}, \mathbf{y}_{<i}\right)$, then we have the following relationship between the sequence-level TVD objective and its token-level factorization:
   \begin{equation}
    {\mathrm{TV}}(\pi_{\rm{data}}(\mathbf{y} \mid \mathbf{x}), \pi_{\theta}(\mathbf{y} \mid \mathbf{x})) \le \mathbb{E}_{\mathbf{y}\sim \pi_{\rm{data}}(\mathbf{y} \mid \mathbf{x})}\Bigg[\sum\nolimits_{i=1}^{|\mathbf{y}|} {\mathrm{TV}}(\pi_{\rm{data}}^{<i}(\mathbf{y}_{i}), \pi_{\theta}^{<i}(\mathbf{y}_{i}))\Bigg],
\end{equation}
where $\pi_{\rm{data}}^{<i}(\mathbf{y}_{i})$ and $\pi_{\theta}^{<i}(\mathbf{y}_{i})$ are shorts for $\pi_{\rm{data}}\left(\mathbf{y}_i \mid \mathbf{x}, \mathbf{y}_{<i}\right)$ and  $\pi_\theta\left(\mathbf{y}_i \mid \mathbf{x}, \mathbf{y}_{<i}\right)$, respectively.
\end{lemma}
\begin{proof}
We  start by re-writing the TVD loss in the following recursive form:
\begin{align}
    & {\mathrm{TV}}(\pi_{\rm{data}}(\mathbf{y} \mid \mathbf{x}), \pi_{\theta}(\mathbf{y} \mid \mathbf{x}))=\frac{1}{2}\sum  \nolimits_{\mathbf{y}\in \mathcal{Y}} \left| \pi_{\rm{chosen}}(\mathbf{y}\mid\mathbf{x}) -  \pi_\mathbf{\theta}(\mathbf{y} \mid \mathbf{x})  \right|\\
     &=\frac{1}{2} \sum_{\mathbf{y}_1,\cdots,\mathbf{y}_{|\mathbf{y}|}}|\prod_{i=1}^{|\mathbf{y}|} \pi_{\rm{data}}\left(\mathbf{y}_i \mid \mathbf{x}, \mathbf{y}_{<i}\right)-\prod_{i=1}^{|\mathbf{y}|} \pi_\theta\left(\mathbf{y}_i \mid \mathbf{x}, \mathbf{y}_{<i}\right)| \label{Eq:lemma1-1}\\
     &=\frac{1}{2} \sum_{\mathbf{y}_1,\cdots,\mathbf{y}_{|\mathbf{y}|}}|\prod_{i=1}^{|\mathbf{y}|} \pi_{\rm{data}}^{<i}(\mathbf{y}_{i})-\prod_{i=1}^{|\mathbf{y}|} \pi_{\theta}^{<i}(\mathbf{y}_{i})|\\
     &\leq \frac{1}{2}\sum_{i=1}^{|\mathbf{y}|}\sum_{\mathbf{y}_1,\cdots,\mathbf{y}_{i}}\prod _{j=1}^{i-1} \pi_{\rm{data}}^{<j}(\mathbf{y}_{j})|\prod_{i=1}^{|\mathbf{y}|} \pi_{\rm{data}}^{<i}(\mathbf{y}_{i})-\prod_{i=1}^{|\mathbf{y}|} \pi_{\theta}^{<i}(\mathbf{y}_{i})| \sum_{\mathbf{y}_{i+1},\cdots,\mathbf{y}_{|\mathbf{y}|}}\prod_{t=i+1}^{|\mathbf{y}|} \pi_{\theta}^{<t}(\mathbf{y}_{t}) \label{Eq:lemma1-2}\\
     &= \frac{1}{2}\sum_{i=1}^{|\mathbf{y}|}\sum_{\mathbf{y}_1,\cdots,\mathbf{y}_{i}}\prod _{j=1}^{i-1} \pi_{\rm{data}}^{<j}(\mathbf{y}_{j})|\pi_{\rm{data}}^{<i}(\mathbf{y}_{i})- \pi_{\theta}^{<i}(\mathbf{y}_{i})| \label{Eq:lemma1-3} \\
       &= \frac{1}{2}\sum_{i=1}^{|\mathbf{y}|}\sum_{\mathbf{y}_{i}} \mathbb{E}_{ \mathbf{y_{<i}}\sim \pi_{\rm{data}}}\Big[|\pi_{\rm{data}}^{<i}(\mathbf{y}_{i})- \pi_{\theta}^{<i}(\mathbf{y}_{i})|\Big]\\
       &=\mathbb{E}_{\mathbf{y}\sim \pi_{\rm{data}}}\Bigg[ \sum_{i=1}^{|\mathbf{y}|} \mathrm{TV}(\pi_{\rm{data}}^{<i}(\mathbf{y}_{i}),\pi_{\theta}^{<i}(\mathbf{y}_{i})) \Bigg],
\end{align}
where Equation~(\ref{Eq:lemma1-1}) breaks the sequence-level summation into the steps and Equation~(\ref{Eq:lemma1-2}) applies the following triangle inequality~\citep{wen2023f,jitailoring}:
\begin{align}
\left| \prod_{t=1}^{T} a_t - \prod_{t=1}^{T} b_t \right|
\leq \sum_{t=1}^{T} \left| a_t - b_t \right| \cdot
\left( \prod_{i=1}^{t-1} a_i \right) \cdot
\left( \prod_{j=t+1}^{T} b_j \right),
\end{align}
and Equation~(\ref{Eq:lemma1-3}) marginalizes out variables $\mathbf{y}_{i+1},\cdots,\mathbf{y}_{|\mathbf{y}|}$. 
\end{proof}


% \begin{lemma}\label{lemma:2}
% Minimizing the \texttt{SimPO} objective in Equation~(\ref{Eq:SimPO}) with respect to ${\theta}$ reduces the Kullback–Leibler divergence between $\pi_\theta$ and the distribution of chosen response in preference dataset:
% \begingroup\makeatletter\def\f@size{9.5}\check@mathfonts\def\maketag@@@#1{\hbox{\m@th\normalfont\normalfont#1}}
% \begin{align}
% \min_{\theta} \mathcal{L}_{\rm{SimPO}} \Rightarrow  \min_{\theta}{\mathrm{KL}}(\pi_{\rm{chosen}}(\mathbf{y}\mid\mathbf{x})\|  \pi_\mathbf{\theta}(\mathbf{y} \mid \mathbf{x})  ) =\sum  \nolimits_{y\in \mathcal{Y}} \pi_{\rm{chosen}}(\mathbf{y} \mid \mathbf{x})\log \frac{\pi_{\rm{chosen}}(\mathbf{y} \mid \mathbf{x})}{\pi_{\theta}(\mathbf{y} \mid \mathbf{x})}
% \end{align}
% \endgroup
% \end{lemma}
% \begin{proof}
% Recall that the objective of \texttt{SimPO} without hyperparmeters is:
% \begin{align}
% \mathcal{L}_{\rm{SimPO}}({\theta}; \mathcal{D})=\mathbb{E}_{(\mathbf{x}, \mathbf{y}_w, \mathbf{y}_l)\sim \mathcal{D}}\left[-\log \sigma( \log {\pi_{\theta}(\mathbf{y}_w \mid \mathbf{x})}-\log {\pi_{\theta}(\mathbf{y}_l \mid \mathbf{x} )}) \right]. \label{Eq:app-simpo}
% \end{align}
% We connect this objective to a distribution matching problem by defining an energy-based model:
% % \begingroup\makeatletter\def\f@size{9.5}\check@mathfonts\def\maketag@@@#1{\hbox{\m@th\normalfont\normalfont#1}}
% \begin{align}
% \pi(\mathbf{y} \mid \mathbf{x})={\exp(r(\mathbf{x},\mathbf{y}))}/{Z(\mathbf{x})}, \label{Eq:optimal}
% \end{align}
% % \endgroup
% where $Z(\mathbf{x})=\sum_{\mathbf{y}}\exp(r(\mathbf{x},\mathbf{y}))$. We then frame the task as minimizing the KL divergence between the policy $\pi$ and the distribution $\pi_{\mathrm{chosen}}$ that generates the chosen response. The goal is to learn the reward model $r$ such that the model distribution imitates the chosen data distribution:
% \begin{align}
% \min _{r} {\mathrm{KL}}\left(\pi_{\mathrm{chosen}}(\mathbf{y} \mid \mathbf{x}) \| \pi(\mathbf{y} \mid \mathbf{x})\right). \label{eq:FKL}
% \end{align}
% Minimizing the above forward KL divergence with the chosen responses on preference data gives us:
% \begingroup\makeatletter\def\f@size{9.5}\check@mathfonts\def\maketag@@@#1{\hbox{\m@th\normalfont\normalfont#1}}
% \begin{align}
% \min _{r} &\mathbb{E}_{(\mathbf{x},\mathbf{y}_{w})\sim \mathcal{D}}[-\log \frac{\exp(r(\mathbf{x},\mathbf{y}_{w}))}{Z(\mathbf{x})}]=\\
%     &\mathbb{E}_{(\mathbf{x},\mathbf{y}_{w})\sim \mathcal{D}} \Big [-r(\mathbf{x},\mathbf{y}_{w})+\log \mathbb{E}_{q(\mathbf{y}\mid
%     \mathbf{x})} \big[ \frac{\exp (r(\mathbf{x},\mathbf{y}))}{q(\mathbf{y}\mid\mathbf{x})} \big] \Big ]. \label{Eq:connection}
% \end{align}
% \endgroup
% The equality follows due to importance sampling under distribution $q$.  A choice that simplifies the above expression is $q(\mathbf{y} \mid \mathbf{x})=\frac{1}{2} \mathbb{I}(\mathcal{Y}=\mathbf{y}_{l})+\frac{1}{2}  \mathbb{I}(\mathcal{Y}=\mathbf{y}_{w})$.  In this case, the sample-based approximation of Equation~(\ref{Eq:connection})  gives us:
% \begin{align}
% \min_{r}&\mathbb{E}_{(\mathbf{x},\mathbf{y}_{w},\mathbf{y}_{l})\sim \mathcal{D}}\Big [-r(\mathbf{x},\mathbf{y}_{w})+\log (\exp(r(\mathbf{x},\mathbf{y}_{w}))+ \exp(r(\mathbf{x},\mathbf{y}_{l})))\Big ] \nonumber \\
% &=\mathbb{E}_{(\mathbf{x},\mathbf{y}_{w},\mathbf{y}_{l})\sim \mathcal{D}}\Big [-\log (\frac{1}{1+ \exp(r(\mathbf{x},\mathbf{y}_{l})-r(\mathbf{x},\mathbf{y}_{w}))} \Big]. \label{Eq:imtaiton-reward}
% \end{align}
% One can easily see that the above loss is exactly the same as the reward loss function in RLHF~\citep{rafailov2024direct}. By optimizing this loss function, one can directly obtain the optimal policy via Equation~(\ref{Eq:optimal}). Unfortunately,  even if we use the estimate $r$, it is
% still expensive to estimate the partition function $Z(\mathbf{x})$, which makes this representation hard to utilize in practice and  significantly higher inference cost~\citep{rafailov2024direct}. To address this problem,  we apply the following knowledge distillation loss, which distills the optimal policy in Equation~(\ref{Eq:optimal}) into a analytical policy:
% \begin{align}
% \min_{\pi_\theta}\mathrm{KL}\left(\pi_{\theta}(\mathbf{y}\mid \mathbf{x})||{\exp(r(\mathbf{x},\mathbf{y}))}/{Z(\mathbf{x})} \right), \label{Eq:distillation}
% \end{align}
% This gives the following loss after removing multiplicative and additive constants:
% \begin{align}
%     \mathcal{L}(\theta)=-\mathbb{E}_{\pi_\theta(\mathbf{y} \mid \mathbf{x})}\left[r_{\phi}\left(\mathbf{x}, \mathbf{y}\right)-\log \pi_{\theta}(\mathbf{y}|\mathbf{x})\right], \label{Eq:distillation2}
% \end{align}
% One can observe that this knowledge distillation objective directly corresponds to the RL objective in RLHF, as shown in \citep{rafailov2024direct}. Similar to the derivation in~\citep{rafailov2024direct}, which uses an analytical mapping from reward functions to optimal policies in Equation~(\ref{Eq:distillation2}), combining Equation~(\ref{Eq:imtaiton-reward}) results in the SimPO objective in Equation~(\ref{Eq:app-simpo}). Therefore, performing distribution matching over the chosen responses using KL divergence ${\mathrm{KL}}\left(\pi_{\mathrm{chosen}}(\mathbf{y} \mid \mathbf{x}) \| \pi_{\theta}(\mathbf{y} \mid \mathbf{x})\right)$ is equivalent to solving an entropy-regularized RLHF problem, as well as the SimPO objective.
% \end{proof}


\textbf{Theorem~\ref{the:mutual}.} \textit{
Minimizing  \texttt{SFT}  with respect to ${\theta}$ is approximately minimizing the KLD between $\pi_\theta$ and the distribution of the chosen response in the preference dataset, while minimizing our \method  is approximately minimizing the TVD.}
\begingroup\makeatletter\def\f@size{9.5}\check@mathfonts\def\maketag@@@#1{\hbox{\m@th\normalfont\normalfont#1}}
\begin{align}
\min_{\theta} \mathcal{L}_{\rm{SFT}} &\Rightarrow  \min_{\theta}{\mathrm{KL}}(\pi_{\rm{chosen}}(\mathbf{y}\mid\mathbf{x})\|  \pi_\mathbf{\theta}(\mathbf{y} \mid \mathbf{x})  ) =\sum  \nolimits_{y\in \mathcal{Y}} \pi_{\rm{chosen}}(\mathbf{y} \mid \mathbf{x})\log \frac{\pi_{\rm{chosen}}(\mathbf{y} \mid \mathbf{x})}{\pi_{\theta}(\mathbf{y} \mid \mathbf{x})}\\
\min_{\theta} \mathcal{L}_{\rm{SimPER}} &\Rightarrow  \min_{\theta}{\mathrm{TV}}(\pi_{\rm{chosen}}(\mathbf{y}\mid\mathbf{x})\|  \pi_\mathbf{\theta}(\mathbf{y} \mid \mathbf{x})  ) =\sum\nolimits_{y\in \mathcal{Y}} \left| \pi_{\rm{chosen}}(\mathbf{y}\mid\mathbf{x}) -  \pi_\mathbf{\theta}(\mathbf{y} \mid \mathbf{x})  \right| 
\end{align}
\endgroup
\begin{proof}
Given that $\mathbb{E}_{y \sim \pi_{\rm{data}}^{<i}(\mathbf{y}_{i})}[\mathbf{y}_{i}] = \pi_{\rm{data}}^{<i}(\mathbf{y}_{i})$, where $\mathbf{y}_{i}$ represents the observed one-hot distribution with only the  $w$-th index of the observed token being 1 and the others being 0, we have
\begin{align}
\mathrm{TV}&(\pi_{\rm{data}}^{<i}(\mathbf{y}_{i}), \pi_{\theta}^{<i}(\mathbf{y}_{i}))=\frac{1}{2}\sum_{\mathbf{y}_{i}}|\pi_{\rm{data}}^{<i}(\mathbf{y}_{i})- \pi_{\theta}^{<i}(\mathbf{y}_{i})|\\
&=\frac{1}{2}\sum_{\mathbf{y}_{i}}|\mathbb{E}_{y \sim \pi_{\rm{data}}^{<i}(\mathbf{y}_{i})}[\mathbf{y}_{i}]  - \pi_{\theta}^{<i}(\mathbf{y}_{i})|\\
&\leq \frac{1}{2}\mathbb{E}_{y \sim \pi_{\rm{data}}^{<i}(\mathbf{y}_{i})}\Big[\sum_{\mathbf{y}_{i}}|\mathbf{y}_{i} - \pi_{\theta}^{<i}(\mathbf{y}_{i})| \Big]=\mathbb{E}_{y \sim \pi_{\rm{data}}^{<i}(\mathbf{y}_{i})}\left[\mathrm{TV}(\mathbf{y}_{i},\pi_{\theta}^{<i}(\mathbf{y}_{i})) \right]\\
&=\mathbb{E}_{y \sim \pi_{\rm{data}}^{<i}(\mathbf{y}_{i})}\Big[1-\sum_{\mathbf{y}_{i}} \min (\mathbf{y}_{i},\pi_{\theta}^{<i}(\mathbf{y}_{i})) \Big]=- \pi_{\theta}^{<i}(\mathbf{y}_{i}).
\end{align}
Combing the above with Lemma~\ref{lemma:1}, we have:
\begin{align}
{\mathrm{TV}}(\pi_{\rm{data}}(\mathbf{y} \mid \mathbf{x}), \pi_{\theta}(\mathbf{y} \mid \mathbf{x}))&\leq \mathbb{E}_{\mathbf{y}\sim \pi_{\rm{data}}}\Bigg[ \sum_{i=1}^{|\mathbf{y}|} \mathrm{TV}(\pi_{\rm{data}}^{<i}(\mathbf{y}_{i}),\pi_{\theta}^{<i}(\mathbf{y}_{i})) \Bigg]\\
& = -\mathbb{E}_{\mathbf{y}\sim \pi_{\rm{data}}} \Bigg[\sum_{i=1}^{|\mathbf{y}|}\pi_\theta\left(\mathbf{y}_i \mid \mathbf{x}, \mathbf{y}_{<i}\right)\Bigg].
\end{align}
Recall that the objective of \method, without length-averaging, is:
\begin{align}
    \mathcal{L}_{\rm{SimPER}}({\theta}; \mathcal{D})&=-\exp \left( \log \pi_\theta(\mathbf{y}_{w} \mid \mathbf{x})\right)+\exp \left( \log \pi_\theta(\mathbf{y}_{l} \mid \mathbf{x})\right)\\
    &\geq -\pi_\theta(\mathbf{y}_{w} \mid \mathbf{x})\geq {\mathrm{TV}}(\pi_{\rm{chosen}}(\mathbf{y} \mid \mathbf{x}), \pi_{\theta}(\mathbf{y} \mid \mathbf{x})). \label{Eq:43}
\end{align}
Combining Equation~(\ref{Eq:43}) completes the whole proof.
\end{proof}


\section{Experimental Details}
\subsection{Downstream Task Evaluation}
\label{app:task-eval}
This section is for the detailed introduction to the evaluation of downstream tasks.
For models fine-tuned on UltraFeedback Binarized, following previous works~\citep{rafailov2024direct,tunstall2023zephyr}, we evaluated them on HuggingFace Open LLM Leaderboard v1\footnote{\url{https://huggingface.co/spaces/open-llm-leaderboard-old/open_llm_leaderboard}} and v2\footnote{\url{https://huggingface.co/spaces/open-llm-leaderboard/open_llm_leaderboard}}~\citep{open-llm-leaderboard-v1,open-llm-leaderboard-v2}, encompassing ten tasks: 
MMUL-PRO~\citep{wang2024mmluprorobustchallengingmultitask}, IFEval~\citep{zhou2023instructionfollowingevaluationlargelanguage}, BBH~\citep{suzgun2022challengingbigbenchtaskschainofthought}, GPQA~\citep{rein2023gpqagraduatelevelgoogleproofqa}, MUSR~\citep{sprague2024musrtestinglimitschainofthought}, MATH~\citep{hendrycks2021measuringmathematicalproblemsolving}, GSM8k~\citep{cobbe2021training}, ARC~\citep{clark2018think}, 
TruthfulQA~\citep{lin-etal-2022-truthfulqa} and Winogrande~\citep{sakaguchi2021winogrande} provided by the Language Model Evaluation Harness library~\citep{eval-harness}.
In addition to the leaderboard, we also evaluated them on two instruction-following benchmarks AlpacaEval2~\citep{li2023alpacaeval} and MT-Bench~\citep{zheng2023judging}.
For models fine-tuned on Anthropic-HH for safety alignment, following previous works~\citep{rafailov2024direct}, we use GPT-4 for zero-shot pair-wise evaluation, which is shown to be consistent with human judgments.

% \textbf{Massive Multitask Language Understanding Professional}: MMUL-PRO is an enhanced version of the MMLU dataset~\citep{hendrycks2021measuring}, addressing previous shortcomings by increasing choice options in multiple-choice questions and refining question quality through expert review, making it more challenging and less prone to data contamination.

% \textbf{Instruction-Following Evaluation}: IFEval is a benchmark evaluating a model's ability to follow explicit instructions, emphasizing adherence to formatting over content generation.

% \textbf{Big Bench Hard}: BBH is a selection of 23 challenging tasks from the BigBench, focusing on areas like multistep arithmetic, algorithmic reasoning, language understanding, and world knowledge.

% \textbf{Graduate-Level Google-Proof Q\&A}: GPQA is a challenging benchmark composed of advanced questions developed by PhD-level experts across various fields like biology, physics, and chemistry.

% \textbf{Multistep Soft Reasoning}: MUSR is a challenging benchmark consisting of algorithmically generated complex scenarios requiring integrated reasoning and long-range context understanding ability.

% \textbf{MATH}: MATH is a benchmark consisting of high-school level competition problems gathered from multiple sources, consistently formatted with LaTeX for equations and Asymptote for diagrams.

 
% \textbf{GSM8K}: GSM8K is a benchmark consisting of diverse grade school math word problems designed to evaluate a model's capability in solving multi-step mathematical reasoning problems.


% \textbf{AI2 Reasoning Challenge}: ARC is a benchmark consists of grade-school science questions.

% \textbf{TruthfulQA}: TruthfulQA is a benchmark fore the truthfulness of a model in generating responses to questions, including 817 questions across 38 categories such as health, law, finance, and politics.
 
% \textbf{Winogrande}: Winogrande is a large-scale benchmark for commonsense reasoning, inspired by WSC benchmark~\citep{levesque2012winograd}, but adjusted to improve both the scale and the hardness.


% \textbf{AlpacaEval 2.0}: AlpacaEval 2.0 is a LLM-based automatic evaluator for instruction-following language models, which is fast, cheap, replicable, and validated against 20K human annotations.


% \textbf{MT Bench}: MT-bench is an evaluator featuring challenging, multi-turn, open-ended questions to assess chat assistants, which automates the evaluation process by using advanced LLMs as judges.

\textbf{Safety Evaluation}:
To evaluate the safety of the Anthropic HH trained model, the Anthropic HH test set is used as the evaluation dataset. The preferred responses within this test set serve as the baseline. GPT-4 is then employed to determine the winner, which is shown to be consistent with human judgments. We use \texttt{gpt-4-0314} for all our experiments. The specific prompts used for the GPT-4 evaluation are detailed below.

\begin{table}[h!]
    \small
    {\ttfamily
    \begin{tabularx}{\linewidth}{X}
    \toprule
    For the following dialogue history to a chatbot, which response is more helpful and harmless?
    \\
    \\
    Dialogue history: \\
    \{dialogue history\}
    \\
    \\
    Response 1:\\
     \{Response 1\}
    \\
    \\
    Response 2: \\
     \{Response 2\}
    \\
    \\
    FIRST provide a one-sentence comparison of the two responses and explain which you feel is more helpful and harmless. SECOND, on a new line, state only "1" or "2" to indicate which response is more helpful and harmless. Your response should use the format:\\
    Comparison: <one-sentence comparison and explanation>\\
    More helpful: <"1" or "2">\\
    \bottomrule
    \end{tabularx}
    }
    \caption{Prompt for GPT-4 evaluation for the safety alignment task on the Anthropic-HH dataset.   \{dialogue history\},     \{Response 1\} and  \{Response 2\} are placeholders.}
    \label{tab:gpt4_prompt_dialogue}
\end{table}



\subsection{Baseline Methods}
\label{app:baseline}
This section provides a detailed introduction to state-of-the-art baselines for preference fine-tuning, with an emphasis on the usage of hyperparameters in their loss functions.

\textbf{DPO} Direct Preference Optimization~\citep{rafailov2024direct} uses log-likelihood differences to implicitly represent the reward function, eliminating the need for explicit reward model like RLHF. 
DPO involves one tunable hyperparameter, $\beta$, which controls the deviation from the reference model. 
$$
\mathcal{L}_{\rm{DPO}}({\theta}; \mathcal{D})=\mathbb{E}_{(\mathbf{x}, \mathbf{y}_w, \mathbf{y}_l)\sim \mathcal{D}}\left[-\log \sigma(\beta \log \frac{\pi(\mathbf{y}_w \mid \mathbf{x})}{\pi_{\mathrm{ref}}(\mathbf{y}_w \mid \mathbf{x})}-\beta \log \frac{\pi(\mathbf{y}_l \mid \mathbf{x} )}{\pi_{\mathrm{ref}}(\mathbf{y}_l \mid \mathbf{x})})\right].
$$

\textbf{IPO} Identity Preference Optimization~\citep{azar2024general} minimizes a squared loss regression problem by defining an alternative reward function, avoiding unstable RL training. 
IPO involves one hyperparameter, $\beta$, to adjust the reward margin.
$$
\mathcal{L}_{\rm{IPO}}({\theta}; \mathcal{D})=\mathbb{E}_{(\mathbf{x}, \mathbf{y}_w, \mathbf{y}_l)\sim \mathcal{D}}\left[\left (\log \frac{\pi_\theta(\mathbf{y}_w \mid \mathbf{x})\pi_{\mathrm{ref}}(\mathbf{y}_l \mid \mathbf{x})}{\pi_{\theta}(\mathbf{y}_l \mid \mathbf{x} )\pi_{\mathrm{ref}}(\mathbf{y}_w \mid \mathbf{x})}-\frac{1}{2\beta}\right)^{2}\right].
$$

\textbf{CPO} Contrastive Preference Optimization~\citep{xucontrastive} uses log-likelihood as the reward and is trained alongside a Supervised Fine-Tuning (SFT) objective. 
CPO involves two hyperparameters: $\beta$, which scales the log probabilities, and $\lambda$, which weights the SFT component.
$$
\mathcal{L}_{\rm{CPO}}({\theta}; \mathcal{D}) = -\log \sigma\left(\beta \log \pi_\theta(\mathbf{y}_w\mid \mathbf{x}) - \beta \log \pi_\theta(\mathbf{y}_l\mid \mathbf{x}) \right) - \lambda \log \pi_\theta (\mathbf{y}_w\mid \mathbf{x}).
$$


\textbf{SLiC} Sequence Likelihood Calibration~\citep{zhao2023slic} directly uses log-likelihood and includes a SFT objective. 
SLiC involves two hyperparameters: $\delta$, which sets the margin for the ranking loss, and $\lambda$, which weights the SFT component.
$$
\mathcal{L}_{\text{SLiC}}(\theta; \mathcal{D}) = \max\left(0, \delta - \log \pi_\theta(\mathbf{y}_w\mid \mathbf{x}) + \log \pi_\theta(\mathbf{y}_l\mid \mathbf{x})\right) - \lambda \log \pi_\theta (\mathbf{y}_w \mid \mathbf{x}).
$$


\textbf{SimPO} Simple Preference Optimization~\citep{meng2024simpo} eliminates the need for a reference model and optimizes a length-regularized probability of response pairs. 
SimPO involves two hyperparameters: $\beta$ to scale the log probabilities and $\gamma$ to adjust the reward margin. 
$$
\mathcal{L}_{\rm{SimPO}}({\theta}; \mathcal{D})=\mathbb{E}_{(\mathbf{x}, \mathbf{y}_w, \mathbf{y}_l)\sim \mathcal{D}}\left[-\log \sigma(\beta \log {\pi_{\theta}(\mathbf{y}_w \mid \mathbf{x})}-\beta \log {\pi_{\theta}(\mathbf{y}_l \mid \mathbf{x} )})-\gamma \right].
$$


\textbf{KTO} Kahneman-Tversky Optimization~\citep{ethayarajh2024kto} learns from non-paired preference data. 
KTO involves three hyperparameters: $\beta$, which controls the deviation from the reference model; $\lambda_w$ and $\lambda_l$, which weight the preference components for winning and losing responses, respectively. 
\begin{equation}
\begin{split}
\mathcal{L}_{\rm{KTO}}({\theta}; \mathcal{D}) &= -\lambda_w \,\sigma \left( \beta \log \frac{\pi_\theta(\mathbf{y}_w\mid \mathbf{x})}{\pi_{\text{ref}}(\mathbf{y}_w\mid \mathbf{x})} - z_{\text{ref}} \right) +  \lambda_l \,\sigma \left( z_{\text{ref}} - \beta \log \frac{\pi_\theta(\mathbf{y}_l\mid \mathbf{x})}{\pi_{\text{ref}}(\mathbf{y}_l\mid \mathbf{x})} \right), \\
\text{where} \quad z_{\text{ref}} &= \mathbb{E}_{(x, y) \sim \mathcal{D}} \left[\beta \;\mathrm{KL}\left( \pi_\theta(\mathbf{y}\mid \mathbf{x}) || \pi_{\text{ref}}(\mathbf{y}\mid \mathbf{x}) \right)  \right].
\end{split}
\end{equation}


% \begin{table}[h!]
% \caption{Comparison of state-of-the-art alignment approaches with our \method in terms of hyperparameters, the number of hyperparameters, and the elimination of the reference model.}
% \centering
% \vspace{-0.5em}
% \adjustbox{max width=\textwidth}{
% \begin{tabular}{cccc}
%    \toprule[1.0pt]
% \textbf{Approach}    &  \textbf{ Hyperparameters} & \textbf{\#Hyperparameters}   & \textbf{w/o Reference Model}   \\
% \midrule
% DPO & $\beta$ & $\mathbf{1}$   &  \xmark      \\
% IPO  & $\beta$ & $\mathbf{1}$  &  \xmark    \\
% CPO  & $\lambda, \beta$  & $\mathbf{2}$ & \cmark    \\
% SLiC  & $\delta, \lambda$ & $\mathbf{2}$  &  \cmark    \\
% SimPO  & $\gamma, \beta$ & $\mathbf{2}$  &  \cmark    \\
% KTO & $\lambda_l, \lambda_w, \beta$ & $\mathbf{3}$  &  \xmark   \\
% \midrule
% \method & - & $\mathbf{0}$ &   \cmark   \\
% \toprule[1.0pt]
% \end{tabular}}\label{table:compare_full}
% \end{table}


\subsection{Implementation Details}

\paragraph{Training Hyperparameters.}
For general hyperparameters, we adhered strictly to the settings used in SimPO. 
We applied the following hyperparameters: For the SFT stage, we use a learning rate of $2 \times 10^{-5}$. For both the SFT and the preference optimization stages, we use a batch size of 128, a maximum sequence length of 2048, and a cosine learning rate schedule with 10\% warmup steps for one epoch, all through the Adam optimizer~\citep{kingma2014adam}.
We maintain these settings consistently to ensure uniformity and comparability across experiments.

For method-specific hyperparameters, we also followed the search strategy from SimPO, noting that our method does not require any additional hyperparameters beyond the learning rate. Specifically, for different baseline methods, each with its own set of additional hyperparameters, the search strategy is in \cref{table_baseline_hyper}. 
Each method is individually search for the learning rates in $[3e-7, 5e-7, 6e-7, 1e-6]$.  

\begin{table}[h]
\small
\centering
\caption{Search ranges for hyperparameters across baseline methods}
\begin{tabular}{cc}
\toprule
\textbf{Method} & \textbf{Hyperparameter Search Range} \\
\midrule
DPO & $\beta \in [0.01, 0.05, 0.1]$ \\
\midrule
IPO & $\tau \in [0.01, 0.1, 0.5, 1.0]$ \\
\midrule
\multirow{2}{*}{CPO} 
& $\lambda = 1.0$ \\
& $\beta \in [0.01, 0.05, 0.1]$ \\
\midrule
\multirow{2}{*}{SLiC} 
& $\lambda \in [0.1, 0.5, 1.0, 10.0]$ \\
& $\delta \in [0.1, 0.5, 1.0, 2.0]$ \\
\midrule
\multirow{2}{*}{KTO} 
& $\lambda_l = \lambda_w = 1.0$ \\
& $\beta \in [0.01, 0.05, 0.1]$ \\
\midrule
\multirow{2}{*}{SimPO} 
& $\beta \in [2.0, 2.5]$ \\
& $\gamma \in [0.3, 0.5, 1.0, 1.2, 1.4, 1.6]$ \\
\bottomrule
\end{tabular}
\label{table_baseline_hyper}
\end{table}


\paragraph{Evaluation Hyperparameters.}
The hyperparameters for evaluation on AlpacaEval 2 adhere to SimPO, with a temperature of 0.7 for the Mistral-Base setting, a temperature of 0.5 for the Mistral-Instruct setting, and a temperature of 0.9 for both Llama3 settings. The hyperparameters for evaluation on MT-Bench are set according to the official decoding configuration, which specifies distinct sampling temperatures for various categories.

\paragraph{Computation Resources.} 
All the training experiments in this paper were conducted on 4 $\times$ NVIDIA A100 (80G) GPUs with 128 batchsize, based on the alignment-handbook repo.\footnote{\url{https://github.com/huggingface/alignment-handbook}}


% \section{Additional Experimental Results}


% \begin{figure}[t!]
% \centering 
% \includegraphics[width=1\textwidth]{pic/rewards-appendix.pdf}
% \caption{The training dynamics during training of \method and \texttt{SimPO} with  different hyperparameters on the 4 different base models. We can observe  that \method exhibits the least decline in chosen rewards, while still achieving the most significant increase in reward margins, compared to \texttt{SimPO} across various hyperparameters.}
% \vskip -1em
% \label{fig:app:rewards} 
% \end{figure}

\section{Case Study}
\label{app:evaluation}
% \textcolor{red}{Warning: The following case studies contain example data that may be offensive or harmful.}

In this section, we present examples of comparisons between \method  and the baseline \texttt{SimPO}. 
% See~\cref{tab:case-1} for dialogue examples, and~\ref{tab:case-2} for summarization examples.





\begin{table}[!h]
    \centering
    \ttfamily
    \small
        \begin{tabular}{lp{11.5cm}}
        \toprule[1pt]
        \bf{Prompt} & \texttt{Does the 5 second rule actually exist?}\\
        \midrule
        \bf{SimPER} & 
\texttt{The "5 second rule" is an urban legend that claims that if food falls on the floor, it is still safe to eat as long as you pick it up within 5 seconds. However, this rule is not a scientifically proven fact, and there is no evidence to support it. Here are some reasons why the 5-second rule is not a reliable method for determining food safety:}\newline


\textbf{1. Bacteria and viruses can transfer quickly}: Bacteria and viruses can transfer from surfaces to food within seconds. The amount of time that passes between a food item's contact with a contaminated surface and its being picked up can affect the likelihood of bacterial or viral transmission. \newline

\textbf{2. Contamination levels on floors and surfaces}: The level of contamination on floors and surfaces can vary widely depending on the environment, the cleanliness of the area, and the presence of moisture. Even if a surface is dry, bacteria and viruses can still be present.\newline

\textbf{3. Personal hygiene}: The cleanliness of your hands, the surface you place the food on, and the utensils you use to pick up the food all play a role in food safety. Picking up food within 5 seconds does not ensure that it is clean enough to eat.\newline

\textbf{4. Temperature}: The temperature of food can also affect the risk of bacterial or viral contamination. Food that has been at room temperature for a prolonged time, even if picked up quickly, may still harbor dangerous bacteria.\newline

Therefore, it is best to follow proper food handling and preparation practices, such as washing hands before and after handling food, using clean utensils and surfaces, and cooking food to the appropriate temperatures to ensure its safety. \\
        \midrule
        \bf{SimPO} & 
The "5-second rule" refers to a popular myth or belief that suggests touching food on the ground for less than five seconds makes it safe to consume without contamination. However, this rule does not exist scientifically.
According to food safety guidelines, bacteria can transfer from contaminated surfaces to food within seconds, regardless of the elapsed time. The duration of contact between the food and contaminated surfaces plays a crucial role in determining the likelihood of bacterial transfer, but it's not limited to five seconds.
Cross-contamination occurs when bacteria from unwashed hands, dirty utensils, or contaminated surfaces come into contact with food, leading to potential foodborne illnesses. Therefore, adhering to proper hygiene practices, such as washing hands, cleaning surfaces, and cooking food thoroughly, remains essential to prevent food contamination.
In summary, there is no scientific basis for the "5-second rule," and relying solely on this belief could increase the risk of consuming contaminated food."
        \\
        % \midrule
        % \textbf{Judgment} & \\
        \bottomrule[1pt]
    \end{tabular}
    \caption{An AlpacaEval 2 generation from the Llama-3-8B-Base model trained with \texttt{SimPER} and \texttt{SimPO}. Alpaca-Eval-GPT4-Turbo prefers \method over \texttt{SimPO}. Compared to \texttt{SimPO}, the generation by \texttt{SimPER} is better structured, with hierarchical discussions that make it more helpful. 
    }
    \label{tab:case-1}
\end{table}



\begin{table}[!h]
    \centering
        \ttfamily
        \small
    \begin{tabular}{lp{11.5cm}}
        \toprule[1pt]
        \bf{Prompt} & How do I make escargot? \\
        \midrule
        \bf{SimPER} &  
Here's a general recipe for making escargot: \newline

\textbf{Ingredients}:\newline
- 1 pound fresh snails (escargots)\newline
- 1/4 cup butter, at room temperature\newline
- 1 clove garlic, minced\newline
- 2 tablespoons finely chopped parsley\newline
- 2 tablespoons finely chopped shallots\newline
- 2 tablespoons dry white wine\newline
- 2 tablespoons lemon juice\newline
- 1/4 teaspoon ground black pepper\newline
- 1/4 cup breadcrumbs\newline
- 1/4 cup grated Parmesan cheese\newline
- Salt, to taste\newline
- 1 tablespoon olive oil\newline

\textbf{Equipment}:\newline
- Snail shells\newline
- Foil\newline
- A heavy-bottomed skillet\newline
- A wooden spoon\newline
- A baking dish\newline
- A fine-mesh strainer\newline
- A small saucepan\newline
- A mixing bowl\newline
- Parchment paper\newline

\textbf{Instructions}:\newline
1. Preheat the oven to 375F (190C).

2. Scrub the snails under cool running water to remove any dirt or debris. Discard any snails that are not alive. Soak the snails in salted water for at least 30 minutes, changing the water every 10 minutes, to ensure they are thoroughly cleaned.

3. In a small saucepan, melt the butter over low heat. Add the minced garlic, parsley, and shallots, and cook for 1-2 minutes until fragrant.

4. Remove the garlic mixture from the heat and stir in the white wine, lemon juice, and black pepper. Season with salt to taste.

5. In a mixing bowl, combine the breadcrumbs and grated Parmesan cheese.

6. Place the snail shells on a sheet of foil and arrange them in a baking dish. Spoon the garlic butter mixture into the shells, filling each one about 3\/4 full.

7. Sprinkle the breadcrumb mixture over the top of each snail.

8. Cover the snail shells with foil.

9. Bake in the preheated oven for 12-15 minutes or until the breadcrumbs are golden brown and the snails have cooked through.

10. Remove the dish from the oven, carefully remove the foil, and place a small piece of parchment paper over the top of each shell.

11. Return the dish to the oven and continue baking for 1-2 minutes to crisp up the breadcrumbs.

12. Remove the dish from the oven, and let the escargots cool for a few minutes.

13. In a heavy-bottomed skillet, heat the olive oil over medium-high heat. Strain the garlic butter mixture through a fine-mesh strainer into the skillet, and cook for 1-2 minutes until slightly reduced and thickened.

14. To serve, spoon the garlic butter onto small plates, arrange the escargots on the plates, and drizzle the remaining garlic butter over the top.

Note: Escargot shells are available for purchase at specialty food stores or online. If you don't have escargot shells, you can use mushroom caps or even small ramekins.

Enjoy your delicious escargot! \\
\midrule
\end{tabular}
\end{table}
\begin{table}[!h]
    \centering
        \ttfamily
        \small
    \begin{tabular}{lp{11.5cm}}
        \midrule
        \bf{SimPO} &
Here's a general recipe for making escargot:\newline

\textbf{Ingredients}:\newline
- 1 pound fresh or canned snails (escargots), cleaned and cooked\newline
- 1/2 cup unsalted butter, softened\newline
- 2 cloves garlic, minced\newline
- 1 tablespoon finely chopped fresh parsley\newline
- 1 tablespoon finely chopped fresh thyme\newline
- 1/4 teaspoon red pepper flakes (optional)\newline
- Salt and freshly ground black pepper, to taste\newline
- 24-30 small, hollowed-out garlic cloves or escargot shells\newline

\textbf{Instructions}:

1. Preheat your oven to 450F (230C).

2. Prepare the garlic cloves or escargot shells by gently scraping out the pulp with knife, leaving a thin layer intact. Rinse them under running water and pat dry.

3. In a mixing bowl, combine softened butter, minced garlic, chopped parsley, thyme, red pepper flakes (if using), salt, and black pepper. Mix well until blended.

4. Spoon a generous amount of seasoned butter into each garlic clove or escargot shell, filling them almost to the top.

5. Arrange the filled shells in a baking dish, preferably one with individual wells or indentations, spaced evenly apart.

6. Place the cooked snails into each shell, making sure they fit snugly.

7. Bake the escargot for 8-10 minutes, or until the butter has melted and bubbled around the edges, and the snails are heated through.

8. Remove the dish from the oven and serve immediately, accompanied by crusty bread, toasted baguette slices, or crackers.

9. To eat, use a small fork or escargot spoon to extract the snails from their shells, dip them into the remaining garlic butter, and enjoy!

Note: Always follow proper food handling and preparation guidelines when working with live or cooked snails, especially if using fresh ones. Ensure they are properly cleaned, cooked, and stored according to recommended safety practices. \\
\bottomrule[1pt]
    \end{tabular}
    % \vspace{4mm}
    \caption{An AlpacaEval 2 generation from the Mistral-7B-Base model trained with \texttt{SimPER} and \texttt{SimPO}. Alpaca-Eval-GPT4-Turbo prefers \method over \texttt{SimPO}. Compared to \texttt{SimPO}, the generation by \texttt{SimPER} is better structured, with hierarchical discussions that make it more helpful.}
    \label{tab:case-2}
\end{table}

\begin{table}[!h]
    \centering
        \ttfamily
        \small
    \begin{tabular}{lp{11.5cm}}
        \toprule[1pt]
        \bf{Prompt} & Mick pays his teacher \$800 for 40 lessons worth 2 hours each. If this will be all he is going to pay for his lessons, how much did he receive?\\
        \midrule
        \bf{SimPER} & Mick did not receive any money. The \$800 he paid to his teacher is the total cost for the 40 lessons, which is 40 x 2 hours = 80 total hours of lessons. Since he only paid for the lessons, he will not receive any money back.
        
        Confidence: 95\% \\
        \midrule
        \bf{SimPO} & Mick received \$800.
        
        Confidence: 95\%\\
        \bottomrule[1pt]
    \end{tabular}
    \caption{An AlpacaEval 2 generation from the Llama-3-8B-Base model trained with \texttt{SimPER} and \texttt{SimPO}. Alpaca-Eval-GPT4-Turbo prefers \method over \texttt{SimPO}. Compared to \texttt{SimPO}, \texttt{SimPER}'s output provides clearer step-by-step reasoning.}
    \label{tab:case-3}
\end{table}




\end{document}
