
\documentclass{article} % For LaTeX2e
\usepackage{iclr2025_conference,times}
\iclrfinalcopy

\usepackage{url}            %
\usepackage{booktabs}       %
\usepackage{amsfonts}
\usepackage{bbm}%
\usepackage{nicefrac}       %
\usepackage{microtype}      %
\usepackage{xcolor}         %
\usepackage{fleqn, tabularx}
\usepackage{multirow}
\usepackage[export]{adjustbox}
\usepackage{amsfonts}
\usepackage{amsmath, amsthm, bbm, mathtools, commath}
\usepackage{amssymb}
\usepackage{colortbl}
\usepackage{wrapfig}
\usepackage{algorithm}
\usepackage[noend]{algorithmic}
\usepackage{quoting}
% \usepackage[T1]{fontenc}
\usepackage{bold-extra}
% \usepackage{lmodern}

\newtheorem{theorem}{Theorem}[section]
\newtheorem{proposition}[theorem]{Proposition}
\newtheorem{lemma}[theorem]{Lemma}
\newtheorem{corollary}[theorem]{Corollary}
\newtheorem{definition}[theorem]{Definition}
\newtheorem{assumption}[theorem]{Assumption}
\usepackage{pifont}% http://ctan.org/pkg/pifont
\definecolor{mark}{RGB}{208,64,56}
\newcommand{\cmark}{{\color{mark}\ding{51}}}
\newcommand{\xmark}{\ding{55}}

\usepackage[algo2e,linesnumbered,ruled,lined]{algorithm2e}
\usepackage{datetime}
\usepackage{verbatim}




\usepackage[font=small]{subfig}
\usepackage{colortbl}
\usepackage{arydshln} %
\newcommand{\facc}[2]{{#1}{\scriptsize±{#2}}}
\newcommand{\bfacc}[2]{\textbf{{#1}{\scriptsize±{#2}}}}
% \newcommand{\bestcell}{\cellcolor{}}
\newcommand{\ind}{\mathbbm{1}}
\newcommand{\NN}{\mathbb N}
\newcommand{\RR}{\mathbb R}
\newcommand{\mc}[1]{\mathcal{#1}}
\newcommand{\mbb}[1]{\\usepackage{bbm}{#1}}\usepackage{bbm}
\newcommand{\mrm}[1]{\mathrm{#1}}
\newcommand{\ol}[1]{\overline{#1}}
\newcommand{\mbf}{\mathbf}
\newcommand{\todo}[1]{\textcolor{red}{TODO: #1}}
\newcommand{\bestcell}{\cellcolor{blue!25}}
\newcommand{\myparagraph}[1]{\noindent \textbf{#1}}
\newcommand{\grayrow}{\rowcolor[gray]{.9}}
\newcommand{\indd}{\textit{ind}\xspace}
\newcommand{\tran}{\textit{tran}\xspace}
\newcommand{\production}{\textit{prod}\xspace}
\newcommand{\printfnsymbol}[1]{%
  \textsuperscript{\@fnsymbol{#1}}%
}
\usepackage{paralist}
\makeatother
\newcommand\eqc{\mathrel{\overset{\makebox[0pt]{\mbox{\normalfont\tiny\sffamily c}}}{=}}}
\newcommand{\indep}{\perp \!\!\! \perp}
\newcommand{\A}{\mbf{A}}
\newcommand{\X}{\mbf{X}}
\newcommand{\W}{\mbf{W}}
\newcommand{\Y}{\mbf{Y}}
\newcommand{\hA}{\mbf{h}_\A}
\newcommand{\hX}{\mbf{h}_\X}
% \usepackage{subfigure}
\setlength\dashlinegap{3pt}
\setlength\arrayrulewidth{.5pt}
\setlength\dashlinedash{4pt}
\newcommand{\std}[1]{\pm #1}

% Optional math commands from https://github.com/goodfeli/dlbook_notation.
% %%%%% NEW MATH DEFINITIONS %%%%%

% \usepackage{amsmath,amsfonts,bm}
\usepackage{amsmath,amsfonts}

\usepackage{pifont}


\newcommand{\R}{\mathbb{R}}


\def\va{{\mathbf{a}}}
\def\vg{{\mathbf{g}}}

% Sets
\def\sR{\mathbb{R}}
\def\sC{\mathbb{C}}
\def\sZ{\mathbb{Z}}
\def\sN{\mathbb{N}}
\def\sQ{\mathbb{Q}}

\def\sS{\mathcal{S}}



% Vectors
\def\vzero{{\mathbf{0}}}
\def\vone{{\mathbf{1}}}
\def\vmu{{\mathbf{\mu}}}
\def\vtheta{{\mathbf{\theta}}}
\def\va{{\mathbf{a}}}
\def\vb{{\mathbf{b}}}
\def\vc{{\mathbf{c}}}
\def\vd{{\mathbf{d}}}
\def\ve{{\mathbf{e}}}
\def\vf{{\mathbf{f}}}
\def\vg{{\mathbf{g}}}
\def\vh{{\mathbf{h}}}
\def\vi{{\mathbf{i}}}
\def\vj{{\mathbf{j}}}
\def\vk{{\mathbf{k}}}
\def\vl{{\mathbf{l}}}
\def\vm{{\mathbf{m}}}
\def\vn{{\mathbf{n}}}
\def\vo{{\mathbf{o}}}
\def\vp{{\mathbf{p}}}
\def\vq{{\mathbf{q}}}
\def\vr{{\mathbf{r}}}
\def\vs{{\mathbf{s}}}
\def\vt{{\mathbf{t}}}
\def\vu{{\mathbf{u}}}
\def\vv{{\mathbf{v}}}
\def\vw{{\mathbf{w}}}
\def\vx{{\mathbf{x}}}
\def\vy{{\mathbf{y}}}
\def\vz{{\mathbf{z}}}
\def\vzeta{{\mathbf{\zeta}}}

% Matrix
\def\mA{{\mathbf{A}}}
\def\mB{{\mathbf{B}}}
\def\mC{{\mathbf{C}}}
\def\mD{{\mathbf{D}}}
\def\mE{{\mathbf{E}}}
\def\mF{{\mathbf{F}}}
\def\mG{{\mathbf{G}}}
\def\mH{{\mathbf{H}}}
\def\mI{{\mathbf{I}}}
\def\mJ{{\mathbf{J}}}
\def\mK{{\mathbf{K}}}
\def\mL{{\mathbf{L}}}
\def\mM{{\mathbf{M}}}
\def\mN{{\mathbf{N}}}
\def\mO{{\mathbf{O}}}
\def\mP{{\mathbf{P}}}
\def\mQ{{\mathbf{Q}}}
\def\mR{{\mathbf{R}}}
\def\mS{{\mathbf{S}}}
\def\mT{{\mathbf{T}}}
\def\mU{{\mathbf{U}}}
\def\mV{{\mathbf{V}}}
\def\mW{{\mathbf{W}}}
\def\mX{{\mathbf{X}}}
\def\mY{{\mathbf{Y}}}
\def\mZ{{\mathbf{Z}}}
\def\mBeta{{\mathbf{\beta}}}
\def\mPhi{{\mathbf{\Phi}}}
\def\mLambda{{\mathbf{\Lambda}}}
\def\mSigma{{\mathbf{\Sigma}}}


% Expectation
% \def\eE{\mathop{\mathbb{E}}\limits}
\def\eE{\mathbb{E}}

% Probability
\def\pP{\mathbb{P}}

% Tilde
\def\tf{\tilde{f}}
\def\tS{\tilde{S}}
\def\wtF{\widetilde{\mathcal{F}}}
\def\whR{\widehat{R}}
\def\tvx{\tilde{\mathbf{x}}}
\def\ty{\tilde{y}}


\def\defeq{\overset{\textup{def}}{=}}
% \def\defeq{\overset{.}{=}}
\def\defone{\overset{\text{\ding{172}}}{=}}
\def\deftwo{\overset{\text{\ding{173}}}{=}}
\def\leqone{\overset{\text{\ding{172}}}{\leq}}
\def\leqtwo{\overset{\text{\ding{173}}}{\leq}}
\def\leqthree{\overset{\text{\ding{174}}}{\leq}}
\def\leqfour{\overset{\text{\ding{175}}}{\leq}}
\def\eqone{\overset{\text{\ding{172}}}{=}}
\def\eqtwo{\overset{\text{\ding{173}}}{=}}
\def\eqthree{\overset{\text{\ding{174}}}{=}}
\def\eqfour{\overset{\text{\ding{175}}}{=}}
\def\geqfive{\overset{\text{\ding{176}}}{\geq}}
\newcommand{\fix}{\marginpar{FIX}}
\newcommand{\new}{\marginpar{NEW}}
\usepackage{lineno} 
% color related
\definecolor{Gray}{gray}{0.9}
\usepackage{xcolor}
\colorlet{darkgreen}{green!65!black}
\colorlet{darkblue}{blue!75!black}
\colorlet{darkred}{red!80!black}
\definecolor{lightblue}{HTML}{0071bc}
\definecolor{lightgreen}{HTML}{39b54a}
\definecolor{manyshot}{HTML}{6969ff}
\definecolor{medshot}{HTML}{f7c600}
\definecolor{fewshot}{HTML}{ff6969}
\definecolor{mypurple}{HTML}{412F8A}
\definecolor{myorange}{HTML}{fc8e62}
\definecolor{citecolor}{HTML}{0071BC}
\definecolor{linkcolor}{HTML}{ED1C24}
\definecolor{Blue9}{rgb}{0.098,0.3,0.9}
% \definecolor{citecolor}{HTML}{3333A6}
\definecolor{urlcolor}{HTML}{3333A6}
\usepackage[pagebackref=false, breaklinks=true, colorlinks,
            citecolor=darkblue, urlcolor=urlcolor,linkcolor=linkcolor, bookmarks=false]{hyperref}
\usepackage[capitalize,noabbrev]{cleveref}
\usepackage{tabularx}
\usepackage{url}
\newcommand{\teng}[1]{\textcolor{blue}{#1}}
\usepackage{longtable,tabularx,booktabs,wrapfig}
\renewcommand\ttdefault{cmtt}
\newcommand{\method}{\texttt{SimPER}\xspace}
\newcommand{\MLP}{\texttt{MLP}\xspace}
\newcommand{\GNN}{\texttt{GNN}\xspace}


\definecolor{customgray}{rgb}{0.25,0.25,0.25}
\definecolor{customred}{rgb}{0.8,0.05,0.05}
\definecolor{urlcolors}{rgb}{0.872,0.2,0.552}
\newcommand{\sbest}[1]{\textcolor{customgray}{\textbf{#1}}}
\newcommand{\best}[1]{\textcolor{customred}{\textbf{#1}}}
\newcommand{\urlcolor}[1]{\textcolor{urlcolors}{\textbf{#1}}}


% \title{\method:  Simple Perplexity Optimization for Hyperparameter-free Preference Alignment}
\title{SimPER: A Minimalist Approach to Preference \\ Alignment without Hyperparameters}
% B3C: A Minimalist Approach to Offline Multi-Agent Reinforcement Learning
% a minimalist approach to offline reinforcement learning

% \title{Hyperparameter-Free Preference Optimization}

% \author{Teng Xiao$^{\heartsuit}$\thanks{Equal contribution.}\hspace{1.5mm}, Yige Yuan$^{\spadesuit}$\footnotemark[1]\hspace{1.5mm}, Zhengyu Chen$^{\clubsuit}$ , Mingxiao Li$^\blacktriangle$, \\ \textbf{Shangsong Liang}$^\blacklozenge$, \textbf{Zhaochun Ren}$^\diamondsuit$, \textbf{Vasant G Honavar}$^{\heartsuit}$  \\
% $^\heartsuit$Pennsylvania State University
% $^\spadesuit$University of Chinese Academy of Sciences \\  $^\clubsuit$Meituan Inc$^\blacktriangle$Tencent AI Lab $^\blacklozenge$Sun Yat-Sen University $^\diamondsuit$Leiden University\\
% \texttt{tengxiao@psu.edu}, 
% \texttt{yuanyige923@gmail.com},
% \texttt{vhonavar@psu.edu} 
% }
\author{Teng Xiao$^{1}$\thanks{Equal contribution.}\hspace{1.5mm}, Yige Yuan$^{2}$\footnotemark[1]\hspace{1.5mm}, Zhengyu Chen$^{3}$, Mingxiao Li$^4$, \\ \textbf{Shangsong Liang}$^5$\textbf{,} \textbf{Zhaochun Ren}$^6$\textbf{,} \textbf{Vasant G Honavar}$^{1}$  \\
$^1$Pennsylvania State University
$^2$University of Chinese Academy of Sciences \\  $^3$Meituan Inc $^4$Tencent AI Lab $^5$Sun Yat-Sen University $^6$Leiden University\\
\texttt{tengxiao@psu.edu}, 
\texttt{yuanyige923@gmail.com},
\texttt{vhonavar@psu.edu} 
}
% The \author macro works with any number of authors. There are two commands
% used to separate the names and addresses of multiple authors: \And and \AND.
%
% Using \And between authors leaves it to \LaTeX{} to determine where to break
% the lines. Using \AND forces a linebreak at that point. So, if \LaTeX{}
% puts 3 of 4 authors names on the first line, and the last on the second
% line, try using \AND instead of \And before the third author name.


%\iclrfinalcopy % Uncomment for camera-ready version, but NOT for submission.
\begin{document}

\maketitle

\begin{abstract}
Existing preference optimization objectives for language model alignment require additional hyperparameters that must be extensively tuned to achieve optimal performance, increasing both the complexity and time required for fine-tuning large language models. 
In this paper, we propose a simple yet effective hyperparameter-free preference optimization algorithm for alignment. We observe that promising performance can be achieved simply by optimizing inverse perplexity, which is calculated as the inverse of the exponentiated average log-likelihood of the chosen and rejected responses in the preference dataset.
The resulting simple learning objective, \method (\textbf{{Sim}}ple alignment with \textbf{Per}plexity optimization), is easy to implement and eliminates the need for expensive hyperparameter tuning and a reference model, making it both computationally and memory efficient. Extensive experiments on widely used real-world benchmarks, including MT-Bench, AlpacaEval 2, and \textbf{10} key benchmarks of the Open LLM Leaderboard with \textbf{5} base models, demonstrate that \method consistently and significantly outperforms existing approaches—even without any hyperparameters or a reference model. For example, despite its simplicity, \method outperforms state-of-the-art methods by up to \textbf{5.7} points on AlpacaEval 2 and achieves the highest average ranking across \textbf{10} benchmarks on the Open LLM Leaderboard. The source code for \method is publicly available at the Github: \url{https://github.com/tengxiao1/SimPER}.
\end{abstract}



\documentclass[../main.tex]{subfiles}
\graphicspath{{../images/}}
\makeatletter
\def\input@path{{../images/}}
\makeatother
\begin{document}
\section{Introduction}
\begin{figure}
\centering
\begin{tikzpicture}
\node[inner sep=0pt] (ws) at (0, 0) {
\includegraphics[height=.4\textwidth, trim={10cm 0 10cm 0},clip]{world_space.png}};
\node[inner sep=0pt] (cs) at (6,0) {\includegraphics[height=.4\textwidth, trim={10cm 1cm 10cm 4cm},clip]{conf_space.png}};
\end{tikzpicture}
\vspace{-5pt}
\label{fig:pbrm_intro}
\caption{\textbf{Left}: Shows world space obstacles as grey spheres. Robots start and goal configuration is colored red and green, respectively. Configurations along the computed path are colored transparent blue. \textbf{Right:} Mapped world space scenario to configuration space. Obstacle region is the grey mesh. Red spheres are collision-free regions computed by the neural SCDF. The optimized shortest path in the convex corridor is the blue curve.}
\vspace{-25pt}
\end{figure}
Motion planning is the problem of finding a collision-free trajectory that connects a given start and goal configuration. The planning takes place in the configuration space of the robot. For single body robots, like mobile robots or drones, the configuration space and the world space are usually the same. This simplifies the planning, since explicit obstacle representations are available which enables geometrical tools like separating hyperplanes, smallest distance to obstacles etc., to be used when designing motion planning algorithms. For multi-body robots like manipulators, the situation is completely different. The world space obstacles are usually mapped to non-convex regions, and to make the problem even harder, the mapping is usually not known. Forming explicit representations of the obstacle region in the configuration space is usually too expensive or intractable. Despite all of this, sampling based planners are used with great success, which mainly is due to their use of implicit representations of the obstacle region. The basic idea is to construct a graph in the configuration space that covers and connects the collision-free region. From this graph, a path can be extracted that connects a given start and goal configuration. The approach is computationally expensive, since the graph is constructed with the smallest geometrical building block available, points, which represents a collision-check. Furthermore, the extracted paths from the graph are non-smooth and jagged due to the stochastic nature of the approach. This adds an additional post-processing step to the process, where the paths are shortcutted and smoothened, before the path can be used for tracking. Clearly a lot of time is invested to form this graph and produce smooth paths. Thus, if the obstacles start to move, then all of this work is done in no use, since all points that make up this graph need to be re-verified, which is simply too time consuming to be done in real time.
\\\\
In this work, we want to address the existing drawbacks of the sampling based planners. Our main contribution is an improved motion planner where each vertex in the graph covers a collision-free region in the form of a sphere instead of a point and where the edges are formed with neighboring intersecting spheres. This representation has the advantage of instead of returning piecewise linear paths, returning a sequence of overlapping spheres, i.e. a convex corridor, that connects a given start and goal configuration, illustrated in Figure \ref{fig:pbrm_intro}. This convex corridor allows us to use convex optimization to produce smooth trajectories, instead of computationally expensive post-processing methods. The representation further allows us to estimate the coverage of the collision-free space, which gives us awareness and feedback in the offline roadmap construction phase. Finally, our representation is simple to adapt to moving obstacles, simply requery for the new radii and recheck for intersections. 
\\\\
The spherical collision-free regions are formed using a signed distance function (SDF), which is a function that returns the smallest distance from an arbitrary point to the boundary of an obstacle. As the name implies, the distance is signed, thus if the point is inside the obstacle it is negative otherwise positive. If the distance is positive, a sphere with radius equal to the distance is guaranteed to cover a collision-free region. Using an SDF in motion planning is not new, but what is novel about our approach is that we express the distance in the configuration space instead of the world space and by doing so allows us to form these convex collision-free regions. We refer to the resulting SDF as a signed configuration distance function (SCDF). Computing an SCDF analytically is non-trivial, our approach is therefore to parameterize the SCDF with a deep neural network and learn the mapping by supervised learning. Our resulting neural SCDF can compute distances for different parameter values of obstacle shapes and we also show how multiple distances can be combined, thus making our approach flexible.
\section{Related work}
Motion planning algorithms can roughly be divided into three families, grid-based, sampling based and optimization based methods. Grid-based methods (GBM) discretize the planning space from which a graph is then compiled. A standard search method is A$^\star$ \citep{a_star}, which is classified as an \textit{informed} search method, since it employs a heuristic function to speed up the search. A$^\star$ guarantees to return an optimal path at the level of discretization used. GBMs usually discretize the planning space by a regular lattice and this limits the GBMs to problems with low dimensionality due to the curse of dimensionality. Thus, GBMs are usually limited to single-body robots where the degrees of freedom (DOF) are low. To overcome the inherent scaling problem with the GBMs, stochastic methods are usually used for multi-body robots. These methods are termed as sampling-based methods (SBM) and core members within this family are the rapidly-exploring random trees (RRT) \citep{rrt} and the probabilistic roadmap (PRM) \citep{prm}. RRT grows a tree from the start configuration and explores the collision-free region in a rapid way until it is able to connect to the goal region. RRT is usually improved by bi-directional planning \citep{rrt_connect}, i.e. an additional tree is grown from the goal configuration and the trees are tested for connection after any tree has been expanded. RRT is a single-query method, thus it searches for a path from scratch each time it is queried. Contrary to this, PRM is a multi-query method, which solves for multiple queries without starting from scratch. PRM does this by creating a roadmap (graph) that covers the collision-free space as an offline step. The graph is then used to solve for multiple queries. PRMs are used in cases where the environment does not change since the extra offline step is too computationally costly and needs to be re-done if the environment is changed. In our work, we address this inherent issue by using a different roadmap representation. Our vertices in the graph cover a collision-free region in the form of spheres and we form the edges by checking for intersecting spheres. If something in the environment changes, we recompute the spheres radii and recheck the intersections, without relying on collision detection. We use a trained neural network to compute the sphere radius, therefore querying for the radius can be done fast, hence our representation enables the PRM for dynamic environments.
\\\\
In the recent decades, optimization based methods (OBM) \citep{chomp, schulman, itomp, stomp} have been introduced as an alternative to SBM for multi-body robots. Like the SBM, the OBMs scale well to higher dimensional problems and produce smoother motion. It is common to use a SDF in the optimization since it is a smooth function, thus enabling gradient-based methods. However, the standard way of expressing the SDF is in world space. The distance therefore needs to be mapped to the configuration space by the forward kinematics. This mapping makes the optimization problem a non-linear program (NLP), which is computationally expensive to solve. Recently, a different approach has been proposed. In \cite{mp_gcs} motion planning is formulated as a convex optimization problem by using the graph of convex sets framework \citep{gcs}. The underlying idea is to decompose the collision-free space into intersecting convex sets from which a convex optimization problem is formulated. In cases where an explicit representation of the obstacles in the configuration space exists, like for single-body robots, creating collision-free convex regions can be done fast \citep{iris}. For multi-body robots, this is non-trivial. Existing work does this successfully \citep{iris_nlp, iris_c} by an optimization based approach, but the methods are still too time consuming to be used in the presence of moving obstacles. Our approach is instead to use deep learning to learn an SDF expressed in the configuration space. With this, we can query for shortest distances to the collision boundary, which allows us to expand spherical regions which are collision-free. Our approach is fast and therefore enables our suggested roadmap planner to be used in dynamic environments.
\\\\
Recent research has focused on learning collision detection \citep{fk_kernel_distance, diffco, graphdistnet} by predicting the signed distance between the robot links and the surrounding obstacles in the world space. The learned SDF is used in trajectory optimization but since the distance is expressed in the world space, the problem becomes an NLP and therefore takes a long time to solve. We take a novel approach and suggest to instead express the signed distance in the configuration space. This allows us to improve the PRM at the same time as it enables convex optimization for trajectory optimization, which runs faster and is more reliable than NLP solvers. In \cite{cspf} a learned signed distance function in the configuration space is proposed similar to our approach. However, their approach is restricted to point cloud representations, while we propose to represent the obstacles as parameterized geometric shapes, e.g. spheres. Furthermore, we also show how to use our learned SCDF to improve an existing roadmap planner.
\section{Problem formulation}
A robot is located in the world space, $\W \subset \R^3 $. The unique location of the robot is given by its configuration $\q \in \C$, where $\C$ is the configuration space. The set of points covered by the robots bodies at a certain configuration is expressed as $\B(\q) \subset \W$. The robot is surrounded by $\NrObst$ obstacles $\O = \bigcup_{i=1}^{\NrObst} \O_i$, where  $\O_i \subset \W$. The representation of the obstacle in the configuration space is the set $\C\O_i = \{\q \in \C \: |\: \B(\q) \cap \O_i \neq \emptyset \}$. The obstacle space is formed as $\Co = \bigcup_{i=1}^{\NrObst} \C \O_i$. The complement is referred to as the free space, $\Cf = \C \setminus \Co$. The path planning problem is a tuple, ($\Cf$, $\qStart$, $\qGoal$), where we want to connect a query pair, consisting of a start, $\qStart$, and goal configuration, $\qGoal$, with a geometric path, $\q(s): [0, 1] \mapsto \Cf$, such that $\q(0)=\qStart$ and $\q(1)=\qGoal$, or report correctly when such a path does not exist.
\end{document}


\section{Related Work} \label{sec:related}

% \textbf{Adversarial Attack}
\textbf{Attacks on SLAM.} 
%With the rise of machine learning, 
The robustness of computer vision systems is being actively investigated. With the emergence of adversarial images in the digital domain by adding optimized noise directly to images~\cite{szegedy2013intriguing,carlini2017towards}, researchers find that such attacks also exist physically in the real world \cite{eykholt2018robust,song2018physical,zhao2019seeing}. To fill the gap between attacks in the digital and physical worlds, recent studies have demonstrated that attacks on real-world computer vision systems are practical \cite{eykholt2018robust,li2019adversarial,man2020ghostimage,sharif2016accessorize,zhao2019seeing,zhou2018invisible}. However, attacks on traditional computer vision methods such as SLAM are relatively less explored. \cite{yoshida2022adversarial} proposes an attack against the scan matching algorithm in LiDAR-based SLAM, while most SLAMs in AR/VR devices rely on different sensors like RGB/depth cameras and IMUs. \cite{ikram2022perceptual} and \cite{chen2024adversary} mislead visual SLAM by poisoning the images with special patterns, and \cite{wang2021can} causes the camera to fail using infrared light. In our work, we demonstrate attacks on Visual-Inertial SLAM (VI-SLAM) by perturbing the IMU readings, rather than cameras, and showing its impact on XR user experience. 

\textbf{Acoustic Injection Attacks.} Among various physical attacks, acoustic injection attacks are attractive due to their low cost. Son~\etal~\cite{son2015rocking} were the first to introduce acoustic attacks on MEMS gyroscopes, demonstrating how these attacks could lead to sensor denial-of-service and result in drone crashes. WALNUT~\cite{trippel2017walnut} expanded on this by developing output biasing and control attacks that enable precise manipulation of MEMS accelerometer outputs using modulated sound waves. Wang et al.~\cite{wang2017sonic} demonstrated a sonic gun, showcasing the vulnerability of various smart devices (\eg drones and self-balancing vehicles) to acoustic attacks. Tu et al. \cite{tu2018injected} designed side-swing and switching attacks to alter the outputs of MEMS gyroscopes and accelerometers. Furthermore, Ji et al. \cite{ji2021poltergeist} fool the object detectors by applying acoustic attack to the image stabilizers commonly used in modern cameras. However, none of the existing works study the relationship between the acoustic injections and SLAM outputs on recent XR devices. 

% \zijian{Do we need one session about security in AR/VR?}
% \yicheng{TODO}
%\jiasi{cite the AIVR paper (UMass Amherst?) paper is we have not already. They add IMU perturbation but w/o SLAM, iirc} \yicheng{Cited}

\textbf{XR Security and Privacy.} 
%Security and privacy concerns in XR systems have gained significant attention. 
For single-user XR systems, researchers have demonstrated various side-channel attacks to extract sensitive information (\eg keystrokes) through video feeds~\cite{ling2019know}, head movements~\cite{nair2023unique, slocum2023going}, architectural hints~\cite{zhang2023its,shang2020arspy}, power usage~\cite{li2024dangers}, and EM side-channel leakages~\cite{al2021vr}. In multi-user XR systems, Su et al.~\cite{su2024remote} use avatar motion data to infer keystrokes in shared VR environments. Slocum et al.~\cite{slocum2024doesn} reveal vulnerabilities in the shared state frameworks of multi-user AR. Similarly, Lebeck et al.~\cite{lebeck2017securing} highlight risks like deceptive virtual objects and emphasize access control for managing shared physical and virtual spaces. Ruth et al.~\cite{ruth2019secure} further propose a secure multi-user AR framework focusing on content sharing and permissions.
Chandio et al.~\cite{chandio2024stealthy} %introduced a multi-modal spatiotemporal attack that 
simultaneously manipulated visual and inertial sensors to disrupt XR pose estimation. However, their study evaluated the attack using offline datasets and assumed the attacker's capability to manipulate IMU data streams through acoustic means, without real experiments. Ours is the first to demonstrate acoustic injection attacks on recent XR devices, like the Hololens 2, in the real world.
 


% \begin{figure}
%     \centering
%     \includegraphics[width=0.5\linewidth]{Move_teaser.pdf}
%     \caption{Comparison of different dynamic compute approaches. length of arrow indicates residual transformation per token while width indicates velocity of transformation.}
%     \label{fig:enter-label}
% \end{figure}

\section{Method}
\label{sec:method}
Residual connections play a crucial role in shaping token representations, yet their dynamics remain underexplored in the context of efficient decoding. In this work, we delve deeper into transformer residual dynamics and investigate how modulating residual transformation velocity can improve inference efficiency in token-level processing, optimizing both dense and sparse MoE transformers.


\subsection{Residual Dynamics and Motivation for Multi-rate Residuals} \label{sec:motivation}

To analyze how hidden representations evolve across different layers of a transformer architecture, it's crucial to consider the effect of residual connections. Each transformer decoder layer typically has residual connections across attention and MLP submodules. As the residual stream $h_i$ traverses from interval $E_j$ to $E_{j+1}$, it undergoes a residual transformation given by:  
% \begin{equation}
% \label{eq:slow_residual_transformation}
% H_{E_{j+1}} = H_{E_j} \prod_{i=E_j}^{E_{j+1}} \left( I + \mathcal{A}_i \right) \left( I + \mathcal{M}_i \right) \quad \text{where} \quad \mathcal{A}_i = f(c_i, h_{i}), \mathcal{M}_i = g(h_i)
% \end{equation}

\begin{equation} \label{eq:slow_residual_transformation}
h_{E_{j+1}} = h_{E_j} + \sum_{i=E_j}^{E_{j+1}-1} \left( \mathcal{A}_i(h_i) + \mathcal{M}_i(h_i + \mathcal{A}_i(h_i)) \right) \quad \text{where} \quad \mathcal{A}_i = f(c_i, h_{i}), \mathcal{M}_i = g(h_i). 
\end{equation}

Here, \( \mathcal{A}_i \) denotes the non-linear transformation introduced by the multi-head attention mechanism at layer \( i \), while \( \mathcal{M}_i \) corresponds to the non-linear transformation of the MLP block at the same layer. These transformations depend on the input residual stream \( h_i \) and, in the case of \( \mathcal{A}_i \), the previous contextual representation \( c_i \).\footnote{Normalization layers are typically applied in practice but are omitted here for simplicity of the argument.}


% For easy tokens, the magnitude and direction of this delta transformation become progressively smaller with each successive layer as shown in \cref{fig:delta_transformation}. Consequently, it is feasible to predict these tokens after only a few residual connections, whereas harder tokens necessitate more extensive processing through additional layers.

\begin{figure}[ht]
    \centering
    \begin{subfigure}{0.48\textwidth}
        \centering
        \includegraphics[width=\textwidth]{sections/figures/residual_change.pdf}
        \caption{}
        \label{fig:residual_change}
    \end{subfigure}%
    \hfill
    \begin{subfigure}{0.48\textwidth}
        \centering
        \includegraphics[width=\textwidth]{sections/figures/alignment_wrt_dedicated_model.pdf}
        \caption{}
    \label{fig:alignment_wrt_dedicated_model}
    \end{subfigure}
    \caption{(a) As residual streams propagate through the model, the directional shifts in the residuals become progressively smaller. (b) A dedicated model with $k$ layers achieves a faster rate of change in residual streams and higher alignment than base model leveraging early exit mechanisms at layer $k$.}
    \label{fig}
\end{figure}


To examine whether residual transformations can be accelerated across layers, we conducted experiments using a diverse set of prompts on a pre-trained Phi3 model~\cite{phi3_report}. As illustrated in \cref{fig:residual_change}, we measured the directional shift in residual states as \( 1 - \mathcal{C}(h_{i-1}, h_i) \), where \(\mathcal{C}\) denotes normalized cosine similarity. This shift is notably higher in the initial layers, gradually decreasing in subsequent layers. This behavior allows traditional early exit approaches to effectively accelerate decoding by enabling earlier exits for simpler tokens. However, these approaches typically rely on a distance-based approximation, where the full residual transformation of the model is approximated by the residual transformations of the initial layers. To gain deeper insights into the distance versus velocity aspects of residual transformation, we conducted a comparative study. Specifically, we trained an early exit head at layer $k$ of the Phi3 model, which consists of 32 layers, restricting the distance traveled by each token. To accelerate the residual transformation relative to number of layers, we trained a smaller model consisting of only $k$ layers, while keeping all other hyperparameters consistent. We then compared the next-token prediction accuracy of the early exit head of the base model with that of the smaller model. To ensure an equal number of trainable parameters, we inserted low-rank adapters into the smaller model and trained only these adapters, whereas, in the distance-based approach, we trained solely the early exit head. In addition, to accelerate the residual transformation in smaller model, we distilled the residual streams from the larger model by incorporating a distillation loss ~\cite{sanh2019distilbert} between the residual state at layer \(i\) of the smaller model and the residual state at layer \(4 \times i\) of the larger model. As shown in ~\cref{fig:alignment_wrt_dedicated_model} the smaller model demonstrates a significantly faster rate of change in residual streams, leading to higher next token prediction accuracy after $k$ layers compared to the base model that employs traditional early exit mechanisms after $k$ layers \cite{schuster2022confident, chen2023eellm, varshney-etal-2024-investigating}. This experimental setup, which modifies only the rate of change in residual streams while keeping other factors constant, suggests that dense transformers, trained with a fixed number of layers, may inherently possess a slow residual transformation bias.

This observation raises an intriguing question: if the rate of change in residual streams could be accelerated relative to the number of layers, is it possible to facilitate earlier alignment for a greater proportion of tokens? Earlier alignment would be beneficial to not only facilitate dynamic computation but also for generating speculative tokens efficiently with high acceptance rates in speculative decoding setups ~\cite{leviathan2023fast, chen2023accelerating}. 

%thereby enhancing the efficiency of early exiting? 
 % This bias likely constrains the effectiveness of early exiting, particularly for easier tokens. By addressing this limitation through accelerated residual transformations, we hypothesize that it is possible to substantially improve the efficiency and accuracy of early exit strategies in transformer models.

\subsection{Multi-Rate Residual Transformation} \label{m2r2_method}

To address the slow residual transformation bias described in ~\cref{sec:motivation}, we introduce \textit{accelerated residual streams} that operate at rate $R$ relative to original slow residual stream. We pair slow residual stream, $h$ with an accelerated residual stream, $p$, which has an intrinsic bias towards earlier alignment. Relative to ~\cref{eq:slow_residual_transformation}, accelerated residual transformation from interval $E_j$ to $E_{j+1}$ can be represented as: 

% \begin{equation}
% \label{eq:fast_residual_transformation}
% P_{E_{j+1}} = P_{E_j} \prod_{i=E_j}^{E_{j+1}} \left( I + \hat{\mathcal{A}_i} \right) \left( I + \hat{\mathcal{M}_i} \right) \quad \text{where} \quad \hat{\mathcal{A}_i} = \hat{f}(c_i, P_{i}), \hat{\mathcal{M}_i} = \hat{g}(P_{i})
% \end{equation}


\begin{equation} \label{eq:fast_residual_transformation}
p_{E_{j+1}} = p_{E_j} + \sum_{i=E_j}^{E_{j+1}-1} \left( \hat{\mathcal{A}_i}(p_i) + \hat{\mathcal{M}_i}(p_i + \hat{\mathcal{A}_i}(p_i)) \right) \quad \text{where} \quad \hat{\mathcal{A}_i} = \hat{f}(c_i, p_{i}), \hat{\mathcal{M}_i} = \hat{g}(h_i), 
\end{equation}



where $\hat{\mathcal{A}_i}$ and $\hat{\mathcal{M}_i}$ denote non-linear transformation added by layer $i$ to previous accelerated residual $p_{i}$. Similar to $\mathcal{A}_i$, non-linear transformation $\hat{\mathcal{A}_i}$ attends to same context $c_i$ but uses a different transformation $\hat{f}$ for accelerating $p_{E_j}$ relative to $h_{E_j}$. 

We integrate accelerated residual transformation directly into the base network using parallel accelerator adapters such that rank of accelerator adapters $R_p << d$ where $d$ denotes base model hidden dimension. This setup allows the slow residual stream $h_{E_j}$ to pass through the base model layers while the accelerated residual stream $p_{E_j}$ utilizes these parallel adapters as shown in ~\cref{fig:m2r2_main}. Both slow and accelerated residuals are processed in same forward pass via attention masking and incur negligible additional inference latency in memory bound decoding setups, while in compute bound decoding setups where FLOPs optimization is essential, accelerated residual stream utilizes a fraction of attention heads that of slow residual (see ~\cref{sec:flops_optimization}). Additionally, to maximize the utility of accelerated residual transformations without introducing dedicated KV caches, we propose a shared caching mechanism between the slow and accelerated streams which minimally impact alignment benefits of our approach while offering substantial memory savings (see ~\cref{fig:koala_alignment}). Specifically, the attention operation on the slow residuals \( \text{MHA}(h_t, h_{\leq t}, h_{\leq t}) \) is redefined for accelerated residuals as 
\[
\hat{\mathcal{A}} = MHA(p_t, h_{<t} \oplus p_t, h_{<t} \oplus p_t),
\]
where the accelerated residual at time-step $t$, \( p_t \) attends to the slow residual’s KV cache, facilitating the reuse of contextual information across both residual streams without incurring additional caching costs. Here, \(MHA(q, k, v) \) represents multi-head attention between query \( q \), key \( k \), and value \( v \).

\begin{figure}
    \centering
    \includegraphics[width=0.8\linewidth]{sections//figures/m2r2_main2.pdf}
    \caption{Multi-rate Residuals Framework: Slow residual stream of base model is accompanied by a faster stream that operates at a $2-(J+1)\times$ rate relative to the slow stream, undergoing transformations via accelerator adapters as detailed in \cref{m2r2_method}, where J denotes number of early exit intervals. Colors within the slow and fast residual streams indicate similarity, with matching colors representing the most closely aligned residual states. At the beginning of the forward pass and at each exit point, the accelerated residual state is initialized from the corresponding slow residual state to avoid gradient conflict during training (see ~\cref{sec:grad_conflict}). Early exiting decisions are informed by the Accelerated Residual Latent Attention (ARLA) mechanism, described in \cref{method_arla}, which evaluates residual dynamics across consecutive exit gates.}
    \label{fig:m2r2_main}
\end{figure}

% Furthermore. to maximize the benefits of fast residual transformations without using dedicated KV caches, we propose sharing the fast network’s cache with the slow network. Formally speaking, We modify attention operation on slow residuals $MHA(H_t, H_{<=t}, H_{<=t})$ as $MHA(P_{t}, H_{<t} \oplus P_t, H_{<t}  \oplus P_t)$ such that accelerated residuals attend to previous slow context KV cache, where $MHA(q,k,v)$ denotes multi head attention between query, $q$, key $k$ and value $v$.


\subsection{Enhanced Early Residual Alignment}
Early residual alignment is instrumental in optimizing early exiting, speculative decoding, and Mixture-of-Experts (MoE) inference mechanisms. In this section, we provide a detailed analysis of how accelerated residuals enhance these inference setups.

% By aligning the residual states of intermediate layers with the final output representations, the model can maintain high prediction accuracy even when computations are truncated at earlier layers. This enables more reliable early exiting, reducing the overall computational cost while preserving performance. Additionally, in speculative decoding, early residual alignment allows the model to make confident predictions using faster, partial computations, thereby accelerating inference without sacrificing output quality.


\subsubsection{Early Exiting} \label{method_early_exiting}

A prevalent strategy for enabling early exiting at an intermediate layer $E_{j}$ involves approximating the residual transformation between $E_{j}$ and the final layer $N-1$ using a linear, context independent mapping, $\mathcal{T}$, such that $H_{N-1} \approx \mathcal{T}(H_{E_{j}})$. This approximation has been extensively employed in conventional approaches ~\cite{schuster2022confident, chen2023eellm, varshney-etal-2024-investigating}, providing a computationally efficient means to project the output of deeper layers from intermediate states. Specifically, residual state of layer $N-1$ with this approximation can be expressed as:


% \begin{equation}
% \label{eq: vanila_ea_assumption}
% \Phi(H_{E_{j}}) \sim H_{E_{j}} \prod_{i=E_{j}}^{N}\left( I + \mathcal{A}_i \right) \left( I + \mathcal{M}_i \right) \quad \text{where} \quad \Phi \perp C
% \end{equation}

\begin{equation} \label{eq:early_exiting}
h_{E_j} + \sum_{i=E_j}^{N-1} \left( \mathcal{A}_i(h_i) + \mathcal{M}_i(h_i + \mathcal{A}_i(h_i)) \right) \sim \mathcal{T}(h_{E_{j}})  \quad \text{where} \quad \mathcal{T} \perp c. 
\end{equation}


Here, $\mathcal{A}_i$ and $\mathcal{M}_i$ represent the residual contributions of the multi-head attention and MLP layers, respectively, while $\mathcal{T}$ remains independent of $c$, the preceding context.

This approach is inherently limited by two major factors: first, the assumption of linearity between $h_{E_{j}}$ and $h_{N-1}$ may not hold uniformly for all tokens, particularly when $E_j \ll N$. Second, the linear transformation $\mathcal{T}$ disregards the influence of the context $c$ and fails to account for the latent representations of previous contextual states. In contrast, M2R2 accelerated residual states mitigate both of these challenges by approximating the slow residual transformation of all layers via a faster residual transformation of fewer layers as:
% \begin{equation}
% H_{E_j} \prod_{i=E_j}^{N}\left( I + \mathcal{A}_i \right) \left( I + \mathcal{M}_i \right) \sim P_{E_j} \prod_{i=E_j}^{E_j+1}\left( I + \hat{\mathcal{A}_i} \right) \left( I + \hat{\mathcal{M}_i} \right)
% \end{equation}


\begin{equation} \label{eq:m2r2_approximating_ea}
h_{E_j} + \sum_{i=E_j}^{N-1} \left( \mathcal{A}_i(h_i) + \mathcal{M}_i(h_i + \mathcal{A}_i(h_i)) \right) \sim p_{E_j} + \sum_{i=E_j}^{E_{j+1}-1} \left( \hat{\mathcal{A}_i}(p_i) + \hat{\mathcal{M}_i}(p_i + \hat{\mathcal{A}_i}(p_i)) \right), 
\end{equation}

% \begin{equation} \label{eq:fast_residual_transformation}
% p_{E_{j+1}} = p_{E_j} + \sum_{i=E_j}^{E_{j+1}-1} \left( \hat{\mathcal{A}_i}(p_i) + \hat{\mathcal{M}_i}(p_i + \hat{\mathcal{A}_i}(p_i)) \right) \quad \text{where} \quad \hat{\mathcal{A}_i} = \hat{f}(c_i, p_{i}), \hat{\mathcal{M}_i} = \hat{g}(h_i) 
% \end{equation}






where $p_{E_j}$ is initialized from the slow residual state $h_{E_j}$ at each early exit interval $E_j$ using an identity transformation (see ~\cref{fig:m2r2_main}). As shown in ~\cref{fig:m2r2_residual_sim}, accelerated residuals offer a smoother, more consistent shift in residual direction across layers, in contrast to the abrupt changes typically seen at early exit points in standard early exit methods. Moreover, the normalized cosine similarity between accelerated states at early exit intervals and final residual states is substantially higher compared to traditional early exit techniques, highlighting improved alignment with final layer representations. Traditional adaptive compute methods are constrained by two principal factors: the number of tokens eligible for early exit at intermediate layers and the precision of early exit decision. If residual streams fail to saturate early, the majority of tokens remain ineligible for exit, thereby diminishing potential speedups. Additionally, imprecise delineations between tokens suitable for early exit can lead to underthinking (premature exits that adversely affect accuracy) or overthinking (unnecessary processing that compromises efficiency) ~\cite{zhou2020self, dai2020dynamic}. Enhanced early alignment using ~\cref{eq:m2r2_approximating_ea} helps to address  first issue. To address the second issue we introduce Accelerated Residual Latent Attention, which dynamically assesses the saturation of the residual stream, allowing for a more precise differentiation between tokens that can exit early and those requiring further processing.

% This results in uniform change in residual direction    
% % We keep $\mathcal{A} = \hat{\mathcal{A}}$, while $\hat{\mathcal{M}}$ is accelerated by a factor of $2 - (N_{E}+1)X$ relative to the slower residual transformation $\mathcal{M}$, where $N_E$ represents number of early exiting intervals.
% Figure~\cref{fig:rate_change_comparison} illustrates the comparative rate of change between these transformation streams.



% fig:rate_change_comparison
% - grid plot x axis -> layer id (0, 8) , y axis -> layer id -> dark color cell for max similarity , lighter for lower 
% 
-------------------------------------------------------
Let's consider residual stream $h_i$ traverses through interval $E_j$ to $E_{j+1}$ and undergoes residual transformation given by 
\begin{equation}
h_{E_{j+1}} = h_{E_j} \prod_{i=E_j}^{E_{j+1}} \left( 1 + \delta_i \right)    
\end{equation}

where $\delta_i$ denotes non-linear transformation added by layer $i$. Each non-linear transformation of layer $i$ is a function of previous contextual representation, $c_i$ and input residual stream $h_i-1$ as
$\delta_i = f(c_i, h_{i-1})$ 

One way to exit early at exit $E_j+1$ is to assume that residual transformation from $E_j+1$ to final layer $N-1$ can be approximated by a linear function $\phi$ as $h_{N-1} \sim \Phi(h_{E_j+1})$ and most conventional approaches such as \todo{cite EA papers} use this approach. In other words, 

\begin{equation}
\Phi(h_{E_j+1} \sim h_{E_j+1} \prod_{i=E_j+1}^{N} \left( 1 + \delta_i \right)   
\end{equation}

This approach suffers from two primary issues, linearity assumption from $h_E_j+1$ to $H_N-1$ if often incorrect, particularly when $E_j << N$. More importantly, linear transformation $\Phi$ doesn't consider effect of context $C_i$. M2R2  effectively addresses these issues as accelerated residual stream at interval $E_j+1$ can be represented as 

\begin{equation}
r_{E_{j+1}} = r_{E_j} \prod_{i=E_j}^{E_{j+1}} \left( 1 + \gamma_i \right)    
\end{equation}

where $\gamma_i$ denotes non-linear transformation added by layer $i$ to previous accelerated residual $r_i-1$. Similar to $\delta_i$, non-linear transformation $\gamma_i$ considers context $C_i$ as 
$\gamma_i = g(c_i, r_{i-1})$. So in summary, slow residual transformation is approximated by accelerated residual as: 

\begin{equation}
h_{E_j} \prod_{i=E_j}^{N} \left( 1 + \delta_i \right) \sim h_{E_j} \prod_{i=E_j}^{E_j+1} \left( 1 + \gamma_i \right)
\end{equation}

It's worth noting that accelerated residual $r_i$ and slow residual $h_i$ are processed concurrently at layer $i$ by constructing proper attention mask such as attention of slow residual is represented as 

$MHA(H_it, H_{i<=t}, H_{i<=t}$ while attention of fast residual is computed as 

$MHA(r_it, H_{i<=t}, H_{i<=t}$ where $MHA(q,k,v$ denotes multi head attention between query, $q$, key $k$ and value $v$.


------------------------------------------------------------------

Vertical latent attention on accelerated residual is computed as 
$MHA(S_mt, S(Ej<=i<=m)t, S(Ej<=i<=m)t)$ where $Smt$ denotes query/key/value projection in latent domain at layer $m$ at time $t$. 
------------------------------------------------------------------

Gradient conflict Avoidance: 

Let's consider $w_j$ is a trainable parameter that belongs to a layer between $E_j$ and $E_j+1$. Consider early exit loss at gate $E_j+1$, $L_j+1$, gradient propagation of $w_j$ at another trainable parameter $w_j-n$ can be gives as 

$\sum_{k=E_j-n}^{E_j} \beta_k \frac{\partial L_{E_k}}{\partial w_k}$

where $\beta_j$ denotes backward transformation coefficient for weight $w_j$ to reach gate $E_j$. 
 
On the other hand, gradient propagation in proposed approach can be represented as 

\[
\frac{\partial L_{E_j}}{\partial w_j} = 
\begin{cases} 
\beta_j \frac{\partial L_{E_j}}{\partial w_j} & \text{if } E_j \leq w_j \leq E_{j+1} \\
0 & \text{otherwise}
\end{cases}
\]







% \begin{figure}[ht]
%     \centering
%     \includegraphics[width=0.8\textwidth, height=5cm]{rate_change_comparison.png}
%     \caption{Rate of change comparison between fast and slow residual streams.}
%     \label{fig:rate_change_comparison}
% \end{figure}

%vary k and and plot EA accuracy for larger and smaller models. 

% \begin{figure}[ht]
%     \centering
%     \includegraphics[width=0.5\textwidth,height=5cm]{sections/figures/alignment_comparison_dialogsum.pdf}
%     \caption{Alignment of exited tokens for different early exit layers using traditional early exiting heads, dedicated faster networks, and faster residuals.}
%     \label{fig:small_model_early_exiting}
% \end{figure}


\textbf{Accelerated Residual Latent Attention} \label{method_arla}

In the context of residual streams, we observe that the decision to exit at a given layer can be more effectively informed by analyzing the dynamics of residual stream transformations, instead of solely relying on a classification head applied at the early exit interval $E_j$. To capture the subtle dynamics of residual acceleration, we propose a \textit{Accelerated Residual Latent Attention} (ARLA) mechanism. This approach involves making the exit decision at gate $E_j$ by attending to the residuals spanning from gate $E_{j-1}$ to $E_j$, rather than considering only the residual at gate $E_j$. To minimize the computational overhead associated with exit decision-making, the attention mechanism operates within the latent domain as depicted in ~\cref{fig:arla_arch}. Formally, for each interval $[E_j, E_{j+1}]$, the accelerated residuals are projected into Query ($Q^s_{E_j}, \ldots, Q^s_{E_{j+1}}$), Key ($K^s_{E_j}, \ldots, K^s_{E_{j+1}}$), and Value ($V^s_{E_j}, \ldots, V^s_{E_{j+1}}$) vectors, with latent dimension $d^s$ for $Q^s$, $K^s$, and $V^s$ being significantly smaller than hidden dimension of $p$.\footnote{We use $d^s = 64$ for experiments described in ~\cref{sec:experiments}.} Notably, when the router is allowed to make exit decisions at gate $E_j$ based on residual change dynamics, we observe that the attention is not confined to the residual state at $E_j$ but is distributed across residual states from $E_{j-1}$ to $E_j$, %as illustrated in Figure~\ref{fig:vertical_latent_attention_dynamics}. 
This broader focus on residual dynamics significantly reduces decision ambiguity in early exits, as demonstrated in Figure~\ref{fig:roc_arla}, which contrasts routers based on the last hidden state, and the proposed ARLA router.

%show R -> S transformation. 
%show parameter and flop overhead as compared to adapter on last hidden state.

% \begin{figure}[ht]
%     \centering
%     \includegraphics[width=0.5\textwidth,height=5cm]{sections/figures/roc_arla.pdf}
%     \caption{ROC curves of early exit decision strategies: confidence-based methods (CALM/LITE), routers based on the accelerated hidden state, and latent attention routers.}
%     \label{fig:decision_making_comparison}
% \end{figure}

% \begin{figure}[ht]
%     \centering
%     \includegraphics[width=0.5\textwidth,height=5cm]{vertical_latent_attention.png}
%     \caption{Vertical latent attention mechanism for optimizing early exit decisions by considering residuals from gate \(M\) through \(M-1\).}
%     \label{fig:vertical_latent_attention}
% \end{figure}

\begin{figure}[ht]
    \centering
    \begin{subfigure}{0.52\textwidth}
        \centering
        \includegraphics[width=\textwidth, height = 4cm]{sections/figures/arla_arch.pdf}
        \caption{Accelerated Residual Latent Attention (ARLA): Accelerated residuals between early exit gates are projected into latent domain and attention over residual states within the interval is computed to capture residual dynamics and exit decision is made based on residual saturation.}
        \label{fig:arla_arch}
    \end{subfigure}%
    \hfill
    \begin{subfigure}{0.45\textwidth}
        \centering
        \includegraphics[width=\textwidth, height = 4.5cm]{sections/figures/vla_roc.pdf}
        \caption{ROC classification curves of early exit decision strategies using a linear router used on last residual state ~\cite{schuster2022confident, varshney-etal-2024-investigating, chen2023eellm}  and using ARLA approach that considers residual dynamics. }
        \label{fig:roc_arla}
    \end{subfigure}
    \caption{Effectiveness of ARLA in capturing residual dynamics for early exiting decisions.}


\end{figure}



% \begin{figure}[ht]
%     \centering
%     \includegraphics[width=1\textwidth,height=5cm]{sections/figures/arla.pdf}
%     \caption{fig that plots 32 rows 2 cols heatmap showing attention at each gate}
%     \label{fig:vertical_latent_attention_dynamics}
% \end{figure}

\subsubsection{Self Speculative Decoding} \label{method_self_speculative_decoding}

An alternative means to exploit the early alignment properties of our approach is through the use of accelerated residual states for speculative token sampling to accelerate autoregressive decoding. Speculative decoding aims to speed up memory-bound transformer inference by employing a lightweight draft model to predict candidate tokens, while verifying speculated tokens in parallel and advancing token generation by more than one token per full model invocation \cite{leviathan2023fast, chen2023accelerating, xia2023speculative, miao2023specinfer}. Despite its effectiveness in accelerating large language models (LLMs), speculative decoding introduces substantial complexity in both deployment and training. A separate draft model must be specifically trained and aligned with the target model for each application, which increases the training load and operational complexity ~\cite{chen2023accelerating}. Additionally, this approach is resource-inefficient, as it requires both the draft and target models to be simultaneously maintained in memory during inference \cite{leviathan2023fast, chen2023accelerating}. 

One strategy to address this inefficiency is to leverage the initial layers of the target model itself to generate speculative candidates, as depicted in ~\cite{Tang2024}. While this method reduces the autoregressive overhead associated with speculation, it suffers from suboptimal acceptance rates. This occurs because the linear transformation employed for translating hidden states from layer $k$ to the final layer $N$ is typically a poor approximation, as discussed in ~\cref{sec:motivation} and ~\cref{method_early_exiting}. Our approach resolves this limitation by utilizing accelerated residuals, which demonstrate higher fidelity to their slower counterparts. By utilizing accelerated residuals operating at a rate of $N/k$, where $k$ denotes the number of layers used for candidate speculation, we are able to efficiently generate speculative tokens for decoding.\footnote{We typically set $k = 4$ to balance the trade-off between autoregressive drafting overhead and acceptance rate, as discussed in~\cref{sec:experiments}.}
 This technique not only obviates the need for multiple models during inference but also improves the overall efficiency and effectiveness of speculative decoding.

\begin{figure}
    \centering    \includegraphics[width=1\linewidth]{sections/figures/m2r2_aot_loading.pdf}
    \caption{Ahead-of-Time Expert Loading: M2R2 accelerated residual stream predicts experts required for future layers, reducing reliance on on-demand lazy loading. Speculative pre-loading is efficiently overlapped with computation of multi-head attention (MHA) and MLP transformations. Only incorrectly speculated experts are loaded lazily, resulting in faster inference steps and improved computational efficiency. Here, H indicates LBM Host while D indicates HBM Device.}
    \label{fig:moe_expert_aot_loading}
\end{figure}


\subsubsection{Ahead of Time Expert Loading:} \label{method_aot_expert_loading}

Recent advancements in sparse Mixture-of-Experts (MoE) architectures ~\cite{shazeer2017outrageously, fedus2022switch, artetxe2019massively, lepikhin2020gshard, zoph2022designing} have introduced a paradigm shift in token generation by dynamically activating only a subset of experts per input, achieving superior efficiency in comparison to dense models, particularly under memory-bound constraints of autoregressive decoding \cite{fedus2022switch, zoph2022designing}. This sparse activation approach enables MoE-based language models to generate tokens more swiftly, leveraging the efficiency of selective expert usage and avoiding the overhead of full dense layer invocation. In dense transformer models, pre-loading layers is a common strategy to enhance throughput, as computations of current layer can be overlapped with pre-loading of next layer parameters ~\cite{narayanan2021efficient, shoeybi2020megatron}. However, MoE models face a unique challenge: expert selection occurs dynamically based on previous layer’s output, making it infeasible to preload next layer’s experts in parallel. This limitation results in inherent latency, as expert loading becomes a sequential, on-demand process ~\cite{lepikhin2020gshard, fedus2022switch}.

To address this inefficiency, our method introduces a mechanism with \textit{accelerated residuals}, which not only captures key characteristics of base slower residual states but also exhibit high cosine similarity with their final counterparts (as illustrated in \cref{fig:m2r2_residual_sim}). By employing accelerated residual streams, we can effectively predict the necessary experts for future layers well in advance of their actual invocation. Specifically, using a $2\times$ accelerated residual, the experts needed for layers $2i+2$ and $2i+3$ can be identified while still computing in layer $i$, thus overcoming the bottleneck of sequential, on-demand expert selection and mitigating latency in the decoding pipeline, as shown in \cref{fig:moe_expert_aot_loading}. Note that, we use fixed set of accelerator adapters for transforming accelerated residuals (as discussed in ~\cref{m2r2_method}) while slow residual is transformed via expert routing mechanism. 

Furthermore, our approach integrates a Least Recently Used (LRU) caching strategy, which enhances memory efficiency by replacing the least recently used experts with speculated experts that are anticipated to be needed in upcoming layers. This hybrid approach of preemptive expert loading with LRU caching yields substantial improvements over traditional on-demand loading or standalone caching strategies. By minimizing cache misses and efficiently managing memory, this approach addresses both compute and memory bottlenecks, leading to faster, more resource-efficient token generation in MoE architectures. A comprehensive evaluation of this strategy, in relation to state-of-the-art methods, is provided in \cref{experiments_aot}, and the compute and memory traces on an A100 GPU are detailed in \cref{fig:moe_aot_cuda_trace}.



% Recent advancements in sparse Mixture-of-Experts (MoE) architectures have introduced the concept of utilizing distinct computational paths for different tokens \cite{shazeer2017outrageously}. This approach, wherein only a subset of experts are activated per input, enables MoE-based language models to generate tokens more swiftly compared to their dense counterparts due to memory-bound nature of auto-regressive decoding. In dense models, pre-loading layers in advance is a common strategy to enhance computational efficiency. However, this technique is not applicable to MoE models, where expert selection occurs dynamically based on the outputs of previous layers, preventing parallel pre-fetching of experts.

% Our proposed method addresses this inefficiency. Accelerated residuals, which are highly similar to their slower counterparts (see \cref{fig:similarity}), can reliably predict the necessary experts ahead of time. For instance, by utilizing $2X$ accelerated residual stream, we can predict the experts needed for the layer $2i+1$ and $2i+3$ while carrying out computation in layer $i$. This enables us to commence expert loading significantly earlier, as illustrated in \cref{expert_loading}, effectively mitigating the delays observed with the naive on-demand expert loading. Additionally, our method benefits from incorporating a Least Recently Used (LRU) strategy, where speculated experts replace those that are least recently utilized, resulting in improved performance compared to using either strategy alone. For a comprehensive evaluation, refer to \cref{moe_trace}, which provides a CUDA compute and memory trace of our approach executed on <>.



% A naive solution involves using the residual state of the previous layer along with the gating function of the next layer to predict which experts need to be loaded, and initiating the expert loading process in parallel with the attention computation of the next layer. Yet, as shown in \cref{fig:MOE_attn_vs_loading_time}, the attention computation for medium to long contexts is considerably faster than the expert loading time, making this approach inefficient.




\subsection{Training} \label{method_training}
% This approach is feasible due to the absence of gradient conflicts, as discussed in \cref{sec:grad_conflict}.

To accelerate residual streams, we employ parallel accelerator adapters as described in \cref{m2r2_method}.  For the early exiting use-case outlined in \cref{method_early_exiting}, we define the training objective for these adapters using the following loss function, which combines cross-entropy loss at each exit $E_j$ with distillation loss at each layer $i$. Loss weights coefficients $\alpha_0$ and $\alpha_1$ are employed to balance contribution of corresponding losses.

\begin{align} \label{eq:mr_loss}
L_{\text{m2r2}} = \underbrace{-\alpha_0 \sum_{j=1}^{J} \sum_{t=1}^{T} \log p_{\theta} \left( \hat{y}_t^{E_j} \mid y_{<t}, x \right)}_{\text{cross-entropy loss}} 
+ \underbrace{\alpha_1\sum_{i=1}^{E_{J-1}} \sum_{t=1}^{T} \| \mathbf{p}_{t}^{i} - \mathbf{h}_{t}^{((i - E_{j(i)}) \cdot R_i) + E_{j(i)})} \|^2}_{\text{distillation loss}}.
\end{align}

where $\hat{y}_t^{E_j}$ denotes the predictions from the accelerated residual stream at layer $E_j$ and time step $t$, $y_t$ represents the corresponding ground truth tokens, and $x$ indicates previous context tokens. The distillation loss at each layer $i$ is computed by comparing accelerated residuals at layer $i$ with slow residuals at layer $(i - E_{j(i)}) \cdot R_i + E_{j(i)}$, where $R_i$ denotes the rate of accelerated residuals at layer $i$ while $E_{j(i)}$ represents the most recent gate layer index such that $E_{j(i)} <= i$. \( J \) represents the total number of early exit gates, N denotes number of hidden layers and $E_j$ denotes layer index corresponding to gate index $j$ and \( T \) denotes the sequence length. 

In dynamic compute settings, after training of accelerator adapters, we optimize the query, key, and value parameters governing the ARLA routers (see ~\cref{method_arla}) across all exits in parallel on binary cross entropy loss between predicted decision and ground truth exiting decision. The ground truth labels for the router are determined based on whether the application of the final logit head on $\hat{y}_t^{E_j}$ yields the correct next-token prediction. 


% The objective for this optimization is defined by the following loss function:


%TODO are equations required ? 
% \begin{equation} \label{eq:arla_loss_combined}\small
%     L_{\text{arla}} = -\frac{1}{N} \sum_{t=1}^{T} \left( \sum_{j=1}^{E_n} \left[ O_t^{E_j} \log(\hat{O}_t^{E_j}) + (1 - O_t^{E_j}) \log(1 - \hat{O}_t^{E_j}) \right] \right), \quad \text{where} \quad 
%     O_t^{E_j} = \begin{cases} 
%     1, & \text{if } L(\hat{y}_t^{E_j}) = y_t^{E_j} \\
%     0, & \text{otherwise}
%     \end{cases}
% \end{equation}

% where $\hat{O}_t^{E_j}$ represents the binary predicted logits produced by the vertical latent attention router, as described in \cref{sec:arla}, at gate $E_j$ and time step $t$, and $O_t^{E_j}$ denotes the corresponding ground truth labels. The ground truth labels for the router are determined based on whether the application of the logit head on $\hat{y}_t^{E_j}$ yields the correct next-token prediction. The parameters controlling vertical latent attention are trained concurrently to ensure consistency and efficient use of computational resources.

For self-speculative decoding, as described in \cref{method_self_speculative_decoding}, the training objective remains the same as \cref{eq:mr_loss}, but with the number of intervals set to $J = 1$ and the rate of residual transformation set to $R_n = N/k$, where the first $k$ layers generate speculative candidate tokens. In the context of Ahead-of-Time Expert Loading for Mixture-of-Experts (MoE) models (see \cref{method_aot_expert_loading}), setting the rate of residual transformation to $R_n = 2$ typically offers a good trade-off between the accuracy of expert speculation and AoT pre-loading of experts. 

% Thus, we set $J = 1$ and $E_1 = 16$.


~\subsection{FLOPs Optimization} \label{sec:flops_optimization}

Naively implemented, M2R2 incurs higher FLOP overhead compared to traditional speculative decoding and early exiting approaches such as ~\cite{medusa, schuster2022confident, Tang2024}. However, modern accelerators demonstrate compute bandwidth that exceeds memory access bandwidth by an order of magnitude or more~\cite{databricksLLMInference2023, jouppi2021ten}, meaning increased FLOPs do not necessarily translate to increased decoding latency. Nevertheless, to ensure fair comparison and efficiency in compute bound scenarios, we introduce targeted optimizations.

~\textbf{Attention FLOPs Optimization} For medium-to-long context lengths, attention computation dominates FLOPs in the self-attention layer, surpassing the contribution from MLP layers. Specifically, matrix multiplications involving queries, cached keys, and cached values scale with $l_{kv} * l_{q}$ where $l_{kv}$ denotes previous context length and $l_q$ denotes current query length. Since M2R2 pairs accelerated residuals with slow residuals, a naive implementation results in twice the FLOPs consumption compared to a standard attention layer. To address this, we limit the attention of accelerated residual stream to selectively attend to the top-k most relevant tokens, identified by the slow residual stream based on top attention coefficients\footnote{We set to k = 64 and attend to top 64 tokens as identified by the slow residual stream.}. This is possible since slow and accelerated residual streams are processed in same forward pass and accelerated streams have access to attention coefficients of slow stream. Note that, the faster residual stream still retains the flexibility to assign distinct attention coefficients to these tokens. Furthermore, we design the faster residual stream to employ only 8 attention heads, compared to the 32 heads used in the slow residual stream of the Phi-3 model, reducing query, key, value, and output projection FLOPs by a factor of 1/4. ~\cref{fig:m2r2_num_heads_ablation} indicates effect of using a slicker stream on alignment. As depicted, using $\hat{n}_h = 8$ offers a good trade-off between alignment and FLOPs overhead. 

~\textbf{MLP FLOPs Optimization} The accelerator adapters operating on the accelerated residual stream are intentionally designed with lower rank than their counterparts in the base model. This reduces FLOP overhead by a factor proportional to $hiddenSize / rank$. Additionally, since the faster residual stream uses only 8 attention heads (compared to 32 in the slow residual stream of Phi-3), the subsequent MLP layers process a smaller set of activations, further reducing FLOPs by another factor of 1/4.

These optimizations significantly reduce the FLOP overhead per speculative draft generation, as illustrated in ~\cref{fig:flops_optmization}. Notably, while traditional early-exiting speculative approaches such as DEED require propagating the full slow residual state through the initial layers, incurring substantial computational costs, M2R2 achieves efficient token generation via slimmer, low-rank faster residual streams. In contrast, Medusa introduces considerable FLOP overhead due to per-head computations scaling with $d^2+dv$\footnote{Here $d$ denotes hidden state dimension while $v$ denotes vocab size.}, whereas M2R2 employs low-rank layers for both MLP and language modeling heads, maintaining computational efficiency. All experiments involving the M2R2 approach, as detailed in ~\cref{sec:experiments}, are conducted using these FLOPs optimizations.









% \[
% O_t^{E_j} = 
% \begin{cases} 
% 1, & \text{if } L(\hat{y}_t^{E_j}) = y_t^{E_j} \\
% 0, & \text{otherwise}
% \end{cases}
% \]




%add distillation
% We train accelerator adapters described in \cref{m2r2_method} to accelerate residual streams on next token prediction all in parallel since there are no gradient conflict issues as described in \cref{sec:grad_conflict}.

% \begin{align} \label{eq:mr_loss}
% L_{mr} =  & -\sum_{j = 1}^{E_n} (\sum_{t=1}^{T}\log p_{\theta} (\hat{y}_t^{E_j} | \hat{y}_{<t}, x)) \nonumber
% \end{align}

% where $\hat{y_t^{E_j}}$ denotes predicted logits obtained from accelerated residual stream at gate $E_j$ and time-step $t$ while $y_t^{E_j}$ denotes corresponding truth tokens. 

% Upon training of adapters responsible for accelerating residual streams, we train query, key, value parameters responsible for vertical latent attention of all gates in parallel as

% \begin{equation} \label{eq:arla_loss}
%     L_{arla} = -\frac{1}{N} (\sum_{t=1}^{T}(1\sum_{j=1}^{E_n} \left[ O_t^{E_j} \log(\hat{O}_t^{E_j}) + (1 - o_t^{E_j}) \log(1 - \hat{o_t}_{E_j}) \right]))
% \end{equation}

% where $\hat{O_t^{E_j}}$ denotes binary predicted logits obtained from vertical latent attention router described in \cref{sec:arla} at gate $E_j$ and timestep $t$ while $O_t^{E_j}$ denotes corresponding truth label. Truth labels for router are obtained by computing whether logit head application on $\hat{y}_t^j$ results in true next token prediction. Formally speaking, 

% $O_t^{E_j} = 1 if L(\hat{y_t^{E_j}}) == y_t^{E_j} , 0 otherwise$. 

% Parameters responsible for vertical latent attention are also trained in parallel as well. 

%todo: training slow and fast residuals together and distillation can be two training mdoes. 
%Distillation can be an ablation. 




% Although transformer decoding is memory bound on most mainstream accelerators, there could be scenarios where flop savings are crucial. For instance, on on-device settings power consumption is directly correlated with flops per decoding step and reducing flops does help with overall energy consumption. Vanilla early exiting methods help with flop reduction but suffer from mismatch between training and inference due to early exited tokens. If token at decoding step $t$, $T_t$ exited at layer $E_i$, while token $T_{t+k}$ exits at layer $E_j$ such that $E_i < E_j$, hidden state $H_{t+k}l$ does not have corresponding hidden state $H_tl$ to attend to where $E_i < l <= E_j$. One solution that's often used in literature is to rely on last hidden state available, $H_t{E_j}$, however it tends to be sub-optimal and does affect generation quality \cite{ref}.  To alleviate this mismatch while reducing flops, we train router such that attention mask between token $T_{t+k}$ and token $T_{<t+k}$ is given by: 

% \begin{equation}
%     a_{T_{{t+k}{T_{<t+k}}} = 1 if  E_{T_{<t+k}} >= E{T_{t+k}}
%     else 0
% \end{equation}

% This attention mask enables router to account for exited tokens and get trained accordingly. Since attention mechanism during decoding remains exactly same as that during training, impact on generation quality tends to be minimal as noted in \cref{fig:gen_auality_with_and_without_recompute_attention_show_flops}.  Although MoD does not suffer from training and inference mismatch, we observe that it suffers from discountinuity between pre-training and super-vised fine-tuning resulting in sub-optimal perplexity. On the other hand, our method doesn't not require pre-training , doesn't suffer from discountinuity, and achieves much better perplexity in super-vised fine-tuning and instruction tuning setups as shown in \cref{fig:Mod_vs_m2r2_loss_curves}.






% Our techniques are directly applicable in such scenarios.    




%expert loading with cuda streams in experiments
\section{Experiments: Planning outperforms Heuristics}
\label{sec:experiment}

We begin our empirical demonstrations by showcasing the effectiveness of our planning framework on both synthetic and real datasets. We focus on the simplest planning algorithm, 1-step lookaheads (Algorithm~\ref{alg:complete}), and show that even basic planning can hold great promise. 
We illustrate our framework using two uncertainty quantification modules---GPs and 
\ensembles/ \ensembleplus. 

Throughout this section, we focus on evaluating the mean squared error of 
a regression model $\model$,  and develop adaptive policies that minimize uncertainty on $g(f)$ defined in~\eqref{eqn:l2-g-f}.
When GPs provide a valid model of uncertainty, 
our experiments show that our planning framework significantly outperforms other baselines. 
We further demonstrate that our conceptual framework extends to deep learning-based uncertainty quantification methods such as  \ensembleplus while highlighting computational challenges that need to be resolved in order to scale our ideas. 
For simplicity, we assume a naive predictor, i.e., $\psi(\cdot) \equiv 0$. However, we emphasize that this problem is just as complex as if we were using a sophisticated model $\psi(.)$. The performance gap between the algorithms 
primarily depends
on the level  of uncertainty in our prior beliefs.

To evaluate the performance of our algorithm, we benchmark it against several baselines. 
%Active learning baselines use an acquisition function $\ac$ to select points that have the highest   function value: $X\opt_t \in \argmax_{X \in \xpoolj{t}} \ac({X})$ at every step $t$. These methods may also need an UQ module, which we simply use the same UQ module as in our algorithm, and it  outputs $V(X)$ that measures the the uncertainty of each point $X \in \xpoolj{t}$.
Our first set of baselines are from active learning~\citep{AggarwalKoGuHaPh14}:
\\ % \noindent\textbf{Active Learning Heuristics:} 
\textbf{(1)} 
\textsf{Uncertainty Sampling (Static):}  In this approach, we query the samples for which the model is least certain about. Specifically, we estimate the variance of the latent output $f(X)$ for each $X \in \xpool$ using the UQ module and select the top-$K$ points with the highest uncertainty. \\
\textbf{(2)} \textsf{Uncertainty Sampling (Sequential):} This is a greedy heuristic that sequentially selects the points with the highest uncertainty within a batch, while updating the posterior beliefs using pseudo labels from the current posterior state. Unlike \textsf{Uncertainty Sampling (Static)}, this method takes into account the information gained from each point within batch, and hence tries to diversify the selected points within a batch. 

 
We also compare our approach to the  \textbf{(3)} \textsf{Random Sampling}, which selects each batch uniformly at random from the pool. Additionally, we compare solving the planning problem using  \textsf{REINFORCE}-based policy gradients with   $\mathsf{Smoothed\text{-}Autodiff}$ policy gradients.\footnote{Our code repository is available at
  \url{https://github.com/namkoong-lab/adaptive-labeling}.}
%Detailed experimental setups are provided in Section \ref{sec:details-experiments}.

%We repeat all experiments with 10 random seeds.




\begin{figure}[t]
\centering
\begin{minipage}[b]{0.49\textwidth}
\centering
\includegraphics[width=\textwidth, height=5cm]{figures/original_scale/Var_of_l_2_loss.pdf}
\caption{(Synthetic data) Variance of mean squared loss evaluated through the posterior belief $\mu_t$ at each horizon $t$. This is the objective that policy gradient methods like \textsf{REINFORCE} and $\ouralgo$ optimizes. 1-step lookaheads are surprisingly effective even in long horizons.}
\label{fig:var-l2-sim}
\end{minipage}
\hfill
\begin{minipage}[b]{0.49\textwidth}
\centering \includegraphics[width=\textwidth, height=5cm]{figures/original_scale/Error_of_estimated_model_l_2_loss.pdf}
\caption{(Synthetic data) Error between MSE calculated based on collected data $\mc{D}^{0:T}$ vs. population oracle MSE over $\mc{D}_{\rm eval} \sim P_X$. Reducing uncertainty over posteriors directly leads to better OOD evaluations. 1-step lookaheads significantly outperform active learning heuristics in small horizons.}
\label{fig:mean-l2-sim}
\end{minipage}
%\caption{Simulated data for GPs}
%\label{fig:both_plots}
\end{figure}

\subsection{Planning with Gaussian processes}
\label{sec:experiment-plan-GP}
We now briefly describe the data generation process for the GP experiments,  deferring a more detailed discussion of the dataset generation to Section~\ref{sec:details-experiments}. 
We use both the synthetic data and the real data to test our methodology.
For the \emph{simulated data},  we construct a setting where the general population is distributed across \emph{51 non-overlapping clusters} while the initial labeled data $\dtrain$ just comes from one cluster. In contrast, both $\dpool \defeq (\xpool,\ypool),\deval \defeq (\xeval,\yeval)$ are generated   from all the clusters. 
We begin with a low-dimensional scenario, generating a one-dimensional regression setting using a GP. %Gaussian Process (GP).
Although the data-generating process is not known to the algorithms,  we assume that the GP hyperparameters are known to all the algorithms
to ensure fair comparisons. This can be viewed as a setting where our prior is well-specified, allowing us to isolate the effects
of different policy optimization approaches
 without any concerns about the misspecified priors. We select $10$ batches, each of size $K=5$ across $T = 10$ time horizons.

To examine the robustness of our method against the distributional assumptions made  in the simulated case, we then move to a real dataset where the correct prior is not known. We simulate selection bias from the eICU dataset~\citep{PollardJoRaCeMaBa18}, which contains real-world patient data with in-hospital mortality outcomes. 
We conduct a $k$-means clustering to generate 51 clusters and then select data from those clusters. We view this to be a credible replication of practice, as severe distribution shifts are common due to selection bias in clinical labels.  To convert the binary mortality labels into a regression setting, we train a  random forest classifier and fit a GP on predicted scores, which serves as the UQ module for all the algorithms. As before, the task is to select 10 batches, each consisting of 5 samples, across 10 time horizons.

 In Figures~\ref{fig:var-l2-sim} and~\ref{fig:mean-l2-sim}, we present results for the simulated data. 
Figure~\ref{fig:var-l2-sim} shows the variance of $\ell_2$ loss, and Figure~\ref{fig:mean-l2-sim} presents the error in the estimated $\ell_2$ loss using $\mu_t$ (relative to true $\ell_2$ loss, that is unknown to the algorithm). 
As we can see from these plots, our method one-step lookahead  gives substantial improvements  over active learning baselines and random sampling. In addition,
compared to the one-step lookahead planning approach using \textsf{REINFORCE}-based policy gradients, 
we observe that $\mathsf{Smoothed\text{-}Autodiff}$-based policy gradients provide significantly more robust performance over all horizons.

In Figures~\ref{fig:var-l2-real}~and~\ref{fig:mean-l2-real}, we observe similar findings on the eICU data. We see that planning policies (\textsf{REINFORCE} and $\mathsf{Smoothed\text{-}Autodiff}$) consistently outperform other heuristics by a large margin.  Active learning baselines perform poorly in these small-horizon batched problems and can sometimes be even worse than the random search baselines.  Overall, our results show the importance of careful planning in adaptive labeling for reliable model evaluation. 

We offer some intuition as to why one-step lookahead planning may outperform other heuristic algorithms. 
 First,  \textsf{Uncertainty sampling (Static)} while myopically selects the
 top-$K$ inputs with the highest uncertainty, it fails to consider 
the overlap in information content among the ``best” instances; see \citep{AggarwalKoGuHaPh14} for more details. 
In other words,  it might acquire points from the same region with high uncertainty while failing to induce diversity among the batch.
Although \textsf{Uncertainty Sampling (Sequential)} somewhat addresses the issue of information overlap, a significant drawback of 
this algorithm
is the disconnect between the objective we aim to optimize and the algorithm. For example, it might sample from a region with high uncertainty but very low density. 

\begin{figure}[t]
\centering
\begin{minipage}[b]{0.48\textwidth}
\centering
\includegraphics[width=\textwidth, height=5cm]{figures/original_scale/Var_of_l_2_loss_real.pdf}
\caption{(Real-world eICU data) Variance of mean squared loss evaluated through the posterior belief $\mu_t$ at each horizon $t$. Even 1-step lookaheads are extremely effective planners, and auto-differentiation-based pathwise policy gradients provide a reliable optimization algorithm based on low-variance gradient estimates.}
\label{fig:var-l2-real}
\end{minipage}
\hfill
\begin{minipage}[b]{0.48\textwidth}
\centering \includegraphics[width=\textwidth, height=5cm]{figures/original_scale/Error_of_estimated_model_l_2_loss_real.pdf}
\caption{(Real-world eICU data) Error between MSE calculated based on collected data $\mc{D}^{0:T}$ vs. population oracle MSE over $\mc{D}_{\rm eval} \sim P_X$. Reducing uncertainty over posteriors directly leads to better OOD evaluations. Our method significantly outperforms active learning-based heuristics, and random sampling.}
\label{fig:mean-l2-real}
\end{minipage}
%\caption{Real data for GPs}
\end{figure}
 
%\vspace{-1.5cm}
% \begin{wrapfigure}{r}{.32\columnwidth}
%   \vspace{-.5cm} 
%   \centering
% \includegraphics[scale=.29]{figures/Var of l2l_2 loss.pdf}
%   \vspace{-0.2cm}
%   \caption{Results of GP}
% \label{fig:var-l2-gp}
%   \vspace{-0.1cm}
% \end{wrapfigure}


% Attempts have been made  in the past to address these  drawbacks heuristically  (see \citep{AggarwalKoGuHaPh14}). We give a unified computational framework while approaching the problem in a more principled manner and solving it more optimally.




\subsection{Planning with  neural network-based uncertainty quantification methods ($\ensembleplus$)}


We now provide a proof-of-concept that shows the generalizability of our conceptual framework  to the deep learning-based UQ modules, specifically focusing on $\ensembleplus$ due to their previously observed superior performance~\citep{OsbandWenAsDwIbLuRo23}. Recall that implementing our framework with deep learning-based UQ modules  requires us to retrain the model across multiple possible random actions $\bm{a}(\theta)$ sampled from the current policy $\pi_\theta$.
This requires significant computational resources, in sharp contrast to the GPs where the posteriors are in closed form and can be readily updated and differentiated. 

Due to the computational constraints, we test $\ensembleplus$ on a toy setting to demonstrate the generalizability of our framework. We consider a setting where the general population consists of four clusters, while the initial labeled data only comes from one cluster. Again we generate data using GPs.  The task is to select a batch of 2 points in one horizon. We detail the $\ensembleplus$ architecture in Section \ref{sec:details-experiments}, and we assume prior uncertainty to be large (depends on the scaling of the prior generating functions). 
The results are summarized in the Table~\ref{tab:UQ_ensemble}.

% \begin{table}[H]
% \vspace{-10pt}
% \caption{Performance under \ensembleplus as UQ module}
%     \centering
%     \begin{tabular}{|m{3cm}|m{2.5cm}|m{2cm}|} 
%     \hline
%       Algorithm   & Variance of $\loss_2$ loss estimate & Error of $\loss_2$ loss estimate  \\ \hline Random Sampling 
%          & $1710.9 \pm 1352.1$ & $8.67\pm6.62$ 
%       \\ \hline \ouralgo & $1.30 \pm 0.68$ & $0.91\pm0.25$ \\ \hline
%     \end{tabular}
%     \label{tab:UQ_ensemble}
%     %\vspace{-10pt}
% \end{table}




\begin{table}[h]
\vspace{-10pt}
\caption{Performance under \ensembleplus as the UQ module}
\centering
\begin{tabular}{|l|l|l|}
\hline
Algorithm   & Variance of $\loss_2$ loss estimate & Error of $\loss_2$ loss estimate  \\
\hline
\textsf{Random sampling} & 7129.8 $\pm$ 1027.0 & 136.2 $\pm$ 8.28 \\ \hline
\textsf{Uncertainty sampling (Static)} & 10852 $\pm$ 0.0 & 162.156 $\pm$ 0.0 \\ \hline
\textsf{Uncertainty sampling (Sequential)} & 8585.5 $\pm$ 898.9 & 144 $\pm$ 6.93 \\ \hline
\textsf{REINFORCE} & 1697.1 $\pm$ 0.0 & 45.27 $\pm$ 0.0 \\ \hline
\ouralgo & 1697.1 $\pm$ 0.0 & 45.27 $\pm$ 0.0 \\ \hline
\end{tabular}
%\caption{Comparison of different algorithms based on variance   and   error in $\ell_2$ loss estimation with Ensemble $+$ as the UQ module. Our results demonstrate that {\ouralgo} and REINFORCE outperformthe other active learning based heuristics, confirming the benefits of our MDP formulation for the adaptive labeling problem, as also demonstrated in Section 4.\\
%\footnotesize{Experimental details: We use Gaussian Processes as our data generating process, GP parameters are the same as in Section D.3.  The task is to select a batch of 2 points along one horizon.The marginal distribution $p_X$ has 4 \textit{non-overlapping} clusters. Initial data comes from one cluster, while pool and evaluation points comes from all the clusters. We have $20$ initial labeled data points, $10$ pool points, and $252$ evaluation points.  Training procedures are similar to the one in Section D.3.} }
\label{tab:UQ_ensemble}
\end{table}



% We faced  issues in scaling up these experiments which will be our focus in the future. 





% \begin{itemize}
%     \item Posteriors should be consistent. Two dimensions: even with less training,  
%     \item the inference should be  fast enough
% \end{itemize}


% Potential research directions for uncertainty quantification

% In this section we consider a simple setting We consider a simpler setting and 


% For synthetic dataset generation, we use ...... For real datasets, we use ...... We compare our methodolgy to several baselines ()    This Section is structured as follows:
% \begin{itemize}
%     \item \textbf{GPs, square loss objective} (Section \ref{}): 
%     %the broad aim of the experiments  in this section is to isolate the performance of our methodology without any concerns for the inefficiencies induced due to a mis-specified prior or imperfect posterior inference. To accomplish this we generate synthetic datasets using GPs (detailed later). We use the well specified prior (GPs - with same hyperparameter setting) as our UQ module.   
%      As GPs provide differentaible posterior inference - any errors induced due to imperfect posterior updates are also isolated. We note that under this setting
%      \item In Section\ref{} we demonstrate why our methodology performs better than other baselines - by devising various synthetic experiments ()
%     \item  \textbf{UQ Benchmarking }(Section \ref{}): Before diving into the experiments using $\ensembleplus$ and ENNs,  we showcase our benchmarking experiments in Section \ref{}. We use real datasets We observe that ENNs perform better
%      \item \textbf{Ensemble $+$}, objective: recall, accuracy
%     \item \textbf{ENN}, objective: recall, accuracy
% \end{itemize}




% In Section {}, we test 
% \subsection{Experimental details}

% \begin{itemize}
%     \item UQ methodologies - GPs, ENNs
%     \item Objectives - Recall,  ATE
%     \item Datasets - ATE-synthetic datasets, Recall-synthetic, real datasets
%     \item Baselines - 
%     \begin{itemize}
%         \item Random sampling
%         \item Active learning - Uncertainty based sampling - In regression setting almost all of the 
%         \item Myopic greedy - Greedy Batch based sampling
%         \item Policy Gradient
%     \end{itemize}
    
% \end{itemize}

% \subsection{Experiments}
%     \begin{itemize}
%     \item GPs with square loss
%     \item Benchmarking ENN
%         \item ENNs with ATE
%         \item ENNs with Recall
%     \end{itemize}

% \subsection{Benefits over other algorithms - intuition and experiments}

%Active learning - Myopic greedy / Don't rely on the objective rather some entropy version.


%%% Local Variables:
%%% mode: latex
%%% TeX-master: "main"
%%% End:



\section{Conclusion}
In this paper, we propose a simple yet effective alignment method for large language models, named \method. \method eliminates the need for both hyperparameters and a reference model while achieving strong performance. The key idea of \method is to directly optimize the reverse perplexity of both the chosen and rejected responses in the preference dataset. Specifically, we minimize the perplexity over the chosen response while maximizing the perplexity over the rejected response, ensuring that the model produces responses with high preference scores. We also provide a theoretical understanding of \method, demonstrating that it can mitigate the problem of gradient imbalance between the chosen and rejected responses during optimization. Furthermore, we show that \method implicitly exhibits mode-seeking behavior, leading to strong alignment performance. Extensive experiments on widely used benchmarks demonstrate that \method significantly outperforms state-of-the-art methods.

\section*{Acknowledgment}
The work of Vasant G Honavar and Teng Xiao was supported in part by grants from the National Science Foundation
(2226025, 2225824), the National Center for Advancing Translational Sciences, and the National
Institutes of Health (UL1 TR002014).
% \clearpage
\bibliography{iclr2025_conference}
\bibliographystyle{iclr2025_conference}

\newpage
\appendix
\section{Proofs of Theorem~\ref{the:mutual}}
\label{app:proof}
To prove Theorem~\ref{the:mutual} in the main paper, we first present the following  Lemmas:

\begin{lemma}\label{lemma:1}
Given the model conditional distribution $\pi_{\theta}(\mathbf{y} \mid \mathbf{x})=\prod_{i=1}^{|\mathbf{y}|} \pi_\theta\left(\mathbf{y}_i \mid \mathbf{x}, \mathbf{y}_{<i}\right)$ and  data distribution $\pi_{\rm{data}}(\mathbf{y} \mid \mathbf{x})=\prod_{i=1}^{|\mathbf{y}|} \pi_{\rm{data}}\left(\mathbf{y}_i \mid \mathbf{x}, \mathbf{y}_{<i}\right)$, then we have the following relationship between the sequence-level TVD objective and its token-level factorization:
   \begin{equation}
    {\mathrm{TV}}(\pi_{\rm{data}}(\mathbf{y} \mid \mathbf{x}), \pi_{\theta}(\mathbf{y} \mid \mathbf{x})) \le \mathbb{E}_{\mathbf{y}\sim \pi_{\rm{data}}(\mathbf{y} \mid \mathbf{x})}\Bigg[\sum\nolimits_{i=1}^{|\mathbf{y}|} {\mathrm{TV}}(\pi_{\rm{data}}^{<i}(\mathbf{y}_{i}), \pi_{\theta}^{<i}(\mathbf{y}_{i}))\Bigg],
\end{equation}
where $\pi_{\rm{data}}^{<i}(\mathbf{y}_{i})$ and $\pi_{\theta}^{<i}(\mathbf{y}_{i})$ are shorts for $\pi_{\rm{data}}\left(\mathbf{y}_i \mid \mathbf{x}, \mathbf{y}_{<i}\right)$ and  $\pi_\theta\left(\mathbf{y}_i \mid \mathbf{x}, \mathbf{y}_{<i}\right)$, respectively.
\end{lemma}
\begin{proof}
We  start by re-writing the TVD loss in the following recursive form:
\begin{align}
    & {\mathrm{TV}}(\pi_{\rm{data}}(\mathbf{y} \mid \mathbf{x}), \pi_{\theta}(\mathbf{y} \mid \mathbf{x}))=\frac{1}{2}\sum  \nolimits_{\mathbf{y}\in \mathcal{Y}} \left| \pi_{\rm{chosen}}(\mathbf{y}\mid\mathbf{x}) -  \pi_\mathbf{\theta}(\mathbf{y} \mid \mathbf{x})  \right|\\
     &=\frac{1}{2} \sum_{\mathbf{y}_1,\cdots,\mathbf{y}_{|\mathbf{y}|}}|\prod_{i=1}^{|\mathbf{y}|} \pi_{\rm{data}}\left(\mathbf{y}_i \mid \mathbf{x}, \mathbf{y}_{<i}\right)-\prod_{i=1}^{|\mathbf{y}|} \pi_\theta\left(\mathbf{y}_i \mid \mathbf{x}, \mathbf{y}_{<i}\right)| \label{Eq:lemma1-1}\\
     &=\frac{1}{2} \sum_{\mathbf{y}_1,\cdots,\mathbf{y}_{|\mathbf{y}|}}|\prod_{i=1}^{|\mathbf{y}|} \pi_{\rm{data}}^{<i}(\mathbf{y}_{i})-\prod_{i=1}^{|\mathbf{y}|} \pi_{\theta}^{<i}(\mathbf{y}_{i})|\\
     &\leq \frac{1}{2}\sum_{i=1}^{|\mathbf{y}|}\sum_{\mathbf{y}_1,\cdots,\mathbf{y}_{i}}\prod _{j=1}^{i-1} \pi_{\rm{data}}^{<j}(\mathbf{y}_{j})|\prod_{i=1}^{|\mathbf{y}|} \pi_{\rm{data}}^{<i}(\mathbf{y}_{i})-\prod_{i=1}^{|\mathbf{y}|} \pi_{\theta}^{<i}(\mathbf{y}_{i})| \sum_{\mathbf{y}_{i+1},\cdots,\mathbf{y}_{|\mathbf{y}|}}\prod_{t=i+1}^{|\mathbf{y}|} \pi_{\theta}^{<t}(\mathbf{y}_{t}) \label{Eq:lemma1-2}\\
     &= \frac{1}{2}\sum_{i=1}^{|\mathbf{y}|}\sum_{\mathbf{y}_1,\cdots,\mathbf{y}_{i}}\prod _{j=1}^{i-1} \pi_{\rm{data}}^{<j}(\mathbf{y}_{j})|\pi_{\rm{data}}^{<i}(\mathbf{y}_{i})- \pi_{\theta}^{<i}(\mathbf{y}_{i})| \label{Eq:lemma1-3} \\
       &= \frac{1}{2}\sum_{i=1}^{|\mathbf{y}|}\sum_{\mathbf{y}_{i}} \mathbb{E}_{ \mathbf{y_{<i}}\sim \pi_{\rm{data}}}\Big[|\pi_{\rm{data}}^{<i}(\mathbf{y}_{i})- \pi_{\theta}^{<i}(\mathbf{y}_{i})|\Big]\\
       &=\mathbb{E}_{\mathbf{y}\sim \pi_{\rm{data}}}\Bigg[ \sum_{i=1}^{|\mathbf{y}|} \mathrm{TV}(\pi_{\rm{data}}^{<i}(\mathbf{y}_{i}),\pi_{\theta}^{<i}(\mathbf{y}_{i})) \Bigg],
\end{align}
where Equation~(\ref{Eq:lemma1-1}) breaks the sequence-level summation into the steps and Equation~(\ref{Eq:lemma1-2}) applies the following triangle inequality~\citep{wen2023f,jitailoring}:
\begin{align}
\left| \prod_{t=1}^{T} a_t - \prod_{t=1}^{T} b_t \right|
\leq \sum_{t=1}^{T} \left| a_t - b_t \right| \cdot
\left( \prod_{i=1}^{t-1} a_i \right) \cdot
\left( \prod_{j=t+1}^{T} b_j \right),
\end{align}
and Equation~(\ref{Eq:lemma1-3}) marginalizes out variables $\mathbf{y}_{i+1},\cdots,\mathbf{y}_{|\mathbf{y}|}$. 
\end{proof}


% \begin{lemma}\label{lemma:2}
% Minimizing the \texttt{SimPO} objective in Equation~(\ref{Eq:SimPO}) with respect to ${\theta}$ reduces the Kullback–Leibler divergence between $\pi_\theta$ and the distribution of chosen response in preference dataset:
% \begingroup\makeatletter\def\f@size{9.5}\check@mathfonts\def\maketag@@@#1{\hbox{\m@th\normalfont\normalfont#1}}
% \begin{align}
% \min_{\theta} \mathcal{L}_{\rm{SimPO}} \Rightarrow  \min_{\theta}{\mathrm{KL}}(\pi_{\rm{chosen}}(\mathbf{y}\mid\mathbf{x})\|  \pi_\mathbf{\theta}(\mathbf{y} \mid \mathbf{x})  ) =\sum  \nolimits_{y\in \mathcal{Y}} \pi_{\rm{chosen}}(\mathbf{y} \mid \mathbf{x})\log \frac{\pi_{\rm{chosen}}(\mathbf{y} \mid \mathbf{x})}{\pi_{\theta}(\mathbf{y} \mid \mathbf{x})}
% \end{align}
% \endgroup
% \end{lemma}
% \begin{proof}
% Recall that the objective of \texttt{SimPO} without hyperparmeters is:
% \begin{align}
% \mathcal{L}_{\rm{SimPO}}({\theta}; \mathcal{D})=\mathbb{E}_{(\mathbf{x}, \mathbf{y}_w, \mathbf{y}_l)\sim \mathcal{D}}\left[-\log \sigma( \log {\pi_{\theta}(\mathbf{y}_w \mid \mathbf{x})}-\log {\pi_{\theta}(\mathbf{y}_l \mid \mathbf{x} )}) \right]. \label{Eq:app-simpo}
% \end{align}
% We connect this objective to a distribution matching problem by defining an energy-based model:
% % \begingroup\makeatletter\def\f@size{9.5}\check@mathfonts\def\maketag@@@#1{\hbox{\m@th\normalfont\normalfont#1}}
% \begin{align}
% \pi(\mathbf{y} \mid \mathbf{x})={\exp(r(\mathbf{x},\mathbf{y}))}/{Z(\mathbf{x})}, \label{Eq:optimal}
% \end{align}
% % \endgroup
% where $Z(\mathbf{x})=\sum_{\mathbf{y}}\exp(r(\mathbf{x},\mathbf{y}))$. We then frame the task as minimizing the KL divergence between the policy $\pi$ and the distribution $\pi_{\mathrm{chosen}}$ that generates the chosen response. The goal is to learn the reward model $r$ such that the model distribution imitates the chosen data distribution:
% \begin{align}
% \min _{r} {\mathrm{KL}}\left(\pi_{\mathrm{chosen}}(\mathbf{y} \mid \mathbf{x}) \| \pi(\mathbf{y} \mid \mathbf{x})\right). \label{eq:FKL}
% \end{align}
% Minimizing the above forward KL divergence with the chosen responses on preference data gives us:
% \begingroup\makeatletter\def\f@size{9.5}\check@mathfonts\def\maketag@@@#1{\hbox{\m@th\normalfont\normalfont#1}}
% \begin{align}
% \min _{r} &\mathbb{E}_{(\mathbf{x},\mathbf{y}_{w})\sim \mathcal{D}}[-\log \frac{\exp(r(\mathbf{x},\mathbf{y}_{w}))}{Z(\mathbf{x})}]=\\
%     &\mathbb{E}_{(\mathbf{x},\mathbf{y}_{w})\sim \mathcal{D}} \Big [-r(\mathbf{x},\mathbf{y}_{w})+\log \mathbb{E}_{q(\mathbf{y}\mid
%     \mathbf{x})} \big[ \frac{\exp (r(\mathbf{x},\mathbf{y}))}{q(\mathbf{y}\mid\mathbf{x})} \big] \Big ]. \label{Eq:connection}
% \end{align}
% \endgroup
% The equality follows due to importance sampling under distribution $q$.  A choice that simplifies the above expression is $q(\mathbf{y} \mid \mathbf{x})=\frac{1}{2} \mathbb{I}(\mathcal{Y}=\mathbf{y}_{l})+\frac{1}{2}  \mathbb{I}(\mathcal{Y}=\mathbf{y}_{w})$.  In this case, the sample-based approximation of Equation~(\ref{Eq:connection})  gives us:
% \begin{align}
% \min_{r}&\mathbb{E}_{(\mathbf{x},\mathbf{y}_{w},\mathbf{y}_{l})\sim \mathcal{D}}\Big [-r(\mathbf{x},\mathbf{y}_{w})+\log (\exp(r(\mathbf{x},\mathbf{y}_{w}))+ \exp(r(\mathbf{x},\mathbf{y}_{l})))\Big ] \nonumber \\
% &=\mathbb{E}_{(\mathbf{x},\mathbf{y}_{w},\mathbf{y}_{l})\sim \mathcal{D}}\Big [-\log (\frac{1}{1+ \exp(r(\mathbf{x},\mathbf{y}_{l})-r(\mathbf{x},\mathbf{y}_{w}))} \Big]. \label{Eq:imtaiton-reward}
% \end{align}
% One can easily see that the above loss is exactly the same as the reward loss function in RLHF~\citep{rafailov2024direct}. By optimizing this loss function, one can directly obtain the optimal policy via Equation~(\ref{Eq:optimal}). Unfortunately,  even if we use the estimate $r$, it is
% still expensive to estimate the partition function $Z(\mathbf{x})$, which makes this representation hard to utilize in practice and  significantly higher inference cost~\citep{rafailov2024direct}. To address this problem,  we apply the following knowledge distillation loss, which distills the optimal policy in Equation~(\ref{Eq:optimal}) into a analytical policy:
% \begin{align}
% \min_{\pi_\theta}\mathrm{KL}\left(\pi_{\theta}(\mathbf{y}\mid \mathbf{x})||{\exp(r(\mathbf{x},\mathbf{y}))}/{Z(\mathbf{x})} \right), \label{Eq:distillation}
% \end{align}
% This gives the following loss after removing multiplicative and additive constants:
% \begin{align}
%     \mathcal{L}(\theta)=-\mathbb{E}_{\pi_\theta(\mathbf{y} \mid \mathbf{x})}\left[r_{\phi}\left(\mathbf{x}, \mathbf{y}\right)-\log \pi_{\theta}(\mathbf{y}|\mathbf{x})\right], \label{Eq:distillation2}
% \end{align}
% One can observe that this knowledge distillation objective directly corresponds to the RL objective in RLHF, as shown in \citep{rafailov2024direct}. Similar to the derivation in~\citep{rafailov2024direct}, which uses an analytical mapping from reward functions to optimal policies in Equation~(\ref{Eq:distillation2}), combining Equation~(\ref{Eq:imtaiton-reward}) results in the SimPO objective in Equation~(\ref{Eq:app-simpo}). Therefore, performing distribution matching over the chosen responses using KL divergence ${\mathrm{KL}}\left(\pi_{\mathrm{chosen}}(\mathbf{y} \mid \mathbf{x}) \| \pi_{\theta}(\mathbf{y} \mid \mathbf{x})\right)$ is equivalent to solving an entropy-regularized RLHF problem, as well as the SimPO objective.
% \end{proof}


\textbf{Theorem~\ref{the:mutual}.} \textit{
Minimizing  \texttt{SFT}  with respect to ${\theta}$ is approximately minimizing the KLD between $\pi_\theta$ and the distribution of the chosen response in the preference dataset, while minimizing our \method  is approximately minimizing the TVD.}
\begingroup\makeatletter\def\f@size{9.5}\check@mathfonts\def\maketag@@@#1{\hbox{\m@th\normalfont\normalfont#1}}
\begin{align}
\min_{\theta} \mathcal{L}_{\rm{SFT}} &\Rightarrow  \min_{\theta}{\mathrm{KL}}(\pi_{\rm{chosen}}(\mathbf{y}\mid\mathbf{x})\|  \pi_\mathbf{\theta}(\mathbf{y} \mid \mathbf{x})  ) =\sum  \nolimits_{y\in \mathcal{Y}} \pi_{\rm{chosen}}(\mathbf{y} \mid \mathbf{x})\log \frac{\pi_{\rm{chosen}}(\mathbf{y} \mid \mathbf{x})}{\pi_{\theta}(\mathbf{y} \mid \mathbf{x})}\\
\min_{\theta} \mathcal{L}_{\rm{SimPER}} &\Rightarrow  \min_{\theta}{\mathrm{TV}}(\pi_{\rm{chosen}}(\mathbf{y}\mid\mathbf{x})\|  \pi_\mathbf{\theta}(\mathbf{y} \mid \mathbf{x})  ) =\sum\nolimits_{y\in \mathcal{Y}} \left| \pi_{\rm{chosen}}(\mathbf{y}\mid\mathbf{x}) -  \pi_\mathbf{\theta}(\mathbf{y} \mid \mathbf{x})  \right| 
\end{align}
\endgroup
\begin{proof}
Given that $\mathbb{E}_{y \sim \pi_{\rm{data}}^{<i}(\mathbf{y}_{i})}[\mathbf{y}_{i}] = \pi_{\rm{data}}^{<i}(\mathbf{y}_{i})$, where $\mathbf{y}_{i}$ represents the observed one-hot distribution with only the  $w$-th index of the observed token being 1 and the others being 0, we have
\begin{align}
\mathrm{TV}&(\pi_{\rm{data}}^{<i}(\mathbf{y}_{i}), \pi_{\theta}^{<i}(\mathbf{y}_{i}))=\frac{1}{2}\sum_{\mathbf{y}_{i}}|\pi_{\rm{data}}^{<i}(\mathbf{y}_{i})- \pi_{\theta}^{<i}(\mathbf{y}_{i})|\\
&=\frac{1}{2}\sum_{\mathbf{y}_{i}}|\mathbb{E}_{y \sim \pi_{\rm{data}}^{<i}(\mathbf{y}_{i})}[\mathbf{y}_{i}]  - \pi_{\theta}^{<i}(\mathbf{y}_{i})|\\
&\leq \frac{1}{2}\mathbb{E}_{y \sim \pi_{\rm{data}}^{<i}(\mathbf{y}_{i})}\Big[\sum_{\mathbf{y}_{i}}|\mathbf{y}_{i} - \pi_{\theta}^{<i}(\mathbf{y}_{i})| \Big]=\mathbb{E}_{y \sim \pi_{\rm{data}}^{<i}(\mathbf{y}_{i})}\left[\mathrm{TV}(\mathbf{y}_{i},\pi_{\theta}^{<i}(\mathbf{y}_{i})) \right]\\
&=\mathbb{E}_{y \sim \pi_{\rm{data}}^{<i}(\mathbf{y}_{i})}\Big[1-\sum_{\mathbf{y}_{i}} \min (\mathbf{y}_{i},\pi_{\theta}^{<i}(\mathbf{y}_{i})) \Big]=- \pi_{\theta}^{<i}(\mathbf{y}_{i}).
\end{align}
Combing the above with Lemma~\ref{lemma:1}, we have:
\begin{align}
{\mathrm{TV}}(\pi_{\rm{data}}(\mathbf{y} \mid \mathbf{x}), \pi_{\theta}(\mathbf{y} \mid \mathbf{x}))&\leq \mathbb{E}_{\mathbf{y}\sim \pi_{\rm{data}}}\Bigg[ \sum_{i=1}^{|\mathbf{y}|} \mathrm{TV}(\pi_{\rm{data}}^{<i}(\mathbf{y}_{i}),\pi_{\theta}^{<i}(\mathbf{y}_{i})) \Bigg]\\
& = -\mathbb{E}_{\mathbf{y}\sim \pi_{\rm{data}}} \Bigg[\sum_{i=1}^{|\mathbf{y}|}\pi_\theta\left(\mathbf{y}_i \mid \mathbf{x}, \mathbf{y}_{<i}\right)\Bigg].
\end{align}
Recall that the objective of \method, without length-averaging, is:
\begin{align}
    \mathcal{L}_{\rm{SimPER}}({\theta}; \mathcal{D})&=-\exp \left( \log \pi_\theta(\mathbf{y}_{w} \mid \mathbf{x})\right)+\exp \left( \log \pi_\theta(\mathbf{y}_{l} \mid \mathbf{x})\right)\\
    &\geq -\pi_\theta(\mathbf{y}_{w} \mid \mathbf{x})\geq {\mathrm{TV}}(\pi_{\rm{chosen}}(\mathbf{y} \mid \mathbf{x}), \pi_{\theta}(\mathbf{y} \mid \mathbf{x})). \label{Eq:43}
\end{align}
Combining Equation~(\ref{Eq:43}) completes the whole proof.
\end{proof}


\section{Experimental Details}
\subsection{Downstream Task Evaluation}
\label{app:task-eval}
This section is for the detailed introduction to the evaluation of downstream tasks.
For models fine-tuned on UltraFeedback Binarized, following previous works~\citep{rafailov2024direct,tunstall2023zephyr}, we evaluated them on HuggingFace Open LLM Leaderboard v1\footnote{\url{https://huggingface.co/spaces/open-llm-leaderboard-old/open_llm_leaderboard}} and v2\footnote{\url{https://huggingface.co/spaces/open-llm-leaderboard/open_llm_leaderboard}}~\citep{open-llm-leaderboard-v1,open-llm-leaderboard-v2}, encompassing ten tasks: 
MMUL-PRO~\citep{wang2024mmluprorobustchallengingmultitask}, IFEval~\citep{zhou2023instructionfollowingevaluationlargelanguage}, BBH~\citep{suzgun2022challengingbigbenchtaskschainofthought}, GPQA~\citep{rein2023gpqagraduatelevelgoogleproofqa}, MUSR~\citep{sprague2024musrtestinglimitschainofthought}, MATH~\citep{hendrycks2021measuringmathematicalproblemsolving}, GSM8k~\citep{cobbe2021training}, ARC~\citep{clark2018think}, 
TruthfulQA~\citep{lin-etal-2022-truthfulqa} and Winogrande~\citep{sakaguchi2021winogrande} provided by the Language Model Evaluation Harness library~\citep{eval-harness}.
In addition to the leaderboard, we also evaluated them on two instruction-following benchmarks AlpacaEval2~\citep{li2023alpacaeval} and MT-Bench~\citep{zheng2023judging}.
For models fine-tuned on Anthropic-HH for safety alignment, following previous works~\citep{rafailov2024direct}, we use GPT-4 for zero-shot pair-wise evaluation, which is shown to be consistent with human judgments.

% \textbf{Massive Multitask Language Understanding Professional}: MMUL-PRO is an enhanced version of the MMLU dataset~\citep{hendrycks2021measuring}, addressing previous shortcomings by increasing choice options in multiple-choice questions and refining question quality through expert review, making it more challenging and less prone to data contamination.

% \textbf{Instruction-Following Evaluation}: IFEval is a benchmark evaluating a model's ability to follow explicit instructions, emphasizing adherence to formatting over content generation.

% \textbf{Big Bench Hard}: BBH is a selection of 23 challenging tasks from the BigBench, focusing on areas like multistep arithmetic, algorithmic reasoning, language understanding, and world knowledge.

% \textbf{Graduate-Level Google-Proof Q\&A}: GPQA is a challenging benchmark composed of advanced questions developed by PhD-level experts across various fields like biology, physics, and chemistry.

% \textbf{Multistep Soft Reasoning}: MUSR is a challenging benchmark consisting of algorithmically generated complex scenarios requiring integrated reasoning and long-range context understanding ability.

% \textbf{MATH}: MATH is a benchmark consisting of high-school level competition problems gathered from multiple sources, consistently formatted with LaTeX for equations and Asymptote for diagrams.

 
% \textbf{GSM8K}: GSM8K is a benchmark consisting of diverse grade school math word problems designed to evaluate a model's capability in solving multi-step mathematical reasoning problems.


% \textbf{AI2 Reasoning Challenge}: ARC is a benchmark consists of grade-school science questions.

% \textbf{TruthfulQA}: TruthfulQA is a benchmark fore the truthfulness of a model in generating responses to questions, including 817 questions across 38 categories such as health, law, finance, and politics.
 
% \textbf{Winogrande}: Winogrande is a large-scale benchmark for commonsense reasoning, inspired by WSC benchmark~\citep{levesque2012winograd}, but adjusted to improve both the scale and the hardness.


% \textbf{AlpacaEval 2.0}: AlpacaEval 2.0 is a LLM-based automatic evaluator for instruction-following language models, which is fast, cheap, replicable, and validated against 20K human annotations.


% \textbf{MT Bench}: MT-bench is an evaluator featuring challenging, multi-turn, open-ended questions to assess chat assistants, which automates the evaluation process by using advanced LLMs as judges.

\textbf{Safety Evaluation}:
To evaluate the safety of the Anthropic HH trained model, the Anthropic HH test set is used as the evaluation dataset. The preferred responses within this test set serve as the baseline. GPT-4 is then employed to determine the winner, which is shown to be consistent with human judgments. We use \texttt{gpt-4-0314} for all our experiments. The specific prompts used for the GPT-4 evaluation are detailed below.

\begin{table}[h!]
    \small
    {\ttfamily
    \begin{tabularx}{\linewidth}{X}
    \toprule
    For the following dialogue history to a chatbot, which response is more helpful and harmless?
    \\
    \\
    Dialogue history: \\
    \{dialogue history\}
    \\
    \\
    Response 1:\\
     \{Response 1\}
    \\
    \\
    Response 2: \\
     \{Response 2\}
    \\
    \\
    FIRST provide a one-sentence comparison of the two responses and explain which you feel is more helpful and harmless. SECOND, on a new line, state only "1" or "2" to indicate which response is more helpful and harmless. Your response should use the format:\\
    Comparison: <one-sentence comparison and explanation>\\
    More helpful: <"1" or "2">\\
    \bottomrule
    \end{tabularx}
    }
    \caption{Prompt for GPT-4 evaluation for the safety alignment task on the Anthropic-HH dataset.   \{dialogue history\},     \{Response 1\} and  \{Response 2\} are placeholders.}
    \label{tab:gpt4_prompt_dialogue}
\end{table}



\subsection{Baseline Methods}
\label{app:baseline}
This section provides a detailed introduction to state-of-the-art baselines for preference fine-tuning, with an emphasis on the usage of hyperparameters in their loss functions.

\textbf{DPO} Direct Preference Optimization~\citep{rafailov2024direct} uses log-likelihood differences to implicitly represent the reward function, eliminating the need for explicit reward model like RLHF. 
DPO involves one tunable hyperparameter, $\beta$, which controls the deviation from the reference model. 
$$
\mathcal{L}_{\rm{DPO}}({\theta}; \mathcal{D})=\mathbb{E}_{(\mathbf{x}, \mathbf{y}_w, \mathbf{y}_l)\sim \mathcal{D}}\left[-\log \sigma(\beta \log \frac{\pi(\mathbf{y}_w \mid \mathbf{x})}{\pi_{\mathrm{ref}}(\mathbf{y}_w \mid \mathbf{x})}-\beta \log \frac{\pi(\mathbf{y}_l \mid \mathbf{x} )}{\pi_{\mathrm{ref}}(\mathbf{y}_l \mid \mathbf{x})})\right].
$$

\textbf{IPO} Identity Preference Optimization~\citep{azar2024general} minimizes a squared loss regression problem by defining an alternative reward function, avoiding unstable RL training. 
IPO involves one hyperparameter, $\beta$, to adjust the reward margin.
$$
\mathcal{L}_{\rm{IPO}}({\theta}; \mathcal{D})=\mathbb{E}_{(\mathbf{x}, \mathbf{y}_w, \mathbf{y}_l)\sim \mathcal{D}}\left[\left (\log \frac{\pi_\theta(\mathbf{y}_w \mid \mathbf{x})\pi_{\mathrm{ref}}(\mathbf{y}_l \mid \mathbf{x})}{\pi_{\theta}(\mathbf{y}_l \mid \mathbf{x} )\pi_{\mathrm{ref}}(\mathbf{y}_w \mid \mathbf{x})}-\frac{1}{2\beta}\right)^{2}\right].
$$

\textbf{CPO} Contrastive Preference Optimization~\citep{xucontrastive} uses log-likelihood as the reward and is trained alongside a Supervised Fine-Tuning (SFT) objective. 
CPO involves two hyperparameters: $\beta$, which scales the log probabilities, and $\lambda$, which weights the SFT component.
$$
\mathcal{L}_{\rm{CPO}}({\theta}; \mathcal{D}) = -\log \sigma\left(\beta \log \pi_\theta(\mathbf{y}_w\mid \mathbf{x}) - \beta \log \pi_\theta(\mathbf{y}_l\mid \mathbf{x}) \right) - \lambda \log \pi_\theta (\mathbf{y}_w\mid \mathbf{x}).
$$


\textbf{SLiC} Sequence Likelihood Calibration~\citep{zhao2023slic} directly uses log-likelihood and includes a SFT objective. 
SLiC involves two hyperparameters: $\delta$, which sets the margin for the ranking loss, and $\lambda$, which weights the SFT component.
$$
\mathcal{L}_{\text{SLiC}}(\theta; \mathcal{D}) = \max\left(0, \delta - \log \pi_\theta(\mathbf{y}_w\mid \mathbf{x}) + \log \pi_\theta(\mathbf{y}_l\mid \mathbf{x})\right) - \lambda \log \pi_\theta (\mathbf{y}_w \mid \mathbf{x}).
$$


\textbf{SimPO} Simple Preference Optimization~\citep{meng2024simpo} eliminates the need for a reference model and optimizes a length-regularized probability of response pairs. 
SimPO involves two hyperparameters: $\beta$ to scale the log probabilities and $\gamma$ to adjust the reward margin. 
$$
\mathcal{L}_{\rm{SimPO}}({\theta}; \mathcal{D})=\mathbb{E}_{(\mathbf{x}, \mathbf{y}_w, \mathbf{y}_l)\sim \mathcal{D}}\left[-\log \sigma(\beta \log {\pi_{\theta}(\mathbf{y}_w \mid \mathbf{x})}-\beta \log {\pi_{\theta}(\mathbf{y}_l \mid \mathbf{x} )})-\gamma \right].
$$


\textbf{KTO} Kahneman-Tversky Optimization~\citep{ethayarajh2024kto} learns from non-paired preference data. 
KTO involves three hyperparameters: $\beta$, which controls the deviation from the reference model; $\lambda_w$ and $\lambda_l$, which weight the preference components for winning and losing responses, respectively. 
\begin{equation}
\begin{split}
\mathcal{L}_{\rm{KTO}}({\theta}; \mathcal{D}) &= -\lambda_w \,\sigma \left( \beta \log \frac{\pi_\theta(\mathbf{y}_w\mid \mathbf{x})}{\pi_{\text{ref}}(\mathbf{y}_w\mid \mathbf{x})} - z_{\text{ref}} \right) +  \lambda_l \,\sigma \left( z_{\text{ref}} - \beta \log \frac{\pi_\theta(\mathbf{y}_l\mid \mathbf{x})}{\pi_{\text{ref}}(\mathbf{y}_l\mid \mathbf{x})} \right), \\
\text{where} \quad z_{\text{ref}} &= \mathbb{E}_{(x, y) \sim \mathcal{D}} \left[\beta \;\mathrm{KL}\left( \pi_\theta(\mathbf{y}\mid \mathbf{x}) || \pi_{\text{ref}}(\mathbf{y}\mid \mathbf{x}) \right)  \right].
\end{split}
\end{equation}


% \begin{table}[h!]
% \caption{Comparison of state-of-the-art alignment approaches with our \method in terms of hyperparameters, the number of hyperparameters, and the elimination of the reference model.}
% \centering
% \vspace{-0.5em}
% \adjustbox{max width=\textwidth}{
% \begin{tabular}{cccc}
%    \toprule[1.0pt]
% \textbf{Approach}    &  \textbf{ Hyperparameters} & \textbf{\#Hyperparameters}   & \textbf{w/o Reference Model}   \\
% \midrule
% DPO & $\beta$ & $\mathbf{1}$   &  \xmark      \\
% IPO  & $\beta$ & $\mathbf{1}$  &  \xmark    \\
% CPO  & $\lambda, \beta$  & $\mathbf{2}$ & \cmark    \\
% SLiC  & $\delta, \lambda$ & $\mathbf{2}$  &  \cmark    \\
% SimPO  & $\gamma, \beta$ & $\mathbf{2}$  &  \cmark    \\
% KTO & $\lambda_l, \lambda_w, \beta$ & $\mathbf{3}$  &  \xmark   \\
% \midrule
% \method & - & $\mathbf{0}$ &   \cmark   \\
% \toprule[1.0pt]
% \end{tabular}}\label{table:compare_full}
% \end{table}


\subsection{Implementation Details}

\paragraph{Training Hyperparameters.}
For general hyperparameters, we adhered strictly to the settings used in SimPO. 
We applied the following hyperparameters: For the SFT stage, we use a learning rate of $2 \times 10^{-5}$. For both the SFT and the preference optimization stages, we use a batch size of 128, a maximum sequence length of 2048, and a cosine learning rate schedule with 10\% warmup steps for one epoch, all through the Adam optimizer~\citep{kingma2014adam}.
We maintain these settings consistently to ensure uniformity and comparability across experiments.

For method-specific hyperparameters, we also followed the search strategy from SimPO, noting that our method does not require any additional hyperparameters beyond the learning rate. Specifically, for different baseline methods, each with its own set of additional hyperparameters, the search strategy is in \cref{table_baseline_hyper}. 
Each method is individually search for the learning rates in $[3e-7, 5e-7, 6e-7, 1e-6]$.  

\begin{table}[h]
\small
\centering
\caption{Search ranges for hyperparameters across baseline methods}
\begin{tabular}{cc}
\toprule
\textbf{Method} & \textbf{Hyperparameter Search Range} \\
\midrule
DPO & $\beta \in [0.01, 0.05, 0.1]$ \\
\midrule
IPO & $\tau \in [0.01, 0.1, 0.5, 1.0]$ \\
\midrule
\multirow{2}{*}{CPO} 
& $\lambda = 1.0$ \\
& $\beta \in [0.01, 0.05, 0.1]$ \\
\midrule
\multirow{2}{*}{SLiC} 
& $\lambda \in [0.1, 0.5, 1.0, 10.0]$ \\
& $\delta \in [0.1, 0.5, 1.0, 2.0]$ \\
\midrule
\multirow{2}{*}{KTO} 
& $\lambda_l = \lambda_w = 1.0$ \\
& $\beta \in [0.01, 0.05, 0.1]$ \\
\midrule
\multirow{2}{*}{SimPO} 
& $\beta \in [2.0, 2.5]$ \\
& $\gamma \in [0.3, 0.5, 1.0, 1.2, 1.4, 1.6]$ \\
\bottomrule
\end{tabular}
\label{table_baseline_hyper}
\end{table}


\paragraph{Evaluation Hyperparameters.}
The hyperparameters for evaluation on AlpacaEval 2 adhere to SimPO, with a temperature of 0.7 for the Mistral-Base setting, a temperature of 0.5 for the Mistral-Instruct setting, and a temperature of 0.9 for both Llama3 settings. The hyperparameters for evaluation on MT-Bench are set according to the official decoding configuration, which specifies distinct sampling temperatures for various categories.

\paragraph{Computation Resources.} 
All the training experiments in this paper were conducted on 4 $\times$ NVIDIA A100 (80G) GPUs with 128 batchsize, based on the alignment-handbook repo.\footnote{\url{https://github.com/huggingface/alignment-handbook}}


% \section{Additional Experimental Results}


% \begin{figure}[t!]
% \centering 
% \includegraphics[width=1\textwidth]{pic/rewards-appendix.pdf}
% \caption{The training dynamics during training of \method and \texttt{SimPO} with  different hyperparameters on the 4 different base models. We can observe  that \method exhibits the least decline in chosen rewards, while still achieving the most significant increase in reward margins, compared to \texttt{SimPO} across various hyperparameters.}
% \vskip -1em
% \label{fig:app:rewards} 
% \end{figure}

\section{Case Study}
\label{app:evaluation}
% \textcolor{red}{Warning: The following case studies contain example data that may be offensive or harmful.}

In this section, we present examples of comparisons between \method  and the baseline \texttt{SimPO}. 
% See~\cref{tab:case-1} for dialogue examples, and~\ref{tab:case-2} for summarization examples.





\begin{table}[!h]
    \centering
    \ttfamily
    \small
        \begin{tabular}{lp{11.5cm}}
        \toprule[1pt]
        \bf{Prompt} & \texttt{Does the 5 second rule actually exist?}\\
        \midrule
        \bf{SimPER} & 
\texttt{The "5 second rule" is an urban legend that claims that if food falls on the floor, it is still safe to eat as long as you pick it up within 5 seconds. However, this rule is not a scientifically proven fact, and there is no evidence to support it. Here are some reasons why the 5-second rule is not a reliable method for determining food safety:}\newline


\textbf{1. Bacteria and viruses can transfer quickly}: Bacteria and viruses can transfer from surfaces to food within seconds. The amount of time that passes between a food item's contact with a contaminated surface and its being picked up can affect the likelihood of bacterial or viral transmission. \newline

\textbf{2. Contamination levels on floors and surfaces}: The level of contamination on floors and surfaces can vary widely depending on the environment, the cleanliness of the area, and the presence of moisture. Even if a surface is dry, bacteria and viruses can still be present.\newline

\textbf{3. Personal hygiene}: The cleanliness of your hands, the surface you place the food on, and the utensils you use to pick up the food all play a role in food safety. Picking up food within 5 seconds does not ensure that it is clean enough to eat.\newline

\textbf{4. Temperature}: The temperature of food can also affect the risk of bacterial or viral contamination. Food that has been at room temperature for a prolonged time, even if picked up quickly, may still harbor dangerous bacteria.\newline

Therefore, it is best to follow proper food handling and preparation practices, such as washing hands before and after handling food, using clean utensils and surfaces, and cooking food to the appropriate temperatures to ensure its safety. \\
        \midrule
        \bf{SimPO} & 
The "5-second rule" refers to a popular myth or belief that suggests touching food on the ground for less than five seconds makes it safe to consume without contamination. However, this rule does not exist scientifically.
According to food safety guidelines, bacteria can transfer from contaminated surfaces to food within seconds, regardless of the elapsed time. The duration of contact between the food and contaminated surfaces plays a crucial role in determining the likelihood of bacterial transfer, but it's not limited to five seconds.
Cross-contamination occurs when bacteria from unwashed hands, dirty utensils, or contaminated surfaces come into contact with food, leading to potential foodborne illnesses. Therefore, adhering to proper hygiene practices, such as washing hands, cleaning surfaces, and cooking food thoroughly, remains essential to prevent food contamination.
In summary, there is no scientific basis for the "5-second rule," and relying solely on this belief could increase the risk of consuming contaminated food."
        \\
        % \midrule
        % \textbf{Judgment} & \\
        \bottomrule[1pt]
    \end{tabular}
    \caption{An AlpacaEval 2 generation from the Llama-3-8B-Base model trained with \texttt{SimPER} and \texttt{SimPO}. Alpaca-Eval-GPT4-Turbo prefers \method over \texttt{SimPO}. Compared to \texttt{SimPO}, the generation by \texttt{SimPER} is better structured, with hierarchical discussions that make it more helpful. 
    }
    \label{tab:case-1}
\end{table}



\begin{table}[!h]
    \centering
        \ttfamily
        \small
    \begin{tabular}{lp{11.5cm}}
        \toprule[1pt]
        \bf{Prompt} & How do I make escargot? \\
        \midrule
        \bf{SimPER} &  
Here's a general recipe for making escargot: \newline

\textbf{Ingredients}:\newline
- 1 pound fresh snails (escargots)\newline
- 1/4 cup butter, at room temperature\newline
- 1 clove garlic, minced\newline
- 2 tablespoons finely chopped parsley\newline
- 2 tablespoons finely chopped shallots\newline
- 2 tablespoons dry white wine\newline
- 2 tablespoons lemon juice\newline
- 1/4 teaspoon ground black pepper\newline
- 1/4 cup breadcrumbs\newline
- 1/4 cup grated Parmesan cheese\newline
- Salt, to taste\newline
- 1 tablespoon olive oil\newline

\textbf{Equipment}:\newline
- Snail shells\newline
- Foil\newline
- A heavy-bottomed skillet\newline
- A wooden spoon\newline
- A baking dish\newline
- A fine-mesh strainer\newline
- A small saucepan\newline
- A mixing bowl\newline
- Parchment paper\newline

\textbf{Instructions}:\newline
1. Preheat the oven to 375F (190C).

2. Scrub the snails under cool running water to remove any dirt or debris. Discard any snails that are not alive. Soak the snails in salted water for at least 30 minutes, changing the water every 10 minutes, to ensure they are thoroughly cleaned.

3. In a small saucepan, melt the butter over low heat. Add the minced garlic, parsley, and shallots, and cook for 1-2 minutes until fragrant.

4. Remove the garlic mixture from the heat and stir in the white wine, lemon juice, and black pepper. Season with salt to taste.

5. In a mixing bowl, combine the breadcrumbs and grated Parmesan cheese.

6. Place the snail shells on a sheet of foil and arrange them in a baking dish. Spoon the garlic butter mixture into the shells, filling each one about 3\/4 full.

7. Sprinkle the breadcrumb mixture over the top of each snail.

8. Cover the snail shells with foil.

9. Bake in the preheated oven for 12-15 minutes or until the breadcrumbs are golden brown and the snails have cooked through.

10. Remove the dish from the oven, carefully remove the foil, and place a small piece of parchment paper over the top of each shell.

11. Return the dish to the oven and continue baking for 1-2 minutes to crisp up the breadcrumbs.

12. Remove the dish from the oven, and let the escargots cool for a few minutes.

13. In a heavy-bottomed skillet, heat the olive oil over medium-high heat. Strain the garlic butter mixture through a fine-mesh strainer into the skillet, and cook for 1-2 minutes until slightly reduced and thickened.

14. To serve, spoon the garlic butter onto small plates, arrange the escargots on the plates, and drizzle the remaining garlic butter over the top.

Note: Escargot shells are available for purchase at specialty food stores or online. If you don't have escargot shells, you can use mushroom caps or even small ramekins.

Enjoy your delicious escargot! \\
\midrule
\end{tabular}
\end{table}
\begin{table}[!h]
    \centering
        \ttfamily
        \small
    \begin{tabular}{lp{11.5cm}}
        \midrule
        \bf{SimPO} &
Here's a general recipe for making escargot:\newline

\textbf{Ingredients}:\newline
- 1 pound fresh or canned snails (escargots), cleaned and cooked\newline
- 1/2 cup unsalted butter, softened\newline
- 2 cloves garlic, minced\newline
- 1 tablespoon finely chopped fresh parsley\newline
- 1 tablespoon finely chopped fresh thyme\newline
- 1/4 teaspoon red pepper flakes (optional)\newline
- Salt and freshly ground black pepper, to taste\newline
- 24-30 small, hollowed-out garlic cloves or escargot shells\newline

\textbf{Instructions}:

1. Preheat your oven to 450F (230C).

2. Prepare the garlic cloves or escargot shells by gently scraping out the pulp with knife, leaving a thin layer intact. Rinse them under running water and pat dry.

3. In a mixing bowl, combine softened butter, minced garlic, chopped parsley, thyme, red pepper flakes (if using), salt, and black pepper. Mix well until blended.

4. Spoon a generous amount of seasoned butter into each garlic clove or escargot shell, filling them almost to the top.

5. Arrange the filled shells in a baking dish, preferably one with individual wells or indentations, spaced evenly apart.

6. Place the cooked snails into each shell, making sure they fit snugly.

7. Bake the escargot for 8-10 minutes, or until the butter has melted and bubbled around the edges, and the snails are heated through.

8. Remove the dish from the oven and serve immediately, accompanied by crusty bread, toasted baguette slices, or crackers.

9. To eat, use a small fork or escargot spoon to extract the snails from their shells, dip them into the remaining garlic butter, and enjoy!

Note: Always follow proper food handling and preparation guidelines when working with live or cooked snails, especially if using fresh ones. Ensure they are properly cleaned, cooked, and stored according to recommended safety practices. \\
\bottomrule[1pt]
    \end{tabular}
    % \vspace{4mm}
    \caption{An AlpacaEval 2 generation from the Mistral-7B-Base model trained with \texttt{SimPER} and \texttt{SimPO}. Alpaca-Eval-GPT4-Turbo prefers \method over \texttt{SimPO}. Compared to \texttt{SimPO}, the generation by \texttt{SimPER} is better structured, with hierarchical discussions that make it more helpful.}
    \label{tab:case-2}
\end{table}

\begin{table}[!h]
    \centering
        \ttfamily
        \small
    \begin{tabular}{lp{11.5cm}}
        \toprule[1pt]
        \bf{Prompt} & Mick pays his teacher \$800 for 40 lessons worth 2 hours each. If this will be all he is going to pay for his lessons, how much did he receive?\\
        \midrule
        \bf{SimPER} & Mick did not receive any money. The \$800 he paid to his teacher is the total cost for the 40 lessons, which is 40 x 2 hours = 80 total hours of lessons. Since he only paid for the lessons, he will not receive any money back.
        
        Confidence: 95\% \\
        \midrule
        \bf{SimPO} & Mick received \$800.
        
        Confidence: 95\%\\
        \bottomrule[1pt]
    \end{tabular}
    \caption{An AlpacaEval 2 generation from the Llama-3-8B-Base model trained with \texttt{SimPER} and \texttt{SimPO}. Alpaca-Eval-GPT4-Turbo prefers \method over \texttt{SimPO}. Compared to \texttt{SimPO}, \texttt{SimPER}'s output provides clearer step-by-step reasoning.}
    \label{tab:case-3}
\end{table}




\end{document}
