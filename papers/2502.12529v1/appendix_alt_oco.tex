\section{Omitted Details in \pref{sec:SC}}\label{app: alt-oco}

\begin{lemma}[Conjugate Duality]\label{lem:conj_dual}
    Suppose that $\psi:\R^d\mapsto\R$ is differentiable and strictly convex. Let $\psi^*(w)=\sup_{x\in\R^d}\left(\inner{x,w}-\psi(x)\right)$. Then, for any $x_1,x_2\in\R^d$, we have $D_{\psi}(x_1,x_2) = D_{\psi^*}(\nabla\psi(x_2),\nabla\psi(x_1))$.
\end{lemma}
\begin{proof}
    Direct calculation shows that
    \begin{align*}
        D_{\psi}(x_1,x_2) &= \psi(x_1) - \psi(x_2) - \inner{\nabla\psi(x_2),x_1-x_2} \\
        &= \psi(x_1) - \psi(x_2) - \inner{\nabla\psi(x_2),x_1-x_2} + \inner{x_1,\nabla\psi(x_1)} - \inner{x_1,\nabla\psi(x_1)}\\
        &=(\inner{\nabla\psi(x_2),x_2}-\psi(x_2)) - (\inner{\nabla\psi(x_1),x_1}-\psi(x_1)) - \inner{x_1,\nabla\psi(x_2) - \nabla\psi(x_1)} \\
        &=D_{\psi^*}(\nabla\psi(x_2),\nabla\psi(x_1)),
    \end{align*}
    where the last equality is because $\nabla\psi^*(\nabla\psi(x))=x$ and $\psi(x)+\psi^*(\nabla\psi(x))=\inner{x,\nabla\psi(x)}$.
\end{proof}

\begin{lemma}[Stability of FTRL]\label{lem:path_length}
    \pref{alg:OCO} ensures
    $
        \norm{x_t-x_{t+1}}_2 \le \frac{\eta}{\sigma}\norm{\nabla f_t(x_t)}_2,
    $
    for all $t$.
\end{lemma}

\begin{proof}
    Let $F_{t+1}(x)\triangleq\sum_{\tau=1}^tf_\tau(x)+\frac{1}{\eta}\psi(x)$. Since $x_{t+1}$ is the minimizer of $F_{t+1}$ in $\calX$, using first-order optimality, we have $F_{t+1}(x_{t+1})\le F_{t+1}(x)-D_{F_{t+1}}(x,x_{t+1})$ for all $x\in\calX$. Picking $x=x_{t}$, we get
    \begin{equation}\label{eqn:bregman_1}
        F_{t+1}(x_{t+1}) - F_t(x_t) \le f_t(x_t) - D_{F_{t+1}}(x_t,x_{t+1}).
    \end{equation}
    Similarly, for $F_t$, we have
    \begin{equation}\label{eqn:bregman_2}
        F_t(x_t) - F_{t+1}(x_{t+1}) \le -f_t(x_{t+1}) - D_{F_t}(x_{t+1},x_t).
    \end{equation}
    Summing up \pref{eqn:bregman_1} and \pref{eqn:bregman_2}, we can obtain that
    \begin{align*}
        f_t(x_t)-f_t(x_{t+1}) \ge D_{F_{t+1}}(x_t,x_{t+1}) + D_{F_t}(x_{t+1},x_t)\ge \frac{\sigma}{\eta}\norm{x_t-x_{t+1}}_2^2,
    \end{align*}
    where the last inequality is because $\psi$ is $\sigma$-strongly convex with respect to $\|\cdot\|_2$ within domain $\calX$, implying that both $F_t$ and $F_{t+1}$ are $\frac{\sigma}{\eta}$-strongly convex.
    Using the convexity of $f_t$, we further have $f_t(x_t)-f_t(x_{t+1})\le \innerp{\nabla f_t(x_t)}{x_t-x_{t+1}}\le \norm{\nabla f_t(x_t)}_2\cdot \norm{x_t-x_{t+1}}_2$. Combining these two inequalities, we get $\norm{x_t-x_{t+1}}_2 \le \frac{\eta}{\sigma}\norm{\nabla f_t(x_t)}_2$.
    \end{proof}

The rest of the section includes the omitted proofs for several lemmas.
\begin{proof}[Proof of \pref{lem:entropic}]
    Given any $\theta\in\R^d$, define $p_{\theta}(x) = \exp\rbr{\innerp{\theta}{x}-\psi^*(x)}\mathbbm{1}\cbr{x\in\calX}$ to be the exponential distribution over $\calX$ with parameter $\theta$. Then, according to Lemma 1 of \citet{bubeck2014entropic}, we know that
\begin{align*}
    &\nabla^3\psi^*(\theta) 
    = \E_{X\sim p_\theta}\sbr{\rbr{X-x(\theta)}\otimes\rbr{X-x(\theta)}\otimes\rbr{X-x(\theta)}},\\
    &\nabla^2\psi(x)=\rbr{\E_{X\sim p_\theta}\sbr{\rbr{X-x}\rbr{X-x}^\T}}^{-1},
\end{align*}
where $x(\theta)=\E_{X\sim p_{\theta}}\sbr{X}$. 
Therefore, for any $h\in\R^d$ such that $\|h\|_2\leq 1$, we know that
\begin{align*}
    \sum_{i=1}^d\sum^d_{j=1}\sum^d_{k=1}\frac{\partial^3\psi^*(\theta)}{\partial \theta_i\partial \theta_j\partial \theta_k}h_ih_jh_k \leq \max_{x,y\in\calX}|\inner{h,x-y}|^3\leq D^3,
\end{align*}
where $D$ is the diameter of $\calX$. \\In addition, since the maximum eigenvalue of $\nabla^{-2}\psi(x)=\E_{X\sim p_\theta}\sbr{\rbr{X-x}\rbr{X-x}^\T}$ is upper bounded by $D^2$, we know that $\nabla^2\psi(x)$ has minimum eigenvalue lower bounded by $1/D^2$. Combining both arguments, we see that $\psi$ satisfies \pref{asp: psi} with $M=D^3$ and $\sigma=1/D^2$. 
\end{proof}


\begin{proof}[Proof of \pref{lem:smoothness of ball case}]
    We can see that $\psi(x)=-\ln\rbr{1-\norm{x}_2^2}$ is a Legendre function. Now, we show that it is $2$-strongly convex with respect to the $\ell_2$-norm.
    \begin{align*}
        \nabla\psi(x) &= \frac{2x}{1-\norm{x}_2^2},\\
        \nabla^2\psi(x) &= \frac{2\mathbf{I}}{1-\norm{x}_2^2}+\frac{4xx^\T}{\rbr{1-\norm{x}_2^2}^2},
    \end{align*}
    where $\mathbf{I}$ represents the $d$-dimensional identity matrix.
    Thus, for any $h\in\fR^d$ and $\norm{h}_2=1$, we have
    \begin{align*}
        \nabla^2\psi(x)[h,h] = \frac{2\norm{h}_2^2}{1-\norm{x}_2^2}+\frac{4\rbr{\innerp{x}{h}}^2}{\rbr{1-\norm{x}_2^2}^2}\ge 2.
    \end{align*}
    It remains to show that $\psi^*$ is $4$-smooth of order $3$ with respect to the $\ell_2$-norm. Below, we calculate the derivatives of $\psi^*$:
    \begin{align*}
    \nabla \psi^*(w)
    &=\frac{w}{1+\sqrt{1+\norm{w}_2^2}},\\
    \nabla^2 \psi^*(w)
    &=\frac{I}{1+\sqrt{1+\norm{w}_2^2}}-\frac{ww^\top}{\rbr{1+\sqrt{1+\norm{w}_2^2}}^2\cdot \sqrt{1+\norm{w}_2^2}}.
    \end{align*}
    Denoting $g(w)=\sqrt{1+\norm{w}_2^2}$, we have
    \begin{align*}
    \nabla^2 \psi^*(w)
    =\frac{\mathbf{I}}{1+g(w)}-\frac{ww^\top}{\rbr{1+g(w)}^2\cdot g(w)}.
    \end{align*}
    Then, we have
    \begin{align*}
    \nabla^3 \psi^*(w)&= -\frac{\textrm{Sym}(\mathbf{I}\otimes w)}{g(w)\rbr{1+g(w)}^2} 
    + \frac{\rbr{1+3g(w)}w\otimes w\otimes w}{\rbr{1+g(w)}^3 g^3(w)}.
    \end{align*}
    Here, $\textrm{Sym}(\mathbf{I}\otimes w)$ denotes symmetrization over all permutations of the indices of the tensor product $I\otimes w$. 
    Then, we check the condition in \pref{def:3rd-order smoothness} for any $h\in\fR^d$ and $\norm{h}_2=1$.
    \begin{align*}
    &\sum_{i=1}^d\sum^d_{j=1}\sum^d_{k=1}\frac{\partial^3\psi(w)}{\partial w_i\partial w_j\partial w_k}h_ih_jh_k \\& =\sum_{i=1}^d\sum^d_{j=1}\sum^d_{k=1} -\frac{\rbr{\delta_{ij}w_k+\delta_{ik}w_j+\delta_{jk}w_i}h_ih_jh_k}{g(w)\rbr{1+g(w)}^2}+\frac{\rbr{1+3g(w)}w_ih_i w_jh_j w_kh_k}{g^3(w)\rbr{1+g(w)}^3}\\
    &= -\frac{3\sum_{i=1}^d w_ih_i\cdot \sum_{k=1}^d h_k^2 }{g(w)\rbr{1+g(w)}^2}+\frac{\rbr{1+3g(w)}\rbr{\sum_{i=1}^d w_ih_i}^3}{g^3(w)\rbr{1+g(w)}^3}.
    \end{align*}
    By Cauchy-Schwarz inequality, we have $\abs{\innerp{w}{h}}\le \norm{w}_2\cdot  \norm{h}_2$. Combining with the fact that $\norm{h}_2=1$,
    it holds that
    \begin{align*}
    &\sum_{i=1}^d\sum^d_{j=1}\sum^d_{k=1}\frac{\partial^3\psi(w)}{\partial w_i\partial w_j\partial w_k}h_ih_jh_k\\ &\le \frac{3 \norm{w}_2 }{g(w)\rbr{1+g(w)}^2}+\frac{\rbr{1+3g(w)}{\norm{w}_2^3}}{g^3(w)\rbr{1+g(w)}^3}.
    \end{align*}
    Recall that $g(w)=\sqrt{1+\norm{w}_2^2}$, it holds that $g(w)\ge \max(1,\norm{w}_2)$. Therefore, we have
    \begin{align*}
    \sum_{i=1}^d\sum^d_{j=1}\sum^d_{k=1}\frac{\partial^3\psi(w)}{\partial w_i\partial w_j\partial w_k}h_ih_jh_k\le 4.
    \end{align*}
\end{proof}
\begin{proof}[proof of \pref{lem:negative entropy}]
    Direct calculation shows that the gradient and Hessian of $
        \psi(x)$ are as follows:
    \begin{align*}
    \frac{\partial \psi}{\partial x_j}&=\log x_j+1-\log(1-\sum_{i=1}^{d-1}x_i)-1=\log x_j-\log(1-\sum_{i=1}^{d-1}x_i),\\
    \nabla^2\psi(x)&=\begin{bmatrix}
    \frac{1}{x_1}&&&\\
    &\frac{1}{x_2}&&\\
    &&\cdots&\\
    &&&\frac{1}{x_{d-1}}
    \end{bmatrix}+\frac{1}{1-\sum_{i=1}^{d-1}x_i}\cdot 
    \begin{bmatrix}
    1&1&\cdots&1\\
    1&1&\cdots&1\\
    \vdots&\vdots&\vdots&\vdots\\
    1&1&\cdots&1
    \end{bmatrix}\succeq \begin{bmatrix}
    \frac{1}{x_1}&&&\\
    &\frac{1}{x_2}&&\\
    &&\cdots&\\
    &&&\frac{1}{x_{d-1}}
    \end{bmatrix}.
    \end{align*}
    Since $x_i\leq 1$ for all $i\in[d-1]$, we have $\nabla^2\psi(x)\succeq I$, meaning that $\psi(x)$ is $1$-strongly convex with respect to $\|\cdot\|_2$.    

    Direct calculation also shows that $\psi^*(w)=\log\left(1+\sum_{i=1}^{d-1}\exp(w_i)\right)$ is the convex conjugate of $\psi(x)$. Define a distribution $p_w$ over $[d]$ where $p_w(i)=\exp(w_i-\psi^*(w))$ for $i\in [d-1]$ and $p_w(d)=\exp(-\psi^*(w))$. For any $v\in\R^{d-1}$, $\nabla^3\psi^*(w)[v,v,v]$ can be calculated as follows:
    \begin{align*}
        \nabla^3\psi^*(w)[v,v,v] = \E_{I\sim p_w}[(v_I-\bar{v})^3],
    \end{align*}
    where $v_d=0$ and $\bar{v}=\sum_{i=1}^{d-1}v_i\cdot p_w(i)$. Therefore, when $\|v\|_2=1$, we have $|v_I-\bar{v}|\leq 2$ for all $I\in[d]$ and 
    \[
    |\nabla^3\psi^*(w)[v,v,v]|\le \mathbb{E}_{I\sim p_w}[|v_I-\Bar{v}|^3]\le  2^3=8.
    \]
    This implies that $\psi^*$ is $8$-smooth of order $3$ with respect to $\ell_2$-norm.
\end{proof}


\begin{lemma}\label{lem:3rd-order-smooth-F}
    Let $F_t(x)\triangleq \sum_{\tau=1}^t f_\tau(x)+\frac{1}{\eta}\psi(x)$. If $\psi$ is Legendre, $\sigma$-strongly convex and its convex conjugate $\psi^*$ is $M$-smooth of order $3$ with respect to $\|\cdot\|_2$, and $f_\tau$ is $C$-self-concordant for all $\tau\in[t]$, then $F_t^*$ is $\rbr{\frac{2C\eta^{3/2}}{\sigma^{3/2}}+M\eta^2}$-smooth of order $3$ with respect to $\|\cdot\|_2$.
\end{lemma}

\begin{proof}
    By definition of convex conjugate, we know that
    \begin{align*}
            \nabla F_t^*(y) &= \rbr{\nabla F_t}^{-1}(y),\\
    \nabla^2 F_t^*(y) &= \rbr{\nabla^2 F_t(x)}^{-1},
    \end{align*}
    where $x = \rbr{\nabla F_t}^{-1}(y)$, the dual variable of $y$.
    To obtain the third derivative, we differentiate both sides of the equation above and obtain that:
    \[
    \nabla^3 F_t^*(y)[h, h, h] = - \nabla^3 F_t(x) \left[ (\nabla^2 F_t(x))^{-1} h, (\nabla^2 F_t(x))^{-1} h, (\nabla^2 F_t(x))^{-1} h \right], \forall~h\in\fR^d.
    \]
    We want to bound $\abs{\nabla^3 F_t^*(y)[h, h, h]}$ for all $h$. Taking $A=\frac{1}{\eta}\nabla^2 \psi(x)$ and $B=\nabla^2 \rbr{\sum_{\tau=1}^tf_\tau(x)}$, we have
    \begin{align*}
    &-\nabla^3 \frac{\psi(x)}{\eta}\left[(\nabla^2 F_t(x))^{-1} h, (\nabla^2 F_t(x))^{-1} h, (\nabla^2 F_t(x))^{-1} h\right] \\
    &= -\nabla^3 \frac{\psi(x)}{\eta}\left[A^{-1}A(A+B)^{-1} h, A^{-1}A(A+B)^{-1} h, A^{-1}A(A+B)^{-1} h\right] \\
    &= \nabla^3\rbr{\frac{\psi}{\eta}}^*(y)[A(A+B)^{-1}h,A(A+B)^{-1}h,A(A+B)^{-1}h] \\
    &\leq \eta^2 M\|A(A+B)^{-1}h\|_2^3\tag{by \pref{lem:psi-by-eta-smoothness}}\\&\leq \eta^2M\norm{h}_2^3.\tag{by \pref{lem:max_singular_val}}
    \end{align*}

    Since for all $\tau\in[t]$, $f_\tau$ is $C$-self-concordant, we have
    \begin{align*}
        &-\sum_{\tau=1}^t\nabla^3 f_\tau(x)\sbr{(\nabla^2 F_t(x))^{-1} h, (\nabla^2 F_t(x))^{-1} h, (\nabla^2 F_t(x))^{-1} h} \\
        &\le \sum_{\tau=1}^t2C\rbr{\nabla^2 f_\tau(x)\sbr{(\nabla^2 F_t(x))^{-1} h, (\nabla^2 F_t(x))^{-1} h}}^{3/2}\tag{by \pref{def:self-concordance}}\\
        &\le 2C\rbr{\sum_{\tau=1}^t\nabla^2f_\tau(x)\sbr{(\nabla^2 F_t(x))^{-1} h, (\nabla^2 F_t(x))^{-1} h}}^{3/2} \tag{since $\nabla^2f_\tau(x)$ is PSD}\\
        &= 2C\rbr{B\sbr{(A+B)^{-1}h,(A+B)^{-1}h}}^{3/2}\\
        &\le 2C\rbr{(A+B)\sbr{(A+B)^{-1}h,(A+B)^{-1}h}}^{3/2} \tag{since $A$ is PSD}\\
        &\le 2C\rbr{\frac{\eta}{\sigma}\norm{h}_2^2}^{3/2}\tag{since $A\succeq \frac{\sigma}{\eta} I$}\\
        &\le \frac{2C\eta^{3/2}}{\sigma^{3/2}} \norm{h}_2^3.
    \end{align*}
    Now we are ready to estimate the third-order smoothness of $F_t^*$. Expanding the third derivative of $F_t$ in terms of the derivatives of $f_\tau$ and $\frac{1}{\eta}\psi$, and bounding each of them, gives us the following:
    \begin{align*}
        \nabla^3 F_t^*(y)[h, h, h] &= - \nabla^3 F_t(x) \left[ (\nabla^2 F_t(x))^{-1} h, (\nabla^2 F_t(x))^{-1} h, (\nabla^2 F_t(x))^{-1} h \right]\\
        &= -\sum_{\tau=1}^t\nabla^3 f_\tau(x)\sbr{(\nabla^2 F_t(x))^{-1} h, (\nabla^2 F_t(x))^{-1} h, (\nabla^2 F_t(x))^{-1} h} \\&- \nabla^3\frac{1}{\eta}\psi(x)\sbr{(\nabla^2 F_t(x))^{-1} h, (\nabla^2 F_t(x))^{-1} h, (\nabla^2 F_t(x))^{-1} h}\\
        &\le \rbr{\frac{2C\eta^{3/2}}{\sigma^{3/2}}+M\eta^2}\norm{h}_2^3.
    \end{align*}
    
    Therefore, $F_t^*$ is $\rbr{\frac{2C\eta^{3/2}}{\sigma^{3/2}}+M\eta^2}$-smooth of order $3$ with respect to $\|\cdot\|_2$.
\end{proof}

\begin{lemma}\label{lem:psi-by-eta-smoothness}
   If $\psi^*$ is $M$-smooth of order $3$ with respect to $\|\cdot\|_2$, then the convex conjugate of $\frac{1}{\eta}\psi$ is $\eta^2M$-smooth of order $3$.
\end{lemma}
\begin{proof}
    By the definition of convex conjugate, we know that
    \[
    \rbr{\frac{1}{\eta}\psi}^*(y)=\sup_x \left(\langle y,x\rangle -\frac{1}{\eta}\psi(x)\right).
    \]
    Rearranging the terms, we can obtain that
    \[
    \eta\cdot\rbr{\frac{1}{\eta}\psi}^*(y)=\sup_x \left(\langle \eta y,x\rangle -\psi(x)\right)=\psi^*(\eta y).
    \]
    Taking the third derivative, we have
    \[
     \nabla^3\rbr{\frac{1}{\eta}\psi}^*(y)=\eta^2\nabla^3\psi(\eta y),
    \]
    which concludes the result in the lemma.
\end{proof}

\begin{lemma}\label{lem:max_singular_val}
    Assume that $A$ is a positive definite matrix, while $B$ is a positive semi-definite matrix. Then, it holds that the maximum singular value of $A(A+B)^{-1}$ is bounded by $1$.
\end{lemma}
\begin{proof}
    Let $C=A+B$, we have $C$ is a positive definite matrix and
    \[
    A(A+B)^{-1}=(C-B)C^{-1}=I-BC^{-1}.
    \]
    Since $C^{-1}$ is symmetric, there exists orthogonal matrix $M$ such that $M^{-1}C^{-1}M=D$, where $D$ is a diagonal matrix with positive diagonal entries. The singular value of $A(A+B)^{-1}$ is same as
    \[
    M^{-1}A(A+B)^{-1}M=I-B'D
    \]
    where $B'=M^{-1}BM$ is also symmetric. Since similarity transformation does not change the singular value, we only need to calculate the singular value of the following matrix:
    \[
    D^{\frac{1}{2}}(I-B'D)D^{-\frac{1}{2}}=I-D^\frac{1}{2}B'D^{\frac{1}{2}},
    \]
    where $D^\frac{1}{2}B'D^{\frac{1}{2}}$ is symmetric. Therefore, the singular value of $I-D^\frac{1}{2}B'D^{\frac{1}{2}}$ is bounded by $1$, which implies that the maximum singular value of $A(A+B)^{-1}$ is bounded by $1$.
\end{proof}