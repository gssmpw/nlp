\section{Preliminary for Self-Concordant Barrier}\label{app:scb}
\begin{definition}\label{def:scb}
    Let $\psi:\intO\to \fR$ be a $C^3$-smooth convex function. $\psi$ is called a self-concordant barrier on $\Omega$ if it satisfies: 
    \begin{itemize}
    \item $\psi(x_i)\to \infty$ as $i\to \infty$ for any sequence $x_1,x_2,\dots\in\intO\subset \fR^d$ converging to the boundary of $\Omega$;
    \item for all $w\in\intO$ and $h\in\fR^d$, the following inequality always holds:
    $$
    \sum^d_{i=1}\sum^d_{j=1}\sum^d_{k=1}\frac{\partial^3\psi(w)}{\partial w_i\partial w_j\partial w_k}h_ih_jh_k\le 2\|h\|_{\nabla^2\psi(w)}^3.
    $$
    
    \end{itemize}
    We further call $\psi$ is a $\nu$-self-concordant barrier if it satisfies the conditions above and also
    $$\inner{\nabla\psi(w),h}\le \sqrt{\nu}\|h\|_{\nabla^2\psi(w)}$$
    for all $w\in\intO$ and $h\in\fR^d$.
\end{definition}

To deal with the infinite range of the barrier $\psi$ on $\Omega$, we define a shrunk version of the decision domain $\Omega$ using Minkowsky functions.

\begin{definition}\label{def:minkowsky-functions}
    Define the Minkowsky function $\pi_w:\Omega\mapsto\fR$ associated with a point $w\in\intO$ and $\Omega$ as:
    \begin{equation*}
        \pi_w(u) = \inf\cbr{s>0~\Big|~w+\frac{u-w}{s}\in\Omega}.
    \end{equation*}
\end{definition}

\begin{lemma}[Proposition 2.3.2 in \citep{nesterov1994interior}]\label{lem:self_concordant}
    Let $\psi$ be a $\nu$-self-concordant barrier on $\Omega\in\fR^d$. Then, for any $u,w\in\intO$, we have
    \begin{equation*}
        \psi(u)-\psi(w)\le\nu\ln\rbr{\frac{1}{1-\pi_w(u)}}.
    \end{equation*}
    This means that for all $u\in\Omega'\triangleq\cbr{(1-\epsilon)x+\epsilon w:x\in\Omega}$ where $\epsilon>0$, we have
    \begin{equation*}
        \psi(u)-\psi(w)\le\nu\ln\rbr{1/\epsilon},
    \end{equation*}
    since $\pi_{w}(u)\leq 1-\epsilon$.
\end{lemma}



