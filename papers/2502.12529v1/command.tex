%!TEX root=main.tex
\newif\ifspacehack
%\spacehacktrue
\usepackage{natbib}
\usepackage{hyperref}
\hypersetup{
    colorlinks = blue,
    breaklinks,
    linkcolor = blue,
    citecolor = blue,
    urlcolor  = blue,
}
\usepackage{url} 
\usepackage{graphicx}
\usepackage{mathtools}
\usepackage{footnote}
\usepackage{float}
\usepackage{xspace}
\usepackage{multirow}
\usepackage{xcolor}
\usepackage{wrapfig}
\usepackage{framed}
\usepackage{bbm}
\usepackage[most]{tcolorbox}

\usepackage{footnote}
\usepackage{nicefrac}
\usepackage{makecell}
\usepackage{amssymb}
\usepackage{bm}
\makesavenoteenv{tabular}
\makesavenoteenv{table}

\newcommand{\hl}[1]{{\color{red}[HL: #1]}}
\newcommand{\mznote}[1]{{\color{blue}[MZ: #1]}}
\newcommand{\ping}[1]{{\color{pink}[Ping: #1]}}
\newcommand{\interior}{\ensuremath{\text{Int}}}
% macros@Peng
\newcommand\innerp[2]{\langle #1, #2 \rangle}
\renewcommand{\tilde}{\widetilde}
\renewcommand{\hat}{\widehat}


\newcommand{\TVD}[1]{\norm{#1}_\text{TV}}
\newcommand{\corral}{\textsc{Corral}\xspace}
\newcommand{\expthree}{\textsc{Exp3}\xspace}
\newcommand{\expfour}{\ensuremath{\mathsf{Exp4}}\xspace}
\newcommand{\expthreeP}{\textsc{Exp3.P}\xspace}
\newcommand{\scrible}{\textsc{SCRiBLe}\xspace}

\def \R {\mathbb{R}}
\newcommand{\eps}{\epsilon}
\newcommand{\vecc}{\mathrm{vec}}
\newcommand{\LS}{\mathrm{LS}}
\newcommand{\FG}{\mathrm{FG}}
\newcommand{\DL}{\Delta \ellhat}
\newcommand{\calA}{{\mathcal{A}}}
\newcommand{\smax}{{\mathrm{smax}}}
\newcommand{\calB}{{\mathcal{B}}}
\newcommand{\calX}{{\mathcal{X}}}
\newcommand{\calS}{{\mathcal{S}}}
\newcommand{\calF}{{\mathcal{F}}}
\newcommand{\calI}{{\mathcal{I}}}
\newcommand{\calJ}{{\mathcal{J}}}
\newcommand{\calK}{{\mathcal{K}}}
\newcommand{\calH}{{\mathcal{H}}}
\newcommand{\calD}{{\mathcal{D}}}
\newcommand{\calE}{{\mathcal{E}}}
\newcommand{\calG}{{\mathcal{G}}}
\newcommand{\calU}{{\mathcal{U}}}
\newcommand{\calR}{{\mathcal{R}}}
\newcommand{\calT}{{\mathcal{T}}}
\newcommand{\calP}{{\mathcal{P}}}
\newcommand{\calQ}{{\mathcal{Q}}}
\newcommand{\calZ}{{\mathcal{Z}}}
\newcommand{\calM}{{\mathcal{M}}}
\newcommand{\calN}{{\mathcal{N}}}
\newcommand{\calW}{{\mathcal{W}}}
\newcommand{\calY}{{\mathcal{Y}}}
\newcommand{\cD}{{\mathcal{D}_{\mathcal{X}}}}
\newcommand{\mcD}{{\mathcal{D}}}
\newcommand{\cF}{{\mathcal{F}}}
\newcommand{\cA}{{\mathcal{A}}}
\newcommand{\cX}{{\mathcal{X}}}
\newcommand{\cE}{{\mathcal{E}}}
\newcommand{\cV}{{\mathcal{V}}}
\newcommand{\cR}{{\mathcal{R}}}
\newcommand{\wcR}{\widehat{\mathcal{R}}}
\newcommand{\Reg}{{\mathrm{Reg}}}
\newcommand{\RegAlt}{\mathrm{Reg}_{\mathrm{Alt}}}
\newcommand{\RegCht}{{\mathrm{Reg}_{\mathrm{Cht}}}}
\newcommand{\Alg}{{\mathsf{Alg}}}
\newcommand{\wReg}{\widehat{\mathrm{Reg}}}
\newcommand{\cB}{\mathcal{B}}
\newcommand{\cP}{\mathcal{P}}

\newcommand{\E}{{\mathbb{E}}}
\newcommand{\V}{\mathbb{V}}
\newcommand{\Prob}{\mathbb{P}}
\newcommand{\1}{\mathbb{I}}
\newcommand{\N}{\mathbb{N}}
\newcommand{\tup}[1]{t^{(#1)}}
\newcommand{\gup}[1]{g^{(#1)}}
\newcommand{\hatfm}{\widehat{f}_m}
\newcommand{\haty}{\widehat{y}}
\newcommand{\hatx}{\widehat{x}}
\newcommand{\yhat}{\widehat{y}}
\newcommand{\xhat}{\widehat{x}}
\newcommand{\fhat}{\widehat{f}}
\newcommand{\ghat}{\widehat{g}}

\newcommand{\inner}[1]{ \left\langle {#1} \right\rangle }
\newcommand{\ind}{\mathbb{I}}
\newcommand{\diag}{\textrm{diag}}
\newcommand{\Nout}{N_{\textrm{out}}}
\newcommand{\nout}{N_{\textrm{out}}}
\newcommand{\Nin}{{\textrm{Nin}}}
\newcommand{\nin}{{\textrm{Nin}}}
\newcommand{\order}{\mathcal{O}}

\renewcommand{\ln}{\log}
\newcommand{\Acal}{\mathcal{A}}
\newcommand{\Bcal}{\mathcal{B}}
\newcommand{\Ccal}{\mathcal{C}}
\newcommand{\Dcal}{\mathcal{D}}
\newcommand{\Ecal}{\mathcal{E}}
\newcommand{\Fcal}{\mathcal{F}}
\newcommand{\Gcal}{\mathcal{G}}
\newcommand{\Hcal}{\mathcal{H}}
\newcommand{\Ical}{\mathcal{I}}
\newcommand{\Jcal}{\mathcal{J}}
\newcommand{\Kcal}{\mathcal{K}}
\newcommand{\Lcal}{\mathcal{L}}
\newcommand{\Mcal}{\mathcal{M}}
\newcommand{\Ncal}{\mathcal{N}}
\newcommand{\Ocal}{\mathcal{O}}
\newcommand{\Pcal}{\mathcal{P}}
\newcommand{\Qcal}{\mathcal{Q}}
\newcommand{\Rcal}{\mathcal{R}}
\newcommand{\Scal}{\mathcal{S}}
\newcommand{\Tcal}{\mathcal{T}}
\newcommand{\Ucal}{\mathcal{U}}
\newcommand{\Vcal}{\mathcal{V}}
\newcommand{\Wcal}{\mathcal{W}}
\newcommand{\Xcal}{\mathcal{X}}
\newcommand{\Ycal}{\mathcal{Y}}
\newcommand{\Zcal}{\mathcal{Z}}
\newcommand{\wkdn}{d}


\newcommand{\avgR}{\wh{\cal{R}}}
%\newcommand{\ips}{\wh{r}}
\newcommand{\whpi}{\wh{\pi}}
\newcommand{\whE}{\wh{\E}}
\newcommand{\whV}{\wh{V}}

\newcommand{\whReg}{\wh{\text{\rm Reg}}}
\newcommand{\flg}{\text{\rm flag}}
\newcommand{\one}{\boldsymbol{1}}
\newcommand{\var}{\Delta}
\newcommand{\Var}{\mathrm{Var}}
\newcommand{\bvar}{\bar{\Delta}}
\newcommand{\p}{\prime}
\newcommand{\evt}{\textsc{Event}}
\newcommand{\unif}{\text{\rm Unif}}
\newcommand{\KL}{\text{\rm KL}}
\newcommand{\Lstar}{{L^\star}}
\newcommand{\istar}{{i^\star}}
\newcommand{\dynreg}{\text{Dyn-Reg}}
\newcommand{\tildedynreg}{\widetilde{\text{Dyn-Reg}}}
\newcommand{\Bstar}{{B^\star}}
\newcommand{\Ustar}{\rho}
\newcommand{\Aconst}{a}
\newcommand{\dplus}[1]{\bm{#1}}
\newcommand{\lambdamax}{\lambda_\text{\rm max}}
\newcommand{\biasone}{\textsc{Deviation}\xspace}
\newcommand{\bias}{\textsc{Bias-1}\xspace}
\newcommand{\biastwo}{\textsc{Bias-2}\xspace}
\newcommand{\biasthree}{\textsc{Bias-3}\xspace}
\newcommand{\term}[1]{\texttt{Term} ~(#1)\xspace}
\newcommand{\x}{\mathbf{x}}
\newcommand{\errorterm}{\textsc{Error}\xspace}
\newcommand{\regterm}{\textsc{Reg-Term}\xspace}
\newcommand{\LTtilde}{\wt{L}_T}
\newcommand{\Bomega}{B_{\Omega}}
\newcommand{\UOB}{UOB-REPS\xspace}
\newcommand{\Holder}{{H{\"o}lder}\xspace}
\newcommand{\dpg}{\dplus{g}}
\newcommand{\dpM}{\dplus{M}}
\newcommand{\dpf}{\dplus{f}}
\newcommand{\dpX}{\dplus{\calX}}
\newcommand{\dpw}{\dplus{w}}
\newcommand{\dpF}{\dplus{F}}
\newcommand{\dpu}{\dplus{u}}
\newcommand{\dpwtilde}{\dplus{\wtilde}}
\newcommand{\dps}{\dplus{s}}
\newcommand{\dpe}{\dplus{e}}
\newcommand{\dpx}{\dplus{x}}
\newcommand{\dpy}{\dplus{y}}
\newcommand{\dpH}{\dplus{H}}
\newcommand{\dpOmega}{\dplus{\Omega}}
\newcommand{\dpellhat}{\dplus{\ellhat}}
\newcommand{\dpell}{\dplus{\ell}}
\newcommand{\dpr}{\dplus{r}}
\newcommand{\dpxi}{\dplus{\xi}}
\newcommand{\dpv}{\dplus{v}}
\newcommand{\dpI}{\dplus{I}}
\newcommand{\dpA}{\dplus{A}}
\newcommand{\dph}{\dplus{h}}
\newcommand{\cprob}{6}
\newcommand{\sigmoid}{\ensuremath{\mathsf{Sigmoid}}\xspace}
\newcommand{\relu}{\ensuremath{\mathsf{ReLU}}\xspace}

\DeclareMathOperator*{\argmin}{argmin}
\DeclareMathOperator*{\argmax}{argmax}
\DeclareMathOperator*{\argsmax}{argsmax}
%\DeclareMathOperator*{\range}{range}
%\DeclareMathOperator*{\mydet}{det_{+}}
%\DeclarePairedDelimiter\abs{\lvert}{\rvert}
%\DeclarePairedDelimiter\bigabs{\big\lvert}{\big\rvert}
\DeclarePairedDelimiter\ceil{\lceil}{\rceil}
%\DeclarePairedDelimiter\floor{\lfloor}{\rfloor}
%\DeclarePairedDelimiter\bigceil{\big\lceil}{\big\rceil}
%\DeclarePairedDelimiter\bigfloor{\big\lfloor}{\big\rfloor}

\newcommand{\field}[1]{\mathbb{#1}}
\newcommand{\fY}{\field{Y}}
\newcommand{\fX}{\field{X}}
\newcommand{\fH}{\field{H}}
\newcommand{\fR}{\field{R}}
\newcommand{\fN}{\field{N}}
\newcommand{\fS}{\field{S}}
\newcommand{\testblock}{\textsc{EndofBlockTest}\xspace}
\newcommand{\testreplay}{\textsc{EndofReplayTest}\xspace}

\newcommand{\theset}[2]{ \left\{ {#1} \,:\, {#2} \right\} }
% \newcommand{\inner}[1]{ \langle {#1} \rangle }
\newcommand{\inn}[1]{ \langle {#1} \rangle }
\newcommand{\Ind}[1]{ \field{I}_{\{{#1}\}} }
\newcommand{\eye}[1]{ \boldsymbol{I}_{#1} }
\newcommand{\norm}[1]{\left\|{#1}\right\|}
%\newcommand{\trace}[1]{\text{tr}\left({#1}\right)}
\newcommand{\trace}[1]{\textsc{tr}({#1})}


\newcommand{\defeq}{\stackrel{\rm def}{=}}
\newcommand{\sgn}{\mbox{\sc sgn}}
\newcommand{\scI}{\mathcal{I}}
\newcommand{\scO}{\mathcal{O}}
\newcommand{\scN}{\mathcal{N}}
\newcommand{\msmwc}{\textsc{MsMwC}}

\newcommand{\dt}{\displaystyle}
\renewcommand{\ss}{\subseteq}
\newcommand{\wh}{\widehat}
\newcommand{\wt}{\widetilde}
\newcommand{\wb}{\overline}
\newcommand{\ve}{\varepsilon}
\newcommand{\hlambda}{\wh{\lambda}}

\newcommand{\Jd}{J}
\newcommand{\ellhat}{\wh{\ell}}
\newcommand{\rhat}{\wh{r}}
\newcommand{\elltilde}{\wt{\ell}}
\newcommand{\wtilde}{\wt{w}}
\newcommand{\what}{\wh{w}}

\DeclareMathOperator{\conv}{conv}
\newcommand{\ellprime}{\ellhat^\prime}

\newcommand{\upconf}{\phi}

%\newcommand{\Ltilde}{\wt{L}}

\newcommand{\hDelta}{\wh{\Delta}}
\newcommand{\hdelta}{\wh{\delta}}
\newcommand{\spin}{\{-1,+1\}}

\newcommand{\ep}[1]{\E\!\left[#1\right]}
\newcommand{\LT}{L_T}
\newcommand{\LTbar}{\overline{L}_T}
\newcommand{\LTbarep}{\mathring{L}_T}
\newcommand{\circxhat}{\mathring{\xh}}
\newcommand{\circx}{\mathring{x}}
\newcommand{\circu}{\mathring{u}}
\newcommand{\circcalX}{\mathring{\calX}}
\newcommand{\circg}{\mathring{g}}
\newcommand{\Lubar}{\overline{L}_{u}}
%\newcommand{\Lustarbar}{\overline{L}_{u^\star}}

\newcommand{\Lyr}{J}
\newcommand{\QQ}{{w}}
\newcommand{\Qt}{{\QQ_t}}
\newcommand{\Qstar}{{u}}
\newcommand{\Qpistar}{{\Qstar^{\star}}}
\newcommand{\Qhat}{\wh{\QQ}}
\newcommand{\Ut}{{\upconf_t}}
\newcommand{\intO}{\mathrm{int}(\Omega)}
\newcommand{\intK}{\mathrm{int}(K)}

\newcommand{\squareCB}{\ensuremath{\mathsf{SquareCB}}\xspace}
\newcommand{\feelgood}{\ensuremath{\mathsf{FGTS}}\xspace}
\newcommand{\graphCB}{\ensuremath{\mathsf{SquareCB.G}}\xspace}
\newcommand{\squareCBAuc}{\ensuremath{\mathsf{SquareCB.A}}\xspace}
\newcommand{\AlgSq}{\ensuremath{\mathsf{AlgSq}}\xspace}
\newcommand{\AlgLog}{\ensuremath{\mathsf{AlgLog}}\xspace}
\newcommand{\ips}{\ensuremath{\mathsf{(IPS)}}\xspace}
\newcommand{\optsq}{\ensuremath{\mathsf{(OptSq)}}\xspace}
\newcommand{\sq}{\ensuremath{\mathsf{(Sq)}}\xspace}
\newcommand{\dec}{\ensuremath{\mathsf{dec}_\gamma}\xspace}
\newcommand{\dectwo}{\ensuremath{\mathsf{dec}_{\gamma_1,\gamma_2}}\xspace}
%\newcommand{\theHalgorithm}{\arabic{algorithm}}

\newcommand{\context}{\text{ctx}}
\newcommand{\noncontext}{\text{n-ctx}}
%\newtheorem{remark}{Remark}
%\newtheorem{prop}{Proposition}
%\newtheorem{definition}{Definition}
\newtheorem{event}{Event}
%\newtheorem*{main}{Main Result}
% \newtheorem{fact}[theorem]{Fact}

\newcommand{\paren}[1]{\left({#1}\right)}
\newcommand{\brackets}[1]{\left[{#1}\right]}
\newcommand{\braces}[1]{\left\{{#1}\right\}}

\newcommand{\normt}[1]{\norm{#1}_{t}}
\newcommand{\dualnormt}[1]{\norm{#1}_{t,*}}

\newcommand{\otil}{\ensuremath{\tilde{\mathcal{O}}}}


%%%%  brackets
\newcommand{\rbr}[1]{\left(#1\right)}
\newcommand{\sbr}[1]{\left[#1\right]}
\newcommand{\cbr}[1]{\left\{#1\right\}}
\newcommand{\nbr}[1]{\left\|#1\right\|}
\newcommand{\abr}[1]{\left|#1\right|}

\usepackage{lipsum,booktabs}
\usepackage{amsmath,mathrsfs,amssymb,amsfonts,bm,enumitem}
\usepackage{rotating}
\usepackage{pdflscape}
\usepackage{hyperref,url}
\hypersetup{
    colorlinks,
    breaklinks,
    linkcolor = blue,
    citecolor = blue,
    urlcolor  = blue,
}
\allowdisplaybreaks
\usepackage{appendix}
\usepackage{multirow,makecell}

%\usepackage{algorithmic,algorithm}
%\renewcommand{\algorithmicrequire}{ \textbf{Input:}}
%\renewcommand{\algorithmicensure}{ \textbf{Output:}}

\renewcommand{\tilde}{\widetilde}
\renewcommand{\hat}{\widehat}
\newcommand{\obs}{O}
\newcommand{\unobs}{E}
\newcommand{\unbiasSize}{c}
\newcommand{\unbias}{C}
\newcommand{\cnt}{k}

% define some macros
\def \A {\mathcal{A}}

\def \B {\mathbb{B}}
\def \B {\mathcal{B}}
\def \C {\mathcal{C}}
\def \D {\mathcal{D}}
\def \E {\mathbb{E}}
\def \F {\mathcal{F}}
\def \G {\mathcal{G}}
\def \H {\mathcal{H}}
\def \I {\mathcal{I}}
\def \J {\mathcal{J}}
\def \K {\mathcal{K}}
\def \L {\mathcal{L}}
\def \M {\mathcal{M}}
\def \N {\mathcal{N}}
\def \O {\mathcal{O}}
\def \P {\mathcal{P}}
\def \Q {\mathcal{Q}}
\def \R {\mathbb{R}}
\def \S {\mathcal{S}}
% \def \T {\mathrm{T}}
\def \T {\top}
\def \U {\mathcal{U}}
\def \V {\mathcal{V}}
\def \W {\mathcal{W}}
\def \X {\mathcal{X}}
\def \Y {\mathcal{Y}}
\def \Z {\mathcal{Z}}

\def \a {\mathbf{a}}
\def \b {\mathbf{b}}
\def \c {\mathbf{c}}
\def \d {\mathbf{d}}
\def \e {\mathbf{e}}
\def \f {\mathbf{f}}
\def \g {\mathbf{g}}
\def \h {\mathbf{h}}
\def \m {\mathbf{m}}
\def \p {\mathbf{p}}
\def \q {\mathbf{q}}
\def \u {\mathbf{u}}
\def \w {\mathbf{w}}
\def \s {\mathbf{s}}
\def \t {\mathbf{t}}
% \def \v {\mathbf{v}}
\def \x {\mathbf{x}}
\def \y {y}
\def \z {\mathbf{z}}

\def \ph {\hat{p}}

\def \fh {\hat{f}}
\def \fb {\bar{f}}
\def \ft{\tilde{f}}

\def \gh {\hat{\g}}
\def \gb {\bar{\g}}
\def \gt {\tilde{g}}

\def \uh {\hat{\u}}
\def \ub {\bar{\u}}
\def \ut{\tilde{\u}}

\def \vh {\hat{\v}}
\def \vb {\bar{\v}}
\def \vt{\tilde{\v}}

\def \xh {\hat{x}}
\def \xb {\bar{\x}}
\def \xt {\tilde{\x}}

\def \zh {\hat{\z}}
\def \zb {\bar{\z}}
\def \zt {\tilde{\z}}

\def \Ecal {\mathcal{E}}
\def \Rcal {\mathcal{R}}
\def \Ot {\tilde{\O}}
\def \indicator {\mathds{1}}
\def \regret {\mbox{Regret}}
\def \proj {\mbox{Proj}}
\def \Pr {\mathsf{Pr}}
\def \ellb {\boldsymbol{\ell}}
\def \thetah {\hat{\theta}}

\newcommand{\RegSq}{\ensuremath{\mathrm{\mathbf{Reg}}_{\mathsf{Sq}}}\xspace}
\newcommand{\RegCB}{\ensuremath{\mathrm{\mathbf{Reg}}_{\mathsf{CB}}}\xspace}
\newcommand{\RegDyn}{\ensuremath{\mathrm{\mathbf{Reg}}_{\mathsf{Dyn}}}\xspace}
\usepackage{mathtools}
\let\oldnorm\norm   % <-- Store original \norm as \oldnorm
\let\norm\undefined % <-- "Undefine" \norm
\DeclarePairedDelimiter\norm{\lVert}{\rVert}
\DeclarePairedDelimiter\abs{\lvert}{\rvert}
%\newcommand\inner[2]{\langle #1, #2 \rangle}
\newcommand*\diff{\mathop{}\!\mathrm{d}}
\newcommand*\Diff[1]{\mathop{}\!\mathrm{d^#1}}

%\DeclareMathOperator*{\Reg}{Regret}
\DeclareMathOperator*{\AReg}{A-Regret}
\DeclareMathOperator*{\WAReg}{WA-Regret}
\DeclareMathOperator*{\SAReg}{SA-Regret}
\DeclareMathOperator*{\DReg}{\mbox{D-Regret}}
\DeclareMathOperator*{\poly}{poly}
%\DeclareMathOperator*{\argmax}{arg\,max}
%\DeclareMathOperator*{\argmin}{arg\,min}

% define new theorem environments
% \let\proof\relax
% \let\endproof\relax
% \newenvironment{proof}{\par\noindent{\bf Proof\ }}{\hfill\BlackBox\\[2mm]}
% \renewcommand\qedsymbol{$\blacksquare$}
\newtheorem{myThm}{Theorem}
\newtheorem{assumption}{Assumption}
\newtheorem{myFact}{Fact}
\newtheorem{myClaim}{Claim}
\newtheorem{myLemma}[myThm]{Lemma}
\newtheorem{myObservation}{Observation}
\newtheorem{myProp}[myThm]{Proposition}
\newtheorem{myProperty}{Property}

% Define a custom environment for prompts
\newtcolorbox{promptbox}[1][]{
  colback=blue!5!white, colframe=blue!75!black,
  fonttitle=\bfseries, title=Prompt,
  left=2mm, right=2mm, top=2mm, bottom=2mm,
  boxrule=0.5mm,  % Thickness of the frame
  coltitle=black, % Color of the title text
  colbacktitle=blue!15!white, % Background color of the title
  breakable,      % Allows the box to break across pages
  #1
}
\newtheorem{myAssum}{Assumption}
\newtheorem{myConj}{Conjecture}
\newtheorem{myCor}{Corollary}
\newtheorem{myDef}{Definition}
\newtheorem{myExample}{Example}
\newtheorem{myNote}{Note}
\newtheorem{myProblem}{Problem}

\newtheorem{myRemark}{Remark}

% add comments
\usepackage{graphicx,color} % more modern
\newcommand{\red}{\color{red}}
\newcommand{\blue}{\color{blue}}
\definecolor{wine_red}{RGB}{228,48,64}
\definecolor{DSgray}{cmyk}{0,1,0,0}
%\newcommand{\Authornote}[2]{{\small\textcolor{NavyBlue}{\sf$<<<${  #1: #2 }$>>>$}}}
% \newcommand{\Authormarginnote}[2]{\marginpar{\parbox{2cm}{\raggedright\tiny \textcolor{DSgray}{#1: #2}}}}
% \newcommand{\pnote}[1]{{\Authornote{Peng}{#1}}}
% \newcommand{\pmarginnote}[1]{{\Authormarginnote{Peng}{#1}}}

\usepackage{prettyref}
\newcommand{\pref}[1]{\prettyref{#1}}
\newcommand{\pfref}[1]{Proof of \prettyref{#1}}
\newcommand{\savehyperref}[2]{\texorpdfstring{\hyperref[#1]{#2}}{#2}}
\newrefformat{eq}{\savehyperref{#1}{Eq. \textup{(\ref*{#1})}}}
\newrefformat{eqn}{\savehyperref{#1}{Eq.~(\ref*{#1})}}
\newrefformat{lem}{\savehyperref{#1}{Lemma~\ref*{#1}}}
\newrefformat{event}{\savehyperref{#1}{Event~\ref*{#1}}}
\newrefformat{def}{\savehyperref{#1}{Definition~\ref*{#1}}}
\newrefformat{line}{\savehyperref{#1}{Line~\ref*{#1}}}
\newrefformat{thm}{\savehyperref{#1}{Theorem~\ref*{#1}}}
\newrefformat{tab}{\savehyperref{#1}{Table~\ref*{#1}}}
\newrefformat{corr}{\savehyperref{#1}{Corollary~\ref*{#1}}}
\newrefformat{cor}{\savehyperref{#1}{Corollary~\ref*{#1}}}
\newrefformat{sec}{\savehyperref{#1}{Section~\ref*{#1}}}
\newrefformat{app}{\savehyperref{#1}{Appendix~\ref*{#1}}}
\newrefformat{assum}{\savehyperref{#1}{Assumption~\ref*{#1}}}
\newrefformat{asm}{\savehyperref{#1}{Assumption~\ref*{#1}}}
\newrefformat{asp}{\savehyperref{#1}{Assumption~\ref*{#1}}}
% \newrefformat{fact}{\savehyperref{#1}{Fact~\ref*{#1}}}
\newrefformat{ex}{\savehyperref{#1}{Example~\ref*{#1}}}
\newrefformat{fig}{\savehyperref{#1}{Figure~\ref*{#1}}}
\newrefformat{alg}{\savehyperref{#1}{Algorithm~\ref*{#1}}}
\newrefformat{rem}{\savehyperref{#1}{Remark~\ref*{#1}}}
\newrefformat{conj}{\savehyperref{#1}{Conjecture~\ref*{#1}}}
\newrefformat{prop}{\savehyperref{#1}{Proposition~\ref*{#1}}}
\newrefformat{proto}{\savehyperref{#1}{Protocol~\ref*{#1}}}
\newrefformat{prob}{\savehyperref{#1}{Problem~\ref*{#1}}}
\newrefformat{claim}{\savehyperref{#1}{Claim~\ref*{#1}}}
\newrefformat{que}{\savehyperref{#1}{Question~\ref*{#1}}}
\newrefformat{op}{\savehyperref{#1}{Open Problem~\ref*{#1}}}
\newrefformat{fn}{\savehyperref{#1}{Footnote~\ref*{#1}}}

\def \p {\boldsymbol{p}}
\def \s {\boldsymbol{s}}
\def \m {\boldsymbol{m}}
\def \epsilon {\varepsilon}

% \def \base {\mathtt{base}\mbox{-}\mathtt{regret}}
% \def \meta {\mathtt{meta}\mbox{-}\mathtt{regret}}
\def \base {\textsc{base-regret}}
\def \meta {\textsc{meta-regret}}
\def \xref {\x_{\text{ref}}}
\def \fb {\bar{f}}
\def \interior {\text{int}}
\def \yh {\hat{\y}}
\def \RegLog {\Reg_{\log}^G}
\newcommand{\bra}[1]{\left[#1\right]}
\newcommand{\pa}[1]{\left(#1\right)}
\newcommand{\hhat}{\wh{h}}
\newcommand{\epsn}{\epsilon_N}
\newcommand{\rad}{\mathsf{rad}}
\newcommand{\hatr}{\wh{r}}
\newcommand{\fl}{\underline{f}^\star}
\newcommand{\PRM}{PRM$^+$~}