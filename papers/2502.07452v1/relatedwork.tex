\section{Related work}
\label{sec:related_worl}

In \cite{DBLP:conf/nmr/EspinozaNT23}, the authors introduce credal support argumentation frameworks where each argument is associated with a credal set and an imprecise base score obtained from its credal set. While they only consider a support relation, they show how to compute the imprecise strength of arguments (as an interval) and study  theoretical properties.
%
The epistemic approach to probabilistic argumentation \cite{hunter21probabilistic} aims to determine valid probabilities for arguments given some properties, similar to our intervals. In the context of fuzzy argumentation, legal argument weights can be computed according to the approach of \cite{DBLP:conf/comma/WuLON16}. However, in all these cases, the properties of the interval are not explicitly considered, and neither is correcting systems which do not comply with the underlying properties.
%
Enforcement in abstract argumentation (e.g., \cite{baumann:hal-03541704}) considers how argumentation frameworks can be modified to guarantee an argument's status. Our refinements can be viewed similarly in a gradual semantics context.