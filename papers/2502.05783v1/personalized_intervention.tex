%
% The first command in your LaTeX source must be the \documentclass command.
% \documentclass[acmlarge, anonymous]{acmart}
\documentclass[acmlarge]{acmart}
% \documentclass[acmlarge]{acmart}
\newcommand{\irale}{\textsc{iral}{\small e}}
\newcommand{\expo}{\textsc{expo}}
\newcommand{\rex}{\textsc{rex}}
\newcommand{\ilm}{\textsc{ilm}}
\newcommand{\glib}[1]{\textsc{glib{\small {#1}}}}
\newcommand{\glibg}[1]{\textsc{glib-g{\small {#1}}}}
\newcommand{\glibl}[1]{\textsc{glib-l{\small {#1}}}}
% Useful in captions / small text:
\newcommand{\glibgc}[1]{\textsc{glib-g\scriptsize{{#1}}}}
\newcommand{\gliblc}[1]{\textsc{glib-l\scriptsize{{#1}}}}
\newcommand{\zpk}{\textsc{lndr}}
\newcommand{\fail}{\text{\emph{fail}}}
\newcommand{\ofail}{\O_\fail}

\newcommand{\states}{\mathcal{X}}
\newcommand{\state}{x}
\newcommand{\staterv}{X}
\newcommand{\parameterizedskills}{\mathcal{U}}
\newcommand{\parameterizedskill}{u}
\newcommand{\actions}{\mathcal{A}}
\newcommand{\action}{a}
\newcommand{\actionrv}{A}
\newcommand{\goal}{g}
\newcommand{\goalrv}{G}
\newcommand{\horizoneval}{H_{\text{eval}}}
\newcommand{\horizonfree}{H_{\text{free}}}
\newcommand{\parameterprior}{\pi^0}
\newcommand{\parameterpolicy}{\pi}
\newcommand{\parameterpolicies}{\Pi}
\newcommand{\learnedparameterpolicy}{\pi^*}
\newcommand{\exploreparameterpolicy}{\pi^+}
\newcommand{\competencerv}{C}
\newcommand{\outcome}{S}


\newcommand{\F}{\mathcal{F}}
\newcommand{\I}{\mathcal{I}}
\newcommand{\G}{\mathcal{G}}
\renewcommand{\P}{\mathcal{P}}
\renewcommand{\O}{\mathcal{O}}
\newcommand{\Q}{\mathcal{Q}}
\renewcommand{\S}{\mathcal{S}} 
\newcommand{\A}{\mathcal{A}}
\newcommand{\T}{\mathcal{T}}
\newcommand{\Loss}{\mathcal{L}}
\newcommand{\Gr}{\mathcal{G}}
\newcommand{\V}{\mathcal{V}}
\newcommand{\E}{\mathcal{E}}
\newcommand{\tuple}[1]{\langle #1 \rangle}
\newcommand{\X}{\mathcal{X}}
\newcommand{\U}{\mathcal{U}}
\newcommand{\B}{\mathcal{B}}
\newcommand{\C}{\mathcal{C}}
\newcommand{\pomdp}{{\sc pomdp}}
\newcommand{\plan}{{\sc plan}}
\newcommand{\ploi}{{\sc ploi}}
\newcommand{\momdp}{{\sc momdp}}
\newcommand{\mdp}{{\sc mdp}}
\renewcommand{\dag}{{\sc dag}}
\newcommand{\Obs}{\Omega}
\newcommand{\indep}{\perp\!\!\!\perp}
\newcommand{\Ex}{\mathop{\mathbb{E}}}
\newcommand{\var}{\mathrm{Var}}
\newcommand{\cov}{\mathrm{Cov}}
\newcommand{\eps}{\epsilon}
\newcommand{\N}{\mathcal{N}}
\newcommand{\D}{\mathcal{D}}
\DeclareMathOperator{\plim}{plim}
\DeclareMathOperator{\rank}{Rank}
\newcommand{\tr}{\mathrm{Tr}}
\newcommand{\toprob}{\overset{p}{\to}}
\newcommand{\todist}{\overset{d}{\to}}
\DeclareMathOperator*{\argmin}{argmin}
\DeclareMathOperator*{\argmax}{argmax}
\DeclarePairedDelimiterX{\infdivx}[2]{(}{)}{%
  #1\;\delimsize\|\;#2%
}
\newcommand{\kl}{D_{\text{KL}}\infdivx}
\newcommand{\js}{D_{\text{JS}}\infdivx}
\newcommand{\objects}{\mathcal{O}}
\newcommand{\object}{o}
\newcommand{\predicates}{\Psi}
\newcommand{\predicate}{\psi}
\newcommand{\abstractfn}{\texttt{abstract}}
\newcommand{\termination}{\beta}
\newcommand{\successfn}{J}
\newcommand{\precondition}{I}
\newcommand{\abstractstate}{s}
\newcommand{\abstractstates}{\mathcal{S}}
\newcommand{\abstractstatespace}{\mathcal{S}_{\predicates}}
\newcommand{\skills}{\Pi}
\newcommand{\skill}{\pi}
\newcommand{\dataset}{\mathcal{D}}
\newcommand{\energyfn}{E}
\newcommand{\lowlevelpolicy}{\mu}

\newcommand{\frobs}{operators}
\newcommand{\Frobs}{Operators}

\newcommand{\types}{\Lambda}
\newcommand{\type}{\lambda}

\newcommand{\ground}{\underline}
\newcommand{\arguments}{\overline{v}}
\newcommand{\params}{\textsc{Par}}
\newcommand{\preconditions}{\textsc{Pre}}
\newcommand{\addeffects}{\textsc{Eff}^{+}}
\newcommand{\deleteeffects}{\textsc{Eff}^{-}}
\newcommand{\controllerspec}{\textsc{Con}}
\newcommand{\substitution}{\delta}
\newcommand{\operator}{\omega}
\newcommand{\operators}{\Omega}
\newcommand{\sampler}{\sigma}
\newcommand{\samplers}{\Sigma}
\newcommand{\competence}{c}
\newcommand{\paramscore}{\kappa}
\newcommand{\practicetimesteps}{n_p}
\newcommand{\practicenumepisodes}{e_p}
\newcommand{\skilloutcome}{y}
\newcommand{\episodesperlearning}{e_l}
\newcommand{\actionhistory}{h^{a}}
\newcommand{\cpm}{f_{\upsilon}}


\newcommand{\notimplies}{%
  \mathrel{{\ooalign{\hidewidth$\not\phantom{=}$\hidewidth\cr$\implies$}}}}
  
\DeclareMathOperator*{\Pre}{Pre}
\DeclareMathOperator*{\Act}{Act}
\DeclareMathOperator*{\Eff}{Eff}
\DeclareMathOperator*{\Inv}{Inv}

\newcommand\projectname{WatchGuardian\xspace}

% Rights management information. 
% This information is sent to you when you complete the rights form.
% These commands have SAMPLE values in them; it is your responsibility as an author to replace
% the commands and values with those provided to you when you complete the rights form.
%
% These commands are for a PROCEEDINGS abstract or paper.
% \copyrightyear{2018}
% \acmYear{2018}
% \setcopyright{acmlicensed}
% \acmConference[Woodstock '18]{Woodstock '18: ACM Symposium on Neural Gaze Detection}{June 03--05, 2018}{Woodstock, NY}
% \acmBooktitle{Woodstock '18: ACM Symposium on Neural Gaze Detection, June 03--05, 2018, Woodstock, NY}
% \acmPrice{15.00}
% \acmDOI{10.1145/1122445.1122456}
% \acmISBN{978-1-4503-9999-9/18/06}

%
% These commands are for a JOURNAL article.

% \setcopyright{rightsretained}
% \acmJournal{IMWUT}
% \acmYear{2023}
% \acmVolume{6} \acmNumber{4} \acmArticle{190} \acmMonth{12} \acmPrice{}\acmDOI{10.1145/XXXXXXX}

% \setcopyright{acmlicensed}
% \acmJournal{IMWUT}
% \acmYear{2024} \acmVolume{8} \acmNumber{1} \acmArticle{32} \acmMonth{3}\acmDOI{10.1145/3643540}

%
% Submission ID. 
% Use this when submitting an article to a sponsored event. You'll receive a unique submission ID from the organizers
% of the event, and this ID should be used as the parameter to this command.
%\acmSubmissionID{123-A56-BU3}

%
% The majority of ACM publications use numbered citations and references. If you are preparing content for an event
% sponsored by ACM SIGGRAPH, you must use the "author year" style of citations and references. Uncommenting
% the next command will enable that style.
%\citestyle{acmauthoryear}

%
% end of the preamble, start of the body of the document source.
\begin{document}

%
% The "title" command has an optional parameter, allowing the author to define a "short title" to be used in page headers.

\title{\projectname: Enabling User-Defined Personalized Just-in-Time Intervention on Smartwatch}

%
% The "author" command and its associated commands are used to define the authors and their affiliations.
% Of note is the shared affiliation of the first two authors, and the "authornote" and "authornotemark" commands
% used to denote shared contribution to the research.

\author{Ying Lei}
\authornote{Mark co-first authors with equal contribution.}
\orcid{0000-0001-8326-1369}
\affiliation{%
  \institution{Simon Fraser University}
  \country{Canada}
}

\author{Yancheng Cao}
\authornotemark[1]
\orcid{0000-0003-3033-8881}
\affiliation{%
  \institution{Columbia University}
  \country{USA}
}

\author{Will Ke Wang}
\orcid{0000-0003-1444-5468}
\affiliation{%
  \institution{Columbia University}
  \country{USA}
}

\author{Yuanzhe Dong}
\orcid{0009-0006-2013-1157}
\affiliation{%
  \institution{Stanford University}
  \country{USA}
}

\author{Changchang Yin}
\orcid{0000-0002-6540-6365}
\author{Weidan Cao} 
\orcid{0000-0001-5417-2121}
\author{Ping Zhang}
\orcid{0000-0002-4601-0779}
\affiliation{%
  \institution{The Ohio State University}
  \country{USA}
}

\author{Jingzhe Yang}
\orcid{0000-0003-4019-0999}
\affiliation{%
  \institution{Nationwide Children's Hospital}
  \country{USA}
}


\author{Bingsheng Yao}
\orcid{0009-0004-8329-4610}
\affiliation{%
  \institution{Northeastern University}
  \country{USA}
}

\author{Yifan Peng}
\orcid{0000-0001-9309-8331}
\affiliation{%
  \institution{Weill Cornell Medicine}
  \country{USA}
}

\author{Chunhua Weng}
\orcid{0000-0002-9624-0214}
\author{Randy Auerbach}
\orcid{0000-0003-2319-4744}
\author{Lena Mamykina}
\orcid{0000-0001-5203-274X}
\affiliation{%
  \institution{Columbia University}
  \country{USA}
}

\author{Dakuo Wang}
\authornote{Mark corresponding authors.}
\orcid{0000-0001-9371-9441}
\affiliation{%
  \institution{Northeastern University}
  \country{USA}
}

\author{Yuntao Wang}
\authornotemark[2]
\orcid{0000-0002-4249-8893}
\affiliation{%
  \institution{University of Washington}
  \country{USA}
}

\author{Xuhai Xu}
\authornotemark[2]
\email{xx2489@columbia.edu}
\orcid{0000-0001-5930-3899}
\affiliation{%
  \institution{Columbia University}
  \country{USA}
}


% Ying Lei *
% Yancheng Cao *
% Will Wang
% Yuanzhe Dong
% Changchang Yin
% Weidan Cao
% Ping Zhang
% Jingzhen (Ginger) Yang
% Bingsheng Yao
% Yifan Peng
% Chunhua Weng
% Randy Auerbach
% Lena Mamykina
% Dakuo Wang
% Yuntao Wang
% Xuhai Xu

%
% By default, the full list of authors will be used in the page headers. Often, this list is too long, and will overlap
% other information printed in the page headers. This command allows the author to define a more concise list
% of authors' names for this purpose.
\renewcommand{\shortauthors}{Ying \& Cao et al.}
\renewcommand{\shorttitle}{\projectname{}}

%
% The abstract is a short summary of the work to be presented in the article.
\begin{abstract}
\begin{abstract}
Out-of-distribution (OOD) detection and OOD generalization are widely studied in Deep Neural Networks (DNNs), yet their relationship remains poorly understood. We empirically show that the degree of Neural Collapse (NC) in a network layer is inversely related with these objectives: stronger NC improves OOD detection but degrades generalization, while weaker NC enhances generalization at the cost of detection. This trade-off suggests that a single feature space cannot simultaneously achieve both tasks. To address this, we develop a theoretical framework linking NC to OOD detection and generalization. We show that entropy regularization mitigates NC to improve generalization, while a fixed Simplex Equiangular Tight Frame (ETF) projector enforces NC for better detection. Based on these insights, we propose a method to control NC at different DNN layers. In experiments, our method excels at both tasks across OOD datasets and DNN architectures. 

\begin{comment}   

Out-of-distribution (OOD) detection and OOD generalization are critical for deploying machine learning models in real-world scenarios. While substantial progress has been made in addressing these problems independently, few works have attempted to tackle them jointly. However, existing methods often rely on auxiliary OOD training data and primarily focus on covariate-shifted OOD data that share labels with in-distribution (ID) data. In contrast, we tackle the more realistic and challenging task of jointly detecting and generalizing to semantic OOD data with disjoint labels from the ID data, without auxiliary OOD training data.
Achieving both objectives simultaneously is inherently difficult due to a fundamental conflict — OOD generalization requires enhanced transferability, while OOD detection necessitates the inhibition of transfer.
To address this, we leverage insights from neural collapse (NC) — a phenomenon in deep networks where top-layer representations suppress feature variability and adopt a Simplex Equiangular Tight Frame (ETF) structure, impairing transferability. By controlling NC, we unify OOD detection and generalization: preventing NC enhances OOD transfer while inducing NC improves OOD detection.
Our proposed method excels at both tasks across various OOD datasets and architectures. 

\end{comment}


\end{abstract}
\end{abstract}

%
% The code below is generated by the tool at http://dl.acm.org/ccs.cfm.
% Please copy and paste the code instead of the example below.
%
\begin{CCSXML}
<ccs2012>
<concept>
<concept_id>10003120.10003138</concept_id>
<concept_desc>Human-centered computing~Ubiquitous and mobile computing</concept_desc>
<concept_significance>500</concept_significance>
</concept>
<concept>
<concept_id>10010405.10010444</concept_id>
<concept_desc>Applied computing~Life and medical sciences</concept_desc>
<concept_significance>500</concept_significance>
</concept>
</ccs2012>
\end{CCSXML}
\ccsdesc[500]{Human-centered computing~Ubiquitous and mobile computing}
\ccsdesc[500]{Applied computing~Life and medical sciences}
%
% Keywords. The author(s) should pick words that accurately describe the work being
% presented. Separate the keywords with commas.
\keywords{Few-shot learning, Just-in-time intervention, Personalized intervention}

%
% This command processes the author and affiliation and title information and builds
% the first part of the formatted document.
\maketitle

\section{Introduction}

% State of the world (robots for creative activites)
The term ``robot,'' originally signifying `forced labor,' has long been associated with labor and work. Robots have demonstrated their utility in various automated productive and social contexts, where the primary goals are improving productivity, safety, and fostering social interactions with humans~\cite{simoes2022designing, weidemann2021role, honig2018understanding}. However, an increasing number of cases feature using of robots in creative settings. Unlike productive contexts, where the focus is on efficiency and task completion~\cite{arents2022smart}, or social contexts, where communication and trust are prioritized~\cite{nam2020trust, saunderson2019robots}, creative environments prioritize artistic innovation and expression~\cite{hsueh2024counts}. This shift fundamentally alters the dynamics of human-robot interaction, redefining the roles and expectations for both humans and robots.

For instance, robots’ social behaviors are leveraged to support the generation and expression of creative ideas~\cite{hu2021exploring, sandoval2022human, alves2020creativity}, and programmable robotic movements and trajectories are employed to inspire artistic activities such as sketching~\cite{lin2020your}. These studies often engage participants from creative fields who possess limited prior experience with robotics, and are typically conducted in short-term, experimental settings. Consequently, the findings from these studies remain constrained since much can be learned from professional practitioners' experiences to inform system design such as digital fabrication~\cite{hirsch2023nothing}. There is a notable gap in research examining the long-term, active, and practical experience of integrating robotic systems into the creative processes. As a result, the deeper insights into how robots facilitate and shape creative processes, beyond simply augmenting human creativity, remain underexplored. In this study, we aim to better understand the impacts of robots on creative processes and outcomes.

As early as Leonardo da Vinci's 16th century ``Automaton,'' artists have explored the creative affordances of robotic systems~\cite{shanken2002cybernetics, pagliarini2009development, jeon2017robotic}. The artistic creation process typically encompasses various stages, including the exploration of materials and techniques, ongoing experimentation and iteration, and the continual refinement of the artists' insights into their creative subjects~\cite{lewis2023art, sturdee2022state}. Therefore, investigating the artistic process involving robots offers an opportunity to gain deeper insights into robots' creative potential. Robotic art, in particular, provides a compelling case for this exploration.

We define robotic art as artworks that utilize robotic or automated machines to create artistic experiences and tangible artifacts. One example is robotic installation art, in which robots are programmed to follow specific rules that embody the artist’s expression (\autoref{fig:teaser} (a)). Another example is responsive art, in which robots react to their environment, with behaviors that change over time or in response to spectators (\autoref{fig:teaser} (b)). Additionally, there are robotic creators, which possess a degree of agency, allowing them to collaborate with human artists and produce works that extend beyond mere replication of human-created art (\autoref{fig:teaser} (c) and (d)). As such, robotic art becomes a rich case for exploring human-machine interactions in creative contexts. Gaining a deeper understanding of how robots facilitate artistic expression can provide insights for designing computing systems to support creative activities~\cite{gomez2021robot}.

% Therefore, we did...
We draw on semi-structured, in-depth interviews with renowned professional robotic artists to investigate the use of robots in artistic practice. Specifically, our goal is to understand how artistic exploration of robotic systems challenges conventional assumptions about the functions of robots, such as their roles in automating repetitive tasks or serving human needs. We also explore the implications of robots in the artistic process and examine how creativity may emerge within robotic art. To address these interrelated inquiries, our study focuses on the practice of robotic art, posing the research question: \textit{How do robotic artists utilize robots in their artistic practice?} We approach this inquiry through the perspectives and experiences of robotic artists, who creatively design, modify, and repurpose robotic systems for artistic expression and exploration.

% The key findings are...
Our findings highlight the social, material, and temporal dimensions of artists' practices that shape their creativity and artistic outcomes. The creation of robotic art is largely a social process, as artists receive both explicit and implicit feedback through the audience's reactions and reception of their work. Simultaneously, the embodiment and malfunctions inherent to robotic systems drive artistic experimentation. The temporal processes of creation and exhibition, beyond just the final product, further enhance the creative value. Our empirical analysis presents how creativity emerges through the interplay of social, material, and temporal interactions among artists, robots, audiences, and the environment.

% The contributions of this work are...
We make two main contributions to HCI in this study. 
First, we elucidate the interactive mechanisms among key actors---human creators, machines, audiences, and environments---within the practice of robotic art, a topic that remains underexplored in HCI. Our findings reveal the significance of sociality (e.g., interactions between artists and audiences), materiality (e.g., the embodiment and malfunctions of robots), and temporality (e.g., the processes of creation and exhibition) in shaping creative values. We propose that these three facets are central to the creative process and facilitate the emergence of creativity in robotic art.
Second, drawing from the findings, we offer implications for \textit{socially informed}, \textit{material-attentive}, and \textit{process-oriented} creation with computing systems. We suggest leveraging these three aspects to enhance creativity and the creative experience. Specifically, we discuss the value of incorporating implicit audience feedback, designing with technical malfunctions, and focusing on the post-creation process to foster alternative creative experiences with machines~\cite{alter2010designing, juarez2022glitch}.




\section{Background} \label{sec:background}

% \subsection{Capture the Flag (CTF) Challenges}

% CTF challenges simulate real-world cyber-attack scenarios and have emerged as a popular medium for practical cybersecurity training, evaluation, and research. These challenges can simulate real-world attack and defense scenarios and thus assist competitors in developing practical skills in areas such as cryptography, binary exploitation, and reverse engineering. 
% Evaluation of autonomous LLM agents works best with jeopardy-style CTF challenges that focus on standalone software that must be compromised \cite{shao2024nyu,pieterse2024friend}.
% The standalone software may be a binary that can be reverse engineered or exploited, encrypted data that can be decrypted, or a web server whose authentication can be bypassed. After successfully compromising the software, a unique ``flag'' string is either found or revealed by the software server.
% The unique flag string is a concrete indicator of the success of a CTF challenge.
% Recent studies use benchmarks of CTF challenges to evaluate LLM agents on their ability to solve complex tasks and demonstrate practical skills in cybersecurity \cite{shao2024nyu,shao2024empirical,abramovich2024enigma, muzsai2024hacksynth, zhang2024cybenchframeworkevaluatingcybersecurity,yang2023language,turtayev2024hacking}
% Platforms like PicoCTF~\cite{picoctf}, TryHackMe~\cite{tryhackme}, CTFTime~\cite{ctftime} and HackTheBox~\cite{hackthebox} have popularized these formats by providing structured challenges for learners at various skill levels.

% Research indicates that CTF challenges can foster cybersecurity expertise and serve as tools for evaluating facility with cybersecurity skills~\cite{chicone2018using}. They are widely used in academia to enhance learning outcomes in cybersecurity education, with studies demonstrating their effectiveness in promoting analytical thinking and teamwork~\cite{hanafi2021ctf,leune2017using,vykopal2020benefits}. Furthermore, the integration of CTF challenges into research environments enables benchmarking of advanced AI systems like LLMs. .

% Yet, challenges in CTF design persist. These include achieving significant performance, preserving context across tasks, and handling complex, dynamic CTFs that rely on multidisciplinary approaches. Implementing strategies to address these issues enhances problem-solving efficiency, enabling more accurate, adaptive, and effective responses to evolving challenges within CTF environments.


% \subsection{Prompt Engineering}
% \subsection{Prompt Engineering for CTF}
% \subsection{LLM Agents}

% As the use of LLMs to solve CFT challenges expands, prompt engineering is becoming a critical technique for enhancing performance. Various methods have been explored to craft prompts that effectively guide LLMs to the solution of complex cybersecurity problems. Each of these solutions have their own unique strengths and limitations.
%\meet{add more references for LLM agents in other domains, like SWE-Agent, also talk about use of function calling}
Text-based LLMs take a text prompt as input from the user, and produce a text output that follows the user prompt.
LLMs have a finite length of text tokens that they can process called the context.
An alternating sequence of user prompts and LLM outputs makes a conversation and is the basis of chat-based LLM interfaces like ChatGPT.
To remove the user from the loop and create autonomous agents, a feedback mechanism is added based on the LLM outputs, so that the LLM can autonomously continue the conversation.
\citet{yang2023intercode} introduce iterative feedback prompting where the LLM is tasked with writing a piece of code, and the code's compilation and execution logs are provided as feedback, which the LLM uses to iteratively refine it's output.
Recent LLMs support function calling, a way to provide a set of actions to the LLM that it may choose to ``call'' as a function.
In this manner, LLM agents can be provided with many ``tools'' such as a command line, web search, file editing, and code execution \cite{wang2024surveyllmagents}, so that they can autonomously perform various tasks like software development \cite{yang2024sweagent}, web browsing \cite{yoran2024assistantbench}, or solve CTF challenges~\cite{shao2024nyu, abramovich2024enigma}.

With access to the command line and file editing tools, LLM agents can autonomously solve many tasks, but they still struggle on complex long-horizon tasks such as CTF challenges that require multiple steps.
Plan-and-solve prompting \cite{wang2023planandsolve} enhances long-term focus of the agent by incorporating a planning phase before iterative execution. This helps agents tackle ambiguous or complex tasks by structured strategies \cite{turtayev2024hacking}.
ReAct (reasoning + action) \cite{yao2022react} combines step-by-step reasoning with action, allowing the agent to adjust dynamically through iterative cycles. ReWOO (Reasoning without Observation) \cite{xu2023rewoo} separates the reasoning process from tool outputs and observations, allowing it to handle multi-step reasoning tasks efficiently while maintaining focus.
The prompting methods in these agents involve static hard-coded templates where environment and task information is filled in.
While static prompts provide straightforward guidance, they often fail to adapt to different problems and complex tasks, limiting their effectiveness.
Auto-prompting~\cite{shin-etal-2020-autoprompt, zhou-etal-2023-revisiting, zhang2023automatic} is a technique to allow the LLM itself to generate a highly-relevant prompt. Auto-prompting invokes more factual responses and reduces hallucinations in LLMs.
D-CIPHER incorporates auto-prompting as a separate agent that can explore the environment and generate a better prompt.
%Based on the given prompt, LLM agents make a decision and proceed further to find flags.  To address this gap, we propose \textbf{dynamic prompting}, where the LLM agent autonomously generates prompts based on the CTF challenge's context and stage.
%include a static template which needs to be given to LLM to solve the CTF challenges. For instance, the NYU CTF framework provides a static prompt as \emph{``Please proceed to the next step using your best judgment"} for each decision making point. 

% To address this gap, we introduce a novel approach where the LLM agent generates the next prompt autonomously based on the current context and stage of the CTF challenge, a technique we call \textbf{dynamic prompting}.


Expanding on single LLM agents, multi-agent LLM systems are a powerful approach to enhance problem-solving by simulating team-based collaboration. Specialized agents, each with distinct objectives, work together to tackle different aspects of complex tasks \cite{guo2024largelanguagemodelbased}
Multi-agent systems are effective in cybersecurity applications. For instance, Audit-LLM~\cite{song2024audit} deploys a  multi-agent system for insider threat detection by employing agents to decompose tasks, build tools, and use collaborative reasoning to enhance detection accuracy. Liu~\cite{liu2024multi} explores multi-agent systems to enhance incident response in cybersecurity by examining centralized, decentralized, and hybrid team structures to assess how LLM agents can improve decision-making, adaptability, and coordination during cyber-attack scenarios. AutoSafeCoder~\cite{nunez2024autosafecoder} enhances the security of code generated by LLMs by incorporating a coding agent for code generation, a static analyzer agent that identifies vulnerabilities, and a fuzz testing agent for dynamic testing to detect runtime errors. Division of responsibilities among different agents allows AutoSafeCoder to produce secure, functionally correct code. 

% With the growing use of LLMs in CTF challenges, prompt engineering is key to enhancing performance. Various methods guide LLMs in solving complex cybersecurity tasks, each with distinct strengths and limitations.

% \textbf{Single Turn (Zero-Shot Prompting)} involves providing the model with a one-time task description that outputs  an immediate solution. This is efficient for straightforward tasks~\cite{yang2023intercode}. In contrast, \textbf{Try Again (Iterative Feedback Prompting)} uses iterative feedback to refine responses over multiple attempts, mimicking real-world problem-solving~\cite{yang2023intercode}. The \textbf{Plan \& Solve} enhances adaptability by incorporating a planning phase before iterative execution. This helps models tackle ambiguous or complex tasks by  structured strategies~\cite{turtayev2024hacking}. Additionally, \textbf{ReAct (Reasoning + Action)} combines step-by-step reasoning with action, allowing the model to adjust dynamically through iterative cycles. This makes it particularly effective for evolving and complex challenges like CTFs~\cite{yao2023react}. 
% These prompting techniques highlight diverse approaches to optimizing LLM performance in cybersecurity tasks. 

% Multi-agents!


%\meet{Add references for auto-prompting, and shorten this para}
%\nanda{Maybe we can add this to previous paragraphs which discusses other prompting methods such as plan-and-solve and ReAct method}
% All of these prompting methods include a static template which needs to be given to LLM to solve the CTF challenges. For instance, the NYU CTF framework provides a static prompt as \emph{``Please proceed to the next step using your best judgment"} for each decision making point. 
% Based on the given prompt, LLM agents make a decision and proceed further to find flags. While static prompts provide straightforward guidance, they often fail to account for the evolving nature of complex tasks, limiting their effectiveness in multi-step or ambiguous CTF challenges. To address this gap, we propose \textbf{dynamic prompting}, where the LLM agent autonomously generates prompts based on the CTF challenge's context and stage.
% % To address this gap, we introduce a novel approach where the LLM agent generates the next prompt autonomously based on the current context and stage of the CTF challenge, a technique we call \textbf{dynamic prompting}.
% Dynamic prompting adapts instructions to task progress, ensuring instructions are contextually relevant and reflective of the specific obstacles encountered. By iterating based on feedback and intermediate outputs, it continuously refines the LLM’s approach, enhancing problem-solving for dynamic tasks like CTFs.
% This adaptive process not only mirrors how humans tackle complex problems but also improves the model’s ability to handle unpredictable scenarios, making it particularly advantageous for cybersecurity tasks like CTFs where conditions change dynamically.


% The very first prompt type used in several applications is \textbf{Single Turn (Zero-Shot Prompting)}~\cite{yang2023intercode}. In single-turn prompting, the model receives a one-time, straightforward task description and is expected to generate a complete response without further interaction. The initial output is directly assessed, making this approach efficient for tasks where minimal feedback or iteration is required. This method tests the model’s ability to understand and respond to tasks immediately, relying heavily on the model's pre-trained knowledge and generalization capabilities.

% Along with this, The prompting method named \textbf{Try Again (Iterative Feedback Prompting)}~\cite{yang2023intercode} has been also used in several appreciations specially to solve CTF challenges. It is an iterative prompting method involves continuous interaction, where the model is provided with feedback after each attempt. The model can refine its responses over multiple turns based on the observations or execution results from previous outputs. This iterative process continues until the task is successfully completed or a maximum number of interactions is reached. This approach closely mirrors real-world problem-solving, where adjustments are made iteratively based on evolving circumstances or feedback.

% Some application are also using \textbf{Plan \& Solve}~\cite{turtayev2024hacking} prompting method which enhances problem-solving by dividing the process into a planning phase followed by execution. Initially, the model formulates a strategy based on the task description and available information, allowing for a structured approach to ambiguous or complex problems. This plan guides the subsequent execution phase, where the model carries out actions iteratively, refining its approach based on feedback. In more challenging scenarios, re-planning mid-task further improves adaptability and performance. This method proves effective in tasks like CTF challenges, where vague instructions require careful analysis and step-by-step resolution.

% Further some application are also using \textbf{ReAct (Reasoning + Action)}~\cite{yao2023react} prompting method blends reasoning with action by guiding the model to think through tasks step-by-step before executing actions. At each step, the model generates a thought based on the task and observations, which informs the next action. The action is executed, and the resulting feedback refines the model’s understanding for the next cycle. This continuous process helps the model adapt dynamically to complex tasks, making it effective for CTF challenges where logical reasoning and step-by-step execution are essential.

\section{Related Works} \label{sec:related_work}


\begin{table}[htpb]
    \centering
    \caption{Feature comparison of LLM agents for solving CTFs.}
    \label{tab:related_work_comparison}
    \begin{tabular}{lcccccc}
    \toprule
         \textbf{Study} & \rotatebox{90}{\textbf{\# CTFs}} & \rotatebox{90}{\textbf{Open bench}} & \rotatebox{90}{\textbf{Tool use}}  & \rotatebox{90}{\textbf{Autonomous}} & \rotatebox{90}{\textbf{Multi-agent}} &\rotatebox{90}{\textbf{Auto-prompt}} \\
    \cmidrule{2-7}
     % \textbf{Study} & \textbf{Dynamic} & \textbf{Used} & \textbf{Multi-} & \textbf{Automatic} & \textbf{Tool} & \textbf{\# of} \\
         Tann et al. \cite{tann2023using} &  $7$ & \purplecross & \purplecross & \purplecross & \purplecross & \purplecross  \\
         Shao et al. \cite{shao2024empirical} & $26$ & \purplecross & \tealcheck & \tealcheck & \purplecross & \purplecross  \\
         InterCode-CTF\cite{yang2023language} & $100$ & \tealcheck & \tealcheck & \tealcheck & \purplecross & \purplecross   \\
         NYU CTF Bench \cite{shao2024nyu} & $200$ & \tealcheck & \tealcheck & \tealcheck & \purplecross & \purplecross \\
         Turtayev et al. \cite{turtayev2024hacking} & $100$ & \tealcheck & \tealcheck & \tealcheck & \purplecross & \purplecross\\
         Cybench \cite{zhang2024cybenchframeworkevaluatingcybersecurity} & $40$ & \tealcheck & \tealcheck & \tealcheck & \purplecross & \purplecross \\
         EnIGMA \cite{abramovich2024enigma} & $350$ & \tealcheck & \tealcheck & \tealcheck & \purplecross & \purplecross\\
         HackSynth \cite{muzsai2024hacksynth} & $200$ & \tealcheck & \tealcheck & \tealcheck & \tealcheck & \purplecross \\
         \textbf{D-CIPHER (ours)} & $290$ & \tealcheck & \tealcheck & \tealcheck & \tealcheck & \tealcheck \\
    \bottomrule
    \end{tabular}
\end{table}



% \subsection{LLMs on Cybersecurity}
% \subsection{LLM Agents for CTF}

%LLMs have a vast knowledge base that can be tapped for cybersecurity use.
Tann et al.~\cite{tann2023using} evaluate early LLMs such as ChatGPT and Google Bard in solving CTF challenges and answering professional certification questions, showing that LLM responses contain key task information.
%Many works extend the LLM capabilities by providing them access to programming and command execution tools, to form autonomous agents. 
The InterCode-CTF agent~\cite{yang2023intercode} reveals that LLM agents demonstrate basic cybersecurity skills, however they struggle with more complex tasks.
The NYU CTF baseline agent~\cite{shao2024empirical} integrates external tools into the LLM's function-calling features and demonstrate improved potential of tool-assisted LLMs to solve CTFs, however it exhausts the LLM context length when command output history becomes very long. InterCode-CTF manages this issue by truncating the history to only show the LLM the last few iterations. Even so, LLM agents face issues with longer tasks.
%NYU CTF Bench~\cite{shao2024nyu}, a benchmark of 200 CTF challenges, presents a baseline agent with specialized reverse engineering tools and category-specific prompts, demonstrating their importance to solve CTFs.
% The NYU CTF baseline agent faces issues of LLM context length when complex tasks run for several iterations and the entire command and output history becomes longer than the LLM's context window size. The InterCode agent manages this issue by truncating the history to only show the LLM the last few iterations.


Excessive tool availability and verbose interfaces can overwhelm agents, leading to inefficiencies. Agents perform better with a focused set of tools with well-defined interfaces~\cite{yang2024sweagent}.
EnIGMA~\cite{abramovich2024enigma} agent incorporates interactive tools and in-context learning techniques to achieve state-of-the-art results. % on the NYU CTF Bench, HackTheBox, and Cybench benchmarks.
For better context management, EnIGMA also uses an LLM summarizer that summarizes the command outputs for the main agent.

HackSynth~\cite{muzsai2024hacksynth}, an LLM agent for autonomous penetration testing, shows that iterative planning and feedback summarization stages help the agent finish multiple tasks and improves overall problem solving.
Similarly, Cybench~\cite{zhang2024cybenchframeworkevaluatingcybersecurity} introduces a benchmark of 40 CTF challenges augmented with step-by-step tasks, demonstrating better focus of LLM agents on smaller tasks, leading to improved success and alleviating the context length issue.
\citet{turtayev2024hacking} expand on InterCode-CTF by implementing plan-and-solve prompting, achieve significant improvement on the InterCode-CTF benchmark. They show that prompting techniques can improve performance even with simple toolsets.
% . Their baseline agent is evaluated in unguided mode (i.e. fully autonomous), and guided mode where the agent is given one task at a time. Their results indicate that providing smaller tasks to the LLM agents improve their focus yielding improved success on complex challenges while .

These works highlight that LLM agents excel at implementing code and executing commands to accomplish small concrete tasks when provided with dynamic feedback and task-specific toolsets. While these works  involved using multiple LLMs with different tasks such as planning and summarizing along-side a main agent, D-CIPHER is the first work to formulate a multi-agent system where there is a bifurcation of responsibilities between agents and meaningful well-defined interactions for dynamic feedback.
Table~\ref{tab:related_work_comparison} shows a feature comparison of D-CIPHER with related works on LLM agents for autonomous CTF solving.
%\meet{some description of the feature comparison?}
% Recent research has focused on enable autonomous solving of CTF challenges~\cite{shao2024empirical,shao2024nyu,abramovich2024enigma}. These agents typically operate in containerized environments to ensure reproducibility and modularity. 

% As an early effort, Tann et al.~\cite{tann2023using} evaluated the effectiveness of LLMs, such as OpenAI's ChatGPT, Google Bard, and Microsoft Bing, in solving cybersecurity CTF challenges and answering professional certification questions. 
% % Their study results show that LLMs performed well on $7$ CTF test cases, with ChatGPT solving $6$, Bard $2$, and Bing $1$. 
% The study shows that LLM responses often contain key information essential for solving tasks.

% The InterCode framework~\cite{yang2023intercode} approaches coding as an interactive process and uses execution feedback to improve code generation. As described in Yang et al.~\cite{yang2023intercode}, InterCode-CTF integrates CTF benchmarks into a reinforcement learning environment that can evaluate the cybersecurity capabilities of language agents. It features $100$ tasks that tapskills such as reverse engineering, forensics, and binary exploitation. While existing language agents demonstrate basic cybersecurity skills, evaluations indicate they struggle with more complicated complex tasks unless the system is fine-tuned or given external support. 
% cite Intercode: Standardizing and benchmarking interactive coding with execution feedback

% Another notable example is an LM agent developed by Shao et al. specifically to automate CTF tasks. 
% Shao et al.~\cite{shao2024empirical} developed a LM agent to automate CTF tasks.
% % They report an accuracy rate of  $46\%$ on $26$ CTF challenges sourced from CSAW'23 Qualifying round competition using GPT-4.
% By effectively combining LLM capabilities with external tools, the researchers demonstrated the potential of tool-assisted LLMs to solve complex problems. Building on this, the team incorporated a broader range of cybersecurity tools and interfaces that enhance both accuracy and versatility. 
% Empirical results show their system outperforms baselines on both the InterCode CTF benchmark and the NYU CTF benchmark.

% Shao et al.~\cite{shao2024nyu} presented a diverse, open-source database of CTF challenges that can be used to benchmark an LLM's ability to solve cybersecurity problems.
% It provides a scalable platform for developing and testing AI-driven approaches for vulnerability detection and resolution, facilitating advancements in automated cybersecurity tasks. The benchmark database and automated framework were successfully applied to the performance of five LLMs. 

% The Cybench benchmark~\cite{zhang2024cybenchframeworkevaluatingcybersecurity} provides another significant contribution by creating a framework tailored to solving CTF challenges. % Cybench: A framework for evaluating cybersecurity capabilities and risk
% % Their benchmark environment achieves an accuracy of $17.5\%$ using Claude 3.5 Sonnet. 
% Such frameworks operate in Linux-based containerized environments, such as Kali Linux, which includes pre-installed cybersecurity tools. However, excessive tool availability can overwhelm agents, leading to inefficiencies. Research indicates that agents perform better with a focused set of tools that have well-defined interfaces~\cite{yang2024sweagent}. % Swe-agent: Agent-computer interfaces enable automated software engineering



% Muzsai et al. introduced HackSynth~\cite{muzsai2024hacksynth}, an LLM-based agent for autonomous penetration testing. It uses a dual-module architecture that consists of a Planner and a Summarizer, allowing for iterative command generation and feedback processing. The framework is evaluated using two benchmark sets from platforms like PicoCTF~\cite{picoctf} and OverTheWire~\cite{overthewire}. These benchmarks address $200$ challenges drawn from various domains and difficulty levels. Results of their study show that HackSynth, especially with the GPT-4o model, achieves the best performance. This highlights the potential of LLM-based agents in advancing autonomous penetration testing.
 % Using basic prompting techniques and expanding tool availability, the study highlights how straightforward approaches can unlock the latent potential of LLMs for cybersecurity tasks. Their work emphasizes that simple LLM designs can effectively solve CTF challenges, and thus broaden the number of cybersecurity applications without the need for advanced engineering.

% \begin{table*}[]
%     \centering
%     \begin{tabular}{|c|c|>{\centering\arraybackslash}p{4.5cm}|c|c|c|c|c|c|}
%     \hline
%          \textbf{Study} & \textbf{Dynamic} & \textbf{Used} & \textbf{Multi-} & \textbf{Open} & \textbf{Automatic} & \textbf{Tool} & \textbf{\# of} & \textbf{\# of} \\
%          & \textbf{Prompt} & \textbf{Benchmarks} & \textbf{Agents} & \textbf{Dataset} & \textbf{Framework} & \textbf{Use} & \textbf{LLMs} & \textbf{CTFs}\\
%          \hline
%          Tann et al.~\cite{tann2023using} & \purplecross & Manual collected & \purplecross & \purplecross & \purplecross & \purplecross & $3$ & $7$ \\
%          \hline
%          InterCode-CTF~\cite{yang2023language} & \purplecross &  PicoCTF~\cite{picoctf} & \purplecross & \purplecross& \purplecross & \purplecross & $1$ & $100$  \\
%          \hline
%          Shao et al.~\cite{shao2024empirical} & \purplecross & CSAW 2023 & \purplecross & \purplecross & \tealcheck & \tealcheck & $4$ & $26$ \\
%          \hline
%          Shao et al.~\cite{shao2024nyu} & \purplecross & NYU CTF~\cite{shao2024nyu} & \purplecross & \tealcheck & \tealcheck & \tealcheck & $5$ & $200$ \\
%          \hline
%          Cybench~\cite{zhang2024cybenchframeworkevaluatingcybersecurity} & \purplecross & Cybench~\cite{zhang2024cybenchframeworkevaluatingcybersecurity}  & \purplecross & \tealcheck & \tealcheck & & $8$ & $40$ \\
%          \hline
%          EnIGMA~\cite{abramovich2024enigma} & \purplecross & NYU CTF~\cite{shao2024nyu}, InterCode-CTF~\cite{yang2023language},  HackTheBox~\cite{hackthebox} & \purplecross & \purplecross & \tealcheck & \tealcheck & $3$ & $350$ \\
%          \hline
%          HackSynth~\cite{muzsai2024hacksynth} & \purplecross & PicoCTF~\cite{picoctf}, OverTheWire~\cite{overthewire} & \tealcheck & \tealcheck & \tealcheck & \tealcheck & $8$ & $200$ \\
%          \hline
%          Turtayev et al.~\cite{turtayev2024hacking} & \purplecross & InterCode-CTF~\cite{yang2023language} & \purplecross & \purplecross & \purplecross & \purplecross & $4$ & $100$ \\
%          \hline
%          \textbf{D-CIPHER (Proposed)} & \tealcheck & NYU CTF~\cite{shao2024nyu}, Cybench \cite{zhang2024cybenchframeworkevaluatingcybersecurity}, HackTheBox \cite{hackthebox} & \tealcheck & \tealcheck & \tealcheck & \tealcheck & 5 & 290 \\
%          \hline
%     \end{tabular}
%     \caption{Comparison with LLM-based CTF solving Literature}
%     \label{tab:related_work_comparison}
% \end{table*}




% \subsection{Multi-agent framework}

% The use of multi-agent LLM systems in Capture the Flag (CTF) challenges is emerging as a powerful approach to enhance cybersecurity problem-solving. Multi-agent frameworks mimic team-based collaboration, where multiple LLM agents, each with specialized expertise, work together to tackle complex tasks. This approach reflects real-world cybersecurity operations, where success often depends on coordinated efforts from teams with diverse skills, each addressing different components of a security challenge.
% Multi-agent LLM systems are emerging as a powerful approach to enhance cybersecurity problem-solving by simulating team-based collaboration. Specialized agents, each with distinct objectives, work together to tackle different aspects of complex security tasks. This mirrors real-world cybersecurity operations, where coordinated efforts and diverse skills are essential for addressing evolving threats and vulnerabilities.

% CTF challenges cover a wide range of domains, including cryptography, reverse engineering, forensics, and web exploitation. Multi-agent systems can distribute the workload by assigning agents to handle specific tasks. This enables parallel problem-solving and emulates the collaborative nature of human teams. For example, one agent may specialize in guiding the fellow agents to what needs to be done, while another executes the instructions, ensuring that tasks are addressed without losing the context, and implementing reasoning from multiple LLMs. This division of labor boosts efficiency and enables problem-solving from multiple perspectives.
% This division of labor enhances efficiency and allows the system to approach problems from multiple perspectives, reflecting the interdisciplinary approach often used in cybersecurity teams.

% Guo et al.~\cite{guo2024largelanguagemodelbased} highlight the strengths of multi-agent LLMs in complex, multi-step tasks where different agents handle specific roles The framework HackSynth~\cite{muzsai2024hacksynth} is a multi-agent penetration testing framework in which agents operate collaboratively to address vulnerabilities in staged environments. Their work emphasizes that when agents work as a cohesive team, they outperform single-agent approaches. This is particularly true when facing layered, iterative challenges. 
% This team-based model of problem-solving aligns closely with how cybersecurity professionals approach real-world security incidents and penetration testing exercises.

% Multi-agent LLM systems have shown effectiveness in various other applications. For instance,  Audit-LLM~\cite{song2024audit} presents a multi-agent framework for insider threat detection using log analysis. It employs agents to decompose tasks, build tools, and use collaborative reasoning to enhance detection accuracy. Liu~\cite{liu2024multi} explores the application of LLM-based multi-agent systems to enhance incident response (IR) in cybersecurity. Utilizing the ``Backdoors \& Breaches" tabletop game as a simulation environment, the study examines centralized, decentralized, and hybrid team structures to assess how LLM agents can improve decision-making, adaptability, and coordination during cyberattack scenarios. AutoSafeCoder~\cite{nunez2024autosafecoder} is a multi-agent system designed to enhance the security of code generated by LLMs. The framework comprises three agents: a Coding Agent responsible for code generation, a Static Analyzer Agent that identifies vulnerabilities through static analysis, and a Fuzzing Agent that performs dynamic testing using mutation-based fuzzing to detect runtime errors. By integrating both static and dynamic testing in an iterative process, AutoSafeCoder aims to produce secure, functionally correct code. 

% To enhance CTF-solving by promoting team-based specialization, we employ a multi-agent CTF solving agent. Within this framework, agents tackle tasks aligned with their strengths. Tasks are executed in parallel, improving efficiency and accelerating progress. Agents share insights, adapt refining strategies based on feedback, and overcome obstacles collectively. This collaborative approach boosts scalability, adaptability, and and resilience, and improves performance in complex challenges.

% This paper presents a comprehensive comparison of D-CIPHER with existing LLM-based CTF-solving literature, as shown in Table~\ref{tab:related_work_comparison}.
% This paper documents the results of  our comprehensive comparison of D-CIPHER with existing LLM-based CTF-solving literature. These results are presented in Table~\ref{tab:related_work_comparison}.



% \subsection{Notations}  %% commented out as we do not use them

% The notations used throughout this paper are summarized in Table ~\ref{t:notations}.

% \begin{table}
%     \centering
%     \small % Reduce font size for the table (optional)
%     \begin{tabular}{|l|c|}
%         \hline
%         \textbf{Notation} & \textbf{Description}  \\
%         \hline
%         $X_{\text{tr}}$ & Training set inputs (messages) 
%         \\\hline
%         $y_{\text{tr}}^{\text{gold}}$ & Gold labels for $X_{\text{tr}}$\\
%         \hline
%         $y_{\text{tr}}^{\text{llm}}$ & Synthetic labels for $X_{\text{tr}}$ \\ \hline
%          $X_{\text{val}}$ & Validation set inputs \\
%          \hline
%         $y_{\text{val}}^{\text{gold}}$ & Gold labels for $X_{\text{val}}$ \\
%         \hline
%         $y_{\text{val}}^{\text{llm}}$ & Synthetic labels for $X_{\text{val}}$ \\ \hline
%         $X_{\text{test}}$ & Test set inputs \\
%         \hline
%         $y_{\text{test}}^{\text{gold}}$ & Gold labels for $X_{\text{test}}$ \\
%         \hline
%         $y_{\text{test}}^{\text{llm}}$ & Synthetic labels for $X_{\text{test}}$ \\ \hline
%         $(X, y)_{\text{tr}}^{\text{llm}}$ & Synthetic training data \\
%         \hline
%          $(X, y)_{\text{val}}^{\text{llm}}$ & Synthetic validation data \\  
%          \hline
%     \end{tabular}
%     % ACL style has the caption below the table or figure
%     \caption{Summary of notations used in the paper}
%     \label{t:notations} 
% \end{table}


\subsection{Overview of Scenarios}

%In this study, we
We investigate the role of LLMs in CB detection, focusing on their utility under varying data availability conditions
and under the assumption that direct use of LLMs as a classifier is too expensive due to the high volume of messages to be checked.
%To establish
As a baseline for comparison, we %first
evaluate a scenario in which a
lightweight, BERT-based
classifier is trained exclusively on gold-standard, manually labeled authentic data without %any
LLM involvement.
We then define three additional scenarios with different data availability
and that use LLMs in different ways.
%, each illustrating how LLMs can aid in CB detection depending on the availability and quantity of authentic data.
%The scenarios are as follows.

%To establish a baseline for comparison, in the first scenario, we evaluate a setup that relies exclusively on training a classifier using gold-standard, manually labeled authentic data with no LLM involvement. We then define three other distinct scenarios, each corresponding to
% %% JW: The following is unneccesary vague as the scenarios are more specifically
% %% about the way the synthetic data is used, apart from the zero-shot LLM.
% a unique way LLMs can be integrated into the detection pipeline.
% These scenarios range from directly serving as classifiers to generating synthetic data or labels for training. 

\paragraph{Scenario 1: Baseline}

This scenario represents the ideal situation where sufficient
%manually labeled (
gold-standard data is available for fine-tuning %a classic encoder such as
BERT.
It serves as the benchmark for evaluating the effectiveness of other approaches.
In this setup, no synthetic data or LLMs are involved.
%The system relies entirely on human annotations.
This scenario is feasible if resources such as time, budget and expert annotators are abundant. However, it often proves impractical due to the
%high costs and scalability
challenges of manual labeling.



\paragraph{Scenario 2: LLM as Classifier}  \label{s:m:sc2}

This scenario applies when labeled authentic data is unavailable, and there is no intention to train a separate classifier for CB detection. Instead, an instruction-tuned
LLM is used directly as a classifier, leveraging its pre-trained knowledge and its ability to follow instructions
to identify CB instances.
%This approach is particularly useful in contexts that require rapid deployment or when computational or time resources are limited for training a new model. 
The primary advantage of this method is its elimination of the need for labeled data and training time. However, there are trade-offs. While an LLM can handle nuanced language patterns, it may be less efficient and incur higher computational costs
compared to simpler BERT-based classifiers with a classification head and fine-
tuned on a labeled dataset.
%% JW: add reference to large zero-shot study in NLP
We explore two prompting strategies for generating synthetic labels:
\textit{(a)} guideline-enhanced (GE) prompts, guiding the LLM with detailed labeling instructions and
\textit{(b)} guidelne-free (GF) prompts, allowing the LLM to generate labels without such guidelines.

\paragraph{Scenario 3: Fully Synthetic Data}

In this scenario, only a small set of manually labeled gold data is available for testing, with no access to authentic data for training or validation.
%To address this, we
We
use an LLM to generate a fully synthetic dataset, consisting of both synthetic messages and corresponding labels, for training and validation.
This approach is particularly valuable in low-resource domains or emerging tasks where authentic data is scarce or difficult to collect.
It is especially useful in situations where creating authentic datasets is costly, time-consuming, or ethically challenging, such as annotating harmful or sensitive content or working with vulnerable populations.
The effective

%Commented for indusrty track \subsubsection{Scenario 4: Data Augmentation with Synthetic Data}
%This scenario assumes the availability of a moderate amount of gold-labeled data for training and validation, which may be insufficient to achieve optimal performance. To augment the dataset, we use an LLM to generate additional synthetic data, which is then combined with the gold-labeled data during training and validation. The experiment systematically varies the ratio of synthetic-to-gold data to evaluate its impact on model performance. This scenario explores how LLMs can supplement authentic data, striking a balance between scalability and accuracy.


\paragraph{Scenario 4: Synthetic Labels for Unlabeled Data} \label{s:m:sc4}

This scenario addresses the common situation where resources for manual annotation are limited. Here, gold-standard labeled data is available only for the test set, while a significant amount of unlabeled authentic data is available for training and validation.
%This scenario demonstrates the utility of LLMs in resource-constrained settings, enabling cost-effective dataset creation from unannotated corpora.
To utilize the unlabeled data, we label it using the best prompting strategy (GE or GF) from scenario~2.

% \subsubsection{Summary of Scenarios}
% Table~\ref{t:scenario-summary} presents an overview of the data used in the baseline system and each scenario, specifying the datasets utilized for training, validation, and testing. For Scenario 2, where no classifier is trained and the LLM is used directly as a classifier, only the test set is included.
% \begin{table}
%     \centering
%     \small % Reduce font size for the table (optional)
%     \begin{tabularx}{\columnwidth}{|X|X|X|X|}
%         \hline
%         \textbf{Scenario} & \textbf{Train} & \textbf{Validation} & \textbf{Test} \\
%         \hline
%          1 & $X_{\text{tr}}, y_{\text{tr}}^{\text{gold}}$ & $X_{\text{val}}, y_{\text{val}}^{\text{gold}}$ &  $X_{\text{test}}, y_{\text{test}}^{\text{gold}}$ \\
%         \hline
%            2 & - & - & $X_{\text{test}}, y_{\text{test}}^{\text{gold}}$ \\
%         \hline
%           3 & $(X, y)_{\text{tr}}^{\text{llm}}$ & $(X, y)_{\text{val}}^{\text{llm}}$ & $X_{\text{test}}, y_{\text{test}}^{\text{gold}}$ \\
%         \hline
%          4 & $X_{\text{tr}}, y_{\text{tr}}^{\text{llm}}$ & $X_{\text{val}}, y_{\text{val}}^{\text{llm}}$ &  $X_{\text{test}}, y_{\text{test}}^{\text{gold}}$ \\ \hline
       
%         % 4 & $X_{\text{tr}}, y_{\text{tr}}^{\text{llm}} + X_{\text{tr}}, y_{\text{tr}}^{\text{gold}}$ & $X_{\text{val}}, y_{\text{val}}^{\text{llm}}+ X_{\text{val}}, y_{\text{val}}^{\text{gold}}$ & $X_{\text{test}}, y_{\text{test}}^{\text{gold}}$ \\ \hline
%         % 4 & $(X, y)_{\text{tr}}^{\text{llm}} + X_{\text{tr}}, y_{\text{tr}}^{\text{gold}}$ & $(X, y)_{\text{val}}^{\text{llm}}+ X_{\text{val}}, y_{\text{val}}^{\text{gold}}$ & $X_{\text{test}}, y_{\text{test}}^{\text{gold}}$ \\ \hline
%     \end{tabularx}
%     % ACL style has the caption below the table or figure
%     \caption{Overview of data used in each scenario}
% \label{t:scenario-summary} 
% \end{table}





% \subsection{Intrinsic Evaluation Metrics}

% Intrinsic evaluation examines the inherent qualities of datasets, enabling the assessment of linguistic diversity, emotional tone, and conversational structure independently from task-specific performance. For our CB detection task, we utilize \textbf{four} categories of intrinsic metrics to compare the authentic dataset with LLM-generated synthetic data. These categories are: 1) lexical and linguistic characteristics, %including metrics such as Mean Words per Message, Mean Word Length, and Type-Token Ratio; 
% 2) content and CB indicators, 
% %such as rate of Harmful Messages, Bully Messages, Victim Messages, and Toxicity; 
% 3) sentiment and emotional tone, 
% %which classifies messages into negative, positive, or neutral; 
% and 4) dialogue act distribution.
% %categorizing messages into types such as Question, Statement, Greeting, Accept/Reject, and Other. 
% These categories are critical for understanding the fundamental differences between authentic and synthetic data in the context of CB detection, as they provide insight into how well the synthetic data replicates the linguistic, emotional, and conversational behaviors that are typically present in real-world online interactions.

% To ensure a fair comparison between the authentic and synthetic datasets, we first normalize both dataset by employing pre-processing techniques such as tokenization using NLTK \cite{loper-bird-2002-nltk} and punctuation handling. Additionally, data is segmented into equal-sized token slices to account for metrics that are influenced by corpus size.

% Sentiment scores are measured using VADER \cite{hutto2014vader}, a sentiment analysis tool optimized for short social media texts. Dialogue acts are classified using a Naive Bayes model trained on the NLTK \texttt{nps-chat} corpus,
% following \newcite[Chp.~6, Sec.~2.2]{bird2009natural}.\footnote{
%     While no citation is provided by \newcite{bird2009natural}, the source
%     of this corpus seems to be
%     \newcite{forsyth-martell-2007-lexical,forsyth-etal-2010-nps}.
% }
%



% Natural Language Processing with Python, by Steven Bird, Ewan Klein and Edward Loper
% Chapter 6, section 2.2 "Identifying Dialogue Act Types"
% refers to Chapter 2, section 1.2 "Web and Chat Text", for the
% NPS Chat Corpus but provides no source or citation.
%  
% An unrelated 2011 paper cites an "NPS Chat Corpus of North American English chat
% conversations (Forsyth and Martell 2007)".
%   * Forsyth, Eric. M. and Craig H. Martell (2007), Lexical and discourse analysis
%     of online chat dialog, Proceedings of the First IEEE International Conference
%     on Semantic Computing (ICSC) 2007, pp. 19–26.
%   * data collected in 2006
%   * approximately 500,000 chat posts gathered from various online services
%   * 10,567 posts tagged in Release 1.0
%   * available on http://faculty.nps.edu/cmartell/NPSChat.htm (page no longer
%     exists but is archived, e.g. on
%     http://web.archive.org/web/20190510121556/http://faculty.nps.edu/cmartell/NPSChat.htm
%        - "If you want just the data, you can get it through the Linguistic Data
%          Consortium.  It is catalog number LDC2010T05."
%        - This page asked for the 2007 paper above to be cited "when referring to
%          the NPS Chat Corpus".
%
% There is a 2010 thesis from Naval Postgraduate School, Monterey, California, by
% J. R. Hitt entitled "Implementation and Performance exploration of a cross-genre
% part of speech tagging methodology to determine dialog act tags in the chat
% domain".
%   * credits Lin and Forsyth
%


% Type-Token Ratio (TTR), which is calculated by dividing the number of unique words by the total tokens in fixed-size slices, serves as a normalized measure of vocabulary diversity. Toxicity scores, which represent the ratio of messages containing profanity, are derived using a publicly available profanity list \cite{surge2023profanity}.


\subsection{Evaluation Metrics}

We choose accuracy of label prediction for development decisions and reporting since the labels are reasonably balanced in the authentic test data with 30.3\% items labeled with the minority
label.\footnote{In the appendix, we further report macro average F1 scores that are also widely used in the area of harmful content detection.}
In scenarios 1, 3 and 4,
we train BERT\_base\_uncased \cite{devlin-etal-2019-bert}, a 110M parameter transformer model, with a linear classification head
% using
% the HuggingFace transformers library \cite{wolf-etal-2020-huggingface}
to detect harm, assigning binary labels to text messages.
To address noise from randomness in training, we train at least 45 models for each setting and report average accuracy and standard deviation.

\section{Model Evaluation}
\label{sec:model_evaluation}
In this section, we report the evaluation of \projectname's few-shot learning pipeline offline performance. We will further elaborate on the evaluation of \projectname's intervention effectiveness in section \ref{sec:intervention_evaluation}.

\subsection{Data Collection}
\label{sub:model_evaluation:data_collection}
\subsubsection{Participants}
We recruited 26 users (14 females, 12 males, age 22$\pm$2) for data collection via social media platforms.
We focused on users who were aware of their own undesirable actions and had the intention to reduce these actions. These are the target users of our intervention system.
Our study was IRB-approved by the local institution, and participants were compensated with \$10 for this data collection study (around 45 minutes).

\subsubsection{Personal Undesirable Action Customization}

Participants were asked to record five pre-determined target actions that are commonly recognized as undesirable actions~\cite{teng2002body,oshio2018shake}, including \textit{Face Scratching}, \textit{Nail Biting}, \textit{Eye Rubbing}, \textit{Lip Tearing}, and \textit{Leg Shaking}. The first five figures in \autoref{fig:evaluation_actions} illustrate these actions.

Moreover, each participant was asked to define a new undesirable action tailored to their own personal needs.
In total, 26 participants designed an additional set of 12 actions, including \textit{Finger Lipping} (designed by N=5 participants), \textit{Head Scratching} (N=5), \textit{Nose Rubbing} (N=4), \textit{Finger Picking} (N=3), \textit{Hair Scratching} (N=2), \textit{Face Rubbing} (N=1), \textit{Finger Biting} (N=1), \textit{Hair Pulling} (N=1), \textit{Hair Rubbing} (N=1), \textit{Lip Biting} (N=1), \textit{Nail Picking} (N=1), and \textit{Neck Scar Scratching} (N=1).
We only grouped identical actions and distinguished actions as long as they differed slightly. For instance, \textit{Head Scratching} and  \textit{Hair Scratching} were similar, but one involved contacts between fingers and 
scalp, while the other one did not.
Similarly, \textit{Finger Picking} and \textit{Nail Picking} were also quite close, yet one solely focused on the skin on the finger, while the other focused on nails.
These actions were visualized in the second half of \autoref{fig:evaluation_actions}.


\begin{figure*}[]
\centering 
\includegraphics[width=1\linewidth]{figure/action.pdf}
\caption{Target Actions for Evaluation. (1-5) presents the five pre-determined actions. (6-17) visualizes new target behaviors defined by participants.
Only identical actions are grouped as one. Actions that have minor differences are counted separately, as each of them could be highly personal.
}
\label{fig:evaluation_actions}
\Description{}
\end{figure*}

\subsubsection{Data Collection Procedure}
For each action, participants followed a consistent protocol (briefly mentioned in Sec.~\ref{subsub:methods:system:fewshot}) comprising two phases per shot: a 5-second \textit{free mode} and a 10-second \textit{record mode}. In the free mode, participants were free to rest or perform natural daily activities (negative data). Once entering the record mode, they performed the target actions (positive data).
This process was repeated across five rounds, with each round consisting of five consecutive shots.
Participants took a short break between two rounds to prevent physical fatigue and were asked to freely adjust the watch position between each round to increase data variance.
In total, we collected 25 shots for each target action.
Moreover, we leveraged the onboarding process at the beginning of the data collection to passively record participants' natural activities (about 5 minutes). This was used as additional data to augment the negative class\footnote{In real-world applications, we envision that such negative data can also be passively collected and implicitly embedded in the instruction process, thereby introducing minimal additional workload for the user}.

The \textit{free mode} segment was labeled as negative data, while the \textit{record mode} segment was labeled as positive data.
To prevent data contamination, the first two seconds during the record mode were excluded from training because these recordings were mixed with postural changes and arm movement.


\subsection{Offline Performance Evaluation}
\label{sub:model_evaluation:pipeline_evaluation}
We evaluated our pipeline by adding one or more actions as target actions.
For each action, we randomly selected two rounds of recordings as the training set (up to 10 shots), one round as the validation set (5 shots), and the remaining two rounds as the test set (10 shots). We repeated the training three times and calculated the average performance.

It is noteworthy that the model performance has two aspects: the window level and the action level. 
For the window level, each sliding window is counted as a binary classification data point (same as the model training process).
For the action level, windows are aggregated with a smoothing threshold of 3 (Sec.~\ref{subsub:methods:system:fewshot}) and represent a closer experience as real-life applications. Such aggregation significantly reduces the false negative and false positive.

\subsubsection{Prediction Performance with Different Number of Shots and Actions}
\label{subsub:model_evaluation:pipeline_evaluation:shots and action}
We evaluated the model performance by training on one to ten shots of the data.
For action recognition, we started by adding one action for each participant (\ie training binary classification models).
To evaluate the performance of multi-class classification models, we also experimented with customizing multiple actions (up to six, as each participant recorded five pre-designed actions and one custom action). This led to a total number of 63 combinations from one to six actions (\(\sum_{k=1}^6 \binom{6}{k}\)).
% We conducted this evaluation by systematically exploring all possible combinations of actions, ranging from one new action to six new actions. Subsequently, we trained models using different numbers of training samples, with the number of shots varying from one to ten.
In total, we trained and evaluated 49,140 models = 10 shot numbers $\times$ 63 action combinations $\times$ 26 participants $\times$ 3 repetitions. 
% We keep the original recognition results on these sliding window samples as  window level

We mainly focused on the action-level performance. \autoref{tab:action_shot_study} presents both the window-level and action-level results.
As shown in \autoref{fig:all_result_ges_num_action}, when using only one shot to add a new action (\ie the user performs the action only once), our framework achieved an average accuracy of 76.8\% and an F1 score of 74.8\%.
The recognition performance became better with more shots for training the model. With five shots of a new action, our framework attained an average accuracy of 84.7\% and an F1 score of 84.2\%. When using ten shots, our model's performance achieved 87.7\% and 87.2\%, respectively.

Recognizing multiple new actions simultaneously presented a greater challenge. However, compared to the performance of adding one action with five shots (84.7\% and 84.2\%), introducing three new actions (\ie four-class classification) with five shots each, the framework maintained a good average accuracy of 79.1\% and an F1 score of 78.1\%.
Even with six additional new actions and five shots each, the framework still achieved an average accuracy of 73.7\% and an F1 score of 72.3\%.
These results demonstrated the robustness and effectiveness of our pipeline for data-efficient action recognition.

\begin{figure*}[]
\centering 
\includegraphics[width=\linewidth]{figure/all_result_ges_num_action.png}
\caption{Few-shot Learning Pipeline Performance of Accuracy and F1 Score. We experimented with different numbers of shots using 1 to 10 samples to train a custom model. We also experimented with adding more than one target action simultaneously (\ie multi-class classification). Error bars indicate standard error. The same below.
}
\label{fig:all_result_ges_num_action}
\Description{}
\end{figure*}

% We report both window level and action level results in \autoref{tab:action_shot_study}
% \renewcommand{\arraystretch}{1.3}
% \begin{table}[]
% \centering
% \resizebox{1\textwidth}{!}{
% \begin{tabular}[t]{l|cccc|cccc}
% \toprule
% \multirow{2}{*}{\textbf{Shots}}  & \multicolumn{4}{c}{\textbf{Window-level}}      & \multicolumn{4}{c}{\textbf{Action-level}}       \\ \cline{2-9}
%  & \textbf{Acc}   & \textbf{Prec}  & \textbf{Rec}   & \textbf{F1}    & \textbf{Acc}   & \textbf{Prec}  & \textbf{Rec}   & \textbf{F1}      \\ \hline
% 1    & 0.614$\pm$0.006 & 0.700$\pm$0.007 & 0.614$\pm$0.006 & 0.571$\pm$0.008 & 0.768$\pm$0.007 & 0.810$\pm$0.007 & 0.768$\pm$0.007 & 0.748$\pm$0.009 \\
% 3    & 0.658$\pm$0.005 & 0.736$\pm$0.005 & 0.658$\pm$0.005 & 0.634$\pm$0.006 & 0.832$\pm$0.006 & 0.860$\pm$0.005 & 0.832$\pm$0.006 & 0.825$\pm$0.006 \\
% 5    & 0.670$\pm$0.005 & 0.746$\pm$0.005 & 0.670$\pm$0.005 & 0.648$\pm$0.006 & 0.847$\pm$0.005 & 0.871$\pm$0.005 & 0.847$\pm$0.005 & 0.842$\pm$0.006 \\
% 7    & 0.685$\pm$0.005 & 0.755$\pm$0.005 & 0.685$\pm$0.005 & 0.667$\pm$0.006 & 0.864$\pm$0.005 & 0.883$\pm$0.005 & 0.864$\pm$0.005 & 0.860$\pm$0.006 \\
% 10   & 0.702$\pm$0.005 & 0.763$\pm$0.005 & 0.702$\pm$0.005 & 0.688$\pm$0.006 & 0.877$\pm$0.005 & 0.890$\pm$0.005 & 0.877$\pm$0.005 & 0.873$\pm$0.006 \\
% \bottomrule
% \end{tabular}
% }
% \caption{Prediction Performance with Different Number of shots for Supporting One New Action.}
% \label{tab:action_shot_study}
% \end{table}
% \renewcommand{\arraystretch}{1.0}

\renewcommand{\arraystretch}{1.3}
\begin{table}[]
\centering
\caption{Detailed Few-shot Pipeline Performance with Different Numbers of Shots when Adding Personal Action.
Window-level results are based on each sliding window as a data point.
Action-level results are the aggregation of the sliding windows after smoothing post-processing (threshold=3) and are closer to real-life application scenarios.
}
\label{tab:action_shot_study}
\resizebox{1\textwidth}{!}{
\begin{tabular}[t]{l|cccc|cccc}
\toprule
\multirow{2}{*}{\textbf{Shots}}  & \multicolumn{4}{c}{\textbf{Window-level}}      & \multicolumn{4}{c}{\textbf{Action-level}}       \\ \cline{2-9}
 & \textbf{Acc}   & \textbf{Prec}  & \textbf{Rec}   & \textbf{F1}    & \textbf{Acc}   & \textbf{Prec}  & \textbf{Rec}   & \textbf{F1}      \\ \hline
1    & 0.614$\pm$0.006 & 0.700$\pm$0.007 & 0.614$\pm$0.006 & 0.571$\pm$0.008 & 0.768$\pm$0.007 & 0.810$\pm$0.007 & 0.768$\pm$0.007 & 0.748$\pm$0.009 \\
3    & 0.658$\pm$0.005 & 0.736$\pm$0.005 & 0.658$\pm$0.005 & 0.634$\pm$0.006 & 0.832$\pm$0.006 & 0.860$\pm$0.005 & 0.832$\pm$0.006 & 0.825$\pm$0.006 \\
5    & 0.670$\pm$0.005 & 0.746$\pm$0.005 & 0.670$\pm$0.005 & 0.648$\pm$0.006 & 0.847$\pm$0.005 & 0.871$\pm$0.005 & 0.847$\pm$0.005 & 0.842$\pm$0.006 \\
7    & 0.685$\pm$0.005 & 0.755$\pm$0.005 & 0.685$\pm$0.005 & 0.667$\pm$0.006 & 0.864$\pm$0.005 & 0.883$\pm$0.005 & 0.864$\pm$0.005 & 0.860$\pm$0.006 \\
10   & 0.702$\pm$0.005 & 0.763$\pm$0.005 & 0.702$\pm$0.005 & 0.688$\pm$0.006 & 0.877$\pm$0.005 & 0.890$\pm$0.005 & 0.877$\pm$0.005 & 0.873$\pm$0.006 \\
\bottomrule
\end{tabular}
}
\end{table}
\renewcommand{\arraystretch}{1.0}

% \renewcommand{\arraystretch}{1.3}
% \begin{table}[]
% \centering
% \resizebox{1\textwidth}{!}{
% \begin{tabular}[t]{c|c|cccc|cccc}
% \toprule
% \multirow{2}{*}{\textbf{Action Num}} & \multirow{2}{*}{\textbf{Shot Num}}  & \multicolumn{4}{c}{\textbf{Window-level}}      & \multicolumn{4}{c}{\textbf{Action-level}}       \\ \cline{3-10}
%  & & \textbf{Acc}   & \textbf{Prec}  & \textbf{Rec}   & \textbf{F1}    & \textbf{Acc}   & \textbf{Prec}  & \textbf{Rec}   & \textbf{F1}      \\ \hline
% \multirow{5}{*}{1} & 1    & 0.614{\tiny$\pm$0.006} & 0.700{\tiny$\pm$0.007} & 0.614{\tiny$\pm$0.006} & 0.571{\tiny$\pm$0.008} & 0.768{\tiny$\pm$0.007} & 0.810{\tiny$\pm$0.007} & 0.768{\tiny$\pm$0.007} & 0.748{\tiny$\pm$0.009} \\
%  & 3    & 0.658{\tiny$\pm$0.005} & 0.736{\tiny$\pm$0.005} & 0.658{\tiny$\pm$0.005} & 0.634{\tiny$\pm$0.006} & 0.832{\tiny$\pm$0.006} & 0.860{\tiny$\pm$0.005} & 0.832{\tiny$\pm$0.006} & 0.825{\tiny$\pm$0.006} \\
%  & 5    & 0.670{\tiny$\pm$0.005} & 0.746{\tiny$\pm$0.005} & 0.670{\tiny$\pm$0.005} & 0.648{\tiny$\pm$0.006} & 0.847{\tiny$\pm$0.005} & 0.871{\tiny$\pm$0.005} & 0.847{\tiny$\pm$0.005} & 0.842{\tiny$\pm$0.006} \\
%  & 7    & 0.685{\tiny$\pm$0.005} & 0.755{\tiny$\pm$0.005} & 0.685{\tiny$\pm$0.005} & 0.667{\tiny$\pm$0.006} & 0.864{\tiny$\pm$0.005} & 0.883{\tiny$\pm$0.005} & 0.864{\tiny$\pm$0.005} & 0.860{\tiny$\pm$0.006} \\
%  & 10   & 0.702{\tiny$\pm$0.005} & 0.763{\tiny$\pm$0.005} & 0.702{\tiny$\pm$0.005} & 0.688{\tiny$\pm$0.006} & 0.877{\tiny$\pm$0.005} & 0.890{\tiny$\pm$0.005} & 0.877{\tiny$\pm$0.005} & 0.873{\tiny$\pm$0.006} \\
% \hline
% \multirow{5}{*}{2} & 1    & 0.578{\tiny$\pm$0.003} & 0.622{\tiny$\pm$0.004} & 0.578{\tiny$\pm$0.003} & 0.521{\tiny$\pm$0.004} & 0.701{\tiny$\pm$0.004} & 0.717{\tiny$\pm$0.004} & 0.701{\tiny$\pm$0.004} & 0.667{\tiny$\pm$0.004} \\
%  & 3    & 0.639{\tiny$\pm$0.003} & 0.684{\tiny$\pm$0.003} & 0.639{\tiny$\pm$0.003} & 0.608{\tiny$\pm$0.003} & 0.783{\tiny$\pm$0.003} & 0.797{\tiny$\pm$0.003} & 0.783{\tiny$\pm$0.003} & 0.770{\tiny$\pm$0.003} \\
%  & 5    & 0.667{\tiny$\pm$0.003} & 0.707{\tiny$\pm$0.003} & 0.667{\tiny$\pm$0.003} & 0.643{\tiny$\pm$0.003} & 0.817{\tiny$\pm$0.003} & 0.827{\tiny$\pm$0.003} & 0.817{\tiny$\pm$0.003} & 0.808{\tiny$\pm$0.003} \\
%  & 7    & 0.690{\tiny$\pm$0.003} & 0.725{\tiny$\pm$0.003} & 0.690{\tiny$\pm$0.003} & 0.672{\tiny$\pm$0.003} & 0.838{\tiny$\pm$0.003} & 0.847{\tiny$\pm$0.003} & 0.838{\tiny$\pm$0.003} & 0.832{\tiny$\pm$0.003} \\
%  & 10   & 0.708{\tiny$\pm$0.003} & 0.738{\tiny$\pm$0.003} & 0.708{\tiny$\pm$0.003} & 0.695{\tiny$\pm$0.003} & 0.854{\tiny$\pm$0.003} & 0.862{\tiny$\pm$0.003} & 0.854{\tiny$\pm$0.003} & 0.849{\tiny$\pm$0.003} \\
% \hline
% \multirow{5}{*}{3} & 1    & 0.566{\tiny$\pm$0.002} & 0.572{\tiny$\pm$0.003} & 0.566{\tiny$\pm$0.002} & 0.511{\tiny$\pm$0.003} & 0.670{\tiny$\pm$0.003} & 0.667{\tiny$\pm$0.004} & 0.670{\tiny$\pm$0.003} & 0.636{\tiny$\pm$0.003} \\
%  & 3    & 0.630{\tiny$\pm$0.002} & 0.654{\tiny$\pm$0.003} & 0.630{\tiny$\pm$0.002} & 0.596{\tiny$\pm$0.003} & 0.758{\tiny$\pm$0.002} & 0.766{\tiny$\pm$0.003} & 0.758{\tiny$\pm$0.002} & 0.742{\tiny$\pm$0.003} \\
%  & 5    & 0.660{\tiny$\pm$0.002} & 0.683{\tiny$\pm$0.002} & 0.660{\tiny$\pm$0.002} & 0.635{\tiny$\pm$0.003} & 0.791{\tiny$\pm$0.002} & 0.799{\tiny$\pm$0.003} & 0.791{\tiny$\pm$0.002} & 0.781{\tiny$\pm$0.003} \\
%  & 7    & 0.682{\tiny$\pm$0.002} & 0.701{\tiny$\pm$0.002} & 0.682{\tiny$\pm$0.002} & 0.662{\tiny$\pm$0.003} & 0.811{\tiny$\pm$0.002} & 0.819{\tiny$\pm$0.002} & 0.811{\tiny$\pm$0.002} & 0.803{\tiny$\pm$0.002} \\
%  & 10   & 0.701{\tiny$\pm$0.002} & 0.718{\tiny$\pm$0.002} & 0.701{\tiny$\pm$0.002} & 0.686{\tiny$\pm$0.003} & 0.826{\tiny$\pm$0.002} & 0.835{\tiny$\pm$0.002} & 0.826{\tiny$\pm$0.002} & 0.822{\tiny$\pm$0.002} \\
% \hline
% \multirow{5}{*}{4} & 1    & 0.557{\tiny$\pm$0.002} & 0.553{\tiny$\pm$0.003} & 0.557{\tiny$\pm$0.002} & 0.501{\tiny$\pm$0.003} & 0.659{\tiny$\pm$0.003} & 0.656{\tiny$\pm$0.003} & 0.659{\tiny$\pm$0.003} & 0.627{\tiny$\pm$0.003} \\
%  & 3    & 0.619{\tiny$\pm$0.002} & 0.635{\tiny$\pm$0.003} & 0.619{\tiny$\pm$0.002} & 0.583{\tiny$\pm$0.003} & 0.739{\tiny$\pm$0.003} & 0.745{\tiny$\pm$0.003} & 0.739{\tiny$\pm$0.003} & 0.723{\tiny$\pm$0.003} \\
%  & 5    & 0.647{\tiny$\pm$0.002} & 0.663{\tiny$\pm$0.003} & 0.647{\tiny$\pm$0.002} & 0.621{\tiny$\pm$0.003} & 0.770{\tiny$\pm$0.002} & 0.778{\tiny$\pm$0.003} & 0.770{\tiny$\pm$0.002} & 0.761{\tiny$\pm$0.003} \\
%  & 7    & 0.667{\tiny$\pm$0.002} & 0.681{\tiny$\pm$0.003} & 0.667{\tiny$\pm$0.002} & 0.646{\tiny$\pm$0.003} & 0.790{\tiny$\pm$0.002} & 0.799{\tiny$\pm$0.002} & 0.790{\tiny$\pm$0.002} & 0.783{\tiny$\pm$0.002} \\
%  & 10   & 0.689{\tiny$\pm$0.002} & 0.701{\tiny$\pm$0.003} & 0.689{\tiny$\pm$0.002} & 0.673{\tiny$\pm$0.003} & 0.807{\tiny$\pm$0.002} & 0.817{\tiny$\pm$0.002} & 0.807{\tiny$\pm$0.002} & 0.803{\tiny$\pm$0.002} \\
% \hline
% \multirow{5}{*}{5} & 1    & 0.542{\tiny$\pm$0.003} & 0.524{\tiny$\pm$0.004} & 0.542{\tiny$\pm$0.003} & 0.481{\tiny$\pm$0.004} & 0.636{\tiny$\pm$0.004} & 0.618{\tiny$\pm$0.005} & 0.636{\tiny$\pm$0.004} & 0.600{\tiny$\pm$0.005} \\
%  & 3    & 0.610{\tiny$\pm$0.003} & 0.612{\tiny$\pm$0.004} & 0.610{\tiny$\pm$0.003} & 0.573{\tiny$\pm$0.004} & 0.722{\tiny$\pm$0.004} & 0.724{\tiny$\pm$0.004} & 0.722{\tiny$\pm$0.004} & 0.705{\tiny$\pm$0.004} \\
%  & 5    & 0.643{\tiny$\pm$0.004} & 0.649{\tiny$\pm$0.004} & 0.643{\tiny$\pm$0.004} & 0.614{\tiny$\pm$0.004} & 0.755{\tiny$\pm$0.004} & 0.762{\tiny$\pm$0.004} & 0.755{\tiny$\pm$0.004} & 0.745{\tiny$\pm$0.004} \\
%  & 7    & 0.663{\tiny$\pm$0.004} & 0.670{\tiny$\pm$0.004} & 0.663{\tiny$\pm$0.004} & 0.640{\tiny$\pm$0.004} & 0.774{\tiny$\pm$0.004} & 0.782{\tiny$\pm$0.004} & 0.774{\tiny$\pm$0.004} & 0.768{\tiny$\pm$0.004} \\
%  & 10   & 0.684{\tiny$\pm$0.004} & 0.690{\tiny$\pm$0.004} & 0.684{\tiny$\pm$0.004} & 0.666{\tiny$\pm$0.004} & 0.795{\tiny$\pm$0.004} & 0.803{\tiny$\pm$0.004} & 0.795{\tiny$\pm$0.004} & 0.790{\tiny$\pm$0.004} \\
% \hline
% \multirow{5}{*}{6} & 1    & 0.533{\tiny$\pm$0.009} & 0.510{\tiny$\pm$0.010} & 0.533{\tiny$\pm$0.009} & 0.479{\tiny$\pm$0.010} & 0.617{\tiny$\pm$0.010} & 0.613{\tiny$\pm$0.011} & 0.617{\tiny$\pm$0.010} & 0.589{\tiny$\pm$0.011} \\
%  & 3    & 0.599{\tiny$\pm$0.009} & 0.592{\tiny$\pm$0.010} & 0.599{\tiny$\pm$0.009} & 0.561{\tiny$\pm$0.010} & 0.703{\tiny$\pm$0.010} & 0.706{\tiny$\pm$0.011} & 0.703{\tiny$\pm$0.010} & 0.686{\tiny$\pm$0.011} \\
%  & 5    & 0.630{\tiny$\pm$0.009} & 0.628{\tiny$\pm$0.011} & 0.630{\tiny$\pm$0.009} & 0.601{\tiny$\pm$0.010} & 0.734{\tiny$\pm$0.009} & 0.739{\tiny$\pm$0.011} & 0.734{\tiny$\pm$0.009} & 0.723{\tiny$\pm$0.010} \\
%  & 7    & 0.650{\tiny$\pm$0.009} & 0.649{\tiny$\pm$0.010} & 0.650{\tiny$\pm$0.009} & 0.625{\tiny$\pm$0.010} & 0.756{\tiny$\pm$0.009} & 0.764{\tiny$\pm$0.010} & 0.756{\tiny$\pm$0.009} & 0.749{\tiny$\pm$0.009} \\
%  & 10   & 0.673{\tiny$\pm$0.009} & 0.673{\tiny$\pm$0.010} & 0.673{\tiny$\pm$0.009} & 0.653{\tiny$\pm$0.010} & 0.776{\tiny$\pm$0.009} & 0.785{\tiny$\pm$0.010} & 0.776{\tiny$\pm$0.009} & 0.771{\tiny$\pm$0.010} \\
% \bottomrule
% \end{tabular}
% }
% \caption{Prediction Performance with Different Number of Shots and Actions.}
% \label{tab:action_shot_study}
% \end{table}
% \renewcommand{\arraystretch}{1.0}


\subsubsection{Prediction Performance of Each New Gesture with Different Number of Shots}
\label{subsub:model_evaluation:pipeline_evaluation:one new gesture}


We further compared the recognition performance across actions.
As shown in \autoref{fig:all_result_shot_num_action}, most of the 17 actions exhibited good performance. Using only one shot, about half of the actions achieved an F1 score above 75\%. When the number of shots increased to five, 14 out of 17 actions surpassed this threshold. With ten shots, performance improved further for most actions, with 12 out of 17 actions achieving an F1 score above 85\%.
\textit{Hair Pulling} appeared to be an exception. Its performance did not improve with more samples after five shots. This was probably due to the overly large variance of the \textit{Hair Pulling} action, even performed by the same individual, and it was challenging for a model to achieve reliable performance even with a limited amount of additional data.

Overall, these results indicate our framework has good learning ability for new actions.

\begin{figure*}[]
\centering 
\includegraphics[width=\linewidth]{figure/all_result_shot_num_action.png}
\caption{Model Performance of Recognizing Each Action with 1, 5, or 10 Shots. For consistency, each action was added alone (\ie binary classification model). The ``(User Num)'' indicates how many users did this action. The five pre-determined actions (Lip Tearing, Nail Biting, Face Scratching, Eye Rubbing, and Leg Shaking) have the total number of participants (26), and other self-defined actions are more scattered.}
\label{fig:all_result_shot_num_action}
\Description{}
\end{figure*}

% \subsubsection{Ablation Studies of Pipeline Stages}
% \label{subsub:model_evaluation:pipeline_evaluation:ablation}
% \orson{@Ying to add more details}

% To measure the effectiveness of different stages in our few-shot pipeline, we further conducted an ablation study by removing different stages in our pipeline.
% \autoref{tab:ablation_study} indicates xxxx. \orson{@Ying to add more details of the results interpretation.}

% \begin{table}[t]
\vspace{-13pt}
\caption{Ablation results with response number under fine-tuning setting. See Reward Margins in~\Cref{tab:rm}. \vspace{-0.5em}}
\label{tab:ablation}
\vskip 0.1in
\begin{center}
\scalebox{0.75}{
\begin{tabular}{clcccc}
\toprule
\multirow{1}{*}{\textbf{Number}} & \multirow{1}{*}{\textbf{Method}} & \textbf{BLEU}$\uparrow$ & \textbf{Reward} & $\textbf{RM}_{\text{DPO}}$$\uparrow$ & $\textbf{RM}_{\text{R-DPO}}$$\uparrow$ \\
\midrule
\multirow{2}{*}{\textbf{5}} & \textbf{DPO-BT} & \textbf{0.229} & \textbf{{0.432}} & 0.166 & -0.516 \\

& \textbf{DPO-HPS} & \textbf{0.229} & 0.431 & \textbf{0.600} & \textbf{-0.273} \\
\midrule
\multirow{2}{*}{\textbf{20}} & \textbf{DPO-BT} & \textbf{0.231} & 0.430 & 0.227 & -0.490 \\

& \textbf{DPO-HPS} & 0.224 & \textbf{{0.432}} & \textbf{0.822} & \textbf{-0.181} \\
\midrule
\multirow{2}{*}{\textbf{50}} & \textbf{DPO-BT} & \textbf{0.230} & \textbf{0.431} & 0.279 & -0.507 \\

& \textbf{DPO-HPS} & \textbf{0.230} & \textbf{0.431} & \textbf{1.645} & \textbf{1.037} \\
\midrule
\multirow{2}{*}{\textbf{100}} & \textbf{DPO-BT} & 0.230 & \textbf{0.431} & 0.349 & -0.455 \\

& \textbf{DPO-HPS} & \textbf{{0.232}} & 0.430 & \textbf{{2.723}} & \textbf{{2.040}} \\
\bottomrule
\end{tabular}}
\end{center}
\vspace{-1em}
\vspace{-9pt}
\end{table}


\section{Intervention Evaluation}
\label{sec:intervention_evaluation}
The promising model performance in Sec.~\ref{sub:model_evaluation:pipeline_evaluation} has validated the effectiveness of our few-shot learning pipeline.
Building upon the pipeline, we further conducted a user study to evaluate the effectiveness of \projectname and compared it against a rule-based baseline intervention system.

\subsection{Participants}
\label{sub:intervention_evaluation:participants}
With IRB approval, we recruited the same set of participants in Sec.~\ref{sub:model_evaluation:data_collection} for a follow-up intervention study. 
In the previous data collection, participants performed five per-determined actions and a self-defined action. In this study, they were asked to select one of the six actions that they had the strongest need for intervention.
This action was set as the target action for intervention during the study.
Among the 26 participants, 5 of them did not follow the study protocol. Their results were removed as outliers.  This section focused on the findings based on the remaining 21 participants.

\subsection{Intervention Setting}
\label{sub:intervention_evaluation:setting}
Since personal undesirable actions are inherently difficult to predict or control, we designed an intervention experience that closely mirrors real-life contexts to enhance ecological validity, encouraging participants to perform these actions under more natural conditions.
Our initial conversation with participants indicated two common scenarios where they tended to perform these actions: when they were in an engaging task with a relaxing state (\eg watching an interesting movie or a reality show with dramatic twists and turns); and when they were bored or disengaged (\eg mindlessly scrolling through social media or watching a tedious video) \footnote{Several participants also mentioned the scenarios under pressure or stress. Considering the feasibility and ethics of a multi-hour intervention study, we did not provide this option.}.
Therefore, we set up two types of video-watching tasks and allowed participants to pick the type in which they tended to perform more undesirable actions.

The first type included \textit{engaging} videos. We prepare a set of multi-hour videos for participants to choose from, such as the Harry Potter movie series, sports competitions, and mystery/detective shows.
The second type was watching \textit{disengaging} videos. Examples include cycling or driving route videos, math problem explanations, and public health lecture videos.
Participants sat in a quiet room with a laptop on the table and watched the video they selected, as shown in \autoref{fig:intervention_setup}(a) and (b).
During the video-watching, participants were not interrupted by the experimenter, simulating the real-life setting. 

\begin{figure*}[]
\centering 
\begin{subfigure}[t]{1\textwidth}
    \centering
    \includegraphics[width=\textwidth]{figure/setting.jpg}
    \caption{Intervention Room Setup}
    \label{subfig:intervention_setup:design}
\end{subfigure}
\caption{\projectname Intervention Evaluation Setup. (a) The sketch of the study room and apparatus setup for intervention. (b) The photo of a participant in the room. (c) The video from the camera on the corner that records the ground truth.}
\label{fig:intervention_setup}
\Description{}
\end{figure*}

\subsection{Study Design and Procedure}
\label{sub:intervention_evaluation:design}
We adopted a within-subject design and compared our AI-powered \projectname against a rule-based intervention system.
In the rule-based system, a regular notification (the same interface as \autoref{subfig:interface_design:intervention}) was delivered every 10 minutes, regardless of whether the user did the action.
To mitigate the effect of the two systems outputting different numbers of notifications, we further added restrictions in \projectname so that the number of delivered notifications would be in the range of $\times 0.5$ to $\times 2$ as the baseline system.
This was achieved by forcefully delivering a notification if there was no intervention by the end of each 20-minute window ($\times 0.5$ times of interventions in minimum).
With the 5-min cool-down setup, \projectname can only deliver up to one intervention every 5 minutes, which would be no more than $\times 2$ times of interventions as the baseline.

Our study procedure was designed as follows. After selecting the personal target undesirable action and the task type (engaging vs disengaging), participants would calibrate and familiarize themselves with the intervention system and study setup. They then attended two intervention sessions in total, one session per day. We counterbalanced the order between \projectname and the baseline system, and participants were blind to the order of the two systems.
After familiarizing themselves with the room environment and setup, participants went through each intervention session with three stages (in total 130 minutes): (1) a 30-minute \textit{pre-intervention stage}, where there was no intervention delivered; (2) a 90-minute \textit{intervention stage}, where \projectname or the baseline system would deliver interventions as designed; and (3) a 10-minute \textit{post-intervention} stage, where no more intervention was delivered to observe any lasting effect \footnote{Due to the restrictions of the room booking time and device battery, we regretfully could not do a longer post-intervention stage. We recognize this as a limitation of our study in discussion.}.

The whole intervention session was video-recorded by a camera from the ceiling, positioned at an angle to capture participants' micro-actions and collect ground truth (see \autoref{fig:intervention_setup}(c)). We manually annotated the video and calculated the number and duration of the target actions during the three stages.
We collected participants' Self-Report of Habit Strength of the target action~\cite{verplanken2003reflections} before and after each session.
After the post-intervention stage, we further collected quantitative data from participants with a questionnaire that includes System Usability Scale (SUS) survey~\cite{bangor2008empirical} and Working Alliance Inventory (WAI, short revision)~\cite{munder2010working}.
In addition, we conducted a brief semi-structured interview to collect qualitative feedback on the intervention experience from each participant.

In total, the two sessions took around 5 hours for each participant. To reduce user fatigue, the two sessions were scheduled on two different days within a week. Participants were compensated with \$50 for the intervention study.
% \orson{@Yancheng to double-check and fill in the content in this section.}

\subsection{Intervention Results}
\label{sub:intervention_evaluation:intervention_results}
We first summarize the quantitative results from our study.
We coded the recorded videos by documenting the duration of target actions performed by participants every 10 minutes across the three stages.
Since participants had diverse behavior patterns, we normalized the results with each individual's target action duration in the pre-intervention stage as the reference.
The \textit{relative duration} was calculated by dividing the average duration of target actions per 10 minutes in both the intervention and post-intervention stages by that of the pre-intervention stage. A lower relative duration means more reduction of the target actions compared to the pre-intervention stage.

% Overall, the real-time pipeline of \projectname had a consistent performance as Sec.~\ref{sec:model_evaluation}.
% , with an overall accuracy of XX.X\% and a false positive rate of XX.

% \subsubsection{Controlled Effects on the Number of Interventions.}
% As the frequency of reminders could be one of the factors affecting intervention effectiveness, we compared the number of interventions between AI-based and rule-based intervention modes. A significant difference was found between the two modes. The mean number of interventions in the AI-based group was 11.76 and in the rule-based group was 9.00. This difference was statistically significant according to an independent-samples t-test (t(20) = 2.97, p < .01). As a result, we included the number of interventions as a covariate in our statistical model to control for its effect.

% \orson{@Yancheeng, check whether there is a significant difference between the number of interventions. Hopefully no. If it is, we would say that we will add the number of interventions in the statistical model to control its effect. I don't think we would need a figure here.}
\begin{figure*}[]
\centering 
\includegraphics[width=\linewidth]{figure/duration_result.pdf}
\caption{(a) Relative Duration of target action every 10 minutes in intervention and post-intervention stages (compared to the pre-intervention stage). A number lower than 1.0 means that an individual performed fewer target actions after intervention.
(b) Average Relative Duration of target action over time. The dashed lines fit the last 10 minutes of the pre-intervention stage and the rest of the session.
}
\label{fig:duration_result}
\Description{}
\end{figure*}

\subsubsection{Reduction of the Duration of Target Actions by Intervention.}
% After the study, we coded the recorded videos by documenting the start and end times of target actions performed by all participants during the last three stages.
We compare the relative duration between \projectname and the baseline in both intervention and post-intervention stages.
Since participants received slightly more notifications in \projectname during the intervention stage (on average 11.8 \vs 9.0 times per session), we controlled the effect of the number of notifications by using generalized linear mixed models (GLMMs).
Specifically, a GLMM had relative duration as the dependent variable, with the intervention method (AI-based in \projectname \vs rule-based in baseline) and the number of notifications as the main factors.

As shown in the left of Fig.\ref{fig:duration_result}(a), during the intervention stage, \projectname resulted in 36.0 $\pm$ 22.6\% of the duration compared to the pre-intervention stage (\ie a reduction of 64.0\% of the target undesirable action), and the baseline system led to 65.0 $\pm$ 47.5\% of the duration (\ie a reduction of 35.0\%).
We fitted a GLMM to compare the two intervention methods.
Our results revealed the significant difference between the two methods: \projectname significantly outperformed the baseline by 29.0\% more reduction of the target undesirable action ($\chi^2_1=6.32, p < .05$). Meanwhile, the number of notifications does not show significance ($\chi^2_1=0.53, p = 0.47$). These results suggest that the advantage of \projectname was mainly attributed to the AI-based intervention method.

In addition, although our post-intervention stage was short, both methods showed promising signals of a potential lasting effect when the intervention was gone (13.9 $\pm$ 16.8\% for the \projectname; 37.7 $\pm$ 37.2\% for the baseline), as shown in the right of Fig.\ref{fig:duration_result}(a).
We fitted another GLMM on the post-intervention data. The results also indicate the significance of the intervention method ($\chi^2_1=10.04, p<0.01$), but not the number of notifications ($\chi^2_1=0.12, p=0.73$).
This is consistent with the result of the intervention stage, further demonstrating the superior performance of \projectname over the baseline method.


% The model also included a non-significant intercept (b = 0.38, OR = 1.46, 95\% CI [0.90, 2.37], p = .131) and intervention times (b = 0.00, OR = 1.00, 95\% CI [0.96, 1.04], p = .933).
% These results indicate that \projectname was more effective in reducing participants' target actions compared to scheduled reminders, and this effect was not significantly associated with the number of intervention times. The GLMM analysis also showed that the relative duration of target actions during the post-intervention stage remained significantly lower in the AI-based condition compared to rule-based intervention (b = 0.25, OR = 1.28, 95\% CI [1.05, 1.55], p < .05). The model included a non-significant intercept (b = 0.12, OR = 1.13, 95\% CI [0.78, 1.63], p = .537) and intervention times (b = 0.00, OR = 1.00, 95\% CI [0.97, 1.03], p = .910). These findings suggest that \projectname demonstrated superior lasting effects in reducing target actions compared to the rule-based condition during the short-term post-intervention period.

% \orson{@Yancheng to add the results figure and write up the findings}

\subsubsection{Intervention Effectiveness over Time.}
To investigate changes in the duration of target action during the study session, we visualize the change of participants' target action duration throughout the study (see \autoref{fig:duration_result}(b)).
Both intervention methods showed a clear and significant decreasing trend once participants entered the intervention stage.
The fitted lines in \autoref{fig:duration_result}(b) indicate that \projectname achieved more duration reduction ($m=-4.8\%$ per 10-minute) compared to the baseline ($m=-4.1\%$) over the intervention session.
In particular, \projectname had a more rapid initial decrease and maintained consistently lower levels throughout the rest of the session compared to the rule-based baseline.
Overall, \projectname demonstrated stronger cumulative effects.

\subsubsection{Difference across Task Types.}
During the study, we asked participants to pick their own preferred task types between watching engaging (N=11) or disengaging videos (N=10).
\autoref{fig:duration_result_breakdown} presents the breakdown of the task type in \autoref{fig:duration_result}(a).
We fitted GLMMs with task type as another main factor and observed a marginal significance of the interaction between the intervention method and the task type ($\chi^2_1=3.27, p=0.07<0.1$). This was only during the intervention stage, but not the post-intervention stage.
\autoref{fig:duration_result_breakdown}(a) and (b) indicate that the advantage of \projectname during the intervention stage was more salient when participants were watching engaging videos ($\Delta=42.3\pm49.6\%$) compared to when they were watching disengaging videos ($\Delta=14.2\pm22.5\%$).
This could be due to the fact that participants were more interruptable or receptive in less engaging tasks~\cite{pielot2017beyond,mishra_detecting_2021,choi_multi-stage_2019}, thus even a basic rule-based intervention could effectively reduce the target actions. However, in more engaging tasks, accurate and just-in-time reminders are more effective than basic ones.

\begin{figure*}[]
\centering 
\includegraphics[width=0.7\linewidth]{figure/engage_result.pdf}
\caption{
(a) Relative duration of target action for participants who watched engaging videos.
(b) Relative duration of target action for participants who watched disengaging videos.
}
\label{fig:duration_result_breakdown}
\Description{}
\end{figure*}

\subsubsection{Survey Outcomes.}
In addition to the objective measurement, we also compare participants' subjective reports on the SUS, WAI, and the change of the habit strength.
Overall, participants reported that \projectname had better usability (SUS: 73.3$\pm$12.8) than the baseline (66.8$\pm$15.9), with significance through a Wilcoxon rank sum test ($p<0.05$).
\projectname achieved a SUS score over 70, indicating acceptable usability. In both methods, false positive notifications were inevitable and could introduce participants' confusion or surprise, which could explain the subpar SUS scores in general.

Interestingly, the results of WAI and habit strength did not indicate such a difference. Participants had similar reports of the relationship with the system (WAI score: 42.1$\pm$7.1 for the \projectname \vs 41.9$\pm$9.3 for the baseline, $p=0.58$).
The change of habit strength between the pre- and post-intervention stages is also minimal ($\Delta$ of habit strength score: $-4.2\pm5.5$ for the \projectname \vs $-5.5\pm6.6$ for the baseline, $p=0.25$).
This was probably due to the fact that the intervention sessions were not long enough to form a long-term alliance between users and the system, or to influence longitudinal behaviors or habits.
Our qualitative results from semi-structured interviews provide more nuanced insights into these results.


\subsection{Qualitative Results}
\label{sub:intervention_evaluation:qualitative_results}
% Moreover, our interview results also revealed interesting insights into \projectname that enables a human-AI collaborative intervention experience.
All interviews were recorded and transcribed. We adopted a simple content analysis framework~\cite{prior2014content}. One author took extensive notes during the interviews, went through the scripts to categorize themes and count their frequency, and discussed with two other authors until convergence.
We summarize our key findings below.

% \orson{@yancheng to add more details and quotes.}

\subsubsection{Perception of AI-powered Intervention}
Multiple participants reported that the AI-powered system possessed a sense of presence or "\textit{having a soul}". For instance, P10 noted, "\textit{[\projectname{}] resembles a habit instructor, or even like my mom... who would gently remind me when I scratch my head.}"
P18 remarked, "\textit{This system seems to read my mind, anticipating when I'm about to bite my lips and reminding me just in time. Sometimes I felt like I was sneaking around when making these actions.}"
Compared to the rule-based condition, \projectname's interventions appeared to foster greater self-reflection among users.
Notably, P19 even perceived the AI's reminders as rewards: "\textit{After being caught [touching my face] several times initially, I managed to control myself for a while. Then, even if the system reminded me again, I felt it was affirming my progress, like receiving a reward.}"
In contrast, the rule-based condition yielded opposite effects, "\textit{This mode of notification felt random to me - it was just like a machine}" (P02).

However, some participants also had a negative experience with \projectname, especially when it did not detect the actions accurately (mostly false positive). For example, P08 mentioned that \projectname had limited impact, and that they also felt a sense of distrust. "\textit{At first, when it reported errors a few times, I tried to look for reasons elsewhere. But it kept making mistakes, which became frustrating. When it occasionally got something right, I thought it was just luck!}" Participants could lose trust in \projectname when the system made mistakes at the beginning of their interaction.
This is supported by prior research in other human-AI interaction systems~\cite{swaroop2024accuracy,jacobs2021designing}.


\subsubsection{Illusory Amplification of Intervention Strength}
We noticed a surprisingly interesting phenomenon: 
Several users (P09, P10, and P16) reported that the vibration strength of the AI-based intervention in \projectname felt stronger than that of the rule-based intervention. However, the vibration setup was identical in the two sessions.
Even after we explained the specific intervention methods after the two study sessions, P16 stated, "\textit{Not only did I subjectively feel that Mode B [our \projectname method] gave me a stronger sense of motion restraint, but it also seemed to vibrate more intensely. Are you sure it's really the same setting?}"
This indicated that participants might develop an illusory or distorted perception of the intervention's strength when the interventions were delivered just-in-time.
We discuss this more in Sec.~\ref{sub:discussion:distorted}.

\subsubsection{Diverse Patterns of Human-AI Collaborative Relationship}
Users exhibited diverse patterns of engagement with the AI system. Some participants demonstrated adaptive behavioral modification in response to \projectname's reminders. As P14 described, "\textit{Every time I shook my leg, it would remind me, which made me increasingly hesitant to move}". This was aligned with our original design goal of introducing AI-powered JITI.

Other than reducing the target actions, we also observed other behavior patterns. One pattern emerged where participants developed an interesting competitive relationship with the AI for user agency. For instance, P8 articulated this sentiment: "\textit{I wanted to compete with it - I tried to resist the urge just so it wouldn't catch me.}" This competitive spirit evolved into experimental behavior for some users, who attempted to understand and control the system's underlying logic. P18's experience exemplified this progression: "\textit{Initially, I felt caught red-handed with every reminder. Later, I noticed it wouldn't always detect my subtle movements, so I started experimenting with the notification logic, trying to gain control over the reminders. Eventually, though, I made peace with it and lost the urge to perform the action altogether.}" These participants wanted to gain better agency in this human-AI relationship.

In addition, some participants developed playful interactions with the system, treating it as an engaging companion rather than a mere monitoring tool. For example, P17 shared: "\textit{When the video was boring... I just wanted to goof around a bit. This thing was actually keeping an eye on me, so I'd mess with it for fun, play around with it, and boom - it would react right away. Kinda helped wake me up a bit? It was basically like playing a game.}"
Overall, these diverse patterns between users and \projectname suggest a set of potential collaborative relationships between the two sides. We discuss this finding in Sec.~\ref{sub:discussion:collaboration}.

\section{Discussion}
\label{sec:discussion}
In this work, we propose to leverage few-shot learning to enable users to self-define personal undesirable actions for personalized intervention on smartwatches.
We developed a three-stage pipeline that began with a self-supervised, pre-trained IMU model for robust feature extraction, then fine-tuned it for accurate human activity recognition, and finally enhanced it with data augmentation and synthesis that enabled rapid customization of new user-defined actions using only a small number of examples. 
We implemented this pipeline on a smartwatch as a real-time intervention system, \projectname, and demonstrated its effectiveness and advantages over the rule-based method through a multi-hour user study.
In this section, we discuss some interesting takeaways from our study, together with our vision of how \projectname can be generally applied to other health domains. We also briefly summarize the limitations of our work.


\subsection{Distorted Perception with AI-powered Intervention}
\label{sub:discussion:distorted}
During the study, we observed an interesting phenomenon where some participants developed a distorted perception towards their own actions or the intervention (see Sec.~\ref{sub:intervention_evaluation:qualitative_results}).
For instance, several participants felt \projectname's vibrations were stronger than the baseline (yet the actual strength of vibration remained constant), and some felt they did the target actions more frequently with \projectname (yet the objective data indicated otherwise).
There are several potential interpretations of such interesting observations.
The distorted perception might be caused by participants' heightened awareness of the AI-guided interventions: because \projectname more accurately and promptly caught the target actions, users started to pay extra and prolonged attention to any intervention. This could leave a stronger impression on them, and subsequently, they found it stronger or more frequent.
Another potential explanation is that the participants, often associating their personal and idiosyncratic undesirable actions with ``wrong-doing'' and thus responding with negative emotions, might have subconsciously perceived their undesirable actions as being more frequent due to the \projectname's more precise and timely feedback eliciting stronger negative emotions. This, combined with an emotional interpretation of being 'corrected', may have amplified their perception of the intervention's intensity (vibration strength) and created the mistaken impression of performing these actions excessively.

Meanwhile, it is an interesting open question of how long such perception will last from a longitudinal intervention perspective. Depending on the cases, the growing self-awareness and/or reliability of AI could potentially assist users in building a long-term habit to reduce the target action, or on the contrary, the effect may fade away due to the AI intervention method no longer being novel or enticing.
Future work can explore the lasting effect of the intervention, the corresponding perception, as well as user engagement in a long-term, field-based intervention study.~\cite{middleton2013long, short2018measuring, wei2020design}.


\subsection{Towards Human-AI Collaborative Interventions}
\label{sub:discussion:collaboration}
Users' mental models of \projectname varied significantly. Some viewed it as a passive watchdog, and some viewed it as a playful interactive system, while others sought to take greater agency in the moment of intervention delivery.
Our findings show the potential for and benefit of developing a collaborative relationship between humans and AI for behavioral intervention.
An AI system can provide appropriate support to users and help them achieve effective intervention outcomes.
Such collaboration is closely relevant to the vision of just-in-time adaptive interventions (JITAIs)~\cite{nahum-shani_translating_2021, nahum2018just}, where the delivery timing and methods of intervention are designed to be dynamically adapting to an individual's internal state and surrounding context.

For instance, for users who see the system as a passive monitor, the system can provide the option for them to configure the frequency and style of intervention (\eg higher/lower-intensity vibrations or consolidated notifications), ensuring the AI remains in the background but still provides supportive nudges.
Taking one step further, the AI system may analyze user behavior over time and suggest new setups or goals for users with transparency (\eg transitioning from nail-biting to managing stress). Users can accept, modify, or reject these suggestions, creating a dialogue where AI acts as a coach or collaborator rather than a rigid enforcer of predefined behaviors.
Meanwhile, for those who see AI as a proactive system, one promising avenue is to incorporate user feedback into the AI's learning process~\cite{orzikulova2024time2stop}. Users can label the AI's predictions as accurate or not, which could serve as input for the model to further adapt to the user and improve performance over time (\eg through reinforcement learning).
Combined with contextual information that can potentially be inferred from sensors~\cite{xu2023globem}, such feedback can enable more precise, context-sensitive and personalized JITIs.
In addition, the system would periodically prompt users to reassess their goals and update intervention targets, ensuring long-term relevance and efficacy.

It is noteworthy that such a human-AI collaboration paradigm needs to follow the principles of transparency and ethical design.
Other than the options mentioned above, namely custom configurations and continuous feedback, users should have visibility into the system's functionality and action logic regardless of the specific collaboration setup. This is important to provide users with agency and build their trust in the system.

\subsection{Beyond Smartwatch and Broader Customization}
In this work, our real-time intervention was implemented on a smartwatch. However, our proposed idea of empowering users to define any personal action and achieve a personalized intervention system can be more broadly applied to other domains.
Instead of relying solely on a watch-based IMU, we can explore other body-based sensor arrays (\eg headbands, rings, or joint sensors) to capture a more diverse range of behaviors in real time.
This would enable the system to accommodate a wide variety of undesirable actions or habits, such as posture corrections and fidgeting management.
In addition, beyond physical interventions, future customization can also delve into psychological or mental health support.
For instance, individuals dealing with obsessive-compulsive disorder (OCD) or other habitual thought/action patterns could define personal triggers (\eg a particular repetitive motion or behavioral cue) and receive timely AI-driven interventions.
Such holistic approaches highlight the flexibility and scalability of our pipeline.
By enabling user-defined actions, we open up possibilities for long-term and effective management of both physical and psychological well-being using a multitude of wearable and sensor-based platforms.

\subsection{Limitations}

Despite \projectname's positive outcome and the promising insights generated, we recognize some limitations in our study design.
As mentioned above, our current model relies solely on accelerometer data for action recognition, which may limit its ability to capture the full range of motion characteristics or other physiology. Future work can explore additional sensing modalities, such as gyroscope, photoplethysmography (PPG), joint locations, to enhance the accuracy and robustness of action recognition. 
Besides, the study was conducted with a relatively small number of participants and a limited set of actions, which may not fully capture the variability and diversity of human activities in real-world scenarios \cite{trapp2015individual, narayanan2013behavioral}.
Additionally, although we tried to simulate real-life scenarios, our intervention study was conducted over a limited duration and in controlled experimental settings, which may not fully reflect the complexities and dynamics of real-life environments. 
Real-world contexts introduce factors such as environmental noise, varying sensor placements, and user behavior changes over time \cite{trapp2015individual,truong2015deployment,mejia2023enhancing,mills2022development}, which were not thoroughly simulated in this study. Future research should conduct longitudinal field experiments with real-world deployment of the system.





\section{Conclusion}\label{sec:conclusion}

In this paper, we proposed a prototype ASL generation system aimed at improving the naturalness, comprehensiveness, and overall quality of generated signs, addressing key limitations in existing approaches. Our technical evaluations indicate that our proposed approaches improve these aspects, enhancing the quality of generated ASL content. Feedback from DHH participants was mixed; while there was general interest in the system, concerns regarding visual quality and naturalness were noted. Reflecting on our design process and study findings, we discuss key insights and identify key areas for future improvement. While further work is needed, our study takes an initial step toward developing sign language generation systems that better meet the needs of the DHH and signing communities, offering real-world value.

\bibliographystyle{ACM-Ref-Format}
\bibliography{
bib/Orson/BehaviorIntervention,
bib/Orson/HumanComputerInteraction,
bib/Orson/Modeling_Behavior-General,
bib/Orson/MachineLearning,
bib/Orson/OrsonPublication,
bib/ly/BehaviorIntervention,
bib/ly/MachineLearning,
bib/ly/Dataset,
bib/ly/others,
bib/others
}

\appendix
\newpage
\appendix
\onecolumn
\section{Full Results on Longbench}
\label{appendix}
% \renewcommand{\arraystretch}{1.2} % 设置行高
\begin{table*}[ht]
\setlength{\tabcolsep}{2.5pt} % 设置列间距
\caption{\textbf{Result on Longbench.} The highest score in each task is marked in bold (except for "Full"). We also note the relative error of Twilight when integrated with the corresponding base algorithm. Green indicates an increase in score, while red indicates a decrease.}
\label{table:longbench}
    \centering
    \scalebox{0.69}{
    \begin{tabular}{lcccccccccccccc}
        \toprule
        \multirow{2}*{\textbf{Methods}} &
        \multirow{2}*{\textbf{Budget }} &
        \multicolumn{2}{c}{\textbf{Single-Doc. QA}} & \multicolumn{3}{c}{\textbf{Multi-Doc. QA}} & \multicolumn{3}{c}{\textbf{Summarization}} & \multicolumn{1}{c}{\textbf{Few-shot}} & \multicolumn{2}{c}{\textbf{Code}} & \multicolumn{1}{c}{\textbf{Synthetic}} & \multirow{2}*{\textbf{Avg. Score}}  \\
        \cmidrule(lr){3-4}\cmidrule(lr){5-7}\cmidrule(lr){8-10} \cmidrule(lr){11-11} \cmidrule(lr){12-13} \cmidrule(lr){14-14} 
        & & \textit{Qasper} & \textit{MF-en} & \textit{HotpotQA} & \textit{2WikiMQA} &  \textit{Musique} & \textit{GovReport} & \textit{QMSum} & \textit{MultiNews} & \textit{TriviaQA} &  \textit{LCC} & \textit{Repobench-P} & \textit{PR-en} \\
        \midrule
        \multicolumn{15}{c}{\textsc{Longchat-7B-32k}} \\
        \midrule
        \multirow{2}*{Full} & 32k & 29.48 & 42.11 & 30.97 & 23.74 & 13.11 & 31.03 & 22.77 & 26.09 & 83.25 & 30.50 & 52.70 & 55.62 & 36.78 \\
         & \textbf{Twilight (Avg. 146)} & 31.74 & \textbf{43.91} & 33.59 & \textbf{25.65} & \textbf{13.93} & 32.19 & \textbf{23.15} & 26.30 & 85.14 & 34.50 & 54.98 & 57.12 & 38.52\textcolor{teal}{(+4.7\%)}\\
        \midrule
        \multirow{5}*{Quest}
         & 256 & 26.00 & 32.83 & 23.23 & 22.14 & 7.45 & 22.64 & 20.98 & 25.05 & 67.40 & 33.60 & 48.70 & 45.07 & 31.26 \\
      & 1024 & 31.63 & 42.36 & 30.47 & 24.42 & 10.11 & 29.94 & 22.70 & 26.39 & 84.21 & 34.5 & 51.52 & 53.95 & 36.85 \\
       & 4096 & 29.77 & 42.71 & 32.94 & 23.94 & 13.24 & 31.54 & 22.86 & 26.45 & 84.37 & 31.50 & 53.17 & 55.52 & 37.33 \\
        & 8192 & 29.34 & 41.70 & 33.27 & 23.46 & 13.51 & 31.18 & 23.02 & 26.48 & 84.70 & 30.00 & 53.02 & 55.57 & 37.10 \\
             & \textbf{Twilight (Avg. 131)} & 31.95 & 43.28 & 31.62 & 24.87 & 13.48 & \textbf{32.21} & 22.79 & 26.33 & 84.93 & 33.50 & 54.86 & 56.70 & 38.04\textcolor{teal}{(+2.5\%)} \\
        \midrule
    \multirow{5}*{DS}
         & 256 & 28.28 & 39.78 & 27.10 & 20.75 & 9.34 & 29.68 & 21.79 & 25.69 & 83.97 & 32.00 & 52.01 & 53.44 & 35.32 \\
      & 1024 & 30.55 & 41.27 & 30.85 & 21.87 & 7.27 & 26.82 & 22.95 & 26.51 & 83.22 & 31.50 & 53.23 & 55.50 & 35.96 \\
       & 4096 & 28.95 & 41.90 & 32.52 & 23.65 & 8.07 & 29.68 & 22.75 & \textbf{26.55} & 83.34 & 30.00 & 52.77 & 55.48 & 36.31 \\
        & 8192 & 29.05 & 41.42 & 31.79 & 22.95 & 12.50 & 30.44 & 22.50 & 26.43 & 83.63 & 30.50 & 52.87 & 55.33 & 36.62 \\
             & \textbf{Twilight (Avg. 126)} & \textbf{32.34} & 43.89 & \textbf{34.67} & 25.43 & 13.84 & 31.88 & 23.01 & 26.32 & \textbf{85.29} & \textbf{35.50} & \textbf{55.03} & \textbf{57.27} & \textbf{38.71}\textcolor{teal}{(+5.7\%)} \\
        \midrule
        \multicolumn{15}{c}{\textsc{Llama-3.1-8B-Instruct}} \\
        \midrule
        \multirow{2}*{Full} & 128k & 46.17 & 53.33 & 55.36 & 43.95 & 27.08 & 35.01 & 25.24 & 27.37 & 91.18 & 99.50 & 62.17 & 57.76 & 52.01 \\
         & \textbf{Twilight (Avg. 478)} & 43.08 & 52.99 & 52.22 & 44.83 & 25.79 & 34.21 & \textbf{25.47} & 26.98 & 91.85 & \textbf{100.00} & \textbf{64.06} & 58.22 & 51.64\textcolor{red}{(-0.7\%)} \\
        \midrule
        \multirow{5}*{Quest}
         & 256 & 24.65 & 37.50 & 30.12 & 23.60 & 12.93 & 27.53 & 20.11 & 26.59 & 65.34 & 95.00 & 49.70 & 45.27 & 38.20 \\
      & 1024 & 38.47 & 49.32 & 47.43 & 38.48 & 20.59 & 33.71 & 23.67 & 26.60 & 81.94 & 99.50 & 60.78 & 52.96 & 47.79 \\
       & 4096 & 43.97 & 53.64 & 51.94 & 42.54 & 24.00 & 34.34 & 24.36 & 26.75 & 90.96 & 99.50 & 62.03 & 55.49 & 50.79 \\
        & 8192 &\textbf{44.34} & 53.25 & 54.72 & 44.84 & \textbf{25.98} & 34.62 & 24.98 & 26.70 & 91.61 & \textbf{100.00} & 62.02 & 54.20 & 51.44 \\
         & \textbf{Twilight (Avg. 427)} & 43.44 & 53.2 & 53.77 & 43.56 & 25.42 & 34.39 & 25.23 & 26.99 & 91.25 & 100.0 & 63.55 & 58.06 & 51.57\textcolor{teal}{(+0.3\%)} \\
        \midrule
    \multirow{5}*{DS}
         & 256 & 38.24 & 49.58 & 43.38 & 31.98 & 15.52 & 33.40 & 24.06 & 26.86 & 84.41 & 99.50 & 53.28 & 48.64 & 45.74 \\
      & 1024 & 42.97 & \textbf{54.65} & 51.75 & 33.92 & 20.39 & 34.50 & 24.92 & 26.71 & \textbf{92.81} & 99.50 & 62.66 & 48.37 & 49.43 \\
       & 4096 & 43.50 & 53.17 & 54.21 & 44.70 & 23.14 & \textbf{34.73} & 25.40 & 26.71 & 92.78 & 99.50 & 62.59 & 51.31 & 50.98 \\
        & 8192 & 43.82 & 53.71 & 54.19 & \textbf{45.13} & 23.72 & 34.27 & 24.98 & 26.69 & 91.61 & \textbf{100.00} & 62.40 & 52.87 & 51.14 \\
             & \textbf{Twilight (Avg. 446)} & 43.08 & 52.89 & \textbf{54.68} & 44.86 & 24.88 & 34.09 & 25.20 & \textbf{27.00} & 91.20 & \textbf{100.00} & 63.95 & \textbf{58.93} & \textbf{51.73}\textcolor{teal}{(+1.2\%)} \\
\bottomrule
\end{tabular}
}
\end{table*}

\end{document}