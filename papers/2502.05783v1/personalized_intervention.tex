%
% The first command in your LaTeX source must be the \documentclass command.
% \documentclass[acmlarge, anonymous]{acmart}
\documentclass[acmlarge]{acmart}
% \documentclass[acmlarge]{acmart}
% Load basic packages
\usepackage{balance}       % to better equalize the last page
\usepackage{graphics}      % for EPS, load graphicx instead 
\usepackage{hyperref}
\usepackage{color}
\usepackage{booktabs}
\usepackage{textcomp}
\usepackage{subcaption}
\usepackage{enumerate}
\usepackage{xcolor}
\usepackage{lipsum}% http://ctan.org/pkg/lipsum
\usepackage{makecell}
\usepackage{multicol}
\usepackage{multirow}
\usepackage{array}
\usepackage{verbatimbox}
\usepackage{enumitem}
\usepackage{amsmath}
\usepackage{stfloats}
\usepackage{graphicx}
\usepackage{amsthm}
\usepackage{listings}
\usepackage{caption} 
\usepackage[export]{adjustbox}
\usepackage{xspace}
\usepackage{epsfig}
\usepackage[linesnumbered]{algorithm2e}
\usepackage{algpseudocode}
\usepackage{tabularx}
\usepackage{arydshln}
\usepackage[bottom]{footmisc}
\usepackage{tcolorbox}
\usepackage{stackengine}
\usepackage{placeins}
\usepackage{graphicx}

\newcommand*{\eg}{\textit{e.g.},\xspace}
\newcommand*{\ie}{\textit{i.e.},\xspace}
\newcommand*{\vs}{\textit{vs.}\xspace}
\newcommand*{\etc}{\textit{etc.}}
\newcommand*{\st}{\textit{s.t.},\xspace}
\newcommand*{\etal}{\textit{et~al.}\xspace}
\newcommand*{\hlinespace}{\addlinespace[1ex]\hline\addlinespace[1ex]}
\newcommand*{\hdashlinespace}{\addlinespace[1ex]\hdashline\addlinespace[1ex]}
\newcommand*{\cdashlinespace}[1]{\addlinespace[1ex]\cdashline{#1}\addlinespace[1ex]}
\newcommand{\clinespace}[1]{\addlinespace[1ex]\cline{#1}\addlinespace[1ex]}

\newcolumntype{L}[1]{>{\raggedright\let\newline\\\arraybackslash\hspace{0pt}}m{#1}}
\newcolumntype{C}[1]{>{\centering\let\newline\\\arraybackslash\hspace{0pt}}m{#1}}
\newcolumntype{R}[1]{>{\raggedleft\let\newline\\\arraybackslash\hspace{0pt}}m{#1}}



\makeatletter
\def\thickhline{%
  \noalign{\ifnum0=`}\fi\hrule \@height \thickarrayrulewidth \futurelet
   \reserved@a\@xthickhline}
\def\@xthickhline{\ifx\reserved@a\thickhline
               \vskip\doublerulesep
               \vskip-\thickarrayrulewidth
             \fi
      \ifnum0=`{\fi}}
\makeatother

\makeatletter
\def\thickhlinespace{%
  \addlinespace[1ex]
  \noalign{\ifnum0=`}\fi\hrule \@height \thickarrayrulewidth \futurelet
   \reserved@a\@xthickhline
   \addlinespace[1ex]
   }
\def\@xthickhlinespace{\ifx\reserved@a\thickhline
               \vskip\doublerulesep
               \vskip-\thickarrayrulewidth
             \fi
      \ifnum0=`{\fi}}
\makeatother

\newlength{\thickarrayrulewidth}
\setlength{\thickarrayrulewidth}{3\arrayrulewidth}


\newlength\Origarrayrulewidth

% horizontal rule equivalent to \cline but with 2pt width
\newcommand{\Cline}[1]{%
 \noalign{\global\setlength\Origarrayrulewidth{\arrayrulewidth}}%
 \noalign{\global\setlength\arrayrulewidth{2pt}}\cline{#1}%
 \noalign{\global\setlength\arrayrulewidth{\Origarrayrulewidth}}%
}

% draw a vertical rule of width 2pt on both sides of a cell
\newcommand\Thickvrule[1]{%
  \multicolumn{1}{!{\vrule width 2pt}c!{\vrule width 2pt}}{#1}%
}

% draw a vertical rule of width 2pt on the left side of a cell
\newcommand\Thickvrulel[1]{%
  \multicolumn{1}{!{\vrule width 2pt}c|}{#1}%
}

% draw a vertical rule of width 2pt on the right side of a cell
\newcommand\Thickvruler[1]{%
  \multicolumn{1}{|c!{\vrule width 2pt}}{#1}%
}

\DeclareMathOperator*{\argmin}{argmin}   % Jan Hlavacek
\DeclareMathOperator*{\argmax}{argmax}   % Jan Hlavacek

\newcommand{\algrule}[1][.2pt]{\par\vskip.5\baselineskip\hrule height #1\par\vskip.5\baselineskip}

\algnewcommand{\IfThenElse}[3]{% \IfThenElse{<if>}{<then>}{<else>}
  \State \algorithmicif\ #1\ \algorithmicthen\ #2\ \algorithmicelse\ #3}
  
\newenvironment{s_itemize}{
\begin{itemize}[leftmargin=*]
  \setlength{\itemsep}{3pt}
  \setlength{\parskip}{0pt}
  \setlength{\parsep}{0pt}
}{\end{itemize}}

\newenvironment{s_enumerate}{
\begin{enumerate}[leftmargin=*]
  \setlength{\itemsep}{3pt}
  \setlength{\parskip}{0pt}
  \setlength{\parsep}{0pt}
}{\end{enumerate}}

\newcommand\acomment[1]{\textcolor{orange}{\textit{Anind: #1}}}
\newcommand\anind[1]{\textcolor{orange}{\textit{Anind: #1}}}
\newcommand\jcomment[1]{\textcolor{red}{\textit{Jen: #1}}}
\newcommand\jm[1]{\textcolor{red}{\textit{Jen: #1}}}
\newcommand\jen[1]{\textcolor{red}{\textit{Jen: #1}}}
\newcommand\tim[1]{\textcolor{magenta}{\textit{Tim: #1}}}
\newcommand\ocomment[1]{\textcolor{blue}{\textit{Orson: #1}}}
\newcommand\orson[1]{\textcolor{blue}{\textit{Orson: #1}}}
\newcommand\needinput[1]{\textcolor{red}{\textit{#1}}}


\newcommand\downred[1]{\textcolor{downredcolor}{#1}}
\newcommand\upgreen[1]{\textcolor{upgreencolor}{#1}}
\definecolor{downredcolor}{HTML}{e31a1c}
\definecolor{upgreencolor}{HTML}{33a02c}

\definecolor{DarkGreen}{HTML}{5DAC81}
% \newcommand\review[1]{\textcolor{DarkGreen}{#1}}
% \newcommand\minorreview[1]{\textcolor{DarkGreen}{#1}}
\newcommand\review[1]{\textcolor{black}{#1}}
\newcommand\minorreview[1]{\textcolor{black}{#1}}
% \newcommand\review[1]{\textcolor{black}{#1}}


\newcommand\projectname{WatchGuardian\xspace}

% Rights management information. 
% This information is sent to you when you complete the rights form.
% These commands have SAMPLE values in them; it is your responsibility as an author to replace
% the commands and values with those provided to you when you complete the rights form.
%
% These commands are for a PROCEEDINGS abstract or paper.
% \copyrightyear{2018}
% \acmYear{2018}
% \setcopyright{acmlicensed}
% \acmConference[Woodstock '18]{Woodstock '18: ACM Symposium on Neural Gaze Detection}{June 03--05, 2018}{Woodstock, NY}
% \acmBooktitle{Woodstock '18: ACM Symposium on Neural Gaze Detection, June 03--05, 2018, Woodstock, NY}
% \acmPrice{15.00}
% \acmDOI{10.1145/1122445.1122456}
% \acmISBN{978-1-4503-9999-9/18/06}

%
% These commands are for a JOURNAL article.

% \setcopyright{rightsretained}
% \acmJournal{IMWUT}
% \acmYear{2023}
% \acmVolume{6} \acmNumber{4} \acmArticle{190} \acmMonth{12} \acmPrice{}\acmDOI{10.1145/XXXXXXX}

% \setcopyright{acmlicensed}
% \acmJournal{IMWUT}
% \acmYear{2024} \acmVolume{8} \acmNumber{1} \acmArticle{32} \acmMonth{3}\acmDOI{10.1145/3643540}

%
% Submission ID. 
% Use this when submitting an article to a sponsored event. You'll receive a unique submission ID from the organizers
% of the event, and this ID should be used as the parameter to this command.
%\acmSubmissionID{123-A56-BU3}

%
% The majority of ACM publications use numbered citations and references. If you are preparing content for an event
% sponsored by ACM SIGGRAPH, you must use the "author year" style of citations and references. Uncommenting
% the next command will enable that style.
%\citestyle{acmauthoryear}

%
% end of the preamble, start of the body of the document source.
\begin{document}

%
% The "title" command has an optional parameter, allowing the author to define a "short title" to be used in page headers.

\title{\projectname: Enabling User-Defined Personalized Just-in-Time Intervention on Smartwatch}

%
% The "author" command and its associated commands are used to define the authors and their affiliations.
% Of note is the shared affiliation of the first two authors, and the "authornote" and "authornotemark" commands
% used to denote shared contribution to the research.

\author{Ying Lei}
\authornote{Mark co-first authors with equal contribution.}
\orcid{0000-0001-8326-1369}
\affiliation{%
  \institution{Simon Fraser University}
  \country{Canada}
}

\author{Yancheng Cao}
\authornotemark[1]
\orcid{0000-0003-3033-8881}
\affiliation{%
  \institution{Columbia University}
  \country{USA}
}

\author{Will Ke Wang}
\orcid{0000-0003-1444-5468}
\affiliation{%
  \institution{Columbia University}
  \country{USA}
}

\author{Yuanzhe Dong}
\orcid{0009-0006-2013-1157}
\affiliation{%
  \institution{Stanford University}
  \country{USA}
}

\author{Changchang Yin}
\orcid{0000-0002-6540-6365}
\author{Weidan Cao} 
\orcid{0000-0001-5417-2121}
\author{Ping Zhang}
\orcid{0000-0002-4601-0779}
\affiliation{%
  \institution{The Ohio State University}
  \country{USA}
}

\author{Jingzhe Yang}
\orcid{0000-0003-4019-0999}
\affiliation{%
  \institution{Nationwide Children's Hospital}
  \country{USA}
}


\author{Bingsheng Yao}
\orcid{0009-0004-8329-4610}
\affiliation{%
  \institution{Northeastern University}
  \country{USA}
}

\author{Yifan Peng}
\orcid{0000-0001-9309-8331}
\affiliation{%
  \institution{Weill Cornell Medicine}
  \country{USA}
}

\author{Chunhua Weng}
\orcid{0000-0002-9624-0214}
\author{Randy Auerbach}
\orcid{0000-0003-2319-4744}
\author{Lena Mamykina}
\orcid{0000-0001-5203-274X}
\affiliation{%
  \institution{Columbia University}
  \country{USA}
}

\author{Dakuo Wang}
\authornote{Mark corresponding authors.}
\orcid{0000-0001-9371-9441}
\affiliation{%
  \institution{Northeastern University}
  \country{USA}
}

\author{Yuntao Wang}
\authornotemark[2]
\orcid{0000-0002-4249-8893}
\affiliation{%
  \institution{University of Washington}
  \country{USA}
}

\author{Xuhai Xu}
\authornotemark[2]
\email{xx2489@columbia.edu}
\orcid{0000-0001-5930-3899}
\affiliation{%
  \institution{Columbia University}
  \country{USA}
}


% Ying Lei *
% Yancheng Cao *
% Will Wang
% Yuanzhe Dong
% Changchang Yin
% Weidan Cao
% Ping Zhang
% Jingzhen (Ginger) Yang
% Bingsheng Yao
% Yifan Peng
% Chunhua Weng
% Randy Auerbach
% Lena Mamykina
% Dakuo Wang
% Yuntao Wang
% Xuhai Xu

%
% By default, the full list of authors will be used in the page headers. Often, this list is too long, and will overlap
% other information printed in the page headers. This command allows the author to define a more concise list
% of authors' names for this purpose.
\renewcommand{\shortauthors}{Ying \& Cao et al.}
\renewcommand{\shorttitle}{\projectname{}}

%
% The abstract is a short summary of the work to be presented in the article.
\begin{abstract}
Large language model (LLM)-based agents have shown promise in tackling complex tasks by interacting dynamically with the environment. 
Existing work primarily focuses on behavior cloning from expert demonstrations and preference learning through exploratory trajectory sampling. However, these methods often struggle in long-horizon tasks, where suboptimal actions accumulate step by step, causing agents to deviate from correct task trajectories.
To address this, we highlight the importance of \textit{timely calibration} and the need to automatically construct calibration trajectories for training agents. We propose \textbf{S}tep-Level \textbf{T}raj\textbf{e}ctory \textbf{Ca}libration (\textbf{\model}), a novel framework for LLM agent learning. 
Specifically, \model identifies suboptimal actions through a step-level reward comparison during exploration. It constructs calibrated trajectories using LLM-driven reflection, enabling agents to learn from improved decision-making processes. These calibrated trajectories, together with successful trajectory data, are utilized for reinforced training.
Extensive experiments demonstrate that \model significantly outperforms existing methods. Further analysis highlights that step-level calibration enables agents to complete tasks with greater robustness. 
Our code and data are available at \url{https://github.com/WangHanLinHenry/STeCa}.
\end{abstract}

%
% The code below is generated by the tool at http://dl.acm.org/ccs.cfm.
% Please copy and paste the code instead of the example below.
%
\begin{CCSXML}
<ccs2012>
<concept>
<concept_id>10003120.10003138</concept_id>
<concept_desc>Human-centered computing~Ubiquitous and mobile computing</concept_desc>
<concept_significance>500</concept_significance>
</concept>
<concept>
<concept_id>10010405.10010444</concept_id>
<concept_desc>Applied computing~Life and medical sciences</concept_desc>
<concept_significance>500</concept_significance>
</concept>
</ccs2012>
\end{CCSXML}
\ccsdesc[500]{Human-centered computing~Ubiquitous and mobile computing}
\ccsdesc[500]{Applied computing~Life and medical sciences}
%
% Keywords. The author(s) should pick words that accurately describe the work being
% presented. Separate the keywords with commas.
\keywords{Few-shot learning, Just-in-time intervention, Personalized intervention}

%
% This command processes the author and affiliation and title information and builds
% the first part of the formatted document.
\maketitle

\section{Introduction}

Despite the remarkable capabilities of large language models (LLMs)~\cite{DBLP:conf/emnlp/QinZ0CYY23,DBLP:journals/corr/abs-2307-09288}, they often inevitably exhibit hallucinations due to incorrect or outdated knowledge embedded in their parameters~\cite{DBLP:journals/corr/abs-2309-01219, DBLP:journals/corr/abs-2302-12813, DBLP:journals/csur/JiLFYSXIBMF23}.
Given the significant time and expense required to retrain LLMs, there has been growing interest in \emph{model editing} (a.k.a., \emph{knowledge editing})~\cite{DBLP:conf/iclr/SinitsinPPPB20, DBLP:journals/corr/abs-2012-00363, DBLP:conf/acl/DaiDHSCW22, DBLP:conf/icml/MitchellLBMF22, DBLP:conf/nips/MengBAB22, DBLP:conf/iclr/MengSABB23, DBLP:conf/emnlp/YaoWT0LDC023, DBLP:conf/emnlp/ZhongWMPC23, DBLP:conf/icml/MaL0G24, DBLP:journals/corr/abs-2401-04700}, 
which aims to update the knowledge of LLMs cost-effectively.
Some existing methods of model editing achieve this by modifying model parameters, which can be generally divided into two categories~\cite{DBLP:journals/corr/abs-2308-07269, DBLP:conf/emnlp/YaoWT0LDC023}.
Specifically, one type is based on \emph{Meta-Learning}~\cite{DBLP:conf/emnlp/CaoAT21, DBLP:conf/acl/DaiDHSCW22}, while the other is based on \emph{Locate-then-Edit}~\cite{DBLP:conf/acl/DaiDHSCW22, DBLP:conf/nips/MengBAB22, DBLP:conf/iclr/MengSABB23}. This paper primarily focuses on the latter.

\begin{figure}[t]
  \centering
  \includegraphics[width=0.48\textwidth]{figures/demonstration.pdf}
  \vspace{-4mm}
  \caption{(a) Comparison of regular model editing and EAC. EAC compresses the editing information into the dimensions where the editing anchors are located. Here, we utilize the gradients generated during training and the magnitude of the updated knowledge vector to identify anchors. (b) Comparison of general downstream task performance before editing, after regular editing, and after constrained editing by EAC.}
  \vspace{-3mm}
  \label{demo}
\end{figure}

\emph{Sequential} model editing~\cite{DBLP:conf/emnlp/YaoWT0LDC023} can expedite the continual learning of LLMs where a series of consecutive edits are conducted.
This is very important in real-world scenarios because new knowledge continually appears, requiring the model to retain previous knowledge while conducting new edits. 
Some studies have experimentally revealed that in sequential editing, existing methods lead to a decrease in the general abilities of the model across downstream tasks~\cite{DBLP:journals/corr/abs-2401-04700, DBLP:conf/acl/GuptaRA24, DBLP:conf/acl/Yang0MLYC24, DBLP:conf/acl/HuC00024}. 
Besides, \citet{ma2024perturbation} have performed a theoretical analysis to elucidate the bottleneck of the general abilities during sequential editing.
However, previous work has not introduced an effective method that maintains editing performance while preserving general abilities in sequential editing.
This impacts model scalability and presents major challenges for continuous learning in LLMs.

In this paper, a statistical analysis is first conducted to help understand how the model is affected during sequential editing using two popular editing methods, including ROME~\cite{DBLP:conf/nips/MengBAB22} and MEMIT~\cite{DBLP:conf/iclr/MengSABB23}.
Matrix norms, particularly the L1 norm, have been shown to be effective indicators of matrix properties such as sparsity, stability, and conditioning, as evidenced by several theoretical works~\cite{kahan2013tutorial}. In our analysis of matrix norms, we observe significant deviations in the parameter matrix after sequential editing.
Besides, the semantic differences between the facts before and after editing are also visualized, and we find that the differences become larger as the deviation of the parameter matrix after editing increases.
Therefore, we assume that each edit during sequential editing not only updates the editing fact as expected but also unintentionally introduces non-trivial noise that can cause the edited model to deviate from its original semantics space.
Furthermore, the accumulation of non-trivial noise can amplify the negative impact on the general abilities of LLMs.

Inspired by these findings, a framework termed \textbf{E}diting \textbf{A}nchor \textbf{C}ompression (EAC) is proposed to constrain the deviation of the parameter matrix during sequential editing by reducing the norm of the update matrix at each step. 
As shown in Figure~\ref{demo}, EAC first selects a subset of dimension with a high product of gradient and magnitude values, namely editing anchors, that are considered crucial for encoding the new relation through a weighted gradient saliency map.
Retraining is then performed on the dimensions where these important editing anchors are located, effectively compressing the editing information.
By compressing information only in certain dimensions and leaving other dimensions unmodified, the deviation of the parameter matrix after editing is constrained. 
To further regulate changes in the L1 norm of the edited matrix to constrain the deviation, we incorporate a scored elastic net ~\cite{zou2005regularization} into the retraining process, optimizing the previously selected editing anchors.

To validate the effectiveness of the proposed EAC, experiments of applying EAC to \textbf{two popular editing methods} including ROME and MEMIT are conducted.
In addition, \textbf{three LLMs of varying sizes} including GPT2-XL~\cite{radford2019language}, LLaMA-3 (8B)~\cite{llama3} and LLaMA-2 (13B)~\cite{DBLP:journals/corr/abs-2307-09288} and \textbf{four representative tasks} including 
natural language inference~\cite{DBLP:conf/mlcw/DaganGM05}, 
summarization~\cite{gliwa-etal-2019-samsum},
open-domain question-answering~\cite{DBLP:journals/tacl/KwiatkowskiPRCP19},  
and sentiment analysis~\cite{DBLP:conf/emnlp/SocherPWCMNP13} are selected to extensively demonstrate the impact of model editing on the general abilities of LLMs. 
Experimental results demonstrate that in sequential editing, EAC can effectively preserve over 70\% of the general abilities of the model across downstream tasks and better retain the edited knowledge.

In summary, our contributions to this paper are three-fold:
(1) This paper statistically elucidates how deviations in the parameter matrix after editing are responsible for the decreased general abilities of the model across downstream tasks after sequential editing.
(2) A framework termed EAC is proposed, which ultimately aims to constrain the deviation of the parameter matrix after editing by compressing the editing information into editing anchors. 
(3) It is discovered that on models like GPT2-XL and LLaMA-3 (8B), EAC significantly preserves over 70\% of the general abilities across downstream tasks and retains the edited knowledge better.

\section{Background}
\label{sec:background}


\subsection{Code Review Automation}
Code review is a widely adopted practice among software developers where a reviewer examines changes submitted in a pull request \cite{hong2022commentfinder, ben2024improving, siow2020core}. If the pull request is not approved, the reviewer must describe the issues or improvements required, providing constructive feedback and identifying potential issues. This step involves review commment generation, which play a key role in the review process by generating review comments for a given code difference. These comments can be descriptive, offering detailed explanations of the issues, or actionable, suggesting specific solutions to address the problems identified \cite{ben2024improving}.


Various approaches have been explored to automate the code review comments process  \cite{tufano2023automating, tufano2024code, yang2024survey}. 
Early efforts centered on knowledge-based systems, which are designed to detect common issues in code. Although these traditional tools provide some support to programmers, they often fall short in addressing complex scenarios encountered during code reviews \cite{dehaerne2022code}. More recently, with advancements in deep learning, researchers have shifted their focus toward using large-language models to enhance the effectiveness of code issue detection and code review comment generation.

\subsection{Knowledge-based Code Review Comments Automation}

Knowledge-based systems (KBS) are software applications designed to emulate human expertise in specific domains by using a collection of rules, logic, and expert knowledge. KBS often consist of facts, rules, an explanation facility, and knowledge acquisition. In the context of software development, these systems are used to analyze the source code, identifying issues such as coding standard violations, bugs, and inefficiencies~\cite{singh2017evaluating, delaitre2015evaluating, ayewah2008using, habchi2018adopting}. By applying a vast set of predefined rules and best practices, they provide automated feedback and recommendations to developers. Tools such as FindBugs \cite{findBugs}, PMD \cite{pmd}, Checkstyle \cite{checkstyle}, and SonarQube \cite{sonarqube} are prominent examples of knowledge-based systems in code analysis, often referred to as static analyzers. These tools have been utilized since the early 1960s, initially to optimize compiler operations, and have since expanded to include debugging tools and software development frameworks \cite{stefanovic2020static, beller2016analyzing}.



\subsection{LLMs-based Code Review Comments Automation}
As the field of machine learning in software engineering evolves, researchers are increasingly leveraging machine learning (ML) and deep learning (DL) techniques to automate the generation of review comments \cite{li2022automating, tufano2022using, balachandran2013reducing, siow2020core, li2022auger, hong2022commentfinder}. Large language models (LLMs) are large-scale Transformer models, which are distinguished by their large number of parameters and extensive pre-training on diverse datasets.  Recently, LLMs have made substantial progress and have been applied across a broad spectrum of domains. Within the software engineering field, LLMs can be categorized into two main types: unified language models and code-specific models, each serving distinct purposes \cite{lu2023llama}.

Code-specific LLMs, such as CodeGen \cite{nijkamp2022codegen}, StarCoder \cite{li2023starcoder} and CodeLlama \cite{roziere2023code} are optimized to excel in code comprehension, code generation, and other programming-related tasks. These specialized models are increasingly utilized in code review activities to detect potential issues, suggest improvements, and automate review comments \cite{yang2024survey, lu2023llama}. 




\subsection{Retrieval-Augmented Generation}
Retrieval-Augmented Generation (RAG) is a general paradigm that enhances LLMs outputs by including relevant information retrieved from external databases into the input prompt \cite{gao2023retrieval}. Traditional LLMs generate responses based solely on the extensive data used in pre-training, which can result in limitations, especially when it comes to domain-specific, time-sensitive, or highly specialized information. RAG addresses these limitations by dynamically retrieving pertinent external knowledge, expanding the model's informational scope and allowing it to generate responses that are more accurate, up-to-date, and contextually relevant \cite{arslan2024business}. 

To build an effective end-to-end RAG pipeline, the system must first establish a comprehensive knowledge base. It requires a retrieval model that captures the semantic meaning of presented data, ensuring relevant information is retrieved. Finally, a capable LLM integrates this retrieved knowledge to generate accurate and coherent results \cite{ibtasham2024towards}.




\subsection{LLM as a Judge Mechanism}

LLM evaluators, often referred to as LLM-as-a-Judge, have gained significant attention due to their ability to align closely with human evaluators' judgments \cite{zhu2023judgelm, shi2024judging}. Their adaptability and scalability make them highly suitable for handling an increasing volume of evaluative tasks. 

Recent studies have shown that certain LLMs, such as Llama-3 70B and GPT-4 Turbo, exhibit strong alignment with human evaluators, making them promising candidates for automated judgment tasks \cite{thakur2024judging}

To enable such evaluations, a proper benchmarking system should be set up with specific components: \emph{prompt design}, which clearly instructs the LLM to evaluate based on a given metric, such as accuracy, relevance, or coherence; \emph{response presentation}, guiding the LLM to present its verdicts in a structured format; and \emph{scoring}, enabling the LLM to assign a score according to a predefined scale \cite{ibtasham2024towards}. Additionally, this evaluation system can be enriched with the ability to explain reasoning behind verdicts, which is a significant advantage of LLM-based evaluation \cite{zheng2023judging}. The LLM can outline the criteria it used to reach its judgment, offering deeper insights into its decision-making process.






\section{Method} \label{section: method}

\begin{figure*}[t]
    \centering
    \includegraphics[width=\textwidth]{figures/overview.pdf}
    \caption{Overview of \ours and cross-token prefetching framework. (a) \textbf{\ours}  formulates the attention history as a spatiotemporal sequence, and predicts the attention at the next step with a pre-trained model. To enhance efficiency, the attention history is updated in a compressed form at each decoding step. (b) \textbf{The cross-token prefetching framework} asynchronously evaluates critical tokens and fetches KV for the next token during the LLM inference, thereby accelerating the decoding stage.}
    \label{fig:prefetch_overview}  
\end{figure*}


In this section, we introduce \ours, the first learning-based method for identifying critical tokens, along with the cross-token prefetch framework for improved cache management. We begin with the problem formulation for attention prediction in Section~\ref{section:formulation}, followed by a description of our novel \ours in Section~\ref{sec:attention_predictor}. Finally, Section~\ref{section: prefetch} presents a cross-token prefetch framework that efficiently hides both evaluation and cache loading latencies.

\subsection{Problem Formulation}
\label{section:formulation}

In the language model decoding stage, we denote $\mathbf{Q}_t \in \mathbb{R}^{1 \times d}$, $\mathbf{K} \in \mathbb{R}^{t \times d}$ as the query tensor and key tensor used for generate token $t$, respectively. Specifically, we denote \( \mathbf{K}_i \in \mathbb{R}^{1 \times d} \), where \( i \in \{1, 2, \dots, t\} \), as the key tensor for token \( i \), and \( \mathbf{K} = \mathbf{K}_{1:t} \) as the complete key tensor. The attention at step $t$ is calculated as:
\begin{equation}
    A_t=\text{Softmax}\left(\frac{1}{\sqrt{d}} \mathbf{Q}_t \mathbf{K}^\top \right), A_t \in \mathbb{R}^{1 \times t}. 
\end{equation}


The sparsity-based KV cache compression seeks to find a subset of keys with budget $B$ that preserves the most important attention values.
Specifically, the set of selectable key positions is $\Gamma=\{\{\mathbf{p}\}=\left\{p_i\right\}_{i=1}^B|p_i\in\{ 1,2,\ldots,t\} ,p_i\neq p_j,\forall i,j=1,2,\ldots,B\}$. 
We define the \textbf{attention recovery rate} as:
\begin{equation}
\label{eq:attention_recovery_score}
R_{rec} = \frac{\sum_{i=0}^{B}{A_{t, p_i}}}{||A_t||_1},
\end{equation}
which reflects the amount of information preserved after compression. A higher recovery rate $R_{rec}$ indicates less information loss caused by KV cache compression.
Therefore, the goal of KV cache compression can be formulated as finding the positions $\mathbf{p}$ that maximize $R_{rec}$, i.e.,
\begin{equation}
\label{eq:find_p}
\underset{\mathbf{p} \in \Gamma }{\max} \,R_{rec}. 
\end{equation}

To determine the positions $\mathbf{p}$, existing methods typically employ heuristic approaches to score the attention at step $t$, represented as $S_t \in \mathbb{R}^{1 \times t}$, and then select the top $B$ positions. 
For example, the well-known method H2O~\citep{zhang2023h2o} accumulates historical attention scores, where $S_t = \sum_{n=1}^{t-1}{A_n}$. 
In this paper, we predict the attention of step $t$ as $\hat{A_t}$ and use it as $S_t$.

After identifying the critical token positions $\mathbf{p}$, the attention is computed sparsely $A^\text{sparse} = \text{Softmax}\left(\frac{1}{\sqrt{d}} \mathbf{Q} {\mathbf{K}^{\text{sparse}}}^\top \right)$, with selected keys $\mathbf{K}^{\text{sparse}} = \text{concate}\{\mathbf{K}_{p_i}\}$.


\subsection{\ours: A Spatiotemporal Predictor}
\label{sec:attention_predictor}

\textbf{Prediction formulation.} We formulate the attention history $A_H \in \mathbb{R}^{t\times t}$ as a spatiotemporal sequence.
The first dimension of $A_H$ corresponds to the time series over the decoding steps,
while the second dimension represents a sparse series over different keys.
We then train a model to predict the attention for step $t$ as $\hat{A}_{t+1} = F(A_H)$, where $F(\cdot)$ denotes the model function.
For efficiency, we limit the time steps of $A_H$ using a hyperparameter $H$, so that the input to the predictor is $A_H \in \mathbb{R}^{H \times t}$.
\begin{figure*}[t]
    \centering
    \includegraphics[width=0.8\textwidth]{figures/prefetch_timeline.pdf}
    \caption{Timeline of our proposed cross-token prefetching. By asynchronously loading the critical KV cache for the next token, our framework hides the token evaluation and transfer latency, accelerating the decoding stage of LLM inference.}
    \label{fig:prefetch_timeline}
\end{figure*}


\textbf{Model design.} To capture spatiotemporal features, we use a convolutional neural network (CNN) composed of two 2D convolution layers followed by a 1D convolution layer. 
The 2D convolutions capture spatiotemporal features at multiple scales, while the 1D convolution focuses on the time dimension, extracting temporal patterns across time steps. By replacing the fully connected layer with a 1D convolution kernel, the model adapts to the increasing spatial dimension, without data segmentation or training multiple models.
Compared to an auto-regressive LSTM~\citep{graves2012lstm}, the CNN is more lightweight and offers faster predictions, maintaining a prediction time shorter than the single-token inference latency. Additionally, when compared to an MLP~\citep{rumelhart1986MLP} on time-series dimension, the CNN is more effective at capturing spatial features, which improves prediction accuracy. 


\textbf{Training strategy.} Our model is both data-efficient and generalizable.
We train the model only on a small subset of attention data, specifically approximately 3\% extracted from the dataset. The model performance on the entire dataset shows our model effectively captures the patterns (see Section \ref{sec:exp_main}). 
Additionally, due to the temporal characteristics of attention inherent in the LLM, a single model can generalize well across various datasets. For example, our model trained on LongBench also performs well on the GSM8K dataset, highlighting the generalization capability of \ours.



\textbf{Block-wise attention compression.}
To speed up prediction, we apply attention compression before computation. 
By taking advantage of the
attention's locality,
\ours predict attention and identify critical tokens in blocks. Inspired by~\citet{tang2024quest}, we use the maximum attention value in each block as its representative.
Specifically, max-pooling is applied on $A$ with a kernel size equal to the block size $b$, as $A_t^{comp} = Maxpooling(A_t,b)$, reducing prediction computation to roughly $\frac{1}{b}$. 

\textbf{Distribution error calibration.}
Due to the sparsity of attention computation, the distribution of attention history $A_H$ used for prediction may deviate from the distribution of dense attention. This deviation tends to accumulate over decoding, particularly as the output length increases. To mitigate this issue and enhance prediction accuracy, we introduce a distribution error calibration technique to correct these deviations. Specifically, we calculate and store the full attention score every $M$ steps, effectively balancing accuracy with computational efficiency.

\textbf{Overall process.}
As shown in \autoref{fig:prefetch_overview} and Algorithm \ref{alg:predict}, \ours prepares an attention history queue in the prefilling stage, and predicts attention during the decoding stage. First, the $A_t$ from the LLM is compressed to $A_t^{comp}$ using block-wise attention compression. Next, $A_H$ is updated with $A_t^{comp}$. The next step attention $\hat{A}_{t+1}$ is then predicted with the pretrained model. From $\hat{A}_{t+1}$, the top-K positions are selected with a budget of $B/b$, since $\hat{A}_{t+1}$ is in compressed form. Finally, the indices are expanded with $b$ to obtain the final critical token positions
$\mathbf{p}$.

\begin{algorithm}[ht!]
   \caption{Identify Critical Tokens}
   \label{alg:predict}
   
    \textbf{Input}: Attention scores $A_t$, Attention history $A_H$, Block size $b$, KV budget $B$
    \\
    \textbf{Output}: Critical KV token positions $\mathbf{p}$
    
    \begin{algorithmic}[1]
    \STATE Pad $A_t$ to the nearest multiple of $b$ with zero
    \STATE $A_t^{comp} \gets \text{MaxPooling}(A_t, b)$
    \STATE $A_H \gets \text{Update}(A_h, A_t^{comp})$
    \STATE $\hat{A}_{t+1} \gets \text{Prediction model}(A_H)$
    \STATE $\text{Positions} \gets \text{Top-K}(\hat{A}_{t+1}, B / b)$
    \STATE $\mathbf{p} \gets \text{Expand}(\text{Positions}, b)$ \\
    \textbf{Return} positions $\mathbf{p} $
    \end{algorithmic}
\end{algorithm}


\subsection{KV Cache Cross-token Prefetching} \label{section: prefetch}

To address the increased memory cost of longer contexts, current LLM systems offload the KV cache to the CPU, but I/O transfer latency becomes the new significant bottleneck in inference. KV cache prefetching offers a solution by asynchronously loading important cache portions in advance, hiding retrieval time. We introduce the cross-token KV cache prefetching framework, which differs from the cross-layer method in Infinigen \citep{lee2024infinigen} by leveraging longer transfer times and enhancing data integration. 
Specifically, our implementation involves a prefetching process for each layer. As illustrated in Figure \ref{fig:prefetch_overview}, during the prefill phase, the computed KV cache is completely offloaded to the CPU without compression. Then, \ours forecasts the critical token indices $\mathbf{p}$ for the next step. The framework then prefetches the KV cache with $\mathbf{p}$ for the next step onto the GPU. Concurrently, the GPU processes inference for other layers, so the maximum time available for prediction and cache loading corresponds to the inference time per token. Subsequently, the GPU utilizes the query for the next step along with the prefetched partial KV cache to calculate the sparse attention. The attention history is then updated with the newly computed attention scores. The timeline of cross-token prefetching can be seen in \autoref{fig:prefetch_timeline}.



\section{Model Evaluation}
\label{sec:model_evaluation}
In this section, we report the evaluation of \projectname's few-shot learning pipeline offline performance. We will further elaborate on the evaluation of \projectname's intervention effectiveness in section \ref{sec:intervention_evaluation}.

\subsection{Data Collection}
\label{sub:model_evaluation:data_collection}
\subsubsection{Participants}
We recruited 26 users (14 females, 12 males, age 22$\pm$2) for data collection via social media platforms.
We focused on users who were aware of their own undesirable actions and had the intention to reduce these actions. These are the target users of our intervention system.
Our study was IRB-approved by the local institution, and participants were compensated with \$10 for this data collection study (around 45 minutes).

\subsubsection{Personal Undesirable Action Customization}

Participants were asked to record five pre-determined target actions that are commonly recognized as undesirable actions~\cite{teng2002body,oshio2018shake}, including \textit{Face Scratching}, \textit{Nail Biting}, \textit{Eye Rubbing}, \textit{Lip Tearing}, and \textit{Leg Shaking}. The first five figures in \autoref{fig:evaluation_actions} illustrate these actions.

Moreover, each participant was asked to define a new undesirable action tailored to their own personal needs.
In total, 26 participants designed an additional set of 12 actions, including \textit{Finger Lipping} (designed by N=5 participants), \textit{Head Scratching} (N=5), \textit{Nose Rubbing} (N=4), \textit{Finger Picking} (N=3), \textit{Hair Scratching} (N=2), \textit{Face Rubbing} (N=1), \textit{Finger Biting} (N=1), \textit{Hair Pulling} (N=1), \textit{Hair Rubbing} (N=1), \textit{Lip Biting} (N=1), \textit{Nail Picking} (N=1), and \textit{Neck Scar Scratching} (N=1).
We only grouped identical actions and distinguished actions as long as they differed slightly. For instance, \textit{Head Scratching} and  \textit{Hair Scratching} were similar, but one involved contacts between fingers and 
scalp, while the other one did not.
Similarly, \textit{Finger Picking} and \textit{Nail Picking} were also quite close, yet one solely focused on the skin on the finger, while the other focused on nails.
These actions were visualized in the second half of \autoref{fig:evaluation_actions}.


\begin{figure*}[]
\centering 
\includegraphics[width=1\linewidth]{figure/action.pdf}
\caption{Target Actions for Evaluation. (1-5) presents the five pre-determined actions. (6-17) visualizes new target behaviors defined by participants.
Only identical actions are grouped as one. Actions that have minor differences are counted separately, as each of them could be highly personal.
}
\label{fig:evaluation_actions}
\Description{}
\end{figure*}

\subsubsection{Data Collection Procedure}
For each action, participants followed a consistent protocol (briefly mentioned in Sec.~\ref{subsub:methods:system:fewshot}) comprising two phases per shot: a 5-second \textit{free mode} and a 10-second \textit{record mode}. In the free mode, participants were free to rest or perform natural daily activities (negative data). Once entering the record mode, they performed the target actions (positive data).
This process was repeated across five rounds, with each round consisting of five consecutive shots.
Participants took a short break between two rounds to prevent physical fatigue and were asked to freely adjust the watch position between each round to increase data variance.
In total, we collected 25 shots for each target action.
Moreover, we leveraged the onboarding process at the beginning of the data collection to passively record participants' natural activities (about 5 minutes). This was used as additional data to augment the negative class\footnote{In real-world applications, we envision that such negative data can also be passively collected and implicitly embedded in the instruction process, thereby introducing minimal additional workload for the user}.

The \textit{free mode} segment was labeled as negative data, while the \textit{record mode} segment was labeled as positive data.
To prevent data contamination, the first two seconds during the record mode were excluded from training because these recordings were mixed with postural changes and arm movement.


\subsection{Offline Performance Evaluation}
\label{sub:model_evaluation:pipeline_evaluation}
We evaluated our pipeline by adding one or more actions as target actions.
For each action, we randomly selected two rounds of recordings as the training set (up to 10 shots), one round as the validation set (5 shots), and the remaining two rounds as the test set (10 shots). We repeated the training three times and calculated the average performance.

It is noteworthy that the model performance has two aspects: the window level and the action level. 
For the window level, each sliding window is counted as a binary classification data point (same as the model training process).
For the action level, windows are aggregated with a smoothing threshold of 3 (Sec.~\ref{subsub:methods:system:fewshot}) and represent a closer experience as real-life applications. Such aggregation significantly reduces the false negative and false positive.

\subsubsection{Prediction Performance with Different Number of Shots and Actions}
\label{subsub:model_evaluation:pipeline_evaluation:shots and action}
We evaluated the model performance by training on one to ten shots of the data.
For action recognition, we started by adding one action for each participant (\ie training binary classification models).
To evaluate the performance of multi-class classification models, we also experimented with customizing multiple actions (up to six, as each participant recorded five pre-designed actions and one custom action). This led to a total number of 63 combinations from one to six actions (\(\sum_{k=1}^6 \binom{6}{k}\)).
% We conducted this evaluation by systematically exploring all possible combinations of actions, ranging from one new action to six new actions. Subsequently, we trained models using different numbers of training samples, with the number of shots varying from one to ten.
In total, we trained and evaluated 49,140 models = 10 shot numbers $\times$ 63 action combinations $\times$ 26 participants $\times$ 3 repetitions. 
% We keep the original recognition results on these sliding window samples as  window level

We mainly focused on the action-level performance. \autoref{tab:action_shot_study} presents both the window-level and action-level results.
As shown in \autoref{fig:all_result_ges_num_action}, when using only one shot to add a new action (\ie the user performs the action only once), our framework achieved an average accuracy of 76.8\% and an F1 score of 74.8\%.
The recognition performance became better with more shots for training the model. With five shots of a new action, our framework attained an average accuracy of 84.7\% and an F1 score of 84.2\%. When using ten shots, our model's performance achieved 87.7\% and 87.2\%, respectively.

Recognizing multiple new actions simultaneously presented a greater challenge. However, compared to the performance of adding one action with five shots (84.7\% and 84.2\%), introducing three new actions (\ie four-class classification) with five shots each, the framework maintained a good average accuracy of 79.1\% and an F1 score of 78.1\%.
Even with six additional new actions and five shots each, the framework still achieved an average accuracy of 73.7\% and an F1 score of 72.3\%.
These results demonstrated the robustness and effectiveness of our pipeline for data-efficient action recognition.

\begin{figure*}[]
\centering 
\includegraphics[width=\linewidth]{figure/all_result_ges_num_action.png}
\caption{Few-shot Learning Pipeline Performance of Accuracy and F1 Score. We experimented with different numbers of shots using 1 to 10 samples to train a custom model. We also experimented with adding more than one target action simultaneously (\ie multi-class classification). Error bars indicate standard error. The same below.
}
\label{fig:all_result_ges_num_action}
\Description{}
\end{figure*}

% We report both window level and action level results in \autoref{tab:action_shot_study}
% \renewcommand{\arraystretch}{1.3}
% \begin{table}[]
% \centering
% \resizebox{1\textwidth}{!}{
% \begin{tabular}[t]{l|cccc|cccc}
% \toprule
% \multirow{2}{*}{\textbf{Shots}}  & \multicolumn{4}{c}{\textbf{Window-level}}      & \multicolumn{4}{c}{\textbf{Action-level}}       \\ \cline{2-9}
%  & \textbf{Acc}   & \textbf{Prec}  & \textbf{Rec}   & \textbf{F1}    & \textbf{Acc}   & \textbf{Prec}  & \textbf{Rec}   & \textbf{F1}      \\ \hline
% 1    & 0.614$\pm$0.006 & 0.700$\pm$0.007 & 0.614$\pm$0.006 & 0.571$\pm$0.008 & 0.768$\pm$0.007 & 0.810$\pm$0.007 & 0.768$\pm$0.007 & 0.748$\pm$0.009 \\
% 3    & 0.658$\pm$0.005 & 0.736$\pm$0.005 & 0.658$\pm$0.005 & 0.634$\pm$0.006 & 0.832$\pm$0.006 & 0.860$\pm$0.005 & 0.832$\pm$0.006 & 0.825$\pm$0.006 \\
% 5    & 0.670$\pm$0.005 & 0.746$\pm$0.005 & 0.670$\pm$0.005 & 0.648$\pm$0.006 & 0.847$\pm$0.005 & 0.871$\pm$0.005 & 0.847$\pm$0.005 & 0.842$\pm$0.006 \\
% 7    & 0.685$\pm$0.005 & 0.755$\pm$0.005 & 0.685$\pm$0.005 & 0.667$\pm$0.006 & 0.864$\pm$0.005 & 0.883$\pm$0.005 & 0.864$\pm$0.005 & 0.860$\pm$0.006 \\
% 10   & 0.702$\pm$0.005 & 0.763$\pm$0.005 & 0.702$\pm$0.005 & 0.688$\pm$0.006 & 0.877$\pm$0.005 & 0.890$\pm$0.005 & 0.877$\pm$0.005 & 0.873$\pm$0.006 \\
% \bottomrule
% \end{tabular}
% }
% \caption{Prediction Performance with Different Number of shots for Supporting One New Action.}
% \label{tab:action_shot_study}
% \end{table}
% \renewcommand{\arraystretch}{1.0}

\renewcommand{\arraystretch}{1.3}
\begin{table}[]
\centering
\caption{Detailed Few-shot Pipeline Performance with Different Numbers of Shots when Adding Personal Action.
Window-level results are based on each sliding window as a data point.
Action-level results are the aggregation of the sliding windows after smoothing post-processing (threshold=3) and are closer to real-life application scenarios.
}
\label{tab:action_shot_study}
\resizebox{1\textwidth}{!}{
\begin{tabular}[t]{l|cccc|cccc}
\toprule
\multirow{2}{*}{\textbf{Shots}}  & \multicolumn{4}{c}{\textbf{Window-level}}      & \multicolumn{4}{c}{\textbf{Action-level}}       \\ \cline{2-9}
 & \textbf{Acc}   & \textbf{Prec}  & \textbf{Rec}   & \textbf{F1}    & \textbf{Acc}   & \textbf{Prec}  & \textbf{Rec}   & \textbf{F1}      \\ \hline
1    & 0.614$\pm$0.006 & 0.700$\pm$0.007 & 0.614$\pm$0.006 & 0.571$\pm$0.008 & 0.768$\pm$0.007 & 0.810$\pm$0.007 & 0.768$\pm$0.007 & 0.748$\pm$0.009 \\
3    & 0.658$\pm$0.005 & 0.736$\pm$0.005 & 0.658$\pm$0.005 & 0.634$\pm$0.006 & 0.832$\pm$0.006 & 0.860$\pm$0.005 & 0.832$\pm$0.006 & 0.825$\pm$0.006 \\
5    & 0.670$\pm$0.005 & 0.746$\pm$0.005 & 0.670$\pm$0.005 & 0.648$\pm$0.006 & 0.847$\pm$0.005 & 0.871$\pm$0.005 & 0.847$\pm$0.005 & 0.842$\pm$0.006 \\
7    & 0.685$\pm$0.005 & 0.755$\pm$0.005 & 0.685$\pm$0.005 & 0.667$\pm$0.006 & 0.864$\pm$0.005 & 0.883$\pm$0.005 & 0.864$\pm$0.005 & 0.860$\pm$0.006 \\
10   & 0.702$\pm$0.005 & 0.763$\pm$0.005 & 0.702$\pm$0.005 & 0.688$\pm$0.006 & 0.877$\pm$0.005 & 0.890$\pm$0.005 & 0.877$\pm$0.005 & 0.873$\pm$0.006 \\
\bottomrule
\end{tabular}
}
\end{table}
\renewcommand{\arraystretch}{1.0}

% \renewcommand{\arraystretch}{1.3}
% \begin{table}[]
% \centering
% \resizebox{1\textwidth}{!}{
% \begin{tabular}[t]{c|c|cccc|cccc}
% \toprule
% \multirow{2}{*}{\textbf{Action Num}} & \multirow{2}{*}{\textbf{Shot Num}}  & \multicolumn{4}{c}{\textbf{Window-level}}      & \multicolumn{4}{c}{\textbf{Action-level}}       \\ \cline{3-10}
%  & & \textbf{Acc}   & \textbf{Prec}  & \textbf{Rec}   & \textbf{F1}    & \textbf{Acc}   & \textbf{Prec}  & \textbf{Rec}   & \textbf{F1}      \\ \hline
% \multirow{5}{*}{1} & 1    & 0.614{\tiny$\pm$0.006} & 0.700{\tiny$\pm$0.007} & 0.614{\tiny$\pm$0.006} & 0.571{\tiny$\pm$0.008} & 0.768{\tiny$\pm$0.007} & 0.810{\tiny$\pm$0.007} & 0.768{\tiny$\pm$0.007} & 0.748{\tiny$\pm$0.009} \\
%  & 3    & 0.658{\tiny$\pm$0.005} & 0.736{\tiny$\pm$0.005} & 0.658{\tiny$\pm$0.005} & 0.634{\tiny$\pm$0.006} & 0.832{\tiny$\pm$0.006} & 0.860{\tiny$\pm$0.005} & 0.832{\tiny$\pm$0.006} & 0.825{\tiny$\pm$0.006} \\
%  & 5    & 0.670{\tiny$\pm$0.005} & 0.746{\tiny$\pm$0.005} & 0.670{\tiny$\pm$0.005} & 0.648{\tiny$\pm$0.006} & 0.847{\tiny$\pm$0.005} & 0.871{\tiny$\pm$0.005} & 0.847{\tiny$\pm$0.005} & 0.842{\tiny$\pm$0.006} \\
%  & 7    & 0.685{\tiny$\pm$0.005} & 0.755{\tiny$\pm$0.005} & 0.685{\tiny$\pm$0.005} & 0.667{\tiny$\pm$0.006} & 0.864{\tiny$\pm$0.005} & 0.883{\tiny$\pm$0.005} & 0.864{\tiny$\pm$0.005} & 0.860{\tiny$\pm$0.006} \\
%  & 10   & 0.702{\tiny$\pm$0.005} & 0.763{\tiny$\pm$0.005} & 0.702{\tiny$\pm$0.005} & 0.688{\tiny$\pm$0.006} & 0.877{\tiny$\pm$0.005} & 0.890{\tiny$\pm$0.005} & 0.877{\tiny$\pm$0.005} & 0.873{\tiny$\pm$0.006} \\
% \hline
% \multirow{5}{*}{2} & 1    & 0.578{\tiny$\pm$0.003} & 0.622{\tiny$\pm$0.004} & 0.578{\tiny$\pm$0.003} & 0.521{\tiny$\pm$0.004} & 0.701{\tiny$\pm$0.004} & 0.717{\tiny$\pm$0.004} & 0.701{\tiny$\pm$0.004} & 0.667{\tiny$\pm$0.004} \\
%  & 3    & 0.639{\tiny$\pm$0.003} & 0.684{\tiny$\pm$0.003} & 0.639{\tiny$\pm$0.003} & 0.608{\tiny$\pm$0.003} & 0.783{\tiny$\pm$0.003} & 0.797{\tiny$\pm$0.003} & 0.783{\tiny$\pm$0.003} & 0.770{\tiny$\pm$0.003} \\
%  & 5    & 0.667{\tiny$\pm$0.003} & 0.707{\tiny$\pm$0.003} & 0.667{\tiny$\pm$0.003} & 0.643{\tiny$\pm$0.003} & 0.817{\tiny$\pm$0.003} & 0.827{\tiny$\pm$0.003} & 0.817{\tiny$\pm$0.003} & 0.808{\tiny$\pm$0.003} \\
%  & 7    & 0.690{\tiny$\pm$0.003} & 0.725{\tiny$\pm$0.003} & 0.690{\tiny$\pm$0.003} & 0.672{\tiny$\pm$0.003} & 0.838{\tiny$\pm$0.003} & 0.847{\tiny$\pm$0.003} & 0.838{\tiny$\pm$0.003} & 0.832{\tiny$\pm$0.003} \\
%  & 10   & 0.708{\tiny$\pm$0.003} & 0.738{\tiny$\pm$0.003} & 0.708{\tiny$\pm$0.003} & 0.695{\tiny$\pm$0.003} & 0.854{\tiny$\pm$0.003} & 0.862{\tiny$\pm$0.003} & 0.854{\tiny$\pm$0.003} & 0.849{\tiny$\pm$0.003} \\
% \hline
% \multirow{5}{*}{3} & 1    & 0.566{\tiny$\pm$0.002} & 0.572{\tiny$\pm$0.003} & 0.566{\tiny$\pm$0.002} & 0.511{\tiny$\pm$0.003} & 0.670{\tiny$\pm$0.003} & 0.667{\tiny$\pm$0.004} & 0.670{\tiny$\pm$0.003} & 0.636{\tiny$\pm$0.003} \\
%  & 3    & 0.630{\tiny$\pm$0.002} & 0.654{\tiny$\pm$0.003} & 0.630{\tiny$\pm$0.002} & 0.596{\tiny$\pm$0.003} & 0.758{\tiny$\pm$0.002} & 0.766{\tiny$\pm$0.003} & 0.758{\tiny$\pm$0.002} & 0.742{\tiny$\pm$0.003} \\
%  & 5    & 0.660{\tiny$\pm$0.002} & 0.683{\tiny$\pm$0.002} & 0.660{\tiny$\pm$0.002} & 0.635{\tiny$\pm$0.003} & 0.791{\tiny$\pm$0.002} & 0.799{\tiny$\pm$0.003} & 0.791{\tiny$\pm$0.002} & 0.781{\tiny$\pm$0.003} \\
%  & 7    & 0.682{\tiny$\pm$0.002} & 0.701{\tiny$\pm$0.002} & 0.682{\tiny$\pm$0.002} & 0.662{\tiny$\pm$0.003} & 0.811{\tiny$\pm$0.002} & 0.819{\tiny$\pm$0.002} & 0.811{\tiny$\pm$0.002} & 0.803{\tiny$\pm$0.002} \\
%  & 10   & 0.701{\tiny$\pm$0.002} & 0.718{\tiny$\pm$0.002} & 0.701{\tiny$\pm$0.002} & 0.686{\tiny$\pm$0.003} & 0.826{\tiny$\pm$0.002} & 0.835{\tiny$\pm$0.002} & 0.826{\tiny$\pm$0.002} & 0.822{\tiny$\pm$0.002} \\
% \hline
% \multirow{5}{*}{4} & 1    & 0.557{\tiny$\pm$0.002} & 0.553{\tiny$\pm$0.003} & 0.557{\tiny$\pm$0.002} & 0.501{\tiny$\pm$0.003} & 0.659{\tiny$\pm$0.003} & 0.656{\tiny$\pm$0.003} & 0.659{\tiny$\pm$0.003} & 0.627{\tiny$\pm$0.003} \\
%  & 3    & 0.619{\tiny$\pm$0.002} & 0.635{\tiny$\pm$0.003} & 0.619{\tiny$\pm$0.002} & 0.583{\tiny$\pm$0.003} & 0.739{\tiny$\pm$0.003} & 0.745{\tiny$\pm$0.003} & 0.739{\tiny$\pm$0.003} & 0.723{\tiny$\pm$0.003} \\
%  & 5    & 0.647{\tiny$\pm$0.002} & 0.663{\tiny$\pm$0.003} & 0.647{\tiny$\pm$0.002} & 0.621{\tiny$\pm$0.003} & 0.770{\tiny$\pm$0.002} & 0.778{\tiny$\pm$0.003} & 0.770{\tiny$\pm$0.002} & 0.761{\tiny$\pm$0.003} \\
%  & 7    & 0.667{\tiny$\pm$0.002} & 0.681{\tiny$\pm$0.003} & 0.667{\tiny$\pm$0.002} & 0.646{\tiny$\pm$0.003} & 0.790{\tiny$\pm$0.002} & 0.799{\tiny$\pm$0.002} & 0.790{\tiny$\pm$0.002} & 0.783{\tiny$\pm$0.002} \\
%  & 10   & 0.689{\tiny$\pm$0.002} & 0.701{\tiny$\pm$0.003} & 0.689{\tiny$\pm$0.002} & 0.673{\tiny$\pm$0.003} & 0.807{\tiny$\pm$0.002} & 0.817{\tiny$\pm$0.002} & 0.807{\tiny$\pm$0.002} & 0.803{\tiny$\pm$0.002} \\
% \hline
% \multirow{5}{*}{5} & 1    & 0.542{\tiny$\pm$0.003} & 0.524{\tiny$\pm$0.004} & 0.542{\tiny$\pm$0.003} & 0.481{\tiny$\pm$0.004} & 0.636{\tiny$\pm$0.004} & 0.618{\tiny$\pm$0.005} & 0.636{\tiny$\pm$0.004} & 0.600{\tiny$\pm$0.005} \\
%  & 3    & 0.610{\tiny$\pm$0.003} & 0.612{\tiny$\pm$0.004} & 0.610{\tiny$\pm$0.003} & 0.573{\tiny$\pm$0.004} & 0.722{\tiny$\pm$0.004} & 0.724{\tiny$\pm$0.004} & 0.722{\tiny$\pm$0.004} & 0.705{\tiny$\pm$0.004} \\
%  & 5    & 0.643{\tiny$\pm$0.004} & 0.649{\tiny$\pm$0.004} & 0.643{\tiny$\pm$0.004} & 0.614{\tiny$\pm$0.004} & 0.755{\tiny$\pm$0.004} & 0.762{\tiny$\pm$0.004} & 0.755{\tiny$\pm$0.004} & 0.745{\tiny$\pm$0.004} \\
%  & 7    & 0.663{\tiny$\pm$0.004} & 0.670{\tiny$\pm$0.004} & 0.663{\tiny$\pm$0.004} & 0.640{\tiny$\pm$0.004} & 0.774{\tiny$\pm$0.004} & 0.782{\tiny$\pm$0.004} & 0.774{\tiny$\pm$0.004} & 0.768{\tiny$\pm$0.004} \\
%  & 10   & 0.684{\tiny$\pm$0.004} & 0.690{\tiny$\pm$0.004} & 0.684{\tiny$\pm$0.004} & 0.666{\tiny$\pm$0.004} & 0.795{\tiny$\pm$0.004} & 0.803{\tiny$\pm$0.004} & 0.795{\tiny$\pm$0.004} & 0.790{\tiny$\pm$0.004} \\
% \hline
% \multirow{5}{*}{6} & 1    & 0.533{\tiny$\pm$0.009} & 0.510{\tiny$\pm$0.010} & 0.533{\tiny$\pm$0.009} & 0.479{\tiny$\pm$0.010} & 0.617{\tiny$\pm$0.010} & 0.613{\tiny$\pm$0.011} & 0.617{\tiny$\pm$0.010} & 0.589{\tiny$\pm$0.011} \\
%  & 3    & 0.599{\tiny$\pm$0.009} & 0.592{\tiny$\pm$0.010} & 0.599{\tiny$\pm$0.009} & 0.561{\tiny$\pm$0.010} & 0.703{\tiny$\pm$0.010} & 0.706{\tiny$\pm$0.011} & 0.703{\tiny$\pm$0.010} & 0.686{\tiny$\pm$0.011} \\
%  & 5    & 0.630{\tiny$\pm$0.009} & 0.628{\tiny$\pm$0.011} & 0.630{\tiny$\pm$0.009} & 0.601{\tiny$\pm$0.010} & 0.734{\tiny$\pm$0.009} & 0.739{\tiny$\pm$0.011} & 0.734{\tiny$\pm$0.009} & 0.723{\tiny$\pm$0.010} \\
%  & 7    & 0.650{\tiny$\pm$0.009} & 0.649{\tiny$\pm$0.010} & 0.650{\tiny$\pm$0.009} & 0.625{\tiny$\pm$0.010} & 0.756{\tiny$\pm$0.009} & 0.764{\tiny$\pm$0.010} & 0.756{\tiny$\pm$0.009} & 0.749{\tiny$\pm$0.009} \\
%  & 10   & 0.673{\tiny$\pm$0.009} & 0.673{\tiny$\pm$0.010} & 0.673{\tiny$\pm$0.009} & 0.653{\tiny$\pm$0.010} & 0.776{\tiny$\pm$0.009} & 0.785{\tiny$\pm$0.010} & 0.776{\tiny$\pm$0.009} & 0.771{\tiny$\pm$0.010} \\
% \bottomrule
% \end{tabular}
% }
% \caption{Prediction Performance with Different Number of Shots and Actions.}
% \label{tab:action_shot_study}
% \end{table}
% \renewcommand{\arraystretch}{1.0}


\subsubsection{Prediction Performance of Each New Gesture with Different Number of Shots}
\label{subsub:model_evaluation:pipeline_evaluation:one new gesture}


We further compared the recognition performance across actions.
As shown in \autoref{fig:all_result_shot_num_action}, most of the 17 actions exhibited good performance. Using only one shot, about half of the actions achieved an F1 score above 75\%. When the number of shots increased to five, 14 out of 17 actions surpassed this threshold. With ten shots, performance improved further for most actions, with 12 out of 17 actions achieving an F1 score above 85\%.
\textit{Hair Pulling} appeared to be an exception. Its performance did not improve with more samples after five shots. This was probably due to the overly large variance of the \textit{Hair Pulling} action, even performed by the same individual, and it was challenging for a model to achieve reliable performance even with a limited amount of additional data.

Overall, these results indicate our framework has good learning ability for new actions.

\begin{figure*}[]
\centering 
\includegraphics[width=\linewidth]{figure/all_result_shot_num_action.png}
\caption{Model Performance of Recognizing Each Action with 1, 5, or 10 Shots. For consistency, each action was added alone (\ie binary classification model). The ``(User Num)'' indicates how many users did this action. The five pre-determined actions (Lip Tearing, Nail Biting, Face Scratching, Eye Rubbing, and Leg Shaking) have the total number of participants (26), and other self-defined actions are more scattered.}
\label{fig:all_result_shot_num_action}
\Description{}
\end{figure*}

% \subsubsection{Ablation Studies of Pipeline Stages}
% \label{subsub:model_evaluation:pipeline_evaluation:ablation}
% \orson{@Ying to add more details}

% To measure the effectiveness of different stages in our few-shot pipeline, we further conducted an ablation study by removing different stages in our pipeline.
% \autoref{tab:ablation_study} indicates xxxx. \orson{@Ying to add more details of the results interpretation.}

% \begin{table}[t]
\vspace{-13pt}
\caption{Ablation results with response number under fine-tuning setting. See Reward Margins in~\Cref{tab:rm}. \vspace{-0.5em}}
\label{tab:ablation}
\vskip 0.1in
\begin{center}
\scalebox{0.75}{
\begin{tabular}{clcccc}
\toprule
\multirow{1}{*}{\textbf{Number}} & \multirow{1}{*}{\textbf{Method}} & \textbf{BLEU}$\uparrow$ & \textbf{Reward} & $\textbf{RM}_{\text{DPO}}$$\uparrow$ & $\textbf{RM}_{\text{R-DPO}}$$\uparrow$ \\
\midrule
\multirow{2}{*}{\textbf{5}} & \textbf{DPO-BT} & \textbf{0.229} & \textbf{{0.432}} & 0.166 & -0.516 \\

& \textbf{DPO-HPS} & \textbf{0.229} & 0.431 & \textbf{0.600} & \textbf{-0.273} \\
\midrule
\multirow{2}{*}{\textbf{20}} & \textbf{DPO-BT} & \textbf{0.231} & 0.430 & 0.227 & -0.490 \\

& \textbf{DPO-HPS} & 0.224 & \textbf{{0.432}} & \textbf{0.822} & \textbf{-0.181} \\
\midrule
\multirow{2}{*}{\textbf{50}} & \textbf{DPO-BT} & \textbf{0.230} & \textbf{0.431} & 0.279 & -0.507 \\

& \textbf{DPO-HPS} & \textbf{0.230} & \textbf{0.431} & \textbf{1.645} & \textbf{1.037} \\
\midrule
\multirow{2}{*}{\textbf{100}} & \textbf{DPO-BT} & 0.230 & \textbf{0.431} & 0.349 & -0.455 \\

& \textbf{DPO-HPS} & \textbf{{0.232}} & 0.430 & \textbf{{2.723}} & \textbf{{2.040}} \\
\bottomrule
\end{tabular}}
\end{center}
\vspace{-1em}
\vspace{-9pt}
\end{table}


\section{Intervention Evaluation}
\label{sec:intervention_evaluation}
The promising model performance in Sec.~\ref{sub:model_evaluation:pipeline_evaluation} has validated the effectiveness of our few-shot learning pipeline.
Building upon the pipeline, we further conducted a user study to evaluate the effectiveness of \projectname and compared it against a rule-based baseline intervention system.

\subsection{Participants}
\label{sub:intervention_evaluation:participants}
With IRB approval, we recruited the same set of participants in Sec.~\ref{sub:model_evaluation:data_collection} for a follow-up intervention study. 
In the previous data collection, participants performed five per-determined actions and a self-defined action. In this study, they were asked to select one of the six actions that they had the strongest need for intervention.
This action was set as the target action for intervention during the study.
Among the 26 participants, 5 of them did not follow the study protocol. Their results were removed as outliers.  This section focused on the findings based on the remaining 21 participants.

\subsection{Intervention Setting}
\label{sub:intervention_evaluation:setting}
Since personal undesirable actions are inherently difficult to predict or control, we designed an intervention experience that closely mirrors real-life contexts to enhance ecological validity, encouraging participants to perform these actions under more natural conditions.
Our initial conversation with participants indicated two common scenarios where they tended to perform these actions: when they were in an engaging task with a relaxing state (\eg watching an interesting movie or a reality show with dramatic twists and turns); and when they were bored or disengaged (\eg mindlessly scrolling through social media or watching a tedious video) \footnote{Several participants also mentioned the scenarios under pressure or stress. Considering the feasibility and ethics of a multi-hour intervention study, we did not provide this option.}.
Therefore, we set up two types of video-watching tasks and allowed participants to pick the type in which they tended to perform more undesirable actions.

The first type included \textit{engaging} videos. We prepare a set of multi-hour videos for participants to choose from, such as the Harry Potter movie series, sports competitions, and mystery/detective shows.
The second type was watching \textit{disengaging} videos. Examples include cycling or driving route videos, math problem explanations, and public health lecture videos.
Participants sat in a quiet room with a laptop on the table and watched the video they selected, as shown in \autoref{fig:intervention_setup}(a) and (b).
During the video-watching, participants were not interrupted by the experimenter, simulating the real-life setting. 

\begin{figure*}[]
\centering 
\begin{subfigure}[t]{1\textwidth}
    \centering
    \includegraphics[width=\textwidth]{figure/setting.jpg}
    \caption{Intervention Room Setup}
    \label{subfig:intervention_setup:design}
\end{subfigure}
\caption{\projectname Intervention Evaluation Setup. (a) The sketch of the study room and apparatus setup for intervention. (b) The photo of a participant in the room. (c) The video from the camera on the corner that records the ground truth.}
\label{fig:intervention_setup}
\Description{}
\end{figure*}

\subsection{Study Design and Procedure}
\label{sub:intervention_evaluation:design}
We adopted a within-subject design and compared our AI-powered \projectname against a rule-based intervention system.
In the rule-based system, a regular notification (the same interface as \autoref{subfig:interface_design:intervention}) was delivered every 10 minutes, regardless of whether the user did the action.
To mitigate the effect of the two systems outputting different numbers of notifications, we further added restrictions in \projectname so that the number of delivered notifications would be in the range of $\times 0.5$ to $\times 2$ as the baseline system.
This was achieved by forcefully delivering a notification if there was no intervention by the end of each 20-minute window ($\times 0.5$ times of interventions in minimum).
With the 5-min cool-down setup, \projectname can only deliver up to one intervention every 5 minutes, which would be no more than $\times 2$ times of interventions as the baseline.

Our study procedure was designed as follows. After selecting the personal target undesirable action and the task type (engaging vs disengaging), participants would calibrate and familiarize themselves with the intervention system and study setup. They then attended two intervention sessions in total, one session per day. We counterbalanced the order between \projectname and the baseline system, and participants were blind to the order of the two systems.
After familiarizing themselves with the room environment and setup, participants went through each intervention session with three stages (in total 130 minutes): (1) a 30-minute \textit{pre-intervention stage}, where there was no intervention delivered; (2) a 90-minute \textit{intervention stage}, where \projectname or the baseline system would deliver interventions as designed; and (3) a 10-minute \textit{post-intervention} stage, where no more intervention was delivered to observe any lasting effect \footnote{Due to the restrictions of the room booking time and device battery, we regretfully could not do a longer post-intervention stage. We recognize this as a limitation of our study in discussion.}.

The whole intervention session was video-recorded by a camera from the ceiling, positioned at an angle to capture participants' micro-actions and collect ground truth (see \autoref{fig:intervention_setup}(c)). We manually annotated the video and calculated the number and duration of the target actions during the three stages.
We collected participants' Self-Report of Habit Strength of the target action~\cite{verplanken2003reflections} before and after each session.
After the post-intervention stage, we further collected quantitative data from participants with a questionnaire that includes System Usability Scale (SUS) survey~\cite{bangor2008empirical} and Working Alliance Inventory (WAI, short revision)~\cite{munder2010working}.
In addition, we conducted a brief semi-structured interview to collect qualitative feedback on the intervention experience from each participant.

In total, the two sessions took around 5 hours for each participant. To reduce user fatigue, the two sessions were scheduled on two different days within a week. Participants were compensated with \$50 for the intervention study.
% \orson{@Yancheng to double-check and fill in the content in this section.}

\subsection{Intervention Results}
\label{sub:intervention_evaluation:intervention_results}
We first summarize the quantitative results from our study.
We coded the recorded videos by documenting the duration of target actions performed by participants every 10 minutes across the three stages.
Since participants had diverse behavior patterns, we normalized the results with each individual's target action duration in the pre-intervention stage as the reference.
The \textit{relative duration} was calculated by dividing the average duration of target actions per 10 minutes in both the intervention and post-intervention stages by that of the pre-intervention stage. A lower relative duration means more reduction of the target actions compared to the pre-intervention stage.

% Overall, the real-time pipeline of \projectname had a consistent performance as Sec.~\ref{sec:model_evaluation}.
% , with an overall accuracy of XX.X\% and a false positive rate of XX.

% \subsubsection{Controlled Effects on the Number of Interventions.}
% As the frequency of reminders could be one of the factors affecting intervention effectiveness, we compared the number of interventions between AI-based and rule-based intervention modes. A significant difference was found between the two modes. The mean number of interventions in the AI-based group was 11.76 and in the rule-based group was 9.00. This difference was statistically significant according to an independent-samples t-test (t(20) = 2.97, p < .01). As a result, we included the number of interventions as a covariate in our statistical model to control for its effect.

% \orson{@Yancheeng, check whether there is a significant difference between the number of interventions. Hopefully no. If it is, we would say that we will add the number of interventions in the statistical model to control its effect. I don't think we would need a figure here.}
\begin{figure*}[]
\centering 
\includegraphics[width=\linewidth]{figure/duration_result.pdf}
\caption{(a) Relative Duration of target action every 10 minutes in intervention and post-intervention stages (compared to the pre-intervention stage). A number lower than 1.0 means that an individual performed fewer target actions after intervention.
(b) Average Relative Duration of target action over time. The dashed lines fit the last 10 minutes of the pre-intervention stage and the rest of the session.
}
\label{fig:duration_result}
\Description{}
\end{figure*}

\subsubsection{Reduction of the Duration of Target Actions by Intervention.}
% After the study, we coded the recorded videos by documenting the start and end times of target actions performed by all participants during the last three stages.
We compare the relative duration between \projectname and the baseline in both intervention and post-intervention stages.
Since participants received slightly more notifications in \projectname during the intervention stage (on average 11.8 \vs 9.0 times per session), we controlled the effect of the number of notifications by using generalized linear mixed models (GLMMs).
Specifically, a GLMM had relative duration as the dependent variable, with the intervention method (AI-based in \projectname \vs rule-based in baseline) and the number of notifications as the main factors.

As shown in the left of Fig.\ref{fig:duration_result}(a), during the intervention stage, \projectname resulted in 36.0 $\pm$ 22.6\% of the duration compared to the pre-intervention stage (\ie a reduction of 64.0\% of the target undesirable action), and the baseline system led to 65.0 $\pm$ 47.5\% of the duration (\ie a reduction of 35.0\%).
We fitted a GLMM to compare the two intervention methods.
Our results revealed the significant difference between the two methods: \projectname significantly outperformed the baseline by 29.0\% more reduction of the target undesirable action ($\chi^2_1=6.32, p < .05$). Meanwhile, the number of notifications does not show significance ($\chi^2_1=0.53, p = 0.47$). These results suggest that the advantage of \projectname was mainly attributed to the AI-based intervention method.

In addition, although our post-intervention stage was short, both methods showed promising signals of a potential lasting effect when the intervention was gone (13.9 $\pm$ 16.8\% for the \projectname; 37.7 $\pm$ 37.2\% for the baseline), as shown in the right of Fig.\ref{fig:duration_result}(a).
We fitted another GLMM on the post-intervention data. The results also indicate the significance of the intervention method ($\chi^2_1=10.04, p<0.01$), but not the number of notifications ($\chi^2_1=0.12, p=0.73$).
This is consistent with the result of the intervention stage, further demonstrating the superior performance of \projectname over the baseline method.


% The model also included a non-significant intercept (b = 0.38, OR = 1.46, 95\% CI [0.90, 2.37], p = .131) and intervention times (b = 0.00, OR = 1.00, 95\% CI [0.96, 1.04], p = .933).
% These results indicate that \projectname was more effective in reducing participants' target actions compared to scheduled reminders, and this effect was not significantly associated with the number of intervention times. The GLMM analysis also showed that the relative duration of target actions during the post-intervention stage remained significantly lower in the AI-based condition compared to rule-based intervention (b = 0.25, OR = 1.28, 95\% CI [1.05, 1.55], p < .05). The model included a non-significant intercept (b = 0.12, OR = 1.13, 95\% CI [0.78, 1.63], p = .537) and intervention times (b = 0.00, OR = 1.00, 95\% CI [0.97, 1.03], p = .910). These findings suggest that \projectname demonstrated superior lasting effects in reducing target actions compared to the rule-based condition during the short-term post-intervention period.

% \orson{@Yancheng to add the results figure and write up the findings}

\subsubsection{Intervention Effectiveness over Time.}
To investigate changes in the duration of target action during the study session, we visualize the change of participants' target action duration throughout the study (see \autoref{fig:duration_result}(b)).
Both intervention methods showed a clear and significant decreasing trend once participants entered the intervention stage.
The fitted lines in \autoref{fig:duration_result}(b) indicate that \projectname achieved more duration reduction ($m=-4.8\%$ per 10-minute) compared to the baseline ($m=-4.1\%$) over the intervention session.
In particular, \projectname had a more rapid initial decrease and maintained consistently lower levels throughout the rest of the session compared to the rule-based baseline.
Overall, \projectname demonstrated stronger cumulative effects.

\subsubsection{Difference across Task Types.}
During the study, we asked participants to pick their own preferred task types between watching engaging (N=11) or disengaging videos (N=10).
\autoref{fig:duration_result_breakdown} presents the breakdown of the task type in \autoref{fig:duration_result}(a).
We fitted GLMMs with task type as another main factor and observed a marginal significance of the interaction between the intervention method and the task type ($\chi^2_1=3.27, p=0.07<0.1$). This was only during the intervention stage, but not the post-intervention stage.
\autoref{fig:duration_result_breakdown}(a) and (b) indicate that the advantage of \projectname during the intervention stage was more salient when participants were watching engaging videos ($\Delta=42.3\pm49.6\%$) compared to when they were watching disengaging videos ($\Delta=14.2\pm22.5\%$).
This could be due to the fact that participants were more interruptable or receptive in less engaging tasks~\cite{pielot2017beyond,mishra_detecting_2021,choi_multi-stage_2019}, thus even a basic rule-based intervention could effectively reduce the target actions. However, in more engaging tasks, accurate and just-in-time reminders are more effective than basic ones.

\begin{figure*}[]
\centering 
\includegraphics[width=0.7\linewidth]{figure/engage_result.pdf}
\caption{
(a) Relative duration of target action for participants who watched engaging videos.
(b) Relative duration of target action for participants who watched disengaging videos.
}
\label{fig:duration_result_breakdown}
\Description{}
\end{figure*}

\subsubsection{Survey Outcomes.}
In addition to the objective measurement, we also compare participants' subjective reports on the SUS, WAI, and the change of the habit strength.
Overall, participants reported that \projectname had better usability (SUS: 73.3$\pm$12.8) than the baseline (66.8$\pm$15.9), with significance through a Wilcoxon rank sum test ($p<0.05$).
\projectname achieved a SUS score over 70, indicating acceptable usability. In both methods, false positive notifications were inevitable and could introduce participants' confusion or surprise, which could explain the subpar SUS scores in general.

Interestingly, the results of WAI and habit strength did not indicate such a difference. Participants had similar reports of the relationship with the system (WAI score: 42.1$\pm$7.1 for the \projectname \vs 41.9$\pm$9.3 for the baseline, $p=0.58$).
The change of habit strength between the pre- and post-intervention stages is also minimal ($\Delta$ of habit strength score: $-4.2\pm5.5$ for the \projectname \vs $-5.5\pm6.6$ for the baseline, $p=0.25$).
This was probably due to the fact that the intervention sessions were not long enough to form a long-term alliance between users and the system, or to influence longitudinal behaviors or habits.
Our qualitative results from semi-structured interviews provide more nuanced insights into these results.


\subsection{Qualitative Results}
\label{sub:intervention_evaluation:qualitative_results}
% Moreover, our interview results also revealed interesting insights into \projectname that enables a human-AI collaborative intervention experience.
All interviews were recorded and transcribed. We adopted a simple content analysis framework~\cite{prior2014content}. One author took extensive notes during the interviews, went through the scripts to categorize themes and count their frequency, and discussed with two other authors until convergence.
We summarize our key findings below.

% \orson{@yancheng to add more details and quotes.}

\subsubsection{Perception of AI-powered Intervention}
Multiple participants reported that the AI-powered system possessed a sense of presence or "\textit{having a soul}". For instance, P10 noted, "\textit{[\projectname{}] resembles a habit instructor, or even like my mom... who would gently remind me when I scratch my head.}"
P18 remarked, "\textit{This system seems to read my mind, anticipating when I'm about to bite my lips and reminding me just in time. Sometimes I felt like I was sneaking around when making these actions.}"
Compared to the rule-based condition, \projectname's interventions appeared to foster greater self-reflection among users.
Notably, P19 even perceived the AI's reminders as rewards: "\textit{After being caught [touching my face] several times initially, I managed to control myself for a while. Then, even if the system reminded me again, I felt it was affirming my progress, like receiving a reward.}"
In contrast, the rule-based condition yielded opposite effects, "\textit{This mode of notification felt random to me - it was just like a machine}" (P02).

However, some participants also had a negative experience with \projectname, especially when it did not detect the actions accurately (mostly false positive). For example, P08 mentioned that \projectname had limited impact, and that they also felt a sense of distrust. "\textit{At first, when it reported errors a few times, I tried to look for reasons elsewhere. But it kept making mistakes, which became frustrating. When it occasionally got something right, I thought it was just luck!}" Participants could lose trust in \projectname when the system made mistakes at the beginning of their interaction.
This is supported by prior research in other human-AI interaction systems~\cite{swaroop2024accuracy,jacobs2021designing}.


\subsubsection{Illusory Amplification of Intervention Strength}
We noticed a surprisingly interesting phenomenon: 
Several users (P09, P10, and P16) reported that the vibration strength of the AI-based intervention in \projectname felt stronger than that of the rule-based intervention. However, the vibration setup was identical in the two sessions.
Even after we explained the specific intervention methods after the two study sessions, P16 stated, "\textit{Not only did I subjectively feel that Mode B [our \projectname method] gave me a stronger sense of motion restraint, but it also seemed to vibrate more intensely. Are you sure it's really the same setting?}"
This indicated that participants might develop an illusory or distorted perception of the intervention's strength when the interventions were delivered just-in-time.
We discuss this more in Sec.~\ref{sub:discussion:distorted}.

\subsubsection{Diverse Patterns of Human-AI Collaborative Relationship}
Users exhibited diverse patterns of engagement with the AI system. Some participants demonstrated adaptive behavioral modification in response to \projectname's reminders. As P14 described, "\textit{Every time I shook my leg, it would remind me, which made me increasingly hesitant to move}". This was aligned with our original design goal of introducing AI-powered JITI.

Other than reducing the target actions, we also observed other behavior patterns. One pattern emerged where participants developed an interesting competitive relationship with the AI for user agency. For instance, P8 articulated this sentiment: "\textit{I wanted to compete with it - I tried to resist the urge just so it wouldn't catch me.}" This competitive spirit evolved into experimental behavior for some users, who attempted to understand and control the system's underlying logic. P18's experience exemplified this progression: "\textit{Initially, I felt caught red-handed with every reminder. Later, I noticed it wouldn't always detect my subtle movements, so I started experimenting with the notification logic, trying to gain control over the reminders. Eventually, though, I made peace with it and lost the urge to perform the action altogether.}" These participants wanted to gain better agency in this human-AI relationship.

In addition, some participants developed playful interactions with the system, treating it as an engaging companion rather than a mere monitoring tool. For example, P17 shared: "\textit{When the video was boring... I just wanted to goof around a bit. This thing was actually keeping an eye on me, so I'd mess with it for fun, play around with it, and boom - it would react right away. Kinda helped wake me up a bit? It was basically like playing a game.}"
Overall, these diverse patterns between users and \projectname suggest a set of potential collaborative relationships between the two sides. We discuss this finding in Sec.~\ref{sub:discussion:collaboration}.

\section{Discussion}
Through our application of personalized accessibility maps and routing applications, we showed how data and insights from our survey findings can help inform the development of more accurate navigation and analytical tools. 
We now situate our findings in related work, highlight how this survey contributes to personalized routing and accessibility mapping for mobility disability groups as well as present directions for future research.

\subsection{Online Image Survey Method}
In this study, we conducted a large-scale image survey (\textit{N=}190) to gather perceptions of sidewalk barriers from different mobility aid user groups. 
This approach helped us to collect insights on the differences between mobility aid user groups as well as shared challenges.
Previous research exploring the relationship between mobility aids and physical environment have mainly employed methods including in-person interviews~\cite{rosenberg_outdoor_2013}, GPS tracking~\cite{prescott_exploration_2021, prescott_factors_2020,rosenberg_outdoor_2013}, and online questionnaires~\cite{carlson_wheelchair_2002}. While interviews and tracking studies typically yield rich detailed information, they are limited to a small sample size. Online text based questionnaires often achieve larger sample sizes but at a cost of depth and nuance. Our image survey method struck a balance between sample size and detail. We collected a large sample within a relatively short time frame, enabling us to gather valuable insights and synthesize patterns across user groups.

Despite advantages, our approach has some limitations. Although street view images help situate and ground a participant's response---as one pilot participant said ``\textit{You're triggering a similar response to a real-life scenario''}, they cannot fully replicate the experience of evaluating a sidewalk \textit{in situ}. The lack of physical interaction with the environment limits the assessment of certain factors. For instance, one of our pilot participants noted that determining whether they could navigate past an obstacle like a trash can varies depending on \sayit{whether the trash can is light enough so I can push it away.} Using our findings as a backdrop, future work should conduct follow-up interviews and in-person evaluations. Such approaches would complement the quantitative data with richer qualitative insights, allowing researchers to better understand the patterns observed in quantitative data as well as the reasoning behind mobility aids users’ assessment.

\subsection{Personalized Accessibility Maps}
Our approach to infuse accessibility maps and routing algorithms with personalized information contributes to the field of accessible urban navigation and analytics. 
Based on our findings, we implemented two accessibility-oriented mapping prototypes, which demonstrate how our data can be used in urban accessibility analytics and personalized routing algorithms. While our current implementation serves as a proof of concept, future research could explore using our findings with more advanced modeling methods such as fuzzy logic~\cite{kasemsuppakorn_personalised_2009, gharebaghi_user-specific_2021, hashemi_collaborative_2017} and AHP~\cite{kasemsuppakorn_personalised_2009,kasemsuppakorn_understanding_2015, hashemi_collaborative_2017}. 

For our current map applications, we used a single set of open-source sidewalk data from Project Sidewalk. However, we acknowledge that other important factors are not included, such as sidewalk topography, width, stairs, crossing conditions, paving material, lighting conditions, weather, and pedestrian traffic~\cite{rosenberg_outdoor_2013,kasemsuppakorn_personalised_2009,darko_adaptive_2022,hashemi_collaborative_2017,sobek_u-access_2006,bigonnesse_role_2018}. 
Future work should build upon our foundation by incorporating more crowdsourced and government official datasets.

While mobility aids play a crucial role in determining accessibility needs, we must recognize that individuals using the same type of mobility aid may have diverse preferences. As one of our pilot participants stated, \sayit{your wheelchair has to be shaped and fitted to your body similar to how you need shoes specifically for your feet.} This insight underscores the need for personalization beyond broad mobility aid categories. Other factors including age~\cite{rosenberg_outdoor_2013}, disability type~\cite{prescott_factors_2020}, body strength~\cite{prescott_factors_2020}, and route familiarity~\cite{kasemsuppakorn_understanding_2015} should be explored in the future. Our attempt in creating personalized maps is not to provide a one-size-fits-all solution for generalized mobility aid groups, but rather to leverage the power of defaults~\cite{nielsen_power_2005} and offer users an improved baseline from which they can easily customize based on their individual needs.

\subsection{Limitations and Future Work}
Due to the visual nature of our survey—images were the primary stimuli—we specifically excluded people who are blind or have low vision\footnote{That said, the custom online survey was made fully screen reader accessible; see \href{https://sidewalk-survey.github.io/}{https://sidewalk-survey.github.io/} for the images and alt text.}. However, as noted previously, many different disabilities can impact mobility, including sensory, physical, and cognitive. Prior research has explored the incorporation of visually impaired or blind individuals into route generation~\cite{volkel_routecheckr_2008}, recognizing shared barriers and the prevalence of multiple disabilities among users. Building upon this foundation, future work should expand the participant pool to include a broader range of disabilities, thereby providing a more comprehensive understanding of diverse accessibility needs.

While we demonstrated two basic scenario applications, our survey findings and personalized mapping approach have potential for broader implementation. One promising direction is in developing barrier removal strategies for policymakers~\cite{eisenberg_barrier-removal_2022}. Current government plans often rely on simple metrics, such as population density or proximity to public buildings~\cite{seattle_department_of_transportation_seattle_2021}. Our methodology could enhance these efforts by identifying sidewalk barriers whose removal would yield the greatest overall benefit to the largest percentage of mobility aid users in the form of connected, safe, accessible routes.

\section{Conclusion}

In this paper, we introduce STeCa, a novel agent learning framework designed to enhance the performance of LLM agents in long-horizon tasks. 
STeCa identifies deviated actions through step-level reward comparisons and constructs calibration trajectories via reflection. 
These trajectories serve as critical data for reinforced training. Extensive experiments demonstrate that STeCa significantly outperforms baseline methods, with additional analyses underscoring its robust calibration capabilities.

\bibliographystyle{ACM-Ref-Format}
\bibliography{
bib/Orson/BehaviorIntervention,
bib/Orson/HumanComputerInteraction,
bib/Orson/Modeling_Behavior-General,
bib/Orson/MachineLearning,
bib/Orson/OrsonPublication,
bib/ly/BehaviorIntervention,
bib/ly/MachineLearning,
bib/ly/Dataset,
bib/ly/others,
bib/others
}

\appendix
\section{Hard Threshold of EAC}\label{threshhold}
In constructing a weighted-gradient saliency map, the value of \(\gamma\) determines the number of the dimensions we select where important feature anchors are located. As the value of \(\gamma\) increases, the number of selected dimensions decreases, requiring the editing information to be compressed into a smaller space during the compression process. 
During compression, it is desired for the compression space to be as small as possible to preserve the general abilities of the model. However, reducing the compression space inevitably increases the loss of editing information, which reduces the editing performance of the model.
Therefore, to ensure editing performance in a single editing scenario, different values of \(\gamma\) are determined for various models, methods, and datasets. Fifty pieces of knowledge were randomly selected from the dataset, and reliability, generalization, and locality were measured after editing. The averages of these metrics were then taken as a measure of the editing performance of the model.
Table~\ref{value} presents the details of \(\gamma\), while Table~\ref{s} illustrates the corresponding editing performance before and after the introduction of EAC. $P_{x}$ denotes the value below which x\% of the values in the dataset.


\begin{table}[!htb]
\caption{The value of $\gamma$.}
\centering
\resizebox{0.45\textwidth}{!}{
\begin{tabular}{lcccc}
\toprule
\textbf{Datasets} & \textbf{Model} & \textbf{ROME} & \textbf{MEMIT} \\
\midrule
\multirow{2}{*}{\textbf{ZSRE}} & GPT-2 XL & $P_{80}$ & $P_{80}$ \\
 & LLaMA-3 (8B) & $P_{90}$ & $P_{95}$ \\
\midrule
\multirow{2}{*}{\textbf{COUNTERFACT}} & GPT-2 XL & $P_{85}$ & $P_{85}$ \\
 & LLaMA-3 (8B) & $P_{95}$ & $P_{95}$ \\
\bottomrule
\end{tabular}}
\label{value}
\end{table}


\begin{table}[!htb]
\caption{The value of $\gamma$.}
\centering
\resizebox{\textwidth}{!}{%
\begin{tabular}{lccccccccccccc}
\toprule
\multirow{1}{*}{Dataset} & \multirow{1}{*}{Method} & \multicolumn{3}{c}{\textbf{GPT-2 XL}} & \multicolumn{3}{c}{\textbf{LLaMA-3 (8B)}} \\
\cmidrule(lr){3-5} \cmidrule(lr){6-8}
& & \multicolumn{1}{c}{Reliability} & \multicolumn{1}{c}{Generalization} & \multicolumn{1}{c}{Locality} & \multicolumn{1}{c}{Reliability} & \multicolumn{1}{c}{Generalization} & \multicolumn{1}{c}{Locality} \\
\midrule
\multirow{1}{*}{ZsRE} & ROME & 1.0000 & 0.9112 & 0.9661 & 1.0000 & 0.9883 & 0.9600  \\
& ROME-EAC & 1.0000 & 0.8923 & 0.9560 & 0.9933 & 0.9733 & 0.9742  \\
\cmidrule(lr){2-8}
& MEMIT & 0.6928 & 0.5208 & 1.0000 & 0.9507 & 0.9333 & 0.9688  \\
& MEMIT-EAC & 0.6614 & 0.4968 & 0.9971 & 0.9503 & 0.9390 & 0.9767  \\
\midrule
\multirow{1}{*}{CounterFact} & ROME & 1.0000 & 0.4200 & 0.9600 & 1.0000 & 0.3600 & 0.7800  \\
& ROME-EAC & 0.9800 & 0.3800 & 0.9600 & 1.0000 & 0.3200 & 0.8800  \\
\cmidrule(lr){2-8}
& MEMIT & 0.9000 & 0.2200 & 1.0000 & 1.0000 & 0.3800 & 0.9500  \\
& MEMIT-EAC & 0.8000 & 0.1800 & 1.0000 & 1.0000 & 0.3200 & 0.9800  \\
\bottomrule
\end{tabular}%
}
\label{s}
\end{table}

\section{Optimization Details}\label{b}
ROME derives a closed-form solution to achieve the optimization:
\begin{equation}
\text{minimize} \ \| \widehat{W}K - V \| \ \text{such that} \ \widehat{W}\mathbf{k}_* = \mathbf{v}_* \ \text{by setting} \ \widehat{W} = W + \Lambda (C^{-1}\mathbf{k}_*)^T.
\end{equation}

Here \( W \) is the original matrix, \( C = KK^T \) is a constant that is pre-cached by estimating the uncentered covariance of \( \mathbf{k} \) from a sample of Wikipedia text, and \( \Lambda = (\mathbf{v}_* - W\mathbf{k}_*) / ( (C^{-1}\mathbf{k}_*)^T \mathbf{k}_*) \) is a vector proportional to the residual error of the new key-value pair on the original memory matrix.

In ROME, \(\mathbf{k}_*\) is derived from the following equation:
\begin{equation}
\mathbf{k}_* = \frac{1}{N} \sum_{j=1}^{N} \mathbf{k}(x_j + s), \quad \text{where} \quad \mathbf{k}(x) = \sigma \left( W_{f_c}^{(l^*)} \gamma \left( a_{[x],i}^{(l^*)} + h_{[x],i}^{(l^*-1)} \right) \right).
\end{equation}

ROME set $\mathbf{v}_* = \arg\min_z \mathcal{L}(z)$, where the objective $\mathcal{L}(z)$ is:
\begin{equation}
\frac{1}{N} \sum_{j=1}^{N} -\log \mathbb{P}_{G(m_{i}^{l^*}:=z))} \left[ o^* \mid x_j + p \right] + D_{KL} \left( \mathbb{P}_{G(m_{i}^{l^*}:=z)} \left[ x \mid p' \right] \parallel \mathbb{P}_{G} \left[ x \mid p' \right] \right).
\end{equation}

\section{Experimental Setup} \label{detail}

\subsection{Editing Methods}\label{EM}

In our experiments, Two popular editing methods including ROME and MEMIT were selected as baselines.

\textbf{ROME} \cite{DBLP:conf/nips/MengBAB22}: it first localized the factual knowledge at a specific layer in the transformer MLP modules, and then updated the knowledge by directly writing new key-value pairs in the MLP module.

\textbf{MEMIT} \cite{DBLP:conf/iclr/MengSABB23}: it extended ROME to edit a large set of facts and updated a set of MLP layers to update knowledge.

The ability of these methods was assessed based on EasyEdit~\cite{DBLP:journals/corr/abs-2308-07269}, an easy-to-use knowledge editing framework which integrates the released codes and hyperparameters from previous methods.

\subsection{Editing Datasets}\label{dat}
In our experiment, two popular model editing datasets \textsc{ZsRE}~\cite{DBLP:conf/conll/LevySCZ17} and \textsc{CounterFact}~\cite{DBLP:conf/nips/MengBAB22} were adopted.

\textbf{\textsc{ZsRE}} is a QA dataset using question rephrasings generated by back-translation as the equivalence neighborhood.
Each input is a question about an entity, and plausible alternative edit labels are sampled from the top-ranked predictions of a BART-base model trained on \textsc{ZsRE}.

\textbf{\textsc{CounterFact}} accounts for counterfacts that start with low scores in comparison to correct facts. It constructs out-of-scope data by substituting the subject entity for a proximate subject entity sharing a predicate. This alteration enables us to differentiate between superficial wording changes and more significant modifications that correspond to a meaningful shift in a fact. 

\subsection{Metrics for Evaluating Editing Performance}\label{Mediting performance}
\paragraph{Reliability} means that given an editing factual knowledge, the edited model should produce the expected predictions. The reliability is measured as the average accuracy on the edit case:
\begin{equation}
\mathbb{E}_{(x'_{ei}, y'_{ei}) \sim \{(x_{ei}, y_{ei})\}} \mathbf{1} \left\{ \arg\max_y f_{\theta_{i}} \left( y \mid x'_{ei} \right) = y'_{ei} \right\}.
\label{rel}
\end{equation}

\paragraph{Generalization} means that edited models should be able to recall the updated knowledge when prompted within the editing scope. The generalization is assessed by the average accuracy of the model on examples uniformly sampled from the equivalence neighborhood:
\begin{equation}
\mathbb{E}_{(x'_{ei}, y'_{ei}) \sim N(x_{ei}, y_{ei})} \mathbf{1} \left\{ \arg\max_y f_{\theta_{i}} \left( y \mid x'_{ei} \right) = y'_{ei} \right\}.
\label{gen}
\end{equation}

\paragraph{Locality} means that the edited model should remain unchanged in response to prompts that are irrelevant or the out-of-scope. The locality is evaluated by the rate at which the edited model's predictions remain unchanged compared to the pre-edit model.
\begin{equation}
\mathbb{E}_{(x'_{ei}, y'_{ei}) \sim O(x_{ei}, y_{ei})} \mathbf{1} \left\{ f_{\theta_{i}} \left( y \mid x'_{ei} \right) = f_{\theta_{i-1}} \left( y \mid x'_{ei} \right) \right\}.
\label{loc}
\end{equation}

\subsection{Downstream Tasks}\label{pro}

Four downstream tasks were selected to measure the general abilities of models before and after editing:
\textbf{Natural language inference (NLI)} on the RTE~\cite{DBLP:conf/mlcw/DaganGM05}, and the results were measured by accuracy of two-way classification.
\textbf{Open-domain QA} on the Natural Question~\cite{DBLP:journals/tacl/KwiatkowskiPRCP19}, and the results were measured by exact match (EM) with the reference answer after minor normalization as in \citet{DBLP:conf/acl/ChenFWB17} and \citet{DBLP:conf/acl/LeeCT19}.
\textbf{Summarization} on the SAMSum~\cite{gliwa-etal-2019-samsum}, and the results were measured by the average of ROUGE-1, ROUGE-2 and ROUGE-L as in \citet{lin-2004-rouge}.
\textbf{Sentiment analysis} on the SST2~\cite{DBLP:conf/emnlp/SocherPWCMNP13}, and the results were measured by accuracy of two-way classification.

The prompts for each task were illustrated in Table~\ref{tab-prompt}.

\begin{table*}[!htb]
% \small
\centering
\begin{tabular}{p{0.95\linewidth}}
\toprule

NLI:\\
\{\texttt{SENTENCE1}\} entails the \{\texttt{SENTENCE2}\}. True or False? answer:\\

\midrule

Open-domain QA:\\
Refer to the passage below and answer the following question. Passage: \{\texttt{DOCUMENT}\} Question: \{\texttt{QUESTION}\}\\

\midrule

Summarization:\\
\{\texttt{DIALOGUE}\} TL;DR:\\

\midrule


Sentiment analysis:\\
For each snippet of text, label the sentiment of the text as positive or negative. The answer should be exact 'positive' or 'negative'. text: \{\texttt{TEXT}\} answer:\\

\bottomrule
\end{tabular}
\caption{The prompts to LLMs for evaluating their zero-shot performance on these general tasks.}
\label{tab-prompt}
\end{table*}

\subsection{Hyper-parameters for Elastic Net}\label{hy}

In our experiment, we set \(\lambda = 5 \times 10^{-7} \), \(\mu = 5 \times 10^{-1} \) for GPT2-XL\cite{radford2019language} and \(\lambda = 5 \times 10^{-7} \), \(\mu = 1 \times 10^{-3} \) for LLaMA-3 (8B)\cite{llama3}.

\begin{figure*}[!hbt]
  \centering
  \includegraphics[width=0.5\textwidth]{figures/legend_edit.pdf}
  \vspace{-4mm}
\end{figure*}

\begin{figure*}[!hbt]
  \centering
  \subfigure{
  \includegraphics[width=0.23\textwidth]{figures/ROME-GPT2XL-CF-edit.pdf}}
  \subfigure{
  \includegraphics[width=0.23\textwidth]{figures/ROME-LLaMA3-8B-CF-edit.pdf}}
  \subfigure{
  \includegraphics[width=0.23\textwidth]{figures/MEMIT-GPT2XL-CF-edit.pdf}}
  \subfigure{
  \includegraphics[width=0.23\textwidth]{figures/MEMIT-LLaMA3-8B-CF-edit.pdf}}
  \caption{Edited on CounterFact, editing performance of edited models using the ROME~\cite{DBLP:conf/nips/MengBAB22} and MEMIT~\cite{DBLP:conf/iclr/MengSABB23} on GPT2-XL~\cite{radford2019language} and LLaMA-3 (8B)~\cite{llama3}, as the number of edits increases before and after the introduction of EAC.}
  \vspace{-4mm}
  \label{edit-performance-cf}
\end{figure*}

\begin{figure*}[!hbt]
  \centering
  \includegraphics[width=0.75\textwidth]{figures/legend.pdf}
  \vspace{-4mm}
\end{figure*}

\begin{figure*}[!htb]
  \centering
  \subfigure{
  \includegraphics[width=0.23\textwidth]{figures/ROME-GPT2XL-CounterFact.pdf}}
  \subfigure{
  \includegraphics[width=0.23\textwidth]{figures/ROME-LLaMA3-8B-CounterFact.pdf}}
  \subfigure{
  \includegraphics[width=0.23\textwidth]{figures/MEMIT-GPT2XL-CounterFact.pdf}}
  \subfigure{
  \includegraphics[width=0.23\textwidth]{figures/MEMIT-LLaMA3-8B-CounterFact.pdf}}
  \caption{Edited on CounterFact, performance on general tasks using the ROME~\cite{DBLP:conf/nips/MengBAB22} and MEMIT~\cite{DBLP:conf/iclr/MengSABB23} on GPT2-XL~\cite{radford2019language} and LLaMA-3 (8B)~\cite{llama3}, as the number of edits increases before and after the introduction of EAC.}
  \vspace{-4mm}
  \label{task-performance-cf}
\end{figure*}

\section{Experimental Results}\label{app}

\subsection{Results of Editing Performance}\label{cf-performance}
Applying CounterFact as the editing dataset, Figure~\ref{edit-performance-cf} presents the editing performance of the ROME~\cite{DBLP:conf/nips/MengBAB22} and MEMIT~\cite{DBLP:conf/iclr/MengSABB23} methods on GPT2-XL~\cite{radford2019language} and LLaMA-3 (8B)~\cite{llama3}, respectively, as the number of edits increases before and after the introduction of EAC.
The dashed line represents the ROME or MEMIT, while the solid line represents the ROME or MEMIT applying the EAC.


\subsection{Results of General Abilities}\label{cf-ability}
Applying CounterFact as the editing dataset, Figure~\ref{task-performance-cf} presents the performance on general tasks of edited models using the ROME~\cite{DBLP:conf/nips/MengBAB22} and MEMIT~\cite{DBLP:conf/iclr/MengSABB23} methods on GPT2-XL~\cite{radford2019language} and LLaMA-3 (8B)~\cite{llama3}, respectively, as the number of edits increases before and after the introduction of EAC. 
The dashed line represents the ROME or MEMIT, while the solid line represents the ROME or MEMIT applying the EAC.

\subsection{Results of Larger Model}\label{13 B}
To better demonstrate the scalability and efficiency of our approach, we conducted experiments using the LLaMA-2 (13B)~\cite{DBLP:journals/corr/abs-2307-09288}.
Figure~\ref{13B-edit} presents the editing performance of the ROME~\cite{DBLP:conf/nips/MengBAB22} and MEMIT~\cite{DBLP:conf/iclr/MengSABB23} methods on LLaMA-2 (13B)
~\cite{DBLP:journals/corr/abs-2307-09288}, as the number of edits increases before and after the introduction of EAC.
Figure~\ref{13B} presents the performance on general tasks of edited models using the ROME and MEMIT methods on LLaMA-2 (13B), as the number of edits increases before and after the introduction of EAC.
The dashed line represents the ROME or MEMIT, while the solid line represents the ROME or MEMIT applying the EAC.

\begin{figure*}[!hbt]
  \centering
  \includegraphics[width=0.5\textwidth]{figures/legend_edit.pdf}
  \vspace{-4mm}
\end{figure*}

\begin{figure*}[!hbt]
  \centering
  \subfigure{
  \includegraphics[width=0.23\textwidth]{figures/ROME-LLaMA2-13B-ZsRE-edit.pdf}}
  \subfigure{
  \includegraphics[width=0.23\textwidth]{figures/MEMIT-LLaMA2-13B-ZsRE-edit.pdf}}
  \subfigure{
  \includegraphics[width=0.23\textwidth]{figures/ROME-LLaMA2-13B-CF-edit.pdf}}
  \subfigure{
  \includegraphics[width=0.23\textwidth]{figures/MEMIT-LLaMA2-13B-CF-edit.pdf}}
  \caption{Editing performance of edited models using the ROME~\cite{DBLP:conf/nips/MengBAB22} and MEMIT~\cite{DBLP:conf/iclr/MengSABB23} on LLaMA-2 (13B)~\cite{DBLP:journals/corr/abs-2307-09288}, as the number of edits increases before and after the introduction of EAC.}
  \vspace{-4mm}
  \label{13B-edit}
\end{figure*}

\begin{figure*}[!hbt]
  \centering
  \includegraphics[width=0.75\textwidth]{figures/legend.pdf}
  \vspace{-4mm}
\end{figure*}

\begin{figure*}[!htb]
  \centering
  \subfigure{
  \includegraphics[width=0.23\textwidth]{figures/ROME-LLaMA2-13B-ZsRE.pdf}}
  \subfigure{
  \includegraphics[width=0.23\textwidth]{figures/MEMIT-LLaMA2-13B-ZsRE.pdf}}
  \subfigure{
  \includegraphics[width=0.23\textwidth]{figures/ROME-LLaMA2-13B-CounterFact.pdf}}
  \subfigure{
  \includegraphics[width=0.23\textwidth]{figures/MEMIT-LLaMA2-13B-CounterFact.pdf}}
  \caption{Performance on general tasks using the ROME~\cite{DBLP:conf/nips/MengBAB22} and MEMIT~\cite{DBLP:conf/iclr/MengSABB23} on LLaMA-2 (13B)~\cite{DBLP:journals/corr/abs-2307-09288}, as the number of edits increases before and after the introduction of EAC.}
  \vspace{-4mm}
  \label{13B}
\end{figure*}

\section{Analysis of Elastic Net}
\label{FT}
It is worth noting that the elastic net introduced in EAC can be applied to methods beyond ROME and MEMIT, such as FT~\cite{DBLP:conf/emnlp/CaoAT21}, to preserve the general abilities of the model.
Unlike the previously mentioned fine-tuning, FT is a model editing approach. It utilized the gradient to gather information about the knowledge to be updated and applied this information directly to the model parameters for updates.
Similar to the approaches of ROME and MEMIT, which involve locating parameters and modifying them, the FT method utilizes gradient information to directly update the model parameters for editing. Therefore, we incorporate an elastic net during the training process to constrain the deviation of the edited matrix.
Figure~\ref{ft} shows the sequential editing performance of FT on GPT2-XL and LLaMA-3 (8B) before and after the introduction of elastic net.
The dashed line represents the FT, while the solid line represents the FT applying the elastic net.
The experimental results indicate that when using the FT method to edit the model, the direct use of gradient information to modify the parameters destroys the general ability of the model. By constraining the deviation of the edited matrix, the incorporation of the elastic net effectively preserves the general abilities of the model.

\begin{figure*}[t]
  \centering
  \subfigure{
  \includegraphics[width=0.43\textwidth]{figures/legend_FT.pdf}}
\end{figure*}

\begin{figure*}[t]%[!ht]
  \centering
  \subfigure{
  \includegraphics[width=0.22\textwidth]{figures/FT-GPT2XL-ZsRE.pdf}}
  \subfigure{
  \includegraphics[width=0.22\textwidth]{figures/FT-GPT2XL-CounterFact.pdf}}
  \vspace{-2mm}
  \caption{Edited on the ZsRE or CounterFact datasets, the sequential editing performance of FT~\cite{DBLP:conf/emnlp/CaoAT21} and FT with elastic net on GPT2-XL before and after the introduction of elastic net.}
  \vspace{-2mm}
  \label{ft}
\end{figure*}


\end{document}