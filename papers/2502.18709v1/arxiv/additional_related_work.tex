\section{Additional Related Work}\label{app:additional_related_work}
We discuss additional related work that could not be included above.

\paragraph{Structured prediction}
Before the introduction of the Fenchel--Young loss framework, \citet{Niculae18sparse} proposed SparseMAP, which used the squared $\ell_2$-norm regularization.
The Fenchel--Young loss, described in \cref{subsec:fenchel-young}, is built upon the idea of SparseMAP. 
The Structure Encoding Loss Function (SELF) was introduced by \citet{ciliberto16consistent,ciliberto20general} to analyze the relationship between surrogate and target losses, a concept known as Fisher consistency.
For more extensive literature, we refer the reader to \citet[Appendix A]{pmlr-v247-sakaue24a}.

\paragraph{Online classification with full and bandit feedback}
In the full information setup, PERCEPTRON is one of the most representative algorithms for binary classification \citep{Rosenblatt1958-sh}, and the multiclass setting has also been extensively studied \citep{,crammer2003ultraconservative,Fink_2006}.
Online logistic regression is another relevant research stream, with \citet{foster18logistic} being a particularly representative study. 
The study of the bandit setup was initiated by \citet{Kakade2008EfficientBA}, and it has since been extensively explored in subsequent research \citep{hazan11newtron,beygelzimer17efficient,foster18logistic}. However, to the best of our knowledge, no prior work has addressed general structured prediction under bandit feedback. A most related study is the work by \citet{gentile14multilabel}, which investigated online multilabel classification and ranking. 
However, their setting assumes access to feedback of the form $\set{\ind[\bmy_{t,i} \neq \hat{\bmy}_{t,i}]}_{i}$, which is more informative than bandit feedback and differs from our setup.
\Citet{NEURIPS2020_Hoeven} explicitly introduced the surrogate regret in the context of online multiclass classification. This study has been extended to the setting where observations are determined by a directed graph \Citep{NEURIPS2021_Hoeven} and to structured prediction scenarios \citep{pmlr-v247-sakaue24a}. For a more extensive overview of the literature on online classification, we refer the reader to \Citet{NEURIPS2020_Hoeven}.


\paragraph{Delayed feedback}
The study of delayed feedback began with \citet{Weinberger_2002_delay}. 
Since then, it has been extensively explored in various online learning settings, primarily in the full information setup of online convex optimization \citep{Mesterharm05online,joulani13online,joulani16delay,pmlr-v139-flaspohler21a}. 
Algorithms for delayed bandit feedback have been studied mainly in the context of multi-armed bandits and their variants \citep{cesabianchi16delay,zimmert20optimal,ito20delay,masoudian22best,hoeven23unified}. In the context of online classification, research considering delay is scarce; the only work is \citet{manwani2022delaytronefficientlearningmulticlass} to our knowledge.


