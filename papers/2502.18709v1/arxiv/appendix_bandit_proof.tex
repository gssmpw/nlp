\section{Omitted Details of \cref{sec:bandit}}
\label{app:proof bandit}
This section provides the omitted details of \cref{sec:bandit}.

\subsection{Concentration inequality}
To prove high probability regret bounds, we use the following concentration inequality.
\begin{lemma}[{Bernstein's inequality, e.g., \citealt[Lemma A.8]{Cesa-Bianchi_Lugosi_2006}}]
    \label{lem:Bernstein}
    Let $Z_1,\hdots,Z_T$ be a martingale difference sequence and $\delta \in (0,1)$.
    If there exist $a$ and $v$ which satisfy $|Z_t|\leq a$ for any $t \in \brk{T}$ and $\sumt{\expect{Z_t^2}}\leq v$ , then with probability at least $1-\delta$, it holds that
    \[
        \sumt{Z_t}\leq\sqrt{2v\log\frac{1}{\delta}}+\frac{\sqrt{2}}{3}a\log\frac{1}{\delta}.
    \]
\end{lemma}


\subsection{Proof of \cref{thm:bandit_high_prob}}\label{app:proof_bandit_high_prob}
Here, we provide the proof of \cref{thm:bandit_high_prob}.
Hereafter, we let $S_{\max} = \max_{\W \in \ww} S_t(\W)$ and $\hat{S}_t(\W) = v_t S_t(\W) = \frac{\ind\brk{\yt = \yht}}{p_t(\yht)} S_t(\W)$.
The following theorem is the formal version of \cref{thm:bandit_high_prob}:
\begin{theorem}[Formal version of \cref{thm:bandit_high_prob}]\label{thm:bandit_high_prob_formal}
Consider the bandit and non-delayed setup.
Let 
\begin{equation}
    \mathcal{C}
    =
    \prn*{
        \frac{3}{2 (a + \xi - 1)}  
        +
        1
    }
    K \Smax \log(2/\delta) 
    +
    \frac{B^2 K b}{2 (1 - \xi)}
    .
    \nonumber
\end{equation}

Then, for any $T \geq \mathcal{C}$ and $\delta \in (0,1/2)$, with probability at least $1-\delta$, the algorithm in \cref{subsec:Bandit_Structured_Prediction_with_General_Losses} with $q = \sqrt{\mathcal{C} / T}$ achieves
\begin{equation}
    \mathcal{R}_T
    \leq
    2
    \sqrt{
        \mathcal{C}
        T
    }
    +
    \sqrt{2 \log (2/ \delta)} \prn{\mathcal{C} T}^{1/4}
    +
    \prn*{ \frac{1-a}{2 (a + \xi - 1)} + 2 } \log (2/\delta).
    \nonumber
\end{equation}
\end{theorem}


Before proving this theorem, we provide the following lemma:
\begin{lemma}\label{lem:hp_pre}
It holds that 
\begin{equation}
    \sum_{t=1}^T \prn*{ \E_t\brk*{L_t(\yht)} - \hat{S}_t(\U) }
    \leq
    \sum_{t=1}^T \prn*{ (1-a) S_t(\W_t) - \hat{S}_t(\W_t)  } + q T
    +
    \sqrt{2} B \sqrt{\frac{b}{q} \sum_{t=1}^T v_t S_t(\W_t) }
    .
    \nonumber
\end{equation}
\end{lemma}
\begin{proof}
We have 
\begin{equation}
    \sum_{t=1}^T \prn*{ \E_t\brk*{L_t(\yht)} - \hat{S}_t(\U) }
    =
    \sum_{t=1}^T \prn*{ \E_t\brk*{L_t(\yht)} - \hat{S}_t(\W_t)  }
    +
    \sum_{t=1}^T \prn*{ \hat{S}_t(\W_t) - \hat{S}_t(\U) }.
    \nonumber
\end{equation}
From \cref{asp:bandit_a}, the first term is bounded as 
\begin{align}
    \sum_{t=1}^T \prn*{ \E_t\brk*{L_t(\yht)} - \hat{S}_t(\W_t)  }
    \leq
    \sum_{t=1}^T \prn*{ (1-a) S_t(\W_t) - \hat{S}_t(\W_t)  } + q T,
    \nonumber
\end{align}
and 
the second term is bounded as 
\begin{align}
    \sum_{t=1}^T \prn*{ \hat{S}_t(\W_t) - \hat{S}_t(\U) }
    &\leq
    \sqrt{2} B \sqrt{\sum_{t=1}^T \nrm{\tilde{\G}_t}_{\F}^2 }
    =
    \sqrt{2} B \sqrt{\sum_{t=1}^T v_t^2 \nrm{\G_t}_{\F}^2 }
    \nonumber \\
    &\leq
    \sqrt{2} B \sqrt{b \sum_{t=1}^T v_t^2 S_t(\W_t) }
    \leq
    \sqrt{2} B \sqrt{\frac{b K}{q} \sum_{t=1}^T v_t S_t(\W_t) },
    \nonumber
\end{align}
where we used \cref{lem:ogd} and $v_t \leq K / q$.
Combining the above three, we obtain
\begin{equation}
    \sum_{t=1}^T \prn*{ \E_t\brk*{L_t(\yht)} - \hat{S}_t(\U) }
    \leq
    \sum_{t=1}^T \prn*{ (1-a) S_t(\W_t) - \hat{S}_t(\W_t)  } + q T
    +
    \sqrt{2} B \sqrt{\frac{b K}{q} \sum_{t=1}^T v_t S_t(\W_t) }
    ,
    \nonumber
\end{equation}
which completes the proof.
\end{proof}


\begin{proof}[Proof of \cref{thm:bandit_high_prob_formal}]
The surrogate regret can be decomposed as
\begin{equation}\label{eq:reg_dec_highp}
    \mathcal{R}_T 
    =
    \sum_{t=1}^T \prn*{ L_t(\yht) - \E_t\brk*{ L_t(\yht)} }
    +
    \sum_{t=1}^T \prn*{ \E_t\brk*{ L_t(\yht)} - S_t(\U) }
    .
\end{equation}
We first upper bound the first term in \eqref{eq:reg_dec_highp}.
Let $Z_t = L_t(\yht) - \E_t\brk*{ L_t(\yht)}$ for simplicity.
Then, we have $Z_t \leq 1$, $\E_t\brk*{Z_t} = 0$, and
$\E_t\brk*{Z_t^2} 
\leq 
\E_t\brk*{ \prn{L_t(\yht)}^2 }
\leq 
(1-a) S_t(\W_t) + q.
$
Hence, from Bernstein's inequality in \cref{lem:Bernstein}, for any $\delta' \in (0,1)$, at least $1 - \delta'$ we have 
\begin{equation}\label{eq:conc_zt}
    \sum_{t=1}^T Z_t
    \leq 
    \sqrt{2 \log (1/\delta') \sum_{t=1}^T \prn*{(1-a) S_t(\W_t) + q} }
    +
    \frac{\sqrt{2}}{3} \log (1/\delta')
    .
\end{equation}
We next consider the second term in \eqref{eq:reg_dec_highp}.
Define $r_t = S_t(\U) - \xi S_t(\W_t)$ for some $\xi \in (0, 1)$, which will be determined later,
and let $v_t = \ind[ \yt = \yht ] / p_t(\yht) \leq K/q$ for simplicity.
Then, we have $\E_t\brk{v_t r_t - r_t} = 0$, $\abs{v_t r_t - r_t} \leq K S_{\max} / q$, and
\begin{equation}
    \E_t\brk{ (v_t r_t - r_t)^2}
    \leq
    \E_t\brk{(v_t r_t)^2}
    \leq 
    \frac{K \Smax}{q} \abs{r_t}
    \leq 
    \frac{K \Smax}{q} \prn*{S_t(\U) + S_t(\W_t)}
    .
    \nonumber
\end{equation}
Hence from Bernstein's inequality in \cref{lem:Bernstein}, for any $\delta'' \in (0,1)$, with probability at least $1 - \delta''$ we have 
\begin{equation}\label{eq:conc_vr}
    \sum_{t=1}^T \prn{v_t r_t - r_t} 
    \leq 
    \sqrt{3 \log (1/\delta'') \sum_{t=1}^T \frac{K \Smax}{q} \prn{S_t(\U) + S_t(\W_t)} }
    +
    \frac{\sqrt{2} K \Smax}{3q} \log(1/\delta'')
    .
\end{equation}


\textbf{When $\sum_{t=1}^T S_t(\U) \leq \sum_{t=1}^T S_t(\W_t)$.}
We first consider the case of $\sum_{t=1}^T S_t(\U) \leq \sum_{t=1}^T S_t(\W_t)$.
From \cref{lem:hp_pre}, we have
\begin{align}
    &\sum_{t=1}^T \E_t\brk*{L_t(\yht)} - q T
    \leq
    \sum_{t=1}^T v_t S_t(\U) 
    +
    \sum_{t=1}^T \prn*{ (1-a) S_t(\W_t) - v_t S_t(\W_t)  } 
    +
    \sqrt{2} B \sqrt{\frac{b K}{q} \sum_{t=1}^T v_t S_t(\W_t) }
    \nonumber \\
    &=
    \sum_{t=1}^T v_t \underbrace{\prn*{ S_t(\U) - \xi S_t(\W_t)  } }_{= r_t}
    -
    (1 - \xi) \sum_{t=1}^T v_t S_t(\W_t)
    +
    (1-a) \sum_{t=1}^T S_t(\W_t)
    +
    \sqrt{2} B \sqrt{\frac{b K}{q} \sum_{t=1}^T v_t S_t(\W_t) }
    \nonumber \\
    &\leq
    \sum_{t=1}^T v_t r_t
    +
    (1-a) \sum_{t=1}^T S_t(\W_t)
    +
    \frac{B^2 K b}{2 q (1 - \xi)},
    \nonumber
\end{align}
where the last inequality follows from 
$c_1\sqrt{x}-c_2x\leq{c_1^2}/\prn{4c_2}$ for $x \geq 0$, $c_1 \geq 0$, and $c_2 > 0$.
From the concentration result provided in \eqref{eq:conc_vr}, this is further bounded as
\begin{align}
    \sum_{t=1}^T \E_t\brk*{L_t(\yht)} - q T
    &\leq
    \sum_{t=1}^T (S_t(\U) - \xi S_t(\W_t))
    +
    \sqrt{3 \log (1/\delta'') \sum_{t=1}^T \frac{K \Smax}{q} \prn{S_t(\U) + S_t(\W_t)} }
    \nonumber \\
    &\qquad
    +
    \frac{\sqrt{2} K \Smax}{3q} \log(1/\delta'') 
    +
    (1-a) \sum_{t=1}^T S_t(\W_t)
    +
    \frac{B^2 K b}{2 q (1 - \xi)}
    ,
    \nonumber
\end{align}
where we recall that $r_t = S_t(\U) - \xi S_t(\W_t)$.
Rearranging the last inequality and using $\sum_{t=1}^T S_t(\U) \leq \sum_{t=1}^T S_t(\W_t)$ give
\begin{align}
    \sum_{t=1}^T \prn*{ \E_t\brk*{L_t(\yht)} - S_t(\U) } 
    &\leq
    q T 
    +
    \sqrt{6 \log (1/\delta'') \sum_{t=1}^T \frac{K \Smax}{q} S_t(\W_t) }
    +
    \frac{\sqrt{2} K \Smax}{3 q} \log(1/\delta'') 
    \nonumber \\
    &\qquad
    +
    (1 - a - \xi) \sum_{t=1}^T S_t(\W_t)
    +
    \frac{B^2 K b}{2 q (1 - \xi)}
    .
    \nonumber
\end{align}
In what follows, we let $\delta' = \delta'' = \delta / 2$ and $\xi = \frac{\prn{4 + c} \gamma}{\lambda \nu}$ for a sufficiently small constant $c > 0$, which satisfies $a + \xi > 1$.
Then, plugging \eqref{eq:conc_zt} and the last inequality in \eqref{eq:reg_dec_highp}, with probability at least $1 - \delta$, we obtain
\begin{align}
    \mathcal{R}_T
    &\leq
    \sqrt{2 \log (2/\delta) \sum_{t=1}^T \prn*{(1-a) S_t(\W_t) + q} }
    +
    \frac{\sqrt{2}}{3} \log (2/\delta)
    +
    q T 
    +
    \sqrt{6 \log (2/\delta) \sum_{t=1}^T \frac{K \Smax}{q} S_t(\W_t) }
    \nonumber \\
    &\qquad
    +
    \frac{\sqrt{2} K \Smax}{3 q} \log(2/\delta) 
    +
    (1 - a - \xi) \sum_{t=1}^T S_t(\W_t)
    +
    \frac{B^2 K b}{2 q (1 - \xi)}
    \nonumber \\
    &\leq
    \frac{1}{2 (a + \xi - 1)}
    \prn*{(1-a) + \frac{3 K \Smax}{q}} \log (2/\delta)
    +
    \sqrt{2 q T \log (2 / \delta)} 
    +
    \frac{\sqrt{2}}{3} \log (2/\delta)
    +
    q T 
    \nonumber \\
    &\qquad
    +
    \frac{\sqrt{2} K \Smax}{3 q} \log(2/\delta) 
    +
    \frac{B^2 b}{2 q (1 - \xi)}
    \nonumber \\
    &\leq
    \frac{1}{q}
    \prn*{
        \frac{3 K \Smax \log (2/\delta)}{2 (a + \xi - 1)}  
        +
        K \Smax \log(2/\delta) 
        +
        \frac{B^2 K b}{2 (1 - \xi)}
    }
    +
    q T 
    +
    \sqrt{2 q T \log (2 / \delta)} 
    \nonumber \\
    &\qquad
    +
    \frac{1}{2 (a + \xi - 1)} (1-a) \log (2/\delta)
    +
    \frac{\sqrt{2}}{3} \log (2/\delta)
    \nonumber \\
    &=
    \frac{\mathcal{C}}{q}
    +
    q T 
    +
    \sqrt{2 q T \log (2 / \delta)} 
    +
    \frac{1}{2 (a + \xi - 1)} (1-a) \log (2/\delta)
    +
    \frac{\sqrt{2}}{3} \log (2/\delta).
    \nonumber
\end{align}
Using the definition of $q = \sqrt{\mathcal{C} / T}$ with the last inequality,
we obtain
\begin{equation}
    \mathcal{R}_T
    \leq
    2
    \sqrt{
        \mathcal{C}
        T
    }
    +
    \prn{\mathcal{C} T}^{1/4} \sqrt{\log (2/ \delta)}
    +
    \prn*{ \frac{1-a}{2 (a + \xi - 1)} + \frac{\sqrt{2}}{3} } \log (2/\delta).
    \nonumber
\end{equation}

\textbf{When $\sum_{t=1}^T S_t(\U) > \sum_{t=1}^T S_t(\W_t)$.}
We next consider the case of $\sum_{t=1}^T S_t(\U) > \sum_{t=1}^T S_t(\W_t)$.
We have
\begin{align}
    \mathcal{R}_T 
    &=
    \sum_{t=1}^T \prn*{ L_t(\yht) - \E_t\brk*{ L_t(\yht)} }
    +
    \sum_{t=1}^T \prn*{ \E_t\brk*{ L_t(\yht)} - S_t(\U) }
    \nonumber \\
    &\leq
    \sqrt{2 \log (1/\delta') \sum_{t=1}^T \prn*{(1-a) S_t(\W_t) + q} }
    +
    \frac{\sqrt{2}}{3} \log (1/\delta')
    +
    \sum_{t=1}^T \prn*{ \E_t\brk*{ L_t(\yht)} - S_t(\W_t) }
    \nonumber \\
    &\leq
    \sqrt{2 \log (1/\delta') \sum_{t=1}^T \prn*{(1-a) S_t(\W_t) + q} }
    +
    \frac{\sqrt{2}}{3} \log (1/\delta')
    +
    \sum_{t=1}^T \prn*{ - a S_t(\W_t) + q }
    \nonumber \\
    &\leq
    \frac{(1 - a) \log (1/ \delta')}{ 2 a }
    +
    \sqrt{2 q T \log (1/\delta') }
    +
    \frac{\sqrt{2}}{3} \log (1/\delta')
    +
    q T
    ,
    \nonumber
\end{align}
where the first inequality follows from \eqref{eq:conc_zt} and $\sum_{t=1}^T S_t(\U) > \sum_{t=1}^T S_t(\W_t)$,
and the second inequality follows from \cref{asp:bandit_a},
the last inequality follows from $c_1\sqrt{x}-c_2x\leq{c_1^2}/\prn{4c_2}$ for $x \geq 0$, $c_1 \geq 0$, and $c_2 > 0$.
Substituting $q = \sqrt{\mathcal{C} / 2}$ and $\delta' = \delta/2$ and  in the last inequality, we obtain
\begin{equation}
    \mathcal{R}_T 
    \leq
    \frac{(1 - a) \log (2/ \delta)}{ 2 a }
    +
    \sqrt{2 \log (2/\delta) } \prn{ \mathcal{C} T }^{1/4}
    +
    \frac{\sqrt{2}}{3} \log (2/\delta)
    +
    \sqrt{\mathcal{C} T}
    .
    \nonumber
\end{equation}
This completes the proof.    
\end{proof}







\subsection{Proof of \cref{thm:bandit_regret_pseudo_estimator}}
\label{app:sub_bandit_regret_pseudo_estimator}


Here, we provide the proof of \cref{thm:bandit_regret_pseudo_estimator}.
We recall that $\pt=\expect{\yht\yht^\top}$.
We then estimate $\yt$ by $\ytilde=\inverse{\V}\Pplus_t\yht\inpr{\yht,\V\yt}$
and $\G_t$ by $\gtil\coloneqq(\yho(\tht)-\ytilde)\xt^\top$ under \cref{asp:self}.
This $\gtil$ satisfies 
$
    \expect{\gtil}=\G_t. 
$
To prove \cref{thm:bandit_regret_pseudo_estimator}, we will upper bound $\expect{\nrm{\gtil}_\F^2}$.
To do so, we begin by proving the following lemma:
\begin{lemma}\label{lem:pseudo_inverse_order}
    Let $\bm{A}$ and $\bm{B}$ positive semi-definite matrices with $\image(\bm{A}) = \image(\bm{B})$ with $\bm{A} \succeq \bm{B}$.
    Then, it holds that $\bm{A}^+ \preceq \bm{B}^+$.
\end{lemma}
\begin{proof}
Since $\image(\bm{A}) = \image(\bm{B})$, there exists an orthogonal matrix $\bm{R}$, a diagonal matrix $\bm{\Lambda}$, and an invertible matrix $\bm{B}'$ that has same dimensions as $\bm{\Lambda}$ such that 
\begin{equation}
    \bm{A}
    =
    \bm{R}
    \begin{pmatrix}
        O & O \\
        O & \bm{\Lambda} 
    \end{pmatrix}
    \bm{R}^\top
    \quad 
    \mbox{and}
    \quad
    \bm{B}
    =
    \bm{R}
    \begin{pmatrix}
        O & O \\
        O & \bm{B}'
    \end{pmatrix}
    \bm{R}^\top
    .
    \nonumber
\end{equation}
Then, 
\begin{equation}\label{eq:Aplus_Bplus}
    \bm{A}^+
    =
    \bm{R}
    \begin{pmatrix}
        O & O \\
        O & \bm{\Lambda}^{-1}
    \end{pmatrix}
    \bm{R}^\top
    \quad 
    \mbox{and}
    \quad
    \bm{B}^+
    =
    \bm{R}
    \begin{pmatrix}
        O & O \\
        O & {\bm{B}'}^{-1}
    \end{pmatrix}
    \bm{R}^\top
    .
\end{equation}
From $\bm{A} \succeq \bm{B}$,
we have $\bm{\Lambda} \succeq \bm{B}'$, which implies $\bm{\Lambda}^{-1} \preceq {\bm{B}'}^{-1}$.
From this and \eqref{eq:Aplus_Bplus}, we have $\bm{A}^+ \preceq \bm{B}^+$, as desired.
\end{proof}

Using this lemma we prove a property of $\pt$ and an upper bound of $\expect{\tr\prn*{\yht\yht^\top}}$.
In what follows, we use $\lambda_\min(\bm{A})$ to denote the minimum eigenvalue of a matrix $\bm{A}$.

\begin{lemma}
    \label{lem:bound of trace}
    Suppose that $\tr \prn*{ \V^{-1} \bm{Q} \prn{\V^{-1}}^\top } \leq \omega$ for $\bm{Q} = \E_{\bm{y} \sim \mu} \brk{ \bm{y} \bm{y}^\top }$, where we recall $\mu$ is the uniform distribution over $\yy$.
    Then, we have
    \[
    \expect{\tr(\ytt\ytt^\top)}\leq \frac{\omega}{q}.
    \]
\end{lemma}
\begin{proof}    
    By the linearity of expectation and the trace property, we have
    \begin{align*}
        \expect{\tr(\ytilde\ytilde^\top)}&\leq \tr\prn*{\inverse{\V}\Pplus_t\expect{\yht\yht^\top}\Pplus_t\prn*{\inverse{\V}}^\top}
        = \tr\prn*{\inverse{\V}\Pplus_t \bm{P}_t \Pplus_t \prn*{\inverse{\V}}^\top}\\
        &= 
        \tr\prn*{\inverse{\V}\Pplus_t\prn*{\inverse{\V}}^\top},
    \end{align*}
    where the first inequality follows from $\inpr{\yht,\V\yt} = L_t(\yht) - \inpr{\yht,\bm{b}} - c(\yt) \leq L_t(\yht) \leq 1$ since $\bm{b} \geq 0$ and $c(\cdot)$ is non-negative
    and
    the last equality follows from $\Pplus_t \bm{P}_t \Pplus_t = \Pplus_t$.
    Hence,
    \begin{align}
        \tr\prn*{\inverse{\V}\Pplus_t\prn*{\inverse{\V}}^\top}
        &=
        \sum_{i=1}^d
        \bm{\ee}_i^\top \inverse{\V}\Pplus_t\prn*{\inverse{\V}}^\top \bm{\ee}_i
        \leq
        \sum_{i=1}^d
        \bm{\ee}_i^\top \inverse{\V} \prn*{q \bm{Q}}^{+} \prn*{\inverse{\V}}^\top \bm{\ee}_i
        \nonumber \\
        &\leq
        \tr\prn*{\prn*{\inverse{\V}}^\top \inverse{\V} (q \bm{Q})^+ }
        =
        \frac{1}{q}
        \tr \prn*{ \inverse{\V} \bm{Q}^+ \prn{\inverse{\V}}^\top } 
        \leq
        \frac{\omega}{q},
        \nonumber
    \end{align}
    where in the first inequality we used \cref{lem:pseudo_inverse_order} and in the last inequality we used the assumption that $\tr \prn*{ \V^{-1} \bm{Q}^+ \prn{\V^{-1}}^\top } \leq \omega$.
    This completes the proof.
\end{proof}

Now, we are ready to upper bound $\expect{\nrm{\gtil}_\F^2}$.
\begin{lemma}
    \label{thm:evaluation of Gtilde}
    Under the same assumption as \cref{lem:bound of trace}, it holds that 
    \[
        \expect{\nrm{\gtil}_\F^2}\leq2b\sw+ \frac{2 \dix^2 \omega}{q}.
    \]
\end{lemma}
\begin{proof}
    We have 
    \begin{align*}
        \nrm{\gtil}_\F^2&=\nrm{\prn*{\yho(\tht)-\ytilde}\xt^\top}_{\mathrm{F}}^2\leq2\nrm{(\yho(\tht)-\yt)\xt^\top}_\F^2+2\nrm{(\yt-\ytilde)\xt^\top}_\F^2\\
        &\leq 2\nrm{\G_t}_\F^2+2\dix ^2\nrm{\yt-\ytilde}_2^2,
    \end{align*}
    where we recall $\dix =\diam(\xx)$.
    From this inequality, 
    \begin{align}
        \expect{\nrm{\gtil}_\F^2}&\leq2\nrm{\G_t}_{\mathrm{F}}^2+2\dix ^2\expect{\nrm{\yt-\ytilde}_2^2}
        \leq2b\sw+2\dix ^2\prn*{\nrm{\yt}_2^2-2\yt^\top\expect{\ytilde}+\expect{\nrm{\ytilde}_2^2}} \nonumber \\
        &=2b\sw+2\dix ^2\prn*{\nrm{\yt}_2^2-2\nrm{\yt}_2^2 + \expect{\nrm{\ytilde}_2^2}} \nonumber \\ 
        &\leq
        2b\sw
        +2\dix ^2\expect{\tr(\ytilde\ytilde^\top)}
        \leq
        2b\sw
        + \frac{2\dix^2 \omega}{q},
        \nonumber
    \end{align} 
    where in the second inequality we used $\nrm{\G_t}_\F^2 \leq b \sw$, in the equality we used $\expect{\ytilde}=\yt$,
    and in the last inequality we used \cref{lem:bound of trace}.
\end{proof}



Finally, we are ready to prove \cref{thm:bandit_regret_pseudo_estimator}.
\begin{proof}[Proof of \cref{thm:bandit_regret_pseudo_estimator}]
    From \cref{asp:bandit_a}, we have 
    \begin{align*}\label{eq:inverse_reg_expect}
        \E\brk{\reg}
        &\leq\E\brk*{\sumt{(\sw-\su)}}-a\E\brk*{\sumt{\sw}}+qT
        \nonumber \\
        &\leq\E\brk*{\sumt{\inpr*{\G_t,\wt-\U}}}-a\E\brk*{\sumt{\sw}}+qT.
    \end{align*}
    From \cref{thm:evaluation of Gtilde} and the unbiasedness of $\gtil$, 
    the first term in the last inequality is further bounded as
    \begin{align*}
        \E\brk*{\sumt{\inpr*{\G_t,\wt-\U}}}  
        &=\E\brk*{\sumt{\inpr*{\gtil,\wt-\U}}}
        \leq\sqrt{2}B\sqrt{\E\brk*{\sumt{\nrm{\gtil}_{\mathrm{F}}^2}}}
        \nonumber \\
        &\leq
        2 B \sqrt{b\E\brk*{\sumt{\sw}}}
        +
        2 B \dix \sqrt{ \omega / q},
    \end{align*}
    where 
    the first inequality follows from \cref{lem:ogd} and the last inequality follows from \cref{thm:evaluation of Gtilde} and the subadditivity of $x \mapsto \sqrt{x}$ for $x \geq 0$.
    Therefore, by combining  with the last inequality, we have 
    \begin{align}
        \E\brk{\reg}
        &\leq 
        2B \sqrt{b\E\brk*{\sumt{\sw}}} 
        +
        2 B \dix \sqrt{ \omega / q} 
        -a \E\brk*{\sumt{\sw}} + qT \\ 
        &\leq 
        \frac{bB^2}{a}
        +
        2 B \dix \sqrt{ \omega / q} 
        +qT ,
        \nonumber
    \end{align}
    where we used $c_1\sqrt{x}-c_2x\leq{c_1^2}/\prn{4c_2}$ for $x\geq0$, $c_1\geq 0$, and $c_2>0$.
    Finally, substituting 
    $q=\prn*{\frac{4 \omega B^2\dix ^2}{ T }}^{1/3}$ in the last inequality gives
    \[
    \E\brk{\reg}
    \leq
    \frac{bB^2}{a}
    +
    2^{5/3} \omega^{1/3} \prn*{ B \dix T}^{2/3}
    ,
    \]
    which is the desired bound.
\end{proof}



\subsection{Proof of \cref{cor:thm_self}}\label{app:SELF_upper_discussion_deferred}
Here, we derive the regret upper bounds provided by the algorithm established  
in \cref{thm:bandit_regret_pseudo_estimator}  
for online multiclass classification, online multilabel classification, and ranking.
Recall that 
we can achieve
\begin{equation}\label{eq:bound_self_app}
\E\brk{\reg}
\leq
\frac{bB^2}{a}
+
O\prn*{ \omega^{1/3} \prn*{ B \dix T}^{2/3} },    
\end{equation}
where we recall that $\omega$ is defined as
$
    \tr \prn*{ \V^{-1} \bm{Q}^+ \prn{\V^{-1}}^\top } \leq \omega
$
for $\bm{Q} = \E_{\bm{y} \sim \mu} \brk{ \bm{y} \bm{y}^\top }$.
Note that when $\spanx(\yy) = \R^d$, then the matrix $\bm{Q}$ is invertible and $\lambda_{\min}(\bm{Q}) > 0$, and thus 
\begin{equation}\label{eq:trace_upper_invertibleQ}
    \tr \prn*{ \V^{-1} \bm{Q}^+ \prn{\V^{-1}}^\top }
    =
    \sum_{i=1}^d
    \bm{\ee}_i^\top \V^{-1} \bm{Q}^+ \prn{\V^{-1}}^\top \bm{\ee}_i
    \leq
    \frac{1}{\lambda_{\min}(\bm{Q})}
    \sum_{i=1}^d
    \nrm{ \prn{\V^{-1}}^\top \bm{\ee}_i }_2^2
    \leq 
    \frac{1}{\lambda_{\min}(\bm{Q})}
    \nrm{ \V^{-1} }_{\F}^2
    .
\end{equation}
In each problem setting, this regret upper bound can be reduced to the following bounds:

\paragraph{Multiclass classification with 0-1 loss}
We first consider multiclass classification with the 0-1 loss.
Since $\V=\bm{1}\bm{1}^\top-\I$, we have $\nrm{\inverse{\V}}_\F^2\leq d$ for $d \geq 2$.  
Recalling that $\mu$ is the uniform distribution over $\yy=\set{\bm{\ee}_1,\hdots,\bm{\ee}_d}$, we have  
$
\E_{\bmy\sim\mu}\brk{(\bmy^\top\bm x)^2}=\frac{1}{d}\sum_{i=1}^{d}x_i^2
$
for any $\bm x\in\R^d$.
Hence, $\lambda_{\min}(\bm{Q})=\min_{\nrm{\bm x}_2=1} \E_{\bmy\sim\mu}\brk*{(\bmy^\top\bm x)^2}=\frac{1}{d}$, where the first equality is from \citet[Lemma 2]{comband}.  
Since $\spanx(\yy) = \R^d$ in this case,
from \eqref{eq:trace_upper_invertibleQ} we can let $\omega = d / \lambda_{\min}(\bm{Q}) = d^2$.
Substituting these into our upper bound in \eqref{eq:bound_self_app}, we obtain  
\begin{equation*}
    \E\brk{\reg}\leq\frac{bB^2}{a}+ O \prn*{ \prn{B \dix  d T}^{2/3} }.
\end{equation*}  



\paragraph{Online multilabel classification with $m$ correct labels   
and the Hamming loss}
We next consider online multilabel classification with the number of correct labels $m$  
and the Hamming loss.  
Since $\V=-\frac{2}{d}\I$, we have  
$\nrm{\inverse{\V}}_\F^2=\frac{d^3}{4}$.  
Let $\yy\subset\set{0,1}^d$ be the set of all vectors  
where exactly $m$ components are $1$, and the remaining components are all $0$.  
By drawing $\bmy\in\yy$ according to the uniform distribution over $\yy$,  
the probability that a given component of $\bmy$ is $1$ is  
$\binom{m-1}{d-1}/\binom{m}{d}=\frac{m}{d}$.  
Hence, for any $\bm x\in\R^d$ with $\nrm{\bm{x}}_2=1$,  
we have
\[
\E_{\bmy\sim\mu}\brk*{(\bmy^\top\bm x)^2}
=\frac{m}{d}\sum_{i=1}^dx_i^2  
+\frac{m^2}{d^2}\sum_{i\neq j}x_ix_j  
=\prn*{\frac{m}{d}\sum_{i=1}^dx_i}^2+\frac{m(d-m)}{d^2}\nrm{\bm x}_2^2  
\geq 
\frac{m(d-m)}{d^2}.
\]
Hence, we have $\lambda_{\min}(\bm{Q}) = \E_{\bmy\sim\mu}\brk*{(\bmy^\top\bm x)^2} \geq\frac{m(d-m)}{d^2}$, where the equality is from \citet[Lemma 2]{comband}.
Since $\spanx(\yy) = \R^d$,
from \eqref{eq:trace_upper_invertibleQ} we can choose $\omega = d^3 / \prn{4 \lambda_{\min}(\bm{Q})} = \frac{d^5}{4 m (d-m)} $.
Therefore, our regret upper bound in \eqref{eq:bound_self_app} is reduced to
\begin{equation*}
    \E\brk{\reg}
    \leq
    \frac{bB^2}{a}
    +
    O \prn*{ \prn*{\frac{B^2\dix ^2d^5}{m(d-m)}}^{1/3} T^{2/3} }.
\end{equation*}


\paragraph{Ranking with the Hamming loss and the number of items $m$}
We finally consider online ranking with the Hamming loss and the number of items $m$.  
From \citet[Proposition 4]{comband}, the smallest positive eigenvalue is at least $1/m$.
Hence, since $\V=-\frac{1}{m}\I$, we have
\begin{equation*}
\tr\prn{\inverse{\V}\bm{Q}^+(\inverse{\V})^\top}=m^2\tr\prn{\bm{Q}^+}\leq m^2\sum_{i=1}^{\rank(\bm{Q}^+)} m\leq m^5,
\end{equation*}
where we used $\rank(\bm{Q}^+) \leq d = m^2$,
and this allows us to choose $\omega = m^5$.
Substituting these values into our regret upper bound in \eqref{eq:bound_self_app} , we obtain  
\begin{equation*}
    \E\brk{\reg}\leq\frac{bB^2}{a}+ O\prn*{ m^{5/3}\prn{B\dix T}^{2/3} }.
\end{equation*}  

