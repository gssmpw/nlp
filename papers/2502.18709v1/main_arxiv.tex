\documentclass[11pt]{article}

% preamble here
\usepackage[top=30truemm,bottom=30truemm,left=25truemm,right=25truemm]{geometry}
\usepackage[utf8]{inputenc} % allow utf-8 input
\usepackage[T1]{fontenc}    % use 8-bit T1 fonts

\usepackage{microtype}
\usepackage{graphicx}
% \usepackage[dvipdfmx]{graphicx}
\usepackage{subcaption}
\usepackage{booktabs} 

\usepackage[colorlinks=true,citecolor=blue,linkcolor=blue]{hyperref}


\newcommand{\theHalgorithm}{\arabic{algorithm}}

\usepackage{amsmath,amssymb,amsfonts,amsthm}
\usepackage{bm,bbm}
\usepackage{mathcomp}
\usepackage{empheq}
\usepackage{fancybox}
\usepackage{breqn}
\usepackage{mathtools}
\mathtoolsset{centercolon}
\usepackage{tgtermes}

\usepackage[capitalize,noabbrev,nameinlink]{cleveref}

\usepackage[textsize=tiny]{todonotes}

\usepackage{my_macro}
\usepackage[authoryear,round]{natbib} 
\usepackage[font=small,labelfont=bf]{caption}



\theoremstyle{plain}
\newtheorem{theorem}{Theorem}[section]
\newtheorem{proposition}[theorem]{Proposition}
\newtheorem{lemma}[theorem]{Lemma}
\newtheorem{corollary}[theorem]{Corollary}
\theoremstyle{definition}
\newtheorem{definition}[theorem]{Definition}
\newtheorem{method}[theorem]{Method}
\newtheorem{assumption}[theorem]{Assumption}
\theoremstyle{remark}
\newtheorem{remark}[theorem]{Remark}

\title{Bandit and Delayed Feedback in Online Structured Prediction}

\author{
  Yuki Shibukawa\footnote{
    The University of Tokyo; 
    \texttt{shibu-yu762@g.ecc.u-tokyo.ac.jp}.
  }
  \and
  Taira Tsuchiya\footnote{
    The University of Tokyo and RIKEN; \texttt{tsuchiya@mist.i.u-tokyo.ac.jp}.
  }
  \and
  Shinsaku Sakaue\footnote{
    The University of Tokyo and RIKEN; \texttt{sakaue@mist.i.u-tokyo.ac.jp}.
  }
  \and 
  Kenji Yamanishi\footnote{
    The University of Tokyo; \texttt{yamanishi@g.ecc.u-tokyo.ac.jp}.
  }
}


\begin{document}
\maketitle

% abstract here
\begin{abstract}
Online structured prediction is a task of sequentially predicting outputs with complex structures based on inputs and past observations, encompassing online classification. Recent studies showed that in the full information setup, we can achieve finite bounds on the \textit{surrogate regret}, i.e., the extra target loss relative to the best possible surrogate loss. In practice, however, full information feedback is often unrealistic as it requires immediate access to the whole structure of complex outputs. Motivated by this, we propose algorithms that work with less demanding feedback, \textit{bandit} and \textit{delayed} feedback. For the bandit setting, using a standard inverse-weighted gradient estimator, we achieve a surrogate regret bound of $O(\sqrt{KT})$ for the time horizon $T$ and the size of the output set $K$. However, $K$ can be extremely large when outputs are highly complex, making this result less desirable. To address this, we propose an algorithm that achieves a surrogate regret bound of $O(T^{2/3})$, which is independent of $K$. This is enabled with a carefully designed pseudo-inverse matrix estimator. Furthermore, for the delayed full information feedback setup, we obtain a surrogate regret bound of $O(D^{2/3} T^{1/3})$ for the delay time $D$. We also provide algorithms for the delayed bandit feedback setup. Finally, we numerically evaluate the performance of the proposed algorithms in online classification with bandit feedback.
\end{abstract}


\section{Introduction}
In many machine learning problems, given an input vector from a vector space $\xx$, we aim to predict a vector in a finite output space $\yy$.  
Multiclass classification is one of the simplest examples, while in other cases output spaces may have more complex structures. 
\emph{Structured prediction} refers to such a class of problems with structured output spaces, including multiclass classification, multilabel classification, ranking, and ordinal regression, and it has applications in various fields, ranging from natural language processing to bioinformatics \citep{JMLR_Tsochantaridis_2005, bakir_2007_article}.
In structured prediction, training models that directly predict outputs in complex discrete output spaces is typically challenging. 
Therefore, we often adopt the \emph{surrogate loss framework} \citep{Bartlett_2006}---define an intermediate space of score vectors and train models that estimate score vectors from inputs based on surrogate loss functions.
Examples of surrogate losses include squared, logistic, and hinge losses, and a general framework encompassing them is the \emph{Fenchel--Young loss} \citep{JMLR_2020_blondel}, which we rely on in this study.


Structured prediction can be naturally extended to the online setting \citep{pmlr-v247-sakaue24a}.  
In this online setting, at each round $t=1,\dots,T$, the environment selects an input-output pair $(\xt,\yt)\in\xx\times\yy$.  
A learner then predicts $\yht \in \yy$ based on the input $\xt$ and incurs a loss $L(\yht;\yt)$, where $L:\yy\times\yy\to\R_{\geq0}$ is the target loss function. 
As with \citet{pmlr-v247-sakaue24a}, we focus on the simple yet fundamental case where the learner's model for estimating score vectors is linear. 

\begin{table*}[t]
\centering
\caption{Surrogate regret upper and lower bounds in online multiclass classification and online structured prediction. Here, $T$ is the time horizon, $K$ is the size of the set, $\yy$, of output vectors, and $D$ is the fixed delayed time.  
``FI'' is the abbreviation of full information.
Delayed feedback is considered only when ``Delayed'' appears in the feedback column. 
In the target loss column, ``SELF*'' means SELF that satisfies \cref{asp:self}.
Note that the $O(T^{2/3})$ bounds for SELF* in lines 6 and 9 do not explicitly depend on $\K$ but on $d$; 
in the case of multiclass classification with the 0-1 loss, the dependence on $\K$ appears as $d = \K$.
} 
\small
\begin{tabular}{@{}l@{\hspace{1ex}}l@{\hspace{1ex}}l@{\hspace{1ex}}l@{\hspace{1ex}}l@{}}
\toprule
& Problem setup &Feedback & Target loss &  Surrogate regret bound \\ 
\midrule
 \citet[Cor.~1]{NEURIPS2021_Hoeven} & Binary classification & Bandit & 0-1 loss & $\Omega(\sqrt{T})$ ($d=2$) \\ 
 \midrule
 \citet[Thm.~4]{NEURIPS2020_Hoeven} & Multiclass classification & Bandit & 0-1 loss & $O(\K \sqrt{T})$ \\ 
 \citet[Thm.~1]{NEURIPS2021_Hoeven} & Multiclass classification & Bandit & 0-1 loss & $O(\K \sqrt{T})$ \\ 
 \midrule
\citet[Thms.~7 and 8]{pmlr-v247-sakaue24a} & Structured prediction & FI & SELF & $O(1)$  \\
\textbf{This work} (\cref{thm:bandit_regret_expectation_abstract,thm:bandit_high_prob}) & Structured prediction &Bandit & SELF &  $O(\sqrt{\K T})$  \\
\textbf{This work}  (\cref{thm:bandit_regret_pseudo_estimator}) & Structured prediction & Bandit & SELF* &  $O(T^{2/3})$ \\
\textbf{This work}  (\cref{thm:delayed_regret_expectation_abstract,thm:delayed_regret_probability_abstract}) & Structured prediction & FI \& Delayed & SELF & $O(D^{2/3}T^{1/3})$  \\
\textbf{This work}  (\cref{thm:delay_bandit_bound_general_abstract}) & Structured prediction & Bandit \& Delayed & SELF & $O(\sqrt{DKT})$  \\
\textbf{This work}  (\cref{thm:delay_bandit_bound_self_abstract}) & Structured prediction & Bandit \& Delayed & SELF* & $O(D^{1/3}T^{2/3})$  \\
\bottomrule
\end{tabular}
\label{tab: regret order}
\end{table*}


The goal of the learner is to minimize the cumulative loss $\sumt{L(\yht;\yt)}$. 
On the other hand, the best the learner can do in the surrogate loss framework is to minimize the cumulative surrogate loss, namely $\sumt{S(\U\xt;\yt)}$, where $\U:\xx\to\R^d$ is the best offline linear estimator and $S:\R^d\times\yy\to\R_{\geq0}$ is a surrogate loss, which measures the discrepancy between the score vector $\U\xt \in \R^d$ and $\yt\in\yy$. 
Given this, a natural performance measure of the learner's predictions is the following \emph{surrogate regret} $\reg$: 
\begin{equation}\label{eq:sur_regret}
    \sumt{L(\yht;\yt)}=\sumt{S(\U\xt;\yt)}+\reg.
\end{equation}
The surrogate regret was introduced in the seminal paper by \Citet{NEURIPS2020_Hoeven} in the context of online multiclass classification.  
Recently, \citet{pmlr-v247-sakaue24a} showed that a finite surrogate regret bound  
can be achieved for online structured prediction under full information feedback, i.e., the learner can observe $\yt$ at the end of each round $t$.

However, the assumption that full information feedback is available is often demanding, especially when outputs have complex structures.
%
For example, in sequential ad assortment on an advertising platform, we may be able to observe only the click-through rate but not which ads were clicked, which boils down to the \emph{bandit feedback} setting \citep{Kakade2008EfficientBA,JMLR:v15:gentile:class_ranking}. 
Also, we may only have access to feedback from a while ago when designing an ad assortment for a new user—namely, \emph{delayed feedback}  \citep{Weinberger_2002_delay,manwani2022delaytronefficientlearningmulticlass}.
%
Similar situations have led to a plethora of studies in various online learning settings.
In combinatorial bandits, algorithms under bandit feedback (referred to as full-bandit feedback in their context), instead of full information feedback, have been widely studied
\citep{comband,combes15combinatorial,rejwan20topk,du21combinatorial}. 
Delayed feedback is also explored in various settings, including full information and bandit feedback \citep{joulani13online,cesabianchi16delay}.
Due to space limitations, we defer a further discussion of the background to \cref{app:additional_related_work}.


\subsection*{Our Contributions}
To extend the applicability of online structured prediction, this study develops online structured prediction algorithms that can handle weaker feedback---bandit feedback and delayed feedback---instead of full information feedback.  
As with \citet{pmlr-v247-sakaue24a}, we consider the case where target loss functions belong to a class called the Structured Encoding Loss Function (SELF) \citep{ciliberto16consistent,NEURIPS2019_Blondel}, a general class including the 0-1 loss in multiclass classification and the Hamming loss in multilabel classification and ranking (see \cref{subsec:self} for the definition). 
\Cref{tab: regret order} summarizes the surrogate regret bounds provided in this study and comparisons with existing results.


One of the challenges of bandit feedback is that the true output $\yt$ is not observed, making it impossible to compute the true gradient of the surrogate loss. 
To deal with this, we use an inverse-weighted gradient estimator, a typical approach that assigns weights to gradients by the inverse of choosing each output, establishing an $O(\sqrt{\K T})$ surrogate regret upper bounds (\cref{thm:bandit_regret_expectation_abstract,thm:bandit_high_prob}), 
where $K = \abs{\yy}$ is the cardinality of $\yy$.  
The $O(\sqrt{\K T})$ bound has an optimal dependence on $T$; it matches the $\sqrt{T}$ lower bound in the special case of online multiclass classification with bandit feedback \Citep[Corollary 1]{NEURIPS2021_Hoeven}. 
Furthermore, our bound is better than the existing $O(\K\sqrt{T})$ bound of \Citet{NEURIPS2020_Hoeven} by a factor of $\sqrt{\K}$, while it is not directly comparable to the latest $O(\K\sqrt{T})$ bound in \Citet{NEURIPS2021_Hoeven} due to differences in surrogate loss functions. 
See \cref{app:Discussio_on_the_Difference_in_Surrogate_Losses} for a more detailed discussion. 

While the $O(\sqrt{\K T})$ bound is satisfactory when $K = \abs{\yy}$ is small, $K$ can be extremely large in some structured prediction problems: in multilabel classification with $m$ correct labels, we have $\K=\binom{d}{m}$, and in ranking problems with $m$ items, we have $\K=m!$.
%
To address this issue, we consider a special case of SELF (denoted by SELF* in \cref{tab: regret order}), which still includes the aforementioned examples: the 0-1 loss in multiclass classification and the Hamming loss in multilabel classification and ranking. 
A technical challenge to resolve the issue lies in designing an appropriate gradient estimator used in online learning methods.
To this end, we draw inspiration from pseudo-inverse estimators used in the adversarial linear/combinatorial bandit literature \citep{dani07price,abernethy08competing,comband}. 
While we cannot naively apply the existing estimators, we design a new gradient estimator that applies to various specific structured prediction problems belonging to the special SELF framework.
Armed with this gradient estimator, we achieve a surrogate regret upper bound of $O(T^{2/3})$, which does not explicitly depend on~$\K$~(\cref{thm:bandit_regret_pseudo_estimator}).

For the delayed feedback setting with a known fixed delay time of $D$, it is actually not difficult to obtain a surrogate regret bound of $O(\sqrt{D T})$ with standard Online Convex Optimization (OCO) algorithms for delayed feedback. 
Our finding is that we can achieve a surrogate regret bound of $O(D^{2/3} T^{1/3})$ in online structured prediction under delayed full information feedback (\cref{thm:delayed_regret_expectation_abstract,thm:delayed_regret_probability_abstract}) by leveraging ODAFTRL \citep{pmlr-v139-flaspohler21a}, a Follow-the-Regularized-Leader-type algorithm that achieves an AdaGrad-type regret upper bound in OCO under delayed feedback. 
This bound is better than $O(\sqrt{D T})$ as $D \le T$.


Given the contributions so far, it is natural to explore online structured prediction in environments where both delay and bandit feedback are present. 
We obtain algorithms for this setup by combining the theoretical developments for bandit feedback without delay and delayed full information feedback, offering surrogate regret bounds of $O(\sqrt{D K T})$ (\cref{thm:delay_bandit_bound_general_abstract}) and $O(D^{1/3} T^{2/3})$ (\cref{thm:delay_bandit_bound_self_abstract}).

We validate our algorithms through numerical experiments using both synthetic and real-world data.  
Specifically, we consider online multiclass classification with bandit feedback. 
We observe that, depending on the number of classes and the dataset, our algorithm, designed for general structured prediction, can achieve accuracy comparable to existing algorithms specialized for multiclass classification.



\section{Preliminaries}
We describe the detailed setup of online structured prediction and key tools used in this work: the Fenchel--Young loss, SELF, and randomized decoding.
\paragraph{Notation}
For any integer $n > 0$, let $\brk{n} = \set{1,2,\hdots,n}$.
Let $\nrm{\cdot}$ denote a norm with $\kappa\nrm{\bmy}\geq\nrm{\bmy}_2$ for some $\kappa>0$ for any $\bmy\in\mathbb{R}^d$. 
For a matrix $\W$, let $\nrm{\W}_{\mathrm{F}}=\sqrt{\tr\prn*{\W^\top \W}}$ be the Frobenius norm. 
Let $\bm1$ denote the all-ones vector and $\bm{\ee}_i$ the $i$th standard basis vector.
For $\mathcal{\mathcal{K}}\subset \mathbb{R}^d$, let $\conv(\mathcal{K})$ be its convex hull and $I_{\mathcal{K}}:\mathbb{R}^d\to\set{0, +\infty}$ be its indicator function, which takes zero if $\bmy \in \mathcal{K}$ and $+\infty$ otherwise.
For $\Omega:\mathbb{R}^d\to\mathbb{R}\cup\set{+\infty}$, let $\dom(\Omega) \coloneqq \set*{\bmy \in \mathbb{R}^d:\Omega(\bmy) < +\infty}$ be its effective domain and $\Omega^*(\thb) \coloneqq \sup\set*{\inpr{\thb, \bmy} - \Omega(\bmy):\bmy \in \mathbb{R}^d}$ be its convex conjugate.
\cref{tab: notation} in \Cref{app: notation} summarizes the notation used in this paper.


\subsection{Online Structured Prediction}
Here, we describe the problem setting of online structured prediction.
Let $\xx$ be the input vector space and $\yy$ be the output vector space.
Define $\K \coloneqq |\yy|$.
Following \citet{JMLR_2020_blondel} and \citet{pmlr-v247-sakaue24a}, we assume that $\yy$ is embedded into $\mathbb{R}^d$ in a standard manner.
For example, $\yy = \set{\bm{\mathrm{e}}_1,\hdots,\bm{\mathrm{e}}_d}$ in $d$-class multiclass classification.


A linear estimator $\W\in\ww$ for a convex domain $\ww$ is used to transform the input vector $\bm{x}$ into the score vector $\W\bm{x}$.  
In online structured prediction, at each round $t=1,\dots,T$:
\begin{enumerate}%[topsep=2pt,itemsep=0pt, partopsep=0pt, leftmargin=18pt]
    \item The environment selects an input $\xt\in\xx$ and the true output $\yt\in\yy$; 
    \item The learner receives $\xt$ and computes the score vector~$\tht=\wt\xt$ using the linear estimator~$\wt$;
    \item The learner selects a predicted output $\yht$ based on $\tht$ and incurs a loss of $L(\yht;\yt)$;
    \item The learner receives feedback based on the problem setup and updates $\wt$ to $\W_{t+1}$ using an online learning algorithm, $\alg$.
\end{enumerate}
The goal of the learner is to minimize the cumulative prediction loss $\sumt{L(\yht;\yt)}$, which is equivalent to minimizing the surrogate regret $\mathcal{R}_T$ in \eqref{eq:sur_regret}.  
We assume that the input and output are generated in an oblivious manner.  
Note that when $\yy = \set{\bm{\ee}_1, \dots, \bm{\ee}_d}$ and $L(\yht; \yt) = \ind[\yht \neq \yt]$, the above setting reduces to online multiclass classification, which was studied by \Citet{NEURIPS2020_Hoeven} and \Citet{NEURIPS2021_Hoeven}.
We will use $B\coloneqq\diam(\ww)$, $\dix\coloneqq\diam(\xx)$, and $\diy\coloneqq\diam(\yy)$ to denote the diameters of the sets $\ww$, $\xx$, and $\yy$, respectively.


The feedback observed by the learner depends on the problem setting.  
The most fundamental setting is the full information setup, where the true output $\yt$ is observed as feedback at the end of each round $t$. 
This setup was extensively investigated in \citet{pmlr-v247-sakaue24a}.  
By contrast, our study investigates the following weaker feedback:
\begin{itemize}%[topsep=2pt,itemsep=0pt, partopsep=0pt, leftmargin=18pt]
    \item \textbf{Bandit feedback}: Only the value of the loss function is observed. 
    Specifically, at the end of each round $t$, the learner observes the target loss value $L(\yht;\yt)$ as feedback.  
    \item \textbf{Delayed feedback}: The feedback is observed with a certain delay. 
    We consider a fixed $D$-round delay setting, i.e., no feedback is received for round $t\leq D$,  
    and for $t>D$, the learner observes either full information feedback $\bmy_{t-D}$ or bandit feedback $L(\hat{\bm{y}}_{t-D}; \bm{y}_{t-D})$.  
\end{itemize}


In this paper, we make the following assumptions:
\begin{assumption}
    \label{asp:online_structured_prediction}
    (1)~There exists $\nu>0$ such that for any distinct $\bm{y},\bm{y}^\prime\in \mathcal{Y}$, it holds that $\|\bm{y}-\bm{y}^\prime\|\geq\nu$.  
    (2) For each $\bm{y}\in\mathcal{Y}$, the target loss function $L(\cdot;\bm{y})$ is defined on $\conv(\mathcal{Y})$, is non-negative, and is affine with respect to its first argument.  
    (3) There exists $\gamma$ such that for any $\bm{y}^\prime\in\conv(\mathcal{Y})$ and $\bm{y}\in\mathcal{Y}$, it holds that $L(\bm{y}^\prime;\bm{y})\leq\gamma\|\bm{y}^\prime-\bm{y}\|$ and $L(\bmy^{\prime};\bmy)\leq 1$. 
    (4) It holds that $L(\bm{y}'; \bm{y})=0$ only if $\bm{y}'=\bm{y}$.
\end{assumption}

As discussed in \citet[Section 2.3]{pmlr-v247-sakaue24a}, these assumptions are natural and hold for a broad range of problem settings and target loss functions, including SELF (see \cref{subsec:self} for the formal definition).


\subsection{Fenchel--Young Loss}\label{subsec:fenchel-young}

We use the Fenchel--Young loss \citep{JMLR_2020_blondel} as the surrogate loss, which subsumes many representative surrogate losses, such as logistic loss, Conditional Random Field (CRF) loss \citep{lafferty01conditional}, and SparseMAP \citep{Niculae18sparse}.
See \citet[Table 1]{JMLR_2020_blondel} for more examples. 
\begin{definition}[{\citealt[Fenchel--Young loss]{JMLR_2020_blondel}}]
    \label{def: Fenchel--Young Loss}
    Let $\Omega:\mathbb{R}^d\rightarrow\mathbb{R}\cup\set{+\infty}$ be a regularization function with $\mathcal{Y}\subset\operatorname{dom}(\Omega)$.  
    The Fenchel--Young loss generated by $\Omega$, denoted by $S_\Omega:\operatorname{dom}(\Omega^\ast)\times\operatorname{dom}(\Omega)\rightarrow\mathbb{R}_{\geq0}$, is defined as
    \[
    S_{\Omega}(\thb;\bmy)\coloneqq\Omega^\ast(\thb)+\Omega(\bmy)-\inpr{\thb,\bmy}.
    \]
\end{definition}
The Fenchel--Young loss has the following properties, which will be useful in the subsequent discussion:
\begin{proposition}[{\citealt[Propositions 2 and~3]{JMLR_2020_blondel} and \citealt[Proposition 3]{pmlr-v247-sakaue24a}}]
    \label{prop:fenchel}
    Let $\Psi:\mathbb{R}^d\rightarrow\mathbb{R}\cup\set{+\infty}$ be a differentiable, Legendre-type function\footnote{
    A function is called Legendre-type if, for any sequence $x_1,x_2,\hdots$ in $\operatorname{int}(\dom(\Psi))$ that converges to a boundary point of $\operatorname{int}(\dom(\Psi))$, it holds that $\lim_{i\rightarrow\infty}\|\nabla\Psi(x_i)\|_2=+\infty$.}  
    that is $\lambda$-strongly convex with respect to $\|\cdot\|$, and suppose that $\conv(\mathcal{Y})\subset\dom(\Psi)$ and $\dom(\Psi^\ast)=\mathbb{R}^d$.  
    Define $\Omega=\Psi+I_{\conv(\yy)}$ and let $S_\Omega$ be the Fenchel--Young loss generated by $\Omega$.  
    For any $\thb\in\mathbb{R}^d$, we define the regularized prediction function as 
    \begin{align*}
        \yho(\thb)&\coloneqq\argmax\{\inpr{\thb,\bmy}-\Omega(\bmy)\::\:\bmy\in\mathbb{R}^d\}\\
        &=\argmax\set{\inpr{\thb,\bmy}-\Psi(\bmy)\::\:\bmy\in\conv(\mathcal{Y})}.
    \end{align*}
    Then, for any $\bmy\in\mathcal{Y}$, $S_\Omega(\thb,\bmy)$ is differentiable with respect to $\thb$, and it satisfies  
    $
    \nabla S_\Omega(\thb;\bmy)=\yho(\thb)-\bmy.
    $
    Furthermore, it holds that  
    $
    S_\Omega(\thb;\bmy)\geq\frac{\lambda}{2}\|\bmy-\yho(\thb)\|^2.
    $ 
\end{proposition}


In what follows, let $S_t(\W)\coloneqq S_\Omega(\W\xt;\yt)$ for simplicity.
Importantly, from the properties of the Fenchel--Young loss, there exists some $b>0$ such that for any $\W\in\ww$,  
\begin{equation}\label{eq:St_smooth}
    \nrm{\nabla S_t (\W)}_{\mathrm{F}}^2\leq b S_t(\W).
\end{equation}
Indeed, from \cref{prop:fenchel} and \cref{asp:online_structured_prediction}, we have 
$
\nrm{\nabla S_t(\W_t)}_\F^2
=
\nrm{\yho(\tht)-\yt}_2^2 \nrm{\xt}^2
\leq
\dix^2\kappa^2\nrm{\yho(\tht)-\yt}^2
\leq
\frac{2\dix^2\kappa^2}{\lambda}\sw
$, 
where we used $\nabla S_t(\wt) =  \prn{\hat{\bm{y}}_\Omega(\bm{\theta}_t) - \yt}\xt^\top$ and $\nrm{\cdot}_2 \leq \kappa \nrm{\cdot}$. 
Thus, \eqref{eq:St_smooth} holds with  $b=\frac{2\dix^2\kappa^2}{\lambda}$.
Below, let $L_t(\bmy)\coloneqq L(\bmy;\yt)$,
$\G_t\coloneqq\nabla S_t(\wt) =  \prn{\hat{\bm{y}}_\Omega(\bm{\theta}_t) - \yt} \bm{x}_t^\top$, and $\L\coloneqq\max_{\W\in\ww}\nrm{\nabla S_t(\W)}_\F$.


\subsection{Examples of Structured Prediction}\label{subsec:pre_examples} 
We present several special cases of structured prediction  
along with specific parameter values introduced so far; see \citet[Section 2.3]{pmlr-v247-sakaue24a} for further details.
\paragraph{Multiclass classification}
Let $\yy=\set{\mathbf{e}_1,\dots,\mathbf{e}_d}$ and $\|\cdot\| = \|\cdot\|_1$.
When using the 0-1 loss, $L(\bmy^{\prime};\bmy)=\ind\brk{\bmy^{\prime}\neq\bmy}$, the parameters in \cref{asp:online_structured_prediction} are $\nu=2$ and $\gamma=\frac{1}{2}$.
The logistic surrogate loss is a Fenchel--Young loss $S_\Omega$ generated by the entropy regularization function $\Omega=\mathsf{H}^s+I_{\Delta_d}$ (up to a constant factor), where $\mathsf{H}^s(\bmy)\coloneqq-\sum_{i=1}^d y_i\log y_i$ and $\Delta_d$ is the $(d-1)$-dimensional probability simplex.
Since $\Omega$ is a $1$-strongly convex function with respect to $\|\cdot\|_1$, we have $\lambda=1$.

\paragraph{Multilabel classification}
Let $\yy=\set{0,1}^d$ and $\|\cdot\| = \|\cdot\|_2$.
When using the Hamming loss as the target loss function $L(\bmy^{\prime};\bmy)=\frac{1}{d}\sum_{i=1}^{d}\ind\brk{y_{i}^{\prime}\neq y_i}$, \cref{asp:online_structured_prediction} is satisfied with $\nu=1$ and $\gamma=\frac{1}{d}$.
The SparseMAP surrogate loss $S_\Omega(\thb,\bmy)=\frac{1}{2}\nrm{\bmy-\thb}_2^2-\frac{1}{2}\nrm{\yho(\thb)-\thb}_2^2$ is a Fenchel--Young loss generated by $\Omega=\frac{1}{2}\nrm{\cdot}^2+I_{\conv(\yy)}$.
Since $\Omega$ is 1-strongly convex with respect to $\|\cdot\|_2$, we have $\lambda=1$.

\paragraph{Ranking}
We consider predicting the ranking of $m$ items. 
Let $\nrm{\cdot} = \nrm{\cdot}_1$, $d=m^2$, and $\yy\subset\set{0,1}^d$ be the set of all vectors representing $m \times m$ permutation matrices.  
We use the target loss function that counts mismatches, $L(\bmy^{\prime};\bmy)=\frac{1}{m}\sum_{i=1}^{m}\ind\brk{y_{i,j_i}^{\prime}\neq y_{i,j_i}}$, where $j_i\in[m]$ is a unique index with $y_{ij_i}=1$ for each $i\in[m]$.  
In this case, the parameters in \cref{asp:online_structured_prediction} satisfy $\nu=4$ and $\gamma=\frac{1}{2m}$.  
We use a surrogate loss given by $S_\Omega(\thb;\bmy)=\inpr{\thb,\yho(\thb)-\bmy}+\frac{1}{\zeta}\mathsf{H}^s(\yho(\thb))$, where $\Omega=-\frac{1}{\zeta}\mathsf{H}^s+I_{\conv(\yy)}$ and $\zeta$ is a parameter controlling the regularization strength. 
The first term in $S_\Omega$ measures the affinity between $\thb$ and $\bmy$, while the second term evaluates the uncertainty of $\yho(\thb)$.  
Since $\Omega$ is $\frac{1}{m\zeta}$-strongly convex, we have $\lambda=\frac{1}{m\zeta}$.


\subsection{Structured Encoding Loss Function (SELF)}\label{subsec:self}
Here we introduce a common class of target loss functions, called the Structured Encoding Loss Function (SELF).
A target loss function is SELF if it can be expressed as  
\begin{equation}\label{eq:self}
    L(\yt; \yht)=\inpr{\yht,\V\yt+\bm{b}}+c(\yt),
\end{equation}
where $\bm{b} \in \R^d$ is a constant vector, $\V\in\R^{d\times d}$ is a constant matrix, and $c:\yy\to\R$ is a function.  
The following loss examples, taken from \citet[Appendix A]{NEURIPS2019_Blondel}, belong to the SELF class: 
\begin{itemize}%[topsep=2pt,itemsep=0pt, partopsep=0pt, leftmargin=18pt]
\item Multiclass classification: the 0-1 loss is a SELF with $\V=\bm{1}\bm{1}^\top-\I$, $\bb=\bm{0}$, and $c(\bmy)=0$.  

\item Multilabel classification: the Hamming loss, 
$L(\bmy^{\prime};\bmy)=\frac{1}{d}\sum_{i=1}^{d}\ind\brk{y^{\prime}_{t,i}\neq y_{i}}$, is a SELF with $\V=-\frac{2}{d}\I$, $\bb=\frac{\bm{1}}{d}$, and $c(\bmy)=\frac{1}{d}\inpr{\bmy,\bm{1}}$, where the last factor is constant if the number of correct labels is fixed.

\item Ranking: the Hamming loss  
$L(\bmy^{\prime};\bmy)=\frac{1}{m}\sum_{i=1}^{m}\ind\brk{y_{i,j_i}^{\prime}\neq y_{i,j_{i}}}$, where $j_i\in[m]$ is a unique index with $y_{i,j_i}=1$ for each $i\in[m]$, is a SELF with  
$\V=-\frac{1}{m}\I$, $\bb=\bm{0}$, and $c(\bmy)=1$.
\end{itemize}
Following \citet{pmlr-v247-sakaue24a}, this study assumes that the target loss function $L$ is a SELF.

\subsection{Randomized Decoding}\label{subsec:randomized_decoding}
\begin{algorithm}[t]
    \caption{Randomized decoding $\phi_\Omega$}
    \label{ALG: randomized decoding}
    \begin{algorithmic}[1]
        \Require {$\btheta\in\mathbb{R}^d$}
            \State {$\yho(\btheta)\leftarrow\argmax\{\langle\btheta,\bm{y}\rangle-\Psi(\bm{y})\::\:\bm{y}\in\conv(\mathcal{Y})\}$}
            \State {$\bm{y}^\ast\leftarrow\argmin\{\|\bm{y}-\yho(\btheta)\|\::\:\bm{y}\in\mathcal{Y}\}$}
            \State {$\Delta^\ast\leftarrow\|\bm{y}^\ast-\yho(\btheta)\|,\:p\leftarrow\min\{1,2\Delta^\ast/\nu\}$}
            \State {Sample $Z \sim \mathrm{Ber}(p)$}
            \LineIf{$Z=0$}{$\yh\leftarrow\bm{y}^\ast$}
            \LineIf{$Z=1$}{$\yh\leftarrow\bm{\tilde{y}}$ where $\bm{\tilde{y}}$ is randomly drawn from $\yy$ so that $\E\brk*{\bm{\tilde{y}}|Z=1}=\yho(\btheta)$}
            \Ensure{$\phi_\Omega(\thb)=\yh$}
    \end{algorithmic}
\end{algorithm}


The procedure of converting the estimated score $\thb$ into a structured output $\bm{\hat{y}}$ is called decoding.  
For this, we employ randomized decoding \citep{pmlr-v247-sakaue24a},  
which plays an essential role particularly in deriving an upper bound independent of the output set size $K = \abs{\mathcal{Y}}$ in \cref{subsec:Bandit_Structured_Prediction_with_SELF}.
The randomized decoding (\cref{ALG: randomized decoding}) selects either the closest $\bm{y}^* \in \yy$ to $\hat{\bm{y}}_\Omega(\thb) \in \conv(\yy)$ or a random $\widetilde{\bm{y}} \in \yy$ satisfying $\E[\widetilde{\bm{y}} \mid Z=1] = \hat{\bm{y}}_\Omega(\thb)$, where $Z$ follows a Bernoulli distribution with a parameter $p$.  
Intuitively, the parameter $p$ is chosen so that if $\hat{\bm{y}}_\Omega(\thb)$ is close to $\bm{y}^*$, the decoding is more likely to return $\bm{y}^*$; otherwise, it is more likely to return $\widetilde{\bm{y}}$, reflecting uncertainty.  
An important property satisfied by the randomized decoding is the following lemma, which we use in the subsequent analysis:
\begin{lemma}[{\citealt[Lemma 4]{pmlr-v247-sakaue24a}}]
    \label{lem:expected_target_bound}
  For any $(\thb, \bmy) \in \mathbb{R}^d\times\yy$, the randomized decoding $\phi_\Omega$ satisfies
  \[
    \E[L(\phi_\Omega(\thb);\bmy)] \leq \frac{4\gamma}{\lambda\nu} S_\Omega(\thb;\bmy),
  \]
  where the expectation is taken with respect to the internal randomness of the randomized decoding.
\end{lemma}









\section{Bandit Feedback}
\label{sec:bandit}
In this section, we present two online structured prediction algorithms in the bandit feedback setup and analyze their regret.  
Our results here are mostly special cases of the regret bounds obtained when handling bandit and delayed feedback (\cref{sec:bandit_and_delayed}). 
Nevertheless, by focusing on the case without delay, we provide a clearer exposition of the core ideas.

\subsection{Randomized Decoding with Uniform Exploration}
\begin{algorithm}[t]
    \caption{Randomized decoding with uniform exploration (RDUE) $\psi_\Omega$}
    \label{ALG:randomized decoding with uniform exploration}
    \begin{algorithmic}[1]
        \Require{$\thb\in\R^n$, $q \in [0,1]$}
        \State {Sample $X \sim \mathrm{Ber}(q)$ 
        }
        \LineIf{$X=0$}{$\bm{\hat{y}}\leftarrow\phi_\Omega(\thb)$}
        \LineIf{$X=1$}{Sample $\bm{y}^\ast \sim \mathrm{Unif}(\mathcal{Y})$ and $\yh\leftarrow\bm{y}^\ast$}
        \Ensure{$\psi_\Omega(\thb)=\yh$}
    \end{algorithmic}
\end{algorithm}
Here, we discuss the properties of the decoding function, \emph{Randomized Decoding with Uniform Exploration (RDUE)}, which will be used in both algorithms to convert scores into outputs.  
As discussed in \cref{subsec:randomized_decoding}, in online structured prediction with full information feedback, the randomized decoding (\cref{ALG: randomized decoding}) was introduced as a decoding function \citep{pmlr-v247-sakaue24a}.  
However, naively applying the randomized decoding does not lead to a satisfactory regret bound under bandit feedback.  
We extend the framework of the randomized decoding to handle bandit feedback effectively.

RDUE (\cref{ALG:randomized decoding with uniform exploration}) is a procedure that, with probability $q \in [0,1]$, selects $\hat{\bmy}$ uniformly at random from $\yy$,  
and with probability $1-q$, selects the output of the randomized decoding.  
Using RDUE, we define $p_t(\bm{y})$ as the probability that $\hat{\bm{y}}_t$ coincides with $\bm{y}$ at round $t$.  
Note that for any $\bm{y} \in \mathcal{Y}$, it holds that  
$
p_t(\bm{y})\geq\frac{q}{\K}.
$
Furthermore, similar to the property of the randomized decoding in \cref{lem:expected_target_bound}, RDUE satisfies the following property:
\begin{lemma}
    \label{lem:bound of randomized decoding with uniform exploration}
    For any $(\thb,\bm{y})\in\R^d\times\mathcal{Y}$, RDUE $\psi_\Omega$ satisfies
    \[
        \E\brk*{L(\psi_\Omega(\thb);\bm{y})}\leq\frac{4\gamma}{\lambda\nu}(1-q)S_\Omega(\thb;\bmy)+q\frac{\K-1}{\K  },
    \]
  where the expectation is taken with respect to the internal randomness of RDUE. 
\end{lemma}
\begin{proof}
Since the output of the randomized decoding is chosen with probability $1-q$,  
and a uniformly random output is chosen with probability $q$, we have  
$
\E\brk*{L(\psi_\Omega(\thb);\bmy)}=(1-q)\E\brk*{L(\phi_\Omega(\thb);\bmy)}+q\frac{\K-1}{\K},
$
where we used $\L(\cdot;\cdot)\leq 1$ and $\phi_\Omega$ is the randomized decoding.  
Hence, combining this with \cref{lem:expected_target_bound}, we obtain the desired bound.
\end{proof}

Additionally, this section makes the following assumption:
\begin{assumption}
    \label{asp:bandit_a}
    There exists $a\in\prn{0,1}$ such that  
    \[
    \expect{L_t(\yht)}\leq(1-a)\sw+q.
    \]  
    Here, $\expect{\cdot}$ denotes the conditional expectation given the random variables $\hat{\bmy}_1,\dots,\hat{\bmy}_{t-1}$.  
\end{assumption}
This assumption can be satisfied by using RDUE and letting   
$a \leq 1-\frac{4\gamma}{\lambda\nu}(1-q)$  
if $\lambda>\frac{4\gamma}{\nu}(1-q)$, according to \cref{lem:bound of randomized decoding with uniform exploration}.  
In what follows, let $a = 1-\frac{4\gamma}{\lambda\nu}$.
Note that $\lambda \geq \frac{4\gamma}{\nu}$ holds in the cases of multiclass classification, multilabel classification, and ranking, as discussed in \cref{subsec:pre_examples}.


\subsection{Online Gradient Descent}\label{subsec:ogd}
The algorithm in this section uses the adaptive Online Gradient Descent (OGD, \citealt{streeter2010regretonlineconditioning}) as $\alg$.
OGD updates $\wt$ to $\W_{t+1}$ using a gradient $\bm{G}_t$ and learning rate $\eta_t$ by
$
    \W_{t+1} \leftarrow \Pi_{\ww} \prn*{\wt - \eta_{t} \bm{G}_t},
$
where $\Pi_{\ww}(\bm Z) = \argmin_{\bm{X} \in \ww} \nrm{\bm X - \bm Z}_{\F}$.
OGD with appropriately chosen learning rate $\eta_t$ achieves the following bound:
\begin{lemma}[{\citealt[Theorem 4.14]{orabona2023modernintroductiononlinelearning}}]
    \label{lem:ogd}
    Suppose that we set the learning rate to $\eta_t=\frac{B}{\sqrt{2 \sum_{i=1}^t\nrm{\bm{G}_i}_{\mathrm{F}}^2}}$ and do not update on rounds when $\bm{G}_t$ is the all-zero matrix.
    Then, for any $\U\in\ww$, OGD achieves 
    $
        \sumt{\inpr{\bm{G}_t, \W_t - \U}}
        \leq \sqrt{2}B\sqrt{\sumt{\nrm{\bm{G}_t}_{\mathrm{F}}^2}}.
    $
\end{lemma}

\subsection{$O(\sqrt{K T})$ Regret Algorithm}
\label{subsec:Bandit_Structured_Prediction_with_General_Losses}
Here, we present an algorithm that achieves a regret upper bound of $O(\sqrt{\K T})$.
\paragraph{Algorithm based on inverse-weighted estimator}
In the bandit setting, the true output $\yt$ is not observed,  
and thus it is necessary to estimate the gradient required for updating $\wt$.  
To deal with this, we use the following inverse-weighted gradient estimator:
\begin{equation}\label{eq:inverse_weighted_est}
    \gtil\coloneqq\frac{\ind[\yht=\yt]}{p_t(\yt)}\G_t,
\end{equation}
where we recall that $\G_t=\nabla S_t(\wt) = \prn{\hat{\bm{y}}_\Omega(\bm{\theta}_t) - \yt} \bm{x}_t^\top$.
Note that $\gtil$ is unbiased, i.e., $\E\brk[\big]{\gtil}=\G_t$.
We use RDUE with $q=B\sqrt{\K/T}$ as the decoding function (assuming $T \geq B^2 \K$ for simplicity).  
For $\alg$, we employ the adaptive OGD in \cref{subsec:ogd} with the learning rate of
$
\eta_t=\frac{B}{\sqrt{2 \sum_{i=1}^t\nrm{\tilde{\G}_i}_{\mathrm{F}}^2}}.
$

\begin{remark}\label{rem:zero-loss}
This study defines the bandit feedback as the value of the target loss function $L_t(\yht)$.
Note, however, that the above algorithm operates using only the weaker feedback of $\ind\brk*{\yht\neq\yt}$. 
\end{remark}

\paragraph{Regret bounds and analysis}
The above algorithm achieves the following regret bound:
\begin{theorem}\label{thm:bandit_regret_expectation_abstract}
    The regret of the above algorithm is bounded by
    $
        \E\brk{\reg}\leq \prn*{\frac{b}{2a}+1}B\sqrt{\K T}.
    $
\end{theorem}
The $O(\sqrt{\K T})$ bound has an optimal dependency on $T$  
and matches the $\sqrt{T}$ lower bound in the special case of online multiclass classification with bandit feedback \citep[Corollary 1]{NEURIPS2021_Hoeven}.  
Furthermore, our bound improves the existing $O(\sqrt{KT})$ bound by \citet{NEURIPS2020_Hoeven} by a factor of $\sqrt{\K}$. 
Note that, due to differences in the target loss function, our result is not directly comparable to the $\sqrt{\K T}$ bound in \citet{NEURIPS2021_Hoeven}. A more detailed discussion can be found in \cref{app:Discussio_on_the_Difference_in_Surrogate_Losses}.
\begin{proof}
From the convexity of $S_t$ and the unbiasedness of $\gtil$, we have 
$
    \E\brk*{\sumt{\prn{\sw-\su}}}
    \leq
    \E\brk*{\sumt{\inpr{\G_t,\wt-\U}}}
    =
    \E\brk*{\sumt{\inpr{\gtil,\wt-\U}}}.
 $
From \cref{lem:ogd}, this is further upper bounded as
$
    \E\brk*{\sumt{\inpr{\gtil,\wt-\U}}}
    \leq
    \sqrt{2}B\sqrt{\E\brk*{\sumt{{\nrm{\gtil}_{\mathrm{F}}^2}}}}
    \leq
    B\sqrt{\frac{2b\K }{q}\E\brk*{\sumt{\sw}}},
$
where in the first inequality we used Jensen's inequality and 
in the last inequality we used
$
\expect{\|\gtil\|_{\mathrm{F}}^2}
=
\frac{\|\G_t\|_{\mathrm{F}}^2}{p_t(\yt)}
\leq
\frac{\K }{q}\|\G_t\|_{\mathrm{F}}^2
\leq
\frac{b\K }{q}\sw,
$
which follows from $p_t(\bm{y}) \geq K /q$ and \eqref{eq:St_smooth}.
Therefore, from \cref{asp:bandit_a}, 
we have
$
    \E\brk{\reg}
    \leq
    \E\brk*{\sumt{\prn*{(1-a)\sw-\su}}}+qT
    \leq 
    B\sqrt{\frac{2b\K }{q} \E\brk*{\sumt{\sw}}}-a \E\brk*{\sumt{\sw}}+qT
    \leq
    \frac{bB^2\K }{2aq}+qT
    ,
$
where the last inequality follows from $c_1\sqrt{x}-c_2x\leq{c_1^2}/\prn{4c_2}$ for $x \geq 0$, $c_1\geq 0$, and $c_2>0$.
Finally, substituting $q=B\sqrt{\K/T}$ into the last inequality yields the desired bound.
\end{proof}

We can also prove the following high-probability bound:
\begin{theorem}\label{thm:bandit_high_prob}
Let $\delta \in (0,1)$.
Then with probability at least $1 - \delta$, the same algorithm as in \cref{thm:bandit_regret_expectation_abstract}, but with a different choice of $q$, achieves the regret bound of 
$\mathcal{R}_T = O\prn[\Big]{\sqrt{KT \log (1/\delta)} + \log(1/\delta)}$,
where we omit the dependencies on parameters other than $K$, $T$, and $\delta$.
\end{theorem}
A more precise statement and proof of this theorem are provided in \cref{app:proof_bandit_high_prob}.
To prove this theorem, we follow the analysis of \cref{thm:bandit_regret_expectation_abstract} and use Bernstein's inequality.  
To address the challenges posed by the randomness introduced by bandit feedback,  
we adopt an approach similar to that used by \citet{NEURIPS2021_Hoeven}, and
arguably, we have successfully simplified their analysis.


\subsection{$O(T^{2/3})$ Regret Algorithm}
\label{subsec:Bandit_Structured_Prediction_with_SELF}
While the $O(\sqrt{KT})$ regret bound given above is desirable in terms of the dependence on $T$, the dependence on $K = \abs{\yy}$ is undesirable for general structured prediction.  
In fact, we have $\K=\binom{d}{m}$ in multilabel classification with $m$ correct labels and $\K=m!$ in ranking with $m$ items.  
To address this issue, we present an algorithm that significantly improves the dependence on $\K$ when the target loss function belongs to a special class of SELF satisfying the following assumptions:\looseness=-1
\begin{assumption}\label{asp:self}
(i) $\V$ is invertible, and $\bm{b}$ and $c(\cdot)$ are known and non-negative.
(ii) Let $\bm{Q} = \E_{\bm{y} \sim \mu} \brk{ \bm{y} \bm{y}^\top }$, where $\mu$ is the uniform distribution over $\yy$. At least one of the following two conditions holds: 
(ii-a) $\bm{Q}$ is invertible, or 
(ii-b) for any $\bm{y} \in \yy$, $\V \bm{y}$ lies in the linear subspace spanned by vectors in $\yy$. 
(iii) For some $\omega > 0$, it holds that\looseness=-1
\begin{equation}\label{eq:def_omega}
    \tr \prn*{ \V^{-1} \bm{Q}^+ \prn{\V^{-1}}^\top } \leq \omega.
    \nonumber
\end{equation} 
\end{assumption}
The first condition is true in the examples in \cref{subsec:self}, assuming that the number of correct labels is fixed in multilabel classification.
The second one is satisfied if $\yy$ consists of $d$ linearly independent vectors or $\V$ is proportional to the identity matrix; either is true in those examples. 
It is also not difficult to derive bounds on $\omega$ in those examples.\looseness=-1

\paragraph{Algorithm based on pseudo-inverse matrix estimator}
As in the case of \cref{subsec:Bandit_Structured_Prediction_with_General_Losses}, we begin by providing a method to estimate the gradient.  
Define $\pt \coloneqq\E_{\bmy\sim p_t}[\bmy\bmy^\top]$.
Then, we define the estimator $\ytilde$ of $\yt$ by
\[
    \ytilde\coloneqq\inverse{\V}\bm{P}_t^+\yht\inpr{\yht,\V\yt},
\]
where $\bm{P}_t^+$ is the Moore--Penrose pseudo-inverse matrix of $\bm{P}_t$.
It is important to note that, given that $\bm{b}$ and $c(\cdot)$ are known,  
$
\inpr{\yht,\V\yt}=L_t(\yht)-\inpr{\yht,\bm{b}}-c(\yt)
$
can be computed.  
Note that $\ytilde$ is unbiased, i.e.,
$
\expect{\ytilde}=\yt
$
from the first requirement of \cref{asp:self}.

Using this $\ytilde$, we define the gradient estimator $\gtil$ by
\begin{equation}
    \label{eq:gtil_self}
    \gtil\coloneqq\prn*{\yho(\tht)-\ytilde}\xt^\top,
\end{equation}
whose expectation is 
$
    \E\brk[\big]{\gtil}=\G_t.
$
Our estimator is based on the estimators used in adversarial linear bandits and adversarial combinatorial full-bandits \citep{dani07price,abernethy08competing,comband}.  

We use RDUE with $q=\prn*{\frac{4 \omega B^2\dix ^2}{ T }}^{1/3}$ as the decoding function  
(assuming $T \geq 4 \omega B^2\dix ^2$ for simplicity).  
For updating $\wt$, we employ the adaptive OGD in \cref{subsec:ogd} as $\alg$ with the learning rate of 
$
\eta_t=\frac{B}{\sqrt{2 \sum_{i=1}^t\nrm{\tilde{\G}_i}_{\mathrm{F}}^2}}.
$


\paragraph{Regret bounds}
The above algorithm achieves the following regret bound,  
which does not directly depend on $\K$:
\begin{theorem}
    \label{thm:bandit_regret_pseudo_estimator}
    The above algorithm achieves
    $
    \E\brk{\reg}
    \leq
    \frac{bB^2}{a}
    +
    O\prn[\big]{ \omega^{1/3} \prn*{ B \dix T}^{2/3} }.
    $
\end{theorem}
The proof can be found in \cref{app:sub_bandit_regret_pseudo_estimator}.  
By using the estimator based on the pseudo-inverse matrix, 
we can upper bound the second moment of the gradient estimator $\gtil$ without $\K$, which allows us to establish the improved regret bound that does not explicitly depend on $\K$.  

The regret bound in \cref{thm:bandit_regret_pseudo_estimator} yields the different bounds on each problem setup as follows:
\begin{corollary}\label{cor:thm_self}
The above algorithm achieves
$
    \E\brk{\reg}\leq\frac{bB^2}{a}+ O\prn*{ \prn{B \dix  d T}^{2/3}}
$
in multiclass classification with the 0-1 loss,
$
    \E\brk{\reg}\leq\frac{bB^2}{a}
    +
    O\prn*{ \prn{d^5 /m(d-m)}^{1/3} \prn{B \dix T}^{2/3}}
$
in multilabel classification with $m$ correct labels and the Hamming loss,
and 
$
    \E\brk{\reg}\leq\frac{bB^2}{a}+ O \prn*{ m^{5/3}\prn{B\dix T}^{2/3}}
$
in ranking with the number of items $m$ and the Hamming loss.
\end{corollary}
The proof of \cref{cor:thm_self} is deferred to \cref{app:SELF_upper_discussion_deferred}.
The bound for multilabel classification with $m$ correct labels significantly improved the $O(\sqrt{\K T})$ bound in \cref{subsec:Bandit_Structured_Prediction_with_General_Losses},  
which has a dependency of $\sqrt{\binom{d}{m}}$,
and
the bound for ranking significantly improved the $O(\sqrt{\K T})$ bound in \cref{subsec:Bandit_Structured_Prediction_with_General_Losses}, which has a dependency of $\sqrt{m!}$.\looseness=-1

\section{Delayed Full-Information Feedback}
\label{sec:delay}
This section discusses online structured prediction with delayed full-information feedback and provides an algorithm that achieves a surrogate regret bound of $O(D^{2/3} T^{1/3})$, a better bound than $O(\sqrt{D T})$ that can be achieved with a standard OCO algorithm under delayed feedback \citep{joulani13online}. 
Below, we make the following assumption.\looseness=-1
\begin{assumption}
\label{asp:delayed_a}
There exists a constant $a\in\prn{0,1}$ which satisfies
\[
    \expect{L_t(\yht)}\leq(1-a)\sw.
\]
\end{assumption}
From \Cref{lem:expected_target_bound}, if $\lambda>\frac{4\gamma}{\nu}$, this condition is satisfied with $a=1-\frac{4\gamma}{\lambda\nu}$ by using the randomized decoding. 
We suppose that such a decoding function is used in this section.

\paragraph{Algorithm}
For updating $\wt$, we employ the Optimistic Delayed Adaptive FTRL (ODAFTRL) algorithm proposed by \citet{pmlr-v139-flaspohler21a}.  
In ODAFTRL, given a gradient~$\bm{G}_t$ at round $t$, $\W_t$ is updated as
\begin{equation}
    \W_{t+1}
    =
    \argmin_{\W\in \ww} \set*{ \sum_{i=1}^{t-D} \inpr*{\bm{G}_i,\W}+\frac{\lambda_t \nrm{\W}_{\F}^2}{2} },
    \nonumber
\end{equation}
where $\lambda_t\geq0$ is the regularization parameter.  
Due to space constraints, the details of the algorithm are provided in \cref{app:sub_odaftrl}.
By updating $\lambda_t$ using an AdaHedge-type algorithm called AdaHedgeD,  
ODAFTRL achieves the following regret upper bound:
\begin{lemma}[{Informal version of \citealt[Theorem 12]{pmlr-v139-flaspohler21a}}]\label{lem:ODAFTRL_bound}
Consider the setting with delayed full-information feedback.
Then, for any $\U\in\ww$, ODAFTRL with the AdaHedgeD update of $\lambda_t$ achieves
$
    \sumt{(\sw\!-\!\su)}
    \leq
    \sumt{\inpr{\G_t, \W_t - \U}}
    =
    O\prn[\Big]{\sqrt{\sumt{(\nrm{\G_t}_\F^2\!+\!D\nrm{\G_t}_\F)}}}.
$
\end{lemma}

\paragraph{Regret bounds and analysis}
The algorithm described above achieves the following bound:
\begin{theorem}
    \label{thm:delayed_regret_expectation_abstract}
    The above algorithm achieves
    $
        \E\brk{\reg}=O(D^{2/3}T^{1/3}).
    $
\end{theorem}
Here, we provide a proof sketch; the complete proof can be found in \cref{app:sub_delayed_regret_expectation_abstract}.
\begin{proof}[Proof sketch]
    \Cref{lem:ODAFTRL_bound} with $\nrm{\G_t}_\F^2\leq b\sw$ in \eqref{eq:St_smooth} and Cauchy--Schwarz yields
    $\sumt{(\sw-\su)}=O(\sqrt{S_{1:T}}+\prn{D^2TS_{1:T}}^{1/4})$, where $S_{1:T}=\sumt{\sw}$.
    Hence, from \cref{asp:delayed_a}, we have
    $
        \E\brk{\reg}\leq \sumt{(\sw-\su)}-a\sumt{\sw}
        =O(\sqrt{S_{1:T}}+\prn{D^2TS_{1:T}}^{1/4})-aS_{1:T}
        =O\prn{D^{2/3}T^{1/3}},
    $
    where we used $c_1\sqrt{x}-c_2x\leq{c_1^2}/\prn{4c_2}$ and $c_1x-c_2x^4\leq\prn*{{c_1^4}/\prn{4c_2}}^{1/3}$ for $x\geq0$, $c_1\geq 0$, and $c_2>0$.
\end{proof}
We can also prove the following high-probability bound; see \cref{app:sub_delayed_regret_probability_abstract} for the proof:
\begin{theorem}
    \label{thm:delayed_regret_probability_abstract}
    For any $\delta \in (0,1)$,
    with probability at least $1 - \delta$, the above algorithm achieves 
    $
        \reg = O\prn{\log({1}/{\delta}) + D^{2/3}T^{1/3}}.
    $
\end{theorem}


\section{Delayed Bandit Feedback}\label{sec:bandit_and_delayed}
Given the results so far, it is natural to explore online structured prediction with delayed bandit feedback.
We construct algorithms for this setup by combining the theoretical developments from \cref{sec:bandit} and \cref{sec:delay}. 
This section assumes \cref{asp:bandit_a}, as in \cref{sec:bandit}.
\subsection{$O(\sqrt{D K T})$ Regret Algorithm}\label{subsec:bandit_delay_general}
We use RDUE with $q=B\sqrt{DK/T}$ as the decoding function (assuming $T\geq DB^2K$ for simplicity), the gradient estimator $\gtil$ in \eqref{eq:inverse_weighted_est},
and ODAFTRL with the AdaHedgeD update as $\alg$. 
This algorithm attains the following bound:
\begin{theorem}
    \label{thm:delay_bandit_bound_general_abstract}
    The above algorithm achieves
    $
        \E\brk{\reg} = O\prn{\sqrt{DKT}}.
    $
\end{theorem}
The proof can be found in \cref{app:bandit_delayed_general}.  
Due to the introduction of the delay, the regret bound worsens by a factor of $\sqrt{D}$ compared to the non-delayed case,
which is natural when considering analyses of adversarial cases in delayed feedback \citep{joulani13online,zimmert20optimal,manwani2022delaytronefficientlearningmulticlass}.



\subsection{$O(D^{1/3} T^{2/3})$ Regret Algorithm}\label{subsec:bandit_delay_self}
Here, we make the same assumptions on the target loss function as in \cref{subsec:Bandit_Structured_Prediction_with_SELF}.
We provide an algorithm that improves the dependence on $\K$ from \cref{subsec:bandit_delay_general}.
We use RDUE with $q=\prn*{\frac{\omega B^2 \dix^2 D}{T}}^{1/3}$ as the decoding function (assuming $T \geq \omega B^2 \dix^2 D$ for simplicity), the gradient estimator $\gtil$ in \eqref{eq:gtil_self}, and ODAFTRL with the AdaHedgeD update as $\alg$.
This algorithm achieves the following bound:
\begin{theorem}
    \label{thm:delay_bandit_bound_self_abstract}
    The above algorithm achieves
    $
        \E\brk{\reg} = O\prn{
            D^{1/3} T^{2/3}
        }.
    $
\end{theorem}
The proof can be found in \cref{app:bandit_delayed_self}.  
Due to the introduction of the delay, the regret bound worsens by a factor of $D^{1/3}$  
compared to the non-delayed bandit feedback case.


\section{Experiments}
\label{sec: experiment}
\begin{figure}[t]
    \centering
    \includegraphics[keepaspectratio, scale=0.3]{experiment_box_MNISTtheo_B10_rep20_for_arxiv.eps}
    \vspace{-10pt}
     \caption{A box plot of error rate of the MNIST experiment for multiclass classification with bandit feedback.
     }
    \label{fig:experiment mnist}
\end{figure}

This section presents numerical experiment results for online multiclass classification  
under bandit feedback.  
We compare three algorithms:  
Gaptron~\Citep{NEURIPS2020_Hoeven} with logistic loss and hinge loss as surrogate losses,  
Gappletron~\Citep{NEURIPS2021_Hoeven} with logistic loss as the surrogate loss,  
and our proposed algorithm from \cref{subsec:bandit_delay_general}.  
The parameters for each algorithm are set based on their theoretical values.  
We use the MNIST dataset~\citep{lecun2010mnist}, a dataset of digit images.  
The diameter $B$ of $\ww$ is fixed at $10$.  
We repeated the experiment for 20 times, and the boxplot of the obtained misclassification rates is summarized in \cref{fig:experiment mnist}.  
From \cref{fig:experiment mnist}, we observe that our method achieves the lowest misclassification rate.  
Despite not being specialized for multiclass classification,  
our approach outperforms existing algorithms designed for multiclass classification  
on real data with $K = 10$.  
Further experiments can be found in \cref{app: experiment},  
and related discussions are provided in \cref{app:Discussio_on_the_Difference_in_Surrogate_Losses}.\looseness=-1













\vspace{-5pt}
\section{Conclusion}
This paper presents a novel unified framework, UniBrain, the first end-to-end model to jointly perform a diverse set of brain imaging analysis tasks, including extraction, registration, segmentation, parcellation, network generation and classification. UniBrain integrates heterogeneous information into a single system, enabling efficient knowledge transfer across different modules, and avoiding the need for extensive task-specific labels. Experimental results show that UniBrain outperforms state-of-the-art methods in all tasks while also demonstrating robustness and time efficiency.

\section*{Acknowledgments}
TT is supported by JST ACT-X Grant Number JPMJAX210E and JSPS KAKENHI Grant Number JP24K23852,
SS is supported by supported by JST ERATO Grant Number JPMJER1903,
and KY is supported by JSPS KAKENHI Grant Number JP24H00703.

\bibliography{bib_list}
% \bibliographystyle{plainnat}
\bibliographystyle{icml2025}

\newpage
\appendix
%

\section{Appendix Notation and Definitions}
We often use the shorthand $(a)_+ \defeq \max(a,0)$ as well as the shorthand $\k(\xset,\xset)$ to represent the matrix $(\k(\x_i,\x_j))_{i,j=1}^n$. 
In addition, for each kernel $\k$, we let $\rkhs$ and $\knorm{\cdot}$ represent the associated reproducing kernel Hilbert space (RKHS) and RKHS norm, so that $\ball_{\kernel}=\{ f\in\rkhs : \knorm{f} \leq 1\}$ and define
\begin{talign}
(\Pin - \Qout)\k \defeq \frac{1}{\nin}\sum_{x\in\xin} \k(\x,\cdot) - \frac{1}{\nout}\sum_{x\in\xout} \k(\x,\cdot).
\end{talign}
%
We also relate our definition of a sub-Gaussian thinning algorithm (\cref{def:alg-subg}) to several useful notions of sub-Gaussianity.
%
\begin{definition}[\tbf{Sub-Gaussian vector}]\label{def:vector-subg}
We say that a random vector $\diff \in \R^n$ is \emph{$(\K,\subg)$-sub-Gaussian on an event $\event$} if $\K$ is SPSD and $\subg>0$ satisfies 
\begin{talign}\label{eq:vector-subg}
    \Esubarg{\event}{\exp(\bu^\top \K \diff)} \leq \exp(\frac{\subg^2}{2} \cdot \bu^\top \K \bu)
    \qtext{for all}
    \bu \in \reals^n.
\end{talign}
If, in addition, the event has probability $1$, we say that $\w$ is \emph{$(\K,\subg)$-sub-Gaussian}.
\end{definition}
%
Notably, a thinning algorithm is $(\K,\subg,\delta)$-sub-Gaussian if and only if its associated vector $\pin-\qout$ is $(\K,\subg)$-sub-Gaussian on an event $\event$ of probability at least $1-\delta/2$.

%
\begin{definition}[\tbf{Sub-Gaussian function}]\label{def:function-subg}
For a kernel $\kernel$, %
we say that a random function $\fsubg\in \rkhs$ is \emph{$(\kernel,\subg)$-sub-Gaussian on an event $\event$} if $\subg > 0$ satisfies
\begin{talign}\label{eq:function-subg}
    \Esubarg{\event}{\exp(\inner{f}{\fsubg}_{\kernel})} \leq \exp(\frac{\subg^2}{2}\cdot \knorm{f}^2)
    \qtext{for all}
    f \in\rkhs.
\end{talign}
If, in addition, the event has probability $1$, we say that $\fsubg$ is \emph{$(\kernel,\subg)$-sub-Gaussian}.
\end{definition}
Our next two lemmas show that for finitely-supported signed measures like $\Pin-\Qout$, this notion of functional sub-Gaussianity is equivalent to the prior notion of vector sub-Gaussianity, allowing us to use the two notions interchangeably. 
%
%
Hereafter, we say that $\k$ generates a SPSD matrix $\K$ if $\k(\xset,\xset) = \K$. 

\begin{lemma}[\tbf{Functional sub-Gaussianity implies vector sub-Gaussianity}]
\label{lem:funct_subg_vector_subg}
In the notation of \cref{def:alg-subg}, if $(\Pin - \Qout)\kernel$ is $(\kernel,\subg)$-sub-Gaussian on an event $\event$ and $\kernel$ generates $\K$, then the vector $\pin - \qout$ is $(\K,\subg)$-sub-Gaussian on $\event$.
%
\end{lemma}
%

%
%
%
%
%
%

%
%
%
%
%
%
%
%
%
%
\begin{proof}
%
Suppose $(\Pin - \Qout)\kernel$ is $(\kernel,\subg)$-sub-Gaussian on an event $\event$, fix a vector $\bu\in \reals^n$, and define the function
\begin{talign}
    f_{\bu} \defeq \sumn u_i \kernel(\cdot, x_i) \in \rkhs.
\end{talign}
By the reproducing property, 
\begin{talign}\label{eq:hnorm-of-fu}
    \bu^\top \K (\pin -\qout) = \inner{f_{\bu}}{(\Pin-\Qout)\kernel}_{\kernel} \qtext{and} \knorm{f_{\bu}}^2 = \bu^\top \K \bu.
\end{talign}
Invoking the representations \cref{eq:hnorm-of-fu} and the functional sub-Gaussianity condition \cref{eq:function-subg} we therefore obtain
\begin{talign}
    \Esubarg{\event}{\exp(\bu^\top \K(\pin-\qout)} &= \Esubarg{\event}{\exp(\inner{f_{\bu}}{(\Pin-\Qout)\kernel}_{\kernel})} 
    \leq \exp(\knorm{f_{\bu}}^2 \cdot \frac{\subg^2}{2}) 
    = \exp(\bu^\top \K \bu \cdot \frac{\subg^2}{2}),
\end{talign}
so that $\pin-\qout$ is $(\K,\subg)$-sub-Gaussian on the event $\event$ as claimed.
\end{proof}

%

\begin{lemma}[\tbf{Vector sub-Gaussianity implies functional sub-Gaussianity}]
\label{lem:vector_subg_funct_subg}
In the notation of \cref{def:alg-subg}, if $\pin - \qout$ is $(\K,\subg)$-sub-Gaussian on an event $\event$ and $\kernel$ generates $\K$, then $(\Pin - \Qout)\kernel$ is $(\kernel,\subg)$-sub-Gaussian on $\event$.
\end{lemma}
\begin{proof}
Suppose $\pin-\qout$ is $(\K,\subg)$-sub-Gaussian on an event $\event$, fix a function $f\in \rkhs$, and consider the set 
\begin{talign}
\Lset \defeq \braces{f_{\bu} \defeq \sum_{i=1}^n u_i \kernel(\cdot,x_i) : \bu \in \reals^n}.
\end{talign} 
Since $\Lset$ is a closed linear subspace of $\rkhs$, we can decompose $f$ as $f  = f_{\bu} + f_\perp$,
where $\bu\in\Rn$ and $f_\perp$ is orthogonal to $\Lset$ \citep[Theorem 12.4]{rudin1991functional},
%
so that 
\begin{talign}\label{eq:knorm-decomposition}
    \knorm{f}^2 = \knorm{f_{\bu}}^2 + \knorm{f_\perp}^2\qtext{and} \knorm{f_{\bu}}^2 = \bu^\top \K \bu.
\end{talign}
Invoking the orthogonality of $f_\perp$ and $(\Pin - \Qout)\kernel\in \Lset$, the reproducing property representations \cref{eq:hnorm-of-fu}, and the vector sub-Gaussianity condition \cref{eq:vector-subg}, we find that
\begin{talign}
    \Esubarg{\event}{\exp(\inner{f}{(\Pin-\Qout)\kernel}_{\kernel})} 
    &= \Esubarg{\event}{\exp(\inner{f_{\bu} + f_\perp}{(\Pin - \Qout) \kernel}_{\kernel})} 
    = \Esubarg{\event}{\exp(\bu^\top \K (\pin - \qout)})\\
    &\leq \exp(\bu^\top \K \bu \cdot \frac{\subg^2}{2}) 
    \sless{\cref{{eq:knorm-decomposition}}} \exp(\knorm{f}^2 \cdot \frac{\subg^2}{2}),
\end{talign}
so that $(\Pin-\Qout)\kernel$ is $(\kernel,\subg)$-sub-Gaussian on the event $\event$ as claimed.
\end{proof}


We end our discussion about the versions of sub-Gaussianity considered above by presenting the standard fact about the additivity of sub-Gaussianity parameters under summation of independent sub-Gaussian random vectors, adapted to our setting.

\begin{lemma}[\tbf{Vector sub-Gaussian additivity}]\label{lem:K_sub_gsn_additivity}
    Suppose that, for each $j\in [m]$, 
    $\Delta_j\in\reals^n$ is $(\mbf K,\subg_j)$ on an event $\event[j]$ given $\Delta_{1:(j-1)}\defeq (\Delta_1,\ldots,\Delta_{j-1})$ and $\event[\leq j-1]\defeq \bigcap_{i=1}^{j-1}\event[i]$. 
    Then $\sum_{j=1}^m \Delta_j$ is $(\mbf K, (\sum_{j=1}^m \subg_j^2)^{1/2})$-sub-Gaussian on $\event[\leq m]$.
    %
    \end{lemma}
    \begin{proof}
    Let $\event[\leq s] = \bigcap_{j=1}^s\event[j]$ for each $s\in [m]$.
    We prove the result for $\mc Z_s = \sum_{i=1}^s \Delta_j$ by induction on $s\in [m]$. 
    The result holds for the base case of $s=1$ by assumption. For the inductive case, suppose the result holds for $s\in [m-1]$. Fixing $\bu\in \R^n$, we may apply the tower property, our conditional sub-Gaussianity assumption, and our inductive hypothesis in turn to conclude
    \begin{talign}
        \Earg{\exp(\inner{\bu}{\K \sum_{j=1}^{s+1} \Delta_j})\indic{\event[\leq s+1]}} &= \Earg{\exp(\inner{\bu}{\K \sum_{j=1}^{s} \Delta_j})\indic{\event[\leq s]} \Earg{\exp(\inner{\bu}{\Delta_{s+1}})\indic{\event[s+1]} \mid \Delta_{1:s},\event[\leq s]} } \\
        &\leq \Earg{\exp(\inner{\bu}{\K \sum_{j=1}^{s} \Delta_j})\indic{\event[\leq s]}} \exp\parenth{\frac{\subg_{s+1}^2}{2}\cdot \bu^\top \K \bu}
        \leq \exp\big( \frac{\sum_{j=1}^{s+1} \subg_j^2}{2} \cdot \bu^\top \K \bu\big).
    \end{talign}
    Hence, $\mc Z_{s+1}$ is $(\K,(\sum_{j=1}^{s+1} \subg_j^2)^{1/2})$-sub-Gaussian on $\event[\leq s+1]$, and the proof is complete.
    \end{proof}



%



%

%
%
%
%
%
%
%
%
%
%
%
%
%
%
%
%



\section{Additional Related Work}\label{app:additional_related_work}
We discuss additional related work that could not be included above.

\paragraph{Structured prediction}
Before the introduction of the Fenchel--Young loss framework, \citet{Niculae18sparse} proposed SparseMAP, which used the squared $\ell_2$-norm regularization.
The Fenchel--Young loss, described in \cref{subsec:fenchel-young}, is built upon the idea of SparseMAP. 
The Structure Encoding Loss Function (SELF) was introduced by \citet{ciliberto16consistent,ciliberto20general} to analyze the relationship between surrogate and target losses, a concept known as Fisher consistency.
For more extensive literature, we refer the reader to \citet[Appendix A]{pmlr-v247-sakaue24a}.

\paragraph{Online classification with full and bandit feedback}
In the full information setup, PERCEPTRON is one of the most representative algorithms for binary classification \citep{Rosenblatt1958-sh}, and the multiclass setting has also been extensively studied \citep{,crammer2003ultraconservative,Fink_2006}.
Online logistic regression is another relevant research stream, with \citet{foster18logistic} being a particularly representative study. 
The study of the bandit setup was initiated by \citet{Kakade2008EfficientBA}, and it has since been extensively explored in subsequent research \citep{hazan11newtron,beygelzimer17efficient,foster18logistic}. However, to the best of our knowledge, no prior work has addressed general structured prediction under bandit feedback. A most related study is the work by \citet{gentile14multilabel}, which investigated online multilabel classification and ranking. 
However, their setting assumes access to feedback of the form $\set{\ind[\bmy_{t,i} \neq \hat{\bmy}_{t,i}]}_{i}$, which is more informative than bandit feedback and differs from our setup.
\Citet{NEURIPS2020_Hoeven} explicitly introduced the surrogate regret in the context of online multiclass classification. This study has been extended to the setting where observations are determined by a directed graph \Citep{NEURIPS2021_Hoeven} and to structured prediction scenarios \citep{pmlr-v247-sakaue24a}. For a more extensive overview of the literature on online classification, we refer the reader to \Citet{NEURIPS2020_Hoeven}.


\paragraph{Delayed feedback}
The study of delayed feedback began with \citet{Weinberger_2002_delay}. 
Since then, it has been extensively explored in various online learning settings, primarily in the full information setup of online convex optimization \citep{Mesterharm05online,joulani13online,joulani16delay,pmlr-v139-flaspohler21a}. 
Algorithms for delayed bandit feedback have been studied mainly in the context of multi-armed bandits and their variants \citep{cesabianchi16delay,zimmert20optimal,ito20delay,masoudian22best,hoeven23unified}. In the context of online classification, research considering delay is scarce; the only work is \citet{manwani2022delaytronefficientlearningmulticlass} to our knowledge.



\section{Discussion on the Difference in Surrogate Loss Functions}\label{app:Discussio_on_the_Difference_in_Surrogate_Losses}

As in \eqref{eq:sur_regret}, the surrogate regret $\reg$ is defined by $\sumt{L(\yht;\yt)}=\sumt{S(\U\xt;\yt)}+\reg$, which means the choice of the surrogate loss $S$ affects the bound on the cumulative loss $\sumt{L(\yht;\yt)}$.
\Citet[Theorem~1]{NEURIPS2021_Hoeven}, which applies to a more general setup than bandit feedback, implies $\reg = O(K\sqrt{T})$ for the bandit setup with $S$ being a logistic loss defined with the base-$K$ logarithm. 
On the other hand, our bound of $\reg = O(\sqrt{KT})$ applies to the logistic loss $S$ defined with the base-$2$ logarithm. 
As a result, while our bound on $\reg$ is better, the $\sumt{S(\U\xt;\yt)}$ term can be worse; this is why we cannot directly compare our $O(\sqrt{KT})$ bound with the $O(K\sqrt{T})$ bound in \Citet[Theorem~1]{NEURIPS2021_Hoeven}. 
We may use the decoding procedure in \citet{NEURIPS2021_Hoeven}, instead of RDUE, to recover their bound that applies to the base-$K$ logistic loss.
It should be noted that their method is specific to multiclass classification;  
naively extending their method to structured prediction formulated as $|\yy|$-class classification results in the undesirable dependence on $K = |\yy|$, as is also discussed in \citet{pmlr-v247-sakaue24a}. 
By contrast, our pseudo-inverse estimator, combined with RDUE, can rule out the explicit dependence on $K$, at the cost of the increase from $\sqrt{T}$ to $T^{2/3}$.
\section{Omitted Details of \cref{sec:bandit}}
\label{app:proof bandit}
This section provides the omitted details of \cref{sec:bandit}.

\subsection{Concentration inequality}
To prove high probability regret bounds, we use the following concentration inequality.
\begin{lemma}[{Bernstein's inequality, e.g., \citealt[Lemma A.8]{Cesa-Bianchi_Lugosi_2006}}]
    \label{lem:Bernstein}
    Let $Z_1,\hdots,Z_T$ be a martingale difference sequence and $\delta \in (0,1)$.
    If there exist $a$ and $v$ which satisfy $|Z_t|\leq a$ for any $t \in \brk{T}$ and $\sumt{\expect{Z_t^2}}\leq v$ , then with probability at least $1-\delta$, it holds that
    \[
        \sumt{Z_t}\leq\sqrt{2v\log\frac{1}{\delta}}+\frac{\sqrt{2}}{3}a\log\frac{1}{\delta}.
    \]
\end{lemma}


\subsection{Proof of \cref{thm:bandit_high_prob}}\label{app:proof_bandit_high_prob}
Here, we provide the proof of \cref{thm:bandit_high_prob}.
Hereafter, we let $S_{\max} = \max_{\W \in \ww} S_t(\W)$ and $\hat{S}_t(\W) = v_t S_t(\W) = \frac{\ind\brk{\yt = \yht}}{p_t(\yht)} S_t(\W)$.
The following theorem is the formal version of \cref{thm:bandit_high_prob}:
\begin{theorem}[Formal version of \cref{thm:bandit_high_prob}]\label{thm:bandit_high_prob_formal}
Consider the bandit and non-delayed setup.
Let 
\begin{equation}
    \mathcal{C}
    =
    \prn*{
        \frac{3}{2 (a + \xi - 1)}  
        +
        1
    }
    K \Smax \log(2/\delta) 
    +
    \frac{B^2 K b}{2 (1 - \xi)}
    .
    \nonumber
\end{equation}

Then, for any $T \geq \mathcal{C}$ and $\delta \in (0,1/2)$, with probability at least $1-\delta$, the algorithm in \cref{subsec:Bandit_Structured_Prediction_with_General_Losses} with $q = \sqrt{\mathcal{C} / T}$ achieves
\begin{equation}
    \mathcal{R}_T
    \leq
    2
    \sqrt{
        \mathcal{C}
        T
    }
    +
    \sqrt{2 \log (2/ \delta)} \prn{\mathcal{C} T}^{1/4}
    +
    \prn*{ \frac{1-a}{2 (a + \xi - 1)} + 2 } \log (2/\delta).
    \nonumber
\end{equation}
\end{theorem}


Before proving this theorem, we provide the following lemma:
\begin{lemma}\label{lem:hp_pre}
It holds that 
\begin{equation}
    \sum_{t=1}^T \prn*{ \E_t\brk*{L_t(\yht)} - \hat{S}_t(\U) }
    \leq
    \sum_{t=1}^T \prn*{ (1-a) S_t(\W_t) - \hat{S}_t(\W_t)  } + q T
    +
    \sqrt{2} B \sqrt{\frac{b}{q} \sum_{t=1}^T v_t S_t(\W_t) }
    .
    \nonumber
\end{equation}
\end{lemma}
\begin{proof}
We have 
\begin{equation}
    \sum_{t=1}^T \prn*{ \E_t\brk*{L_t(\yht)} - \hat{S}_t(\U) }
    =
    \sum_{t=1}^T \prn*{ \E_t\brk*{L_t(\yht)} - \hat{S}_t(\W_t)  }
    +
    \sum_{t=1}^T \prn*{ \hat{S}_t(\W_t) - \hat{S}_t(\U) }.
    \nonumber
\end{equation}
From \cref{asp:bandit_a}, the first term is bounded as 
\begin{align}
    \sum_{t=1}^T \prn*{ \E_t\brk*{L_t(\yht)} - \hat{S}_t(\W_t)  }
    \leq
    \sum_{t=1}^T \prn*{ (1-a) S_t(\W_t) - \hat{S}_t(\W_t)  } + q T,
    \nonumber
\end{align}
and 
the second term is bounded as 
\begin{align}
    \sum_{t=1}^T \prn*{ \hat{S}_t(\W_t) - \hat{S}_t(\U) }
    &\leq
    \sqrt{2} B \sqrt{\sum_{t=1}^T \nrm{\tilde{\G}_t}_{\F}^2 }
    =
    \sqrt{2} B \sqrt{\sum_{t=1}^T v_t^2 \nrm{\G_t}_{\F}^2 }
    \nonumber \\
    &\leq
    \sqrt{2} B \sqrt{b \sum_{t=1}^T v_t^2 S_t(\W_t) }
    \leq
    \sqrt{2} B \sqrt{\frac{b K}{q} \sum_{t=1}^T v_t S_t(\W_t) },
    \nonumber
\end{align}
where we used \cref{lem:ogd} and $v_t \leq K / q$.
Combining the above three, we obtain
\begin{equation}
    \sum_{t=1}^T \prn*{ \E_t\brk*{L_t(\yht)} - \hat{S}_t(\U) }
    \leq
    \sum_{t=1}^T \prn*{ (1-a) S_t(\W_t) - \hat{S}_t(\W_t)  } + q T
    +
    \sqrt{2} B \sqrt{\frac{b K}{q} \sum_{t=1}^T v_t S_t(\W_t) }
    ,
    \nonumber
\end{equation}
which completes the proof.
\end{proof}


\begin{proof}[Proof of \cref{thm:bandit_high_prob_formal}]
The surrogate regret can be decomposed as
\begin{equation}\label{eq:reg_dec_highp}
    \mathcal{R}_T 
    =
    \sum_{t=1}^T \prn*{ L_t(\yht) - \E_t\brk*{ L_t(\yht)} }
    +
    \sum_{t=1}^T \prn*{ \E_t\brk*{ L_t(\yht)} - S_t(\U) }
    .
\end{equation}
We first upper bound the first term in \eqref{eq:reg_dec_highp}.
Let $Z_t = L_t(\yht) - \E_t\brk*{ L_t(\yht)}$ for simplicity.
Then, we have $Z_t \leq 1$, $\E_t\brk*{Z_t} = 0$, and
$\E_t\brk*{Z_t^2} 
\leq 
\E_t\brk*{ \prn{L_t(\yht)}^2 }
\leq 
(1-a) S_t(\W_t) + q.
$
Hence, from Bernstein's inequality in \cref{lem:Bernstein}, for any $\delta' \in (0,1)$, at least $1 - \delta'$ we have 
\begin{equation}\label{eq:conc_zt}
    \sum_{t=1}^T Z_t
    \leq 
    \sqrt{2 \log (1/\delta') \sum_{t=1}^T \prn*{(1-a) S_t(\W_t) + q} }
    +
    \frac{\sqrt{2}}{3} \log (1/\delta')
    .
\end{equation}
We next consider the second term in \eqref{eq:reg_dec_highp}.
Define $r_t = S_t(\U) - \xi S_t(\W_t)$ for some $\xi \in (0, 1)$, which will be determined later,
and let $v_t = \ind[ \yt = \yht ] / p_t(\yht) \leq K/q$ for simplicity.
Then, we have $\E_t\brk{v_t r_t - r_t} = 0$, $\abs{v_t r_t - r_t} \leq K S_{\max} / q$, and
\begin{equation}
    \E_t\brk{ (v_t r_t - r_t)^2}
    \leq
    \E_t\brk{(v_t r_t)^2}
    \leq 
    \frac{K \Smax}{q} \abs{r_t}
    \leq 
    \frac{K \Smax}{q} \prn*{S_t(\U) + S_t(\W_t)}
    .
    \nonumber
\end{equation}
Hence from Bernstein's inequality in \cref{lem:Bernstein}, for any $\delta'' \in (0,1)$, with probability at least $1 - \delta''$ we have 
\begin{equation}\label{eq:conc_vr}
    \sum_{t=1}^T \prn{v_t r_t - r_t} 
    \leq 
    \sqrt{3 \log (1/\delta'') \sum_{t=1}^T \frac{K \Smax}{q} \prn{S_t(\U) + S_t(\W_t)} }
    +
    \frac{\sqrt{2} K \Smax}{3q} \log(1/\delta'')
    .
\end{equation}


\textbf{When $\sum_{t=1}^T S_t(\U) \leq \sum_{t=1}^T S_t(\W_t)$.}
We first consider the case of $\sum_{t=1}^T S_t(\U) \leq \sum_{t=1}^T S_t(\W_t)$.
From \cref{lem:hp_pre}, we have
\begin{align}
    &\sum_{t=1}^T \E_t\brk*{L_t(\yht)} - q T
    \leq
    \sum_{t=1}^T v_t S_t(\U) 
    +
    \sum_{t=1}^T \prn*{ (1-a) S_t(\W_t) - v_t S_t(\W_t)  } 
    +
    \sqrt{2} B \sqrt{\frac{b K}{q} \sum_{t=1}^T v_t S_t(\W_t) }
    \nonumber \\
    &=
    \sum_{t=1}^T v_t \underbrace{\prn*{ S_t(\U) - \xi S_t(\W_t)  } }_{= r_t}
    -
    (1 - \xi) \sum_{t=1}^T v_t S_t(\W_t)
    +
    (1-a) \sum_{t=1}^T S_t(\W_t)
    +
    \sqrt{2} B \sqrt{\frac{b K}{q} \sum_{t=1}^T v_t S_t(\W_t) }
    \nonumber \\
    &\leq
    \sum_{t=1}^T v_t r_t
    +
    (1-a) \sum_{t=1}^T S_t(\W_t)
    +
    \frac{B^2 K b}{2 q (1 - \xi)},
    \nonumber
\end{align}
where the last inequality follows from 
$c_1\sqrt{x}-c_2x\leq{c_1^2}/\prn{4c_2}$ for $x \geq 0$, $c_1 \geq 0$, and $c_2 > 0$.
From the concentration result provided in \eqref{eq:conc_vr}, this is further bounded as
\begin{align}
    \sum_{t=1}^T \E_t\brk*{L_t(\yht)} - q T
    &\leq
    \sum_{t=1}^T (S_t(\U) - \xi S_t(\W_t))
    +
    \sqrt{3 \log (1/\delta'') \sum_{t=1}^T \frac{K \Smax}{q} \prn{S_t(\U) + S_t(\W_t)} }
    \nonumber \\
    &\qquad
    +
    \frac{\sqrt{2} K \Smax}{3q} \log(1/\delta'') 
    +
    (1-a) \sum_{t=1}^T S_t(\W_t)
    +
    \frac{B^2 K b}{2 q (1 - \xi)}
    ,
    \nonumber
\end{align}
where we recall that $r_t = S_t(\U) - \xi S_t(\W_t)$.
Rearranging the last inequality and using $\sum_{t=1}^T S_t(\U) \leq \sum_{t=1}^T S_t(\W_t)$ give
\begin{align}
    \sum_{t=1}^T \prn*{ \E_t\brk*{L_t(\yht)} - S_t(\U) } 
    &\leq
    q T 
    +
    \sqrt{6 \log (1/\delta'') \sum_{t=1}^T \frac{K \Smax}{q} S_t(\W_t) }
    +
    \frac{\sqrt{2} K \Smax}{3 q} \log(1/\delta'') 
    \nonumber \\
    &\qquad
    +
    (1 - a - \xi) \sum_{t=1}^T S_t(\W_t)
    +
    \frac{B^2 K b}{2 q (1 - \xi)}
    .
    \nonumber
\end{align}
In what follows, we let $\delta' = \delta'' = \delta / 2$ and $\xi = \frac{\prn{4 + c} \gamma}{\lambda \nu}$ for a sufficiently small constant $c > 0$, which satisfies $a + \xi > 1$.
Then, plugging \eqref{eq:conc_zt} and the last inequality in \eqref{eq:reg_dec_highp}, with probability at least $1 - \delta$, we obtain
\begin{align}
    \mathcal{R}_T
    &\leq
    \sqrt{2 \log (2/\delta) \sum_{t=1}^T \prn*{(1-a) S_t(\W_t) + q} }
    +
    \frac{\sqrt{2}}{3} \log (2/\delta)
    +
    q T 
    +
    \sqrt{6 \log (2/\delta) \sum_{t=1}^T \frac{K \Smax}{q} S_t(\W_t) }
    \nonumber \\
    &\qquad
    +
    \frac{\sqrt{2} K \Smax}{3 q} \log(2/\delta) 
    +
    (1 - a - \xi) \sum_{t=1}^T S_t(\W_t)
    +
    \frac{B^2 K b}{2 q (1 - \xi)}
    \nonumber \\
    &\leq
    \frac{1}{2 (a + \xi - 1)}
    \prn*{(1-a) + \frac{3 K \Smax}{q}} \log (2/\delta)
    +
    \sqrt{2 q T \log (2 / \delta)} 
    +
    \frac{\sqrt{2}}{3} \log (2/\delta)
    +
    q T 
    \nonumber \\
    &\qquad
    +
    \frac{\sqrt{2} K \Smax}{3 q} \log(2/\delta) 
    +
    \frac{B^2 b}{2 q (1 - \xi)}
    \nonumber \\
    &\leq
    \frac{1}{q}
    \prn*{
        \frac{3 K \Smax \log (2/\delta)}{2 (a + \xi - 1)}  
        +
        K \Smax \log(2/\delta) 
        +
        \frac{B^2 K b}{2 (1 - \xi)}
    }
    +
    q T 
    +
    \sqrt{2 q T \log (2 / \delta)} 
    \nonumber \\
    &\qquad
    +
    \frac{1}{2 (a + \xi - 1)} (1-a) \log (2/\delta)
    +
    \frac{\sqrt{2}}{3} \log (2/\delta)
    \nonumber \\
    &=
    \frac{\mathcal{C}}{q}
    +
    q T 
    +
    \sqrt{2 q T \log (2 / \delta)} 
    +
    \frac{1}{2 (a + \xi - 1)} (1-a) \log (2/\delta)
    +
    \frac{\sqrt{2}}{3} \log (2/\delta).
    \nonumber
\end{align}
Using the definition of $q = \sqrt{\mathcal{C} / T}$ with the last inequality,
we obtain
\begin{equation}
    \mathcal{R}_T
    \leq
    2
    \sqrt{
        \mathcal{C}
        T
    }
    +
    \prn{\mathcal{C} T}^{1/4} \sqrt{\log (2/ \delta)}
    +
    \prn*{ \frac{1-a}{2 (a + \xi - 1)} + \frac{\sqrt{2}}{3} } \log (2/\delta).
    \nonumber
\end{equation}

\textbf{When $\sum_{t=1}^T S_t(\U) > \sum_{t=1}^T S_t(\W_t)$.}
We next consider the case of $\sum_{t=1}^T S_t(\U) > \sum_{t=1}^T S_t(\W_t)$.
We have
\begin{align}
    \mathcal{R}_T 
    &=
    \sum_{t=1}^T \prn*{ L_t(\yht) - \E_t\brk*{ L_t(\yht)} }
    +
    \sum_{t=1}^T \prn*{ \E_t\brk*{ L_t(\yht)} - S_t(\U) }
    \nonumber \\
    &\leq
    \sqrt{2 \log (1/\delta') \sum_{t=1}^T \prn*{(1-a) S_t(\W_t) + q} }
    +
    \frac{\sqrt{2}}{3} \log (1/\delta')
    +
    \sum_{t=1}^T \prn*{ \E_t\brk*{ L_t(\yht)} - S_t(\W_t) }
    \nonumber \\
    &\leq
    \sqrt{2 \log (1/\delta') \sum_{t=1}^T \prn*{(1-a) S_t(\W_t) + q} }
    +
    \frac{\sqrt{2}}{3} \log (1/\delta')
    +
    \sum_{t=1}^T \prn*{ - a S_t(\W_t) + q }
    \nonumber \\
    &\leq
    \frac{(1 - a) \log (1/ \delta')}{ 2 a }
    +
    \sqrt{2 q T \log (1/\delta') }
    +
    \frac{\sqrt{2}}{3} \log (1/\delta')
    +
    q T
    ,
    \nonumber
\end{align}
where the first inequality follows from \eqref{eq:conc_zt} and $\sum_{t=1}^T S_t(\U) > \sum_{t=1}^T S_t(\W_t)$,
and the second inequality follows from \cref{asp:bandit_a},
the last inequality follows from $c_1\sqrt{x}-c_2x\leq{c_1^2}/\prn{4c_2}$ for $x \geq 0$, $c_1 \geq 0$, and $c_2 > 0$.
Substituting $q = \sqrt{\mathcal{C} / 2}$ and $\delta' = \delta/2$ and  in the last inequality, we obtain
\begin{equation}
    \mathcal{R}_T 
    \leq
    \frac{(1 - a) \log (2/ \delta)}{ 2 a }
    +
    \sqrt{2 \log (2/\delta) } \prn{ \mathcal{C} T }^{1/4}
    +
    \frac{\sqrt{2}}{3} \log (2/\delta)
    +
    \sqrt{\mathcal{C} T}
    .
    \nonumber
\end{equation}
This completes the proof.    
\end{proof}







\subsection{Proof of \cref{thm:bandit_regret_pseudo_estimator}}
\label{app:sub_bandit_regret_pseudo_estimator}


Here, we provide the proof of \cref{thm:bandit_regret_pseudo_estimator}.
We recall that $\pt=\expect{\yht\yht^\top}$.
We then estimate $\yt$ by $\ytilde=\inverse{\V}\Pplus_t\yht\inpr{\yht,\V\yt}$
and $\G_t$ by $\gtil\coloneqq(\yho(\tht)-\ytilde)\xt^\top$ under \cref{asp:self}.
This $\gtil$ satisfies 
$
    \expect{\gtil}=\G_t. 
$
To prove \cref{thm:bandit_regret_pseudo_estimator}, we will upper bound $\expect{\nrm{\gtil}_\F^2}$.
To do so, we begin by proving the following lemma:
\begin{lemma}\label{lem:pseudo_inverse_order}
    Let $\bm{A}$ and $\bm{B}$ positive semi-definite matrices with $\image(\bm{A}) = \image(\bm{B})$ with $\bm{A} \succeq \bm{B}$.
    Then, it holds that $\bm{A}^+ \preceq \bm{B}^+$.
\end{lemma}
\begin{proof}
Since $\image(\bm{A}) = \image(\bm{B})$, there exists an orthogonal matrix $\bm{R}$, a diagonal matrix $\bm{\Lambda}$, and an invertible matrix $\bm{B}'$ that has same dimensions as $\bm{\Lambda}$ such that 
\begin{equation}
    \bm{A}
    =
    \bm{R}
    \begin{pmatrix}
        O & O \\
        O & \bm{\Lambda} 
    \end{pmatrix}
    \bm{R}^\top
    \quad 
    \mbox{and}
    \quad
    \bm{B}
    =
    \bm{R}
    \begin{pmatrix}
        O & O \\
        O & \bm{B}'
    \end{pmatrix}
    \bm{R}^\top
    .
    \nonumber
\end{equation}
Then, 
\begin{equation}\label{eq:Aplus_Bplus}
    \bm{A}^+
    =
    \bm{R}
    \begin{pmatrix}
        O & O \\
        O & \bm{\Lambda}^{-1}
    \end{pmatrix}
    \bm{R}^\top
    \quad 
    \mbox{and}
    \quad
    \bm{B}^+
    =
    \bm{R}
    \begin{pmatrix}
        O & O \\
        O & {\bm{B}'}^{-1}
    \end{pmatrix}
    \bm{R}^\top
    .
\end{equation}
From $\bm{A} \succeq \bm{B}$,
we have $\bm{\Lambda} \succeq \bm{B}'$, which implies $\bm{\Lambda}^{-1} \preceq {\bm{B}'}^{-1}$.
From this and \eqref{eq:Aplus_Bplus}, we have $\bm{A}^+ \preceq \bm{B}^+$, as desired.
\end{proof}

Using this lemma we prove a property of $\pt$ and an upper bound of $\expect{\tr\prn*{\yht\yht^\top}}$.
In what follows, we use $\lambda_\min(\bm{A})$ to denote the minimum eigenvalue of a matrix $\bm{A}$.

\begin{lemma}
    \label{lem:bound of trace}
    Suppose that $\tr \prn*{ \V^{-1} \bm{Q} \prn{\V^{-1}}^\top } \leq \omega$ for $\bm{Q} = \E_{\bm{y} \sim \mu} \brk{ \bm{y} \bm{y}^\top }$, where we recall $\mu$ is the uniform distribution over $\yy$.
    Then, we have
    \[
    \expect{\tr(\ytt\ytt^\top)}\leq \frac{\omega}{q}.
    \]
\end{lemma}
\begin{proof}    
    By the linearity of expectation and the trace property, we have
    \begin{align*}
        \expect{\tr(\ytilde\ytilde^\top)}&\leq \tr\prn*{\inverse{\V}\Pplus_t\expect{\yht\yht^\top}\Pplus_t\prn*{\inverse{\V}}^\top}
        = \tr\prn*{\inverse{\V}\Pplus_t \bm{P}_t \Pplus_t \prn*{\inverse{\V}}^\top}\\
        &= 
        \tr\prn*{\inverse{\V}\Pplus_t\prn*{\inverse{\V}}^\top},
    \end{align*}
    where the first inequality follows from $\inpr{\yht,\V\yt} = L_t(\yht) - \inpr{\yht,\bm{b}} - c(\yt) \leq L_t(\yht) \leq 1$ since $\bm{b} \geq 0$ and $c(\cdot)$ is non-negative
    and
    the last equality follows from $\Pplus_t \bm{P}_t \Pplus_t = \Pplus_t$.
    Hence,
    \begin{align}
        \tr\prn*{\inverse{\V}\Pplus_t\prn*{\inverse{\V}}^\top}
        &=
        \sum_{i=1}^d
        \bm{\ee}_i^\top \inverse{\V}\Pplus_t\prn*{\inverse{\V}}^\top \bm{\ee}_i
        \leq
        \sum_{i=1}^d
        \bm{\ee}_i^\top \inverse{\V} \prn*{q \bm{Q}}^{+} \prn*{\inverse{\V}}^\top \bm{\ee}_i
        \nonumber \\
        &\leq
        \tr\prn*{\prn*{\inverse{\V}}^\top \inverse{\V} (q \bm{Q})^+ }
        =
        \frac{1}{q}
        \tr \prn*{ \inverse{\V} \bm{Q}^+ \prn{\inverse{\V}}^\top } 
        \leq
        \frac{\omega}{q},
        \nonumber
    \end{align}
    where in the first inequality we used \cref{lem:pseudo_inverse_order} and in the last inequality we used the assumption that $\tr \prn*{ \V^{-1} \bm{Q}^+ \prn{\V^{-1}}^\top } \leq \omega$.
    This completes the proof.
\end{proof}

Now, we are ready to upper bound $\expect{\nrm{\gtil}_\F^2}$.
\begin{lemma}
    \label{thm:evaluation of Gtilde}
    Under the same assumption as \cref{lem:bound of trace}, it holds that 
    \[
        \expect{\nrm{\gtil}_\F^2}\leq2b\sw+ \frac{2 \dix^2 \omega}{q}.
    \]
\end{lemma}
\begin{proof}
    We have 
    \begin{align*}
        \nrm{\gtil}_\F^2&=\nrm{\prn*{\yho(\tht)-\ytilde}\xt^\top}_{\mathrm{F}}^2\leq2\nrm{(\yho(\tht)-\yt)\xt^\top}_\F^2+2\nrm{(\yt-\ytilde)\xt^\top}_\F^2\\
        &\leq 2\nrm{\G_t}_\F^2+2\dix ^2\nrm{\yt-\ytilde}_2^2,
    \end{align*}
    where we recall $\dix =\diam(\xx)$.
    From this inequality, 
    \begin{align}
        \expect{\nrm{\gtil}_\F^2}&\leq2\nrm{\G_t}_{\mathrm{F}}^2+2\dix ^2\expect{\nrm{\yt-\ytilde}_2^2}
        \leq2b\sw+2\dix ^2\prn*{\nrm{\yt}_2^2-2\yt^\top\expect{\ytilde}+\expect{\nrm{\ytilde}_2^2}} \nonumber \\
        &=2b\sw+2\dix ^2\prn*{\nrm{\yt}_2^2-2\nrm{\yt}_2^2 + \expect{\nrm{\ytilde}_2^2}} \nonumber \\ 
        &\leq
        2b\sw
        +2\dix ^2\expect{\tr(\ytilde\ytilde^\top)}
        \leq
        2b\sw
        + \frac{2\dix^2 \omega}{q},
        \nonumber
    \end{align} 
    where in the second inequality we used $\nrm{\G_t}_\F^2 \leq b \sw$, in the equality we used $\expect{\ytilde}=\yt$,
    and in the last inequality we used \cref{lem:bound of trace}.
\end{proof}



Finally, we are ready to prove \cref{thm:bandit_regret_pseudo_estimator}.
\begin{proof}[Proof of \cref{thm:bandit_regret_pseudo_estimator}]
    From \cref{asp:bandit_a}, we have 
    \begin{align*}\label{eq:inverse_reg_expect}
        \E\brk{\reg}
        &\leq\E\brk*{\sumt{(\sw-\su)}}-a\E\brk*{\sumt{\sw}}+qT
        \nonumber \\
        &\leq\E\brk*{\sumt{\inpr*{\G_t,\wt-\U}}}-a\E\brk*{\sumt{\sw}}+qT.
    \end{align*}
    From \cref{thm:evaluation of Gtilde} and the unbiasedness of $\gtil$, 
    the first term in the last inequality is further bounded as
    \begin{align*}
        \E\brk*{\sumt{\inpr*{\G_t,\wt-\U}}}  
        &=\E\brk*{\sumt{\inpr*{\gtil,\wt-\U}}}
        \leq\sqrt{2}B\sqrt{\E\brk*{\sumt{\nrm{\gtil}_{\mathrm{F}}^2}}}
        \nonumber \\
        &\leq
        2 B \sqrt{b\E\brk*{\sumt{\sw}}}
        +
        2 B \dix \sqrt{ \omega / q},
    \end{align*}
    where 
    the first inequality follows from \cref{lem:ogd} and the last inequality follows from \cref{thm:evaluation of Gtilde} and the subadditivity of $x \mapsto \sqrt{x}$ for $x \geq 0$.
    Therefore, by combining  with the last inequality, we have 
    \begin{align}
        \E\brk{\reg}
        &\leq 
        2B \sqrt{b\E\brk*{\sumt{\sw}}} 
        +
        2 B \dix \sqrt{ \omega / q} 
        -a \E\brk*{\sumt{\sw}} + qT \\ 
        &\leq 
        \frac{bB^2}{a}
        +
        2 B \dix \sqrt{ \omega / q} 
        +qT ,
        \nonumber
    \end{align}
    where we used $c_1\sqrt{x}-c_2x\leq{c_1^2}/\prn{4c_2}$ for $x\geq0$, $c_1\geq 0$, and $c_2>0$.
    Finally, substituting 
    $q=\prn*{\frac{4 \omega B^2\dix ^2}{ T }}^{1/3}$ in the last inequality gives
    \[
    \E\brk{\reg}
    \leq
    \frac{bB^2}{a}
    +
    2^{5/3} \omega^{1/3} \prn*{ B \dix T}^{2/3}
    ,
    \]
    which is the desired bound.
\end{proof}



\subsection{Proof of \cref{cor:thm_self}}\label{app:SELF_upper_discussion_deferred}
Here, we derive the regret upper bounds provided by the algorithm established  
in \cref{thm:bandit_regret_pseudo_estimator}  
for online multiclass classification, online multilabel classification, and ranking.
Recall that 
we can achieve
\begin{equation}\label{eq:bound_self_app}
\E\brk{\reg}
\leq
\frac{bB^2}{a}
+
O\prn*{ \omega^{1/3} \prn*{ B \dix T}^{2/3} },    
\end{equation}
where we recall that $\omega$ is defined as
$
    \tr \prn*{ \V^{-1} \bm{Q}^+ \prn{\V^{-1}}^\top } \leq \omega
$
for $\bm{Q} = \E_{\bm{y} \sim \mu} \brk{ \bm{y} \bm{y}^\top }$.
Note that when $\spanx(\yy) = \R^d$, then the matrix $\bm{Q}$ is invertible and $\lambda_{\min}(\bm{Q}) > 0$, and thus 
\begin{equation}\label{eq:trace_upper_invertibleQ}
    \tr \prn*{ \V^{-1} \bm{Q}^+ \prn{\V^{-1}}^\top }
    =
    \sum_{i=1}^d
    \bm{\ee}_i^\top \V^{-1} \bm{Q}^+ \prn{\V^{-1}}^\top \bm{\ee}_i
    \leq
    \frac{1}{\lambda_{\min}(\bm{Q})}
    \sum_{i=1}^d
    \nrm{ \prn{\V^{-1}}^\top \bm{\ee}_i }_2^2
    \leq 
    \frac{1}{\lambda_{\min}(\bm{Q})}
    \nrm{ \V^{-1} }_{\F}^2
    .
\end{equation}
In each problem setting, this regret upper bound can be reduced to the following bounds:

\paragraph{Multiclass classification with 0-1 loss}
We first consider multiclass classification with the 0-1 loss.
Since $\V=\bm{1}\bm{1}^\top-\I$, we have $\nrm{\inverse{\V}}_\F^2\leq d$ for $d \geq 2$.  
Recalling that $\mu$ is the uniform distribution over $\yy=\set{\bm{\ee}_1,\hdots,\bm{\ee}_d}$, we have  
$
\E_{\bmy\sim\mu}\brk{(\bmy^\top\bm x)^2}=\frac{1}{d}\sum_{i=1}^{d}x_i^2
$
for any $\bm x\in\R^d$.
Hence, $\lambda_{\min}(\bm{Q})=\min_{\nrm{\bm x}_2=1} \E_{\bmy\sim\mu}\brk*{(\bmy^\top\bm x)^2}=\frac{1}{d}$, where the first equality is from \citet[Lemma 2]{comband}.  
Since $\spanx(\yy) = \R^d$ in this case,
from \eqref{eq:trace_upper_invertibleQ} we can let $\omega = d / \lambda_{\min}(\bm{Q}) = d^2$.
Substituting these into our upper bound in \eqref{eq:bound_self_app}, we obtain  
\begin{equation*}
    \E\brk{\reg}\leq\frac{bB^2}{a}+ O \prn*{ \prn{B \dix  d T}^{2/3} }.
\end{equation*}  



\paragraph{Online multilabel classification with $m$ correct labels   
and the Hamming loss}
We next consider online multilabel classification with the number of correct labels $m$  
and the Hamming loss.  
Since $\V=-\frac{2}{d}\I$, we have  
$\nrm{\inverse{\V}}_\F^2=\frac{d^3}{4}$.  
Let $\yy\subset\set{0,1}^d$ be the set of all vectors  
where exactly $m$ components are $1$, and the remaining components are all $0$.  
By drawing $\bmy\in\yy$ according to the uniform distribution over $\yy$,  
the probability that a given component of $\bmy$ is $1$ is  
$\binom{m-1}{d-1}/\binom{m}{d}=\frac{m}{d}$.  
Hence, for any $\bm x\in\R^d$ with $\nrm{\bm{x}}_2=1$,  
we have
\[
\E_{\bmy\sim\mu}\brk*{(\bmy^\top\bm x)^2}
=\frac{m}{d}\sum_{i=1}^dx_i^2  
+\frac{m^2}{d^2}\sum_{i\neq j}x_ix_j  
=\prn*{\frac{m}{d}\sum_{i=1}^dx_i}^2+\frac{m(d-m)}{d^2}\nrm{\bm x}_2^2  
\geq 
\frac{m(d-m)}{d^2}.
\]
Hence, we have $\lambda_{\min}(\bm{Q}) = \E_{\bmy\sim\mu}\brk*{(\bmy^\top\bm x)^2} \geq\frac{m(d-m)}{d^2}$, where the equality is from \citet[Lemma 2]{comband}.
Since $\spanx(\yy) = \R^d$,
from \eqref{eq:trace_upper_invertibleQ} we can choose $\omega = d^3 / \prn{4 \lambda_{\min}(\bm{Q})} = \frac{d^5}{4 m (d-m)} $.
Therefore, our regret upper bound in \eqref{eq:bound_self_app} is reduced to
\begin{equation*}
    \E\brk{\reg}
    \leq
    \frac{bB^2}{a}
    +
    O \prn*{ \prn*{\frac{B^2\dix ^2d^5}{m(d-m)}}^{1/3} T^{2/3} }.
\end{equation*}


\paragraph{Ranking with the Hamming loss and the number of items $m$}
We finally consider online ranking with the Hamming loss and the number of items $m$.  
From \citet[Proposition 4]{comband}, the smallest positive eigenvalue is at least $1/m$.
Hence, since $\V=-\frac{1}{m}\I$, we have
\begin{equation*}
\tr\prn{\inverse{\V}\bm{Q}^+(\inverse{\V})^\top}=m^2\tr\prn{\bm{Q}^+}\leq m^2\sum_{i=1}^{\rank(\bm{Q}^+)} m\leq m^5,
\end{equation*}
where we used $\rank(\bm{Q}^+) \leq d = m^2$,
and this allows us to choose $\omega = m^5$.
Substituting these values into our regret upper bound in \eqref{eq:bound_self_app} , we obtain  
\begin{equation*}
    \E\brk{\reg}\leq\frac{bB^2}{a}+ O\prn*{ m^{5/3}\prn{B\dix T}^{2/3} }.
\end{equation*}  


\section{Omitted Details of \cref{sec:delay}}
\label{app:proof delay}
This section provides the proofs of the theorems in \cref{sec:delay}.

\subsection{Details of Optimistic Delayed Adaptive FTRL (ODAFTRL)}
\label{app:sub_odaftrl}
We provide a more detailed explanation of the Optimistic Delayed Adaptive FTRL (ODAFTRL) algorithm used for updating $\W_t$ in \cref{sec:delay}.
We recall that ODAFTRL computes $\W_t$ by the following update rule:
\begin{equation}
    \label{eq:odaftrl_2}
    \W_{t+1}=\argmin_{\W\in \ww} \set*{ \sum_{i=1}^{t-D} \inpr{\bm{G}_i ,\W} + \frac{\lambda_t \nrm{\W}_{\F}^2 }{2} },
\end{equation}
where $\lambda_t\geq0$ is a regularization parameter.
The ODAFTRL algorithm, when using this parameter update called AdaHedgeD, satisfies the following lemma:
\begin{lemma}[{\citealt[Theorem 12]{pmlr-v139-flaspohler21a}}]
    \label{thm:AdaHedgeD}
    Fix $\alpha>0$. 
    Let $S_t:\ww\to\R$ be a convex function for each $t=1,\dots,T$.
    Suppose that we update $\lambda_{t}$ in \eqref{eq:odaftrl_2} by the following AdaHedgeD update:
    \begin{align*}
        \lambda_{t+1}&=\frac{1}{\alpha}\sum_{s=1}^{t-D}\delta_s,\\
        \delta_t &= \min\crl*{F_{t+1}(\W_t)-F_{t+1}(\bar{\bm{W}}_t),\inpr{ \bm{G}_t,\W_t-\bar{\bm{W}}_t},F_{t+1}(\hat{\bm{W}}_t)-F_{t+1}(\bar{\bm{W}}_t)+\inpr{ \bm{G}_t,\W_t-\hat{\bm{W}}_t}}_+,\\
        \bar{\bm{W}}_t &= \argmin_{\W\in\ww}F_{t+1}(\W),\\
        \hat{\bm{W}}_t&= \argmin_{\W\in\ww}\set*{F_{t+1}(\W)-\min\crl*{\frac{\nrm{\bm{G}_t}_{\mathrm{F}}}{\nrm{\bm{G}_{t-D:t}}_{\mathrm{F}}},1}\inpr{\bm{G}_{t-D:t},\W}}, \text{  and}\\
        F_{t+1}(\W)&=\frac{\lambda_t\nrm{\W}_\F^2}{2}+\inpr{\G_{1:t},\W}.
    \end{align*}
    Then, for any $\U\in\ww$, ODAFTRL achieves
    \begin{equation*}
        \sumt{\sw}
        \leq
        \sumt{\su}
        +
        \prn*{\frac{B^2}{2\alpha}+1}\prn*{2\max_{s\in[T]}a_{s-D:s-1}+\sqrt{\sum_{t=1}^{T}a_{t}^2+2\alpha b_{t}}},
    \end{equation*}
    where 
    \begin{align*}
        a_{t}&=B\min\crl{\nrm{\bm{G}_{t-D:t}}_{\mathrm{F}},\nrm{\bm{G}_t}_{\mathrm{F}}},\\
        b_{t}&=\operatorname{huber}\prn{\nrm{\bm{G}_{t-D:t}}_{\mathrm{F}},\nrm{\bm{G}_t}_{\mathrm{F}}},\:\text{and} \:
        \operatorname{huber}(x,y)=\frac{1}{2}x^2-\frac{1}{2}(|x|-|y|)^2_+\leq\min\crl*{\frac{1}{2}x^2,|x||y|}.
    \end{align*}
\end{lemma}
In the following, we let $\alpha = \max_{\W \in \ww} \frac{\nrm{\W}_\F^2}{2} = \frac{B^2}{2}$ for simplicity. Then,
\begin{equation*}
        \sumt{\sw}
        \leq
        \sumt{\su}
        +
        2\prn*{2\max_{s\in[T]}a_{s-D:s-1}+\sqrt{\sum_{t=1}^{T}a_{t}^2+B^2 b_{t}}},
\end{equation*}


\subsection{Proof of \cref{thm:delayed_regret_expectation_abstract} }
\label{app:sub_delayed_regret_expectation_abstract}
We present \cref{thm:delayed_regret_expectation_abstract} in a more detailed form and provide its proof. 
\begin{theorem}[Formal version of \cref{thm:delayed_regret_expectation_abstract}]
    \label{thm:delayed_regret_expectation_abstract_detail}
    Let $\alpha=\frac{B^2}{2}$.
    Then, ODAFTRL with the AdaHedgeD update in online structured prediction with a delay of $D$ achieves
    \begin{equation*}
        \E\brk*{\reg}\leq4 B D \L+\frac{2bB^2}{a}
        +\frac{3}{2}\prn*{a^{-1}bB^4\L^2(D+1)^2T}^{1/3}.
    \end{equation*}
\end{theorem}
\begin{proof}
    From \cref{thm:AdaHedgeD} and the definition of $a_t$ and $b_t$, we have
    \begin{align*}
        \sumt{(\sw-\su)}&\leq2\prn*{2B\max_{s\in[T]}\sum_{i=s-D}^{s-1}\nrm{\G_i}_{\mathrm{F}}
        +\sqrt{\sumt{\prn*{B^2\normst^2+B^2\nrm{\G_{t-D:t}}_\F\normst}}}}\\
        &\leq 2\prn*{2BD\L+\sqrt{bB^2\sumt{\sw}}+  \prn*{b B^4 \L^2 (D+1)^2T\sumt{\sw}}^{1/4}},
    \end{align*}
    where we used the Cauchy--Schwarz inequality, $\normst\leq \L$, $\normst^2\leq b\sw$, and the subadditivity of $x \mapsto \sqrt{x}$ for $x \geq 0$ in the last inequality.
    From this inequality and \cref{asp:delayed_a}, we can evaluate surrogate regret as
    \begin{align*}
        \E\brk*{\reg} &\leq \sumt{\prn*{(1-a)\sw-\su}}\\
        &\leq 2\prn*{2BD\L+\sqrt{bB^2\sumt{\sw}} + \prn*{b B^4 \L^2 (D+1)^2 T\sumt{\sw}}^{1/4}} -a\sumt{\sw}\\
        &\leq 4 B D \L+\frac{2bB^2}{a}
        +\frac{3}{2}\prn*{a^{-1}bB^4\L^2(D+1)^2T}^{1/3}.
    \end{align*}
    We used $c_1\sqrt{x}-c_2x\leq{c_1^2}/\prn{4c_2}$ and $c_1x-c_2x^4\leq\prn{{3}/{4}}\prn*{{c_1^4}/\prn{4c_2}}^{1/3}$ that hold for any $x\geq0$, $c_1\geq 0$, and $c_2>0$ in the last inequality.
\end{proof}


\subsection{Proof of \cref{thm:delayed_regret_probability_abstract}}
\label{app:sub_delayed_regret_probability_abstract}
We present \cref{thm:delayed_regret_probability_abstract} in a more detailed form and provide its proof. 
\begin{theorem}
    \label{thm:delayed_regret_probability_abstract_detail}
    Let $\alpha=\frac{B^2}{2}$ and $\delta \in (0,1)$.
    Then, 
    ODAFTRL with the AdaHedgeD update in online structured prediction with a delay of $D$ achieves
    \begin{equation*}
        \reg\leq 4BD\L+\frac{\sqrt{2}}{3}\log\frac{1}{\delta}
        +\frac{\prn*{\sqrt{(1-a)\log\frac{1}{\delta}}+\sqrt{2bB^2}}^2}{a}
        +\frac{3}{2}\prn*{a^{-1}bB^4\L^2(D+1)^2T}^{1/3},
    \end{equation*}
    with probability at least $1 - \delta$. 
\end{theorem}

\begin{proof}
    We decompose $\reg$ into     
    \begin{equation}\label{eq:decompose_reg}
    \reg
    =\sumt{\prn*{L_t(\yht)-\expect{L_t(\yht)}}}+\sumt{\prn{\expect{L_t(\yht)}-\su}}.
    \end{equation}
    Let $Z_t=L_t(\yht)-\expect{L_t(\yht)}$. 
    Then, we have $|Z_t|\leq 1$ and $\expect{Z_t^2}\leq (1-a)\sw$ from \cref{asp:delayed_a}.
    Hence, from \cref{lem:Bernstein}, with probability at least $1 - \delta$, the first term in \eqref{eq:decompose_reg} is upper bounded as 
    \begin{equation}\label{eq:bound_of_zt}
        \sumt{Z_t}\leq\sqrt{2(1-a)\sumt{\sw}\log\frac{1}{\delta}}+\frac{\sqrt{2}}{3}\log\frac{1}{\delta}.
    \end{equation}
    From from \cref{asp:delayed_a} and \cref{thm:AdaHedgeD},
    the second term in \eqref{eq:decompose_reg} is also upper bounded as 
    \begin{align}
        &
        \sumt{(\expect{L_t(\yht)}-\su)}
        \nonumber \\
        &\leq
        2\prn*{2BD\L+\sqrt{bB^2\sumt{\sw}}+  \prn*{b B^4 \L^2 (D+1)^2T\sumt{\sw}}^{1/4}} - a\sumt{\sw},
        \label{eq:bound_of_reg_exp}
    \end{align}
    where we used the subadditivity of $x \mapsto \sqrt{x}$ for $x \geq 0$.
    Therefore, substituting \eqref{eq:bound_of_zt} and \eqref{eq:bound_of_reg_exp} into \eqref{eq:decompose_reg} gives 
    \begin{align*}
        \reg&\leq 4BD\L+\frac{\sqrt{2}}{3}\log\frac{1}{\delta}
        +\prn*{\sqrt{ 2 (1-a) \log \frac{1}{\delta} } + 2\sqrt{bB^2}}\sqrt{\sumt{\sw}}\\
        &\quad+2\prn*{b B^4 (D+1)^2 \L ^2T\sumt{\sw}}^{1/4} - a\sumt{\sw}\\
        &\leq 4BD\L+\frac{\sqrt{2}}{3}\log\frac{1}{\delta}
        +\frac{\prn*{\sqrt{(1-a)\log\frac{1}{\delta}}+\sqrt{2bB^2}}^2}{a}
        +\frac{3}{2}\prn*{a^{-1}bB^4\L^2(D+1)^2T}^{1/3},
    \end{align*}
    where we used $c_1\sqrt{x}-c_2x\leq{c_1^2}/\prn{4c_2}$ and $c_1x-c_2x^4\leq\prn{{3}/{4}}\prn*{{c_1^4}/\prn{4c_2}}^{1/3}$ for $x\geq0$, $c_1\geq 0$, and $c_2>0$ in the last inequality.
    This is the desired bound.
\end{proof}
\section{Omitted Details of \cref{sec:bandit_and_delayed}}\label{app:bandit_and_delayed}
This section provides the omitted proofs of the theorems in \cref{sec:bandit_and_delayed}.

\subsection{Common analysis}
We provide the analysis that is commonly used in the proofs of \cref{thm:delay_bandit_bound_general_abstract,thm:delay_bandit_bound_self_abstract}.
We use ODAFTRL with the AdaHedgeD update in \cref{app:sub_odaftrl} for $\alg$.
From \cref{thm:AdaHedgeD}, it holds that
\begin{equation}\label{eq:inpr_common}
    \sumt{\inpr{\gtil,\W_t-\U}} 
    \leq 2\prn*{2\max_{t\in[T]}a_{t-D:t-1}+\sqrt{\sumt{a_t^2}+B^2 b_t}},
\end{equation}
where
\begin{equation*}
    a_t=B\min\set{\nrm{\tilde{\G}_{t-D:t}}_\F,\nrm{\gtil}_\F} \quad \mbox{and} \quad b_t\leq\min\set*{\frac{1}{2}\nrm{\tilde{\G}_{t-D:t}}_\F^2, \nrm{\tilde{\G}_{t-D:t}}_\F\nrm{\gtil}_\F}.
\end{equation*}
By the definition of $a_t$, we have 
\begin{equation}\label{eq:bound_of_at}
    \E\brk*{\max_{t\in\brk{T}}a_{t-D:t-1}}\leq B\E\brk*{\max_{t\in\brk{T}} \sum_{s=t-D}^{t-1}\nrm{\tilde{\G}_{s}}_\F }\leq B\E\brk*{\sqrt{D\max_{t\in\brk{T}}\sum_{s=t-D}^{t-1}\nrm{\tilde{\G}_s}_\F^2}}\leq B\sqrt{D\E\brk*{\sumt{\nrm{\gtil}_\F^2}}},
\end{equation}
from the Cauchy--Schwarz inequality and Jensen's inequality.
Hence, from \eqref{eq:inpr_common} and \eqref{eq:bound_of_at}, we have
\begin{align}\label{eq:bandit_delayed_expect_inpr}
    \E\brk*{\sumt{\inpr{\gtil,\W_t-\U}}}
    &\leq 2\prn*{2\E\brk*{\max_{t\in[T]}a_{t-D:t-1}}+\sqrt{\E\brk*{\sumt{a_t^2}}}+B\sqrt{ \E\brk*{\sumt{b_t}}}}\nonumber\\
    &\leq 2B\prn*{2\sqrt{D\E\brk*{\sumt{\nrm{\gtil}_\F^2}}}+\sqrt{\E\brk*{\sumt{\nrm{\gtil}_\F^2}}}+\sqrt{\E\brk*{\sumt{b_t}}}},
\end{align}
where we used the subadditivity of $x \mapsto \sqrt{x}$ for $x \geq 0$.
The last term in the last inequality is further bounded as
\begin{align}\label{eq:expect_bt}
    \E\brk*{\sumt{b_t}}&\leq \E\brk*{\sumt{\nrm{\tilde{\G}_{t-D:t}}_\F\nrm{\gtil}_\F}}
    \leq\E\brk*{\sumt{\nrm{\gtil}_\F^2}} +\E\brk*{\sumt{\nrm{\gtil}_\F\sum_{s=t-D}^{t-1}\nrm{\tilde{\G}_s}_\F}}\nonumber\\
    &=\E\brk*{\sumt{\nrm{\gtil}_\F^2}} + \E\brk*{\sumt
    {\E_t\brk*{\nrm{\gtil}_\F}\sum_{s=t-D}^{t-1}\nrm{\tilde{\G}_s}_\F}}, 
\end{align}
where the second inequality follows from the triangle inequality and the equality follows from the law of total expectation.


\subsection{Proof of \cref{thm:delay_bandit_bound_general_abstract}}\label{app:bandit_delayed_general}
We provide the complete version of \cref{thm:delay_bandit_bound_general_abstract}:
\begin{theorem}[Formal version of \cref{thm:delay_bandit_bound_general_abstract}]
    The algorithm in \cref{subsec:bandit_delay_general} achieves
\begin{equation*}
    \E\brk*{\reg}
    \leq \frac{4bB}{a}\prn*{D^{1/4}+D^{-1/4}}^2\sqrt{KT}+2B \L\sqrt{D T}+B\sqrt{DKT}
    = O\prn{D\sqrt{KT}}
    .
\end{equation*}
\end{theorem}

\begin{proof}
First, we will upper bound $\E\brk*{\sumt{b_t}}$.
From \eqref{eq:expect_bt}, we have
\begin{align}\label{eq:bound_of_bt_general}
    \E\brk*{\sumt{b_t}}&\leq \E\brk*{\sumt{\nrm{\gtil}_\F^2}} + \E\brk*{\sumt
    {\E_t\brk*{\nrm{\gtil}_\F}\sum_{s=t-D}^{t-1}\nrm{\tilde{\G}_s}_\F}}\nonumber\\
    &\leq \E\brk*{\sumt{\nrm{\gtil}_\F^2}} + \L\E\brk*{\sumt{\sum_{s=t-D}^{t-1}\E_s\brk*{\nrm{\tilde{\G}_s}_\F}}}\nonumber\\
    &\leq \E\brk*{\sumt{\nrm{\gtil}_\F^2}} + D\L^2 T,
\end{align}
where the second and third inequality follow from the inequality $\expect{\nrm{\gtil}_\F}=\nrm{\G_t}_\F\leq \L$.
Hence, from \eqref{eq:bandit_delayed_expect_inpr} and \eqref{eq:bound_of_bt_general}, it holds that
\begin{align}
    \E\brk*{\sumt{\inpr{\gtil,\W_t-\U}}}
    &\leq 2B\prn*{2\sqrt{D\E\brk*{\sumt{\nrm{\gtil}_\F^2}}}+\sqrt{\E\brk*{\sumt{\nrm{\gtil}_\F^2}}}
    +
    \sqrt{{\E\brk*{\sumt{\nrm{\gtil}_\F^2}} + D\L^2 T}}}\nonumber\\
    &\leq 4B\prn*{\sqrt{D}+1}\sqrt{\E\brk*{\sumt{\nrm{\gtil}_\F^2}}}+2 B\L\sqrt{ D T}\nonumber\\
    &\leq 4B\prn*{\sqrt{D}+1}\sqrt{\frac{bK}{q}\E\brk*{\sumt{\sw}}}+2B\L\sqrt{ D T},
\end{align}
where in the second inequality we used the subadditivity of $x \mapsto \sqrt{x}$ for $x \geq 0$ and in the last inequality we used 
\begin{equation*}
\expect{\|\gtil\|_{\mathrm{F}}^2}
=
\frac{\|\G_t\|_{\mathrm{F}}^2}{p_t(\yt)}
\leq
\frac{\K }{q}\|\G_t\|_{\mathrm{F}}^2
\leq
\frac{b\K }{q}\sw.
\end{equation*}
Therefore, combining all the above arguments yields 
\begin{align*}
    \E\brk*{\reg}&
    \leq \E\brk*{\sumt{(\sw-\su)}} - a\E\brk*{\sumt{\sw}} + q T\\
    &\leq \E\brk*{\sumt{\inpr{\gtil,\W_t-\U}}} - a\E\brk*{\sumt{\sw}} + q T\\
    &\leq 4B\prn*{\sqrt{D}+1}\sqrt{\frac{bK}{q}\E\brk*{\sumt{\sw}}}+2 B\L\sqrt{ D T}- a\E\brk*{\sumt{\sw}} + q T\\
    &\leq \frac{4bB^2}{a}\prn*{\sqrt{D}+1}^2\frac{bK}{q}+2 B\L\sqrt{D T}+ q T,
\end{align*}
where the first inequality follows from \cref{asp:bandit_a},
the second inequality follows from the convexity of $S_t$ and the unbiasedness of $\gtil$,
and the last inequality follows from $c_1\sqrt{x}-c_2x\leq{c_1^2}/\prn{4c_2}$ for $x \geq 0$, $c_1 \geq 0$, and $c_2 > 0$.
Finally, by substituting $q=B\sqrt{DK/T}$, we obtain
\begin{equation*}
    \E\brk*{\reg}
    \leq
    \frac{4bB}{a}\prn*{D^{1/4}+D^{-1/4}}^2\sqrt{KT}+2B \L\sqrt{D T}+B\sqrt{DKT},
\end{equation*}
which is the desired bound.
\end{proof}





\subsection{Proof of \cref{thm:delay_bandit_bound_self_abstract}}\label{app:bandit_delayed_self}
We provide the complete version of \cref{thm:delay_bandit_bound_self_abstract}:
\begin{theorem}[Formal version of \cref{thm:delay_bandit_bound_self_abstract}]
    The algorithm in \cref{subsec:bandit_delay_self} achieves
    \begin{equation*}
        \E\brk{\reg}
        =
        \frac{8bB^2}{a}\prn[\Big]{\sqrt{D} + 1}^2
    +2B (\L+\dix \diy )\sqrt{DT}
        + O\prn*{ B^{2/3} D^{1/3} T^{2/3} }. 
    \end{equation*}
\end{theorem}
\begin{proof}
First, we will derive an upper bound of $\E\brk*{\sumt{b_t}}$.
We first observe that
\begin{align*}
    \expect{\nrm{\gtil}_\F}&=\expect{\nrm{(\yho(\tht)-\ytilde)\xt^\top}_\F}\leq \expect{\nrm{\G_t}_\F + \dix \nrm{\yt-\ytilde}_2}
    \leq \nrm{\G_t}_\F + \dix \expect{\nrm{\yt}_2+\nrm{\ytilde}_2}\\
    &\leq \nrm{\G_t}_\F + \dix \diy + \dix \expect{\sqrt{\tr\prn*{\ytilde\ytilde^\top}}}
    \leq \nrm{\G_t}_\F + \dix \diy + \dix {\sqrt{\expect{\tr\prn*{\ytilde\ytilde^\top}}}}\\
    &\leq \nrm{\G_t}_\F + \dix \diy + \sqrt{\dix^2\omega / q}
    \leq \L+ \dix \diy + \sqrt{\dix^2\omega / q},
\end{align*}
where the first inequality follows from $\dix =\diam(\xx)$, the third inequality follows from $\diy=\diam(\yy)$, the fourth inequality follows from Jensen's inequality, and the fifth inequality follows from \cref{lem:bound of trace}.
Thus, combining \eqref{eq:expect_bt} with the last inequality, we have
\begin{align}
    \E\brk*{\sumt{b_t}}&\leq \E\brk*{\sumt{\nrm{\gtil}_\F^2}} + \E\brk*{\sumt
    {\E_t\brk*{\nrm{\gtil}_\F}\sum_{s=t-D}^{t-1}\nrm{\tilde{\G}_s}_\F}} \nonumber \\
    &\leq \E\brk*{\sumt{\nrm{\gtil}_\F^2}}+ \prn*{\L+ \dix \diy + \sqrt{\frac{\dix^2\omega}{q}} } \E\brk*{\sumt{\sum_{s=t-D}^{t-1}\E_s\brk*{\nrm{\tilde{\G}_s}_\F}}} \nonumber \\
    & \leq\E\brk*{\sumt{\nrm{\gtil}_\F^2}}+ D T \prn*{\L+ \dix \diy + \sqrt{\frac{\dix^2\omega}{q}} }^2.
    \label{eq:expect_gtil_self}
\end{align}
Hence, from \eqref{eq:bandit_delayed_expect_inpr} and \eqref{eq:expect_gtil_self}, we have
\begin{align}\label{eq:expext_inpr_self}
    &\E\brk*{\sumt{\inpr{\gtil,\W_t-\U}}}\nonumber\\
    &\leq 2B\prn*{2\sqrt{D\E\brk*{\sumt{\nrm{\gtil}_\F^2}}}+\sqrt{\E\brk*{\sumt{\nrm{\gtil}_\F^2}}}+\sqrt{\E\brk*{\sumt{b_t}}}}\nonumber\\
    &\leq 4B\prn*{\sqrt{D}+1}\sqrt{\E\brk*{\sumt{\nrm{\gtil}_\F^2}}} + 2B\prn*{\L+ \dix \diy + \sqrt{ \dix^2\omega / q } } \sqrt{DT}\nonumber\\
    &\leq 4B\prn*{\sqrt{D}+1}\sqrt{2\sumt{\prn*{b\sw+\frac{\dix^2\omega}{q}}}
    }
    + 2B\prn*{\L+ \dix \diy + \sqrt{\dix^2\omega / q} } \sqrt{DT},
\end{align}
where the first inequality follows from \eqref{eq:bandit_delayed_expect_inpr}, 
the second inequality follows from \eqref{eq:expect_gtil_self} and the subadditivity of $x \mapsto \sqrt{x}$ for $x \geq 0$, 
and the last inequality follows from \cref{thm:evaluation of Gtilde}.
Therefore, combining all the above arguments yields 
\begin{align*}
    \E\brk*{\reg}
    &\leq \E\brk*{\sumt{(\sw-\su)}} - a\E\brk*{\sumt{\sw}} + q T\\
    &\leq \E\brk*{\sumt{\inpr{\gtil,\wt-\U}}} - a\E\brk*{\sumt{\sw}} + q T\\
    & \leq 4B\prn*{\sqrt{D} + 1}\prn*{\sqrt{2b\sumt{\sw}}+\sqrt{ 2\dix^2\omega T / q}
    }
    \nonumber \\
    &\qquad+ 2B\prn*{\L+ \dix \diy + \sqrt{\dix^2\omega / q}  } \sqrt{DT} - a\E\brk*{\sumt{\sw}} + q T \\
    &\leq
    \frac{8bB^2}{a}\prn[\Big]{\sqrt{D} + 1}^2
    +2B (\L+\dix \diy )\sqrt{DT}
    +2B\dix\prn[\big]{(2\sqrt{2}+1)\sqrt{D}+2\sqrt{2}}\sqrt{\omega T/ q} + q T,
\end{align*}
where the first inequality follows from \cref{asp:bandit_a}, 
the third inequality follows from \eqref{eq:expext_inpr_self} and the subadditivity of $x \mapsto \sqrt{x}$ for $x \geq 0$, 
and the last inequality follows from the definition of $\epsilon$ and $c_1\sqrt{x}-c_2x\leq{c_1^2}/\prn{4c_2}$ for $x \geq 0$, $c_1 \geq 0$, and $c_2 > 0$.
Finally, substituting $q=\prn*{\frac{\omega B^2\dix^2 D}{T}}^{1/3}$ gives the desired bound.
\end{proof}



\section{Overhead of \ourSystem.}
We report the size and inference time of the model for NeRF and \ourSystem in Table~\ref{table_overhead}.  
The results indicate that \ourSystem has a larger model size than NeRF, \ie 27.1 \vs 8.0\,MB.  
Correspondingly, \ourSystem exhibits a longer inference time, \ie 1.79 \vs 0.43\,s.  
Unlike NeRF, \ourSystem requires neighboring spectra as input. 
During inference, the target transmitter's neighbors are extracted from the training dataset, so \ourSystem does not incur additional data burdens.  
Moreover, since \ourSystem can operate in unseen scenes, it significantly reduces the requirement for a time-consuming training process.



\begin{table}[h]
\centering
\caption{Comparison of model size and inference time.}

\begin{tabular}{lC{0.8in}C{0.8in}}
\toprule
     & \nerft    & \ourSystem    
     \\ \midrule
Model size (MB) & 8.0  & 27.1   \\
Inference time (s) & 0.43    & 1.79  \\
\bottomrule
\end{tabular}
\label{table_overhead}
\end{table}






\end{document}