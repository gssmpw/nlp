\section{Related Work}
\subsection{Extended reality for remote collaboration}

Recent advancements in extended reality have demonstrated significant potential for enhancing remote collaboration, particularly in asynchronous settings. Several key systems have explored how immersive technologies can improve knowledge transfer and interaction between users working in different locations and times. VRdeo ____ is an educational tool that allows teachers and students to interact asynchronously through recorded virtual reality (VR) sessions, which can be exported as videos. Creators can teleport around the scene, manipulate 3D objects, and provide voice instructions during recording. A user study comparing VR and 2D recordings found that participants preferred VR for its interactivity and engagement, resulting in improved learning outcomes.

Providing a structured way to access and learn from previous activities can significantly enhance collaboration in XR settings. The Tesseract system ____ improves remote collaboration in spatial design processes, by enabling users to capture and review design recordings, and allowing teams to reflect on workflows and design decisions asynchronously. Its Worlds-in-Miniature-based Search Cube interface lets users stage and query parts of a design, making it easier to retrieve key moments from past sessions. This feature aids knowledge transfer, as junior designers can replay senior designers' work to understand design rationale. 

\subsection{Enhancing user experience in asynchronous extended reality}

Several works have explored innovative methods for enhancing user interaction and understanding in asynchronous XR environments. RealityReplay ____ introduces an end-to-end pipeline that detects and visualises important changes in a user's environment through user-centric sensing. The system combines semantic segmentation and saliency prediction to track significant changes in the environment, filtering out static or irrelevant objects. Users can select an area of interest, and the system generates a summary visualisation, offering options from simple motion lines to more detailed depictions of texture and position. Who Put That There? ____ is a VR system designed to track spatio-temporal changes in objects, allowing users to query a timeline of changes such as location, size, and appearance. It introduces interaction primitives, that include a "trajectory sphere" for exploring object interactions over time and a "preview sphere" that isolates specific events without altering the surrounding virtual environment. An evaluation study showed that users preferred the enhanced navigation tools over conventional timelines, noting increased engagement and a stronger sense of presence.

Spatio-temporal tools can significantly enhance asynchronous learning by helping to lower cognitive load. The XR-LIVE system ____ facilitates this through an assistive toolkit that includes features like a checklist for organising lab tasks and an auto-pause function that halts demonstrations at key moments. Temporal cues guide learners through each step, while spatial features allow students to accurately replicate instructor actions within a virtual environment, facilitating a deeper understanding of complex tasks. Together, these elements support different learning styles, boosting engagement and making asynchronous learning more effective.

\subsection{Design challenges in asynchronous collaboration}

While there is a relative scarcity of research for asynchronous collaboration in XR, the current literature presents distinct design challenges, indicating the field is still developing. A key challenge lies in the lack of established methods for representing changes made in XR environments to users who are not present simultaneously, which is crucial for maintaining continuity in tasks ____. Additionally, asynchronous tools can contribute to information overload, especially when users navigate multiple communication channels. 

Asynchronous collaboration involves more than simply producing and consuming information; it requires precise authoring of spatial annotations that are contextually and temporally accurate ____. Key challenges include creating annotations that reflect both location and time, allowing users to understand the sequence and significance of events. In particular, in remote maintenance, asynchronous collaboration should enable experts to guide on-site technicians through the placement of spatially aligned annotations, that provide clear instructions for equipment repair and servicing ____. In addition, asynchronous AR systems should support user-driven behaviours, such as gaze tracking, and facilitate collaboration across different devices and environments, enhancing the depth and efficiency of communication.

Building on these challenges, the integration of privacy, security, and user data protection in XR collaboration further complicates the design of asynchronous systems. Effective access control is crucial in XR environments to manage permissions and prevent unauthorised access to shared virtual content and physical spaces ____. Privacy concerns arise from environmental sensing, where unintended capture of users' surroundings can occur without consent. In this context, balancing security with usability is essential, as complex interaction techniques may enhance security but complicate the user experience.