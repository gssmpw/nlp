
\documentclass[10pt]{article} % For LaTeX2e
% \usepackage{tmlr}
% If accepted, instead use the following line for the camera-ready submission:
% \usepackage[accepted]{tmlr}
% To de-anonymize and remove mentions to TMLR (for example for posting to preprint servers), instead use the following:
\usepackage[preprint]{tmlr}

% Optional math commands from https://github.com/goodfeli/dlbook_notation.
%%%%% NEW MATH DEFINITIONS %%%%%

% \usepackage{amsmath,amsfonts,bm}
\usepackage{amsmath,amsfonts}

\usepackage{pifont}


\newcommand{\R}{\mathbb{R}}


\def\va{{\mathbf{a}}}
\def\vg{{\mathbf{g}}}

% Sets
\def\sR{\mathbb{R}}
\def\sC{\mathbb{C}}
\def\sZ{\mathbb{Z}}
\def\sN{\mathbb{N}}
\def\sQ{\mathbb{Q}}

\def\sS{\mathcal{S}}



% Vectors
\def\vzero{{\mathbf{0}}}
\def\vone{{\mathbf{1}}}
\def\vmu{{\mathbf{\mu}}}
\def\vtheta{{\mathbf{\theta}}}
\def\va{{\mathbf{a}}}
\def\vb{{\mathbf{b}}}
\def\vc{{\mathbf{c}}}
\def\vd{{\mathbf{d}}}
\def\ve{{\mathbf{e}}}
\def\vf{{\mathbf{f}}}
\def\vg{{\mathbf{g}}}
\def\vh{{\mathbf{h}}}
\def\vi{{\mathbf{i}}}
\def\vj{{\mathbf{j}}}
\def\vk{{\mathbf{k}}}
\def\vl{{\mathbf{l}}}
\def\vm{{\mathbf{m}}}
\def\vn{{\mathbf{n}}}
\def\vo{{\mathbf{o}}}
\def\vp{{\mathbf{p}}}
\def\vq{{\mathbf{q}}}
\def\vr{{\mathbf{r}}}
\def\vs{{\mathbf{s}}}
\def\vt{{\mathbf{t}}}
\def\vu{{\mathbf{u}}}
\def\vv{{\mathbf{v}}}
\def\vw{{\mathbf{w}}}
\def\vx{{\mathbf{x}}}
\def\vy{{\mathbf{y}}}
\def\vz{{\mathbf{z}}}
\def\vzeta{{\mathbf{\zeta}}}

% Matrix
\def\mA{{\mathbf{A}}}
\def\mB{{\mathbf{B}}}
\def\mC{{\mathbf{C}}}
\def\mD{{\mathbf{D}}}
\def\mE{{\mathbf{E}}}
\def\mF{{\mathbf{F}}}
\def\mG{{\mathbf{G}}}
\def\mH{{\mathbf{H}}}
\def\mI{{\mathbf{I}}}
\def\mJ{{\mathbf{J}}}
\def\mK{{\mathbf{K}}}
\def\mL{{\mathbf{L}}}
\def\mM{{\mathbf{M}}}
\def\mN{{\mathbf{N}}}
\def\mO{{\mathbf{O}}}
\def\mP{{\mathbf{P}}}
\def\mQ{{\mathbf{Q}}}
\def\mR{{\mathbf{R}}}
\def\mS{{\mathbf{S}}}
\def\mT{{\mathbf{T}}}
\def\mU{{\mathbf{U}}}
\def\mV{{\mathbf{V}}}
\def\mW{{\mathbf{W}}}
\def\mX{{\mathbf{X}}}
\def\mY{{\mathbf{Y}}}
\def\mZ{{\mathbf{Z}}}
\def\mBeta{{\mathbf{\beta}}}
\def\mPhi{{\mathbf{\Phi}}}
\def\mLambda{{\mathbf{\Lambda}}}
\def\mSigma{{\mathbf{\Sigma}}}


% Expectation
% \def\eE{\mathop{\mathbb{E}}\limits}
\def\eE{\mathbb{E}}

% Probability
\def\pP{\mathbb{P}}

% Tilde
\def\tf{\tilde{f}}
\def\tS{\tilde{S}}
\def\wtF{\widetilde{\mathcal{F}}}
\def\whR{\widehat{R}}
\def\tvx{\tilde{\mathbf{x}}}
\def\ty{\tilde{y}}


\def\defeq{\overset{\textup{def}}{=}}
% \def\defeq{\overset{.}{=}}
\def\defone{\overset{\text{\ding{172}}}{=}}
\def\deftwo{\overset{\text{\ding{173}}}{=}}
\def\leqone{\overset{\text{\ding{172}}}{\leq}}
\def\leqtwo{\overset{\text{\ding{173}}}{\leq}}
\def\leqthree{\overset{\text{\ding{174}}}{\leq}}
\def\leqfour{\overset{\text{\ding{175}}}{\leq}}
\def\eqone{\overset{\text{\ding{172}}}{=}}
\def\eqtwo{\overset{\text{\ding{173}}}{=}}
\def\eqthree{\overset{\text{\ding{174}}}{=}}
\def\eqfour{\overset{\text{\ding{175}}}{=}}
\def\geqfive{\overset{\text{\ding{176}}}{\geq}}

\usepackage{hyperref}
\usepackage{url}
\usepackage{xcolor}
\usepackage{graphicx}
\usepackage{amsmath}
\usepackage{amssymb}
\usepackage{booktabs}
\usepackage{color}
\usepackage{multirow}
\usepackage{pifont}
\usepackage{makecell}
\usepackage{xurl}

\usepackage{tikz}
\newcommand{\cmark}{\ding{51}}%
\newcommand{\xmark}{\ding{55}}%
\usetikzlibrary{tikzmark}

\title{End-to-end Training for Text-to-Image Synthesis using Dual-Text Embeddings}
% \title{Learning Dual Text Embeddings by Synthesising Images Conditioned on Text}

% Authors must not appear in the submitted version. They should be hidden
% as long as the tmlr package is used without the [accepted] or [preprint] options.
% Non-anonymous submissions will be rejected without review.

\author{\name Yeruru Asrar Ahmed \email asrar@cse.iitm.ac.in \\
      \addr Department of Computer Science and Engineering\\
      Indian Institute of Technology Madras
      \AND
      \name Anurag Mittal \email amittal@cse.iitm.ac.in \\
      \addr Department of Computer Science and Engineering\\
      Indian Institute of Technology Madras
}

% The \author macro works with any number of authors. Use \AND 
% to separate the names and addresses of multiple authors.

\newcommand{\fix}{\marginpar{FIX}}
\newcommand{\new}{\marginpar{NEW}}

\def\month{MM}  % Insert correct month for camera-ready version
\def\year{YYYY} % Insert correct year for camera-ready version
\def\openreview{\url{https://openreview.net/forum?id=XXXX}} % Insert correct link to OpenReview for camera-ready version


\begin{document}


\maketitle

\begin{abstract}
Text-to-Image (T2I) synthesis is a challenging task that requires modeling complex interactions between two modalities (\textit{, i.e.}, text and image). A common framework adopted in recent state-of-the-art approaches to achieving such multimodal interactions is to bootstrap the learning process with pre-trained image-aligned text embeddings trained using contrastive loss. Furthermore, these embeddings are typically trained generically and reused across various synthesis models. In contrast, we explore an approach to learning text embeddings specifically tailored to the T2I synthesis network, trained in an end-to-end fashion. Further, we combine generative and contrastive training and use two embeddings, one optimized to enhance the photo-realism of the generated images, and the other seeking to capture text-to-image alignment.  A comprehensive set of experiments on three text-to-image benchmark datasets (Oxford-102, Caltech-UCSD, and MS-COCO) reveal that having two separate embeddings gives better results than using a shared one and that such an approach performs favourably in comparison with methods that use text representations from a pre-trained text encoder trained using a discriminative approach.  Finally, we demonstrate that such learned embeddings can be used in other contexts as well, such as text-to-image manipulation.

\end{abstract}

% \vspace{-1cm}
\section{Introduction}
\label{sec:intro}
% \vspace{-0.2cm}

Visualizing images for any textual statement is elemental to human understanding of the world. Intelligent systems' ability to generate images from the text for human understanding has a wide range of applications such as information sharing, computer-aided design, text-to-image search, and photo editing. Image synthesis from text is a challenging task due to complex interaction and ambiguous association of the text modality with the image modality. For instance, multiple textual descriptions can describe the same image and vice versa. In addition, finer details of the images may not always be well captured in textual descriptions. In this domain, Generative Adversarial Networks (GANs) \citep{GAN_2014} were the go-to method to generate realistic images. Conditional GANs \citep{condtionalgan,acgan,cgans_projections} allow us to generate real images semantically coherent with text \citep{reed2016generative,tacgan,GAWWN} by conditioning the generation process on global sentence embeddings.

Though GANs have been shown to generate meaningful images, naively generating high-resolution images from text leads to subpar visual results and training instability due to the complex nature of the task. A set of methods attempts to solve this problem by \textit{ advancing the visual generation} part of the model. For instance, StackGAN \citep{stack_gan} employs a \textit{hierarchical stage-wise training} of GANs from a low-resolution to a high resolution and conditions the generator at every stage by images generated from the previous stage generator. 

\begin{figure*}[t]
    \centering
    \includegraphics[width=1\textwidth]{images/overall_picture.drawio.pdf}
    % \vspace{-0.5cm}
    \caption{ (a) Text and image are projected into a shared embedding space to enhance mutual information capturing discriminative features (Discriminative Embeddings), (b) Word embeddings are trained by generating images capturing semantic details (Generative Embeddings) and (c) Our Dual Text Embedding approach combining both generative and discriminative embeddings.}
    \label{fig:overall_picture}
    % \vspace{-0.8cm}
\end{figure*}

Another set of methods \citep{AttnGAN,DMGAN,CPGAN} addresses high-quality image generation from a text by improving the compatibility between text and image modalities. It is achieved by semantically aligning the visual features in image sub-regions with the pre-trained word embeddings through attention. These pre-trained embeddings are trained by projecting text and image features into a shared embedding space and maximising mutual information between text and image features using contrastive loss \citep{AttnGAN,clip}. Furthermore, these embeddings are typically trained generically and reused across various synthesis models \citep{AttnGAN,CPGAN,DF_GAN_CVPR,SSA-GAN,DALL-E,Dalle_2,make_a_scene,stable_diffusion}.  This work explores a new direction for learning text representations within the Text-to-Image (T2I) framework through unified end-to-end training, as illustrated in Figure \ref{fig:overall_picture}, to enhance compatibility between image and text modalities. Additionally, we combine generative and contrastive training while utilizing two distinct embeddings: one optimized for enhancing the photorealism of generated images and the other focused on capturing accurate text-to-image alignment as shown in Figure \ref{fig:overall_picture}. We evaluate the proposed approach on three datasets: 1) Oxford-102 \citep{flower_dataset}, 2) Caltech-UCSD Birds 200 (CUB) \citep{CUB_dataset}, and 3) MS-COCO \citep{mscoco}. We use three metrics to assess the generated images: \textit{Inception Score (IS)} \citep{IS_score}, \textit{Fréchet Inception Distance (FID)} \citep{FID_score} for image quality, and \textit{R-precision} \citep{AttnGAN} to measure text-image alignment. Our model reduces the FID score from $14.06$ to $13.67$ on CUB and $40.31$ to $30.07$ on Oxford-102. For MS-COCO, the FID drops from $35.49$ to $25.17$, outperforming AttnGAN \citep{AttnGAN}, which uses embeddings trained generically in discriminative approach using contrastive loss.  Furthermore, we observe that employing separate embeddings gives superior results compared to a shared embedding approach as verified in Section \ref{sec:ablationdualemb}.  Finally, we demonstrate that such learnt dual-text embeddings can be used in other contexts as well, such as for text-to-image manipulation.

The contributions of this paper are summarized as follows:

% \vspace{-0.2cm}
\begin{itemize}
    \item We propose a novel approach for learning text embeddings specifically tailored to the requirements of Text-to-Image synthesis networks, optimized through an end-to-end training paradigm to enhance synthesis performance.
    \item We introduce a dual-embedding framework that integrates generative and contrastive training paradigms, with one embedding optimized for photo-realism and the other for robust text-to-image alignment.
    \item Our approach demonstrates competitive performance compared to methods that utilize text representations derived from pre-trained text encoders optimized using a discriminative training paradigm.
    \item We show the application of such learnt embeddings for additional tasks, such as text-to-image manipulation.
\end{itemize}

% \vspace{-0.8cm}
\section{Related Work}
% \vspace{-0.2cm}
In this section, some of the relevant works in the literature relating to this paper are discussed briefly.

\noindent\textbf{Generative Adversarial Networks:}  In past few years, GANs \citep{GAN_2014} had been the go-to method for generating images and class-specific images \citep{condtionalgan,acgan,cgans_projections} on small datasets such as MNIST \citep{MNIST} and CIFAR \citep{cifar}. However, GAN training is highly unstable when used to generate images on large datasets such as ImageNet \citep{imagenet}. Researchers have explored to fix this training instability by re-framing  GAN loss and regularisation \citep{wgan,I_wgan,lsgan,sn_gan,orthogonal_regularisation} to generate high-resolution images on large datasets \citep{progressive_gan,big_gan}. 

\begin{figure*}[t]
    \centering
    \includegraphics[width=1\textwidth]{images/Dte_arch.pdf}
    %Model_achitecute.drawio.pdf}
    \vspace{-0.2cm}
    \caption{Overview of DTE-GAN architecture. DTE-GAN consists of three core components: i) a single-stage generator $G$ (Section \ref{sec:generator}), ii) a discriminator $D$ (Section \ref{sec:discriminator}), and iii) a dual text embedding setup (Section \ref{sec:dualtextembed}). In the Figure, $W_G$ = generator-side word embeddings, $S_G$ = generator-side sentence embedding, $W_D$ = discriminator-side word embeddings, $S_D$ = discriminator-side sentence embedding. The model is optimised using two objective functions: 1) adversarial loss, and 2) multi-modal contrastive loss.}
    \label{fig:modelarch}
    % \vspace{-0.9cm}
\end{figure*}

\noindent \textbf{Text-to-Image synthesis:} GANs conditioned on global sentence-level embeddings are known to generate meaningful images at low resolutions \citep{GAWWN,tacgan,reed2016generative}. StackGAN \citep{stack_gan} generates high-resolution images in stage-wise approach, where the generator at each stage is conditioned by the image generated from the previous stage. Unlike StackGAN, HDGAN \citep{hd_gan} trains a single generator and multiple discriminators for each resolution. 
AttnGAN \citep{AttnGAN} uses text embeddings to fine-tune image features and also introduces a multimodal contrastive loss (DAMSM loss) to bridge the gap between generated images and words. DM-GAN \citep{DMGAN} refines words and image features using a memory module. MirrorGAN \citep{MirrorGAN} generates a caption for the generated images that improves the text \textit{vs.} image semantic consistency. SD-GAN \citep{SDGAN} introduces a Siamese structure for the generator that uses Conditional Batch Normalization (CBN) \citep{CBN} to improve the alignment of text-image. CPGAN \citep{CPGAN} learns a memory-attended text encoder by attending to salient features in images for each word and fine-grained discriminator \citep{control_gan}.  DTGAN \citep{DTGAN} applies channel and spatial attention, conditioned on sentence vector to focus on important features for each textual representation. XMC-GAN \citep{XMC-GAN} maximises the mutual information between text and image using intra-modality and inter-modality contrastive losses. DF-GAN \citep{DF_GAN_CVPR} uses deep affine transformed global sentence embedding to condition the geand theator and the matching-aware discriminator. 
In this space of text-to-image semantic alignment-based methods, pre-trained embeddings are an inherent prerequisite. These embeddings are only trained by the discriminative approach. Unlike these methods, our apporach (DTE) attempts to learn text embeddings that capture generative and discriminative properties.  

\noindent \textbf{Generative embedding learning:} Some methods attempt to learn embeddings (text / visual) end-to-end as part of a generator. For example, \citep{NDR,VQGAN} learn discrete embeddings for visual representation and show substantial improvement in the performance of text-to-image synthesis \citep{DALL-E,CogView}. Further better and compact representation are learned to improve the quality of image generation \citep{vq_vae_2,res_quatizer,make_a_scene}. Unlike these works that consider only the generation process while learning embeddings, DTE explores the capture of different perspectives of generation and discrimination process by learning dual text embeddings.

\noindent \textbf{Large Scale Text-to-Image Synthesis:} Denoising Diffusion Probabilistic models \citep{denosing_diffusion} have currently achieved remarkable success in image generation \citep{ddpm,imporved_ddpm,diffusion_beats_GAN} by reversing the \textit{forward Markovian process} with noise removal in multiple steps. Though diffusion-based models are able to generate images with complex and varied interactions for text and generate high-quality images \citep{Dalle_2,imagen,Vq_diffusion}, these approaches require large-scale training and exploit pre-trained discriminative language models like CLIP \citep{clip}. Further, CLIP-based language and image encoders are used as a bootstrapping approach for predicting conditional representations in large-scale GAN-based approaches \citep{galip, scaling_up_gan, lafite}. Unlike CLIP, which is trained to capture text-image alignment only, DTE is proposed to learn text-image alignment and text representation toimprove image realism. 

% \vspace{-0.4cm}
\section{Methodology}
% \vspace{-0.2cm}
In this section, an end-to-end framework called "\textit{Dual Text Embedding GAN}" (DTE-GAN) is formulated for learning text embeddings learn tailored to the T2I synthesis network in an end-to-end manner while capturing different representation for text for improving photo-realism and capturing text-image alignment. In the following sub-sections, the overall architecture is introduced followed by specific details on generator and discriminator architectures respectively; then the final sub-section focuses on learning dual text embeddings.

% \vspace{-0.2cm}
\subsection{Model overview}
\label{sec:overview}

DTE-GAN consists of three core components: 1) a  novel dual text embedding setup,  2) a single-stage generator $G$, and 3) a discriminator $D$. The overall architecture is shown in Figure \ref{fig:modelarch}.

First, the given text $T$ is passed through the novel dual text embedding procedure to encode the text into two types of sentence embeddings, namely: 1) Generator-side sentence embedding $S_G$, and 2) Discriminator-side sentence embedding $S_D$. 
The dual text embedding setup consists of two separate Bi-LSTM \citep{bi-lstm} text encoders (\textit{Generator-side} and \textit{Discriminator-side}) and their own independent word embeddings ($W_G$ \& $W_D$). 
As the name suggests, the \textit{Generator-side} word embeddings $W_G$ and its encoder are intended to be trained from the image-generation process (generator $G$) and its losses, while the \textit{Discriminator-side} word embeddings $W_D$ and its encoder are optimised by the contrastive loss at the discriminator $D$. Such a decoupling between these two parts of embedding adds flexibility in capturing different natures of image generations and discrimination processes. Specifically, the image generation process warrants learning representation for creating an image, while the discrimination process strives to learn features for improving text-image alignment. Further, the separation of these two embeddings allows $W_G$ to learn from noisy gradients of $G$ independently (as $G$'s gradients are initially noisy due to fake image - sentence pairs), while $W_D$ learns from stable gradients of $D$ (real image - sentence pairs).

During training, $W_G$, the generator-side text encoder, and Generator $G$ are trained from gradient signals of the generation process \textit{i.e.,} Adversarial loss of fake images ($I_{fake}$)  and multi-modal contrastive loss between the generated image $I_{fake}$ and the given text $T$. Next, $W_D$ and the discriminator-side text encoder are trained from the gradient signals of the multi-modal contrastive loss between the real image $I_{real}$ and the given text $T$. Further, discriminator $D$ is trained from Adversarial loss ($I_{fake}$, $I_{real}$) and multi-modal contrastive loss between the real image $I_{real}$ and the given text $T$.

% \vspace{-0.45cm}
\subsection{Generator}
\label{sec:generator}
% \vspace{-0.35cm}


As opposed to other methods that use stacks of GANs or hierarchical GAN, a single-stage generator is employed that can generate an image at any resolution owing to its simpler design and easier training procedure. 
The generator $G$ takes three inputs: i) a noise vector $z$ of dimension $d_z$ from a Standard Gaussian Distribution $\mathcal{N}(0,1)$ with the Truncation Trick \citep{big_gan,DF_GAN_CVPR,DTGAN}, ii) the generator-side sentence embedding $S_G$ of dimension $d_{SG}$, and iii) the discriminator-side sentence embedding $S_D$ of dimension $d_{SD}$. Next, the two sentence embeddings ($S_G$, $S_D$) together are passed through the conditioning augmentation \citep{stack_gan} to get the conditional vector $C_t$ which is concatenated with a noise vector $z$ sampled from a Standard Gaussian Distribution $\mathcal{N}(0,1)$ to form the input vector $f_G$ and passed through a fully connected layer with reshape to create a low-resolution spatial feature map. It may be noted that, in this step, $S_D$ is detached from the gradient flow of $G$ to avoid getting gradients from the generation process (refer to ablation studies in Section \ref{sec:ablationdualemb}).

Further, this low-resolution feature map is passed through a set of upsampling blocks (\textit{UpBlock}) followed by a convolution layer that accepts the last high-resolution feature map and outputs the generated image $I_{fake}$ of dimension $3 \times h \times w$ ($h=$ height, $w=$ width). Each \textit{UpBlock} is formulated as a residual layer consisting of a bi-linear upsampling step followed by two convolution blocks (convolutional layer + Conditional Batch normalisation (CBN)\citep{CBN} + LeakyReLU \citep{lrelu}). To increase the stochastic capability of the model, scaled noise is added (similar to StyleGAN \citep{styleGAN}) to input before passing to the convolutional layer. The modulation parameters in Conditional Batch normalisation (CBN) \citep{SDGAN,CBN} $\gamma_c$, and $\beta_c$ are calculated from $f_G$ by means of a linear projection layer. The modulation parameters $\gamma_c$, and $\beta_c$ in CBN are calculated as follows:

\vspace{-0.6cm}
\begin{align}
    \operatorname{BN}(x \mid C_t,z) &=\left(\gamma+\gamma_{c}\right) \cdot \frac{x-\mu(x)}{\sigma(x)}+\left(\beta+\beta_{c}\right)\\%\nonumber\\
    f_G &= \text{Concat}[C_t,z]\\%\nonumber\\
    \gamma_{c} &={FC}_{\gamma}(f_G) \\ %\nonumber\\
    \beta_{c} &={FC}_{\beta}(f_G)
\end{align}


The generator is trained to minimise adversarial loss ($\mathcal{L}_{Adv}^G$), and multi-modal contrastive loss ($\mathcal{L}_{\text{cont}}^{G}$).
$\mathcal{L}_{\text{cont}}^{G}$ is formulated as a loss between the features from the generated image and the discriminator-side sentence embeddings. Mathematically, the objective functions can be written as follows:

\vspace{-0.6cm}
\begin{align}
    \mathcal{L}_{Adv}^{G} &= \mathbb{E}_{\hat{x} \sim p_{G}}[-D(\hat{x})] \\%\nonumber\\
    % \mathcal{L}_{CA} = D_{K L}\left(\mathcal{N}\left(\mu\left(C_{t}\right), \Sigma\left(C_{t}\right)\right) || \mathcal{N}(0, I)\right) \\
    \mathcal{L}_{\text{cont}}^{G}\left(\hat{f}_{v_i}, S_{D_i}\right) &=-\log \frac{\exp \left(Sim\left(\hat{f}_{v_i}, S_{D_i}\right)\right)}{\sum_{j=1}^{N} \exp \left(Sim\left(\hat{f}_{v_i}, S_{D_j}\right)\right)}\\%\nonumber\\
    Sim(f_v, S_D) &=\cos \left(f_v, S_D\right) / \tau 
\end{align}

Here, $Sim(.,.)$ is a score function to calculate the similarity between sentence embeddings and image features, $\cos (u, v)=u^{T} v /\|u\|\|v\|$ is the Cosine Similarity between features and $\tau$ denotes the temperature hyper-parameter, and $\hat{f}_{v}$ represents visual features extracted by the discriminator for the generated image $I_{fake}$. We use conditioning augmentation \citep{stack_gan} to sample the sentence condition from an independent Gaussian Distribution $\mathcal{N}\left(\mu\left(s_{t}\right), \Sigma\left(s_{t}\right)\right)$. The regularisation term from conditioning augmentation ($\mathcal{L}_{CA}$) for combined sentence embeddings($s_t$) is:

\vspace{-0.6cm}
\begin{align}
\mathcal{L}_{CA} = D_{K L}\left(\mathcal{N}\left(\mu\left(s_{t}\right), \Sigma\left(s_{t}\right)\right) \| \mathcal{N}(0, I)\right)
\end{align}

Here $\mu(s_t)$ and $\Sigma(s_t)$ are mean and diagonal covaraince matrices that are computed as functions of the combined sentence embedding. The regualarisation term is KL Divergence between the Conditioning Gaussian and a Standard Gaussian Distribution. The final loss for generator is defined as:

\vspace{-0.6cm}
\begin{equation}
     \mathcal{L}_{G} = \mathcal{L}_{Adv}^{G} +\lambda_1 \mathcal{L}_{CA} + \lambda_2 \mathcal{L}_{\text{cont}}^{G} 
 \end{equation}

 % \vspace{-0.4cm}
 \subsection{Discriminator}
\label{sec:discriminator}
% \vspace{-0.2cm}

The discriminator $D$ is designed to serve two purposes: (1) to be a critic to determine whether the image is real or fake, and (2) to be a feature encoder to extract image features for multi-modal contrastive loss. The given image ($I_{real}$ or $I_{fake}$) is passed through a series of downsampling blocks (\textit{DownBlock}s), until the feature map is of size $8 \times 8$. Next, these $8 \times 8$ dimensional spatial features are passed through two separate branches: one for extracting features for the adversarial loss and the other for computing image features for multi-modal contrastive loss. For the adversarial branch, the input is passed through a DownBlock, ResBlock, and a fully connected layer to predict the logit to represent if the given image is real or fake. The predicted logit is used as input to an Adversarial Hinge loss \citep{sn_gan} $\mathcal{L}_{Adv}^{D}$ as follows:  

\vspace{-0.6cm}
\begin{equation}
\begin{split}
\mathcal{L}_{Adv}^{D}=\mathbb{E}_{x \sim p_{\text {data }}}[\max (0,1-D(x))] & \\  +  \mathbb{E}_{\hat{x} \sim p_{G}}[\max (0,1+D(\hat{x}))]& 
\end{split}
\end{equation}

Here, $x$ and $\hat{x}$ are real ($I_{real}$) and generated ($I_{fake}$) images.

In the multi-modal contrastive loss branch, the features are passed through a DownBlock, ResBlock, and a linear projection layer to output visual features $f_v$. The multi-modal contrastive loss $\mathcal{L}_{\text {cont }}^D$ takes as input the real image features $f_{v_i}$ and sentence embeddings $S_{D_i}$ and calculates the contrastive loss to increase the mutual information in text and image as follows:  

\vspace{-0.6cm}
\begin{equation}
    \mathcal{L}_{\text{cont}}^D\left(f_{v_i}, S_{D_i}\right)=-\log \frac{\exp \left(Sim\left(f_{v_i}, S_{D_i}\right)\right)}{\sum_{j=1}^{N} \exp \left(Sim\left(f_{v_i}, S_{D_j}\right)\right)}
\end{equation}



 $\mathcal{L}_{\text {cont }}^D$ is the contrastive loss between real image - text pairs.  The final objective function for the Discriminator is defined as:

 \vspace{-0.6cm}
 \begin{equation}
     \mathcal{L}_{D} = \mathcal{L}_{Adv}^{D} + \lambda_3 \mathcal{L}_{\text{cont}}^D
 \end{equation}


\subsection{Dual text embedding learning}
\label{sec:dualtextembed}
% \vspace{-0.2cm}

Embeddings can be viewed as memory representation learned by reducing a loss. Multiple embeddings, each learned by optimising on different losses, will capture various memory representations for the same word. The goal of the dual text embedding setup is to learn generator-side word embeddings $W_G$ (along with its encoder) to capture complex representation of words to aid improve the photo-realism of the generated images and discriminator-side word embeddings $W_D$ (along with its encoder) to capture distinctive features for words to align text-image associativity. To achieve this, we make sure that $W_G$ receives only the gradients from image-generation process whereas $W_D$ receives gradients from the contrastive loss. Specifically, we formulate the generator-side embedding loss ($\mathcal{L}_{emb}^G$) and the discriminator-side embedding loss ($\mathcal{L}_{emb}^D$) as follows:

% \vspace{-0.5cm}
 \begin{align}
     \mathcal{L}_{emb}^G &= \mathcal{L}_{G} \\%\nonumber\\
     \mathcal{L}_{emb}^D &= \lambda_3 \mathcal{L}_{\text{cont}}^D
 \end{align}
 % \vspace{-0.4cm}

Here, $\mathcal{L}_{G}$ denotes the loss function for the generator $G$, $\mathcal{L}_{\text{cont}}^D$ denotes the multi-modal contrastive loss between real image ($I_{real}$) features and discriminator-side sentence embedding $S_D$ for the given text T.

  \begin{figure*}[!ht]
    % \vspace{-0.4cm}
    \centering
    \includegraphics[width=1\textwidth]{images/bird_results_DFGAN.drawio.pdf}
    % \includegraphics[width=1\textwidth]{images/coco_results_main.drawio.pdf}
    % \vspace{-0.4cm}
    \caption{Visual comparision of the images generated by DF-GAN \citep{DF_GAN_CVPR} and DTE-GAN on CUB\citep{CUB_dataset} and COCO\citep{coco_dataset} Datasets. }
    \label{fig:bird_results}
\end{figure*}

\begin{figure*}[!ht]
    % \vspace{-1.2cm}
    \centering
    \includegraphics[width=1\textwidth]{images/flower_results_3.drawio.pdf}
    % \vspace{-0.4cm}
    \caption{Illustration of the images generated by HDGAN \citep{hd_gan} and those of  DTE-GAN on Oxford-102 Flower Dataset \citep{flower_dataset}.}
    \label{fig:flower_results}
\end{figure*}


% \vspace{-0.4cm}
\section{Experiments}
\label{sec:Experiments}
% \vspace{-0.2cm}
 
 In this section, datasets and evaluation metrics are introduced for experiments. Futher, proposed DTE is evaluated and compared quantitatively and qualitatively with other methods in the literature. The specific training details and hyper-parameters are mentioned in the supplementary material.

\textbf{Datasets:} DTE-GAN is evaluated on three datasets, namely, 1) Caltech-UCSD birds (CUB) \citep{CUB_dataset}, 2) Oxford-102 flowers \citep{flower_dataset}, and 3) MS COCO \citep{mscoco} datasets. For CUB and Oxford-102 datasets, we have similar setup to StackGAN \citep{stack_gan}. Ten captions are provided for each image in both the datasets. The MS-COCO dataset consists of around 80k training and 40k validation images; and for every image, there are 5 captions provided with the dataset.  

\begin{table*}[!ht]
\centering
\begin{tabular}{lcccccc}
\toprule
%\multirow{2}{*}{\textbf{\makecell[l]{Variant \\ name}}} & 
\multirow{2}{*}{\textbf{\makecell{Method}}} & \multicolumn{3}{c}{\textbf{CUB}} & \multicolumn{3}{c}{ \textbf{COCO}} \\
\cmidrule{2-7}
    &\textbf{IS} $\uparrow$   & \textbf{FID} $\downarrow$   & \textbf{R\%} $\uparrow$   & \textbf{FID} $\downarrow$      & \textbf{R\%} $\uparrow$     & \textbf{NoP} $\downarrow$     \\
\midrule
StackGAN \citep{stack_gan} & $3.70 \pm .04$  & - & - & - & - & -\\
AttnGAN \citep{AttnGAN} & $4.36 \pm .02$  & $23.98$ & $67.82$ & $35.49$ & $83.82$ & $230M$\\
MirrorGAN \citep{MirrorGAN} & $4.56 \pm .17$  & $18.32$ & $57.67$ & $34.71$ & $74.53$ & -\\
 DM-GAN \citep{DMGAN} & $4.75 \pm .07$ & $16.09$ & $72.32$ & $32.64$ & $88.56$ & $46M$\\
 KT-GAN \citep{KT-GAN} & $4.85 \pm .05$ & $17.32$ & - & $30.73$ & - & -\\
 TIME \citep{TIME} & $4.91 \pm .04$ & $\color{blue}14.30$ & $71.57$& $31.14$ & - & $120M$\\
 DAE-GAN \citep{DAE-GAN} & $4.42 \pm .04$ & $15.19$ & $\color{green}85.45$ & $28.12$ & $\color{red}92.61$ & $98M$\\
 CSM-GAN \citep{CSM-GAN} & $4.62 \pm .08$ & $20.18$ & - & $33.48$ & - & -\\
 DR-GAN \citep{DR-GAN} & $4.90 \pm .05$ & $14.96$ & - & $27.80$ & - & $73M$\\
 ALR-GAN \citep{ALR-GAN} & $4.96 \pm .04$ & $15.14$ & $77.54$ & $29.04$ & $69.20$ & $76M$\\
 
 DF-GAN \citep{DF_GAN_CVPR} & $4.86 \pm .04$ & $14.81$ & - & $\color{red}19.32$ & - & $\color{green}19M$\\
 SSA-GAN \citep{SSA-GAN} & $ \color{red} 5.17 \pm .08$ & $15.61$ & $\color{blue}85.4$ & $\color{green}19.37$ & $\color{blue}90.6$ & $\color{blue}26M$\\
\midrule

\textbf{DTE-GAN}& $\color{green}5.12 \pm .04$  & $\color{red} 13.67$ & $\color{red}86.64$& $25.17$ & $\color{green}90.82$ & $\color{red}11M$\\
         % \hline
 \textbf{DTE-GAN+MAGP}& $\color{blue}5.09 \pm .02$  & $\color{green}13.94$ & $81.33$  & $\color{blue}19.69$ & $88.39$ & $\color{red}11M$\\
 
      \bottomrule
\end{tabular}
\caption{Quantitative comparison between DTE-GAN and other models on CUB \citep{CUB_dataset} and COCO \citep{coco_dataset} datasets. "-" indicates values are unreported. The best three results are marked with \textcolor{red}{red}, \textcolor{green}{green}, and \textcolor{blue}{blue}, respectively. ‘‘$\uparrow$’’ indicates the higher, the better, while ‘‘$\downarrow$’’ indicates the lower, the better.}
\label{tab:Bird_COCO_table}
% \vspace{-1.2cm}
\end{table*}

\subsection{Visual comparison}
% \vspace{-0.2cm}
The generated images are visually compared between DF-GAN \citep{DFGAN} and DTE-GAN in the CUB and COCO datasets. In Figure \ref{fig:bird_results}, it is evident that DF-GAN struggles to depict complete bird shapes. The presented model, employing dual text embeddings, minimizes image generation loss, improving shape accuracy and realistic fine-grained features in generated images. Furthermore, DTE-GAN outperforms DF-GAN in pose representation, resulting in more natural-looking images. With regard to semantic consistency between images and text, DTE-GAN captures detailed structures and overall coherence compared to DF-GAN. Despite not using word embeddings for image region attention, DTE-GAN's ability to learn embeddings for both generation and discrimination allows it to generate images with finer details. DTE-GAN produces images that resemble real images due to its learned embeddings encompassing generation and discrimination aspects. For the COCO dataset, DTE-GAN generates images of comparable quality to DF-GAN while using significantly fewer parameters. This efficiency is achieved by learning tailored text embeddings specifically for the synthesis model. In Figure \ref{fig:flower_results}, images for the Oxford-102 dataset are generated and compared with those of HDGAN \citep{hd_gan}. We can observe that our model is able to capture the complex variation of the flowers and generate more realistic images than HDGAN.


\begin{table}[!ht]
    \centering
    \begin{tabular}{lcc}
         \toprule
         {\textbf{Method}}& {\textbf{IS} $\uparrow$} & \textbf{FID} $\downarrow$\\ 
         \midrule
         StackGAN \citep{stack_gan} & $3.20 \pm .01$ & $51.89$  \\
         StackGAN++ \citep{stackgan++} & $3.26 \pm .01$ & $48.68$ \\
         HDGAN \citep{hd_gan} & $3.45 \pm .07$ & $43.17$ \\
         % SegAttnGAN \citep{SegAttnGAN} &$ 3.36 \pm .08$ & - \\
         SSTIS \citep{SSTIS} &$ 3.37 \pm .05$ & - \\
         SS-TiGAN \citep{SSBI} &$ 3.45 \pm .04$ & $40.54$ \\
         DualAttn-GAN \citep{Dual_Attn_GAN} &$ 4.06 \pm .05$ & $\color{blue}40.31$ \\
         RAT-GAN \citep{RATGAN} &$ \color{blue}4.09  $& - \\
         \midrule
         \textbf{DTE-GAN}& $\color{green}4.21 \pm .08$  & $\color{red}30.07$\\
         \textbf{DTE-GAN+MAGP}& $\color{red}4.26 \pm .07$  & $\color{green}31.13$\\
         \bottomrule
    \end{tabular}
    \caption{Quantitative comparison between DTE-GAN and other models on Oxford-102 Dataset. The best three results are marked with \textcolor{red}{red}, \textcolor{green}{green}, and \textcolor{blue}{blue}, respectively. ‘‘$\uparrow$’’ indicates the higher, the better, while ‘‘$\downarrow$’’ indicates the lower, the better."-" indicates values are unreported.}
    \label{tab:Flower_table}
    % \vspace{-1.1cm}
\end{table}

% \vspace{-0.5cm}
\subsection{Quantitative Evaluation}
% \vspace{-0.2cm}

\textbf{Evaluation metrics:} To evaluate the quality of images generated, the following metrics used are: \textit{Inception Score (IS)} \citep{IS_score} calculates the Kullback-Leibler (KL) divergence between a conditional distribution and marginal distribution for class probabilities from Inception-v3 \citep{szegedy2016rethinking} model. Higher IS suggest generated images are from higher diverse classes. \textit{Fr\'echet Inception Distance (FID)} \citep{FID_score} calculates the Fr\'echet Distance between two multivariate Gaussians, which are fit to the global features extracted from the Inception-v3 \citep{szegedy2016rethinking}  model on the synthetic and generated images. Lower the FID suggest generated images closer to real images. \textit{R-precision} (R\%) precision evaluates text-to-image alignment, by assessing whether generated images can be used to retrieve the text. 


The performance of proposed model is compared with that of the lightweight GAN approaches (having similar training setups) for the task of text-to-image synthesis on CUB and COCO datasets in Table \ref{tab:Bird_COCO_table}. From Table \ref{tab:Bird_COCO_table}, on the CUB dataset, we observe that DTE-GAN improves \textit{IS} from $4.91$ to $5.12$, achieves the best \textit{R-precision} of $86.64$ and further decreases \textit{FID} from $14.06$ to $13.67$. We also train our DTE-GAN with the proposed regularisation trick Matching-Aware Gradient Penalty (MAGP) \citep{DF_GAN_CVPR}, which smooths out discriminator function and allows to generate more realistic images. On the CUB dataset, compared to AttnGAN \citep{AttnGAN} that employs contrastive loss-based embeddings, our model decreases \textit{FID} from $23.98$ to $13.67$. Such an improvement illustrates the effectiveness of the end-to-end learned dual embeddings over fixed pre-trained embeddings learned on the same data. On MS-COCO \citep{mscoco} dataset (in Table \ref{tab:Bird_COCO_table}), we achieve similar performance of DF-GAN and SSA-GAN with fewer parameters, underscoring the efficiency of the proposed DTE and its ability to learn embeddings tailored specifically to the synthesis model. DTE-GAN+MAGP achieves similar performance as that of SSA-GAN \citep{SSA-GAN} on COCO dataset as SSA-GAN and DF-GAN \citep{DF_GAN_CVPR} both incorporating MAGP. 

As shown in the Table \ref{tab:Bird_COCO_table}, we have employed a network with significantly fewer parameters by reducing the width of the layers by half compared to SSA-GAN \cite{SSA-GAN} and DF-GAN \cite{DFGAN} in their respective UpBlock and DownBlock. Despite this reduction in complexity, our model achieves comparable results by learning more effective text encoding representations, which improve the performance of Text-to-Image synthesis

Following previous works \citep{DF_GAN_CVPR,DTGAN,SSA-GAN}, we report only \textit{FID} scores, as \textit{IS} scores for the MS-COCO dataset do not reflect the quality of the synthesised images. In comparison to TIME \citep{TIME} which learns embeddings along with the Text-to-Image synthesis model, DTE-GAN achieves significant improvement (0.6 in \textit{FID}, +15\% in R-precision) demonstrating the effectiveness of dual text embeddings. In Table \ref{tab:Flower_table}, on the Oxford-102 dataset, we use \textit{IS} and \textit{FID} scores for evaluation, as R-precision scores are not available in the literature. In this dataset, our model improves \textit{IS} score from $4.06$ to $4.21$ over the state-of-the-art (DualAttn-GAN \citep{Dual_Attn_GAN}, LeicaGAN \citep{Leica_gan}) models and remarkably decreases \textit{FID} from $40.31$ to $30.07$. 

\vspace{-0.3cm}
\subsection{Additional Studies}
\vspace{-0.3cm}

\begin{figure*}[t]
    % \vspace{-1.5cm}
    \centering
    \includegraphics[width=1\textwidth]{images/clip_images.drawio.pdf}
    \includegraphics[width=1\textwidth]{images/bird_clip_images.drawio.pdf}
    \vspace{-0.6cm}
    \caption{Images generated using CLIP, CLIP + Learnable Generator side embeddings (CLIP + $\mathbf{G_{emb}}$) and DTE on CUB and Oxford-102 datasets.}
    \label{fig:DTE_vs_CLIp}
    \vspace{-0.3cm}
\end{figure*}

\begin{table}[!ht]
\centering
\begin{tabular}{ccccc}
\toprule
                 \textbf{Dataset} & \textbf{Embeddings}  & \textbf{IS} $\uparrow$ & \textbf{FID} $\downarrow$ & \textbf{R \%} $\uparrow$\\
\midrule
\multirow{3}{*}{\textbf{CUB}} & \textbf{CLIP}  & $4.53$ & $21.33$ & $74.12$\\
%  \cline{2-5}
                     & \textbf{CLIP+$\mathbf{G_{emb}}$ } & $4.51$ & $19.36$ & $78.35$\\
                     
                     & \textbf{DTE} & $\color{red}5.12$ & $\color{red}13.67$ &  $\color{red}86.64$   \\
\midrule
\multirow{3}{*}{\textbf{Oxford}} & \textbf{\textbf{CLIP}}  &  $3.72$ & $38.36$ & $71.78$   \\
%  \cline{2-5}
                     & \textbf{CLIP+$\mathbf{G_{emb}}$} &$3.81$ & $35.93$ &  $73.87$   \\
                     
                     
                     & \textbf{DTE} & $\color{red}4.21$  & $\color{red}30.07$ & $\color{red}83.19$  \\
\bottomrule
\end{tabular}
\caption{We compare quality of T2I generation of our proposed DTE approach with that of models trained using CLIP \citep{clip} and CLIP+$\mathbf{G_emb}$. The best results are \textcolor{red}{red}. ‘‘$\uparrow$’’ indicates the higher, the better, while ‘‘$\downarrow$’’ indicates the lower, the better.}
\label{tab:dte_vs_clip}
\end{table}

% \vspace{-1.5cm}
\subsubsection{DTE vs CLIP:}
\label{sec:DTE_VS_CLIP}

We compare DTE with pre-trained CLIP embeddings by training a GAN for text-to-image generation. Additionally, we train another GAN model resembling the DTE setup. This model uses learnable Generator-side embeddings and pre-trained CLIP embeddings for the discriminator (referred to as CLIP+ $G_{emb}$). Table \ref{tab:dte_vs_clip} compares image quality using different CUB and Oxford-102 datasets embeddings. CLIP+$\mathbf{G_{emb}}$ improves over just CLIP. In Figure \ref{fig:DTE_vs_CLIp}, CLIP images differ from reality, while CLIP+$\mathbf{G_{emb}}$ matches better text and real images due to learned generative embeddings. DTE's combined approach creates images closer to real images. Furthermore, we demonstrate that this approach can be integrated with pre-trained vision-language models, specifically CLIP+$\mathbf{G_{emb}}$ , where only the generator-side embeddings are learned. This method of focusing solely on generator-side embeddings can be seamlessly incorporated into current diffusion-based Text-to-Image models \citep{stable_diffusion,Dalle_2} within the existing training frameworks and emphasises learning embedding as needed for synthesis model.

\subsubsection{Generalisation ability of DTE:}
%AttnGAN with DTE approach

To assess how well the DTE setup applies to other architectures, we integrate it into AttnGAN \citep{AttnGAN}, now called AttnGAN+DTE. In AttnGAN, pre-trained text embeddings (DAMSM embeddings) that are generic is used training synthesis model. Instead, we train embeddings scratch using DTE. Incorporating DTE into AttnGAN involves modifying its discriminators to include a dual branch (adversarial loss and multi-modal contrastive loss) after reaching $8 \times 8$ feature size. As AttnGAN uses words for alignment in the attention layer, we introduce a word-contrastive loss \citep{AttnGAN,XMC-GAN} in the final discriminator as an additional branch when the feature size is $8 \times 8$, aimed at reducing the semantic gap between words and image features. We combine generator-side and discriminator-side word embeddings to provide word features for the Generator's attention mechanism. The results in Table \ref{tab:attn_dte_table} show that AttnGAN+DTE can train without pre-trained embeddings and even improve the results, proving that DTE can be integrated well with other methods.

\begin{table}[!ht]
    % \vspace{-0.5cm}
    \centering
    % \vspace{-0.1cm}
    \begin{tabular}{lccc}
         \toprule
         {\textbf{Method}}&  \textbf{IS} $\uparrow$ &\textbf{FID} $\downarrow$&\textbf{R\%} \\ 
         \midrule
         AttnGAN & $4.36 \pm .02$  & $23.98$& $67.82$\\
         AttnGAN+DTE &  $4.38 \pm .03$ & $21.45$& $71.39$\\ 
         DTE-GAN & $\color{red}5.12 \pm .04$  & $\color{red}13.67$ & $\color{red}86.64$\\
         \bottomrule
    \end{tabular}
    \caption{Impact of DTE approach on AttnGAN \citep{AttnGAN} on CUB Dataset \citep{CUB_dataset}. The best results are \textcolor{red}{red}. ‘‘$\uparrow$’’ indicates the higher, the better, while ‘‘$\downarrow$’’ indicates the lower, the better.}
    \label{tab:attn_dte_table}
    % \vspace{-2cm}
\end{table}

\begin{figure}[!ht]
    \vspace{-0.2cm}
    \centering
    \includegraphics[width=0.6\textwidth]{images/manipulation_images_birds_5_box.drawio.pdf}
    \vspace{-0.2cm}
    \caption{Examples of manipulated images generated by LightWeight GAN  \citep{lightweight_GAN} using DTE-GAN pre-trained embeddings on CUB dataset. Source images are manipulated by the caption of concept images.}
    \vspace{-0.4cm}
    \label{fig:manipulation_results}
\end{figure}


\begin{table}[!ht]

    % \vspace{-0.5cm}
    \centering
    \begin{tabular}{lcc}
        \midrule
         \textbf{Method}& \textbf{IS} $\uparrow$ & \textbf{FID} $\downarrow$ \\
         \midrule
        %  SISGAN \citep{sisgan} & $2.24$ & - \\
        %  TAGAN \citep{tagan} & $3.32$ & - \\
         MANIGAN  & $8.48$ & $9.75$ \\
         LWGAN  & $8.26$ & $8.02$  \\
         LWGAN w/ DTE-EMB & $\color{red}8.56$ & $\color{red}7.77$ \\
         \bottomrule
    \end{tabular}
    \vspace{-0.2cm}
    \caption{Quantitative comparison of Inception score and FID for manipulated images on CUB dataset. We use Lightweight GAN \citep{lightweight_GAN} which we name as LWGAN with the pre-trained embeddings using DTE-GAN (LWGAN w/ DTE-EMB). The best results are \textcolor{red}{red}. ‘‘$\uparrow$’’ indicates the higher, the better, while ‘‘$\downarrow$’’ indicates the lower, the better.}
    \label{tab:manipulations}
\end{table}

\vspace{-0.2cm}
\subsubsection{Application to Text-to-Image manipulation task:} 
\vspace{-0.2cm}
To demonstrate the versatility of the learned dual embeddings, we apply them to text-to-image manipulation tasks. We train dual text embeddings through DTE-GAN on the CUB dataset for text-to-image synthesis. We then utilise the pre-trained word embeddings ($W_G, W_D$) from this synthesis task for text-to-image manipulation in the Lightweight GAN for Text-to-Image manipulations \citep{lightweight_GAN} task on the same dataset. It is important to note that these pre-trained dual text embeddings remain fixed during training of manipulation network. In Table \ref{tab:manipulations}, the quantitative performance of the model using these pre-trained embeddings is compared with that of other text-to-image manipulation models.  The model improves the Inception score from 8.48 (MANIGAN \citep{manigan}) to 8.56 and reduces the \textit{FID} score from 8.02 \citep{lightweight_GAN} to 7.77 on the CUB dataset. This demonstrates the ability of dual text embeddings to generalise effectively across different tasks. Figure \ref{fig:manipulation_results} illustrates few visual examples of the text-to-image manipulations. 


\begin{table*}[!ht]
\vspace{-0.2cm}
\centering
\begin{tabular}{lcccccccccc}
\toprule
%\multirow{2}{*}{\textbf{\makecell[l]{Variant \\ name}}} & 
\multirow{2}{*}{\textbf{\makecell{Emb.\\ type}}} & \multicolumn{4}{c}{\textbf{ Components}} & \multicolumn{3}{c}{\textbf{CUB}} & \multicolumn{3}{c}{ \textbf{Oxford-102}} \\
\cmidrule{2-11}
    & $\mathcal{L}_G$    & $\mathcal{L}_{cont}^D$ &$S_D \rightarrow G$ &$S_G \rightarrow D$   & \textbf{IS} $\uparrow$    & \textbf{FID} $\downarrow$   & \textbf{R\%} $\uparrow$   & \textbf{IS}  $\uparrow$     & \textbf{FID} $\downarrow$     & \textbf{R\%} $\uparrow$     \\
\midrule
    Shared & \xmark  &\cmark & - & - & $4.54 \pm .04$ & $15.81$ & $85.63$ & $3.52 \pm .06$ & $33.57$ & $81.73$\\  
   Shared &    \cmark   &  \cmark & - & - & $4.27 \pm .06$ & $18.38$ & $82.73$ & $3.32 \pm .05$ & $34.83$ & $76.15$\\
  Dual & \cmark  &\cmark & \xmark & \xmark & $ 4.73\pm .05$ & $14.93$ & $63.79$ & $3.84 \pm .03$ & $32.98$ & $54.97$\\
 Dual & \cmark  &\cmark  & \cmark & \cmark & $4.25 \pm .05$ & $18.01 $ & $69.38$ & $3.28 \pm .04$ & $35.41$ & $70.48$\\
 Dual & \cmark & \cmark & \cmark & \xmark &$\color{red}5.12 \pm .04$ & $\color{red}13.67$ & $\color{red}86.64$ & $\color{red}4.21 \pm .08$  & $\color{red}30.07$ & $\color{red}83.19$ \\ 
      \bottomrule
\end{tabular}
\caption{Quantitative comparison of DTE with its variants. Here, $\mathcal{L}_G$ = generator loss, $\mathcal{L}_{cont}^D$ = multi-modal contrastive loss between real image - text pairs, $S_D \rightarrow G$ = whether the generator has access to the discriminator-side sentence embedding, $S_G \rightarrow D$ = whether the discriminator has access to the generator-side sentence embedding. The best results are \textcolor{red}{red}. ‘‘$\uparrow$’’ indicates the higher, the better, while ‘‘$\downarrow$’’ indicates the lower, the better.}
\label{tab:modelvariants}
% \vspace{-1cm}
\end{table*}


\begin{table*}[!ht]
\vspace{-0.2cm}
\centering
\begin{tabular}{lcccccccc}
\toprule
%\multirow{2}{*}{\textbf{\makecell[l]{Variant \\ name}}} & 
\multirow{2}{*}{\textbf{\makecell{Epoch\\ Count}}} & \multicolumn{2}{c}{\textbf{ Components}} & \multicolumn{3}{c}{\textbf{CUB}} & \multicolumn{3}{c}{ \textbf{Oxford-102}} \\
\cmidrule{2-9}
     &$S_D \rightarrow G$ &$S_G \rightarrow D$   & \textbf{IS} $\uparrow$    & \textbf{FID} $\downarrow$   & \textbf{R\%} $\uparrow$   & \textbf{IS}  $\uparrow$     & \textbf{FID} $\downarrow$     & \textbf{R\%} $\uparrow$     \\
\midrule
 0  & \cmark & \cmark & $4.21 \pm .04$ & $18.33 $ & $69.81$ & $3.34 \pm .05$ & $36.18$ & $71.13$\\
 100  & \cmark & \cmark & $4.59 \pm .06$ & $16.24 $ & $74.46$ & $3.56 \pm .05$ & $33.27$ & $75.68$\\
 300  & \cmark & \cmark & $\color{red}4.82 \pm .04$ & $\color{red}14.96 $ & $\color{red}78.57$ & $\color{red}3.91 \pm .04$ & $\color{red}31.38$ & $\color{red}78.92$\\
 \bottomrule
\end{tabular}
\caption{Quantitative comparison of of ($S_G \rightarrow D$) whether the discriminator has access to the generator-side sentence embedding from the epoch count. $S_D \rightarrow G$ represents generator has access to the discriminator-side sentence embedding. The best results are \textcolor{red}{red}. ‘‘$\uparrow$’’ indicates the higher, the better, while ‘‘$\downarrow$’’ indicates the lower, the better.}
\label{tab:modelvariants_gen_emb}
% \vspace{-1cm}
\end{table*}

\vspace{-0.2cm}
\subsubsection{Importance of dual text embeddings}
\label{sec:ablationdualemb}
% Need to add one more result to it 
\vspace{-0.2cm}

To verify the effectiveness of the proposed Dual Text Embeddings (DTE) setup, different ways of organising the word embeddings between generator $G$ and discriminator $D$ are evaluated on CUB and Oxford-102 datasets and results shown in Table \ref{tab:modelvariants}. Specifically, four variants of organising the embeddings are compared, namely: \textit{i}) A shared word embedding layer between $G$ and $D$ that is trained only with a multi-modal contrastive loss $\mathcal{L}_{cont}^D$ between real image-text pairs as it is trained only to capture distinctive features (Table \ref{tab:modelvariants}, row 1). \textit{ii}) A shared word embedding layer between $G$ and $D$ that is trained using both the generator loss $\mathcal{L}_G$ and real image-text pair contrastive loss $\mathcal{L}_{cont}^D$ to capture distinctive and intricate appearance features in single embeddings (Table \ref{tab:modelvariants}, row 2).  \textit{iii}) A dual embedding setup where $G$ doesn't have access to the discriminator-side sentence embedding $S_D$(Table \ref{tab:modelvariants},row 3). \textit{iv}) A dual embedding setup where both $G$ and $D$ have access to the generator-side sentence embeddings $S_G$ and discriminator-side sentence embeddings $S_D$ (Table \ref{tab:modelvariants}, row 4). 
The shared embedding model trained only with $\mathcal{L}_{cont}^D$ to capture distinctive features (Table \ref{tab:modelvariants}, row 1) achieves similar R-precision scores as those of the proposed DTE-GAN, but there is significant drop in \textit{IS} and \textit{FID} scores suggesting that there is drop in the quality of generated images. 
Next, the shared embedding model trained with both $\mathcal{L}_G$ and $\mathcal{L}_{cont}^D$ performs inferior to the one trained only with $\mathcal{L}_{cont}^D$. It proves that capturing generator-side noisy gradient signals into the same word embeddings degrades the performance. Further, the dual embedding model with independent generator- and discriminator-side embeddings achieves better IS and FID scores compared to those of shared embedding models but has a significant drop in R-precision, suggesting that the images generated are realistic but do not capture text-to-image alignment. Further, allowing the discriminator to have access to the generator-side embeddings significantly drops the performance (Table \ref{tab:modelvariants}, row 4). As the Discriminator-side embeddings are learned with ground-truth real image-text pairs, it is found beneficial to allow Generator to have access (\textit{i.e.,} a sneak peek) to Discriminator-side embeddings (Table \ref{tab:modelvariants}, row 5). Discriminator-side embeddings capture distinct features and Generator-side feature captures intricate details to improve photo-realism; providing both the information to Generator allows to generate more realistic and text-aligned images. On the other hand, Generator-side embeddings are learned using noisy gradients from fake images and allowing the discriminator to access them introduces an adverse effect and decreases the performance (Table \ref{tab:modelvariants}, row 4).

\subsubsection{Generator-side embeddings to Discriminator}
\label{sec:gen_emb_to_disc}

When generator-side embeddings are provided to the discriminator, we observe a significant drop in overall image generation quality, as reported in Table \ref{tab:modelvariants}. To further evaluate this effect, we have conducted an experiment in which generator-side embeddings are supplied to the discriminator after a specified number of training epochs, with the results presented in Table \ref{tab:modelvariants_gen_emb}. Specifically, generator-side $S_G$ embeddings are combined with $S_D$ using summation from the start of training (epoch = 0), after 100 epochs when the generator began producing images with plausible structure, and after 300 epochs when more realistic images were generated. The low or noisy quality of generated images during the initial stages affects the learning of generator-side embeddings; consequently, providing these embeddings to the discriminator negatively impacts the overall image generation quality.



\begin{table*}[!ht]
% \vspace{-1.2cm}
\centering
\begin{tabular}{lcccccccc}
\toprule
%\multirow{2}{*}{\textbf{\makecell[l]{Variant \\ name}}} & 
\multirow{2}{*}{\textbf{\makecell{Epoch\\ Count}}} & \multicolumn{2}{c}{\textbf{ loss}} & \multicolumn{3}{c}{\textbf{CUB}} & \multicolumn{3}{c}{ \textbf{Oxford-102}} \\
\cmidrule{2-9}
     &$\mathcal{L}_{cont}^D$ &$\mathcal{L}_G$   & \textbf{IS} $\uparrow$    & \textbf{FID} $\downarrow$   & \textbf{R\%} $\uparrow$   & \textbf{IS}  $\uparrow$     & \textbf{FID} $\downarrow$     & \textbf{R\%} $\uparrow$     \\
\midrule
 0  & \cmark & \cmark & $4.27 \pm .06$ & $18.38 $ & $82.73$ & $3.32 \pm .05$ & $34.83$ & $76.15$\\
 100  & \cmark & \cmark & $4.38 \pm .04$ & $16.82 $ & $83.72$ & $3.39 \pm .04$ & $34.04$ & $78.69$\\
 300  & \cmark & \cmark & $4.47 \pm .05$ & $16.23 $ & $84.51$ & $3.47 \pm .03$ & $33.96$ & $79.94$\\
 -  & \cmark & \xmark & $\color{red}4.54 \pm .04$ & $\color{red}15.81 $ & $\color{red}85.63$ & $\color{red}3.52 \pm .06$ & $\color{red}33.57$ & $\color{red}81.73$\\
 \bottomrule
\end{tabular}
\caption{Quantitative comparison of shared embeddings trained with contrastive loss ($\mathcal{L}_{cont}^D$) and Generator's loss ($\mathcal{L}_G$) after the epoch count. Single embeddings without $\mathcal{L}_G$ training achieves superior performance. The best results are \textcolor{red}{red}. ‘‘$\uparrow$’’ indicates the higher, the better, while ‘‘$\downarrow$’’ indicates the lower, the better.}
\label{tab:modelvariants_dual_loss}
% \vspace{-1cm}
\end{table*}

\subsubsection{Shared Embeddings with Dual Loss}
\label{sec:dual_loss}

As shown in Table \ref{tab:modelvariants}, shared embeddings (or single embeddings for text) trained with both contrastive loss $\mathcal{L}_{cont}^D$ and the generator's loss $\mathcal{L}_G$ yield subpar results compared to embeddings trained solely with contrastive loss. We hypothesise that this is because, during the early stages of GAN training, the generator produces predominantly noisy images, negatively impacting the learning of embeddings. To investigate the impact of the generator's loss, we trained multiple networks with shared embeddings and applied the generator's loss to these embeddings at different training stages. The results, reported in Table \ref{tab:modelvariants_dual_loss}, indicate that applying both losses from the early stages of training adversely affects image generation quality. Moreover, decoupling the embeddings introduces flexibility, enabling them to capture diverse representations, ultimately improving overall image generation quality.



\subsubsection{Dual Captions for Dual Text Encoders}
\label{sec:dual_captions}

Earlier approaches in Text-to-Image synthesis have employed dual \citep{SDGAN} or multiple captions \citep{RifeGAN} to enhance semantic consistency and facilitate richer feature extraction. To analyse the impact of using different captions in a dual text encoder setup, we have conducted experiments and reported the results in Table \ref{tab:Dual Captions}. Consistent with previous findings, our results show a marginal improvement in performance. This improvement is attributed to the ability of each text encoder to independently extract superior features, thereby reinforcing the overall semantic alignment and image quality.

\begin{table*}[!ht]
% \vspace{-1.2cm}
\centering
\begin{tabular}{lcccccc}
\toprule
%\multirow{2}{*}{\textbf{\makecell[l]{Variant \\ name}}} & 
\multirow{2}{*}{\textbf{\makecell{Embedding\\ Type}}}  & \multicolumn{3}{c}{\textbf{CUB}} & \multicolumn{3}{c}{ \textbf{Oxford-102}} \\
\cmidrule{2-7}
       & \textbf{IS} $\uparrow$    & \textbf{FID} $\downarrow$   & \textbf{R\%} $\uparrow$   & \textbf{IS}  $\uparrow$     & \textbf{FID} $\downarrow$     & \textbf{R\%} $\uparrow$     \\
\midrule
 \textbf{Single}   & $5.12 \pm .04$ & $13.67 $ & $86.64$ & $4.21 \pm .08$ & $30.07$ & $83.19$\\
 \textbf{Dual}   & $\color{red}5.14 \pm .04$ & $\color{red}13.37 $ & $\color{red}87.12$ & $\color{red}4.28 \pm .09$ & $\color{red}29.71$ & $\color{red}83.38$\\
 \bottomrule
\end{tabular}
\caption{Quantitative comparison of shared embeddings trained with single and dual captions. Single embeddings with out $\mathcal{L}_G$ training achieves superior performance. The best results are \textcolor{red}{red}. ‘‘$\uparrow$’’ indicates the higher, the better, while ‘‘$\downarrow$’’ indicates the lower, the better.}
\label{tab:Dual Captions}
% \vspace{-1cm}
\end{table*}


% % \vspace{-0.2cm}
% \section{Limitation and Future Scope}
% \label{sec:issues}

% This work proposes an approach to learning vision-language models by generating images. Due to computational constraints (our model is trained on a single 1080Ti graphics card with 12 GB memory), we focus our approach and conduct experiments on smaller datasets. For future work, we aim to explore learning vision-language models by generating images using diffusion-based \citep{ddpm,diffusion_beats_GAN} models on large-scale openly available datasets \citep{webvideo10m,DiffusionDBLargescalePrompt2022}.

\vspace{-0.1cm}
\section{Conclusion}
\vspace{-0.1cm}
This study introduces a novel approach to text-to-image synthesis by proposing Dual Text Embeddings (DTE), which learns text embeddings tailored explicitly for the synthesis task in an end-to-end manner. Unlike traditional methods that depend on pre-trained, generic embeddings, DTE uses two separate embeddings designed for specific purposes: one to improve the photo-realism of generated images and the other to ensure better alignment between text and images. DTE decouples these objectives with dedicated training techniques, leading to enhanced performance.

Our experiments on three benchmark datasets (Oxford-102, Caltech-UCSD, and MS-COCO) demonstrate that having separate embeddings yields better results than using a shared representation. Furthermore, the DTE framework performs favourably compared to methods relying on pre-trained text embeddings optimized with contrastive loss. Additionally, their versatility highlights the adaptability of learned dual embeddings to other language-based vision tasks, such as text-to-image manipulations.

Future work includes extending this dual embedding framework to other multimodal applications, including image or video captioning and visual question answering, further showcasing the potential of task-specific text embeddings in advancing language-vision interactions.


\begin{thebibliography}{78}
\providecommand{\natexlab}[1]{#1}
\providecommand{\url}[1]{\texttt{#1}}
\expandafter\ifx\csname urlstyle\endcsname\relax
  \providecommand{\doi}[1]{doi: #1}\else
  \providecommand{\doi}{doi: \begingroup \urlstyle{rm}\Url}\fi

\bibitem[Arjovsky et~al.(2017)Arjovsky, Chintala, and Bottou]{wgan}
Martin Arjovsky, Soumith Chintala, and Léon Bottou.
\newblock Wasserstein gan, 2017.

\bibitem[Brock et~al.(2017)Brock, Lim, Ritchie, and Weston]{orthogonal_regularisation}
Andrew Brock, Theodore Lim, J.~M. Ritchie, and Nick Weston.
\newblock Neural photo editing with introspective adversarial networks, 2017.

\bibitem[Brock et~al.(2019)Brock, Donahue, and Simonyan]{big_gan}
Andrew Brock, Jeff Donahue, and Karen Simonyan.
\newblock Large scale gan training for high fidelity natural image synthesis, 2019.

\bibitem[Cai et~al.(2019)Cai, Wang, Yu, Li, Xu, Li, and Li]{Dual_Attn_GAN}
Yali Cai, Xiaoru Wang, Zhihong Yu, Fu~Li, Peirong Xu, Yueli Li, and Lixian Li.
\newblock Dualattn-gan: Text to image synthesis with dual attentional generative adversarial network.
\newblock \emph{IEEE Access}, 7:\penalty0 183706--183716, 2019.
\newblock \doi{10.1109/ACCESS.2019.2958864}.

\bibitem[Chen et~al.(2019)Chen, Lučić, Houlsby, and Gelly]{CBN}
Ting Chen, Mario Lučić, Neil Houlsby, and Sylvain Gelly.
\newblock On self-modulation for generative adversarial networks.
\newblock In \emph{International Conference on Learning Representations}, 2019.
\newblock URL \url{https://arxiv.org/pdf/1810.01365.pdf}.

\bibitem[Cheng et~al.(2020)Cheng, Wu, Tian, Wang, and Tao]{RifeGAN}
Jun Cheng, Fuxiang Wu, Yanling Tian, Lei Wang, and Dapeng Tao.
\newblock Rifegan: Rich feature generation for text-to-image synthesis from prior knowledge.
\newblock In \emph{2020 IEEE/CVF Conference on Computer Vision and Pattern Recognition (CVPR)}, pp.\  10908--10917, 2020.
\newblock \doi{10.1109/CVPR42600.2020.01092}.

\bibitem[Dash et~al.(2017)Dash, Gamboa, Ahmed, Liwicki, and Afzal]{tacgan}
Ayushman Dash, John Cristian~Borges Gamboa, Sheraz Ahmed, Marcus Liwicki, and Muhammad~Zeshan Afzal.
\newblock Tac-gan - text conditioned auxiliary classifier generative adversarial network, 2017.

\bibitem[Deng et~al.(2009)Deng, Dong, Socher, Li, Li, and Fei-Fei]{imagenet}
J.~Deng, W.~Dong, R.~Socher, L.-J. Li, K.~Li, and L.~Fei-Fei.
\newblock {ImageNet: A Large-Scale Hierarchical Image Database}.
\newblock In \emph{CVPR09}, 2009.

\bibitem[Dhariwal \& Nichol(2021)Dhariwal and Nichol]{diffusion_beats_GAN}
Prafulla Dhariwal and Alexander~Quinn Nichol.
\newblock Diffusion models beat {GAN}s on image synthesis.
\newblock In A.~Beygelzimer, Y.~Dauphin, P.~Liang, and J.~Wortman Vaughan (eds.), \emph{Advances in Neural Information Processing Systems}, 2021.
\newblock URL \url{https://openreview.net/forum?id=AAWuCvzaVt}.

\bibitem[Ding et~al.(2021)Ding, Yang, Hong, Zheng, Zhou, Yin, Lin, Zou, Shao, Yang, and Tang]{CogView}
Ming Ding, Zhuoyi Yang, Wenyi Hong, Wendi Zheng, Chang Zhou, Da~Yin, Junyang Lin, Xu~Zou, Zhou Shao, Hongxia Yang, and Jie Tang.
\newblock Cogview: Mastering text-to-image generation via transformers.
\newblock In M.~Ranzato, A.~Beygelzimer, Y.~Dauphin, P.S. Liang, and J.~Wortman Vaughan (eds.), \emph{Advances in Neural Information Processing Systems}, pp.\  19822--19835. Curran Associates, Inc., 2021.

\bibitem[Esser et~al.(2020)Esser, Rombach, and Ommer]{VQGAN}
Patrick Esser, Robin Rombach, and Bj{\"{o}}rn Ommer.
\newblock Taming transformers for high-resolution image synthesis.
\newblock \emph{CoRR}, abs/2012.09841, 2020.
\newblock URL \url{https://arxiv.org/abs/2012.09841}.

\bibitem[Gafni et~al.(2022)Gafni, Polyak, Ashual, Sheynin, Parikh, and Taigman]{make_a_scene}
Oran Gafni, Adam Polyak, Oron Ashual, Shelly Sheynin, Devi Parikh, and Yaniv Taigman.
\newblock Make-a-scene: Scene-based text-to-image generation with human priors, 2022.
\newblock URL \url{https://arxiv.org/abs/2203.13131}.

\bibitem[Goodfellow et~al.(2014)Goodfellow, Pouget-Abadie, Mirza, Xu, Warde-Farley, Ozair, Courville, and Bengio]{GAN_2014}
Ian Goodfellow, Jean Pouget-Abadie, Mehdi Mirza, Bing Xu, David Warde-Farley, Sherjil Ozair, Aaron Courville, and Yoshua Bengio.
\newblock Generative adversarial nets.
\newblock In Z.~Ghahramani, M.~Welling, C.~Cortes, N.~Lawrence, and K.~Q. Weinberger (eds.), \emph{Advances in Neural Information Processing Systems}, volume~27, 2014.

\bibitem[Gu et~al.(2022)Gu, Chen, Bao, Wen, Zhang, Chen, Yuan, and Guo]{Vq_diffusion}
Shuyang Gu, Dong Chen, Jianmin Bao, Fang Wen, Bo~Zhang, Dongdong Chen, Lu~Yuan, and Baining Guo.
\newblock Vector quantized diffusion model for text-to-image synthesis.
\newblock In \emph{Proceedings of the IEEE/CVF Conference on Computer Vision and Pattern Recognition (CVPR)}, pp.\  10696--10706, June 2022.

\bibitem[Gulrajani et~al.(2017)Gulrajani, Ahmed, Arjovsky, Dumoulin, and Courville]{I_wgan}
Ishaan Gulrajani, Faruk Ahmed, Martin Arjovsky, Vincent Dumoulin, and Aaron~C Courville.
\newblock Improved training of wasserstein gans.
\newblock In I.~Guyon, U.~V. Luxburg, S.~Bengio, H.~Wallach, R.~Fergus, S.~Vishwanathan, and R.~Garnett (eds.), \emph{Advances in Neural Information Processing Systems}, volume~30. Curran Associates, Inc., 2017.

\bibitem[Heusel et~al.(2017)Heusel, Ramsauer, Unterthiner, Nessler, and Hochreiter]{FID_score}
Martin Heusel, Hubert Ramsauer, Thomas Unterthiner, Bernhard Nessler, and Sepp Hochreiter.
\newblock Gans trained by a two time-scale update rule converge to a local nash equilibrium.
\newblock In I.~Guyon, U.~V. Luxburg, S.~Bengio, H.~Wallach, R.~Fergus, S.~Vishwanathan, and R.~Garnett (eds.), \emph{Advances in Neural Information Processing Systems}. Curran Associates, Inc., 2017.

\bibitem[Ho et~al.(2020)Ho, Jain, and Abbeel]{ddpm}
Jonathan Ho, Ajay Jain, and Pieter Abbeel.
\newblock Denoising diffusion probabilistic models.
\newblock In H.~Larochelle, M.~Ranzato, R.~Hadsell, M.F. Balcan, and H.~Lin (eds.), \emph{Advances in Neural Information Processing Systems}, pp.\  6840--6851. Curran Associates, Inc., 2020.

\bibitem[Kang et~al.(2023)Kang, Zhu, Zhang, Park, Shechtman, Paris, and Park]{scaling_up_gan}
Minguk Kang, Jun-Yan Zhu, Richard Zhang, Jaesik Park, Eli Shechtman, Sylvain Paris, and Taesung Park.
\newblock Scaling up gans for text-to-image synthesis.
\newblock In \emph{Proceedings of the IEEE/CVF Conference on Computer Vision and Pattern Recognition (CVPR)}, pp.\  10124--10134, June 2023.

\bibitem[Karras et~al.(2018)Karras, Aila, Laine, and Lehtinen]{progressive_gan}
Tero Karras, Timo Aila, Samuli Laine, and Jaakko Lehtinen.
\newblock Progressive growing of gans for improved quality, stability, and variation, 2018.

\bibitem[Karras et~al.(2019)Karras, Laine, and Aila]{styleGAN}
Tero Karras, Samuli Laine, and Timo Aila.
\newblock A style-based generator architecture for generative adversarial networks.
\newblock In \emph{Proceedings of the IEEE/CVF Conference on Computer Vision and Pattern Recognition (CVPR)}, June 2019.

\bibitem[Kingma \& Ba(2015)Kingma and Ba]{Adam}
Diederik~P. Kingma and Jimmy Ba.
\newblock Adam: {A} method for stochastic optimization.
\newblock In Yoshua Bengio and Yann LeCun (eds.), \emph{3rd International Conference on Learning Representations, {ICLR} 2015, San Diego, CA, USA, May 7-9, 2015, Conference Track Proceedings}, 2015.
\newblock URL \url{http://arxiv.org/abs/1412.6980}.

\bibitem[Krizhevsky(2009)]{cifar}
Alex Krizhevsky.
\newblock Learning multiple layers of features from tiny images.
\newblock Technical report, 2009.

\bibitem[LeCun \& Cortes(2010)LeCun and Cortes]{MNIST}
Yann LeCun and Corinna Cortes.
\newblock {MNIST} handwritten digit database.
\newblock 2010.
\newblock URL \url{http://yann.lecun.com/exdb/mnist/}.

\bibitem[Lee et~al.(2022)Lee, Kim, Kim, Cho, and Han]{res_quatizer}
Doyup Lee, Chiheon Kim, Saehoon Kim, Minsu Cho, and Wook-Shin Han.
\newblock Autoregressive image generation using residual quantization.
\newblock In \emph{Proceedings of the IEEE/CVF Conference on Computer Vision and Pattern Recognition (CVPR)}, pp.\  11523--11532, June 2022.

\bibitem[Li et~al.(2019)Li, Qi, Lukasiewicz, and Torr]{control_gan}
Bowen Li, Xiaojuan Qi, Thomas Lukasiewicz, and Philip Torr.
\newblock Controllable text-to-image generation.
\newblock In H.~Wallach, H.~Larochelle, A.~Beygelzimer, F.~d\textquotesingle Alch\'{e}-Buc, E.~Fox, and R.~Garnett (eds.), \emph{Advances in Neural Information Processing Systems}. Curran Associates, Inc., 2019.

\bibitem[Li et~al.(2020{\natexlab{a}})Li, Qi, Lukasiewicz, and Torr]{manigan}
Bowen Li, Xiaojuan Qi, Thomas Lukasiewicz, and Philip~H.S. Torr.
\newblock Manigan: Text-guided image manipulation.
\newblock In \emph{Proceedings of the IEEE/CVF Conference on Computer Vision and Pattern Recognition (CVPR)}, June 2020{\natexlab{a}}.

\bibitem[Li et~al.(2020{\natexlab{b}})Li, Qi, Torr, and Lukasiewicz]{lightweight_GAN}
Bowen Li, Xiaojuan Qi, Philip Torr, and Thomas Lukasiewicz.
\newblock Lightweight generative adversarial networks for text-guided image manipulation.
\newblock In H.~Larochelle, M.~Ranzato, R.~Hadsell, M.~F. Balcan, and H.~Lin (eds.), \emph{Advances in Neural Information Processing Systems}, pp.\  22020--22031. Curran Associates, Inc., 2020{\natexlab{b}}.

\bibitem[Liang et~al.(2020)Liang, Pei, and Lu]{CPGAN}
Jiadong Liang, Wenjie Pei, and Feng Lu.
\newblock Cpgan: Content-parsing generative adversarial networks for text-to-image synthesis.
\newblock In Andrea Vedaldi, Horst Bischof, Thomas Brox, and Jan-Michael Frahm (eds.), \emph{Computer Vision -- ECCV 2020}, pp.\  491--508, Cham, 2020. Springer International Publishing.
\newblock ISBN 978-3-030-58548-8.

\bibitem[Liao et~al.(2022)Liao, Hu, Yang, and Rosenhahn]{SSA-GAN}
Wentong Liao, Kai Hu, Michael~Ying Yang, and Bodo Rosenhahn.
\newblock Text to image generation with semantic-spatial aware gan.
\newblock In \emph{Proceedings of the IEEE/CVF Conference on Computer Vision and Pattern Recognition (CVPR)}, pp.\  18187--18196, June 2022.

\bibitem[Lin et~al.(2014{\natexlab{a}})Lin, Maire, Belongie, Hays, Perona, Ramanan, Doll{\'a}r, and Zitnick]{coco_dataset}
Tsung-Yi Lin, Michael Maire, Serge Belongie, James Hays, Pietro Perona, Deva Ramanan, Piotr Doll{\'a}r, and C.~Lawrence Zitnick.
\newblock Microsoft coco: Common objects in context.
\newblock In David Fleet, Tomas Pajdla, Bernt Schiele, and Tinne Tuytelaars (eds.), \emph{Computer Vision -- ECCV 2014}, 2014{\natexlab{a}}.

\bibitem[Lin et~al.(2014{\natexlab{b}})Lin, Maire, Belongie, Hays, Perona, Ramanan, Doll{\'a}r, and Zitnick]{mscoco}
Tsung-Yi Lin, Michael Maire, Serge Belongie, James Hays, Pietro Perona, Deva Ramanan, Piotr Doll{\'a}r, and C.~Lawrence Zitnick.
\newblock Microsoft coco: Common objects in context.
\newblock In David Fleet, Tomas Pajdla, Bernt Schiele, and Tinne Tuytelaars (eds.), \emph{Computer Vision -- ECCV 2014}, pp.\  740--755, Cham, 2014{\natexlab{b}}. Springer International Publishing.
\newblock ISBN 978-3-319-10602-1.

\bibitem[Liu et~al.(2021)Liu, Song, Zhu, de~Melo, and Elgammal]{TIME}
Bingchen Liu, Kunpeng Song, Yizhe Zhu, Gerard de~Melo, and Ahmed Elgammal.
\newblock Time: Text and image mutual-translation adversarial networks.
\newblock \emph{Proceedings of the AAAI Conference on Artificial Intelligence}, 35\penalty0 (3):\penalty0 2082--2090, May 2021.
\newblock URL \url{https://ojs.aaai.org/index.php/AAAI/article/view/16305}.

\bibitem[Maas et~al.(2013)Maas, Hannun, and Ng]{lrelu}
Andrew~L. Maas, Awni~Y. Hannun, and Andrew~Y. Ng.
\newblock Rectifier nonlinearities improve neural network acoustic models.
\newblock In \emph{in ICML Workshop on Deep Learning for Audio, Speech and Language Processing}, 2013.

\bibitem[Mao et~al.(2017)Mao, Li, Xie, Lau, Wang, and Paul~Smolley]{lsgan}
Xudong Mao, Qing Li, Haoran Xie, Raymond~Y.K. Lau, Zhen Wang, and Stephen Paul~Smolley.
\newblock Least squares generative adversarial networks.
\newblock In \emph{Proceedings of the IEEE International Conference on Computer Vision (ICCV)}, Oct 2017.

\bibitem[Mirza \& Osindero(2014)Mirza and Osindero]{condtionalgan}
Mehdi Mirza and Simon Osindero.
\newblock Conditional generative adversarial nets, 2014.

\bibitem[Miyato \& Koyama(2018)Miyato and Koyama]{cgans_projections}
Takeru Miyato and Masanori Koyama.
\newblock cgans with projection discriminator, 2018.

\bibitem[Miyato et~al.(2018)Miyato, Kataoka, Koyama, and Yoshida]{sn_gan}
Takeru Miyato, Toshiki Kataoka, Masanori Koyama, and Yuichi Yoshida.
\newblock Spectral normalization for generative adversarial networks.
\newblock In \emph{International Conference on Learning Representations}, 2018.
\newblock URL \url{https://openreview.net/forum?id=B1QRgziT-}.

\bibitem[Nichol \& Dhariwal(2021)Nichol and Dhariwal]{imporved_ddpm}
Alex Nichol and Prafulla Dhariwal.
\newblock Improved denoising diffusion probabilistic models, 2021.
\newblock URL \url{https://arxiv.org/abs/2102.09672}.

\bibitem[Nilsback \& Zisserman(2008)Nilsback and Zisserman]{flower_dataset}
Maria-Elena Nilsback and Andrew Zisserman.
\newblock Automated flower classification over a large number of classes.
\newblock In \emph{Indian Conference on Computer Vision, Graphics and Image Processing}, Dec 2008.

\bibitem[Odena et~al.(2017)Odena, Olah, and Shlens]{acgan}
Augustus Odena, Christopher Olah, and Jonathon Shlens.
\newblock Conditional image synthesis with auxiliary classifier {GAN}s.
\newblock In Doina Precup and Yee~Whye Teh (eds.), \emph{Proceedings of the 34th International Conference on Machine Learning}, volume~70 of \emph{Proceedings of Machine Learning Research}, pp.\  2642--2651. PMLR, 06--11 Aug 2017.

\bibitem[Paszke et~al.(2019)Paszke, Gross, Massa, Lerer, Bradbury, Chanan, Killeen, Lin, Gimelshein, Antiga, Desmaison, Kopf, Yang, DeVito, Raison, Tejani, Chilamkurthy, Steiner, Fang, Bai, and Chintala]{NEURIPS2019_9015}
Adam Paszke, Sam Gross, Francisco Massa, Adam Lerer, James Bradbury, Gregory Chanan, Trevor Killeen, Zeming Lin, Natalia Gimelshein, Luca Antiga, Alban Desmaison, Andreas Kopf, Edward Yang, Zachary DeVito, Martin Raison, Alykhan Tejani, Sasank Chilamkurthy, Benoit Steiner, Lu~Fang, Junjie Bai, and Soumith Chintala.
\newblock Pytorch: An imperative style, high-performance deep learning library.
\newblock In H.~Wallach, H.~Larochelle, A.~Beygelzimer, F.~d\textquotesingle Alch\'{e}-Buc, E.~Fox, and R.~Garnett (eds.), \emph{Advances in Neural Information Processing Systems 32}, pp.\  8024--8035. Curran Associates, Inc., 2019.

\bibitem[Qiao et~al.(2019{\natexlab{a}})Qiao, Zhang, Xu, and Tao]{Leica_gan}
Tingting Qiao, Jing Zhang, Duanqing Xu, and Dacheng Tao.
\newblock Learn, imagine and create: Text-to-image generation from prior knowledge.
\newblock In H.~Wallach, H.~Larochelle, A.~Beygelzimer, F.~d\textquotesingle Alch\'{e}-Buc, E.~Fox, and R.~Garnett (eds.), \emph{Advances in Neural Information Processing Systems}. Curran Associates, Inc., 2019{\natexlab{a}}.

\bibitem[Qiao et~al.(2019{\natexlab{b}})Qiao, Zhang, Xu, and Tao]{MirrorGAN}
Tingting Qiao, Jing Zhang, Duanqing Xu, and Dacheng Tao.
\newblock Mirrorgan: Learning text-to-image generation by redescription.
\newblock In \emph{Proceedings of the IEEE/CVF Conference on Computer Vision and Pattern Recognition (CVPR)}, June 2019{\natexlab{b}}.

\bibitem[Radford et~al.(2021)Radford, Kim, Hallacy, Ramesh, Goh, Agarwal, Sastry, Askell, Mishkin, Clark, Krueger, and Sutskever]{clip}
Alec Radford, Jong~Wook Kim, Chris Hallacy, Aditya Ramesh, Gabriel Goh, Sandhini Agarwal, Girish Sastry, Amanda Askell, Pamela Mishkin, Jack Clark, Gretchen Krueger, and Ilya Sutskever.
\newblock Learning transferable visual models from natural language supervision, 2021.
\newblock URL \url{https://arxiv.org/abs/2103.00020}.

\bibitem[Ramesh et~al.(2021)Ramesh, Pavlov, Goh, Gray, Voss, Radford, Chen, and Sutskever]{DALL-E}
Aditya Ramesh, Mikhail Pavlov, Gabriel Goh, Scott Gray, Chelsea Voss, Alec Radford, Mark Chen, and Ilya Sutskever.
\newblock Zero-shot text-to-image generation.
\newblock \emph{CoRR}, abs/2102.12092, 2021.
\newblock URL \url{https://arxiv.org/abs/2102.12092}.

\bibitem[Ramesh et~al.(2022)Ramesh, Dhariwal, Nichol, Chu, and Chen]{Dalle_2}
Aditya Ramesh, Prafulla Dhariwal, Alex Nichol, Casey Chu, and Mark Chen.
\newblock Hierarchical text-conditional image generation with clip latents, 2022.
\newblock URL \url{https://arxiv.org/abs/2204.06125}.

\bibitem[Razavi et~al.(2019)Razavi, van~den Oord, and Vinyals]{vq_vae_2}
Ali Razavi, Aaron van~den Oord, and Oriol Vinyals.
\newblock Generating diverse high-fidelity images with vq-vae-2.
\newblock In H.~Wallach, H.~Larochelle, A.~Beygelzimer, F.~d\textquotesingle Alch\'{e}-Buc, E.~Fox, and R.~Garnett (eds.), \emph{Advances in Neural Information Processing Systems}. Curran Associates, Inc., 2019.

\bibitem[Reed et~al.(2016{\natexlab{a}})Reed, Akata, Mohan, Tenka, Schiele, and Lee]{GAWWN}
Scott Reed, Zeynep Akata, Santosh Mohan, Samuel Tenka, Bernt Schiele, and Honglak Lee.
\newblock Learning what and where to draw, 2016{\natexlab{a}}.

\bibitem[Reed et~al.(2016{\natexlab{b}})Reed, Akata, Yan, Logeswaran, Schiele, and Lee]{reed2016generative}
Scott Reed, Zeynep Akata, Xinchen Yan, Lajanugen Logeswaran, Bernt Schiele, and Honglak Lee.
\newblock Generative adversarial text to image synthesis.
\newblock In Maria~Florina Balcan and Kilian~Q. Weinberger (eds.), \emph{Proceedings of The 33rd International Conference on Machine Learning}, volume~48 of \emph{Proceedings of Machine Learning Research}, pp.\  1060--1069. PMLR, 2016{\natexlab{b}}.

\bibitem[Rombach et~al.(2021)Rombach, Blattmann, Lorenz, Esser, and Ommer]{stable_diffusion}
Robin Rombach, Andreas Blattmann, Dominik Lorenz, Patrick Esser, and Björn Ommer.
\newblock High-resolution image synthesis with latent diffusion models, 2021.

\bibitem[Ruan et~al.(2021)Ruan, Zhang, Zhang, Fan, Tang, Liu, and Chen]{DAE-GAN}
Shulan Ruan, Yong Zhang, Kun Zhang, Yanbo Fan, Fan Tang, Qi~Liu, and Enhong Chen.
\newblock Dae-gan: Dynamic aspect-aware gan for text-to-image synthesis.
\newblock In \emph{Proceedings of the IEEE/CVF International Conference on Computer Vision (ICCV)}, pp.\  13960--13969, October 2021.

\bibitem[Saharia et~al.(2022)Saharia, Chan, Saxena, Li, Whang, Denton, Ghasemipour, Ayan, Mahdavi, Lopes, Salimans, Ho, Fleet, and Norouzi]{imagen}
Chitwan Saharia, William Chan, Saurabh Saxena, Lala Li, Jay Whang, Emily Denton, Seyed Kamyar~Seyed Ghasemipour, Burcu~Karagol Ayan, S.~Sara Mahdavi, Rapha~Gontijo Lopes, Tim Salimans, Jonathan Ho, David~J Fleet, and Mohammad Norouzi.
\newblock Photorealistic text-to-image diffusion models with deep language understanding, 2022.
\newblock URL \url{https://arxiv.org/abs/2205.11487}.

\bibitem[Salimans et~al.(2016)Salimans, Goodfellow, Zaremba, Cheung, Radford, and Chen]{IS_score}
Tim Salimans, Ian Goodfellow, Wojciech Zaremba, Vicki Cheung, Alec Radford, and Xi~Chen.
\newblock Improved techniques for training gans.
\newblock In \emph{Proceedings of the 30th International Conference on Neural Information Processing Systems}, NIPS'16, pp.\  2234–2242, Red Hook, NY, USA, 2016. Curran Associates Inc.
\newblock ISBN 9781510838819.

\bibitem[Schuster \& Paliwal(1997)Schuster and Paliwal]{bi-lstm}
M.~Schuster and K.K. Paliwal.
\newblock Bidirectional recurrent neural networks.
\newblock \emph{IEEE Transactions on Signal Processing}, 45\penalty0 (11):\penalty0 2673--2681, 1997.
\newblock \doi{10.1109/78.650093}.

\bibitem[Sohl-Dickstein et~al.(2015)Sohl-Dickstein, Weiss, Maheswaranathan, and Ganguli]{denosing_diffusion}
Jascha Sohl-Dickstein, Eric Weiss, Niru Maheswaranathan, and Surya Ganguli.
\newblock Deep unsupervised learning using nonequilibrium thermodynamics.
\newblock In Francis Bach and David Blei (eds.), \emph{Proceedings of the 32nd International Conference on Machine Learning}, volume~37 of \emph{Proceedings of Machine Learning Research}, pp.\  2256--2265, Lille, France, 07--09 Jul 2015. PMLR.
\newblock URL \url{https://proceedings.mlr.press/v37/sohl-dickstein15.html}.

\bibitem[Szegedy et~al.(2016)Szegedy, Vanhoucke, Ioffe, Shlens, and Wojna]{szegedy2016rethinking}
Christian Szegedy, Vincent Vanhoucke, Sergey Ioffe, Jon Shlens, and Zbigniew Wojna.
\newblock Rethinking the inception architecture for computer vision.
\newblock In \emph{Proceedings of the IEEE conference on computer vision and pattern recognition}, pp.\  2818--2826, 2016.

\bibitem[Tan et~al.(2021)Tan, Liu, Liu, Yin, and Li]{KT-GAN}
Hongchen Tan, Xiuping Liu, Meng Liu, Baocai Yin, and Xin Li.
\newblock Kt-gan: Knowledge-transfer generative adversarial network for text-to-image synthesis.
\newblock \emph{IEEE Transactions on Image Processing}, 30:\penalty0 1275--1290, 2021.
\newblock \doi{10.1109/TIP.2020.3026728}.

\bibitem[Tan et~al.(2022{\natexlab{a}})Tan, Liu, Yin, and Li]{CSM-GAN}
Hongchen Tan, Xiuping Liu, Baocai Yin, and Xin Li.
\newblock Cross-modal semantic matching generative adversarial networks for text-to-image synthesis.
\newblock \emph{IEEE Transactions on Multimedia}, 24:\penalty0 832--845, 2022{\natexlab{a}}.
\newblock \doi{10.1109/TMM.2021.3060291}.

\bibitem[Tan et~al.(2023{\natexlab{a}})Tan, Liu, Yin, and Li]{DR-GAN}
Hongchen Tan, Xiuping Liu, Baocai Yin, and Xin Li.
\newblock Dr-gan: Distribution regularization for text-to-image generation.
\newblock \emph{IEEE Transactions on Neural Networks and Learning Systems}, 34\penalty0 (12):\penalty0 10309--10323, 2023{\natexlab{a}}.
\newblock \doi{10.1109/TNNLS.2022.3165573}.

\bibitem[Tan et~al.(2023{\natexlab{b}})Tan, Yin, Wei, Liu, and Li]{ALR-GAN}
Hongchen Tan, Baocai Yin, Kun Wei, Xiuping Liu, and Xin Li.
\newblock Alr-gan: Adaptive layout refinement for text-to-image synthesis.
\newblock \emph{IEEE Transactions on Multimedia}, 25:\penalty0 8620--8631, 2023{\natexlab{b}}.
\newblock \doi{10.1109/TMM.2023.3238554}.

\bibitem[Tan et~al.(2022{\natexlab{b}})Tan, Lee, Neo, and Lim]{SSTIS}
Yong~Xuan Tan, Chin~Poo Lee, Mai Neo, and Kian~Ming Lim.
\newblock Text-to-image synthesis with self-supervised learning.
\newblock \emph{Pattern Recognition Letters}, 157:\penalty0 119--126, 2022{\natexlab{b}}.
\newblock ISSN 0167-8655.
\newblock \doi{https://doi.org/10.1016/j.patrec.2022.04.010}.
\newblock URL \url{https://www.sciencedirect.com/science/article/pii/S0167865522001064}.

\bibitem[Tan et~al.(2023{\natexlab{c}})Tan, Lee, Neo, Lim, and Lim]{SSBI}
Yong~Xuan Tan, Chin~Poo Lee, Mai Neo, Kian~Ming Lim, and Jit~Yan Lim.
\newblock Text-to-image synthesis with self-supervised bi-stage generative adversarial network.
\newblock \emph{Pattern Recognition Letters}, 169:\penalty0 43--49, 2023{\natexlab{c}}.
\newblock ISSN 0167-8655.
\newblock \doi{https://doi.org/10.1016/j.patrec.2023.03.023}.
\newblock URL \url{https://www.sciencedirect.com/science/article/pii/S0167865523000880}.

\bibitem[Tao et~al.(2020)Tao, Wu, Zhang, and Wang]{DFGAN}
Ming Tao, Songsong Wu, Xiaofeng Zhang, and Cailing Wang.
\newblock Dcfgan: Dynamic convolutional fusion generative adversarial network for text-to-image synthesis.
\newblock pp.\  1250--1254, 11 2020.
\newblock \doi{10.1109/ICIBA50161.2020.9277299}.

\bibitem[Tao et~al.(2022)Tao, Tang, Wu, Jing, Bao, and Xu]{DF_GAN_CVPR}
Ming Tao, Hao Tang, Fei Wu, Xiao-Yuan Jing, Bing-Kun Bao, and Changsheng Xu.
\newblock Df-gan: A simple and effective baseline for text-to-image synthesis.
\newblock In \emph{Proceedings of the IEEE/CVF Conference on Computer Vision and Pattern Recognition (CVPR)}, pp.\  16515--16525, June 2022.

\bibitem[Tao et~al.(2023)Tao, Bao, Tang, and Xu]{galip}
Ming Tao, Bing-Kun Bao, Hao Tang, and Changsheng Xu.
\newblock Galip: Generative adversarial clips for text-to-image synthesis.
\newblock In \emph{Proceedings of the IEEE/CVF Conference on Computer Vision and Pattern Recognition}, pp.\  14214--14223, 2023.

\bibitem[van~den Oord et~al.(2017)van~den Oord, Vinyals, and kavukcuoglu]{NDR}
Aaron van~den Oord, Oriol Vinyals, and koray kavukcuoglu.
\newblock Neural discrete representation learning.
\newblock In I.~Guyon, U.~V. Luxburg, S.~Bengio, H.~Wallach, R.~Fergus, S.~Vishwanathan, and R.~Garnett (eds.), \emph{Advances in Neural Information Processing Systems}. Curran Associates, Inc., 2017.

\bibitem[Vaswani et~al.(2017)Vaswani, Shazeer, Parmar, Uszkoreit, Jones, Gomez, Kaiser, and Polosukhin]{transformers}
Ashish Vaswani, Noam Shazeer, Niki Parmar, Jakob Uszkoreit, Llion Jones, Aidan~N Gomez, \L~ukasz Kaiser, and Illia Polosukhin.
\newblock Attention is all you need.
\newblock In I.~Guyon, U.~Von Luxburg, S.~Bengio, H.~Wallach, R.~Fergus, S.~Vishwanathan, and R.~Garnett (eds.), \emph{Advances in Neural Information Processing Systems}, volume~30. Curran Associates, Inc., 2017.
\newblock URL \url{https://proceedings.neurips.cc/paper/2017/file/3f5ee243547dee91fbd053c1c4a845aa-Paper.pdf}.

\bibitem[Welinder et~al.(2010)Welinder, Branson, Mita, Wah, Schroff, Belongie, and Perona]{CUB_dataset}
P.~Welinder, S.~Branson, T.~Mita, C.~Wah, F.~Schroff, S.~Belongie, and P.~Perona.
\newblock {Caltech-UCSD Birds 200}.
\newblock Technical Report CNS-TR-2010-001, California Institute of Technology, 2010.

\bibitem[Xu et~al.(2018)Xu, Zhang, Huang, Zhang, Gan, Huang, and He]{AttnGAN}
Tao Xu, Pengchuan Zhang, Qiuyuan Huang, Han Zhang, Zhe Gan, Xiaolei Huang, and Xiaodong He.
\newblock Attngan: Fine-grained text to image generation with attentional generative adversarial networks.
\newblock In \emph{Proceedings of the IEEE Conference on Computer Vision and Pattern Recognition (CVPR)}, June 2018.

\bibitem[Ye et~al.(2024)Ye, Wang, Tan, and Liu]{RATGAN}
Senmao Ye, Huan Wang, Mingkui Tan, and Fei Liu.
\newblock Recurrent affine transformation for text-to-image synthesis.
\newblock \emph{IEEE Transactions on Multimedia}, 26:\penalty0 462--473, 2024.
\newblock \doi{10.1109/TMM.2023.3266607}.

\bibitem[Yin et~al.(2019)Yin, Liu, Sheng, Yu, Wang, and Shao]{SDGAN}
Guojun Yin, Bin Liu, Lu~Sheng, Nenghai Yu, Xiaogang Wang, and Jing Shao.
\newblock Semantics disentangling for text-to-image generation.
\newblock In \emph{Proceedings of the IEEE/CVF Conference on Computer Vision and Pattern Recognition (CVPR)}, June 2019.

\bibitem[Zhang et~al.(2017{\natexlab{a}})Zhang, Xu, Li, Zhang, Wang, Huang, and Metaxas]{stackgan++}
Han Zhang, Tao Xu, Hongsheng Li, Shaoting Zhang, Xiaogang Wang, Xiaolei Huang, and Dimitris Metaxas.
\newblock Stackgan++: Realistic image synthesis with stacked generative adversarial networks.
\newblock \emph{IEEE Transactions on Pattern Analysis and Machine Intelligence}, PP, 10 2017{\natexlab{a}}.
\newblock \doi{10.1109/TPAMI.2018.2856256}.

\bibitem[Zhang et~al.(2017{\natexlab{b}})Zhang, Xu, Li, Zhang, Wang, Huang, and Metaxas]{stack_gan}
Han Zhang, Tao Xu, Hongsheng Li, Shaoting Zhang, Xiaogang Wang, Xiaolei Huang, and Dimitris~N. Metaxas.
\newblock Stackgan: Text to photo-realistic image synthesis with stacked generative adversarial networks.
\newblock In \emph{Proceedings of the IEEE International Conference on Computer Vision (ICCV)}, Oct 2017{\natexlab{b}}.

\bibitem[Zhang et~al.(2021)Zhang, Koh, Baldridge, Lee, and Yang]{XMC-GAN}
Han Zhang, Jing~Yu Koh, Jason Baldridge, Honglak Lee, and Yinfei Yang.
\newblock Cross-modal contrastive learning for text-to-image generation.
\newblock In \emph{Proceedings of the IEEE/CVF Conference on Computer Vision and Pattern Recognition (CVPR)}, pp.\  833--842, June 2021.

\bibitem[Zhang \& Schomaker(2020)Zhang and Schomaker]{DTGAN}
Zhenxing Zhang and Lambert Schomaker.
\newblock {DTGAN:} dual attention generative adversarial networks for text-to-image generation.
\newblock \emph{CoRR}, abs/2011.02709, 2020.
\newblock URL \url{https://arxiv.org/abs/2011.02709}.

\bibitem[Zhang et~al.(2018)Zhang, Xie, and Yang]{hd_gan}
Zizhao Zhang, Yuanpu Xie, and Lin Yang.
\newblock Photographic text-to-image synthesis with a hierarchically-nested adversarial network.
\newblock In \emph{Proceedings of the IEEE Conference on Computer Vision and Pattern Recognition (CVPR)}, June 2018.

\bibitem[Zhou et~al.(2021)Zhou, Zhang, Chen, Li, Tensmeyer, Yu, Gu, Xu, and Sun]{lafite}
Yufan Zhou, Ruiyi Zhang, Changyou Chen, Chunyuan Li, Chris Tensmeyer, Tong Yu, Jiuxiang Gu, Jinhui Xu, and Tong Sun.
\newblock Lafite: Towards language-free training for text-to-image generation.
\newblock \emph{arXiv preprint arXiv:2111.13792}, 2021.

\bibitem[Zhu et~al.(2019)Zhu, Pan, Chen, and Yang]{DMGAN}
Minfeng Zhu, Pingbo Pan, Wei Chen, and Yi~Yang.
\newblock Dm-gan: Dynamic memory generative adversarial networks for text-to-image synthesis.
\newblock In \emph{Proceedings of the IEEE/CVF Conference on Computer Vision and Pattern Recognition (CVPR)}, June 2019.

\end{thebibliography}


% \documentclass{MITstyle}

%\usepackage[table]{xcolor}
\usepackage{chngcntr}
\usepackage{hyperref}
\usepackage{microtype}

\title{A Lightweight and Extensible Cell Segmentation and Classification Model for Whole Slide Images}

\author{Nikita Shvetsov~$^{1, }$\footnote{Correspondence e-mail: nikita.shvetsov@uit.no}, Thomas K. Kilvaer~$^{2, 3}$, Masoud Tafavvoghi~$^{4}$, Anders Sildnes~$^{1}$, \\ Kajsa Møllersen~$^{4}$, Lill-Tove Rasmussen Busund~$^{5, 6}$, Lars Ailo Bongo~$^{1}$ \\
%
\vspace{1em} % Space between authors and afilliations
%
\normalfont{\small $^{1}$Department of Computer Science, UiT The Arctic University of Norway}\\
\normalfont{\small $^{2}$Department of Oncology, University Hospital of North Norway}\\
\normalfont{\small $^{3}$Department of Clinical Medicine, UiT The Arctic University of Norway}\\
\normalfont{\small $^{4}$Department of Community Medicine, UiT The Arctic University of Norway}\\
\normalfont{\small $^{5}$Department of Medical Biology, UiT The Arctic University of Norway} \\
\normalfont{\small $^{6}$Department of Clinical Pathology, University Hospital of North Norway} %\vspace{2em}
}

\begin{document}
\maketitle

\section*{Abstract}

% \begin{abstract}
% Developing clinically useful cell-level analysis tools in digital pathology remains challenging due to limitations in dataset granularity, inconsistent annotations, computational demands of advanced models, and difficulties in integrating new technologies into clinical workflows. To address these challenges, we propose a multi-faceted solution that enhances data quality, model performance, and usability to create a lightweight and extensible cell segmentation and classification model.

% First, we update data labels by employing a cross-relabeling process that refines the labels of two existing datasets, PanNuke and MoNuSAC, to create a new unified dataset with enhanced granularity, encompassing seven distinct cell types. Second, we leverage the H-Optimus foundation model as a fixed encoder to improve feature representation for simultaneous cell segmentation and classification tasks. Third, to address the computational demands of foundation models, we employ knowledge distillation to reduce model size and complexity while maintaining comparable performance. Finally, to facilitate integration into clinical workflows, we integrate the distilled model into the QuPath software, a widely used open-source platform in digital pathology.

% Our results demonstrate improvements in cell segmentation and classification performance using the H‑Optimus-based model compared to a CNN-based model. Specifically, the average $R^2$ improved from 0.575 to 0.871, and the average $PQ$ score improved from 0.450 to 0.492, indicating better alignment with actual cell counts and enhanced segmentation and classification quality. Furthermore, the distilled student model maintains performance comparable to the larger foundation model while reducing the parameter count by a factor of 48.
% Overall, by reducing computational complexity and integrating it into existing workflows, the proposed approach may significantly impact diagnostic processes, reduce the workload of pathologists, and contribute to improved patient outcomes. Though our approach shows potential enhancements in efficiency and usability of cell segmentation and classification models in digital pathology, extensive validation is needed to deploy these models in clinical practice.
% \end{abstract}

%%% shortened abstract
\begin{abstract}
Developing clinically useful cell-level analysis tools in digital pathology remains challenging due to limitations in dataset granularity, inconsistent annotations, high computational demands, and difficulties integrating new technologies into workflows. To address these issues, we propose a solution that enhances data quality, model performance, and usability by creating a lightweight, extensible cell segmentation and classification model. 

First, we update data labels through cross-relabeling to refine annotations of PanNuke and MoNuSAC, producing a unified dataset with seven distinct cell types. Second, we leverage the H-Optimus foundation model as a fixed encoder to improve feature representation for simultaneous segmentation and classification tasks. Third, to address foundation models' computational demands, we distill knowledge to reduce model size and complexity while maintaining comparable performance. Finally, we integrate the distilled model into QuPath, a widely used open-source digital pathology platform. 

Results demonstrate improved segmentation and classification performance using the H-Optimus-based model compared to a CNN-based model. Specifically, average $R^2$ improved from 0.575 to 0.871, and average $PQ$ score improved from 0.450 to 0.492, indicating better alignment with actual cell counts and enhanced segmentation quality. The distilled model maintains comparable performance while reducing parameter count by a factor of 48. By reducing computational complexity and integrating into workflows, this approach may significantly impact diagnostics, reduce pathologist workload, and improve outcomes. Although the method shows promise, extensive validation is necessary prior to clinical deployment.
\end{abstract}
\clearpage

\section{Introduction}
In digital pathology, accurate segmentation and classification of cells are crucial for many diagnostic, prognostic, and predictive analyses \cite{Jaber_Beziaeva_etal._2019,Lin_Pan_etal._2022,Park_Ock_etal._2022,Shen_Choi_etal._2024}. Nowadays, developments in computational pathology offer multiple solutions \cite{H._Qu_P._Wu_etal._2020,Javed_Mahmood_etal._2020} to utilize cell-level datasets to train machine learning models that solve these problems. The quality and specificity of training datasets are critical for robust and accurate models. Adhering to the principle of "garbage in, garbage out", it is essential to ensure that these datasets are extensively and accurately labeled with distinct classes that reflect the diverse biological characteristics of different cell types. Unfortunately, the number of open-source datasets comprising such high-quality annotations is limited. Existing cell segmentation datasets \cite{Gamper_Koohbanani_etal._2019,Graham_Vu_etal._2019,Verma_Kumar_etal._2021} may offer extensive annotations for certain cell types while providing more general labels for others. For example, in PanNuke, which is one of the largest open-source datasets comprising labeled cells, various types of morphologically and functionally different inflammatory cells like macrophages and lymphocytes are clustered in a broad "inflammatory" class. Consequently, these classes are frequently omitted from analyses or aggregated into broader meta-classes \cite{Gamper_Koohbanani_etal._2020} and likely interfere with other cell classes included in the dataset. This and similar inconsistencies in annotation granularity limit the ability of machine learning models to learn the comprehensive and nuanced features necessary for accurate cell segmentation and classification. To address these challenges, methods for refining and standardizing dataset annotations are essential to enhance the quality of training data.

A complementary approach to mitigate the absence of high-quality training data is the use of foundation models. Foundation models as encoders are defined as large-scale, versatile networks pre-trained on vast, diverse datasets using self-supervised learning, contrasting with convolutional neural network (CNN) pre-trained encoders that rely on supervised learning with labeled data. In practice, foundation models leverage enormous amounts of weakly or unlabeled data from millions of whole slide images (WSIs) and employ self-attention mechanisms to capture long-range dependencies and global context \cite{Chen_Ding_etal._2024,Saillard_Jenatton_etal._2024,Vorontsov_Bozkurt_etal._2024,Xu_Usuyama_etal._2024}. As a consequence, foundation models are able to produce transferable feature representations across different cell types and tissue environments. The feature representations can be leveraged by decoder networks to produce segmentation masks and pixel-level classifications. Because foundation models have comprehensive feature representations, they can be effectively fine-tuned using much smaller amounts of cell-level data compared to the large datasets needed to train models from scratch. Furthermore, foundation models incorporate adversarial training elements or contrastive learning \cite{Chen_Ding_etal._2024,Xu_Usuyama_etal._2024}, enhancing their resilience and adaptability by exposing them to challenging and varied scenarios during training. This may result in more generalizable models, often making them well-suited for diverse and complex tasks in digital pathology.

Despite the inherent advantages of foundation models, their deployment for practical use faces its own obstacles. In particular, they require substantial computational power, financial investments and rigorous testing to ensure reliability and efficacy for a given task \cite{Akkus_Dangott_etal._2022,Dragomir_Cocuz_etal._2022,Go_2022,Jafri_Farooqui_etal._2024}. Moreover, while foundation models enhance feature representation and performance, they depend on the quality of available annotations for decoder fine-tuning and, like any other model, cannot resolve existing inconsistencies or ambiguities in data labels. Therefore, there remains a critical need for solutions that address both data quality and practical deployment considerations.
Further, integrating new technologies into existing clinical workflows often encounters resistance, as it necessitates adjustments to established diagnostic processes. So, there is a need to develop solutions that could be integrated into current practices, minimizing the burden on medical professionals to adopt new tools \cite{King_Williams_etal._2023}.

Existing solutions \cite{Goldsborough_Philps_etal._2024,Hörst_Rempe_etal._2024}, while addressing some aspects of these challenges, fall short in providing a comprehensive approach. To address the data quality and clinical deployment issues, we propose a multi-faceted solution that encompasses data refinement, model optimization, and integration with existing pathology tools (\hyperref[fig:fig1]{Figure 1}). The outcome is a lightweight cell segmentation and classification model that can be integrated into digital pathology workflows for practical clinical use.

\begin{figure}[h!]
    \centering
    \includegraphics[width=\textwidth, height=0.82\textheight, keepaspectratio]{images/Figure_1.pdf}
    \caption{Overview of the proposed solution, including 1) Data refinement using cross-relabeling, 2) Teacher model development and fine tuning, 3) Student model optimization with knowledge distillation and 4) Student model and QuPath integration}
    \label{fig:fig1}
\end{figure}
\clearpage

Our approach begins with preparing the data for the fine-tuning and training of the machine learning models. We create a refined dataset, acquired via cross-relabeling two cell-level datasets, enhancing annotation specificity and consistency of the labeled data. Subsequently, we create a cell segmentation and classification model based on the foundation model. We leverage the foundation model as a fixed encoder and fine-tune a decoder using the refined dataset to improve generalization across diverse tissue- and cell types.
To ensure that the model remains lightweight and deployable in a possibly resource-constrained environment, we employ knowledge distillation to approximate the functionality of the foundation model. Finally, to facilitate the practical application of our model in digital pathology workflows, we integrate it with the QuPath \cite{Bankhead_Loughrey_etal._2017} application. Each methodological component contributes to the overarching goal of enhancing model performance, generalizability, and usability in clinical settings.

The primary contributions of this paper are:
\begin{enumerate}
    \item \textit{Data labels refinement through cross-relabeling:}
    
    We propose a new method for refining labels of cell-level datasets through cross-relabeling. This method employs classification models to re-label broad and ambiguous instances, resulting in a more diverse dataset. Our evaluation demonstrates that these classification models achieve high accuracy on test subsets, indicating the reliability of the method for label refinement.

    \item \textit{Enhanced model performance via foundation models:}
    
    We employ a foundation model as a feature extractor for the cell segmentation and classification task. In comparison with training a CNN model from scratch, the foundation model backbone only needs fine-tuning, which significantly reduces training time, computational resources and data requirements. We show that using a foundation model encoder leads to better performance in cell segmentation and classification networks than using a CNN-based encoder. This improvement may enable the model to generalize more effectively across various tissue types and imaging methods.
    
    \item \textit{Model optimization through knowledge distillation:}
    
    We show that a smaller student model trained using knowledge distillation on the refined dataset obtained via our cross-relabeling approach from a foundation model achieves comparable performance in cell segmentation and quantification tasks. As a result, this model is more suitable for deployment in environments without high-performance computing resources.
    
    \item \textit{Integration with QuPath:}
    
    We integrate the distilled cell segmentation and classification model into QuPath, a widely used open-source digital pathology platform, to accelerate clinical adaptation by enabling pathologists to more easily incorporate advanced computational tools into their existing workflows.
\end{enumerate}

Through these methodological steps, we aim to bridge the gap between advanced machine learning techniques and practical clinical applications, making accurate and efficient digital pathology accessible in a broader range of healthcare settings.

\section{Refining Existing Datasets Using Cross-Relabeling}
To address the limitations of sparse and ambiguous labeling of cell-level datasets, we propose a generalizable cross-relabeling strategy that can be applied to any dataset containing broadly categorized or imprecisely labeled cell types. This approach involves training and subsequently leveraging classification models to refine broad categories into more specific or biologically relevant classes.
When applied to cell-level data, the methodology includes extracting individual cell images from the dataset patches, preprocessing these images to standardize the size and accommodate partial cells, and then training deep learning classifiers capable of distinguishing between the finer cell subtypes within the coarser categories. 
To illustrate our approach, we focus on the PanNuke \cite{Gamper_Koohbanani_etal._2020, Gamper_Koohbanani_etal._2019} and MoNuSAC \cite{Verma_Kumar_etal._2021} datasets that we have used to train models for cell quantification in our previous works \cite{Shvetsov_Grønnesby_etal._2022,Shvetsov_Sildnes_etal._2024}. We find that for better cell differentiation we have to introduce more granular labels. PanNuke includes a broad classification of "inflammatory" cells, encompassing lymphocytes, macrophages, and neutrophils. Each cell type differs significantly in structure, function, and clinical relevance. Conversely, MoNuSAC uses the label "epithelial" for a class that comprises both benign epithelial cells and malignant neoplastic cells. This practice makes it challenging to differentiate between benign and malignant epithelial cells in the dataset, which is a critical distinction when identifying tumor areas within tissue samples. To address these issues, we implement a cross-relabeling strategy as shown in \hyperref[fig:fig2]{Figure 2}. The key components are two classification models: one is trained on singular cell images from PanNuke data to classify the epithelial meta-class into epithelial and neoplastic classes. The other is trained on MoNuSAC to refine the inflammatory class into lymphocytes, neutrophils, and macrophages.

\begin{figure}[h!]
    \centering
    \includegraphics[width=\textwidth]{images/Figure_2.pdf}
    \caption{Refined dataset generation via cross relabeling}
    \label{fig:fig2}
\end{figure}

The refining approach consists of three consecutive steps. The first is the preprocessing step, in which we extract individual cells from both datasets (\hyperref[fig:fig3]{Figure 3}). The specifics of PanNuke and MoNuSAC patch preparation before cell preprocessing are provided in \hyperref[chap:S1]{Appendix S1}.

\begin{figure}[h!]
    \centering
    \includegraphics[width=\textwidth]{images/Figure_3.pdf}
    \caption{Cell instances preprocessing including (1) cell map extraction, (2) bounding box delineation, (3) adjusting cell boxes and (4) cropping and resizing of cell images}
    \label{fig:fig3}
\end{figure}

During preprocessing, we extract cell type maps from the ground truth label mask and calculate bounding boxes around each cell instance. To accommodate partial cells at patch borders, a common issue in cropped patch images, we employ mirror padding and extend the field of view of the cell label by 15 pixels to capture adjacent cells. We then crop and resize the identified regions to $64 \times 64$ pixels using bicubic interpolation.

The preprocessed PanNuke dataset comprises 68,031 neoplastic and 23,207 epithelial cell images, while MoNuSAC comprises  33,104 lymphocytes, 1,252 neutrophils, and 1,695 macrophages, which we subsequently use in training cell classification models and classifying the cell image data \hyperref[fig:S2]{Appendix Figure S2 (1)}. 

The next step is to train two distinct ResNet50-based classifiers tailored to address the specific labeling challenges inherent in each dataset. We use ResNet50 for classification models due to its proven effectiveness for image classification tasks in histopathology \cite{pan2022reviewmachinelearningapproaches}, and its compatibility with small images. For the PanNuke dataset, we design the classifier, trained on MoNuSAC data, to disaggregate the heterogeneous "inflammatory" cell category into distinct subtypes: lymphocytes, macrophages, and neutrophils. Similarly, for the MoNuSAC dataset, the classifier is trained on PanNuke data and distinguishes between benign and malignant epithelial cells within the overarching "epithelial" label. By applying these targeted classifiers to their respective datasets, we assign more specific labels to individual cell instances, thus enabling us to create a unified dataset.
To ensure a balanced representation of classes, we train both models on datasets that had been equalized to match the size of the least represented class. Thus, we obtain datasets comprising 23,207 samples per class for PanNuke and 1,252 samples per class for MoNuSAC data. Next, we partition both of them into training (70\%), validation (20\%), and testing (10\%) subsets. To mitigate the risk of overfitting, we use a single dropout layer with a rate of p=0.5 in both models and data augmentation using randomized color perturbations, rotation, and horizontal and vertical flipping. We employ AdamW optimizer and the cross-entropy loss function for the training criterion.

To evaluate the two trained models, we measure the classification accuracy on the respective test subsets. The accuracies on the test subset for both classifiers are presented in \hyperref[tab:1]{Table 1}. The PanNuke model achieves an average accuracy of 93.57\%, with higher accuracy for neoplastic cells (96.06\%) compared to epithelial cells (86.26\%). The confusion matrix in Figure A3.1 shows that the model predominantly distinguishes accurately between epithelial and neoplastic tissues, with a substantial number of correct classifications and relatively few misclassifications. The MoNuSAC model demonstrates an average accuracy of 98.92\%, excelling in classifying lymphocytes (99.67\%) and macrophages (94.12\%), with lower performance for neutrophils (85.71\%). The confusion matrix in Figure A3.2 shows that the model identifies lymphocytes and performs reasonably well with macrophages and neutrophils.

\begin{table}[h!]
\renewcommand{\arraystretch}{1.5}
  \centering
  \caption{Cell classification results for PanNuke and MoNuSAC trained models (CI 95\%).}
  \label{tab:1}
  \begin{tabular}{|l|c|c|}
   \hline
   %\rowcolor{gray!30}
    Accuracy               & PanNuke model              & MoNuSAC model              \\
    \hline
    Average      & 0.936 (0.931--0.941)         & 0.989 (0.986--0.993)        \\
    \hline
    Neoplastic   & 0.961 (0.956--0.965)         & -                          \\
    \hline
    Epithelial   & 0.863 (0.849--0.877)         & -                          \\
    \hline
    Lymphocytes  & -                          & 0.997 (0.995--0.999)        \\
    \hline
    Neutrophils  & -                          & 0.857 (0.796--0.918)        \\
    \hline
    Macrophages  & -                          & 0.941 (0.906--0.976)        \\
    \hline
  \end{tabular}
\end{table}

Finally, during the last step, we use the model trained on PanNuke data for epithelial cells in MoNuSAC and the model trained on MoNuSAC for the inflammatory cells class in PanNuke. Specifically, we use classifier models to relabel epithelial cells in MoNuSAC and inflammatory cells in PanNuke data. Then we combine cells with refined labels and the rest of the cells in both datasets to create a refined dataset (\hyperref[fig:S2]{Appendix Figure S2 (2)}). The process of relabeling cells and visualizing them on a patch is shown in \hyperref[fig:fig4]{Figure 4}. The cell counts in the refined dataset are provided in \hyperref[tab:S4]{Appendix Table S4}.

\begin{figure}[h!]
    \centering
    \includegraphics[width=\textwidth, height=0.42\textheight, keepaspectratio]{images/Figure_4.pdf}
    \caption{Cell relabeling procedure for epithelial and inflammatory cell classes}
    \label{fig:fig4}
\end{figure}

%\hfill

Relabeling and combining datasets have been explored in a prior study \cite{Parulekar_Kanwat_etal._2023}, where consecutive fine-tuning on multiple datasets was employed to account for hierarchical class label structures. While the method presented in \cite{Parulekar_Kanwat_etal._2023} is intuitive, it often lacks consistency and requires multiple fine-tuning runs, which can be cumbersome and time-consuming. 
In contrast, cross-relabeling simplifies this process by using specialized classification models tailored to each dataset's specific labeling challenges. This approach provides better transparency and produces a unified dataset encompassing seven distinct cell types across multiple tissue samples, enhancing data diversity for further model training or fine-tuning.

Despite these improvements, cross-relabeling does not entirely resolve issues related to poor labeling quality or the amount of labeled data. Specifically, our results show lower accuracies persist for underrepresented classes, such as macrophages, which may stem from a limited sample availability and intrinsic challenges in distinguishing these cells based solely on H\&E staining. Furthermore, while our method enhances label specificity, it relies on the initial quality of the broad labels; thus, any fundamental inaccuracies in the original annotations can propagate through the relabeling process. Addressing the overall problem of limited data labels may require integrating additional data sources or utilizing complementary immunohistochemical staining methods.
Although the reported performance metrics are obtained from evaluations on the native test sets of each dataset, it is important to note that the primary application of these classifiers is to perform cross-relabeling, where a model trained on one dataset (e.g., PanNuke) is applied to another (e.g., MoNuSAC) and vice versa. We acknowledge that a more systematic evaluation of cross-dataset generalization is needed and could be performed in future work.

Overall, the refined dataset produced by our approach can enhance the supervised training or fine-tuning of cell segmentation and classification models, especially those that utilize pre-trained foundation models to improve feature extraction robustness. In addition, these models can detect nuanced classes that enable researchers to conduct more detailed analyses of biological processes in computational pathology.

\section{Foundation models for robust cell segmentation and classification}

Accurate cell segmentation and classification in digital pathology are hindered by limited labeled data and the fact that conventional CNNs are unable to capture global contextual information due to their local receptive field constraints \cite{Gheflati_Rivaz_2022,Yang_Marcus_etal.}. Traditional approaches in cell quantification have predominantly relied on CNN encoders, such as ResNet50, given their proven effectiveness in semantic segmentation tasks \cite{Deshmane_2023,Graham_Vu_etal._2019,Mukasheva_Koishiyeva_etal._2024,Stringer_Wang_etal._2021}. However, approaches that include fine-tuning of pretrained CNNs, data augmentation, and stain normalization to partially increase data variability and address staining differences often fail to achieve the necessary generalization and robustness across diverse tissue types and staining conditions \cite{G._Wang_W._Li_etal._2018,Gao_Bagci_etal._2018,Karim_El_Khoury_Martin_Fockedey_etal._2021}.

To overcome these challenges, we leverage an encoder-decoder network that uses a foundation model as the encoder and a CNN upsampling decoder (\hyperref[fig:fig5]{Figure 5}) for simultaneous cell segmentation and classification in 2D patches extracted from WSIs. Foundation models with transformer-based architectures are viable alternatives to CNN-based encoders \cite{Shamshad_Khan_etal._2023,Sourget_2023}. They enable the creation of more advanced architectures that can decode or transform learned features more effectively \cite{Chen_Duan_etal._2023,Cheng_Misra_etal._2022,Xie_Wang_etal._2021}.

\begin{figure}[h!]
    \centering
    \includegraphics[width=\textwidth]{images/Figure_5.pdf}
    \caption{UNETR-like model with foundational model as backbone}
    \label{fig:fig5}
\end{figure}

By utilizing a transformer-based encoder, we incorporate global contextual information into the feature extraction process, which is a key advantage of such architectures \cite{Chen_Lu_etal._2021}. This foundation model integration facilitates accurate pixel-wise segmentation and classification without the need for extensive encoder training, thereby potentially improving generalization across varied cellular structures and tissue types.
In our implementation, we employ a modified UNETR \cite{Hatamizadeh_Tang_etal._2021} architecture that combines a vision transformer (ViT) \cite{Dosovitskiy_Beyer_etal._2021} encoder with a CNN-based decoder. The encoder utilizes the pretrained H-Optimus foundation model, which contains 1.1 billion parameters and is trained on over 500,000 H\&E stained WSIs \cite{Saillard_Jenatton_etal._2024}. We extract outputs from four evenly spaced transformer blocks $Z_i$, where $i \in [1, 14, 26, 38]$, to serve as residual connections for the CNN decoder. We select these blocks based on our observation that features from non-adjacent levels of the encoder lead to better overall performance on the test subset.

The CNN decoder upsamples the feature representations, acquired from the transformer blocks, to generate an intermediate vector that is handled by two task-specific layers that generate cell segmentation and classification masks. The first task-specific layer is the ‘Cellpose head’,  which is used to delineate cell instances. The layer generates horizontal and vertical gradient maps to form vector fields that are refined through gradient tracking in a post-processing step using the Cellpose algorithm \cite{Stringer_Wang_etal._2021}, known for its efficacy in cell segmentation tasks and generalizability across multiple domains \cite{Pachitariu_Stringer_2022,Stringer_Pachitariu_2024}. The second task-specific layer is the "Cell type head", which assigns labels to individual pixels. In the post-processing step, we determine the output classification label of each segmented cell instance by majority voting over the labeled pixels that comprise the cell in the segmentation map.

To evaluate model performance and measure the impact of adding a foundation model as backbone, we compare it to a ResNet50-based model. ResNet50 is a widely used solution for encoders in segmentation architectures in the medical domain \cite{Deshmane_2023,Graham_Vu_etal._2019,Mukasheva_Koishiyeva_etal._2024,Stringer_Wang_etal._2021}. For the H-Optimus-based model, we utilize frozen weights for the encoder and only fine-tune the decoder to take advantage of the extensive pre-training of the foundation model. For the ResNet50-based model we start with ImageNet \cite{Deng_Dong_etal.} weights and train both encoder and decoder parts. Hyperparameters for the training step are set to be identical, where possible, for comparable evaluation. 
For this evaluation, we deliberately use the PanNuke dataset to provide a standardized and controlled comparison between the H‑Optimus and ResNet50-based models (\hyperref[fig:S2]{Appendix Figure S2 (3)}). Specifically, we use two of the default PanNuke dataset splits (66\%) for training and validation, and reserve the third split (33\%) for testing.

To address the challenge of cell class imbalance in the PanNuke dataset, which is a common characteristic in most cell-level H\&E patch datasets, both models’ training processes employ a weighted loss function comprising cross-entropy and focal loss \cite{Lin_Goyal_etal._2018}. The focal loss component is adjusted with coefficients derived from each cell class' instance frequency, emphasizing learning from underrepresented classes and enhancing the model's sensitivity to rare but significant cellular patterns. The cross-entropy loss is augmented with spectral decoupling regularization \cite{Pezeshki_Kaba_etal._2021,Pohjonen_Stürenberg_etal._2022} and spatially varying label smoothing \cite{Islam_Glocker_2021}, which potentially stabilizes training and improves generalization in case of complex tissue morphologies. For optimization, we employ the \textit{AdamW} \cite{Loshchilov_Hutter_2019} to counter unbalanced class scenarios, with cosine annealing learning rate scheduler.

We utilize the scikit-learn library \cite{Van_der_Walt_Schönberger_etal._2014} and HoVer-Net \cite{Graham_Vu_etal._2019} implementations of $R^2$ (the coefficient of determination) and $PQ$ (panoptic quality) to evaluate our experiments. Complete mathematical formulations and detailed explanations of these metrics are provided in \hyperref[chap:S5]{Appendix S5}. To compute confidence intervals, we use nonparametric bootstrapping, where after calculating the metric on the full sample, we generated 1000 bootstrap replicates by resampling with replacement and then determined the 95\% confidence intervals as the 2.5th and 97.5th percentiles of the resulting empirical distribution.

%\hfill

The model comparisons are summarized in \hyperref[tab:2]{Table 2}. The H‑Optimus-based model achieves higher $R^2$ across all cell classes compared to the ResNet50-based model, which means that its predictions are more closely aligned with the PanNuke cell counts, indicating a stronger correlation with the observed data. Notably, the improvement of $R^2_{dead}$ may be an indicator of better global contextual representations provided by the foundation model backbone. In terms of segmentation and classification quality combined, measured by the PQ score, the H‑Optimus-based model demonstrates notable improvements across most cell classes. Overall, the average $R^2$ improved from 0.575 to 0.871, while the average $PQ$ score improved from 0.450 to 0.492, demonstrating better performance of the H-Optimus-based model.

\begin{table}[h!]
\renewcommand{\arraystretch}{1.5}
  \centering
  \caption{Cell quantification metrics for baseline and proposed models (CI 95\%).}
  \label{tab:2}
  \begin{tabular}{|l|c|c|}
    \hline
    %\rowcolor{gray!30}
    Metric             & Resnet50-based            & H-optimus-based              \\
    \hline
    $R^2_{neoplastic}$    & 0.681 (0.576--0.769)       & \textbf{0.941 (0.917--0.960)} \\
    \hline
    $R^2_{inflammatory}$  & 0.863 (0.778--0.903)       & \textbf{0.949 (0.918--0.966)} \\
    \hline
    $R^2_{connective}$    & 0.600 (0.488--0.698)       & 0.609 (0.436--0.772)          \\
    \hline
    $R^2_{dead}$          & 0.097 (-11.389--0.669)     & 0.925 (0.404--0.982)          \\
    \hline
    $R^2_{epithelial}$    & 0.635 (0.490--0.747)       & \textbf{0.930 (0.886--0.964)} \\
    \hline
    $PQ_{neoplastic}$       & 0.517 (0.499--0.535)       & \textbf{0.589 (0.575--0.604)} \\
    \hline
    $PQ_{inflammatory}$     & 0.455 (0.429--0.482)       & \textbf{0.528 (0.507--0.549)} \\
    \hline
    $PQ_{connective}$       & 0.416 (0.400--0.431)       & \textbf{0.451 (0.436--0.465)} \\
    \hline
    $PQ_{dead}$             & 0.374 (0.342--0.408)       & 0.292 (0.209--0.365)          \\
    \hline
    $PQ_{epithelial}$       & 0.488 (0.460--0.519)       & \textbf{0.599 (0.579--0.618)} \\
    \hline
  \end{tabular}
\end{table}

Our results  show that integrating the H‑Optimus foundation model within the UNETR architecture enhances the model's ability to segment and classify cells across diverse tissues from PanNuke data. The pretrained transformer encoder provides robust feature representations, resulting in higher average $R^2$ and $PQ$ scores compared to the CNN-based model. This leads to more reliable cell quantification and more accurate downstream analysis. Additionally, the streamlined fine-tuning process reduces computational overhead and training time, making the model more adaptable for new data.

Despite these advancements, the foundation model-based approach does not fully resolve all challenges related to cell segmentation and classification. We observe lower metric scores for underrepresented classes in the training data. Furthermore, foundation models typically encompass billions of parameters, resulting in substantial computational and memory requirements. It therefore poses challenges for deployment in resource-constrained environments, limiting their practical applicability in certain clinical settings.

\section{Model optimization via Knowledge Distillation}

To address the limitations posed by the extensive size of foundation models, we implement knowledge distillation — a model compression technique that leverages the teacher-student paradigm \cite{Hinton_Vinyals_etal._2015}. By training a smaller, more efficient student model to replicate the output of a larger, pre-trained teacher model, we retain performance while significantly reducing the model's complexity and resource requirements (\hyperref[fig:fig6]{Figure 6}).

\begin{figure}[h!]
    \centering
    \includegraphics[width=\textwidth, height=0.45\textheight, keepaspectratio]{images/Figure_6.pdf}
    \caption{Knowledge distillation framework for training a student model using a pre-trained teacher}
    \label{fig:fig6}
\end{figure}

We employ knowledge distillation to compress the H‑Optimus-based teacher model into a more efficient student model. The teacher model is the modified UNETR architecture with the H‑Optimus foundation model described in the previous chapter. The student model is based on a UNet architecture augmented with residual connections and incorporates a smaller ViT encoder with 9 million parameters \cite{Steiner_Kolesnikov_etal._2022,Wightman_2019}. 

First, we fine-tune the teacher model using the refined dataset from the cross-relabeling procedure (Section 2). Initially we train the decoder of the teacher model while keeping the encoder weights frozen. We split the refined dataset into train (70\%), validation (20\%) and test (10\%) subsets (\hyperref[fig:S2]{Appendix Figure S2 (4)}). During fine-tuning, we use the train and validation subsets, while leaving the test subset for model evaluation. We set the training procedure and model hyperparameters to be identical to those that were used to demonstrate the utility of foundation models for the simultaneous cell segmentation and classification task.

Next, we perform knowledge distillation from teacher to student using the refined dataset used to fine-tune the teacher model. The student model is trained to replicate the teacher model's outputs. We utilize a specialized loss function that aligns the student's predicted probability distribution with the teacher's, incorporating the teacher's class probability distribution derived from the output. Following the methodology of Hinton et al. \cite{Hinton_Vinyals_etal._2015}, we experiment with various hyperparameter settings for the temperature ($T$) and the balancing coefficients ($\alpha$ and $\beta$) in the loss function. We vary $T$ from 1 to 20 and adjust $\alpha$ and $\beta$ to balance the distillation and student losses. Through iterative tuning and evaluation, we identify that setting $T=14$, $\alpha=0.3$, and $\beta=0.7$ yields a configuration that converges and closely approximates the teacher model's performance during training.

Finally, we assess the performance of both models using the $R^2$ and $PQ$ (defined in \hyperref[chap:S5]{Appendix S5}) on the test set of the refined dataset (\hyperref[tab:3]{Table 3}). We observe that the 95\% confidence intervals overlap for most cell types, so we cannot claim statistically significant performance differences between the teacher and student models. One exception appears in the neoplastic class. The teacher model produces an $R^2$ of 0.919, while the student model shows an $R^2$ of 0.852. In addition, the student model achieves higher $PQ$ values for the neoplastic and connective classes, though the confidence intervals show overlap.

\begin{table}[h!]
\renewcommand{\arraystretch}{1.5}
  \centering
  \caption{Cell quantification metrics for teacher and distilled student models (CI 95\%).}
  \label{tab:3}
  \begin{tabular}{|l|c|c|}
    \hline
    %\rowcolor{gray!30}
    Metric & Teacher & Student \\
    \hline
    $R^2_{neoplastic}$    & \textbf{0.919} (0.898--0.939) & 0.852 (0.800--0.891) \\
    \hline
    $R^2_{lymphocyte}$    & 0.969 (0.956--0.977)         & 0.969 (0.956--0.978) \\
    \hline
    $R^2_{connective}$    & 0.694 (0.548--0.809)         & 0.618 (0.469--0.741) \\
    \hline
    $R^2_{dead}$          & 0.755 (0.400--0.908)         & 0.424 (0.100--0.731) \\
    \hline
    $R^2_{epithelial}$    & 0.922 (0.870--0.958)         & 0.843 (0.738--0.917) \\
    \hline
    $R^2_{macrophage}$    & 0.384 (-0.369--0.724)        & 0.704 (0.352--0.859) \\
    \hline
    $R^2_{neutrofil}$     & 0.854 (0.578--0.929)         & 0.833 (0.502--0.925) \\
    \hline
    $PQ_{neoplastic}$       & 0.581 (0.569--0.593)         & 0.601 (0.588--0.613) \\
    \hline
    $PQ_{lymphocyte}$       & 0.536 (0.520--0.553)         & 0.563 (0.544--0.579) \\
    \hline
    $PQ_{connective}$       & 0.436 (0.421--0.451)         & 0.457 (0.441--0.474) \\
    \hline
    $PQ_{dead}$             & 0.272 (0.235--0.315)         & 0.279 (0.201--0.369) \\
    \hline
    $PQ_{epithelial}$       & 0.522 (0.500--0.545)         & 0.530 (0.506--0.555) \\
    \hline
    $PQ_{macrophage}$       & 0.524 (0.459--0.588)         & 0.474 (0.405--0.543) \\
    \hline
    $PQ_{neutrofil}$        & 0.541 (0.490--0.592)         & 0.565 (0.522--0.607) \\
    \hline
  \end{tabular}
\end{table}


We further decompose the $PQ$ metric into its $SQ$ and $DQ$ components (\hyperref[tab:S6]{Appendix Table S6}). Both models produce nearly identical $SQ$ values, which indicates that they predict instance boundaries with similar precision. Although the student model shows some improvement in $DQ$ scores for certain classes, the confidence intervals overlap and do not confirm a statistically significant difference.

We observe that the student and teacher models yield comparable detection performance despite the student model using a much smaller and simpler architecture. A model with fewer parameters reduces the risk of overfitting when training data are scarce relative to the model’s complexity \cite{Farias_Ludermir_etal._2022}. The knowledge distillation process also encourages the student model to focus on the most generalizable detection features learned from the teacher. These factors enable the student model to achieve similar detection performance across different cell types.

Additionally, considering the model sizes reported in \hyperref[tab:4]{Table 4}, the distilled model achieves a significant reduction compared to the teacher model, with a 48-fold decrease in parameter count and a 5.5-fold reduction in on-disk size. In inference mode, the teacher model requires 16 GB of VRAM for a batch size of 32, while the distilled model only needs 3 GB of VRAM for the same batch size. These reductions make the distilled model significantly more practical for fine-tuning and deployment in resource-constrained environments.

\begin{table}[h!]
\renewcommand{\arraystretch}{1.5}
  \centering
  \caption{Parameter counts and size of teacher and distilled model}
  \label{tab:4}
  \adjustbox{max width=\textwidth}{%
  \begin{tabular}{|l|c|c|c|}
    \hline
    %\rowcolor{gray!30}
    Metric & H-optimus-based (Teacher) & mobileViT-based (Student) & Magnitude of difference \\
    \hline
    Parameters count       & 1,158,917,906   & \textbf{24,093,393}   & \textbf{48x}  \\
    \hline
    Estimated Total Size (MB) & 87,912       & \textbf{15,935}    & \textbf{5.5x} \\
    \hline
  \end{tabular}%
}
\end{table}

%\hfill

With recent advancements in complex network architectures and the use of pretrained encoders to achieve state-of-the-art performance \cite{Baumann_Dislich_etal._2024,Hörst_Rempe_etal._2024} in cell segmentation and classification tasks, model size, computational complexity, and processing times have increased. This limits the scalability and accessibility of these models. As we demonstrate, this may be mitigated using knowledge distillation. Studies in the field of natural language processing have demonstrated the efficacy of knowledge distillation in retaining the capabilities of the teacher model while achieving significant reductions in size and complexity \cite{Huangpu_Gao_2024,Sun_Yu_etal.}. 

We demonstrate the feasibility of knowledge distillation in digital pathology, specifically for cell segmentation and classification tasks. Moreover, we achieve this performance while also significantly reducing the parameter count. In addressing the challenge of knowledge transfer, we found that distillation from a transformer-based model to a smaller transformer is more straightforward than attempting to map transformer features to CNN blocks. In our experiments, using a CNN-based network as a student results in worse cell quantification performance due to the structural constraints of CNN feature space dimensions. 

Although our primary approach relies on a transformer-based student model that performs well, it can be further optimized to incorporate advantages from CNN architectures. For example, employing alternative techniques such as using ViT adapters \cite{Chen_Duan_etal._2023} or $1 \times 1$ convolutions to adjust feature map sizes may be beneficial for harnessing CNN advantages like enhanced local feature extraction. Moreover, if additional performance improvements are desired, the process can be further enhanced by applying supplementary knowledge distillation techniques, such as self-distillation \cite{Zhang_Song_etal._2019} or online distillation \cite{Houyon_Cioppa_etal._2023}.

Despite these promising results, further validation on independent datasets is necessary to fully understand the model's limitations. Underrepresented classes may pose challenges when addressing complex cases. Pathologists need to validate these models to adopt them in clinical settings. While the distilled models are smaller and more deployable, a technological gap persists because pathologists traditionally rely on established methods for inspecting WSIs and diagnosing diseases. Addressing the complexities involved in deploying models for inference and supporting pathologists in adopting new tools is essential for integrating these models into clinical workflows.

\section{Model integration with QuPath}
Digital pathology tools with graphical user interfaces are essential for visualizing and analyzing WSIs. To make our student model useful in clinical pathology workflows, it needs to be integrated into a tool that enables inspecting regions, creating annotations, and providing quantitative analyses of biomarkers. Therefore, we integrate the trained student model from the previous chapter into the QuPath open‑source platform \cite{Bankhead_Loughrey_etal._2017}. QuPath provides the required annotation, visualization, and analysis tools to interpret complex histological data, including workflows for cell segmentation, classification, and quantification (\hyperref[fig:fig7]{Figure 7}). 

\begin{figure}[h!]
    \centering
    \includegraphics[width=\textwidth]{images/Figure_7.pdf}
    \caption{Visualization of model-generated cell quantification annotations (left) and the corresponding unannotated slide (right) in QuPath}
    \label{fig:fig7}
\end{figure}

To identify the regions in a WSI critical for prognosticating tumor development, such as specific tumor areas or border regions without overlapping healthy tissue, the pathologist uses QuPath to outline these regions. Then, the pathologist initiates a cell segmentation and classification script through the QuPath interface for the selected regions. The resulting annotations and quantified cell information are then directly overlaid onto the WSI in the QuPath interface. Additional design and implementation details are in \hyperref[chap:S7]{Appendix S7}. 

Two common approaches for integrating deep learning models into QuPath are Java‑based native QuPath extensions \cite{Goldsborough_Philps_etal._2024} and the execution of RESTful API requests to a model server coupled with handling the response via an extension, as demonstrated in the application of cell segmentation models applied to immunofluorescence images \cite{Sugawara_2023}. While the community is actively working on these integration strategies, there is currently no universal solution that fully addresses all integration and performance requirements.

Extensions may offer better integration with QuPath, allowing slightly improved performance and more widespread usage of the built-in QuPath models, but they lack the flexibility to customize models and modify their behavior. For example, the newest version of QuPath includes models such as StarDist \cite{Weigert_Schmidt} and InstanSeg \cite{Goldsborough_Philps_etal._2024} that can perform cell segmentation. Both models pose limitations when applied to simultaneous cell segmentation and classification. StarDist performs well only on convex, round shapes by design, whereas some neoplastic, inflammatory, and connective cells exhibit complex and non-convex shapes. InstanSeg provides only semantic segmentation without assigning classes to the segmented cells.

%\hfill

In contrast, our approach offers an alternative integration strategy. It utilizes the paquo library to directly interact with QuPath’s internal application programming interface from within Python. This enables data exchange and processing without the need for intermediate conversion steps and provides greater control over model customization, retraining, and the incorporation of custom processing steps.

The integration of our custom model with QuPath underscores its potential to significantly enhance the diagnostic process by reducing the time burden on pathologists and enabling them to focus on more complex interpretative tasks using familiar software. Leveraging a tool that is already well-established among pathologists increases the likelihood of its adoption into daily clinical workflows. The quantitative data generated through the automated workflow is critical for both clinical decision-making and research, facilitating more accurate biomarker analysis, enabling robust statistical evaluations, and supporting hypothesis generation and testing. Additionally, by streamlining cell segmentation and classification, the tool enhances the scalability and reproducibility of pathological assessments, ultimately contributing to improved diagnostic accuracy and patient outcomes.

\section{Conclusion and future work}

In this study, we address critical challenges in digital pathology and tackle the usability and deployment issues of the developed models in standard computing environments without the need for high-performance computing systems. Our multi-faceted approach encompasses data refinement through cross-relabeling, leveraging foundation models for robust cell segmentation and classification, optimizing model performance via knowledge distillation, and integrating the optimized model into the QuPath software for practical application. This approach is used to construct a capable, versatile, and adjustable model for cell segmentation and classification, with enhanced performance and usability.

\begin{sloppypar}
While our approach shows potential in the field of computational pathology, certain limitations persist. 
For example, our implementation currently exhibits lower performance in detecting macrophages. 
This serves as an instance of the broader challenge of accurately identifying complex cell types. In order to address this issue, extending our approach to incorporate additional data sources, exploring alternative modeling approaches, and integrating other imaging modalities such as immunohistochemical staining may help improve detection accuracy. Moreover, although the distilled model reduces computational demands, integrating advanced deep learning models into clinical practice requires addressing technological gaps and potential resistance to adopting new tools within established diagnostic processes.
\end{sloppypar}

Future work could focus on several key areas to refine the proposed approach and facilitate its adoption in clinical environments. Enhancing the cell-relabeling process with additional datasets \cite{Graham_Jahanifar_etal._2021} could improve the representation of underrepresented cell types and enhance overall model performance. Also, incorporating additional data sources, such as multi-modal imaging or complementary staining methods, may address limitations related to cell type differentiation and class imbalance. Exploring other foundation models \cite{Vorontsov_Bozkurt_etal._2024,Zimmermann_Vorontsov_etal._2024} or introducing additional modalities \cite{Ding_Wagner_etal._2024,Vaidya_Zhang_etal._2025} may provide alternative architectures better suited to specific tasks or offer improved efficiency. Implementing more complex knowledge distillation techniques \cite{Houyon_Cioppa_etal._2023,Zhang_Song_etal._2019} could further optimize the model's performance and adaptability. Additionally, deeper integration with QuPath or other digital pathology software could provide pathologists more control over cell quantification analysis directly within the QuPath interface, thereby increasing accessibility and usability. Such enhancements would not only refine model performance but also ensure greater adaptability and scalability within various clinical environments. Finally, extensive validation of the model by pathologists and benchmarking against independent datasets are essential steps toward establishing the model's reliability and fostering confidence in its clinical utility.

\section*{Acknowledgments} 
This work was funded in part by the Research Council of Norway grant no. 309439 SFI Visual Intelligence, and the North Norwegian Health Authority grant no. HNF1521-20.

\bibliographystyle{IEEEtran}
\begin{sloppypar}
\begin{thebibliography}{99}

\bibitem{chaplot2020neural} Chaplot, Devendra Singh, et al. "Neural topological slam for visual navigation." Proceedings of the IEEE/CVF conference on computer vision and pattern recognition. 2020.

\bibitem{maksymets2021thda} Maksymets, Oleksandr, et al. "Thda: Treasure hunt data augmentation for semantic navigation." Proceedings of the IEEE/CVF International Conference on Computer Vision. 2021.

\bibitem{mezghan2022memory} Mezghan, Lina, et al. "Memory-augmented reinforcement learning for image-goal navigation." 2022 IEEE/RSJ International Conference on Intelligent Robots and Systems (IROS). IEEE, 2022.

\bibitem{al2022zero} Al-Halah, Ziad, Santhosh Kumar Ramakrishnan, and Kristen Grauman. "Zero experience required: Plug \& play modular transfer learning for semantic visual navigation." Proceedings of the IEEE/CVF Conference on Computer Vision and Pattern Recognition. 2022.

\bibitem{ye2021auxiliary} Ye, Joel, et al. "Auxiliary tasks and exploration enable objectgoal navigation." Proceedings of the IEEE/CVF international conference on computer vision. 2021.

\bibitem{chaplot2020object} Chaplot, Devendra Singh, et al. "Object goal navigation using goal-oriented semantic exploration." Advances in Neural Information Processing Systems 33 (2020)

\bibitem{ramakrishnan2022poni} Ramakrishnan, Santhosh Kumar, et al. "Poni: Potential functions for objectgoal navigation with interaction-free learning." Proceedings of the IEEE/CVF Conference on Computer Vision and Pattern Recognition. 2022.

\bibitem{ramrakhya2022habitat} Ramrakhya, Ram, et al. "Habitat-web: Learning embodied object-search strategies from human demonstrations at scale." Proceedings of the IEEE/CVF Conference on Computer Vision and Pattern Recognition. 2022.

\bibitem{mousavian2019visual} Mousavian, Arsalan, et al. "Visual representations for semantic target driven navigation." 2019 International Conference on Robotics and Automation (ICRA). IEEE, 2019.

\bibitem{dhariwal2021diffusion} Dhariwal, Prafulla, and Alexander Nichol. "Diffusion models beat gans on image synthesis." Advances in neural information processing systems 34 (2021)

\bibitem{ho2022classifier} Ho, Jonathan, and Tim Salimans. "Classifier-free diffusion guidance." arXiv preprint arXiv:2207.12598 (2022).

\bibitem{nichol2021glide} Nichol, Alex, et al. "Glide: Towards photorealistic image generation and editing with text-guided diffusion models." arXiv preprint arXiv:2112.10741 (2021)

\bibitem{brooks2023instructpix2pix} Brooks, Tim, Aleksander Holynski, and Alexei A. Efros. "Instructpix2pix: Learning to follow image editing instructions." Proceedings of the IEEE/CVF Conference on Computer Vision and Pattern Recognition. 2023.

\bibitem{fu2023guiding} Fu, Tsu-Jui, et al. "Guiding instruction-based image editing via multimodal large language models." arXiv preprint arXiv:2309.17102 (2023).

\bibitem{geng2024instructdiffusion} Geng, Zigang, et al. "Instructdiffusion: A generalist modeling interface for vision tasks." Proceedings of the IEEE/CVF Conference on Computer Vision and Pattern Recognition. 2024.

\bibitem{zhou2024minedreamer} Zhou, Enshen, et al. "Minedreamer: Learning to follow instructions via chain-of-imagination for simulated-world control." arXiv preprint arXiv:2403.12037 (2024).

\bibitem{zhou2023esc} Zhou, Kaiwen, et al. "Esc: Exploration with soft commonsense constraints for zero-shot object navigation." International Conference on Machine Learning. PMLR, 2023.

\bibitem{yu2023l3mvn} Yu, Bangguo, Hamidreza Kasaei, and Ming Cao. "L3mvn: Leveraging large language models for visual target navigation." 2023 IEEE/RSJ International Conference on Intelligent Robots and Systems (IROS). IEEE, 2023.

\bibitem{gadre2023cows} Gadre, Samir Yitzhak, et al. "Cows on pasture: Baselines and benchmarks for language-driven zero-shot object navigation." Proceedings of the IEEE/CVF Conference on Computer Vision and Pattern Recognition. 2023.

\bibitem{shah2023navigation} Shah, Dhruv, et al. "Navigation with large language models: Semantic guesswork as a heuristic for planning." Conference on Robot Learning. PMLR, 2023.

\bibitem{cai2024bridging} Cai, Wenzhe, et al. "Bridging zero-shot object navigation and foundation models through pixel-guided navigation skill." 2024 IEEE International Conference on Robotics and Automation (ICRA). IEEE, 2024.

\bibitem{yu2023co} Yu, Bangguo, Hamidreza Kasaei, and Ming Cao. "Co-NavGPT: Multi-robot cooperative visual semantic navigation using large language models." arXiv preprint arXiv:2310.07937 (2023).

\bibitem{wu2024voronav} Wu, Pengying, et al. "Voronav: Voronoi-based zero-shot object navigation with large language model." arXiv preprint arXiv:2401.02695 (2024).

\bibitem{qin2023mp5} Qin, Yiran, et al. "Mp5: A multi-modal open-ended embodied system in minecraft via active perception." arXiv preprint arXiv:2312.07472 (2023).

\bibitem{du2024learning} Du, Yilun, et al. "Learning universal policies via text-guided video generation." Advances in Neural Information Processing Systems 36 (2024).

\bibitem{ajay2024compositional} Ajay, Anurag, et al. "Compositional foundation models for hierarchical planning." Advances in Neural Information Processing Systems 36 (2024).

\bibitem{liang2024skilldiffuser} Liang, Zhixuan, et al. "Skilldiffuser: Interpretable hierarchical planning via skill abstractions in diffusion-based task execution." Proceedings of the IEEE/CVF Conference on Computer Vision and Pattern Recognition. 2024.

\bibitem{heusel2017gans} Heusel, Martin, et al. "Gans trained by a two time-scale update rule converge to a local nash equilibrium." Advances in neural information processing systems 30 (2017).

\bibitem{zhang2018unreasonable} Zhang, Richard, et al. "The unreasonable effectiveness of deep features as a perceptual metric." Proceedings of the IEEE conference on computer vision and pattern recognition. 2018.

\bibitem{brown2020language} Brown, Tom B. "Language models are few-shot learners." arXiv preprint arXiv:2005.14165 (2020).

\bibitem{podell2023sdxl} Podell, Dustin, et al. "Sdxl: Improving latent diffusion models for high-resolution image synthesis." arXiv preprint arXiv:2307.01952 (2023).

\bibitem{brohan2022rt} Brohan, Anthony, et al. "Rt-1: Robotics transformer for real-world control at scale." arXiv preprint arXiv:2212.06817 (2022).

\bibitem{brohan2023rt} Brohan, Anthony, et al. "Rt-2: Vision-language-action models transfer web knowledge to robotic control." arXiv preprint arXiv:2307.15818 (2023).

\bibitem{li2024manipllm} Li, Xiaoqi, et al. "Manipllm: Embodied multimodal large language model for object-centric robotic manipulation." Proceedings of the IEEE/CVF Conference on Computer Vision and Pattern Recognition. 2024.

\bibitem{shah2023vint} Shah, Dhruv, et al. "ViNT: A foundation model for visual navigation." arXiv preprint arXiv:2306.14846 (2023).

\bibitem{liu2024visual} Liu, Haotian, et al. "Visual instruction tuning." Advances in neural information processing systems 36 (2024).

\bibitem{hu2021lora} Hu, Edward J., et al. "Lora: Low-rank adaptation of large language models." arXiv preprint arXiv:2106.09685 (2021).

\bibitem{qin2023supfusion} Qin, Yiran, et al. "SupFusion: Supervised LiDAR-camera fusion for 3D object detection." Proceedings of the IEEE/CVF International Conference on Computer Vision. 2023.

\bibitem{qin2024worldsimbench} Qin, Yiran, et al. "Worldsimbench: Towards video generation models as world simulators." arXiv preprint arXiv:2410.18072 (2024).

\bibitem{yu2025gamefactory} Yu, Jiwen, et al. "GameFactory: Creating New Games with Generative Interactive Videos." arXiv preprint arXiv:2501.08325 (2025).

\bibitem{zhou2024code} Zhou, Enshen, et al. "Code-as-Monitor: Constraint-aware Visual Programming for Reactive and Proactive Robotic Failure Detection." arXiv preprint arXiv:2412.04455 (2024).

\bibitem{zhang2024ad} Zhang, Zaibin, et al. "AD-H: Autonomous Driving with Hierarchical Agents." arXiv preprint arXiv:2406.03474 (2024).

\bibitem{wang2024toward} Wang, Chaoqun, et al. "Toward Accurate Camera-based 3D Object Detection via Cascade Depth Estimation and Calibration." arXiv preprint arXiv:2402.04883 (2024).

\bibitem{huang2024story3d} Huang, Yuzhou, et al. "Story3d-agent: Exploring 3d storytelling visualization with large language models." arXiv preprint arXiv:2408.11801 (2024).

\bibitem{savinov2018semi} Savinov, Nikolay, Alexey Dosovitskiy, and Vladlen Koltun. "Semi-parametric topological memory for navigation." arXiv preprint arXiv:1803.00653 (2018).

\bibitem{majumdar2022zson} Majumdar, Arjun, et al. "Zson: Zero-shot object-goal navigation using multimodal goal embeddings." Advances in Neural Information Processing Systems 35 (2022): 32340-32352.

\bibitem{yadav2023offline} Yadav, Karmesh, et al. "Offline visual representation learning for embodied navigation." Workshop on Reincarnating Reinforcement Learning at ICLR 2023. 2023.

\bibitem{yadav2023ovrl} Yadav, Karmesh, et al. "Ovrl-v2: A simple state-of-art baseline for imagenav and objectnav." arXiv preprint arXiv:2303.07798 (2023).

\bibitem{sun2024fgprompt} Sun, Xinyu, et al. "FGPrompt: fine-grained goal prompting for image-goal navigation." Advances in Neural Information Processing Systems 36 (2024).

\bibitem{zhu2017target} Zhu, Yuke, et al. "Target-driven visual navigation in indoor scenes using deep reinforcement learning." 2017 IEEE international conference on robotics and automation (ICRA). IEEE, 2017.

\bibitem{koh2024generating} Koh, Jing Yu, Daniel Fried, and Russ R. Salakhutdinov. "Generating images with multimodal language models." Advances in Neural Information Processing Systems 36 (2024).

\bibitem{krantz2022instance} Krantz, Jacob, et al. "Instance-specific image goal navigation: Training embodied agents to find object instances." arXiv preprint arXiv:2211.15876 (2022).

\bibitem{schulman2017proximal} Schulman, John, et al. "Proximal policy optimization algorithms." arXiv preprint arXiv:1707.06347 (2017).

\bibitem{anderson2018evaluation} Anderson, Peter, et al. "On evaluation of embodied navigation agents." arXiv preprint arXiv:1807.06757 (2018).

\bibitem{lin2024navcot} Lin, Bingqian, et al. "NavCoT: Boosting LLM-Based Vision-and-Language Navigation via Learning Disentangled Reasoning." arXiv preprint arXiv:2403.07376 (2024).

\bibitem{NavGPT} Zhou, Gengze, Yicong Hong, and Qi Wu. "Navgpt: Explicit reasoning in vision-and-language navigation with large language models." Proceedings of the AAAI Conference on Artificial Intelligence.

\bibitem{hahn2021no} Hahn, Meera, et al. "No rl, no simulation: Learning to navigate without navigating." Advances in Neural Information Processing Systems 34 (2021): 26661-26673.

\bibitem{li2025t2isafety} Li, Lijun, et al. "T2ISafety: Benchmark for Assessing Fairness, Toxicity, and Privacy in Image Generation." arXiv preprint arXiv:2501.12612 (2025).

\bibitem{an2024agfsync} An, Jingkun, et al. "AGFSync: Leveraging AI-Generated Feedback for Preference Optimization in Text-to-Image Generation." arXiv preprint arXiv:2403.13352 (2024).


\end{thebibliography}
\end{sloppypar}

\clearpage
\beginsupplement
\section*{Appendix}
\renewcommand{\thesubsection}{S\arabic{subsection}}

\subsection{\label{chap:S1}PanNuke and MoNuSAC preprocessing}
The PanNuke dataset comprises a set of 7,901 RGB patches, each with dimensions of $256 \times 256$ pixels, which we set as the standard patch size for our analysis. In contrast, the MoNuSAC dataset encompasses 294 images of heterogeneous dimensions. To standardize the MoNuSAC images with our experiments, we implement a standardization protocol. Specifically, for images exceeding the dimensions of $256 \times 256$ pixels, we segment them into equal-sized patches and apply mirror padding to the remaining portions to avoid information loss at the peripherals. Patches with dimensions less than $128 \times 128$ pixels are excluded from the dataset due to the insufficient resolution to capture relevant cellular details. For patches where either dimension falls between 128 and 256 pixels, we employ upsampling to achieve the standard patch size. As a result, we obtain a total of 2,823 RGB patches derived from the MoNuSAC dataset for subsequent analysis. For additional details on the MoNuSAC data preparation process, refer to the source code \cite{Shvetsov_2025a}.
\clearpage

\subsection{\label{chap:S2}Data usage for the methodology}

\counterwithin{figure}{subsection}
\renewcommand{\thefigure}{S\arabic{subsection}}

\begin{figure}[h!]
    \centering
    \includegraphics[width=\textwidth, height=0.85\textheight, keepaspectratio]{images/A2.pdf}
    \caption{Overview of the methodology for cross-labeling, dataset refinement, and model comparison. (1) Cross-relabeling - training and testing cell classification models, (2) Cross-relabeling - using cell classification models to create refined dataset, (3) Fine-tuning and training models for comparison, (4) Student knowledge distillation with refined dataset}
    \label{fig:S2}
\end{figure}
\clearpage

\subsection{\label{chap:S3}Confusion matrices for classification models}
\counterwithin{figure}{subsection}
\renewcommand{\thefigure}{S\arabic{subsection}.\arabic{figure}}

\begin{figure}[h!]
    \centering
    \includegraphics[width=\textwidth, height=0.4\textheight, keepaspectratio]{images/A3_1.pdf}
    \caption{Confusion matrix for PanNuke trained model}
    \label{fig:S3.1}
\end{figure}

\begin{figure}[h!]
    \centering
    \includegraphics[width=\textwidth, height=0.4\textheight, keepaspectratio]{images/A3_2.pdf}
    \caption{Confusion matrix for MoNuSAC trained model}
    \label{fig:S3.2}
\end{figure}

\clearpage

\subsection{\label{chap:S4}Datasets cell counts}

\counterwithin{table}{subsection}
\renewcommand{\thetable}{S\arabic{subsection}}

\begin{table}[h!]
\renewcommand{\arraystretch}{2.0}
\centering
\caption{\label{tab:S4}Cell counts for PanNuke, MoNuSAC and refined datasets. Numbers in parentheses indicate preprocessed cell counts for cell classifier models training and testing.}
%\adjustbox{max width=\textwidth}{%
\begin{tabular}{|l|c|c|c|}
\hline
%\rowcolor{gray!30}
Cell type & PanNuke & MoNuSAC & Refined \\
\hline
Neoplastic & 77,403 (68,031) & - & 105,451 \\
\hline
Epithelial & 26,572 (23,207) & - & 29,926 \\
\hline
Epithelial (benign and malignant) & - & 31,402 & - \\
\hline
Inflammatory & 32,276 & - & - \\
\hline
Lymphocytes & - & 37,045 (33,104) & 65,275 \\
\hline
Neutrophils & - & 1,355 (1,252) & 3,833 \\
\hline
Macrophage & - & 1,842 (1,695) & 3,410 \\
\hline
Dead & 2,908 & - & 2,908 \\
\hline
Connective & 50,585 & - & 50,585 \\
\hline
\end{tabular}
%
%}
\end{table}



\clearpage

\subsection{\label{chap:S5}Definition of validation metrics}
\counterwithin{equation}{subsection}
\renewcommand{\theequation}{\arabic{equation}}

\subsubsection{\label{chap:S5.1}R\textsuperscript{2}}
The coefficient of determination, denoted as $R^2$, is a statistical measure that represents the proportion of variance in the dependent variable that is predictable from the independent variables. In the context of cell quantification in pathology, $R^2$ is used to assess how well the predicted quantities of different cell types in a patch align with the actual quantities observed in the ground truth data, with higher values representing more accurate quantification. $R^2$ is defined as
\begin{equation*}
R^2 = 1 - \frac{\sum_{i=1}^n (y_i - \hat{y}_i)^2}{\sum_{i=1}^n (y_i - \bar{y})^2},
\end{equation*}
where $y_i$ represents the actual number of cells of a specific type in the $i$-th image, $\hat{y}_i$ represents the predicted number of cells of that type in the $i$-th image, $\bar{y}$ is the mean of the actual numbers across all images, and $n$ is the total number of images in the dataset.

The $R^2$ metric has a range of $(-\infty, 1]$. An $R^2$ of 1 indicates perfect prediction, where all predicted values exactly match the actual values. An $R^2$ of 0 suggests that the model explains none of the variability of the response data around its mean. If $R^2$ is negative, it indicates that the model performs worse than a model that simply predicts the mean of the actual values for all observations.

\subsubsection{\label{chap:S5.2}PQ}
Panoptic Quality ($PQ$) is a comprehensive metric used to evaluate the performance of segmentation models in tasks that require both instance segmentation and classification. $PQ$ provides a single score that encapsulates both the detection accuracy (i.e., how many objects were correctly identified) and the segmentation quality (i.e., how accurately the objects' boundaries were delineated). This metric is particularly useful in multiclass scenarios where each pixel is classified into distinct categories, such as different cell types in pathology images.

$PQ$ is calculated as the product of two terms: Detection Quality ($DQ$) and Segmentation Quality ($SQ$). It can be expressed as
\begin{equation*}
PQ = DQ \cdot SQ,
\end{equation*}
where
\begin{equation*}
DQ = \frac{TP}{TP + 0.5\, FP + 0.5\, FN},
\end{equation*}
\begin{equation*}
SQ = \frac{\sum_{(p, g) \in \mathcal{M}} IoU(p, g)}{TP}.
\end{equation*}
In these formulas, $TP$ denotes the number of correctly matched instances between ground truth and prediction, $FP$ denotes the predicted instances that have no corresponding ground truth, $FN$ denotes the ground truth instances that were not detected, $IoU(p, g)$ is the Intersection over Union for a pair of matched instances $p$ (prediction) and $g$ (ground truth), and $\mathcal{M}$ is the set of matched pairs.

The $PQ$ metric is calculated for each class and is averaged across classes to provide a global performance measure.

The $PQ$ score has a range of $[0, 1.0]$, where a higher score indicates better performance in both detecting and segmenting the instances correctly. A $PQ$ of 1 signifies perfect identification and segmentation of all instances, whereas a $PQ$ of 0 indicates that no instances were correctly identified and segmented.

\clearpage

\subsection{\label{chap:S6}Segmentation and Detection quality metrics for teacher and student models}

\begin{table}[h!]
\renewcommand{\arraystretch}{2.0}
\centering
\caption{Segmentation and detection quality for student and teacher models (CI 95\%)}
\label{tab:S6}
%\adjustbox{max width=\textwidth}{%
\begin{tabular}{|l|c|c|}
\hline
%\rowcolor{gray!30}
Metric & Teacher & Student \\
\hline
$SQ_{neoplastic}$ & 0.819 (0.815--0.823) & 0.824 (0.819--0.828) \\
\hline
$SQ_{lymphocyte}$ & 0.795 (0.788--0.802) & 0.790 (0.783--0.796) \\
\hline
$SQ_{connective}$ & 0.770 (0.762--0.776) & 0.780 (0.772--0.786) \\
\hline
$SQ_{dead}$ & 0.659 (0.623--0.688) & 0.657 (0.624--0.695) \\
\hline
$SQ_{epithelial}$ & 0.780 (0.770--0.790) & 0.788 (0.779--0.797) \\
\hline
$SQ_{macrophage}$ & 0.788 (0.760--0.810) & 0.757 (0.730--0.783) \\
\hline
$SQ_{neutrofil}$ & 0.782 (0.761--0.801) & 0.775 (0.759--0.792) \\
\hline
$DQ_{neoplastic}$ & 0.706 (0.692--0.719) & 0.727 (0.712--0.741) \\
\hline
$DQ_{lymphocyte}$ & 0.675 (0.656--0.698) & 0.713 (0.691--0.734) \\
\hline
$DQ_{connective}$ & 0.566 (0.546--0.584) & 0.583 (0.565--0.602) \\
\hline
$DQ_{dead}$ & 0.410 (0.361--0.465) & 0.435 (0.306--0.561) \\
\hline
$DQ_{epithelial}$ & 0.668 (0.639--0.694) & 0.673 (0.644--0.702) \\
\hline
$DQ_{macrophage}$ & 0.657 (0.583--0.727) & 0.615 (0.531--0.703) \\
\hline
$DQ_{neutrofil}$ & 0.691 (0.625--0.753) & 0.729 (0.679--0.778) \\
\hline
\end{tabular}
%
%}
\end{table}

\clearpage

\subsection{\label{chap:S7}QuPath integration method}
We adopt an integration strategy leveraging the paquo \cite{Bayer_AG} library, a Python package that enables direct interaction with QuPath’s internal API, thereby facilitating seamless data exchange without intermediate conversion steps. The data processing pipeline (\hyperref[fig:S7]{Appendix Figure S7}) begins with the acquisition of WSIs and their associated annotations from QuPath, which are represented as Shapely \cite{Gillies_Wel_etal._2024} polygons. Utilizing paquo, we directly read, create, and modify these annotations and detections within a QuPath project in the Python environment. Images are then cropped using these polygons and processed by cell segmentation and classification models employing standard vision processing toolkits such as OpenCV, pyvips, and PyTorch. Additionally, QuPath employs Groovy scripts to initiate a Python process that starts the entire pipeline from QuPath graphical interface: fetching polygons, extracting images from them, and running deep learning model inference on the cropped images. 
The results are returned to QuPath, leveraging paquo's Python bindings to manipulate QuPath data while minimizing the computational overhead typically associated with cross-environment communication.

\counterwithin{figure}{subsection}
\renewcommand{\thefigure}{S\arabic{subsection}}

\begin{figure}[h!]
    \centering
    \includegraphics[width=\textwidth]{images/A7.pdf}
    \caption{QuPath integration workflow using Python environment}
    \label{fig:S7}
\end{figure}

Compared to traditional workflows that involve exporting annotations as GeoJSON, classifying them in Python, and reimporting them into QuPath, our approach offers several advantages. We eliminate the need to switch between programming languages, providing a cohesive and streamlined development process entirely within QuPath software and removing the necessity to use other tools. Meanwhile, we avoid storing annotations as intermediate JSON files unless required for external use or archiving. By conducting the entire inference and post-processing workflow within the Python environment, we leverage the power and flexibility of Python libraries for image processing and machine learning. This approach also enables adjustments to any set of labels and models, thereby improving its applicability.

%\hfill

The distilled model and QuPath integration code are packaged into a Docker container, enabling streamlined execution with the Docker engine. Detailed integration code and deployment instructions can be found in the GitHub repository \cite{Shvetsov_2025b}.

Despite these benefits, we acknowledge that the paquo library is a proof‑of‑concept project in its early development stage and has not been tested across all versions of QuPath.

\clearpage

\subsection{\label{chap:S8}Data and code availability statement}
All datasets, models, and code used in this study are publicly available and can be obtained from the repositories listed below. 
The PanNuke \cite{Gamper_Koohbanani_etal._2019} and MoNuSAC \cite{Verma_Kumar_etal._2021} datasets are publicly accessible, and download information along with detailed descriptions can be found in their respective articles. Preprocessing scripts for PanNuke and MoNuSAC data, as well as individual cell extraction scripts, are available on GitHub \cite{Shvetsov_2025a}. The H-Optimus foundation model used in our experiments can be downloaded from the HuggingFace repository \cite{hoptimus2024}, and model information is available on GitHub \cite{Saillard_Jenatton_etal._2024}. In addition, the integration code for QuPath and the distilled model packaged in a Docker container are provided in the repository \cite{Shvetsov_2025b}, and paquo Python library is available from the authors GitHub repository \cite{Bayer_AG}.
\clearpage

\end{document}


\appendix
% \section{Appendix}

\section{DTE-GAN with MA-GP}

Recent methods \citep{DF_GAN_CVPR,DTGAN,SSA-GAN} have significantly improved quality of synthesised images on COCO dataset \citep{mscoco}. Their improved performance may be associated with the Matching-Aware zero-centered Gradient Penalty (MA-GP) term adopted in these methods. MA-GP is applied to Discriminator with real images and its corresponding text to smooth the loss function that allows the Generator to synthesise more realistic images, incorporating MA-GP into DTE-GAN by changing the Discriminator to conditional Discriminator (towards one-way output for gradient penalty). Mismatch pairs are not used as DTE-GAN has a multi-modal contrastive branch for text-image alignment. For conditional prediction, following similar setup of DF-GAN \citep{DF_GAN_CVPR} of replicating sentence features and concatenating with image features to predict logit values for adversarial loss, the modified adversarial loss function of Discriminator for conditional loss with MA-GP is:

\begin{align}
\begin{split}
L^{Adv}_{D}=&-\mathbb{E}_{x \sim \mathbb{P}_{data}}[\max (0,1-D(x, S_D))] \\
&+\mathbb{E}_{\hat{x} \sim \mathbb{P}_{G}}[\max (0,1+D(\hat{x}, S_D))]\\
&+k \mathbb{E}_{x \sim \mathbb{P}_{data}}\left[\left(\left\|\nabla_{x} D(x, S_D)\right\|+\left\|\nabla_{S_D} D(x, S_D)\right\|\right)^{p}\right] \\
\end{split}
\end{align}

Here, \textit{k} and \textit{p} are hyper-parameters (we use the same hyper-parameter values from DF-GAN \citep{DF_GAN_CVPR}). For training discriminator-side word embeddings and their sentence encoder from real image-text pairs, we do not update the weights using the gradient of fake conditional prediction from the adversarial loss. Conditional Adversarial loss for Generator is:
\begin{align}
L^{Adv}_{G}=&-\mathbb{E}_{\hat{x} \sim \mathbb{P}_{G}}[D(\hat{x}, S_D)]    
\end{align}

The final objective function for the Generator and Discriminator is defined as:
 
 \begin{align}
     \mathcal{L}_{G} &= \mathcal{L}_{Adv}^{G} +\lambda_1 \mathcal{L}_{CA} + \lambda_2 \mathcal{L}_{\text{cont}}^{G} \\
     \mathcal{L}_{D} &= \mathcal{L}_{GAN}^{D} + \lambda_3 \mathcal{L}_{\text{cont}}^D 
 \end{align}

\section{Text Encoding Scheme}

We have used a single-stage Bi-LSTM \cite{bi-lstm} for text encoding, following the popular DAMSM \cite{AttnGAN} embeddings commonly employed in lightweight GAN models \cite{AttnGAN,DMGAN,SSA-GAN,DFGAN}. The DAMSM embeddings are trained to learn discriminative features by distinguishing between instances, ensuring a fair comparison focused on design principles rather than simply increasing the number of parameters. Additionally, we have conducted an ablation study by replacing the Bi-LSTM \citep{bi-lstm} with a 4-layer Transformer encoder \cite{transformers} in both the generator and discriminator text encoders, and we have reported the results in Table \ref{tab:bi_lstm vs Transformer}. 

\begin{table}[!ht]
\centering
\begin{tabular}{ccccc}
\toprule
                 \textbf{Dataset} & \textbf{Encodings}  & \textbf{IS} $\uparrow$ & \textbf{FID} $\downarrow$ & \textbf{R \%} $\uparrow$\\
\midrule
\multirow{2}{*}{\textbf{CUB}} & \textbf{Bi-LSTM}  & $5.12$ & $13.67$ & $86.64$\\
%  \cline{2-5}
                     
                     & \textbf{Transformer Encoder} & $\color{red}5.19$ & $\color{red}13.12$ &  $\color{red}87.9$   \\
\midrule
\multirow{2}{*}{\textbf{Oxford}} & \textbf{\textbf{Bi-LSTM}}  &  $4.21$  & $30.07$ & $83.19$  \\
%  \cline{2-5}        
                     
                     & \textbf{Transformer Encoder} & $\color{red}4.27$  & $\color{red}29.61$ & $\color{red}83.94$  \\
\bottomrule
\end{tabular}
\caption{We compare quality of T2I generation using Bi-LSTM and 4 layer Transformer Encoder text encoding scheme and report the results on CUB and Oxford-102 dataset.}
\label{tab:bi_lstm vs Transformer}
\end{table}

 \section{Details of the Proposed Architecture}
\label{sec:intro_arch}

In this section, we elaborate internal architecture details of the DTE-GAN. Proposed model is implemented using Pytorch \citep{NEURIPS2019_9015} framework. DTE-GAN architecture consists of a dual text embedding setup (Section \ref{sec:dualtext}), a single-stage Generator (Section \ref{sec:gen}) and a Discriminator (Section \ref{sec:disc}). 
%DTE-GAN + MA-GP achieves similar performance as SSA-GAN \citep{SSA-GAN} 


\subsection{Dual Text Embeddings}
\label{sec:dualtext}
% \vspace{-0.3cm}

In the Dual Text Embeddings setup, bi-Directional LSTM \citep{bi-lstm} are used as text encoder both generator-side and discriminator-side. For each direction in the LSTM, hidden layer size is set as $128$. The size of word embeddings $W_D$ and $W_G$ is set to $256$. The sentence embeddings $S_G$, $S_D$ are encoded from the output of last hidden state of respective text encoders. For both the text encoders, sentence embedding size is set to $256$.  

\begin{figure}[t]
    \centering
    \includegraphics[scale = 0.40]{images/UpBlock_AAAI.drawio.pdf}
    \caption{UpBlock used in Generator of DTE-GAN.}
    \label{Fig:UpBlock}
\end{figure}

\subsection{Generator}
\label{sec:gen}

Single-stage generator $G$ is used to generate $256\times 256$ resolution images with base channel dimension of $64$. Details of $G$'s architecture are shown in Table \ref{tab:Gen_arch}. The generator $G$ takes noise $z$ along with generator-side sentence embeddings $S_G$ and the discriminator-side sentence embedding $S_D$ and passes them through a set of linear layers followed by a set of upsampling blocks (UpBlocks). UpBlock at each stage is utilised for up sampling spatial features as shown in Figure \ref{Fig:UpBlock}. $S_G$ and $S_D$ are also used to calculate modulation parameters for Conditional Batch Normalisation \citep{SDGAN}. Features are passed through a self modulation convolution and a $1 \times 1$ convolution resulting in generation of final image of dimension $3 \times 256 \times 256$

\begin{table*}[h!]
    \centering
    \begin{tabular}{c}
         \toprule
         \midrule
         \makecell[c]{$z \epsilon \mathbb{R}^{100}$  $\sim$ $\mathcal{N}(0,I)$ , $S_G$ $\epsilon$ $\mathbb{R}^{256}$, $S_D$  $\epsilon$ $\mathbb{R}^{256}$, \\$W_G$  $\epsilon$ $\mathbb{R}^{256}$, $W_D$  $\epsilon$ $\mathbb{R}^{256}$}\\
         \midrule
         Linear(512) $\longrightarrow$ $512$ \\
         \midrule
         Conditional Augmentation(512) $\longrightarrow$ $200$ \\
         \midrule
         Linear(200+100) $\longrightarrow$ $(8*ch) \times 4 \times 4$ \\
         \midrule
         UpBlock $\longrightarrow$ $(8*ch) \times 8 \times 8$  \\
         \midrule
         UpBlock $\longrightarrow$ $(8*ch) \times 16 \times 16$  \\
         \midrule
         UpBlock $\longrightarrow$ $(4*ch) \times 32 \times 32$  \\
         \midrule
         UpBlock $\longrightarrow$ $(2*ch) \times 64 \times 64$  \\
         \midrule
         UpBlock $\longrightarrow$ $(2*ch) \times 128 \times 128$  \\
         \midrule
         UpBlock $\longrightarrow$ $ch \times 256 \times 256$  \\
         \midrule
         Self Modulation Convolution $\longrightarrow$ $ch \times 256 \times 256$  \\
         \midrule
         $1 \times 1$ Convolution $\longrightarrow$ $3 \times 256 \times 256$  \\
         \midrule
         \bottomrule
    \end{tabular}
    \caption{Generator architecture of DTE-GAN. Base channel dimension \it{ch} = $64$.}
    \label{tab:Gen_arch}
\end{table*}




\begin{figure}[h]
    \centering
    \includegraphics[scale = 0.50]{images/DownBlock.drawio.pdf}
    \caption{DownBlock used in Discriminator of DTE-GAN.}
    \label{Fig:DownBlock}
\end{figure}

\subsection{Discriminator}
\label{sec:disc}
% \vspace{-0.3cm}

 Discriminator $D$ is utilised to provide adversarial loss and also act as a feature extractor for multi-modal contrastive loss (as shown in Table \ref{tab:Disc_arch}). Unlike multiple / multi-stage discriminator setup, presented model with a single discriminator, is easy to train and not having a cumbersome training procedure. The discriminator $D$ takes image of dimension $3 \times 256 \times 256$ and passes it through a set of down sampling blocks (DownBlocks- as shown in Figure \ref{Fig:DownBlock}) followed by two branches, one for the adversarial loss and the other multi-modal contrastive loss.

\begin{table*}[h!]
    \centering
    \begin{tabular}{c|c}
         \toprule
         \midrule
         \multicolumn{2}{c}{RGB images $3\times256\times256$, $S_D$ $\epsilon$ $\mathbb{R}^{256}$, $W_D$ $\epsilon$ $\mathbb{R}^{256}$ } \\
         \midrule
         \multicolumn{2}{c}{DownBlock $\longrightarrow$ $ch \times 128 \times 128$} \\
         \midrule
         \multicolumn{2}{c}{DownBlock $\longrightarrow$ $(2*ch) \times 64 \times 64$} \\
         \midrule
         \multicolumn{2}{c}{DownBlock $\longrightarrow$ $(4*ch) \times 32 \times 32$} \\
         \midrule
         \multicolumn{2}{c}{DownBlock $\longrightarrow$ $(4*ch) \times 16 \times 16$} \\
         \midrule
         \multicolumn{2}{c}{DownBlock $\longrightarrow$ $(4*ch) \times 8 \times 8$} \\
         \midrule
         DownBlock $\longrightarrow$ $(8*ch) \times 4 \times 4$ & DownBlock $\longrightarrow$ $(8*ch) \times 4 \times 4$ \\
         \midrule
         ResBlock $\longrightarrow$ $(8*ch) \times 4 \times 4$ & ResBlock $\longrightarrow$ $(8*ch) \times 4 \times 4$ \\
         \midrule
         Linear($(8*ch) \times 4 \times 4$) $\longrightarrow$ $1$ & Linear($(8*ch) \times 4 \times 4$) $\longrightarrow$ $256$ \\
         \midrule
         Adversarial Loss & Multi-Modal Contrastive Loss \\
         \midrule
         \bottomrule

    \end{tabular}
    \caption{Discriminator architecture of DTE-GAN. Base channel dimension \it{ch} = $64$.}
    \label{tab:Disc_arch}
\end{table*}

\subsection{Implementation Details}
Implementation of the models is done using the PyTorch framework \citep{NEURIPS2019_9015} and optimising the network using Adam optimiser \citep{Adam} with the following hyper parameters: $\beta_1 = 0.5$, $\beta_2 = 0.999$, batch size = $24$, learning rate = $0.0002$, $\lambda_1 = 1$, $\lambda_2 = 1$ and $\lambda_3 = 1$. Spectral Normalisation \citep{sn_gan} is used for all convolutions and fully connected layers in generator and discriminator. The model is trained for 600 epochs on CUB and Oxford-102 datasets (takes $\sim$4 days in 2 NVIDIA 1080Ti GPUs) and 120 epochs for COCO dataset (takes $\sim$7 days in 2 NVIDIA 1080Ti GPUs). During inference, we report results with exponential moving average weights, with a decay rate of 0.999. For R-precision, we obtain text features from D-side Bi-LSTM sentence encoder and image features from discriminator network.

\end{document}
