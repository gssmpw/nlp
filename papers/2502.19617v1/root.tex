

\documentclass[letterpaper, 10 pt, conference]{ieeetran}  % Comment this line out if you need a4paper



\IEEEoverridecommandlockouts                             
\usepackage[noend]{algpseudocode}
\usepackage{algorithm}
\usepackage{amssymb}
\usepackage{bbm}
\usepackage{dsfont}
\usepackage[left=1.69cm,right=1.69cm,top=2.12cm,bottom=1.52cm]{geometry}


\newcommand{\algorithmautorefname}{Alg.}

\usepackage{xspace}
\usepackage{xcolor}
\definecolor{softgreen}{RGB}{34,139,34} % A visually appealing forest green
\usepackage[colorlinks=true, linkcolor=softgreen, citecolor=softgreen, urlcolor=softgreen]{hyperref}

\renewcommand{\figureautorefname}{Fig.} % Changes "Figure" to "Fig."
\renewcommand{\equationautorefname}{Eq.} % Changes "Equation" to "Eq."
\usepackage{array}


\newcommand{\R}{\ensuremath{\mathbb{R}}\xspace}
\newcommand{\X}{\ensuremath{\mathcal{X}}\xspace}
\newcommand{\T}{\ensuremath{\mathcal{T}}\xspace}
\newcommand{\Xg}{\X_G} 
\newcommand{\xs}{\ensuremath{\x_\start}} 
\newcommand{\U}{\ensuremath{\mathcal{U}}\xspace}
\newcommand{\uu}{u}
\newcommand{\Xfree}{\ensuremath{\X_{\text{free}}}\xspace}
\newcommand{\Xobs}{\ensuremath{\X_{\text{obs}}}\xspace}
\newcommand{\K}{\ensuremath{I}\xspace}
\newcommand{\Kspace}{\K-space\xspace}
\newcommand{\key}{\ensuremath{\mathbb{K}}\xspace}
\newcommand{\skey}{\ensuremath{\mathds{k}}\xspace}
\newcommand{\keys}{\ensuremath{\key_\text{start}}\xspace}
\newcommand{\keyg}{\ensuremath{\key_\text{goal}}\xspace}
\newcommand{\Kobs}{\ensuremath{\K_{\text{obs}}}\xspace}
\newcommand{\Kfree}{\ensuremath{\K_{\text{free}}}\xspace}
\newcommand{\x}{x}
\newcommand{\xr}{\x_{rand}}
\newcommand{\Kf}{{\ensuremath{\K_\free}}\xspace}
\newcommand{\Ko}{{\ensuremath{\K_\text{obs}}}\xspace}
\renewcommand{\xi}{\x_{I}}
\newcommand{\prm}{\textsc{prm}\xspace}
\newcommand{\lprm}{\textsc{lazy-prm}\xspace}
\newcommand{\euclid}{image distance}
\newcommand{\learned}{learned}

\newcommand{\start}{\text{start}}
\newcommand{\source}{\text{src}}
\newcommand{\destination}{\text{dst}}
\newcommand{\goal}{\text{goal}}
\newcommand{\free}{\text{free}}
\newcommand{\dof}{\textsc{dof}\xspace}

\newcommand{\im}{\ensuremath{Im}\xspace}
\newcommand{\IM}{\ensuremath{\mathcal{IM}}\xspace}


% Configuration Space
\newcommand{\C}{\X}
\newcommand{\Cs}{\X}
\newcommand{\Cf}{\Xfree}
\newcommand{\Co}{\Xobs}

\usepackage[textsize=small]{todonotes}
\setlength{\marginparwidth}{1.2cm}

\usepackage{graphics} 
\usepackage{listings}
\usepackage{comment}
\usepackage{epsfig} 

\usepackage{amsmath} 
\usepackage{cite}
\usepackage{subcaption}
\usepackage{xcolor}
 \usepackage{url}
\definecolor{gg}{RGB}{0, 155, 85} 
\definecolor{json-key}{rgb}{0.13,0.55,0.13}
\definecolor{json-value}{rgb}{0.25,0.25,0.25}
\definecolor{json-string}{rgb}{0.9,0.3,0.3}
\usepackage{caption}
\captionsetup[figure]{font=small, labelfont=small, labelsep=period}
\captionsetup[table]{font=small, labelfont=small, labelsep=period}

\lstdefinelanguage{json}{
  basicstyle=\ttfamily,
    commentstyle=\color{gray},
    stringstyle=\color{orange},
    numbers=left,
    numberstyle=,
    numbersep=5pt,
    breaklines=true,
    frame=lines,
    backgroundcolor=\color{white!10},
    captionpos=b
}

\newcommand\centerImage[2][]%
  {\raisebox
    {-\dimexpr0.5\height}%
    [\dimexpr0.5\height+1mm]%
    [\dimexpr0.5\height+1mm]%
    {\includegraphics[#1]{#2}}%
  }


\title{\LARGE \bf  Image-Based Roadmaps for Vision-Only Planning and Control of Robotic Manipulators}

\author{Sreejani Chatterjee$^{1}$, Abhinav Gandhi $^{1}$, Berk Calli$^{1}$ and Constantinos Chamzas $^{1}$ % stops a space
% \thanks{*This paper was supported in part by the National Science Foundation under grant
% IIS-1900953 and CMMI-1928506.}% <-this % stops a space
\thanks{$^{1}$S Chatterjee, $^{1}$A Gandhi,  $^{1}$C Chamzas and $^{1}$B Calli are with Department of Robotics Engineering,
        Worcester Polytechnic Institute, Worcester, MA 01609, USA
        {\tt\small schatterjee@wpi.edu, agandhi@wpi.edu}}%
}

\begin{document}
\maketitle
\thispagestyle{empty}
\pagestyle{empty}


\begin{abstract}

This work presents a motion planning framework for robotic manipulators that computes collision-free paths directly in image space. The generated paths can then be tracked using vision-based control, eliminating the need for an explicit robot model or proprioceptive sensing.
At the core of our approach is the construction of a roadmap entirely in image space. To achieve this, we explicitly define sampling, nearest-neighbor selection, and collision checking based on visual features rather than geometric models. We first collect a set of image-space samples by moving the robot within its workspace, capturing keypoints along its body at different configurations. These samples serve as nodes in the roadmap, which we construct using either learned or predefined distance metrics.
At runtime, the roadmap generates collision-free paths directly in image space, removing the need for a robot model or joint encoders. We validate our approach through an experimental study in which a robotic arm follows planned paths using an adaptive vision-based control scheme to avoid obstacles. The results show that paths generated with the learned-distance roadmap achieved 100\% success in control convergence, whereas the predefined image-space distance roadmap enabled faster transient responses but had a lower success rate in convergence.
 
\end{abstract}

\section{Introduction}

Vision-based control techniques~\cite{hutchinson1996tutorial, hashimoto2003review}, offer significant advantages for robotic manipulators in unstructured and cluttered environments by enabling closed-loop control using task-relevant visual information. These techniques also enhance robustness against model inaccuracies, beneficial for robots with complex or variable dynamics \cite{Calli2016, cuevas2018hybrid, ardon2018reaching}. This strategy is especially useful for robots that are difficult to model accurately, e.g. soft robots \cite{Lai2020, luo2018orisnake}, under-actuated robots \cite{liu2020survey, gandhi2023shape}, 3D printed robots \cite{chavdarov2019design, onal2014origami}, and robots with inexpensive hardware \cite{adzeman2020kinematic}. Furthermore, model-free visual servoing, which learns robot-feature motion models during control, reduces reliance on a priori knowledge of the robot model \cite{wang2018adaptive, navarro2017fourier}.

The goal of our research is to push the boundaries of purely vision-based control and motion planning for robotic manipulators, by decreasing reliance on explicit robot modeling or proprioceptive sensing. In doing so, we strive to use natural visual features along the robot's body in image space, without attaching any external markers. While the works in \cite{gandhi2022skeleton,chatterjee2023keypoints, chatterjee2024utilizing} provided algorithms to track natural keypoints and use them to achieve vision-based model-free control with decent transient responses, these algorithms are only designed to run in obstacle-free space and do not provide motion planning capability to avoid obstacles in the scene. 

\begin{figure}[t]
        \centering
    \includegraphics[width=0.65\columnwidth]{images/fig_1_with_obs_v4.png}
    \caption{
    The robot follows a collision-free path using visual keypoints (colored) planned with the proposed method. 
    The set of keypoints are tracked in order:
    green, magenta, blue, and red. The robot avoids the yellow obstacle while moving between the specified start and goal.  }
    \label{intro-image}
    \vspace{-1em}
\end{figure}

In this work, we tackle the problem of collision free path planning for purely vision-based robot control: we develop a planning and control scheme that does not rely on a robot model or on-board sensors (e.g. joint encoders) in runtime. As such, this strategy is especially useful for (but not limited to) soft/underactuated/inexpensive robots that are hard to model and may not have reliable proprioceptive sensing.

Toward this goal, we propose a novel motion planning formulation that operates only with visual information. We propose a vision-only motion planning methodology using roadmaps \cite{kavraki1996probabilistic}.  We investigated two ways of generating roadmaps using natural features on a robot: 1) directly utilizing the Euclidean distance between keypoints in image space as a distance metric, 2) estimating the joint displacements from image features and utilizing it as a distance metric. For the latter, we first used an automated data collection pipeline to annotate natural keypoints' placement in image space along the robot's body as the robot moves across various configurations.

This dataset was used to develop a simple neural network that approximates joint displacements based on keypoint locations in image space. These distance metrics were integrated into the roadmap construction. Once the roadmaps are constructed, polygon-based collision checking and A* search are employed to ensure collision-free paths. These two approaches have different implications for vision-based robot control. In a nutshell, we observed that utilizing estimated joint distances in the roadmap results in smoother and more accurate tracking of the generated path, allowing the robot to stay in its defined workspace and avoid obstacles. In contrast, roadmaps based on Euclidean distances in image space can offer faster transient responses, albeit with potentially less reliable tracking. Our experiments demonstrate these aspects both in the presence and absence of obstacles.
 
\section{Related Work}

Motion planning is a core problem in robotics that has been extensively studied over the decades. It is generally categorized into three main types: optimization-based \cite{schulman2014motion, zhao2024survey}, sampling-based \cite{orthey2024review, kingston2018sampling}, and search-based planners \cite{likhachev2003ara, cohen2010search}. Recently, Jacobian-based motion planning \cite{park2020trajectory} has also gained traction for obstacle avoidance tasks. While all methods have found widespread success in different applications, all of these rely on having an explicit geometric model of the robot to design feasible paths. In this work we extend the principles of sampling-based planning of the Probabilistic Roadmap (\prm) approach \cite{kavraki1996probabilistic}, to provide a method for motion planning where a robot model is not available. 

Model-free planning, especially without prior knowledge of the robot's geometry, presents unique challenges. Reinforcement learning (RL) has been explored extensively in the recent decades to address such challenges. For instance, \cite{liu2021model} proposed an RL-based framework for generating jerk-free, smooth trajectories. While effective, this method depends on carefully crafted reward functions and extensive datasets generated in simulation, with no real-world validation. Similarly, \cite{zhou2021robotic} introduced a hybrid approach combining RRT*-based trajectories with PPO reinforcement learning for policy refinement. However, this method heavily relies on model-based elements like precomputed trajectories and supervised learning for initial policy design, with experiments confined to simulated environments. In contrast, our approach eliminates reliance on explicit or precomputed models and trajectories by leveraging visual keypoints to construct roadmaps directly in image space, requiring only a small dataset collected from a real robot. This makes our method more adaptable to scenarios without precise geometric models.

Among related works, the approaches proposed in \cite{ichter2019robot} and \cite{556169} are the most closely aligned with ours. In \cite{ichter2019robot}, the authors use sampling-based planning directly from images by leveraging learned forward propagation models, custom distance metrics, and collision checkers. While effective, their method requires extensive simulation-based training, which introduces a significant sim-to-real gap. In contrast, our method uses a limited dataset collected entirely from real robots and constructs a configuration-space roadmap without relying on simulation. Similarly, \cite{556169} combines image features with a robot model to train a neural network for planning a path. Unlike their approach, our method eliminates the need for a robot model, relying solely on image features for motion planning.

Planning a path for image based visual servoing without a-priori knowledge of the robot's model is rarely delved into in the literature. For instance, \cite{mezouar2000path} modeled a path planner for visual servoing to bridge gaps between initial and target positions which are much further apart in configuration space, without addressing collision avoidance. In \cite{lee2011obstacle} authors achieve obstacle avoidance in pose based visual servoing and hence still needed explicit robot model instead of only visual feedback.

\section{Preliminaries and Problem Statement}
In this section we describe the motion planning problem, a brief review of sampling-based methods focusing on probabilistic roadmaps, and finally we introduce the problem statement that we are trying to solve.  

\begin{figure}[t]
      \centering
      \includegraphics[width=0.5\textwidth]{images/config_space_v7.pdf}
      \vspace{-3.85em}
      \caption{The images depict a vision-based motion planning problem. The left image shows the discretized representation of \K  highlighted in gray, where each configuration is sampled as an image frame, resulting in a finite set of \key. The gray region in the right image highlights the same discretized representation of \Kfree, avoiding the obstacle in yellow. \Kfree ensures a collision-free path from the start configuration \keys to the goal configuration \keyg}
      \label{config_space}
\end{figure}

\subsection{Model-based Motion Planning}
\label{ssec:prob-state}

Let $\x \in \X$ denote the state and state-space, and $ \uu \in \U$ the control and control-space of a robot system \cite{orthey2024review}.
The system dynamics equations $f$ of the robot system can be written as 
\begin{equation} \label{eq:motion}
\x(\T) = \x(0)+\int_0^{\T} f(x(t),u(t))dt \end{equation}
If the system dynamics impose constraints on the allowable paths the system is called a non-holonomic system.  Now, Let $\Co \subset \C $ denote the invalid state space, which is the set of states that violates the robots kinematic constraints or collides with the workspace $\mathcal{W} \subseteq \mathbb{R}^3$. e.g.,
\begin{equation}
\Co =\{\x \in \C | R(\x) \bigcap \mathcal{W} \neq \emptyset \}
\end{equation}
where $R(\x) \subseteq \mathbb{R}^3$ is the set of all the points of the robot in the workspace at state \x \footnote{This equation only encodes collisions but joint/kinematic constraints can be encoded in a similar manner}. Then let, $\Cf = \C \setminus \Co$ denote the collision free state-space. Also, let $x_\start \in \Cf $ and \mbox{$X_\goal \subseteq \Cf$} denote the start state and goal regions respectively.

\textit{Motion Planning Problem:}
Given the motion planning tuple $(\X, R, \mathcal{W}, \U, f,  x_\start, \Xg)$ find a time $\T$ and a set of controls $u: [0,\T] \rightarrow \U$ such that the motion described by \autoref{eq:motion} satisfies $ x(0) = \xs $, $ \x(\T) \in \Xg$ and $x(t) \in \Xfree$.   

\subsection{Vision-Only Motion Planning} 
\label{ssec:prob-state-vis}

The problem we are considering in this setting is departing from the above motion planning problem as the geometric model $R(x)$, the dynamics $f$ and the workspace $\mathcal{W}$ is not directly available. Instead the only information available is the space of admissible controls \U and an image $\im \in \IM$.

We will describe the new problem statement by explicitly defining equivalent concepts of obstacles, robot states, and state dynamics directly in the image space.
We assume that we are given a point tracking function that maps a given image to $N$ fixed pixel points on the robot's body.  
This vector of pixel points is denoted as an image state $\key \in \K $.
Where $\K \subset \mathbb{R}^{N \times 2}$ denotes the space of all image states.
In \autoref{intro-image} we can see $4$ different robot configurations in image space. Each configuration is represented by a set of $5$ same-colored circles. Each of these circles on the robot's body in the image is a pixel point and is denoted as a keypoint or $\skey$. The kinematic chain or shape formed by the vector of $5$ circles represents one image state \key.

Given a set of image obstacles $\mathcal{IO} \subset \im$ we define: 
\begin{equation}
\Kobs =\{\key \in \K | RI(\key) \bigcap \mathcal{IO} \neq \emptyset \}
\end{equation}
where $RI(\key) \subseteq \im $ is the set of pixels that the robot occupies at image state \key . Similarly to before, we define $\Kfree  = \K \setminus \Kobs$, $\key_\start$ and $\K_g$. 

We define the unknown system dynamics function as:     
\begin{equation} \label{eq:motion_image}
\key(\T) = \key(0)+\int_0^{\T} g(\key(t),u(t))dt
\end{equation}

Here we note that even if the underlying function f is fully-integrable (holonomic-system) if $dim(\K)> dim(\X)$ the equivalent system dynamics equations $g$ will be non-holonomic as there will be paths in the higher dimensional image-space \K that cannot be followed by the robot. In our approach we consider this issue, and propose a way to produce paths that approximately satisfy these constraints. In the first image of \autoref{config_space}, the grey region illustrates the discretized representation of \K, as each configuration is captured as an image frame, resulting in a finite set of \key sampled at a specific frame rate. The second image of \autoref{config_space} highlights similar discretized representation of \Kfree.

 Given the above definitions, let us define the vision-only motion planning problem. 

\textit{Vision-Only Motion Planning Problem:}
Given the tuple $(\K, RI, \U, \mathcal{IO}, \im, \key_{\start}, \K_g)$ find a time $\T$ and a set of controls $u: [0,\T] \rightarrow \U$ such that the motion described by $g$ satisfies $\key(0) = \key_{\start}$, $ \key(\T) \in \K_g$ and $\key(t) \in \Kfree$.  
%If we had the model of the robot, then we can compute all these different functions and succesfully find a path from start to goal that avoids the obstacles. However since we don't have the model of the robot, we would need to find ways to compute the  and distance for nearest neighbor structure.  
\section{Methodology}
\label{ssec:methodology}

Since the system dynamics equation is unknown, we cannot directly plan in the control space \U. Instead, we will plan a path $\key_0, \key_1 \ldots, \key_n$ directly in the \Kspace and then control the robot to follow the path with a vision-only controller\cite{gandhi2022skeleton}.

To compute the path, which is the main contribution of this work, we propose to use probabilistic road-map planner (\prm) by adapting its subroutines to operate directly in \K-space. Specifically, we opted to adapt the \lprm~\cite{bohlin2000path} planner due to its compatibility with our particular requirements. However, any sampling-based planner which relies on the the same subroutines could be used. 

\begin{algorithm}[H]
   \caption{Build-Lazy-PRM} 
   \label{alg:build-lazy-prm}
    \begin{algorithmic}[1] 
     \Procedure{Build-Lazy-PRM}{N, k}  
        \State {$G$} $\gets$ INIT()   
        \While{$G$.size()$\leq$ N}%$i = 1 ,\ldots, N$} 
          %\State \Comment{generate collision free sample}
           \State $\key_{new} \gets $ {\color{red}{\textsc{sample}}}($\K$) \label{sample} 
           \State $G$.addNode($\key_{new}$) \label{roadmap}
        \EndWhile
       \For{each $\key \in  G$.nodes()} 
           \State $\mathcal{N}(\key) \gets $ {\color{red}{\textsc{K-nearest}}}($\key$, {$G$}) \label{dist} 
           \For{each $\key_{near} \in \mathcal{N}(\key_{new})$} 
            \State $e \gets$ ($\key, \key_{near})$
            \If{$e \notin G$.edges()} \label{collision}
                \State  $G$.addEdge($e$) 
            \EndIf 
          \EndFor 
        \EndFor 
    	\State \Return $G$ 
    \EndProcedure
    \end{algorithmic}
    \end{algorithm}
    \vspace{-0em}

Similar to \prm, \lprm operates in two distinct phases, the building-phase \autoref{alg:build-lazy-prm} and the query-phase \autoref{alg:query-lazy-prm}. In the building-phase, a roadmap is built  without performing any collision checking. In the query-phase, the roadmap is utilized to find a potential path for a given start and goal. The key difference between \lprm and \prm lies in their approach to collision checking. While \prm verifies the collision status of all edges in the roadmap during the building phase, \lprm reverses the order. It first performs the search query and only checks collisions for the found path. If any edges are found to be in collision, the roadmap is updated, and another search query is executed until a valid path is discovered.

We selected probabilistic roadmap method for its efficiency in multi-query scenarios, where a single precomputed graph can handle multiple start and goal configurations, and used the lazy version, as we only have very few varying obstacles between queries. 
Additionally, its ability to operate in high-dimensional spaces makes it particularly well-suited for handling non-traditional representations of configuration space, such as the visual keypoints in \K-space used in our approach. We first describe how \lprm works and then we describe how we modified the necessary subroutines to make it work in \Kspace. 

The building-phase (\autoref{alg:build-lazy-prm}) works with the following procedure. First, in line \ref{sample}, a sample is generated and added in graph $G$. Then the k nearest nearest neighbors are found by using a distance defined in \K (line \ref{dist}) and are connected with edges. This continues until $N$ nodes are in the graph $G$. 


\begin{algorithm}[H]
   \caption{Query-Lazy-PRM} 
   \label{alg:query-lazy-prm}
    \begin{algorithmic}[1] 
     \Procedure{Lazy-Query-PRM}{\keys, \keyg, $G$}   
       \State G.add(\keys) \Comment{Add start, and goal to the Graph}
       \State G.add(\keyg)
       \State Edges, $\gets$ \textsc{Search-Graph} $G$(\keyg, \keys)
       \For{each $e \in  G$.edges()} 
           \If{{\color{red}{\textsc{coll\_free}}}($e$)} \label{collision}
           \For{each $\key_{near} \in \mathcal{N}(x_{new})$} 
            \State $e \gets$ ($\key, \key_{near})$
            \If{{\color{red}{\textsc{coll\_free}}}($e$) and $e \notin G$.edges()} \label{collision}
                \State  $G$.addEdge($e$) 
            \EndIf 
          \EndFor 
    	\State \Return $Path$ 
        \EndIf 
     \EndFor 
    \EndProcedure
    \end{algorithmic}
    \end{algorithm}
    \vspace{-0.75em}



During the query-phase (\autoref{alg:query-lazy-prm} a new motion planning problem is solved. Given a $\key_\start$ and $\key_\goal $ they are added in the  graph, and connected with their nearest-neighbors. Then a graph search algorithm e.g., A* is used to find a path. If edges of the path are in-collision they are updated accordingly in the roadmap, and the process repeats until a collision free-path is found. 
The three operations \textsc{sample},  \textsc{k-nearest},  \textsc{coll\_free},
for \autoref{alg:build-lazy-prm} in lines \ref{sample}, \ref{dist} and \autoref{alg:query-lazy-prm}  typically require a model for the robot.

\begin{figure*}[ht!]
      \centering
      \includegraphics[width=0.8\textwidth]{images/graph_flow_v8.pdf}
          \caption{Overview of the roadmap creation process for motion planning. In this image the purple circles are nodes in image space represented by image state in each frame. Using a distance metric, we connect the nodes to create a roadmap. As observed two different metrics produce different edges on the graph. After an A* search for path finding between a pair of start (\keys) and goal (\keyg) configurations and collision check if required, paths are found for both roadmaps. The green lines denote the final path after collision check. As observed, the \textbf{learned} distance roadmap has a clearly more optimized path than the \textbf{image space} distance roadmap}
      \label{graph}
      \vspace{-1.4em}
\end{figure*}


\textsc{sample}: This function typically samples the configuration space uniformly. However, in the vision-only setting, we can't directly sample in \Kspace as we don't have the model of the robot. Since $dim(\K) > dim(\X)$ the keypoint vectors (image state \key) that correspond to actual configurations of the robot, will lie in a lower-dimensional (equal to $dim(\X)$) manifold in \Kspace. Thus 
randomly sampling \Kspace will have $0$ probability of sampling a valid configuration that lies on the valid manifold \cite{kingston2018sampling}. We describe how to address this issues in \autoref{ssec:dataset}, by collecting and storing valid samples directly from the real robot.  

\textsc{k-nearest}: This function usually relies on a distance defined in \C  and finds the nearest configurations that can be connected. However, since our representation in \Kspace is now a non-holonomic system, defining this distance is very challenging, as a straight line path in \Kspace defined by a simple Euclidean distance, might not be accurately followable by a controller. To mitigate this, we describe a learning-based approach to estimate the unknown joint-distance, in \autoref{ssec:dist_metric}.


\textsc{coll\_free}: This function checks if there is a collision for an edge in \C. Again this typically requires the model $R(x)$ for the robot. In the vision only case, we propose simple yet successful method to do collision checking with an $RI(x)$ directly in \Kspace \autoref{ssec:line-check}. 

In the next section we describe our proposed method, for each, of the aforementioned subroutines. \autoref{graph} visually describes all the steps of the visual motion planning framework.

\subsection{\textsc{sample}: Executed Trajectories as Proxy Samples} 
\label{ssec:dataset}
In the absence of a robot model, we represent each image state $\key \in \K$ by identifying and annotating keypoints (\skey)
 on the robotic arm using an automated data collection pipeline described in \cite{chatterjee2023keypoints, chatterjee2024utilizing}. Each \ensuremath{\key_n} is composed of a set of $N$ keypoints \skey, where each \skey represents a pixel point of a specific location on the robot's body in image space. For instance, if \( N = 5 \), a keypoint vector or image state \ensuremath{\key_n} is represented by a vector of {\ensuremath{\skey_1}, \ensuremath{\skey_2}, \ensuremath{\skey_3}, \ensuremath{\skey_4}, \ensuremath{\skey_5}}, where each \skey is a pixel point in the $n\textsuperscript{th}$ image frame. An example vector of such keypoints is shown in \autoref{config_space} as \keys or \keyg

 To systematically explore the robot's visible workspace in the image space and ensure comprehensive coverage, we compute velocities for each joint \( j \) using:

\begin{equation}
\label{comp_vel}
v_j = \min\left(\frac{M_j}{\ensuremath{res} \cdot t}, v_{\text{max}}\right),
\end{equation}

where, \( M_j \) is the motion range or difference of limits of joint \( j \), \ensuremath{res}, is the number of discrete  steps used to traverse \( M_j \), \( t \) is the duration allocated to complete each step, and \( v_{\text{max}} \) is the maximum allowable velocity for joint \( j \).
By dividing the motion range of each joint into uniform increments based on \ensuremath{res}, we ensure that the robot systematically explores all possible configurations within its workspace. The resulting dataset of image states \key is thus evenly distributed across the image space \K, enabling robust coverage and accurate representation of the robot's motion capabilities. 

Each configuration \ensuremath{\key_n} is computed using the following transformation:

\begin{equation}
\label{2d_eq}
\ensuremath{\key_n} = K \cdot T_{cw} \cdot \ensuremath{x_n}
\end{equation}
where $K$ is the camera intrinsic matrix, $T_{cw}$ is the camera extrinsics matrix derived from calibration processes described in \cite{Lee2020, Zhang2000},  and \ensuremath{x_n} is the 3D configuration robot in workspace. 
This transformation projects the 3D configuration \ensuremath{x_n}, into their corresponding 2D (pixel) projection in the image described in \cite{3dRecon}, resulting in a set of \skey for each image state \ensuremath{\key_n} The process captures the robot's motion across its visible workspace and creates a comprehensive dataset of \key, representing close to all feasible configurations in image space. Dataset of \ensuremath{\key} can also be collected by following the process describe in \cite{chatterjee2024utilizing}. 

The \textbf{Keypoint Dataset Samples} section of \autoref{graph}, represents this part of the workflow. The grey area in the left image of \autoref{config_space} illustrates how \key are distributed within image space with each frame consisting of a set of \skey or keypoints. This collection process yields a large dataset of \key that can be used to enable efficient roadmap construction for vision-based motion planning.

\subsection{\textsc{k-nearest}: Learned and Image-Space Metrics}
\label{ssec:dist_metric}
To search for the K-nearest neighbors of each image state sample we employ the following two distance metrics:

\begin{itemize}
\item \textit{Learned distance},  where the distance is learned by a neural network, trained to predict joint displacements between two image states. The input to the network is a pair of image states $\key_1$ and $\key_2$ and and the output is an estimated joint displacement required to transition between them:
\begin{equation}
\label{cust-eq}
\text{dist}_{learned}() \leftarrow \text{NN}(\key_1, \key_2)
\end{equation}
\item \textit{Distance in Image space}, where we simply calculate the Euclidean distance between $\key_1$ and $\key_2$ in image space:
\begin{equation}
\label{euc-eq}
\text{dist}_{image}() \leftarrow || \key_1 - \key_2 ||_2
\end{equation}
\end{itemize}
Each metric influences the graph's structure by determining the nearest neighbors and defining the edges in the roadmap. The learned distance prioritizes image states with minimal estimated joint displacement, while the image space distance favors image states that are closer in the image. These differences impact the connectivity of the roadmap, as illustrated in the \textbf{Connected Roadmap} and \textbf{Connect Start and Goal} sections of \autoref{graph}.

\subsubsection{Dataset generation for network model}
\label{ssec:approx-joint}
While collecting the dataset of image states \key in \autoref{ssec:dataset}, we recorded the velocity (\autoref{comp_vel}) applied to transition between consecutive frames. Using this data, we created a new dataset that includes pairs of consecutive image states (\ensuremath{\key_1}, \ensuremath{\key_2}) and their estimated joint displacements. This is calculated by multiplying the recorded velocity with the frame rate at which \key was captured, as described in \autoref{2d_eq}.

 Please note here, only pairs of \key captured in consecutive image frames are included in this data generation process.  However, for constructing the graph $G$, it is essential to compute distances between arbitrary \key pairs in \K-space. To achieve this, a neural network is employed to predict the joint displacement across different pairs of image-states \key.
 
To enhance diversity, the dataset is augmented by combining frame sequences where the first frame's image state (\(\key_{\text{start}}\)) and the last frame's image state (\(\key_{\text{end}}\)) act as boundaries, with total joint displacements calculated as sum of displacements across intermediate frames. This approach ensures diverse transitions, enabling the neural network to accurately estimate joint displacements for any image state pair.

\subsubsection{Neural Network for Learned Distance Metric}
\label{ssec:reg-model}
To derive \autoref{cust-eq}, we design a simple neural network using the aforementioned dataset to learn a distance more similar to the joint space distance. The model takes concatenated arrays of the starting image state (\(\key_{\text{start}}\)) and the subsequent image state (\(\key_{\text{next}}\)) as input and predicts the estimated joint displacement.

\subsection{\textsc{coll\_free}:Image-Based Collision Checking}
\label{ssec:line-check} 

To check for collisions in our model-free system, we employ an image-based polygon collision-checking framework. Polygons are formed by connecting pairs of consecutive keypoints ((\(\skey_{n-1}\)), (\(\skey_{n}\))) in one image state (\(\key_1\)) with their corresponding keypoints in the neighboring image state (\(\key_2\)). In \autoref{line-seg}, the polygons created by (\(\skey_1\), \(\skey_2\)),  (\(\skey_2\), \(\skey_3\)), (\(\skey_3\), \(\skey_4\)) and (\(\skey_4\), \(\skey_5\)) of the start image state (\(\key_1\)) in green and its neighboring image state (\(\key_2\)) in red are examples of such polygons. 

Each polygon is then checked for intersections with the obstacle boundaries defined by a safety margin which we consider to account for controller uncertainty. This polygon-based approach ensures a thorough and robust collision-checking process, effectively handling obstacles of any size. In \autoref{line-seg} illustrates an example where the obstacle, enclosed by the red safety margin, intersects with the polygons defined by ([\(\skey_2\), \(\skey_3\)],[\(\skey_3\), \(\skey_4\)] and [\(\skey_4\), \(\skey_5\)]), demonstrating the effectiveness of this method.
The collision checking process and \textbf{A*} search is shown in  \textbf{A* Search with Lazy Collision Checking} of \autoref{graph}.  

\begin{figure}[t]
      \centering
      \includegraphics[width=0.4\textwidth]{images/poly_check_comb_paper.pdf}
      % \vspace{-3.0em}
      \caption{Illustration of polygon-based collision checking for vision-only motion planning. In this image $4$ polygons (bordered with light blue) are defined by connecting consecutive keypoint pairs (\(\skey_1\), \(\skey_2\)),  (\(\skey_2\), \(\skey_3\)), (\(\skey_3\), \(\skey_4\)) and (\(\skey_4\), \(\skey_5\)) of the image state (\(\key_1\)) in green and its neighboring image state (\(\key_2\)) in red. The yellow obstacle enclosed by the purple safety margin is situated right on the polygons formed by ([\(\skey_2\), \(\skey_3\)],[\(\skey_3\), \(\skey_4\)] and [\(\skey_4\), \(\skey_5\)]), rendering these two configurations ineligible for a collision-free path.}
      \vspace{-0.5em}
      \label{line-seg}
\end{figure}

\subsection{Adaptive Visual Servoing}
This work builds on the adaptive visual servoing method described in \cite{gandhi2022skeleton}, using a roadmap of collision-free sequence of goal image states. At each goal, vector of keypoints (\skey) in image state (\key), is tracked as visual features as described in \cite{chatterjee2023keypoints}. The controller moves the arm minimizing the feature error, computed as the difference between the current and the target \key. The Jacobian matrix, estimated online via least-square optimization of recent joint velocities and keypoints vector over a moving window, eliminates the need to read joint position from  encoder. This makes the control pipeline completely model-free. To improve accuracy, we reset the Jacobian estimation window at each new target keeping the estimate unbiased and relevant to the current goal. Since the goals may not be evenly spaced in the image for different experiments, we use a saturation limit on the feature error to prevent sudden spikes in velocity ensuring stable motion and protecting the motor from damage.

\begin{figure}[t]
      \centering
      \includegraphics[width=0.9\columnwidth]{images/histogram_slide.pdf}
      \caption{Histograms of actual joint displacements along the edges for the three roadmaps. The \textbf{Learned} roadmap closely aligns with the distribution of the Joint Space roadmap, showing only slight deviations and indicating reasonable accuracy in the predicted joint displacements. In contrast, the \textbf{Image space} roadmap demonstrates a wider spread and larger joint displacements, reflecting its lack of alignment between image-space proximity and joint-space movements. This misalignment may reduce its efficiency in generating paths suitable for precise control.}
      \vspace{-1.5em}
      \label{histogram}
\end{figure}

\section{Experiments and Observations:}
We experimentally assesed the performance of the proposed vision only motion planning framework on a Franka Emika Panda Arm.

\subsection{Experimental details}

\subsubsection{Generating samples} We generated the required image state samples by collecting a large dataset created by actuating the planar joints (Joints $2,4,6$) of the Franka arm, using velocities computed from \autoref{comp_vel} as explained in \autoref{ssec:dataset}, covering the robot's planar workspace.

\subsubsection{ Generating the roadmap} The collected samples \key were used as nodes to generate roadmaps using the different proposed distances described in \autoref{ssec:dist_metric}. The roadmap generated by the image-space distance is coined as \textbf{Image Space} roadmap and the one generated using the learned distance is coined as \textbf{Learned} roadmap. For benchmarking purposes we also use the actual joint distance to create the Joint Space roadmap. The Joint Space roadmap used actual joint displacements from encoders solely for benchmarking, maintaining the proposed model-free framework. The k-neighbor value for all approaches was set to $25$.
 
\subsubsection{Obstacle Representation and Path Planning} In the experimental setup, obstacles were modeled as virtual yellow rectangles with a safety margin, as shown in \autoref{line-seg}. Paths were generated offline for various start (\keys) and goal (\keyg) incorporating the obstacle avoidance logic from \autoref{ssec:line-check}. These paths were later used in adaptive visual servoing experiments

\subsubsection{Control Experiment Set-up} 
 The real-time control experiments used an Intel Realsense D435i camera in an eye-to-hand setup for visual feedback. The Panda arm followed the planned paths in $16$ obstacle-free and $10$ obstacle-avoidance experiments. The performance of each proposed roadmap was evaluated by comparing the joint position changes between intermediate image states to those of the Joint Space roadmap. The controller’s ability to guide the arm along collision-free paths was evaluated for efficiency and effectiveness.

\subsection{Roadmap and Path Planning Experiments}
In this section, we analyze the joint displacements along the edges of the three roadmaps to evaluate their efficiency. Path planning experiments were conducted in both collision-free and obstacle-avoidance scenarios to compare the joint distances covered by the paths generated from each roadmap.

\subsubsection{Distribution of Joint Distances of Edges for Different Roadmaps}
\autoref{histogram} shows the histograms of joint displacements along the edges of the three roadmaps. The \textbf{Learned} roadmap closely aligns with the joint space roadmap with minor deviations, demonstrating reasonable accuracy in estimated joint displacements. In contrast, the \textbf{Image Space} roadmap exhibits a broader spread and larger joint displacements. This suggests that the Learned roadmap is more effective in accurately capturing transitions and generating paths that minimize joint movements.

\subsubsection{Average Joint Distances for Planned Collision-Free Paths}
\label{random-rm}
We randomly selected $100$ pairs of \keys and \keyg from the roadmaps for path planning without obstacles. The average joint distances over $100$ trials, summarized in \autoref{random_trials}, show that the \textbf{learned} roadmap yields results closer to the joint space while the \textbf{image space} roadmap results in significantly larger joint displacements. These findings suggest that the learned roadmap possibly generates more efficient path, better suited for successful control convergence.  

\begin{table}[t]
\caption{Average Joint Distances (Radians) for Collision-Free Paths}
\label{random_trials}
\centering
\resizebox{0.5\textwidth}{!}{%
\begin{tabular}{|m{2cm}||m{2cm}||m{2cm}||m{2cm}||}
\hline
& \multicolumn{1}{m{1cm}||}{\centering \textbf{Joint Space}} & \multicolumn{1}{m{1cm}||}{\centering \textbf{Learned}} & \multicolumn{1}{m{1cm}|}{\centering \textbf{Image} \\ \centering \textbf{Space}} \\
\hline
\hline
\multicolumn{1}{|m{2cm}||}{\centering \textbf{Mean} \\ \centering \textbf{(radians)}}  & \multicolumn{1}{m{1cm}||} {\centering $\textbf{1.74}$} & \multicolumn{1}{m{1cm}||}{\centering $\textbf{2.19}$} & \multicolumn{1}{m{1cm}|}{\centering $\textbf{3.06}$} \\
\hline
\end{tabular}
} 
\vspace{-0.75em}
\end{table}

\subsubsection{Planned paths for Control Experiments}

\label{ssec:path_planning}
We generated planned paths for two scenarios: $16$ start and goal pairs without checking for collision and $10$ pairs with collision avoidance. Paths were computed using the three roadmaps: Joint Space, Learned, and Image Space. These precomputed paths were used in the control experiments described in \autoref{all_control_exps}. 

For each planned path, we calculated metrics including the average number of waypoints (intermediate image states), joint distances between waypoints, total joint distances for the entire path, and Euclidean distances between image states (keypoint distances) both between waypoints and across the entire path. 

As observed in \autoref{exps_free_and_obs} the \textbf{learned} roadmap consistently resulted in fewer waypoints and shorter joint distances compared to the \textbf{image space} roadmap, which prioritizes minimizing keypoints distances in image space but incurs higher joint displacements.

Notably, the joint distances required to traverse $1000$ pixels in image space were much higher for the Image space roadmap than for the Learned roadmap. This suggests that reliance on image space proximity may lead to less efficient joint-space paths.

\begin{table}[t]
\caption{Comparison of Joint Distances and Keypoints Distance in Image Space over Experiments}
\label{exps_free_and_obs}
\centering
\resizebox{0.45\textwidth}{!}{%
\begin{tabular}{|m{2cm}||m{1cm}|m{1cm}|m{1cm}||m{1cm}|m{1cm}|m{1cm}|}
\hline
& \multicolumn{3}{c||}{\textbf{Without Collision-check}} & \multicolumn{3}{c|}{\textbf{With Collision-check}} \\
\hline
\textbf{Roadmaps} & \textbf{Joint Space} & \textbf{Learned} & \textbf{Image Space} & \textbf{Joint Space} & \textbf{Learned} & \textbf{Image Space} \\
\hline
\textbf{Number of Experiments} & \textbf{16} & \textbf{16} & \textbf{16} & \textbf{10} & \textbf{10} & \textbf{10} \\
\hline
\textbf{Avg No. of Waypoints} & \textbf{8} & \textbf{8} & \textbf{14} & \textbf{11} & \textbf{13} & \textbf{15} \\
\hline
\textbf{Avg. Joint Distances (radians) b/w Waypoints} & \textbf{0.25} & \textbf{0.3} & \textbf{0.32} & \textbf{0.28} & \textbf{0.28} & \textbf{0.35} \\
\hline
\textbf{Avg. Keypoints Distances (pixels) b/w Waypoints} & \textbf{167.99} & \textbf{174.76} & \textbf{84.11} & \textbf{139.53} & \textbf{129.13} & \textbf{89.36} \\
\hline
\textbf{Avg. Joint Distances (radians) over Entire Path} & \textbf{1.92} & \textbf{2.27} & \textbf{4.29} & \textbf{3.04} & \textbf{3.46} & \textbf{5.36} \\
\hline
\textbf{Avg. Keypoints Distances (pixels) over Entire Path} & \textbf{1222.42} & \textbf{1278.92} & \textbf{1157.00} & \textbf{1532.72} & \textbf{1569.37} & \textbf{1361.43} \\
\hline
\textbf{Joint Distance (radians) Traversed To Move 1000 pixels in image space} & \textbf{1.6} & \textbf{1.77} & \textbf{3.79} & \textbf{1.98} & \textbf{2.19} & \textbf{3.9} \\
\hline
\end{tabular}
} 
\vspace{-1.85em}
\end{table}
We have two theories from the above observations:

\begin{itemize}
    \item The larger number of waypoints in the image space roadmap arises from its Euclidean distance-based edge weights, causing A* to prioritize shorter pixel distances and select more intermediate nodes. In contrast, the learned roadmap, with joint displacement-based weights, produces more direct paths with fewer waypoints.
    \item The image states \key which are close in image space may be much further away in joint space as highlighted in \autoref{exps_free_and_obs}. This behavior may lead to increased overall joint movement where non-holonomy may exist in the joint space for the covered joint displacements. This may reduce control accuracy and hinder convergence to the target image state.
\end{itemize}


\subsection{Control Experiments}
\label{all_control_exps}
The adaptive visual servoing experiments\footnote{Planning and Control Experiments videos are available at \href{https://drive.google.com/file/d/1eOoP0dVFz85q4usiLzjlPdA5PTBA3UKU/view?usp=drive_link}{this link}. The details of how to use the link is in the supplementary ReadMe file} used precomputed paths from  \autoref{ssec:path_planning}, with control gains optimized to minimize rise and settling times while keeping overshoot within 5\%, by careful tuning.

In \autoref{performance_data}, the overall control metrics highlight that the image space roadmap succeeded in only $69.2$\% of cases, while the learned roadmap achieved a $100$\% success rate. However, the image space roadmap, when successful, showed faster transient responses compared to the Learned roadmap. \autoref{exps_success_failure} uses identical metrics as in \autoref{exps_free_and_obs}, categorizing experiments into successful and failed cases

Failures in the image space roadmap were characterized by large joint displacements relative to smaller image space distances, as noted earlier in \autoref{ssec:path_planning}. Optimal execution of the references generated by using image space distances requires all the keypoints to follow straight paths. This is not possible for the robot due to the non-holonomic constraints of its kinematics (if defined directly in image space). As a result, the robot deviates from the planned path. While the robot reaches the reference locations in most cases, the deviations cause it to go out of its workspace (due to joint limits) in some others. This effect is especially visible when obstacles exist in the workspace since the robot needs to travel near the edges of its workspace to avoid them. When estimated joint distances are used the resulting trajectories are more suitable to the robot's kinematics, which prevents such failures at large. 


\begin{table}[t]
\caption{Performance results for the control experiments with and without obstacle}
\label{performance_data}
\centering
\resizebox{0.45\textwidth}{!}{%
\begin{tabular}{|m{1cm}||m{1cm}||m{1cm}||m{1cm}||m{1cm}||m{1cm}||m{1cm}|}
\hline
& \multicolumn{3}{c|||}{\textbf{Without Collision-check}} & \multicolumn{3}{c|}{\textbf{With Collision-check}} \\
\hline
 \multicolumn{1}{|m{2cm}||}{\centering \textbf{Performance} \\ \centering \textbf{Metrics}} & \multicolumn{1}{m{1cm}||}{\centering \textbf{Joint} \\ \centering \textbf{Space}} & \multicolumn{1}{m{1cm}||}{\centering \textbf{Learned}} & \multicolumn{1}{m{1cm}|||}{\centering \textbf{Image} \\ \centering \textbf{Space}} & \multicolumn{1}{m{1cm}||}{\centering \textbf{Joint} \\ \textbf{Space}} & \multicolumn{1}{m{1cm}||}{\centering \textbf{Learned}} & \multicolumn{1}{m{1cm}|}{\centering \textbf{Image} \\ \centering \textbf{Space}} \\
\hline
\hline
\multicolumn{1}{|m{2cm}||}{\centering \textbf{Successful} \\ \textbf{Experiments}} & \multicolumn{1}{m{1cm}||} {\centering \textbf{16/16}} & \multicolumn{1}{m{1cm}||}{\centering \textbf{16/16}} & \multicolumn{1}{m{1cm}|||}{\centering \textbf{13/16}} & \multicolumn{1}{m{1cm}||}{\centering \textbf{10/10}} & \multicolumn{1}{m{1cm}||}{\centering \textbf{10/10}} & \multicolumn{1}{m{1cm}|}{\centering \textbf{5/10}} \\
\hline
\hline
\multicolumn{1}{|m{2cm}||}{\centering \textbf{System} \\ \textbf{Rise time (s)}} & \multicolumn{1}{m{1cm}||}{\centering \textbf{74.92}} & \multicolumn{1}{m{1cm}||}{\centering \textbf{97.53}} & \multicolumn{1}{m{1cm}|||}{\centering \textbf{94.75}} & \multicolumn{1}{m{1cm}||}{\centering \textbf{105.38}} & \multicolumn{1}{m{1cm}||}{\centering \textbf{125.38}} & \multicolumn{1}{m{1cm}|}{\centering \textbf{90.46}} \\
\hline
\multicolumn{1}{|m{2cm}||}{\centering \textbf{System} \\ \textbf{Settling time (s)}} & \multicolumn{1}{m{1cm}||}{\centering \textbf{94.37}} & \multicolumn{1}{m{1cm}||}{\centering \textbf{118.62}} & \multicolumn{1}{m{1cm}|||}{\centering \textbf{101.99}} & \multicolumn{1}{m{1cm}||}{\centering \textbf{118.18}} & \multicolumn{1}{m{1cm}||}{\centering \textbf{155.14}} & \multicolumn{1}{m{1cm}|}{\centering \textbf{96.44}} \\
\hline
\multicolumn{1}{|m{2cm}||}{\centering \textbf{End effector} \\ \textbf{Rise time (s)}} & \multicolumn{1}{m{1cm}||}{\centering \textbf{74.86}} & \multicolumn{1}{m{1cm}||}{\centering \textbf{97.53}} & \multicolumn{1}{m{1cm}|||}{\centering \textbf{94.39}} & \multicolumn{1}{m{1cm}||}{\centering \textbf{105.02}} & \multicolumn{1}{m{1cm}||}{\centering \textbf{125.38}} & \multicolumn{1}{m{1cm}|}{\centering \textbf{90.46}} \\
\hline
\multicolumn{1}{|m{2cm}||}{\centering \textbf{End effector} \\ \textbf{Settling time (s)}} & \multicolumn{1}{m{1cm}||}{\centering \textbf{94.37}} & \multicolumn{1}{m{1cm}||}{\centering \textbf{114.02}} & \multicolumn{1}{m{1cm}|||}{\centering \textbf{99.81}} & \multicolumn{1}{m{1cm}||}{\centering \textbf{118.18}} & \multicolumn{1}{m{1cm}||}{\centering \textbf{147.14}} & \multicolumn{1}{m{1cm}|}{\centering \textbf{94.22}} \\
\hline
\multicolumn{1}{|m{2cm}||}{\centering \textbf{Overshoot (\%)}} & \multicolumn{1}{m{1cm}||}{\centering \textbf{1.94}} & \multicolumn{1}{m{1cm}||}{\centering \textbf{2.61}} & \multicolumn{1}{m{1cm}|||}{\centering \textbf{1.89}} & \multicolumn{1}{m{1cm}||}{\centering \textbf{1.93}} & \multicolumn{1}{m{1cm}||}{\centering \textbf{2.36}} & \multicolumn{1}{m{1cm}|}{\centering \textbf{1.35}} \\
\hline
\multicolumn{1}{|m{2cm}||}{\centering \textbf{Execution} \\ \textbf{time (s)}} & \multicolumn{1}{m{1cm}||}{\centering \textbf{125.17}} & \multicolumn{1}{m{1cm}||}{\centering \textbf{148.92}} & \multicolumn{1}{m{1cm}|||}{\centering \textbf{140.78}} & \multicolumn{1}{m{1cm}||}{\centering \textbf{146.02}} & \multicolumn{1}{m{1cm}||}{\centering \textbf{201.68}} & \multicolumn{1}{m{1cm}|}{\centering \textbf{124.70}} \\
\hline
\end{tabular}
} 
\end{table}

\begin{table}[t]
\caption{Comparison of Joint Distances and Keypoints Distance in Image Space over Experiments for Successful and Failed Experiments}
\label{exps_success_failure}
\centering
\resizebox{0.45\textwidth}{!}{%
\begin{tabular}{|m{2cm}||m{1cm}|m{1cm}|m{1cm}||m{1cm}|m{1cm}|m{1cm}|}
\hline
& \multicolumn{3}{c||}{\textbf{Successful}} & \multicolumn{3}{c|}{\textbf{Failed (Image Space)}} \\
\hline
\textbf{Roadmaps} & \textbf{Joint Space} & \textbf{Learned} & \textbf{Image Space} & \textbf{Joint Space} & \textbf{Learned} & \textbf{Image Space} \\
\hline
\textbf{Number of Experiments} & \textbf{18} & \textbf{18} & \textbf{18} & \textbf{8} & \textbf{8} & \textbf{8} \\
\hline
\textbf{Avg No. of Waypoints} & \textbf{8} & \textbf{8} & \textbf{14} & \textbf{11} & \textbf{13} & \textbf{15} \\
\hline
\textbf{Avg. Joint Distances (radians) b/w Waypoints} & \textbf{0.25} & \textbf{0.3} & \textbf{0.32} & \textbf{0.27} & \textbf{0.29} & \textbf{0.35} \\
\hline
\textbf{Avg. Keypoints Distances (pixels) b/w Waypoints} & \textbf{166.77} & \textbf{163.41} & \textbf{86.03} & \textbf{135.17} & \textbf{143.24} & \textbf{86.35} \\
\hline
\textbf{Avg. Joint Distances (radians) over Entire Path} & \textbf{2.10} & \textbf{2.37} & \textbf{4.42} & \textbf{2.92} & \textbf{3.52} & \textbf{5.33} \\
\hline
\textbf{Avg. Keypoints Distances (pixels) over Entire Path} & \textbf{1307.02} & \textbf{1257.02} & \textbf{1210.93} & \textbf{1419.93} & \textbf{1691.26} & \textbf{1291.2} \\
\hline
\textbf{Joint Distance (radians) Traversed To Move 1000 pixels in image space} & \textbf{1.6} & \textbf{1.9} & \textbf{3.71} & \textbf{2.1} & \textbf{2.1} & \textbf{4.12} \\
\hline
\end{tabular}
} 
\vspace{-1.05em}

\end{table}


To summarize, both the \textbf{image space} and \textbf{learned} roadmaps exhibit unique benefits. The image space roadmap provides faster execution when successful but struggles in reliability and path convergence. The \textbf{learned} roadmap, using joint displacement-based distances, avoids non-holonomic constraints and ensures robustness, particularly in complex paths. The choice between the two ultimately depends on the specific application context.

\section{Conclusion and Future Work}
In conclusion, this work introduced a novel framework for collision-free motion planning of robotic manipulators that relied solely on visual features, eliminating the need for explicit robot models or encoder feedback. 

The \textbf{learned} roadmap offered smoother, more reliable transitions, and due to its joint displacement-based distance definition, the paths it generated maintained joint-space holonomy when it existed. In contrast, the paths produced by the \textbf{image space} roadmap sometimes failed to maintain joint-space holonomy, even when holonomy existed in the image space. However, the \textbf{image space} roadmap provided faster transient responses and simplicity, making it advantageous for applications where speed and computational efficiency were prioritized.

Future work will explore extending this approach to out-of-plane motion and incorporating real-world obstacles to develop a fully integrated control and manipulation pipeline.

\bibliographystyle{IEEEtran}
%%%%%%%%%%%%%%%%%%%%%%%%%%%%%%%%%%%%%%%%%%%%%%%%%%%%%%%%%%%%%%%%%%%%%%%%%%%%%%%%
%2345678901234567890123456789012345678901234567890123456789012345678901234567890
%        1         2         3         4         5         6         7         8

\documentclass[letterpaper, 10 pt, conference]{ieeeconf}  % Comment this line out if you need a4paper

%\documentclass[a4paper, 10pt, conference]{ieeeconf}      % Use this line for a4 paper

\IEEEoverridecommandlockouts                              % This command is only needed if 
                                                          % you want to use the \thanks command

\overrideIEEEmargins
% Needed to meet printer requirements.

%In case you encounter the following error:
%Error 1010 The PDF file may be corrupt (unable to open PDF file) OR
%Error 1000 An error occurred while parsing a contents stream. Unable to analyze the PDF file.
%This is a known problem with pdfLaTeX conversion filter. The file cannot be opened with acrobat reader
%Please use one of the alternatives below to circumvent this error by uncommenting one or the other
%\pdfobjcompresslevel=0
%\pdfminorversion=4

% See the \addtolength command later in the file to balance the column lengths
% on the last page of the document

% The following packages can be found on http:\\www.ctan.org
%\usepackage{graphics} % for pdf, bitmapped graphics files
%\usepackage{epsfig} % for postscript graphics files
%\usepackage{mathptmx} % assumes new font selection scheme installed
%\usepackage{times} % assumes new font selection scheme installed
%\usepackage{amsmath} % assumes amsmath package installed
%\usepackage{amssymb}  % assumes amsmath package installed

% Package
\usepackage{algorithm}
\usepackage{algpseudocode}
\usepackage{kotex}
\usepackage{xcolor}
\usepackage{booktabs}
\usepackage{xspace}
\usepackage{microtype}
\usepackage{booktabs}
\usepackage{siunitx}
\usepackage{multirow}
\usepackage{cite}
\usepackage{amsmath}
\usepackage{amssymb}
\usepackage{amsfonts}
% \usepackage{algorithmic}
% \usepackage[ruled,linesnumbered,vlined]{algorithm2e}
\usepackage[final]{graphicx}
\usepackage{textcomp}
\usepackage{etoolbox}
% \usepackage{subcaption} % removed due to package caption warning
\usepackage{comment}
\usepackage{hyperref}
\usepackage{cleveref}
\usepackage{flushend}

\title{\LARGE \bf

Do Looks Matter? Exploring Functional and Aesthetic

 Design Preferences for a Robotic Guide Dog
 % from 
 
 % Visually Impaired Users' Perspectives


}

\author{Aviv L. Cohav$^{*1}$, A. Xinran Gong$^{*1}$, J. Taery Kim$^{1}$, Clint Zeagler$^{1}$, Sehoon Ha$^{1}$, and Bruce N. Walker$^{1}$% <-this % stops a space
\thanks{*co-first authors}% <-this % stops a space
\thanks{$^{1}$Georgia Institute of Technology, Atlanta, GA, USA}
}


\begin{document}



\maketitle
\thispagestyle{empty}
\pagestyle{empty}


%%%%%%%%%%%%%%%%%%%%%%%%%%%%%%%%%%%%%%%%%%%%%%%%%%%%%%%%%%%%%%%%%%%%%%%%%%%%%%%%
\begin{abstract}

Dog guides offer an effective mobility solution for blind or visually impaired (BVI) individuals, but conventional dog guides have limitations including the need for care, potential distractions, societal prejudice, high costs, and limited availability. To address these challenges, we seek to develop a robot dog guide capable of performing the tasks of a conventional dog guide, enhanced with additional features. In this work, we focus on design research to identify functional and aesthetic design concepts to implement into a quadrupedal robot. The aesthetic design remains relevant even for BVI users due to their sensitivity toward societal perceptions and the need for smooth integration into society. We collected data through interviews and surveys to answer specific design questions pertaining to the appearance, texture, features, and method of controlling and communicating with the robot. Our study identified essential and preferred features for a future robot dog guide, which are supported by relevant statistics aligning with each suggestion. These findings will inform the future development of user-centered designs to effectively meet the needs of BVI individuals. 
% Dog guides offer an effective mobility solution for blind or visually impaired (BVI) individuals, but the high costs and length of training significantly limit their availability. Consequently, only a small fraction of BVI individuals who could benefit from a dog guide are able to access them. To address these challenges, we seek to develop a robot dog guide capable of performing the tasks of a conventional dog guide, enhanced with additional features. We are currently focused on design research, with the goal of identifying functional and aesthetic design concepts to implement into a quadruped robot. We collected data through interviews and surveys to answer specific design questions pertaining to the appearance, texture, features, and method of controlling and communicating with the robot. Some key findings from the BVI participants included the following: the participants had an overall favorable impression of a future robot dog guide and acknowledged numerous potential advantages that it could offer; most participants showed some preference for the robot to resemble the appearance of a real dog and look "cute" or "approachable", as well as for it to have a uniform identifier indicating it as a guide; they were in favor of the robot having a built-in GPS and Bluetooth connectivity and being made of waterproof material; they showed some preference for a softer external texture rather than a rigid one; they wanted to have the multiple options of voice command, motion gesture, and buttons on the harness as methods of giving commands to the robot, and both auditory signals and haptic feedback as methods of receiving signals from the robot; they emphasized the need for a long battery life and reliable battery percentage indicator and were highly favorable toward a self-charging capability. These data inform us on directions for prototyping, enabling us to narrow in on designs that effectively meet the needs of BVI individuals. 


\end{abstract}


\usepackage{bbm}
\usepackage{graphicx}
\usepackage{amsmath,amssymb,amsthm,amsfonts}

\usepackage{paralist}
\usepackage{bm}
\usepackage{xspace}
\usepackage{url}
\usepackage{prettyref}
\usepackage{boxedminipage}
\usepackage{wrapfig}
\usepackage{ifthen}
\usepackage{color}
\usepackage{xspace}

\newcommand{\ii}{{\sc Indicator-Instance}\xspace}
\newcommand{\midd}{{\sf mid}}


\usepackage{amsmath,amsthm,amsfonts,amssymb}
\usepackage{mathtools}
\usepackage{graphicx}


% \usepackage{fullpage}

\usepackage{nicefrac}

\newtheorem{inftheorem}{Informal Theorem}
\newtheorem{claim}{Claim}
\newtheorem*{definition*}{Definition}
\newtheorem{example}{Example}

\DeclareMathOperator*{\argmax}{arg\,max}
\DeclareMathOperator*{\argmin}{arg\,min}
\usepackage{subcaption}

\newtheorem{problem}{Problem}
\usepackage[utf8]{inputenc}
\newcommand{\rank}{\mathsf{rank}}
\newcommand{\tr}{\mathsf{Tr}}
\newcommand{\tv}{\mathsf{TV}}
\newcommand{\opt}{\mathsf{OPT}}
\newcommand{\rr}{\textsc{R}\space}
\newcommand{\alg}{\textsf{Alg}\space}
\newcommand{\sd}{\textsf{sd}_\lambda}
\newcommand{\lblq}{\mathfrak{lq} (X_1)}
\newcommand{\diag}{\textsf{diag}}
\newcommand{\sign}{\textsf{sgn}}
\newcommand{\BC}{\texttt{BC} }
\newcommand{\MM}{\texttt{MM} }
\newcommand{\Nexp}{N_{\mathrm{exp}}}
\newcommand{\Nrep}{N_{\mathrm{replay}}}
\newcommand{\Drep}{D_{\mathrm{replay}}}
\newcommand{\Nsim}{N_{\mathrm{sim}}}
\newcommand{\piBC}{\pi^{\texttt{BC}}}
\newcommand{\piRE}{\pi^{\texttt{RE}}}
\newcommand{\piEMM}{\pi^{\texttt{MM}}}
\newcommand{\mmd}{\texttt{Mimic-MD} }
\newcommand{\RE}{\texttt{RE} }
\newcommand{\dem}{\pi^E}
\newcommand{\Rlint}{\mathcal{R}_{\mathrm{lin,t}}}
\newcommand{\Rlipt}{\mathcal{R}_{\mathrm{lip,t}}}
\newcommand{\Rlin}{\mathcal{R}_{\mathrm{lin}}}
\newcommand{\Rlip}{\mathcal{R}_{\mathrm{lip}}}
\newcommand{\Rmax}{R_{\mathrm{max}}}
\newcommand{\Rall}{\mathcal{R}_{\mathrm{all}}}
\newcommand{\Rdet}{\mathcal{R}_{\mathrm{det}}}
\newcommand{\Fmax}{F_{\mathrm{max}}}
\newcommand{\Nmax}{\mathcal{N}_{\mathrm{max}}}
\newcommand{\piref}{\pi^{\mathrm{ref}}}
\newcommand{\green}{\text{\color{green!75!black} green}\;}
\newcommand{\thetaBC}{\widehat{\theta}^{\textsf{BC}}}
\newcommand{\ent}{\mathcal{E}_{\Theta,n,\delta}}
\newcommand{\eNt}{\mathcal{E}_{\Theta_t,\Nexp,\delta}}
\newcommand{\eNtH}{\mathcal{E}_{\Theta_t,\Nexp,\delta/H}}

\newcommand{\eref}[1]{(\ref{#1})}
\newcommand{\sref}[1]{Sec. \ref{#1}}
\newcommand{\dr}{\widehat{d}_{\mathrm{replay}}}
\newcommand{\figref}[1]{Fig. \ref{#1}}

\usepackage{xcolor}
\definecolor{expert}{HTML}{008000}
\definecolor{error}{HTML}{f96565}
\newcommand{\GKS}[1]{{\textcolor{violet}{\textbf{GKS: #1}}}}
\newcommand{\Q}[1]{{\textcolor{red}{\textbf{Question #1}}}}
\newcommand{\ZSW}[1]{{\textcolor{orange}{\textbf{ZSW: #1}}}}
\newcommand{\JAB}[1]{{\textcolor{teal}{\textbf{JAB: #1}}}}
\newcommand{\jab}[1]{{\textcolor{teal}{\textbf{JAB: #1}}}}
\newcommand{\SAN}[1]{{\textcolor{blue}{\textbf{SC: #1}}}}
\newcommand{\scnote}[1]{\SAN{#1}}
\newcommand{\norm}[1]{\left\lVert #1 \right\rVert}

\usepackage{color-edits}
\addauthor{sw}{blue}

\usepackage{thmtools}
\usepackage{thm-restate}

\usepackage{tikz}
\usetikzlibrary{arrows,calc} 
\newcommand{\tikzAngleOfLine}{\tikz@AngleOfLine}
\def\tikz@AngleOfLine(#1)(#2)#3{%
\pgfmathanglebetweenpoints{%
\pgfpointanchor{#1}{center}}{%
\pgfpointanchor{#2}{center}}
\pgfmathsetmacro{#3}{\pgfmathresult}%
}

\declaretheoremstyle[
    headfont=\normalfont\bfseries, 
    bodyfont = \normalfont\itshape]{mystyle} 
\declaretheorem[name=Theorem,style=mystyle,numberwithin=section]{thm}

% \usepackage{algorithm}
% \usepackage{algorithmic}
\usepackage[linesnumbered,algoruled,boxed,lined,noend]{algorithm2e}

\usepackage{listings}
\usepackage{amsmath}
\usepackage{amsthm}
\usepackage{tikz}
\usepackage{caption}
\usepackage{mdwmath}
\usepackage{multirow}
\usepackage{mdwtab}
\usepackage{eqparbox}
\usepackage{multicol}
\usepackage{amsfonts}
\usepackage{tikz}
\usepackage{multirow,bigstrut,threeparttable}
\usepackage{amsthm}
\usepackage{bbm}
\usepackage{epstopdf}
\usepackage{mdwmath}
\usepackage{mdwtab}
\usepackage{eqparbox}
\usetikzlibrary{topaths,calc}
\usepackage{latexsym}
\usepackage{cite}
\usepackage{amssymb}
\usepackage{bm}
\usepackage{amssymb}
\usepackage{graphicx}
\usepackage{mathrsfs}
\usepackage{epsfig}
\usepackage{psfrag}
\usepackage{setspace}
\usepackage[%dvips,
            CJKbookmarks=true,
            bookmarksnumbered=true,
            bookmarksopen=true,
%						bookmarks=false,
            colorlinks=true,
            citecolor=red,
            linkcolor=blue,
            anchorcolor=red,
            urlcolor=blue
            ]{hyperref}
%\usepackage{algorithm}
\usepackage[linesnumbered,algoruled,boxed,lined]{algorithm2e}
\usepackage{algpseudocode}
\usepackage{stfloats}
\RequirePackage[numbers]{natbib}

\usepackage{comment}
\usepackage{mathtools}
\usepackage{blkarray}
\usepackage{multirow,bigdelim,dcolumn,booktabs}

\usepackage{xparse}
\usepackage{tikz}
\usetikzlibrary{calc}
\usetikzlibrary{decorations.pathreplacing,matrix,positioning}

\usepackage[T1]{fontenc}
\usepackage[utf8]{inputenc}
\usepackage{mathtools}
\usepackage{blkarray, bigstrut}
\usepackage{gauss}

\newenvironment{mygmatrix}{\def\mathstrut{\vphantom{\big(}}\gmatrix}{\endgmatrix}

\newcommand{\tikzmark}[1]{\tikz[overlay,remember picture] \node (#1) {};}

%% Adapted form https://tex.stackexchange.com/questions/206898/braces-for-cases-in-tabular-environment/207704#207704
\newcommand*{\BraceAmplitude}{0.4em}%
\newcommand*{\VerticalOffset}{0.5ex}%  
\newcommand*{\HorizontalOffset}{0.0em}% 
\newcommand*{\blocktextwid}{3.0cm}%
\NewDocumentCommand{\InsertLeftBrace}{%
	O{} % #1 = draw options
	O{\HorizontalOffset,\VerticalOffset} % #2 = optional brace shift options
	O{\blocktextwid} % #3 = optional text width
	m   % #4 = top tikzmark
	m   % #5 = bottom tikzmark
	m   % #6 = node text
}{%
	\begin{tikzpicture}[overlay,remember picture]
	\coordinate (Brace Top)    at ($(#4.north) + (#2)$);
	\coordinate (Brace Bottom) at ($(#5.south) + (#2)$);
	\draw [decoration={brace, amplitude=\BraceAmplitude}, decorate, thick, draw=black, #1]
	(Brace Bottom) -- (Brace Top) 
	node [pos=0.5, anchor=east, align=left, text width=#3, color=black, xshift=\BraceAmplitude] {#6};
	\end{tikzpicture}%
}%
\NewDocumentCommand{\InsertRightBrace}{%
	O{} % #1 = draw options
	O{\HorizontalOffset,\VerticalOffset} % #2 = optional brace shift options
	O{\blocktextwid} % #3 = optional text width
	m   % #4 = top tikzmark
	m   % #5 = bottom tikzmark
	m   % #6 = node text
}{%
	\begin{tikzpicture}[overlay,remember picture]
	\coordinate (Brace Top)    at ($(#4.north) + (#2)$);
	\coordinate (Brace Bottom) at ($(#5.south) + (#2)$);
	\draw [decoration={brace, amplitude=\BraceAmplitude}, decorate, thick, draw=black, #1]
	(Brace Top) -- (Brace Bottom) 
	node [pos=0.5, anchor=west, align=left, text width=#3, color=black, xshift=\BraceAmplitude] {#6};
	\end{tikzpicture}%
}%
\NewDocumentCommand{\InsertTopBrace}{%
	O{} % #1 = draw options
	O{\HorizontalOffset,\VerticalOffset} % #2 = optional brace shift options
	O{\blocktextwid} % #3 = optional text width
	m   % #4 = top tikzmark
	m   % #5 = bottom tikzmark
	m   % #6 = node text
}{%
	\begin{tikzpicture}[overlay,remember picture]
	\coordinate (Brace Top)    at ($(#4.west) + (#2)$);
	\coordinate (Brace Bottom) at ($(#5.east) + (#2)$);
	\draw [decoration={brace, amplitude=\BraceAmplitude}, decorate, thick, draw=black, #1]
	(Brace Top) -- (Brace Bottom) 
	node [pos=0.5, anchor=south, align=left, text width=#3, color=black, xshift=\BraceAmplitude] {#6};
	\end{tikzpicture}%
}%

\usetikzlibrary{patterns}

\definecolor{cof}{RGB}{219,144,71}
\definecolor{pur}{RGB}{186,146,162}
\definecolor{greeo}{RGB}{91,173,69}
\definecolor{greet}{RGB}{52,111,72}

% provide arXiv number if available:
% \arxiv{cs.IT/1502.00326}

% put your definitions there:

%\newtheorem{remark}{Remark} \def\remref#1{Remark~\ref{#1}}
%\newtheorem{conjecture}{Conjecture} \def\remref#1{Remark~\ref{#1}}
%\newtheorem{example}{Example}

%\theorembodyfont{\itshape}
%\newtheorem{theorem}{Theorem}
%\newtheorem{proposition}{Proposition}
%\newtheorem{lemma}{Lemma} \def\lemref#1{Lemma~\ref{#1}}
%\newtheorem{corollary}{Corollary}


%\theorembodyfont{\rmfamily}
%\newtheorem{definition}{Definition}
%\numberwithin{equation}{section}
% \theoremstyle{plain}
% \newtheorem{theorem}{Theorem}
% \newtheorem{Example}{Example}
% \newtheorem{lemma}{Lemma}
% \newtheorem{remark}{Remark}
% \newtheorem{corollary}{Corollary}
% \newtheorem{definition}{Definition}
% \newtheorem{conjecture}{Conjecture}
% \newtheorem{question}{Question}
% \newtheorem*{induction}{Induction Hypothesis}
% \newtheorem*{folklore}{Folklore}
% \newtheorem{assumption}{Assumption}

\def \by {\bar{y}}
\def \bx {\bar{x}}
\def \bh {\bar{h}}
\def \bz {\bar{z}}
\def \cF {\mathcal{F}}
\def \bP {\mathbb{P}}
\def \bE {\mathbb{E}}
\def \bR {\mathbb{R}}
\def \bF {\mathbb{F}}
\def \cG {\mathcal{G}}
\def \cM {\mathcal{M}}
\def \cB {\mathcal{B}}
\def \cN {\mathcal{N}}
\def \var {\mathsf{Var}}
\def\1{\mathbbm{1}}
\def \FF {\mathbb{F}}


\newenvironment{keywords}
{\bgroup\leftskip 20pt\rightskip 20pt \small\noindent{\bfseries
Keywords:} \ignorespaces}%
{\par\egroup\vskip 0.25ex}
\newlength\aftertitskip     \newlength\beforetitskip
\newlength\interauthorskip  \newlength\aftermaketitskip















%%%%%%%%%%%%%%%%%%%%%%%%%%%% by Wu %%%%%%%%%%%%%%%%%%%%%%%%%%%%
\usepackage{xspace}

\newcommand{\Lip}{\mathrm{Lip}}
\newcommand{\stepa}[1]{\overset{\rm (a)}{#1}}
\newcommand{\stepb}[1]{\overset{\rm (b)}{#1}}
\newcommand{\stepc}[1]{\overset{\rm (c)}{#1}}
\newcommand{\stepd}[1]{\overset{\rm (d)}{#1}}
\newcommand{\stepe}[1]{\overset{\rm (e)}{#1}}
\newcommand{\stepf}[1]{\overset{\rm (f)}{#1}}


\newcommand{\floor}[1]{{\left\lfloor {#1} \right \rfloor}}
\newcommand{\ceil}[1]{{\left\lceil {#1} \right \rceil}}

\newcommand{\blambda}{\bar{\lambda}}
\newcommand{\reals}{\mathbb{R}}
\newcommand{\naturals}{\mathbb{N}}
\newcommand{\integers}{\mathbb{Z}}
\newcommand{\Expect}{\mathbb{E}}
\newcommand{\expect}[1]{\mathbb{E}\left[#1\right]}
\newcommand{\Prob}{\mathbb{P}}
\newcommand{\prob}[1]{\mathbb{P}\left[#1\right]}
\newcommand{\pprob}[1]{\mathbb{P}[#1]}
\newcommand{\intd}{{\rm d}}
\newcommand{\TV}{{\sf TV}}
\newcommand{\LC}{{\sf LC}}
\newcommand{\PW}{{\sf PW}}
\newcommand{\htheta}{\hat{\theta}}
\newcommand{\eexp}{{\rm e}}
\newcommand{\expects}[2]{\mathbb{E}_{#2}\left[ #1 \right]}
\newcommand{\diff}{{\rm d}}
\newcommand{\eg}{e.g.\xspace}
\newcommand{\ie}{i.e.\xspace}
\newcommand{\iid}{i.i.d.\xspace}
\newcommand{\fracp}[2]{\frac{\partial #1}{\partial #2}}
\newcommand{\fracpk}[3]{\frac{\partial^{#3} #1}{\partial #2^{#3}}}
\newcommand{\fracd}[2]{\frac{\diff #1}{\diff #2}}
\newcommand{\fracdk}[3]{\frac{\diff^{#3} #1}{\diff #2^{#3}}}
\newcommand{\renyi}{R\'enyi\xspace}
\newcommand{\lpnorm}[1]{\left\|{#1} \right\|_{p}}
\newcommand{\linf}[1]{\left\|{#1} \right\|_{\infty}}
\newcommand{\lnorm}[2]{\left\|{#1} \right\|_{{#2}}}
\newcommand{\Lploc}[1]{L^{#1}_{\rm loc}}
\newcommand{\hellinger}{d_{\rm H}}
\newcommand{\Fnorm}[1]{\lnorm{#1}{\rm F}}
%% parenthesis
\newcommand{\pth}[1]{\left( #1 \right)}
\newcommand{\qth}[1]{\left[ #1 \right]}
\newcommand{\sth}[1]{\left\{ #1 \right\}}
\newcommand{\bpth}[1]{\Bigg( #1 \Bigg)}
\newcommand{\bqth}[1]{\Bigg[ #1 \Bigg]}
\newcommand{\bsth}[1]{\Bigg\{ #1 \Bigg\}}
\newcommand{\xxx}{\textbf{xxx}\xspace}
\newcommand{\toprob}{{\xrightarrow{\Prob}}}
\newcommand{\tolp}[1]{{\xrightarrow{L^{#1}}}}
\newcommand{\toas}{{\xrightarrow{{\rm a.s.}}}}
\newcommand{\toae}{{\xrightarrow{{\rm a.e.}}}}
\newcommand{\todistr}{{\xrightarrow{{\rm D}}}}
\newcommand{\eqdistr}{{\stackrel{\rm D}{=}}}
\newcommand{\iiddistr}{{\stackrel{\text{\iid}}{\sim}}}
%\newcommand{\var}{\mathsf{var}}
\newcommand\indep{\protect\mathpalette{\protect\independenT}{\perp}}
\def\independenT#1#2{\mathrel{\rlap{$#1#2$}\mkern2mu{#1#2}}}
\newcommand{\Bern}{\text{Bern}}
\newcommand{\Poi}{\mathsf{Poi}}
\newcommand{\iprod}[2]{\left \langle #1, #2 \right\rangle}
\newcommand{\Iprod}[2]{\langle #1, #2 \rangle}
\newcommand{\indc}[1]{{\mathbf{1}_{\left\{{#1}\right\}}}}
\newcommand{\Indc}{\mathbf{1}}
\newcommand{\regoff}[1]{\textsf{Reg}_{\mathcal{F}}^{\text{off}} (#1)}
\newcommand{\regon}[1]{\textsf{Reg}_{\mathcal{F}}^{\text{on}} (#1)}

\definecolor{myblue}{rgb}{.8, .8, 1}
\definecolor{mathblue}{rgb}{0.2472, 0.24, 0.6} % mathematica's Color[1, 1--3]
\definecolor{mathred}{rgb}{0.6, 0.24, 0.442893}
\definecolor{mathyellow}{rgb}{0.6, 0.547014, 0.24}


\newcommand{\red}{\color{red}}
\newcommand{\blue}{\color{blue}}
\newcommand{\nb}[1]{{\sf\blue[#1]}}
\newcommand{\nbr}[1]{{\sf\red[#1]}}

\newcommand{\tmu}{{\tilde{\mu}}}
\newcommand{\tf}{{\tilde{f}}}
\newcommand{\tp}{\tilde{p}}
\newcommand{\tilh}{{\tilde{h}}}
\newcommand{\tu}{{\tilde{u}}}
\newcommand{\tx}{{\tilde{x}}}
\newcommand{\ty}{{\tilde{y}}}
\newcommand{\tz}{{\tilde{z}}}
\newcommand{\tA}{{\tilde{A}}}
\newcommand{\tB}{{\tilde{B}}}
\newcommand{\tC}{{\tilde{C}}}
\newcommand{\tD}{{\tilde{D}}}
\newcommand{\tE}{{\tilde{E}}}
\newcommand{\tF}{{\tilde{F}}}
\newcommand{\tG}{{\tilde{G}}}
\newcommand{\tH}{{\tilde{H}}}
\newcommand{\tI}{{\tilde{I}}}
\newcommand{\tJ}{{\tilde{J}}}
\newcommand{\tK}{{\tilde{K}}}
\newcommand{\tL}{{\tilde{L}}}
\newcommand{\tM}{{\tilde{M}}}
\newcommand{\tN}{{\tilde{N}}}
\newcommand{\tO}{{\tilde{O}}}
\newcommand{\tP}{{\tilde{P}}}
\newcommand{\tQ}{{\tilde{Q}}}
\newcommand{\tR}{{\tilde{R}}}
\newcommand{\tS}{{\tilde{S}}}
\newcommand{\tT}{{\tilde{T}}}
\newcommand{\tU}{{\tilde{U}}}
\newcommand{\tV}{{\tilde{V}}}
\newcommand{\tW}{{\tilde{W}}}
\newcommand{\tX}{{\tilde{X}}}
\newcommand{\tY}{{\tilde{Y}}}
\newcommand{\tZ}{{\tilde{Z}}}

\newcommand{\sfa}{{\mathsf{a}}}
\newcommand{\sfb}{{\mathsf{b}}}
\newcommand{\sfc}{{\mathsf{c}}}
\newcommand{\sfd}{{\mathsf{d}}}
\newcommand{\sfe}{{\mathsf{e}}}
\newcommand{\sff}{{\mathsf{f}}}
\newcommand{\sfg}{{\mathsf{g}}}
\newcommand{\sfh}{{\mathsf{h}}}
\newcommand{\sfi}{{\mathsf{i}}}
\newcommand{\sfj}{{\mathsf{j}}}
\newcommand{\sfk}{{\mathsf{k}}}
\newcommand{\sfl}{{\mathsf{l}}}
\newcommand{\sfm}{{\mathsf{m}}}
\newcommand{\sfn}{{\mathsf{n}}}
\newcommand{\sfo}{{\mathsf{o}}}
\newcommand{\sfp}{{\mathsf{p}}}
\newcommand{\sfq}{{\mathsf{q}}}
\newcommand{\sfr}{{\mathsf{r}}}
\newcommand{\sfs}{{\mathsf{s}}}
\newcommand{\sft}{{\mathsf{t}}}
\newcommand{\sfu}{{\mathsf{u}}}
\newcommand{\sfv}{{\mathsf{v}}}
\newcommand{\sfw}{{\mathsf{w}}}
\newcommand{\sfx}{{\mathsf{x}}}
\newcommand{\sfy}{{\mathsf{y}}}
\newcommand{\sfz}{{\mathsf{z}}}
\newcommand{\sfA}{{\mathsf{A}}}
\newcommand{\sfB}{{\mathsf{B}}}
\newcommand{\sfC}{{\mathsf{C}}}
\newcommand{\sfD}{{\mathsf{D}}}
\newcommand{\sfE}{{\mathsf{E}}}
\newcommand{\sfF}{{\mathsf{F}}}
\newcommand{\sfG}{{\mathsf{G}}}
\newcommand{\sfH}{{\mathsf{H}}}
\newcommand{\sfI}{{\mathsf{I}}}
\newcommand{\sfJ}{{\mathsf{J}}}
\newcommand{\sfK}{{\mathsf{K}}}
\newcommand{\sfL}{{\mathsf{L}}}
\newcommand{\sfM}{{\mathsf{M}}}
\newcommand{\sfN}{{\mathsf{N}}}
\newcommand{\sfO}{{\mathsf{O}}}
\newcommand{\sfP}{{\mathsf{P}}}
\newcommand{\sfQ}{{\mathsf{Q}}}
\newcommand{\sfR}{{\mathsf{R}}}
\newcommand{\sfS}{{\mathsf{S}}}
\newcommand{\sfT}{{\mathsf{T}}}
\newcommand{\sfU}{{\mathsf{U}}}
\newcommand{\sfV}{{\mathsf{V}}}
\newcommand{\sfW}{{\mathsf{W}}}
\newcommand{\sfX}{{\mathsf{X}}}
\newcommand{\sfY}{{\mathsf{Y}}}
\newcommand{\sfZ}{{\mathsf{Z}}}


\newcommand{\calA}{{\mathcal{A}}}
\newcommand{\calB}{{\mathcal{B}}}
\newcommand{\calC}{{\mathcal{C}}}
\newcommand{\calD}{{\mathcal{D}}}
\newcommand{\calE}{{\mathcal{E}}}
\newcommand{\calF}{{\mathcal{F}}}
\newcommand{\calG}{{\mathcal{G}}}
\newcommand{\calH}{{\mathcal{H}}}
\newcommand{\calI}{{\mathcal{I}}}
\newcommand{\calJ}{{\mathcal{J}}}
\newcommand{\calK}{{\mathcal{K}}}
\newcommand{\calL}{{\mathcal{L}}}
\newcommand{\calM}{{\mathcal{M}}}
\newcommand{\calN}{{\mathcal{N}}}
\newcommand{\calO}{{\mathcal{O}}}
\newcommand{\calP}{{\mathcal{P}}}
\newcommand{\calQ}{{\mathcal{Q}}}
\newcommand{\calR}{{\mathcal{R}}}
\newcommand{\calS}{{\mathcal{S}}}
\newcommand{\calT}{{\mathcal{T}}}
\newcommand{\calU}{{\mathcal{U}}}
\newcommand{\calV}{{\mathcal{V}}}
\newcommand{\calW}{{\mathcal{W}}}
\newcommand{\calX}{{\mathcal{X}}}
\newcommand{\calY}{{\mathcal{Y}}}
\newcommand{\calZ}{{\mathcal{Z}}}

\newcommand{\bara}{{\bar{a}}}
\newcommand{\barb}{{\bar{b}}}
\newcommand{\barc}{{\bar{c}}}
\newcommand{\bard}{{\bar{d}}}
\newcommand{\bare}{{\bar{e}}}
\newcommand{\barf}{{\bar{f}}}
\newcommand{\barg}{{\bar{g}}}
\newcommand{\barh}{{\bar{h}}}
\newcommand{\bari}{{\bar{i}}}
\newcommand{\barj}{{\bar{j}}}
\newcommand{\bark}{{\bar{k}}}
\newcommand{\barl}{{\bar{l}}}
\newcommand{\barm}{{\bar{m}}}
\newcommand{\barn}{{\bar{n}}}
\newcommand{\baro}{{\bar{o}}}
\newcommand{\barp}{{\bar{p}}}
\newcommand{\barq}{{\bar{q}}}
\newcommand{\barr}{{\bar{r}}}
\newcommand{\bars}{{\bar{s}}}
\newcommand{\bart}{{\bar{t}}}
\newcommand{\baru}{{\bar{u}}}
\newcommand{\barv}{{\bar{v}}}
\newcommand{\barw}{{\bar{w}}}
\newcommand{\barx}{{\bar{x}}}
\newcommand{\bary}{{\bar{y}}}
\newcommand{\barz}{{\bar{z}}}
\newcommand{\barA}{{\bar{A}}}
\newcommand{\barB}{{\bar{B}}}
\newcommand{\barC}{{\bar{C}}}
\newcommand{\barD}{{\bar{D}}}
\newcommand{\barE}{{\bar{E}}}
\newcommand{\barF}{{\bar{F}}}
\newcommand{\barG}{{\bar{G}}}
\newcommand{\barH}{{\bar{H}}}
\newcommand{\barI}{{\bar{I}}}
\newcommand{\barJ}{{\bar{J}}}
\newcommand{\barK}{{\bar{K}}}
\newcommand{\barL}{{\bar{L}}}
\newcommand{\barM}{{\bar{M}}}
\newcommand{\barN}{{\bar{N}}}
\newcommand{\barO}{{\bar{O}}}
\newcommand{\barP}{{\bar{P}}}
\newcommand{\barQ}{{\bar{Q}}}
\newcommand{\barR}{{\bar{R}}}
\newcommand{\barS}{{\bar{S}}}
\newcommand{\barT}{{\bar{T}}}
\newcommand{\barU}{{\bar{U}}}
\newcommand{\barV}{{\bar{V}}}
\newcommand{\barW}{{\bar{W}}}
\newcommand{\barX}{{\bar{X}}}
\newcommand{\barY}{{\bar{Y}}}
\newcommand{\barZ}{{\bar{Z}}}

\newcommand{\hX}{\hat{X}}
\newcommand{\Ent}{\mathsf{Ent}}
\newcommand{\awarm}{{A_{\text{warm}}}}
\newcommand{\thetaLS}{{\widehat{\theta}^{\text{\rm LS}}}}

\newcommand{\jiao}[1]{\langle{#1}\rangle}
\newcommand{\gaht}{\textsc{GoodActionHypTest}\;}
\newcommand{\iaht}{\textsc{InitialActionHypTest}\;}
\newcommand{\true}{\textsf{True}\;}
\newcommand{\false}{\textsf{False}\;}

% \usepackage[capitalize,noabbrev]{cleveref}
% \crefname{lemma}{Lemma}{Lemmas}
% \Crefname{lemma}{Lemma}{Lemmas}
% \crefname{thm}{Theorem}{Theorems}
% \Crefname{thm}{Theorem}{Theorems}
% \Crefname{assumption}{Assumption}{Assumptions}
% \Crefname{inftheorem}{Informal Theorem}{Informal Theorems}
% \crefformat{equation}{(#2#1#3)}

% % if you use cleveref..
% \usepackage[capitalize,noabbrev]{cleveref}
% \crefname{lemma}{Lemma}{Lemmas}
% \crefname{proposition}{Proposition}{Propositions}
% \crefname{remark}{Remark}{Remarks}
% \crefname{corollary}{Corollary}{Corollaries}
% \crefname{definition}{Definition}{Definitions}
% \crefname{conjecture}{Conjecture}{Conjectures}
% \crefname{figure}{Fig.}{Figures}

\section{Introduction}
Backdoor attacks pose a concealed yet profound security risk to machine learning (ML) models, for which the adversaries can inject a stealth backdoor into the model during training, enabling them to illicitly control the model's output upon encountering predefined inputs. These attacks can even occur without the knowledge of developers or end-users, thereby undermining the trust in ML systems. As ML becomes more deeply embedded in critical sectors like finance, healthcare, and autonomous driving \citep{he2016deep, liu2020computing, tournier2019mrtrix3, adjabi2020past}, the potential damage from backdoor attacks grows, underscoring the emergency for developing robust defense mechanisms against backdoor attacks.

To address the threat of backdoor attacks, researchers have developed a variety of strategies \cite{liu2018fine,wu2021adversarial,wang2019neural,zeng2022adversarial,zhu2023neural,Zhu_2023_ICCV, wei2024shared,wei2024d3}, aimed at purifying backdoors within victim models. These methods are designed to integrate with current deployment workflows seamlessly and have demonstrated significant success in mitigating the effects of backdoor triggers \cite{wubackdoorbench, wu2023defenses, wu2024backdoorbench,dunnett2024countering}.  However, most state-of-the-art (SOTA) backdoor purification methods operate under the assumption that a small clean dataset, often referred to as \textbf{auxiliary dataset}, is available for purification. Such an assumption poses practical challenges, especially in scenarios where data is scarce. To tackle this challenge, efforts have been made to reduce the size of the required auxiliary dataset~\cite{chai2022oneshot,li2023reconstructive, Zhu_2023_ICCV} and even explore dataset-free purification techniques~\cite{zheng2022data,hong2023revisiting,lin2024fusing}. Although these approaches offer some improvements, recent evaluations \cite{dunnett2024countering, wu2024backdoorbench} continue to highlight the importance of sufficient auxiliary data for achieving robust defenses against backdoor attacks.

While significant progress has been made in reducing the size of auxiliary datasets, an equally critical yet underexplored question remains: \emph{how does the nature of the auxiliary dataset affect purification effectiveness?} In  real-world  applications, auxiliary datasets can vary widely, encompassing in-distribution data, synthetic data, or external data from different sources. Understanding how each type of auxiliary dataset influences the purification effectiveness is vital for selecting or constructing the most suitable auxiliary dataset and the corresponding technique. For instance, when multiple datasets are available, understanding how different datasets contribute to purification can guide defenders in selecting or crafting the most appropriate dataset. Conversely, when only limited auxiliary data is accessible, knowing which purification technique works best under those constraints is critical. Therefore, there is an urgent need for a thorough investigation into the impact of auxiliary datasets on purification effectiveness to guide defenders in  enhancing the security of ML systems. 

In this paper, we systematically investigate the critical role of auxiliary datasets in backdoor purification, aiming to bridge the gap between idealized and practical purification scenarios.  Specifically, we first construct a diverse set of auxiliary datasets to emulate real-world conditions, as summarized in Table~\ref{overall}. These datasets include in-distribution data, synthetic data, and external data from other sources. Through an evaluation of SOTA backdoor purification methods across these datasets, we uncover several critical insights: \textbf{1)} In-distribution datasets, particularly those carefully filtered from the original training data of the victim model, effectively preserve the model’s utility for its intended tasks but may fall short in eliminating backdoors. \textbf{2)} Incorporating OOD datasets can help the model forget backdoors but also bring the risk of forgetting critical learned knowledge, significantly degrading its overall performance. Building on these findings, we propose Guided Input Calibration (GIC), a novel technique that enhances backdoor purification by adaptively transforming auxiliary data to better align with the victim model’s learned representations. By leveraging the victim model itself to guide this transformation, GIC optimizes the purification process, striking a balance between preserving model utility and mitigating backdoor threats. Extensive experiments demonstrate that GIC significantly improves the effectiveness of backdoor purification across diverse auxiliary datasets, providing a practical and robust defense solution.

Our main contributions are threefold:
\textbf{1) Impact analysis of auxiliary datasets:} We take the \textbf{first step}  in systematically investigating how different types of auxiliary datasets influence backdoor purification effectiveness. Our findings provide novel insights and serve as a foundation for future research on optimizing dataset selection and construction for enhanced backdoor defense.
%
\textbf{2) Compilation and evaluation of diverse auxiliary datasets:}  We have compiled and rigorously evaluated a diverse set of auxiliary datasets using SOTA purification methods, making our datasets and code publicly available to facilitate and support future research on practical backdoor defense strategies.
%
\textbf{3) Introduction of GIC:} We introduce GIC, the \textbf{first} dedicated solution designed to align auxiliary datasets with the model’s learned representations, significantly enhancing backdoor mitigation across various dataset types. Our approach sets a new benchmark for practical and effective backdoor defense.



\section{Related Work}
\label{sec:related-works}
\subsection{Novel View Synthesis}
Novel view synthesis is a foundational task in the computer vision and graphics, which aims to generate unseen views of a scene from a given set of images.
% Many methods have been designed to solve this problem by posing it as 3D geometry based rendering, where point clouds~\cite{point_differentiable,point_nfs}, mesh~\cite{worldsheet,FVS,SVS}, planes~\cite{automatci_photo_pop_up,tour_into_the_picture} and multi-plane images~\cite{MINE,single_view_mpi,stereo_magnification}, \etal
Numerous methods have been developed to address this problem by approaching it as 3D geometry-based rendering, such as using meshes~\cite{worldsheet,FVS,SVS}, MPI~\cite{MINE,single_view_mpi,stereo_magnification}, point clouds~\cite{point_differentiable,point_nfs}, etc.
% planes~\cite{automatci_photo_pop_up,tour_into_the_picture}, 


\begin{figure*}[!t]
    \centering
    \includegraphics[width=1.0\linewidth]{figures/overview-v7.png}
    %\caption{\textbf{Overview.} Given a set of images, our method obtains both camera intrinsics and extrinsics, as well as a 3DGS model. First, we obtain the initial camera parameters, global track points from image correspondences and monodepth with reprojection loss. Then we incorporate the global track information and select Gaussian kernels associated with track points. We jointly optimize the parameters $K$, $T_{cw}$, 3DGS through multi-view geometric consistency $L_{t2d}$, $L_{t3d}$, $L_{scale}$ and photometric consistency $L_1$, $L_{D-SSIM}$.}
    \caption{\textbf{Overview.} Given a set of images, our method obtains both camera intrinsics and extrinsics, as well as a 3DGS model. During the initialization, we extract the global tracks, and initialize camera parameters and Gaussians from image correspondences and monodepth with reprojection loss. We determine Gaussian kernels with recovered 3D track points, and then jointly optimize the parameters $K$, $T_{cw}$, 3DGS through the proposed global track constraints (i.e., $L_{t2d}$, $L_{t3d}$, and $L_{scale}$) and original photometric losses (i.e., $L_1$ and $L_{D-SSIM}$).}
    \label{fig:overview}
\end{figure*}

Recently, Neural Radiance Fields (NeRF)~\cite{2020NeRF} provide a novel solution to this problem by representing scenes as implicit radiance fields using neural networks, achieving photo-realistic rendering quality. Although having some works in improving efficiency~\cite{instant_nerf2022, lin2022enerf}, the time-consuming training and rendering still limit its practicality.
Alternatively, 3D Gaussian Splatting (3DGS)~\cite{3DGS2023} models the scene as explicit Gaussian kernels, with differentiable splatting for rendering. Its improved real-time rendering performance, lower storage and efficiency, quickly attract more attentions.
% Different from NeRF-based methods which need MLPs to model the scene and huge computational cost for rendering, 3DGS has stronger real-time performance, higher storage and computational efficiency, benefits from its explicit representation and gradient backpropagation.

\subsection{Optimizing Camera Poses in NeRFs and 3DGS}
Although NeRF and 3DGS can provide impressive scene representation, these methods all need accurate camera parameters (both intrinsic and extrinsic) as additional inputs, which are mostly obtained by COLMAP~\cite{colmap2016}.
% This strong reliance on COLMAP significantly limits their use in real-world applications, so optimizing the camera parameters during the scene training becomes crucial.
When the prior is inaccurate or unknown, accurately estimating camera parameters and scene representations becomes crucial.

% In early works, only photometric constraints are used for scene training and camera pose estimation. 
% iNeRF~\cite{iNerf2021} optimizes the camera poses based on a pre-trained NeRF model.
% NeRFmm~\cite{wang2021nerfmm} introduce a joint optimization process, which estimates the camera poses and trains NeRF model jointly.
% BARF~\cite{barf2021} and GARF~\cite{2022GARF} provide new positional encoding strategy to handle with the gradient inconsistency issue of positional embedding and yield promising results.
% However, they achieve satisfactory optimization results when only the pose initialization is quite closed to the ground-truth, as the photometric constrains can only improve the quality of camera estimation within a small range.
% Later, more prior information of geometry and correspondence, \ie monocular depth and feature matching, are introduced into joint optimisation to enhance the capability of camera poses estimation.
% SC-NeRF~\cite{SCNeRF2021} minimizes a projected ray distance loss based on correspondence of adjacent frames.
% NoPe-NeRF~\cite{bian2022nopenerf} chooses monocular depth maps as geometric priors, and defines undistorted depth loss and relative pose constraints for joint optimization.
In earlier studies, scene training and camera pose estimation relied solely on photometric constraints. iNeRF~\cite{iNerf2021} refines the camera poses using a pre-trained NeRF model. NeRFmm~\cite{wang2021nerfmm} introduces a joint optimization approach that simultaneously estimates camera poses and trains the NeRF model. BARF~\cite{barf2021} and GARF~\cite{2022GARF} propose a new positional encoding strategy to address the gradient inconsistency issues in positional embedding, achieving promising results. However, these methods only yield satisfactory optimization when the initial pose is very close to the ground truth, as photometric constraints alone can only enhance camera estimation quality within a limited range. Subsequently, 
% additional prior information on geometry and correspondence, such as monocular depth and feature matching, has been incorporated into joint optimization to improve the accuracy of camera pose estimation. 
SC-NeRF~\cite{SCNeRF2021} minimizes a projected ray distance loss based on correspondence between adjacent frames. NoPe-NeRF~\cite{bian2022nopenerf} utilizes monocular depth maps as geometric priors and defines undistorted depth loss and relative pose constraints.

% With regard to 3D Gaussian Splatting, CF-3DGS~\cite{CF-3DGS-2024} also leverages mono-depth information to constrain the optimization of local 3DGS for relative pose estimation and later learn a global 3DGS progressively in a sequential manner.
% InstantSplat~\cite{fan2024instantsplat} focus on sparse view scenes, first use DUSt3R~\cite{dust3r2024cvpr} to generate a set of densely covered and pixel-aligned points for 3D Gaussian initialization, then introduce a parallel grid partitioning strategy in joint optimization to speed up.
% % Jiang et al.~\cite{Jiang_2024sig} proposed to build the scene continuously and progressively, to next unregistered frame, they use registration and adjustment to adjust the previous registered camera poses and align unregistered monocular depths, later refine the joint model by matching detected correspondences in screen-space coordinates.
% \gjh{Jiang et al.~\cite{Jiang_2024sig} also implemented an incremental approach for reconstructing camera poses and scenes. Initially, they perform feature matching between the current image and the image rendered by a differentiable surface renderer. They then construct matching point errors, depth errors, and photometric errors to achieve the registration and adjustment of the current image. Finally, based on the depth map, the pixels of the current image are projected as new 3D Gaussians. However, this method still exhibits limitations when dealing with complex scenes and unordered images.}
% % CG-3DGS~\cite{sun2024correspondenceguidedsfmfree3dgaussian} follows CF-3DGS, first construct a coarse point cloud from mono-depth maps to train a 3DGS model, then progressively estimate camera poses based on this pre-trained model by constraining the correspondences between rendering view and ground-truth.
% \gjh{Similarly, CG-3DGS~\cite{sun2024correspondenceguidedsfmfree3dgaussian} first utilizes monocular depth estimation and the camera parameters from the first frame to initialize a set of 3D Gaussians. It then progressively estimates camera poses based on this pre-trained model by constraining the correspondences between the rendered views and the ground truth.}
% % Free-SurGS~\cite{freesurgs2024} matches the projection flow derived from 3D Gaussians with optical flow to estimate the poses, to compensate for the limitations of photometric loss.
% \gjh{Free-SurGS~\cite{freesurgs2024} introduces the first SfM-free 3DGS approach for surgical scene reconstruction. Due to the challenges posed by weak textures and photometric inconsistencies in surgical scenes, Free-SurGS achieves pose estimation by minimizing the flow loss between the projection flow and the optical flow. Subsequently, it keeps the camera pose fixed and optimizes the scene representation by minimizing the photometric loss, depth loss and flow loss.}
% \gjh{However, most current works assume camera intrinsics are known and primarily focus on optimizing camera poses. Additionally, these methods typically rely on sequentially ordered image inputs and incrementally optimize camera parameters and scene representation. This inevitably leads to drift errors, preventing the achievement of globally consistent results. Our work aims to address these issues.}

Regarding 3D Gaussian Splatting, CF-3DGS~\cite{CF-3DGS-2024} utilizes mono-depth information to refine the optimization of local 3DGS for relative pose estimation and subsequently learns a global 3DGS in a sequential manner. InstantSplat~\cite{fan2024instantsplat} targets sparse view scenes, initially employing DUSt3R~\cite{dust3r2024cvpr} to create a densely covered, pixel-aligned point set for initializing 3D Gaussian models, and then implements a parallel grid partitioning strategy to accelerate joint optimization. Jiang \etal~\cite{Jiang_2024sig} develops an incremental method for reconstructing camera poses and scenes, but it struggles with complex scenes and unordered images. 
% Similarly, CG-3DGS~\cite{sun2024correspondenceguidedsfmfree3dgaussian} progressively estimates camera poses using a pre-trained model by aligning the correspondences between rendered views and actual scenes. Free-SurGS~\cite{freesurgs2024} pioneers an SfM-free 3DGS method for reconstructing surgical scenes, overcoming challenges such as weak textures and photometric inconsistencies by minimizing the discrepancy between projection flow and optical flow.
%\pb{SF-3DGS-HT~\cite{ji2024sfmfree3dgaussiansplatting} introduced VFI into training as additional photometric constraints. They separated the whole scene into several local 3DGS models and then merged them hierarchically, which leads to a significant improvement on simple and dense view scenes.}
HT-3DGS~\cite{ji2024sfmfree3dgaussiansplatting} interpolates frames for training and splits the scene into local clips, using a hierarchical strategy to build 3DGS model. It works well for simple scenes, but fails with dramatic motions due to unstable interpolation and low efficiency.
% {While effective for simple scenes, it struggles with dramatic motion due to unstable view interpolation and suffers from low computational efficiency.}

However, most existing methods generally depend on sequentially ordered image inputs and incrementally optimize camera parameters and 3DGS, which often leads to drift errors and hinders achieving globally consistent results. Our work seeks to overcome these limitations.

\section{Study Design}
% robot: aliengo 
% We used the Unitree AlienGo quadruped robot. 
% See Appendix 1 in AlienGo Software Guide PDF
% Weight = 25kg, size (L,W,H) = (0.55, 0.35, 06) m when standing, (0.55, 0.35, 0.31) m when walking
% Handle is 0.4 m or 0.5 m. I'll need to check it to see which type it is.
We gathered input from primary stakeholders of the robot dog guide, divided into three subgroups: BVI individuals who have owned a dog guide, BVI individuals who were not dog guide owners, and sighted individuals with generally low degrees of familiarity with dog guides. While the main focus of this study was on the BVI participants, we elected to include survey responses from sighted participants given the importance of social acceptance of the robot by the general public, which could reflect upon the BVI users themselves and affect their interactions with the general population \cite{kayukawa2022perceive}. 

The need-finding processes consisted of two stages. During Stage 1, we conducted in-depth interviews with BVI participants, querying their experiences in using conventional assistive technologies and dog guides. During Stage 2, a large-scale survey was distributed to both BVI and sighted participants. 

This study was approved by the University’s Institutional Review Board (IRB), and all processes were conducted after obtaining the participants' consent.

\subsection{Stage 1: Interviews}
We recruited nine BVI participants (\textbf{Table}~\ref{tab:bvi-info}) for in-depth interviews, which lasted 45-90 minutes for current or former dog guide owners (DO) and 30-60 minutes for participants without dog guides (NDO). Group DO consisted of five participants, while Group NDO consisted of four participants.
% The interview participants were divided into two groups. Group DO (Dog guide Owner) consisted of five participants who were current or former dog guide owners and Group NDO (Non Dog guide Owner) consisted of three participants who were not dog guide owners. 
All participants were familiar with using white canes as a mobility aid. 

We recruited participants in both groups, DO and NDO, to gather data from those with substantial experience with dog guides, offering potentially more practical insights, and from those without prior experience, providing a perspective that may be less constrained and more open to novel approaches. 

We asked about the participants' overall impressions of a robot dog guide, expectations regarding its potential benefits and challenges compared to a conventional dog guide, their desired methods of giving commands and communicating with the robot dog guide, essential functionalities that the robot dog guide should offer, and their preferences for various aspects of the robot dog guide's form factors. 
For Group DO, we also included questions that asked about the participants' experiences with conventional dog guides. 

% We obtained permission to record the conversations for our records while simultaneously taking notes during the interviews. The interviews lasted 30-60 minutes for NDO participants and 45-90 minutes for DO participants. 

\subsection{Stage 2: Large-Scale Surveys} 
After gathering sufficient initial results from the interviews, we created an online survey for distributing to a larger pool of participants. The survey platform used was Qualtrics. 

\subsubsection{Survey Participants}
The survey had 100 participants divided into two primary groups. Group BVI consisted of 42 blind or visually impaired participants, and Group ST consisted of 58 sighted participants. \textbf{Table}~\ref{tab:survey-demographics} shows the demographic information of the survey participants. 

\subsubsection{Question Differentiation} 
Based on their responses to initial qualifying questions, survey participants were sorted into three subgroups: DO, NDO, and ST. Each participant was assigned one of three different versions of the survey. The surveys for BVI participants mirrored the interview categories (overall impressions, communication methods, functionalities, and form factors), but with a more quantitative approach rather than the open-ended questions used in interviews. The DO version included additional questions pertaining to their prior experience with dog guides. The ST version revolved around the participants' prior interactions with and feelings toward dog guides and dogs in general, their thoughts on a robot dog guide, and broad opinions on the aesthetic component of the robot's design. 

\section{Experimental Results}
In this section, we present the main results in~\secref{sec:main}, followed by ablation studies on key design choices in~\secref{sec:ablation}.

\begin{table*}[t]
\renewcommand\arraystretch{1.05}
\centering
\setlength{\tabcolsep}{2.5mm}{}
\begin{tabular}{l|l|c|cc|cc}
type & model     & \#params      & FID$\downarrow$ & IS$\uparrow$ & Precision$\uparrow$ & Recall$\uparrow$ \\
\shline
GAN& BigGAN~\cite{biggan} & 112M & 6.95  & 224.5       & 0.89 & 0.38     \\
GAN& GigaGAN~\cite{gigagan}  & 569M      & 3.45  & 225.5       & 0.84 & 0.61\\  
GAN& StyleGan-XL~\cite{stylegan-xl} & 166M & 2.30  & 265.1       & 0.78 & 0.53  \\
\hline
Diffusion& ADM~\cite{adm}    & 554M      & 10.94 & 101.0        & 0.69 & 0.63\\
Diffusion& LDM-4-G~\cite{ldm}   & 400M  & 3.60  & 247.7       & -  & -     \\
Diffusion & Simple-Diffusion~\cite{diff1} & 2B & 2.44 & 256.3 & - & - \\
Diffusion& DiT-XL/2~\cite{dit} & 675M     & 2.27  & 278.2       & 0.83 & 0.57     \\
Diffusion&L-DiT-3B~\cite{dit-github}  & 3.0B    & 2.10  & 304.4       & 0.82 & 0.60    \\
Diffusion&DiMR-G/2R~\cite{liu2024alleviating} &1.1B& 1.63& 292.5& 0.79 &0.63 \\
Diffusion & MDTv2-XL/2~\cite{gao2023mdtv2} & 676M & 1.58 & 314.7 & 0.79 & 0.65\\
Diffusion & CausalFusion-H$^\dag$~\cite{deng2024causal} & 1B & 1.57 & - & - & - \\
\hline
Flow-Matching & SiT-XL/2~\cite{sit} & 675M & 2.06 & 277.5 & 0.83 & 0.59 \\
Flow-Matching&REPA~\cite{yu2024representation} &675M& 1.80 & 284.0 &0.81 &0.61\\    
Flow-Matching&REPA$^\dag$~\cite{yu2024representation}& 675M& 1.42&  305.7& 0.80& 0.65 \\
\hline
Mask.& MaskGIT~\cite{maskgit}  & 227M   & 6.18  & 182.1        & 0.80 & 0.51 \\
Mask. & TiTok-S-128~\cite{yu2024image} & 287M & 1.97 & 281.8 & - & - \\
Mask. & MAGVIT-v2~\cite{yu2024language} & 307M & 1.78 & 319.4 & - & - \\ 
Mask. & MaskBit~\cite{weber2024maskbit} & 305M & 1.52 & 328.6 & - & - \\
\hline
AR& VQVAE-2~\cite{vqvae2} & 13.5B    & 31.11           & $\sim$45     & 0.36           & 0.57          \\
AR& VQGAN~\cite{vqgan}& 227M  & 18.65 & 80.4         & 0.78 & 0.26   \\
AR& VQGAN~\cite{vqgan}   & 1.4B     & 15.78 & 74.3   & -  & -     \\
AR&RQTran.~\cite{rq}     & 3.8B    & 7.55  & 134.0  & -  & -    \\
AR& ViTVQ~\cite{vit-vqgan} & 1.7B  & 4.17  & 175.1  & -  & -    \\
AR & DART-AR~\cite{gu2025dart} & 812M & 3.98 & 256.8 & - & - \\
AR & MonoFormer~\cite{zhao2024monoformer} & 1.1B & 2.57 & 272.6 & 0.84 & 0.56\\
AR & Open-MAGVIT2-XL~\cite{luo2024open} & 1.5B & 2.33 & 271.8 & 0.84 & 0.54\\
AR&LlamaGen-3B~\cite{llamagen}  &3.1B& 2.18& 263.3 &0.81& 0.58\\
AR & FlowAR-H~\cite{flowar} & 1.9B & 1.65 & 296.5 & 0.83 & 0.60\\
AR & RAR-XXL~\cite{yu2024randomized} & 1.5B & 1.48 & 326.0 & 0.80 & 0.63 \\
\hline
MAR & MAR-B~\cite{mar} & 208M & 2.31 &281.7 &0.82 &0.57 \\
MAR & MAR-L~\cite{mar} &479M& 1.78 &296.0& 0.81& 0.60 \\
MAR & MAR-H~\cite{mar} & 943M&1.55& 303.7& 0.81 &0.62 \\
\hline
VAR&VAR-$d16$~\cite{var}   & 310M  & 3.30& 274.4& 0.84& 0.51    \\
VAR&VAR-$d20$~\cite{var}   &600M & 2.57& 302.6& 0.83& 0.56     \\
VAR&VAR-$d30$~\cite{var}   & 2.0B      & 1.97  & 323.1 & 0.82 & 0.59      \\
\hline
\modelname& \modelname-B    &172M   &1.72&280.4&0.82&0.59 \\
\modelname& \modelname-L   & 608M   & 1.28& 292.5&0.82&0.62\\
\modelname& \modelname-H    & 1.1B    & 1.24 &301.6&0.83&0.64\\
\end{tabular}
\caption{
\textbf{Generation Results on ImageNet-256.}
Metrics include Fréchet Inception Distance (FID), Inception Score (IS), Precision, and Recall. $^\dag$ denotes the use of guidance interval sampling~\cite{guidance}. The proposed \modelname-H achieves a state-of-the-art 1.24 FID on the ImageNet-256 benchmark without relying on vision foundation models (\eg, DINOv2~\cite{dinov2}) or guidance interval sampling~\cite{guidance}, as used in REPA~\cite{yu2024representation}.
}\label{tab:256}
\end{table*}

\subsection{Main Results}
\label{sec:main}
We conduct experiments on ImageNet~\cite{deng2009imagenet} at 256$\times$256 and 512$\times$512 resolutions. Following prior works~\cite{dit,mar}, we evaluate model performance using FID~\cite{fid}, Inception Score (IS)~\cite{is}, Precision, and Recall. \modelname is trained with the same hyper-parameters as~\cite{mar,dit} (\eg, 800 training epochs), with model sizes ranging from 172M to 1.1B parameters. See Appendix~\secref{sec:sup_hyper} for hyper-parameter details.





\begin{table}[t]
    \centering
    \begin{tabular}{c|c|c|c}
      model    &  \#params & FID$\downarrow$ & IS$\uparrow$ \\
      \shline
      VQGAN~\cite{vqgan}&227M &26.52& 66.8\\
      BigGAN~\cite{biggan}& 158M&8.43 &177.9\\
      MaskGiT~\cite{maskgit}& 227M&7.32& 156.0\\
      DiT-XL/2~\cite{dit} &675M &3.04& 240.8 \\
     DiMR-XL/3R~\cite{liu2024alleviating}& 525M&2.89 &289.8 \\
     VAR-d36~\cite{var}  & 2.3B& 2.63 & 303.2\\
     REPA$^\ddagger$~\cite{yu2024representation}&675M &2.08& 274.6 \\
     \hline
     \modelname-L & 608M&1.70& 281.5 \\
    \end{tabular}
    \caption{
    \textbf{Generation Results on ImageNet-512.} $^\ddagger$ denotes the use of DINOv2~\cite{dinov2}.
    }
    \label{tab:512}
\end{table}

\noindent\textbf{ImageNet-256.}
In~\tabref{tab:256}, we compare \modelname with previous state-of-the-art generative models.
Out best variant, \modelname-H, achieves a new state-of-the-art-performance of 1.24 FID, outperforming the GAN-based StyleGAN-XL~\cite{stylegan-xl} by 1.06 FID, masked-prediction-based MaskBit~\cite{maskgit} by 0.28 FID, AR-based RAR~\cite{yu2024randomized} by 0.24 FID, VAR~\cite{var} by 0.73 FID, MAR~\cite{mar} by 0.31 FID, and flow-matching-based REPA~\cite{yu2024representation} by 0.18 FID.
Notably, \modelname does not rely on vision foundation models~\cite{dinov2} or guidance interval sampling~\cite{guidance}, both of which were used in REPA~\cite{yu2024representation}, the previous best-performing model.
Additionally, our lightweight \modelname-B (172M), surpasses DiT-XL (675M)~\cite{dit} by 0.55 FID while achieving an inference speed of 9.8 images per second—20$\times$ faster than DiT-XL (0.5 images per second). Detailed speed comparison can be found in Appendix \ref{sec:speed}.



\noindent\textbf{ImageNet-512.}
In~\tabref{tab:512}, we report the performance of \modelname on ImageNet-512.
Similarly, \modelname-L sets a new state-of-the-art FID of 1.70, outperforming the diffusion based DiT-XL/2~\cite{dit} and DiMR-XL/3R~\cite{liu2024alleviating} by a large margin of 1.34 and 1.19 FID, respectively.
Additionally, \modelname-L also surpasses the previous best autoregressive model VAR-d36~\cite{var} and flow-matching-based REPA~\cite{yu2024representation} by 0.93 and 0.38 FID, respectively.




\noindent\textbf{Qualitative Results.}
\figref{fig:qualitative} presents samples generated by \modelname (trained on ImageNet) at 512$\times$512 and 256$\times$256 resolutions. These results highlight \modelname's ability to produce high-fidelity images with exceptional visual quality.

\begin{figure*}
    \centering
    \vspace{-6pt}
    \includegraphics[width=1\linewidth]{figures/qualitative.pdf}
    \caption{\textbf{Generated Samples.} \modelname generates high-quality images at resolutions of 512$\times$512 (1st row) and 256$\times$256 (2nd and 3rd row).
    }
    \label{fig:qualitative}
\end{figure*}

\subsection{Ablation Studies}
\label{sec:ablation}
In this section, we conduct ablation studies using \modelname-B, trained for 400 epochs to efficiently iterate on model design.

\noindent\textbf{Prediction Entity X.}
The proposed \modelname extends next-token prediction to next-X prediction. In~\tabref{tab:X}, we evaluate different designs for the prediction entity X, including an individual patch token, a cell (a group of surrounding tokens), a subsample (a non-local grouping), a scale (coarse-to-fine resolution), and an entire image.

Among these variants, cell-based \modelname achieves the best performance, with an FID of 2.48, outperforming the token-based \modelname by 1.03 FID and surpassing the second best design (scale-based \modelname) by 0.42 FID. Furthermore, even when using standard prediction entities such as tokens, subsamples, images, or scales, \modelname consistently outperforms existing methods while requiring significantly fewer parameters. These results highlight the efficiency and effectiveness of \modelname across diverse prediction entities.






\begin{table}[]
    \centering
    \scalebox{0.92}{
    \begin{tabular}{c|c|c|c|c}
        model & \makecell[c]{prediction\\entity} & \#params & FID$\downarrow$ & IS$\uparrow$\\
        \shline
        LlamaGen-L~\cite{llamagen} & \multirow{2}{*}{token} & 343M & 3.80 &248.3\\
        \modelname-B& & 172M&3.51&251.4\\
        \hline
        PAR-L~\cite{par} & \multirow{2}{*}{subsample}& 343M & 3.76 & 218.9\\
        \modelname-B&  &172M& 3.58&231.5\\
        \hline
        DiT-L/2~\cite{dit}& \multirow{2}{*}{image}& 458M&5.02&167.2 \\
         \modelname-B& & 172M&3.13&253.4 \\
        \hline
        VAR-$d16$~\cite{var} & \multirow{2}{*}{scale} & 310M&3.30 &274.4\\
        \modelname-B& &172M&2.90&262.8\\
        \hline
        \baseline{\modelname-B}& \baseline{cell} & \baseline{172M}&\baseline{2.48}&\baseline{269.2} \\
    \end{tabular}
    }
    \caption{\textbf{Ablation on Prediction Entity X.} Using cells as the prediction entity outperforms alternatives such as tokens or entire images. Additionally, under the same prediction entity, \modelname surpasses previous methods, demonstrating its effectiveness across different prediction granularities. }%
    \label{tab:X}
\end{table}

\noindent\textbf{Cell Size.}
A prediction entity cell is formed by grouping spatially adjacent $k\times k$ tokens, where a larger cell size incorporates more tokens and thus captures a broader context within a single prediction step.
For a $256\times256$ input image, the encoded continuous latent representation has a spatial resolution of $16\times16$. Given this, the image can be partitioned into an $m\times m$ grid, where each cell consists of $k\times k$ neighboring tokens. As shown in~\tabref{tab:cell}, we evaluate different cell sizes with $k \in \{1,2,4,8,16\}$, where $k=1$ represents a single token and $k=16$ corresponds to the entire image as a single entity. We observe that performance improves as $k$ increases, peaking at an FID of 2.48 when using cell size $8\times8$ (\ie, $k=8$). Beyond this, performance declines, reaching an FID of 3.13 when the entire image is treated as a single entity.
These results suggest that using cells rather than the entire image as the prediction unit allows the model to condition on previously generated context, improving confidence in predictions while maintaining both rich semantics and local details.





\begin{table}[t]
    \centering
    \scalebox{0.98}{
    \begin{tabular}{c|c|c|c}
    cell size ($k\times k$ tokens) & $m\times m$ grid & FID$\downarrow$ & IS$\uparrow$ \\
       \shline
       $1\times1$ & $16\times16$ &3.51&251.4 \\
       $2\times2$ & $8\times8$ & 3.04& 253.5\\
       $4\times4$ & $4\times4$ & 2.61&258.2 \\
       \baseline{$8\times8$} & \baseline{$2\times2$} & \baseline{2.48} & \baseline{269.2}\\
       $16\times16$ & $1\times1$ & 3.13&253.4  \\
    \end{tabular}
    }
    \caption{\textbf{Ablation on the cell size.}
    In this study, a $16\times16$ continuous latent representation is partitioned into an $m\times m$ grid, where each cell consits of $k\times k$ neighboring tokens.
    A cell size of $8\times8$ achieves the best performance, striking an optimal balance between local structure and global context.
    }
    \label{tab:cell}
\end{table}



\begin{table}[t]
    \centering
    \scalebox{0.95}{
    \begin{tabular}{c|c|c|c}
      previous cell & noise time step &  FID$\downarrow$ & IS$\uparrow$ \\
       \shline
       clean & $t_i=0, \forall i<n$& 3.45& 243.5\\
       increasing noise & $t_1<t_2<\cdots<t_{n-1}$& 2.95&258.8 \\
       decreasing noise & $t_1>t_2>\cdots>t_{n-1}$&2.78 &262.1 \\
      \baseline{random noise}  & \baseline{no constraint} &\baseline{2.48} & \baseline{269.2}\\
    \end{tabular}
    }
    \caption{
    \textbf{Ablation on Noisy Context Learning.}
    This study examines the impact of noise time steps ($t_1, \cdots, t_{n-1} \subset [0, 1]$) in previous entities ($t=0$ represents pure Gaussian noise).
    Conditioning on all clean entities (the ``clean'' variant) results in suboptimal performance.
    Imposing an order on noise time steps, either ``increasing noise'' or ``decreasing noise'', also leads to inferior results. The best performance is achieved with the "random noise" setting, where no constraints are imposed on noise time steps.
    }
    \label{tab:ncl}
\end{table}


\noindent\textbf{Noisy Context Learning.}
During training, \modelname employs Noisy Context Learning (NCL), predicting $X_n$ by conditioning on all previous noisy entities, unlike Teacher Forcing.
The noise intensity of previous entities is contorlled by noise time steps $\{t_1, \dots, t_{n-1}\} \subset [0, 1]$, where $t=0$ corresponds to pure Gaussian noise.
We analyze the impact of NCL in~\tabref{tab:ncl}.
When conditioning on all clean entities (\ie, the ``clean'' variant, where $t_i=0, \forall i<n$), which is equivalent to vanilla AR (\ie, Teacher Forcing), the suboptimal performance is obtained.
We also evaluate two constrained noise schedules: the ``increasing noise'' variant, where noise time steps increase over AR steps ($t_1<t_2< \cdots < t_{n-1}$), and the `` decreasing noise'' variant, where noise time steps decrease ($t_1>t_2> \cdots > t_{n-1}$).
While both settings improve over the ``clean'' variant, they remain inferior to our final ``random noise'' setting, where no constraints are imposed on noise time steps, leading to the best performance.




        


\section{Discussion and Conclusion}

% \begin{quote}
% \textit{"We believe it is unethical for social workers not to learn... about technology-mediated social work."} (\citeauthor{singer_ai_2023}, 2023)
% \end{quote}

In this study, we uncovered multiple ways in which GenAI can be used in social service practice. While some concerns did arise, practitioners by and large seemed optimistic about the possibilities of such tools, and that these issues could be overcome. We note that while most participants found the tool useful, it was far from perfect in its outputs. This is not surprising, since it was powered by a generic LLM rather than one fine-tuned for social service case management. However, despite these inadequacies, our participants still found many uses for most of the tool's outputs. Many flaws pointed out by our participants related to highly contextualised, local knowledge. To tune an AI system for this would require large amounts of case files as training data; given the privacy concerns associated with using client data, this seems unlikely to happen in the near future. What our study shows, however, is that GenAI systems need not aim to be perfect to be useful to social service practitioners, and can instead serve as a complement to the critical "human touch" in social service.

We draw both inspiration and comparisons with prior work on AI in other settings. Studies on creative writing tools showed how the "uncertainty" \cite{wan2024felt} and "randomness" \cite{clark2018creative} of AI outputs aid creativity. Given the promise that our tool shows in aiding brainstorming and discussion, future social service studies could consider AI tools explicitly geared towards creativity - for instance, providing side-by-side displays of how a given case would fit into different theoretical frameworks, prompting users to compare, contrast, and adopt the best of each framework; or allowing users to play around with combining different intervention modalities to generate eclectic (i.e. multi-modal) interventions.

At the same time, the concept of supervision creates a different interaction paradigm to other uses of AI in brainstorming. Past work (e.g. \cite{shaer2024ai}) has explored the use of GenAI for ideation during brainstorming sessions, wherein all users present discuss the ideas generated by the system. With supervision in social service practice, however, there is a marked information and role asymmetry: supervisors may not have had the time to fully read up on their supervisee's case beforehand, yet have to provide guidance and help to the latter. We suggest that GenAI can serve a dual purpose of bringing supervisors up to speed quickly by summarising their supervisee's case data, while simultaneously generating a list of discussion and talking points that can improve the quality of supervision. Generalising, this interaction paradigm has promise in many other areas: senior doctors reviewing medical procedures with newer ones \cite{snowdon2017does} could use GenAI to generate questions about critical parts of a procedure to ask the latter, confirming they have been correctly understood or executed; game studio directors could quickly summarise key developmental pipeline concerns to raise at meetings and ensure the team is on track; even in academia, advisors involved in rather too many projects to keep track of could quickly summarise each graduate student's projects and identify potential concerns to address at their next meeting.

In closing, we are optimistic about the potential for GenAI to significantly enhance social service practice and the quality of care to clients. Future studies could focus on 1) longitudinal investigations into the long-term impact of GenAI on practitioner skills, client outcomes, and organisational workflows, and 2) optimising workflows to best integrate GenAI into casework and supervision, understanding where best to harness the speed and creativity of such systems in harmony with the experience and skills of practitioners at all levels.

% GenAI here thus serves as a tool that supervisors can use before rather then using the session, taking just a few minutes of their time to generate a list of discussion points with their supervisees.

% Traditional brainstorming comes up with new things that users discuss. In supervision, supervisors can use AI to more efficiently generate talking points with their supervisees. These are generally not novel ideas, since an experienced worker would be able to come up with these on their own. However, the interesting and novel use of AI here is in its use as a preparation tool, efficiently generating talking and discussion points, saving supervisors' time in preparing for a session, while still serving as a brainstorming tool during the session itself.

% The idea of embracing imperfect AI echoes the findings of \citeauthor{bossen2023batman} (2023) in a clinical decision setting, which examined the successful implementation of an "error-prone but useful AI tool". This study frames human-AI collaboration as "Batman and Robin", where AI is a useful but ultimately less skilled sidekick that plays second fiddle to Batman. This is similar to \citeauthor{yang2019unremarkable}'s (2019) idea of "unremarkable AI", systems designed to be unobtrusive and only visible to the user when they add some value. As compared to \citeauthor{bossen2023batman}, however, we see fewer instances of our AI system producing errors, and more examples of it providing learning and collaborative opportunities and other new use cases. We build on the idea of "complementary performance" \cite{bansal2021does}, which discusses how the unique expertise of AI enhances human decision-making performance beyond what humans can achieve alone. Beyond decision-making, GenAI can now enable "complementary work patterns", where the nature of its outputs enables humans to carry out their work in entirely new ways. Our study suggests that rather being a sidekick - Robin - AI is growing into the role of a "second Batman" or "AI-Batman": an entity with distinct abilities and expertise from humans, and that contributes in its own unique way. There is certainly still a time and place for unremarkable AI, but exploring uses beyond that paradigm uncovers entirely new areas of system design.

% % \cite{gero2022sparks} found AI to be useful for science writers to translate ideas already in their head into words, and to provide new perspectives to spark further inspiration. \textit{But how is ours different from theirs?}

% \subsection{New Avenues of Human-AI Collaboration}
% \label{subsubsec:discussionhaicollaboration}

% Past HCI literature in other areas \cite{nah2023generative} has suggested that GenAI represents a "leap" \cite{singh2023hide} in human-AI collaboration, 
% % Even when an AI system sometimes produces irrelevant outputs, it can still provide users 
% % Such systems have been proposed as ways to 
% helping users discover new viewpoints \cite{singh2023hide}, scour existing literature to suggest new hypotheses 
% \cite{cascella2023evaluating} and answer questions \cite{biswas2023role}, stimulate their cognitive processes \cite{memmert2023towards}, and overcome "writer's block" \cite{singh2023hide, cooper2023examining} (particularly relevant to SSPs and the vast amount of writing required of them). Our study finds promise for AI to help SSPs in all of these areas. By nature of being more verbose and capable of generating large amounts of content, GenAI seems to create a new way in which AI can complement human work and expertise. Our system, as LLMs tend to do, produced a lot of "bullshit" (S6) \cite{frankfurt2005bullshit} - superficially true statements that were often only "tangentially related" and "devoid of meaning" \cite{halloran2023ai}. Yet, many participants cited the page-long analyses and detailed multi-step intervention plans generated by the AI system to be a good starting point for further discussion, both to better conceptualize a particular case and to facilitate general worker growth and development. Almost like throwing mud at a wall to see what sticks, GenAI can quickly produce a long list of ideas or information, before the worker glances through it and quickly identifies the more interesting points to discuss. Playing the proposed role as a "scaffold" for further work \cite{cooper2023examining}, GenAI, literally, generates new opportunities for novel and more effective processes and perspectives that previous systems (e.g., PRMs) could not. This represents an entirely new mode of human-AI collaboration.
% % This represents a new mode of collaboration not possible with the largely quantitative AI models (like PRMs) of the past.

% Our work therefore supports and extends prior research that have postulated the the potential of AI's shifting roles from decision-maker to human-supporter \cite{wang_human-human_2020}. \citeauthor{siemon2022elaborating} (2022) suggests the role of AI as a "creator" or "coordinator", rather than merely providing "process guidance" \cite{memmert2023towards} that does not contribute to brainstorming. Similarly, \citeauthor{memmert2023towards} (2023) propose GenAI as a step forward from providing meta-level process guidance (i.e. facilitating user tasks) to actively contributing content and aiding brainstorming. We suggest that beyond content-support, AI can even create new work processes that were not possible without GenAI. In this sense, AI has come full circle, becoming a "meta-facilitator".

% % --- WIP BELOW ---

% % Our work echoes and extends previous research on HAI collaboration in tasks requiring a human touch. \cite{gero2023social} found AI to be a safe space for creative writers to bounce ideas off of and document their inner thoughts. \cite{dhillon2024shaping} reference the idea of appropriate scaffolding in argumentative writing, where the user is providing with guidance appropriate for their competency level, and also warns of decreased satisfaction and ownership from AI use. 

% Separately, we draw parallels with the field of creative writing, where HAI collaboration has been extensively researched. Writers note the "irreducibly human" aspect of creativity in writing \cite{gero2023social}, similar to the "human touch" core to social service practice (D1); both groups therefore expressed few concerns about AI taking over core aspects of their jobs. Another interesting parallel was how writers often appreciated the "uncertainty" \cite{wan2024felt} and "randomness" \cite{clark2018creative} of AI systems, which served as a source of inspiration. This echoes the idea of "imperfect AI" "expanding [the] perspective[s]" (S4) of our participants when they simply skimmed through what the AI produced. \cite{wan2024felt} cited how the "duality of uncertainty in the creativity process advances the exploration of the imperfection of GenAI models". While social service work is not typically regarded as "creative", practitioners nonetheless go through processes of ideation and iteration while formulating a case. Our study showed hints of how AI can help with various forms of ideation, but, drawing inspiration from creative writing tools, future studies could consider designs more explicitly geared towards creativity - for instance, by attempting to fit a given case into a number of different theories or modalities, and displaying them together for the user to consider. While many of these assessments may be imperfect or even downnright nonsensical, they may contain valuable ideas and new angles on viewing the case that the practitioner can integrate into their own assessment.

% % \cite{foong2024designing}, describing the design of caregiver-facing values elicitation tools, cites the "twin scenarios" that caregivers face - private use, where they might use a tool to discover their patient's values, and collaborative use, where they discuss the resulting values with other parties close to the patient. This closely mirrors how SSPs in our study reference both individual and collaborative uses of our tool. Unlike in \cite{foong2024designing}, however, we do not see a resulting need to design a "staged approach" with distinct interface features for both stages.

% % --- END OF WIP ---

% Having mentioned algorithm aversion previously, we also make a quick point here on the other end of the spectrum - automation bias, or blind trust in an automated system \cite{brown2019toward}. LLMs risk being perceived as an "ultimate epistemic authority" \cite{cooper2023examining} due to their detailed, life-like outputs. While automation bias has been studied in many contexts, including in the social sector or adjacent areas, we suggest that the very nature of GenAI systems fundamentally inhibits automation bias. The tendency of GenAI to produce verbose, lengthy explanations prompts users to read and think through the machine's judgement before accepting it, bringing up opportunities to disagree with the machine's opinion. This guards against blind acceptance of the system's recommendations, particularly in the culture of a social work agency where constant dialogue - including discussing AI-produced work - is the norm.


% % : Perception of AI in Social Service Work ??

% \subsection{Redefining the Boundary}
% \label{subsubsec:discussiontheoretical}

% As \citeauthor{meilvang_working_2023} (2023) describes, the social service profession has sought to distance itself from comprising mostly "administrative work" \cite{abbott2016boundaries}, and workers have long tried to tried to reduce their considerable time \cite{socialraadgiverforening2010notat} spent on such tasks in favour of actual casework with clients \cite{toren1972social}. Our study, however, suggests a blurring of the line between "manual" administrative tasks and "mental" casework that draws on practitioner expertise. Many tasks our participants cited involve elements of both: for instance, documenting a case recording requires selecting only the relevant information to include, and planning an intervention can be an iterative process of drafting a plan and discussing it with colleagues and superiors. This all stems from the fact that GenAI can produce virtually any document required by the user, but this document almost always requires revision under a watchful human eye.

% \citeauthor{meilvang_working_2023} (2023) also describes a more recent shift in the perceived accepted boundary of AI interventions in social service work. From "defending [the] jurisdiction [of social service work] against artificial intelligence" in the early days of PRM and other statistical assessment tools, the community has started to embrace AI as a "decision-support ... element in the assessment process". Our study concurs and frames GenAI as a source of information that can be used to support and qualify the assessments of SSPs \cite{meilvang_working_2023}, but suggests that we can take a step further: AI can be viewed as a \textit{facilitator} rather than just a supporter. GenAI can facilitate a wide range of discussions that promote efficiency, encourage worker learning and growth, and ultimately enhance client outcomes. This entails a much larger scope of AI use, where practitioners use the information provided by AI in a range of new scenarios. 

% Taken together, these suggest a new focus for boundary work and, more broadly, HCI research. GAI can play a role not just in menial documentation or decision-support, but can be deeply ingrained into every facet of the social service workflow to open new opportunities for worker growth, workflow optimisation, and ultimately improved client outcomes. Future research can therefore investigate the deeper, organisational-level effects of these new uses of AI, and their resulting impact on the role of profession discretion in effective social service work.

% % MH: oh i feel this paragraph is quite new to me! Could we elaborate this more, and truncate the first two paragraphs a bit to adjust the word propotion?


% % Our study extensively documents this for the first time in social service practice, and in the process reveals new insights about how AI can play such a role.



% \subsection{Design Implications}

% % Add link from ACE diagram?

% % EJ: it would be interesting to discuss how LLMs could help "hands-on experience" in the discussion section

% Addressing the struggle of integrating AI amidst the tension between machine assessment and expert judgement, we reframe AI as an \textit{facilitator} rather than an algorithm or decision-support tool, alleviating many concerns about trust and explainablity. We now present a high-level framework (Figure \ref{fig:hai-collaboration}) on human-AI collaboration, presenting a new perspective on designing effective AI systems that can be applied to both the social service sector and beyond.

% \begin{figure}
%     \centering
%     \includegraphics[scale=0.15]{images/designframework.png}
%     \caption{Framework for Human-AI Collaboration}
%     \label{fig:hai-collaboration}
%     \Description{An image showing our framework for Human-AI Collaboration. It shows that as stakeholder level increases from junior to senior, the directness of use shifts from co-creation to provision.}
% \end{figure}
% % MH: so this paradigm is proposed by us? I wonder if this could a part of results as well..?

% % \subsubsection{From Creation to Provision}

% In Section \ref{sec:stage2findings}, we uncovered the different ways in which SSPs of varying seniorities use, evaluate, and suggest uses of AI. These are intrinsically tied to the perspectives and levels of expertise that each stakeholder possesses. We therefore position the role of AI along the scale of \textit{creation} to \textit{provision}. 

% With junior workers, we recommend \textbf{designing tools for co-creation}: systems that aid the least experienced workers in creating the required deliverables for their work. Rather than \textit{telling} workers what to do - a difficult task in any case given the complexity of social work solutions - AI systems should instead \textit{co-create} deliverables required of these workers. These encompass the multitude of use cases that junior workers found useful: creating reports, suggesting perspectives from which to formulate a case, and providing a starting template for possible intervention plans. Notably, since AI outputs are not perfect, we emphasise the "co" in "co-creation": AI should only be a part of the workflow that also includes active engagement on the part of the SSPs and proactive discussion with supervisors. 

% For more experienced SSPs, we recommend \textbf{designing tools for provision}. Again, this is not the mere provision of recommendations or courses of action with clients, but rather that of resources which complement the needs of workers with greater responsibilities. This notably includes supplying materials to aid with supervision, a novel use case that to our knowledge has not surfaced in previous literature. In addition, senior workers also benefit greatly from manual tasks such as routine report writing and data processing. Since these workers are more experienced and can better spot inaccuracies in AI output, we suggest that AI can "provide" a more finished product that requires less vetting and corrections, and which can be used more directly as part of required deliverables.

% % MH: can we seperate here? above is about the guidance to paradigm, below is the practical roadmap for implementation
% In terms of concrete design features, given the constant focus on discussing AI outputs between colleagues in our FGDs, we recommend that AI tools, particularly those for junior workers, \textbf{include collaborative features} that facilitate feedback and idea sharing between users. We also suggest that designers work closely with domain experts (i.e. social work practitioners and agencies) to identify areas where the given AI model tends to make more mistakes, and to build in features that \textbf{highlight potential mistakes or inadequacies} in the AI's output to facilitate further discussion and avoid workers adopting suboptimal suggestions. 

% We also point out a fundamental difference between GenAI systems and previous systems: that GenAI can now play an important role in aiding users \textit{regardless of its flaws}. The nature of GenAI means that it promotes discussion and opens up new workflows by nature of its verbose and potentially incomplete outputs. Rather than working towards more accurate or explainable outcomes, which may in any case have minimal improvement on worker outcomes \cite{li2024advanced}, designers can also focus on \textbf{understanding how GenAI outputs can augment existing user flows and create new ones}.

% % for more senior workers...

% % how to differentiate levels of workers?

% % \subsubsection{Provider}

% % The most basic and obvious role of modern AI that we identify leverages the main strength of LLMs. They have the ability to produce high-quality writing from short, point-form, or otherwise messy and disjoint case notes that user often have \textit{[cite participant here]}. 

% Finally, given the limited expertise of many workers at using AI, it is important that systems \textbf{explicitly guide users to the features they need}, rather than simply relying on the ability of GAI to understand complex user instructions. For example, in the case of flexibility in use cases (Section \ref{subsubsec:control}), systems should include user flows that help combine multiple intervention and assessment modalities in order to directly meet the needs of workers.

% \subsection{Limitations and Future Work}

% While we attempt to mimic a contextual inquiry and work environment in our study design, there is no substitute for real data from actual system deployment. The use of an AI system in day-to-day work could reveal a different set of insights. Future studies could in particular study how the longitudinal context of how user attitudes, behaviours, preferences, and work outputs change with extended use of AI. 

% While we tried to include practitioners from different agencies, roles, and seniorities, social service practice may differ culturally or procedurally in other agencies or countries. Future studies could investigate different kinds of social service agencies and in different cultures to see if AI is similarly useful there.

% As the study was conducted in a country with relatively high technology literacy, participants naturally had a higher baseline understanding and acceptance of AI and other computer systems. However, we emphasise that our findings are not contingent on this - rather, we suggest that our proposed lens of viewing AI in the social sector is a means for engaging in relevant stakeholders and ensuring the effective design and implementation of AI in the social sector, regardless of how participants feel about AI to begin with. 



% % \subsection{Notes}

% % 1) safety and risks and 2) privacy - what does the emphasis on this say about a) design recommendations and b) approach to designing/PD of such systems?


% % W9 was presented with "Strengths" and "SFBT" output options. They commented, "solution focus is always building on the person's strengths". W9 therefore requested being able to output strengths and SFBT at the same time. But this would suggest that the SFBT output does not currently emphasise strengths strongly enough. However, W9 did not specifically evaluate that, and only made this comment because they saw the "strengths" option available, and in their head, strengths are key to SFBT.
% % What does this say about system design and UI in relation to user mental models?
\paragraph{Summary}
Our findings provide significant insights into the influence of correctness, explanations, and refinement on evaluation accuracy and user trust in AI-based planners. 
In particular, the findings are three-fold: 
(1) The \textbf{correctness} of the generated plans is the most significant factor that impacts the evaluation accuracy and user trust in the planners. As the PDDL solver is more capable of generating correct plans, it achieves the highest evaluation accuracy and trust. 
(2) The \textbf{explanation} component of the LLM planner improves evaluation accuracy, as LLM+Expl achieves higher accuracy than LLM alone. Despite this improvement, LLM+Expl minimally impacts user trust. However, alternative explanation methods may influence user trust differently from the manually generated explanations used in our approach.
% On the other hand, explanations may help refine the trust of the planner to a more appropriate level by indicating planner shortcomings.
(3) The \textbf{refinement} procedure in the LLM planner does not lead to a significant improvement in evaluation accuracy; however, it exhibits a positive influence on user trust that may indicate an overtrust in some situations.
% This finding is aligned with prior works showing that iterative refinements based on user feedback would increase user trust~\cite{kunkel2019let, sebo2019don}.
Finally, the propensity-to-trust analysis identifies correctness as the primary determinant of user trust, whereas explanations provided limited improvement in scenarios where the planner's accuracy is diminished.

% In conclusion, our results indicate that the planner's correctness is the dominant factor for both evaluation accuracy and user trust. Therefore, selecting high-quality training data and optimizing the training procedure of AI-based planners to improve planning correctness is the top priority. Once the AI planner achieves a similar correctness level to traditional graph-search planners, strengthening its capability to explain and refine plans will further improve user trust compared to traditional planners.

\paragraph{Future Research} Future steps in this research include expanding user studies with larger sample sizes to improve generalizability and including additional planning problems per session for a more comprehensive evaluation. Next, we will explore alternative methods for generating plan explanations beyond manual creation to identify approaches that more effectively enhance user trust. 
Additionally, we will examine user trust by employing multiple LLM-based planners with varying levels of planning accuracy to better understand the interplay between planning correctness and user trust. 
Furthermore, we aim to enable real-time user-planner interaction, allowing users to provide feedback and refine plans collaboratively, thereby fostering a more dynamic and user-centric planning process.

\subsection{Lloyd-Max Algorithm}
\label{subsec:Lloyd-Max}
For a given quantization bitwidth $B$ and an operand $\bm{X}$, the Lloyd-Max algorithm finds $2^B$ quantization levels $\{\hat{x}_i\}_{i=1}^{2^B}$ such that quantizing $\bm{X}$ by rounding each scalar in $\bm{X}$ to the nearest quantization level minimizes the quantization MSE. 

The algorithm starts with an initial guess of quantization levels and then iteratively computes quantization thresholds $\{\tau_i\}_{i=1}^{2^B-1}$ and updates quantization levels $\{\hat{x}_i\}_{i=1}^{2^B}$. Specifically, at iteration $n$, thresholds are set to the midpoints of the previous iteration's levels:
\begin{align*}
    \tau_i^{(n)}=\frac{\hat{x}_i^{(n-1)}+\hat{x}_{i+1}^{(n-1)}}2 \text{ for } i=1\ldots 2^B-1
\end{align*}
Subsequently, the quantization levels are re-computed as conditional means of the data regions defined by the new thresholds:
\begin{align*}
    \hat{x}_i^{(n)}=\mathbb{E}\left[ \bm{X} \big| \bm{X}\in [\tau_{i-1}^{(n)},\tau_i^{(n)}] \right] \text{ for } i=1\ldots 2^B
\end{align*}
where to satisfy boundary conditions we have $\tau_0=-\infty$ and $\tau_{2^B}=\infty$. The algorithm iterates the above steps until convergence.

Figure \ref{fig:lm_quant} compares the quantization levels of a $7$-bit floating point (E3M3) quantizer (left) to a $7$-bit Lloyd-Max quantizer (right) when quantizing a layer of weights from the GPT3-126M model at a per-tensor granularity. As shown, the Lloyd-Max quantizer achieves substantially lower quantization MSE. Further, Table \ref{tab:FP7_vs_LM7} shows the superior perplexity achieved by Lloyd-Max quantizers for bitwidths of $7$, $6$ and $5$. The difference between the quantizers is clear at 5 bits, where per-tensor FP quantization incurs a drastic and unacceptable increase in perplexity, while Lloyd-Max quantization incurs a much smaller increase. Nevertheless, we note that even the optimal Lloyd-Max quantizer incurs a notable ($\sim 1.5$) increase in perplexity due to the coarse granularity of quantization. 

\begin{figure}[h]
  \centering
  \includegraphics[width=0.7\linewidth]{sections/figures/LM7_FP7.pdf}
  \caption{\small Quantization levels and the corresponding quantization MSE of Floating Point (left) vs Lloyd-Max (right) Quantizers for a layer of weights in the GPT3-126M model.}
  \label{fig:lm_quant}
\end{figure}

\begin{table}[h]\scriptsize
\begin{center}
\caption{\label{tab:FP7_vs_LM7} \small Comparing perplexity (lower is better) achieved by floating point quantizers and Lloyd-Max quantizers on a GPT3-126M model for the Wikitext-103 dataset.}
\begin{tabular}{c|cc|c}
\hline
 \multirow{2}{*}{\textbf{Bitwidth}} & \multicolumn{2}{|c|}{\textbf{Floating-Point Quantizer}} & \textbf{Lloyd-Max Quantizer} \\
 & Best Format & Wikitext-103 Perplexity & Wikitext-103 Perplexity \\
\hline
7 & E3M3 & 18.32 & 18.27 \\
6 & E3M2 & 19.07 & 18.51 \\
5 & E4M0 & 43.89 & 19.71 \\
\hline
\end{tabular}
\end{center}
\end{table}

\subsection{Proof of Local Optimality of LO-BCQ}
\label{subsec:lobcq_opt_proof}
For a given block $\bm{b}_j$, the quantization MSE during LO-BCQ can be empirically evaluated as $\frac{1}{L_b}\lVert \bm{b}_j- \bm{\hat{b}}_j\rVert^2_2$ where $\bm{\hat{b}}_j$ is computed from equation (\ref{eq:clustered_quantization_definition}) as $C_{f(\bm{b}_j)}(\bm{b}_j)$. Further, for a given block cluster $\mathcal{B}_i$, we compute the quantization MSE as $\frac{1}{|\mathcal{B}_{i}|}\sum_{\bm{b} \in \mathcal{B}_{i}} \frac{1}{L_b}\lVert \bm{b}- C_i^{(n)}(\bm{b})\rVert^2_2$. Therefore, at the end of iteration $n$, we evaluate the overall quantization MSE $J^{(n)}$ for a given operand $\bm{X}$ composed of $N_c$ block clusters as:
\begin{align*}
    \label{eq:mse_iter_n}
    J^{(n)} = \frac{1}{N_c} \sum_{i=1}^{N_c} \frac{1}{|\mathcal{B}_{i}^{(n)}|}\sum_{\bm{v} \in \mathcal{B}_{i}^{(n)}} \frac{1}{L_b}\lVert \bm{b}- B_i^{(n)}(\bm{b})\rVert^2_2
\end{align*}

At the end of iteration $n$, the codebooks are updated from $\mathcal{C}^{(n-1)}$ to $\mathcal{C}^{(n)}$. However, the mapping of a given vector $\bm{b}_j$ to quantizers $\mathcal{C}^{(n)}$ remains as  $f^{(n)}(\bm{b}_j)$. At the next iteration, during the vector clustering step, $f^{(n+1)}(\bm{b}_j)$ finds new mapping of $\bm{b}_j$ to updated codebooks $\mathcal{C}^{(n)}$ such that the quantization MSE over the candidate codebooks is minimized. Therefore, we obtain the following result for $\bm{b}_j$:
\begin{align*}
\frac{1}{L_b}\lVert \bm{b}_j - C_{f^{(n+1)}(\bm{b}_j)}^{(n)}(\bm{b}_j)\rVert^2_2 \le \frac{1}{L_b}\lVert \bm{b}_j - C_{f^{(n)}(\bm{b}_j)}^{(n)}(\bm{b}_j)\rVert^2_2
\end{align*}

That is, quantizing $\bm{b}_j$ at the end of the block clustering step of iteration $n+1$ results in lower quantization MSE compared to quantizing at the end of iteration $n$. Since this is true for all $\bm{b} \in \bm{X}$, we assert the following:
\begin{equation}
\begin{split}
\label{eq:mse_ineq_1}
    \tilde{J}^{(n+1)} &= \frac{1}{N_c} \sum_{i=1}^{N_c} \frac{1}{|\mathcal{B}_{i}^{(n+1)}|}\sum_{\bm{b} \in \mathcal{B}_{i}^{(n+1)}} \frac{1}{L_b}\lVert \bm{b} - C_i^{(n)}(b)\rVert^2_2 \le J^{(n)}
\end{split}
\end{equation}
where $\tilde{J}^{(n+1)}$ is the the quantization MSE after the vector clustering step at iteration $n+1$.

Next, during the codebook update step (\ref{eq:quantizers_update}) at iteration $n+1$, the per-cluster codebooks $\mathcal{C}^{(n)}$ are updated to $\mathcal{C}^{(n+1)}$ by invoking the Lloyd-Max algorithm \citep{Lloyd}. We know that for any given value distribution, the Lloyd-Max algorithm minimizes the quantization MSE. Therefore, for a given vector cluster $\mathcal{B}_i$ we obtain the following result:

\begin{equation}
    \frac{1}{|\mathcal{B}_{i}^{(n+1)}|}\sum_{\bm{b} \in \mathcal{B}_{i}^{(n+1)}} \frac{1}{L_b}\lVert \bm{b}- C_i^{(n+1)}(\bm{b})\rVert^2_2 \le \frac{1}{|\mathcal{B}_{i}^{(n+1)}|}\sum_{\bm{b} \in \mathcal{B}_{i}^{(n+1)}} \frac{1}{L_b}\lVert \bm{b}- C_i^{(n)}(\bm{b})\rVert^2_2
\end{equation}

The above equation states that quantizing the given block cluster $\mathcal{B}_i$ after updating the associated codebook from $C_i^{(n)}$ to $C_i^{(n+1)}$ results in lower quantization MSE. Since this is true for all the block clusters, we derive the following result: 
\begin{equation}
\begin{split}
\label{eq:mse_ineq_2}
     J^{(n+1)} &= \frac{1}{N_c} \sum_{i=1}^{N_c} \frac{1}{|\mathcal{B}_{i}^{(n+1)}|}\sum_{\bm{b} \in \mathcal{B}_{i}^{(n+1)}} \frac{1}{L_b}\lVert \bm{b}- C_i^{(n+1)}(\bm{b})\rVert^2_2  \le \tilde{J}^{(n+1)}   
\end{split}
\end{equation}

Following (\ref{eq:mse_ineq_1}) and (\ref{eq:mse_ineq_2}), we find that the quantization MSE is non-increasing for each iteration, that is, $J^{(1)} \ge J^{(2)} \ge J^{(3)} \ge \ldots \ge J^{(M)}$ where $M$ is the maximum number of iterations. 
%Therefore, we can say that if the algorithm converges, then it must be that it has converged to a local minimum. 
\hfill $\blacksquare$


\begin{figure}
    \begin{center}
    \includegraphics[width=0.5\textwidth]{sections//figures/mse_vs_iter.pdf}
    \end{center}
    \caption{\small NMSE vs iterations during LO-BCQ compared to other block quantization proposals}
    \label{fig:nmse_vs_iter}
\end{figure}

Figure \ref{fig:nmse_vs_iter} shows the empirical convergence of LO-BCQ across several block lengths and number of codebooks. Also, the MSE achieved by LO-BCQ is compared to baselines such as MXFP and VSQ. As shown, LO-BCQ converges to a lower MSE than the baselines. Further, we achieve better convergence for larger number of codebooks ($N_c$) and for a smaller block length ($L_b$), both of which increase the bitwidth of BCQ (see Eq \ref{eq:bitwidth_bcq}).


\subsection{Additional Accuracy Results}
%Table \ref{tab:lobcq_config} lists the various LOBCQ configurations and their corresponding bitwidths.
\begin{table}
\setlength{\tabcolsep}{4.75pt}
\begin{center}
\caption{\label{tab:lobcq_config} Various LO-BCQ configurations and their bitwidths.}
\begin{tabular}{|c||c|c|c|c||c|c||c|} 
\hline
 & \multicolumn{4}{|c||}{$L_b=8$} & \multicolumn{2}{|c||}{$L_b=4$} & $L_b=2$ \\
 \hline
 \backslashbox{$L_A$\kern-1em}{\kern-1em$N_c$} & 2 & 4 & 8 & 16 & 2 & 4 & 2 \\
 \hline
 64 & 4.25 & 4.375 & 4.5 & 4.625 & 4.375 & 4.625 & 4.625\\
 \hline
 32 & 4.375 & 4.5 & 4.625& 4.75 & 4.5 & 4.75 & 4.75 \\
 \hline
 16 & 4.625 & 4.75& 4.875 & 5 & 4.75 & 5 & 5 \\
 \hline
\end{tabular}
\end{center}
\end{table}

%\subsection{Perplexity achieved by various LO-BCQ configurations on Wikitext-103 dataset}

\begin{table} \centering
\begin{tabular}{|c||c|c|c|c||c|c||c|} 
\hline
 $L_b \rightarrow$& \multicolumn{4}{c||}{8} & \multicolumn{2}{c||}{4} & 2\\
 \hline
 \backslashbox{$L_A$\kern-1em}{\kern-1em$N_c$} & 2 & 4 & 8 & 16 & 2 & 4 & 2  \\
 %$N_c \rightarrow$ & 2 & 4 & 8 & 16 & 2 & 4 & 2 \\
 \hline
 \hline
 \multicolumn{8}{c}{GPT3-1.3B (FP32 PPL = 9.98)} \\ 
 \hline
 \hline
 64 & 10.40 & 10.23 & 10.17 & 10.15 &  10.28 & 10.18 & 10.19 \\
 \hline
 32 & 10.25 & 10.20 & 10.15 & 10.12 &  10.23 & 10.17 & 10.17 \\
 \hline
 16 & 10.22 & 10.16 & 10.10 & 10.09 &  10.21 & 10.14 & 10.16 \\
 \hline
  \hline
 \multicolumn{8}{c}{GPT3-8B (FP32 PPL = 7.38)} \\ 
 \hline
 \hline
 64 & 7.61 & 7.52 & 7.48 &  7.47 &  7.55 &  7.49 & 7.50 \\
 \hline
 32 & 7.52 & 7.50 & 7.46 &  7.45 &  7.52 &  7.48 & 7.48  \\
 \hline
 16 & 7.51 & 7.48 & 7.44 &  7.44 &  7.51 &  7.49 & 7.47  \\
 \hline
\end{tabular}
\caption{\label{tab:ppl_gpt3_abalation} Wikitext-103 perplexity across GPT3-1.3B and 8B models.}
\end{table}

\begin{table} \centering
\begin{tabular}{|c||c|c|c|c||} 
\hline
 $L_b \rightarrow$& \multicolumn{4}{c||}{8}\\
 \hline
 \backslashbox{$L_A$\kern-1em}{\kern-1em$N_c$} & 2 & 4 & 8 & 16 \\
 %$N_c \rightarrow$ & 2 & 4 & 8 & 16 & 2 & 4 & 2 \\
 \hline
 \hline
 \multicolumn{5}{|c|}{Llama2-7B (FP32 PPL = 5.06)} \\ 
 \hline
 \hline
 64 & 5.31 & 5.26 & 5.19 & 5.18  \\
 \hline
 32 & 5.23 & 5.25 & 5.18 & 5.15  \\
 \hline
 16 & 5.23 & 5.19 & 5.16 & 5.14  \\
 \hline
 \multicolumn{5}{|c|}{Nemotron4-15B (FP32 PPL = 5.87)} \\ 
 \hline
 \hline
 64  & 6.3 & 6.20 & 6.13 & 6.08  \\
 \hline
 32  & 6.24 & 6.12 & 6.07 & 6.03  \\
 \hline
 16  & 6.12 & 6.14 & 6.04 & 6.02  \\
 \hline
 \multicolumn{5}{|c|}{Nemotron4-340B (FP32 PPL = 3.48)} \\ 
 \hline
 \hline
 64 & 3.67 & 3.62 & 3.60 & 3.59 \\
 \hline
 32 & 3.63 & 3.61 & 3.59 & 3.56 \\
 \hline
 16 & 3.61 & 3.58 & 3.57 & 3.55 \\
 \hline
\end{tabular}
\caption{\label{tab:ppl_llama7B_nemo15B} Wikitext-103 perplexity compared to FP32 baseline in Llama2-7B and Nemotron4-15B, 340B models}
\end{table}

%\subsection{Perplexity achieved by various LO-BCQ configurations on MMLU dataset}


\begin{table} \centering
\begin{tabular}{|c||c|c|c|c||c|c|c|c|} 
\hline
 $L_b \rightarrow$& \multicolumn{4}{c||}{8} & \multicolumn{4}{c||}{8}\\
 \hline
 \backslashbox{$L_A$\kern-1em}{\kern-1em$N_c$} & 2 & 4 & 8 & 16 & 2 & 4 & 8 & 16  \\
 %$N_c \rightarrow$ & 2 & 4 & 8 & 16 & 2 & 4 & 2 \\
 \hline
 \hline
 \multicolumn{5}{|c|}{Llama2-7B (FP32 Accuracy = 45.8\%)} & \multicolumn{4}{|c|}{Llama2-70B (FP32 Accuracy = 69.12\%)} \\ 
 \hline
 \hline
 64 & 43.9 & 43.4 & 43.9 & 44.9 & 68.07 & 68.27 & 68.17 & 68.75 \\
 \hline
 32 & 44.5 & 43.8 & 44.9 & 44.5 & 68.37 & 68.51 & 68.35 & 68.27  \\
 \hline
 16 & 43.9 & 42.7 & 44.9 & 45 & 68.12 & 68.77 & 68.31 & 68.59  \\
 \hline
 \hline
 \multicolumn{5}{|c|}{GPT3-22B (FP32 Accuracy = 38.75\%)} & \multicolumn{4}{|c|}{Nemotron4-15B (FP32 Accuracy = 64.3\%)} \\ 
 \hline
 \hline
 64 & 36.71 & 38.85 & 38.13 & 38.92 & 63.17 & 62.36 & 63.72 & 64.09 \\
 \hline
 32 & 37.95 & 38.69 & 39.45 & 38.34 & 64.05 & 62.30 & 63.8 & 64.33  \\
 \hline
 16 & 38.88 & 38.80 & 38.31 & 38.92 & 63.22 & 63.51 & 63.93 & 64.43  \\
 \hline
\end{tabular}
\caption{\label{tab:mmlu_abalation} Accuracy on MMLU dataset across GPT3-22B, Llama2-7B, 70B and Nemotron4-15B models.}
\end{table}


%\subsection{Perplexity achieved by various LO-BCQ configurations on LM evaluation harness}

\begin{table} \centering
\begin{tabular}{|c||c|c|c|c||c|c|c|c|} 
\hline
 $L_b \rightarrow$& \multicolumn{4}{c||}{8} & \multicolumn{4}{c||}{8}\\
 \hline
 \backslashbox{$L_A$\kern-1em}{\kern-1em$N_c$} & 2 & 4 & 8 & 16 & 2 & 4 & 8 & 16  \\
 %$N_c \rightarrow$ & 2 & 4 & 8 & 16 & 2 & 4 & 2 \\
 \hline
 \hline
 \multicolumn{5}{|c|}{Race (FP32 Accuracy = 37.51\%)} & \multicolumn{4}{|c|}{Boolq (FP32 Accuracy = 64.62\%)} \\ 
 \hline
 \hline
 64 & 36.94 & 37.13 & 36.27 & 37.13 & 63.73 & 62.26 & 63.49 & 63.36 \\
 \hline
 32 & 37.03 & 36.36 & 36.08 & 37.03 & 62.54 & 63.51 & 63.49 & 63.55  \\
 \hline
 16 & 37.03 & 37.03 & 36.46 & 37.03 & 61.1 & 63.79 & 63.58 & 63.33  \\
 \hline
 \hline
 \multicolumn{5}{|c|}{Winogrande (FP32 Accuracy = 58.01\%)} & \multicolumn{4}{|c|}{Piqa (FP32 Accuracy = 74.21\%)} \\ 
 \hline
 \hline
 64 & 58.17 & 57.22 & 57.85 & 58.33 & 73.01 & 73.07 & 73.07 & 72.80 \\
 \hline
 32 & 59.12 & 58.09 & 57.85 & 58.41 & 73.01 & 73.94 & 72.74 & 73.18  \\
 \hline
 16 & 57.93 & 58.88 & 57.93 & 58.56 & 73.94 & 72.80 & 73.01 & 73.94  \\
 \hline
\end{tabular}
\caption{\label{tab:mmlu_abalation} Accuracy on LM evaluation harness tasks on GPT3-1.3B model.}
\end{table}

\begin{table} \centering
\begin{tabular}{|c||c|c|c|c||c|c|c|c|} 
\hline
 $L_b \rightarrow$& \multicolumn{4}{c||}{8} & \multicolumn{4}{c||}{8}\\
 \hline
 \backslashbox{$L_A$\kern-1em}{\kern-1em$N_c$} & 2 & 4 & 8 & 16 & 2 & 4 & 8 & 16  \\
 %$N_c \rightarrow$ & 2 & 4 & 8 & 16 & 2 & 4 & 2 \\
 \hline
 \hline
 \multicolumn{5}{|c|}{Race (FP32 Accuracy = 41.34\%)} & \multicolumn{4}{|c|}{Boolq (FP32 Accuracy = 68.32\%)} \\ 
 \hline
 \hline
 64 & 40.48 & 40.10 & 39.43 & 39.90 & 69.20 & 68.41 & 69.45 & 68.56 \\
 \hline
 32 & 39.52 & 39.52 & 40.77 & 39.62 & 68.32 & 67.43 & 68.17 & 69.30  \\
 \hline
 16 & 39.81 & 39.71 & 39.90 & 40.38 & 68.10 & 66.33 & 69.51 & 69.42  \\
 \hline
 \hline
 \multicolumn{5}{|c|}{Winogrande (FP32 Accuracy = 67.88\%)} & \multicolumn{4}{|c|}{Piqa (FP32 Accuracy = 78.78\%)} \\ 
 \hline
 \hline
 64 & 66.85 & 66.61 & 67.72 & 67.88 & 77.31 & 77.42 & 77.75 & 77.64 \\
 \hline
 32 & 67.25 & 67.72 & 67.72 & 67.00 & 77.31 & 77.04 & 77.80 & 77.37  \\
 \hline
 16 & 68.11 & 68.90 & 67.88 & 67.48 & 77.37 & 78.13 & 78.13 & 77.69  \\
 \hline
\end{tabular}
\caption{\label{tab:mmlu_abalation} Accuracy on LM evaluation harness tasks on GPT3-8B model.}
\end{table}

\begin{table} \centering
\begin{tabular}{|c||c|c|c|c||c|c|c|c|} 
\hline
 $L_b \rightarrow$& \multicolumn{4}{c||}{8} & \multicolumn{4}{c||}{8}\\
 \hline
 \backslashbox{$L_A$\kern-1em}{\kern-1em$N_c$} & 2 & 4 & 8 & 16 & 2 & 4 & 8 & 16  \\
 %$N_c \rightarrow$ & 2 & 4 & 8 & 16 & 2 & 4 & 2 \\
 \hline
 \hline
 \multicolumn{5}{|c|}{Race (FP32 Accuracy = 40.67\%)} & \multicolumn{4}{|c|}{Boolq (FP32 Accuracy = 76.54\%)} \\ 
 \hline
 \hline
 64 & 40.48 & 40.10 & 39.43 & 39.90 & 75.41 & 75.11 & 77.09 & 75.66 \\
 \hline
 32 & 39.52 & 39.52 & 40.77 & 39.62 & 76.02 & 76.02 & 75.96 & 75.35  \\
 \hline
 16 & 39.81 & 39.71 & 39.90 & 40.38 & 75.05 & 73.82 & 75.72 & 76.09  \\
 \hline
 \hline
 \multicolumn{5}{|c|}{Winogrande (FP32 Accuracy = 70.64\%)} & \multicolumn{4}{|c|}{Piqa (FP32 Accuracy = 79.16\%)} \\ 
 \hline
 \hline
 64 & 69.14 & 70.17 & 70.17 & 70.56 & 78.24 & 79.00 & 78.62 & 78.73 \\
 \hline
 32 & 70.96 & 69.69 & 71.27 & 69.30 & 78.56 & 79.49 & 79.16 & 78.89  \\
 \hline
 16 & 71.03 & 69.53 & 69.69 & 70.40 & 78.13 & 79.16 & 79.00 & 79.00  \\
 \hline
\end{tabular}
\caption{\label{tab:mmlu_abalation} Accuracy on LM evaluation harness tasks on GPT3-22B model.}
\end{table}

\begin{table} \centering
\begin{tabular}{|c||c|c|c|c||c|c|c|c|} 
\hline
 $L_b \rightarrow$& \multicolumn{4}{c||}{8} & \multicolumn{4}{c||}{8}\\
 \hline
 \backslashbox{$L_A$\kern-1em}{\kern-1em$N_c$} & 2 & 4 & 8 & 16 & 2 & 4 & 8 & 16  \\
 %$N_c \rightarrow$ & 2 & 4 & 8 & 16 & 2 & 4 & 2 \\
 \hline
 \hline
 \multicolumn{5}{|c|}{Race (FP32 Accuracy = 44.4\%)} & \multicolumn{4}{|c|}{Boolq (FP32 Accuracy = 79.29\%)} \\ 
 \hline
 \hline
 64 & 42.49 & 42.51 & 42.58 & 43.45 & 77.58 & 77.37 & 77.43 & 78.1 \\
 \hline
 32 & 43.35 & 42.49 & 43.64 & 43.73 & 77.86 & 75.32 & 77.28 & 77.86  \\
 \hline
 16 & 44.21 & 44.21 & 43.64 & 42.97 & 78.65 & 77 & 76.94 & 77.98  \\
 \hline
 \hline
 \multicolumn{5}{|c|}{Winogrande (FP32 Accuracy = 69.38\%)} & \multicolumn{4}{|c|}{Piqa (FP32 Accuracy = 78.07\%)} \\ 
 \hline
 \hline
 64 & 68.9 & 68.43 & 69.77 & 68.19 & 77.09 & 76.82 & 77.09 & 77.86 \\
 \hline
 32 & 69.38 & 68.51 & 68.82 & 68.90 & 78.07 & 76.71 & 78.07 & 77.86  \\
 \hline
 16 & 69.53 & 67.09 & 69.38 & 68.90 & 77.37 & 77.8 & 77.91 & 77.69  \\
 \hline
\end{tabular}
\caption{\label{tab:mmlu_abalation} Accuracy on LM evaluation harness tasks on Llama2-7B model.}
\end{table}

\begin{table} \centering
\begin{tabular}{|c||c|c|c|c||c|c|c|c|} 
\hline
 $L_b \rightarrow$& \multicolumn{4}{c||}{8} & \multicolumn{4}{c||}{8}\\
 \hline
 \backslashbox{$L_A$\kern-1em}{\kern-1em$N_c$} & 2 & 4 & 8 & 16 & 2 & 4 & 8 & 16  \\
 %$N_c \rightarrow$ & 2 & 4 & 8 & 16 & 2 & 4 & 2 \\
 \hline
 \hline
 \multicolumn{5}{|c|}{Race (FP32 Accuracy = 48.8\%)} & \multicolumn{4}{|c|}{Boolq (FP32 Accuracy = 85.23\%)} \\ 
 \hline
 \hline
 64 & 49.00 & 49.00 & 49.28 & 48.71 & 82.82 & 84.28 & 84.03 & 84.25 \\
 \hline
 32 & 49.57 & 48.52 & 48.33 & 49.28 & 83.85 & 84.46 & 84.31 & 84.93  \\
 \hline
 16 & 49.85 & 49.09 & 49.28 & 48.99 & 85.11 & 84.46 & 84.61 & 83.94  \\
 \hline
 \hline
 \multicolumn{5}{|c|}{Winogrande (FP32 Accuracy = 79.95\%)} & \multicolumn{4}{|c|}{Piqa (FP32 Accuracy = 81.56\%)} \\ 
 \hline
 \hline
 64 & 78.77 & 78.45 & 78.37 & 79.16 & 81.45 & 80.69 & 81.45 & 81.5 \\
 \hline
 32 & 78.45 & 79.01 & 78.69 & 80.66 & 81.56 & 80.58 & 81.18 & 81.34  \\
 \hline
 16 & 79.95 & 79.56 & 79.79 & 79.72 & 81.28 & 81.66 & 81.28 & 80.96  \\
 \hline
\end{tabular}
\caption{\label{tab:mmlu_abalation} Accuracy on LM evaluation harness tasks on Llama2-70B model.}
\end{table}

%\section{MSE Studies}
%\textcolor{red}{TODO}


\subsection{Number Formats and Quantization Method}
\label{subsec:numFormats_quantMethod}
\subsubsection{Integer Format}
An $n$-bit signed integer (INT) is typically represented with a 2s-complement format \citep{yao2022zeroquant,xiao2023smoothquant,dai2021vsq}, where the most significant bit denotes the sign.

\subsubsection{Floating Point Format}
An $n$-bit signed floating point (FP) number $x$ comprises of a 1-bit sign ($x_{\mathrm{sign}}$), $B_m$-bit mantissa ($x_{\mathrm{mant}}$) and $B_e$-bit exponent ($x_{\mathrm{exp}}$) such that $B_m+B_e=n-1$. The associated constant exponent bias ($E_{\mathrm{bias}}$) is computed as $(2^{{B_e}-1}-1)$. We denote this format as $E_{B_e}M_{B_m}$.  

\subsubsection{Quantization Scheme}
\label{subsec:quant_method}
A quantization scheme dictates how a given unquantized tensor is converted to its quantized representation. We consider FP formats for the purpose of illustration. Given an unquantized tensor $\bm{X}$ and an FP format $E_{B_e}M_{B_m}$, we first, we compute the quantization scale factor $s_X$ that maps the maximum absolute value of $\bm{X}$ to the maximum quantization level of the $E_{B_e}M_{B_m}$ format as follows:
\begin{align}
\label{eq:sf}
    s_X = \frac{\mathrm{max}(|\bm{X}|)}{\mathrm{max}(E_{B_e}M_{B_m})}
\end{align}
In the above equation, $|\cdot|$ denotes the absolute value function.

Next, we scale $\bm{X}$ by $s_X$ and quantize it to $\hat{\bm{X}}$ by rounding it to the nearest quantization level of $E_{B_e}M_{B_m}$ as:

\begin{align}
\label{eq:tensor_quant}
    \hat{\bm{X}} = \text{round-to-nearest}\left(\frac{\bm{X}}{s_X}, E_{B_e}M_{B_m}\right)
\end{align}

We perform dynamic max-scaled quantization \citep{wu2020integer}, where the scale factor $s$ for activations is dynamically computed during runtime.

\subsection{Vector Scaled Quantization}
\begin{wrapfigure}{r}{0.35\linewidth}
  \centering
  \includegraphics[width=\linewidth]{sections/figures/vsquant.jpg}
  \caption{\small Vectorwise decomposition for per-vector scaled quantization (VSQ \citep{dai2021vsq}).}
  \label{fig:vsquant}
\end{wrapfigure}
During VSQ \citep{dai2021vsq}, the operand tensors are decomposed into 1D vectors in a hardware friendly manner as shown in Figure \ref{fig:vsquant}. Since the decomposed tensors are used as operands in matrix multiplications during inference, it is beneficial to perform this decomposition along the reduction dimension of the multiplication. The vectorwise quantization is performed similar to tensorwise quantization described in Equations \ref{eq:sf} and \ref{eq:tensor_quant}, where a scale factor $s_v$ is required for each vector $\bm{v}$ that maps the maximum absolute value of that vector to the maximum quantization level. While smaller vector lengths can lead to larger accuracy gains, the associated memory and computational overheads due to the per-vector scale factors increases. To alleviate these overheads, VSQ \citep{dai2021vsq} proposed a second level quantization of the per-vector scale factors to unsigned integers, while MX \citep{rouhani2023shared} quantizes them to integer powers of 2 (denoted as $2^{INT}$).

\subsubsection{MX Format}
The MX format proposed in \citep{rouhani2023microscaling} introduces the concept of sub-block shifting. For every two scalar elements of $b$-bits each, there is a shared exponent bit. The value of this exponent bit is determined through an empirical analysis that targets minimizing quantization MSE. We note that the FP format $E_{1}M_{b}$ is strictly better than MX from an accuracy perspective since it allocates a dedicated exponent bit to each scalar as opposed to sharing it across two scalars. Therefore, we conservatively bound the accuracy of a $b+2$-bit signed MX format with that of a $E_{1}M_{b}$ format in our comparisons. For instance, we use E1M2 format as a proxy for MX4.

\begin{figure}
    \centering
    \includegraphics[width=1\linewidth]{sections//figures/BlockFormats.pdf}
    \caption{\small Comparing LO-BCQ to MX format.}
    \label{fig:block_formats}
\end{figure}

Figure \ref{fig:block_formats} compares our $4$-bit LO-BCQ block format to MX \citep{rouhani2023microscaling}. As shown, both LO-BCQ and MX decompose a given operand tensor into block arrays and each block array into blocks. Similar to MX, we find that per-block quantization ($L_b < L_A$) leads to better accuracy due to increased flexibility. While MX achieves this through per-block $1$-bit micro-scales, we associate a dedicated codebook to each block through a per-block codebook selector. Further, MX quantizes the per-block array scale-factor to E8M0 format without per-tensor scaling. In contrast during LO-BCQ, we find that per-tensor scaling combined with quantization of per-block array scale-factor to E4M3 format results in superior inference accuracy across models. 






\bibliographystyle{IEEEtran}
\bibliography{bib}


%%%%%%%%%%%%%%%%%%%%%%%%%%%%%%%%%%%%%%%%%%%%%%%%%%%%%%%%%%%%%%%%%%%%%%%%%%%%%%%%



\addtolength{\textheight}{-12cm}   % This command serves to balance the column lengths
                                  % on the last page of the document manually. It shortens
                                  % the textheight of the last page by a suitable amount.
                                  % This command does not take effect until the next page
                                  % so it should come on the page before the last. Make
                                  % sure that you do not shorten the textheight too much.

%%%%%%%%%%%%%%%%%%%%%%%%%%%%%%%%%%%%%%%%%%%%%%%%%%%%%%%%%%%%%%%%%%%%%%%%%%%%%%%%

\end{document}

\end{document}
