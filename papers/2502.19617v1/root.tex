

\documentclass[letterpaper, 10 pt, conference]{ieeetran}  % Comment this line out if you need a4paper



\IEEEoverridecommandlockouts                             
\usepackage[noend]{algpseudocode}
\usepackage{algorithm}
\usepackage{amssymb}
\usepackage{bbm}
\usepackage{dsfont}
\usepackage[left=1.69cm,right=1.69cm,top=2.12cm,bottom=1.52cm]{geometry}


\newcommand{\algorithmautorefname}{Alg.}

\usepackage{xspace}
\usepackage{xcolor}
\definecolor{softgreen}{RGB}{34,139,34} % A visually appealing forest green
\usepackage[colorlinks=true, linkcolor=softgreen, citecolor=softgreen, urlcolor=softgreen]{hyperref}

\renewcommand{\figureautorefname}{Fig.} % Changes "Figure" to "Fig."
\renewcommand{\equationautorefname}{Eq.} % Changes "Equation" to "Eq."
\usepackage{array}


\newcommand{\R}{\ensuremath{\mathbb{R}}\xspace}
\newcommand{\X}{\ensuremath{\mathcal{X}}\xspace}
\newcommand{\T}{\ensuremath{\mathcal{T}}\xspace}
\newcommand{\Xg}{\X_G} 
\newcommand{\xs}{\ensuremath{\x_\start}} 
\newcommand{\U}{\ensuremath{\mathcal{U}}\xspace}
\newcommand{\uu}{u}
\newcommand{\Xfree}{\ensuremath{\X_{\text{free}}}\xspace}
\newcommand{\Xobs}{\ensuremath{\X_{\text{obs}}}\xspace}
\newcommand{\K}{\ensuremath{I}\xspace}
\newcommand{\Kspace}{\K-space\xspace}
\newcommand{\key}{\ensuremath{\mathbb{K}}\xspace}
\newcommand{\skey}{\ensuremath{\mathds{k}}\xspace}
\newcommand{\keys}{\ensuremath{\key_\text{start}}\xspace}
\newcommand{\keyg}{\ensuremath{\key_\text{goal}}\xspace}
\newcommand{\Kobs}{\ensuremath{\K_{\text{obs}}}\xspace}
\newcommand{\Kfree}{\ensuremath{\K_{\text{free}}}\xspace}
\newcommand{\x}{x}
\newcommand{\xr}{\x_{rand}}
\newcommand{\Kf}{{\ensuremath{\K_\free}}\xspace}
\newcommand{\Ko}{{\ensuremath{\K_\text{obs}}}\xspace}
\renewcommand{\xi}{\x_{I}}
\newcommand{\prm}{\textsc{prm}\xspace}
\newcommand{\lprm}{\textsc{lazy-prm}\xspace}
\newcommand{\euclid}{image distance}
\newcommand{\learned}{learned}

\newcommand{\start}{\text{start}}
\newcommand{\source}{\text{src}}
\newcommand{\destination}{\text{dst}}
\newcommand{\goal}{\text{goal}}
\newcommand{\free}{\text{free}}
\newcommand{\dof}{\textsc{dof}\xspace}

\newcommand{\im}{\ensuremath{Im}\xspace}
\newcommand{\IM}{\ensuremath{\mathcal{IM}}\xspace}


% Configuration Space
\newcommand{\C}{\X}
\newcommand{\Cs}{\X}
\newcommand{\Cf}{\Xfree}
\newcommand{\Co}{\Xobs}

\usepackage[textsize=small]{todonotes}
\setlength{\marginparwidth}{1.2cm}

\usepackage{graphics} 
\usepackage{listings}
\usepackage{comment}
\usepackage{epsfig} 

\usepackage{amsmath} 
\usepackage{cite}
\usepackage{subcaption}
\usepackage{xcolor}
 \usepackage{url}
\definecolor{gg}{RGB}{0, 155, 85} 
\definecolor{json-key}{rgb}{0.13,0.55,0.13}
\definecolor{json-value}{rgb}{0.25,0.25,0.25}
\definecolor{json-string}{rgb}{0.9,0.3,0.3}
\usepackage{caption}
\captionsetup[figure]{font=small, labelfont=small, labelsep=period}
\captionsetup[table]{font=small, labelfont=small, labelsep=period}

\lstdefinelanguage{json}{
  basicstyle=\ttfamily,
    commentstyle=\color{gray},
    stringstyle=\color{orange},
    numbers=left,
    numberstyle=,
    numbersep=5pt,
    breaklines=true,
    frame=lines,
    backgroundcolor=\color{white!10},
    captionpos=b
}

\newcommand\centerImage[2][]%
  {\raisebox
    {-\dimexpr0.5\height}%
    [\dimexpr0.5\height+1mm]%
    [\dimexpr0.5\height+1mm]%
    {\includegraphics[#1]{#2}}%
  }


\title{\LARGE \bf  Image-Based Roadmaps for Vision-Only Planning and Control of Robotic Manipulators}

\author{Sreejani Chatterjee$^{1}$, Abhinav Gandhi $^{1}$, Berk Calli$^{1}$ and Constantinos Chamzas $^{1}$ % stops a space
% \thanks{*This paper was supported in part by the National Science Foundation under grant
% IIS-1900953 and CMMI-1928506.}% <-this % stops a space
\thanks{$^{1}$S Chatterjee, $^{1}$A Gandhi,  $^{1}$C Chamzas and $^{1}$B Calli are with Department of Robotics Engineering,
        Worcester Polytechnic Institute, Worcester, MA 01609, USA
        {\tt\small schatterjee@wpi.edu, agandhi@wpi.edu}}%
}

\begin{document}
\maketitle
\thispagestyle{empty}
\pagestyle{empty}


\begin{abstract}

This work presents a motion planning framework for robotic manipulators that computes collision-free paths directly in image space. The generated paths can then be tracked using vision-based control, eliminating the need for an explicit robot model or proprioceptive sensing.
At the core of our approach is the construction of a roadmap entirely in image space. To achieve this, we explicitly define sampling, nearest-neighbor selection, and collision checking based on visual features rather than geometric models. We first collect a set of image-space samples by moving the robot within its workspace, capturing keypoints along its body at different configurations. These samples serve as nodes in the roadmap, which we construct using either learned or predefined distance metrics.
At runtime, the roadmap generates collision-free paths directly in image space, removing the need for a robot model or joint encoders. We validate our approach through an experimental study in which a robotic arm follows planned paths using an adaptive vision-based control scheme to avoid obstacles. The results show that paths generated with the learned-distance roadmap achieved 100\% success in control convergence, whereas the predefined image-space distance roadmap enabled faster transient responses but had a lower success rate in convergence.
 
\end{abstract}

\section{Introduction}

Vision-based control techniques~\cite{hutchinson1996tutorial, hashimoto2003review}, offer significant advantages for robotic manipulators in unstructured and cluttered environments by enabling closed-loop control using task-relevant visual information. These techniques also enhance robustness against model inaccuracies, beneficial for robots with complex or variable dynamics \cite{Calli2016, cuevas2018hybrid, ardon2018reaching}. This strategy is especially useful for robots that are difficult to model accurately, e.g. soft robots \cite{Lai2020, luo2018orisnake}, under-actuated robots \cite{liu2020survey, gandhi2023shape}, 3D printed robots \cite{chavdarov2019design, onal2014origami}, and robots with inexpensive hardware \cite{adzeman2020kinematic}. Furthermore, model-free visual servoing, which learns robot-feature motion models during control, reduces reliance on a priori knowledge of the robot model \cite{wang2018adaptive, navarro2017fourier}.

The goal of our research is to push the boundaries of purely vision-based control and motion planning for robotic manipulators, by decreasing reliance on explicit robot modeling or proprioceptive sensing. In doing so, we strive to use natural visual features along the robot's body in image space, without attaching any external markers. While the works in \cite{gandhi2022skeleton,chatterjee2023keypoints, chatterjee2024utilizing} provided algorithms to track natural keypoints and use them to achieve vision-based model-free control with decent transient responses, these algorithms are only designed to run in obstacle-free space and do not provide motion planning capability to avoid obstacles in the scene. 

\begin{figure}[t]
        \centering
    \includegraphics[width=0.65\columnwidth]{images/fig_1_with_obs_v4.png}
    \caption{
    The robot follows a collision-free path using visual keypoints (colored) planned with the proposed method. 
    The set of keypoints are tracked in order:
    green, magenta, blue, and red. The robot avoids the yellow obstacle while moving between the specified start and goal.  }
    \label{intro-image}
    \vspace{-1em}
\end{figure}

In this work, we tackle the problem of collision free path planning for purely vision-based robot control: we develop a planning and control scheme that does not rely on a robot model or on-board sensors (e.g. joint encoders) in runtime. As such, this strategy is especially useful for (but not limited to) soft/underactuated/inexpensive robots that are hard to model and may not have reliable proprioceptive sensing.

Toward this goal, we propose a novel motion planning formulation that operates only with visual information. We propose a vision-only motion planning methodology using roadmaps \cite{kavraki1996probabilistic}.  We investigated two ways of generating roadmaps using natural features on a robot: 1) directly utilizing the Euclidean distance between keypoints in image space as a distance metric, 2) estimating the joint displacements from image features and utilizing it as a distance metric. For the latter, we first used an automated data collection pipeline to annotate natural keypoints' placement in image space along the robot's body as the robot moves across various configurations.

This dataset was used to develop a simple neural network that approximates joint displacements based on keypoint locations in image space. These distance metrics were integrated into the roadmap construction. Once the roadmaps are constructed, polygon-based collision checking and A* search are employed to ensure collision-free paths. These two approaches have different implications for vision-based robot control. In a nutshell, we observed that utilizing estimated joint distances in the roadmap results in smoother and more accurate tracking of the generated path, allowing the robot to stay in its defined workspace and avoid obstacles. In contrast, roadmaps based on Euclidean distances in image space can offer faster transient responses, albeit with potentially less reliable tracking. Our experiments demonstrate these aspects both in the presence and absence of obstacles.
 
\section{Related Work}

Motion planning is a core problem in robotics that has been extensively studied over the decades. It is generally categorized into three main types: optimization-based \cite{schulman2014motion, zhao2024survey}, sampling-based \cite{orthey2024review, kingston2018sampling}, and search-based planners \cite{likhachev2003ara, cohen2010search}. Recently, Jacobian-based motion planning \cite{park2020trajectory} has also gained traction for obstacle avoidance tasks. While all methods have found widespread success in different applications, all of these rely on having an explicit geometric model of the robot to design feasible paths. In this work we extend the principles of sampling-based planning of the Probabilistic Roadmap (\prm) approach \cite{kavraki1996probabilistic}, to provide a method for motion planning where a robot model is not available. 

Model-free planning, especially without prior knowledge of the robot's geometry, presents unique challenges. Reinforcement learning (RL) has been explored extensively in the recent decades to address such challenges. For instance, \cite{liu2021model} proposed an RL-based framework for generating jerk-free, smooth trajectories. While effective, this method depends on carefully crafted reward functions and extensive datasets generated in simulation, with no real-world validation. Similarly, \cite{zhou2021robotic} introduced a hybrid approach combining RRT*-based trajectories with PPO reinforcement learning for policy refinement. However, this method heavily relies on model-based elements like precomputed trajectories and supervised learning for initial policy design, with experiments confined to simulated environments. In contrast, our approach eliminates reliance on explicit or precomputed models and trajectories by leveraging visual keypoints to construct roadmaps directly in image space, requiring only a small dataset collected from a real robot. This makes our method more adaptable to scenarios without precise geometric models.

Among related works, the approaches proposed in \cite{ichter2019robot} and \cite{556169} are the most closely aligned with ours. In \cite{ichter2019robot}, the authors use sampling-based planning directly from images by leveraging learned forward propagation models, custom distance metrics, and collision checkers. While effective, their method requires extensive simulation-based training, which introduces a significant sim-to-real gap. In contrast, our method uses a limited dataset collected entirely from real robots and constructs a configuration-space roadmap without relying on simulation. Similarly, \cite{556169} combines image features with a robot model to train a neural network for planning a path. Unlike their approach, our method eliminates the need for a robot model, relying solely on image features for motion planning.

Planning a path for image based visual servoing without a-priori knowledge of the robot's model is rarely delved into in the literature. For instance, \cite{mezouar2000path} modeled a path planner for visual servoing to bridge gaps between initial and target positions which are much further apart in configuration space, without addressing collision avoidance. In \cite{lee2011obstacle} authors achieve obstacle avoidance in pose based visual servoing and hence still needed explicit robot model instead of only visual feedback.

\section{Preliminaries and Problem Statement}
In this section we describe the motion planning problem, a brief review of sampling-based methods focusing on probabilistic roadmaps, and finally we introduce the problem statement that we are trying to solve.  

\begin{figure}[t]
      \centering
      \includegraphics[width=0.5\textwidth]{images/config_space_v7.pdf}
      \vspace{-3.85em}
      \caption{The images depict a vision-based motion planning problem. The left image shows the discretized representation of \K  highlighted in gray, where each configuration is sampled as an image frame, resulting in a finite set of \key. The gray region in the right image highlights the same discretized representation of \Kfree, avoiding the obstacle in yellow. \Kfree ensures a collision-free path from the start configuration \keys to the goal configuration \keyg}
      \label{config_space}
\end{figure}

\subsection{Model-based Motion Planning}
\label{ssec:prob-state}

Let $\x \in \X$ denote the state and state-space, and $ \uu \in \U$ the control and control-space of a robot system \cite{orthey2024review}.
The system dynamics equations $f$ of the robot system can be written as 
\begin{equation} \label{eq:motion}
\x(\T) = \x(0)+\int_0^{\T} f(x(t),u(t))dt \end{equation}
If the system dynamics impose constraints on the allowable paths the system is called a non-holonomic system.  Now, Let $\Co \subset \C $ denote the invalid state space, which is the set of states that violates the robots kinematic constraints or collides with the workspace $\mathcal{W} \subseteq \mathbb{R}^3$. e.g.,
\begin{equation}
\Co =\{\x \in \C | R(\x) \bigcap \mathcal{W} \neq \emptyset \}
\end{equation}
where $R(\x) \subseteq \mathbb{R}^3$ is the set of all the points of the robot in the workspace at state \x \footnote{This equation only encodes collisions but joint/kinematic constraints can be encoded in a similar manner}. Then let, $\Cf = \C \setminus \Co$ denote the collision free state-space. Also, let $x_\start \in \Cf $ and \mbox{$X_\goal \subseteq \Cf$} denote the start state and goal regions respectively.

\textit{Motion Planning Problem:}
Given the motion planning tuple $(\X, R, \mathcal{W}, \U, f,  x_\start, \Xg)$ find a time $\T$ and a set of controls $u: [0,\T] \rightarrow \U$ such that the motion described by \autoref{eq:motion} satisfies $ x(0) = \xs $, $ \x(\T) \in \Xg$ and $x(t) \in \Xfree$.   

\subsection{Vision-Only Motion Planning} 
\label{ssec:prob-state-vis}

The problem we are considering in this setting is departing from the above motion planning problem as the geometric model $R(x)$, the dynamics $f$ and the workspace $\mathcal{W}$ is not directly available. Instead the only information available is the space of admissible controls \U and an image $\im \in \IM$.

We will describe the new problem statement by explicitly defining equivalent concepts of obstacles, robot states, and state dynamics directly in the image space.
We assume that we are given a point tracking function that maps a given image to $N$ fixed pixel points on the robot's body.  
This vector of pixel points is denoted as an image state $\key \in \K $.
Where $\K \subset \mathbb{R}^{N \times 2}$ denotes the space of all image states.
In \autoref{intro-image} we can see $4$ different robot configurations in image space. Each configuration is represented by a set of $5$ same-colored circles. Each of these circles on the robot's body in the image is a pixel point and is denoted as a keypoint or $\skey$. The kinematic chain or shape formed by the vector of $5$ circles represents one image state \key.

Given a set of image obstacles $\mathcal{IO} \subset \im$ we define: 
\begin{equation}
\Kobs =\{\key \in \K | RI(\key) \bigcap \mathcal{IO} \neq \emptyset \}
\end{equation}
where $RI(\key) \subseteq \im $ is the set of pixels that the robot occupies at image state \key . Similarly to before, we define $\Kfree  = \K \setminus \Kobs$, $\key_\start$ and $\K_g$. 

We define the unknown system dynamics function as:     
\begin{equation} \label{eq:motion_image}
\key(\T) = \key(0)+\int_0^{\T} g(\key(t),u(t))dt
\end{equation}

Here we note that even if the underlying function f is fully-integrable (holonomic-system) if $dim(\K)> dim(\X)$ the equivalent system dynamics equations $g$ will be non-holonomic as there will be paths in the higher dimensional image-space \K that cannot be followed by the robot. In our approach we consider this issue, and propose a way to produce paths that approximately satisfy these constraints. In the first image of \autoref{config_space}, the grey region illustrates the discretized representation of \K, as each configuration is captured as an image frame, resulting in a finite set of \key sampled at a specific frame rate. The second image of \autoref{config_space} highlights similar discretized representation of \Kfree.

 Given the above definitions, let us define the vision-only motion planning problem. 

\textit{Vision-Only Motion Planning Problem:}
Given the tuple $(\K, RI, \U, \mathcal{IO}, \im, \key_{\start}, \K_g)$ find a time $\T$ and a set of controls $u: [0,\T] \rightarrow \U$ such that the motion described by $g$ satisfies $\key(0) = \key_{\start}$, $ \key(\T) \in \K_g$ and $\key(t) \in \Kfree$.  
%If we had the model of the robot, then we can compute all these different functions and succesfully find a path from start to goal that avoids the obstacles. However since we don't have the model of the robot, we would need to find ways to compute the  and distance for nearest neighbor structure.  
\section{Methodology}
\label{ssec:methodology}

Since the system dynamics equation is unknown, we cannot directly plan in the control space \U. Instead, we will plan a path $\key_0, \key_1 \ldots, \key_n$ directly in the \Kspace and then control the robot to follow the path with a vision-only controller\cite{gandhi2022skeleton}.

To compute the path, which is the main contribution of this work, we propose to use probabilistic road-map planner (\prm) by adapting its subroutines to operate directly in \K-space. Specifically, we opted to adapt the \lprm~\cite{bohlin2000path} planner due to its compatibility with our particular requirements. However, any sampling-based planner which relies on the the same subroutines could be used. 

\begin{algorithm}[H]
   \caption{Build-Lazy-PRM} 
   \label{alg:build-lazy-prm}
    \begin{algorithmic}[1] 
     \Procedure{Build-Lazy-PRM}{N, k}  
        \State {$G$} $\gets$ INIT()   
        \While{$G$.size()$\leq$ N}%$i = 1 ,\ldots, N$} 
          %\State \Comment{generate collision free sample}
           \State $\key_{new} \gets $ {\color{red}{\textsc{sample}}}($\K$) \label{sample} 
           \State $G$.addNode($\key_{new}$) \label{roadmap}
        \EndWhile
       \For{each $\key \in  G$.nodes()} 
           \State $\mathcal{N}(\key) \gets $ {\color{red}{\textsc{K-nearest}}}($\key$, {$G$}) \label{dist} 
           \For{each $\key_{near} \in \mathcal{N}(\key_{new})$} 
            \State $e \gets$ ($\key, \key_{near})$
            \If{$e \notin G$.edges()} \label{collision}
                \State  $G$.addEdge($e$) 
            \EndIf 
          \EndFor 
        \EndFor 
    	\State \Return $G$ 
    \EndProcedure
    \end{algorithmic}
    \end{algorithm}
    \vspace{-0em}

Similar to \prm, \lprm operates in two distinct phases, the building-phase \autoref{alg:build-lazy-prm} and the query-phase \autoref{alg:query-lazy-prm}. In the building-phase, a roadmap is built  without performing any collision checking. In the query-phase, the roadmap is utilized to find a potential path for a given start and goal. The key difference between \lprm and \prm lies in their approach to collision checking. While \prm verifies the collision status of all edges in the roadmap during the building phase, \lprm reverses the order. It first performs the search query and only checks collisions for the found path. If any edges are found to be in collision, the roadmap is updated, and another search query is executed until a valid path is discovered.

We selected probabilistic roadmap method for its efficiency in multi-query scenarios, where a single precomputed graph can handle multiple start and goal configurations, and used the lazy version, as we only have very few varying obstacles between queries. 
Additionally, its ability to operate in high-dimensional spaces makes it particularly well-suited for handling non-traditional representations of configuration space, such as the visual keypoints in \K-space used in our approach. We first describe how \lprm works and then we describe how we modified the necessary subroutines to make it work in \Kspace. 

The building-phase (\autoref{alg:build-lazy-prm}) works with the following procedure. First, in line \ref{sample}, a sample is generated and added in graph $G$. Then the k nearest nearest neighbors are found by using a distance defined in \K (line \ref{dist}) and are connected with edges. This continues until $N$ nodes are in the graph $G$. 


\begin{algorithm}[H]
   \caption{Query-Lazy-PRM} 
   \label{alg:query-lazy-prm}
    \begin{algorithmic}[1] 
     \Procedure{Lazy-Query-PRM}{\keys, \keyg, $G$}   
       \State G.add(\keys) \Comment{Add start, and goal to the Graph}
       \State G.add(\keyg)
       \State Edges, $\gets$ \textsc{Search-Graph} $G$(\keyg, \keys)
       \For{each $e \in  G$.edges()} 
           \If{{\color{red}{\textsc{coll\_free}}}($e$)} \label{collision}
           \For{each $\key_{near} \in \mathcal{N}(x_{new})$} 
            \State $e \gets$ ($\key, \key_{near})$
            \If{{\color{red}{\textsc{coll\_free}}}($e$) and $e \notin G$.edges()} \label{collision}
                \State  $G$.addEdge($e$) 
            \EndIf 
          \EndFor 
    	\State \Return $Path$ 
        \EndIf 
     \EndFor 
    \EndProcedure
    \end{algorithmic}
    \end{algorithm}
    \vspace{-0.75em}



During the query-phase (\autoref{alg:query-lazy-prm} a new motion planning problem is solved. Given a $\key_\start$ and $\key_\goal $ they are added in the  graph, and connected with their nearest-neighbors. Then a graph search algorithm e.g., A* is used to find a path. If edges of the path are in-collision they are updated accordingly in the roadmap, and the process repeats until a collision free-path is found. 
The three operations \textsc{sample},  \textsc{k-nearest},  \textsc{coll\_free},
for \autoref{alg:build-lazy-prm} in lines \ref{sample}, \ref{dist} and \autoref{alg:query-lazy-prm}  typically require a model for the robot.

\begin{figure*}[ht!]
      \centering
      \includegraphics[width=0.8\textwidth]{images/graph_flow_v8.pdf}
          \caption{Overview of the roadmap creation process for motion planning. In this image the purple circles are nodes in image space represented by image state in each frame. Using a distance metric, we connect the nodes to create a roadmap. As observed two different metrics produce different edges on the graph. After an A* search for path finding between a pair of start (\keys) and goal (\keyg) configurations and collision check if required, paths are found for both roadmaps. The green lines denote the final path after collision check. As observed, the \textbf{learned} distance roadmap has a clearly more optimized path than the \textbf{image space} distance roadmap}
      \label{graph}
      \vspace{-1.4em}
\end{figure*}


\textsc{sample}: This function typically samples the configuration space uniformly. However, in the vision-only setting, we can't directly sample in \Kspace as we don't have the model of the robot. Since $dim(\K) > dim(\X)$ the keypoint vectors (image state \key) that correspond to actual configurations of the robot, will lie in a lower-dimensional (equal to $dim(\X)$) manifold in \Kspace. Thus 
randomly sampling \Kspace will have $0$ probability of sampling a valid configuration that lies on the valid manifold \cite{kingston2018sampling}. We describe how to address this issues in \autoref{ssec:dataset}, by collecting and storing valid samples directly from the real robot.  

\textsc{k-nearest}: This function usually relies on a distance defined in \C  and finds the nearest configurations that can be connected. However, since our representation in \Kspace is now a non-holonomic system, defining this distance is very challenging, as a straight line path in \Kspace defined by a simple Euclidean distance, might not be accurately followable by a controller. To mitigate this, we describe a learning-based approach to estimate the unknown joint-distance, in \autoref{ssec:dist_metric}.


\textsc{coll\_free}: This function checks if there is a collision for an edge in \C. Again this typically requires the model $R(x)$ for the robot. In the vision only case, we propose simple yet successful method to do collision checking with an $RI(x)$ directly in \Kspace \autoref{ssec:line-check}. 

In the next section we describe our proposed method, for each, of the aforementioned subroutines. \autoref{graph} visually describes all the steps of the visual motion planning framework.

\subsection{\textsc{sample}: Executed Trajectories as Proxy Samples} 
\label{ssec:dataset}
In the absence of a robot model, we represent each image state $\key \in \K$ by identifying and annotating keypoints (\skey)
 on the robotic arm using an automated data collection pipeline described in \cite{chatterjee2023keypoints, chatterjee2024utilizing}. Each \ensuremath{\key_n} is composed of a set of $N$ keypoints \skey, where each \skey represents a pixel point of a specific location on the robot's body in image space. For instance, if \( N = 5 \), a keypoint vector or image state \ensuremath{\key_n} is represented by a vector of {\ensuremath{\skey_1}, \ensuremath{\skey_2}, \ensuremath{\skey_3}, \ensuremath{\skey_4}, \ensuremath{\skey_5}}, where each \skey is a pixel point in the $n\textsuperscript{th}$ image frame. An example vector of such keypoints is shown in \autoref{config_space} as \keys or \keyg

 To systematically explore the robot's visible workspace in the image space and ensure comprehensive coverage, we compute velocities for each joint \( j \) using:

\begin{equation}
\label{comp_vel}
v_j = \min\left(\frac{M_j}{\ensuremath{res} \cdot t}, v_{\text{max}}\right),
\end{equation}

where, \( M_j \) is the motion range or difference of limits of joint \( j \), \ensuremath{res}, is the number of discrete  steps used to traverse \( M_j \), \( t \) is the duration allocated to complete each step, and \( v_{\text{max}} \) is the maximum allowable velocity for joint \( j \).
By dividing the motion range of each joint into uniform increments based on \ensuremath{res}, we ensure that the robot systematically explores all possible configurations within its workspace. The resulting dataset of image states \key is thus evenly distributed across the image space \K, enabling robust coverage and accurate representation of the robot's motion capabilities. 

Each configuration \ensuremath{\key_n} is computed using the following transformation:

\begin{equation}
\label{2d_eq}
\ensuremath{\key_n} = K \cdot T_{cw} \cdot \ensuremath{x_n}
\end{equation}
where $K$ is the camera intrinsic matrix, $T_{cw}$ is the camera extrinsics matrix derived from calibration processes described in \cite{Lee2020, Zhang2000},  and \ensuremath{x_n} is the 3D configuration robot in workspace. 
This transformation projects the 3D configuration \ensuremath{x_n}, into their corresponding 2D (pixel) projection in the image described in \cite{3dRecon}, resulting in a set of \skey for each image state \ensuremath{\key_n} The process captures the robot's motion across its visible workspace and creates a comprehensive dataset of \key, representing close to all feasible configurations in image space. Dataset of \ensuremath{\key} can also be collected by following the process describe in \cite{chatterjee2024utilizing}. 

The \textbf{Keypoint Dataset Samples} section of \autoref{graph}, represents this part of the workflow. The grey area in the left image of \autoref{config_space} illustrates how \key are distributed within image space with each frame consisting of a set of \skey or keypoints. This collection process yields a large dataset of \key that can be used to enable efficient roadmap construction for vision-based motion planning.

\subsection{\textsc{k-nearest}: Learned and Image-Space Metrics}
\label{ssec:dist_metric}
To search for the K-nearest neighbors of each image state sample we employ the following two distance metrics:

\begin{itemize}
\item \textit{Learned distance},  where the distance is learned by a neural network, trained to predict joint displacements between two image states. The input to the network is a pair of image states $\key_1$ and $\key_2$ and and the output is an estimated joint displacement required to transition between them:
\begin{equation}
\label{cust-eq}
\text{dist}_{learned}() \leftarrow \text{NN}(\key_1, \key_2)
\end{equation}
\item \textit{Distance in Image space}, where we simply calculate the Euclidean distance between $\key_1$ and $\key_2$ in image space:
\begin{equation}
\label{euc-eq}
\text{dist}_{image}() \leftarrow || \key_1 - \key_2 ||_2
\end{equation}
\end{itemize}
Each metric influences the graph's structure by determining the nearest neighbors and defining the edges in the roadmap. The learned distance prioritizes image states with minimal estimated joint displacement, while the image space distance favors image states that are closer in the image. These differences impact the connectivity of the roadmap, as illustrated in the \textbf{Connected Roadmap} and \textbf{Connect Start and Goal} sections of \autoref{graph}.

\subsubsection{Dataset generation for network model}
\label{ssec:approx-joint}
While collecting the dataset of image states \key in \autoref{ssec:dataset}, we recorded the velocity (\autoref{comp_vel}) applied to transition between consecutive frames. Using this data, we created a new dataset that includes pairs of consecutive image states (\ensuremath{\key_1}, \ensuremath{\key_2}) and their estimated joint displacements. This is calculated by multiplying the recorded velocity with the frame rate at which \key was captured, as described in \autoref{2d_eq}.

 Please note here, only pairs of \key captured in consecutive image frames are included in this data generation process.  However, for constructing the graph $G$, it is essential to compute distances between arbitrary \key pairs in \K-space. To achieve this, a neural network is employed to predict the joint displacement across different pairs of image-states \key.
 
To enhance diversity, the dataset is augmented by combining frame sequences where the first frame's image state (\(\key_{\text{start}}\)) and the last frame's image state (\(\key_{\text{end}}\)) act as boundaries, with total joint displacements calculated as sum of displacements across intermediate frames. This approach ensures diverse transitions, enabling the neural network to accurately estimate joint displacements for any image state pair.

\subsubsection{Neural Network for Learned Distance Metric}
\label{ssec:reg-model}
To derive \autoref{cust-eq}, we design a simple neural network using the aforementioned dataset to learn a distance more similar to the joint space distance. The model takes concatenated arrays of the starting image state (\(\key_{\text{start}}\)) and the subsequent image state (\(\key_{\text{next}}\)) as input and predicts the estimated joint displacement.

\subsection{\textsc{coll\_free}:Image-Based Collision Checking}
\label{ssec:line-check} 

To check for collisions in our model-free system, we employ an image-based polygon collision-checking framework. Polygons are formed by connecting pairs of consecutive keypoints ((\(\skey_{n-1}\)), (\(\skey_{n}\))) in one image state (\(\key_1\)) with their corresponding keypoints in the neighboring image state (\(\key_2\)). In \autoref{line-seg}, the polygons created by (\(\skey_1\), \(\skey_2\)),  (\(\skey_2\), \(\skey_3\)), (\(\skey_3\), \(\skey_4\)) and (\(\skey_4\), \(\skey_5\)) of the start image state (\(\key_1\)) in green and its neighboring image state (\(\key_2\)) in red are examples of such polygons. 

Each polygon is then checked for intersections with the obstacle boundaries defined by a safety margin which we consider to account for controller uncertainty. This polygon-based approach ensures a thorough and robust collision-checking process, effectively handling obstacles of any size. In \autoref{line-seg} illustrates an example where the obstacle, enclosed by the red safety margin, intersects with the polygons defined by ([\(\skey_2\), \(\skey_3\)],[\(\skey_3\), \(\skey_4\)] and [\(\skey_4\), \(\skey_5\)]), demonstrating the effectiveness of this method.
The collision checking process and \textbf{A*} search is shown in  \textbf{A* Search with Lazy Collision Checking} of \autoref{graph}.  

\begin{figure}[t]
      \centering
      \includegraphics[width=0.4\textwidth]{images/poly_check_comb_paper.pdf}
      % \vspace{-3.0em}
      \caption{Illustration of polygon-based collision checking for vision-only motion planning. In this image $4$ polygons (bordered with light blue) are defined by connecting consecutive keypoint pairs (\(\skey_1\), \(\skey_2\)),  (\(\skey_2\), \(\skey_3\)), (\(\skey_3\), \(\skey_4\)) and (\(\skey_4\), \(\skey_5\)) of the image state (\(\key_1\)) in green and its neighboring image state (\(\key_2\)) in red. The yellow obstacle enclosed by the purple safety margin is situated right on the polygons formed by ([\(\skey_2\), \(\skey_3\)],[\(\skey_3\), \(\skey_4\)] and [\(\skey_4\), \(\skey_5\)]), rendering these two configurations ineligible for a collision-free path.}
      \vspace{-0.5em}
      \label{line-seg}
\end{figure}

\subsection{Adaptive Visual Servoing}
This work builds on the adaptive visual servoing method described in \cite{gandhi2022skeleton}, using a roadmap of collision-free sequence of goal image states. At each goal, vector of keypoints (\skey) in image state (\key), is tracked as visual features as described in \cite{chatterjee2023keypoints}. The controller moves the arm minimizing the feature error, computed as the difference between the current and the target \key. The Jacobian matrix, estimated online via least-square optimization of recent joint velocities and keypoints vector over a moving window, eliminates the need to read joint position from  encoder. This makes the control pipeline completely model-free. To improve accuracy, we reset the Jacobian estimation window at each new target keeping the estimate unbiased and relevant to the current goal. Since the goals may not be evenly spaced in the image for different experiments, we use a saturation limit on the feature error to prevent sudden spikes in velocity ensuring stable motion and protecting the motor from damage.

\begin{figure}[t]
      \centering
      \includegraphics[width=0.9\columnwidth]{images/histogram_slide.pdf}
      \caption{Histograms of actual joint displacements along the edges for the three roadmaps. The \textbf{Learned} roadmap closely aligns with the distribution of the Joint Space roadmap, showing only slight deviations and indicating reasonable accuracy in the predicted joint displacements. In contrast, the \textbf{Image space} roadmap demonstrates a wider spread and larger joint displacements, reflecting its lack of alignment between image-space proximity and joint-space movements. This misalignment may reduce its efficiency in generating paths suitable for precise control.}
      \vspace{-1.5em}
      \label{histogram}
\end{figure}

\section{Experiments and Observations:}
We experimentally assesed the performance of the proposed vision only motion planning framework on a Franka Emika Panda Arm.

\subsection{Experimental details}

\subsubsection{Generating samples} We generated the required image state samples by collecting a large dataset created by actuating the planar joints (Joints $2,4,6$) of the Franka arm, using velocities computed from \autoref{comp_vel} as explained in \autoref{ssec:dataset}, covering the robot's planar workspace.

\subsubsection{ Generating the roadmap} The collected samples \key were used as nodes to generate roadmaps using the different proposed distances described in \autoref{ssec:dist_metric}. The roadmap generated by the image-space distance is coined as \textbf{Image Space} roadmap and the one generated using the learned distance is coined as \textbf{Learned} roadmap. For benchmarking purposes we also use the actual joint distance to create the Joint Space roadmap. The Joint Space roadmap used actual joint displacements from encoders solely for benchmarking, maintaining the proposed model-free framework. The k-neighbor value for all approaches was set to $25$.
 
\subsubsection{Obstacle Representation and Path Planning} In the experimental setup, obstacles were modeled as virtual yellow rectangles with a safety margin, as shown in \autoref{line-seg}. Paths were generated offline for various start (\keys) and goal (\keyg) incorporating the obstacle avoidance logic from \autoref{ssec:line-check}. These paths were later used in adaptive visual servoing experiments

\subsubsection{Control Experiment Set-up} 
 The real-time control experiments used an Intel Realsense D435i camera in an eye-to-hand setup for visual feedback. The Panda arm followed the planned paths in $16$ obstacle-free and $10$ obstacle-avoidance experiments. The performance of each proposed roadmap was evaluated by comparing the joint position changes between intermediate image states to those of the Joint Space roadmap. The controller’s ability to guide the arm along collision-free paths was evaluated for efficiency and effectiveness.

\subsection{Roadmap and Path Planning Experiments}
In this section, we analyze the joint displacements along the edges of the three roadmaps to evaluate their efficiency. Path planning experiments were conducted in both collision-free and obstacle-avoidance scenarios to compare the joint distances covered by the paths generated from each roadmap.

\subsubsection{Distribution of Joint Distances of Edges for Different Roadmaps}
\autoref{histogram} shows the histograms of joint displacements along the edges of the three roadmaps. The \textbf{Learned} roadmap closely aligns with the joint space roadmap with minor deviations, demonstrating reasonable accuracy in estimated joint displacements. In contrast, the \textbf{Image Space} roadmap exhibits a broader spread and larger joint displacements. This suggests that the Learned roadmap is more effective in accurately capturing transitions and generating paths that minimize joint movements.

\subsubsection{Average Joint Distances for Planned Collision-Free Paths}
\label{random-rm}
We randomly selected $100$ pairs of \keys and \keyg from the roadmaps for path planning without obstacles. The average joint distances over $100$ trials, summarized in \autoref{random_trials}, show that the \textbf{learned} roadmap yields results closer to the joint space while the \textbf{image space} roadmap results in significantly larger joint displacements. These findings suggest that the learned roadmap possibly generates more efficient path, better suited for successful control convergence.  

\begin{table}[t]
\caption{Average Joint Distances (Radians) for Collision-Free Paths}
\label{random_trials}
\centering
\resizebox{0.5\textwidth}{!}{%
\begin{tabular}{|m{2cm}||m{2cm}||m{2cm}||m{2cm}||}
\hline
& \multicolumn{1}{m{1cm}||}{\centering \textbf{Joint Space}} & \multicolumn{1}{m{1cm}||}{\centering \textbf{Learned}} & \multicolumn{1}{m{1cm}|}{\centering \textbf{Image} \\ \centering \textbf{Space}} \\
\hline
\hline
\multicolumn{1}{|m{2cm}||}{\centering \textbf{Mean} \\ \centering \textbf{(radians)}}  & \multicolumn{1}{m{1cm}||} {\centering $\textbf{1.74}$} & \multicolumn{1}{m{1cm}||}{\centering $\textbf{2.19}$} & \multicolumn{1}{m{1cm}|}{\centering $\textbf{3.06}$} \\
\hline
\end{tabular}
} 
\vspace{-0.75em}
\end{table}

\subsubsection{Planned paths for Control Experiments}

\label{ssec:path_planning}
We generated planned paths for two scenarios: $16$ start and goal pairs without checking for collision and $10$ pairs with collision avoidance. Paths were computed using the three roadmaps: Joint Space, Learned, and Image Space. These precomputed paths were used in the control experiments described in \autoref{all_control_exps}. 

For each planned path, we calculated metrics including the average number of waypoints (intermediate image states), joint distances between waypoints, total joint distances for the entire path, and Euclidean distances between image states (keypoint distances) both between waypoints and across the entire path. 

As observed in \autoref{exps_free_and_obs} the \textbf{learned} roadmap consistently resulted in fewer waypoints and shorter joint distances compared to the \textbf{image space} roadmap, which prioritizes minimizing keypoints distances in image space but incurs higher joint displacements.

Notably, the joint distances required to traverse $1000$ pixels in image space were much higher for the Image space roadmap than for the Learned roadmap. This suggests that reliance on image space proximity may lead to less efficient joint-space paths.

\begin{table}[t]
\caption{Comparison of Joint Distances and Keypoints Distance in Image Space over Experiments}
\label{exps_free_and_obs}
\centering
\resizebox{0.45\textwidth}{!}{%
\begin{tabular}{|m{2cm}||m{1cm}|m{1cm}|m{1cm}||m{1cm}|m{1cm}|m{1cm}|}
\hline
& \multicolumn{3}{c||}{\textbf{Without Collision-check}} & \multicolumn{3}{c|}{\textbf{With Collision-check}} \\
\hline
\textbf{Roadmaps} & \textbf{Joint Space} & \textbf{Learned} & \textbf{Image Space} & \textbf{Joint Space} & \textbf{Learned} & \textbf{Image Space} \\
\hline
\textbf{Number of Experiments} & \textbf{16} & \textbf{16} & \textbf{16} & \textbf{10} & \textbf{10} & \textbf{10} \\
\hline
\textbf{Avg No. of Waypoints} & \textbf{8} & \textbf{8} & \textbf{14} & \textbf{11} & \textbf{13} & \textbf{15} \\
\hline
\textbf{Avg. Joint Distances (radians) b/w Waypoints} & \textbf{0.25} & \textbf{0.3} & \textbf{0.32} & \textbf{0.28} & \textbf{0.28} & \textbf{0.35} \\
\hline
\textbf{Avg. Keypoints Distances (pixels) b/w Waypoints} & \textbf{167.99} & \textbf{174.76} & \textbf{84.11} & \textbf{139.53} & \textbf{129.13} & \textbf{89.36} \\
\hline
\textbf{Avg. Joint Distances (radians) over Entire Path} & \textbf{1.92} & \textbf{2.27} & \textbf{4.29} & \textbf{3.04} & \textbf{3.46} & \textbf{5.36} \\
\hline
\textbf{Avg. Keypoints Distances (pixels) over Entire Path} & \textbf{1222.42} & \textbf{1278.92} & \textbf{1157.00} & \textbf{1532.72} & \textbf{1569.37} & \textbf{1361.43} \\
\hline
\textbf{Joint Distance (radians) Traversed To Move 1000 pixels in image space} & \textbf{1.6} & \textbf{1.77} & \textbf{3.79} & \textbf{1.98} & \textbf{2.19} & \textbf{3.9} \\
\hline
\end{tabular}
} 
\vspace{-1.85em}
\end{table}
We have two theories from the above observations:

\begin{itemize}
    \item The larger number of waypoints in the image space roadmap arises from its Euclidean distance-based edge weights, causing A* to prioritize shorter pixel distances and select more intermediate nodes. In contrast, the learned roadmap, with joint displacement-based weights, produces more direct paths with fewer waypoints.
    \item The image states \key which are close in image space may be much further away in joint space as highlighted in \autoref{exps_free_and_obs}. This behavior may lead to increased overall joint movement where non-holonomy may exist in the joint space for the covered joint displacements. This may reduce control accuracy and hinder convergence to the target image state.
\end{itemize}


\subsection{Control Experiments}
\label{all_control_exps}
The adaptive visual servoing experiments\footnote{Planning and Control Experiments videos are available at \href{https://drive.google.com/file/d/1eOoP0dVFz85q4usiLzjlPdA5PTBA3UKU/view?usp=drive_link}{this link}. The details of how to use the link is in the supplementary ReadMe file} used precomputed paths from  \autoref{ssec:path_planning}, with control gains optimized to minimize rise and settling times while keeping overshoot within 5\%, by careful tuning.

In \autoref{performance_data}, the overall control metrics highlight that the image space roadmap succeeded in only $69.2$\% of cases, while the learned roadmap achieved a $100$\% success rate. However, the image space roadmap, when successful, showed faster transient responses compared to the Learned roadmap. \autoref{exps_success_failure} uses identical metrics as in \autoref{exps_free_and_obs}, categorizing experiments into successful and failed cases

Failures in the image space roadmap were characterized by large joint displacements relative to smaller image space distances, as noted earlier in \autoref{ssec:path_planning}. Optimal execution of the references generated by using image space distances requires all the keypoints to follow straight paths. This is not possible for the robot due to the non-holonomic constraints of its kinematics (if defined directly in image space). As a result, the robot deviates from the planned path. While the robot reaches the reference locations in most cases, the deviations cause it to go out of its workspace (due to joint limits) in some others. This effect is especially visible when obstacles exist in the workspace since the robot needs to travel near the edges of its workspace to avoid them. When estimated joint distances are used the resulting trajectories are more suitable to the robot's kinematics, which prevents such failures at large. 


\begin{table}[t]
\caption{Performance results for the control experiments with and without obstacle}
\label{performance_data}
\centering
\resizebox{0.45\textwidth}{!}{%
\begin{tabular}{|m{1cm}||m{1cm}||m{1cm}||m{1cm}||m{1cm}||m{1cm}||m{1cm}|}
\hline
& \multicolumn{3}{c|||}{\textbf{Without Collision-check}} & \multicolumn{3}{c|}{\textbf{With Collision-check}} \\
\hline
 \multicolumn{1}{|m{2cm}||}{\centering \textbf{Performance} \\ \centering \textbf{Metrics}} & \multicolumn{1}{m{1cm}||}{\centering \textbf{Joint} \\ \centering \textbf{Space}} & \multicolumn{1}{m{1cm}||}{\centering \textbf{Learned}} & \multicolumn{1}{m{1cm}|||}{\centering \textbf{Image} \\ \centering \textbf{Space}} & \multicolumn{1}{m{1cm}||}{\centering \textbf{Joint} \\ \textbf{Space}} & \multicolumn{1}{m{1cm}||}{\centering \textbf{Learned}} & \multicolumn{1}{m{1cm}|}{\centering \textbf{Image} \\ \centering \textbf{Space}} \\
\hline
\hline
\multicolumn{1}{|m{2cm}||}{\centering \textbf{Successful} \\ \textbf{Experiments}} & \multicolumn{1}{m{1cm}||} {\centering \textbf{16/16}} & \multicolumn{1}{m{1cm}||}{\centering \textbf{16/16}} & \multicolumn{1}{m{1cm}|||}{\centering \textbf{13/16}} & \multicolumn{1}{m{1cm}||}{\centering \textbf{10/10}} & \multicolumn{1}{m{1cm}||}{\centering \textbf{10/10}} & \multicolumn{1}{m{1cm}|}{\centering \textbf{5/10}} \\
\hline
\hline
\multicolumn{1}{|m{2cm}||}{\centering \textbf{System} \\ \textbf{Rise time (s)}} & \multicolumn{1}{m{1cm}||}{\centering \textbf{74.92}} & \multicolumn{1}{m{1cm}||}{\centering \textbf{97.53}} & \multicolumn{1}{m{1cm}|||}{\centering \textbf{94.75}} & \multicolumn{1}{m{1cm}||}{\centering \textbf{105.38}} & \multicolumn{1}{m{1cm}||}{\centering \textbf{125.38}} & \multicolumn{1}{m{1cm}|}{\centering \textbf{90.46}} \\
\hline
\multicolumn{1}{|m{2cm}||}{\centering \textbf{System} \\ \textbf{Settling time (s)}} & \multicolumn{1}{m{1cm}||}{\centering \textbf{94.37}} & \multicolumn{1}{m{1cm}||}{\centering \textbf{118.62}} & \multicolumn{1}{m{1cm}|||}{\centering \textbf{101.99}} & \multicolumn{1}{m{1cm}||}{\centering \textbf{118.18}} & \multicolumn{1}{m{1cm}||}{\centering \textbf{155.14}} & \multicolumn{1}{m{1cm}|}{\centering \textbf{96.44}} \\
\hline
\multicolumn{1}{|m{2cm}||}{\centering \textbf{End effector} \\ \textbf{Rise time (s)}} & \multicolumn{1}{m{1cm}||}{\centering \textbf{74.86}} & \multicolumn{1}{m{1cm}||}{\centering \textbf{97.53}} & \multicolumn{1}{m{1cm}|||}{\centering \textbf{94.39}} & \multicolumn{1}{m{1cm}||}{\centering \textbf{105.02}} & \multicolumn{1}{m{1cm}||}{\centering \textbf{125.38}} & \multicolumn{1}{m{1cm}|}{\centering \textbf{90.46}} \\
\hline
\multicolumn{1}{|m{2cm}||}{\centering \textbf{End effector} \\ \textbf{Settling time (s)}} & \multicolumn{1}{m{1cm}||}{\centering \textbf{94.37}} & \multicolumn{1}{m{1cm}||}{\centering \textbf{114.02}} & \multicolumn{1}{m{1cm}|||}{\centering \textbf{99.81}} & \multicolumn{1}{m{1cm}||}{\centering \textbf{118.18}} & \multicolumn{1}{m{1cm}||}{\centering \textbf{147.14}} & \multicolumn{1}{m{1cm}|}{\centering \textbf{94.22}} \\
\hline
\multicolumn{1}{|m{2cm}||}{\centering \textbf{Overshoot (\%)}} & \multicolumn{1}{m{1cm}||}{\centering \textbf{1.94}} & \multicolumn{1}{m{1cm}||}{\centering \textbf{2.61}} & \multicolumn{1}{m{1cm}|||}{\centering \textbf{1.89}} & \multicolumn{1}{m{1cm}||}{\centering \textbf{1.93}} & \multicolumn{1}{m{1cm}||}{\centering \textbf{2.36}} & \multicolumn{1}{m{1cm}|}{\centering \textbf{1.35}} \\
\hline
\multicolumn{1}{|m{2cm}||}{\centering \textbf{Execution} \\ \textbf{time (s)}} & \multicolumn{1}{m{1cm}||}{\centering \textbf{125.17}} & \multicolumn{1}{m{1cm}||}{\centering \textbf{148.92}} & \multicolumn{1}{m{1cm}|||}{\centering \textbf{140.78}} & \multicolumn{1}{m{1cm}||}{\centering \textbf{146.02}} & \multicolumn{1}{m{1cm}||}{\centering \textbf{201.68}} & \multicolumn{1}{m{1cm}|}{\centering \textbf{124.70}} \\
\hline
\end{tabular}
} 
\end{table}

\begin{table}[t]
\caption{Comparison of Joint Distances and Keypoints Distance in Image Space over Experiments for Successful and Failed Experiments}
\label{exps_success_failure}
\centering
\resizebox{0.45\textwidth}{!}{%
\begin{tabular}{|m{2cm}||m{1cm}|m{1cm}|m{1cm}||m{1cm}|m{1cm}|m{1cm}|}
\hline
& \multicolumn{3}{c||}{\textbf{Successful}} & \multicolumn{3}{c|}{\textbf{Failed (Image Space)}} \\
\hline
\textbf{Roadmaps} & \textbf{Joint Space} & \textbf{Learned} & \textbf{Image Space} & \textbf{Joint Space} & \textbf{Learned} & \textbf{Image Space} \\
\hline
\textbf{Number of Experiments} & \textbf{18} & \textbf{18} & \textbf{18} & \textbf{8} & \textbf{8} & \textbf{8} \\
\hline
\textbf{Avg No. of Waypoints} & \textbf{8} & \textbf{8} & \textbf{14} & \textbf{11} & \textbf{13} & \textbf{15} \\
\hline
\textbf{Avg. Joint Distances (radians) b/w Waypoints} & \textbf{0.25} & \textbf{0.3} & \textbf{0.32} & \textbf{0.27} & \textbf{0.29} & \textbf{0.35} \\
\hline
\textbf{Avg. Keypoints Distances (pixels) b/w Waypoints} & \textbf{166.77} & \textbf{163.41} & \textbf{86.03} & \textbf{135.17} & \textbf{143.24} & \textbf{86.35} \\
\hline
\textbf{Avg. Joint Distances (radians) over Entire Path} & \textbf{2.10} & \textbf{2.37} & \textbf{4.42} & \textbf{2.92} & \textbf{3.52} & \textbf{5.33} \\
\hline
\textbf{Avg. Keypoints Distances (pixels) over Entire Path} & \textbf{1307.02} & \textbf{1257.02} & \textbf{1210.93} & \textbf{1419.93} & \textbf{1691.26} & \textbf{1291.2} \\
\hline
\textbf{Joint Distance (radians) Traversed To Move 1000 pixels in image space} & \textbf{1.6} & \textbf{1.9} & \textbf{3.71} & \textbf{2.1} & \textbf{2.1} & \textbf{4.12} \\
\hline
\end{tabular}
} 
\vspace{-1.05em}

\end{table}


To summarize, both the \textbf{image space} and \textbf{learned} roadmaps exhibit unique benefits. The image space roadmap provides faster execution when successful but struggles in reliability and path convergence. The \textbf{learned} roadmap, using joint displacement-based distances, avoids non-holonomic constraints and ensures robustness, particularly in complex paths. The choice between the two ultimately depends on the specific application context.

\section{Conclusion and Future Work}
In conclusion, this work introduced a novel framework for collision-free motion planning of robotic manipulators that relied solely on visual features, eliminating the need for explicit robot models or encoder feedback. 

The \textbf{learned} roadmap offered smoother, more reliable transitions, and due to its joint displacement-based distance definition, the paths it generated maintained joint-space holonomy when it existed. In contrast, the paths produced by the \textbf{image space} roadmap sometimes failed to maintain joint-space holonomy, even when holonomy existed in the image space. However, the \textbf{image space} roadmap provided faster transient responses and simplicity, making it advantageous for applications where speed and computational efficiency were prioritized.

Future work will explore extending this approach to out-of-plane motion and incorporating real-world obstacles to develop a fully integrated control and manipulation pipeline.

\bibliographystyle{IEEEtran}
%%%%%%%%%%%%%%%%%%%%%%%%%%%%%%%%%%%%%%%%%%%%%%%%%%%%%%%%%%%%%%%%%%%%%%%%%%%%%%%%
%2345678901234567890123456789012345678901234567890123456789012345678901234567890
%        1         2         3         4         5         6         7         8

\documentclass[letterpaper, 10 pt, conference]{ieeeconf}  % Comment this line out if you need a4paper

%\documentclass[a4paper, 10pt, conference]{ieeeconf}      % Use this line for a4 paper

\IEEEoverridecommandlockouts                              % This command is only needed if 
                                                          % you want to use the \thanks command

\overrideIEEEmargins                                      % Needed to meet printer requirements.

%In case you encounter the following error:
%Error 1010 The PDF file may be corrupt (unable to open PDF file) OR
%Error 1000 An error occurred while parsing a contents stream. Unable to analyze the PDF file.
%This is a known problem with pdfLaTeX conversion filter. The file cannot be opened with acrobat reader
%Please use one of the alternatives below to circumvent this error by uncommenting one or the other
%\pdfobjcompresslevel=0
%\pdfminorversion=4

% See the \addtolength command later in the file to balance the column lengths
% on the last page of the document

% The following packages can be found on http:\\www.ctan.org
%\usepackage{graphics} % for pdf, bitmapped graphics files
%\usepackage{epsfig} % for postscript graphics files
%\usepackage{mathptmx} % assumes new font selection scheme installed
%\usepackage{times} % assumes new font selection scheme installed
%\usepackage{amsmath} % assumes amsmath package installed
%\usepackage{amssymb}  % assumes amsmath package installed
\usepackage{amsmath}
\usepackage{amssymb}
\usepackage{hyperref}
\usepackage{multirow}
\usepackage{graphicx}


\title{\LARGE \bf
4DR P2T: 4D Radar Tensor Synthesis with Point Clouds
}


\author{Woo-Jin Jung, Dong-Hee Paek, and Seung-Hyun Kong% <-this % stops a space
\thanks{This work was supported by the National Research Foundation of Korea(NRF) grant funded by the Korea government(MSIT) (No. 2021R1A2C3008370).}% <-this % stops a space
\thanks{Woo-Jin Jung, Dong-Hee Paek, and Seung-Hyun Kong are with the CCS Graduate School of Mobility, Korea Advanced Institute of Science and Technology, Daejeon, Korea, 34051 
        {\tt\small \{woo-jin.jung, donghee.paek, skong\}@kaist.ac.kr}}}%



\begin{document}



\maketitle
\thispagestyle{empty}
\pagestyle{empty}


%%%%%%%%%%%%%%%%%%%%%%%%%%%%%%%%%%%%%%%%%%%%%%%%%%%%%%%%%%%%%%%%%%%%%%%%%%%%%%%%
\begin{abstract}

In four-dimensional (4D) Radar-based point cloud generation, clutter removal is commonly performed using the constant false alarm rate (CFAR) algorithm. However, CFAR may not fully capture the spatial characteristics of objects. To address limitation, this paper proposes the 4D Radar Point-to-Tensor (4DR P2T) model, which generates tensor data suitable for deep learning applications while minimizing measurement loss. Our method employs a conditional generative adversarial network (cGAN), modified to effectively process 4D Radar point cloud data and generate tensor data. Experimental results on the K-Radar dataset validate the effectiveness of the 4DR P2T model, achieving an average PSNR of 30.39dB and SSIM of 0.96. Additionally, our analysis of different point cloud generation methods highlights that the 5\% percentile method provides the best overall performance, while the 1\% percentile method optimally balances data volume reduction and performance, making it well-suited for deep learning applications.

\end{abstract}


%%%%%%%%%%%%%%%%%%%%%%%%%%%%%%%%%%%%%%%%%%%%%%%%%%%%%%%%%%%%%%%%%%%%%%%%%%%%%%%%
\section{INTRODUCTION}

In recent autonomous driving research, 4D Radar has gained increasing attention as an advanced sensing technology. Traditional Radar sensors, often employed as auxiliary sensors, measure range, azimuth, and Doppler information. In contrast, 4D Radar incorporates elevation into these measurements, enabling more precise spatial perception. Consequently, 4D Radar provides more robust measurements than cameras and LiDAR under adverse weather conditions such as snow or rain. Furthermore, it surpasses traditional Radar in detecting object contours, demonstrating superior object detection capabilities. Owing to these advantages, 4D Radar has emerged as a key sensing modality in autonomous driving systems, offering enhanced object detection across diverse operational environments.

Most 4D Radar data are provided as point clouds, which are typically generated by traditional handcrafted methods such as CFAR to remove clutter \cite{clutter} from the tensor data. However, CFAR processes each cell independently, disregarding spatial continuity across adjacent cells. As a result, CFAR-generated point clouds often fail to preserve essential spatial characteristics—such as object size, shape, and continuous contours—thereby limiting their ability to accurately represent complex objects \cite{4D_radar_survey}. In autonomous driving scenarios where objects vary in size and shape, this limitation constrains environmental perception. Moreover, CFAR-based point clouds typically exhibit much lower point density than LiDAR, reducing the fidelity of captured object features \cite{rpfa_net} and complicating subsequent sensor fusion processes \cite{dpft}.

\begin{figure}[t!]
  \centering
  \includegraphics[width=1.0\columnwidth]{fig/fig0.png}
  \caption{
   4DR P2T overview. The 4DR P2T model generates tensor data from 4D Radar point clouds, which are represented in bird’s-eye view (BEV) as a 2D projection. Traditional point cloud generation methods often suffer from measurement loss, which may affect their suitability for deep learning training. To mitigate this limitation, the model generates tensor data to prevent measurement loss, ensuring that crucial information is retained for deep learning tasks.}
  \label{fig0.overview}
\end{figure}

\begin{figure*}[!th]
 \centering
\vspace{1mm} 
 \includegraphics[width=1.0\textwidth]{fig/fig1.png}
    \caption{4D Radar signal processing and data representation \cite{4D_radar_survey, 4dradar_tutorial, 4dradar_data_representation}. The Radar power values are normalized and represented using colors. The Radar point cloud is shown as black points, and the bounding box for the objects is indicated with a red box.}
  \label{fig1.4d_radar_data}

\end{figure*}

To mitigate these limitations, previous studies have proposed methods for reconstructing points representing objects \cite{3DRIMR} or generating tensor data prior to CFAR \cite{radarpointgenerator1} \cite{4D_radar_survey}. One notable method \cite{radarpointgenerator1} utilizes a conditional generative adversarial network (cGAN) with a UNet \cite{unet} architecture, leveraging LiDAR data to supervise the generation of denser Radar point clouds. However, fundamental differences between LiDAR (near-infrared) and Radar (electromagnetic waves) result in heterogeneous data characteristics, leading to distortions in power values and contour representations, which may degrade the reliability of the generated Radar data.

As shown in Fig. \ref{fig0.overview}, a method is required to directly generate tensor data using the original 4D Radar tensor as supervision, thereby avoiding cross-sensor inconsistencies. In this study, we leverage the K-Radar dataset \cite{KRadar}, currently the only publicly available dataset that provides 4D tensor data. Prior to its release in 2023, no dataset included 4D tensor data, making direct data-driven methods infeasible. With this new dataset, it is now possible to train models that generate tensor representations from 4D Radar point clouds collected by the same sensor.

Accordingly, we propose the 4D Radar Point cloud-to-Tensor (4DR P2T) model, which utilizes a cGAN-based architecture to generate tensor data from 4D Radar point clouds. This study conducts two primary investigations. First, we identify the point cloud generation method that achieves the best tensor generation performance—measured by peak signal-to-noise ratio (PSNR) and structural similarity index measure (SSIM)—among CFAR \cite{rtnh+} and percentile-based methods \cite{KRadar, enhancekradar} with different densities. Second, we determine the optimal point cloud generation method for deep learning applications, specifically the one that minimizes data volume while preserving sufficiently high tensor generation performance. To enable these investigations, we interpret point cloud data as the encoded version of tensor data, with our 4DR P2T model serving as a decoder that generates the original tensor. Consequently, the tensor generation performance of the 4DR P2T model serves as a proxy for assessing how well a given point cloud preserves environmental information, which in turn facilitates the selection of the most suitable point cloud generation method for deep learning model training and interpretation. Through our experiments, the proposed 4DR P2T model achieves an average PSNR of 30.39dB and SSIM of 0.96, demonstrating its effectiveness and stability. Our findings reveal that the percentile 5\% method yields the best tensor generation performance, while the percentile 1\% method offers an optimal balance between data volume reduction and performance, making it well-suited for deep learning training.

The key contributions of this study are summarized as follows:
\begin{itemize}
    \item Development of the 4DR P2T model, which generates tensor data from 4D Radar point cloud data.
    \item Experimental validation showing that the percentile 5\% data provides the best tensor generation performance.
    \item Confirmation that the percentile 1\% method effectively reduces data volume while maintaining high tensor generation performance.
\end{itemize}

This paper is organized as follows. Section \ref{sec:related_works} discusses 4D Radar signal processing and data generation processes, and reviews related models that convert point cloud data into tensors. Section \ref{sec:method} describes the proposed model architecture. Section \ref{sec:experiments} presents and analyzes the quantitative and qualitative experimental results. Finally, Section \ref{sec:conclusion} concludes the paper and discusses future research directions.

\section{Related work} \label{sec:related_works}
In this section, we provide an overview of related works, focusing specifically on 4D Radar signal processing and data generation, as well as previous studies on data translation methods using cGANs, which form the basis for developing models that generate tensor data from point cloud data.

\begin{figure*}[!th]
    \centering
    \vspace{1mm} % 위쪽 여백 추가
    \includegraphics[width=1\textwidth]{fig/fig2.png} 
    \caption{Overall structure of 4DR P2T. The encoder utilizes 3D sparse convolution to process 4D Radar point cloud data, while the decoder employs 3D dense convolution to generate tensor data.}
    \label{fig2.model}
\end{figure*}

\subsection{4D Radar signal processing and data generation}
The 4D Radar signal processing and data generation process, as applied in autonomous driving, is illustrated in Figure \ref{fig1.4d_radar_data}. The core analog components of a 4D Radar system consist of a synthesizer, transmission (TX) antennas, reception (RX) antennas, and a mixer. The TX antennas emit electromagnetic waves, which reflect off objects in the environment and are received by the RX antennas. The transmitted signal is generated by the synthesizer and radiated through the TX antennas. This signal is a frequency-modulated continuous wave (FMCW), composed of a sequence of frequency-modulated signals, commonly referred to as chirps.

The signal emitted by the TX antennas and the signal received by the RX antennas are combined using a mixer, producing an intermediate frequency (IF) signal. This IF signal represents the frequency difference between the transmitted and received signals, which is used to extract the distance and velocity of the reflected objects. The generated IF signal is then converted into a digital form through an analog-to-digital converter (ADC), creating ADC sample data. This data is separated into a fast time axis, which calculates range information through chirp sampling, and a slow time axis, which calculates Doppler information through frame sampling.

The ADC sample data is processed through a 2D Fast Fourier Transform (FFT), which is applied to perform range FFT and Doppler FFT. The range FFT estimates the distance to objects, while the Doppler FFT estimates their relative velocity, resulting in the generation of an RD heatmap. Although the RD heatmap contains information about range and velocity, it does not include azimuth or elevation information, making it less intuitive to interpret.

To extract azimuth and elevation information, an additional angle FFT is applied to the RD heatmap. The angle FFT utilizes the positional information of the TX and RX antennas arrays in a multiple-input and multiple-output (MIMO) antennas design to analyze the phase differences in the reflected signals. This process generates a 4D tensor that includes range, azimuth, elevation, and Doppler information, with each tensor cell representing the corresponding signal strength. The 4D tensor is represented in a polar coordinate system, but for better interpretability, the visualization shown in the Fig. \ref{fig1.4d_radar_data} is converted into a Cartesian coordinate system.

The generated 4D tensor data is filtered using the CFAR method. CFAR dynamically adjusts the threshold by comparing the signal strength of each cell to its surrounding cells, effectively removing noise and identifying actual targets. This filtering process is applied across all dimensions of the tensor, ultimately producing point cloud data that contains information about actual targets.

The resulting point cloud data includes the position (range, azimuth, elevation) and Doppler of the detected objects and is utilized in various autonomous driving applications, such as object detection and tracking \cite{4dradar_tutorial, 4D_radar_survey, FFT-RadNet}.

\subsection{Image translation}

Image translation focuses on style translation while preserving key information. Notable methods include pix2pix \cite{pix2pix} and pix2pixHD \cite{pix2pixhd}. These methods utilize cGANs to translate input images into output images. These methods have been successfully applied to various image synthesis and transformation tasks \cite{radsimreal, l2r, radarpointgenerator1}. Pix2pix employs a U-Net-based generator and a patch-based discriminator, enabling applications such as image synthesis and color translation. Pix2pixHD extends this framework to handle high-resolution images by incorporating boundary maps, multi-scale generators, and multi-scale discriminators, achieving improved quality. These methods excel at 2D image-to-image translation while maintaining structural information.

However, this study deals with generating tensor data from 4D Radar point cloud data, making it challenging to directly apply conventional image translation models. Existing methods are primarily optimized for 2D image data, necessitating structural modifications to handle higher-dimensional data such as point clouds and tensors. To address this, this study extends the fundamental method of pix2pixHD by modifying the model architecture to effectively process 3D or higher-dimensional data.

\section{Method} \label{sec:method}

This section outlines the data dimensions used for training, the model architecture, and the objective function.

\subsection{Data preparation}
In this study, the 4D Radar tensor data is reduced to 3D spatial information by excluding Doppler information for the training process. As a result, the input data for training consists of four channels, including \(x\), \(y\), \(z\) coordinates, and power values. Including Doppler data would require processing additional values beyond the existing spatial information \((x, y, z)\) and power, necessitating the use of convolution layers with at least four dimensions. This would significantly increase the complexity of the model and computational costs, making it challenging to achieve the primary goal of verifying implementation feasibility in the initial stage of the research. Moreover, according to RTNH \cite{KRadar}, an early model utilizing 4D Radar tensor data, excluding Doppler information still achieves sufficient object detection performance. Therefore, this study focuses on minimizing model complexity while verifying the feasibility of generating tensor data from 4D Radar point cloud data. This method also lays the foundation for future studies incorporating Doppler information.


\subsection{Model structure}
The proposed model is inspired by image translation methods, such as pix2pix and pix2pixHD, and referenced recent studies like l2r \cite{l2r} and RadSimReal \cite{radsimreal} to balance generative feasibility and structural simplicity, while optimizing for input and output data dimensions. To capture the spatial characteristics of Radar point cloud data, the model employs an encoding method based on Voxelnet \cite{voxelnet}, drawing from Lee’s method \cite{l2rtranslation_voxel}. The 4DR P2T model extends the U-Net structure \cite{unet}, commonly used in image translation tasks, with modifications to process 3D data.

As illustrated in Figure \ref{fig2.model}, the encoder uses 3D sparse convolution layers to account for the sparsity of 4D Radar point cloud data. Sparse convolution layers \cite{sparse_conv} are employed in stages where spatial resolution is reduced, while submanifold sparse convolution layers \cite{submaninfold} are utilized for operations where spatial resolution remains unchanged, thereby enhancing feature representation. In the decoder, 3D dense convolution layers are used to generate a dense 3D tensor. This method performs computations across all regions, making it suitable for producing complete tensors.

The generated tensor data is evaluated using a multi-scale discriminator  \cite{pix2pixhd}, which determines the authenticity of the data. To handle dense data, the discriminator also incorporates 3D dense convolution layers.

\subsection{Objective functions}
4DR P2T adopts a training framework using a Generator \(G\) and a Discriminator \(D\), inspired by traditional image translation methods. While image translation typically aims for a one-to-many mapping to generate diverse outputs, this study focuses on a one-to-one mapping, necessitating the design of appropriate loss functions. Following the method by Wang \cite{l2r}, the final loss function is defined as follows:
\begin{align}
\mathcal{L}(G, D) &= \mathcal{L}_{cGAN}(G, D) + \lambda_{L1} \mathcal{L}_{L1}(G) \nonumber \\
&\quad + \lambda_{perc} \mathcal{L}_{perc}(G) \label{eq:loss}
\end{align}
where \(\lambda_{L1}\) and \(\lambda_{perc}\) are weights that control the importance of each loss component, ensuring balanced training.

% \subsubsubsection{Conditional Adversarial Loss ({L}_{cGAN})}
First, a conditional adversarial loss is used, where the \(G\) synthesizes data, and the \(D\) learns to distinguish between real and synthesized data. This is the core loss of cGANs, defined as:
\begin{align}
\mathcal{L}_{cGAN}(G, D) &= \mathbb{E}_{x,y}[\log D(x, y)] \nonumber \\
&\quad + \mathbb{E}_{x}[\log (1 - D(x, G(x)))] \label{eq:cgan_loss}
\end{align}
Here, \(x\) represents input data, \(y\) is the GT, and \(G(x)\) is the output of the \(G\). The \(D\) learns to differentiate \(G(x)\) from \(y\), while the \(G(x)\) is trained to deceive \(D\) by making \(G(x)\) resemble \(y\).

% \subsubsubsection{L1 Loss ({L}_{L1})}
Second, L1 loss minimizes the absolute error between synthesized data \(G(x)\) and GT \(y\). This simple and stable loss function ensures that the synthesized data closely resembles real data:
\begin{equation}
\mathcal{L}_{L1}(G) = \mathbb{E}_{x,y}\left[|G(x) - y|_1\right]
\label{eq:l1_loss}
\end{equation}
% \subsubsubsection{Perceptual Loss ({L}_{perc})}
Third, perceptual loss introduced in pix2pixHD is used to compare high-level feature distributions between synthesized and real data. By leveraging intermediate layer outputs from a pre-trained neural network, perceptual loss measures semantic differences, guiding the synthesized data to have similar high-level features to real data.

\begin{figure*}[!th]
{
  \centering
\vspace{1mm} 
 \includegraphics[width=1.0\textwidth]{fig/fig3.png}
    \caption{Qualitative experimental results of 4DR P2T. The top part shows the front camera image and LiDAR point cloud as reference data to understand the scene of the 4D Radar GT tensor data, while the bottom part presents the tensor data results generated by 4DR P2T under different point cloud generation methods and density conditions.}
  \label{fig3.result}
}
\end{figure*}

\section{Experiments} \label{sec:experiments}

This section describes the datasets and implementation details for training the 4DR P2T model, the evaluation metrics used, and the results and analysis of tensor generation performance.

\subsection{K-Radar dataset}
The K-Radar dataset \cite{KRadar}, which was used to train the 4DR P2T model, is the only dataset that provides 4D Radar tensor data (4DRT) consisting of the four dimensions: range, azimuth, elevation, and Doppler. This makes it of significant value. Additionally, K-Radar includes data from various weather conditions (clear, cloudy, fog, rain, sleet, light snow, and heavy snow), which distinguishes it from other autonomous driving 4D Radar datasets. Furthermore, the dataset includes 4D Radar data, high-resolution LiDAR data, and camera data, with 93.3K object labels for 35K frames, distributed across 58 different driving scenes.

K-Radar includes not only 4DRT tensor data but also point cloud data with CFAR applied, as used in the experiments of RTNH+ \cite{rtnh+}, and point cloud data with the percentile method applied, as used in the RTNH model \cite{KRadar}. The percentile method is effective in reducing memory and computational complexity while preserving the structure of tensor data, and thus was used as input data for training the RTNH model.

\subsection{Implementation details}
The experiments in this study were conducted using the K-Radar dataset, with data generated using various point cloud generation methods and density conditions for comparison. Specifically, point cloud data generated using the percentile method (top 0.1\%, 1\%, 5\%, 10\%) from enhanced K-Radar \cite{enhancekradar}, and point cloud data with hyper-parameter ($K_{1}$) of 2.5\% ($N_{2.5,a}$) and 10\% ($N_{10,a}$) using constant average CFAR from RTNH+ \cite{rtnh+} were used. These datasets were selected due to the significant differences in the point cloud distribution, making them suitable for comparison analysis.

The point cloud data used for training was extracted from 4D tensors in polar coordinates using CFAR or the percentile method and then converted into Cartesian coordinates. In this process, it can be observed that points become increasingly sparse as the range (distance) increases (Fig. \ref{fig0.overview}). The tensor data used for training was reconstructed into a dense cube shape through interpolation after converting from polar to Cartesian coordinates \cite{KRadar}. This data preparation process was set up to verify whether sparse point cloud data could be transformed into dense Cartesian tensor data and to expand its range of applicability.

The Region of Interest (ROI) was set as $x$-axis [0, 76.8], $y$-axis [-16, 16], and $z$-axis [-2, 10.8]. This range was chosen considering the scope of the RTNH\_WIDE \cite{kradargithub} object detection model trained with the widest range. All sequence data were used in the experiment, with the 4DR P2T model trained using the train set and performance evaluated using the test set. Model training was performed on an NVIDIA 3090 GPU, with a batch size of 8, a learning rate of 0.001, and Adam optimizer \cite{adam}, running for 20 epochs.

\subsection{Metrics}
For evaluation metrics, PSNR and SSIM were used, referencing \cite{l2r}. PSNR measures the signal-to-noise ratio between the synthetic data and the ground truth data, while SSIM measures the structural similarity between the two datasets. Both metrics indicate better performance with higher values. Although the generated data is a 3D tensor, the evaluation was performed by converting it to a 2D image through mean pooling along the height axis, and then calculating the metrics.

The deep-learning efficiency score (DES) metric, defined in Eq. \ref{eq:DES}, was used to identify efficient point cloud generation methods for deep learning. This metric aims to reduce data volume, which is related to point cloud density (PCD), while maintaining high tensor generation performance. First, the PSNR and SSIM values are normalized using min-max scaling, as shown in Eq. \ref{eq:psnr_norm} and Eq. \ref{eq:ssim_norm}, to ensure a fair comparison.
\begin{equation}
\text{PSNR}_{\text{norm}}^{(i)} = \frac{\text{PSNR}^{(i)} - \text{PSNR}_{\min}}{\text{PSNR}_{\max} - \text{PSNR}_{\min}}
\label{eq:psnr_norm}
\end{equation}
\begin{equation}
\text{SSIM}_{\text{norm}}^{(i)} = \frac{\text{SSIM}^{(i)} - \text{SSIM}_{\min}}{\text{SSIM}_{\max} - \text{SSIM}_{\min}}
\label{eq:ssim_norm}
\end{equation}
Where \( \text{PSNR}^{(i)} \) and \( \text{SSIM}^{(i)} \) represent the PSNR and SSIM for the \textit{\( i \)-th method}, respectively. \( \text{PSNR}_{\min} \) and \( \text{PSNR}_{\max} \) denote the minimum and maximum PSNR values across all evaluated methods, and similarly \( \text{SSIM}_{\min} \) and \( \text{SSIM}_{\max} \) represent the minimum and maximum SSIM values. Using these normalized values, the DES metric is computed as shown in Eq. \ref{eq:DES}.
\begin{equation}
M = \alpha \times \frac{\text{PSNR}_{\text{norm}}^{(i)}}{D^{(i)}} 
+ \beta \times \frac{\text{SSIM}_{\text{norm}}^{(i)}}{D^{(i)}}
\label{eq:DES}
\end{equation}
Where \( D^{(i)} \) is the PCD, defined as the ratio of detected points to the total possible points within the ROI for method \( i \). The weighting factors \( \alpha \) and \( \beta \), which control the relative importance of PSNR and SSIM, satisfy the constraint \( \alpha + \beta = 1 \). In this study, equal weights of 0.5 were assigned to both PSNR and SSIM.

\subsection{Results}

\begin{table}[ht]
\caption{Quantitative experimental results of 4DR P2T. 'Method' refers to the main categories of point cloud generation methods, while 'Hyper.' denotes the subcategories of point cloud generation methods, representing the hyper-parameters used in each point generation method. }
\label{tab:result}
\centering
\renewcommand{\arraystretch}{1.4} 
\begin{tabular}{c|c|c|c|c|c}
\hline \hline
\begin{tabular}[c]{@{}c@{}}Method\end{tabular} &
  Hyper. &
  \begin{tabular}[c]{@{}c@{}} PCD (\%)\end{tabular} &
  PSNR (dB) ↑ &
  SSIM ↑ &
  \begin{tabular}[c]{@{}c@{}}DES ↑\end{tabular} \\ \hline
\multirow{2}{*}{CFAR}       & 2.5 & 1.22   & 30.00 & 0.96 & 0.33 \\
                            & 10  & 2.42   & 28.14 & 0.96  & 0.11 \\ \hline
\multirow{4}{*}{Percentile} & 0.1 & 0.12   & 28.08 & 0.94 & 0.00 \\
                            & 1   & 1.11  & 31.66 & 0.96 & \textbf{0.48} \\
                            & 5   & 4.46 & \textbf{34.43} & \textbf{0.98} & 0.22 \\
                            & 10  & 8.17 & 30.00 & 0.96 & 0.05 \\ \hline \hline
\end{tabular}

\end{table}

Tab. \ref{tab:result} summarizes the tensor generation performance of the 4DR P2T model on 4D Radar point cloud data generated by various methods, evaluated using PSNR, SSIM, and DES. The average PSNR across all methods is 30.39dB—exceeding the 20–25 dB threshold commonly considered acceptable in wireless communication quality \cite{PSNR_1, PSNR_2}—indicating that the generated tensor data is of sufficiently high performance. As shown in Fig. \ref{fig3.result}, the percentile 5\% method achieves the best tensor generation performance, with a PSNR of 34.43dB and SSIM of 0.98. Meanwhile, the percentile 1\% method attains the highest DES value of 0.48, while also demonstrating superior point generation ability while reducing data volume, making it well-suited for deep learning model training.

\section{CONCLUSIONS} \label{sec:conclusion}

This study introduces the 4DR P2T model, which generates tensor data from 4D Radar point cloud data to address the limitation of inadequate spatial characteristic capture when CFAR is applied to 4D Radar data. By leveraging a cGAN-based architecture, our model effectively generates tensor data, as demonstrated by an average PSNR of 30.39dB and SSIM of 0.96. In addition, our comparative experiments show that the percentile 5\% method yields the best tensor generation performance,  while the percentile 1\% method offers an optimal balance between data volume reduction and performance, making it well-suited for deep learning training.

Future research will extend the 4DR P2T model to accommodate unpaired data, enabling tensor generation even for datasets lacking original tensor data.  Additionally, Doppler information will be incorporated to further enhance object representation. These advancements aim to improve the preservation of critical object features, enhance sensor fusion, and ultimately strengthen perception capabilities in autonomous driving systems.

\addtolength{\textheight}{-12cm}   % This command serves to balance the column lengths
                                  % on the last page of the document manually. It shortens
                                  % the textheight of the last page by a suitable amount.
                                  % This command does not take effect until the next page
                                  % so it should come on the page before the last. Make
                                  % sure that you do not shorten the textheight too much.

%%%%%%%%%%%%%%%%%%%%%%%%%%%%%%%%%%%%%%%%%%%%%%%%%%%%%%%%%%%%%%%%%%%%%%%%%%%%%%%%



%%%%%%%%%%%%%%%%%%%%%%%%%%%%%%%%%%%%%%%%%%%%%%%%%%%%%%%%%%%%%%%%%%%%%%%%%%%%%%%%



%%%%%%%%%%%%%%%%%%%%%%%%%%%%%%%%%%%%%%%%%%%%%%%%%%%%%%%%%%%%%%%%%%%%%%%%%%%%%%%%
\section*{ACKNOWLEDGMENT}

This work was supported by the National Research Foundation of Korea (NRF) grant funded by the Korea government (MSIT) (No. 2021R1A2C3008370).



%%%%%%%%%%%%%%%%%%%%%%%%%%%%%%%%%%%%%%%%%%%%%%%%%%%%%%%%%%%%%%%%%%%%%%%%%%%%%%%%

\bibliographystyle{unsrt}
\bibliography{ref}






\end{document}

\end{document}
