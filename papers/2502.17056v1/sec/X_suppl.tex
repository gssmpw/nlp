\clearpage
\setcounter{page}{1}
\maketitlesupplementary
% \section{Rationale}
% \label{sec:rationale}
% % 
% Having the supplementary compiled together with the main paper means that:
% % 
% \begin{itemize}
% \item The supplementary can back-reference sections of the main paper, for example, we can refer to \cref{sec:intro};
% \item The main paper can forward reference sub-sections within the supplementary explicitly (e.g. referring to a particular experiment); 
% \item When submitted to arXiv, the supplementary will already included at the end of the paper.
% \end{itemize}
% % 
% To split the supplementary pages from the main paper, you can use \href{https://support.apple.com/en-ca/guide/preview/prvw11793/mac#:~:text=Delete%20a%20page%20from%20a,or%20choose%20Edit%20%3E%20Delete).}{Preview (on macOS)}, \href{https://www.adobe.com/acrobat/how-to/delete-pages-from-pdf.html#:~:text=Choose%20%E2%80%9CTools%E2%80%9D%20%3E%20%E2%80%9COrganize,or%20pages%20from%20the%20file.}{Adobe Acrobat} (on all OSs), as well as \href{https://superuser.com/questions/517986/is-it-possible-to-delete-some-pages-of-a-pdf-document}{command line tools}.

\begin{table*}[h]
\centering
    \resizebox{1.0\textwidth}{!}{
    \begin{tabular}{l|cc|ccccccccccccc|cc}
    \hline
        ~   &~&~&\multicolumn{13}{c|}{per class IoU}&~&~\\
         Method &Real Data   &Synthetic Data & BG & SW & UF & BD & MD & AV & AL & PC & PT & FR & SR & OS & WL &mIoU$\uparrow$ &F1$\uparrow$\\

    \hline
        \multirow{3}*{PFSegNet-r50 \cite{pfsegnet}}    &2k &-  &0.2462&0.5617&0.4973&0.3349&0.2436&0.0402&0.6391&0.0452&0.4395&0.7762&0.0279&0.2022&0.1709&0.3250 &0.4444\\
         ~  &-  &10k&0.1506&0.5371&0.5351&0.3785&0.1778&0.0468&0.6430&0.0000&0.4280&0.7820&0.0120&0.2551&0.2052    &0.3193  &0.4327\\
         ~  &2k &10k&\textbf{0.3118}&\textbf{0.5993}&\textbf{0.6113}&\textbf{0.4153}&\textbf{0.2718}&\textbf{0.0627}&\textbf{0.6732}&\textbf{0.0696}&\textbf{0.4640}&\textbf{0.8082}&\textbf{0.0584}&\textbf{0.2793}&\textbf{0.2669}    &\textbf{0.3763} &\textbf{0.5021}\\
     \hline
        \multirow{3}*{PFSegNet-r101 \cite{pfsegnet}}    &2k &-  &0.2702&\textbf{0.6010}&0.5804&\textbf{0.4139}&\textbf{0.2639}&\textbf{0.1031}&0.6647&0.0737&0.4449&0.7927&0.0337&0.2627&0.2458&0.3654 &0.4943\\
         ~  &-  &10k&0.0970&0.4374&0.4116&0.2408&0.0686&0.0585&0.4950&0.0185&0.3722&0.7462&0.0191&0.1485&0.1787 &0.2532  &0.3609\\
         ~  &2k &10k&\textbf{0.2836}&0.5785&\textbf{0.5982}&0.3977&0.2478&0.0966&\textbf{0.6705}&\textbf{0.0843}&\textbf{0.4548}&\textbf{0.8007}&\textbf{0.0488}&\textbf{0.2644}&\textbf{0.2799}    &\textbf{0.3697} &\textbf{0.4987}\\
    \hline
        \multirow{3}*{SegFormer-B0 \cite{segformer}}    &2k &-  &0.2733&0.5679&\textbf{0.5950}&0.3929&0.2421&0.0765&0.6252&\textbf{0.0280}&\textbf{0.4027}&0.7313&0.0258&0.2784&0.1505&0.3377 &0.4605\\
         ~  &-  &10k&0.1571&0.4728&0.4630&0.3227&0.2134&0.0701&0.5234&0.0013&0.3354&0.6558&0.0124&0.1876&0.1267  &0.2724  &0.3879\\
         ~  &2k &10k&\textbf{0.3065}&\textbf{0.5686}&0.5948&\textbf{0.4102}&\textbf{0.2646}&\textbf{0.0849}&\textbf{0.6285}&0.0222&0.3775&\textbf{0.7389}&\textbf{0.0446}&\textbf{0.3018}&\textbf{0.2230}    &\textbf{0.3512} &\textbf{0.4788}\\
    \hline
        \multirow{3}*{SegFormer-B5 \cite{segformer}}    &2k &-  &0.2628&0.5971&0.6104&0.4124&0.2359&\textbf{0.0990}&0.6487&0.0818&0.4213&0.7611&0.0502&0.2871&0.1781&0.3574 &0.4852\\
         ~  &-  &10k&0.1247&0.5265&0.4935&0.3218&0.2277&0.0882&0.6098&0.0060&0.3768&0.7018&0.0242&0.2006&0.1103  &0.2932  &0.4088\\
         ~  &2k &10k&\textbf{0.2987}&\textbf{0.6156}&\textbf{0.6212}&\textbf{0.4386}&\textbf{0.2987}&0.0960&\textbf{0.6504}&\textbf{0.1096}&\textbf{0.4285}&\textbf{0.7775}&\textbf{0.0706}&\textbf{0.2922}&\textbf{0.2061}  &\textbf{0.3772} &\textbf{0.5092}\\
    \hline  
    \end{tabular}}
\caption{\textbf{Downstream task evaluation results} of (a) semantic segmantion and (b) change detection. With the augmentation of our synthetic data, the performance of downstream tasks on all methods get improvement, highlighted in \textbf{Bold}.}
\label{addtional ss}
\end{table*}

\begin{figure*}[h]
    \centering
    \includegraphics[width=1.0\linewidth]{figs/diffusion.pdf}
    \vspace{-10mm}
    \caption{\textbf{Spectral profile comparison.} We visualize the spectral response of our generated samples, comparing to real samples. We sample the pixels of several typical landforms according to the annotations. The intensity of spectral responses of the same landform keep consistent in different HSIs and are close to the real samples.}
    \label{fig:diffusion}
\end{figure*}

\begin{figure*}[h]
    \centering
    \includegraphics[width=1.0\linewidth]{figs/denoising.pdf}
    \vspace{-10mm}
    \caption{\textbf{Spectral profile comparison.} We visualize the spectral response of our generated samples, comparing to real samples. We sample the pixels of several typical landforms according to the annotations. The intensity of spectral responses of the same landform keep consistent in different HSIs and are close to the real samples.}
    \label{fig:denoising}
\end{figure*}

\begin{figure*}[h]
    \centering
    \includegraphics[width=1.0\linewidth]{figs/comparison_to_ldm.pdf}
    \vspace{-10mm}
    \caption{\textbf{Spectral profile comparison.} We visualize the spectral response of our generated samples, comparing to real samples. We sample the pixels of several typical landforms according to the annotations. The intensity of spectral responses of the same landform keep consistent in different HSIs and are close to the real samples.}
    \label{fig:comparison to ldm}
\end{figure*}