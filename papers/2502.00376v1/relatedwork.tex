\section{Literature Review}
\label{lr}
The identification of ADHD is currently the focus of numerous studies \cite{cortese2022half,loh2022automated,galvez2022therapeutic,faraone2021world}. For example, several research groups have started to develop advanced deep learning and machine learning algorithms based on ADHD data \cite{wang2022attention,hernandez2023machine}. It is worth noting that these algorithms have been studied to improve performance in the diagnosis of ADHD. While ADHD cannot be treated, early detection can contribute to how these ADHD individuals function in society \cite{carpentier2012adhd}. Researchers are trying to identify risk factors to reduce the number of children diagnosed with ADHD. One study found a strong correlation between ADHD and hereditary variables \cite{faraone2021world}. In younger children, around 75\% of the risk of ADHD is caused by genetic factors \cite{faraone2021world}. Brain damage and alcohol or tobacco use during pregnancy and early birth are among the risk factors for ADHD, along with genetics. Several other factors have been linked to ADHD in children \cite{claussen2022all}. These include gender, age, race, asthma, anxiety, depression, obesity, smoking and socioeconomic status. Machine learning (ML) models can be used instead of classical approaches for prediction. The mental health \cite{kessler2019role,zea2022machine} have all used ML-based models for identification and prediction.

Various machine learning classifiers were used to predict which children would raise ADHD \cite{kim2021can,zhang2021evidence}. Uluyagmur-Ozturk et al. \cite{uluyagmur2016adhd} categorized children as having ASD, ADHD or no diagnosis in a study examining their emotional well-being. Information was obtained on 61 children from Marmara University Medical Hospital. 18 children had autism spectrum disorder, 30 children had ADHD and 13 completely healthy children. To identify the hallmarks of ASD and ADHD, they turned to ReliefF. They used five ML-based algorithms to further categorize the children as to whether they were ASD, ADHD or healthy. The results show an 80\% success rate in classifying children as healthy with ASD or ADHD.

In another study, a Continuous Performance Test (CPT) was used to diagnose children with ADHD \cite{slobodin2020machine}. They selected 458 children aged 6 to 12 years. With a mean age of $8.7\pm$1.8 years, 59.0\% of the selected children were male, and 46.51\% of these children struggled with ADHD. Their proposed ML classifier, MOXO, showed an impressive accuracy of 87.0\%.
%
In addition, Morrow et al. conducted a study of children being treated for ADHD \cite{morrow2020leveraging} in which they retrieved information on 6,630 children aged 3 to 17 years who were diagnosed with ADHD from the National Survey of Children's Health (NSCH), 2016-2017. The children diagnosed with ADHD were, on average, 12.4 years old. They identified the characteristics of children who were receiving treatment for ADHD. They demonstrated that the DeepNet-based classifier achieved the best AUC at 0.72\%.


The application of ML-based classifiers to ADHD diagnosis is still a challenge despite its rapid growth. However, using different ADHD datasets in different countries, deep learning-based classifiers have been used to predict children with ADHD \cite{ahmadi2021computer,kim2021can, zhang2021evidence}. However, there is a need to improve the performance of the models. This study identifies which variables in children are at risk for developing ADHD and proposes a deep learning (DL) classifier that can identify and predict whether a child is healthy or has ADHD. Furthermore, it is possible to provide a generalizable and pre-trained framework that can be used for the classification of almost all downstream tasks in ADHD.