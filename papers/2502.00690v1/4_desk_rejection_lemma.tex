\section{The Desk Rejection Dilemma}\label{sec:dr_dilemma}

In this section, we define the concept of an ideal desk-rejection system in Section~\ref{sec:good_solution} and formally demonstrate in Section~\ref{sec:good_solution_hard} that no algorithm can achieve this ideal system.

\subsection{Ideal Desk-Rejection}\label{sec:good_solution}
An ideal desk-rejection system should avoid unfairly rejecting papers from authors who either comply with the submission limit or exceed it by only one or two papers. Otherwise, authors may face consequences due to co-authors with an excessively high number of submissions. This issue is particularly problematic for early-career researchers, as such collective penalties can have a significant negative impact on their careers. 

To address this, we formally define the criteria for an ideal desk-rejection outcome for the problem in Definition~\ref{def:submit_limit_problem}, where rejections are based solely on an author’s excessive submissions, without unfairly penalizing others.

\begin{definition}[Ideal desk-rejection] \label{def:good_solution}
An ideal solution for the submission limit problem in Definition~\ref{def:submit_limit_problem} is a paper subset $S\subseteq \mathcal{P}$ such that every author has exactly $\min\{x, |P_i|\}$ papers remaining after desk rejection. 

\end{definition}
\begin{remark}
    The ideal desk-rejection in Definition~\ref{def:good_solution} ensures that innocent authors with less than $x$ submissions will retain all their papers, and a non-compliant author $a_i$ with more than $x$ submissions will be desk-rejected exact $(|P_i|-x)$ papers.
\end{remark}
Thus, if there exists an algorithm that can reach the aforementioned ideal solution, we can ensure that no author is unfairly penalized due to their co-authors' submission behavior, achieving both fairness and individual accountability.
\subsection{Hardness of Ideal Desk-Rejection}\label{sec:good_solution_hard}

Unfortunately, we find that achieving an ideal desk-rejection system is fundamentally intractable. The main result regarding this hardness is presented in the following theorem:

\begin{theorem}[Hardness of Ideal Desk-Rejection]\label{thm:main_res_general}
Let $n = |\mathcal{A}|$ denote the number of authors in Definition~\ref{def:submit_limit_problem}. We can show that

\begin{itemize}
    \item {\bf Part 1:} For $n \le 2$, there always exists an algorithm that can achieve the ideal desk-rejection in Definition~\ref{def:good_solution}.
    \item {\bf Part 2:} For $n \geq 3$, there exists at least one problem instance where no algorithm can guarantee achieving the ideal desk-rejection in Definition~\ref{def:good_solution}.
\end{itemize}
\end{theorem}

\begin{proof} For {\bf Part 1}, the result follows directly from Lemma~\ref{lem:n_eq_1_positive_general} and Lemma~\ref{lem:n_eq_2_positive_general}. For {\bf Part 2}, the result is established using Lemma~\ref{lem:n_eq_3_negative} and Lemma~\ref{lem:n_geq_3_negative}. 
Detailed technical proofs for these lemmas are provided in Appendix~\ref{sec:dilemma_proof}.
\end{proof}

Therefore, since an ideal desk-rejection system is not achievable, it is inevitable that some authors may face excessive desk-rejections due to collective punishments. This challenge is particularly concerning for early-career researchers with only one or two submissions, motivating the need to seek an approximate solution that optimizes fairness in desk-rejection systems.
