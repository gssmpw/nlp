\begin{algorithm}[!ht]
\caption{Conventional Desk-Reject Algorithm}
\label{alg:conventional_desk_reject_algo}
\begin{algorithmic}[1]

\Procedure{DeskRejct}{$\mathcal{A}, \mathcal{P}, x$} 
\State {\color{blue} /*  Initialize registered paper set for each author. */}
\For {$i = 1 \to n$}
\State $R_i \gets \emptyset$
\EndFor
\State {\color{blue} /* Initialize the subset of remaining papers. */}
\State $S \gets \mathcal{P}$
\State {\color{blue} /*  Process each paper in submission order.\. */}
\For {$j = 1 \to m$}
\For {$i \in A_j$}
\State {\color{blue} /* If author $a_i$ has reached the submission limit, the paper will be rejected.*/}
\If {$|R_i| \geq x$}
\State $S \gets S \setminus \{p_j\}$
\State \textbf{break}
\EndIf
\EndFor
\State {\color{blue} /* If paper $p_j$ is not rejected, we add it to each co-author's registered paper set.*/}
\If{$p_j \in S$}
\For {$i \in A_j$}
\State $R_i \gets R_i \cup \{ j \}$
\EndFor
\EndIf
\EndFor
\State \Return $S$
\EndProcedure
\end{algorithmic}
\end{algorithm}

\section{Case Study}\label{sec:case_study}

Since desk-rejection data from top AI conferences is not publicly available, and fully open-review conferences like ICLR do not impose submission limits, evaluating real-world conference submissions is impractical. Therefore, we present a case study to demonstrate how our proposed desk-rejection algorithm more effectively addresses fairness issues. Additional case studies are provided in Appendix~\ref{sec:more_case_study}.


Let the paper subscript $j$ in $p_j \in \mathcal{P}$ denote the submission order. We analyze the widely used desk-rejection system (e.g., CVPR 2025) in Algorithm~\ref{alg:conventional_desk_reject_algo}, which rejects all papers submitted after an author’s $x$-th submission. To highlight its limitations, we present a minimal working example:

\begin{example}
    Consider a submission limit problem as defined in Definition~\ref{def:submit_limit_problem} with $n=2$, $x=25$, and $m=26$. Author $a_1$ submits all papers $p_1, \ldots, p_{26}$, while author $a_2$ submits only $p_{26}$.
\end{example}

Given the ideal desk-rejection criteria in Definition~\ref{def:good_solution}, it is evident that we can reject any papers in $\{p_1, p_2, \ldots, p_{25}\}$ following the techniques in Lemma~\ref{lem:n_eq_2_positive}. After rejection, since $a_1$ retains 25 papers and $a_2$ retains 1 paper, the fairness metrics are $\zeta_{\mathrm{ind}}(S) = \max\{1/26, 0\} = 1/26$ and $\zeta_{\mathrm{group}}(S) = \frac{1}{2}(1/26 + 0) = 1/52$.

On the other hand, the CVPR 2025 algorithm, as described in Algorithm~\ref{alg:conventional_desk_reject_algo}, rejects $p_{26}$, retaining $S = \{p_1, \ldots, p_{25}\}$. This unfairly penalizes $a_2$, resulting in $\zeta_{\mathrm{ind}}(S) = \max\{1/26, 1\}= 1$ and $\zeta_{\mathrm{group}}(S) = \frac{1}{2}(1/26 + 1) = 27/52$,
which is much worse compared with the ideal results. In contrast, our method in Algorithm~\ref{alg:fair_desk_reject_algo} solves the linear program and recovers the ideal solution, achieving the same fairness metrics as the optimal case.

A simple workaround to mitigate unfairness in conventional desk-rejection systems is the roulette algorithm, which randomly rejects papers from non-compliant authors like $a_1$ until the submission limit $x$ is reached. However, this heuristic cannot fully prevent the rejection of the undesirable paper $p_{26}$ and results in suboptimal fairness outcomes compared to our fairness-aware rejection, since the expected fairness metrics under the roulette algorithm satisfy $\mathbb{E}[\zeta_{\mathrm{ind}}] = (25/26) \cdot(1/26) + (1/26) \cdot 1 \leq 1/26$ and $\mathbb{E}[\zeta_{\mathrm{group}}] = (25/26) \cdot (1/52) + (1/26) \cdot (27/52) \leq 27/52$.

Thus, this example illustrates that conventional desk-rejection systems in top conferences such as CVPR can suffer from severe fairness issues, whereas our proposed method effectively mitigates these problems. 

Additionally, this example also highlights another noteworthy consequence of the conventional desk-rejection system. Specifically, authors collaborating with senior researchers who have numerous submissions may have to compete for earlier submission slots to avoid desk rejection. However, the submission order should not influence whether a paper is accepted, which reveals the unintended implications of the order-based desk-rejection system.
