\section{Missing Proofs in Section 5}\label{sec:fair_proof}
In this section, we first present the missing proofs for fairness metrics in Section~\ref{sec:fair_metric_append}, and then present the supplementary proofs for the hardness of individual fairness optimization in Section~\ref{sec:indi_fair_hard_append}. Finally, we show the additional proofs for the group fairness optimization problem in Section~\ref{sec:fair_optim_append}.

\subsection{Fairness Metrics}\label{sec:fair_metric_append}
We present the relationship between the fairness metrics.  

\begin{proposition}[Relationship of Fairness Metrics, formal version of Proposition~\ref{lem:fair_metric_ineq} in Section~\ref{sec:fair_metric}]\label{lem:fair_metric_ineq_append}
    For any solution $S\subseteq \mathcal{P}$ for the submission limit problem in Definition~\ref{def:submit_limit_problem}, we have 
    \begin{align*}
        \zeta_{\mathrm{group}}(S) \leq \zeta_{\mathrm{ind}}(S).
    \end{align*}
\end{proposition}
\begin{proof}
    By Definition~\ref{def:group_fair}, we have:
    \begin{align*}
        \zeta_{\mathrm{group}}(S) &=~ \frac{1}{n}\sum_{i \in [n]} c(a_i,S) \\
        &\leq~ \frac{1}{n}\sum_{i \in [n]} \max_{i\in[n]}c(a_i, S) \\ 
        &=~\frac{1}{n}\cdot n \cdot \max_{i\in[n]}c(a_i, S) \\ 
        &=~ \zeta_{\mathrm{ind}}(S),
    \end{align*}
where the first equality directly follows from Definition~\ref{def:group_fair}, the second and the third inequality follow from basic algebra, and the last equality follows from Definition~\ref{def:individual_fair}. Thus, we complete the proof.
\end{proof}

\subsection{Hardness of Individual Fairness-Aware Submission Limit Problem}\label{sec:indi_fair_hard_append}

Before proving the theoretical results in Section~\ref{sec:indi_fair_hard}, we first introduce a useful fact that serves as a foundation for the subsequent proofs.   

\begin{fact}\label{fact:author_paper_count}
For each author $a_i\in\mathcal{A}$, the number of papers after desk-rejection (i.e., $|\{p_j \in  S : a_i \in A_j\}|$) can be written as $W_i^\top r$.
\end{fact}
\begin{proof}
    This simply follows from:
    \begin{align*}
        W^\top_ir &=~ \sum_{j\in[m]} W_{i,j}\cdot r_j \\ 
        &=~ |\{j\in[m]:W_{i, j}=1,r_j=1\}| \\ 
        &=~ |\{p_j\in\mathcal{P}:a_i\in A_j, p_j\in S\}| \\ 
        &=~|\{p_j \in  S : a_i \in A_j\}|,
    \end{align*}
where the first and the second equality follow from basic algebra and set theory, and the third and the fourth equality follow from Definition~\ref{def:submit_limit_problem}.  
\end{proof}

With the help of the aforementioned fact, we now prove the equivalence of the matrix form for the individual fairness problem.


\begin{proposition}[Matrix Form Equivalence for $\zeta_{\mathrm{ind}}$, formal version of Proposition~\ref{prop:equiv_individual} in Section~\ref{sec:indi_fair_hard}]\label{prop:equiv_individual_append}
    The individual fairness-aware submission limit problem in Definition~\ref{def:ind_fair_min} and the matrix form integer programming problem in Definition~\ref{def:ind_fair_min_matrix} are equivalent.
\end{proposition}
\begin{proof}
    In Definition~\ref{def:ind_fair_min}, the paper set $\mathcal{P}$ consists of $m$ papers, each of which can either be maintained or desk-rejected. Thus, the subset of maintained papers, $\mathcal{S}$, can be represented by a 0-1 vector $r \in \{0, 1\}^m$, where $r_j = 1$ indicates that paper $p_j$ is maintained, and $r_j = 0$ indicates that it is desk-rejected. We now establish the equivalence of both the objective function and the constraints in these two formulations.


    \paragraph{Part 1: Optimization Objective.} We first consider the objective function $\mathbf{1}^\top_nD^{-1}Wr$ in Definition~\ref{def:ind_fair_min_matrix}:
    \begin{align*}
        \min_{r \in \{0,1\}^m} \| \mathbf{1}_n - D^{-1}Wr\|_\infty &=~ \min_{r \in \{0,1\}^m} \max_{i\in[n]} (1 - (D^{-1}Wr)_i) \\ 
        &=~ \min_{r \in \{0,1\}^m} \max_{i\in[n]} (1 - (W_i^\top r)_i/D_{i,i}) \\
        &=~ \min_{r \in \{0,1\}^m} \max_{i\in[n]} (1 - (W_i^\top r)_i/|P_i|) \\ 
        &=~ \min_{r \in \{0,1\}^m} \max_{i\in[n]} (1 - |\{p_j \in  S : a_i \in A_j\}|/|P_i|) \\ 
        &=~ \min_{r \in \{0,1\}^m} \max_{i\in[n]}c(a_i, S) \\ 
        &=~ \min_{r \in \{0,1\}^m} \zeta_{\mathrm{ind}}(S),
    \end{align*}
where the first equality follows from the definition of infinity norm, the second equality follows from basic algebra, the third equality follows from Definition~\ref{def:ind_fair_min_matrix}, the fourth equality follows from Fact~\ref{fact:author_paper_count}, the fifth equality follows from Definition~\ref{def:cost}, and the last equality follows from Definition~\ref{def:individual_fair}. By decoding $r$ back into the paper subset $S$, we recover the original optimization objective in Definition~\ref{def:ind_fair_min}.

\paragraph{Part 2: Constraints.} The constraint in Definition~\ref{def:ind_fair_min_matrix} can be rewritten using basic algebra as:
     \begin{align*}
         W_i \cdot r \leq x, \quad \forall i \in [n].
     \end{align*}
     By applying Fact~\ref{fact:author_paper_count}, we see that this constraint is equivalent to its counterpart in Definition~\ref{def:ind_fair_min}.

Since both the objective function and constraints in Definition~\ref{def:ind_fair_min} and Definition~\ref{def:ind_fair_min_matrix} are equivalent, the proof is complete.
\end{proof}

To show the hardness of the individual fairness problem, we first present a classical set cover problem with well-established hardness. 

\begin{definition}[Set Cover Problem \cite{k72,gj79}]\label{def:set_cover}
The Set Cover problem is the following: 
\begin{itemize}
    \item {\bf Input:} A universe $U = \{1, \ldots, n\}$, a family of sets 
    $\{S_1, \ldots, S_m\} \subseteq 2^U$, and a integer $K > 0$. 
    \item {\bf Question:} Is there a subfamily $\{S_j : j \in J\}$ for some 
    $J \subseteq \{1,\ldots,m\}$ and $|J| \leq K$ that covers $U$, i.e., $\bigcup_{j \in J} S_j = U$?
\end{itemize}
\end{definition}

\begin{lemma}[Folklore \cite{k72,gj79}]\label{lem:set_cover_np_hard}
    The Set Cover problem defined in Definition~\ref{def:set_cover} is $\NPhard$.
\end{lemma}

Additionally, we also present a technical lemma which is useful for showing the hardness of the individual fairness problem. 

\begin{lemma}\label{lem:useful_lemma}
    For any $r \in \{0,1\}^m$, the following two statements are equivalent:
    \begin{itemize}
       \item {\bf Part 1.} $\|\mathbf{1}_n - D^{-1}Wr\|_\infty \leq 1 - \frac{1}{\min_{i \in [n]}|P_i|}$.
       \item {\bf Part 2.} $\min_{i\in[n]} (Wr)_i \geq 1$.
    \end{itemize}
\end{lemma}
\begin{proof}
    We first show that Part 1 implies Part 2.
    Suppose that
    \begin{align*}
        \|\mathbf{1}_n - D^{-1}Wr\|_\infty \leq 1 - \frac{1}{\min_{i \in [n]}|P_i|}.
    \end{align*}
    By the definition of the infinity norm, we have
    \begin{align*}
         1- \frac{(Wr)_{i'}}{|P_{i'}|} \leq 1 - \frac{1}{\min_{i \in [n]}|P_i|}, \quad \forall i' \in [n].
    \end{align*}
    Rearranging gives
    \begin{align*}
        (Wr)_{i'} \geq \frac{|P_{i'}|}{\min_{i \in [n]}|P_i|} \geq 1, \quad \forall i' \in [n].
    \end{align*}
    Since for all $i'\in[n]$, we have $(Wr)_{i'} \geq 1$, we can conclude that $\min_{i\in[n]} (Wr)_i \geq 1$.

    Now we show that that Part 2 implies Part 1.
    Suppose that $\min_{i\in[n]} (Wr)_i \geq 1$,  then we have $(Wr)_{i} \geq 1$ for all $i \in [n]$, which implies that for all $i \in [n]$, 
    \begin{align*}
        1 - \frac{(Wr)_i}{|P_i|} \leq  1 - \frac{1}{|P_i|} \leq  1 - \frac{1}{\max_{i' \in [n]} |P_{i'}|}.
    \end{align*}
    Hence
    \begin{align*}
        \|\mathbf{1}_n - D^{-1}Wr\|_\infty \leq 1 - \frac{1}{\min_{i \in [n]}|P_i|}.
    \end{align*}
    Thus the proof is complete.
\end{proof}

\begin{theorem}[Hardness, formal version of Theorem~\ref{thm:indi_nphard} in Section~\ref{sec:indi_fair_hard}]\label{thm:indi_nphard_append}
    The Individual Fairness-Aware Submission Limit Problem defined in Definition~\ref{def:ind_fair_min} is $\NPhard$.
\end{theorem}
\begin{proof}
    By Proposition~\ref{prop:equiv_individual}, it sufficies to reduce Set Cover problem to the integer optimization problem of the matrix form in Definition~\ref{def:ind_fair_min_matrix}.

    Given an instance of Set Cover, we build the matrix $W \in \{0, 1\}^{n\times m}$ by defining $W_{i,j} = 1$ if element $i \in S_j$, and 0 otherwise.
    Now set $|P_i| = \sum_{j \in [m]} W_{i,j}$ for every row $i \in [n]$. Finally, we choose $x = m$. We reduce the Set Cover problem to the following optimization problem:
    \begin{align*}
        & ~ \min_{r \in \{0,1\}^m} \| \mathbf{1}_n - D^{-1}Wr\|_\infty \\
    \mathrm{s.t.}
    & ~ W r \leq m \mathbf{1}_n, \\
    & ~ \|r\|_1 \leq K.
    \end{align*}
    Note that this problem is easier than the optimization problem defined in Definition~\ref{def:ind_fair_min}. The constraint $ W r \leq m \mathbf{1}_n$ is always satisfied, so we can drop it out. Now, it suffices to consider the decision problem:
    \begin{align*}
        &~ \mathrm{Find~} r \in \{0,1\}^m \\ \mathrm{s.t.~} &~
       \|\mathbf{1}_n - D^{-1}Wr\|_\infty \leq 1 - \frac{1}{\min_{i \in [n]}|P_i|}, \\
       &~ \|r\|_1 \leq K.
    \end{align*}

    Note that $\|\mathbf{1}_n - D^{-1}Wr\|_\infty \leq 1 - \frac{1}{\min_{i \in [n]}|P_i|}$ is equivalent to 
       $\min_{i\in[n]} (Wr)_i \geq 1$ by Lemma~\ref{lem:useful_lemma}.

    Hence the problem is equivalent to
    \begin{align*}
       \mathrm{Find~} r \in \{0,1\}^m  \mathrm{~~~s.t.~~~}
       \min_{i\in[n]} (Wr)_i \geq 1 \mathrm{~~and~~} 
        \|r\|_1 \leq K.
    \end{align*}
    It is not hard to see that the Set Cover problem has a solution if and only if the above problem has a solution.
    Requiring $\min_{i \in [n]} (W r)_i > 1$ exactly means that each element $i$ in the universe is covered by at least set $S_j$. The constraint $\|r\|_1 \leq K$ means that the size of cover is at most $K$. In other words, there exists a subfamily of size at most $K$ covering all elements if and only if there is an $r \in \{0,1\}^m$ with $\min_{i \in [n]} (W r)_i > 1$ and $\|r\|_1 \leq K$.
    
    Therefore, by Lemma~\ref{lem:set_cover_np_hard}, the individual fairness-aware submission limit problem is $\NPhard$.
\end{proof}

\subsection{Group Fairness Optimization}\label{sec:fair_optim_append}

Now, we present the missing proofs on both matrix form equivalence and linear programming optimal solution equivalence for the group fairness optimization problem. 

\begin{proposition}[Matrix Form Equivalence for $\zeta_{\mathrm{group}}$, formal version of Proposition~\ref{lem:group_fair_min_equiv} in Section~\ref{sec:fair_optim}]\label{lem:group_fair_min_equiv_append}
    The problem in Definition~\ref{def:group_fair_min} and the problem in Definition~\ref{def:group_fair_min_mat_new} are equivalent.
\end{proposition}
\begin{proof}
     In Definition~\ref{def:group_fair_min}, there are $m$ papers in $\mathcal{P}$, where each paper can either be maintained or rejected. Thus, we can encode the paper subset $S$ using a binary vector $r \in \{0, 1\}^m$, where $r_j = 1$ indicates that paper $p_j$ is maintained, and $r_j = 0$ indicates that it is desk-rejected. We now demonstrate that both the objective function and the constraints are equivalent.

     \paragraph{Part 1: Optimization Objective.} We first examine the objective function $\mathbf{1}^\top_nD^{-1}Wr$ in Definition~\ref{def:group_fair_min_mat_new}:
     \begin{align*}
         \mathbf{1}^\top_nD^{-1}Wr &=~ \sum_{i\in[n]}(D^{-1}Wr)_i \\ 
         &=~\sum_{i\in[n]}(W\cdot r)_i / |P_i| \\
         &=~\sum_{i\in[n]}(W_i^\top\cdot r) / |P_i| \\
         &=~\sum_{i\in[n]}\frac{|\{p_j \in S: a_i \in A_j\}|}{|P_i|} \\ 
         &=~ \sum_{i\in[n]}(1-c(a_i,S)),
     \end{align*}
     where the first equality follows from basic algebra, the second follows from Definition~\ref{def:group_fair_min_mat_new}, the third follows from matrix-vector multiplication, the fourth follows from Fact~\ref{fact:author_paper_count}, and the final equality follows from Definition~\ref{def:cost}. Consequently, the maximization problem in Definition~\ref{def:group_fair_min_mat_new} can be rewritten as:
     \begin{align*}
        \max_{r \in \{0, 1\}^m} \sum_{i\in[n]}(1-c(a_i,S)).
     \end{align*}
     Since maximizing this objective is equivalent to minimizing $\sum_{i\in[n]} c(a_i,S)$, we can reformulate it as:
     \begin{align*}
         \min_{r \in \{0, 1\}^m} \sum_{i\in[n]} c(a_i,S).
     \end{align*}
     By decoding $r$ back into the paper subset $S$, we recover the original optimization objective in Definition~\ref{def:group_fair_min}. 

     \paragraph{Part 2: Constraints.} Since the constraint is identical to that in the individual fairness minimization problem in Definition~\ref{def:ind_fair_min}, this result follows directly from Part 2 in the proof of Proposition~\ref{prop:equiv_individual_append}.

     Since both the objective function and constraints in Definition~\ref{def:group_fair_min} and Definition~\ref{def:group_fair_min_mat_new} are equivalent, the proof is complete.
\end{proof}

\begin{theorem}[Optimal Solution Equivalence of the Relaxed Problem, formal version of Theorem~\ref{thm:lp_equiv} in Section~\ref{sec:fair_optim}]\label{thm:lp_equiv_append}
    The optimal solution of the relaxed problem in Definition~\ref{def:group_fair_min_mat_relax_new} is equivalent to the optimal solution of the original problem in Definition~\ref{def:group_fair_min_mat_new}.
\end{theorem}
\begin{proof} 
    The problem in Definition~\ref{def:group_fair_min_mat_relax_new} is a linear program since it has a linear objective function $\mathbf{1}^\top_nD^{-1}Wr$ and linear constraints: the box constraint $r\in[0,1]^m$ and a linear inequality constraint $(Wr)/x \leq \mathbf{1}_n$.

    Furthermore, the problem is convex because the objective function is linear, the constraint $(Wr)/x \leq \mathbf{1}_n$ is affine, and the feasible region defined by $r\in[0,1]^m$ is a convex set.

    By the fundamental theorem of linear programming (see Page 23 of~\cite{ly84}), the optimal solution must occur at an extreme point of the convex polytope defined by the constraints. This implies that for all $i \in [m]$, we must have either $r_i = 0$ or $r_i = 1$. Therefore, the optimal solution of the relaxed linear program coincides with that of the original integer program, which finishes the proof.
\end{proof}
