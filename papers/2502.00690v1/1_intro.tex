\section{Introduction}

\begin{table}[!ht]\caption{ 
In this table, we summarize the submission limits of top conferences in recent years. For details of each conference website, we refer the readers to Section~\ref{app:sec:conference_links} in the Appendix. 
Some conferences (CVPR, ICCV, WSDM, KDD) employ a conventional desk-reject algorithm (Algorithm~\ref{alg:conventional_desk_reject_algo}), where papers are desk-rejected once an author has registered more than $x$ (the submission limit) papers.
}  \label{tab:conference_submission_limit}
\begin{center}
\begin{tabular}{ |c|c|c|c|c| } 
 \hline
 {\bf Conference Name} & {\bf Year} & {\bf Upper Bound} \\ \hline
 CVPR & 2025 & 25 \\ \hline
 CVPR & 2024 & N/A \\ \hline
 ICCV & 2025 & 25 \\ \hline
 ICCV & 2023 & N/A \\ \hline
 AAAI & 2023-2025 & 10 \\ \hline
 AAAI & 2022 & N/A \\ \hline
 WSDM & 2021-2025 & 10 \\ \hline
 WSDM & 2020 & N/A \\ \hline
 IJCAI & 2021-2025 & 8 \\ \hline
 IJCAI & 2020 & 6 \\ \hline
 IJCAI & 2018-2019 & 10 \\ \hline
 IJCAI & 2017 & N/A \\ \hline
 KDD & 2024-2025 & 7 \\ \hline
 KDD & 2023 & N/A \\ \hline
\end{tabular}
\end{center}
\end{table}

We are living in an era shaped by the unprecedented advancements of Artificial Intelligence (AI), where transformative breakthroughs have emerged across various domains in just a few years. A key driving force behind AI's rapid progress is the prevalence of top conferences held frequently throughout the year, offering dynamic platforms to present many of the field’s most influential papers. For example, ResNet~\cite{hzrs16}, a foundational milestone in deep learning with over 250,000 citations, was first introduced at CVPR 2016. Similarly, the Transformer architecture~\cite{vsp+17}, the backbone of modern large language models, emerged at NeurIPS 2017. More recently, diffusion models~\cite{hja20}, which represent the state-of-the-art in image generation, were presented at NeurIPS 2020, while CLIP~\cite{rkh+21}, a leading model for image-text pretraining, was showcased at ICML 2021. These groundbreaking contributions from top-tier conferences have significantly accelerated the advancement of AI, enriching both theoretical insights and practical applications.


As AI continues to expand its applications and capabilities in real-world domains such as dialogue systems~\cite{szk+22,aaa+23,a24}, image generation~\cite{hja20,sme20,wsd+24,wcz+23,wxz+24}, and video generation~\cite{hsg+22,bdk+23}, its immense potential for commercialization has raised growing enthusiasm in AI research. This enthusiasm has led to a rapid, rocket-like increase in the number of AI-related papers in 2024, as witnessed by recent studies~\cite{stanford_ai_index}. A direct consequence of this surge is the significant rise in submissions to AI conferences, which has placed a heavy burden on program committees tasked with selecting papers for acceptance. To address these challenges and maintain the quality of accepted papers, many leading conferences have introduced submission limits per author. In 2025, a wide range of leading AI conferences, including CVPR, ICCV, AAAI, WSDM, IJCAI, and KDD, have introduced submission limits per author in their guidelines, ranging from a maximum of $x=7$ to $x=25$. Table~\ref{tab:conference_submission_limit} provides an overview of these submission limits across major AI conferences.

However, such a desk-rejection mechanism may result in unintended negative societal impacts due to the Matthew effect in the research community~\cite{bvr18}, as illustrated in Figure~\ref{fig:matthew_effect}. Recent research has shown that the impact of a setback (e.g., a paper rejection) is often much greater for early-career researchers than for senior researchers~\cite{wjw19,scl23}, which shows that the effect of a desk rejection can vary significantly depending on the author’s career stage. For instance, as illustrated in Figure~\ref{fig:unfairness_of_desk_rejection}, consider the case of a young student submitting their only draft to the conference, co-authored with a renowned researcher who submits numerous papers annually. If the paper is desk-rejected due to exceeding submission limits, the senior researcher might view this as a neglectable inconvenience. In contrast, the rejection could have severe consequences for the student, as the paper might be crucial for applying to graduate programs, securing employment, or forming a chapter of their thesis. This disparity in the impact of desk rejections may worsen the Matthew effect in the AI community by disproportionately disadvantaging researchers with only one or two submitted papers, while having little effect on prolific senior researchers. Such outcomes raise important concerns about fairness and equity in current desk-rejection policies.

\begin{figure}[!ht]
    \centering
    \includegraphics[width=0.95\linewidth]{matthew_effect.pdf}
    \caption{
    The Matthew Effect in the AI community. This figure illustrates the worsening Matthew Effect in the AI community, where senior researchers tend to have a significantly higher number of submissions, while junior researchers have relatively few. 
    }
    \label{fig:matthew_effect}
\end{figure}

In response to the challenges posed by paper-limit-based desk-rejection systems, this work investigates an important and practical problem: ensuring fairness in desk-rejection systems for AI conferences under submission limits. As illustrated in Figure~\ref{fig:our_research}, our goal is to design a fair desk-rejection system that prioritizes rejecting submissions from authors with many papers while protecting those with fewer submissions, particularly early-career researchers. Our key contributions are as follows:

\begin{itemize}
\item We formally define the paper submission limit problem in desk-rejection systems and prove that an ideal system that rejects papers solely based on each author's excessive submissions is mathematically impossible when there are more than three authors.
\item We introduce two fairness metrics: individual fairness and group fairness. We formulate the fairness-aware paper submission limit problem as an integer programming problem. Specifically, we formally prove that optimizing individual fairness is NP-hard, while the group fairness optimization problem can be solved efficiently using any off-the-shelf linear programming solver.
\item Through case studies, we demonstrate that our proposed system achieves greater fairness compared to existing approaches used in top AI conferences such as CVPR 2025, promoting social justice and fostering a more inclusive ML research community.
\end{itemize}

\paragraph{Roadmap.} Our paper is organized as follows: In Section~\ref{sec:related_works}, we review related literature. In Section~\ref{sec:preliminary}, we present the key definition of the paper submission limit problem. In Section~\ref{sec:dr_dilemma}, we show that no algorithm can reach the ideal desk-rejection system without unfair collective punishments. In Section~\ref{sec:fair}, we present our new fairness-aware desk-rejection system. In Section~\ref{sec:case_study}, we show by case studies that our system is better than existing systems. In Section~\ref{sec:conclusion}, we present our conclusions. 


\begin{figure*}[!ht]
    \centering
    \includegraphics[width=0.85\linewidth]{unfairness_of_desk_rejection.pdf}
    \caption{ The unfairness of desk rejection based on submission limits. {\bf Left: A careless mistake.} In this scenario, a young student submits the only paper, co-authored with a professor who submits numerous papers, and carelessly exceeds the submission limit. The paper, which may aim to apply to graduate programs, secure employment, or form a chapter of the thesis, is very important for the student but may not be for the professor. {\bf Right: The desk rejection.} If the paper is desk-rejected due to submission limits, it poses a minor inconvenience to the professor, and the professor can shrug about it due to his remaining papers. However, it could have severe consequences for the students, as the paper is crucial for the student's future plans.
    }
    \label{fig:unfairness_of_desk_rejection}
\end{figure*}
