\def\isarxiv{1} %%% for icml submission version, we comment this line

\ifdefined\isarxiv
\documentclass[11pt]{article}

\usepackage[numbers]{natbib}


\else
\documentclass{article}
\usepackage{algorithm}
\usepackage{microtype}
\usepackage{graphicx}
\usepackage{subfigure}
\usepackage{booktabs}
\usepackage{hyperref}
\usepackage{icml2025}



\usepackage{amsmath}
\usepackage{amssymb}
\usepackage{mathtools}
\usepackage{amsthm}
\usepackage{algorithm}
\usepackage{algpseudocode}
\fi

\ifdefined\isarxiv
\usepackage{amsmath}
\usepackage{amsthm}
\usepackage{amssymb}
\usepackage{algorithm}
\usepackage{subfig}
\usepackage{algpseudocode}
\usepackage{graphicx}
\usepackage{grffile}
\usepackage{wrapfig,epsfig}
\usepackage{url}
\usepackage{xcolor}
\usepackage{epstopdf}


\usepackage{bbm}
\usepackage{dsfont}
\fi
 
 
\allowdisplaybreaks
 

\ifdefined\isarxiv

\let\C\relax
\usepackage{tikz}
\usepackage{hyperref}  %%% arxiv don't allow this.
\hypersetup{colorlinks=true,citecolor=blue,linkcolor=blue} %%% Zhao : maybe we should comment this in submission.
\usetikzlibrary{arrows}
\usepackage[margin=1in]{geometry}

\else

% \usepackage{microtype}
% \usepackage{hyperref}
% \definecolor{mydarkblue}{rgb}{0,0.08,0.45}
% \hypersetup{colorlinks=true, citecolor=mydarkblue,linkcolor=mydarkblue}
 

\fi



 
\graphicspath{{./figs/}}

\theoremstyle{plain}
\newtheorem{theorem}{Theorem}[section]
\newtheorem{lemma}[theorem]{Lemma}
\newtheorem{definition}[theorem]{Definition}
\newtheorem{notation}[theorem]{Notation}
%\newtheorem{proof}[theorem]{Proof}
\newtheorem{proposition}[theorem]{Proposition}
\newtheorem{corollary}[theorem]{Corollary}
\newtheorem{conjecture}[theorem]{Conjecture}
\newtheorem{assumption}[theorem]{Assumption}
\newtheorem{observation}[theorem]{Observation}
\newtheorem{fact}[theorem]{Fact}
\newtheorem{remark}[theorem]{Remark}
\newtheorem{claim}[theorem]{Claim}
\newtheorem{example}[theorem]{Example}
\newtheorem{problem}[theorem]{Problem}
\newtheorem{open}[theorem]{Open Problem}
\newtheorem{property}[theorem]{Property}
\newtheorem{hypothesis}[theorem]{Hypothesis}

\newtheorem{question}[theorem]{Question}

\newcommand{\wh}{\widehat}
\newcommand{\wt}{\widetilde}
\newcommand{\ov}{\overline}
\newcommand{\N}{\mathcal{N}}
\newcommand{\R}{\mathbb{R}}
\newcommand{\F}{\mathcal{F}}
\newcommand{\G}{\mathcal{G}}
\newcommand{\True}{\mathrm{true}}
\newcommand{\Noisy}{\mathrm{noisy}}
\newcommand{\Clean}{\mathrm{clean}}
\newcommand{\Est}{\mathrm{est}}
\newcommand{\RHS}{\mathrm{RHS}}
\newcommand{\LHS}{\mathrm{LHS}}
\renewcommand{\d}{\mathrm{d}}
\renewcommand{\i}{\mathbf{i}}
\renewcommand{\tilde}{\wt}
\renewcommand{\hat}{\wh}
\newcommand{\Tmat}{{\cal T}_{\mathrm{mat}}}
\newcommand{\NPhard}{\mathsf{NP}\text{-}\mathsf{hard}}

\DeclareMathOperator*{\E}{{\mathbb{E}}}
\DeclareMathOperator*{\var}{\mathrm{Var}}
\DeclareMathOperator*{\Z}{\mathbb{Z}}
\DeclareMathOperator*{\C}{\mathbb{C}}
\DeclareMathOperator*{\D}{\mathcal{D}}
\DeclareMathOperator*{\median}{median}
\DeclareMathOperator*{\mean}{mean}
\DeclareMathOperator{\OPT}{OPT}
\DeclareMathOperator{\supp}{supp}
\DeclareMathOperator{\poly}{poly}

\DeclareMathOperator{\nnz}{nnz}
\DeclareMathOperator{\sparsity}{sparsity}
\DeclareMathOperator{\rank}{rank}
\DeclareMathOperator{\diag}{diag}
\DeclareMathOperator{\Diag}{Diag}
\DeclareMathOperator{\dist}{dist}
\DeclareMathOperator{\cost}{cost}
\DeclareMathOperator{\vect}{vec}
\DeclareMathOperator{\tr}{tr}
\DeclareMathOperator{\dis}{dis}
\DeclareMathOperator{\cts}{cts}



\makeatletter
\newcommand*{\RN}[1]{\expandafter\@slowromancap\romannumeral #1@}
\makeatother
\newcommand{\Zhao}[1]{{\color{red}[Zhao: #1]}}
\newcommand{\Zhenmei}[1]{{\color{purple}[Zhenmei: #1]}}
\newcommand{\Zhizhou}[1]{{\color{blue}[Zhizhou: #1]}}
\newcommand{\Yuefan}[1]{{\color{brown}[Yuefan: #1]}} 
\newcommand{\Jiahao}[1]{{\color{orange}[Jiahao: #1]}} %%%Change to intern name
\newcommand{\Xiaoyu}[1]{{\color{purple}[Xiaoyu: #1]}} 


\usepackage{lineno}
\def\linenumberfont{\normalfont\small}


\ifdefined\isarxiv
\else
\icmltitlerunning{Dissecting Submission Limit in Desk-Rejections: A Mathematical Analysis of Fairness in AI Conference Policies}
\fi
\begin{document}

\ifdefined\isarxiv

\date{}

\title{Dissecting Submission Limit in Desk-Rejections: A Mathematical Analysis of Fairness in AI Conference Policies}
\author{
Yuefan Cao\thanks{\texttt{ ralph1997off@gmail.com}. Zhejiang University.}
\and
Xiaoyu Li\thanks{\texttt{
xiaoyu.li2@student.unsw.edu.au}. University of New South Wales.}
\and
Yingyu Liang\thanks{\texttt{
yingyul@hku.hk}. The University of Hong Kong. \texttt{
yliang@cs.wisc.edu}. University of Wisconsin-Madison.} 
\and
Zhizhou Sha\thanks{\texttt{
shazz20@mails.tsinghua.edu.cn}. Tsinghua University.}
\and
Zhenmei Shi\thanks{\texttt{
zhmeishi@cs.wisc.edu}. University of Wisconsin-Madison.}
\and
Zhao Song\thanks{\texttt{ magic.linuxkde@gmail.com}. The Simons Institute for the Theory of Computing at UC Berkeley.}
\and
Jiahao Zhang\thanks{\texttt{ ml.jiahaozhang02@gmail.com}. Independent Researcher.}
}

\else

% \title{Intern Project} 
% \maketitle 
% \iffalse
%\linenumbers

\twocolumn[

\icmltitle{Dissecting Submission Limit in Desk-Rejections: A Mathematical Analysis\\ of Fairness in AI Conference Policies}
% It is OKAY to include author information, even for blind
% submissions: the style file will automatically remove it for you
% unless you've provided the [accepted] option to the icml2019
% package.

% List of affiliations: The first argument should be a (short)
% identifier you will use later to specify author affiliations
% Academic affiliations should list Department, University, City, Region, Country
% Industry affiliations should list Company, City, Region, Country

% You can specify symbols, otherwise they are numbered in order.
% Ideally, you should not use this facility. Affiliations will be numbered
% in order of appearance and this is the preferred way.
\icmlsetsymbol{equal}{*}

\begin{icmlauthorlist}
\icmlauthor{Aeiau Zzzz}{equal,to}
\icmlauthor{Bauiu C.~Yyyy}{equal,to,goo}
\icmlauthor{Cieua Vvvvv}{goo}
\icmlauthor{Iaesut Saoeu}{ed}
\icmlauthor{Fiuea Rrrr}{to}
\icmlauthor{Tateu H.~Yasehe}{ed,to,goo}
\icmlauthor{Aaoeu Iasoh}{goo}
\icmlauthor{Buiui Eueu}{ed}
\icmlauthor{Aeuia Zzzz}{ed}
\icmlauthor{Bieea C.~Yyyy}{to,goo}
\icmlauthor{Teoau Xxxx}{ed}\label{eq:335_2}
\icmlauthor{Eee Pppp}{ed}
\end{icmlauthorlist}

\icmlaffiliation{to}{Department of Computation, University of Torontoland, Torontoland, Canada}
\icmlaffiliation{goo}{Googol ShallowMind, New London, Michigan, USA}
\icmlaffiliation{ed}{School of Computation, University of Edenborrow, Edenborrow, United Kingdom}

\icmlcorrespondingauthor{Cieua Vvvvv}{c.vvvvv@googol.com}
\icmlcorrespondingauthor{Eee Pppp}{ep@eden.co.uk}

% You may provide any keywords that you
% find helpful for describing your paper; these are used to populate
% the "keywords" metadata in the PDF but will not be shown in the document
\icmlkeywords{Machine Learning, ICML}

\vskip 0.3in
]

\printAffiliationsAndNotice{\icmlEqualContribution} 
% \fi
\fi





\ifdefined\isarxiv
\begin{titlepage}
  \maketitle
  \begin{abstract}
\begin{abstract}

% Recent works to jointly reconstruct 3D human and object from a single RGB image, are mostly model-based, that fail to capture the fine details of the clothed human body and object surface. In this paper, we introduce ReCHOR, a novel, model-free, first-method to produce realistic clothed human-object reconstructions from a monocular view. This is extremely challenging due to human-object occlusions, diverse interactions and depth ambiguity, as it needs to infer both 3D spatial awareness and high resolution details. Our core idea is based on estimating neural implicit representations for human and object respectively by an attention-based neural implicit model that attends to pixel-aligned features from both the global human-object image for spatial awareness and  the local separate view of human and object images for high quality details. Additionally, the network is conditioned on semantic features from an initial estimated human-object pose prior and a generative diffusion model that inpaints occluded regions, thus enabling the retrieval of details from them.
% We also propose a synthetic dataset with rendered scenes of diverse, inter-occluded 3D human and object scans, to train our network. We evaluate our method on the synthetic and real world BEHAVE dataset. Our experiments show that our method outperforms the SOTA in achieving realistic clothed human-object reconstructions.
Recent approaches to jointly reconstruct 3D humans and objects from a single RGB image represent 3D shapes with template-based or coarse models, which fail to capture details of loose clothing on human bodies. In this paper, we introduce a novel implicit approach for jointly reconstructing realistic 3D clothed humans and objects from a monocular view. For the first time, we model both the human and the object with an implicit representation, allowing to capture more realistic details such as clothing. This task is extremely challenging due to human-object occlusions and the lack of 3D information in 2D images, often leading to poor detail reconstruction and depth ambiguity. To address these problems, we propose a novel attention-based neural implicit model that leverages image pixel alignment from both the input human-object image for a global understanding of the human-object scene and from local separate views of the human and object images to improve realism with, for example, clothing details. Additionally, the network is conditioned on semantic features derived from an estimated human-object pose prior, which provides 3D spatial information about the shared space of humans and objects. To handle human occlusion caused by objects, we use a generative diffusion model that inpaints the occluded regions, recovering otherwise lost details. For training and evaluation, we introduce a synthetic dataset featuring rendered scenes of inter-occluded 3D human scans and diverse objects. Extensive evaluation on both synthetic and real-world datasets demonstrates the superior quality of the proposed human-object reconstructions over competitive methods.
\end{abstract}

  \end{abstract}
  \thispagestyle{empty}
\end{titlepage}

{\hypersetup{linkcolor=black}
\tableofcontents
}
\newpage

\else

\begin{abstract}
\begin{abstract}

% Recent works to jointly reconstruct 3D human and object from a single RGB image, are mostly model-based, that fail to capture the fine details of the clothed human body and object surface. In this paper, we introduce ReCHOR, a novel, model-free, first-method to produce realistic clothed human-object reconstructions from a monocular view. This is extremely challenging due to human-object occlusions, diverse interactions and depth ambiguity, as it needs to infer both 3D spatial awareness and high resolution details. Our core idea is based on estimating neural implicit representations for human and object respectively by an attention-based neural implicit model that attends to pixel-aligned features from both the global human-object image for spatial awareness and  the local separate view of human and object images for high quality details. Additionally, the network is conditioned on semantic features from an initial estimated human-object pose prior and a generative diffusion model that inpaints occluded regions, thus enabling the retrieval of details from them.
% We also propose a synthetic dataset with rendered scenes of diverse, inter-occluded 3D human and object scans, to train our network. We evaluate our method on the synthetic and real world BEHAVE dataset. Our experiments show that our method outperforms the SOTA in achieving realistic clothed human-object reconstructions.
Recent approaches to jointly reconstruct 3D humans and objects from a single RGB image represent 3D shapes with template-based or coarse models, which fail to capture details of loose clothing on human bodies. In this paper, we introduce a novel implicit approach for jointly reconstructing realistic 3D clothed humans and objects from a monocular view. For the first time, we model both the human and the object with an implicit representation, allowing to capture more realistic details such as clothing. This task is extremely challenging due to human-object occlusions and the lack of 3D information in 2D images, often leading to poor detail reconstruction and depth ambiguity. To address these problems, we propose a novel attention-based neural implicit model that leverages image pixel alignment from both the input human-object image for a global understanding of the human-object scene and from local separate views of the human and object images to improve realism with, for example, clothing details. Additionally, the network is conditioned on semantic features derived from an estimated human-object pose prior, which provides 3D spatial information about the shared space of humans and objects. To handle human occlusion caused by objects, we use a generative diffusion model that inpaints the occluded regions, recovering otherwise lost details. For training and evaluation, we introduce a synthetic dataset featuring rendered scenes of inter-occluded 3D human scans and diverse objects. Extensive evaluation on both synthetic and real-world datasets demonstrates the superior quality of the proposed human-object reconstructions over competitive methods.
\end{abstract}
\end{abstract}

\fi


\section{Introduction}
\label{sec:intro}
% Image editing methods in diffusion models depend on user-defined control directions - users can unlock their creativity using these methods by specifying the desired manipulation through prompts~\cite{gandikota2023concept}, reference images~\cite{ruiz2022dreambooth, kumari2022customdiffusion, gal2022image, chen2024trainingfreeregionalpromptingdiffusion}, or attribute vectors~\cite{parmar2023zero,hertz2022prompt}. In this work, we ask a fundamentally different question: \emph{Can we automatically discover the underlying visual structure of a concept within diffusion model's knowledge?} %Rather than requiring user-specified controls, we aim to decompose the model's internal knowledge into meaningful directions.

% This question touches on a fundamental limitation in how we interact with diffusion models. Current control methods ~\cite{zhang2023addingconditionalcontroltexttoimage, gandikota2023concept, ye2023ipadaptertextcompatibleimage,ye2023ipadaptertextcompatibleimage, hertz2024stylealignedimagegeneration, li2023photomaker, shi2024instantbooth, chen2024trainingfreeregionalpromptingdiffusion} require users to specify their desired manipulations in advance, limiting interactive creativity. This contrasts with natural human artistic workflows, where creators dynamically explore creative ideas while jointly refining them toward meaningful artistic outcomes~\cite{hoffmann2016modeling}. This synergy between specification and exploration is not new to generative models. Early GAN architectures naturally developed disentangled latent spaces that enabled continuous\cite{harkonen2020ganspace,radford2015unsupervised, wu2021stylespace, shen2020interfacegan}, compositional control over generated images. Users could explore these spaces to discover interesting variations that would be difficult to describe in words~\cite{wu2021stylespace}, then combine them to achieve their creative goals~\cite{grabe2022towards}. 


% While diffusion models have largely superseded GANs in conditional image synthesis~\cite{dhariwal2021diffusion},  their underlying structure remains less understood. Diffusion models achieve remarkable diversity through high-dimensional latents, unlike GANs' compact latent spaces.  With a single prompt, diffusion models can generate radically different variations through different random initializations of input noise. We ask - Is it possible to discover interpretable structure within this vast space of variations?

Text-to-image diffusion models are capable of generating remarkable visual variations from a single prompt through different random initializations. However, this vast creative potential remains largely opaque to users---while we can generate diverse images, we lack understanding of the underlying structure of these variations. This presents a fundamental challenge: how can we discover and expose the latent visual capabilities encoded within these models?

\let\thefootnote\relax \footnote{$^{*}$Correspondence to \texttt{gandikota.ro@northeastern.edu}}

The challenge touches on a key limitation in how we interact with diffusion models today. Current control methods require users to explicitly specify their desired edits in advance through prompts~\cite{gandikota2023concept}, reference images~\cite{zhang2023addingconditionalcontroltexttoimage, chen2024trainingfreeregionalpromptingdiffusion, ruiz2022dreambooth,kumari2022customdiffusion, Ryu_lora, hu2021lora}, or attribute vectors~\cite{ye2023ipadaptertextcompatibleimage, hertz2024stylealignedimagegeneration, li2023photomaker, shi2024instantbooth,parmar2023zero,hertz2022prompt}. That contrasts sharply with natural human creative workflows, where artists dynamically explore creative ideas and jointly refine them toward meaningful artistic outcomes~\cite{hoffmann2016modeling}. The need for pre-specified controls creates a barrier between users and the full creative potential of these models.

Interestingly, earlier generative models like GANs~\cite{gans,karras2019style,brock2018large} naturally developed more interpretable internal structures. Their compact latent spaces often exhibited emergent disentanglement~\cite{harkonen2020ganspace,radford2015unsupervised, wu2021stylespace, shen2020interfacegan}, enabling continuous and compositional control over generated images. Users could explore these spaces to discover interesting variations that would be difficult to describe in words~\cite{wu2021stylespace}, then combine them to achieve their creative goals~\cite{grabe2022towards}.

Diffusion models have largely superseded GANs in conditional image synthesis~\cite{dhariwal2021diffusion}, achieving greater diversity through much higher-dimensional latents. And yet an understanding of the underlying structure of these larger latent spaces has remained elusive. In this work, we ask a fundamental question: \emph{Can we automatically discover the visual structure within a diffusion model's knowledge of a concept?} Rather than requiring user-specified controls, we aim to decompose the model's internal representations into expressive directions that users can explore and combine.

To address these needs, we present \textbf{SliderSpace}, a framework that brings systematic explorability to diffusion models. Given just a text prompt, SliderSpace discovers a canonical set of meaningful, diverse, and controllable directions within the model's knowledge of that concept. Each direction is implemented as a low-rank adapter~\cite{hu2021lora} that can be scaled and composed with others, allowing users to explore and smoothly combine different aspects of variation, as shown in Figure~\ref{fig:intro}.

We ground SliderSpace discovery in three key requirements for meaningful decomposition of a diffusion model's visual manifold: 
\begin{enumerate}
    \item \textbf{Unsupervised Discovery:} The decomposition process should emerge from the intrinsic structure of the model's learned representation, rather than being guided by predefined attributes. This ensures we capture the true topology of the model's knowledge space rather than projecting our assumptions onto it.
    
    \item \textbf{Semantic Orthogonality:} Each discovered control must represent a distinct semantic direction. This is enforced in a semantic feature space, like CLIP, where every slider has an orthogonal effect in embeddings. This prevents discovering multiple controls that create similar semantic effects, making the system more efficient and easier.
    
    \item \textbf{Distribution Consistency:} Directions must induce consistent transformations across both random seeds and prompt variations. 
\end{enumerate}

These requirements naturally lead to our proposed framework, which we formalize in Section~\ref{sec:method}. As we show in our experiments, SliderSpace is architecture-agnostic, working with both conventional U-Net based models like Stable Diffusion~\cite{rombach2022high, rombach2022sd20, podell2023sdxl, turbo, dmd} and recent transformer-based architectures like Flux~\cite{flux}.

We demonstrate the expressiveness of SliderSpace through three applications: First, we show how SliderSpace can decompose high-level concepts into diverse and expressive components, revealing the natural axes of variation in the model's understanding. Second, we explore artistic style variation, where SliderSpace discovers directions that match or exceed the diversity of manually curated artist lists while being judged more useful by human evaluators. Finally, we show how SliderSpace can help reverse the mode collapse commonly observed in distilled diffusion models, restoring diversity while maintaining generation speed.

Beyond providing practical creative control, SliderSpace opens new avenues for understanding and utilizing the latent capabilities of diffusion models. By mapping these models' visual potential into intuitive, composable directions, we take a step toward making their creative possibilities more accessible and interpretable to users.

% Image editing methods in diffusion models unlock the creativity of users. In this work we ask an alternate question: \emph{Can we organize and expose what of the diffusion model is already capable of?}.
% Existing methods for controlling image generation typically require users to manually specify edit directions for desired changes. This process is time-consuming, requires technical expertise, and limits the spontaneity of the creative process. For instance, if a user wants to adjust the smile of a generated person, they must explicitly request this edit, often through imprecise prompt engineering or model fine-tuning. This approach of predefined controls or manual specifications restricts users from fully exploring the latent capabilities of the model. There may be interesting stylistic variations or attributes that the model can generate, but users have no easy way to discover or utilize these.

% Natural visual disentanglement was an emergent property in the latent space of Generative Adversarial Models (GANs) \cite{harkonen2020ganspace,radford2015unsupervised, wu2021stylespace, shen2020interfacegan}. In particular, it has been observed that StyleGAN~\cite{karras2019style} stylespace neurons offer detailed control over many meaningful aspects of images that would be difficult to describe in words~\cite{wu2021stylespace}. However, diffusion models do not share such a compact latent space~\cite{park2023unsupervised}; and efforts to uncover such a space in the semantic embeddings of the text conditioning have met with limited success \nik{Nick - is there a specific citation you were thinking about?}.

% In this work we introduce \textbf{SliderSpace}, which takes a step towards uncovering an analogous low dimensional representation of diffusion models' visual breadth; in essence treating the diffusion model as many generators sharing parameters, where a particular generator is defined by a specific prompt. For a given prompt we sample many random seeds (and optionally prompt expansions using an LLM), generate the corresponding images, and apply an off the shelf feature extractor (in this work CLIP, but our method can be applied to any differentiable feature extractor). We use PCA to analyze these features, and for each of the leading $k$ principal components we train a LoRA \cite{} which causes the diffusion model to produces images which increase the feature magnitude along that component when passed back through the same feature extractor. This leads to a 'Slider' for each principal component, because each LoRA can be scaled and applied to the original diffusion model, continuously varying those visual features in the generated results (as measured, in our case, by CLIP).

% There are many other works that enhance the controllability of diffusion models. One common approach is enabling users to add spatial constraints to a generation either manually, or via a reference image \cite{zhang2023addingconditionalcontroltexttoimage, chen2024trainingfreeregionalpromptingdiffusion}, a second is leveraging more abstract embeddings (e.g. identity, style) extracted from a reference image \cite{ye2023ipadaptertextcompatibleimage, hertz2024stylealignedimagegeneration, li2023photomaker, shi2024instantbooth}, a third is finetuning a foundation model to better generate a concept important to the user \cite{ruiz2022dreambooth, kumari2022customdiffusion, Ryu_lora, hu2021lora}, and a fourth (most relevant to this work) is finding low-rank adaptors of the model based on a prompt or small training set which can be scaled to provide continous control over one aspect of generated image (e.g. night vs day, basic vs luxury, etc.) \cite{gandikota2023concept}. SliderSpace is complementary to all of these methods and offers something distinct. All of the other methods we are aware require the user (and / or model designer) to know in advance what type of control they want. In contrast SliderSpace assists users in discovering and controlling hidden capabilities present in the diffusion model's distribution of possible generations.

%We propose that truly intuitive creative control in a text-to-image model should meet three key criteria: \emph{discoverability}, \emph{intuitiveness}, and \emph{specificity}. The model should reveal controllable attributes that may not be immediately obvious, offer controls that are easy to understand and manipulate, and ensure each control affects a distinct attribute of the generated image.

% We demonstrate the utility and power of SliderSpace using three applications built on top of SDXL-DMD \cite{dmd}, because its fast generation speed lends itself well to the continuous control offered by SliderSpace.

% First, we study concept decomposition (Section \ref{sec:concept_exp}), where we learn sliders for a specific concept (e.g. 'monster', 'waterfall', 'car'). Through quantitative metrics of diversity and text alignment we demonstrate that the learned sliders dramatically boost the diversity of generations when randomly applied without harming text alignment; we also ask humans to qualitatively judge these results in a user study where they find the SliderSpace results to be more 'Diverse', 'Useful', and 'Creative' than our baselines.

% Second, we attempt to compare the automatic discoveries of SliderSpace to a large scale manual study of artistic styles (Section \ref{sec:art_exp}), open-sourced by ParrotZone \cite{parrotzone}. In this study SDXL was prompted with over 4300 artist names,  and based on visual inspection the cases of successful stylistic mimicry recorded. Quantitatively SliderSpace more closely matches the distribution of artistic variation discovered by ParrotZone than other baselines, and in our user studies was judged to be significantly more 'Diverse' and 'Useful' than the baselines. To our surprise humans even judged SliderSpace results to be slightly more 'Diverse' than the results generated by the manually discovered artist names of \cite{parrotzone}.

% Third, we attempt to use SliderSpace to reverse the mode collapse commonly observed in distilled few-step diffusion models relative to the original teacher model (Section \ref{sec:diverse_exp}). We quantitatively demonstrate that applying SliderSpace to SDXL-DMD leads to more closely matching the distribution of images by the original teacher, SDXL.

%Through extensive experiments on various state-of-the-art text-to-image models, we demonstrate that SliderSpace significantly enhances user control and creative expression in AI-assisted image generation tasks. Our method enables a range of applications, including concept decomposition and control, diversity improvement in generated images, customization dissection and edits, and the exploration of artistic styles inherent in the model.

% SliderSpace goes beyond providing a practical tool for enhanced creative control. By mapping the visual potential of diffusion models it can open new avenues for generative creativity and deepens our understanding of each model's hidden potential. %%% Section 1. Introduction
\section{Related Work}

\subsection{First-order logic for natural entailment}

Since the start of the RTE challenge \citep{rte}, multiple works have attempted using FOL representations to solve natural language entailment. These methods first obtain the syntactic/semantic parse tree and apply a rule-based transformation to get the FOL representation \citep{bos-markert-2005-recognising, bos-nli}. However, it was repeatedly shown that these FOL representations are not empirically effective in solving natural language entailment. For instance, \citet{bos-nli} reported that FOL representations translated from the discourse representation structure (DRS) yield only 1.9\% recall in detecting the entailment in the single-premise RTE benchmark \citep{rte}.

Independently from these works, multi-premise logical entailment benchmarks \citep{tafjord-etal-2021-proofwriter, logicnli, folio} were developed to evaluate the reasoning ability of generative models. These benchmarks adopt the classic 3-way entailment label classification format (\textit{entailment, contradiction, neutral}) of single-premise RTE tasks, in which both the NL sentences and their gold FOL representations point to the same entailment label. 

Recent works have applied LLMs to obtain FOL representations for these multi-premise logical entailment tasks \citep{logiclm, linc, divide-and-translate}, fueled by the code generation ability of LLMs. While they achieve significant performance in synthetic, controlled logical reasoning benchmarks, whether they can generalize to natural entailment has remained unanswered. Furthermore, \citet{linc} observed that LLMs are highly susceptible to \textit{arbitrariness}, as they fail to produce coherent predicate names or numbers of arguments even when generating FOL representations of premises and hypotheses in a single inference.

\subsection{Executable semantic representations}

Apart from FOL, a stream of research focuses on the \textit{executability} of semantic representations. From this perspective, semantic representations are \textit{program codes} that can be executed to solve downstream tasks, such as query intent analysis \citep{spider, dligach-etal-2022-exploring} and question answering \citep{semparse-qa}. The performance of the semantic parser is directly assessed by the accuracy of execution results for the downstream tasks, rather than the similarity between the prediction and the reference parse.

To improve the execution accuracy that is often non-differentiable, reinforcement learning (RL) and its variants have been applied to train neural semantic parsers \citep{cheng-etal-2019-learning, cheng-lapata-2018-weakly}. Using only the input sentence and the desired execution result, these methods learn to maximize the probability of the representations that lead to the correct execution result. However, these approaches are not directly applicable to EPF, as EPF requires taking account of \textit{interactions between premises and hypotheses} during execution (\textit{i.e.} theorem proving) while these methods assume that sentences are isolated.


\section{Preliminary} \label{sec:preli}

In Section~\ref{sub:notation}, we introduce all the notations we used in our paper. Then, in Section~\ref{sub:flow_matching}, we show the basic facts about flow matching. In Section~\ref{sub:special_relativity}, we present the basic background of special relativity and define the relativistic force.

\subsection{Notations} \label{sub:notation}

For any positive integer $n$, we use $[n]$ to denote set $\{1,2,\cdots, n\}$. 
For two vectors $x \in \R^n$ and $y \in \R^n$, we use $\langle x, y \rangle$ to denote the inner product between $x,y$.
For a vector $v \in \R^n$, we use $\|v\|_2$ to denote the $\ell_2$-norm of $v$.
We use ${\bf 1}_n$ to denote a length-$n$ vector where all the entries are ones.
We use the symbol $ \perp $ to represent a component that is perpendicular to the direction of velocity, as exemplified by $ a_{\perp t} $, which denotes the perpendicular acceleration. Similarly, the symbol $ \parallel $ is employed to indicate a component parallel to the direction of velocity, such as $ f_{\parallel t} $, which represents the parallel force. We use $\dot{x}_t$ to denote $\frac{\d x_t}{\d t}$, and $\ddot{x}_t$ to denote $\frac{\d^2 x_t}{\d t^2}$.


\subsection{Flow Matching} \label{sub:flow_matching}

Flow Matching (FM) \cite{lcb+22,lgl22} is a generative modeling technique that constructs a smooth, invertible (i.e., diffeomorphic) mapping from a simple prior distribution to a complex target distribution. In FM, a time-dependent mapping $Z_t$ is defined to evolve according to an ordinary differential equation (ODE) driven by a vector field:
\begin{align*}
    \frac{\d x_t}{\d t} = V_t(x_t), \quad t \in [0, T].
\end{align*}
The goal is to ensure that, at the terminal time $T$, the ODE transforms a sample $x_0$ from a simple distribution (e.g., a Gaussian) into a sample $x_T$ from the target data distribution $\mathcal{D}$.

To achieve this, Flow Matching (FM) constructs a stochastic interpolation between a sample $x_1 \sim \mathcal{D}$ and a sample $x_0$ drawn from a known prior distribution, typically $\N(0,I)$. The interpolation is defined as
\begin{align*}
    x_t := \alpha_t x_1 + \sigma_t x_0, \quad t\in [0,T],
\end{align*}
where the time-dependent coefficients $\alpha_t$ and $\sigma_t$ are chosen so that
\begin{align*}
    \alpha_0 = 0,\quad \sigma_0 = 1,\quad \alpha_T = 1,\quad \sigma_T = 0.
\end{align*}
Thus, at $t=0$ the interpolated sample is purely the prior ($x_0$), and at $t=T$ it becomes a data sample ($x_1$).

The instantaneous change of $x$ is obtained by differentiating the interpolation:
\begin{align*}
    \frac{\d x_t}{\d t} = \frac{\d \alpha_t}{\d t} x_1 + \frac{\d \sigma_t}{\d t} x_0.
\end{align*}

The vector field is approximated by a neural network $V_t(x_t)$ with learnable parameters $\theta$. The FM training objective is then given by
\begin{align*}
    \mathcal{L}_\mathrm{FM}(\theta) := \E_{t\sim {\sf Uniform}[0,T], x_1 \sim \mathcal{D}} [\| V_t(x_t) - v_t(x_t) \|_2^2 ].
\end{align*}
This loss ensures that the learned velocity field $V_t(x_t)$ closely tracks the conditional dynamics $v_t(x_t)$ along the interpolation path.

After training, samples are generated by solving the ODE
\begin{align*}
    \frac{\d x_t}{\d t} = V_t(x_t),
\end{align*}
starting from an initial sample $x_0 \sim \N(0,I)$. Integrating this ODE from $t=0$ to $t=T$ yields a sample $x_T$ that approximates a draw from the target distribution. This ODE-based formulation offers a flexible and powerful framework for modeling complex data distributions while naturally incorporating conditional sampling.

\subsection{Background on Special Relativity} \label{sub:special_relativity}

We first introduce several essential ideas of special relativity \cite{e+05}.

\begin{definition}[Lorentz Factor]
\label{def:LorentzFactor}
According to special relativity~\cite{e+05}, the Lorentz factor at lab time $t$ is given by
\begin{align*}
\gamma_t := \frac{1}{\sqrt{1 - {\|v_t^{\rm lab}\|_2^2}/{c^2}}},
\end{align*}
where $v_t^{\rm lab}$ is the velocity at lab frame of reference, $c = 3 \times 10^8$ is the speed of light in vacuum.
\end{definition}

Then, we introduce the proper time of special relativity.

\begin{definition}[Proper Time]
\label{def:ProperTime}
The proper time is defined as the time interval measured in the rest frame of a moving object according to special relativity~\cite{e+05}. The differential form of the proper time is given by
\begin{align*}
    \d \tau = \frac{\d t}{\gamma_t},
\end{align*}
where $\d t$ is the time interval in the laboratory frame of reference, and $\gamma_t$ is the Lorentz factor at time lab time $t$ as defined in Definition~\ref{def:LorentzFactor}.
\end{definition}

Next, we define the force under special relativity here.

\begin{definition}[Relativistic Force]
\label{def:RelativisticForce}
In the framework of special relativity, the \emph{local force} (i.e., the force measured in the instantaneous rest frame of the particle) denoted as $f^{\rm local}$ has
\begin{align}
    f^{\rm local} := \frac{\d p^{\rm lab}}{\d \tau}, \label{eq:f_local}
\end{align}
where $p^{\rm lab}$ is the momentum at lab frame of reference, $\tau$ denotes the proper time defined in Definition~\ref{def:ProperTime}.

The momentum in the lab frame is defined as
\begin{align}
    p^{\rm lab} := m^{\rm lab} v_t^{\rm lab}, \label{eq:p}
\end{align}
where $m^{\rm lab}$ is the mass at lab frame of reference, and $v_t^{\rm lab}$ is the velocity at lab frame of reference.
\end{definition}

We state an equivalence lemma. Due to the space limitation, we delayed the proofs into the appendix.
\begin{lemma}[Equivalent Form of Relativistic Force, informal version of Lemma~\ref{lem:equiv_relativistic_force:formal}]\label{lem:equiv_relativistic_force:informal}
Let $p^{\rm lab}$ be the momentum defined in Eq.~\eqref{eq:p}, $\gamma_t$ be the Lorentz factor at lab time $t$ defined in Definition~\ref{def:LorentzFactor}, $\tau$ denotes the proper time, $v_t^{\rm lab} = \dot{x}_t$ denotes the velocity, 
$a_t^{\rm lab} = \ddot{x}_t$ denotes the acceleration.
The relativistic force, defined as the time derivative of the momentum in the lab frame, can be written as
\begin{align*}
f^{\rm local} =  m^{\rm lab}  (\gamma_t a_t^{\rm lab} + \gamma_t^3 \frac{ \langle v_t^{\rm lab}, a_t^{\rm lab} \rangle}{c^2} v_t^{\rm lab}).
\end{align*}

\end{lemma} 
\section{The Desk Rejection Dilemma}\label{sec:dr_dilemma}

In this section, we define the concept of an ideal desk-rejection system in Section~\ref{sec:good_solution} and formally demonstrate in Section~\ref{sec:good_solution_hard} that no algorithm can achieve this ideal system.

\subsection{Ideal Desk-Rejection}\label{sec:good_solution}
An ideal desk-rejection system should avoid unfairly rejecting papers from authors who either comply with the submission limit or exceed it by only one or two papers. Otherwise, authors may face consequences due to co-authors with an excessively high number of submissions. This issue is particularly problematic for early-career researchers, as such collective penalties can have a significant negative impact on their careers. 

To address this, we formally define the criteria for an ideal desk-rejection outcome for the problem in Definition~\ref{def:submit_limit_problem}, where rejections are based solely on an author’s excessive submissions, without unfairly penalizing others.

\begin{definition}[Ideal desk-rejection] \label{def:good_solution}
An ideal solution for the submission limit problem in Definition~\ref{def:submit_limit_problem} is a paper subset $S\subseteq \mathcal{P}$ such that every author has exactly $\min\{x, |P_i|\}$ papers remaining after desk rejection. 

\end{definition}
\begin{remark}
    The ideal desk-rejection in Definition~\ref{def:good_solution} ensures that innocent authors with less than $x$ submissions will retain all their papers, and a non-compliant author $a_i$ with more than $x$ submissions will be desk-rejected exact $(|P_i|-x)$ papers.
\end{remark}
Thus, if there exists an algorithm that can reach the aforementioned ideal solution, we can ensure that no author is unfairly penalized due to their co-authors' submission behavior, achieving both fairness and individual accountability.
\subsection{Hardness of Ideal Desk-Rejection}\label{sec:good_solution_hard}

Unfortunately, we find that achieving an ideal desk-rejection system is fundamentally intractable. The main result regarding this hardness is presented in the following theorem:

\begin{theorem}[Hardness of Ideal Desk-Rejection]\label{thm:main_res_general}
Let $n = |\mathcal{A}|$ denote the number of authors in Definition~\ref{def:submit_limit_problem}. We can show that

\begin{itemize}
    \item {\bf Part 1:} For $n \le 2$, there always exists an algorithm that can achieve the ideal desk-rejection in Definition~\ref{def:good_solution}.
    \item {\bf Part 2:} For $n \geq 3$, there exists at least one problem instance where no algorithm can guarantee achieving the ideal desk-rejection in Definition~\ref{def:good_solution}.
\end{itemize}
\end{theorem}

\begin{proof} For {\bf Part 1}, the result follows directly from Lemma~\ref{lem:n_eq_1_positive_general} and Lemma~\ref{lem:n_eq_2_positive_general}. For {\bf Part 2}, the result is established using Lemma~\ref{lem:n_eq_3_negative} and Lemma~\ref{lem:n_geq_3_negative}. 
Detailed technical proofs for these lemmas are provided in Appendix~\ref{sec:dilemma_proof}.
\end{proof}

Therefore, since an ideal desk-rejection system is not achievable, it is inevitable that some authors may face excessive desk-rejections due to collective punishments. This challenge is particularly concerning for early-career researchers with only one or two submissions, motivating the need to seek an approximate solution that optimizes fairness in desk-rejection systems.

\section{Fairness-Aware Desk-Rejection}\label{sec:fair}

In this section, we first introduce two fairness metrics in Section~\ref{sec:fair_metric}, and then present the hardness result on minimizing one of them in Section~\ref{sec:good_solution_hard}. In Section~\ref{sec:fair_optim}, we show our optimization-based fairness-aware desk-rejection framework. 

\subsection{Fairness Metrics}\label{sec:fair_metric}

As discussed earlier, achieving an ideal desk-rejection system is practically infeasible, as unintended rejections due to collective punishments are unavoidable. To address this, we relax the ideal system into an approximate form, where some unfair desk-rejections are permitted, while these rejections should be proportional to each author's total number of submissions.

Specifically, we introduce a cost function for each author, which estimates the impact of desk-rejection on each author:

\begin{definition}[Cost Function]\label{def:cost}
Considering the submission limit problem in Definition~\ref{def:submit_limit_problem}, we define the cost function $c: [n] \times 2^{[m]} \to [0,1]$ for a specific author $a_i$ and a set of remaining paper $S$ as
\begin{align*}
    c(a_i, S) := \frac{|P_i| - |\{p_j \in S: a_i \in A_j\}|}{|P_i|}.
\end{align*}
\end{definition}

\begin{remark}
    The cost function $c(a_i,S)$ measures the proportion of papers authored by $a_i$ that are rejected, prioritizing fairness for early-career authors with fewer submissions and aiming to reduce setbacks for them.
\end{remark}

To further demonstrate how this author-wise cost function could benefit fairness, we present the following example:
\begin{example}
    Consider a submission limit problem with $x = 10$ and $n = 2$. Suppose author $a_1$ submits papers $p_1, p_2, \ldots, p_{11}$, and author $a_2$ submits only paper $p_{11}$. Rejecting paper $p_{11}$ (i.e., $S = \mathcal{P} \setminus \{p_{11}\}$) results in a cost of $c(a_1, S) = 1/11$ for $a_1$ but a cost of $c(a_2, S) = 1$ for $a_2$, which is unfair to $a_2$. On the other hand, if we reject paper $p_1$ (i.e., $S' = \mathcal{P} \setminus \{p_1\}$), the cost for $a_1$ remains $c(a_1, S') = 1/11$, while the cost for $a_2$ becomes $c(a_2, S') = 0$. This minimizes both the highest cost and the total cost. This example demonstrates that our cost function encourages rejecting papers from authors with many submissions while protecting authors with few submissions.
\end{example}

To ensure fair treatment for all authors and avoid imposing excessive setbacks on early-career researchers, we introduce two fairness metrics based on our cost function. These metrics are inspired by the principles of utilitarian social welfare and egalitarian social welfare~\cite{ams24}. We begin by defining individual fairness, which is a strict worst-case fairness metric that aligns with the egalitarian social welfare framework by estimating the individual cost among all authors.

\begin{definition}[Individual Fairness]\label{def:individual_fair}
Let $c: [n] \times 2^{[m]} \to [0,1] $ be the cost function defined in Definition~\ref{def:cost}.
We define function $\zeta_{\mathrm{ind}}: 2^{[m]} \to [0,1]$ to measure the individual fairness:
\begin{align*}
    & ~ \zeta_{\mathrm{ind}}(S) := \max_{i \in [n]} c(a_i,S).
\end{align*}
\end{definition}
Next, we present the concept of group fairness, which aligns with utilitarian social welfare and measures the total cost across all authors.

\begin{definition}[Group Fairness]\label{def:group_fair}
Let $c: [n] \times 2^{[m]} \to [0,1] $ be the cost function defined in Definition~\ref{def:cost}.
We define function $\zeta_{\mathrm{group}}: 2^{[m]} \to [0,1]$ to measure the group fairness:
\begin{align*}
    & ~ \zeta_{\mathrm{group}}(S) := \frac{1}{n}\sum_{i \in [n]} c(a_i,S).
\end{align*}
 \end{definition}

To show the relationship between these two fairness metrics, we have the following proposition:

\begin{proposition}[Relationship of Fairness Metrics, informal version of Proposition~\ref{lem:fair_metric_ineq_append} in Appendix~\ref{sec:fair_proof}]\label{lem:fair_metric_ineq}
    For any solution $S\subseteq \mathcal{P}$ to the submission limit problem in Definition~\ref{def:submit_limit_problem}, we have 
    \begin{align*}
        \zeta_{\mathrm{ind}}(S) \leq \zeta_{\mathrm{group}}(S).
    \end{align*}
\end{proposition}


\subsection{Hardness of Individual Fairness-Aware Submission Limit Problem}\label{sec:indi_fair_hard}

After presenting fairness metrics for the desk-rejection system, we introduce an optimization-based framework to address these metrics. We first study the individual fairness-aware submission limit problem to minimize the individual fairness measure $\zeta_{\mathrm{ind}}$ in Definition~\ref{def:individual_fair}. 


\begin{definition}[Individual Fairness-Aware Submission Limit Problem]\label{def:ind_fair_min}
    We consider the following optimization problem:
\begin{align*}
    & ~ \min_{S \subseteq \mathcal{P}} \zeta_{\mathrm{ind}}(S) \\
    \mathrm{s.t.} & ~ |\{p_j \in S : a_i \in A_j\}| \leq x, \quad \forall a_i \in \mathcal{A}.
\end{align*}
\end{definition}

To represent the fairness metric minimization problem in matrix form, we introduce the following definition:

\begin{definition}[Author-Paper Matrix]\label{def:author_paper_mat}
    Let $W \in \{0, 1\}^{n \times m}$ denote the author-paper matrix for the author set $\mathcal{A}$ and paper set $\mathcal{P}$. Then, we define $W_{i,j} = 1$ if author $a_i$ is a coauthor of paper $p_j$, and $W_{i,j} = 0$ otherwise.
\end{definition}

Therefore, we present a more tractable integer
programming form of the original problem and prove its equivalence to the original
formulation:

\begin{definition}[Individual Fairness-Aware Submission Limit Problem, Matrix Form]\label{def:ind_fair_min_matrix}
    We consider the following integer optimization problem:
    \begin{align*}
        & ~ \min_{r \in \{0,1\}^m} \| \mathbf{1}_n - D^{-1}Wr\|_\infty \\
    \mathrm{s.t.} & ~ 
     (W r) / x \leq \mathbf{1}_n
    \end{align*}
    where $D = \Diag(|P_1|, \cdots, |P_n|)$, and the rejection vector $r \in \{0, 1\}^m$ is a 0-1 vector, with $r_j = 1$ indicating that paper $p_j$ is remained, and $r_j = 0$ indicating that it is desk-rejected. 
\end{definition}

\begin{proposition}[Matrix Form Equivalence for $\zeta_{\mathrm{ind}}$, informal version of Proposition~\ref{prop:equiv_individual_append} in Appendix~\ref{sec:fair_proof}]\label{prop:equiv_individual}
    The individual fairness-aware submission limit problem in Definition~\ref{def:ind_fair_min} and the matrix form integer programming problem in Definition~\ref{def:ind_fair_min_matrix} are equivalent.
\end{proposition}

Unfortunately, solving this integer programming problem is highly non-trivial, which means it may not yield a feasible solution within a reasonable time for large-scale conference submission systems. We establish the computational hardness of this problem in the following theorem:
 
\begin{theorem}[Hardness, informal version of Theorem~\ref{thm:indi_nphard_append} in Appendix~\ref{sec:indi_fair_hard_append}]\label{thm:indi_nphard}
    The Individual Fairness-Aware Submission Limit Problem defined in Definition~\ref{def:ind_fair_min} is $\NPhard$.
\end{theorem}

Since minimizing individual fairness is computationally intractable, our fairness-aware desk-rejection system instead focuses on minimizing group fairness. 


\subsection{Group Fairness Optimization}\label{sec:fair_optim}

Given the inherent hardness of individual fairness optimization, we address the fairness problem using an alternative yet equally important metric: group fairness, as defined in Definition~\ref{def:group_fair}. This metric is not only a crucial fairness measure in its own right but also serves as a lower bound for individual fairness as stated in Proposition~\ref{lem:fair_metric_ineq}, potentially improving individual fairness implicitly. 

Following a similar approach in Section~\ref{sec:indi_fair_hard}, we first formulate the submission limit problem with respect to group fairness and derive a more tractable integer programming formulation in matrix form:

\begin{definition}[Group Fairness-Aware Submission Limit Problem]\label{def:group_fair_min}
    We consider the following optimization problem:
\begin{align*}
    & ~ \min_{S \subseteq \mathcal{P}} \zeta_{\mathrm{group}}(S) \\
    \mathrm{s.t.} & ~ |\{p_j \in S : a_i \in A_j\}| \leq x, \quad \forall a_i \in \mathcal{A}.
\end{align*}
\end{definition}


\begin{definition}[Group Fairness-Aware Submission Limit Problem, Matrix Form]\label{def:group_fair_min_mat_new}
    We consider the following integer programming problem:
    \begin{align*}
        &~ \max_{r \in \{0, 1\}^m} \mathbf{1}^\top_n D^{-1} W r \\ 
        \mathrm{s.t.} 
        & ~ (W r) / x \leq \mathbf{1}_n,
    \end{align*}
    where $D = \Diag(|P_1|, \cdots, |P_n|)$, and the rejection vector $r \in \{0, 1\}^m$ is a 0-1 vector, with $r_j = 1$ indicating that paper $p_j$ is remained, and $r_j = 0$ indicating that it is desk-rejected. 
\end{definition}

\begin{proposition}[Matrix Form Equivalence for $\zeta_{\mathrm{group}}$, informal version of Proposition~\ref{lem:group_fair_min_equiv_append} in Appendix~\ref{sec:fair_proof}]\label{lem:group_fair_min_equiv}
    The fairness-aware submission limit problem in Definition~\ref{def:group_fair_min} and the matrix form integer programming problem in Definition~\ref{def:group_fair_min_mat_new} are equivalent.
\end{proposition}

However, solving integer programming problems is practically challenging. To this end, we first relax the feasible region of $r$ to $[0,1]^m$, and then analyze the resulting relaxed problem.  

\begin{definition}[Group Fairness-Aware Submission Limit Problem, Relaxation]\label{def:group_fair_min_mat_relax_new}
    We consider the optimization problem
    \begin{align*}
    &~ \max_{r \in [0, 1]^m}  \mathbf{1}^\top_nD^{-1}Wr \\ 
        \mathrm{s.t.} 
        & ~ (Wr)/x\le \mathbf{1}_n,
\end{align*}
where $D = \Diag(|P_1|, \cdots, |P_n|)$, and the rejection vector $r \in \{0, 1\}^m$ is a 0-1 vector, with $r_j = 1$ indicating that paper $p_j$ is remained, and $r_j = 0$ indicating that it is desk-rejected. 
\end{definition}

Fortunately, the relaxed problem is a linear program, which can be efficiently solved using standard linear programming solvers. Moreover, its optimal solution is equivalent to that of the original integer programming problem, an this result is formalized in the following theorem:

\begin{theorem}[Optimal Solution Equivalence of the Relaxed Problem, informal version of Theorem~\ref{thm:lp_equiv_append} in Appendix~\ref{sec:fair_proof}]\label{thm:lp_equiv}
    The optimal solution of the relaxed linear programming problem in Definition~\ref{def:group_fair_min_mat_relax_new} is equivalent to the optimal solution of the original integer programming problem in Definition~\ref{def:group_fair_min_mat_new}.
\end{theorem}

\begin{algorithm}[!ht]
\caption{Fairness-Aware Desk-Reject Algorithm}
\label{alg:fair_desk_reject_algo}
\begin{algorithmic}[1]

\State {\color{blue} /* $\mathcal{A}$ denotes the set of $n$ authors. */}
\State {\color{blue} /* $\mathcal{P}$ denote the set of $m$ papers. */}
\State {\color{blue} /* Author $a_i \in \mathcal{N}$ has a subset of papers $P_i \subset \mathcal{P}$. */}
\State {\color{blue} /* Paper $p_j \in \mathcal{P}$ is coauthored by a subset of authors $A_j \subseteq \mathcal{A}$.*/}
\State {\color{blue} /* $x$ represents the submission limit for each author.*/}

\Procedure{FairDeskReject}{$\mathcal{A}, \mathcal{P}, x$} 
\State {\color{blue} /* Initialize the constants of the problem. */}
\For{$i \in [n], j \in [m]$}
\If{$p_j \in \mathcal{A}_i$}
\State $W_{i,j}\gets 1$
\Else
\State $W_{i,j}\gets 0$
\EndIf
\EndFor
\State $D\gets\Diag(|P_1|, \ldots, |P_n|)$
\State {\color{blue} /* Solve the linear programming problem in Definition~\ref{def:group_fair_min_mat_relax_new}. */}
\State $r^{\star} \gets \mathsf{LPSolver}(W, D, x, r^0)$
\State {\color{blue} /* Transform the solution. */}
\State $S\gets\emptyset$
\For {$j\in[m]$}
\If {$r_j = 1$}
\State $S\gets S \cup \{p_j\}$
\EndIf
\EndFor
\State \Return $S$ 
\EndProcedure
\end{algorithmic}
\end{algorithm}

This theoretical result is significant as we formally establish that the group fairness-aware submission problem in Definition~\ref{def:group_fair_min} reduces to a linear programming (LP) problem with guaranteed optimality, solvable using off-the-shelf LP solvers. We formalize this procedure in Algorithm~\ref{alg:fair_desk_reject_algo}, where $\mathsf{LPSolver}$ denotes any standard LP solver, including but not limited to the simplex method~\cite{bg69}, interior-point path-finding methods~\cite{ls14}, and state-of-the-art stochastic central path methods~\cite{cls19, jswz21}.

\begin{remark}
    The time complexity of our fairness-aware desk-rejection algorithm in Algorithm~\ref{alg:fair_desk_reject_algo} aligns with modern linear programming solvers. For instance, using the stochastic central path method~\cite{cls21,jswz21,vlss20,sy21}, it achieves a time complexity of $O^*(m^{2.37} \log(m/\delta))$, where $\delta$ represents the relative accuracy corresponding to a \((1+\delta)\)-approximation guarantee.
\end{remark}

\begin{remark}
    In practice, major AI conferences routinely process submissions at the scale of $m \sim 10^4$~\cite{stanford_ai_index}. Given this regime, our algorithm guarantees efficient computation, enabling fairness-aware desk rejection within tractable timeframes, even for large-scale conferences.
\end{remark}

\subsection{Analysis}


\begin{table}[h]
    \centering
    \small
    \begin{tabular}{lccc}
        \toprule
        Score & cos $\theta$ &\# of Generated EX & \%  Filtered EX \\
        \midrule
        \textbf{$\geq 0$} &0.581& 10340 & 0 \% \\ 
        \textbf{$\geq 2$} &0.625& 10185  & 1.50\% (-155) \\
        \textbf{$\geq 4$} &0.744& 9883 & 4.41\% (-457)  \\
        \textbf{$\geq 6$} &0.762 & 9378 & 9.30\% (-962)  \\
        \textbf{$\geq 8$}&0.765& 8606 & 16.76\% (-1734)\\
        \textbf{$\geq 10$} &0.769& 6795 & 34.28\% (-3545)  \\
        \bottomrule
    \end{tabular}
    %\caption{A summary of the data generation and filtering result, along with an embedding similarity analysis of the filtered examples, categorized by their respective scores.}
\caption{Summary of data generation, filtering results, and embedding similarity analysis by score.}
    \label{tab:number_of_generated}
    % \vspace{-4mm}
\end{table}

% & 0.581          & 0.625            &  0.744         & 0.762          & 0.765    &  \textbf{0.769}  

\begin{figure*}[t]
\centerline{\includegraphics[scale=0.48]{Pictures/corr_bin.pdf}}

\caption{(Left) Correlation between question embedding similarity and average EX, (Right) Average EX across embedding similarity bins}
% \vspace{-4mm}
\label{fig:corr_bin}
\end{figure*}

\paragraph{Number of generated and filtered examples per score, along with an embedding similarity analysis of the filtered examples}
For each test question in the Spider dev set, 10 examples are generated, resulting in a total of 10,340 examples. The quality of these examples is assessed using a relevance score ranging from 0 to 10. As shown in Table~\ref{tab:number_of_generated}, the 65.71\% of examples are assigned a score of 10, while the 0.59\% of examples are received a score of 0. This trend suggests that the LLM tends to assign high relevance to its own generated examples. The similarity is computed using cosine similarity, where higher scores indicate greater semantic alignment between the test questions and the retained examples. As the filtering threshold increases, the embedding similarity also increases, suggesting that higher-relevance examples exhibit stronger semantic consistency with the test questions. However, we also observe that overly strict filtering—selecting only examples with a perfect score of 10—leads to a decline in performance. This drop occurs because an excessively high threshold significantly reduces the number of available examples, limiting the diversity.


\paragraph{Effect of question embedding similarity on Execution Accuracy.}
In Figure~\ref{fig:corr_bin}, the left graph illustrates the correlation between embedding similarity and EX. Each point represents one of the 11 data points obtained by filtering examples based on different threshold scores (0 to 10). The data points follow an upward trend, suggesting that higher similarity tends to result in better EX. The red line indicates the overall correlation, with a coefficient of 0.82, showing a relatively strong positive relationship. Building on this analysis, the right graph provides a more fine-grained view by examining the execution accuracy of individual generated examples based on their embedding similarity with test questions. The x-axis represents the normalized similarity between the test question and the generated question, and the y-axis indicates EX. The results show that EX is lowest in the 0.0-0.1 similarity range, suggesting that examples with very low similarity to test questions tend to be less useful. As similarity increases, EX generally improves, peaking in the 0.7-0.8 range. This suggests that examples with a moderate to high similarity to test questions are more effective in generating executable SQL queries. However, accuracy drops slightly in the 0.8-0.9 range before rising again in the 0.9-1.0 range. This indicates that excessively high similarity can reduce diversity, potentially limiting the model’s generalization ability. 


\begin{figure}[t]
\centerline{\includegraphics[scale=0.36]{Pictures/Diff_threshold_GPT4o.png}}
\caption{Performance of GPT-4o at different relevance score thresholds.}
% \vspace{-5mm}
\label{tab:diff_thres}
\end{figure}


\paragraph{Effect of Relevance Scoring Thresholds on Performance.}

To further evaluate the effectiveness of SAFE-SQL, we conduct a detailed case study using varying thresholds for the relevance scoring mechanism as shown in Figure~\ref{tab:diff_thres}.  The self-generated examples are filtered based on relevance scores, with thresholds ranging from 0 to 10. For each test question, the number of high-scoring examples varied due to the specific content and schema structure (e.g., some test questions had six examples with scores $\geq 8$, while others had three). The selected examples are then used during the final inference stage to generate SQL queries. The $\geq 8$ threshold consistently produced the best results, validating the robustness of SAFE-SQL’s relevance score filtering. The results demonstrate that selecting high-quality examples plays a critical role in guiding LLMs to generate accurate SQL queries, regardless of the underlying model.


\begin{comment}
\begin{table*}[h]
    \centering
    \renewcommand{\arraystretch}{1.3}  % 행 간격 조정
    \begin{tabularx}{\textwidth}{p{4cm} p{6cm} p{4cm} p{6cm}}
        \toprule
        \textbf{Original SQL Question} & \textbf{Original SQL Query} & \textbf{Generated SQL Question} & \textbf{Generated Reasoning Path} \\
        \midrule
        What are all the flights that leave from Aberdeen? & 
        \lstinline|SELECT * FROM flights WHERE departure_city = 'Aberdeen'| & 
        What are all the flights departing from Aberdeen? & 
        Identify all flights with Aberdeen as the departure city. \\
        
        Of those, which land in Ashley? & 
        \lstinline|SELECT * FROM flights WHERE departure_city = 'Aberdeen' AND arrival_city = 'Ashley'| & 
        Which flights leave from Aberdeen and land in Ashley? & 
        Filter previous results to include only flights arriving in Ashley. \\
        
        How many are there? & 
        \lstinline|SELECT COUNT(*) FROM flights WHERE departure_city = 'Aberdeen' AND arrival_city = 'Ashley'| & 
        How many flights travel from Aberdeen to Ashley? & 
        Count the number of flights from the filtered list. \\
        \midrule
        
        What are all the airlines? & 
        \lstinline|SELECT DISTINCT airline FROM flights| & 
        What airlines operate flights? & 
        Retrieve distinct airline names from the flights table. \\
        
        Of these, which is JetBlue Airways? & 
        \lstinline|SELECT * FROM flights WHERE airline = 'JetBlue Airways'| & 
        Which flights are operated by JetBlue Airways? & 
        Filter flights to include only those operated by JetBlue Airways. \\
        
        What is the country corresponding it? & 
        \lstinline|SELECT country FROM airlines WHERE name = 'JetBlue Airways'| & 
        What country is JetBlue Airways based in? & 
        Retrieve the country associated with JetBlue Airways from the airlines table. \\
        \bottomrule
    \end{tabularx}
    \caption{Examples of original and generated SQL questions with reasoning paths.}
    \label{tab:sql_examples}
\end{table*}
\end{comment}

\begin{comment}
\begin{table*}[h]
    \centering
    \small
    \renewcommand{\arraystretch}{1.3}  % Adjust row spacing
    \begin{tabularx}{\textwidth}{X X X X X}
        \toprule
        \textbf{SQL Question} & \textbf{GOLD SQL Query} & \textbf{Augmented SQL Question} & \textbf{Generated Reasoning Path} & \textbf{Relevance Score} \\
        \midrule
        \hl{Question1:}
        What are the names, countries, and ages for every singer in descending order of age? & 
        \texttt{SELECT name, country, age FROM singer ORDER BY age DESC} & 
        \sethlcolor{lime!50}
        \hl{What are the names, ages, and countries of all singers from a specific country, sorted by age in descending order?} & 
        \sethlcolor{violet!20}
        \hl{1.Identify the desired columns: name, age, and country. 
        2.Specify the table: singer. 
        3.Sort the results by age in descending order.}& semantic similarity:3   Structure \& key word 
 score: 3  Reasoning patt score:4 Relevance score = 10
        \\
        \midrule
        \hl{Question2:}
        What is the number of car models that are produced by each maker and what is the id and full name of each maker?
        &  
        \texttt{SELECT Count(*), T2.FullName , T2.id FROM MODEL\_LIST AS T1 JOIN CAR\_MAKERS AS T2 ON T1.Maker = T2.id GROUP BY T2.id;} & 
               \sethlcolor{lime!50}
 \hl{Could you provide the count of car models produced by each manufacturer, along with the ID and full name of each manufacturer?} & 
 \sethlcolor{violet!20}
 \hl{1.Retrieve Required Information: Count car models per maker and get each maker's ID and full name. 2.Join Tables: Link MODEL\_LIST (T1) and CAR\_MAKERS (T2) using T1.Maker = T2.Id. 3.Group and Aggregate: Use COUNT(*) to count models and group by T2.id. 4.Select Output: Return the model count, maker’s full name, and ID.} & semantic similarity:1   Structure \& key word 
 score: 2  Reasoning patt score:3 Relevance score = 6 \\ 
        \midrule
        \hl{Question3:} Return the names and template ids for documents that contain the letter w in their description. & 
        \texttt{SELECT document\_name , template\_id FROM Documents WHERE Document\_Description LIKE "\%w\%"} & 
        \sethlcolor{lime!50}
        \hl{Retrieve the names and template IDs of documents whose descriptions include the letter 'w'.} & 
        \sethlcolor{violet!20}
        \hl{1.Extract the document\_name and template\_id columns.
        2.Search within the Documents table. 3.Filter results where Document\_Description contains the letter "w".
        Return the matching records.} & semantic similarity:3   Structure \& key word 
 score: 2  Reasoning path score:3 Relevance score = 8  \\

        \bottomrule
    \end{tabularx}
    \caption{Examples of original and Augmented SQL questions with reasoning paths by GPT-4o mini.}
    \label{tab:sql_examples}
\end{table*}
\end{comment}

\begin{table*}[t]
    \centering
    \scriptsize
    \renewcommand{\arraystretch}{1.2}
    \resizebox{1.97\columnwidth}{!}{% Adjust row spacing
    \begin{tabularx}{\textwidth}{X X X >{\raggedright\arraybackslash}p{0.3\textwidth} X}
        \toprule
        \textbf{SQL Question} & \textbf{GOLD SQL Query} & \textbf{Augmented SQL Question} & \textbf{Generated Reasoning Path} & \textbf{Relevance Score} \\
        \midrule
        \hl{Question1:} What are the names, countries, and ages for every singer in descending order of age? & 
        \texttt{SELECT name, country, age FROM singer ORDER BY age DESC} & 
        \sethlcolor{lime!50}\hl{What are the names, ages, and countries of all singers from a specific country, sorted by age in descending order?} & 
        \sethlcolor{violet!20}\hl{1. Identify the desired columns: name, age, and country. \\
        2. Specify the table: singer. \\
        3. Sort the results by age in descending order.} &  
        Semantic similarity = 10 
        Structural Similarity = 10  
        Reasoning path = 10  
        Relevance score = (10+10+10)/3 = 10
                \\ 
        \midrule
        \hl{Question2:} Return the names and template ids for documents that contain the letter w in their description. & 
        \texttt{SELECT document\_name, template\_id FROM Documents WHERE Document\_Description LIKE "\%w\%"} & 
        \sethlcolor{lime!50}\hl{Retrieve the titles and category IDs of articles whose summaries contain the word "data".} &
        \sethlcolor{violet!20}\hl{1. Identify the necessary columns: extract title and category\_id from the Articles table. \\
         2. Locate the relevant table: use the Articles table as it contains the required data. \\
        3. Define the filtering condition: apply a WHERE clause to check if the summary column contains the substring "data". \\
         4. Use the LIKE operator: employ LIKE '\%data\%' to search for any instance of "data" within the summary. \\
        5. Retrieve the results: return the title and category\_id values for all matching records.} & Semantic similarity = 7
                    Structural Similarity = 9
Reasoning path = 8
Relevance score = (7+9+8)/3 = 8
        \\
        \midrule
        \hl{Question3:} What is the number of car models that are produced by each maker and what is the id and full name of each maker? &  
        \texttt{SELECT Count(*), T2.FullName, T2.id FROM MODEL\_LIST AS T1 JOIN CAR\_MAKERS AS T2 ON T1.Maker = T2.id GROUP BY T2.id;} & 
        \sethlcolor{lime!50}\hl{List all employees who work in the IT department along with their employee ID and hire date.} & 
        \sethlcolor{violet!20}\hl{1. Identify required details: employee ID and hire date. \\
        2. Filter condition: find employees who work in IT. \\
        3. Retrieve data: select only emp\_id and hire\_date.} & 
        Semantic similarity = 6
        Structural Similarity = 3
        Reasoning path = 2
        Relevance score = (6+3+2)/3 = 3.67
  \\
        \bottomrule
    \end{tabularx}
    }
    %\vspace{-2mm}
    \caption{Examples of original and augmented SQL questions with reasoning paths by GPT-4o.}
    \label{tab:sql_examples}
    \vspace{-4mm}
\end{table*}


%This experiment is performed across multiple models, including GPT 4o Mini, Deepseek Coder 6.7B, %Llama 3.1 8B, and Starcoder 7B.
\begin{table}[t]
    \centering
    \small
    \resizebox{0.48\textwidth}{!}{
    \begin{tabular}{lcc||ccccc}
        \toprule
        \textbf{$\alpha$} & \textbf{$\beta$} &\textbf{$\gamma$}& \textbf{Easy}& \textbf{Medium}& \textbf{Hard} &\textbf{Extra}& \textbf{EX} \\
        \midrule
        0.33 & 0.33 & 0.33 & \textbf{93.4} & \textbf{89.3} & \underline{88.4} & \textbf{75.8} & \textbf{87.9} \\ 
        \midrule
        1 & 0 & 0 & 90.7& 84.2& 82.3& 68.3&  82.8 \\ 
        0 & 1 & 0 & 89.8& 85.6& 81.2& 69.2&  83.1 \\ 
        0 & 0 & 1 & 89.2& 85.1& 84.3& 71.7& 83.7  \\ 
        \midrule
        0.5 & 0.5 & 0& 91.2& 87.3& 82.5& 69.4& 84.4 \\ 
        0.5 & 0 & 0.5& 92.5& \underline{87.9}& 83.5& 70.3& 85.3 \\ 
        0 & 0.5 & 0.5& \underline{92.7}& 86.8& \textbf{88.5}& \underline{72.4}& \underline{86.1} \\ 
        \bottomrule
    \end{tabular}
    }
    %\vspace{-2mm}
    %\caption{Execution accuracy across different difficulty levels with varying weights of semantic similarity ($\alpha$), keyword \& structural similarity ($\beta$), and reasoning path quality ($\gamma$).}
    \caption{Execution accuracy across difficulty levels under different weights: semantic similarity ($\alpha$), Structural similarity ($\beta$), and reasoning path quality ($\gamma$).}
    % \vspace{-4mm}
    \label{tab:filtering_score_ablation}
\end{table} 

\paragraph{Effect of three measuring components on Performance.}

To assess the impact of the three measuring components—semantic similarity ($\alpha$), keyword \& structural similarity ($\beta$), and reasoning path quality ($\gamma$)—on EX, we conduct experiments by varying their respective weightings. The results, presented in Table~\ref{tab:filtering_score_ablation}, highlight distinct performance trends across different difficulty levels. Notably, the exclusion of reasoning path quality leads to a drop in EX, particularly in the Hard and Extra Hard. This suggests that a well-structured reasoning path is crucial for handling complex queries, as it provides essential logical steps that bridge the gap between natural language understanding and SQL formulation. Conversely, semantic similarity and structural SQL query similarity have a greater influence on the Easy and Medium levels. This is because these queries tend to be relatively straightforward, meaning that having structurally similar SQL questions in the example set often provides sufficient guidance for generating correct queries. In these cases, direct pattern matching and schema alignment play a larger role. Overall, the findings demonstrate that a balanced combination of all three components is essential for optimizing performance across different levels of query complexity.

% Simply maximizing similarity may not always yield the best results, and a balanced approach that considers both relevance and diversity could be more effective.


%: Qwen2.5-3B-instruct, Qwen2.5-7B-instruct, and Qwen2.5-14B-instruct 


\subsection{Case Study}
As shown in Table~\ref{tab:sql_examples}, test questions from the Spider dev set alongside their generated similar examples, evaluated based on semantic similarity, structural similarity, and the reasoning path score, which together determine the relevance score. The first example achieves a perfect relevance score of 10, as the generated question closely aligns with the original in meaning, structure, and reasoning. The SQL formulation remains nearly identical, and the reasoning path explicitly details each step, ensuring full alignment. The second example receives a relevance score of 8, with semantic similarity of 7 due to minor differences in terminology ("documents" vs. "articles" and "letter 'w'" vs. "word 'data'"). However, its structural similarity remains high, as the SQL structure is nearly identical. The reasoning path score of 8 reflects a clear explanation of query formulation, though slightly less detailed than the first example. The third example has the lowest relevance score due to significant differences. The generated question shifts focus from counting car models to listing IT employees, resulting in semantic similarity of 6 and structural similarity of 3. These results emphasize the importance of fine-grained example selection due to the varing quality of generated examples.
% Table 6번 언급되는 곳이 하나도 없었습니다. 맨뒤로 빼고 Case Study 만들어서 설명할 필요가 있습니다. 또한 Relevance Score 변경했는데 확인해주셔야합니다
\section{Conclusion and future directions} \label{sec:conclusion}

In this paper we proposed a nested MLMC framework that offers important computational savings by performing most calculations in low precision and exploiting approximate random normal variables for the low precision path calculations. The low precision calculations could be performed in fixed precision on an FPGA for greater efficiency, and we suggested a procedure to optimise the bit-widths of every variable at each Monte Carlo level. This is an important improvement over previous mixed precision MLMC frameworks which held the lower precision fixed \cite{Rounding_error_oliver} or defined uniform bit-width at every level heuristically \cite{brugger2014mixed}. Our numerical results suggest that for the first levels our procedure reduces the cost at these levels by a factor 5 or 7. Hence the overall savings are significant since most paths are calculated on the first levels. Our approach would be even more efficient for the Milstein scheme because its higher order strong convergence leads to a greater proportion of the computational costs being on the coarsest levels.

The next stage of the research project will be to implement the RNG methods and the nested framework on FPGAs to determine the hardware requirements and confirm the extent of the computational savings. It would also be good to compare the performance benefits to using half-precision floating point arithmetic on GPUs or CPUs for the low-accuracy computations.




\ifdefined\isarxiv
%\section*{Acknowledgments}
\bibliographystyle{alpha}
\bibliography{ref} 
\else
\bibliography{ref}
\bibliographystyle{icml2025}
% \bibliographystyle{alpg mha}

\fi



\newpage
\onecolumn
\appendix




%%%% Cut-line between first 10 pages and appendix
%\input{11_backup_for_pos_results}
\clearpage
\newpage
\begin{center}
    \textbf{\LARGE Appendix}
\end{center}

\paragraph{Roadmap.} In Section~\ref{sec:dilemma_proof}, we supplement the missing proofs in Section~\ref{sec:dr_dilemma}. In Section~\ref{sec:fair_proof}, we present the missing proofs in Section~\ref{sec:fair}. In Section~\ref{sec:more_case_study}, we show additional case studies. In Section~\ref{app:sec:conference_links}, we provide the details related to conference submission limits. 

\section{Missing Proofs in Section 4} \label{sec:dilemma_proof}

In this section, we provide the complete technical proofs for Theorem~\ref{thm:main_res_general} in Section~\ref{sec:dr_dilemma}. 
In Section~\ref{sec:dilemma_proof_defs}, we first introduce key definitions that will be useful 
To structure our analysis, we in the subsequent proofs. We then establish positive results for the cases where $n \leq 2$ in Section~\ref{sec:positive_results}, followed by negative results for $n \geq 3$ in Section~\ref{sec:negative_results}.


\subsection{Basic Definitions}\label{sec:dilemma_proof_defs}

To systematically analyze the desk-rejection problem, we begin by classifying authors based on their submission behavior and their relationship to co-authors. This classification will help us organize and present the proofs in a more structured and readable manner.

\begin{definition}[Author Categories]\label{def:three_kinds_authors}
For any author $a_i \in \mathcal{A}$, we define the following categories:
\begin{itemize}
    \item \textbf{Non-compliant}: An author $a_i$ is non-compliant if they have submitted more than $x$ papers, i.e., $|P_i| > x$. Such authors exceed the submission limit and are subject to desk-rejection under the policy.

    \item \textbf{Vulnerable}: An author $a_i$ is vulnerable if they have submitted no more than $x$ papers ($|P_i| \leq x$) but have at least one non-compliant co-author, i.e., $\exists k \in C_i$ such that $|P_k| > x$. Although these authors comply with the submission limit, they are at risk of being unfairly penalized due to their co-authors' non-compliance.

    \item \textbf{Safe}: An author $a_i$ is safe if they have submitted no more than $x$ papers ($|P_i| \leq x$) and all their co-authors are also compliant, i.e., $\forall k \in C_i$, $|P_k| \leq x$. These authors are guaranteed to retain all their submissions, as neither they nor their co-authors violate the submission limit.
\end{itemize}
\end{definition}

Next, we formalize the notion of achievability for the ideal desk-rejection system.

\begin{definition}[Achievability]\label{def:achievability}
Given a submission limit problem instance as defined in Definition~\ref{def:submit_limit_problem}:
\begin{itemize}
    \item \textbf{Positive result}: A problem instance is a positive result if there exists an algorithm that can achieve the ideal desk-rejection as defined in Definition~\ref{def:good_solution}.
    
    \item \textbf{Negative result}: A problem instance is a negative result if,  under proper conditions, no algorithm can achieve the ideal desk-rejection as defined in Definition~\ref{def:good_solution}.
\end{itemize}
\end{definition}

In the following sections, we will use these definitions to systematically prove the positive results for small numbers of authors ($n \leq 2$) and the negative results for larger numbers of authors ($n \geq 3$), which covers two cases in Theorem~\ref{thm:main_res_general}.


\subsection{Positive Results} \label{sec:positive_results}
In this subsection, we present two positive results that support the $n \leq 2$ case in Theorem~\ref{thm:main_res_general}. We begin with the positive result for $n = 1$ and any $x \in \mathbb{N}_+$.

\begin{lemma}[Positive result for $n = 1$ and any $x \in \mathbb{N}_+$, general case]\label{lem:n_eq_1_positive_general}
    If the following conditions hold:
    \begin{itemize}
        \item Let $n = 1$ denote the number of authors as defined in Definition~\ref{def:submit_limit_problem}.
        \item Let $x \in \mathbb{N}_+$ denote the maximum number of submissions allowed for each author in the conference.
    \end{itemize}
    Then, there exists an algorithm that achieves the ideal desk-rejection as defined in Definition~\ref{def:good_solution}.
\end{lemma}

\begin{proof}
We consider the three cases for the only author $a_1$: non-compliant, vulnerable, and safe, as defined in Definition~\ref{def:three_kinds_authors}.

\paragraph{Case 1: Non-compliant author.} If author $a_1$ is non-compliant, we desk-reject $(|P_1| - x)$ papers. This ensures that exactly $x$ papers remain, satisfying the ideal desk-rejection condition.

\paragraph{Case 2: Vulnerable author.} Since $n = 1$ and there is only one author, author $a_1$ has no co-authors to make itself vulnerable. Therefore, this case cannot happen.

\paragraph{Case 3: Safe author.} If author $a_1$ is safe, no papers need to be rejected. The ideal desk-rejection condition is trivially satisfied.

In all possible cases, we can achieve the ideal desk-rejection. Thus, the proof is finished. 
\end{proof}

To present the positive result for $n=2$ and any $x \in \mathbb{N}_+$, we first discuss a specific case where all authors are non-compliant.

\begin{lemma} [Positive result for $n=2$ and any $x\in\mathbb{N}_+$, non-compliant author only case] \label{lem:n_eq_2_positive}
If the following conditions hold:
\begin{itemize}
    \item Let $n = 1$ denote the number of authors as defined in Definition~\ref{def:submit_limit_problem}.
    \item All the authors are non-compliant authors as defined in Definition~\ref{def:three_kinds_authors}.
    \item Let $x \in \mathbb{N}_+$ denote the maximum number of submissions allowed for each author in the conference. 
\end{itemize}

Then, there exists an algorithm that achieves the ideal desk-rejection as defined in Definition~\ref{def:good_solution}.
\end{lemma}

\begin{proof}
Let $c \in \mathbb{N}$ denote the number of papers co-authored by both author $a_1$ and author $a_2$. For $i \in \{1, 2\}$, let $b_i \in \mathbb{N}$ denote the number of single-authored papers by author $a_i$. 

We then have:
\begin{align*}
    b_1 + c = |P_1|
\end{align*}
and
\begin{align*}
    b_2 + c = |P_2|.
\end{align*}

\paragraph{Case 1: $c \leq x$.} 
In this case, we have $b_1 \geq |P_1| - x$ and $b_2 \geq |P_2| - x$. Since $b_i$ represents the number of single-authored papers by author $a_i$, we can desk reject exactly $(|P_i| - x)$ papers from author $a_i$.

\paragraph{Case 2: $c > x$.} 
Here, we have $b_1 < |P_1| - x$ and $b_2 < |P_2| - x$. We first desk reject all $b_1$ single-authored papers from author $a_1$ and all $b_2$ single-authored papers from author $a_2$. Next, we desk reject $(c - x)$ co-authored papers from both authors. This ensures that the remaining $x$ papers are co-authored by both $a_1$ and $a_2$. Thus, we have successfully rejected exactly $(|P_i| - x)$ papers from each author $a_i$.

By combining the two cases above, the proof is complete.
\end{proof}

With the help of Lemma~\ref{lem:n_eq_2_positive}, we now establish the positive result for $n=2$ and any $x \in \mathbb{N}_+$.

\begin{lemma} [Positive result for $n=2$ and any $x\in\mathbb{N}_+$, general case] \label{lem:n_eq_2_positive_general}
If the following conditions hold:
\begin{itemize}
    \item Let $n = 1$ denote the number of authors as defined in Definition~\ref{def:submit_limit_problem}.
    \item Let $x \in \mathbb{N}_+$ denote the maximum number of submissions allowed for each author in the conference. 
\end{itemize}

Then, there exists an algorithm that achieves the ideal desk-rejection as defined in Definition~\ref{def:good_solution}.
\end{lemma}

\begin{proof}
We consider two authors, $a_1$ and $a_2$. Without loss of generality, we assume that $a_1$ has at least as many papers as $a_2$, i.e., $|P_1| \geq |P_2|$. By exhaustively enumerating all possible compositions of author types (i.e., non-compliant, vulnerable, or safe) for $a_1$ and $a_2$, we observe that the vulnerable-safe composition is impossible. This is because a vulnerable author must co-author at least one paper with a non-compliant author. After excluding this case, we analyze the remaining possible scenarios as follows:

\paragraph{Case 1: Both $a_1$ and $a_2$ are safe authors.} 
In this case, no papers need to be rejected, and the ideal desk-rejection trivially holds.

\paragraph{Case 2: $a_1$ is a non-compliant author and $a_2$ is a safe author.} 
Since rejecting papers from $a_1$ does not affect $a_2$'s submissions, we can simply reject $(|P_1| - x)$ papers from $a_1$ to achieve the ideal desk-rejection.

\paragraph{Case 3: $a_1$ is a non-compliant author and $a_2$ is a vulnerable author.} 
By Definition~\ref{def:three_kinds_authors}, we have $|P_1| > x$ and $|P_2| \le x$. Let $c := |\{p_j \in S: p_j \in P_1, p_j \in P_2\}|$ denote the number of co-authored papers by $a_1$ and $a_2$. From basic set theory, we know that $c \leq |P_2|$. Since $|P_2| \le x$, it follows that $c \le x$. Therefore, we have:
\begin{align*}
    \underbrace{|P_1| - c}_{\text{Individual papers of } a_1} \ge \underbrace{|P_1| - x}_{\text{Excess papers of } a_1},
\end{align*}
which implies that the number of individual papers authored solely by $a_1$ exceeds the number of over-limit papers for $a_1$. Thus, we can first reject $a_1$'s individual papers without affecting $a_2$'s submissions, thereby achieving the desired ideal desk-rejection.

\paragraph{Case 4: Both $a_1$ and $a_2$ are non-compliant authors.} 
This case directly follows from Lemma~\ref{lem:n_eq_2_positive}.

Combining all the cases above, we conclude that the ideal desk-rejection can always be achieved, which finishes the proof.
\end{proof}

\subsection{Negative Results} \label{sec:negative_results}
In this subsection, we present two positive results that support the $n \ge 3$ case in Theorem~\ref{thm:main_res_general}. We commence by showing the negative result for $n = 3$ and $x=1$.

\begin{lemma}[Negative result for $n=3$ and $x=1$] \label{lem:n_eq_3_negative}
If the following conditions hold:
\begin{itemize}
    \item Let $n = 3$ denote the number of authors as defined in Definition~\ref{def:submit_limit_problem}.
    \item Let $x=1$ denote the maximum number of submissions allowed for each author in the conference. 
\end{itemize}

Then, under proper conditions, no algorithm can achieve the ideal desk-rejection as defined
in Definition~\ref{def:good_solution}.
\end{lemma}

\begin{proof}
Let all the authors be non-compliant authors as defined in Definition~\ref{def:three_kinds_authors}, and let the number of papers be $m=3$. We suppose the three papers $p_1$, $p_2$, and $p_3$ have the following authorship:

\begin{itemize}
    \item Paper $p_1$ is co-authored by $a_1$ and $a_2$.
    \item Paper $p_2$ is co-authored by $a_1$ and $a_3$.
    \item Paper $p_3$ is co-authored by $a_2$ and $a_3$.
\end{itemize}

From the authors' perspective, the relationships are as follows:
\begin{itemize}
    \item Author $a_1$ has papers $p_1$ and $p_2$.
    \item Author $a_2$ has papers $p_1$ and $p_3$.
    \item Author $a_3$ has papers $p_2$ and $p_3$.
\end{itemize}

We enumerate all possible rejection plans and their outcomes in Table~\ref{tab:all_possible_rejections}.

\begin{table}[!ht]
\caption{Remaining number of papers for each author after desk rejection.}
\label{tab:all_possible_rejections}
\begin{center}
\begin{tabular}{|c|c|c|c|}
 \hline
 Rejected Papers & Author $a_1$ & Author $a_2$ & Author $a_3$ \\ \hline
 N/A             & 2            & 2            & 2            \\ \hline
 $p_1$           & 1            & 1            & 2            \\ \hline
 $p_2$           & 1            & 2            & 1            \\ \hline
 $p_3$           & 2            & 1            & 1            \\ \hline
 $p_1, p_2$      & 0            & 1            & 1            \\ \hline
 $p_1, p_3$      & 1            & 0            & 1            \\ \hline
 $p_2, p_3$      & 1            & 1            & 0            \\ \hline
 $p_1, p_2, p_3$ & 0            & 0            & 0            \\ \hline
\end{tabular}
\end{center}
\end{table}

First, suppose we desk reject paper $p_3$. Then, authors $a_2$ and $a_3$ each have one paper remaining, but author $a_1$ still has two papers. To satisfy the constraint $x=1$, we must reject one of $p_1$ or $p_2$.

If we reject $p_1$, author $a_2$ is left with no papers, which is unfair. If we reject $p_2$, author $a_3$ is left with no papers, which is also unfair.

Thus, no rejection plan satisfies the ideal desk rejection condition for all authors. This completes the proof.
\end{proof}

Next, we present the negative result for any $n \geq 3$ and $x = n-2$.

\begin{lemma}[Negative result for any $n \geq 3$ and $x = n-2$] \label{lem:n_geq_3_negative}
If the following conditions hold:
\begin{itemize}
    \item Let $n \ge 3$ denote the number of authors as defined in Definition~\ref{def:submit_limit_problem}.
    \item Let $x=n-2$ denote the maximum number of submissions allowed for each author in the conference. 
\end{itemize}

Then, under proper conditions, no algorithm can achieve the ideal desk-rejection as defined
in Definition~\ref{def:good_solution}.
\end{lemma}

\begin{proof}
In this negative problem instance, we choose the number of papers to be the same as the number of authors, i.e., $m=n$, and we assume all the $n$ authors are non-compliant authors as defined in Definition~\ref{def:three_kinds_authors}. 


For each of the $n$ papers $p_i \in \mathcal{P}$, we let $i$-th paper $p_i$ contain $n-1$ authors, excluding only the $i$-th author $a_i$. Specifically, we have:  
\begin{itemize}
    \item The first paper $p_1$ has authors $a_2, a_3, \cdots, a_n$. 
    \item The second paper $p_2$ has authors $a_1, a_3, a_4, \cdots, a_n$.
    \item $\cdots\cdots$
    \item The $(n-1)$-th paper has authors $a_1, a_2, \cdots , a_{n-2}, a_n$.
    \item The $n$-th paper has authors $a_1, a_2, \cdots , a_{n-2}, a_{n-1}$.
\end{itemize}

Since each author is allowed to submit at most $x=n-2$ papers, we must desk-reject at least two papers. We analyze the process of desk-rejecting these two papers step by step.


\textbf{Step 1: Desk-reject the first paper.}

Without loss of generality, we consider rejecting paper $p_1$ first. After this operation, authors $a_2, a_3, \cdots a_n$, will have $n-2$ submitted papers, while author $a_1$ will have $n-1$ submitted papers. 


\textbf{Step 2: Desk-reject the second paper.}

Without loss of generality, we consider rejecting paper $p_2$ next. After this operation, authors $a_3, a_4, \cdots a_n$, will have $n-3$ submitted papers, while author $a_1$ and $a_2$ will have $n-2$ submitted papers. 

At this point, it is impossible for authors $a_3, a_4, a_5 \cdots , a_n$ to have exactly $(n-2)$ submitted papers. Therefore, no algorithm can achieve the ideal desk-rejection under the given conditions. This completes the proof.
\end{proof}


\section{Missing Proofs in Section 5}\label{sec:fair_proof}
In this section, we first present the missing proofs for fairness metrics in Section~\ref{sec:fair_metric_append}, and then present the supplementary proofs for the hardness of individual fairness optimization in Section~\ref{sec:indi_fair_hard_append}. Finally, we show the additional proofs for the group fairness optimization problem in Section~\ref{sec:fair_optim_append}.

\subsection{Fairness Metrics}\label{sec:fair_metric_append}
We present the relationship between the fairness metrics.  

\begin{proposition}[Relationship of Fairness Metrics, formal version of Proposition~\ref{lem:fair_metric_ineq} in Section~\ref{sec:fair_metric}]\label{lem:fair_metric_ineq_append}
    For any solution $S\subseteq \mathcal{P}$ for the submission limit problem in Definition~\ref{def:submit_limit_problem}, we have 
    \begin{align*}
        \zeta_{\mathrm{group}}(S) \leq \zeta_{\mathrm{ind}}(S).
    \end{align*}
\end{proposition}
\begin{proof}
    By Definition~\ref{def:group_fair}, we have:
    \begin{align*}
        \zeta_{\mathrm{group}}(S) &=~ \frac{1}{n}\sum_{i \in [n]} c(a_i,S) \\
        &\leq~ \frac{1}{n}\sum_{i \in [n]} \max_{i\in[n]}c(a_i, S) \\ 
        &=~\frac{1}{n}\cdot n \cdot \max_{i\in[n]}c(a_i, S) \\ 
        &=~ \zeta_{\mathrm{ind}}(S),
    \end{align*}
where the first equality directly follows from Definition~\ref{def:group_fair}, the second and the third inequality follow from basic algebra, and the last equality follows from Definition~\ref{def:individual_fair}. Thus, we complete the proof.
\end{proof}

\subsection{Hardness of Individual Fairness-Aware Submission Limit Problem}\label{sec:indi_fair_hard_append}

Before proving the theoretical results in Section~\ref{sec:indi_fair_hard}, we first introduce a useful fact that serves as a foundation for the subsequent proofs.   

\begin{fact}\label{fact:author_paper_count}
For each author $a_i\in\mathcal{A}$, the number of papers after desk-rejection (i.e., $|\{p_j \in  S : a_i \in A_j\}|$) can be written as $W_i^\top r$.
\end{fact}
\begin{proof}
    This simply follows from:
    \begin{align*}
        W^\top_ir &=~ \sum_{j\in[m]} W_{i,j}\cdot r_j \\ 
        &=~ |\{j\in[m]:W_{i, j}=1,r_j=1\}| \\ 
        &=~ |\{p_j\in\mathcal{P}:a_i\in A_j, p_j\in S\}| \\ 
        &=~|\{p_j \in  S : a_i \in A_j\}|,
    \end{align*}
where the first and the second equality follow from basic algebra and set theory, and the third and the fourth equality follow from Definition~\ref{def:submit_limit_problem}.  
\end{proof}

With the help of the aforementioned fact, we now prove the equivalence of the matrix form for the individual fairness problem.


\begin{proposition}[Matrix Form Equivalence for $\zeta_{\mathrm{ind}}$, formal version of Proposition~\ref{prop:equiv_individual} in Section~\ref{sec:indi_fair_hard}]\label{prop:equiv_individual_append}
    The individual fairness-aware submission limit problem in Definition~\ref{def:ind_fair_min} and the matrix form integer programming problem in Definition~\ref{def:ind_fair_min_matrix} are equivalent.
\end{proposition}
\begin{proof}
    In Definition~\ref{def:ind_fair_min}, the paper set $\mathcal{P}$ consists of $m$ papers, each of which can either be maintained or desk-rejected. Thus, the subset of maintained papers, $\mathcal{S}$, can be represented by a 0-1 vector $r \in \{0, 1\}^m$, where $r_j = 1$ indicates that paper $p_j$ is maintained, and $r_j = 0$ indicates that it is desk-rejected. We now establish the equivalence of both the objective function and the constraints in these two formulations.


    \paragraph{Part 1: Optimization Objective.} We first consider the objective function $\mathbf{1}^\top_nD^{-1}Wr$ in Definition~\ref{def:ind_fair_min_matrix}:
    \begin{align*}
        \min_{r \in \{0,1\}^m} \| \mathbf{1}_n - D^{-1}Wr\|_\infty &=~ \min_{r \in \{0,1\}^m} \max_{i\in[n]} (1 - (D^{-1}Wr)_i) \\ 
        &=~ \min_{r \in \{0,1\}^m} \max_{i\in[n]} (1 - (W_i^\top r)_i/D_{i,i}) \\
        &=~ \min_{r \in \{0,1\}^m} \max_{i\in[n]} (1 - (W_i^\top r)_i/|P_i|) \\ 
        &=~ \min_{r \in \{0,1\}^m} \max_{i\in[n]} (1 - |\{p_j \in  S : a_i \in A_j\}|/|P_i|) \\ 
        &=~ \min_{r \in \{0,1\}^m} \max_{i\in[n]}c(a_i, S) \\ 
        &=~ \min_{r \in \{0,1\}^m} \zeta_{\mathrm{ind}}(S),
    \end{align*}
where the first equality follows from the definition of infinity norm, the second equality follows from basic algebra, the third equality follows from Definition~\ref{def:ind_fair_min_matrix}, the fourth equality follows from Fact~\ref{fact:author_paper_count}, the fifth equality follows from Definition~\ref{def:cost}, and the last equality follows from Definition~\ref{def:individual_fair}. By decoding $r$ back into the paper subset $S$, we recover the original optimization objective in Definition~\ref{def:ind_fair_min}.

\paragraph{Part 2: Constraints.} The constraint in Definition~\ref{def:ind_fair_min_matrix} can be rewritten using basic algebra as:
     \begin{align*}
         W_i \cdot r \leq x, \quad \forall i \in [n].
     \end{align*}
     By applying Fact~\ref{fact:author_paper_count}, we see that this constraint is equivalent to its counterpart in Definition~\ref{def:ind_fair_min}.

Since both the objective function and constraints in Definition~\ref{def:ind_fair_min} and Definition~\ref{def:ind_fair_min_matrix} are equivalent, the proof is complete.
\end{proof}

To show the hardness of the individual fairness problem, we first present a classical set cover problem with well-established hardness. 

\begin{definition}[Set Cover Problem \cite{k72,gj79}]\label{def:set_cover}
The Set Cover problem is the following: 
\begin{itemize}
    \item {\bf Input:} A universe $U = \{1, \ldots, n\}$, a family of sets 
    $\{S_1, \ldots, S_m\} \subseteq 2^U$, and a integer $K > 0$. 
    \item {\bf Question:} Is there a subfamily $\{S_j : j \in J\}$ for some 
    $J \subseteq \{1,\ldots,m\}$ and $|J| \leq K$ that covers $U$, i.e., $\bigcup_{j \in J} S_j = U$?
\end{itemize}
\end{definition}

\begin{lemma}[Folklore \cite{k72,gj79}]\label{lem:set_cover_np_hard}
    The Set Cover problem defined in Definition~\ref{def:set_cover} is $\NPhard$.
\end{lemma}

Additionally, we also present a technical lemma which is useful for showing the hardness of the individual fairness problem. 

\begin{lemma}\label{lem:useful_lemma}
    For any $r \in \{0,1\}^m$, the following two statements are equivalent:
    \begin{itemize}
       \item {\bf Part 1.} $\|\mathbf{1}_n - D^{-1}Wr\|_\infty \leq 1 - \frac{1}{\min_{i \in [n]}|P_i|}$.
       \item {\bf Part 2.} $\min_{i\in[n]} (Wr)_i \geq 1$.
    \end{itemize}
\end{lemma}
\begin{proof}
    We first show that Part 1 implies Part 2.
    Suppose that
    \begin{align*}
        \|\mathbf{1}_n - D^{-1}Wr\|_\infty \leq 1 - \frac{1}{\min_{i \in [n]}|P_i|}.
    \end{align*}
    By the definition of the infinity norm, we have
    \begin{align*}
         1- \frac{(Wr)_{i'}}{|P_{i'}|} \leq 1 - \frac{1}{\min_{i \in [n]}|P_i|}, \quad \forall i' \in [n].
    \end{align*}
    Rearranging gives
    \begin{align*}
        (Wr)_{i'} \geq \frac{|P_{i'}|}{\min_{i \in [n]}|P_i|} \geq 1, \quad \forall i' \in [n].
    \end{align*}
    Since for all $i'\in[n]$, we have $(Wr)_{i'} \geq 1$, we can conclude that $\min_{i\in[n]} (Wr)_i \geq 1$.

    Now we show that that Part 2 implies Part 1.
    Suppose that $\min_{i\in[n]} (Wr)_i \geq 1$,  then we have $(Wr)_{i} \geq 1$ for all $i \in [n]$, which implies that for all $i \in [n]$, 
    \begin{align*}
        1 - \frac{(Wr)_i}{|P_i|} \leq  1 - \frac{1}{|P_i|} \leq  1 - \frac{1}{\max_{i' \in [n]} |P_{i'}|}.
    \end{align*}
    Hence
    \begin{align*}
        \|\mathbf{1}_n - D^{-1}Wr\|_\infty \leq 1 - \frac{1}{\min_{i \in [n]}|P_i|}.
    \end{align*}
    Thus the proof is complete.
\end{proof}

\begin{theorem}[Hardness, formal version of Theorem~\ref{thm:indi_nphard} in Section~\ref{sec:indi_fair_hard}]\label{thm:indi_nphard_append}
    The Individual Fairness-Aware Submission Limit Problem defined in Definition~\ref{def:ind_fair_min} is $\NPhard$.
\end{theorem}
\begin{proof}
    By Proposition~\ref{prop:equiv_individual}, it sufficies to reduce Set Cover problem to the integer optimization problem of the matrix form in Definition~\ref{def:ind_fair_min_matrix}.

    Given an instance of Set Cover, we build the matrix $W \in \{0, 1\}^{n\times m}$ by defining $W_{i,j} = 1$ if element $i \in S_j$, and 0 otherwise.
    Now set $|P_i| = \sum_{j \in [m]} W_{i,j}$ for every row $i \in [n]$. Finally, we choose $x = m$. We reduce the Set Cover problem to the following optimization problem:
    \begin{align*}
        & ~ \min_{r \in \{0,1\}^m} \| \mathbf{1}_n - D^{-1}Wr\|_\infty \\
    \mathrm{s.t.}
    & ~ W r \leq m \mathbf{1}_n, \\
    & ~ \|r\|_1 \leq K.
    \end{align*}
    Note that this problem is easier than the optimization problem defined in Definition~\ref{def:ind_fair_min}. The constraint $ W r \leq m \mathbf{1}_n$ is always satisfied, so we can drop it out. Now, it suffices to consider the decision problem:
    \begin{align*}
        &~ \mathrm{Find~} r \in \{0,1\}^m \\ \mathrm{s.t.~} &~
       \|\mathbf{1}_n - D^{-1}Wr\|_\infty \leq 1 - \frac{1}{\min_{i \in [n]}|P_i|}, \\
       &~ \|r\|_1 \leq K.
    \end{align*}

    Note that $\|\mathbf{1}_n - D^{-1}Wr\|_\infty \leq 1 - \frac{1}{\min_{i \in [n]}|P_i|}$ is equivalent to 
       $\min_{i\in[n]} (Wr)_i \geq 1$ by Lemma~\ref{lem:useful_lemma}.

    Hence the problem is equivalent to
    \begin{align*}
       \mathrm{Find~} r \in \{0,1\}^m  \mathrm{~~~s.t.~~~}
       \min_{i\in[n]} (Wr)_i \geq 1 \mathrm{~~and~~} 
        \|r\|_1 \leq K.
    \end{align*}
    It is not hard to see that the Set Cover problem has a solution if and only if the above problem has a solution.
    Requiring $\min_{i \in [n]} (W r)_i > 1$ exactly means that each element $i$ in the universe is covered by at least set $S_j$. The constraint $\|r\|_1 \leq K$ means that the size of cover is at most $K$. In other words, there exists a subfamily of size at most $K$ covering all elements if and only if there is an $r \in \{0,1\}^m$ with $\min_{i \in [n]} (W r)_i > 1$ and $\|r\|_1 \leq K$.
    
    Therefore, by Lemma~\ref{lem:set_cover_np_hard}, the individual fairness-aware submission limit problem is $\NPhard$.
\end{proof}

\subsection{Group Fairness Optimization}\label{sec:fair_optim_append}

Now, we present the missing proofs on both matrix form equivalence and linear programming optimal solution equivalence for the group fairness optimization problem. 

\begin{proposition}[Matrix Form Equivalence for $\zeta_{\mathrm{group}}$, formal version of Proposition~\ref{lem:group_fair_min_equiv} in Section~\ref{sec:fair_optim}]\label{lem:group_fair_min_equiv_append}
    The problem in Definition~\ref{def:group_fair_min} and the problem in Definition~\ref{def:group_fair_min_mat_new} are equivalent.
\end{proposition}
\begin{proof}
     In Definition~\ref{def:group_fair_min}, there are $m$ papers in $\mathcal{P}$, where each paper can either be maintained or rejected. Thus, we can encode the paper subset $S$ using a binary vector $r \in \{0, 1\}^m$, where $r_j = 1$ indicates that paper $p_j$ is maintained, and $r_j = 0$ indicates that it is desk-rejected. We now demonstrate that both the objective function and the constraints are equivalent.

     \paragraph{Part 1: Optimization Objective.} We first examine the objective function $\mathbf{1}^\top_nD^{-1}Wr$ in Definition~\ref{def:group_fair_min_mat_new}:
     \begin{align*}
         \mathbf{1}^\top_nD^{-1}Wr &=~ \sum_{i\in[n]}(D^{-1}Wr)_i \\ 
         &=~\sum_{i\in[n]}(W\cdot r)_i / |P_i| \\
         &=~\sum_{i\in[n]}(W_i^\top\cdot r) / |P_i| \\
         &=~\sum_{i\in[n]}\frac{|\{p_j \in S: a_i \in A_j\}|}{|P_i|} \\ 
         &=~ \sum_{i\in[n]}(1-c(a_i,S)),
     \end{align*}
     where the first equality follows from basic algebra, the second follows from Definition~\ref{def:group_fair_min_mat_new}, the third follows from matrix-vector multiplication, the fourth follows from Fact~\ref{fact:author_paper_count}, and the final equality follows from Definition~\ref{def:cost}. Consequently, the maximization problem in Definition~\ref{def:group_fair_min_mat_new} can be rewritten as:
     \begin{align*}
        \max_{r \in \{0, 1\}^m} \sum_{i\in[n]}(1-c(a_i,S)).
     \end{align*}
     Since maximizing this objective is equivalent to minimizing $\sum_{i\in[n]} c(a_i,S)$, we can reformulate it as:
     \begin{align*}
         \min_{r \in \{0, 1\}^m} \sum_{i\in[n]} c(a_i,S).
     \end{align*}
     By decoding $r$ back into the paper subset $S$, we recover the original optimization objective in Definition~\ref{def:group_fair_min}. 

     \paragraph{Part 2: Constraints.} Since the constraint is identical to that in the individual fairness minimization problem in Definition~\ref{def:ind_fair_min}, this result follows directly from Part 2 in the proof of Proposition~\ref{prop:equiv_individual_append}.

     Since both the objective function and constraints in Definition~\ref{def:group_fair_min} and Definition~\ref{def:group_fair_min_mat_new} are equivalent, the proof is complete.
\end{proof}

\begin{theorem}[Optimal Solution Equivalence of the Relaxed Problem, formal version of Theorem~\ref{thm:lp_equiv} in Section~\ref{sec:fair_optim}]\label{thm:lp_equiv_append}
    The optimal solution of the relaxed problem in Definition~\ref{def:group_fair_min_mat_relax_new} is equivalent to the optimal solution of the original problem in Definition~\ref{def:group_fair_min_mat_new}.
\end{theorem}
\begin{proof} 
    The problem in Definition~\ref{def:group_fair_min_mat_relax_new} is a linear program since it has a linear objective function $\mathbf{1}^\top_nD^{-1}Wr$ and linear constraints: the box constraint $r\in[0,1]^m$ and a linear inequality constraint $(Wr)/x \leq \mathbf{1}_n$.

    Furthermore, the problem is convex because the objective function is linear, the constraint $(Wr)/x \leq \mathbf{1}_n$ is affine, and the feasible region defined by $r\in[0,1]^m$ is a convex set.

    By the fundamental theorem of linear programming (see Page 23 of~\cite{ly84}), the optimal solution must occur at an extreme point of the convex polytope defined by the constraints. This implies that for all $i \in [m]$, we must have either $r_i = 0$ or $r_i = 1$. Therefore, the optimal solution of the relaxed linear program coincides with that of the original integer program, which finishes the proof.
\end{proof}

\section{Additional Case Studies}\label{sec:more_case_study}

As discussed in Section~\ref{sec:good_solution_hard}, optimizing the individual fairness metric is computationally challenging. Therefore, we minimize the group fairness metric, which serves as a lower bound for individual fairness, as a practical alternative. In this subsection, we present case studies demonstrating the relationship between both types of fairness metrics. 

\begin{example}
Consider a submission limit problem as defined in Definition~\ref{def:submit_limit_problem} with $x = 2$, $n = 3$ authors, and $m = 6$ papers. Let author $a_1$ submit four papers $p_1, p_2, p_3, p_4$, author $a_2$ submit two papers $p_3, p_5$, and author $a_3$ submit two papers $p_4, p_6$. 
\end{example}

In this case, the ideal desk-rejection criteria in Definition~\ref{def:good_solution} reject $p_1$ and $p_2$ (i.e., $S = \{p_3, p_4, p_5, p_6\}$), yielding fairness metrics $\zeta_{\mathrm{ind}}(S) = \max\{1/2, 0, 0\} = 1/2$ and $\zeta_{\mathrm{group}}(S) = \frac{1}{3}(1/2 + 0 + 0) = 1/6$. By applying an LP solver to minimize group fairness using Algorithm~\ref{alg:fair_desk_reject_algo} and enumerating all rejection strategies to verify individual fairness minimization, we observe that minimizing group fairness in this case aligns with minimizing individual fairness as defined in Definition~\ref{def:ind_fair_min_matrix}. This case illustrates that minimizing group fairness can sometimes benefit individual fairness.

However, group fairness and individual fairness are not always consistent. In some cases, prioritizing group fairness may disproportionately burden certain individuals. To illustrate this, we consider the following example.

\begin{example}
    Consider a submission limit problem as defined in Definition~\ref{def:submit_limit_problem} with $x = 2$, $n = 5$ authors, and $m = 4$ papers. Let author $a_1$ submit four papers $p_1, p_2, p_3, p_4$, author $a_2$ submit two papers $p_1, p_2$, and authors $a_3, a_4, a_5$ be coauthors of papers $p_3, p_4$. 
\end{example}

In this scenario, an ideal desk-rejection is impossible because $a_1$ must have two papers rejected, but rejecting any papers would cause at least one of the authors in $a_2, \ldots, a_5$ to fall below the submission limit of $x=2$. Here, group fairness and individual fairness diverge: Algorithm~\ref{alg:fair_desk_reject_algo} minimizes group fairness by rejecting $p_1$ and $p_2$ (i.e., $S = \{p_3, p_4\}$), which unfairly excludes all of $a_2$'s submissions. This results in fairness metrics $\zeta_{\mathrm{group}}(S) = \frac{1}{4}(1/2 + 1 + 0 + 0) = 3/8$ and $\zeta_{\mathrm{ind}}(S) = \max\{1/2, 1, 0, 0\} = 1$. 

Conversely, the individual fairness minimization problem in Definition~\ref{def:ind_fair_min_matrix} rejects one paper from $a_1, a_2$ and another from $a_3, a_4$, leading to $\zeta_{\mathrm{group}}(S) = \frac{1}{4}(1/2 + 1/2 + 1/2 + 1/2) = 1/2$ and $\zeta_{\mathrm{ind}}(S) = \max\{1/2, 1/2, 1/2, 1/2\} = 1/2$.

This example highlights an unintended consequence of minimizing group fairness: it may unfairly penalize authors with fewer coauthors, as rejecting their papers incurs a smaller total cost. On the other hand, optimizing individual fairness inevitably spreads rejections across a broader set of authors, potentially leading to a higher overall fairness cost. Balancing individual and group fairness remains an open challenge, which we leave for future work.


\section{Summary of Conference Links} \label{app:sec:conference_links}

In the introduction, Table~\ref{tab:conference_submission_limit} only gives a brief summary of the conference year and its limitation of per-author submission. 
Thus, we provide a detailed list of conferences in each year in this section, and then summarize the submission limits in Table.~\ref{tab:conference_submission_limit_full}. 
\begin{itemize}
    \item CVPR 
    \begin{itemize}
        \item 2025, \url{https://cvpr.thecvf.com/Conferences/2025/CVPRChanges} 
        \item 2024, \url{https://cvpr.thecvf.com/Conferences/2024/AuthorGuidelines}
    \end{itemize}

    \item ICCV
    \begin{itemize}
        \item 2025, \url{https://iccv.thecvf.com/Conferences/2025/AuthorGuidelines}
        \item 2023, \url{https://iccv2023.thecvf.com/policies-361500-2-20-15.php}
    \end{itemize}
    
    \item AAAI
    \begin{itemize}
        \item 2025, \url{https://aaai.org/conference/aaai/aaai-25/submission-instructions/} 
        \item 2024, \url{https://aaai.org/aaai-24-conference/submission-instructions/} 
        \item 2023, \url{https://aaai-23.aaai.org/submission-guidelines/} 
        \item 2022, \url{https://aaai.org/conference/aaai/aaai-22/}
    \end{itemize}
    
    \item WSDM 
    \begin{itemize}
        \item 2025, \url{https://www.wsdm-conference.org/2025/call-for-papers/} 
        \item 2024, \url{https://www.wsdm-conference.org/2024/call-for-papers/} 
        \item 2023, \url{https://www.wsdm-conference.org/2023/calls/call-papers/} 
        \item 2022, \url{https://www.wsdm-conference.org/2022/calls/} 
        \item 2021, \url{https://www.wsdm-conference.org/2021/call-for-papers.php} 
        \item 2020, \url{https://www.wsdm-conference.org/2020/call-for-papers.php} 
    \end{itemize}
    
    \item IJCAI
    \begin{itemize}
        \item 2025, \url{https://2025.ijcai.org/call-for-papers-main-track/} 
        \item 2024, \url{https://ijcai24.org/call-for-papers/}
        \item 2023, \url{https://ijcai-23.org/call-for-papers/}
        \item 2022, \url{https://ijcai-22.org/calls-papers}
        \item 2021, \url{https://ijcai-21.org/cfp/index.html}
        \item 2020, \url{https://ijcai20.org/call-for-papers/index.html}
        \item 2019, \url{https://www.ijcai19.org/call-for-papers.html}
        \item 2018, \url{https://www.ijcai-18.org/cfp/index.html}
        \item 2017, \url{https://ijcai-17.org/MainTrackCFP.html}
    \end{itemize}
    
    \item KDD 
    \begin{itemize}
        \item 2025, \url{https://kdd2025.kdd.org/research-track-call-for-papers/} 
        \item 2024, \url{https://kdd2024.kdd.org/research-track-call-for-papers/}
        \item 2023, \url{https://kdd.org/kdd2023/call-for-research-track-papers/index.html}
    \end{itemize}
\end{itemize}

\newpage
\begin{table}[!ht]\caption{ 
In this table, we summarize the submission limits of top conferences in recent years. For details of each conference website, we refer the readers to Section~\ref{app:sec:conference_links} in Appendix. 
}  \label{tab:conference_submission_limit_full}
\begin{center}
\begin{tabular}{ |c|c|c|c|c| } 
 \hline
 {\bf Conference Name} & {\bf Year} & {\bf Upper Bound} \\ \hline
 CVPR & 2025 & 25 \\ \hline
 CVPR & 2024 & N/A \\ \hline
 ICCV & 2025 & 25 \\ \hline
 ICCV & 2023 & N/A \\ \hline
 AAAI & 2025 & 10 \\ \hline
 AAAI & 2024 & 10 \\ \hline
 AAAI & 2023 & 10 \\ \hline
 AAAI & 2022 & N/A \\ \hline
 WSDM & 2025 & 10 \\ \hline
 WSDM & 2024 & 10 \\ \hline
 WSDM & 2023 & 10 \\ \hline
 WSDM & 2022 & 10 \\ \hline
 WSDM & 2021 & 10 \\ \hline
 WSDM & 2020 & N/A \\ \hline
 IJCAI & 2025 & 8 \\ \hline
 IJCAI & 2024 & 8 \\ \hline
 IJCAI & 2023 & 8 \\ \hline
 IJCAI & 2022 & 8 \\ \hline
 IJCAI & 2021 & 8 \\ \hline
 IJCAI & 2020 & 6 \\ \hline
 IJCAI & 2019 & 10 \\ \hline
 IJCAI & 2018 & 10 \\ \hline
 IJCAI & 2017 & N/A \\ \hline
 KDD & 2025 & 7 \\ \hline
 KDD & 2024 & 7 \\ \hline
 KDD & 2023 & N/A \\ \hline
 \iffalse
 ICML & 2025 & None \\ \hline
 ICML & 2024 & None \\ \hline
 ICML & 2023 & None \\ \hline
 ICLR & 2025 & None \\ \hline
 ICLR & 2024 & None \\ \hline
 ICLR & 2023 & None \\ \hline
 ICCV & 2023 & None \\ \hline
 ICCV & 2021 & None \\ \hline
 ICCV & 2019 & None \\ \hline
 ECCV & 2024 & None \\ \hline
 ECCV & 2022 & None \\ \hline
 ECCV & 2020 & None \\ \hline
 WACV & 2025 & None \\ \hline
 WACV & 2024 & None \\ \hline
 WACV & 2023 & None \\ \hline
 ICRA & 2025 & None \\ \hline
 ICRA & 2024 & None \\ \hline
 ICRA & 2023 & None \\ \hline
 CoRL & 2024 & None \\ \hline
 CoRL & 2023 & None \\ \hline
 CoRL & 2022 & None \\ \hline
 EMNLP & 2024 & None \\ \hline
 EMNLP & 2023 & None \\ \hline
 EMNLP & 2022 & None \\ \hline
 ACL & 2025 & None \\ \hline
 ACL & 2024 & None \\ \hline
 ACL & 2023 & None \\ \hline
 BigData & 2025 & None \\ \hline
 BigData & 2024 & None \\ \hline
 BigData & 2023 & None \\ \hline
 \fi
\end{tabular}
\end{center}
\end{table}






%%% some writing rules

%% Writing rule for creating tags.
%% Tags :
%% Theorem    \ref{thm:bla_bla}
%% Lemma      \ref{lem:bla_bla}
%% Claim      \ref{cla:bla_bla}
%% Corollary  \ref{cor:bla_bla}
%% Fact       \ref{fac:bla_bla}
%% Definition \ref{def:bla_bla}
%% Section    \ref{sec:bla_bla}
%% Subsection \ref{sub:bla_bla}
%% Equation   \ref{eq:bla_bla}



\end{document}



%%%%%%%%%%%%%%%%%%%%%%%%%%%%%%%%%%%%%%%%%%%%%%%%%%%%%%%%%%%%%%%%%%%%%%%%%%%%%%%%%%%%%%%%%%%%%%%%%%%%%%%%%%%%%%%%%%%%%%%%%%%%%%%%%%%%%%%%%%%%%%%%%%%%%%%%%%%%%%%%%%%%%%%%%%%%%%%%%%%%%%%%%%%%%%%%%%%%%%%%%%%%%%%%%%%%%%%%%%%%%%%%%%%%%%%%%%%%%%%%%%%%%%%%%%%%%%%%%%%%%%%%%%%%%%%%%%%%%%%%%%%%%%%%%%%%%%%%%%%%%%%%%%%%%%%%%%%%%%%%%%%%%%%%%%%%%%%%%%%%%%%%%%%%%%%%%%%%%%%%%%%%%%%%%%%%%%%%%%%%%%%%%%%%%%%%%%%%%%%%%%%%%%%%%%%%%%%%%%%%%%%%%%%%%%%%%%%%%%%%%%%%%%%%%%%%%%%%%%%%%%
