\def\isarxiv{1} %%% for icml submission version, we comment this line

\ifdefined\isarxiv
\documentclass[11pt]{article}

\usepackage[numbers]{natbib}


\else
\documentclass{article}
\usepackage{algorithm}
\usepackage{microtype}
\usepackage{graphicx}
\usepackage{subfigure}
\usepackage{booktabs}
\usepackage{hyperref}
\usepackage{icml2025}



\usepackage{amsmath}
\usepackage{amssymb}
\usepackage{mathtools}
\usepackage{amsthm}
\usepackage{algorithm}
\usepackage{algpseudocode}
\fi

\ifdefined\isarxiv
\usepackage{amsmath}
\usepackage{amsthm}
\usepackage{amssymb}
\usepackage{algorithm}
\usepackage{subfig}
\usepackage{algpseudocode}
\usepackage{graphicx}
\usepackage{grffile}
\usepackage{wrapfig,epsfig}
\usepackage{url}
\usepackage{xcolor}
\usepackage{epstopdf}


\usepackage{bbm}
\usepackage{dsfont}
\fi
 
 
\allowdisplaybreaks
 

\ifdefined\isarxiv

\let\C\relax
\usepackage{tikz}
\usepackage{hyperref}  %%% arxiv don't allow this.
\hypersetup{colorlinks=true,citecolor=blue,linkcolor=blue} %%% Zhao : maybe we should comment this in submission.
\usetikzlibrary{arrows}
\usepackage[margin=1in]{geometry}

\else

% \usepackage{microtype}
% \usepackage{hyperref}
% \definecolor{mydarkblue}{rgb}{0,0.08,0.45}
% \hypersetup{colorlinks=true, citecolor=mydarkblue,linkcolor=mydarkblue}
 

\fi



 
\graphicspath{{./figs/}}

\theoremstyle{plain}
\newtheorem{theorem}{Theorem}[section]
\newtheorem{lemma}[theorem]{Lemma}
\newtheorem{definition}[theorem]{Definition}
\newtheorem{notation}[theorem]{Notation}
%\newtheorem{proof}[theorem]{Proof}
\newtheorem{proposition}[theorem]{Proposition}
\newtheorem{corollary}[theorem]{Corollary}
\newtheorem{conjecture}[theorem]{Conjecture}
\newtheorem{assumption}[theorem]{Assumption}
\newtheorem{observation}[theorem]{Observation}
\newtheorem{fact}[theorem]{Fact}
\newtheorem{remark}[theorem]{Remark}
\newtheorem{claim}[theorem]{Claim}
\newtheorem{example}[theorem]{Example}
\newtheorem{problem}[theorem]{Problem}
\newtheorem{open}[theorem]{Open Problem}
\newtheorem{property}[theorem]{Property}
\newtheorem{hypothesis}[theorem]{Hypothesis}

\newtheorem{question}[theorem]{Question}

\newcommand{\wh}{\widehat}
\newcommand{\wt}{\widetilde}
\newcommand{\ov}{\overline}
\newcommand{\N}{\mathcal{N}}
\newcommand{\R}{\mathbb{R}}
\newcommand{\F}{\mathcal{F}}
\newcommand{\G}{\mathcal{G}}
\newcommand{\True}{\mathrm{true}}
\newcommand{\Noisy}{\mathrm{noisy}}
\newcommand{\Clean}{\mathrm{clean}}
\newcommand{\Est}{\mathrm{est}}
\newcommand{\RHS}{\mathrm{RHS}}
\newcommand{\LHS}{\mathrm{LHS}}
\renewcommand{\d}{\mathrm{d}}
\renewcommand{\i}{\mathbf{i}}
\renewcommand{\tilde}{\wt}
\renewcommand{\hat}{\wh}
\newcommand{\Tmat}{{\cal T}_{\mathrm{mat}}}
\newcommand{\NPhard}{\mathsf{NP}\text{-}\mathsf{hard}}

\DeclareMathOperator*{\E}{{\mathbb{E}}}
\DeclareMathOperator*{\var}{\mathrm{Var}}
\DeclareMathOperator*{\Z}{\mathbb{Z}}
\DeclareMathOperator*{\C}{\mathbb{C}}
\DeclareMathOperator*{\D}{\mathcal{D}}
\DeclareMathOperator*{\median}{median}
\DeclareMathOperator*{\mean}{mean}
\DeclareMathOperator{\OPT}{OPT}
\DeclareMathOperator{\supp}{supp}
\DeclareMathOperator{\poly}{poly}

\DeclareMathOperator{\nnz}{nnz}
\DeclareMathOperator{\sparsity}{sparsity}
\DeclareMathOperator{\rank}{rank}
\DeclareMathOperator{\diag}{diag}
\DeclareMathOperator{\Diag}{Diag}
\DeclareMathOperator{\dist}{dist}
\DeclareMathOperator{\cost}{cost}
\DeclareMathOperator{\vect}{vec}
\DeclareMathOperator{\tr}{tr}
\DeclareMathOperator{\dis}{dis}
\DeclareMathOperator{\cts}{cts}



\makeatletter
\newcommand*{\RN}[1]{\expandafter\@slowromancap\romannumeral #1@}
\makeatother
\newcommand{\Zhao}[1]{{\color{red}[Zhao: #1]}}
\newcommand{\Zhenmei}[1]{{\color{purple}[Zhenmei: #1]}}
\newcommand{\Zhizhou}[1]{{\color{blue}[Zhizhou: #1]}}
\newcommand{\Yuefan}[1]{{\color{brown}[Yuefan: #1]}} 
\newcommand{\Jiahao}[1]{{\color{orange}[Jiahao: #1]}} %%%Change to intern name
\newcommand{\Xiaoyu}[1]{{\color{purple}[Xiaoyu: #1]}} 


\usepackage{lineno}
\def\linenumberfont{\normalfont\small}


\ifdefined\isarxiv
\else
\icmltitlerunning{Dissecting Submission Limit in Desk-Rejections: A Mathematical Analysis of Fairness in AI Conference Policies}
\fi
\begin{document}

\ifdefined\isarxiv

\date{}

\title{Dissecting Submission Limit in Desk-Rejections: A Mathematical Analysis of Fairness in AI Conference Policies}
\author{
Yuefan Cao\thanks{\texttt{ ralph1997off@gmail.com}. Zhejiang University.}
\and
Xiaoyu Li\thanks{\texttt{
xiaoyu.li2@student.unsw.edu.au}. University of New South Wales.}
\and
Yingyu Liang\thanks{\texttt{
yingyul@hku.hk}. The University of Hong Kong. \texttt{
yliang@cs.wisc.edu}. University of Wisconsin-Madison.} 
\and
Zhizhou Sha\thanks{\texttt{
shazz20@mails.tsinghua.edu.cn}. Tsinghua University.}
\and
Zhenmei Shi\thanks{\texttt{
zhmeishi@cs.wisc.edu}. University of Wisconsin-Madison.}
\and
Zhao Song\thanks{\texttt{ magic.linuxkde@gmail.com}. The Simons Institute for the Theory of Computing at UC Berkeley.}
\and
Jiahao Zhang\thanks{\texttt{ ml.jiahaozhang02@gmail.com}. Independent Researcher.}
}

\else

% \title{Intern Project} 
% \maketitle 
% \iffalse
%\linenumbers

\twocolumn[

\icmltitle{Dissecting Submission Limit in Desk-Rejections: A Mathematical Analysis\\ of Fairness in AI Conference Policies}
% It is OKAY to include author information, even for blind
% submissions: the style file will automatically remove it for you
% unless you've provided the [accepted] option to the icml2019
% package.

% List of affiliations: The first argument should be a (short)
% identifier you will use later to specify author affiliations
% Academic affiliations should list Department, University, City, Region, Country
% Industry affiliations should list Company, City, Region, Country

% You can specify symbols, otherwise they are numbered in order.
% Ideally, you should not use this facility. Affiliations will be numbered
% in order of appearance and this is the preferred way.
\icmlsetsymbol{equal}{*}

\begin{icmlauthorlist}
\icmlauthor{Aeiau Zzzz}{equal,to}
\icmlauthor{Bauiu C.~Yyyy}{equal,to,goo}
\icmlauthor{Cieua Vvvvv}{goo}
\icmlauthor{Iaesut Saoeu}{ed}
\icmlauthor{Fiuea Rrrr}{to}
\icmlauthor{Tateu H.~Yasehe}{ed,to,goo}
\icmlauthor{Aaoeu Iasoh}{goo}
\icmlauthor{Buiui Eueu}{ed}
\icmlauthor{Aeuia Zzzz}{ed}
\icmlauthor{Bieea C.~Yyyy}{to,goo}
\icmlauthor{Teoau Xxxx}{ed}\label{eq:335_2}
\icmlauthor{Eee Pppp}{ed}
\end{icmlauthorlist}

\icmlaffiliation{to}{Department of Computation, University of Torontoland, Torontoland, Canada}
\icmlaffiliation{goo}{Googol ShallowMind, New London, Michigan, USA}
\icmlaffiliation{ed}{School of Computation, University of Edenborrow, Edenborrow, United Kingdom}

\icmlcorrespondingauthor{Cieua Vvvvv}{c.vvvvv@googol.com}
\icmlcorrespondingauthor{Eee Pppp}{ep@eden.co.uk}

% You may provide any keywords that you
% find helpful for describing your paper; these are used to populate
% the "keywords" metadata in the PDF but will not be shown in the document
\icmlkeywords{Machine Learning, ICML}

\vskip 0.3in
]

\printAffiliationsAndNotice{\icmlEqualContribution} 
% \fi
\fi





\ifdefined\isarxiv
\begin{titlepage}
  \maketitle
  \begin{abstract}
\begin{abstract}


The choice of representation for geographic location significantly impacts the accuracy of models for a broad range of geospatial tasks, including fine-grained species classification, population density estimation, and biome classification. Recent works like SatCLIP and GeoCLIP learn such representations by contrastively aligning geolocation with co-located images. While these methods work exceptionally well, in this paper, we posit that the current training strategies fail to fully capture the important visual features. We provide an information theoretic perspective on why the resulting embeddings from these methods discard crucial visual information that is important for many downstream tasks. To solve this problem, we propose a novel retrieval-augmented strategy called RANGE. We build our method on the intuition that the visual features of a location can be estimated by combining the visual features from multiple similar-looking locations. We evaluate our method across a wide variety of tasks. Our results show that RANGE outperforms the existing state-of-the-art models with significant margins in most tasks. We show gains of up to 13.1\% on classification tasks and 0.145 $R^2$ on regression tasks. All our code and models will be made available at: \href{https://github.com/mvrl/RANGE}{https://github.com/mvrl/RANGE}.

\end{abstract}



  \end{abstract}
  \thispagestyle{empty}
\end{titlepage}

{\hypersetup{linkcolor=black}
\tableofcontents
}
\newpage

\else

\begin{abstract}
\begin{abstract}


The choice of representation for geographic location significantly impacts the accuracy of models for a broad range of geospatial tasks, including fine-grained species classification, population density estimation, and biome classification. Recent works like SatCLIP and GeoCLIP learn such representations by contrastively aligning geolocation with co-located images. While these methods work exceptionally well, in this paper, we posit that the current training strategies fail to fully capture the important visual features. We provide an information theoretic perspective on why the resulting embeddings from these methods discard crucial visual information that is important for many downstream tasks. To solve this problem, we propose a novel retrieval-augmented strategy called RANGE. We build our method on the intuition that the visual features of a location can be estimated by combining the visual features from multiple similar-looking locations. We evaluate our method across a wide variety of tasks. Our results show that RANGE outperforms the existing state-of-the-art models with significant margins in most tasks. We show gains of up to 13.1\% on classification tasks and 0.145 $R^2$ on regression tasks. All our code and models will be made available at: \href{https://github.com/mvrl/RANGE}{https://github.com/mvrl/RANGE}.

\end{abstract}


\end{abstract}

\fi


\section{Introduction}
Backdoor attacks pose a concealed yet profound security risk to machine learning (ML) models, for which the adversaries can inject a stealth backdoor into the model during training, enabling them to illicitly control the model's output upon encountering predefined inputs. These attacks can even occur without the knowledge of developers or end-users, thereby undermining the trust in ML systems. As ML becomes more deeply embedded in critical sectors like finance, healthcare, and autonomous driving \citep{he2016deep, liu2020computing, tournier2019mrtrix3, adjabi2020past}, the potential damage from backdoor attacks grows, underscoring the emergency for developing robust defense mechanisms against backdoor attacks.

To address the threat of backdoor attacks, researchers have developed a variety of strategies \cite{liu2018fine,wu2021adversarial,wang2019neural,zeng2022adversarial,zhu2023neural,Zhu_2023_ICCV, wei2024shared,wei2024d3}, aimed at purifying backdoors within victim models. These methods are designed to integrate with current deployment workflows seamlessly and have demonstrated significant success in mitigating the effects of backdoor triggers \cite{wubackdoorbench, wu2023defenses, wu2024backdoorbench,dunnett2024countering}.  However, most state-of-the-art (SOTA) backdoor purification methods operate under the assumption that a small clean dataset, often referred to as \textbf{auxiliary dataset}, is available for purification. Such an assumption poses practical challenges, especially in scenarios where data is scarce. To tackle this challenge, efforts have been made to reduce the size of the required auxiliary dataset~\cite{chai2022oneshot,li2023reconstructive, Zhu_2023_ICCV} and even explore dataset-free purification techniques~\cite{zheng2022data,hong2023revisiting,lin2024fusing}. Although these approaches offer some improvements, recent evaluations \cite{dunnett2024countering, wu2024backdoorbench} continue to highlight the importance of sufficient auxiliary data for achieving robust defenses against backdoor attacks.

While significant progress has been made in reducing the size of auxiliary datasets, an equally critical yet underexplored question remains: \emph{how does the nature of the auxiliary dataset affect purification effectiveness?} In  real-world  applications, auxiliary datasets can vary widely, encompassing in-distribution data, synthetic data, or external data from different sources. Understanding how each type of auxiliary dataset influences the purification effectiveness is vital for selecting or constructing the most suitable auxiliary dataset and the corresponding technique. For instance, when multiple datasets are available, understanding how different datasets contribute to purification can guide defenders in selecting or crafting the most appropriate dataset. Conversely, when only limited auxiliary data is accessible, knowing which purification technique works best under those constraints is critical. Therefore, there is an urgent need for a thorough investigation into the impact of auxiliary datasets on purification effectiveness to guide defenders in  enhancing the security of ML systems. 

In this paper, we systematically investigate the critical role of auxiliary datasets in backdoor purification, aiming to bridge the gap between idealized and practical purification scenarios.  Specifically, we first construct a diverse set of auxiliary datasets to emulate real-world conditions, as summarized in Table~\ref{overall}. These datasets include in-distribution data, synthetic data, and external data from other sources. Through an evaluation of SOTA backdoor purification methods across these datasets, we uncover several critical insights: \textbf{1)} In-distribution datasets, particularly those carefully filtered from the original training data of the victim model, effectively preserve the model’s utility for its intended tasks but may fall short in eliminating backdoors. \textbf{2)} Incorporating OOD datasets can help the model forget backdoors but also bring the risk of forgetting critical learned knowledge, significantly degrading its overall performance. Building on these findings, we propose Guided Input Calibration (GIC), a novel technique that enhances backdoor purification by adaptively transforming auxiliary data to better align with the victim model’s learned representations. By leveraging the victim model itself to guide this transformation, GIC optimizes the purification process, striking a balance between preserving model utility and mitigating backdoor threats. Extensive experiments demonstrate that GIC significantly improves the effectiveness of backdoor purification across diverse auxiliary datasets, providing a practical and robust defense solution.

Our main contributions are threefold:
\textbf{1) Impact analysis of auxiliary datasets:} We take the \textbf{first step}  in systematically investigating how different types of auxiliary datasets influence backdoor purification effectiveness. Our findings provide novel insights and serve as a foundation for future research on optimizing dataset selection and construction for enhanced backdoor defense.
%
\textbf{2) Compilation and evaluation of diverse auxiliary datasets:}  We have compiled and rigorously evaluated a diverse set of auxiliary datasets using SOTA purification methods, making our datasets and code publicly available to facilitate and support future research on practical backdoor defense strategies.
%
\textbf{3) Introduction of GIC:} We introduce GIC, the \textbf{first} dedicated solution designed to align auxiliary datasets with the model’s learned representations, significantly enhancing backdoor mitigation across various dataset types. Our approach sets a new benchmark for practical and effective backdoor defense.


 %%% Section 1. Introduction
\begin{figure*}[t]
  \centering
    \includegraphics[width=1\linewidth]{visuals/final_registration.png}
    \caption{Target measurement process on low-cost scan data using ICP and Coloured ICP. (1) Initialisation: The source point cloud (checkerboard) is misaligned with the target point cloud. (2) Initial Registration using Point-to-Plane ICP: Standard ICP leads to suboptimal registration. (3) Final Registration using Coloured ICP: Colour information is incorporated after pre-processing with RANSAC and Binarisation with Otsu Thresholding for real data, resulting in improved alignment.}
    \label{fig:Registration_visualisation}
\end{figure*}

\subsection{Iterative Closest Point (ICP) Algorithm}
The Iterative Closest Point (ICP) algorithm has been a fundamental technique in 3D computer vision and robotics for point cloud. Originally proposed by \cite{besl_method_1992}, ICP aims to minimise the distance between two datasets, typically referred to as the source and the target. The algorithm operates in an iterative manner, identifying correspondences by matching each source point with its nearest target point \citep{survey_ICP}. It then computes the rigid transformation, usually involving both rotation and translation, to achieve the best alignment of these matched points \citep{survey_ICP}. This process is repeated until convergence, where the change in the alignment parameters or the overall alignment error becomes smaller than a predefined threshold.

One key advantage of the ICP framework lies in its simplicity: the algorithm is conceptually straightforward, and its basic version is relatively easy to implement. However, traditional ICP can be sensitive to local minima, often requiring a good initial alignment \citep{zhang2021fast}. Furthermore, outliers, noise, and partial overlaps between datasets can significantly degrade its performance \citep{zhang2021fast, bouaziz2013sparse}. Over the years, various modifications and improvements \citep{gelfand2005robust, rusu2009fast, aiger20084, gruen2005least, fitzgibbon2003robust} have been proposed to mitigate these issues. Among the most common strategies are robust cost functions \citep{fitzgibbon2003robust}, weighting schemes for correspondences \citep{rusu2009fast}, and more sophisticated techniques \citep{gelfand2005robust, bouaziz2013sparse} to reject outliers. 

In addition, there is significant interest in integrating additional information into the ICP pipeline. Instead of solely relying on geometric cues such as point coordinates or surface normals, recent approaches have proposed incorporating colour (RGB) or intensity data to enhance correspondence accuracy. These methods \citep{park_colored_2017, 5980407}, commonly known as "Colored ICP" employ differences in pixel intensities or colour values as additional constraints. This is particularly beneficial in situations where geometric attributes alone are inadequate for accurate alignment or where surfaces possess complex texture patterns that can assist in the matching process.

\subsection{Applications of Target Measurement}

One approach relies on the use of physical checkerboard targets for registration. \cite{fryskowska2019} analyse checkerboard target identification for terrestrial laser scanning. They propose a geometric method to determine the target centre with higher precision, demonstrating that their approach can reduce errors by up to 6 mm compared to conventional automatic methods.

\cite{becerik2011assessment} examines data acquisition errors in 3D laser scanning for construction by evaluating how different target types (paper, paddle, and sphere) and layouts impact registration accuracy in both indoor and outdoor environments and presents guidelines for optimal target configuration.

\citet{Liang2024} propose to use Coloured ICP to measure target centres for checkerboard targets, similar to our investigation. They use data from a survey-grade terrestrial laser scanner. Their intended application is structural bridge monitoring purposes. They report an average accuracy of the measurement below 1.3 millimetres.

Where targets cannot be placed in the scene, the intensity information form the scanner can still be used to identify distinctive points. For point cloud data that is captured with a regular pattern, standard image processing can be used in a similar way to target detection. For example, \citet{wendt_automation_2004} proposes to use the SUSAN operator on a co-registered image from a camera, \citet{bohm_automatic_2007} proposes to use the SIFT operator on the LIDAR reflectance directly and \citet{theiler_markerless_2013} propose to use a Difference-of-Gaussian approach on the reflectance information.
Most of these methods extract image features to find reliable 3D correspondences for the purpose of registration.

In the following we describe our approach to the measurement of the target centre. In contrast to most proposed methods above we focus on unordered point clouds, where raster-based methods are not available, and low-cost sensors, where increased measurement noise and outliers are expected. As we are not aware of a commercial reference solution to this problem, we start by conducting a series of synthetic experiments to explore the viability and accuracy potential of the approach.



%The reviewed studies primarily rely on physical targets or target-free methods and do not utilise 3D synthetic point cloud checkerboards. In contrast, our approach introduces synthetic point cloud checkerboards, which offer controlled and consistent target geometry and reduce variability caused by physical targets. This innovation has significant potential for commercialisation and industrial application.



\section{Preliminary} \label{sec:preliminary}
In this section, we first introduce the notations in Section~\ref{sec:notations}. Then, we present the general problem formulation in Section~\ref{sec:problem_formulation}.

\subsection{Notations}\label{sec:notations}
For any positive integer $n$, we use $[n]$ to denote the set $\{1, 2, \ldots, n\}$. We use $\mathbb{N}_+$ to represent the set of all positive integers. For two sets $\mathcal{B}$ and $\mathcal{C}$, we denote the set difference as $\mathcal{B} \setminus \mathcal{C}:=\{x\in \mathcal{B}:x\notin\mathcal{C}\}$. For a vector $x \in \mathbb{R}^d$, $\Diag(d)$ denotes a diagonal matrix $X \in \mathbb{R}^{d \times d}$, where the diagonal entries satisfy $X_{i,i} = x_i$ for all $i \in [d]$, and all off-diagonal entries are zero. We use $\mathbf{1}_n$ to denote an $n$-dimensional column vector with all entries equal to one.


\begin{figure}[!ht]
    \centering
    \includegraphics[width=0.95\linewidth]{our_research.pdf}
    \caption{
    Our research objective. This figure presents the goal of our study: creating a more equitable desk-rejection system. Consider Professor A, who has carelessly submitted numerous papers exceeding the submission limit, collaborating with another senior researcher (Professor B) with many submissions, and a young student with only one paper. Our proposed system prioritizes desk-rejecting papers from authors with a large number of submissions first, thereby increasing the student’s chances of having their paper accepted. This approach aims to mitigate the disparity in the impact of desk rejections and promote fairness.
    }
    \label{fig:our_research}
\end{figure}

\subsection{Problem Formulation} \label{sec:problem_formulation}

In this section, we further introduce the actual problem we will investigate in this paper, where we begin with introducing the definition for three kinds of authors that will appear later in our discussion. 


\begin{definition}[Submission Limit Problem]\label{def:submit_limit_problem}
    Let $\mathcal{A} = \{a_1, a_2, \dots, a_n\}$ denote the set of $n$ authors, and let $\mathcal{P} = \{p_1, p_2, \dots, p_m\}$ denote the set of $m$ papers. Each author $a_i \in \mathcal{A}$ has a subset of papers $P_i \subseteq \mathcal{P}$, and each paper $p_j \in \mathcal{P}$ is authored by a subset of authors $A_j \subseteq \mathcal{A}$. For each author, $a_i \in \mathcal{A}$, let $C_i$ denote the set of all coauthors of $a_i$ and let $x \in \mathbb{N}_+$ denote the maximum number of papers each author can submit. 

    The goal is to find a subset $S \subseteq \mathcal{P}$ of papers (to keep) such that for every $a_i\in\mathcal{A},$
    \begin{align*}
        \underbrace{|\{p_j \in  S : a_i \in A_j\}|}_{\#\mathrm{remained~papers~of~author}~a_i} \leq x.  
    \end{align*}
    or equivalently find a subset $\ov{S} \subseteq \mathcal{P}$ of papers (to reject) such that for every $a_i \in \mathcal{A}$,
        \begin{align*}
        |P_i| - \underbrace{|\{j \in  \ov{S} : i \in A_j\}|}_{\#{\mathrm{rejected~papers~of~author}~}a_i} \leq x.  
    \end{align*}
\end{definition}

We now present several fundamental facts related to Definition~\ref{def:submit_limit_problem}, which can be easily verified through basic set theory. 
\begin{fact}
    For any author $a_i \in \mathcal{A}$ and paper $p_j \in \mathcal{P}$, $a_i \in A_j$ if and only if $p_j \in P_i$.
\end{fact}

\begin{fact}
    For each author $a_i \in \mathcal{A}$, the number of papers submitted by the author can be formulated as:
    \begin{align*}
        |P_i| = |\{p_j \in \mathcal{P} : a_i \in A_j\}|.
    \end{align*}
\end{fact}

\begin{fact}
    For each paper $j \in [m]$, the number of authors of this paper can be formulated as:
    \begin{align*}
        |A_j| = |\{a_i \in \mathcal{A} : p_j \in P_i\}|.
    \end{align*}
\end{fact}

\begin{fact}
    For each author $a_i \in \mathcal{A}$, the set of coauthors for author $a_i$ can be formulated as:
    \begin{align*}
        C_i = (\bigcup_{p_j \in P_i} A_j) \setminus \{ a_i \}.
    \end{align*}
\end{fact}


\section{The Desk Rejection Dilemma}\label{sec:dr_dilemma}

In this section, we define the concept of an ideal desk-rejection system in Section~\ref{sec:good_solution} and formally demonstrate in Section~\ref{sec:good_solution_hard} that no algorithm can achieve this ideal system.

\subsection{Ideal Desk-Rejection}\label{sec:good_solution}
An ideal desk-rejection system should avoid unfairly rejecting papers from authors who either comply with the submission limit or exceed it by only one or two papers. Otherwise, authors may face consequences due to co-authors with an excessively high number of submissions. This issue is particularly problematic for early-career researchers, as such collective penalties can have a significant negative impact on their careers. 

To address this, we formally define the criteria for an ideal desk-rejection outcome for the problem in Definition~\ref{def:submit_limit_problem}, where rejections are based solely on an author’s excessive submissions, without unfairly penalizing others.

\begin{definition}[Ideal desk-rejection] \label{def:good_solution}
An ideal solution for the submission limit problem in Definition~\ref{def:submit_limit_problem} is a paper subset $S\subseteq \mathcal{P}$ such that every author has exactly $\min\{x, |P_i|\}$ papers remaining after desk rejection. 

\end{definition}
\begin{remark}
    The ideal desk-rejection in Definition~\ref{def:good_solution} ensures that innocent authors with less than $x$ submissions will retain all their papers, and a non-compliant author $a_i$ with more than $x$ submissions will be desk-rejected exact $(|P_i|-x)$ papers.
\end{remark}
Thus, if there exists an algorithm that can reach the aforementioned ideal solution, we can ensure that no author is unfairly penalized due to their co-authors' submission behavior, achieving both fairness and individual accountability.
\subsection{Hardness of Ideal Desk-Rejection}\label{sec:good_solution_hard}

Unfortunately, we find that achieving an ideal desk-rejection system is fundamentally intractable. The main result regarding this hardness is presented in the following theorem:

\begin{theorem}[Hardness of Ideal Desk-Rejection]\label{thm:main_res_general}
Let $n = |\mathcal{A}|$ denote the number of authors in Definition~\ref{def:submit_limit_problem}. We can show that

\begin{itemize}
    \item {\bf Part 1:} For $n \le 2$, there always exists an algorithm that can achieve the ideal desk-rejection in Definition~\ref{def:good_solution}.
    \item {\bf Part 2:} For $n \geq 3$, there exists at least one problem instance where no algorithm can guarantee achieving the ideal desk-rejection in Definition~\ref{def:good_solution}.
\end{itemize}
\end{theorem}

\begin{proof} For {\bf Part 1}, the result follows directly from Lemma~\ref{lem:n_eq_1_positive_general} and Lemma~\ref{lem:n_eq_2_positive_general}. For {\bf Part 2}, the result is established using Lemma~\ref{lem:n_eq_3_negative} and Lemma~\ref{lem:n_geq_3_negative}. 
Detailed technical proofs for these lemmas are provided in Appendix~\ref{sec:dilemma_proof}.
\end{proof}

Therefore, since an ideal desk-rejection system is not achievable, it is inevitable that some authors may face excessive desk-rejections due to collective punishments. This challenge is particularly concerning for early-career researchers with only one or two submissions, motivating the need to seek an approximate solution that optimizes fairness in desk-rejection systems.

\section{Fairness-Aware Desk-Rejection}\label{sec:fair}

In this section, we first introduce two fairness metrics in Section~\ref{sec:fair_metric}, and then present the hardness result on minimizing one of them in Section~\ref{sec:good_solution_hard}. In Section~\ref{sec:fair_optim}, we show our optimization-based fairness-aware desk-rejection framework. 

\subsection{Fairness Metrics}\label{sec:fair_metric}

As discussed earlier, achieving an ideal desk-rejection system is practically infeasible, as unintended rejections due to collective punishments are unavoidable. To address this, we relax the ideal system into an approximate form, where some unfair desk-rejections are permitted, while these rejections should be proportional to each author's total number of submissions.

Specifically, we introduce a cost function for each author, which estimates the impact of desk-rejection on each author:

\begin{definition}[Cost Function]\label{def:cost}
Considering the submission limit problem in Definition~\ref{def:submit_limit_problem}, we define the cost function $c: [n] \times 2^{[m]} \to [0,1]$ for a specific author $a_i$ and a set of remaining paper $S$ as
\begin{align*}
    c(a_i, S) := \frac{|P_i| - |\{p_j \in S: a_i \in A_j\}|}{|P_i|}.
\end{align*}
\end{definition}

\begin{remark}
    The cost function $c(a_i,S)$ measures the proportion of papers authored by $a_i$ that are rejected, prioritizing fairness for early-career authors with fewer submissions and aiming to reduce setbacks for them.
\end{remark}

To further demonstrate how this author-wise cost function could benefit fairness, we present the following example:
\begin{example}
    Consider a submission limit problem with $x = 10$ and $n = 2$. Suppose author $a_1$ submits papers $p_1, p_2, \ldots, p_{11}$, and author $a_2$ submits only paper $p_{11}$. Rejecting paper $p_{11}$ (i.e., $S = \mathcal{P} \setminus \{p_{11}\}$) results in a cost of $c(a_1, S) = 1/11$ for $a_1$ but a cost of $c(a_2, S) = 1$ for $a_2$, which is unfair to $a_2$. On the other hand, if we reject paper $p_1$ (i.e., $S' = \mathcal{P} \setminus \{p_1\}$), the cost for $a_1$ remains $c(a_1, S') = 1/11$, while the cost for $a_2$ becomes $c(a_2, S') = 0$. This minimizes both the highest cost and the total cost. This example demonstrates that our cost function encourages rejecting papers from authors with many submissions while protecting authors with few submissions.
\end{example}

To ensure fair treatment for all authors and avoid imposing excessive setbacks on early-career researchers, we introduce two fairness metrics based on our cost function. These metrics are inspired by the principles of utilitarian social welfare and egalitarian social welfare~\cite{ams24}. We begin by defining individual fairness, which is a strict worst-case fairness metric that aligns with the egalitarian social welfare framework by estimating the individual cost among all authors.

\begin{definition}[Individual Fairness]\label{def:individual_fair}
Let $c: [n] \times 2^{[m]} \to [0,1] $ be the cost function defined in Definition~\ref{def:cost}.
We define function $\zeta_{\mathrm{ind}}: 2^{[m]} \to [0,1]$ to measure the individual fairness:
\begin{align*}
    & ~ \zeta_{\mathrm{ind}}(S) := \max_{i \in [n]} c(a_i,S).
\end{align*}
\end{definition}
Next, we present the concept of group fairness, which aligns with utilitarian social welfare and measures the total cost across all authors.

\begin{definition}[Group Fairness]\label{def:group_fair}
Let $c: [n] \times 2^{[m]} \to [0,1] $ be the cost function defined in Definition~\ref{def:cost}.
We define function $\zeta_{\mathrm{group}}: 2^{[m]} \to [0,1]$ to measure the group fairness:
\begin{align*}
    & ~ \zeta_{\mathrm{group}}(S) := \frac{1}{n}\sum_{i \in [n]} c(a_i,S).
\end{align*}
 \end{definition}

To show the relationship between these two fairness metrics, we have the following proposition:

\begin{proposition}[Relationship of Fairness Metrics, informal version of Proposition~\ref{lem:fair_metric_ineq_append} in Appendix~\ref{sec:fair_proof}]\label{lem:fair_metric_ineq}
    For any solution $S\subseteq \mathcal{P}$ to the submission limit problem in Definition~\ref{def:submit_limit_problem}, we have 
    \begin{align*}
        \zeta_{\mathrm{ind}}(S) \leq \zeta_{\mathrm{group}}(S).
    \end{align*}
\end{proposition}


\subsection{Hardness of Individual Fairness-Aware Submission Limit Problem}\label{sec:indi_fair_hard}

After presenting fairness metrics for the desk-rejection system, we introduce an optimization-based framework to address these metrics. We first study the individual fairness-aware submission limit problem to minimize the individual fairness measure $\zeta_{\mathrm{ind}}$ in Definition~\ref{def:individual_fair}. 


\begin{definition}[Individual Fairness-Aware Submission Limit Problem]\label{def:ind_fair_min}
    We consider the following optimization problem:
\begin{align*}
    & ~ \min_{S \subseteq \mathcal{P}} \zeta_{\mathrm{ind}}(S) \\
    \mathrm{s.t.} & ~ |\{p_j \in S : a_i \in A_j\}| \leq x, \quad \forall a_i \in \mathcal{A}.
\end{align*}
\end{definition}

To represent the fairness metric minimization problem in matrix form, we introduce the following definition:

\begin{definition}[Author-Paper Matrix]\label{def:author_paper_mat}
    Let $W \in \{0, 1\}^{n \times m}$ denote the author-paper matrix for the author set $\mathcal{A}$ and paper set $\mathcal{P}$. Then, we define $W_{i,j} = 1$ if author $a_i$ is a coauthor of paper $p_j$, and $W_{i,j} = 0$ otherwise.
\end{definition}

Therefore, we present a more tractable integer
programming form of the original problem and prove its equivalence to the original
formulation:

\begin{definition}[Individual Fairness-Aware Submission Limit Problem, Matrix Form]\label{def:ind_fair_min_matrix}
    We consider the following integer optimization problem:
    \begin{align*}
        & ~ \min_{r \in \{0,1\}^m} \| \mathbf{1}_n - D^{-1}Wr\|_\infty \\
    \mathrm{s.t.} & ~ 
     (W r) / x \leq \mathbf{1}_n
    \end{align*}
    where $D = \Diag(|P_1|, \cdots, |P_n|)$, and the rejection vector $r \in \{0, 1\}^m$ is a 0-1 vector, with $r_j = 1$ indicating that paper $p_j$ is remained, and $r_j = 0$ indicating that it is desk-rejected. 
\end{definition}

\begin{proposition}[Matrix Form Equivalence for $\zeta_{\mathrm{ind}}$, informal version of Proposition~\ref{prop:equiv_individual_append} in Appendix~\ref{sec:fair_proof}]\label{prop:equiv_individual}
    The individual fairness-aware submission limit problem in Definition~\ref{def:ind_fair_min} and the matrix form integer programming problem in Definition~\ref{def:ind_fair_min_matrix} are equivalent.
\end{proposition}

Unfortunately, solving this integer programming problem is highly non-trivial, which means it may not yield a feasible solution within a reasonable time for large-scale conference submission systems. We establish the computational hardness of this problem in the following theorem:
 
\begin{theorem}[Hardness, informal version of Theorem~\ref{thm:indi_nphard_append} in Appendix~\ref{sec:indi_fair_hard_append}]\label{thm:indi_nphard}
    The Individual Fairness-Aware Submission Limit Problem defined in Definition~\ref{def:ind_fair_min} is $\NPhard$.
\end{theorem}

Since minimizing individual fairness is computationally intractable, our fairness-aware desk-rejection system instead focuses on minimizing group fairness. 


\subsection{Group Fairness Optimization}\label{sec:fair_optim}

Given the inherent hardness of individual fairness optimization, we address the fairness problem using an alternative yet equally important metric: group fairness, as defined in Definition~\ref{def:group_fair}. This metric is not only a crucial fairness measure in its own right but also serves as a lower bound for individual fairness as stated in Proposition~\ref{lem:fair_metric_ineq}, potentially improving individual fairness implicitly. 

Following a similar approach in Section~\ref{sec:indi_fair_hard}, we first formulate the submission limit problem with respect to group fairness and derive a more tractable integer programming formulation in matrix form:

\begin{definition}[Group Fairness-Aware Submission Limit Problem]\label{def:group_fair_min}
    We consider the following optimization problem:
\begin{align*}
    & ~ \min_{S \subseteq \mathcal{P}} \zeta_{\mathrm{group}}(S) \\
    \mathrm{s.t.} & ~ |\{p_j \in S : a_i \in A_j\}| \leq x, \quad \forall a_i \in \mathcal{A}.
\end{align*}
\end{definition}


\begin{definition}[Group Fairness-Aware Submission Limit Problem, Matrix Form]\label{def:group_fair_min_mat_new}
    We consider the following integer programming problem:
    \begin{align*}
        &~ \max_{r \in \{0, 1\}^m} \mathbf{1}^\top_n D^{-1} W r \\ 
        \mathrm{s.t.} 
        & ~ (W r) / x \leq \mathbf{1}_n,
    \end{align*}
    where $D = \Diag(|P_1|, \cdots, |P_n|)$, and the rejection vector $r \in \{0, 1\}^m$ is a 0-1 vector, with $r_j = 1$ indicating that paper $p_j$ is remained, and $r_j = 0$ indicating that it is desk-rejected. 
\end{definition}

\begin{proposition}[Matrix Form Equivalence for $\zeta_{\mathrm{group}}$, informal version of Proposition~\ref{lem:group_fair_min_equiv_append} in Appendix~\ref{sec:fair_proof}]\label{lem:group_fair_min_equiv}
    The fairness-aware submission limit problem in Definition~\ref{def:group_fair_min} and the matrix form integer programming problem in Definition~\ref{def:group_fair_min_mat_new} are equivalent.
\end{proposition}

However, solving integer programming problems is practically challenging. To this end, we first relax the feasible region of $r$ to $[0,1]^m$, and then analyze the resulting relaxed problem.  

\begin{definition}[Group Fairness-Aware Submission Limit Problem, Relaxation]\label{def:group_fair_min_mat_relax_new}
    We consider the optimization problem
    \begin{align*}
    &~ \max_{r \in [0, 1]^m}  \mathbf{1}^\top_nD^{-1}Wr \\ 
        \mathrm{s.t.} 
        & ~ (Wr)/x\le \mathbf{1}_n,
\end{align*}
where $D = \Diag(|P_1|, \cdots, |P_n|)$, and the rejection vector $r \in \{0, 1\}^m$ is a 0-1 vector, with $r_j = 1$ indicating that paper $p_j$ is remained, and $r_j = 0$ indicating that it is desk-rejected. 
\end{definition}

Fortunately, the relaxed problem is a linear program, which can be efficiently solved using standard linear programming solvers. Moreover, its optimal solution is equivalent to that of the original integer programming problem, an this result is formalized in the following theorem:

\begin{theorem}[Optimal Solution Equivalence of the Relaxed Problem, informal version of Theorem~\ref{thm:lp_equiv_append} in Appendix~\ref{sec:fair_proof}]\label{thm:lp_equiv}
    The optimal solution of the relaxed linear programming problem in Definition~\ref{def:group_fair_min_mat_relax_new} is equivalent to the optimal solution of the original integer programming problem in Definition~\ref{def:group_fair_min_mat_new}.
\end{theorem}

\begin{algorithm}[!ht]
\caption{Fairness-Aware Desk-Reject Algorithm}
\label{alg:fair_desk_reject_algo}
\begin{algorithmic}[1]

\State {\color{blue} /* $\mathcal{A}$ denotes the set of $n$ authors. */}
\State {\color{blue} /* $\mathcal{P}$ denote the set of $m$ papers. */}
\State {\color{blue} /* Author $a_i \in \mathcal{N}$ has a subset of papers $P_i \subset \mathcal{P}$. */}
\State {\color{blue} /* Paper $p_j \in \mathcal{P}$ is coauthored by a subset of authors $A_j \subseteq \mathcal{A}$.*/}
\State {\color{blue} /* $x$ represents the submission limit for each author.*/}

\Procedure{FairDeskReject}{$\mathcal{A}, \mathcal{P}, x$} 
\State {\color{blue} /* Initialize the constants of the problem. */}
\For{$i \in [n], j \in [m]$}
\If{$p_j \in \mathcal{A}_i$}
\State $W_{i,j}\gets 1$
\Else
\State $W_{i,j}\gets 0$
\EndIf
\EndFor
\State $D\gets\Diag(|P_1|, \ldots, |P_n|)$
\State {\color{blue} /* Solve the linear programming problem in Definition~\ref{def:group_fair_min_mat_relax_new}. */}
\State $r^{\star} \gets \mathsf{LPSolver}(W, D, x, r^0)$
\State {\color{blue} /* Transform the solution. */}
\State $S\gets\emptyset$
\For {$j\in[m]$}
\If {$r_j = 1$}
\State $S\gets S \cup \{p_j\}$
\EndIf
\EndFor
\State \Return $S$ 
\EndProcedure
\end{algorithmic}
\end{algorithm}

This theoretical result is significant as we formally establish that the group fairness-aware submission problem in Definition~\ref{def:group_fair_min} reduces to a linear programming (LP) problem with guaranteed optimality, solvable using off-the-shelf LP solvers. We formalize this procedure in Algorithm~\ref{alg:fair_desk_reject_algo}, where $\mathsf{LPSolver}$ denotes any standard LP solver, including but not limited to the simplex method~\cite{bg69}, interior-point path-finding methods~\cite{ls14}, and state-of-the-art stochastic central path methods~\cite{cls19, jswz21}.

\begin{remark}
    The time complexity of our fairness-aware desk-rejection algorithm in Algorithm~\ref{alg:fair_desk_reject_algo} aligns with modern linear programming solvers. For instance, using the stochastic central path method~\cite{cls21,jswz21,vlss20,sy21}, it achieves a time complexity of $O^*(m^{2.37} \log(m/\delta))$, where $\delta$ represents the relative accuracy corresponding to a \((1+\delta)\)-approximation guarantee.
\end{remark}

\begin{remark}
    In practice, major AI conferences routinely process submissions at the scale of $m \sim 10^4$~\cite{stanford_ai_index}. Given this regime, our algorithm guarantees efficient computation, enabling fairness-aware desk rejection within tractable timeframes, even for large-scale conferences.
\end{remark}

\subsection{Analysis}


\begin{table}[h]
    \centering
    \small
    \begin{tabular}{lccc}
        \toprule
        Score & cos $\theta$ &\# of Generated EX & \%  Filtered EX \\
        \midrule
        \textbf{$\geq 0$} &0.581& 10340 & 0 \% \\ 
        \textbf{$\geq 2$} &0.625& 10185  & 1.50\% (-155) \\
        \textbf{$\geq 4$} &0.744& 9883 & 4.41\% (-457)  \\
        \textbf{$\geq 6$} &0.762 & 9378 & 9.30\% (-962)  \\
        \textbf{$\geq 8$}&0.765& 8606 & 16.76\% (-1734)\\
        \textbf{$\geq 10$} &0.769& 6795 & 34.28\% (-3545)  \\
        \bottomrule
    \end{tabular}
    %\caption{A summary of the data generation and filtering result, along with an embedding similarity analysis of the filtered examples, categorized by their respective scores.}
\caption{Summary of data generation, filtering results, and embedding similarity analysis by score.}
    \label{tab:number_of_generated}
    % \vspace{-4mm}
\end{table}

% & 0.581          & 0.625            &  0.744         & 0.762          & 0.765    &  \textbf{0.769}  

\begin{figure*}[t]
\centerline{\includegraphics[scale=0.48]{Pictures/corr_bin.pdf}}

\caption{(Left) Correlation between question embedding similarity and average EX, (Right) Average EX across embedding similarity bins}
% \vspace{-4mm}
\label{fig:corr_bin}
\end{figure*}

\paragraph{Number of generated and filtered examples per score, along with an embedding similarity analysis of the filtered examples}
For each test question in the Spider dev set, 10 examples are generated, resulting in a total of 10,340 examples. The quality of these examples is assessed using a relevance score ranging from 0 to 10. As shown in Table~\ref{tab:number_of_generated}, the 65.71\% of examples are assigned a score of 10, while the 0.59\% of examples are received a score of 0. This trend suggests that the LLM tends to assign high relevance to its own generated examples. The similarity is computed using cosine similarity, where higher scores indicate greater semantic alignment between the test questions and the retained examples. As the filtering threshold increases, the embedding similarity also increases, suggesting that higher-relevance examples exhibit stronger semantic consistency with the test questions. However, we also observe that overly strict filtering—selecting only examples with a perfect score of 10—leads to a decline in performance. This drop occurs because an excessively high threshold significantly reduces the number of available examples, limiting the diversity.


\paragraph{Effect of question embedding similarity on Execution Accuracy.}
In Figure~\ref{fig:corr_bin}, the left graph illustrates the correlation between embedding similarity and EX. Each point represents one of the 11 data points obtained by filtering examples based on different threshold scores (0 to 10). The data points follow an upward trend, suggesting that higher similarity tends to result in better EX. The red line indicates the overall correlation, with a coefficient of 0.82, showing a relatively strong positive relationship. Building on this analysis, the right graph provides a more fine-grained view by examining the execution accuracy of individual generated examples based on their embedding similarity with test questions. The x-axis represents the normalized similarity between the test question and the generated question, and the y-axis indicates EX. The results show that EX is lowest in the 0.0-0.1 similarity range, suggesting that examples with very low similarity to test questions tend to be less useful. As similarity increases, EX generally improves, peaking in the 0.7-0.8 range. This suggests that examples with a moderate to high similarity to test questions are more effective in generating executable SQL queries. However, accuracy drops slightly in the 0.8-0.9 range before rising again in the 0.9-1.0 range. This indicates that excessively high similarity can reduce diversity, potentially limiting the model’s generalization ability. 


\begin{figure}[t]
\centerline{\includegraphics[scale=0.36]{Pictures/Diff_threshold_GPT4o.png}}
\caption{Performance of GPT-4o at different relevance score thresholds.}
% \vspace{-5mm}
\label{tab:diff_thres}
\end{figure}


\paragraph{Effect of Relevance Scoring Thresholds on Performance.}

To further evaluate the effectiveness of SAFE-SQL, we conduct a detailed case study using varying thresholds for the relevance scoring mechanism as shown in Figure~\ref{tab:diff_thres}.  The self-generated examples are filtered based on relevance scores, with thresholds ranging from 0 to 10. For each test question, the number of high-scoring examples varied due to the specific content and schema structure (e.g., some test questions had six examples with scores $\geq 8$, while others had three). The selected examples are then used during the final inference stage to generate SQL queries. The $\geq 8$ threshold consistently produced the best results, validating the robustness of SAFE-SQL’s relevance score filtering. The results demonstrate that selecting high-quality examples plays a critical role in guiding LLMs to generate accurate SQL queries, regardless of the underlying model.


\begin{comment}
\begin{table*}[h]
    \centering
    \renewcommand{\arraystretch}{1.3}  % 행 간격 조정
    \begin{tabularx}{\textwidth}{p{4cm} p{6cm} p{4cm} p{6cm}}
        \toprule
        \textbf{Original SQL Question} & \textbf{Original SQL Query} & \textbf{Generated SQL Question} & \textbf{Generated Reasoning Path} \\
        \midrule
        What are all the flights that leave from Aberdeen? & 
        \lstinline|SELECT * FROM flights WHERE departure_city = 'Aberdeen'| & 
        What are all the flights departing from Aberdeen? & 
        Identify all flights with Aberdeen as the departure city. \\
        
        Of those, which land in Ashley? & 
        \lstinline|SELECT * FROM flights WHERE departure_city = 'Aberdeen' AND arrival_city = 'Ashley'| & 
        Which flights leave from Aberdeen and land in Ashley? & 
        Filter previous results to include only flights arriving in Ashley. \\
        
        How many are there? & 
        \lstinline|SELECT COUNT(*) FROM flights WHERE departure_city = 'Aberdeen' AND arrival_city = 'Ashley'| & 
        How many flights travel from Aberdeen to Ashley? & 
        Count the number of flights from the filtered list. \\
        \midrule
        
        What are all the airlines? & 
        \lstinline|SELECT DISTINCT airline FROM flights| & 
        What airlines operate flights? & 
        Retrieve distinct airline names from the flights table. \\
        
        Of these, which is JetBlue Airways? & 
        \lstinline|SELECT * FROM flights WHERE airline = 'JetBlue Airways'| & 
        Which flights are operated by JetBlue Airways? & 
        Filter flights to include only those operated by JetBlue Airways. \\
        
        What is the country corresponding it? & 
        \lstinline|SELECT country FROM airlines WHERE name = 'JetBlue Airways'| & 
        What country is JetBlue Airways based in? & 
        Retrieve the country associated with JetBlue Airways from the airlines table. \\
        \bottomrule
    \end{tabularx}
    \caption{Examples of original and generated SQL questions with reasoning paths.}
    \label{tab:sql_examples}
\end{table*}
\end{comment}

\begin{comment}
\begin{table*}[h]
    \centering
    \small
    \renewcommand{\arraystretch}{1.3}  % Adjust row spacing
    \begin{tabularx}{\textwidth}{X X X X X}
        \toprule
        \textbf{SQL Question} & \textbf{GOLD SQL Query} & \textbf{Augmented SQL Question} & \textbf{Generated Reasoning Path} & \textbf{Relevance Score} \\
        \midrule
        \hl{Question1:}
        What are the names, countries, and ages for every singer in descending order of age? & 
        \texttt{SELECT name, country, age FROM singer ORDER BY age DESC} & 
        \sethlcolor{lime!50}
        \hl{What are the names, ages, and countries of all singers from a specific country, sorted by age in descending order?} & 
        \sethlcolor{violet!20}
        \hl{1.Identify the desired columns: name, age, and country. 
        2.Specify the table: singer. 
        3.Sort the results by age in descending order.}& semantic similarity:3   Structure \& key word 
 score: 3  Reasoning patt score:4 Relevance score = 10
        \\
        \midrule
        \hl{Question2:}
        What is the number of car models that are produced by each maker and what is the id and full name of each maker?
        &  
        \texttt{SELECT Count(*), T2.FullName , T2.id FROM MODEL\_LIST AS T1 JOIN CAR\_MAKERS AS T2 ON T1.Maker = T2.id GROUP BY T2.id;} & 
               \sethlcolor{lime!50}
 \hl{Could you provide the count of car models produced by each manufacturer, along with the ID and full name of each manufacturer?} & 
 \sethlcolor{violet!20}
 \hl{1.Retrieve Required Information: Count car models per maker and get each maker's ID and full name. 2.Join Tables: Link MODEL\_LIST (T1) and CAR\_MAKERS (T2) using T1.Maker = T2.Id. 3.Group and Aggregate: Use COUNT(*) to count models and group by T2.id. 4.Select Output: Return the model count, maker’s full name, and ID.} & semantic similarity:1   Structure \& key word 
 score: 2  Reasoning patt score:3 Relevance score = 6 \\ 
        \midrule
        \hl{Question3:} Return the names and template ids for documents that contain the letter w in their description. & 
        \texttt{SELECT document\_name , template\_id FROM Documents WHERE Document\_Description LIKE "\%w\%"} & 
        \sethlcolor{lime!50}
        \hl{Retrieve the names and template IDs of documents whose descriptions include the letter 'w'.} & 
        \sethlcolor{violet!20}
        \hl{1.Extract the document\_name and template\_id columns.
        2.Search within the Documents table. 3.Filter results where Document\_Description contains the letter "w".
        Return the matching records.} & semantic similarity:3   Structure \& key word 
 score: 2  Reasoning path score:3 Relevance score = 8  \\

        \bottomrule
    \end{tabularx}
    \caption{Examples of original and Augmented SQL questions with reasoning paths by GPT-4o mini.}
    \label{tab:sql_examples}
\end{table*}
\end{comment}

\begin{table*}[t]
    \centering
    \scriptsize
    \renewcommand{\arraystretch}{1.2}
    \resizebox{1.97\columnwidth}{!}{% Adjust row spacing
    \begin{tabularx}{\textwidth}{X X X >{\raggedright\arraybackslash}p{0.3\textwidth} X}
        \toprule
        \textbf{SQL Question} & \textbf{GOLD SQL Query} & \textbf{Augmented SQL Question} & \textbf{Generated Reasoning Path} & \textbf{Relevance Score} \\
        \midrule
        \hl{Question1:} What are the names, countries, and ages for every singer in descending order of age? & 
        \texttt{SELECT name, country, age FROM singer ORDER BY age DESC} & 
        \sethlcolor{lime!50}\hl{What are the names, ages, and countries of all singers from a specific country, sorted by age in descending order?} & 
        \sethlcolor{violet!20}\hl{1. Identify the desired columns: name, age, and country. \\
        2. Specify the table: singer. \\
        3. Sort the results by age in descending order.} &  
        Semantic similarity = 10 
        Structural Similarity = 10  
        Reasoning path = 10  
        Relevance score = (10+10+10)/3 = 10
                \\ 
        \midrule
        \hl{Question2:} Return the names and template ids for documents that contain the letter w in their description. & 
        \texttt{SELECT document\_name, template\_id FROM Documents WHERE Document\_Description LIKE "\%w\%"} & 
        \sethlcolor{lime!50}\hl{Retrieve the titles and category IDs of articles whose summaries contain the word "data".} &
        \sethlcolor{violet!20}\hl{1. Identify the necessary columns: extract title and category\_id from the Articles table. \\
         2. Locate the relevant table: use the Articles table as it contains the required data. \\
        3. Define the filtering condition: apply a WHERE clause to check if the summary column contains the substring "data". \\
         4. Use the LIKE operator: employ LIKE '\%data\%' to search for any instance of "data" within the summary. \\
        5. Retrieve the results: return the title and category\_id values for all matching records.} & Semantic similarity = 7
                    Structural Similarity = 9
Reasoning path = 8
Relevance score = (7+9+8)/3 = 8
        \\
        \midrule
        \hl{Question3:} What is the number of car models that are produced by each maker and what is the id and full name of each maker? &  
        \texttt{SELECT Count(*), T2.FullName, T2.id FROM MODEL\_LIST AS T1 JOIN CAR\_MAKERS AS T2 ON T1.Maker = T2.id GROUP BY T2.id;} & 
        \sethlcolor{lime!50}\hl{List all employees who work in the IT department along with their employee ID and hire date.} & 
        \sethlcolor{violet!20}\hl{1. Identify required details: employee ID and hire date. \\
        2. Filter condition: find employees who work in IT. \\
        3. Retrieve data: select only emp\_id and hire\_date.} & 
        Semantic similarity = 6
        Structural Similarity = 3
        Reasoning path = 2
        Relevance score = (6+3+2)/3 = 3.67
  \\
        \bottomrule
    \end{tabularx}
    }
    %\vspace{-2mm}
    \caption{Examples of original and augmented SQL questions with reasoning paths by GPT-4o.}
    \label{tab:sql_examples}
    \vspace{-4mm}
\end{table*}


%This experiment is performed across multiple models, including GPT 4o Mini, Deepseek Coder 6.7B, %Llama 3.1 8B, and Starcoder 7B.
\begin{table}[t]
    \centering
    \small
    \resizebox{0.48\textwidth}{!}{
    \begin{tabular}{lcc||ccccc}
        \toprule
        \textbf{$\alpha$} & \textbf{$\beta$} &\textbf{$\gamma$}& \textbf{Easy}& \textbf{Medium}& \textbf{Hard} &\textbf{Extra}& \textbf{EX} \\
        \midrule
        0.33 & 0.33 & 0.33 & \textbf{93.4} & \textbf{89.3} & \underline{88.4} & \textbf{75.8} & \textbf{87.9} \\ 
        \midrule
        1 & 0 & 0 & 90.7& 84.2& 82.3& 68.3&  82.8 \\ 
        0 & 1 & 0 & 89.8& 85.6& 81.2& 69.2&  83.1 \\ 
        0 & 0 & 1 & 89.2& 85.1& 84.3& 71.7& 83.7  \\ 
        \midrule
        0.5 & 0.5 & 0& 91.2& 87.3& 82.5& 69.4& 84.4 \\ 
        0.5 & 0 & 0.5& 92.5& \underline{87.9}& 83.5& 70.3& 85.3 \\ 
        0 & 0.5 & 0.5& \underline{92.7}& 86.8& \textbf{88.5}& \underline{72.4}& \underline{86.1} \\ 
        \bottomrule
    \end{tabular}
    }
    %\vspace{-2mm}
    %\caption{Execution accuracy across different difficulty levels with varying weights of semantic similarity ($\alpha$), keyword \& structural similarity ($\beta$), and reasoning path quality ($\gamma$).}
    \caption{Execution accuracy across difficulty levels under different weights: semantic similarity ($\alpha$), Structural similarity ($\beta$), and reasoning path quality ($\gamma$).}
    % \vspace{-4mm}
    \label{tab:filtering_score_ablation}
\end{table} 

\paragraph{Effect of three measuring components on Performance.}

To assess the impact of the three measuring components—semantic similarity ($\alpha$), keyword \& structural similarity ($\beta$), and reasoning path quality ($\gamma$)—on EX, we conduct experiments by varying their respective weightings. The results, presented in Table~\ref{tab:filtering_score_ablation}, highlight distinct performance trends across different difficulty levels. Notably, the exclusion of reasoning path quality leads to a drop in EX, particularly in the Hard and Extra Hard. This suggests that a well-structured reasoning path is crucial for handling complex queries, as it provides essential logical steps that bridge the gap between natural language understanding and SQL formulation. Conversely, semantic similarity and structural SQL query similarity have a greater influence on the Easy and Medium levels. This is because these queries tend to be relatively straightforward, meaning that having structurally similar SQL questions in the example set often provides sufficient guidance for generating correct queries. In these cases, direct pattern matching and schema alignment play a larger role. Overall, the findings demonstrate that a balanced combination of all three components is essential for optimizing performance across different levels of query complexity.

% Simply maximizing similarity may not always yield the best results, and a balanced approach that considers both relevance and diversity could be more effective.


%: Qwen2.5-3B-instruct, Qwen2.5-7B-instruct, and Qwen2.5-14B-instruct 


\subsection{Case Study}
As shown in Table~\ref{tab:sql_examples}, test questions from the Spider dev set alongside their generated similar examples, evaluated based on semantic similarity, structural similarity, and the reasoning path score, which together determine the relevance score. The first example achieves a perfect relevance score of 10, as the generated question closely aligns with the original in meaning, structure, and reasoning. The SQL formulation remains nearly identical, and the reasoning path explicitly details each step, ensuring full alignment. The second example receives a relevance score of 8, with semantic similarity of 7 due to minor differences in terminology ("documents" vs. "articles" and "letter 'w'" vs. "word 'data'"). However, its structural similarity remains high, as the SQL structure is nearly identical. The reasoning path score of 8 reflects a clear explanation of query formulation, though slightly less detailed than the first example. The third example has the lowest relevance score due to significant differences. The generated question shifts focus from counting car models to listing IT employees, resulting in semantic similarity of 6 and structural similarity of 3. These results emphasize the importance of fine-grained example selection due to the varing quality of generated examples.
% Table 6번 언급되는 곳이 하나도 없었습니다. 맨뒤로 빼고 Case Study 만들어서 설명할 필요가 있습니다. 또한 Relevance Score 변경했는데 확인해주셔야합니다
\section{Discussion}\label{sec:discussion}



\subsection{From Interactive Prompting to Interactive Multi-modal Prompting}
The rapid advancements of large pre-trained generative models including large language models and text-to-image generation models, have inspired many HCI researchers to develop interactive tools to support users in crafting appropriate prompts.
% Studies on this topic in last two years' HCI conferences are predominantly focused on helping users refine single-modality textual prompts.
Many previous studies are focused on helping users refine single-modality textual prompts.
However, for many real-world applications concerning data beyond text modality, such as multi-modal AI and embodied intelligence, information from other modalities is essential in constructing sophisticated multi-modal prompts that fully convey users' instruction.
This demand inspires some researchers to develop multimodal prompting interactions to facilitate generation tasks ranging from visual modality image generation~\cite{wang2024promptcharm, promptpaint} to textual modality story generation~\cite{chung2022tale}.
% Some previous studies contributed relevant findings on this topic. 
Specifically, for the image generation task, recent studies have contributed some relevant findings on multi-modal prompting.
For example, PromptCharm~\cite{wang2024promptcharm} discovers the importance of multimodal feedback in refining initial text-based prompting in diffusion models.
However, the multi-modal interactions in PromptCharm are mainly focused on the feedback empowered the inpainting function, instead of supporting initial multimodal sketch-prompt control. 

\begin{figure*}[t]
    \centering
    \includegraphics[width=0.9\textwidth]{src/img/novice_expert.pdf}
    \vspace{-2mm}
    \caption{The comparison between novice and expert participants in painting reveals that experts produce more accurate and fine-grained sketches, resulting in closer alignment with reference images in close-ended tasks. Conversely, in open-ended tasks, expert fine-grained strokes fail to generate precise results due to \tool's lack of control at the thin stroke level.}
    \Description{The comparison between novice and expert participants in painting reveals that experts produce more accurate and fine-grained sketches, resulting in closer alignment with reference images in close-ended tasks. Novice users create rougher sketches with less accuracy in shape. Conversely, in open-ended tasks, expert fine-grained strokes fail to generate precise results due to \tool's lack of control at the thin stroke level, while novice users' broader strokes yield results more aligned with their sketches.}
    \label{fig:novice_expert}
    % \vspace{-3mm}
\end{figure*}


% In particular, in the initial control input, users are unable to explicitly specify multi-modal generation intents.
In another example, PromptPaint~\cite{promptpaint} stresses the importance of paint-medium-like interactions and introduces Prompt stencil functions that allow users to perform fine-grained controls with localized image generation. 
However, insufficient spatial control (\eg, PromptPaint only allows for single-object prompt stencil at a time) and unstable models can still leave some users feeling the uncertainty of AI and a varying degree of ownership of the generated artwork~\cite{promptpaint}.
% As a result, the gap between intuitive multi-modal or paint-medium-like control and the current prompting interface still exists, which requires further research on multi-modal prompting interactions.
From this perspective, our work seeks to further enhance multi-object spatial-semantic prompting control by users' natural sketching.
However, there are still some challenges to be resolved, such as consistent multi-object generation in multiple rounds to increase stability and improved understanding of user sketches.   


% \new{
% From this perspective, our work is a step forward in this direction by allowing multi-object spatial-semantic prompting control by users' natural sketching, which considers the interplay between multiple sketch regions.
% % To further advance the multi-modal prompting experience, there are some aspects we identify to be important.
% % One of the important aspects is enhancing the consistency and stability of multiple rounds of generation to reduce the uncertainty and loss of control on users' part.
% % For this purpose, we need to develop techniques to incorporate consistent generation~\cite{tewel2024training} into multi-modal prompting framework.}
% % Another important aspect is improving generative models' understanding of the implicit user intents \new{implied by the paint-medium-like or sketch-based input (\eg, sketch of two people with their hands slightly overlapping indicates holding hand without needing explicit prompt).
% % This can facilitate more natural control and alleviate users' effort in tuning the textual prompt.
% % In addition, it can increase users' sense of ownership as the generated results can be more aligned with their sketching intents.
% }
% For example, when users draw sketches of two people with their hands slightly overlapping, current region-based models cannot automatically infer users' implicit intention that the two people are holding hands.
% Instead, they still require users to explicitly specify in the prompt such relationship.
% \tool addresses this through sketch-aware prompt recommendation to fill in the necessary semantic information, alleviating users' workload.
% However, some users want the generative AI in the future to be able to directly infer this natural implicit intentions from the sketches without additional prompting since prompt recommendation can still be unstable sometimes.


% \new{
% Besides visual generation, 
% }
% For example, one of the important aspect is referring~\cite{he2024multi}, linking specific text semantics with specific spatial object, which is partly what we do in our sketch-aware prompt recommendation.
% Analogously, in natural communication between humans, text or audio alone often cannot suffice in expressing the speakers' intentions, and speakers often need to refer to an existing spatial object or draw out an illustration of her ideas for better explanation.
% Philosophically, we HCI researchers are mostly concerned about the human-end experience in human-AI communications.
% However, studies on prompting is unique in that we should not just care about the human-end interaction, but also make sure that AI can really get what the human means and produce intention-aligned output.
% Such consideration can drastically impact the design of prompting interactions in human-AI collaboration applications.
% On this note, although studies on multi-modal interactions is a well-established topic in HCI community, it remains a challenging problem what kind of multi-modal information is really effective in helping humans convey their ideas to current and next generation large AI models.




\subsection{Novice Performance vs. Expert Performance}\label{sec:nVe}
In this section we discuss the performance difference between novice and expert regarding experience in painting and prompting.
First, regarding painting skills, some participants with experience (4/12) preferred to draw accurate and fine-grained shapes at the beginning. 
All novice users (5/12) draw rough and less accurate shapes, while some participants with basic painting skills (3/12) also favored sketching rough areas of objects, as exemplified in Figure~\ref{fig:novice_expert}.
The experienced participants using fine-grained strokes (4/12, none of whom were experienced in prompting) achieved higher IoU scores (0.557) in the close-ended task (0.535) when using \tool. 
This is because their sketches were closer in shape and location to the reference, making the single object decomposition result more accurate.
Also, experienced participants are better at arranging spatial location and size of objects than novice participants.
However, some experienced participants (3/12) have mentioned that the fine-grained stroke sometimes makes them frustrated.
As P1's comment for his result in open-ended task: "\emph{It seems it cannot understand thin strokes; even if the shape is accurate, it can only generate content roughly around the area, especially when there is overlapping.}" 
This suggests that while \tool\ provides rough control to produce reasonably fine results from less accurate sketches for novice users, it may disappoint experienced users seeking more precise control through finer strokes. 
As shown in the last column in Figure~\ref{fig:novice_expert}, the dragon hovering in the sky was wrongly turned into a standing large dragon by \tool.

Second, regarding prompting skills, 3 out of 12 participants had one or more years of experience in T2I prompting. These participants used more modifiers than others during both T2I and R2I tasks.
Their performance in the T2I (0.335) and R2I (0.469) tasks showed higher scores than the average T2I (0.314) and R2I (0.418), but there was no performance improvement with \tool\ between their results (0.508) and the overall average score (0.528). 
This indicates that \tool\ can assist novice users in prompting, enabling them to produce satisfactory images similar to those created by users with prompting expertise.



\subsection{Applicability of \tool}
The feedback from user study highlighted several potential applications for our system. 
Three participants (P2, P6, P8) mentioned its possible use in commercial advertising design, emphasizing the importance of controllability for such work. 
They noted that the system's flexibility allows designers to quickly experiment with different settings.
Some participants (N = 3) also mentioned its potential for digital asset creation, particularly for game asset design. 
P7, a game mod developer, found the system highly useful for mod development. 
He explained: "\emph{Mods often require a series of images with a consistent theme and specific spatial requirements. 
For example, in a sacrifice scene, how the objects are arranged is closely tied to the mod's background. It would be difficult for a developer without professional skills, but with this system, it is possible to quickly construct such images}."
A few participants expressed similar thoughts regarding its use in scene construction, such as in film production. 
An interesting suggestion came from participant P4, who proposed its application in crime scene description. 
She pointed out that witnesses are often not skilled artists, and typically describe crime scenes verbally while someone else illustrates their account. 
With this system, witnesses could more easily express what they saw themselves, potentially producing depictions closer to the real events. "\emph{Details like object locations and distances from buildings can be easily conveyed using the system}," she added.

% \subsection{Model Understanding of Users' Implicit Intents}
% In region-sketch-based control of generative models, a significant gap between interaction design and actual implementation is the model's failure in understanding users' naturally expressed intentions.
% For example, when users draw sketches of two people with their hands slightly overlapping, current region-based models cannot automatically infer users' implicit intention that the two people are holding hands.
% Instead, they still require users to explicitly specify in the prompt such relationship.
% \tool addresses this through sketch-aware prompt recommendation to fill in the necessary semantic information, alleviating users' workload.
% However, some users want the generative AI in the future to be able to directly infer this natural implicit intentions from the sketches without additional prompting since prompt recommendation can still be unstable sometimes.
% This problem reflects a more general dilemma, which ubiquitously exists in all forms of conditioned control for generative models such as canny or scribble control.
% This is because all the control models are trained on pairs of explicit control signal and target image, which is lacking further interpretation or customization of the user intentions behind the seemingly straightforward input.
% For another example, the generative models cannot understand what abstraction level the user has in mind for her personal scribbles.
% Such problems leave more challenges to be addressed by future human-AI co-creation research.
% One possible direction is fine-tuning the conditioned models on individual user's conditioned control data to provide more customized interpretation. 

% \subsection{Balance between recommendation and autonomy}
% AIGC tools are a typical example of 
\subsection{Progressive Sketching}
Currently \tool is mainly aimed at novice users who are only capable of creating very rough sketches by themselves.
However, more accomplished painters or even professional artists typically have a coarse-to-fine creative process. 
Such a process is most evident in painting styles like traditional oil painting or digital impasto painting, where artists first quickly lay down large color patches to outline the most primitive proportion and structure of visual elements.
After that, the artists will progressively add layers of finer color strokes to the canvas to gradually refine the painting to an exquisite piece of artwork.
One participant in our user study (P1) , as a professional painter, has mentioned a similar point "\emph{
I think it is useful for laying out the big picture, give some inspirations for the initial drawing stage}."
Therefore, rough sketch also plays a part in the professional artists' creation process, yet it is more challenging to integrate AI into this more complex coarse-to-fine procedure.
Particularly, artists would like to preserve some of their finer strokes in later progression, not just the shape of the initial sketch.
In addition, instead of requiring the tool to generate a finished piece of artwork, some artists may prefer a model that can generate another more accurate sketch based on the initial one, and leave the final coloring and refining to the artists themselves.
To accommodate these diverse progressive sketching requirements, a more advanced sketch-based AI-assisted creation tool should be developed that can seamlessly enable artist intervention at any stage of the sketch and maximally preserve their creative intents to the finest level. 

\subsection{Ethical Issues}
Intellectual property and unethical misuse are two potential ethical concerns of AI-assisted creative tools, particularly those targeting novice users.
In terms of intellectual property, \tool hands over to novice users more control, giving them a higher sense of ownership of the creation.
However, the question still remains: how much contribution from the user's part constitutes full authorship of the artwork?
As \tool still relies on backbone generative models which may be trained on uncopyrighted data largely responsible for turning the sketch into finished artwork, we should design some mechanisms to circumvent this risk.
For example, we can allow artists to upload backbone models trained on their own artworks to integrate with our sketch control.
Regarding unethical misuse, \tool makes fine-grained spatial control more accessible to novice users, who may maliciously generate inappropriate content such as more realistic deepfake with specific postures they want or other explicit content.
To address this issue, we plan to incorporate a more sophisticated filtering mechanism that can detect and screen unethical content with more complex spatial-semantic conditions. 
% In the future, we plan to enable artists to upload their own style model

% \subsection{From interactive prompting to interactive spatial prompting}


\subsection{Limitations and Future work}

    \textbf{User Study Design}. Our open-ended task assesses the usability of \tool's system features in general use cases. To further examine aspects such as creativity and controllability across different methods, the open-ended task could be improved by incorporating baselines to provide more insightful comparative analysis. 
    Besides, in close-ended tasks, while the fixing order of tool usage prevents prior knowledge leakage, it might introduce learning effects. In our study, we include practice sessions for the three systems before the formal task to mitigate these effects. In the future, utilizing parallel tests (\textit{e.g.} different content with the same difficulty) or adding a control group could further reduce the learning effects.

    \textbf{Failure Cases}. There are certain failure cases with \tool that can limit its usability. 
    Firstly, when there are three or more objects with similar semantics, objects may still be missing despite prompt recommendations. 
    Secondly, if an object's stroke is thin, \tool may incorrectly interpret it as a full area, as demonstrated in the expert results of the open-ended task in Figure~\ref{fig:novice_expert}. 
    Finally, sometimes inclusion relationships (\textit{e.g.} inside) between objects cannot be generated correctly, partially due to biases in the base model that lack training samples with such relationship. 

    \textbf{More support for single object adjustment}.
    Participants (N=4) suggested that additional control features should be introduced, beyond just adjusting size and location. They noted that when objects overlap, they cannot freely control which object appears on top or which should be covered, and overlapping areas are currently not allowed.
    They proposed adding features such as layer control and depth control within the single-object mask manipulation. Currently, the system assigns layers based on color order, but future versions should allow users to adjust the layer of each object freely, while considering weighted prompts for overlapping areas.

    \textbf{More customized generation ability}.
    Our current system is built around a single model $ColorfulXL-Lightning$, which limits its ability to fully support the diverse creative needs of users. Feedback from participants has indicated a strong desire for more flexibility in style and personalization, such as integrating fine-tuned models that cater to specific artistic styles or individual preferences. 
    This limitation restricts the ability to adapt to varied creative intents across different users and contexts.
    In future iterations, we plan to address this by embedding a model selection feature, allowing users to choose from a variety of pre-trained or custom fine-tuned models that better align with their stylistic preferences. 
    
    \textbf{Integrate other model functions}.
    Our current system is compatible with many existing tools, such as Promptist~\cite{hao2024optimizing} and Magic Prompt, allowing users to iteratively generate prompts for single objects. However, the integration of these functions is somewhat limited in scope, and users may benefit from a broader range of interactive options, especially for more complex generation tasks. Additionally, for multimodal large models, users can currently explore using affordable or open-source models like Qwen2-VL~\cite{qwen} and InternVL2-Llama3~\cite{llama}, which have demonstrated solid inference performance in our tests. While GPT-4o remains a leading choice, alternative models also offer competitive results.
    Moving forward, we aim to integrate more multimodal large models into the system, giving users the flexibility to choose the models that best fit their needs. 
    


\section{Conclusion}\label{sec:conclusion}
In this paper, we present \tool, an interactive system designed to help novice users create high-quality, fine-grained images that align with their intentions based on rough sketches. 
The system first refines the user's initial prompt into a complete and coherent one that matches the rough sketch, ensuring the generated results are both stable, coherent and high quality.
To further support users in achieving fine-grained alignment between the generated image and their creative intent without requiring professional skills, we introduce a decompose-and-recompose strategy. 
This allows users to select desired, refined object shapes for individual decomposed objects and then recombine them, providing flexible mask manipulation for precise spatial control.
The framework operates through a coarse-to-fine process, enabling iterative and fine-grained control that is not possible with traditional end-to-end generation methods. 
Our user study demonstrates that \tool offers novice users enhanced flexibility in control and fine-grained alignment between their intentions and the generated images.


\ifdefined\isarxiv
%\section*{Acknowledgments}
\bibliographystyle{alpha}
\bibliography{ref} 
\else
\bibliography{ref}
\bibliographystyle{icml2025}
% \bibliographystyle{alpg mha}

\fi



\newpage
\onecolumn
\appendix




%%%% Cut-line between first 10 pages and appendix
%\input{11_backup_for_pos_results}
\clearpage
\newpage
\begin{center}
    \textbf{\LARGE Appendix}
\end{center}

\paragraph{Roadmap.} In Section~\ref{sec:dilemma_proof}, we supplement the missing proofs in Section~\ref{sec:dr_dilemma}. In Section~\ref{sec:fair_proof}, we present the missing proofs in Section~\ref{sec:fair}. In Section~\ref{sec:more_case_study}, we show additional case studies. In Section~\ref{app:sec:conference_links}, we provide the details related to conference submission limits. 

\section{Missing Proofs in Section 4} \label{sec:dilemma_proof}

In this section, we provide the complete technical proofs for Theorem~\ref{thm:main_res_general} in Section~\ref{sec:dr_dilemma}. 
In Section~\ref{sec:dilemma_proof_defs}, we first introduce key definitions that will be useful 
To structure our analysis, we in the subsequent proofs. We then establish positive results for the cases where $n \leq 2$ in Section~\ref{sec:positive_results}, followed by negative results for $n \geq 3$ in Section~\ref{sec:negative_results}.


\subsection{Basic Definitions}\label{sec:dilemma_proof_defs}

To systematically analyze the desk-rejection problem, we begin by classifying authors based on their submission behavior and their relationship to co-authors. This classification will help us organize and present the proofs in a more structured and readable manner.

\begin{definition}[Author Categories]\label{def:three_kinds_authors}
For any author $a_i \in \mathcal{A}$, we define the following categories:
\begin{itemize}
    \item \textbf{Non-compliant}: An author $a_i$ is non-compliant if they have submitted more than $x$ papers, i.e., $|P_i| > x$. Such authors exceed the submission limit and are subject to desk-rejection under the policy.

    \item \textbf{Vulnerable}: An author $a_i$ is vulnerable if they have submitted no more than $x$ papers ($|P_i| \leq x$) but have at least one non-compliant co-author, i.e., $\exists k \in C_i$ such that $|P_k| > x$. Although these authors comply with the submission limit, they are at risk of being unfairly penalized due to their co-authors' non-compliance.

    \item \textbf{Safe}: An author $a_i$ is safe if they have submitted no more than $x$ papers ($|P_i| \leq x$) and all their co-authors are also compliant, i.e., $\forall k \in C_i$, $|P_k| \leq x$. These authors are guaranteed to retain all their submissions, as neither they nor their co-authors violate the submission limit.
\end{itemize}
\end{definition}

Next, we formalize the notion of achievability for the ideal desk-rejection system.

\begin{definition}[Achievability]\label{def:achievability}
Given a submission limit problem instance as defined in Definition~\ref{def:submit_limit_problem}:
\begin{itemize}
    \item \textbf{Positive result}: A problem instance is a positive result if there exists an algorithm that can achieve the ideal desk-rejection as defined in Definition~\ref{def:good_solution}.
    
    \item \textbf{Negative result}: A problem instance is a negative result if,  under proper conditions, no algorithm can achieve the ideal desk-rejection as defined in Definition~\ref{def:good_solution}.
\end{itemize}
\end{definition}

In the following sections, we will use these definitions to systematically prove the positive results for small numbers of authors ($n \leq 2$) and the negative results for larger numbers of authors ($n \geq 3$), which covers two cases in Theorem~\ref{thm:main_res_general}.


\subsection{Positive Results} \label{sec:positive_results}
In this subsection, we present two positive results that support the $n \leq 2$ case in Theorem~\ref{thm:main_res_general}. We begin with the positive result for $n = 1$ and any $x \in \mathbb{N}_+$.

\begin{lemma}[Positive result for $n = 1$ and any $x \in \mathbb{N}_+$, general case]\label{lem:n_eq_1_positive_general}
    If the following conditions hold:
    \begin{itemize}
        \item Let $n = 1$ denote the number of authors as defined in Definition~\ref{def:submit_limit_problem}.
        \item Let $x \in \mathbb{N}_+$ denote the maximum number of submissions allowed for each author in the conference.
    \end{itemize}
    Then, there exists an algorithm that achieves the ideal desk-rejection as defined in Definition~\ref{def:good_solution}.
\end{lemma}

\begin{proof}
We consider the three cases for the only author $a_1$: non-compliant, vulnerable, and safe, as defined in Definition~\ref{def:three_kinds_authors}.

\paragraph{Case 1: Non-compliant author.} If author $a_1$ is non-compliant, we desk-reject $(|P_1| - x)$ papers. This ensures that exactly $x$ papers remain, satisfying the ideal desk-rejection condition.

\paragraph{Case 2: Vulnerable author.} Since $n = 1$ and there is only one author, author $a_1$ has no co-authors to make itself vulnerable. Therefore, this case cannot happen.

\paragraph{Case 3: Safe author.} If author $a_1$ is safe, no papers need to be rejected. The ideal desk-rejection condition is trivially satisfied.

In all possible cases, we can achieve the ideal desk-rejection. Thus, the proof is finished. 
\end{proof}

To present the positive result for $n=2$ and any $x \in \mathbb{N}_+$, we first discuss a specific case where all authors are non-compliant.

\begin{lemma} [Positive result for $n=2$ and any $x\in\mathbb{N}_+$, non-compliant author only case] \label{lem:n_eq_2_positive}
If the following conditions hold:
\begin{itemize}
    \item Let $n = 1$ denote the number of authors as defined in Definition~\ref{def:submit_limit_problem}.
    \item All the authors are non-compliant authors as defined in Definition~\ref{def:three_kinds_authors}.
    \item Let $x \in \mathbb{N}_+$ denote the maximum number of submissions allowed for each author in the conference. 
\end{itemize}

Then, there exists an algorithm that achieves the ideal desk-rejection as defined in Definition~\ref{def:good_solution}.
\end{lemma}

\begin{proof}
Let $c \in \mathbb{N}$ denote the number of papers co-authored by both author $a_1$ and author $a_2$. For $i \in \{1, 2\}$, let $b_i \in \mathbb{N}$ denote the number of single-authored papers by author $a_i$. 

We then have:
\begin{align*}
    b_1 + c = |P_1|
\end{align*}
and
\begin{align*}
    b_2 + c = |P_2|.
\end{align*}

\paragraph{Case 1: $c \leq x$.} 
In this case, we have $b_1 \geq |P_1| - x$ and $b_2 \geq |P_2| - x$. Since $b_i$ represents the number of single-authored papers by author $a_i$, we can desk reject exactly $(|P_i| - x)$ papers from author $a_i$.

\paragraph{Case 2: $c > x$.} 
Here, we have $b_1 < |P_1| - x$ and $b_2 < |P_2| - x$. We first desk reject all $b_1$ single-authored papers from author $a_1$ and all $b_2$ single-authored papers from author $a_2$. Next, we desk reject $(c - x)$ co-authored papers from both authors. This ensures that the remaining $x$ papers are co-authored by both $a_1$ and $a_2$. Thus, we have successfully rejected exactly $(|P_i| - x)$ papers from each author $a_i$.

By combining the two cases above, the proof is complete.
\end{proof}

With the help of Lemma~\ref{lem:n_eq_2_positive}, we now establish the positive result for $n=2$ and any $x \in \mathbb{N}_+$.

\begin{lemma} [Positive result for $n=2$ and any $x\in\mathbb{N}_+$, general case] \label{lem:n_eq_2_positive_general}
If the following conditions hold:
\begin{itemize}
    \item Let $n = 1$ denote the number of authors as defined in Definition~\ref{def:submit_limit_problem}.
    \item Let $x \in \mathbb{N}_+$ denote the maximum number of submissions allowed for each author in the conference. 
\end{itemize}

Then, there exists an algorithm that achieves the ideal desk-rejection as defined in Definition~\ref{def:good_solution}.
\end{lemma}

\begin{proof}
We consider two authors, $a_1$ and $a_2$. Without loss of generality, we assume that $a_1$ has at least as many papers as $a_2$, i.e., $|P_1| \geq |P_2|$. By exhaustively enumerating all possible compositions of author types (i.e., non-compliant, vulnerable, or safe) for $a_1$ and $a_2$, we observe that the vulnerable-safe composition is impossible. This is because a vulnerable author must co-author at least one paper with a non-compliant author. After excluding this case, we analyze the remaining possible scenarios as follows:

\paragraph{Case 1: Both $a_1$ and $a_2$ are safe authors.} 
In this case, no papers need to be rejected, and the ideal desk-rejection trivially holds.

\paragraph{Case 2: $a_1$ is a non-compliant author and $a_2$ is a safe author.} 
Since rejecting papers from $a_1$ does not affect $a_2$'s submissions, we can simply reject $(|P_1| - x)$ papers from $a_1$ to achieve the ideal desk-rejection.

\paragraph{Case 3: $a_1$ is a non-compliant author and $a_2$ is a vulnerable author.} 
By Definition~\ref{def:three_kinds_authors}, we have $|P_1| > x$ and $|P_2| \le x$. Let $c := |\{p_j \in S: p_j \in P_1, p_j \in P_2\}|$ denote the number of co-authored papers by $a_1$ and $a_2$. From basic set theory, we know that $c \leq |P_2|$. Since $|P_2| \le x$, it follows that $c \le x$. Therefore, we have:
\begin{align*}
    \underbrace{|P_1| - c}_{\text{Individual papers of } a_1} \ge \underbrace{|P_1| - x}_{\text{Excess papers of } a_1},
\end{align*}
which implies that the number of individual papers authored solely by $a_1$ exceeds the number of over-limit papers for $a_1$. Thus, we can first reject $a_1$'s individual papers without affecting $a_2$'s submissions, thereby achieving the desired ideal desk-rejection.

\paragraph{Case 4: Both $a_1$ and $a_2$ are non-compliant authors.} 
This case directly follows from Lemma~\ref{lem:n_eq_2_positive}.

Combining all the cases above, we conclude that the ideal desk-rejection can always be achieved, which finishes the proof.
\end{proof}

\subsection{Negative Results} \label{sec:negative_results}
In this subsection, we present two positive results that support the $n \ge 3$ case in Theorem~\ref{thm:main_res_general}. We commence by showing the negative result for $n = 3$ and $x=1$.

\begin{lemma}[Negative result for $n=3$ and $x=1$] \label{lem:n_eq_3_negative}
If the following conditions hold:
\begin{itemize}
    \item Let $n = 3$ denote the number of authors as defined in Definition~\ref{def:submit_limit_problem}.
    \item Let $x=1$ denote the maximum number of submissions allowed for each author in the conference. 
\end{itemize}

Then, under proper conditions, no algorithm can achieve the ideal desk-rejection as defined
in Definition~\ref{def:good_solution}.
\end{lemma}

\begin{proof}
Let all the authors be non-compliant authors as defined in Definition~\ref{def:three_kinds_authors}, and let the number of papers be $m=3$. We suppose the three papers $p_1$, $p_2$, and $p_3$ have the following authorship:

\begin{itemize}
    \item Paper $p_1$ is co-authored by $a_1$ and $a_2$.
    \item Paper $p_2$ is co-authored by $a_1$ and $a_3$.
    \item Paper $p_3$ is co-authored by $a_2$ and $a_3$.
\end{itemize}

From the authors' perspective, the relationships are as follows:
\begin{itemize}
    \item Author $a_1$ has papers $p_1$ and $p_2$.
    \item Author $a_2$ has papers $p_1$ and $p_3$.
    \item Author $a_3$ has papers $p_2$ and $p_3$.
\end{itemize}

We enumerate all possible rejection plans and their outcomes in Table~\ref{tab:all_possible_rejections}.

\begin{table}[!ht]
\caption{Remaining number of papers for each author after desk rejection.}
\label{tab:all_possible_rejections}
\begin{center}
\begin{tabular}{|c|c|c|c|}
 \hline
 Rejected Papers & Author $a_1$ & Author $a_2$ & Author $a_3$ \\ \hline
 N/A             & 2            & 2            & 2            \\ \hline
 $p_1$           & 1            & 1            & 2            \\ \hline
 $p_2$           & 1            & 2            & 1            \\ \hline
 $p_3$           & 2            & 1            & 1            \\ \hline
 $p_1, p_2$      & 0            & 1            & 1            \\ \hline
 $p_1, p_3$      & 1            & 0            & 1            \\ \hline
 $p_2, p_3$      & 1            & 1            & 0            \\ \hline
 $p_1, p_2, p_3$ & 0            & 0            & 0            \\ \hline
\end{tabular}
\end{center}
\end{table}

First, suppose we desk reject paper $p_3$. Then, authors $a_2$ and $a_3$ each have one paper remaining, but author $a_1$ still has two papers. To satisfy the constraint $x=1$, we must reject one of $p_1$ or $p_2$.

If we reject $p_1$, author $a_2$ is left with no papers, which is unfair. If we reject $p_2$, author $a_3$ is left with no papers, which is also unfair.

Thus, no rejection plan satisfies the ideal desk rejection condition for all authors. This completes the proof.
\end{proof}

Next, we present the negative result for any $n \geq 3$ and $x = n-2$.

\begin{lemma}[Negative result for any $n \geq 3$ and $x = n-2$] \label{lem:n_geq_3_negative}
If the following conditions hold:
\begin{itemize}
    \item Let $n \ge 3$ denote the number of authors as defined in Definition~\ref{def:submit_limit_problem}.
    \item Let $x=n-2$ denote the maximum number of submissions allowed for each author in the conference. 
\end{itemize}

Then, under proper conditions, no algorithm can achieve the ideal desk-rejection as defined
in Definition~\ref{def:good_solution}.
\end{lemma}

\begin{proof}
In this negative problem instance, we choose the number of papers to be the same as the number of authors, i.e., $m=n$, and we assume all the $n$ authors are non-compliant authors as defined in Definition~\ref{def:three_kinds_authors}. 


For each of the $n$ papers $p_i \in \mathcal{P}$, we let $i$-th paper $p_i$ contain $n-1$ authors, excluding only the $i$-th author $a_i$. Specifically, we have:  
\begin{itemize}
    \item The first paper $p_1$ has authors $a_2, a_3, \cdots, a_n$. 
    \item The second paper $p_2$ has authors $a_1, a_3, a_4, \cdots, a_n$.
    \item $\cdots\cdots$
    \item The $(n-1)$-th paper has authors $a_1, a_2, \cdots , a_{n-2}, a_n$.
    \item The $n$-th paper has authors $a_1, a_2, \cdots , a_{n-2}, a_{n-1}$.
\end{itemize}

Since each author is allowed to submit at most $x=n-2$ papers, we must desk-reject at least two papers. We analyze the process of desk-rejecting these two papers step by step.


\textbf{Step 1: Desk-reject the first paper.}

Without loss of generality, we consider rejecting paper $p_1$ first. After this operation, authors $a_2, a_3, \cdots a_n$, will have $n-2$ submitted papers, while author $a_1$ will have $n-1$ submitted papers. 


\textbf{Step 2: Desk-reject the second paper.}

Without loss of generality, we consider rejecting paper $p_2$ next. After this operation, authors $a_3, a_4, \cdots a_n$, will have $n-3$ submitted papers, while author $a_1$ and $a_2$ will have $n-2$ submitted papers. 

At this point, it is impossible for authors $a_3, a_4, a_5 \cdots , a_n$ to have exactly $(n-2)$ submitted papers. Therefore, no algorithm can achieve the ideal desk-rejection under the given conditions. This completes the proof.
\end{proof}


\section{Missing Proofs in Section 5}\label{sec:fair_proof}
In this section, we first present the missing proofs for fairness metrics in Section~\ref{sec:fair_metric_append}, and then present the supplementary proofs for the hardness of individual fairness optimization in Section~\ref{sec:indi_fair_hard_append}. Finally, we show the additional proofs for the group fairness optimization problem in Section~\ref{sec:fair_optim_append}.

\subsection{Fairness Metrics}\label{sec:fair_metric_append}
We present the relationship between the fairness metrics.  

\begin{proposition}[Relationship of Fairness Metrics, formal version of Proposition~\ref{lem:fair_metric_ineq} in Section~\ref{sec:fair_metric}]\label{lem:fair_metric_ineq_append}
    For any solution $S\subseteq \mathcal{P}$ for the submission limit problem in Definition~\ref{def:submit_limit_problem}, we have 
    \begin{align*}
        \zeta_{\mathrm{group}}(S) \leq \zeta_{\mathrm{ind}}(S).
    \end{align*}
\end{proposition}
\begin{proof}
    By Definition~\ref{def:group_fair}, we have:
    \begin{align*}
        \zeta_{\mathrm{group}}(S) &=~ \frac{1}{n}\sum_{i \in [n]} c(a_i,S) \\
        &\leq~ \frac{1}{n}\sum_{i \in [n]} \max_{i\in[n]}c(a_i, S) \\ 
        &=~\frac{1}{n}\cdot n \cdot \max_{i\in[n]}c(a_i, S) \\ 
        &=~ \zeta_{\mathrm{ind}}(S),
    \end{align*}
where the first equality directly follows from Definition~\ref{def:group_fair}, the second and the third inequality follow from basic algebra, and the last equality follows from Definition~\ref{def:individual_fair}. Thus, we complete the proof.
\end{proof}

\subsection{Hardness of Individual Fairness-Aware Submission Limit Problem}\label{sec:indi_fair_hard_append}

Before proving the theoretical results in Section~\ref{sec:indi_fair_hard}, we first introduce a useful fact that serves as a foundation for the subsequent proofs.   

\begin{fact}\label{fact:author_paper_count}
For each author $a_i\in\mathcal{A}$, the number of papers after desk-rejection (i.e., $|\{p_j \in  S : a_i \in A_j\}|$) can be written as $W_i^\top r$.
\end{fact}
\begin{proof}
    This simply follows from:
    \begin{align*}
        W^\top_ir &=~ \sum_{j\in[m]} W_{i,j}\cdot r_j \\ 
        &=~ |\{j\in[m]:W_{i, j}=1,r_j=1\}| \\ 
        &=~ |\{p_j\in\mathcal{P}:a_i\in A_j, p_j\in S\}| \\ 
        &=~|\{p_j \in  S : a_i \in A_j\}|,
    \end{align*}
where the first and the second equality follow from basic algebra and set theory, and the third and the fourth equality follow from Definition~\ref{def:submit_limit_problem}.  
\end{proof}

With the help of the aforementioned fact, we now prove the equivalence of the matrix form for the individual fairness problem.


\begin{proposition}[Matrix Form Equivalence for $\zeta_{\mathrm{ind}}$, formal version of Proposition~\ref{prop:equiv_individual} in Section~\ref{sec:indi_fair_hard}]\label{prop:equiv_individual_append}
    The individual fairness-aware submission limit problem in Definition~\ref{def:ind_fair_min} and the matrix form integer programming problem in Definition~\ref{def:ind_fair_min_matrix} are equivalent.
\end{proposition}
\begin{proof}
    In Definition~\ref{def:ind_fair_min}, the paper set $\mathcal{P}$ consists of $m$ papers, each of which can either be maintained or desk-rejected. Thus, the subset of maintained papers, $\mathcal{S}$, can be represented by a 0-1 vector $r \in \{0, 1\}^m$, where $r_j = 1$ indicates that paper $p_j$ is maintained, and $r_j = 0$ indicates that it is desk-rejected. We now establish the equivalence of both the objective function and the constraints in these two formulations.


    \paragraph{Part 1: Optimization Objective.} We first consider the objective function $\mathbf{1}^\top_nD^{-1}Wr$ in Definition~\ref{def:ind_fair_min_matrix}:
    \begin{align*}
        \min_{r \in \{0,1\}^m} \| \mathbf{1}_n - D^{-1}Wr\|_\infty &=~ \min_{r \in \{0,1\}^m} \max_{i\in[n]} (1 - (D^{-1}Wr)_i) \\ 
        &=~ \min_{r \in \{0,1\}^m} \max_{i\in[n]} (1 - (W_i^\top r)_i/D_{i,i}) \\
        &=~ \min_{r \in \{0,1\}^m} \max_{i\in[n]} (1 - (W_i^\top r)_i/|P_i|) \\ 
        &=~ \min_{r \in \{0,1\}^m} \max_{i\in[n]} (1 - |\{p_j \in  S : a_i \in A_j\}|/|P_i|) \\ 
        &=~ \min_{r \in \{0,1\}^m} \max_{i\in[n]}c(a_i, S) \\ 
        &=~ \min_{r \in \{0,1\}^m} \zeta_{\mathrm{ind}}(S),
    \end{align*}
where the first equality follows from the definition of infinity norm, the second equality follows from basic algebra, the third equality follows from Definition~\ref{def:ind_fair_min_matrix}, the fourth equality follows from Fact~\ref{fact:author_paper_count}, the fifth equality follows from Definition~\ref{def:cost}, and the last equality follows from Definition~\ref{def:individual_fair}. By decoding $r$ back into the paper subset $S$, we recover the original optimization objective in Definition~\ref{def:ind_fair_min}.

\paragraph{Part 2: Constraints.} The constraint in Definition~\ref{def:ind_fair_min_matrix} can be rewritten using basic algebra as:
     \begin{align*}
         W_i \cdot r \leq x, \quad \forall i \in [n].
     \end{align*}
     By applying Fact~\ref{fact:author_paper_count}, we see that this constraint is equivalent to its counterpart in Definition~\ref{def:ind_fair_min}.

Since both the objective function and constraints in Definition~\ref{def:ind_fair_min} and Definition~\ref{def:ind_fair_min_matrix} are equivalent, the proof is complete.
\end{proof}

To show the hardness of the individual fairness problem, we first present a classical set cover problem with well-established hardness. 

\begin{definition}[Set Cover Problem \cite{k72,gj79}]\label{def:set_cover}
The Set Cover problem is the following: 
\begin{itemize}
    \item {\bf Input:} A universe $U = \{1, \ldots, n\}$, a family of sets 
    $\{S_1, \ldots, S_m\} \subseteq 2^U$, and a integer $K > 0$. 
    \item {\bf Question:} Is there a subfamily $\{S_j : j \in J\}$ for some 
    $J \subseteq \{1,\ldots,m\}$ and $|J| \leq K$ that covers $U$, i.e., $\bigcup_{j \in J} S_j = U$?
\end{itemize}
\end{definition}

\begin{lemma}[Folklore \cite{k72,gj79}]\label{lem:set_cover_np_hard}
    The Set Cover problem defined in Definition~\ref{def:set_cover} is $\NPhard$.
\end{lemma}

Additionally, we also present a technical lemma which is useful for showing the hardness of the individual fairness problem. 

\begin{lemma}\label{lem:useful_lemma}
    For any $r \in \{0,1\}^m$, the following two statements are equivalent:
    \begin{itemize}
       \item {\bf Part 1.} $\|\mathbf{1}_n - D^{-1}Wr\|_\infty \leq 1 - \frac{1}{\min_{i \in [n]}|P_i|}$.
       \item {\bf Part 2.} $\min_{i\in[n]} (Wr)_i \geq 1$.
    \end{itemize}
\end{lemma}
\begin{proof}
    We first show that Part 1 implies Part 2.
    Suppose that
    \begin{align*}
        \|\mathbf{1}_n - D^{-1}Wr\|_\infty \leq 1 - \frac{1}{\min_{i \in [n]}|P_i|}.
    \end{align*}
    By the definition of the infinity norm, we have
    \begin{align*}
         1- \frac{(Wr)_{i'}}{|P_{i'}|} \leq 1 - \frac{1}{\min_{i \in [n]}|P_i|}, \quad \forall i' \in [n].
    \end{align*}
    Rearranging gives
    \begin{align*}
        (Wr)_{i'} \geq \frac{|P_{i'}|}{\min_{i \in [n]}|P_i|} \geq 1, \quad \forall i' \in [n].
    \end{align*}
    Since for all $i'\in[n]$, we have $(Wr)_{i'} \geq 1$, we can conclude that $\min_{i\in[n]} (Wr)_i \geq 1$.

    Now we show that that Part 2 implies Part 1.
    Suppose that $\min_{i\in[n]} (Wr)_i \geq 1$,  then we have $(Wr)_{i} \geq 1$ for all $i \in [n]$, which implies that for all $i \in [n]$, 
    \begin{align*}
        1 - \frac{(Wr)_i}{|P_i|} \leq  1 - \frac{1}{|P_i|} \leq  1 - \frac{1}{\max_{i' \in [n]} |P_{i'}|}.
    \end{align*}
    Hence
    \begin{align*}
        \|\mathbf{1}_n - D^{-1}Wr\|_\infty \leq 1 - \frac{1}{\min_{i \in [n]}|P_i|}.
    \end{align*}
    Thus the proof is complete.
\end{proof}

\begin{theorem}[Hardness, formal version of Theorem~\ref{thm:indi_nphard} in Section~\ref{sec:indi_fair_hard}]\label{thm:indi_nphard_append}
    The Individual Fairness-Aware Submission Limit Problem defined in Definition~\ref{def:ind_fair_min} is $\NPhard$.
\end{theorem}
\begin{proof}
    By Proposition~\ref{prop:equiv_individual}, it sufficies to reduce Set Cover problem to the integer optimization problem of the matrix form in Definition~\ref{def:ind_fair_min_matrix}.

    Given an instance of Set Cover, we build the matrix $W \in \{0, 1\}^{n\times m}$ by defining $W_{i,j} = 1$ if element $i \in S_j$, and 0 otherwise.
    Now set $|P_i| = \sum_{j \in [m]} W_{i,j}$ for every row $i \in [n]$. Finally, we choose $x = m$. We reduce the Set Cover problem to the following optimization problem:
    \begin{align*}
        & ~ \min_{r \in \{0,1\}^m} \| \mathbf{1}_n - D^{-1}Wr\|_\infty \\
    \mathrm{s.t.}
    & ~ W r \leq m \mathbf{1}_n, \\
    & ~ \|r\|_1 \leq K.
    \end{align*}
    Note that this problem is easier than the optimization problem defined in Definition~\ref{def:ind_fair_min}. The constraint $ W r \leq m \mathbf{1}_n$ is always satisfied, so we can drop it out. Now, it suffices to consider the decision problem:
    \begin{align*}
        &~ \mathrm{Find~} r \in \{0,1\}^m \\ \mathrm{s.t.~} &~
       \|\mathbf{1}_n - D^{-1}Wr\|_\infty \leq 1 - \frac{1}{\min_{i \in [n]}|P_i|}, \\
       &~ \|r\|_1 \leq K.
    \end{align*}

    Note that $\|\mathbf{1}_n - D^{-1}Wr\|_\infty \leq 1 - \frac{1}{\min_{i \in [n]}|P_i|}$ is equivalent to 
       $\min_{i\in[n]} (Wr)_i \geq 1$ by Lemma~\ref{lem:useful_lemma}.

    Hence the problem is equivalent to
    \begin{align*}
       \mathrm{Find~} r \in \{0,1\}^m  \mathrm{~~~s.t.~~~}
       \min_{i\in[n]} (Wr)_i \geq 1 \mathrm{~~and~~} 
        \|r\|_1 \leq K.
    \end{align*}
    It is not hard to see that the Set Cover problem has a solution if and only if the above problem has a solution.
    Requiring $\min_{i \in [n]} (W r)_i > 1$ exactly means that each element $i$ in the universe is covered by at least set $S_j$. The constraint $\|r\|_1 \leq K$ means that the size of cover is at most $K$. In other words, there exists a subfamily of size at most $K$ covering all elements if and only if there is an $r \in \{0,1\}^m$ with $\min_{i \in [n]} (W r)_i > 1$ and $\|r\|_1 \leq K$.
    
    Therefore, by Lemma~\ref{lem:set_cover_np_hard}, the individual fairness-aware submission limit problem is $\NPhard$.
\end{proof}

\subsection{Group Fairness Optimization}\label{sec:fair_optim_append}

Now, we present the missing proofs on both matrix form equivalence and linear programming optimal solution equivalence for the group fairness optimization problem. 

\begin{proposition}[Matrix Form Equivalence for $\zeta_{\mathrm{group}}$, formal version of Proposition~\ref{lem:group_fair_min_equiv} in Section~\ref{sec:fair_optim}]\label{lem:group_fair_min_equiv_append}
    The problem in Definition~\ref{def:group_fair_min} and the problem in Definition~\ref{def:group_fair_min_mat_new} are equivalent.
\end{proposition}
\begin{proof}
     In Definition~\ref{def:group_fair_min}, there are $m$ papers in $\mathcal{P}$, where each paper can either be maintained or rejected. Thus, we can encode the paper subset $S$ using a binary vector $r \in \{0, 1\}^m$, where $r_j = 1$ indicates that paper $p_j$ is maintained, and $r_j = 0$ indicates that it is desk-rejected. We now demonstrate that both the objective function and the constraints are equivalent.

     \paragraph{Part 1: Optimization Objective.} We first examine the objective function $\mathbf{1}^\top_nD^{-1}Wr$ in Definition~\ref{def:group_fair_min_mat_new}:
     \begin{align*}
         \mathbf{1}^\top_nD^{-1}Wr &=~ \sum_{i\in[n]}(D^{-1}Wr)_i \\ 
         &=~\sum_{i\in[n]}(W\cdot r)_i / |P_i| \\
         &=~\sum_{i\in[n]}(W_i^\top\cdot r) / |P_i| \\
         &=~\sum_{i\in[n]}\frac{|\{p_j \in S: a_i \in A_j\}|}{|P_i|} \\ 
         &=~ \sum_{i\in[n]}(1-c(a_i,S)),
     \end{align*}
     where the first equality follows from basic algebra, the second follows from Definition~\ref{def:group_fair_min_mat_new}, the third follows from matrix-vector multiplication, the fourth follows from Fact~\ref{fact:author_paper_count}, and the final equality follows from Definition~\ref{def:cost}. Consequently, the maximization problem in Definition~\ref{def:group_fair_min_mat_new} can be rewritten as:
     \begin{align*}
        \max_{r \in \{0, 1\}^m} \sum_{i\in[n]}(1-c(a_i,S)).
     \end{align*}
     Since maximizing this objective is equivalent to minimizing $\sum_{i\in[n]} c(a_i,S)$, we can reformulate it as:
     \begin{align*}
         \min_{r \in \{0, 1\}^m} \sum_{i\in[n]} c(a_i,S).
     \end{align*}
     By decoding $r$ back into the paper subset $S$, we recover the original optimization objective in Definition~\ref{def:group_fair_min}. 

     \paragraph{Part 2: Constraints.} Since the constraint is identical to that in the individual fairness minimization problem in Definition~\ref{def:ind_fair_min}, this result follows directly from Part 2 in the proof of Proposition~\ref{prop:equiv_individual_append}.

     Since both the objective function and constraints in Definition~\ref{def:group_fair_min} and Definition~\ref{def:group_fair_min_mat_new} are equivalent, the proof is complete.
\end{proof}

\begin{theorem}[Optimal Solution Equivalence of the Relaxed Problem, formal version of Theorem~\ref{thm:lp_equiv} in Section~\ref{sec:fair_optim}]\label{thm:lp_equiv_append}
    The optimal solution of the relaxed problem in Definition~\ref{def:group_fair_min_mat_relax_new} is equivalent to the optimal solution of the original problem in Definition~\ref{def:group_fair_min_mat_new}.
\end{theorem}
\begin{proof} 
    The problem in Definition~\ref{def:group_fair_min_mat_relax_new} is a linear program since it has a linear objective function $\mathbf{1}^\top_nD^{-1}Wr$ and linear constraints: the box constraint $r\in[0,1]^m$ and a linear inequality constraint $(Wr)/x \leq \mathbf{1}_n$.

    Furthermore, the problem is convex because the objective function is linear, the constraint $(Wr)/x \leq \mathbf{1}_n$ is affine, and the feasible region defined by $r\in[0,1]^m$ is a convex set.

    By the fundamental theorem of linear programming (see Page 23 of~\cite{ly84}), the optimal solution must occur at an extreme point of the convex polytope defined by the constraints. This implies that for all $i \in [m]$, we must have either $r_i = 0$ or $r_i = 1$. Therefore, the optimal solution of the relaxed linear program coincides with that of the original integer program, which finishes the proof.
\end{proof}

\section{Additional Case Studies}\label{sec:more_case_study}

As discussed in Section~\ref{sec:good_solution_hard}, optimizing the individual fairness metric is computationally challenging. Therefore, we minimize the group fairness metric, which serves as a lower bound for individual fairness, as a practical alternative. In this subsection, we present case studies demonstrating the relationship between both types of fairness metrics. 

\begin{example}
Consider a submission limit problem as defined in Definition~\ref{def:submit_limit_problem} with $x = 2$, $n = 3$ authors, and $m = 6$ papers. Let author $a_1$ submit four papers $p_1, p_2, p_3, p_4$, author $a_2$ submit two papers $p_3, p_5$, and author $a_3$ submit two papers $p_4, p_6$. 
\end{example}

In this case, the ideal desk-rejection criteria in Definition~\ref{def:good_solution} reject $p_1$ and $p_2$ (i.e., $S = \{p_3, p_4, p_5, p_6\}$), yielding fairness metrics $\zeta_{\mathrm{ind}}(S) = \max\{1/2, 0, 0\} = 1/2$ and $\zeta_{\mathrm{group}}(S) = \frac{1}{3}(1/2 + 0 + 0) = 1/6$. By applying an LP solver to minimize group fairness using Algorithm~\ref{alg:fair_desk_reject_algo} and enumerating all rejection strategies to verify individual fairness minimization, we observe that minimizing group fairness in this case aligns with minimizing individual fairness as defined in Definition~\ref{def:ind_fair_min_matrix}. This case illustrates that minimizing group fairness can sometimes benefit individual fairness.

However, group fairness and individual fairness are not always consistent. In some cases, prioritizing group fairness may disproportionately burden certain individuals. To illustrate this, we consider the following example.

\begin{example}
    Consider a submission limit problem as defined in Definition~\ref{def:submit_limit_problem} with $x = 2$, $n = 5$ authors, and $m = 4$ papers. Let author $a_1$ submit four papers $p_1, p_2, p_3, p_4$, author $a_2$ submit two papers $p_1, p_2$, and authors $a_3, a_4, a_5$ be coauthors of papers $p_3, p_4$. 
\end{example}

In this scenario, an ideal desk-rejection is impossible because $a_1$ must have two papers rejected, but rejecting any papers would cause at least one of the authors in $a_2, \ldots, a_5$ to fall below the submission limit of $x=2$. Here, group fairness and individual fairness diverge: Algorithm~\ref{alg:fair_desk_reject_algo} minimizes group fairness by rejecting $p_1$ and $p_2$ (i.e., $S = \{p_3, p_4\}$), which unfairly excludes all of $a_2$'s submissions. This results in fairness metrics $\zeta_{\mathrm{group}}(S) = \frac{1}{4}(1/2 + 1 + 0 + 0) = 3/8$ and $\zeta_{\mathrm{ind}}(S) = \max\{1/2, 1, 0, 0\} = 1$. 

Conversely, the individual fairness minimization problem in Definition~\ref{def:ind_fair_min_matrix} rejects one paper from $a_1, a_2$ and another from $a_3, a_4$, leading to $\zeta_{\mathrm{group}}(S) = \frac{1}{4}(1/2 + 1/2 + 1/2 + 1/2) = 1/2$ and $\zeta_{\mathrm{ind}}(S) = \max\{1/2, 1/2, 1/2, 1/2\} = 1/2$.

This example highlights an unintended consequence of minimizing group fairness: it may unfairly penalize authors with fewer coauthors, as rejecting their papers incurs a smaller total cost. On the other hand, optimizing individual fairness inevitably spreads rejections across a broader set of authors, potentially leading to a higher overall fairness cost. Balancing individual and group fairness remains an open challenge, which we leave for future work.


\section{Summary of Conference Links} \label{app:sec:conference_links}

In the introduction, Table~\ref{tab:conference_submission_limit} only gives a brief summary of the conference year and its limitation of per-author submission. 
Thus, we provide a detailed list of conferences in each year in this section, and then summarize the submission limits in Table.~\ref{tab:conference_submission_limit_full}. 
\begin{itemize}
    \item CVPR 
    \begin{itemize}
        \item 2025, \url{https://cvpr.thecvf.com/Conferences/2025/CVPRChanges} 
        \item 2024, \url{https://cvpr.thecvf.com/Conferences/2024/AuthorGuidelines}
    \end{itemize}

    \item ICCV
    \begin{itemize}
        \item 2025, \url{https://iccv.thecvf.com/Conferences/2025/AuthorGuidelines}
        \item 2023, \url{https://iccv2023.thecvf.com/policies-361500-2-20-15.php}
    \end{itemize}
    
    \item AAAI
    \begin{itemize}
        \item 2025, \url{https://aaai.org/conference/aaai/aaai-25/submission-instructions/} 
        \item 2024, \url{https://aaai.org/aaai-24-conference/submission-instructions/} 
        \item 2023, \url{https://aaai-23.aaai.org/submission-guidelines/} 
        \item 2022, \url{https://aaai.org/conference/aaai/aaai-22/}
    \end{itemize}
    
    \item WSDM 
    \begin{itemize}
        \item 2025, \url{https://www.wsdm-conference.org/2025/call-for-papers/} 
        \item 2024, \url{https://www.wsdm-conference.org/2024/call-for-papers/} 
        \item 2023, \url{https://www.wsdm-conference.org/2023/calls/call-papers/} 
        \item 2022, \url{https://www.wsdm-conference.org/2022/calls/} 
        \item 2021, \url{https://www.wsdm-conference.org/2021/call-for-papers.php} 
        \item 2020, \url{https://www.wsdm-conference.org/2020/call-for-papers.php} 
    \end{itemize}
    
    \item IJCAI
    \begin{itemize}
        \item 2025, \url{https://2025.ijcai.org/call-for-papers-main-track/} 
        \item 2024, \url{https://ijcai24.org/call-for-papers/}
        \item 2023, \url{https://ijcai-23.org/call-for-papers/}
        \item 2022, \url{https://ijcai-22.org/calls-papers}
        \item 2021, \url{https://ijcai-21.org/cfp/index.html}
        \item 2020, \url{https://ijcai20.org/call-for-papers/index.html}
        \item 2019, \url{https://www.ijcai19.org/call-for-papers.html}
        \item 2018, \url{https://www.ijcai-18.org/cfp/index.html}
        \item 2017, \url{https://ijcai-17.org/MainTrackCFP.html}
    \end{itemize}
    
    \item KDD 
    \begin{itemize}
        \item 2025, \url{https://kdd2025.kdd.org/research-track-call-for-papers/} 
        \item 2024, \url{https://kdd2024.kdd.org/research-track-call-for-papers/}
        \item 2023, \url{https://kdd.org/kdd2023/call-for-research-track-papers/index.html}
    \end{itemize}
\end{itemize}

\newpage
\begin{table}[!ht]\caption{ 
In this table, we summarize the submission limits of top conferences in recent years. For details of each conference website, we refer the readers to Section~\ref{app:sec:conference_links} in Appendix. 
}  \label{tab:conference_submission_limit_full}
\begin{center}
\begin{tabular}{ |c|c|c|c|c| } 
 \hline
 {\bf Conference Name} & {\bf Year} & {\bf Upper Bound} \\ \hline
 CVPR & 2025 & 25 \\ \hline
 CVPR & 2024 & N/A \\ \hline
 ICCV & 2025 & 25 \\ \hline
 ICCV & 2023 & N/A \\ \hline
 AAAI & 2025 & 10 \\ \hline
 AAAI & 2024 & 10 \\ \hline
 AAAI & 2023 & 10 \\ \hline
 AAAI & 2022 & N/A \\ \hline
 WSDM & 2025 & 10 \\ \hline
 WSDM & 2024 & 10 \\ \hline
 WSDM & 2023 & 10 \\ \hline
 WSDM & 2022 & 10 \\ \hline
 WSDM & 2021 & 10 \\ \hline
 WSDM & 2020 & N/A \\ \hline
 IJCAI & 2025 & 8 \\ \hline
 IJCAI & 2024 & 8 \\ \hline
 IJCAI & 2023 & 8 \\ \hline
 IJCAI & 2022 & 8 \\ \hline
 IJCAI & 2021 & 8 \\ \hline
 IJCAI & 2020 & 6 \\ \hline
 IJCAI & 2019 & 10 \\ \hline
 IJCAI & 2018 & 10 \\ \hline
 IJCAI & 2017 & N/A \\ \hline
 KDD & 2025 & 7 \\ \hline
 KDD & 2024 & 7 \\ \hline
 KDD & 2023 & N/A \\ \hline
 \iffalse
 ICML & 2025 & None \\ \hline
 ICML & 2024 & None \\ \hline
 ICML & 2023 & None \\ \hline
 ICLR & 2025 & None \\ \hline
 ICLR & 2024 & None \\ \hline
 ICLR & 2023 & None \\ \hline
 ICCV & 2023 & None \\ \hline
 ICCV & 2021 & None \\ \hline
 ICCV & 2019 & None \\ \hline
 ECCV & 2024 & None \\ \hline
 ECCV & 2022 & None \\ \hline
 ECCV & 2020 & None \\ \hline
 WACV & 2025 & None \\ \hline
 WACV & 2024 & None \\ \hline
 WACV & 2023 & None \\ \hline
 ICRA & 2025 & None \\ \hline
 ICRA & 2024 & None \\ \hline
 ICRA & 2023 & None \\ \hline
 CoRL & 2024 & None \\ \hline
 CoRL & 2023 & None \\ \hline
 CoRL & 2022 & None \\ \hline
 EMNLP & 2024 & None \\ \hline
 EMNLP & 2023 & None \\ \hline
 EMNLP & 2022 & None \\ \hline
 ACL & 2025 & None \\ \hline
 ACL & 2024 & None \\ \hline
 ACL & 2023 & None \\ \hline
 BigData & 2025 & None \\ \hline
 BigData & 2024 & None \\ \hline
 BigData & 2023 & None \\ \hline
 \fi
\end{tabular}
\end{center}
\end{table}






%%% some writing rules

%% Writing rule for creating tags.
%% Tags :
%% Theorem    \ref{thm:bla_bla}
%% Lemma      \ref{lem:bla_bla}
%% Claim      \ref{cla:bla_bla}
%% Corollary  \ref{cor:bla_bla}
%% Fact       \ref{fac:bla_bla}
%% Definition \ref{def:bla_bla}
%% Section    \ref{sec:bla_bla}
%% Subsection \ref{sub:bla_bla}
%% Equation   \ref{eq:bla_bla}



\end{document}



%%%%%%%%%%%%%%%%%%%%%%%%%%%%%%%%%%%%%%%%%%%%%%%%%%%%%%%%%%%%%%%%%%%%%%%%%%%%%%%%%%%%%%%%%%%%%%%%%%%%%%%%%%%%%%%%%%%%%%%%%%%%%%%%%%%%%%%%%%%%%%%%%%%%%%%%%%%%%%%%%%%%%%%%%%%%%%%%%%%%%%%%%%%%%%%%%%%%%%%%%%%%%%%%%%%%%%%%%%%%%%%%%%%%%%%%%%%%%%%%%%%%%%%%%%%%%%%%%%%%%%%%%%%%%%%%%%%%%%%%%%%%%%%%%%%%%%%%%%%%%%%%%%%%%%%%%%%%%%%%%%%%%%%%%%%%%%%%%%%%%%%%%%%%%%%%%%%%%%%%%%%%%%%%%%%%%%%%%%%%%%%%%%%%%%%%%%%%%%%%%%%%%%%%%%%%%%%%%%%%%%%%%%%%%%%%%%%%%%%%%%%%%%%%%%%%%%%%%%%%%%
