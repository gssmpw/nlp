\section{Additional Case Studies}\label{sec:more_case_study}

As discussed in Section~\ref{sec:good_solution_hard}, optimizing the individual fairness metric is computationally challenging. Therefore, we minimize the group fairness metric, which serves as a lower bound for individual fairness, as a practical alternative. In this subsection, we present case studies demonstrating the relationship between both types of fairness metrics. 

\begin{example}
Consider a submission limit problem as defined in Definition~\ref{def:submit_limit_problem} with $x = 2$, $n = 3$ authors, and $m = 6$ papers. Let author $a_1$ submit four papers $p_1, p_2, p_3, p_4$, author $a_2$ submit two papers $p_3, p_5$, and author $a_3$ submit two papers $p_4, p_6$. 
\end{example}

In this case, the ideal desk-rejection criteria in Definition~\ref{def:good_solution} reject $p_1$ and $p_2$ (i.e., $S = \{p_3, p_4, p_5, p_6\}$), yielding fairness metrics $\zeta_{\mathrm{ind}}(S) = \max\{1/2, 0, 0\} = 1/2$ and $\zeta_{\mathrm{group}}(S) = \frac{1}{3}(1/2 + 0 + 0) = 1/6$. By applying an LP solver to minimize group fairness using Algorithm~\ref{alg:fair_desk_reject_algo} and enumerating all rejection strategies to verify individual fairness minimization, we observe that minimizing group fairness in this case aligns with minimizing individual fairness as defined in Definition~\ref{def:ind_fair_min_matrix}. This case illustrates that minimizing group fairness can sometimes benefit individual fairness.

However, group fairness and individual fairness are not always consistent. In some cases, prioritizing group fairness may disproportionately burden certain individuals. To illustrate this, we consider the following example.

\begin{example}
    Consider a submission limit problem as defined in Definition~\ref{def:submit_limit_problem} with $x = 2$, $n = 5$ authors, and $m = 4$ papers. Let author $a_1$ submit four papers $p_1, p_2, p_3, p_4$, author $a_2$ submit two papers $p_1, p_2$, and authors $a_3, a_4, a_5$ be coauthors of papers $p_3, p_4$. 
\end{example}

In this scenario, an ideal desk-rejection is impossible because $a_1$ must have two papers rejected, but rejecting any papers would cause at least one of the authors in $a_2, \ldots, a_5$ to fall below the submission limit of $x=2$. Here, group fairness and individual fairness diverge: Algorithm~\ref{alg:fair_desk_reject_algo} minimizes group fairness by rejecting $p_1$ and $p_2$ (i.e., $S = \{p_3, p_4\}$), which unfairly excludes all of $a_2$'s submissions. This results in fairness metrics $\zeta_{\mathrm{group}}(S) = \frac{1}{4}(1/2 + 1 + 0 + 0) = 3/8$ and $\zeta_{\mathrm{ind}}(S) = \max\{1/2, 1, 0, 0\} = 1$. 

Conversely, the individual fairness minimization problem in Definition~\ref{def:ind_fair_min_matrix} rejects one paper from $a_1, a_2$ and another from $a_3, a_4$, leading to $\zeta_{\mathrm{group}}(S) = \frac{1}{4}(1/2 + 1/2 + 1/2 + 1/2) = 1/2$ and $\zeta_{\mathrm{ind}}(S) = \max\{1/2, 1/2, 1/2, 1/2\} = 1/2$.

This example highlights an unintended consequence of minimizing group fairness: it may unfairly penalize authors with fewer coauthors, as rejecting their papers incurs a smaller total cost. On the other hand, optimizing individual fairness inevitably spreads rejections across a broader set of authors, potentially leading to a higher overall fairness cost. Balancing individual and group fairness remains an open challenge, which we leave for future work.

