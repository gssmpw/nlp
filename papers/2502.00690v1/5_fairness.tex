\section{Fairness-Aware Desk-Rejection}\label{sec:fair}

In this section, we first introduce two fairness metrics in Section~\ref{sec:fair_metric}, and then present the hardness result on minimizing one of them in Section~\ref{sec:good_solution_hard}. In Section~\ref{sec:fair_optim}, we show our optimization-based fairness-aware desk-rejection framework. 

\subsection{Fairness Metrics}\label{sec:fair_metric}

As discussed earlier, achieving an ideal desk-rejection system is practically infeasible, as unintended rejections due to collective punishments are unavoidable. To address this, we relax the ideal system into an approximate form, where some unfair desk-rejections are permitted, while these rejections should be proportional to each author's total number of submissions.

Specifically, we introduce a cost function for each author, which estimates the impact of desk-rejection on each author:

\begin{definition}[Cost Function]\label{def:cost}
Considering the submission limit problem in Definition~\ref{def:submit_limit_problem}, we define the cost function $c: [n] \times 2^{[m]} \to [0,1]$ for a specific author $a_i$ and a set of remaining paper $S$ as
\begin{align*}
    c(a_i, S) := \frac{|P_i| - |\{p_j \in S: a_i \in A_j\}|}{|P_i|}.
\end{align*}
\end{definition}

\begin{remark}
    The cost function $c(a_i,S)$ measures the proportion of papers authored by $a_i$ that are rejected, prioritizing fairness for early-career authors with fewer submissions and aiming to reduce setbacks for them.
\end{remark}

To further demonstrate how this author-wise cost function could benefit fairness, we present the following example:
\begin{example}
    Consider a submission limit problem with $x = 10$ and $n = 2$. Suppose author $a_1$ submits papers $p_1, p_2, \ldots, p_{11}$, and author $a_2$ submits only paper $p_{11}$. Rejecting paper $p_{11}$ (i.e., $S = \mathcal{P} \setminus \{p_{11}\}$) results in a cost of $c(a_1, S) = 1/11$ for $a_1$ but a cost of $c(a_2, S) = 1$ for $a_2$, which is unfair to $a_2$. On the other hand, if we reject paper $p_1$ (i.e., $S' = \mathcal{P} \setminus \{p_1\}$), the cost for $a_1$ remains $c(a_1, S') = 1/11$, while the cost for $a_2$ becomes $c(a_2, S') = 0$. This minimizes both the highest cost and the total cost. This example demonstrates that our cost function encourages rejecting papers from authors with many submissions while protecting authors with few submissions.
\end{example}

To ensure fair treatment for all authors and avoid imposing excessive setbacks on early-career researchers, we introduce two fairness metrics based on our cost function. These metrics are inspired by the principles of utilitarian social welfare and egalitarian social welfare~\cite{ams24}. We begin by defining individual fairness, which is a strict worst-case fairness metric that aligns with the egalitarian social welfare framework by estimating the individual cost among all authors.

\begin{definition}[Individual Fairness]\label{def:individual_fair}
Let $c: [n] \times 2^{[m]} \to [0,1] $ be the cost function defined in Definition~\ref{def:cost}.
We define function $\zeta_{\mathrm{ind}}: 2^{[m]} \to [0,1]$ to measure the individual fairness:
\begin{align*}
    & ~ \zeta_{\mathrm{ind}}(S) := \max_{i \in [n]} c(a_i,S).
\end{align*}
\end{definition}
Next, we present the concept of group fairness, which aligns with utilitarian social welfare and measures the total cost across all authors.

\begin{definition}[Group Fairness]\label{def:group_fair}
Let $c: [n] \times 2^{[m]} \to [0,1] $ be the cost function defined in Definition~\ref{def:cost}.
We define function $\zeta_{\mathrm{group}}: 2^{[m]} \to [0,1]$ to measure the group fairness:
\begin{align*}
    & ~ \zeta_{\mathrm{group}}(S) := \frac{1}{n}\sum_{i \in [n]} c(a_i,S).
\end{align*}
 \end{definition}

To show the relationship between these two fairness metrics, we have the following proposition:

\begin{proposition}[Relationship of Fairness Metrics, informal version of Proposition~\ref{lem:fair_metric_ineq_append} in Appendix~\ref{sec:fair_proof}]\label{lem:fair_metric_ineq}
    For any solution $S\subseteq \mathcal{P}$ to the submission limit problem in Definition~\ref{def:submit_limit_problem}, we have 
    \begin{align*}
        \zeta_{\mathrm{ind}}(S) \leq \zeta_{\mathrm{group}}(S).
    \end{align*}
\end{proposition}


\subsection{Hardness of Individual Fairness-Aware Submission Limit Problem}\label{sec:indi_fair_hard}

After presenting fairness metrics for the desk-rejection system, we introduce an optimization-based framework to address these metrics. We first study the individual fairness-aware submission limit problem to minimize the individual fairness measure $\zeta_{\mathrm{ind}}$ in Definition~\ref{def:individual_fair}. 


\begin{definition}[Individual Fairness-Aware Submission Limit Problem]\label{def:ind_fair_min}
    We consider the following optimization problem:
\begin{align*}
    & ~ \min_{S \subseteq \mathcal{P}} \zeta_{\mathrm{ind}}(S) \\
    \mathrm{s.t.} & ~ |\{p_j \in S : a_i \in A_j\}| \leq x, \quad \forall a_i \in \mathcal{A}.
\end{align*}
\end{definition}

To represent the fairness metric minimization problem in matrix form, we introduce the following definition:

\begin{definition}[Author-Paper Matrix]\label{def:author_paper_mat}
    Let $W \in \{0, 1\}^{n \times m}$ denote the author-paper matrix for the author set $\mathcal{A}$ and paper set $\mathcal{P}$. Then, we define $W_{i,j} = 1$ if author $a_i$ is a coauthor of paper $p_j$, and $W_{i,j} = 0$ otherwise.
\end{definition}

Therefore, we present a more tractable integer
programming form of the original problem and prove its equivalence to the original
formulation:

\begin{definition}[Individual Fairness-Aware Submission Limit Problem, Matrix Form]\label{def:ind_fair_min_matrix}
    We consider the following integer optimization problem:
    \begin{align*}
        & ~ \min_{r \in \{0,1\}^m} \| \mathbf{1}_n - D^{-1}Wr\|_\infty \\
    \mathrm{s.t.} & ~ 
     (W r) / x \leq \mathbf{1}_n
    \end{align*}
    where $D = \Diag(|P_1|, \cdots, |P_n|)$, and the rejection vector $r \in \{0, 1\}^m$ is a 0-1 vector, with $r_j = 1$ indicating that paper $p_j$ is remained, and $r_j = 0$ indicating that it is desk-rejected. 
\end{definition}

\begin{proposition}[Matrix Form Equivalence for $\zeta_{\mathrm{ind}}$, informal version of Proposition~\ref{prop:equiv_individual_append} in Appendix~\ref{sec:fair_proof}]\label{prop:equiv_individual}
    The individual fairness-aware submission limit problem in Definition~\ref{def:ind_fair_min} and the matrix form integer programming problem in Definition~\ref{def:ind_fair_min_matrix} are equivalent.
\end{proposition}

Unfortunately, solving this integer programming problem is highly non-trivial, which means it may not yield a feasible solution within a reasonable time for large-scale conference submission systems. We establish the computational hardness of this problem in the following theorem:
 
\begin{theorem}[Hardness, informal version of Theorem~\ref{thm:indi_nphard_append} in Appendix~\ref{sec:indi_fair_hard_append}]\label{thm:indi_nphard}
    The Individual Fairness-Aware Submission Limit Problem defined in Definition~\ref{def:ind_fair_min} is $\NPhard$.
\end{theorem}

Since minimizing individual fairness is computationally intractable, our fairness-aware desk-rejection system instead focuses on minimizing group fairness. 


\subsection{Group Fairness Optimization}\label{sec:fair_optim}

Given the inherent hardness of individual fairness optimization, we address the fairness problem using an alternative yet equally important metric: group fairness, as defined in Definition~\ref{def:group_fair}. This metric is not only a crucial fairness measure in its own right but also serves as a lower bound for individual fairness as stated in Proposition~\ref{lem:fair_metric_ineq}, potentially improving individual fairness implicitly. 

Following a similar approach in Section~\ref{sec:indi_fair_hard}, we first formulate the submission limit problem with respect to group fairness and derive a more tractable integer programming formulation in matrix form:

\begin{definition}[Group Fairness-Aware Submission Limit Problem]\label{def:group_fair_min}
    We consider the following optimization problem:
\begin{align*}
    & ~ \min_{S \subseteq \mathcal{P}} \zeta_{\mathrm{group}}(S) \\
    \mathrm{s.t.} & ~ |\{p_j \in S : a_i \in A_j\}| \leq x, \quad \forall a_i \in \mathcal{A}.
\end{align*}
\end{definition}


\begin{definition}[Group Fairness-Aware Submission Limit Problem, Matrix Form]\label{def:group_fair_min_mat_new}
    We consider the following integer programming problem:
    \begin{align*}
        &~ \max_{r \in \{0, 1\}^m} \mathbf{1}^\top_n D^{-1} W r \\ 
        \mathrm{s.t.} 
        & ~ (W r) / x \leq \mathbf{1}_n,
    \end{align*}
    where $D = \Diag(|P_1|, \cdots, |P_n|)$, and the rejection vector $r \in \{0, 1\}^m$ is a 0-1 vector, with $r_j = 1$ indicating that paper $p_j$ is remained, and $r_j = 0$ indicating that it is desk-rejected. 
\end{definition}

\begin{proposition}[Matrix Form Equivalence for $\zeta_{\mathrm{group}}$, informal version of Proposition~\ref{lem:group_fair_min_equiv_append} in Appendix~\ref{sec:fair_proof}]\label{lem:group_fair_min_equiv}
    The fairness-aware submission limit problem in Definition~\ref{def:group_fair_min} and the matrix form integer programming problem in Definition~\ref{def:group_fair_min_mat_new} are equivalent.
\end{proposition}

However, solving integer programming problems is practically challenging. To this end, we first relax the feasible region of $r$ to $[0,1]^m$, and then analyze the resulting relaxed problem.  

\begin{definition}[Group Fairness-Aware Submission Limit Problem, Relaxation]\label{def:group_fair_min_mat_relax_new}
    We consider the optimization problem
    \begin{align*}
    &~ \max_{r \in [0, 1]^m}  \mathbf{1}^\top_nD^{-1}Wr \\ 
        \mathrm{s.t.} 
        & ~ (Wr)/x\le \mathbf{1}_n,
\end{align*}
where $D = \Diag(|P_1|, \cdots, |P_n|)$, and the rejection vector $r \in \{0, 1\}^m$ is a 0-1 vector, with $r_j = 1$ indicating that paper $p_j$ is remained, and $r_j = 0$ indicating that it is desk-rejected. 
\end{definition}

Fortunately, the relaxed problem is a linear program, which can be efficiently solved using standard linear programming solvers. Moreover, its optimal solution is equivalent to that of the original integer programming problem, an this result is formalized in the following theorem:

\begin{theorem}[Optimal Solution Equivalence of the Relaxed Problem, informal version of Theorem~\ref{thm:lp_equiv_append} in Appendix~\ref{sec:fair_proof}]\label{thm:lp_equiv}
    The optimal solution of the relaxed linear programming problem in Definition~\ref{def:group_fair_min_mat_relax_new} is equivalent to the optimal solution of the original integer programming problem in Definition~\ref{def:group_fair_min_mat_new}.
\end{theorem}

\begin{algorithm}[!ht]
\caption{Fairness-Aware Desk-Reject Algorithm}
\label{alg:fair_desk_reject_algo}
\begin{algorithmic}[1]

\State {\color{blue} /* $\mathcal{A}$ denotes the set of $n$ authors. */}
\State {\color{blue} /* $\mathcal{P}$ denote the set of $m$ papers. */}
\State {\color{blue} /* Author $a_i \in \mathcal{N}$ has a subset of papers $P_i \subset \mathcal{P}$. */}
\State {\color{blue} /* Paper $p_j \in \mathcal{P}$ is coauthored by a subset of authors $A_j \subseteq \mathcal{A}$.*/}
\State {\color{blue} /* $x$ represents the submission limit for each author.*/}

\Procedure{FairDeskReject}{$\mathcal{A}, \mathcal{P}, x$} 
\State {\color{blue} /* Initialize the constants of the problem. */}
\For{$i \in [n], j \in [m]$}
\If{$p_j \in \mathcal{A}_i$}
\State $W_{i,j}\gets 1$
\Else
\State $W_{i,j}\gets 0$
\EndIf
\EndFor
\State $D\gets\Diag(|P_1|, \ldots, |P_n|)$
\State {\color{blue} /* Solve the linear programming problem in Definition~\ref{def:group_fair_min_mat_relax_new}. */}
\State $r^{\star} \gets \mathsf{LPSolver}(W, D, x, r^0)$
\State {\color{blue} /* Transform the solution. */}
\State $S\gets\emptyset$
\For {$j\in[m]$}
\If {$r_j = 1$}
\State $S\gets S \cup \{p_j\}$
\EndIf
\EndFor
\State \Return $S$ 
\EndProcedure
\end{algorithmic}
\end{algorithm}

This theoretical result is significant as we formally establish that the group fairness-aware submission problem in Definition~\ref{def:group_fair_min} reduces to a linear programming (LP) problem with guaranteed optimality, solvable using off-the-shelf LP solvers. We formalize this procedure in Algorithm~\ref{alg:fair_desk_reject_algo}, where $\mathsf{LPSolver}$ denotes any standard LP solver, including but not limited to the simplex method~\cite{bg69}, interior-point path-finding methods~\cite{ls14}, and state-of-the-art stochastic central path methods~\cite{cls19, jswz21}.

\begin{remark}
    The time complexity of our fairness-aware desk-rejection algorithm in Algorithm~\ref{alg:fair_desk_reject_algo} aligns with modern linear programming solvers. For instance, using the stochastic central path method~\cite{cls21,jswz21,vlss20,sy21}, it achieves a time complexity of $O^*(m^{2.37} \log(m/\delta))$, where $\delta$ represents the relative accuracy corresponding to a \((1+\delta)\)-approximation guarantee.
\end{remark}

\begin{remark}
    In practice, major AI conferences routinely process submissions at the scale of $m \sim 10^4$~\cite{stanford_ai_index}. Given this regime, our algorithm guarantees efficient computation, enabling fairness-aware desk rejection within tractable timeframes, even for large-scale conferences.
\end{remark}
