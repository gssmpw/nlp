\clearpage
\newpage
\begin{center}
    \textbf{\LARGE Appendix}
\end{center}

\paragraph{Roadmap.} In Section~\ref{sec:dilemma_proof}, we supplement the missing proofs in Section~\ref{sec:dr_dilemma}. In Section~\ref{sec:fair_proof}, we present the missing proofs in Section~\ref{sec:fair}. In Section~\ref{sec:more_case_study}, we show additional case studies. In Section~\ref{app:sec:conference_links}, we provide the details related to conference submission limits. 

\section{Missing Proofs in Section 4} \label{sec:dilemma_proof}

In this section, we provide the complete technical proofs for Theorem~\ref{thm:main_res_general} in Section~\ref{sec:dr_dilemma}. 
In Section~\ref{sec:dilemma_proof_defs}, we first introduce key definitions that will be useful 
To structure our analysis, we in the subsequent proofs. We then establish positive results for the cases where $n \leq 2$ in Section~\ref{sec:positive_results}, followed by negative results for $n \geq 3$ in Section~\ref{sec:negative_results}.


\subsection{Basic Definitions}\label{sec:dilemma_proof_defs}

To systematically analyze the desk-rejection problem, we begin by classifying authors based on their submission behavior and their relationship to co-authors. This classification will help us organize and present the proofs in a more structured and readable manner.

\begin{definition}[Author Categories]\label{def:three_kinds_authors}
For any author $a_i \in \mathcal{A}$, we define the following categories:
\begin{itemize}
    \item \textbf{Non-compliant}: An author $a_i$ is non-compliant if they have submitted more than $x$ papers, i.e., $|P_i| > x$. Such authors exceed the submission limit and are subject to desk-rejection under the policy.

    \item \textbf{Vulnerable}: An author $a_i$ is vulnerable if they have submitted no more than $x$ papers ($|P_i| \leq x$) but have at least one non-compliant co-author, i.e., $\exists k \in C_i$ such that $|P_k| > x$. Although these authors comply with the submission limit, they are at risk of being unfairly penalized due to their co-authors' non-compliance.

    \item \textbf{Safe}: An author $a_i$ is safe if they have submitted no more than $x$ papers ($|P_i| \leq x$) and all their co-authors are also compliant, i.e., $\forall k \in C_i$, $|P_k| \leq x$. These authors are guaranteed to retain all their submissions, as neither they nor their co-authors violate the submission limit.
\end{itemize}
\end{definition}

Next, we formalize the notion of achievability for the ideal desk-rejection system.

\begin{definition}[Achievability]\label{def:achievability}
Given a submission limit problem instance as defined in Definition~\ref{def:submit_limit_problem}:
\begin{itemize}
    \item \textbf{Positive result}: A problem instance is a positive result if there exists an algorithm that can achieve the ideal desk-rejection as defined in Definition~\ref{def:good_solution}.
    
    \item \textbf{Negative result}: A problem instance is a negative result if,  under proper conditions, no algorithm can achieve the ideal desk-rejection as defined in Definition~\ref{def:good_solution}.
\end{itemize}
\end{definition}

In the following sections, we will use these definitions to systematically prove the positive results for small numbers of authors ($n \leq 2$) and the negative results for larger numbers of authors ($n \geq 3$), which covers two cases in Theorem~\ref{thm:main_res_general}.


\subsection{Positive Results} \label{sec:positive_results}
In this subsection, we present two positive results that support the $n \leq 2$ case in Theorem~\ref{thm:main_res_general}. We begin with the positive result for $n = 1$ and any $x \in \mathbb{N}_+$.

\begin{lemma}[Positive result for $n = 1$ and any $x \in \mathbb{N}_+$, general case]\label{lem:n_eq_1_positive_general}
    If the following conditions hold:
    \begin{itemize}
        \item Let $n = 1$ denote the number of authors as defined in Definition~\ref{def:submit_limit_problem}.
        \item Let $x \in \mathbb{N}_+$ denote the maximum number of submissions allowed for each author in the conference.
    \end{itemize}
    Then, there exists an algorithm that achieves the ideal desk-rejection as defined in Definition~\ref{def:good_solution}.
\end{lemma}

\begin{proof}
We consider the three cases for the only author $a_1$: non-compliant, vulnerable, and safe, as defined in Definition~\ref{def:three_kinds_authors}.

\paragraph{Case 1: Non-compliant author.} If author $a_1$ is non-compliant, we desk-reject $(|P_1| - x)$ papers. This ensures that exactly $x$ papers remain, satisfying the ideal desk-rejection condition.

\paragraph{Case 2: Vulnerable author.} Since $n = 1$ and there is only one author, author $a_1$ has no co-authors to make itself vulnerable. Therefore, this case cannot happen.

\paragraph{Case 3: Safe author.} If author $a_1$ is safe, no papers need to be rejected. The ideal desk-rejection condition is trivially satisfied.

In all possible cases, we can achieve the ideal desk-rejection. Thus, the proof is finished. 
\end{proof}

To present the positive result for $n=2$ and any $x \in \mathbb{N}_+$, we first discuss a specific case where all authors are non-compliant.

\begin{lemma} [Positive result for $n=2$ and any $x\in\mathbb{N}_+$, non-compliant author only case] \label{lem:n_eq_2_positive}
If the following conditions hold:
\begin{itemize}
    \item Let $n = 1$ denote the number of authors as defined in Definition~\ref{def:submit_limit_problem}.
    \item All the authors are non-compliant authors as defined in Definition~\ref{def:three_kinds_authors}.
    \item Let $x \in \mathbb{N}_+$ denote the maximum number of submissions allowed for each author in the conference. 
\end{itemize}

Then, there exists an algorithm that achieves the ideal desk-rejection as defined in Definition~\ref{def:good_solution}.
\end{lemma}

\begin{proof}
Let $c \in \mathbb{N}$ denote the number of papers co-authored by both author $a_1$ and author $a_2$. For $i \in \{1, 2\}$, let $b_i \in \mathbb{N}$ denote the number of single-authored papers by author $a_i$. 

We then have:
\begin{align*}
    b_1 + c = |P_1|
\end{align*}
and
\begin{align*}
    b_2 + c = |P_2|.
\end{align*}

\paragraph{Case 1: $c \leq x$.} 
In this case, we have $b_1 \geq |P_1| - x$ and $b_2 \geq |P_2| - x$. Since $b_i$ represents the number of single-authored papers by author $a_i$, we can desk reject exactly $(|P_i| - x)$ papers from author $a_i$.

\paragraph{Case 2: $c > x$.} 
Here, we have $b_1 < |P_1| - x$ and $b_2 < |P_2| - x$. We first desk reject all $b_1$ single-authored papers from author $a_1$ and all $b_2$ single-authored papers from author $a_2$. Next, we desk reject $(c - x)$ co-authored papers from both authors. This ensures that the remaining $x$ papers are co-authored by both $a_1$ and $a_2$. Thus, we have successfully rejected exactly $(|P_i| - x)$ papers from each author $a_i$.

By combining the two cases above, the proof is complete.
\end{proof}

With the help of Lemma~\ref{lem:n_eq_2_positive}, we now establish the positive result for $n=2$ and any $x \in \mathbb{N}_+$.

\begin{lemma} [Positive result for $n=2$ and any $x\in\mathbb{N}_+$, general case] \label{lem:n_eq_2_positive_general}
If the following conditions hold:
\begin{itemize}
    \item Let $n = 1$ denote the number of authors as defined in Definition~\ref{def:submit_limit_problem}.
    \item Let $x \in \mathbb{N}_+$ denote the maximum number of submissions allowed for each author in the conference. 
\end{itemize}

Then, there exists an algorithm that achieves the ideal desk-rejection as defined in Definition~\ref{def:good_solution}.
\end{lemma}

\begin{proof}
We consider two authors, $a_1$ and $a_2$. Without loss of generality, we assume that $a_1$ has at least as many papers as $a_2$, i.e., $|P_1| \geq |P_2|$. By exhaustively enumerating all possible compositions of author types (i.e., non-compliant, vulnerable, or safe) for $a_1$ and $a_2$, we observe that the vulnerable-safe composition is impossible. This is because a vulnerable author must co-author at least one paper with a non-compliant author. After excluding this case, we analyze the remaining possible scenarios as follows:

\paragraph{Case 1: Both $a_1$ and $a_2$ are safe authors.} 
In this case, no papers need to be rejected, and the ideal desk-rejection trivially holds.

\paragraph{Case 2: $a_1$ is a non-compliant author and $a_2$ is a safe author.} 
Since rejecting papers from $a_1$ does not affect $a_2$'s submissions, we can simply reject $(|P_1| - x)$ papers from $a_1$ to achieve the ideal desk-rejection.

\paragraph{Case 3: $a_1$ is a non-compliant author and $a_2$ is a vulnerable author.} 
By Definition~\ref{def:three_kinds_authors}, we have $|P_1| > x$ and $|P_2| \le x$. Let $c := |\{p_j \in S: p_j \in P_1, p_j \in P_2\}|$ denote the number of co-authored papers by $a_1$ and $a_2$. From basic set theory, we know that $c \leq |P_2|$. Since $|P_2| \le x$, it follows that $c \le x$. Therefore, we have:
\begin{align*}
    \underbrace{|P_1| - c}_{\text{Individual papers of } a_1} \ge \underbrace{|P_1| - x}_{\text{Excess papers of } a_1},
\end{align*}
which implies that the number of individual papers authored solely by $a_1$ exceeds the number of over-limit papers for $a_1$. Thus, we can first reject $a_1$'s individual papers without affecting $a_2$'s submissions, thereby achieving the desired ideal desk-rejection.

\paragraph{Case 4: Both $a_1$ and $a_2$ are non-compliant authors.} 
This case directly follows from Lemma~\ref{lem:n_eq_2_positive}.

Combining all the cases above, we conclude that the ideal desk-rejection can always be achieved, which finishes the proof.
\end{proof}

\subsection{Negative Results} \label{sec:negative_results}
In this subsection, we present two positive results that support the $n \ge 3$ case in Theorem~\ref{thm:main_res_general}. We commence by showing the negative result for $n = 3$ and $x=1$.

\begin{lemma}[Negative result for $n=3$ and $x=1$] \label{lem:n_eq_3_negative}
If the following conditions hold:
\begin{itemize}
    \item Let $n = 3$ denote the number of authors as defined in Definition~\ref{def:submit_limit_problem}.
    \item Let $x=1$ denote the maximum number of submissions allowed for each author in the conference. 
\end{itemize}

Then, under proper conditions, no algorithm can achieve the ideal desk-rejection as defined
in Definition~\ref{def:good_solution}.
\end{lemma}

\begin{proof}
Let all the authors be non-compliant authors as defined in Definition~\ref{def:three_kinds_authors}, and let the number of papers be $m=3$. We suppose the three papers $p_1$, $p_2$, and $p_3$ have the following authorship:

\begin{itemize}
    \item Paper $p_1$ is co-authored by $a_1$ and $a_2$.
    \item Paper $p_2$ is co-authored by $a_1$ and $a_3$.
    \item Paper $p_3$ is co-authored by $a_2$ and $a_3$.
\end{itemize}

From the authors' perspective, the relationships are as follows:
\begin{itemize}
    \item Author $a_1$ has papers $p_1$ and $p_2$.
    \item Author $a_2$ has papers $p_1$ and $p_3$.
    \item Author $a_3$ has papers $p_2$ and $p_3$.
\end{itemize}

We enumerate all possible rejection plans and their outcomes in Table~\ref{tab:all_possible_rejections}.

\begin{table}[!ht]
\caption{Remaining number of papers for each author after desk rejection.}
\label{tab:all_possible_rejections}
\begin{center}
\begin{tabular}{|c|c|c|c|}
 \hline
 Rejected Papers & Author $a_1$ & Author $a_2$ & Author $a_3$ \\ \hline
 N/A             & 2            & 2            & 2            \\ \hline
 $p_1$           & 1            & 1            & 2            \\ \hline
 $p_2$           & 1            & 2            & 1            \\ \hline
 $p_3$           & 2            & 1            & 1            \\ \hline
 $p_1, p_2$      & 0            & 1            & 1            \\ \hline
 $p_1, p_3$      & 1            & 0            & 1            \\ \hline
 $p_2, p_3$      & 1            & 1            & 0            \\ \hline
 $p_1, p_2, p_3$ & 0            & 0            & 0            \\ \hline
\end{tabular}
\end{center}
\end{table}

First, suppose we desk reject paper $p_3$. Then, authors $a_2$ and $a_3$ each have one paper remaining, but author $a_1$ still has two papers. To satisfy the constraint $x=1$, we must reject one of $p_1$ or $p_2$.

If we reject $p_1$, author $a_2$ is left with no papers, which is unfair. If we reject $p_2$, author $a_3$ is left with no papers, which is also unfair.

Thus, no rejection plan satisfies the ideal desk rejection condition for all authors. This completes the proof.
\end{proof}

Next, we present the negative result for any $n \geq 3$ and $x = n-2$.

\begin{lemma}[Negative result for any $n \geq 3$ and $x = n-2$] \label{lem:n_geq_3_negative}
If the following conditions hold:
\begin{itemize}
    \item Let $n \ge 3$ denote the number of authors as defined in Definition~\ref{def:submit_limit_problem}.
    \item Let $x=n-2$ denote the maximum number of submissions allowed for each author in the conference. 
\end{itemize}

Then, under proper conditions, no algorithm can achieve the ideal desk-rejection as defined
in Definition~\ref{def:good_solution}.
\end{lemma}

\begin{proof}
In this negative problem instance, we choose the number of papers to be the same as the number of authors, i.e., $m=n$, and we assume all the $n$ authors are non-compliant authors as defined in Definition~\ref{def:three_kinds_authors}. 


For each of the $n$ papers $p_i \in \mathcal{P}$, we let $i$-th paper $p_i$ contain $n-1$ authors, excluding only the $i$-th author $a_i$. Specifically, we have:  
\begin{itemize}
    \item The first paper $p_1$ has authors $a_2, a_3, \cdots, a_n$. 
    \item The second paper $p_2$ has authors $a_1, a_3, a_4, \cdots, a_n$.
    \item $\cdots\cdots$
    \item The $(n-1)$-th paper has authors $a_1, a_2, \cdots , a_{n-2}, a_n$.
    \item The $n$-th paper has authors $a_1, a_2, \cdots , a_{n-2}, a_{n-1}$.
\end{itemize}

Since each author is allowed to submit at most $x=n-2$ papers, we must desk-reject at least two papers. We analyze the process of desk-rejecting these two papers step by step.


\textbf{Step 1: Desk-reject the first paper.}

Without loss of generality, we consider rejecting paper $p_1$ first. After this operation, authors $a_2, a_3, \cdots a_n$, will have $n-2$ submitted papers, while author $a_1$ will have $n-1$ submitted papers. 


\textbf{Step 2: Desk-reject the second paper.}

Without loss of generality, we consider rejecting paper $p_2$ next. After this operation, authors $a_3, a_4, \cdots a_n$, will have $n-3$ submitted papers, while author $a_1$ and $a_2$ will have $n-2$ submitted papers. 

At this point, it is impossible for authors $a_3, a_4, a_5 \cdots , a_n$ to have exactly $(n-2)$ submitted papers. Therefore, no algorithm can achieve the ideal desk-rejection under the given conditions. This completes the proof.
\end{proof}

