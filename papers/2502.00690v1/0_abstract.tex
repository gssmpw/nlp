

As AI research surges in both impact and volume, conferences have imposed submission limits to maintain paper quality and alleviate organizational pressure. 
In this work, we examine the fairness of desk-rejection systems under submission limits and reveal that existing practices can result in substantial inequities. Specifically, we formally define the paper submission limit problem and identify a critical dilemma: when the number of authors exceeds three, it becomes impossible to reject papers solely based on excessive submissions without negatively impacting innocent authors.
Thus, this issue may unfairly affect early-career researchers, as their submissions may be penalized due to co-authors with significantly higher submission counts, while senior researchers with numerous papers face minimal consequences. To address this, we propose an optimization-based fairness-aware desk-rejection mechanism and formally define two fairness metrics: individual fairness and group fairness. We prove that optimizing individual fairness is NP-hard, whereas group fairness can be efficiently optimized via linear programming. Through case studies, we demonstrate that our proposed system ensures greater equity than existing methods, including those used in CVPR 2025, offering a more socially just approach to managing excessive submissions in AI conferences.


