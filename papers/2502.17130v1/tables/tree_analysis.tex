\begin{table}[h]
    \centering
    \resizebox{0.83\linewidth}{!}{
    \begin{tabular}{lccccc}
        \toprule
        Tree & Nodes & Unique degrees & Theoretical worst-case & $\deg_{\max}$ & Longest path length \\
        \midrule
        $m$-ary trees & Varying & 2 & Varying & $m + 1$ & varying \\
        Mosses & 344 & 11 & 38 & 16 & 51 \\
        Weevils & 195 & 5 & 29 & 8 & 29 \\
        Carnivora & 548 & 3 & 45 & 4 & 192.4 \\
        Lichen & 481 & 3 & 48 & 4 & 0.972 \\
        \bottomrule
    \end{tabular}
    }
    \caption{\textbf{Statistics for the trees used in the experiments.} The number of unique degrees is excluding nodes with a degree of 1. This number is equal to the total number of optimizations that has to be performed when embedding the tree using HS-DTE. The theoretical worst-case shows the worst-case number of optimizations that has to be performed according to Theorem \ref{thm:mhs_complexity}. Note that the true number of optimizations is often significantly lower than this worst-case number.}
    \label{tab:tree_analysis}
\end{table}