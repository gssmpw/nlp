\begin{table}[h]
    \centering
    \resizebox{0.83\linewidth}{!}{
    \begin{tabular}{l @{\hskip 0.5cm} cccc @{\hskip 0.5cm} cccc}
        \toprule
        \multirow{2}{*}{$D_{ave}$} & \multicolumn{4}{c}{\hspace{-0.7cm} Mosses} & \multicolumn{4}{c}{\hspace{-0.7cm} Weevils} \\
        & 4 & 7 & 10 & 20 & 4 & 7 & 10 & 20 \\
        \midrule
        HypFPE + \cite{sala2018representation} $\ddagger$ & - & - & - & 0.09 & - & - & 0.07 & 0.07 \\
        HypFPE + \cite{sala2018representation} $\star$ & 0.06 & 0.10 & 0.08 & 0.10 & \textbf{0.03} & 0.05 & 0.05 & 0.10 \\
        \rowcolor{Gray}
        HypFPE + HS-DTE & \textbf{0.04} & \textbf{0.04} & \textbf{0.04} & \textbf{0.04} & \textbf{0.03} & \textbf{0.03} & \textbf{0.03} & \textbf{0.03} \\
        \bottomrule
    \end{tabular}
    }
    \caption{\textbf{Comparison of average distortion of hyperbolic embedding algorithms on the mosses and weevils trees.} $\ddagger$ represents the construction with Hadamard hyperspherical points and $\star$ the construction with points sampled from a set precomputed with \citep{lovisolo2001uniform}. The best performance is in bold. The embeddings are performed in a 4, 7, 10 or 20-dimensional space. Overall, we find that HS-DTE works best.}
    \label{tab:phylo_tree_dim_res_ave_1}
\end{table}

\begin{table}[h]
    \centering
    \resizebox{0.83\linewidth}{!}{
    \begin{tabular}{l @{\hskip 0.5cm} cccc @{\hskip 0.5cm} cccc @{\hskip 0.5cm} cccc}
        \toprule
        \multirow{2}{*}{$D_{ave}$} & \multicolumn{4}{c}{\hspace{-0.7cm} Carnivora} & \multicolumn{4}{c}{\hspace{-0.7cm} Lichen} \\
        & 4 & 7 & 10 & 20 & 4 & 7 & 10 & 20 \\
        \midrule
        HypFPE + \cite{sala2018representation} $\ddagger$ & 0.04 & 0.04 & 0.04 & 0.04 & 0.12 & 0.12 & 0.12 & 0.12 \\
        HypFPE + \cite{sala2018representation} $\star$ & \textbf{0.01} & 0.03 & \textbf{0.03} & 0.06 & 0.05 & 0.10 & 0.11 & 0.19 \\
        \rowcolor{Gray}
        HypFPE + HS-DTE & 0.02 & \textbf{0.02} & \textbf{0.03} & \textbf{0.02} & \textbf{0.06} & \textbf{0.06} & \textbf{0.05} & \textbf{0.05} \\
        \bottomrule
    \end{tabular}
    }
    \caption{\textbf{Comparison of average distortion of hyperbolic embedding algorithms on the carnivora and lichen trees.} $\ddagger$ represents the construction with Hadamard hyperspherical points and $\star$ the construction with points sampled from a set precomputed with \citep{lovisolo2001uniform}. The best performance is in bold. The embeddings are performed in a 4, 7, 10 or 20-dimensional space. Overall, we find that HS-DTE works best.}
    \label{tab:phylo_tree_dim_res_ave_2}
\end{table}

\begin{table}[h]
    \centering
    \resizebox{0.83\linewidth}{!}{
    \begin{tabular}{l @{\hskip 0.5cm} cccc @{\hskip 0.5cm} cccc @{\hskip 0.5cm} cccc}
        \toprule
        \multirow{2}{*}{$D_{wc}$} & \multicolumn{4}{c}{\hspace{-0.7cm} Mosses} & \multicolumn{4}{c}{\hspace{-0.7cm} Weevils} \\
        & 4 & 7 & 10 & 20 & 4 & 7 & 10 & 20 \\
        \midrule
        HypFPE + \cite{sala2018representation} $\ddagger$ & - & - & - & 1.10 & - & - & 1.09 & 1.09 \\
        HypFPE + \cite{sala2018representation} $\star$ & 1.36 & 1.21 & 1.14 & 1.16 & 1.25 & 1.12 & 1.11 & 1.13 \\
        \rowcolor{Gray}
        HypFPE + HS-DTE & \textbf{1.09} & \textbf{1.07} & \textbf{1.06} & \textbf{1.07} & \textbf{1.05} & \textbf{1.05} & \textbf{1.04} & \textbf{1.04} \\
        \bottomrule
    \end{tabular}
    }
    \caption{\textbf{Comparison of worst-case distortion of hyperbolic embedding algorithms on the mosses and weevils trees.} $\ddagger$ represents the construction with Hadamard hyperspherical points and $\star$ the construction with points sampled from a set precomputed with \citep{lovisolo2001uniform}. The best performance is in bold. The embeddings are performed in a 4, 7, 10 or 20-dimensional space. Overall, we find that HS-DTE works best.}
    \label{tab:phylo_tree_dim_res_wc_1}
\end{table}

\begin{table}[h]
    \centering
    \resizebox{0.83\linewidth}{!}{
    \begin{tabular}{l @{\hskip 0.5cm} cccc @{\hskip 0.5cm} cccc @{\hskip 0.5cm} cccc}
        \toprule
        \multirow{2}{*}{$D_{wc}$} & \multicolumn{4}{c}{\hspace{-0.7cm} Carnivora} & \multicolumn{4}{c}{\hspace{-0.7cm} Lichen} \\
        & 4 & 7 & 10 & 20 & 4 & 7 & 10 & 20 \\
        \midrule
        HypFPE + \cite{sala2018representation} $\ddagger$ & 6.76 & 6.76 & 6.76 & 6.76 & 43.4 & 43.4 & 43.4 & 43.4 \\
        HypFPE + \cite{sala2018representation} $\star$ & 3.50 & 4.06 & 4.87 & 13.0 & 4.73 & 5.44 & 6.43 & 36.0 \\
        \rowcolor{Gray}
        HypFPE + HS-DTE & \textbf{2.46} & \textbf{2.45} & \textbf{2.03} & \textbf{2.35} & \textbf{4.07} & \textbf{4.63} & \textbf{3.30} & \textbf{7.17} \\
        \bottomrule
    \end{tabular}
    }
    \caption{\textbf{Comparison of worst-case distortion of hyperbolic embedding algorithms on the carnivora and lichen trees.} $\ddagger$ represents the construction with Hadamard hyperspherical points and $\star$ the construction with points sampled from a set precomputed with \citep{lovisolo2001uniform}. The best performance is in bold. The embeddings are performed in a 4, 7, 10 or 20-dimensional space. Overall, we find that HS-DTE works best.}
    \label{tab:phylo_tree_dim_res_wc_2}
\end{table}
