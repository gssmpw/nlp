\begin{figure*}
    \centering
    \begin{minipage}[t]{0.67\textwidth}
        \begin{subfigure}[t]{0.49\textwidth}
            \centering
            \includegraphics[width=\textwidth]{assets/25_points_v2.png}
            \caption{Minimal hyperspherical energy ablation.}
            \label{fig:hyperspherical_separation}
        \end{subfigure}
        \hfill
        \begin{subfigure}[t]{0.49\textwidth}
            \centering
            \includegraphics[width=\textwidth]{assets/fpe_ablation.png}
            \caption{Floating point expansion ablation.}
            \label{fig:fpe_ablation}
        \end{subfigure}
        \caption{\textbf{Ablation studies on our construction and floating point expansion.} (a) Minimal pairwise angle ($\uparrow$) of the hyperspherical points generated in step \ref{ln:hyperspherical_gen} of Algorithm \ref{alg:sarkar} using the various generation methods. The dimension of the space is set to 8, so the Hadamard method cannot generate more than 8 points. The MAM objective consistently leads to a higher separation angle. (b) The worst-case distortion ($\downarrow$, $D_{wc}$) of the constructed embedding of the phylogenetic tree with the maximal admissable $\tau$ given the number of bits. The vertical dashed line shows the limit with standard GPU floating point formats (float64). The horizontal dashed line is the best possible result $D_{wc} = 1$. FPE representations are required to get high quality embeddings without losing GPU-compatibility.}
        \label{fig:ablations}
    \end{minipage}
    \hfill
    \begin{minipage}[h]{0.3\textwidth}
        \centering
        \includegraphics[width=0.925\textwidth]{assets/hmds_vs_ms_dte_vertical.png}
        \vspace{-0.45cm}
        \captionof{figure}{\textbf{Pairwise relative distortions of h-MDS (top) and HS-DTE (bottom)} applied to the 5-ary tree with a scaling factor $\tau = 5.0$. Axes are ordered using a breadth-first search of the tree.}\label{fig:hmds_vs_ms_dte}
    \end{minipage}
    \vspace{-0.3cm}
\end{figure*}