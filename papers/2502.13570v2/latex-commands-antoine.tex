\usepackage{xspace}
\usepackage{braket} 	% for \Set{ … | … }
\usepackage{commath} % for \dif,  \od[2]{f}{x},  \pd[2]{f}{x}

%\usepackage{booktabs} % For nice tables
%\usepackage{boldline} % For hlineB{2}, \clineB{2-8}{2}: useful for rows with colored bg without extra spaces

\newcommand{\AC}[1]{{\color{purple}[A: #1]}}    
\newcommand{\NS}[1]{{\color{orange}[N: #1]}}
 

%%%%%%%%%%%%%%%%%%%%%%%%%%%%%%%%%%%%%%%%%%%%%%%%%%%%%%%%%%%%%%%%%%%%%%%
%                                Fonts                                %
%%%%%%%%%%%%%%%%%%%%%%%%%%%%%%%%%%%%%%%%%%%%%%%%%%%%%%%%%%%%%%%%%%%%%%%

%\DeclareMathAlphabet{\mathcal}{OMS}{cmsy}{m}{n}
%\let\mathbb\relax % remove the definition by unicode-math
%\DeclareMathAlphabet{\mathbb}{U}{msb}{m}{n}

%%%%%%%%%%%%%%%%%%%%%%%%%%%%%%%%%%%%%%%%%%%%%%%%%%%%%%%%%%%%%%%%%%%%%%%
%                Short macros mathbb/mathcal/mathfrak                 %
%%%%%%%%%%%%%%%%%%%%%%%%%%%%%%%%%%%%%%%%%%%%%%%%%%%%%%%%%%%%%%%%%%%%%%%

% Math Bb
\newcommand{\bC}{\mathbb{C}}
\newcommand{\bE}{\mathbb{E}}
\newcommand{\bN}{\mathbb{N}}
\newcommand{\bF}{\mathbb{F}}
\newcommand{\bK}{\mathbb{K}}
\newcommand{\bP}{\mathbb{P}}
\newcommand{\bR}{\mathbb{R}}
\newcommand{\bZ}{\mathbb{Z}}

\usepackage{pgffor}
\foreach \x in {a,...,z}{
  \expandafter\xdef\csname V\x \endcsname{\noexpand\ensuremath{\noexpand\V{\x}}}
}
\foreach \x in {A,...,Z}{
  \expandafter\xdef\csname V\x \endcsname{\noexpand\ensuremath{\noexpand\V{\x}}}
  \expandafter\xdef\csname c\x \endcsname{\noexpand\ensuremath{\noexpand\mathcal{\x}}}
  \expandafter\xdef\csname f\x \endcsname{\noexpand\ensuremath{\noexpand\mathfrak{\x}}}
}

%%%%%%%%%%%%%%%%%%%%%%%%%%%%%%%%%%%%%%%%%%%%%%%%%%%%%%%%%%%%%%%%%%%%%%%
%             Standard mathematical operators / commands              %
%%%%%%%%%%%%%%%%%%%%%%%%%%%%%%%%%%%%%%%%%%%%%%%%%%%%%%%%%%%%%%%%%%%%%%%

\DeclareMathOperator*{\Gr}{Gr}
\DeclareMathOperator*{\intr}{int}
\DeclareMathOperator*{\relint}{relint}
\DeclareMathOperator*{\relintr}{relint}
\DeclareMathOperator*{\sign}{sign}
\DeclareMathOperator*{\trc}{tr}
\DeclareMathOperator*{\im}{Im}
\DeclareMathOperator*{\rg}{rg}
\DeclareMathOperator*{\flp}{flip}
\DeclareMathOperator*{\adj}{adj}
%\DeclareMathOperator*{\argmin}{arg\,min}
%\DeclareMathOperator*{\argmax}{arg\,max}
\DeclareMathOperator*{\supp}{supp}
\DeclareMathOperator*{\rk}{rk}
\DeclareMathOperator*{\esssup}{ess\,sup}
%\DeclareMathOperator*{\ex}{exp}
\DeclareMathOperator*{\colspan}{colspan}
\DeclareMathOperator*{\rowspan}{rowspan}
%\DeclareMathOperator*{\var}{Var}
\newcommand{\var}{\text{Var}}
\newcommand{\Var}{\text{Var}}
\DeclareMathOperator*{\cov}{Cov}
\DeclareMathOperator*{\Cov}{Cov}
%\DeclareMathOperator*{\di}{dim}
\DeclareMathOperator*{\vol}{Vol}
\DeclareMathOperator*{\Tr}{tr}
\DeclareMathOperator*{\spa}{span}
%\DeclareMathOperator*{\vect}{vec}
%\newcommand{\V}[1]{\symbf{#1}} % for pdflatex
\ifLuaTeX
	\newcommand{\V}[1]{\symbf{#1}} % depends on unicode-math?
\else
	\usepackage{bm}
	\newcommand{\V}[1]{\bm{#1}} % for pdflatex
\fi
\DeclareMathOperator*{\veh}{vech}
\DeclareMathOperator*{\diag}{diag}
\DeclareMathOperator*{\erfc}{erfc}
\DeclareMathOperator*{\conv}{Conv}
%\DeclareMathOperator*{\ran}{ran}
\DeclareMathOperator*{\polylog}{polylog}

%%%%%%%%%%%%%%%%%%%%%%%%%%%%%%%%%%%%%%%%%%%%%%%%%%%%%%%%%%%%%%%%%%%%%%%
%                           Useful commands                           %
%%%%%%%%%%%%%%%%%%%%%%%%%%%%%%%%%%%%%%%%%%%%%%%%%%%%%%%%%%%%%%%%%%%%%%%

\newcommand{\cred}[1]{{\color{red}#1}}
\newcommand{\cgreen}[1]{{\color{ForestGreen}#1}}
\newcommand{\cblue}[1]{{\color{blue}#1}}
\newcommand{\note}[1]{{\color{blue}[#1]}}
\newcommand{\todo}[1]{{\color{red}[TODO: #1]}}
\newcommand{\ttodo}[1]{\text{{\color{red}[TODO: #1]}}}
\newcommand{\shorttodo}[1]{{\color{red}(#1)}}

% Useful 
\newcommand{\tbf}[1]{\textbf{#1}}			
\newcommand\tlt{$\leadsto$\ }
\newcommand\fblah[2]{
	\noindent\begin{tabularx}{\linewidth}{@{}r@{\ }X@{}}
		#1 & #2
	\end{tabularx}
}
\newcommand\lac[1]{\begin{center}{\Large #1}\end{center}}
%\newcommand\de{\triangleq}
\newcommand\de{:=}
\renewcommand\eg{e.g.\xspace }
\renewcommand\ie{i.e.\xspace }
\newcommand\wrt{w.r.t.\xspace }
\newcommand\whp{w.h.p.\xspace}
\newcommand{\st}{\text{ s.t. }\xspace}
%\newcommand\irange[1]{\llbracket #1 \rrbracket}
%\newcommand\irange[1]{\brk*{#1}}
\NewDocumentCommand\irange{O{1}m}{\brk*{#1,…,#2}}
\newcommand{\diid}{\overset{{\small{i.i.d.}}}{\sim}}
\newcommand{\diidl}{\sim_{i.i.d.}}
\newcommand{\tiid}{i.i.d.\ }

\newcommand{\hyp}{\textbf{Hypothesis:}\ }
%\newcommand{\nt}{\textbf{Notation:}\ }
\providecommand{\overbar}[1]{\mkern 1.8mu\overline{\mkern-1.8mu#1\mkern-1.8mu}\mkern 1.8mu}
\newcommand\numberthis{\addtocounter{equation}{1}\tag{\theequation}}

\newcommand\warning{\raisebox{0.2ex}{{\fontencoding{U}\fontfamily{futs}\selectfont\char 66\relax}}}
%\newcommand\wa\warning
\newcommand{\ra}[1]{\renewcommand{\arraystretch}{#1}}
\newcommand{\na}{n/a}
\newcommand\restr[2]{{% we make the whole thing an ordinary symbol
  \left.\kern-\nulldelimiterspace % automatically resize the bar with \right
  #1 % the function
  \vphantom{\big|} % pretend it's a little taller at normal size
  \right|_{#2} % this is the delimiter
  }}
\newcommand\citel[1]{\cblue{[\citetitle{#1}, \citeyear{#1} \cite{#1}]}}
\newcommand\citeb[1]{[\citetitle{#1} \citeyear{#1} \citeauthor{#1}]}
%\newcommand\citetya[1]{\cblue{[\citetitle{#1}, \citeyear{#1}, \citeauthor{#1} \cite{#1}]}} % cite title year author
\newcommand\citetya[1]{\cblue{[\citetitle{#1}, \citeyear{#1}, \cite{#1}]}} % cite title year author
\newcommand{\cleq}{\preccurlyeq}			

% French
\newcommand\cad{c.-à-d.\xspace}
\newcommand\pex{p.ex.\xspace}

%\newcommand*\diff{\mathop{}\!\mathrm{d}} % use \dif instead
\newcommand{\had}{\circ}
\newcommand{\kron}{\otimes}
\newcommand{\teq}{\triangleq}
\newcommand{\bI}{\mathds{1}}	% ??
\newcommand{\vvec}[1]{\begin{bmatrix}#1\end{bmatrix}}

% Environments 
% Depending on the environment, this might already be defined
%\ifcsmacro{definition}{}{
	%\let\endmyenvironment\undefined%
	%\newtheorem{definition}{Definition}
%}
%\ifcsmacro{lemma}{}{
	%\let\endmyenvironment\undefined%
	%\newtheorem{lemma}{Lemma}
%}
%\ifcsmacro{theorem}{}{
	%\let\endmyenvironment\undefined%
	%\newtheorem{theorem}{Theorem}
%}

\newcommand{\E}{\mathbf{E}}
\newcommand{\GMM}{\text{GMM}}
\newcommand{\MMD}{\textup{MMD}}
%\renewcommand{\P}{\mathbb{P}}

%\RenewDocumentCommand{\P}{e{_}sd[]}{%
	%\IfBooleanTF{#2}{%
	%\mathbb{P}\IfValueT{#1}{_{#1}}\IfValueT{#3}{\brk*{#3}}%
	%}{%
	%\mathbb{P}\IfValueT{#1}{_{#1}}\IfValueT{#3}{\brk{#3}}%
	%}
%}

\providecommand\given{}
% \brkg is similar to \brk, but with \given defined in the body
\DeclarePairedDelimiterX{\brkg}[1]{[}{]}{
\renewcommand\given{\nonscript\:\delimsize\vert\nonscript\:\mathopen{}}%
#1}
% Macro for probas and conditional probas (using the \given macro), accepting embelishment and a star (to scale the deliminters and the bar of the conditional or not)
\RenewDocumentCommand{\P}{se{_^}sd[]}{%
	\mathbb{P}\IfValueT{#2}{_{#2}}\IfValueT{#3}{^{#3}}%
	\IfBooleanTF{#1}{%
		% There is (at least) one star
		\IfValueT{#5}{\brkg*{#5}}%
	}{%
		\IfBooleanTF{#4}{%
			% Same code as above, but for when the star is given after embelishments
			\IfValueT{#5}{\brkg*{#5}}%
		}{%
			% No star
			\IfValueT{#5}{\brkg{#5}}%
		}%
	}
}
%\NewDocumentCommand{\E}{se{_^}sd[]}{%
	%\mathbf{E}\IfValueT{#2}{_{#2}}\IfValueT{#3}{^{#3}}%
	%\IfBooleanTF{#1}{%
		%% There is (at least) one star
		%\IfValueT{#5}{\brkg*{#5}}%
	%}{%
		%\IfBooleanTF{#4}{%
			%% Same code as above, but for when the star is given after embelishments
			%\IfValueT{#5}{\brkg*{#5}}%
		%}{%
			%% No star
			%\IfValueT{#5}{\brkg{#5}}%
		%}%
	%}
%}

% Norms
\DeclarePairedDelimiter{\ve}{[}{]}
\DeclarePairedDelimiter{\prt}{(}{)}
\DeclarePairedDelimiter{\brk}{[}{]}
\DeclarePairedDelimiter{\cb}{\{}{\}}
\let\norm\relax
\DeclarePairedDelimiter{\norm}{\lVert}{\rVert}
\DeclarePairedDelimiter{\n}{\lVert}{\rVert}
\DeclarePairedDelimiter{\ip}{\langle}{\rangle}
\DeclarePairedDelimiter{\ipHS}{\langle}{\rangle_{\cL²(\rkhs)}}
\DeclarePairedDelimiter{\absv}{|}{|}
\DeclarePairedDelimiter{\nHS}{\lVert}{\rVert_{\cL²(\rkhs)}}	% HS
\DeclarePairedDelimiter{\nF}{\lVert}{\rVert_{\textup{F}}}	% Frobenius

% For normvvv (triple norm). Alternatively just \usepackage{mathabx} %%%%%%%%%%%%%%%%%%%%%%%%%%%%%%%%%%%%%%
\DeclareFontFamily{U}{matha}{\hyphenchar\font45}
\DeclareFontShape{U}{matha}{m}{n}{
<-6> matha5 <6-7> matha6 <7-8> matha7
<8-9> matha8 <9-10> matha9
<10-12> matha10 <12-> matha12
}{}
\DeclareSymbolFont{matha}{U}{matha}{m}{n}

\DeclareFontFamily{U}{mathx}{\hyphenchar\font45}
\DeclareFontShape{U}{mathx}{m}{n}{
<-6> mathx5 <6-7> mathx6 <7-8> mathx7
<8-9> mathx8 <9-10> mathx9
<10-12> mathx10 <12-> mathx12
}{}
\DeclareSymbolFont{mathx}{U}{mathx}{m}{n}

\DeclareMathDelimiter{\vvvert} {0}{matha}{"7E}{mathx}{"17}%
\DeclarePairedDelimiterX{\normiii}[1]
{\vvvert}
{\vvvert}
{\ifblank{#1}{\:\cdot\:}{#1}}
%%%%%%%%%%%%%%%%%%%%%%%%%%%%%%%%%%%%%%%%%%%%%%%%%%%%%%%%%%%%%%%%%%%%%%%%%%%%

\newcommand{\nfc}[1]{\nocst{#1}_{\cF}}

\newcommand*\circled[1]{\tikz[baseline=(char.base)]{
   \node[shape=circle,draw,inner sep=1.5pt,thick] (char) {#1};}}



%%%%%%%%%%%%%%%%%%%%%%%%%%%%%%%%%%%%%%%%%%%%%%%%%%%%%%%%%%%%%%%%%%%%%%%
%                            Image loading                            %
%%%%%%%%%%%%%%%%%%%%%%%%%%%%%%%%%%%%%%%%%%%%%%%%%%%%%%%%%%%%%%%%%%%%%%%

%\newcommand\mathregular[1]{#1}

%\NewDocumentCommand\inputpgf{O{.}m}{
%\let\pgfimageWithoutPath\pgfimage
%\renewcommand{\pgfimage}[2][]{\pgfimageWithoutPath[##1]{#1/##2}}
%\let\includegraphicsWithoutPath\includegraphics
%\renewcommand{\includegraphics}[2][]{\includegraphicsWithoutPath[##1]{#1/##2}}
%\begingroup\renewcommand\sffamily{}\input{#1/#2}\endgroup
%}
%\renewcommand\inputpgfc[1]{\inputpgf[figures/]{#1}}

%%%%%%%%%%%%%%%%%%%%%%%%%%%%%%%%%%%%%%%%%%%%%%%%%%%%%%%%%%%%%%%%%%%%%%%
%                               Tables                                %
%%%%%%%%%%%%%%%%%%%%%%%%%%%%%%%%%%%%%%%%%%%%%%%%%%%%%%%%%%%%%%%%%%%%%%%

\usepackage{pifont}
\newcommand{\cmark}{\ding{51}}%
\newcommand{\xmark}{\ding{55}}%
\newcommand{\yes}{{\color{ForestGreen}\cmark}}
\newcommand{\no}{{\color{red}\xmark}}%

\newcommand\fitline[1]{\resizebox{\linewidth}{!}{#1}}

