
\documentclass[]{fairmeta}


\usepackage{microtype}
\usepackage{graphicx}
\usepackage{subcaption}
\usepackage{booktabs} %

\usepackage{hyperref}


\newcommand{\theHalgorithm}{\arabic{algorithm}}


\usepackage{tcolorbox}

\usepackage{amsmath}
\usepackage{amssymb}
\usepackage{mathtools}
\usepackage{amsthm}

\newcommand{\kd}[1]{\textcolor{blue}{kd: #1}}
\newcommand{\ca}[1]{\textcolor{red}{CA: #1}}
\newcommand{\viv}[1]{\textcolor{red}{VC: #1}}
\definecolor{myred}{HTML}{CB4154}

\usepackage{amsmath}
\usepackage{bbm}
\usepackage{booktabs}
\newcommand{\indicator}[1]{\mathbbm{1}\left\lbrace #1\right\rbrace}
\newcommand{\thrsoracle}{\tau_{\text{oracle}}}
\newcommand{\thrsapprox}{\tau_{\text{approx}}}
\newcommand{\thrsstatic}{\tau_{\text{static}}}

\newcommand{\tbulk}{T^{\text{bulk}}}

\newcommand{\oraclecriterion}{c^h_{\text{oracle}}}


\usepackage{listings}
\usepackage{xcolor}
\definecolor{keyword}{rgb}{0.75, 0.13, 0.13}
\definecolor{comment}{rgb}{0.25, 0.5, 0.35}
\definecolor{string}{rgb}{0.6, 0.1, 0.1}
\usepackage{listings}
\usepackage{xcolor}







\definecolor{codeblue}{rgb}{0.13, 0.13, 0.75}
\definecolor{codegray}{rgb}{0.5, 0.5, 0.5}
\definecolor{codepurple}{rgb}{0.58, 0.0, 0.82}
\definecolor{backcolour}{rgb}{0.98, 0.98, 0.98}

\definecolor{functionblue}{RGB}{67,110,238}    %
\definecolor{variablegreen}{RGB}{102,153,0}    %
\definecolor{torchfunc}{RGB}{255,140,0}        %
\definecolor{outputpurple}{RGB}{147,112,219}   %
\definecolor{keywordcolor}{RGB}{0,119,170}     %


\definecolor{lightgreen}{RGB}{76,175,80}     %
\definecolor{darkgreen}{RGB}{46,125,50}      %
\definecolor{torchfunc}{RGB}{255,140,0}      %
\definecolor{outputpurple}{RGB}{147,112,219} %

\usepackage{listings}
\usepackage{xcolor}

\definecolor{backcolour}{rgb}{0.99,0.99,0.99}
\definecolor{codegray}{rgb}{0.5,0.5,0.5}
\definecolor{codeblue}{rgb}{0.0,0.0,0.5}
\definecolor{codepurple}{rgb}{0.58,0.0,0.82}
\definecolor{orange}{rgb}{1.0,0.5,0.0}

\lstset{
    language=Python,
    backgroundcolor=\color{backcolour},
    basicstyle=\ttfamily\footnotesize,
    breakatwhitespace=true,
    breaklines=true,
    captionpos=b,
    commentstyle=\color{codegray}\itshape,
    keepspaces=true,
    keywordstyle=\color{codeblue}\bfseries,
    numbers=left,
    numbersep=5pt,
    numberstyle=\tiny\color{codegray},
    showspaces=false,
    showstringspaces=false,
    showtabs=false,
    stringstyle=\color{codepurple},
    tabsize=4,
    frame=single,
    rulecolor=\color{black},
    emphstyle=\color{orange},
    emph={einsum, lse, stack, local_attn_, dense_attn_, sqrt, log},
    morekeywords={def, return}
}
\theoremstyle{plain}
\newtheorem{theorem}{Theorem}[section]
\newtheorem{proposition}[theorem]{Proposition}
\newtheorem{lemma}[theorem]{Lemma}
\newtheorem{corollary}[theorem]{Corollary}
\theoremstyle{definition}
\newtheorem{definition}[theorem]{Definition}
\newtheorem{assumption}[theorem]{Assumption}
\theoremstyle{remark}
\newtheorem{remark}[theorem]{Remark}



\usepackage[textsize=tiny]{todonotes}


\title{Unveiling Simplicities of Attention:\\ Adaptive Long-Context Head Identification}

\author[2,*]{Konstantin Donhauser}
\author[1]{Charles Arnal}
\author[1]{Mohammad Pezeshki}
\author[1]{Vivien Cabannes}
\author[1]{David Lopez-Paz}
\author[1]{Kartik Ahuja}

\affiliation[1]{FAIR at Meta}
\affiliation[2]{ETH Zurich}

\contribution[*]{Work done at Meta}

\abstract{
The ability to process long contexts is crucial for many natural language processing tasks, yet it remains a significant challenge. While substantial progress has been made in enhancing the efficiency of attention mechanisms, there is still a gap in understanding how attention heads function in long-context settings. In this paper, we observe that while certain heads consistently attend to local information only, others swing between attending to local and long-context information depending on the query. This raises the question: can we identify which heads require long-context information to predict the next token accurately? We demonstrate that it's possible to predict which heads are crucial for long-context processing using only local keys. The core idea here is to exploit a simple model for the long-context scores via second moment approximations. These findings unveil simple properties of attention in the context of long sequences, and open the door to potentially significant gains in efficiency.
}
\usepackage{amsthm}
\usepackage{amsmath}
\usepackage{amssymb}
\usepackage{algorithm}
\usepackage{algorithmic}
\usepackage{subcaption}

\date{\today}
\correspondence{First Author at \email{konstantin.donhauser@ai.ethz.ch}}


\let\oldparagraph\paragraph
\renewcommand{\paragraph}[1]{\textbf{#1}}
\usepackage{amsmath}



\begin{document}

\maketitle






\section{Introduction}
\label{section:intro}



The landscape of large language models (LLMs) is rapidly evolving, with modern architectures capable of generating text from vast contexts. Recent advances have led to a significant increase in context window sizes, with Llama 3 \citep{dubey2024llama}, DeepSeekv3 \citep{liu2024deepseek}, and Gemini \citep{team2024gemini} supporting windows of at least 128k. 
However, long context modeling still poses significant challenges \citep{hsieh2024ruler} in terms of both accuracy  and  the substantial cost of processing long contexts in terms of memory and run-time compute. 
In spite of their importance, our current comprehension of the attention mechanism in long-context tasks remains incomplete. This work aims to address some of these knowledge gaps.

Despite the overwhelming complexity of state-of-the-art models, certain simple behaviors in the attention mechanism are strikingly consistent. In particular, many forms of sparse behaviors have been consistently observed, and exploited by numerous methods for efficient inference (see Section~\ref{sec:relatedworks}). 
Among them, \citet{xiao2023efficient} showed that even when computing the attention only using tokens close to the current token plus initial ``sink'' tokens, as illustrated in Figure~\ref{fig:gaussian},  the model is still capable of generating fluent text. We refer to these tokens as local window, and always implicitly include the   initial tokens as they play a crucial role as an attention ``sink'' (see also \citet{chen2024magicpig,gu2024attention,sun2024massive}). 

However, such a local window approximation, if applied to every attention head simultaneously, necessarily harms the capabilities of LLMs to retrieve and process long-context information (see e.g., \citet{xiao2024duoattention}).
Instead, to overcome such limitations, we aim to identify the heads whose output can be well-approximated via a local window attention, and apply the approximation to those only. If a head can be approximated via a local approximation, we call it a \textbf{local head}, and otherwise it is a \textbf{long-context head}. 
In particular, we ask:     Which heads can be approximated using a local window with minimal impact on downstream task performance?

 \begin{figure*}[ht]
\centering
\begin{subfigure}[b]{\linewidth}
\centering
        \includegraphics[width=1.0\linewidth]{plots/Figure1.pdf}\\
        \begin{subfigure}[b]{\linewidth}
    \end{subfigure}
\end{subfigure}
    \vspace{-1cm}
     \caption{
    \small{ 
    \textbf{Attention sparsity and its impact on efficiency.} 
    \textit{Left:} Attention scores are split into \textit{bulk} ($A^{\text{bulk}}$) for distant tokens and \textit{local window} ($A^{\text{local}}$) for nearby ones. A head is considered local if most of its attention mass falls within the local window. The static criterion pre-assigns local heads, while the adaptive oracle query-dependently compares bulk and local contributions but is computationally expensive. Our approximation models $A^{\text{bulk}}$ using a Gaussian distribution for efficiency.
    \textit{Middle:} Oracle-based classification with $\tau = 0.6$ (see Figure~\ref{fig:compare-approx} for the threshold) reveals three types of heads: consistently local (heads labeled more than $95\%$ of the times as local), often long-context (less than $50\%$), and varying, which switch behavior dynamically.
    \textit{Right:} Comparison of three methods: Static (green) removes a fixed fraction of heads, the adaptive oracle (blue) dynamically selects heads but is costly, and our adaptive method (purple) achieves near-oracle performance with significantly lower cost. As sparsity increases, static pruning degrades performance, while our adaptive method remains robust.
    These results show that \textit{most attention heads do not need to attend to the entire context}, enabling significant efficiency gains with \textit{query-adaptive} head classification.} }
    \label{fig:gaussian}
\end{figure*}



Two approaches to this problem can be distinguished:
  \textit{Static} criteria label the heads -- local vs long-context --  once for all queries, while \textit{query-adaptive} criteria change the labels from query to query.  Static criteria, as used by \citet{xiao2024duoattention,tang2024razorattention}, have the advantage that all key-value pairs (except for the few in the local window) of local heads can be discarded, thus saving memory. While recent works \citep{wu2024retrieval,tang2024razorattention,hong2024token} 
 provide some evidence that a \textit{fixed} small subset of the heads are particularly relevant for processing long-context information, the following question remains unclear:
 \begin{center}
     \textit{How much sparsity (measured as the average percentage of local heads) can we gain using query-adaptive criteria compared to static criteria?}
     \end{center} 


\paragraph{Contribution 1.} We present an extensive analysis comparing a query-adaptive oracle criterion, which selects local heads independently for each token, with static criteria. We make two observations: first, we find that static criteria can label up to 60\% of the heads as local heads without impacting downstream task evaluations, which confirms the intuition from \citep{wu2024retrieval}. Nevertheless, we find that a query-adaptive oracle criterion allows to label a substantially higher percentage of heads as local heads (up to 90\%) without sacrificing performance (see Figure~\ref{fig:gaussian}).


Unfortunately, the oracle requires the computation of the full attention scores. This leads to the following question:
\begin{center}
    \textit{ For each query, can we determine which heads are long-context and which are local without computing the full attention scores?}
\end{center}

The relevance of this question is twofold: on one hand, answering it helps guide further research toward developing more compute-efficient attention mechanisms. On the other hand, it advances our understanding of the inner workings of attention mechanisms, which is central to mechanistic interpretability (see, e.g., \citet{olsson2022context}). 

\paragraph{Contribution 2.} We address this question by proposing a novel query-adaptive attention criterion (QAdA) based on second-order statistics of the attention scores (briefly summarized in Figure~\ref{fig:gaussian}).
Our experiments on three families of LLMs, Llama \citep{dubey2024llama}, Qwen \citep{bai2023qwen} and Mistral \citep{jiang2023mistral} applied to a variety of standard long-context benchmarks, as well as hard  reasoning tasks embedded in long-context prompts, show that this relatively simple criterion allows to efficiently identify long-context heads: our method increased sparsity at a smaller loss in downstream performance than oracle static approaches. 
Along with our other experiments, it sheds light onto certain simple behaviors of attention heads in long-context settings. 


 



\section{RELATED WORK}
\label{sec:relatedwork}
In this section, we describe the previous works related to our proposal, which are divided into two parts. In Section~\ref{sec:relatedwork_exoplanet}, we present a review of approaches based on machine learning techniques for the detection of planetary transit signals. Section~\ref{sec:relatedwork_attention} provides an account of the approaches based on attention mechanisms applied in Astronomy.\par

\subsection{Exoplanet detection}
\label{sec:relatedwork_exoplanet}
Machine learning methods have achieved great performance for the automatic selection of exoplanet transit signals. One of the earliest applications of machine learning is a model named Autovetter \citep{MCcauliff}, which is a random forest (RF) model based on characteristics derived from Kepler pipeline statistics to classify exoplanet and false positive signals. Then, other studies emerged that also used supervised learning. \cite{mislis2016sidra} also used a RF, but unlike the work by \citet{MCcauliff}, they used simulated light curves and a box least square \citep[BLS;][]{kovacs2002box}-based periodogram to search for transiting exoplanets. \citet{thompson2015machine} proposed a k-nearest neighbors model for Kepler data to determine if a given signal has similarity to known transits. Unsupervised learning techniques were also applied, such as self-organizing maps (SOM), proposed \citet{armstrong2016transit}; which implements an architecture to segment similar light curves. In the same way, \citet{armstrong2018automatic} developed a combination of supervised and unsupervised learning, including RF and SOM models. In general, these approaches require a previous phase of feature engineering for each light curve. \par

%DL is a modern data-driven technology that automatically extracts characteristics, and that has been successful in classification problems from a variety of application domains. The architecture relies on several layers of NNs of simple interconnected units and uses layers to build increasingly complex and useful features by means of linear and non-linear transformation. This family of models is capable of generating increasingly high-level representations \citep{lecun2015deep}.

The application of DL for exoplanetary signal detection has evolved rapidly in recent years and has become very popular in planetary science.  \citet{pearson2018} and \citet{zucker2018shallow} developed CNN-based algorithms that learn from synthetic data to search for exoplanets. Perhaps one of the most successful applications of the DL models in transit detection was that of \citet{Shallue_2018}; who, in collaboration with Google, proposed a CNN named AstroNet that recognizes exoplanet signals in real data from Kepler. AstroNet uses the training set of labelled TCEs from the Autovetter planet candidate catalog of Q1–Q17 data release 24 (DR24) of the Kepler mission \citep{catanzarite2015autovetter}. AstroNet analyses the data in two views: a ``global view'', and ``local view'' \citep{Shallue_2018}. \par


% The global view shows the characteristics of the light curve over an orbital period, and a local view shows the moment at occurring the transit in detail

%different = space-based

Based on AstroNet, researchers have modified the original AstroNet model to rank candidates from different surveys, specifically for Kepler and TESS missions. \citet{ansdell2018scientific} developed a CNN trained on Kepler data, and included for the first time the information on the centroids, showing that the model improves performance considerably. Then, \citet{osborn2020rapid} and \citet{yu2019identifying} also included the centroids information, but in addition, \citet{osborn2020rapid} included information of the stellar and transit parameters. Finally, \citet{rao2021nigraha} proposed a pipeline that includes a new ``half-phase'' view of the transit signal. This half-phase view represents a transit view with a different time and phase. The purpose of this view is to recover any possible secondary eclipse (the object hiding behind the disk of the primary star).


%last pipeline applies a procedure after the prediction of the model to obtain new candidates, this process is carried out through a series of steps that include the evaluation with Discovery and Validation of Exoplanets (DAVE) \citet{kostov2019discovery} that was adapted for the TESS telescope.\par
%



\subsection{Attention mechanisms in astronomy}
\label{sec:relatedwork_attention}
Despite the remarkable success of attention mechanisms in sequential data, few papers have exploited their advantages in astronomy. In particular, there are no models based on attention mechanisms for detecting planets. Below we present a summary of the main applications of this modeling approach to astronomy, based on two points of view; performance and interpretability of the model.\par
%Attention mechanisms have not yet been explored in all sub-areas of astronomy. However, recent works show a successful application of the mechanism.
%performance

The application of attention mechanisms has shown improvements in the performance of some regression and classification tasks compared to previous approaches. One of the first implementations of the attention mechanism was to find gravitational lenses proposed by \citet{thuruthipilly2021finding}. They designed 21 self-attention-based encoder models, where each model was trained separately with 18,000 simulated images, demonstrating that the model based on the Transformer has a better performance and uses fewer trainable parameters compared to CNN. A novel application was proposed by \citet{lin2021galaxy} for the morphological classification of galaxies, who used an architecture derived from the Transformer, named Vision Transformer (VIT) \citep{dosovitskiy2020image}. \citet{lin2021galaxy} demonstrated competitive results compared to CNNs. Another application with successful results was proposed by \citet{zerveas2021transformer}; which first proposed a transformer-based framework for learning unsupervised representations of multivariate time series. Their methodology takes advantage of unlabeled data to train an encoder and extract dense vector representations of time series. Subsequently, they evaluate the model for regression and classification tasks, demonstrating better performance than other state-of-the-art supervised methods, even with data sets with limited samples.

%interpretation
Regarding the interpretability of the model, a recent contribution that analyses the attention maps was presented by \citet{bowles20212}, which explored the use of group-equivariant self-attention for radio astronomy classification. Compared to other approaches, this model analysed the attention maps of the predictions and showed that the mechanism extracts the brightest spots and jets of the radio source more clearly. This indicates that attention maps for prediction interpretation could help experts see patterns that the human eye often misses. \par

In the field of variable stars, \citet{allam2021paying} employed the mechanism for classifying multivariate time series in variable stars. And additionally, \citet{allam2021paying} showed that the activation weights are accommodated according to the variation in brightness of the star, achieving a more interpretable model. And finally, related to the TESS telescope, \citet{morvan2022don} proposed a model that removes the noise from the light curves through the distribution of attention weights. \citet{morvan2022don} showed that the use of the attention mechanism is excellent for removing noise and outliers in time series datasets compared with other approaches. In addition, the use of attention maps allowed them to show the representations learned from the model. \par

Recent attention mechanism approaches in astronomy demonstrate comparable results with earlier approaches, such as CNNs. At the same time, they offer interpretability of their results, which allows a post-prediction analysis. \par


\section{Method}\label{sec:method}
\begin{figure}
    \centering
    \includegraphics[width=0.85\textwidth]{imgs/heatmap_acc.pdf}
    \caption{\textbf{Visualization of the proposed periodic Bayesian flow with mean parameter $\mu$ and accumulated accuracy parameter $c$ which corresponds to the entropy/uncertainty}. For $x = 0.3, \beta(1) = 1000$ and $\alpha_i$ defined in \cref{appd:bfn_cir}, this figure plots three colored stochastic parameter trajectories for receiver mean parameter $m$ and accumulated accuracy parameter $c$, superimposed on a log-scale heatmap of the Bayesian flow distribution $p_F(m|x,\senderacc)$ and $p_F(c|x,\senderacc)$. Note the \emph{non-monotonicity} and \emph{non-additive} property of $c$ which could inform the network the entropy of the mean parameter $m$ as a condition and the \emph{periodicity} of $m$. %\jj{Shrink the figures to save space}\hanlin{Do we need to make this figure one-column?}
    }
    \label{fig:vmbf_vis}
    \vskip -0.1in
\end{figure}
% \begin{wrapfigure}{r}{0.5\textwidth}
%     \centering
%     \includegraphics[width=0.49\textwidth]{imgs/heatmap_acc.pdf}
%     \caption{\textbf{Visualization of hyper-torus Bayesian flow based on von Mises Distribution}. For $x = 0.3, \beta(1) = 1000$ and $\alpha_i$ defined in \cref{appd:bfn_cir}, this figure plots three colored stochastic parameter trajectories for receiver mean parameter $m$ and accumulated accuracy parameter $c$, superimposed on a log-scale heatmap of the Bayesian flow distribution $p_F(m|x,\senderacc)$ and $p_F(c|x,\senderacc)$. Note the \emph{non-monotonicity} and \emph{non-additive} property of $c$. \jj{Shrink the figures to save space}}
%     \label{fig:vmbf_vis}
%     \vspace{-30pt}
% \end{wrapfigure}


In this section, we explain the detailed design of CrysBFN tackling theoretical and practical challenges. First, we describe how to derive our new formulation of Bayesian Flow Networks over hyper-torus $\mathbb{T}^{D}$ from scratch. Next, we illustrate the two key differences between \modelname and the original form of BFN: $1)$ a meticulously designed novel base distribution with different Bayesian update rules; and $2)$ different properties over the accuracy scheduling resulted from the periodicity and the new Bayesian update rules. Then, we present in detail the overall framework of \modelname over each manifold of the crystal space (\textit{i.e.} fractional coordinates, lattice vectors, atom types) respecting \textit{periodic E(3) invariance}. 

% In this section, we first demonstrate how to build Bayesian flow on hyper-torus $\mathbb{T}^{D}$ by overcoming theoretical and practical problems to provide a low-noise parameter-space approach to fractional atom coordinate generation. Next, we present how \modelname models each manifold of crystal space respecting \textit{periodic E(3) invariance}. 

\subsection{Periodic Bayesian Flow on Hyper-torus \texorpdfstring{$\mathbb{T}^{D}$}{}} 
For generative modeling of fractional coordinates in crystal, we first construct a periodic Bayesian flow on \texorpdfstring{$\mathbb{T}^{D}$}{} by designing every component of the totally new Bayesian update process which we demonstrate to be distinct from the original Bayesian flow (please see \cref{fig:non_add}). 
 %:) 
 
 The fractional atom coordinate system \citep{jiao2023crystal} inherently distributes over a hyper-torus support $\mathbb{T}^{3\times N}$. Hence, the normal distribution support on $\R$ used in the original \citep{bfn} is not suitable for this scenario. 
% The key problem of generative modeling for crystal is the periodicity of Cartesian atom coordinates $\vX$ requiring:
% \begin{equation}\label{eq:periodcity}
% p(\vA,\vL,\vX)=p(\vA,\vL,\vX+\vec{LK}),\text{where}~\vec{K}=\vec{k}\vec{1}_{1\times N},\forall\vec{k}\in\mathbb{Z}^{3\times1}
% \end{equation}
% However, there does not exist such a distribution supporting on $\R$ to model such property because the integration of such distribution over $\R$ will not be finite and equal to 1. Therefore, the normal distribution used in \citet{bfn} can not meet this condition.

To tackle this problem, the circular distribution~\citep{mardia2009directional} over the finite interval $[-\pi,\pi)$ is a natural choice as the base distribution for deriving the BFN on $\mathbb{T}^D$. 
% one natural choice is to 
% we would like to consider the circular distribution over the finite interval as the base 
% we find that circular distributions \citep{mardia2009directional} defined on a finite interval with lengths of $2\pi$ can be used as the instantiation of input distribution for the BFN on $\mathbb{T}^D$.
Specifically, circular distributions enjoy desirable periodic properties: $1)$ the integration over any interval length of $2\pi$ equals 1; $2)$ the probability distribution function is periodic with period $2\pi$.  Sharing the same intrinsic with fractional coordinates, such periodic property of circular distribution makes it suitable for the instantiation of BFN's input distribution, in parameterizing the belief towards ground truth $\x$ on $\mathbb{T}^D$. 
% \yuxuan{this is very complicated from my perspective.} \hanlin{But this property is exactly beautiful and perfectly fit into the BFN.}

\textbf{von Mises Distribution and its Bayesian Update} We choose von Mises distribution \citep{mardia2009directional} from various circular distributions as the form of input distribution, based on the appealing conjugacy property required in the derivation of the BFN framework.
% to leverage the Bayesian conjugacy property of von Mises distribution which is required by the BFN framework. 
That is, the posterior of a von Mises distribution parameterized likelihood is still in the family of von Mises distributions. The probability density function of von Mises distribution with mean direction parameter $m$ and concentration parameter $c$ (describing the entropy/uncertainty of $m$) is defined as: 
\begin{equation}
f(x|m,c)=vM(x|m,c)=\frac{\exp(c\cos(x-m))}{2\pi I_0(c)}
\end{equation}
where $I_0(c)$ is zeroth order modified Bessel function of the first kind as the normalizing constant. Given the last univariate belief parameterized by von Mises distribution with parameter $\theta_{i-1}=\{m_{i-1},\ c_{i-1}\}$ and the sample $y$ from sender distribution with unknown data sample $x$ and known accuracy $\alpha$ describing the entropy/uncertainty of $y$,  Bayesian update for the receiver is deducted as:
\begin{equation}
 h(\{m_{i-1},c_{i-1}\},y,\alpha)=\{m_i,c_i \}, \text{where}
\end{equation}
\begin{equation}\label{eq:h_m}
m_i=\text{atan2}(\alpha\sin y+c_{i-1}\sin m_{i-1}, {\alpha\cos y+c_{i-1}\cos m_{i-1}})
\end{equation}
\begin{equation}\label{eq:h_c}
c_i =\sqrt{\alpha^2+c_{i-1}^2+2\alpha c_{i-1}\cos(y-m_{i-1})}
\end{equation}
The proof of the above equations can be found in \cref{apdx:bayesian_update_function}. The atan2 function refers to  2-argument arctangent. Independently conducting  Bayesian update for each dimension, we can obtain the Bayesian update distribution by marginalizing $\y$:
\begin{equation}
p_U(\vtheta'|\vtheta,\bold{x};\alpha)=\mathbb{E}_{p_S(\bold{y}|\bold{x};\alpha)}\delta(\vtheta'-h(\vtheta,\bold{y},\alpha))=\mathbb{E}_{vM(\bold{y}|\bold{x},\alpha)}\delta(\vtheta'-h(\vtheta,\bold{y},\alpha))
\end{equation} 
\begin{figure}
    \centering
    \vskip -0.15in
    \includegraphics[width=0.95\linewidth]{imgs/non_add.pdf}
    \caption{An intuitive illustration of non-additive accuracy Bayesian update on the torus. The lengths of arrows represent the uncertainty/entropy of the belief (\emph{e.g.}~$1/\sigma^2$ for Gaussian and $c$ for von Mises). The directions of the arrows represent the believed location (\emph{e.g.}~ $\mu$ for Gaussian and $m$ for von Mises).}
    \label{fig:non_add}
    \vskip -0.15in
\end{figure}
\textbf{Non-additive Accuracy} 
The additive accuracy is a nice property held with the Gaussian-formed sender distribution of the original BFN expressed as:
\begin{align}
\label{eq:standard_id}
    \update(\parsn{}'' \mid \parsn{}, \x; \alpha_a+\alpha_b) = \E_{\update(\parsn{}' \mid \parsn{}, \x; \alpha_a)} \update(\parsn{}'' \mid \parsn{}', \x; \alpha_b)
\end{align}
Such property is mainly derived based on the standard identity of Gaussian variable:
\begin{equation}
X \sim \mathcal{N}\left(\mu_X, \sigma_X^2\right), Y \sim \mathcal{N}\left(\mu_Y, \sigma_Y^2\right) \Longrightarrow X+Y \sim \mathcal{N}\left(\mu_X+\mu_Y, \sigma_X^2+\sigma_Y^2\right)
\end{equation}
The additive accuracy property makes it feasible to derive the Bayesian flow distribution $
p_F(\boldsymbol{\theta} \mid \mathbf{x} ; i)=p_U\left(\boldsymbol{\theta} \mid \boldsymbol{\theta}_0, \mathbf{x}, \sum_{k=1}^{i} \alpha_i \right)
$ for the simulation-free training of \cref{eq:loss_n}.
It should be noted that the standard identity in \cref{eq:standard_id} does not hold in the von Mises distribution. Hence there exists an important difference between the original Bayesian flow defined on Euclidean space and the Bayesian flow of circular data on $\mathbb{T}^D$ based on von Mises distribution. With prior $\btheta = \{\bold{0},\bold{0}\}$, we could formally represent the non-additive accuracy issue as:
% The additive accuracy property implies the fact that the "confidence" for the data sample after observing a series of the noisy samples with accuracy ${\alpha_1, \cdots, \alpha_i}$ could be  as the accuracy sum  which could be  
% Here we 
% Here we emphasize the specific property of BFN based on von Mises distribution.
% Note that 
% \begin{equation}
% \update(\parsn'' \mid \parsn, \x; \alpha_a+\alpha_b) \ne \E_{\update(\parsn' \mid \parsn, \x; \alpha_a)} \update(\parsn'' \mid \parsn', \x; \alpha_b)
% \end{equation}
% \oyyw{please check whether the below equation is better}
% \yuxuan{I fill somehow confusing on what is the update distribution with $\alpha$. }
% \begin{equation}
% \update(\parsn{}'' \mid \parsn{}, \x; \alpha_a+\alpha_b) \ne \E_{\update(\parsn{}' \mid \parsn{}, \x; \alpha_a)} \update(\parsn{}'' \mid \parsn{}', \x; \alpha_b)
% \end{equation}
% We give an intuitive visualization of such difference in \cref{fig:non_add}. The untenability of this property can materialize by considering the following case: with prior $\btheta = \{\bold{0},\bold{0}\}$, check the two-step Bayesian update distribution with $\alpha_a,\alpha_b$ and one-step Bayesian update with $\alpha=\alpha_a+\alpha_b$:
\begin{align}
\label{eq:nonadd}
     &\update(c'' \mid \parsn, \x; \alpha_a+\alpha_b)  = \delta(c-\alpha_a-\alpha_b)
     \ne  \mathbb{E}_{p_U(\parsn' \mid \parsn, \x; \alpha_a)}\update(c'' \mid \parsn', \x; \alpha_b) \nonumber \\&= \mathbb{E}_{vM(\bold{y}_b|\bold{x},\alpha_a)}\mathbb{E}_{vM(\bold{y}_a|\bold{x},\alpha_b)}\delta(c-||[\alpha_a \cos\y_a+\alpha_b\cos \y_b,\alpha_a \sin\y_a+\alpha_b\sin \y_b]^T||_2)
\end{align}
A more intuitive visualization could be found in \cref{fig:non_add}. This fundamental difference between periodic Bayesian flow and that of \citet{bfn} presents both theoretical and practical challenges, which we will explain and address in the following contents.

% This makes constructing Bayesian flow based on von Mises distribution intrinsically different from previous Bayesian flows (\citet{bfn}).

% Thus, we must reformulate the framework of Bayesian flow networks  accordingly. % and do necessary reformulations of BFN. 

% \yuxuan{overall I feel this part is complicated by using the language of update distribution. I would like to suggest simply use bayesian update, to provide intuitive explantion.}\hanlin{See the illustration in \cref{fig:non_add}}

% That introduces a cascade of problems, and we investigate the following issues: $(1)$ Accuracies between sender and receiver are not synchronized and need to be differentiated. $(2)$ There is no tractable Bayesian flow distribution for a one-step sample conditioned on a given time step $i$, and naively simulating the Bayesian flow results in computational overhead. $(3)$ It is difficult to control the entropy of the Bayesian flow. $(4)$ Accuracy is no longer a function of $t$ and becomes a distribution conditioned on $t$, which can be different across dimensions.
%\jj{Edited till here}

\textbf{Entropy Conditioning} As a common practice in generative models~\citep{ddpm,flowmatching,bfn}, timestep $t$ is widely used to distinguish among generation states by feeding the timestep information into the networks. However, this paper shows that for periodic Bayesian flow, the accumulated accuracy $\vc_i$ is more effective than time-based conditioning by informing the network about the entropy and certainty of the states $\parsnt{i}$. This stems from the intrinsic non-additive accuracy which makes the receiver's accumulated accuracy $c$ not bijective function of $t$, but a distribution conditioned on accumulated accuracies $\vc_i$ instead. Therefore, the entropy parameter $\vc$ is taken logarithm and fed into the network to describe the entropy of the input corrupted structure. We verify this consideration in \cref{sec:exp_ablation}. 
% \yuxuan{implement variant. traditionally, the timestep is widely used to distinguish the different states by putting the timestep embedding into the networks. citation of FM, diffusion, BFN. However, we find that conditioned on time in periodic flow could not provide extra benefits. To further boost the performance, we introduce a simple yet effective modification term entropy conditional. This is based on that the accumulated accuracy which represents the current uncertainty or entropy could be a better indicator to distinguish different states. + Describe how you do this. }



\textbf{Reformulations of BFN}. Recall the original update function with Gaussian sender distribution, after receiving noisy samples $\y_1,\y_2,\dots,\y_i$ with accuracies $\senderacc$, the accumulated accuracies of the receiver side could be analytically obtained by the additive property and it is consistent with the sender side.
% Since observing sample $\y$ with $\alpha_i$ can not result in exact accuracy increment $\alpha_i$ for receiver, the accuracies between sender and receiver are not synchronized which need to be differentiated. 
However, as previously mentioned, this does not apply to periodic Bayesian flow, and some of the notations in original BFN~\citep{bfn} need to be adjusted accordingly. We maintain the notations of sender side's one-step accuracy $\alpha$ and added accuracy $\beta$, and alter the notation of receiver's accuracy parameter as $c$, which is needed to be simulated by cascade of Bayesian updates. We emphasize that the receiver's accumulated accuracy $c$ is no longer a function of $t$ (differently from the Gaussian case), and it becomes a distribution conditioned on received accuracies $\senderacc$ from the sender. Therefore, we represent the Bayesian flow distribution of von Mises distribution as $p_F(\btheta|\x;\alpha_1,\alpha_2,\dots,\alpha_i)$. And the original simulation-free training with Bayesian flow distribution is no longer applicable in this scenario.
% Different from previous BFNs where the accumulated accuracy $\rho$ is not explicitly modeled, the accumulated accuracy parameter $c$ (visualized in \cref{fig:vmbf_vis}) needs to be explicitly modeled by feeding it to the network to avoid information loss.
% the randomaccuracy parameter $c$ (visualized in \cref{fig:vmbf_vis}) implies that there exists information in $c$ from the sender just like $m$, meaning that $c$ also should be fed into the network to avoid information loss. 
% We ablate this consideration in  \cref{sec:exp_ablation}. 

\textbf{Fast Sampling from Equivalent Bayesian Flow Distribution} Based on the above reformulations, the Bayesian flow distribution of von Mises distribution is reframed as: 
\begin{equation}\label{eq:flow_frac}
p_F(\btheta_i|\x;\alpha_1,\alpha_2,\dots,\alpha_i)=\E_{\update(\parsnt{1} \mid \parsnt{0}, \x ; \alphat{1})}\dots\E_{\update(\parsn_{i-1} \mid \parsnt{i-2}, \x; \alphat{i-1})} \update(\parsnt{i} | \parsnt{i-1},\x;\alphat{i} )
\end{equation}
Naively sampling from \cref{eq:flow_frac} requires slow auto-regressive iterated simulation, making training unaffordable. Noticing the mathematical properties of \cref{eq:h_m,eq:h_c}, we  transform \cref{eq:flow_frac} to the equivalent form:
\begin{equation}\label{eq:cirflow_equiv}
p_F(\vec{m}_i|\x;\alpha_1,\alpha_2,\dots,\alpha_i)=\E_{vM(\y_1|\x,\alpha_1)\dots vM(\y_i|\x,\alpha_i)} \delta(\vec{m}_i-\text{atan2}(\sum_{j=1}^i \alpha_j \cos \y_j,\sum_{j=1}^i \alpha_j \sin \y_j))
\end{equation}
\begin{equation}\label{eq:cirflow_equiv2}
p_F(\vec{c}_i|\x;\alpha_1,\alpha_2,\dots,\alpha_i)=\E_{vM(\y_1|\x,\alpha_1)\dots vM(\y_i|\x,\alpha_i)}  \delta(\vec{c}_i-||[\sum_{j=1}^i \alpha_j \cos \y_j,\sum_{j=1}^i \alpha_j \sin \y_j]^T||_2)
\end{equation}
which bypasses the computation of intermediate variables and allows pure tensor operations, with negligible computational overhead.
\begin{restatable}{proposition}{cirflowequiv}
The probability density function of Bayesian flow distribution defined by \cref{eq:cirflow_equiv,eq:cirflow_equiv2} is equivalent to the original definition in \cref{eq:flow_frac}. 
\end{restatable}
\textbf{Numerical Determination of Linear Entropy Sender Accuracy Schedule} ~Original BFN designs the accuracy schedule $\beta(t)$ to make the entropy of input distribution linearly decrease. As for crystal generation task, to ensure information coherence between modalities, we choose a sender accuracy schedule $\senderacc$ that makes the receiver's belief entropy $H(t_i)=H(p_I(\cdot|\vtheta_i))=H(p_I(\cdot|\vc_i))$ linearly decrease \emph{w.r.t.} time $t_i$, given the initial and final accuracy parameter $c(0)$ and $c(1)$. Due to the intractability of \cref{eq:vm_entropy}, we first use numerical binary search in $[0,c(1)]$ to determine the receiver's $c(t_i)$ for $i=1,\dots, n$ by solving the equation $H(c(t_i))=(1-t_i)H(c(0))+tH(c(1))$. Next, with $c(t_i)$, we conduct numerical binary search for each $\alpha_i$ in $[0,c(1)]$ by solving the equations $\E_{y\sim vM(x,\alpha_i)}[\sqrt{\alpha_i^2+c_{i-1}^2+2\alpha_i c_{i-1}\cos(y-m_{i-1})}]=c(t_i)$ from $i=1$ to $i=n$ for arbitrarily selected $x\in[-\pi,\pi)$.

After tackling all those issues, we have now arrived at a new BFN architecture for effectively modeling crystals. Such BFN can also be adapted to other type of data located in hyper-torus $\mathbb{T}^{D}$.

\subsection{Equivariant Bayesian Flow for Crystal}
With the above Bayesian flow designed for generative modeling of fractional coordinate $\vF$, we are able to build equivariant Bayesian flow for each modality of crystal. In this section, we first give an overview of the general training and sampling algorithm of \modelname (visualized in \cref{fig:framework}). Then, we describe the details of the Bayesian flow of every modality. The training and sampling algorithm can be found in \cref{alg:train} and \cref{alg:sampling}.

\textbf{Overview} Operating in the parameter space $\bthetaM=\{\bthetaA,\bthetaL,\bthetaF\}$, \modelname generates high-fidelity crystals through a joint BFN sampling process on the parameter of  atom type $\bthetaA$, lattice parameter $\vec{\theta}^L=\{\bmuL,\brhoL\}$, and the parameter of fractional coordinate matrix $\bthetaF=\{\bmF,\bcF\}$. We index the $n$-steps of the generation process in a discrete manner $i$, and denote the corresponding continuous notation $t_i=i/n$ from prior parameter $\thetaM_0$ to a considerably low variance parameter $\thetaM_n$ (\emph{i.e.} large $\vrho^L,\bmF$, and centered $\bthetaA$).

At training time, \modelname samples time $i\sim U\{1,n\}$ and $\bthetaM_{i-1}$ from the Bayesian flow distribution of each modality, serving as the input to the network. The network $\net$ outputs $\net(\parsnt{i-1}^\mathcal{M},t_{i-1})=\net(\parsnt{i-1}^A,\parsnt{i-1}^F,\parsnt{i-1}^L,t_{i-1})$ and conducts gradient descents on loss function \cref{eq:loss_n} for each modality. After proper training, the sender distribution $p_S$ can be approximated by the receiver distribution $p_R$. 

At inference time, from predefined $\thetaM_0$, we conduct transitions from $\thetaM_{i-1}$ to $\thetaM_{i}$ by: $(1)$ sampling $\y_i\sim p_R(\bold{y}|\thetaM_{i-1};t_i,\alpha_i)$ according to network prediction $\predM{i-1}$; and $(2)$ performing Bayesian update $h(\thetaM_{i-1},\y^\calM_{i-1},\alpha_i)$ for each dimension. 

% Alternatively, we complete this transition using the flow-back technique by sampling 
% $\thetaM_{i}$ from Bayesian flow distribution $\flow(\btheta^M_{i}|\predM{i-1};t_{i-1})$. 

% The training objective of $\net$ is to minimize the KL divergence between sender distribution and receiver distribution for every modality as defined in \cref{eq:loss_n} which is equivalent to optimizing the negative variational lower bound $\calL^{VLB}$ as discussed in \cref{sec:preliminaries}. 

%In the following part, we will present the Bayesian flow of each modality in detail.

\textbf{Bayesian Flow of Fractional Coordinate $\vF$}~The distribution of the prior parameter $\bthetaF_0$ is defined as:
\begin{equation}\label{eq:prior_frac}
    p(\bthetaF_0) \defeq \{vM(\vm_0^F|\vec{0}_{3\times N},\vec{0}_{3\times N}),\delta(\vc_0^F-\vec{0}_{3\times N})\} = \{U(\vec{0},\vec{1}),\delta(\vc_0^F-\vec{0}_{3\times N})\}
\end{equation}
Note that this prior distribution of $\vm_0^F$ is uniform over $[\vec{0},\vec{1})$, ensuring the periodic translation invariance property in \cref{De:pi}. The training objective is minimizing the KL divergence between sender and receiver distribution (deduction can be found in \cref{appd:cir_loss}): 
%\oyyw{replace $\vF$ with $\x$?} \hanlin{notations follow Preliminary?}
\begin{align}\label{loss_frac}
\calL_F = n \E_{i \sim \ui{n}, \flow(\parsn{}^F \mid \vF ; \senderacc)} \alpha_i\frac{I_1(\alpha_i)}{I_0(\alpha_i)}(1-\cos(\vF-\predF{i-1}))
\end{align}
where $I_0(x)$ and $I_1(x)$ are the zeroth and the first order of modified Bessel functions. The transition from $\bthetaF_{i-1}$ to $\bthetaF_{i}$ is the Bayesian update distribution based on network prediction:
\begin{equation}\label{eq:transi_frac}
    p(\btheta^F_{i}|\parsnt{i-1}^\calM)=\mathbb{E}_{vM(\bold{y}|\predF{i-1},\alpha_i)}\delta(\btheta^F_{i}-h(\btheta^F_{i-1},\bold{y},\alpha_i))
\end{equation}
\begin{restatable}{proposition}{fracinv}
With $\net_{F}$ as a periodic translation equivariant function namely $\net_F(\parsnt{}^A,w(\parsnt{}^F+\vt),\parsnt{}^L,t)=w(\net_F(\parsnt{}^A,\parsnt{}^F,\parsnt{}^L,t)+\vt), \forall\vt\in\R^3$, the marginal distribution of $p(\vF_n)$ defined by \cref{eq:prior_frac,eq:transi_frac} is periodic translation invariant. 
\end{restatable}
\textbf{Bayesian Flow of Lattice Parameter \texorpdfstring{$\boldsymbol{L}$}{}}   
Noting the lattice parameter $\bm{L}$ located in Euclidean space, we set prior as the parameter of a isotropic multivariate normal distribution $\btheta^L_0\defeq\{\vmu_0^L,\vrho_0^L\}=\{\bm{0}_{3\times3},\bm{1}_{3\times3}\}$
% \begin{equation}\label{eq:lattice_prior}
% \btheta^L_0\defeq\{\vmu_0^L,\vrho_0^L\}=\{\bm{0}_{3\times3},\bm{1}_{3\times3}\}
% \end{equation}
such that the prior distribution of the Markov process on $\vmu^L$ is the Dirac distribution $\delta(\vec{\mu_0}-\vec{0})$ and $\delta(\vec{\rho_0}-\vec{1})$, 
% \begin{equation}
%     p_I^L(\boldsymbol{L}|\btheta_0^L)=\mathcal{N}(\bm{L}|\bm{0},\bm{I})
% \end{equation}
which ensures O(3)-invariance of prior distribution of $\vL$. By Eq. 77 from \citet{bfn}, the Bayesian flow distribution of the lattice parameter $\bm{L}$ is: 
\begin{align}% =p_U(\bmuL|\btheta_0^L,\bm{L},\beta(t))
p_F^L(\bmuL|\bm{L};t) &=\mathcal{N}(\bmuL|\gamma(t)\bm{L},\gamma(t)(1-\gamma(t))\bm{I}) 
\end{align}
where $\gamma(t) = 1 - \sigma_1^{2t}$ and $\sigma_1$ is the predefined hyper-parameter controlling the variance of input distribution at $t=1$ under linear entropy accuracy schedule. The variance parameter $\vrho$ does not need to be modeled and fed to the network, since it is deterministic given the accuracy schedule. After sampling $\bmuL_i$ from $p_F^L$, the training objective is defined as minimizing KL divergence between sender and receiver distribution (based on Eq. 96 in \citet{bfn}):
\begin{align}
\mathcal{L}_{L} = \frac{n}{2}\left(1-\sigma_1^{2/n}\right)\E_{i \sim \ui{n}}\E_{\flow(\bmuL_{i-1} |\vL ; t_{i-1})}  \frac{\left\|\vL -\predL{i-1}\right\|^2}{\sigma_1^{2i/n}},\label{eq:lattice_loss}
\end{align}
where the prediction term $\predL{i-1}$ is the lattice parameter part of network output. After training, the generation process is defined as the Bayesian update distribution given network prediction:
\begin{equation}\label{eq:lattice_sampling}
    p(\bmuL_{i}|\parsnt{i-1}^\calM)=\update^L(\bmuL_{i}|\predL{i-1},\bmuL_{i-1};t_{i-1})
\end{equation}
    

% The final prediction of the lattice parameter is given by $\bmuL_n = \predL{n-1}$.
% \begin{equation}\label{eq:final_lattice}
%     \bmuL_n = \predL{n-1}
% \end{equation}

\begin{restatable}{proposition}{latticeinv}\label{prop:latticeinv}
With $\net_{L}$ as  O(3)-equivariant function namely $\net_L(\parsnt{}^A,\parsnt{}^F,\vQ\parsnt{}^L,t)=\vQ\net_L(\parsnt{}^A,\parsnt{}^F,\parsnt{}^L,t),\forall\vQ^T\vQ=\vI$, the marginal distribution of $p(\bmuL_n)$ defined by \cref{eq:lattice_sampling} is O(3)-invariant. 
\end{restatable}


\textbf{Bayesian Flow of Atom Types \texorpdfstring{$\boldsymbol{A}$}{}} 
Given that atom types are discrete random variables located in a simplex $\calS^K$, the prior parameter of $\boldsymbol{A}$ is the discrete uniform distribution over the vocabulary $\parsnt{0}^A \defeq \frac{1}{K}\vec{1}_{1\times N}$. 
% \begin{align}\label{eq:disc_input_prior}
% \parsnt{0}^A \defeq \frac{1}{K}\vec{1}_{1\times N}
% \end{align}
% \begin{align}
%     (\oh{j}{K})_k \defeq \delta_{j k}, \text{where }\oh{j}{K}\in \R^{K},\oh{\vA}{KD} \defeq \left(\oh{a_1}{K},\dots,\oh{a_N}{K}\right) \in \R^{K\times N}
% \end{align}
With the notation of the projection from the class index $j$ to the length $K$ one-hot vector $ (\oh{j}{K})_k \defeq \delta_{j k}, \text{where }\oh{j}{K}\in \R^{K},\oh{\vA}{KD} \defeq \left(\oh{a_1}{K},\dots,\oh{a_N}{K}\right) \in \R^{K\times N}$, the Bayesian flow distribution of atom types $\vA$ is derived in \citet{bfn}:
\begin{align}
\flow^{A}(\parsn^A \mid \vA; t) &= \E_{\N{\y \mid \beta^A(t)\left(K \oh{\vA}{K\times N} - \vec{1}_{K\times N}\right)}{\beta^A(t) K \vec{I}_{K\times N \times N}}} \delta\left(\parsn^A - \frac{e^{\y}\parsnt{0}^A}{\sum_{k=1}^K e^{\y_k}(\parsnt{0})_{k}^A}\right).
\end{align}
where $\beta^A(t)$ is the predefined accuracy schedule for atom types. Sampling $\btheta_i^A$ from $p_F^A$ as the training signal, the training objective is the $n$-step discrete-time loss for discrete variable \citep{bfn}: 
% \oyyw{can we simplify the next equation? Such as remove $K \times N, K \times N \times N$}
% \begin{align}
% &\calL_A = n\E_{i \sim U\{1,n\},\flow^A(\parsn^A \mid \vA ; t_{i-1}),\N{\y \mid \alphat{i}\left(K \oh{\vA}{KD} - \vec{1}_{K\times N}\right)}{\alphat{i} K \vec{I}_{K\times N \times N}}} \ln \N{\y \mid \alphat{i}\left(K \oh{\vA}{K\times N} - \vec{1}_{K\times N}\right)}{\alphat{i} K \vec{I}_{K\times N \times N}}\nonumber\\
% &\qquad\qquad\qquad-\sum_{d=1}^N \ln \left(\sum_{k=1}^K \out^{(d)}(k \mid \parsn^A; t_{i-1}) \N{\ydd{d} \mid \alphat{i}\left(K\oh{k}{K}- \vec{1}_{K\times N}\right)}{\alphat{i} K \vec{I}_{K\times N \times N}}\right)\label{discdisc_t_loss_exp}
% \end{align}
\begin{align}
&\calL_A = n\E_{i \sim U\{1,n\},\flow^A(\parsn^A \mid \vA ; t_{i-1}),\N{\y \mid \alphat{i}\left(K \oh{\vA}{KD} - \vec{1}\right)}{\alphat{i} K \vec{I}}} \ln \N{\y \mid \alphat{i}\left(K \oh{\vA}{K\times N} - \vec{1}\right)}{\alphat{i} K \vec{I}}\nonumber\\
&\qquad\qquad\qquad-\sum_{d=1}^N \ln \left(\sum_{k=1}^K \out^{(d)}(k \mid \parsn^A; t_{i-1}) \N{\ydd{d} \mid \alphat{i}\left(K\oh{k}{K}- \vec{1}\right)}{\alphat{i} K \vec{I}}\right)\label{discdisc_t_loss_exp}
\end{align}
where $\vec{I}\in \R^{K\times N \times N}$ and $\vec{1}\in\R^{K\times D}$. When sampling, the transition from $\bthetaA_{i-1}$ to $\bthetaA_{i}$ is derived as:
\begin{equation}
    p(\btheta^A_{i}|\parsnt{i-1}^\calM)=\update^A(\btheta^A_{i}|\btheta^A_{i-1},\predA{i-1};t_{i-1})
\end{equation}

The detailed training and sampling algorithm could be found in \cref{alg:train} and \cref{alg:sampling}.









\bibliography{paper}
\bibliographystyle{assets/plainnat}


\appendix
\onecolumn





\section{Additional Experimental Details}
\label{sec:apx-exp-details}
In this section we present additional details for the experiments.



\paragraph{Additional details for the methods }
The best way to select the ``right'' subset of attention heads for the static criterion is still widely understudied. In particular, it poses  the fundamental challenge of which dataset should be chosen to select the heads in advance. Since we are primarily interested in how much query-adaptivity helps to improve, we compare against a \textbf{static oracle} criterion, that uses the prompts for evaluation to decide which heads are sued as static heads. Moreover, we also implement \textbf{static RULER}, using the prompts from the RULER task. We present additional ablations for the choice of the static criterion in Figure~\ref{fig:staticablations}.
Similar to \citet{wu2024retrieval,tang2024razorattention}, we measure head patterns in a synthetic retrieval task, and select heads via the following  simple \textbf{static criterion}: 
\begin{itemize}
    \item \textit{Step 1}: Generate responses for selected prompts using full attention (for LongBench, GSM8k and MBPP tasks) or the approximate attention from the oracle criterion with $\thrsoracle =0.6$ (RULER tasks). Compute the percentage of times each head is labeled as local window by the oracle criterion from Equation~\eqref{eq:oracle} with threshold $\thrsstatic$.
\item
\textit{Step 2}: Calculate the $(1-\alpha)$-quantile of these percentages across all heads $h$. Label heads below the threshold as \textit{long-context} ($c^h_{\text{static}} = 0$) and those above as \textit{local} ($c^h_{\text{static}} = 1$). These labels are query-independent.
\end{itemize}

We further refer the reader to Appendix~\ref{sec:keys} for how we compute the moments used by  \textbf{QAdA}, for which we devote an entire section. 

\paragraph{Choices for thresholds} We ablate over the various thresholds $\thrsoracle, \thrsapprox \in $ (0.1, 0.2, 0.3, 0.4, 0.5, 0.6, 0.7, 0.8, 0.9, 0.95, 0.99, 0.995), as well as 
$\alpha \in $ (0.05, 0.1, 0.15, 0.2, 0.25, 0.3, 0.35, 0.4, 0.45, 0.5, 0.55, 0.6) with $\thrsstatic=0.6$. We ran additional ablations in Figure~\ref{fig:static} for $\thrsstatic$ confirming that the choice $\thrsstatic=0.6$ yields robust performance across all tasks. 


\paragraph{RULER tasks} The RULER benchmark \citep{hsieh2024ruler} consists of a collection of synthetic tasks with varying prompt sizes. These tasks are designed to challenge the model's capabilities in processing long-context information.
We choose the two Q/A tasks, ``qa-1'' and ``qa-2'', the two aggregation tasks: common words extraction ``cwe'' and frequent words extraction ``fwe'', the variable tracing task ``vt'', and the multiquery needle-in-a-haystack task ``niah''. Especially, the two aggregation tasks ``fwe'' and ``cwe'' are known to be difficult baselines for achieving accuracy using efficient sparse attention mechanisms (see the discussion in \citet{chen2024magicpig}).

\paragraph{LongBench tasks} The LongBench benchmark contains a selection of challenging real-world and synthetic tasks, including single-doc QA, multi-doc QA, summarization, and few-shot learning.
We use a selection of tasks from the LongBench dataset for which the standard model achieves at least decent scores. We evaluate on the tasks: (Single-Document QA): ``qasper'', ``multifieldqa-en'', ``multifieldqa-zh'', ``narrativeqa''; (Multi-Document QA): ``2wikimqa'', ``musique'', ``hotpotqa''; (Summarization): ``qmsum'', ``vcsum''; and (Few-shot Learning): ``triviaqa''.





\begin{figure*}[t]
    \centering    
        \centering
            \includegraphics[width=\linewidth]{plots/static/mean_legend.pdf}

                \begin{subfigure}[b]{0.47\linewidth}
\includegraphics[width=\linewidth]{plots/spearman.pdf}

        \caption{ Spearman rank correlation of heads}
        \end{subfigure}
                \begin{subfigure}[b]{0.52\linewidth}
\includegraphics[width=\linewidth]{plots/static/mean.pdf}
    
        \caption{ Ablation over datasets for static criterion}
        \label{fig:static}
        \end{subfigure}
    \caption{\small \textbf{a)}  The Spearman rank correlation of the attention heads ordered by the fraction of times labeled as Local Heads by the oracle criterion with $\tau=0.6$. We see a high correlation among all tasks. b) Ablations for the static criterion using different datasets (LongBench, RULER and specific RULER task, called oracle) and threshold $\thrsstatic$ to label the heads. We use Llama3-8B on RULER 8k.} 
    \label{fig:staticablations}
\end{figure*}



\paragraph{Long-context GSM8k and MBPP datasets}

In addition to the two standard benchmarks, RULER and LongBench, we also construct our own long-context tasks based on the reasoning task GSM8k \citep{cobbe2021training} and the code-generation task MBPP \citep{austin2021program}. We use the standard evaluation protocol, but instead of using only the ``correct'' few-shot examples, we select 55 few-shot examples in the same format generated from the SQUAD \citep{rajpurkar2016squad} dataset, as well as 5 actual few-shot examples (highlighted in green). We provide fragments of the example prompts below. The resulting context lengths are $\approx 10k$ for GSM8k and $\approx 11k$ for MBPP.

For these two tasks, we always use the pre-trained Llama3-8B parameter model \citep{dubey2024llama}, instead of the instruction fine-tuned variant. The reason for choosing the pre-trained model is that the instruction fine-tuned model can solve these tasks without the need for few-shot examples, while the pre-trained model crucially depends on few-shot examples. Since these examples are hidden in a long context, the task becomes challenging, and the model requires retrieving information from tokens far away in order to achieve high accuracy on the task.






\section{Computing the moment statistics}
\label{sec:keys}
We discuss in this section more formally how we obtain the moment statistics as sketched in Section~\ref{sec:moments}.



\begin{figure*}
    \centering    

\begin{subfigure}[b]{0.24\linewidth}
        \centering
        \includegraphics[width=\linewidth]{plots/truepositive.pdf}
        \caption{oracle vs adaptive}
        \label{fig:accuracy}
    \end{subfigure}
        \begin{subfigure}[b]{0.24\linewidth}
        \centering
        \includegraphics[width=\linewidth]{plots/recall/summary.pdf}
        \caption{recall of aggregation}
        \label{fig:recalla}
    \end{subfigure}
        \begin{subfigure}[b]{0.24\linewidth}
        \centering
        \includegraphics[width=\linewidth]{plots/recall/qa.pdf}
        \caption{recall of Q/A}
        \label{fig:recallb}
    \end{subfigure}
        \begin{subfigure}[b]{0.24\linewidth}
        \centering
        \includegraphics[width=\linewidth]{plots/recall/retrieval.pdf}
        \caption{recall of retrieval}
        \label{fig:recallc}
    \end{subfigure}

    
    \caption{\small \textbf{a)}  Accuracy and fraction of true/false  positives/negatives for the 10\% quantiles of the heads (labeled as local heads) for the adaptive criterion with  $\thrsoracle=\thrsapprox=0.6$ on the  RULER benchmark with sequence length 8k. 
    \textbf{b,c,d)} The recall values of long-context heads selected by the oracle criterion for various thresholds $\thrsoracle$ when using the static and adaptive oracle criteria as a function of the average sparsity (percentage of local heads). We adjust the thresholds $\alpha$ (with $\thrsstatic = \thrsoracle$) and $\thrsapprox$ to achieve matching sparsity levels. Annotations indicate the specific oracle thresholds $\thrsoracle$.  We use Llama3-8B on RULER 8k.} 
    \vspace{-0.2in}
\end{figure*}


\paragraph{Option 1 (current prompt):} In this case, after pre-filling, we compute the moment statistics for each head as described in Section~\ref{sec:moments}. Note that for grouped-query attention \citep{ainslie2023gqa}, as used by Llama, we naturally use the same moments for each query in the group since these heads share the same keys. During generation, we keep the moment statistics fixed and do not update them after predicting each token. This is because we always generate sequences of length less than $256$, so updating the statistics has only a limited influence. However, when generating long sequences consisting of thousands of tokens, we would expect that updating the moments during generation becomes beneficial for performance.




\paragraph{Option 2 (other prompt):} In this case, we perform a single forward pass using one of the three choices as prompts: \textit{random word prompt}, which simply permutes words from a Wikipedia article (including the HTML syntax); \textit{wiki prompt}, where we concatenate Wikipedia articles; and \textit{single words prompt}, where we repeat the word "observation." As we showed in Section~\ref{sec:ablations}, the content of the prompt is not important as long as there is enough "diversity." However, we found that the length of the sequence is crucial. Therefore, we store all keys from the forward pass of this prompt. During generation, when predicting the next tokens for a given prompt, we load the keys from the specific \textit{other prompt} and generate the moments using the first $T-1024$ keys, where $T$ is the sequence length of the current prompt. The reason for choosing minus $1024$ is because, as we saw in Figure~\ref{fig:seqlen_prompt}, the performance is robust to keys generated from shorter prompts than the actual sequence but suffers significantly in performance for longer ones. As an alternative implementation, one could also pre-compute the moments for lengths of fixed intervals and load the corresponding moment after pre-filling before starting the generation.














\section{Recall of Attention Heads}
\label{sec:recall}
In this section, we analyze how well our adaptive criterion from Section~\ref{sec:method} can recall the heads selected by the oracle criterion; in other words, how effectively it serves as a proxy for the oracle. We always use the current prompt (Option 1) to generate the moment statistics. 



\paragraph{Accuracy}

We generate responses using standard dense attention and store the scores used to compare the two criteria using the current prompt to generate the moments. For each task, we group the heads into $10\%$ quantiles based on the percentage of times the oracle criterion has been satisfied. For each quantile (averaged over the six selected RULER tasks), we show the fraction of true positives, true negatives, false positives, and false negatives, where a true positive means that both the oracle and adaptive criteria labeled a head as a local head.

We find that the adaptive criterion always correctly identifies the top $50\%$ of the heads that are consistently local heads. Moreover, we find even higher accuracies for the lower quantiles where heads vary between local and long-context. Interestingly, we see that the false negative rate is much lower than the false positive rate for these heads. As a result, the adaptive criterion selects fewer heads than the oracle criterion. This observation is counter-intuitive to the observations made in Section~\ref{sec:rec}, where we observed that our adaptive criterion tends to select more heads than the oracle criterion for the same threshold. The explanation here is that in this section we compare the criterion on scores obtained when using standard full attention. This is necessary to allow a direct comparison between the two criteria. In contrast, in Section~\ref{sec:rec} we compare the average sparsity when using the approximate attention that approximates all labeled heads by a local window.




\paragraph{Recall of long-context heads.} We further compare our adaptive criterion  with the oracle  static criterion in their ability to identify long-context heads selected by the oracle criterion. 
We show in Figure~\ref{fig:recalla}-\ref{fig:recallc} the recall value of long-context heads selected by the oracle criterion for different oracle thresholds $\thrsoracle$ as a function of the sparsity (fraction of heads labeled as local heads by the oracle criterion). 
To allow for a direct comparison between static and adaptive, we  choose $\thrsapprox$, resp. quantile $\alpha$ (with $\thrsstatic = \thrsoracle$), such that the average sparsity is the same as the one of the oracle criterion. We plot the curves for all (selected)  RULER tasks, and find that our test achieves consistently a higher recall value than the oracle static assignment (except for the ``vt'' task, for which the \textit{current prompt} choice for the moments breaks down, as discussed in Section~\ref{sec:ablations}).







\begin{table}[t]
\centering
\begin{tabular}{@{}lccc@{}}
\toprule
Method & all & top 20\% & top 10\% \\
& $\mu \pm \sigma$ & $\mu \pm \sigma$ & $\mu \pm \sigma$ \\
\midrule\midrule
 & \multicolumn{3}{c}{RULER 8k task ``fwe''} \\ 
\midrule\midrule
Log error & $0.41 \pm 0.58$ & $0.50 \pm 0.98$ & $0.57 \pm 1.27$ \\
Dist. local & $3.44 \pm 1.73$ & $1.78 \pm 1.38$ & $1.54 \pm 1.23$ \\
Gaussian opt. & $0.15 \pm 0.18$ & $0.14 \pm 0.21$ & $0.15 \pm 0.25$ \\
\midrule\midrule
 & \multicolumn{3}{c}{RULER 8k task ``Q/A-2''} \\
\midrule\midrule
Log error & $0.37 \pm 0.52$ & $0.63 \pm 0.75$ & $0.74 \pm 0.83$ \\
Dist. local & $2.80 \pm 1.55$ & $1.17 \pm 0.98$ & $1.29 \pm 1.08$ \\
Gaussian opt.  & $0.18 \pm 0.22$ & $0.25 \pm 0.34$ & $0.29 \pm 0.40$ \\
\bottomrule
\end{tabular}
\caption{The mean and standard deviation for the terms  log difference  $|\log A^{\text{bulk}} - (\log(T^{\text{bulk}}) + \mu_s + \sigma_s^2/2)|$ (Log error) and $|\log A^{\text{bulk}} - \log A^{\text{local}}|$ (Dist. local) for all heads (first column) and the 20\% and 10\% percentiles of heads most often labeled as local heads by the oracle criterion with $\thrsoracle=0.6$. We further show the ``Log error'' when replacing the scores by i.i.d.~Gaussian samples instead with matching mean and variance. This indicates the achievable error assuming that the Gaussian approximation holds true.  We use Llama3-8B on RULER 8k.}\label{tab:comparison}
\vspace{-0.1in}
\end{table}





\section{Discussion: Gaussian Approximation}

 \label{apx:gaussian}
In this section, we further discuss the Gaussian approximation exploited  by our criterion in Section~\ref{sec:method}. We divide the discussion into multiple paragraphs.  

\paragraph{Approximatin error} We wonder what is the approximation error arising from Equation~\eqref{eq:gaussianapprox}. 
We show in Table~\ref{tab:comparison}  the average log difference  $|\log A^{\text{bulk}} - (\log(T^{\text{bulk}}) + \mu_s + \sigma_s^2/2)|$  (first row)  between the un-normalized mass of the bulk and our Gaussian approximation from Equation~\eqref{eq:gaussianapprox}.  Taking the exponent, we find that the Gaussian approximation is typically off by a factor of $\approx 2-5$, and thus clearly imprecise. In comparison, in the third row, we show the same statistics, when replacing the scores by i.i.d~samples from a Gaussian distribution with matching mean and variance. This error captures the ``optimal'' error given that Gaussian actually holds. As we can see, this error is significantly smaller. 

Nevertheless, we are effectively interested in whether the Gaussian assumption suffices to make an accurate prediction on whether the head is a local or long-context head. To that end, we also compare in the second row the average log difference  $|\log A^{\text{bulk}} -  \log A^{\text{local}}|$. Indeed, if this distance is much larger than the average log error arising form the Gaussian approximation, we expect our criterion to nevertheless be accurate. As we observe, this is the case. Taking again the exponent, we  find that the $A^{\text{bulk}} $ and $A^{\text{local}}$ typically differ by  factors around $\approx 15-50$. Interestingly, however, we see that the gap becomes more narrow when only considering the top 20\% (resp. 10\%) of heads most frequently selected by the oracle criterion as long-context heads. Finally, we also show the average standard deviation. 














\section{Additional Experiments}

\label{sec:additional_exps}

\paragraph{Ablations for the choice of the prompts}
We show in Figure~\ref{fig:ablations-vt-extra} the plots for the other RULER tasks for the ablations for the choice of the prompt in Figures~\ref{fig:prompts},\ref{fig:promptsfwe} in Section~\ref{sec:ablations}. 



\paragraph{Performances for individual tasks}
We showed in Figures~\ref{fig:compare-approx} and \ref{fig:longbench} the aggregated performances over the tasks. For completeness, we further show in Figures~\ref{fig:llama8k}-\ref{fig:longappendix} the performances for the individual tasks. We further also show the performance of QAdA (current prompt). Interestingly, we observe that the using the random words prompt (Option 2) for generating the keys overwhelmingly often outperforms the use of the current prompt (Option 1). We leave an explanation for this intriguing finding as a task for future work. 











\begin{figure*}[t]
    \centering    
        \centering
        \includegraphics[width=0.8\linewidth]{plots/ablations/mean_legend.pdf}

        \begin{subfigure}[b]{0.24\linewidth}
        \includegraphics[width=\linewidth]{plots/ablations/qa_1.pdf}
                            
        \caption{ ``qa-1'' task}
        \end{subfigure}
        \begin{subfigure}[b]{0.24\linewidth}
        \includegraphics[width=\linewidth]{plots/ablations/qa_2.pdf}
                            
        \caption{ ``qa-2'' task}
        \end{subfigure}
            \begin{subfigure}[b]{0.24\linewidth}
        \includegraphics[width=\linewidth]{plots/ablations/niah_multiquery.pdf}
                            
        \caption{ ``niah'' task}
        \end{subfigure}
            \begin{subfigure}[b]{0.24\linewidth}
        \includegraphics[width=\linewidth]{plots/ablations/cwe.pdf}
                            
        \caption{ ``cwe'' task}
        \end{subfigure}
        

    \caption{\small  Ablations for varying prompts. Same as Figure~\ref{fig:prompts} and \ref{fig:promptsfwe} for the additional RULER $8$k tasks using Llama 3-8B.} 
    \label{fig:ablations-vt-extra}
\end{figure*}




















\begin{figure}
    \centering
    \includegraphics[width=\linewidth]{plots/individual/llama38192.pdf}
    \caption{ Performances for individual tasks for RULER $8$k using Llama-3 8B as in Figure~\ref{fig:compare-approx}}
    \label{fig:llama8k}
\end{figure}

\begin{figure}
    \centering
    \includegraphics[width=\linewidth]{plots/individual/llama316384.pdf}
    \caption{ Performances for individual tasks for RULER $16$k using Llama-3 8B as in Figure~\ref{fig:compare-approx}}
    \label{fig:llama16k}
\end{figure}


\begin{figure}
    \centering
    \includegraphics[width=\linewidth]{plots/individual/mistralai8192.pdf}
    \caption{ Performances for individual tasks for RULER $8$k using Mistral-7B as in Figure~\ref{fig:compare-approx}}
    \label{fig:mistral8k}
\end{figure}

\begin{figure}
    \centering
    \includegraphics[width=\linewidth]{plots/individual/mistralai16384.pdf}
    \caption{ Performances for individual tasks for RULER $16$k using Mistral-7B as in Figure~\ref{fig:compare-approx}}
    \label{fig:mistral16k}
\end{figure}


\begin{figure}
    \centering
    \includegraphics[width=\linewidth]{plots/individual/Qwen8192.pdf}
    \caption{ Performances for individual tasks for RULER $8$k using Qwen-7B as in Figure~\ref{fig:compare-approx}}
    \label{fig:qwen9k}
\end{figure}

\begin{figure}
    \centering
    \includegraphics[width=\linewidth]{plots/individual/Qwen16384.pdf}
    \caption{ Performances for individual tasks for RULER $16$k using Qwen-7B as in Figure~\ref{fig:compare-approx}}
    \label{fig:qwen16k}
\end{figure}


\begin{figure}
    \centering
    \includegraphics[width=\linewidth]{plots/individual/LONG.pdf}
    \caption{\small Performances for individual tasks for LongBench as in Figure~\ref{fig:longbench}}
    \label{fig:longappendix}
\end{figure}


















\begin{figure*}
\begin{tcolorbox}[
    title=Example Prompt for long-context MBPP,
    width=\textwidth,
    colback=white,
    left=5pt,
    right=5pt,
    top=5pt,
    bottom=5pt
]
[...]\\
Q: Due to extreme variation in elevation, great variation occurs in the climatic conditions of Himachal . The climate varies from hot and subhumid tropical in the southern tracts to, with more elevation, cold, alpine, and glacial in the northern and eastern mountain ranges. The state has areas like Dharamsala that receive very heavy rainfall, as well as those like Lahaul and Spiti that are cold and almost rainless. Broadly, Himachal experiences three seasons: summer, winter, and rainy season. Summer lasts from mid-April till the end of June and most parts become very hot (except in the alpine zone which experiences a mild summer) with the average temperature ranging from 28 to 32 °C (82 to 90 °F). Winter lasts from late November till mid March. Snowfall is common in alpine tracts (generally above 2,200 metres (7,218 ft) i.e. in the higher and trans-Himalayan region).\\
What is the climate like?\\
A: varies from hot and subhumid tropical  The answer is varies from hot and subhumid tropical.\\\\
\textcolor{darkgreen}{
Q: James decides to buy a new bed and bed frame.  The bed frame is \$75 and the bed is 10 times that price.  He gets a deal for 20\% off.  How much does he pay for everything?\\
A: The bed cost 75*10=\$750\\
So everything cost 750+75=\$825\\
He gets 825*.2=\$165 off\\
So that means he pays 825-165=\$660 The answer is 660.}
\\\\
\textcolor{darkgreen}{
Q: Liz sold her car at 80\% of what she originally paid. She uses the proceeds of that sale and needs only \$4,000 to buy herself a new \$30,000 car. How much cheaper is her new car versus what she originally paid for her old one?\\
A: If Liz needs only \$4,000 to buy a new \$30,000 car, that means she has \$30,000-\$4,000=\$26,000 from the proceeds of selling her old car\\
If she sold her car at 80\% of what she originally paid for and sold it for \$26,000 then she originally paid \$26,000/80\% = \$32,500 for her old car\\
If she paid \$32,500 for her old car and the new one is \$30,000 then, the new one is \$32,500-\$30,000 = \$2,500 cheaper The answer is 2500.}\\\\
Q: Unlike in multicellular organisms, increases in cell size (cell growth) and reproduction by cell division are tightly linked in unicellular organisms. Bacteria grow to a fixed size and then reproduce through binary fission, a form of asexual reproduction. Under optimal conditions, bacteria can grow and divide extremely rapidly, and bacterial populations can double as quickly as every 9.8 minutes. In cell division, two identical clone daughter cells are produced. Some bacteria, while still reproducing asexually, form more complex reproductive structures that help disperse the newly formed daughter cells. Examples include fruiting body formation by Myxobacteria and aerial hyphae formation by Streptomyces, or budding. Budding involves a cell forming a protrusion that breaks away and produces a daughter cell.\\
What are produced in cell division?\\
A: two identical clone daughter cells  The answer is two identical clone daughter cells.\\\\
\textcolor{darkgreen}{
Q: Janet’s ducks lay 16 eggs per day. She eats three for breakfast every morning and bakes muffins for her friends every day with four. She sells the remainder at the farmers' market daily for \$2 per fresh duck egg. How much in dollars does she make every day at the farmers' market?\\
A:}
\end{tcolorbox}
\end{figure*}



\begin{figure*}

\begin{tcolorbox}[
    title=Example Prompt for long-context GSM8k,
    width=\textwidth,
    colback=white,
    left=5pt,
    right=5pt,
    top=5pt,
    bottom=5pt
]
[...]
You are an expert Python programmer, and here is your task: Due to extreme variation in elevation, great variation occurs in the climatic conditions of Himachal . The climate varies from hot and subhumid tropical in the southern tracts to, with more elevation, cold, alpine, and glacial in the northern and eastern mountain ranges. The state has areas like Dharamsala that receive very heavy rainfall, as well as those like Lahaul and Spiti that are cold and almost rainless. 
Broadly, Himachal experiences three seasons: summer, winter, and rainy season. Summer lasts from mid-April till the end of June and most parts become very hot (except in the alpine zone which experiences a mild summer) with the average temperature ranging from 28 to 32 °C (82 to 90 °F). Winter lasts from late November till mid March. Snowfall is common in alpine tracts (generally above 2,200 metres (7,218 ft) i.e. in the higher and trans-Himalayan region).\\
What is the climate like? Your code should pass these tests:
\\
empty
\\
{[BEGIN]}\\
varies from hot and subhumid tropical\\
{[DONE]}
\\\\
\textcolor{darkgreen}{
You are an expert Python programmer, and here is your task: Write a function to find the similar elements from the given two tuple lists. Your code should pass these tests:\\
assert similar\_elements((3, 4, 5, 6),(5, 7, 4, 10)) == (4, 5)\\
assert similar\_elements((1, 2, 3, 4),(5, 4, 3, 7)) == (3, 4)\\
assert similar\_elements((11, 12, 14, 13),(17, 15, 14, 13)) == (13, 14)\\\\
{[BEGIN]}\\
def similar\_elements(test\_tup1, test\_tup2):\\
  res = tuple(set(test\_tup1) \& set(test\_tup2))\\
  return (res) \\
{[DONE]}}\\\\
You are an expert Python programmer, and here is your task: Unlike in multicellular organisms, increases in cell size (cell growth) and reproduction by cell division are tightly linked in unicellular organisms. Bacteria grow to a fixed size and then reproduce through binary fission, a form of asexual reproduction. Under optimal conditions, bacteria can grow and divide extremely rapidly, and bacterial populations can double as quickly as every 9.8 minutes. In cell division, two identical clone daughter cells are produced. Some bacteria, while still reproducing asexually, form more complex reproductive structures that help disperse the newly formed daughter cells. Examples include fruiting body formation by Myxobacteria and aerial hyphae formation by Streptomyces, or budding. Budding involves a cell forming a protrusion that breaks away and produces a daughter cell.\\
What are produced in cell division? Your code should pass these tests:\\
empty
\\
{[BEGIN]}\\
two identical clone daughter cells\\
{[DONE]}\\\\
\textcolor{darkgreen}{
You are an expert Python programmer, and here is your task: Write a python function to remove first and last occurrence of a given character from the string. Your code should pass these tests:\\
assert remove\_Occ("hello","l") == "heo"\\
assert remove\_Occ("abcda","a") == "bcd"\\
assert remove\_Occ("PHP","P") == "H"\\\\
{[BEGIN]}\\}

\end{tcolorbox}
\end{figure*}


\end{document}
