
\documentclass[]{fairmeta}


\usepackage{microtype}
\usepackage{graphicx}
\usepackage{subcaption}
\usepackage{booktabs} %

\usepackage{hyperref}


\newcommand{\theHalgorithm}{\arabic{algorithm}}


\usepackage{tcolorbox}

\usepackage{amsmath}
\usepackage{amssymb}
\usepackage{mathtools}
\usepackage{amsthm}

\newcommand{\kd}[1]{\textcolor{blue}{kd: #1}}
\newcommand{\ca}[1]{\textcolor{red}{CA: #1}}
\newcommand{\viv}[1]{\textcolor{red}{VC: #1}}
\definecolor{myred}{HTML}{CB4154}

\usepackage{amsmath}
\usepackage{bbm}
\usepackage{booktabs}
\newcommand{\indicator}[1]{\mathbbm{1}\left\lbrace #1\right\rbrace}
\newcommand{\thrsoracle}{\tau_{\text{oracle}}}
\newcommand{\thrsapprox}{\tau_{\text{approx}}}
\newcommand{\thrsstatic}{\tau_{\text{static}}}

\newcommand{\tbulk}{T^{\text{bulk}}}

\newcommand{\oraclecriterion}{c^h_{\text{oracle}}}


\usepackage{listings}
\usepackage{xcolor}
\definecolor{keyword}{rgb}{0.75, 0.13, 0.13}
\definecolor{comment}{rgb}{0.25, 0.5, 0.35}
\definecolor{string}{rgb}{0.6, 0.1, 0.1}
\usepackage{listings}
\usepackage{xcolor}







\definecolor{codeblue}{rgb}{0.13, 0.13, 0.75}
\definecolor{codegray}{rgb}{0.5, 0.5, 0.5}
\definecolor{codepurple}{rgb}{0.58, 0.0, 0.82}
\definecolor{backcolour}{rgb}{0.98, 0.98, 0.98}

\definecolor{functionblue}{RGB}{67,110,238}    %
\definecolor{variablegreen}{RGB}{102,153,0}    %
\definecolor{torchfunc}{RGB}{255,140,0}        %
\definecolor{outputpurple}{RGB}{147,112,219}   %
\definecolor{keywordcolor}{RGB}{0,119,170}     %


\definecolor{lightgreen}{RGB}{76,175,80}     %
\definecolor{darkgreen}{RGB}{46,125,50}      %
\definecolor{torchfunc}{RGB}{255,140,0}      %
\definecolor{outputpurple}{RGB}{147,112,219} %

\usepackage{listings}
\usepackage{xcolor}

\definecolor{backcolour}{rgb}{0.99,0.99,0.99}
\definecolor{codegray}{rgb}{0.5,0.5,0.5}
\definecolor{codeblue}{rgb}{0.0,0.0,0.5}
\definecolor{codepurple}{rgb}{0.58,0.0,0.82}
\definecolor{orange}{rgb}{1.0,0.5,0.0}

\lstset{
    language=Python,
    backgroundcolor=\color{backcolour},
    basicstyle=\ttfamily\footnotesize,
    breakatwhitespace=true,
    breaklines=true,
    captionpos=b,
    commentstyle=\color{codegray}\itshape,
    keepspaces=true,
    keywordstyle=\color{codeblue}\bfseries,
    numbers=left,
    numbersep=5pt,
    numberstyle=\tiny\color{codegray},
    showspaces=false,
    showstringspaces=false,
    showtabs=false,
    stringstyle=\color{codepurple},
    tabsize=4,
    frame=single,
    rulecolor=\color{black},
    emphstyle=\color{orange},
    emph={einsum, lse, stack, local_attn_, dense_attn_, sqrt, log},
    morekeywords={def, return}
}
\theoremstyle{plain}
\newtheorem{theorem}{Theorem}[section]
\newtheorem{proposition}[theorem]{Proposition}
\newtheorem{lemma}[theorem]{Lemma}
\newtheorem{corollary}[theorem]{Corollary}
\theoremstyle{definition}
\newtheorem{definition}[theorem]{Definition}
\newtheorem{assumption}[theorem]{Assumption}
\theoremstyle{remark}
\newtheorem{remark}[theorem]{Remark}



\usepackage[textsize=tiny]{todonotes}


\title{Unveiling Simplicities of Attention:\\ Adaptive Long-Context Head Identification}

\author[2,*]{Konstantin Donhauser}
\author[1]{Charles Arnal}
\author[1]{Mohammad Pezeshki}
\author[1]{Vivien Cabannes}
\author[1]{David Lopez-Paz}
\author[1]{Kartik Ahuja}

\affiliation[1]{FAIR at Meta}
\affiliation[2]{ETH Zurich}

\contribution[*]{Work done at Meta}

\abstract{
The ability to process long contexts is crucial for many natural language processing tasks, yet it remains a significant challenge. While substantial progress has been made in enhancing the efficiency of attention mechanisms, there is still a gap in understanding how attention heads function in long-context settings. In this paper, we observe that while certain heads consistently attend to local information only, others swing between attending to local and long-context information depending on the query. This raises the question: can we identify which heads require long-context information to predict the next token accurately? We demonstrate that it's possible to predict which heads are crucial for long-context processing using only local keys. The core idea here is to exploit a simple model for the long-context scores via second moment approximations. These findings unveil simple properties of attention in the context of long sequences, and open the door to potentially significant gains in efficiency.
}
\usepackage{amsthm}
\usepackage{amsmath}
\usepackage{amssymb}
\usepackage{algorithm}
\usepackage{algorithmic}
\usepackage{subcaption}

\date{\today}
\correspondence{First Author at \email{konstantin.donhauser@ai.ethz.ch}}


\let\oldparagraph\paragraph
\renewcommand{\paragraph}[1]{\textbf{#1}}
\usepackage{amsmath}



\begin{document}

\maketitle


\section{Introduction}
\label{sec:intro}

The digital era has transformed how individuals engage with public discourse, granting unprecedented access to platforms for exchanging information and debating topical issues. While these online networks facilitate large-scale interactions and democratic participation, they also pose significant challenges to societal cohesion. A prominent concern is the intensification of social polarization---the emergence of ideologically opposed groups that often struggle to find common ground \cite{grover_dilemma_2022, kubin_role_2021, bail_exposure_2018}.

Current research addressing online polarization largely divides into two methodological streams. On one hand, observational analyses leverage large-scale data from social media platforms, applying sophisticated techniques such as sentiment analysis \cite{karjus_evolving_2024, alsinet_measuring_2021, buder_does_2021}, network clustering \cite{treuillier_gaining_2024, bond_political_2022, al_amin_unveiling_2017}, and topic modeling \cite{kim_polarized_2019, chen_modeling_2021}. These studies offer insight into real-world polarization patterns but rely on passively obtained data that cannot be manipulated experimentally, thus limiting causal inferences. On the other hand, theoretical and simulation-based research uses formal opinion dynamics models to explore the underlying mechanisms of polarization \cite{hegselmann_opinion_2002,degroot_reaching_1974,sasahara_social_2021,del_vicario_modeling_2017}. Such models allow for controlled variable manipulation but often simplify social interactions in ways that may underrepresent the complexities of actual communication behaviors.

Bridging these two traditions requires empirical investigations that place human participants into experimentally controlled but contextually realistic environments. Advances in large language models (LLMs) offer new opportunities to develop more nuanced simulations of online behavior and discourse \cite{chuang_simulating_2024,breum_persuasive_2024,ohagi_polarization_2024}. However, the literature still lacks comprehensive user studies that integrate LLM-based artificial agents into a systematic framework for studying polarization. Empirical research with human participants interacting alongside artificial agents has the potential to illuminate the causal pathways of opinion formation and polarization in a manner neither observational data nor simplified simulations can fully capture.

In this article, we present a novel framework that marries mathematical opinion dynamics principles with LLM-based artificial agents in a synthetic social network platform. Building on an underlying agent-based model, we implement a robust offline validation process to verify how well our LLM-driven agents reproduce realistic communication patterns and polarization dynamics. Subsequently, we conduct a user study to assess how real participants engage with the resulting debate space, measuring changes in their opinions, perceptions of the platform, and interaction behaviors before and after exposure to polarized content.

Our study makes three overarching contributions:
\begin{enumerate}
    \item \textbf{Integration of Theory and Advanced Simulation:} We introduce a platform that unites formal opinion dynamics with LLM-based agents, thereby enabling more sophisticated modeling and ecological validity than existing approaches.
    \item \textbf{Comprehensive Offline and Online Validation:} We demonstrate, through extensive offline tests, that our LLM-driven agents can replicate key features of polarized discourse. We then validate these findings in a live user study to illuminate how actual participants respond to such environments.
    \item \textbf{Framework for Future Research:} We provide a reproducible experimental pipeline that other researchers can adopt or extend to study areas like echo-chamber formation, information diffusion, and intervention strategies aimed at mitigating harmful polarization.
\end{enumerate}

Our results suggest that a polarized environment influenced by agent-generated content can intensify perceived emotional stakes and group identity markers among human participants, while recommendation systems further shape patterns of engagement in polarized contexts. Taken together, these findings underscore the methodological value of combining offline simulation techniques with empirical user studies, paving the way for deeper insights into how polarized discourses arise, evolve, and might be counteracted.

The remainder of this paper is organized as follows. Section~\ref{sec:related_work} situates our work within the broader literature on online polarization and LLM-based simulations, underscoring the need for a novel experimental bridge between theoretical and user-focused research. Section~\ref{sec:simulation_model} details our synthetic social network and the integration of LLM-based agents. Section~\ref{sec:offline_evaluation} presents the offline validation of our agents’ polarization behaviors, while Section~\ref{sec:user_study} describes the user study design and key findings. In Section~\ref{sec:discussion}, we reflect on the broader implications of our results and future directions, and we conclude in Section~\ref{sec:conclusion}.

\section{Related Work}
% \subsection{Vision Language Model}
% 시각장애인에서 상황을 설명할 DB가 없으니 만들었다. 그리고 이를 VLM에 튜닝했다.
\subsection{Technical approaches for assisting the visually-impaired}


\subsection{Datasets for visual instruction tuning}

% \begin{figure}
%     \centering
%     \includegraphics[width=0.5\linewidth]{Move_teaser.pdf}
%     \caption{Comparison of different dynamic compute approaches. length of arrow indicates residual transformation per token while width indicates velocity of transformation.}
%     \label{fig:enter-label}
% \end{figure}

\section{Method}
\label{sec:method}
Residual connections play a crucial role in shaping token representations, yet their dynamics remain underexplored in the context of efficient decoding. In this work, we delve deeper into transformer residual dynamics and investigate how modulating residual transformation velocity can improve inference efficiency in token-level processing, optimizing both dense and sparse MoE transformers.


\subsection{Residual Dynamics and Motivation for Multi-rate Residuals} \label{sec:motivation}

To analyze how hidden representations evolve across different layers of a transformer architecture, it's crucial to consider the effect of residual connections. Each transformer decoder layer typically has residual connections across attention and MLP submodules. As the residual stream $h_i$ traverses from interval $E_j$ to $E_{j+1}$, it undergoes a residual transformation given by:  
% \begin{equation}
% \label{eq:slow_residual_transformation}
% H_{E_{j+1}} = H_{E_j} \prod_{i=E_j}^{E_{j+1}} \left( I + \mathcal{A}_i \right) \left( I + \mathcal{M}_i \right) \quad \text{where} \quad \mathcal{A}_i = f(c_i, h_{i}), \mathcal{M}_i = g(h_i)
% \end{equation}

\begin{equation} \label{eq:slow_residual_transformation}
h_{E_{j+1}} = h_{E_j} + \sum_{i=E_j}^{E_{j+1}-1} \left( \mathcal{A}_i(h_i) + \mathcal{M}_i(h_i + \mathcal{A}_i(h_i)) \right) \quad \text{where} \quad \mathcal{A}_i = f(c_i, h_{i}), \mathcal{M}_i = g(h_i). 
\end{equation}

Here, \( \mathcal{A}_i \) denotes the non-linear transformation introduced by the multi-head attention mechanism at layer \( i \), while \( \mathcal{M}_i \) corresponds to the non-linear transformation of the MLP block at the same layer. These transformations depend on the input residual stream \( h_i \) and, in the case of \( \mathcal{A}_i \), the previous contextual representation \( c_i \).\footnote{Normalization layers are typically applied in practice but are omitted here for simplicity of the argument.}


% For easy tokens, the magnitude and direction of this delta transformation become progressively smaller with each successive layer as shown in \cref{fig:delta_transformation}. Consequently, it is feasible to predict these tokens after only a few residual connections, whereas harder tokens necessitate more extensive processing through additional layers.

\begin{figure}[ht]
    \centering
    \begin{subfigure}{0.48\textwidth}
        \centering
        \includegraphics[width=\textwidth]{sections/figures/residual_change.pdf}
        \caption{}
        \label{fig:residual_change}
    \end{subfigure}%
    \hfill
    \begin{subfigure}{0.48\textwidth}
        \centering
        \includegraphics[width=\textwidth]{sections/figures/alignment_wrt_dedicated_model.pdf}
        \caption{}
    \label{fig:alignment_wrt_dedicated_model}
    \end{subfigure}
    \caption{(a) As residual streams propagate through the model, the directional shifts in the residuals become progressively smaller. (b) A dedicated model with $k$ layers achieves a faster rate of change in residual streams and higher alignment than base model leveraging early exit mechanisms at layer $k$.}
    \label{fig}
\end{figure}


To examine whether residual transformations can be accelerated across layers, we conducted experiments using a diverse set of prompts on a pre-trained Phi3 model~\cite{phi3_report}. As illustrated in \cref{fig:residual_change}, we measured the directional shift in residual states as \( 1 - \mathcal{C}(h_{i-1}, h_i) \), where \(\mathcal{C}\) denotes normalized cosine similarity. This shift is notably higher in the initial layers, gradually decreasing in subsequent layers. This behavior allows traditional early exit approaches to effectively accelerate decoding by enabling earlier exits for simpler tokens. However, these approaches typically rely on a distance-based approximation, where the full residual transformation of the model is approximated by the residual transformations of the initial layers. To gain deeper insights into the distance versus velocity aspects of residual transformation, we conducted a comparative study. Specifically, we trained an early exit head at layer $k$ of the Phi3 model, which consists of 32 layers, restricting the distance traveled by each token. To accelerate the residual transformation relative to number of layers, we trained a smaller model consisting of only $k$ layers, while keeping all other hyperparameters consistent. We then compared the next-token prediction accuracy of the early exit head of the base model with that of the smaller model. To ensure an equal number of trainable parameters, we inserted low-rank adapters into the smaller model and trained only these adapters, whereas, in the distance-based approach, we trained solely the early exit head. In addition, to accelerate the residual transformation in smaller model, we distilled the residual streams from the larger model by incorporating a distillation loss ~\cite{sanh2019distilbert} between the residual state at layer \(i\) of the smaller model and the residual state at layer \(4 \times i\) of the larger model. As shown in ~\cref{fig:alignment_wrt_dedicated_model} the smaller model demonstrates a significantly faster rate of change in residual streams, leading to higher next token prediction accuracy after $k$ layers compared to the base model that employs traditional early exit mechanisms after $k$ layers \cite{schuster2022confident, chen2023eellm, varshney-etal-2024-investigating}. This experimental setup, which modifies only the rate of change in residual streams while keeping other factors constant, suggests that dense transformers, trained with a fixed number of layers, may inherently possess a slow residual transformation bias.

This observation raises an intriguing question: if the rate of change in residual streams could be accelerated relative to the number of layers, is it possible to facilitate earlier alignment for a greater proportion of tokens? Earlier alignment would be beneficial to not only facilitate dynamic computation but also for generating speculative tokens efficiently with high acceptance rates in speculative decoding setups ~\cite{leviathan2023fast, chen2023accelerating}. 

%thereby enhancing the efficiency of early exiting? 
 % This bias likely constrains the effectiveness of early exiting, particularly for easier tokens. By addressing this limitation through accelerated residual transformations, we hypothesize that it is possible to substantially improve the efficiency and accuracy of early exit strategies in transformer models.

\subsection{Multi-Rate Residual Transformation} \label{m2r2_method}

To address the slow residual transformation bias described in ~\cref{sec:motivation}, we introduce \textit{accelerated residual streams} that operate at rate $R$ relative to original slow residual stream. We pair slow residual stream, $h$ with an accelerated residual stream, $p$, which has an intrinsic bias towards earlier alignment. Relative to ~\cref{eq:slow_residual_transformation}, accelerated residual transformation from interval $E_j$ to $E_{j+1}$ can be represented as: 

% \begin{equation}
% \label{eq:fast_residual_transformation}
% P_{E_{j+1}} = P_{E_j} \prod_{i=E_j}^{E_{j+1}} \left( I + \hat{\mathcal{A}_i} \right) \left( I + \hat{\mathcal{M}_i} \right) \quad \text{where} \quad \hat{\mathcal{A}_i} = \hat{f}(c_i, P_{i}), \hat{\mathcal{M}_i} = \hat{g}(P_{i})
% \end{equation}


\begin{equation} \label{eq:fast_residual_transformation}
p_{E_{j+1}} = p_{E_j} + \sum_{i=E_j}^{E_{j+1}-1} \left( \hat{\mathcal{A}_i}(p_i) + \hat{\mathcal{M}_i}(p_i + \hat{\mathcal{A}_i}(p_i)) \right) \quad \text{where} \quad \hat{\mathcal{A}_i} = \hat{f}(c_i, p_{i}), \hat{\mathcal{M}_i} = \hat{g}(h_i), 
\end{equation}



where $\hat{\mathcal{A}_i}$ and $\hat{\mathcal{M}_i}$ denote non-linear transformation added by layer $i$ to previous accelerated residual $p_{i}$. Similar to $\mathcal{A}_i$, non-linear transformation $\hat{\mathcal{A}_i}$ attends to same context $c_i$ but uses a different transformation $\hat{f}$ for accelerating $p_{E_j}$ relative to $h_{E_j}$. 

We integrate accelerated residual transformation directly into the base network using parallel accelerator adapters such that rank of accelerator adapters $R_p << d$ where $d$ denotes base model hidden dimension. This setup allows the slow residual stream $h_{E_j}$ to pass through the base model layers while the accelerated residual stream $p_{E_j}$ utilizes these parallel adapters as shown in ~\cref{fig:m2r2_main}. Both slow and accelerated residuals are processed in same forward pass via attention masking and incur negligible additional inference latency in memory bound decoding setups, while in compute bound decoding setups where FLOPs optimization is essential, accelerated residual stream utilizes a fraction of attention heads that of slow residual (see ~\cref{sec:flops_optimization}). Additionally, to maximize the utility of accelerated residual transformations without introducing dedicated KV caches, we propose a shared caching mechanism between the slow and accelerated streams which minimally impact alignment benefits of our approach while offering substantial memory savings (see ~\cref{fig:koala_alignment}). Specifically, the attention operation on the slow residuals \( \text{MHA}(h_t, h_{\leq t}, h_{\leq t}) \) is redefined for accelerated residuals as 
\[
\hat{\mathcal{A}} = MHA(p_t, h_{<t} \oplus p_t, h_{<t} \oplus p_t),
\]
where the accelerated residual at time-step $t$, \( p_t \) attends to the slow residual’s KV cache, facilitating the reuse of contextual information across both residual streams without incurring additional caching costs. Here, \(MHA(q, k, v) \) represents multi-head attention between query \( q \), key \( k \), and value \( v \).

\begin{figure}
    \centering
    \includegraphics[width=0.8\linewidth]{sections//figures/m2r2_main2.pdf}
    \caption{Multi-rate Residuals Framework: Slow residual stream of base model is accompanied by a faster stream that operates at a $2-(J+1)\times$ rate relative to the slow stream, undergoing transformations via accelerator adapters as detailed in \cref{m2r2_method}, where J denotes number of early exit intervals. Colors within the slow and fast residual streams indicate similarity, with matching colors representing the most closely aligned residual states. At the beginning of the forward pass and at each exit point, the accelerated residual state is initialized from the corresponding slow residual state to avoid gradient conflict during training (see ~\cref{sec:grad_conflict}). Early exiting decisions are informed by the Accelerated Residual Latent Attention (ARLA) mechanism, described in \cref{method_arla}, which evaluates residual dynamics across consecutive exit gates.}
    \label{fig:m2r2_main}
\end{figure}

% Furthermore. to maximize the benefits of fast residual transformations without using dedicated KV caches, we propose sharing the fast network’s cache with the slow network. Formally speaking, We modify attention operation on slow residuals $MHA(H_t, H_{<=t}, H_{<=t})$ as $MHA(P_{t}, H_{<t} \oplus P_t, H_{<t}  \oplus P_t)$ such that accelerated residuals attend to previous slow context KV cache, where $MHA(q,k,v)$ denotes multi head attention between query, $q$, key $k$ and value $v$.


\subsection{Enhanced Early Residual Alignment}
Early residual alignment is instrumental in optimizing early exiting, speculative decoding, and Mixture-of-Experts (MoE) inference mechanisms. In this section, we provide a detailed analysis of how accelerated residuals enhance these inference setups.

% By aligning the residual states of intermediate layers with the final output representations, the model can maintain high prediction accuracy even when computations are truncated at earlier layers. This enables more reliable early exiting, reducing the overall computational cost while preserving performance. Additionally, in speculative decoding, early residual alignment allows the model to make confident predictions using faster, partial computations, thereby accelerating inference without sacrificing output quality.


\subsubsection{Early Exiting} \label{method_early_exiting}

A prevalent strategy for enabling early exiting at an intermediate layer $E_{j}$ involves approximating the residual transformation between $E_{j}$ and the final layer $N-1$ using a linear, context independent mapping, $\mathcal{T}$, such that $H_{N-1} \approx \mathcal{T}(H_{E_{j}})$. This approximation has been extensively employed in conventional approaches ~\cite{schuster2022confident, chen2023eellm, varshney-etal-2024-investigating}, providing a computationally efficient means to project the output of deeper layers from intermediate states. Specifically, residual state of layer $N-1$ with this approximation can be expressed as:


% \begin{equation}
% \label{eq: vanila_ea_assumption}
% \Phi(H_{E_{j}}) \sim H_{E_{j}} \prod_{i=E_{j}}^{N}\left( I + \mathcal{A}_i \right) \left( I + \mathcal{M}_i \right) \quad \text{where} \quad \Phi \perp C
% \end{equation}

\begin{equation} \label{eq:early_exiting}
h_{E_j} + \sum_{i=E_j}^{N-1} \left( \mathcal{A}_i(h_i) + \mathcal{M}_i(h_i + \mathcal{A}_i(h_i)) \right) \sim \mathcal{T}(h_{E_{j}})  \quad \text{where} \quad \mathcal{T} \perp c. 
\end{equation}


Here, $\mathcal{A}_i$ and $\mathcal{M}_i$ represent the residual contributions of the multi-head attention and MLP layers, respectively, while $\mathcal{T}$ remains independent of $c$, the preceding context.

This approach is inherently limited by two major factors: first, the assumption of linearity between $h_{E_{j}}$ and $h_{N-1}$ may not hold uniformly for all tokens, particularly when $E_j \ll N$. Second, the linear transformation $\mathcal{T}$ disregards the influence of the context $c$ and fails to account for the latent representations of previous contextual states. In contrast, M2R2 accelerated residual states mitigate both of these challenges by approximating the slow residual transformation of all layers via a faster residual transformation of fewer layers as:
% \begin{equation}
% H_{E_j} \prod_{i=E_j}^{N}\left( I + \mathcal{A}_i \right) \left( I + \mathcal{M}_i \right) \sim P_{E_j} \prod_{i=E_j}^{E_j+1}\left( I + \hat{\mathcal{A}_i} \right) \left( I + \hat{\mathcal{M}_i} \right)
% \end{equation}


\begin{equation} \label{eq:m2r2_approximating_ea}
h_{E_j} + \sum_{i=E_j}^{N-1} \left( \mathcal{A}_i(h_i) + \mathcal{M}_i(h_i + \mathcal{A}_i(h_i)) \right) \sim p_{E_j} + \sum_{i=E_j}^{E_{j+1}-1} \left( \hat{\mathcal{A}_i}(p_i) + \hat{\mathcal{M}_i}(p_i + \hat{\mathcal{A}_i}(p_i)) \right), 
\end{equation}

% \begin{equation} \label{eq:fast_residual_transformation}
% p_{E_{j+1}} = p_{E_j} + \sum_{i=E_j}^{E_{j+1}-1} \left( \hat{\mathcal{A}_i}(p_i) + \hat{\mathcal{M}_i}(p_i + \hat{\mathcal{A}_i}(p_i)) \right) \quad \text{where} \quad \hat{\mathcal{A}_i} = \hat{f}(c_i, p_{i}), \hat{\mathcal{M}_i} = \hat{g}(h_i) 
% \end{equation}






where $p_{E_j}$ is initialized from the slow residual state $h_{E_j}$ at each early exit interval $E_j$ using an identity transformation (see ~\cref{fig:m2r2_main}). As shown in ~\cref{fig:m2r2_residual_sim}, accelerated residuals offer a smoother, more consistent shift in residual direction across layers, in contrast to the abrupt changes typically seen at early exit points in standard early exit methods. Moreover, the normalized cosine similarity between accelerated states at early exit intervals and final residual states is substantially higher compared to traditional early exit techniques, highlighting improved alignment with final layer representations. Traditional adaptive compute methods are constrained by two principal factors: the number of tokens eligible for early exit at intermediate layers and the precision of early exit decision. If residual streams fail to saturate early, the majority of tokens remain ineligible for exit, thereby diminishing potential speedups. Additionally, imprecise delineations between tokens suitable for early exit can lead to underthinking (premature exits that adversely affect accuracy) or overthinking (unnecessary processing that compromises efficiency) ~\cite{zhou2020self, dai2020dynamic}. Enhanced early alignment using ~\cref{eq:m2r2_approximating_ea} helps to address  first issue. To address the second issue we introduce Accelerated Residual Latent Attention, which dynamically assesses the saturation of the residual stream, allowing for a more precise differentiation between tokens that can exit early and those requiring further processing.

% This results in uniform change in residual direction    
% % We keep $\mathcal{A} = \hat{\mathcal{A}}$, while $\hat{\mathcal{M}}$ is accelerated by a factor of $2 - (N_{E}+1)X$ relative to the slower residual transformation $\mathcal{M}$, where $N_E$ represents number of early exiting intervals.
% Figure~\cref{fig:rate_change_comparison} illustrates the comparative rate of change between these transformation streams.



% fig:rate_change_comparison
% - grid plot x axis -> layer id (0, 8) , y axis -> layer id -> dark color cell for max similarity , lighter for lower 
% 
-------------------------------------------------------
Let's consider residual stream $h_i$ traverses through interval $E_j$ to $E_{j+1}$ and undergoes residual transformation given by 
\begin{equation}
h_{E_{j+1}} = h_{E_j} \prod_{i=E_j}^{E_{j+1}} \left( 1 + \delta_i \right)    
\end{equation}

where $\delta_i$ denotes non-linear transformation added by layer $i$. Each non-linear transformation of layer $i$ is a function of previous contextual representation, $c_i$ and input residual stream $h_i-1$ as
$\delta_i = f(c_i, h_{i-1})$ 

One way to exit early at exit $E_j+1$ is to assume that residual transformation from $E_j+1$ to final layer $N-1$ can be approximated by a linear function $\phi$ as $h_{N-1} \sim \Phi(h_{E_j+1})$ and most conventional approaches such as \todo{cite EA papers} use this approach. In other words, 

\begin{equation}
\Phi(h_{E_j+1} \sim h_{E_j+1} \prod_{i=E_j+1}^{N} \left( 1 + \delta_i \right)   
\end{equation}

This approach suffers from two primary issues, linearity assumption from $h_E_j+1$ to $H_N-1$ if often incorrect, particularly when $E_j << N$. More importantly, linear transformation $\Phi$ doesn't consider effect of context $C_i$. M2R2  effectively addresses these issues as accelerated residual stream at interval $E_j+1$ can be represented as 

\begin{equation}
r_{E_{j+1}} = r_{E_j} \prod_{i=E_j}^{E_{j+1}} \left( 1 + \gamma_i \right)    
\end{equation}

where $\gamma_i$ denotes non-linear transformation added by layer $i$ to previous accelerated residual $r_i-1$. Similar to $\delta_i$, non-linear transformation $\gamma_i$ considers context $C_i$ as 
$\gamma_i = g(c_i, r_{i-1})$. So in summary, slow residual transformation is approximated by accelerated residual as: 

\begin{equation}
h_{E_j} \prod_{i=E_j}^{N} \left( 1 + \delta_i \right) \sim h_{E_j} \prod_{i=E_j}^{E_j+1} \left( 1 + \gamma_i \right)
\end{equation}

It's worth noting that accelerated residual $r_i$ and slow residual $h_i$ are processed concurrently at layer $i$ by constructing proper attention mask such as attention of slow residual is represented as 

$MHA(H_it, H_{i<=t}, H_{i<=t}$ while attention of fast residual is computed as 

$MHA(r_it, H_{i<=t}, H_{i<=t}$ where $MHA(q,k,v$ denotes multi head attention between query, $q$, key $k$ and value $v$.


------------------------------------------------------------------

Vertical latent attention on accelerated residual is computed as 
$MHA(S_mt, S(Ej<=i<=m)t, S(Ej<=i<=m)t)$ where $Smt$ denotes query/key/value projection in latent domain at layer $m$ at time $t$. 
------------------------------------------------------------------

Gradient conflict Avoidance: 

Let's consider $w_j$ is a trainable parameter that belongs to a layer between $E_j$ and $E_j+1$. Consider early exit loss at gate $E_j+1$, $L_j+1$, gradient propagation of $w_j$ at another trainable parameter $w_j-n$ can be gives as 

$\sum_{k=E_j-n}^{E_j} \beta_k \frac{\partial L_{E_k}}{\partial w_k}$

where $\beta_j$ denotes backward transformation coefficient for weight $w_j$ to reach gate $E_j$. 
 
On the other hand, gradient propagation in proposed approach can be represented as 

\[
\frac{\partial L_{E_j}}{\partial w_j} = 
\begin{cases} 
\beta_j \frac{\partial L_{E_j}}{\partial w_j} & \text{if } E_j \leq w_j \leq E_{j+1} \\
0 & \text{otherwise}
\end{cases}
\]







% \begin{figure}[ht]
%     \centering
%     \includegraphics[width=0.8\textwidth, height=5cm]{rate_change_comparison.png}
%     \caption{Rate of change comparison between fast and slow residual streams.}
%     \label{fig:rate_change_comparison}
% \end{figure}

%vary k and and plot EA accuracy for larger and smaller models. 

% \begin{figure}[ht]
%     \centering
%     \includegraphics[width=0.5\textwidth,height=5cm]{sections/figures/alignment_comparison_dialogsum.pdf}
%     \caption{Alignment of exited tokens for different early exit layers using traditional early exiting heads, dedicated faster networks, and faster residuals.}
%     \label{fig:small_model_early_exiting}
% \end{figure}


\textbf{Accelerated Residual Latent Attention} \label{method_arla}

In the context of residual streams, we observe that the decision to exit at a given layer can be more effectively informed by analyzing the dynamics of residual stream transformations, instead of solely relying on a classification head applied at the early exit interval $E_j$. To capture the subtle dynamics of residual acceleration, we propose a \textit{Accelerated Residual Latent Attention} (ARLA) mechanism. This approach involves making the exit decision at gate $E_j$ by attending to the residuals spanning from gate $E_{j-1}$ to $E_j$, rather than considering only the residual at gate $E_j$. To minimize the computational overhead associated with exit decision-making, the attention mechanism operates within the latent domain as depicted in ~\cref{fig:arla_arch}. Formally, for each interval $[E_j, E_{j+1}]$, the accelerated residuals are projected into Query ($Q^s_{E_j}, \ldots, Q^s_{E_{j+1}}$), Key ($K^s_{E_j}, \ldots, K^s_{E_{j+1}}$), and Value ($V^s_{E_j}, \ldots, V^s_{E_{j+1}}$) vectors, with latent dimension $d^s$ for $Q^s$, $K^s$, and $V^s$ being significantly smaller than hidden dimension of $p$.\footnote{We use $d^s = 64$ for experiments described in ~\cref{sec:experiments}.} Notably, when the router is allowed to make exit decisions at gate $E_j$ based on residual change dynamics, we observe that the attention is not confined to the residual state at $E_j$ but is distributed across residual states from $E_{j-1}$ to $E_j$, %as illustrated in Figure~\ref{fig:vertical_latent_attention_dynamics}. 
This broader focus on residual dynamics significantly reduces decision ambiguity in early exits, as demonstrated in Figure~\ref{fig:roc_arla}, which contrasts routers based on the last hidden state, and the proposed ARLA router.

%show R -> S transformation. 
%show parameter and flop overhead as compared to adapter on last hidden state.

% \begin{figure}[ht]
%     \centering
%     \includegraphics[width=0.5\textwidth,height=5cm]{sections/figures/roc_arla.pdf}
%     \caption{ROC curves of early exit decision strategies: confidence-based methods (CALM/LITE), routers based on the accelerated hidden state, and latent attention routers.}
%     \label{fig:decision_making_comparison}
% \end{figure}

% \begin{figure}[ht]
%     \centering
%     \includegraphics[width=0.5\textwidth,height=5cm]{vertical_latent_attention.png}
%     \caption{Vertical latent attention mechanism for optimizing early exit decisions by considering residuals from gate \(M\) through \(M-1\).}
%     \label{fig:vertical_latent_attention}
% \end{figure}

\begin{figure}[ht]
    \centering
    \begin{subfigure}{0.52\textwidth}
        \centering
        \includegraphics[width=\textwidth, height = 4cm]{sections/figures/arla_arch.pdf}
        \caption{Accelerated Residual Latent Attention (ARLA): Accelerated residuals between early exit gates are projected into latent domain and attention over residual states within the interval is computed to capture residual dynamics and exit decision is made based on residual saturation.}
        \label{fig:arla_arch}
    \end{subfigure}%
    \hfill
    \begin{subfigure}{0.45\textwidth}
        \centering
        \includegraphics[width=\textwidth, height = 4.5cm]{sections/figures/vla_roc.pdf}
        \caption{ROC classification curves of early exit decision strategies using a linear router used on last residual state ~\cite{schuster2022confident, varshney-etal-2024-investigating, chen2023eellm}  and using ARLA approach that considers residual dynamics. }
        \label{fig:roc_arla}
    \end{subfigure}
    \caption{Effectiveness of ARLA in capturing residual dynamics for early exiting decisions.}


\end{figure}



% \begin{figure}[ht]
%     \centering
%     \includegraphics[width=1\textwidth,height=5cm]{sections/figures/arla.pdf}
%     \caption{fig that plots 32 rows 2 cols heatmap showing attention at each gate}
%     \label{fig:vertical_latent_attention_dynamics}
% \end{figure}

\subsubsection{Self Speculative Decoding} \label{method_self_speculative_decoding}

An alternative means to exploit the early alignment properties of our approach is through the use of accelerated residual states for speculative token sampling to accelerate autoregressive decoding. Speculative decoding aims to speed up memory-bound transformer inference by employing a lightweight draft model to predict candidate tokens, while verifying speculated tokens in parallel and advancing token generation by more than one token per full model invocation \cite{leviathan2023fast, chen2023accelerating, xia2023speculative, miao2023specinfer}. Despite its effectiveness in accelerating large language models (LLMs), speculative decoding introduces substantial complexity in both deployment and training. A separate draft model must be specifically trained and aligned with the target model for each application, which increases the training load and operational complexity ~\cite{chen2023accelerating}. Additionally, this approach is resource-inefficient, as it requires both the draft and target models to be simultaneously maintained in memory during inference \cite{leviathan2023fast, chen2023accelerating}. 

One strategy to address this inefficiency is to leverage the initial layers of the target model itself to generate speculative candidates, as depicted in ~\cite{Tang2024}. While this method reduces the autoregressive overhead associated with speculation, it suffers from suboptimal acceptance rates. This occurs because the linear transformation employed for translating hidden states from layer $k$ to the final layer $N$ is typically a poor approximation, as discussed in ~\cref{sec:motivation} and ~\cref{method_early_exiting}. Our approach resolves this limitation by utilizing accelerated residuals, which demonstrate higher fidelity to their slower counterparts. By utilizing accelerated residuals operating at a rate of $N/k$, where $k$ denotes the number of layers used for candidate speculation, we are able to efficiently generate speculative tokens for decoding.\footnote{We typically set $k = 4$ to balance the trade-off between autoregressive drafting overhead and acceptance rate, as discussed in~\cref{sec:experiments}.}
 This technique not only obviates the need for multiple models during inference but also improves the overall efficiency and effectiveness of speculative decoding.

\begin{figure}
    \centering    \includegraphics[width=1\linewidth]{sections/figures/m2r2_aot_loading.pdf}
    \caption{Ahead-of-Time Expert Loading: M2R2 accelerated residual stream predicts experts required for future layers, reducing reliance on on-demand lazy loading. Speculative pre-loading is efficiently overlapped with computation of multi-head attention (MHA) and MLP transformations. Only incorrectly speculated experts are loaded lazily, resulting in faster inference steps and improved computational efficiency. Here, H indicates LBM Host while D indicates HBM Device.}
    \label{fig:moe_expert_aot_loading}
\end{figure}


\subsubsection{Ahead of Time Expert Loading:} \label{method_aot_expert_loading}

Recent advancements in sparse Mixture-of-Experts (MoE) architectures ~\cite{shazeer2017outrageously, fedus2022switch, artetxe2019massively, lepikhin2020gshard, zoph2022designing} have introduced a paradigm shift in token generation by dynamically activating only a subset of experts per input, achieving superior efficiency in comparison to dense models, particularly under memory-bound constraints of autoregressive decoding \cite{fedus2022switch, zoph2022designing}. This sparse activation approach enables MoE-based language models to generate tokens more swiftly, leveraging the efficiency of selective expert usage and avoiding the overhead of full dense layer invocation. In dense transformer models, pre-loading layers is a common strategy to enhance throughput, as computations of current layer can be overlapped with pre-loading of next layer parameters ~\cite{narayanan2021efficient, shoeybi2020megatron}. However, MoE models face a unique challenge: expert selection occurs dynamically based on previous layer’s output, making it infeasible to preload next layer’s experts in parallel. This limitation results in inherent latency, as expert loading becomes a sequential, on-demand process ~\cite{lepikhin2020gshard, fedus2022switch}.

To address this inefficiency, our method introduces a mechanism with \textit{accelerated residuals}, which not only captures key characteristics of base slower residual states but also exhibit high cosine similarity with their final counterparts (as illustrated in \cref{fig:m2r2_residual_sim}). By employing accelerated residual streams, we can effectively predict the necessary experts for future layers well in advance of their actual invocation. Specifically, using a $2\times$ accelerated residual, the experts needed for layers $2i+2$ and $2i+3$ can be identified while still computing in layer $i$, thus overcoming the bottleneck of sequential, on-demand expert selection and mitigating latency in the decoding pipeline, as shown in \cref{fig:moe_expert_aot_loading}. Note that, we use fixed set of accelerator adapters for transforming accelerated residuals (as discussed in ~\cref{m2r2_method}) while slow residual is transformed via expert routing mechanism. 

Furthermore, our approach integrates a Least Recently Used (LRU) caching strategy, which enhances memory efficiency by replacing the least recently used experts with speculated experts that are anticipated to be needed in upcoming layers. This hybrid approach of preemptive expert loading with LRU caching yields substantial improvements over traditional on-demand loading or standalone caching strategies. By minimizing cache misses and efficiently managing memory, this approach addresses both compute and memory bottlenecks, leading to faster, more resource-efficient token generation in MoE architectures. A comprehensive evaluation of this strategy, in relation to state-of-the-art methods, is provided in \cref{experiments_aot}, and the compute and memory traces on an A100 GPU are detailed in \cref{fig:moe_aot_cuda_trace}.



% Recent advancements in sparse Mixture-of-Experts (MoE) architectures have introduced the concept of utilizing distinct computational paths for different tokens \cite{shazeer2017outrageously}. This approach, wherein only a subset of experts are activated per input, enables MoE-based language models to generate tokens more swiftly compared to their dense counterparts due to memory-bound nature of auto-regressive decoding. In dense models, pre-loading layers in advance is a common strategy to enhance computational efficiency. However, this technique is not applicable to MoE models, where expert selection occurs dynamically based on the outputs of previous layers, preventing parallel pre-fetching of experts.

% Our proposed method addresses this inefficiency. Accelerated residuals, which are highly similar to their slower counterparts (see \cref{fig:similarity}), can reliably predict the necessary experts ahead of time. For instance, by utilizing $2X$ accelerated residual stream, we can predict the experts needed for the layer $2i+1$ and $2i+3$ while carrying out computation in layer $i$. This enables us to commence expert loading significantly earlier, as illustrated in \cref{expert_loading}, effectively mitigating the delays observed with the naive on-demand expert loading. Additionally, our method benefits from incorporating a Least Recently Used (LRU) strategy, where speculated experts replace those that are least recently utilized, resulting in improved performance compared to using either strategy alone. For a comprehensive evaluation, refer to \cref{moe_trace}, which provides a CUDA compute and memory trace of our approach executed on <>.



% A naive solution involves using the residual state of the previous layer along with the gating function of the next layer to predict which experts need to be loaded, and initiating the expert loading process in parallel with the attention computation of the next layer. Yet, as shown in \cref{fig:MOE_attn_vs_loading_time}, the attention computation for medium to long contexts is considerably faster than the expert loading time, making this approach inefficient.




\subsection{Training} \label{method_training}
% This approach is feasible due to the absence of gradient conflicts, as discussed in \cref{sec:grad_conflict}.

To accelerate residual streams, we employ parallel accelerator adapters as described in \cref{m2r2_method}.  For the early exiting use-case outlined in \cref{method_early_exiting}, we define the training objective for these adapters using the following loss function, which combines cross-entropy loss at each exit $E_j$ with distillation loss at each layer $i$. Loss weights coefficients $\alpha_0$ and $\alpha_1$ are employed to balance contribution of corresponding losses.

\begin{align} \label{eq:mr_loss}
L_{\text{m2r2}} = \underbrace{-\alpha_0 \sum_{j=1}^{J} \sum_{t=1}^{T} \log p_{\theta} \left( \hat{y}_t^{E_j} \mid y_{<t}, x \right)}_{\text{cross-entropy loss}} 
+ \underbrace{\alpha_1\sum_{i=1}^{E_{J-1}} \sum_{t=1}^{T} \| \mathbf{p}_{t}^{i} - \mathbf{h}_{t}^{((i - E_{j(i)}) \cdot R_i) + E_{j(i)})} \|^2}_{\text{distillation loss}}.
\end{align}

where $\hat{y}_t^{E_j}$ denotes the predictions from the accelerated residual stream at layer $E_j$ and time step $t$, $y_t$ represents the corresponding ground truth tokens, and $x$ indicates previous context tokens. The distillation loss at each layer $i$ is computed by comparing accelerated residuals at layer $i$ with slow residuals at layer $(i - E_{j(i)}) \cdot R_i + E_{j(i)}$, where $R_i$ denotes the rate of accelerated residuals at layer $i$ while $E_{j(i)}$ represents the most recent gate layer index such that $E_{j(i)} <= i$. \( J \) represents the total number of early exit gates, N denotes number of hidden layers and $E_j$ denotes layer index corresponding to gate index $j$ and \( T \) denotes the sequence length. 

In dynamic compute settings, after training of accelerator adapters, we optimize the query, key, and value parameters governing the ARLA routers (see ~\cref{method_arla}) across all exits in parallel on binary cross entropy loss between predicted decision and ground truth exiting decision. The ground truth labels for the router are determined based on whether the application of the final logit head on $\hat{y}_t^{E_j}$ yields the correct next-token prediction. 


% The objective for this optimization is defined by the following loss function:


%TODO are equations required ? 
% \begin{equation} \label{eq:arla_loss_combined}\small
%     L_{\text{arla}} = -\frac{1}{N} \sum_{t=1}^{T} \left( \sum_{j=1}^{E_n} \left[ O_t^{E_j} \log(\hat{O}_t^{E_j}) + (1 - O_t^{E_j}) \log(1 - \hat{O}_t^{E_j}) \right] \right), \quad \text{where} \quad 
%     O_t^{E_j} = \begin{cases} 
%     1, & \text{if } L(\hat{y}_t^{E_j}) = y_t^{E_j} \\
%     0, & \text{otherwise}
%     \end{cases}
% \end{equation}

% where $\hat{O}_t^{E_j}$ represents the binary predicted logits produced by the vertical latent attention router, as described in \cref{sec:arla}, at gate $E_j$ and time step $t$, and $O_t^{E_j}$ denotes the corresponding ground truth labels. The ground truth labels for the router are determined based on whether the application of the logit head on $\hat{y}_t^{E_j}$ yields the correct next-token prediction. The parameters controlling vertical latent attention are trained concurrently to ensure consistency and efficient use of computational resources.

For self-speculative decoding, as described in \cref{method_self_speculative_decoding}, the training objective remains the same as \cref{eq:mr_loss}, but with the number of intervals set to $J = 1$ and the rate of residual transformation set to $R_n = N/k$, where the first $k$ layers generate speculative candidate tokens. In the context of Ahead-of-Time Expert Loading for Mixture-of-Experts (MoE) models (see \cref{method_aot_expert_loading}), setting the rate of residual transformation to $R_n = 2$ typically offers a good trade-off between the accuracy of expert speculation and AoT pre-loading of experts. 

% Thus, we set $J = 1$ and $E_1 = 16$.


~\subsection{FLOPs Optimization} \label{sec:flops_optimization}

Naively implemented, M2R2 incurs higher FLOP overhead compared to traditional speculative decoding and early exiting approaches such as ~\cite{medusa, schuster2022confident, Tang2024}. However, modern accelerators demonstrate compute bandwidth that exceeds memory access bandwidth by an order of magnitude or more~\cite{databricksLLMInference2023, jouppi2021ten}, meaning increased FLOPs do not necessarily translate to increased decoding latency. Nevertheless, to ensure fair comparison and efficiency in compute bound scenarios, we introduce targeted optimizations.

~\textbf{Attention FLOPs Optimization} For medium-to-long context lengths, attention computation dominates FLOPs in the self-attention layer, surpassing the contribution from MLP layers. Specifically, matrix multiplications involving queries, cached keys, and cached values scale with $l_{kv} * l_{q}$ where $l_{kv}$ denotes previous context length and $l_q$ denotes current query length. Since M2R2 pairs accelerated residuals with slow residuals, a naive implementation results in twice the FLOPs consumption compared to a standard attention layer. To address this, we limit the attention of accelerated residual stream to selectively attend to the top-k most relevant tokens, identified by the slow residual stream based on top attention coefficients\footnote{We set to k = 64 and attend to top 64 tokens as identified by the slow residual stream.}. This is possible since slow and accelerated residual streams are processed in same forward pass and accelerated streams have access to attention coefficients of slow stream. Note that, the faster residual stream still retains the flexibility to assign distinct attention coefficients to these tokens. Furthermore, we design the faster residual stream to employ only 8 attention heads, compared to the 32 heads used in the slow residual stream of the Phi-3 model, reducing query, key, value, and output projection FLOPs by a factor of 1/4. ~\cref{fig:m2r2_num_heads_ablation} indicates effect of using a slicker stream on alignment. As depicted, using $\hat{n}_h = 8$ offers a good trade-off between alignment and FLOPs overhead. 

~\textbf{MLP FLOPs Optimization} The accelerator adapters operating on the accelerated residual stream are intentionally designed with lower rank than their counterparts in the base model. This reduces FLOP overhead by a factor proportional to $hiddenSize / rank$. Additionally, since the faster residual stream uses only 8 attention heads (compared to 32 in the slow residual stream of Phi-3), the subsequent MLP layers process a smaller set of activations, further reducing FLOPs by another factor of 1/4.

These optimizations significantly reduce the FLOP overhead per speculative draft generation, as illustrated in ~\cref{fig:flops_optmization}. Notably, while traditional early-exiting speculative approaches such as DEED require propagating the full slow residual state through the initial layers, incurring substantial computational costs, M2R2 achieves efficient token generation via slimmer, low-rank faster residual streams. In contrast, Medusa introduces considerable FLOP overhead due to per-head computations scaling with $d^2+dv$\footnote{Here $d$ denotes hidden state dimension while $v$ denotes vocab size.}, whereas M2R2 employs low-rank layers for both MLP and language modeling heads, maintaining computational efficiency. All experiments involving the M2R2 approach, as detailed in ~\cref{sec:experiments}, are conducted using these FLOPs optimizations.









% \[
% O_t^{E_j} = 
% \begin{cases} 
% 1, & \text{if } L(\hat{y}_t^{E_j}) = y_t^{E_j} \\
% 0, & \text{otherwise}
% \end{cases}
% \]




%add distillation
% We train accelerator adapters described in \cref{m2r2_method} to accelerate residual streams on next token prediction all in parallel since there are no gradient conflict issues as described in \cref{sec:grad_conflict}.

% \begin{align} \label{eq:mr_loss}
% L_{mr} =  & -\sum_{j = 1}^{E_n} (\sum_{t=1}^{T}\log p_{\theta} (\hat{y}_t^{E_j} | \hat{y}_{<t}, x)) \nonumber
% \end{align}

% where $\hat{y_t^{E_j}}$ denotes predicted logits obtained from accelerated residual stream at gate $E_j$ and time-step $t$ while $y_t^{E_j}$ denotes corresponding truth tokens. 

% Upon training of adapters responsible for accelerating residual streams, we train query, key, value parameters responsible for vertical latent attention of all gates in parallel as

% \begin{equation} \label{eq:arla_loss}
%     L_{arla} = -\frac{1}{N} (\sum_{t=1}^{T}(1\sum_{j=1}^{E_n} \left[ O_t^{E_j} \log(\hat{O}_t^{E_j}) + (1 - o_t^{E_j}) \log(1 - \hat{o_t}_{E_j}) \right]))
% \end{equation}

% where $\hat{O_t^{E_j}}$ denotes binary predicted logits obtained from vertical latent attention router described in \cref{sec:arla} at gate $E_j$ and timestep $t$ while $O_t^{E_j}$ denotes corresponding truth label. Truth labels for router are obtained by computing whether logit head application on $\hat{y}_t^j$ results in true next token prediction. Formally speaking, 

% $O_t^{E_j} = 1 if L(\hat{y_t^{E_j}}) == y_t^{E_j} , 0 otherwise$. 

% Parameters responsible for vertical latent attention are also trained in parallel as well. 

%todo: training slow and fast residuals together and distillation can be two training mdoes. 
%Distillation can be an ablation. 




% Although transformer decoding is memory bound on most mainstream accelerators, there could be scenarios where flop savings are crucial. For instance, on on-device settings power consumption is directly correlated with flops per decoding step and reducing flops does help with overall energy consumption. Vanilla early exiting methods help with flop reduction but suffer from mismatch between training and inference due to early exited tokens. If token at decoding step $t$, $T_t$ exited at layer $E_i$, while token $T_{t+k}$ exits at layer $E_j$ such that $E_i < E_j$, hidden state $H_{t+k}l$ does not have corresponding hidden state $H_tl$ to attend to where $E_i < l <= E_j$. One solution that's often used in literature is to rely on last hidden state available, $H_t{E_j}$, however it tends to be sub-optimal and does affect generation quality \cite{ref}.  To alleviate this mismatch while reducing flops, we train router such that attention mask between token $T_{t+k}$ and token $T_{<t+k}$ is given by: 

% \begin{equation}
%     a_{T_{{t+k}{T_{<t+k}}} = 1 if  E_{T_{<t+k}} >= E{T_{t+k}}
%     else 0
% \end{equation}

% This attention mask enables router to account for exited tokens and get trained accordingly. Since attention mechanism during decoding remains exactly same as that during training, impact on generation quality tends to be minimal as noted in \cref{fig:gen_auality_with_and_without_recompute_attention_show_flops}.  Although MoD does not suffer from training and inference mismatch, we observe that it suffers from discountinuity between pre-training and super-vised fine-tuning resulting in sub-optimal perplexity. On the other hand, our method doesn't not require pre-training , doesn't suffer from discountinuity, and achieves much better perplexity in super-vised fine-tuning and instruction tuning setups as shown in \cref{fig:Mod_vs_m2r2_loss_curves}.






% Our techniques are directly applicable in such scenarios.    




%expert loading with cuda streams in experiments





\bibliography{paper}
\bibliographystyle{assets/plainnat}


\appendix
\onecolumn





\section{Additional Experimental Details}
\label{sec:apx-exp-details}
In this section we present additional details for the experiments.



\paragraph{Additional details for the methods }
The best way to select the ``right'' subset of attention heads for the static criterion is still widely understudied. In particular, it poses  the fundamental challenge of which dataset should be chosen to select the heads in advance. Since we are primarily interested in how much query-adaptivity helps to improve, we compare against a \textbf{static oracle} criterion, that uses the prompts for evaluation to decide which heads are sued as static heads. Moreover, we also implement \textbf{static RULER}, using the prompts from the RULER task. We present additional ablations for the choice of the static criterion in Figure~\ref{fig:staticablations}.
Similar to \citet{wu2024retrieval,tang2024razorattention}, we measure head patterns in a synthetic retrieval task, and select heads via the following  simple \textbf{static criterion}: 
\begin{itemize}
    \item \textit{Step 1}: Generate responses for selected prompts using full attention (for LongBench, GSM8k and MBPP tasks) or the approximate attention from the oracle criterion with $\thrsoracle =0.6$ (RULER tasks). Compute the percentage of times each head is labeled as local window by the oracle criterion from Equation~\eqref{eq:oracle} with threshold $\thrsstatic$.
\item
\textit{Step 2}: Calculate the $(1-\alpha)$-quantile of these percentages across all heads $h$. Label heads below the threshold as \textit{long-context} ($c^h_{\text{static}} = 0$) and those above as \textit{local} ($c^h_{\text{static}} = 1$). These labels are query-independent.
\end{itemize}

We further refer the reader to Appendix~\ref{sec:keys} for how we compute the moments used by  \textbf{QAdA}, for which we devote an entire section. 

\paragraph{Choices for thresholds} We ablate over the various thresholds $\thrsoracle, \thrsapprox \in $ (0.1, 0.2, 0.3, 0.4, 0.5, 0.6, 0.7, 0.8, 0.9, 0.95, 0.99, 0.995), as well as 
$\alpha \in $ (0.05, 0.1, 0.15, 0.2, 0.25, 0.3, 0.35, 0.4, 0.45, 0.5, 0.55, 0.6) with $\thrsstatic=0.6$. We ran additional ablations in Figure~\ref{fig:static} for $\thrsstatic$ confirming that the choice $\thrsstatic=0.6$ yields robust performance across all tasks. 


\paragraph{RULER tasks} The RULER benchmark \citep{hsieh2024ruler} consists of a collection of synthetic tasks with varying prompt sizes. These tasks are designed to challenge the model's capabilities in processing long-context information.
We choose the two Q/A tasks, ``qa-1'' and ``qa-2'', the two aggregation tasks: common words extraction ``cwe'' and frequent words extraction ``fwe'', the variable tracing task ``vt'', and the multiquery needle-in-a-haystack task ``niah''. Especially, the two aggregation tasks ``fwe'' and ``cwe'' are known to be difficult baselines for achieving accuracy using efficient sparse attention mechanisms (see the discussion in \citet{chen2024magicpig}).

\paragraph{LongBench tasks} The LongBench benchmark contains a selection of challenging real-world and synthetic tasks, including single-doc QA, multi-doc QA, summarization, and few-shot learning.
We use a selection of tasks from the LongBench dataset for which the standard model achieves at least decent scores. We evaluate on the tasks: (Single-Document QA): ``qasper'', ``multifieldqa-en'', ``multifieldqa-zh'', ``narrativeqa''; (Multi-Document QA): ``2wikimqa'', ``musique'', ``hotpotqa''; (Summarization): ``qmsum'', ``vcsum''; and (Few-shot Learning): ``triviaqa''.





\begin{figure*}[t]
    \centering    
        \centering
            \includegraphics[width=\linewidth]{plots/static/mean_legend.pdf}

                \begin{subfigure}[b]{0.47\linewidth}
\includegraphics[width=\linewidth]{plots/spearman.pdf}

        \caption{ Spearman rank correlation of heads}
        \end{subfigure}
                \begin{subfigure}[b]{0.52\linewidth}
\includegraphics[width=\linewidth]{plots/static/mean.pdf}
    
        \caption{ Ablation over datasets for static criterion}
        \label{fig:static}
        \end{subfigure}
    \caption{\small \textbf{a)}  The Spearman rank correlation of the attention heads ordered by the fraction of times labeled as Local Heads by the oracle criterion with $\tau=0.6$. We see a high correlation among all tasks. b) Ablations for the static criterion using different datasets (LongBench, RULER and specific RULER task, called oracle) and threshold $\thrsstatic$ to label the heads. We use Llama3-8B on RULER 8k.} 
    \label{fig:staticablations}
\end{figure*}



\paragraph{Long-context GSM8k and MBPP datasets}

In addition to the two standard benchmarks, RULER and LongBench, we also construct our own long-context tasks based on the reasoning task GSM8k \citep{cobbe2021training} and the code-generation task MBPP \citep{austin2021program}. We use the standard evaluation protocol, but instead of using only the ``correct'' few-shot examples, we select 55 few-shot examples in the same format generated from the SQUAD \citep{rajpurkar2016squad} dataset, as well as 5 actual few-shot examples (highlighted in green). We provide fragments of the example prompts below. The resulting context lengths are $\approx 10k$ for GSM8k and $\approx 11k$ for MBPP.

For these two tasks, we always use the pre-trained Llama3-8B parameter model \citep{dubey2024llama}, instead of the instruction fine-tuned variant. The reason for choosing the pre-trained model is that the instruction fine-tuned model can solve these tasks without the need for few-shot examples, while the pre-trained model crucially depends on few-shot examples. Since these examples are hidden in a long context, the task becomes challenging, and the model requires retrieving information from tokens far away in order to achieve high accuracy on the task.






\section{Computing the moment statistics}
\label{sec:keys}
We discuss in this section more formally how we obtain the moment statistics as sketched in Section~\ref{sec:moments}.



\begin{figure*}
    \centering    

\begin{subfigure}[b]{0.24\linewidth}
        \centering
        \includegraphics[width=\linewidth]{plots/truepositive.pdf}
        \caption{oracle vs adaptive}
        \label{fig:accuracy}
    \end{subfigure}
        \begin{subfigure}[b]{0.24\linewidth}
        \centering
        \includegraphics[width=\linewidth]{plots/recall/summary.pdf}
        \caption{recall of aggregation}
        \label{fig:recalla}
    \end{subfigure}
        \begin{subfigure}[b]{0.24\linewidth}
        \centering
        \includegraphics[width=\linewidth]{plots/recall/qa.pdf}
        \caption{recall of Q/A}
        \label{fig:recallb}
    \end{subfigure}
        \begin{subfigure}[b]{0.24\linewidth}
        \centering
        \includegraphics[width=\linewidth]{plots/recall/retrieval.pdf}
        \caption{recall of retrieval}
        \label{fig:recallc}
    \end{subfigure}

    
    \caption{\small \textbf{a)}  Accuracy and fraction of true/false  positives/negatives for the 10\% quantiles of the heads (labeled as local heads) for the adaptive criterion with  $\thrsoracle=\thrsapprox=0.6$ on the  RULER benchmark with sequence length 8k. 
    \textbf{b,c,d)} The recall values of long-context heads selected by the oracle criterion for various thresholds $\thrsoracle$ when using the static and adaptive oracle criteria as a function of the average sparsity (percentage of local heads). We adjust the thresholds $\alpha$ (with $\thrsstatic = \thrsoracle$) and $\thrsapprox$ to achieve matching sparsity levels. Annotations indicate the specific oracle thresholds $\thrsoracle$.  We use Llama3-8B on RULER 8k.} 
    \vspace{-0.2in}
\end{figure*}


\paragraph{Option 1 (current prompt):} In this case, after pre-filling, we compute the moment statistics for each head as described in Section~\ref{sec:moments}. Note that for grouped-query attention \citep{ainslie2023gqa}, as used by Llama, we naturally use the same moments for each query in the group since these heads share the same keys. During generation, we keep the moment statistics fixed and do not update them after predicting each token. This is because we always generate sequences of length less than $256$, so updating the statistics has only a limited influence. However, when generating long sequences consisting of thousands of tokens, we would expect that updating the moments during generation becomes beneficial for performance.




\paragraph{Option 2 (other prompt):} In this case, we perform a single forward pass using one of the three choices as prompts: \textit{random word prompt}, which simply permutes words from a Wikipedia article (including the HTML syntax); \textit{wiki prompt}, where we concatenate Wikipedia articles; and \textit{single words prompt}, where we repeat the word "observation." As we showed in Section~\ref{sec:ablations}, the content of the prompt is not important as long as there is enough "diversity." However, we found that the length of the sequence is crucial. Therefore, we store all keys from the forward pass of this prompt. During generation, when predicting the next tokens for a given prompt, we load the keys from the specific \textit{other prompt} and generate the moments using the first $T-1024$ keys, where $T$ is the sequence length of the current prompt. The reason for choosing minus $1024$ is because, as we saw in Figure~\ref{fig:seqlen_prompt}, the performance is robust to keys generated from shorter prompts than the actual sequence but suffers significantly in performance for longer ones. As an alternative implementation, one could also pre-compute the moments for lengths of fixed intervals and load the corresponding moment after pre-filling before starting the generation.














\section{Recall of Attention Heads}
\label{sec:recall}
In this section, we analyze how well our adaptive criterion from Section~\ref{sec:method} can recall the heads selected by the oracle criterion; in other words, how effectively it serves as a proxy for the oracle. We always use the current prompt (Option 1) to generate the moment statistics. 



\paragraph{Accuracy}

We generate responses using standard dense attention and store the scores used to compare the two criteria using the current prompt to generate the moments. For each task, we group the heads into $10\%$ quantiles based on the percentage of times the oracle criterion has been satisfied. For each quantile (averaged over the six selected RULER tasks), we show the fraction of true positives, true negatives, false positives, and false negatives, where a true positive means that both the oracle and adaptive criteria labeled a head as a local head.

We find that the adaptive criterion always correctly identifies the top $50\%$ of the heads that are consistently local heads. Moreover, we find even higher accuracies for the lower quantiles where heads vary between local and long-context. Interestingly, we see that the false negative rate is much lower than the false positive rate for these heads. As a result, the adaptive criterion selects fewer heads than the oracle criterion. This observation is counter-intuitive to the observations made in Section~\ref{sec:rec}, where we observed that our adaptive criterion tends to select more heads than the oracle criterion for the same threshold. The explanation here is that in this section we compare the criterion on scores obtained when using standard full attention. This is necessary to allow a direct comparison between the two criteria. In contrast, in Section~\ref{sec:rec} we compare the average sparsity when using the approximate attention that approximates all labeled heads by a local window.




\paragraph{Recall of long-context heads.} We further compare our adaptive criterion  with the oracle  static criterion in their ability to identify long-context heads selected by the oracle criterion. 
We show in Figure~\ref{fig:recalla}-\ref{fig:recallc} the recall value of long-context heads selected by the oracle criterion for different oracle thresholds $\thrsoracle$ as a function of the sparsity (fraction of heads labeled as local heads by the oracle criterion). 
To allow for a direct comparison between static and adaptive, we  choose $\thrsapprox$, resp. quantile $\alpha$ (with $\thrsstatic = \thrsoracle$), such that the average sparsity is the same as the one of the oracle criterion. We plot the curves for all (selected)  RULER tasks, and find that our test achieves consistently a higher recall value than the oracle static assignment (except for the ``vt'' task, for which the \textit{current prompt} choice for the moments breaks down, as discussed in Section~\ref{sec:ablations}).







\begin{table}[t]
\centering
\begin{tabular}{@{}lccc@{}}
\toprule
Method & all & top 20\% & top 10\% \\
& $\mu \pm \sigma$ & $\mu \pm \sigma$ & $\mu \pm \sigma$ \\
\midrule\midrule
 & \multicolumn{3}{c}{RULER 8k task ``fwe''} \\ 
\midrule\midrule
Log error & $0.41 \pm 0.58$ & $0.50 \pm 0.98$ & $0.57 \pm 1.27$ \\
Dist. local & $3.44 \pm 1.73$ & $1.78 \pm 1.38$ & $1.54 \pm 1.23$ \\
Gaussian opt. & $0.15 \pm 0.18$ & $0.14 \pm 0.21$ & $0.15 \pm 0.25$ \\
\midrule\midrule
 & \multicolumn{3}{c}{RULER 8k task ``Q/A-2''} \\
\midrule\midrule
Log error & $0.37 \pm 0.52$ & $0.63 \pm 0.75$ & $0.74 \pm 0.83$ \\
Dist. local & $2.80 \pm 1.55$ & $1.17 \pm 0.98$ & $1.29 \pm 1.08$ \\
Gaussian opt.  & $0.18 \pm 0.22$ & $0.25 \pm 0.34$ & $0.29 \pm 0.40$ \\
\bottomrule
\end{tabular}
\caption{The mean and standard deviation for the terms  log difference  $|\log A^{\text{bulk}} - (\log(T^{\text{bulk}}) + \mu_s + \sigma_s^2/2)|$ (Log error) and $|\log A^{\text{bulk}} - \log A^{\text{local}}|$ (Dist. local) for all heads (first column) and the 20\% and 10\% percentiles of heads most often labeled as local heads by the oracle criterion with $\thrsoracle=0.6$. We further show the ``Log error'' when replacing the scores by i.i.d.~Gaussian samples instead with matching mean and variance. This indicates the achievable error assuming that the Gaussian approximation holds true.  We use Llama3-8B on RULER 8k.}\label{tab:comparison}
\vspace{-0.1in}
\end{table}





\section{Discussion: Gaussian Approximation}

 \label{apx:gaussian}
In this section, we further discuss the Gaussian approximation exploited  by our criterion in Section~\ref{sec:method}. We divide the discussion into multiple paragraphs.  

\paragraph{Approximatin error} We wonder what is the approximation error arising from Equation~\eqref{eq:gaussianapprox}. 
We show in Table~\ref{tab:comparison}  the average log difference  $|\log A^{\text{bulk}} - (\log(T^{\text{bulk}}) + \mu_s + \sigma_s^2/2)|$  (first row)  between the un-normalized mass of the bulk and our Gaussian approximation from Equation~\eqref{eq:gaussianapprox}.  Taking the exponent, we find that the Gaussian approximation is typically off by a factor of $\approx 2-5$, and thus clearly imprecise. In comparison, in the third row, we show the same statistics, when replacing the scores by i.i.d~samples from a Gaussian distribution with matching mean and variance. This error captures the ``optimal'' error given that Gaussian actually holds. As we can see, this error is significantly smaller. 

Nevertheless, we are effectively interested in whether the Gaussian assumption suffices to make an accurate prediction on whether the head is a local or long-context head. To that end, we also compare in the second row the average log difference  $|\log A^{\text{bulk}} -  \log A^{\text{local}}|$. Indeed, if this distance is much larger than the average log error arising form the Gaussian approximation, we expect our criterion to nevertheless be accurate. As we observe, this is the case. Taking again the exponent, we  find that the $A^{\text{bulk}} $ and $A^{\text{local}}$ typically differ by  factors around $\approx 15-50$. Interestingly, however, we see that the gap becomes more narrow when only considering the top 20\% (resp. 10\%) of heads most frequently selected by the oracle criterion as long-context heads. Finally, we also show the average standard deviation. 














\section{Additional Experiments}

\label{sec:additional_exps}

\paragraph{Ablations for the choice of the prompts}
We show in Figure~\ref{fig:ablations-vt-extra} the plots for the other RULER tasks for the ablations for the choice of the prompt in Figures~\ref{fig:prompts},\ref{fig:promptsfwe} in Section~\ref{sec:ablations}. 



\paragraph{Performances for individual tasks}
We showed in Figures~\ref{fig:compare-approx} and \ref{fig:longbench} the aggregated performances over the tasks. For completeness, we further show in Figures~\ref{fig:llama8k}-\ref{fig:longappendix} the performances for the individual tasks. We further also show the performance of QAdA (current prompt). Interestingly, we observe that the using the random words prompt (Option 2) for generating the keys overwhelmingly often outperforms the use of the current prompt (Option 1). We leave an explanation for this intriguing finding as a task for future work. 











\begin{figure*}[t]
    \centering    
        \centering
        \includegraphics[width=0.8\linewidth]{plots/ablations/mean_legend.pdf}

        \begin{subfigure}[b]{0.24\linewidth}
        \includegraphics[width=\linewidth]{plots/ablations/qa_1.pdf}
                            
        \caption{ ``qa-1'' task}
        \end{subfigure}
        \begin{subfigure}[b]{0.24\linewidth}
        \includegraphics[width=\linewidth]{plots/ablations/qa_2.pdf}
                            
        \caption{ ``qa-2'' task}
        \end{subfigure}
            \begin{subfigure}[b]{0.24\linewidth}
        \includegraphics[width=\linewidth]{plots/ablations/niah_multiquery.pdf}
                            
        \caption{ ``niah'' task}
        \end{subfigure}
            \begin{subfigure}[b]{0.24\linewidth}
        \includegraphics[width=\linewidth]{plots/ablations/cwe.pdf}
                            
        \caption{ ``cwe'' task}
        \end{subfigure}
        

    \caption{\small  Ablations for varying prompts. Same as Figure~\ref{fig:prompts} and \ref{fig:promptsfwe} for the additional RULER $8$k tasks using Llama 3-8B.} 
    \label{fig:ablations-vt-extra}
\end{figure*}




















\begin{figure}
    \centering
    \includegraphics[width=\linewidth]{plots/individual/llama38192.pdf}
    \caption{ Performances for individual tasks for RULER $8$k using Llama-3 8B as in Figure~\ref{fig:compare-approx}}
    \label{fig:llama8k}
\end{figure}

\begin{figure}
    \centering
    \includegraphics[width=\linewidth]{plots/individual/llama316384.pdf}
    \caption{ Performances for individual tasks for RULER $16$k using Llama-3 8B as in Figure~\ref{fig:compare-approx}}
    \label{fig:llama16k}
\end{figure}


\begin{figure}
    \centering
    \includegraphics[width=\linewidth]{plots/individual/mistralai8192.pdf}
    \caption{ Performances for individual tasks for RULER $8$k using Mistral-7B as in Figure~\ref{fig:compare-approx}}
    \label{fig:mistral8k}
\end{figure}

\begin{figure}
    \centering
    \includegraphics[width=\linewidth]{plots/individual/mistralai16384.pdf}
    \caption{ Performances for individual tasks for RULER $16$k using Mistral-7B as in Figure~\ref{fig:compare-approx}}
    \label{fig:mistral16k}
\end{figure}


\begin{figure}
    \centering
    \includegraphics[width=\linewidth]{plots/individual/Qwen8192.pdf}
    \caption{ Performances for individual tasks for RULER $8$k using Qwen-7B as in Figure~\ref{fig:compare-approx}}
    \label{fig:qwen9k}
\end{figure}

\begin{figure}
    \centering
    \includegraphics[width=\linewidth]{plots/individual/Qwen16384.pdf}
    \caption{ Performances for individual tasks for RULER $16$k using Qwen-7B as in Figure~\ref{fig:compare-approx}}
    \label{fig:qwen16k}
\end{figure}


\begin{figure}
    \centering
    \includegraphics[width=\linewidth]{plots/individual/LONG.pdf}
    \caption{\small Performances for individual tasks for LongBench as in Figure~\ref{fig:longbench}}
    \label{fig:longappendix}
\end{figure}


















\begin{figure*}
\begin{tcolorbox}[
    title=Example Prompt for long-context MBPP,
    width=\textwidth,
    colback=white,
    left=5pt,
    right=5pt,
    top=5pt,
    bottom=5pt
]
[...]\\
Q: Due to extreme variation in elevation, great variation occurs in the climatic conditions of Himachal . The climate varies from hot and subhumid tropical in the southern tracts to, with more elevation, cold, alpine, and glacial in the northern and eastern mountain ranges. The state has areas like Dharamsala that receive very heavy rainfall, as well as those like Lahaul and Spiti that are cold and almost rainless. Broadly, Himachal experiences three seasons: summer, winter, and rainy season. Summer lasts from mid-April till the end of June and most parts become very hot (except in the alpine zone which experiences a mild summer) with the average temperature ranging from 28 to 32 °C (82 to 90 °F). Winter lasts from late November till mid March. Snowfall is common in alpine tracts (generally above 2,200 metres (7,218 ft) i.e. in the higher and trans-Himalayan region).\\
What is the climate like?\\
A: varies from hot and subhumid tropical  The answer is varies from hot and subhumid tropical.\\\\
\textcolor{darkgreen}{
Q: James decides to buy a new bed and bed frame.  The bed frame is \$75 and the bed is 10 times that price.  He gets a deal for 20\% off.  How much does he pay for everything?\\
A: The bed cost 75*10=\$750\\
So everything cost 750+75=\$825\\
He gets 825*.2=\$165 off\\
So that means he pays 825-165=\$660 The answer is 660.}
\\\\
\textcolor{darkgreen}{
Q: Liz sold her car at 80\% of what she originally paid. She uses the proceeds of that sale and needs only \$4,000 to buy herself a new \$30,000 car. How much cheaper is her new car versus what she originally paid for her old one?\\
A: If Liz needs only \$4,000 to buy a new \$30,000 car, that means she has \$30,000-\$4,000=\$26,000 from the proceeds of selling her old car\\
If she sold her car at 80\% of what she originally paid for and sold it for \$26,000 then she originally paid \$26,000/80\% = \$32,500 for her old car\\
If she paid \$32,500 for her old car and the new one is \$30,000 then, the new one is \$32,500-\$30,000 = \$2,500 cheaper The answer is 2500.}\\\\
Q: Unlike in multicellular organisms, increases in cell size (cell growth) and reproduction by cell division are tightly linked in unicellular organisms. Bacteria grow to a fixed size and then reproduce through binary fission, a form of asexual reproduction. Under optimal conditions, bacteria can grow and divide extremely rapidly, and bacterial populations can double as quickly as every 9.8 minutes. In cell division, two identical clone daughter cells are produced. Some bacteria, while still reproducing asexually, form more complex reproductive structures that help disperse the newly formed daughter cells. Examples include fruiting body formation by Myxobacteria and aerial hyphae formation by Streptomyces, or budding. Budding involves a cell forming a protrusion that breaks away and produces a daughter cell.\\
What are produced in cell division?\\
A: two identical clone daughter cells  The answer is two identical clone daughter cells.\\\\
\textcolor{darkgreen}{
Q: Janet’s ducks lay 16 eggs per day. She eats three for breakfast every morning and bakes muffins for her friends every day with four. She sells the remainder at the farmers' market daily for \$2 per fresh duck egg. How much in dollars does she make every day at the farmers' market?\\
A:}
\end{tcolorbox}
\end{figure*}



\begin{figure*}

\begin{tcolorbox}[
    title=Example Prompt for long-context GSM8k,
    width=\textwidth,
    colback=white,
    left=5pt,
    right=5pt,
    top=5pt,
    bottom=5pt
]
[...]
You are an expert Python programmer, and here is your task: Due to extreme variation in elevation, great variation occurs in the climatic conditions of Himachal . The climate varies from hot and subhumid tropical in the southern tracts to, with more elevation, cold, alpine, and glacial in the northern and eastern mountain ranges. The state has areas like Dharamsala that receive very heavy rainfall, as well as those like Lahaul and Spiti that are cold and almost rainless. 
Broadly, Himachal experiences three seasons: summer, winter, and rainy season. Summer lasts from mid-April till the end of June and most parts become very hot (except in the alpine zone which experiences a mild summer) with the average temperature ranging from 28 to 32 °C (82 to 90 °F). Winter lasts from late November till mid March. Snowfall is common in alpine tracts (generally above 2,200 metres (7,218 ft) i.e. in the higher and trans-Himalayan region).\\
What is the climate like? Your code should pass these tests:
\\
empty
\\
{[BEGIN]}\\
varies from hot and subhumid tropical\\
{[DONE]}
\\\\
\textcolor{darkgreen}{
You are an expert Python programmer, and here is your task: Write a function to find the similar elements from the given two tuple lists. Your code should pass these tests:\\
assert similar\_elements((3, 4, 5, 6),(5, 7, 4, 10)) == (4, 5)\\
assert similar\_elements((1, 2, 3, 4),(5, 4, 3, 7)) == (3, 4)\\
assert similar\_elements((11, 12, 14, 13),(17, 15, 14, 13)) == (13, 14)\\\\
{[BEGIN]}\\
def similar\_elements(test\_tup1, test\_tup2):\\
  res = tuple(set(test\_tup1) \& set(test\_tup2))\\
  return (res) \\
{[DONE]}}\\\\
You are an expert Python programmer, and here is your task: Unlike in multicellular organisms, increases in cell size (cell growth) and reproduction by cell division are tightly linked in unicellular organisms. Bacteria grow to a fixed size and then reproduce through binary fission, a form of asexual reproduction. Under optimal conditions, bacteria can grow and divide extremely rapidly, and bacterial populations can double as quickly as every 9.8 minutes. In cell division, two identical clone daughter cells are produced. Some bacteria, while still reproducing asexually, form more complex reproductive structures that help disperse the newly formed daughter cells. Examples include fruiting body formation by Myxobacteria and aerial hyphae formation by Streptomyces, or budding. Budding involves a cell forming a protrusion that breaks away and produces a daughter cell.\\
What are produced in cell division? Your code should pass these tests:\\
empty
\\
{[BEGIN]}\\
two identical clone daughter cells\\
{[DONE]}\\\\
\textcolor{darkgreen}{
You are an expert Python programmer, and here is your task: Write a python function to remove first and last occurrence of a given character from the string. Your code should pass these tests:\\
assert remove\_Occ("hello","l") == "heo"\\
assert remove\_Occ("abcda","a") == "bcd"\\
assert remove\_Occ("PHP","P") == "H"\\\\
{[BEGIN]}\\}

\end{tcolorbox}
\end{figure*}


\end{document}
