%%%%%%%% ICML 2025 EXAMPLE LATEX SUBMISSION FILE %%%%%%%%%%%%%%%%%

\documentclass{article}

\usepackage[dvipsnames]{xcolor}
% Recommended, but optional, packages for figures and better typesetting:
\usepackage{microtype}
\usepackage{subfigure}
\usepackage{booktabs} % for professional tables
% Optional math commands from https://github.com/goodfeli/dlbook_notation.
%%%%% NEW MATH DEFINITIONS %%%%%

\usepackage{amsmath,amsfonts,bm}
\usepackage{derivative}
% Mark sections of captions for referring to divisions of figures
\newcommand{\figleft}{{\em (Left)}}
\newcommand{\figcenter}{{\em (Center)}}
\newcommand{\figright}{{\em (Right)}}
\newcommand{\figtop}{{\em (Top)}}
\newcommand{\figbottom}{{\em (Bottom)}}
\newcommand{\captiona}{{\em (a)}}
\newcommand{\captionb}{{\em (b)}}
\newcommand{\captionc}{{\em (c)}}
\newcommand{\captiond}{{\em (d)}}

% Highlight a newly defined term
\newcommand{\newterm}[1]{{\bf #1}}

% Derivative d 
\newcommand{\deriv}{{\mathrm{d}}}

% Figure reference, lower-case.
\def\figref#1{figure~\ref{#1}}
% Figure reference, capital. For start of sentence
\def\Figref#1{Figure~\ref{#1}}
\def\twofigref#1#2{figures \ref{#1} and \ref{#2}}
\def\quadfigref#1#2#3#4{figures \ref{#1}, \ref{#2}, \ref{#3} and \ref{#4}}
% Section reference, lower-case.
\def\secref#1{section~\ref{#1}}
% Section reference, capital.
\def\Secref#1{Section~\ref{#1}}
% Reference to two sections.
\def\twosecrefs#1#2{sections \ref{#1} and \ref{#2}}
% Reference to three sections.
\def\secrefs#1#2#3{sections \ref{#1}, \ref{#2} and \ref{#3}}
% Reference to an equation, lower-case.
\def\eqref#1{equation~\ref{#1}}
% Reference to an equation, upper case
\def\Eqref#1{Equation~\ref{#1}}
% A raw reference to an equation---avoid using if possible
\def\plaineqref#1{\ref{#1}}
% Reference to a chapter, lower-case.
\def\chapref#1{chapter~\ref{#1}}
% Reference to an equation, upper case.
\def\Chapref#1{Chapter~\ref{#1}}
% Reference to a range of chapters
\def\rangechapref#1#2{chapters\ref{#1}--\ref{#2}}
% Reference to an algorithm, lower-case.
\def\algref#1{algorithm~\ref{#1}}
% Reference to an algorithm, upper case.
\def\Algref#1{Algorithm~\ref{#1}}
\def\twoalgref#1#2{algorithms \ref{#1} and \ref{#2}}
\def\Twoalgref#1#2{Algorithms \ref{#1} and \ref{#2}}
% Reference to a part, lower case
\def\partref#1{part~\ref{#1}}
% Reference to a part, upper case
\def\Partref#1{Part~\ref{#1}}
\def\twopartref#1#2{parts \ref{#1} and \ref{#2}}

\def\ceil#1{\lceil #1 \rceil}
\def\floor#1{\lfloor #1 \rfloor}
\def\1{\bm{1}}
\newcommand{\train}{\mathcal{D}}
\newcommand{\valid}{\mathcal{D_{\mathrm{valid}}}}
\newcommand{\test}{\mathcal{D_{\mathrm{test}}}}

\def\eps{{\epsilon}}


% Random variables
\def\reta{{\textnormal{$\eta$}}}
\def\ra{{\textnormal{a}}}
\def\rb{{\textnormal{b}}}
\def\rc{{\textnormal{c}}}
\def\rd{{\textnormal{d}}}
\def\re{{\textnormal{e}}}
\def\rf{{\textnormal{f}}}
\def\rg{{\textnormal{g}}}
\def\rh{{\textnormal{h}}}
\def\ri{{\textnormal{i}}}
\def\rj{{\textnormal{j}}}
\def\rk{{\textnormal{k}}}
\def\rl{{\textnormal{l}}}
% rm is already a command, just don't name any random variables m
\def\rn{{\textnormal{n}}}
\def\ro{{\textnormal{o}}}
\def\rp{{\textnormal{p}}}
\def\rq{{\textnormal{q}}}
\def\rr{{\textnormal{r}}}
\def\rs{{\textnormal{s}}}
\def\rt{{\textnormal{t}}}
\def\ru{{\textnormal{u}}}
\def\rv{{\textnormal{v}}}
\def\rw{{\textnormal{w}}}
\def\rx{{\textnormal{x}}}
\def\ry{{\textnormal{y}}}
\def\rz{{\textnormal{z}}}

% Random vectors
\def\rvepsilon{{\mathbf{\epsilon}}}
\def\rvphi{{\mathbf{\phi}}}
\def\rvtheta{{\mathbf{\theta}}}
\def\rva{{\mathbf{a}}}
\def\rvb{{\mathbf{b}}}
\def\rvc{{\mathbf{c}}}
\def\rvd{{\mathbf{d}}}
\def\rve{{\mathbf{e}}}
\def\rvf{{\mathbf{f}}}
\def\rvg{{\mathbf{g}}}
\def\rvh{{\mathbf{h}}}
\def\rvu{{\mathbf{i}}}
\def\rvj{{\mathbf{j}}}
\def\rvk{{\mathbf{k}}}
\def\rvl{{\mathbf{l}}}
\def\rvm{{\mathbf{m}}}
\def\rvn{{\mathbf{n}}}
\def\rvo{{\mathbf{o}}}
\def\rvp{{\mathbf{p}}}
\def\rvq{{\mathbf{q}}}
\def\rvr{{\mathbf{r}}}
\def\rvs{{\mathbf{s}}}
\def\rvt{{\mathbf{t}}}
\def\rvu{{\mathbf{u}}}
\def\rvv{{\mathbf{v}}}
\def\rvw{{\mathbf{w}}}
\def\rvx{{\mathbf{x}}}
\def\rvy{{\mathbf{y}}}
\def\rvz{{\mathbf{z}}}

% Elements of random vectors
\def\erva{{\textnormal{a}}}
\def\ervb{{\textnormal{b}}}
\def\ervc{{\textnormal{c}}}
\def\ervd{{\textnormal{d}}}
\def\erve{{\textnormal{e}}}
\def\ervf{{\textnormal{f}}}
\def\ervg{{\textnormal{g}}}
\def\ervh{{\textnormal{h}}}
\def\ervi{{\textnormal{i}}}
\def\ervj{{\textnormal{j}}}
\def\ervk{{\textnormal{k}}}
\def\ervl{{\textnormal{l}}}
\def\ervm{{\textnormal{m}}}
\def\ervn{{\textnormal{n}}}
\def\ervo{{\textnormal{o}}}
\def\ervp{{\textnormal{p}}}
\def\ervq{{\textnormal{q}}}
\def\ervr{{\textnormal{r}}}
\def\ervs{{\textnormal{s}}}
\def\ervt{{\textnormal{t}}}
\def\ervu{{\textnormal{u}}}
\def\ervv{{\textnormal{v}}}
\def\ervw{{\textnormal{w}}}
\def\ervx{{\textnormal{x}}}
\def\ervy{{\textnormal{y}}}
\def\ervz{{\textnormal{z}}}

% Random matrices
\def\rmA{{\mathbf{A}}}
\def\rmB{{\mathbf{B}}}
\def\rmC{{\mathbf{C}}}
\def\rmD{{\mathbf{D}}}
\def\rmE{{\mathbf{E}}}
\def\rmF{{\mathbf{F}}}
\def\rmG{{\mathbf{G}}}
\def\rmH{{\mathbf{H}}}
\def\rmI{{\mathbf{I}}}
\def\rmJ{{\mathbf{J}}}
\def\rmK{{\mathbf{K}}}
\def\rmL{{\mathbf{L}}}
\def\rmM{{\mathbf{M}}}
\def\rmN{{\mathbf{N}}}
\def\rmO{{\mathbf{O}}}
\def\rmP{{\mathbf{P}}}
\def\rmQ{{\mathbf{Q}}}
\def\rmR{{\mathbf{R}}}
\def\rmS{{\mathbf{S}}}
\def\rmT{{\mathbf{T}}}
\def\rmU{{\mathbf{U}}}
\def\rmV{{\mathbf{V}}}
\def\rmW{{\mathbf{W}}}
\def\rmX{{\mathbf{X}}}
\def\rmY{{\mathbf{Y}}}
\def\rmZ{{\mathbf{Z}}}

% Elements of random matrices
\def\ermA{{\textnormal{A}}}
\def\ermB{{\textnormal{B}}}
\def\ermC{{\textnormal{C}}}
\def\ermD{{\textnormal{D}}}
\def\ermE{{\textnormal{E}}}
\def\ermF{{\textnormal{F}}}
\def\ermG{{\textnormal{G}}}
\def\ermH{{\textnormal{H}}}
\def\ermI{{\textnormal{I}}}
\def\ermJ{{\textnormal{J}}}
\def\ermK{{\textnormal{K}}}
\def\ermL{{\textnormal{L}}}
\def\ermM{{\textnormal{M}}}
\def\ermN{{\textnormal{N}}}
\def\ermO{{\textnormal{O}}}
\def\ermP{{\textnormal{P}}}
\def\ermQ{{\textnormal{Q}}}
\def\ermR{{\textnormal{R}}}
\def\ermS{{\textnormal{S}}}
\def\ermT{{\textnormal{T}}}
\def\ermU{{\textnormal{U}}}
\def\ermV{{\textnormal{V}}}
\def\ermW{{\textnormal{W}}}
\def\ermX{{\textnormal{X}}}
\def\ermY{{\textnormal{Y}}}
\def\ermZ{{\textnormal{Z}}}

% Vectors
\def\vzero{{\bm{0}}}
\def\vone{{\bm{1}}}
\def\vmu{{\bm{\mu}}}
\def\vtheta{{\bm{\theta}}}
\def\vphi{{\bm{\phi}}}
\def\va{{\bm{a}}}
\def\vb{{\bm{b}}}
\def\vc{{\bm{c}}}
\def\vd{{\bm{d}}}
\def\ve{{\bm{e}}}
\def\vf{{\bm{f}}}
\def\vg{{\bm{g}}}
\def\vh{{\bm{h}}}
\def\vi{{\bm{i}}}
\def\vj{{\bm{j}}}
\def\vk{{\bm{k}}}
\def\vl{{\bm{l}}}
\def\vm{{\bm{m}}}
\def\vn{{\bm{n}}}
\def\vo{{\bm{o}}}
\def\vp{{\bm{p}}}
\def\vq{{\bm{q}}}
\def\vr{{\bm{r}}}
\def\vs{{\bm{s}}}
\def\vt{{\bm{t}}}
\def\vu{{\bm{u}}}
\def\vv{{\bm{v}}}
\def\vw{{\bm{w}}}
\def\vx{{\bm{x}}}
\def\vy{{\bm{y}}}
\def\vz{{\bm{z}}}

% Elements of vectors
\def\evalpha{{\alpha}}
\def\evbeta{{\beta}}
\def\evepsilon{{\epsilon}}
\def\evlambda{{\lambda}}
\def\evomega{{\omega}}
\def\evmu{{\mu}}
\def\evpsi{{\psi}}
\def\evsigma{{\sigma}}
\def\evtheta{{\theta}}
\def\eva{{a}}
\def\evb{{b}}
\def\evc{{c}}
\def\evd{{d}}
\def\eve{{e}}
\def\evf{{f}}
\def\evg{{g}}
\def\evh{{h}}
\def\evi{{i}}
\def\evj{{j}}
\def\evk{{k}}
\def\evl{{l}}
\def\evm{{m}}
\def\evn{{n}}
\def\evo{{o}}
\def\evp{{p}}
\def\evq{{q}}
\def\evr{{r}}
\def\evs{{s}}
\def\evt{{t}}
\def\evu{{u}}
\def\evv{{v}}
\def\evw{{w}}
\def\evx{{x}}
\def\evy{{y}}
\def\evz{{z}}

% Matrix
\def\mA{{\bm{A}}}
\def\mB{{\bm{B}}}
\def\mC{{\bm{C}}}
\def\mD{{\bm{D}}}
\def\mE{{\bm{E}}}
\def\mF{{\bm{F}}}
\def\mG{{\bm{G}}}
\def\mH{{\bm{H}}}
\def\mI{{\bm{I}}}
\def\mJ{{\bm{J}}}
\def\mK{{\bm{K}}}
\def\mL{{\bm{L}}}
\def\mM{{\bm{M}}}
\def\mN{{\bm{N}}}
\def\mO{{\bm{O}}}
\def\mP{{\bm{P}}}
\def\mQ{{\bm{Q}}}
\def\mR{{\bm{R}}}
\def\mS{{\bm{S}}}
\def\mT{{\bm{T}}}
\def\mU{{\bm{U}}}
\def\mV{{\bm{V}}}
\def\mW{{\bm{W}}}
\def\mX{{\bm{X}}}
\def\mY{{\bm{Y}}}
\def\mZ{{\bm{Z}}}
\def\mBeta{{\bm{\beta}}}
\def\mPhi{{\bm{\Phi}}}
\def\mLambda{{\bm{\Lambda}}}
\def\mSigma{{\bm{\Sigma}}}

% Tensor
\DeclareMathAlphabet{\mathsfit}{\encodingdefault}{\sfdefault}{m}{sl}
\SetMathAlphabet{\mathsfit}{bold}{\encodingdefault}{\sfdefault}{bx}{n}
\newcommand{\tens}[1]{\bm{\mathsfit{#1}}}
\def\tA{{\tens{A}}}
\def\tB{{\tens{B}}}
\def\tC{{\tens{C}}}
\def\tD{{\tens{D}}}
\def\tE{{\tens{E}}}
\def\tF{{\tens{F}}}
\def\tG{{\tens{G}}}
\def\tH{{\tens{H}}}
\def\tI{{\tens{I}}}
\def\tJ{{\tens{J}}}
\def\tK{{\tens{K}}}
\def\tL{{\tens{L}}}
\def\tM{{\tens{M}}}
\def\tN{{\tens{N}}}
\def\tO{{\tens{O}}}
\def\tP{{\tens{P}}}
\def\tQ{{\tens{Q}}}
\def\tR{{\tens{R}}}
\def\tS{{\tens{S}}}
\def\tT{{\tens{T}}}
\def\tU{{\tens{U}}}
\def\tV{{\tens{V}}}
\def\tW{{\tens{W}}}
\def\tX{{\tens{X}}}
\def\tY{{\tens{Y}}}
\def\tZ{{\tens{Z}}}


% Graph
\def\gA{{\mathcal{A}}}
\def\gB{{\mathcal{B}}}
\def\gC{{\mathcal{C}}}
\def\gD{{\mathcal{D}}}
\def\gE{{\mathcal{E}}}
\def\gF{{\mathcal{F}}}
\def\gG{{\mathcal{G}}}
\def\gH{{\mathcal{H}}}
\def\gI{{\mathcal{I}}}
\def\gJ{{\mathcal{J}}}
\def\gK{{\mathcal{K}}}
\def\gL{{\mathcal{L}}}
\def\gM{{\mathcal{M}}}
\def\gN{{\mathcal{N}}}
\def\gO{{\mathcal{O}}}
\def\gP{{\mathcal{P}}}
\def\gQ{{\mathcal{Q}}}
\def\gR{{\mathcal{R}}}
\def\gS{{\mathcal{S}}}
\def\gT{{\mathcal{T}}}
\def\gU{{\mathcal{U}}}
\def\gV{{\mathcal{V}}}
\def\gW{{\mathcal{W}}}
\def\gX{{\mathcal{X}}}
\def\gY{{\mathcal{Y}}}
\def\gZ{{\mathcal{Z}}}

% Sets
\def\sA{{\mathbb{A}}}
\def\sB{{\mathbb{B}}}
\def\sC{{\mathbb{C}}}
\def\sD{{\mathbb{D}}}
% Don't use a set called E, because this would be the same as our symbol
% for expectation.
\def\sF{{\mathbb{F}}}
\def\sG{{\mathbb{G}}}
\def\sH{{\mathbb{H}}}
\def\sI{{\mathbb{I}}}
\def\sJ{{\mathbb{J}}}
\def\sK{{\mathbb{K}}}
\def\sL{{\mathbb{L}}}
\def\sM{{\mathbb{M}}}
\def\sN{{\mathbb{N}}}
\def\sO{{\mathbb{O}}}
\def\sP{{\mathbb{P}}}
\def\sQ{{\mathbb{Q}}}
\def\sR{{\mathbb{R}}}
\def\sS{{\mathbb{S}}}
\def\sT{{\mathbb{T}}}
\def\sU{{\mathbb{U}}}
\def\sV{{\mathbb{V}}}
\def\sW{{\mathbb{W}}}
\def\sX{{\mathbb{X}}}
\def\sY{{\mathbb{Y}}}
\def\sZ{{\mathbb{Z}}}

% Entries of a matrix
\def\emLambda{{\Lambda}}
\def\emA{{A}}
\def\emB{{B}}
\def\emC{{C}}
\def\emD{{D}}
\def\emE{{E}}
\def\emF{{F}}
\def\emG{{G}}
\def\emH{{H}}
\def\emI{{I}}
\def\emJ{{J}}
\def\emK{{K}}
\def\emL{{L}}
\def\emM{{M}}
\def\emN{{N}}
\def\emO{{O}}
\def\emP{{P}}
\def\emQ{{Q}}
\def\emR{{R}}
\def\emS{{S}}
\def\emT{{T}}
\def\emU{{U}}
\def\emV{{V}}
\def\emW{{W}}
\def\emX{{X}}
\def\emY{{Y}}
\def\emZ{{Z}}
\def\emSigma{{\Sigma}}

% entries of a tensor
% Same font as tensor, without \bm wrapper
\newcommand{\etens}[1]{\mathsfit{#1}}
\def\etLambda{{\etens{\Lambda}}}
\def\etA{{\etens{A}}}
\def\etB{{\etens{B}}}
\def\etC{{\etens{C}}}
\def\etD{{\etens{D}}}
\def\etE{{\etens{E}}}
\def\etF{{\etens{F}}}
\def\etG{{\etens{G}}}
\def\etH{{\etens{H}}}
\def\etI{{\etens{I}}}
\def\etJ{{\etens{J}}}
\def\etK{{\etens{K}}}
\def\etL{{\etens{L}}}
\def\etM{{\etens{M}}}
\def\etN{{\etens{N}}}
\def\etO{{\etens{O}}}
\def\etP{{\etens{P}}}
\def\etQ{{\etens{Q}}}
\def\etR{{\etens{R}}}
\def\etS{{\etens{S}}}
\def\etT{{\etens{T}}}
\def\etU{{\etens{U}}}
\def\etV{{\etens{V}}}
\def\etW{{\etens{W}}}
\def\etX{{\etens{X}}}
\def\etY{{\etens{Y}}}
\def\etZ{{\etens{Z}}}

% The true underlying data generating distribution
\newcommand{\pdata}{p_{\rm{data}}}
\newcommand{\ptarget}{p_{\rm{target}}}
\newcommand{\pprior}{p_{\rm{prior}}}
\newcommand{\pbase}{p_{\rm{base}}}
\newcommand{\pref}{p_{\rm{ref}}}

% The empirical distribution defined by the training set
\newcommand{\ptrain}{\hat{p}_{\rm{data}}}
\newcommand{\Ptrain}{\hat{P}_{\rm{data}}}
% The model distribution
\newcommand{\pmodel}{p_{\rm{model}}}
\newcommand{\Pmodel}{P_{\rm{model}}}
\newcommand{\ptildemodel}{\tilde{p}_{\rm{model}}}
% Stochastic autoencoder distributions
\newcommand{\pencode}{p_{\rm{encoder}}}
\newcommand{\pdecode}{p_{\rm{decoder}}}
\newcommand{\precons}{p_{\rm{reconstruct}}}

\newcommand{\laplace}{\mathrm{Laplace}} % Laplace distribution

\newcommand{\E}{\mathbb{E}}
\newcommand{\Ls}{\mathcal{L}}
\newcommand{\R}{\mathbb{R}}
\newcommand{\emp}{\tilde{p}}
\newcommand{\lr}{\alpha}
\newcommand{\reg}{\lambda}
\newcommand{\rect}{\mathrm{rectifier}}
\newcommand{\softmax}{\mathrm{softmax}}
\newcommand{\sigmoid}{\sigma}
\newcommand{\softplus}{\zeta}
\newcommand{\KL}{D_{\mathrm{KL}}}
\newcommand{\Var}{\mathrm{Var}}
\newcommand{\standarderror}{\mathrm{SE}}
\newcommand{\Cov}{\mathrm{Cov}}
% Wolfram Mathworld says $L^2$ is for function spaces and $\ell^2$ is for vectors
% But then they seem to use $L^2$ for vectors throughout the site, and so does
% wikipedia.
\newcommand{\normlzero}{L^0}
\newcommand{\normlone}{L^1}
\newcommand{\normltwo}{L^2}
\newcommand{\normlp}{L^p}
\newcommand{\normmax}{L^\infty}

\newcommand{\parents}{Pa} % See usage in notation.tex. Chosen to match Daphne's book.

\DeclareMathOperator*{\argmax}{arg\,max}
\DeclareMathOperator*{\argmin}{arg\,min}

\DeclareMathOperator{\sign}{sign}
\DeclareMathOperator{\Tr}{Tr}
\let\ab\allowbreak


% \usepackage{hyperref}
\usepackage[hyperfootnotes=false]{hyperref}
% \usepackage{url}
\usepackage{booktabs}
\usepackage{graphicx}
\usepackage{amsmath}
\usepackage{wrapfig}
\usepackage{longtable}
\usepackage{caption}
\captionsetup[longtable]{width=\textwidth}

\newcommand{\rylan}[1]{\textcolor{red}{rylan: #1}}
\newcommand{\sanmi}[1]{\textcolor{blue}{sanmi: #1}}



% hyperref makes hyperlinks in the resulting PDF.
% If your build breaks (sometimes temporarily if a hyperlink spans a page)
% please comment out the following usepackage line and replace
% \usepackage{icml2025} with \usepackage[nohyperref]{icml2025} above.
\usepackage{hyperref}


% Attempt to make hyperref and algorithmic work together better:
\newcommand{\theHalgorithm}{\arabic{algorithm}}

% Use the following line for the initial blind version submitted for review:
% \usepackage{icml2025}

% If accepted, instead use the following line for the camera-ready submission:
\usepackage[accepted]{arxiv}

% For theorems and such
\usepackage{amsmath}
\usepackage{amssymb}
\usepackage{mathtools}
\usepackage{amsthm}

% if you use cleveref..
\usepackage[capitalize,noabbrev]{cleveref}

%%%%%%%%%%%%%%%%%%%%%%%%%%%%%%%%
% THEOREMS
%%%%%%%%%%%%%%%%%%%%%%%%%%%%%%%%
\theoremstyle{plain}
\newtheorem{theorem}{Theorem}[section]
\newtheorem{proposition}[theorem]{Proposition}
\newtheorem{lemma}[theorem]{Lemma}
\newtheorem{corollary}[theorem]{Corollary}
\theoremstyle{definition}
\newtheorem{definition}[theorem]{Definition}
\newtheorem{assumption}[theorem]{Assumption}
\theoremstyle{remark}
\newtheorem{remark}[theorem]{Remark}

% Todonotes is useful during development; simply uncomment the next line
%    and comment out the line below the next line to turn off comments
%\usepackage[disable,textsize=tiny]{todonotes}
\usepackage[textsize=tiny]{todonotes}


% The \icmltitle you define below is probably too long as a header.
% Therefore, a short form for the running title is supplied here:
\icmltitlerunning{Correlating and Predicting Human Evaluations of Language Models from Natural Language Processing Benchmarks}

\begin{document}

\twocolumn[
\icmltitle{Correlating and Predicting Human Evaluations of Language Models from Natural Language Processing Benchmarks}

% It is OKAY to include author information, even for blind
% submissions: the style file will automatically remove it for you
% unless you've provided the [accepted] option to the icml2025
% package.

% List of affiliations: The first argument should be a (short)
% identifier you will use later to specify author affiliations
% Academic affiliations should list Department, University, City, Region, Country
% Industry affiliations should list Company, City, Region, Country

% You can specify symbols, otherwise they are numbered in order.
% Ideally, you should not use this facility. Affiliations will be numbered
% in order of appearance and this is the preferred way.

% \icmlsetsymbol{equal}{*}

\begin{icmlauthorlist}
\icmlauthor{Rylan Schaeffer}{stanfordcs,note}
\icmlauthor{Punit Singh Koura}{meta}
\icmlauthor{Binh Tang}{meta}
\icmlauthor{Ranjan Subramanian}{meta}
\icmlauthor{Aaditya K. Singh}{ucl,note}
\icmlauthor{Todor Mihaylov}{meta}
\icmlauthor{Prajjwal Bhargava}{meta}
\icmlauthor{Lovish Madaan}{meta}
\icmlauthor{Niladri S. Chatterji}{meta}
\icmlauthor{Vedanuj Goswami}{meta}
\icmlauthor{Sergey Edunov}{meta}
\icmlauthor{Dieuwke Hupkes}{meta}
\icmlauthor{Sanmi Koyejo}{stanfordcs}
\icmlauthor{Sharan Narang}{meta}
\end{icmlauthorlist}

\icmlaffiliation{stanfordcs}{Stanford Computer Science}
\icmlaffiliation{ucl}{University College London}
\icmlaffiliation{meta}{Meta GenAI}
\icmlaffiliation{note}{Work completed during Meta GenAI internship.}

\icmlcorrespondingauthor{Rylan Schaeffer}{rschaef@cs.stanford.edu}
\icmlcorrespondingauthor{Sharan Narang}{sharann@meta.com}

% You may provide any keywords that you
% find helpful for describing your paper; these are used to populate
% the "keywords" metadata in the PDF but will not be shown in the document
\icmlkeywords{Machine Learning, ICML}

\vskip 0.3in
]

% this must go after the closing bracket ] following \twocolumn[ ...

% This command actually creates the footnote in the first column
% listing the affiliations and the copyright notice.
% The command takes one argument, which is text to display at the start of the footnote.
% The \icmlEqualContribution command is standard text for equal contribution.
% Remove it (just {}) if you do not need this facility.

\printAffiliationsAndNotice{}  % leave blank if no need to mention equal contribution
% \printAffiliationsAndNotice{\icmlEqualContribution} % otherwise use the standard text.

\begin{abstract}
    The explosion of high-performing conversational language models (LMs) has spurred a shift from classic natural language processing (NLP) benchmarks to expensive, time-consuming and noisy human evaluations — yet the relationship between these two evaluation strategies remains hazy. In this paper, we conduct a large-scale study of four Chat Llama 2 models, comparing their performance on 160 standard NLP benchmarks (e.g., MMLU, ARC, BIG-Bench Hard) against extensive human preferences on more than 11k single-turn and 2k multi-turn dialogues from over 2k human annotators. Our findings are striking: most NLP benchmarks strongly correlate with human evaluations, suggesting that cheaper, automated metrics can serve as surprisingly reliable predictors of human preferences. Three human evaluations, such as adversarial dishonesty and safety, are anticorrelated with NLP benchmarks, while two are uncorrelated.
    Moreover, through overparameterized linear regressions, we show that NLP scores can accurately predict human evaluations across different model scales, offering a path to reduce costly human annotation without sacrificing rigor. Overall, our results affirm the continued value of classic benchmarks and illuminate how to harness them to anticipate real-world user satisfaction — pointing to how NLP benchmarks can be leveraged to meet evaluation needs of our new era of conversational AI.
\end{abstract}

\section{Introduction}
\label{section:introduction}

% redirection is unique and important in VR
Virtual Reality (VR) systems enable users to embody virtual avatars by mirroring their physical movements and aligning their perspective with virtual avatars' in real time. 
As the head-mounted displays (HMDs) block direct visual access to the physical world, users primarily rely on visual feedback from the virtual environment and integrate it with proprioceptive cues to control the avatar’s movements and interact within the VR space.
Since human perception is heavily influenced by visual input~\cite{gibson1933adaptation}, 
VR systems have the unique capability to control users' perception of the virtual environment and avatars by manipulating the visual information presented to them.
Leveraging this, various redirection techniques have been proposed to enable novel VR interactions, 
such as redirecting users' walking paths~\cite{razzaque2005redirected, suma2012impossible, steinicke2009estimation},
modifying reaching movements~\cite{gonzalez2022model, azmandian2016haptic, cheng2017sparse, feick2021visuo},
and conveying haptic information through visual feedback to create pseudo-haptic effects~\cite{samad2019pseudo, dominjon2005influence, lecuyer2009simulating}.
Such redirection techniques enable these interactions by manipulating the alignment between users' physical movements and their virtual avatar's actions.

% % what is hand/arm redirection, motivation of study arm-offset
% \change{\yj{i don't understand the purpose of this paragraph}
% These illusion-based techniques provide users with unique experiences in virtual environments that differ from the physical world yet maintain an immersive experience. 
% A key example is hand redirection, which shifts the virtual hand’s position away from the real hand as the user moves to enhance ergonomics during interaction~\cite{feuchtner2018ownershift, wentzel2020improving} and improve interaction performance~\cite{montano2017erg, poupyrev1996go}. 
% To increase the realism of virtual movements and strengthen the user’s sense of embodiment, hand redirection techniques often incorporate a complete virtual arm or full body alongside the redirected virtual hand, using inverse kinematics~\cite{hartfill2021analysis, ponton2024stretch} or adjustments to the virtual arm's movement as well~\cite{li2022modeling, feick2024impact}.
% }

% noticeability, motivation of predicting a probability, not a classification
However, these redirection techniques are most effective when the manipulation remains undetected~\cite{gonzalez2017model, li2022modeling}. 
If the redirection becomes too large, the user may not mitigate the conflict between the visual sensory input (redirected virtual movement) and their proprioception (actual physical movement), potentially leading to a loss of embodiment with the virtual avatar and making it difficult for the user to accurately control virtual movements to complete interaction tasks~\cite{li2022modeling, wentzel2020improving, feuchtner2018ownershift}. 
While proprioception is not absolute, users only have a general sense of their physical movements and the likelihood that they notice the redirection is probabilistic. 
This probability of detecting the redirection is referred to as \textbf{noticeability}~\cite{li2022modeling, zenner2024beyond, zenner2023detectability} and is typically estimated based on the frequency with which users detect the manipulation across multiple trials.

% version B
% Prior research has explored factors influencing the noticeability of redirected motion, including the redirection's magnitude~\cite{wentzel2020improving, poupyrev1996go}, direction~\cite{li2022modeling, feuchtner2018ownershift}, and the visual characteristics of the virtual avatar~\cite{ogawa2020effect, feick2024impact}.
% While these factors focus on the avatars, the surrounding virtual environment can also influence the users' behavior and in turn affect the noticeability of redirection.
% One such prominent external influence is through the visual channel - the users' visual attention is constantly distracted by complex visual effects and events in practical VR scenarios.
% Although some prior studies have explored how to leverage user blindness caused by visual distractions to redirect users' virtual hand~\cite{zenner2023detectability}, there remains a gap in understanding how to quantify the noticeability of redirection under visual distractions.

% visual stimuli and gaze behavior
Prior research has explored factors influencing the noticeability of redirected motion, including the redirection's magnitude~\cite{wentzel2020improving, poupyrev1996go}, direction~\cite{li2022modeling, feuchtner2018ownershift}, and the visual characteristics of the virtual avatar~\cite{ogawa2020effect, feick2024impact}.
While these factors focus on the avatars, the surrounding virtual environment can also influence the users' behavior and in turn affect the noticeability of redirection.
This, however, remains underexplored.
One such prominent external influence is through the visual channel - the users' visual attention is constantly distracted by complex visual effects and events in practical VR scenarios.
We thus want to investigate how \textbf{visual stimuli in the virtual environment} affect the noticeability of redirection.
With this, we hope to complement existing works that focus on avatars by incorporating environmental visual influences to enable more accurate control over the noticeability of redirected motions in practical VR scenarios.
% However, in realistic VR applications, the virtual environment often contains complex visual effects beyond the virtual avatar itself. 
% We argue that these visual effects can \textbf{distract users’ visual attention and thus affect the noticeability of redirection offsets}, while current research has yet taken into account.
% For instance, in a VR boxing scenario, a user’s visual attention is likely focused on their opponent rather than on their virtual body, leading to a lower noticeability of redirection offsets on their virtual movements. 
% Conversely, when reaching for an object in the center of their field of view, the user’s attention is more concentrated on the virtual hand’s movement and position to ensure successful interaction, resulting in a higher noticeability of offsets.

Since each visual event is a complex choreography of many underlying factors (type of visual effect, location, duration, etc.), it is extremely difficult to quantify or parameterize visual stimuli.
Furthermore, individuals respond differently to even the same visual events.
Prior neuroscience studies revealed that factors like age, gender, and personality can influence how quickly someone reacts to visual events~\cite{gillon2024responses, gale1997human}. 
Therefore, aiming to model visual stimuli in a way that is generalizable and applicable to different stimuli and users, we propose to use users' \textbf{gaze behavior} as an indicator of how they respond to visual stimuli.
In this paper, we used various gaze behaviors, including gaze location, saccades~\cite{krejtz2018eye}, fixations~\cite{perkhofer2019using}, and the Index of Pupil Activity (IPA)~\cite{duchowski2018index}.
These behaviors indicate both where users are looking and their cognitive activity, as looking at something does not necessarily mean they are attending to it.
Our goal is to investigate how these gaze behaviors stimulated by various visual stimuli relate to the noticeability of redirection.
With this, we contribute a model that allows designers and content creators to adjust the redirection in real-time responding to dynamic visual events in VR.

To achieve this, we conducted user studies to collect users' noticeability of redirection under various visual stimuli.
To simulate realistic VR scenarios, we adopted a dual-task design in which the participants performed redirected movements while monitoring the visual stimuli.
Specifically, participants' primary task was to report if they noticed an offset between the avatar's movement and their own, while their secondary task was to monitor and report the visual stimuli.
As realistic virtual environments often contain complex visual effects, we started with simple and controlled visual stimulus to manage the influencing factors.

% first user study, confirmation study
% collect data under no visual stimuli, different basic visual stimuli
We first conducted a confirmation study (N=16) to test whether applying visual stimuli (opacity-based) actually affects their noticeability of redirection. 
The results showed that participants were significantly less likely to detect the redirection when visual stimuli was presented $(F_{(1,15)}=5.90,~p=0.03)$.
Furthermore, by analyzing the collected gaze data, results revealed a correlation between the proposed gaze behaviors and the noticeability results $(r=-0.43)$, confirming that the gaze behaviors could be leveraged to compute the noticeability.

% data collection study
We then conducted a data collection study to obtain more accurate noticeability results through repeated measurements to better model the relationship between visual stimuli-triggered gaze behaviors and noticeability of redirection.
With the collected data, we analyzed various numerical features from the gaze behaviors to identify the most effective ones. 
We tested combinations of these features to determine the most effective one for predicting noticeability under visual stimuli.
Using the selected features, our regression model achieved a mean squared error (MSE) of 0.011 through leave-one-user-out cross-validation. 
Furthermore, we developed both a binary and a three-class classification model to categorize noticeability, which achieved an accuracy of 91.74\% and 85.62\%, respectively.

% evaluation study
To evaluate the generalizability of the regression model, we conducted an evaluation study (N=24) to test whether the model could accurately predict noticeability with new visual stimuli (color- and scale-based animations).
Specifically, we evaluated whether the model's predictions aligned with participants' responses under these unseen stimuli.
The results showed that our model accurately estimated the noticeability, achieving mean squared errors (MSE) of 0.014 and 0.012 for the color- and scale-based visual stimili, respectively, compared to participants' responses.
Since the tested visual stimuli data were not included in the training, the results suggested that the extracted gaze behavior features capture a generalizable pattern and can effectively indicate the corresponding impact on the noticeability of redirection.

% application
Based on our model, we implemented an adaptive redirection technique and demonstrated it through two applications: adaptive VR action game and opportunistic rendering.
We conducted a proof-of-concept user study (N=8) to compare our adaptive redirection technique with a static redirection, evaluating the usability and benefits of our adaptive redirection technique.
The results indicated that participants experienced less physical demand and stronger sense of embodiment and agency when using the adaptive redirection technique. 
These results demonstrated the effectiveness and usability of our model.

In summary, we make the following contributions.
% 
\begin{itemize}
    \item 
    We propose to use users' gaze behavior as a medium to quantify how visual stimuli influences the noticebility of redirection. 
    Through two user studies, we confirm that visual stimuli significantly influences noticeability and identify key gaze behavior features that are closely related to this impact.
    \item 
    We build a regression model that takes the user's gaze behavioral data as input, then computes the noticeability of redirection.
    Through an evaluation study, we verify that our model can estimate the noticeability with new participants under unseen visual stimuli.
    These findings suggest that the extracted gaze behavior features effectively capture the influence of visual stimuli on noticeability and can generalize across different users and visual stimuli.
    \item 
    We develop an adaptive redirection technique based on our regression model and implement two applications with it.
    With a proof-of-concept study, we demonstrate the effectiveness and potential usability of our regression model on real-world use cases.

\end{itemize}

% \delete{
% Virtual Reality (VR) allows the user to embody a virtual avatar by mirroring their physical movements through the avatar.
% As the user's visual access to the physical world is blocked in tasks involving motion control, they heavily rely on the visual representation of the avatar's motions to guide their proprioception.
% Similar to real-world experiences, the user is able to resolve conflicts between different sensory inputs (e.g., vision and motor control) through multisensory integration, which is essential for mitigating the sensory noise that commonly arises.
% However, it also enables unique manipulations in VR, as the system can intentionally modify the avatar's movements in relation to the user's motions to achieve specific functional outcomes,
% for example, 
% % the manipulations on the avatar's movements can 
% enabling novel interaction techniques of redirected walking~\cite{razzaque2005redirected}, redirected reaching~\cite{gonzalez2022model}, and pseudo haptics~\cite{samad2019pseudo}.
% With small adjustments to the avatar's movements, the user can maintain their sense of embodiment, due to their ability to resolve the perceptual differences.
% % However, a large mismatch between the user and avatar's movements can result in the user losing their sense of embodiment, due to an inability to resolve the perceptual differences.
% }

% \delete{
% However, multisensory integration can break when the manipulation is so intense that the user is aware of the existence of the motion offset and no longer maintains the sense of embodiment.
% Prior research studied the intensity threshold of the offset applied on the avatar's hand, beyond which the embodiment will break~\cite{li2022modeling}. 
% Studies also investigated the user's sensitivity to the offsets over time~\cite{kohm2022sensitivity}.
% Based on the findings, we argue that one crucial factor that affects to what extent the user notices the offset (i.e., \textit{noticeability}) that remains under-explored is whether the user directs their visual attention towards or away from the virtual avatar.
% Related work (e.g., Mise-unseen~\cite{marwecki2019mise}) has showcased applications where adjustments in the environment can be made in an unnoticeable manner when they happen in the area out of the user's visual field.
% We hypothesize that directing the user's visual attention away from the avatar's body, while still partially keeping the avatar within the user's field-of-view, can reduce the noticeability of the offset.
% Therefore, we conduct two user studies and implement a regression model to systematically investigate this effect.
% }

% \delete{
% In the first user study (N = 16), we test whether drawing the user's visual attention away from their body impacts the possibility of them noticing an offset that we apply to their arm motion in VR.
% We adopt a dual-task design to enable the alteration of the user's visual attention and a yes/no paradigm to measure the noticeability of motion offset. 
% The primary task for the user is to perform an arm motion and report when they perceive an offset between the avatar's virtual arm and their real arm.
% In the secondary task, we randomly render a visual animation of a ball turning from transparent to red and becoming transparent again and ask them to monitor and report when it appears.
% We control the strength of the visual stimuli by changing the duration and location of the animation.
% % By changing the time duration and location of the visual animation, we control the strengths of attraction to the users.
% As a result, we found significant differences in the noticeability of the offsets $(F_{(1,15)}=5.90,~p=0.03)$ between conditions with and without visual stimuli.
% Based on further analysis, we also identified the behavioral patterns of the user's gaze (including pupil dilation, fixations, and saccades) to be correlated with the noticeability results $(r=-0.43)$ and they may potentially serve as indicators of noticeability.
% }

% \delete{
% To further investigate how visual attention influences the noticeability, we conduct a data collection study (N = 12) and build a regression model based on the data.
% The regression model is able to calculate the noticeability of the offset applied on the user's arm under various visual stimuli based on their gaze behaviors.
% Our leave-one-out cross-validation results show that the proposed method was able to achieve a mean-squared error (MSE) of 0.012 in the probability regression task.
% }

% \delete{
% To verify the feasibility and extendability of the regression model, we conduct an evaluation study where we test new visual animations based on adjustments on scale and color and invite 24 new participants to attend the study.
% Results show that the proposed method can accurately estimate the noticeability with an MSE of 0.014 and 0.012 in the conditions of the color- and scale-based visual effects.
% Since these animations were not included in the dataset that the regression model was built on, the study demonstrates that the gaze behavioral features we extracted from the data capture a generalizable pattern of the user's visual attention and can indicate the corresponding impact on the noticeability of the offset.
% }

% \delete{
% Finally, we demonstrate applications that can benefit from the noticeability prediction model, including adaptive motion offsets and opportunistic rendering, considering the user's visual attention. 
% We conclude with discussions of our work's limitations and future research directions.
% }

% \delete{
% In summary, we make the following contributions.
% }
% % 
% \begin{itemize}
%     \item 
%     \delete{
%     We quantify the effects of the user's visual attention directed away by stimuli on their noticeability of an offset applied to the avatar's arm motion with respect to the user's physical arm. 
%     Through two user studies, we identified gaze behavioral features that are indicative of the changes in noticeability.
%     }
%     \item 
%     \delete{We build a regression model that takes the user's gaze behavioral data and the offset applied to the arm motion as input, then computes the probability of the user noticing the offset.
%     Through an evaluation study, we verified that the model needs no information about the source attracting the user's visual attention and can be generalizable in different scenarios.
%     }
%     \item 
%     \delete{We demonstrate two applications that potentially benefit from the regression model, including adaptive motion offsets and opportunistic rendering.
%     }

% \end{itemize}

\begin{comment}
However, users will lose the sense of embodiment to the virtual avatars if they notice the offset between the virtual and physical movements.
To address this, researchers have been exploring the noticing threshold of offsets with various magnitudes and proposing various redirection techniques that maintain the sense of embodiment~\cite{}.

However, when users embody virtual avatars to explore virtual environments, they encounter various visual effects and content that can attract their attention~\cite{}.
During this, the user may notice an offset when he observes the virtual movement carefully while ignoring it when the virtual contents attract his attention from the movements.
Therefore, static offset thresholds are not appropriate in dynamic scenarios.

Past research has proposed dynamic mapping techniques that adapted to users' state, such as hand moving speed~\cite{frees2007prism} or ergonomically comfortable poses~\cite{montano2017erg}, but not considering the influence of virtual content.
More specifically, PRISM~\cite{frees2007prism} proposed adjusting the C/D ratio with a non-linear mapping according to users' hand moving speed, but it might not be optimal for various virtual scenarios.
While Erg-O~\cite{montano2017erg} redirected users' virtual hands according to the virtual target's relative position to reduce physical fatigue, neglecting the change of virtual environments. 

Therefore, how to design redirection techniques in various scenarios with different visual attractions remains unknown.
To address this, we investigate how visual attention affects the noticing probability of movement offsets.
Based on our experiments, we implement a computational model that automatically computes the noticing probability of offsets under certain visual attractions.
VR application designers and developers can easily leverage our model to design redirection techniques maintaining the sense of embodiment adapt to the user's visual attention.
We implement a dynamic redirection technique with our model and demonstrate that it effectively reduces the target reaching time without reducing the sense of embodiment compared to static redirection techniques.

% Need to be refined
This paper offers the following contributions.
\begin{itemize}
    \item We investigate how visual attractions affect the noticing probability of redirection offsets.
    \item We construct a computational model to predict the noticing probability of an offset with a given visual background.
    \item We implement a dynamic redirection technique adapting to the visual background. We evaluate the technique and develop three applications to demonstrate the benefits. 
\end{itemize}



First, we conducted a controlled experiment to understand how users perceived the movement offset while subjected to various distractions.
Since hand redirection is one of the most frequently used redirections in VR interactions, we focused on the dynamic arm movements and manually added angular offsets to the' elbow joint~\cite{li2022modeling, gonzalez2022model, zenner2019estimating}. 
We employed flashing spheres in the user's field of view as distractions to attract users' visual attention.
Participants were instructed to report the appearing location of the spheres while simultaneously performing the arm movements and reporting if they perceived an offset during the movement. 
(\zhipeng{Add the results of data collection. Analyze the influence of the distance between the gaze map and the offset.}
We measured the visual attraction's magnitude with the gaze distribution on it.
Results showed that stronger distractions made it harder for users to notice the offset.)
\zhipeng{Need to rewrite. Not sure to use gaze distribution or a metric obtained from the visual content.}
Secondly, we constructed a computational model to predict the noticing probability of offsets with given visual content.
We analyzed the data from the user studies to measure the influence of visual attractions on the noticing probability of offsets.
We built a statistical model to predict the offset's noticing probability with a given visual content.
Based on the model, we implement a dynamic redirection technique to adjust the redirection offset adapted to the user's current field of view.
We evaluated the technique in a target selection task compared to no hand redirection and static hand redirection.
\zhipeng{Add the results of the evaluation.}
Results showed that the dynamic hand redirection technique significantly reduced the target selection time with similar accuracy and a comparable sense of embodiment.
Finally, we implemented three applications to demonstrate the potential benefits of the visual attention adapted dynamic redirection technique.
\end{comment}

% This one modifies arm length, not redirection
% \citeauthor{mcintosh2020iteratively} proposed an adaptation method to iteratively change the virtual avatar arm's length based on the primary tasks' performance~\cite{mcintosh2020iteratively}.



% \zhipeng{TO ADD: what is redirection}
% Redirection enables novel interactions in Virtual Reality, including redirected walking, haptic redirection, and pseudo haptics by introducing an offset to users' movement.
% \zhipeng{TO ADD: extend this sentence}
% The price of this is that users' immersiveness and embodiment in VR can be compromised when they notice the offset and perceive the virtual movement not as theirs~\cite{}.
% \zhipeng{TO ADD: extend this sentence, elaborate how the virtual environment attracts users' attention}
% Meanwhile, the visual content in the virtual environment is abundant and consistently captures users' attention, making it harder to notice the offset~\cite{}.
% While previous studies explored the noticing threshold of the offsets and optimized the redirection techniques to maintain the sense of embodiment~\cite{}, the influence of visual content on the probability of perceiving offsets remains unknown.  
% Therefore, we propose to investigate how users perceive the redirection offset when they are facing various visual attractions.


% We conducted a user study to understand how users notice the shift with visual attractions.
% We used a color-changing ball to attract the user's attention while instructing users to perform different poses with their arms and observe it meanwhile.
% \zhipeng{(Which one should be the primary task? Observe the ball should be the primary one, but if the primary task is too simple, users might allocate more attention on the secondary task and this makes the secondary task primary.)}
% \zhipeng{(We need a good and reasonable dual-task design in which users care about both their pose and the visual content, at least in the evaluation study. And we need to be able to control the visual content's magnitude and saliency maybe?)}
% We controlled the shift magnitude and direction, the user's pose, the ball's size, and the color range.
% We set the ball's color-changing interval as the independent factor.
% We collect the user's response to each shift and the color-changing times.
% Based on the collected data, we constructed a statistical model to describe the influence of visual attraction on the noticing probability.
% \zhipeng{(Are we actually controlling the attention allocation? How do we measure the attracting effect? We need uniform metrics, otherwise it is also hard for others to use our knowledge.)}
% \zhipeng{(Try to use eye gaze? The eye gaze distribution in the last five seconds to decide the attention allocation? Basically constructing a model with eye gaze distribution and noticing probability. But the user's head is moving, so the eye gaze distribution is not aligned well with the current view.)}

% \zhipeng{Saliency and EMD}
% \zhipeng{Gaze is more than just a point: Rethinking visual attention
% analysis using peripheral vision-based gaze mapping}

% Evaluation study(ideal case): based on the visual content, adjusting the redirection magnitude dynamically.

% \zhipeng{(The risk is our model's effect is trivial.)}

% Applications:
% Playing Lego while watching demo videos, we can accelerate the reaching process of bricks, and forbid the redirection during the manipulation.

% Beat saber again: but not make a lot of sense? Difficult game has complicated visual effects, while allows larger shift, but do not need large shift with high difficulty



\section{Methods: Models, Human Evaluations and NLP Benchmarks}
\label{sec:methods}

We briefly outline our methodology here; for additional information, please see Appendix \ref{app:sec:experimental_methodology}.

\begin{figure*}[t!]
    \centering
    % \includegraphics[width=0.55\textwidth]{figures/correlations/correlation_by_human_eval_area_split_corrmethod.pdf}%
    \includegraphics[width=0.9\textwidth]{figures/correlations/correlation_by_human_eval_area_cat_split_corrmethod.pdf}
    \caption{\textbf{Distributions of Correlations between Human Evaluations and NLP benchmarks.} Macroscopically, for each human evaluation area, Chat LM scores are typically highly correlated with NLP benchmarks. Mesoscopically, human and NLP benchmarks remain positively correlated, with notable exceptions: Adversarial Dishonesty, Adversarial Harmfulness and Safety are anticorrelated with most NLP benchmarks, and Language Assistance and Open QA are uncorrelated.}
    \label{fig:corr:human_eval_area}
\end{figure*}

\textbf{Models.} We evaluated four Chat Llama 2 models with 7, 13, 34, and 70 billion parameters pre-trained on 2 trillion tokens and finetuned using supervised finetuning \citep{sanh2021multitask, chung2022scaling, longpre2023flan} and reinforcement learning from human feedback \citep{christiano2017deep, ziegler2019fine, stiennon2020learning}.
We chose the Llama 2 family because, at the time we collected our data, it contained leading open-access chat-finetuned models spanning multiple scales with minimal variations, ensuring a consistent foundation for our investigations.

\textbf{Human Evaluations: Single Turn \& Multi-Turn.} In this work, our aim was specifically to identify which NLP benchmark scores are predictive of human preferences on open-ended prompts representative of real-world chat model usage. We chose this approach to maximize the ecological validity and generalizability of the findings to real-world use cases. For a concrete example, we may want our chat language models (LMs) to excel at providing bespoke career advice; which NLP benchmarks provide useful signals for whether models are improving at such tasks? 



To answer such questions, we created a taxonomy of single-turn and multi-turn interactions (Fig. \ref{fig:human_eval_prompt_taxonomy}) between chat LMs and humans. For single-turn interactions, we generated a diverse set of prompts spanning common areas of interest: Factual Questions, Procedural Questions, Language Assistance, Writing \& Content Creation, Dialogue, Code, Reasoning, Recommendations / Brainstorming and Safety, with nested categories and subcategories. 
For multi-turn prompts, non-annotator humans were asked to have conversations (3 to 15 turns long) with all models on similar topics of interest: Factual Questions, Procedural Questions, Language Assistance, Writing \& Content Creation, Summarization \& Editing, General Dialogue, Reasoning and Recommendations / Brainstorming. This taxonomy was chosen to broadly cover common use-cases of Chat LMs.
Example prompts include: ``What is the tallest mountain in the world?" (Factual Question); ``How do I make minestrone soup?" (Procedural Question); ``Please make this sentence more friendly: I need you to stop parking in my space" (Language Assistance). 
% ``Write me a poem about getting to the weekend after a long day at work" (Writing \& Content Creation).
See Appendix \ref{app:sec:experimental_methodology:human_evals} for more information.

We paid human annotators to evaluate each of the four Chat Llama 2 models against ChatGPT 3.5 \citep{ouyang2022training} (gpt-3.5-0301) on a dataset of single-turn and multi-turn prompts (Fig \ref{fig:human_eval_prompt_taxonomy}). We chose gpt-3.5-0301 because, at the time this data was collected, gpt-3.5-0301 was a good balance of three desirable properties for our study: performant, cheap, and stable.
For each pair of conversations (one conversation with Chat Llama responses and the other with ChatGPT responses), at least three unique human annotators independently indicated which conversation was preferred using a \citet{likert1932technique} scale from 1 to 7, where 1 denotes the Chat Llama model was strongly preferred and 7 denotes gpt-3.5-0301 was strongly preferred.
Across the 11291 single-turn samples and 2081 multi-turn samples, we had at least 3 unique human annotators evaluate each per pairwise comparison, with 2104 unique annotators overall. 
We averaged the annotators' scores for each pairwise comparison to give us an average human evaluation score per datum.



\textbf{Natural Language Processing (NLP) Benchmarks.} We evaluated the four Chat Llama 2 models on large-scale and commonly-used NLP benchmarks: AGI Eval \citep{zhong2023agieval}, AI2 Reasoning Challenge (ARC; both Easy and Hard) \citep{clark2018arc}, BIG Bench Hard \citep{srivastava2022beyond,suzgun2022challenging} BoolQ \citep{clark2019boolq}, CommonSenseQA \citep{talmor2019commonsenseqa}, COPA \citep{roemmele2011choice}, DROP \citep{dua2019drop}, GSM8K \citep{cobbe2021training}, HellaSwag \citep{zellers2019hellaswag}, HumanEval \citep{chen2021evaluatinglargelanguagemodels}, InverseScaling \citep{mckenzie2022inverse,mckenzie2022round1,mckenzie2022round2}, MBPP \citep{austin2021program}, MMLU \citep{hendrycks2020measuring}, Natural Questions \citep{kwiatkowski2019naturalquestions}, OpenbookQA \citep{mihaylov2018openbookqa}, PIQA \citep{bisk2020piqa}, QuAC \citep{choi2018quac}, RACE \citep{lai2017race}, SIQA \citep{sap2019social}, SQUAD \citep{rajpurkar2016squad}, TLDR \citep{volske2017tl}, TriviaQA \citep{joshi2017triviaqa}, WinoGrande \citep{sakaguchi2021winogrande} and XSum \citep{narayan2018xsum}. Some of these benchmarks (e.g., MMLU) contain subsets (e.g., Jurisprudence) that we do not aggregate over.
These tasks cover commonsense reasoning, world knowledge, reading comprehension, coding and more. We used standard evaluation processes for all academic benchmarks including prompt formatting, metrics, 0-shot/few-shot, etc.
This structured approach facilitates an exhaustive examination of model performances across varied metrics.
For more information, see Appendix \ref{app:sec:experimental_methodology:nlp_benchmarks}.

\begin{figure*}[t!]
    \centering
    \includegraphics[width=\textwidth]{figures/correlations/academic_benchmark_subset_by_pearson_avg_over_human.pdf}
    \caption{\textbf{NLP Benchmarks Ranked by Average Pearson Correlation over All Human Evaluations.} Certain benchmarks have higher correlations with human evaluations, including a subset of MMLU, a subset of BIG Bench Hard, HellaSwag, ARC, RACE, PIQA, NaturalQuestions, QuAC, and CommonSenseQA. Other benchmarks were weakly or uncorrelated with human evaluations: ETHOS, Kth Sentence, Inverse Scaling (with the exception of Resisting Correction Classification), OpenBookQA, COPA, SciBench and SIQA.}\label{fig:corr:academic_vs_correlation_split_corrmethod}
\end{figure*}

\textbf{Scores for Subsequent Analyses.} For each dataset and evaluation process, we average each model's scores across all samples, yielding two matrices of scores:
%
$$X_{\text{NLP}} \in \mathbb{R}^{160 \times 4} \quad \quad \quad \quad X_{\text{Human}} \in \mathbb{R}^{55 \times 4}$$
%
Here, $4$ is the number of models, $160$ is the number of NLP benchmarks per model and $55$ is the number of human evaluation area-category-subcategory scores per model. We subsequently study the correlations between $X_{\text{NLP}}$ and $X_{\text{Human}}$, then test how well $X_{\text{NLP}}$ can predict $X_{\text{Human}}$.





\section{Correlating Human Evaluations with NLP Benchmarks}
\label{sec:correlations}

We began by computing correlations between human evaluations and NLP benchmarks over the 4 average scores per model with three standard correlations  --- Pearson \citep{galton1877heredity}, Spearman \citep{spearman1904proof} and Kendall \citep{kendall1938new} --- giving us three correlation matrices of shape $160 \times 55$ between every pair of NLP benchmark and human evaluation area-category-subcategory (Fig. \ref{fig:corr:microscopic}).
% Pearson correlation measures the linear relationship between two continuous variables, whereas Spearman and Kendall correlations assess the monotonic relationship between two variables; Spearman correlation is based on the rank order of the data points, whereas Kendall correlation is determined by the number of concordant and discordant pairs.
By using different correlation metrics, we aim to robustly characterize the relationships between human and NLP benchmarks.

Macroscopically, at the most coarse grouping of human evaluations in our taxonomy (i.e., areas) (Fig.~\ref{fig:human_eval_prompt_taxonomy}), we found that average NLP benchmark scores are highly correlated with average human scores for all human evaluation areas under all three correlation metrics (Fig. \ref{fig:corr:human_eval_area} top).
% Due to the small number of models ($N=4$), Spearman and Kendall correlations suffer discretization effects (Fig. \ref{app:fig:correlation_couplings}), inducing an illusion of undulations.
These strong correlations suggest that, at a high level, NLP benchmarks are reasonable proxies for human judgments of LM quality.

Mesoscopically, at the level of human evaluation areas and categories, NLP benchmarks remain highly correlated with human evaluations, with two notable types of exceptions (Fig. \ref{fig:corr:human_eval_area}). First, Adversarial Dishonesty, Adversarial Harmfulness, and Safety are anti-correlated with most NLP benchmarks, potentially indicating that these adversarial and safety-focused categories are more easily transgressed by more capable LMs; an alternative hypothesis could be that safety benchmarks simply are not especially good, as demonstrated by \citet{ren2024safetywashing}. Second, Language Assistance and Open Question Answering are uncorrelated with most NLP benchmarks, suggesting that these categories may require new NLP benchmarks. Open Question Answering was surprising given that some of our NLP benchmarks are open question answering datasets, e.g., OpenBookQA \citep{mihaylov2018openbookqa}. We found the three correlations metrics visually agreed with one another and were themselves tightly coupled (App. Fig. \ref{app:fig:correlation_couplings}), and so we present only one (Pearson) moving forward, with equivalent plots of the other two (Spearman, Kendall) deferred to the appendix.



\subsection{Which human evaluations have few-to-no correlated NLP benchmarks?}
\label{sec:correlations:subsec:nlp_benchmarks_no_correlations}

To the best of our ability to discern, none. Every human evaluation seemed to have at least some NLP benchmarks that were either correlated or anticorrelated with it. This result is promising because it suggests human evaluations might be predictable from NLP benchmarks (Sec.~\ref{sec:predictions}).  

\subsection{Which NLP benchmarks exhibit high correlations with human evaluations?}
\label{sec:correlations:subsec:nlp_benchmarks_highest_correlations}

\begin{figure*}[t!]
    \centering
    \includegraphics[width=\linewidth]{figures/correlations/human_eval_area_cat_vs_academic_benchmark_subset_by_correlation_method=pearson}
    \caption{\textbf{Pearson Correlations Between Human Evaluations and NLP Benchmarks.} Rows: Human evaluation areas-categories-subcategories. Columns: NLP benchmarks. The heatmap is row-wrapped to fit on the page. \textcolor{red}{Large positive correlations (+1) are shown in red.} \textcolor{blue}{Large negative anticorrelations (-1) are shown in blue.} Low uncorrelations ($\sim$0) are shown in light-white-gray.}
    \label{fig:corr:microscopic}
\end{figure*}

\begin{figure*}[t!]
    \centering
    \includegraphics[width=0.9\textwidth]{figures/correlations/2D_biplot_correlation=Pearson.pdf}
    \caption{\textbf{Matrix Decomposition of Pairwise Pearson Correlations Between Human Evaluations and NLP Benchmarks.} The correlation matrix has 3 non-zero singular values (App. Fig. \ref{app:fig:academic_human_singular_value_spectra}). Bottom: \textcolor{Magenta}{Human evaluations} and \textcolor{Green}{NLP benchmarks} are plotted projected along the (dimension-scaled) first two singular modes of the Pearson correlation matrix. The bulk of evaluations live in one community (left), with smaller communities (top, bottom, right); for an in-depth interpretation, see Sec. \ref{sec:correlations:subsec:community_detection}.}
    \label{fig:corr:singular_modes}
\end{figure*}


To answer this question, we ordered NLP benchmarks based on their average correlation score with all human evaluation areas, categories and subcategories.
We found many NLP benchmarks have high average correlation with human evaluations (Fig. \ref{fig:corr:academic_vs_correlation_split_corrmethod}); the highest average correlation NLP benchmarks include a subset of MMLU (Nutrition, Human Aging, Sociology, Public Relations, Moral Scenarios, College Computer Science), a subset of BIG Bench Hard (Word Sorting, Reasoning About Colored Objects, Logical Deduction), HellaSwag, ARC, RACE, PIQA, NaturalQuestions, QuAC, CommonSenseQA, DROP and TriviaQA. Other benchmarks were less correlated or uncorrelated with human evaluations: ETHOS, Kth Sentence, Inverse Scaling (excluding Resisting Correction Classification), OpenBookQA, COPA, SciBench (excluding Fundamentals of Physics) and SIQA.
Upon investigating, some of the most highly correlated NLP benchmarks make sense. For instance, Inverse Scaling's Resisting Correction Classification ranked second highest for being correlated with human evaluations, and the task measures a highly desirable capability for human users: the LM's ability to follow user instructions that run counter to the LM's natural inclinations.


\subsection{Matrix Decomposition of Correlations between Human Evaluations and NLP Benchmarks} \label{sec:correlations:subsec:community_detection}


To gain a structured view of how human evaluations and NLP benchmarks interrelate, we analyze the $160 \times 55$ Pearson correlation matrix between them by computing the singular value decomposition of the correlation matrix $C = U \Sigma V^T$, where $U \in \mathbb{R}^{160 \times 3}$ and $V \in \mathbb{R}^{55 \times 3}$ are the left and right singular vectors. 
We observe only three non-zero singular values (Appendix Figure~\ref{app:fig:academic_human_singular_value_spectra}), indicating that most of the variance in correlations is captured by three underlying dimensions.
Interpreting each row of $U$ and $V$ as coordinates in the corresponding low-dimensional space provides a way to visualize how tasks and human evaluations group together based on their similarity in correlation patterns, as shown by plotting each NLP benchmark (\textcolor{Green}{green}) and each human evaluation category (\textcolor{Magenta}{magenta}) in the 2D space spanned by the first two singular vectors (scaled by their singular values). (Fig.~\ref{fig:corr:singular_modes}).
% Each correlation is computed across the four Chat Llama 2 models. Consequently, the matrix can have rank at most four. 



\textbf{Overall Alignment (Left Cluster).} The largest group of points, combining both \textcolor{Magenta}{human evaluations} and \textcolor{Green}{benchmarks}, sits on the left side. This indicates that many standard NLP benchmarks (e.g., language understanding, commonsense reasoning, factual QA) tend to move in tandem with broad human-judged performance (e.g., correctness, clarity) across our four models.

\textbf{Dialogue and Context-Oriented Tasks (Top-Left).} Certain dialogue-related human evaluations (e.g., \textcolor{Magenta}{Language Assistance, Coding Assistance}) appear near several benchmarks that emphasize context or discourse (e.g., \textcolor{Green}{OpenBookQA}, \textcolor{Green}{Kth Sentence}). Their proximity suggests a shared correlation pattern across the models, possibly reflecting reliance on social reasoning and contextual cues.

\textbf{Adversarial and Safety-Focused Tasks (Right \& Bottom-Right).} Evaluations tied to \textcolor{Magenta}{Adversarial Harmfulness/Dishonesty} and \textcolor{Magenta}{Safety} show distinct positions, often near benchmarks aimed at exposing errors or biases (e.g., \textcolor{Green}{Inverse Scaling tasks}, \textcolor{Green}{ETHOS} for hate speech). This segregation indicates that safety/adversarial capabilities differ in how they correlate with more conventional tasks.

\textbf{Open QA \& Domain Knowledge (Lower-Left).} Finally, \textcolor{Magenta}{Open QA} and some \textcolor{Magenta}{Writing} evaluations lie closer to benchmarks demanding specialized knowledge (\textcolor{Green}{MMLU.Electrical Engineering}, \textcolor{Green}{SciBench.Quantum Chemistry}), suggesting that open-ended user queries may align more with advanced domain-knowledge benchmarks than with simpler tasks.

Overall, this matrix decomposition shows that while there is a dominant “general ability” factor (represented by the first singular value) that aligns most tasks and evaluations, additional singular vectors capture subtler patterns. These include the safety/adversarial dimension and context-dependent or domain-specialized dimensions. Consequently, the correlations between human evaluations and NLP benchmarks exhibit a rich low-rank structure indicative of multiple underlying performance factors.

\begin{figure*}[t!]
    \centering
    \includegraphics[width=0.9\textwidth]{figures/regressions/pred_vs_true_human_scores_by_human_eval_split_by_left_out_model.pdf}
    \caption{\textbf{Leave-One-Out Cross Validated Linear Regression Predictions of Human Evaluations.} Linear regressions accurately predict human evaluation scores from all NLP benchmark scores. Each subfigure shows predicted human evaluation scores against actual human evaluation scores on each of the four left-out Chat Llama 2 models colored by human evaluation area, category and subcategory.}
    \label{fig:reg:overparameterized_regressions_leave_one_out}
\end{figure*}
\section{Predicting Human Evaluations from NLP Benchmarks}
\label{sec:predictions}

Having established the existence of correlations between human evaluations and NLP benchmarks, we investigated the feasibility of predicting human evaluations from NLP benchmarks. Our goal is to build predictive models that accurately predict a language model's average human evaluation scores per areas and categories using the model's average scores on NLP benchmarks and tasks. However, we faced a significant challenge due to the overparameterized nature of our data: for each target human evaluation area or category, there are approximately 150 covariates (NLP benchmarks and tasks) but only four models.


\textbf{Predictive Modeling: Overparameterized Linear Regressions.}
To predict human evaluations from NLP benchmarks, we used overparameterized linear regression. In general, overparameterized linear regression is known to be capable of generalizing (App. Sec. \ref{app:sec:generalization_of_overparameterized_models}), although whether linear models would generalize in this setting was an empirical question. 
For each human evaluation area and category, we fit a linear model to predict a language model's average human evaluation score from its average scores on all NLP benchmarks and tasks. To assess the predictive accuracy of these overparameterized models, we employed leave-one-out cross validation: we fit four separate linear models, each time fitting on three of the chat LMS' scores and holding out the fourth to test the performance of the linear model. This approach allows us to estimate the models' performance on unseen data, albeit with limitations due to the small sample size.
Before fitting the models, we normalized all human evaluation scores to lie in $[0, 1]$ rather than $[-7, -1]$ (recalling that higher scores indicate the human evaluator prefers the Chat Llama 2 model compared to GPT-3.5). 

\textbf{Results.} Across human evaluation areas and categories, we found that the linear models' predicted average human evaluation scores generally align well with the actual average human evaluation scores, as evidenced by most points falling close to the identity line in the predicted score vs. actual score plane (Fig. \ref{fig:reg:overparameterized_regressions_leave_one_out}). This suggests that, despite the overparameterization, the linear models can capture meaningful relationships between NLP benchmarks and human evaluations. However, we caution against over-interpreting these results, as the small sample size and the assumption of linearity may limit the generalizability of these findings to other language models or evaluation settings.

To gain insight into which NLP benchmarks are most informative for predicting human evaluation scores, we examine the learned weights of the linear models (Fig. \ref{app:fig:linear_regression_coefficients}). NLP benchmarks with consistently high absolute weights across different human evaluation areas and categories are likely to be more predictive of human judgments. However, due to the overparameterized nature of the models, we refrain from drawing strong conclusions about the relative importance of individual benchmarks and instead focus on the overall predictive performance.
These results suggest that scaling up the number of chat LMs and human evaluation data could unlock highly predictive models of slow, noisy and expensive but valuable human evaluations using fast, precise and cheaper NLP benchmarks.
\section{Discussion and Conclusion}
\label{sec:discussion}


\textbf{Conclusion.} In this paper, we propose LRM to utilize diffusion models for step-level reward modeling, based on the insights that diffusion models possess text-image alignment abilities and can perceive noisy latent images across different timesteps. To facilitate the training of LRM, the MPCF strategy is introduced to address the inconsistent preference issue in LRM's training data. We further propose LPO, a method that employs LRM for step-level preference optimization, operating entirely within the latent space. LPO not only significantly reduces training time but also delivers remarkable performance improvements across various evaluation dimensions, highlighting the effectiveness of employing the diffusion model itself to guide its preference optimization. We hope our findings can open new avenues for research in preference optimization for diffusion models and contribute to advancing the field of visual generation.

\textbf{Limitations and Future Work.} (1) The experiments in this work are conducted on UNet-based models and the DDPM scheduling method. Further research is needed to adapt these findings to larger DiT-based models \cite{sd3} and flow matching methods \cite{flow_match}. (2) The Pick-a-Pic dataset mainly contains images generated by SD1.5 and SDXL, which generally exhibit low image quality. Introducing higher-quality images is expected to enhance the generalization of the LRM. (3) As a step-level reward model, the LRM can be easily applied to reward fine-tuning methods \cite{alignprop, draft}, avoiding lengthy inference chain backpropagation and significantly accelerating the training speed. (4) The LRM can also extend the best-of-N approach to a step-level version, enabling exploration and selection at each step of image generation, thereby achieving inference-time optimization similar to GPT-o1 \cite{gpt_o1}.


\clearpage

\bibliography{references_rylan}
\bibliographystyle{icml2025}


%%%%%%%%%%%%%%%%%%%%%%%%%%%%%%%%%%%%%%%%%%%%%%%%%%%%%%%%%%%%%%%%%%%%%%%%%%%%%%%
%%%%%%%%%%%%%%%%%%%%%%%%%%%%%%%%%%%%%%%%%%%%%%%%%%%%%%%%%%%%%%%%%%%%%%%%%%%%%%%
% APPENDIX
%%%%%%%%%%%%%%%%%%%%%%%%%%%%%%%%%%%%%%%%%%%%%%%%%%%%%%%%%%%%%%%%%%%%%%%%%%%%%%%
%%%%%%%%%%%%%%%%%%%%%%%%%%%%%%%%%%%%%%%%%%%%%%%%%%%%%%%%%%%%%%%%%%%%%%%%%%%%%%%
\clearpage
\appendix
\onecolumn


% \section{Proposed model}
% \label{section:app:model}
% Table \ref{table:define} lists the symbols and their definitions used in this paper. \par
% % \vspace{-1.5em}
% % \TSK{
% % \begin{table}[t]
\vspace{-0.5em}
\centering
\small
% \footnotesize
\caption{Symbols and definitions.}
\label{table:define}
\vspace{-1.2em}
\begin{tabular}{l|l}
\toprule
Symbol & Definition \\
\midrule
$d$ & Number of dimensions \\
$t_c$ & Current time point \\
% $N$ & Current window length \\
$\mX$ & Co-evolving multivariate data stream (semi-infinite) \\
$\mX^c$ & Current window, i.e., $\mX^c = \mX[t_m:t_c]\in\R^{d\times N}$ \\
\midrule
$h$ & Embedding dimension \\
% $\mH$ & Hankel matrix, i.e., $\begin{bmatrix}
%             \embed{\vx_1} & \embed{\vx_2} & \cdots & \embed{\vx_{n-h+1}}
%         \end{bmatrix}$ \\
$\embed{\cdot}$ & Observable for time-delay embedding, i.e., $g\colon\R \rightarrow \R^{h}$ \\
% $\mH$ & Hankel matrix of $\mX$, i.e., $\mH = [\embed{\vx_1}~\embed{\vx_2} ~ ... ~ \embed{\vx_{n-h+1}}]$ \\
$\mH$ & Hankel matrix \\
$\nmodes$ & Number of modes \\
% $\imode$ & Modes of the system for $i$-th dimension of $\mX$, i.e., $\imode \in \R^{h \times r_i}$ \\
$\modes$ & Modes of the system, i.e., $\modes \in \R^{h\times\nmodes}$ \\
% $\ieig$ & Eigenvalues of the system for $i$-th dimension of $\mX$, $i.e., \ieig \in \R^{r_i \times r_i}$ \\
$\eigs$ & Eigenvalues of the system, i.e., $\eigs \in \R^{\nmodes\times\nmodes}$ \\
$\demixing$ & Demixing matrix, i.e., $\mW = [\rowvect{w}_1, ..., \rowvect{w}_d]^\top \in \R^{d \times d}$ \\
$\mB$ & Causal adjacency matrix, i.e., $\mB \in \R^{d \times d}$ \\
\midrule
% $\vs(t)$ & Latent variables at time point $t$, i.e., $\vs(t) = \{ \vs_1(t), ..., \vs_d(t) \} $ \\
$\ind(t)$ & Inherent signal at time point $t$, i.e., $\ind(t) = \{ \ith{e}(t) \}_{i=1}^d$ \\
$\mat{S}(t)$ & Latent vectors at time point $t$, i.e., $\mat{S}(t) = \{ 
\ith{\vs}(t) \}_{i=1}^d$ \\
$\vvec(t)$ & Estimated vector at time point $t$, i.e., $\vvec(t) = \{ \ith{v}(t) \}_{i=1}^d$ \\
\midrule
$\mathcal{D}$ & Self-dynamics factor set, i.e., $\mathcal{D} = \{\modes, \eigs\}$\\
$\regime$ & Regime parameter set, i.e., $\regime = \{ \mW, \mathcal{D}_{(1)}, ..., \mathcal{D}_{(d)} \}$\\
\midrule
$R$ & Number of regimes \\
$\regimeset$ & Regime set, i.e., $\regimeset 
 = \{ \regime^1, ..., \regime^R \}$\\
 $\mathcal{B}$ & \Relation, i.e., $\mathcal{B} = \{\mB^1, ..., \mB^R\}$\\
$\updateset$ & Update parameter set,  i.e., $\updateset 
 = \{ \update^1, ..., \update^R \}$ \\
\midrule
 $\modelparam$ & Full parameter set,  i.e., $\modelparam 
 = \{ \regimeset, \updateset \}$ \\

\bottomrule
% \midrule
\end{tabular}
\normalsize
% \vspace{-2.0em}
\vspace{1.0em}
\end{table}

% % }
% \vspace{0.6em}

\section{Optimization Algorithm}
% Algorithm \ref{alg:model} is the overall procedure of \method. Algorithm \ref{alg:estimator}, namely, \modelestimator continuously updates the full parameter set $\modelparam$ and the model candidate $\candparam$, which describes the current window $\mX^c$.
\TSK{
\begin{figure}[!h]
\vspace{-5.0ex}
\begin{algorithm}[H]
    \normalsize
    \caption{\method($\vx(t_c), \modelparam, \candparam$)}
    \label{alg:model}
    \begin{algorithmic}[1]
        \STATE {\bf Input:}
        \hspace{0mm}    (a) New value $\vx(t_c)$ at time point $t_c$ \\
        \hspace{9.5mm} (b) Full parameter set $\modelparam = \{\regimeset, \updateset\}$ \\
        \hspace{9.68mm} (c) Model candidate $\candparam = \{\regime^c, \update^c, \bm{s}^c_{en}\}$
        \STATE {\bf Output:}
        \hspace{0mm}    (a) Updated full parameter set $\modelparam'$ \\
        \hspace{11.8mm} (b) Updated model candidate $\candparam'$ \\
        \hspace{11.9mm} (c) $l_s$-steps-ahead future value $\vect{v}(t_c+l_s)$ \\
        \hspace{11.8mm} (d) Causal adjacency matrix $\mB$
        \STATE /* Update current window $\mX^c$ */
        \STATE $\mX^c \leftarrow \mX[t_m : t_c]$
        \STATE /* Estimate optimal regime $\regime$ */
        \STATE $\{\modelparam', \candparam'\} \leftarrow$ \modelestimator($\mX^c, \modelparam$, $\candparam$)
        \STATE /* Forecast future value and discover causal relationship */
        \STATE $\{\vect{v}(t_c+l_s),~\mB\} \leftarrow$ \modelgenerator($\candparam'$)
        \STATE /* Update regime $\regime$ */
        \IF{NOT create new regime}
            \STATE $\candparam' \leftarrow \regimeupdate(\mX^c, \candparam')$
        \ENDIF
    \RETURN $\{\modelparam', \candparam', \vect{v}(t_c+l_s), \mB\}$
    \end{algorithmic}
\end{algorithm}
\vspace{-4.5em}
\end{figure}
}\par
\TSK{
\begin{figure}[!h]
\vspace{-0.0ex}
\begin{algorithm}[H]
    \normalsize
    \caption{\modelestimator($\mX^c, \modelparam, \candparam$)}
    \label{alg:estimator}
    \begin{algorithmic}[1]
        \STATE {\bf Input:}
        \hspace{0.0mm}  (a) Current window $\mX^c$ \\
        \hspace{9.5mm} (b) Full parameter set $\modelparam$ \\
        \hspace{9.68mm} (c) Model candidate $\candparam$
        \STATE {\bf Output:}
        \hspace{0.0mm}  (a) Updated full parameter set $\modelparam'$ \\
        \hspace{11.8mm} (b) Updated model candidate $\candparam'$
        \STATE /* Calculate optimal initial conditions */
        \STATE $\mat{S}_{0}^c \leftarrow \argmin_{\mat{S}_{0}^c} f(\mX^c; \mat{S}_{0}^c, \regime^c)$
        \IF{$f(\mX^c; \mat{S}_{0}^c, \regime^c) > \tau$}
            \STATE /* Find better regime in $\bm{\Theta}$ */
            \STATE $\{ \mat{S}_{0}^c, \regime^c \} \leftarrow \argmin_{\mat{S}_{0}^c, \regime \in \regimeset} \,f(\mX^c; \mat{S}_{0}^c, \regime^c)$
            \IF{$f(\mX^c; \mat{S}_{0}^c, \regime^c) > \tau$}
                \STATE /* Create new regime */
                \STATE $\{ \regime^c, \update^c \} \leftarrow \textsc{RegimeCreation}(\mX^c)$
                \STATE $\regimeset \leftarrow \regimeset \cup \regime^c$; $\updateset \leftarrow \updateset \cup \update^c$
            \ENDIF
        \ENDIF
        \STATE $\modelparam' \leftarrow \{\regimeset, \updateset\}$; $\candparam' \leftarrow \{\regime^c, \update^c, \mat{S}_{en}^c\}$
        \RETURN $\modelparam', \candparam'$
    \end{algorithmic}
\end{algorithm}
\vspace{-3.3em}
\end{figure}
}
% Algorithm \ref{alg:model} shows the overall procedure for \method,
% including \modelestimator (Algorithm \ref{alg:estimator}).
% \modelestimator continuously updates the full parameter set $\modelparam$ and
% the model candidate $\candparam$, which describes the current window $\mX^c$.
% \par
\label{section:app:algorithm}
% \myparaitemize{Details in Eq. \eqref{eq:update_trans}}
\subsection{Details of Eq. (\ref{eq:update_trans})}
% \myparaitemize{Details of Eq. (\ref{eq:update_trans})}
Here, we introduce the recurrence relation of transition matrix $\ith{\trans}$.
As mentioned earlier, we use the following cost function (below, index $i$ denoting $i$-th dimension is omitted for the sake of simplicity, e.g., we write $\ith{\trans}$ as $\trans$):
\begin{align*}
    \mathcal{E} &= \sum_{t'=t_m+h}^{t_c}\forgetting^{t_c-t'}||\embed{e(t')} - \trans\embed{e(t'-1)}||_2^2 \\
    &= \sum_{l=1}^h (\mat{L}(l, :) - \trans(l, :)\mat{R})\Forgetting(\mat{L}(l, :) - \trans(l, :)\mat{R})^\top
\end{align*}
where,
% $\Forgetting = diag(\forgetting^{N-2}, ..., \forgetting^0) \in \R^{(N-1) \times (N-1)}$ and
$\Forgetting, \mat{L}$ and $\mat{R}$ are synonymous with the definition in Section \ref{section:alg:creation}.
Because we want to obtain $\trans$ that minimizes this cost function $\mathcal{E}$, we differentiate it with respect to $\trans$.
\begin{align*}
    \dfrac{\partial}{\partial\trans(l, :)}\mathcal{E}
    &= -2(\mat{L}(l, :) - \trans(l, :)\mat{R}) \Forgetting \mat{R}^\top
\end{align*}
Solving the equation $\partial\mathcal{E}/\partial\trans(l, :) = 0$ for each $l$, $1 \leq l \leq h$,
the optimal solution for $\trans$ is given by
% the optimal solution is $ \trans = (\mat{L}\Forgetting\mat{R}^\top)(\mat{R}\Forgetting\mat{R}^\top)^{-1} $
$$ \trans = (\mat{L}\Forgetting\mat{R}^\top)(\mat{R}\Forgetting\mat{R}^\top)^{-1} $$
where we define
\begin{align*}
    \mat{Q} = \mat{L}\Forgetting\mat{R}^\top,\quad
    \mat{P} = (\mat{R}\Forgetting\mat{R}^\top)^{-1}
\end{align*}
The recurrence relations of $\mat{Q}$ can be written as
\begin{align*}
    \mat{Q} &= \sum_{t'=t_m+h}^{t_c}\forgetting^{t_c-t'}\embed{e(t')}\embed{e(t'-1)}^\top \\
    &= \forgetting\sum_{t'=t_m+h}^{t_c-1}\forgetting^{t_c-t'-1}\embed{e(t')}\embed{e(t'-1)}^\top + \embed{e(t_c)}\embed{e(t_c-1)}^\top
\end{align*}
\begin{align}
    \label{eq:Q}
    \therefore \mat{Q}^{new} = \forgetting\mat{Q}^{prev} + \embed{e(t_c-1)}\embed{e(t_c)}^\top
\end{align}
and similarly
\begin{align}
    \label{eq:bP}
    \mat{P}^{new} &= (\forgetting{(\mat{P}^{prev})}^{-1} + \embed{e(t_c)}\embed{e(t_c)}^\top)^{-1}
\end{align}Here, we apply the Sherman-Morrison formula~\cite{sherman1950adjustment} to the RHS of Eq. \eqref{eq:bP}.
Note that $\embed{e(t_c)}^\top\mat{P}^{prev}\embed{e(t_c)} > 0$
because $\mat{P}^{-1} = \mat{R}\Forgetting\mat{R}^\top$ is positive definite by definition.
\begin{align}
    \label{eq:P}
    \therefore \mat{P}^{new} = \frac{1}{\forgetting}(\mat{P}^{prev} - \frac{\mat{P}^{prev}\embed{e(t_c-1)}\embed{e(t_c-1)}^\top\mat{P}^{prev}}{\forgetting + \embed{e(t_c-1)}^\top\mat{P}^{prev}\embed{e(t_c-1)}})
\end{align}
Finally, combining Eq. \eqref{eq:Q} and Eq. \eqref{eq:P} gives the recurrence relations of $\trans$ for Eq. \eqref{eq:update_trans}.
\begin{align*}
    \begin{split}
        \trans^{new} &= \trans^{prev} + (\embed{e(t_c)} - \trans^{prev}\embed{e(t_c-1)}\boldsymbol\gamma \\
        \boldsymbol\gamma &= \frac{\embed{e(t_c-1)}^\top\mat{P}^{prev}}{\forgetting + \embed{e(t_c-1)}^\top\mat{P}^{prev}\embed{e(t_c-1)}}
    \end{split}
\end{align*}
    
% \end{enumerate}
\par
% \TSK{
% \begin{figure*}[t]
    \begin{tabular}{cccc}
      \hspace{-1.5em}
      \begin{minipage}[c]{0.24\linewidth}
        \centering
        % \vspace{1em}
        \includegraphics[width=\linewidth]{results/web/original1_ver1.0.pdf}
        \vspace{-2em} \\
        \hspace{1.5em}
        % (a-i) Snapshot
        % (a-i) $l_s$-steps-ahead future value forecasting
        (a-i) Original data $\mX^c$
        \label{fig:web:forecast}
      \end{minipage} &
      \hspace{-1.5em}
      \begin{minipage}[c]{0.24\linewidth}
        \centering
        \includegraphics[width=\linewidth]{results/web/latent1_ver1.2.pdf}
        \vspace{-2em} \\
        \hspace{1.5em}
        (a-ii) Inherent signals $\mE$
      \end{minipage} &
      \hspace{-1.5em}
      \begin{minipage}[c]{0.24\linewidth}
        \centering
        \includegraphics[width=0.8\linewidth]{results/web/causal1_ver1.0.pdf}
        % \vspace{-2em}
        \\
        % \hspace{2.0em}
        (a-iii) Causal relationship $\mB$
      \end{minipage} &
      \hspace{-1.5em}
      \begin{minipage}[c]{0.24\linewidth}
        \centering
        \includegraphics[width=0.95\linewidth]{results/web/mode1_ver1.0.pdf}
        % \vspace{-2em}
        \\
        \hspace{-0.7em}
        (a-iv) Latent dynamics $\eigs$
      \end{minipage} \vspace{0.5em} \\
      % \caption{(a) Snapshot at the current time point $t_c = 208$}
      \multicolumn{4}{c}{\textbf{(a) Snapshots at current time point $t_c=208$.}}
      \vspace{0.5em} \\
      \hspace{-1.5em}
      \begin{minipage}[c]{0.24\linewidth}
        \centering
        % \vspace{1em}
        \includegraphics[width=\linewidth]{results/web/original2_ver1.0.pdf}
        \vspace{-2em} \\
        \hspace{1.5em}
        % (a-i) Snapshot
        % (b-i) $l_s$-steps-ahead future value forecasting
        (b-i) Original data $\mX^c$
      \end{minipage} &
      \hspace{-1.5em}
      \begin{minipage}[c]{0.24\linewidth}
        \centering
        \includegraphics[width=\linewidth]{results/web/latent2_ver1.2.pdf}
        \vspace{-2em} \\
        \hspace{1.5em}
        (b-ii) Inherent signals $\mE$
      \end{minipage} &
      \hspace{-1.5em}
      \begin{minipage}[c]{0.24\linewidth}
        \centering
        \includegraphics[width=0.8\linewidth]{results/web/causal2_ver1.0.pdf}
        % \vspace{-2em}
        \\
        % \hspace{2.0em}
        (b-iii) Causal relationship $\mB$
      \end{minipage} &
      \hspace{-1.5em}
      \begin{minipage}[c]{0.24\linewidth}
        \centering
        \includegraphics[width=0.95\linewidth]{results/web/mode2_ver1.0.pdf}
        % \vspace{-2em} 
        \\
        \hspace{-0.7em}
        (b-iv) Latent dynamics $\eigs$
      \end{minipage} \vspace{0.5em} \\
      \multicolumn{4}{c}{\textbf{(b) Snapshots at current time point $t_c=443$.}}
    \end{tabular}
    \vspace{-1.0em}
    \caption{\method modeling for a web-click activity stream related to beer query sets (i.e., \googletrend).
      Two sets of
      % invaluable knowledge
      snapshots taken
      at two different time points
      % (i.e., $t_c = 208, 443$, respectively)
      % on December 27, 2007 (top) and July 14, 2011 (bottom)
      show:
      (a/b-i) the current window of the original data stream,
      % where, the blue right vertical and red axes represent the current and $l_s$-steps-ahead time points, (i.e., $t_c, t_c+l_s$), respectively;
      % where, the blue vertical line to the right represents the current time point $t_c$;
      % where, the blue right vertical axis represents the current time point $t_c$;
      (a/b-ii) independent signals $\mE$ specific to each observation;
      % (c) time-evolving relationships with each other based on variables generating processes (i.e., \relations) and
      (a/b-iii) causal relationships $\mB\in\mathcal{B}$ and
      (a/b-iv) interpretable latent dynamics $\eigs$,
      where the argument and the absolute value of each point correspond to
      the temporal frequency and decay rate of modes, respectively.
      }
    \label{fig:web}
    \vspace{-1.2em}
\end{figure*}
% }
\setcounter{lemma}{1}
\subsection{Proof of Lemma \ref{lemma:create_time}}
\begin{proof}
The dominant steps in \textsc{RegimeCreation} are I, IV, and VI.
The decomposition $\mX$ into $\demixing^{-1}$ and $\mE$ using ICA requires $O(d^2N)$.
For each observation,
the SVD of $\ith{\mat{R}}\mat{M}$ requires $O(h^2N)$, and the eigendecomposition of $\ith{\tilde{\trans}}$ takes $O(k_i^3)$.
The straightforward way to
process IV and VI
is to perform the calculation $d$ times sequentially, i.e., they require $O(dh^2N+\sum_ik_i^3)$ in total.
However, since these operations do not interfere with each other,
they are simultaneously computed by parallel processing.
Therefore, the time complexity of \textsc{RegimeCreation} is $O(N(d^2+h^2)+k^3)$, where $k=\max_i(k_i)$.
\end{proof}
\subsection{Proof of Lemma \ref{lemma:causal}}
\begin{proof}
First, we need to formulate the causal structure.
Here, we utilize the structural equation model~\cite{pearl2009causality}, denoted by $\mX_{\text{sem}} = \mB_{\text{sem}}\mX_{\text{sem}} + \mE_{\text{sem}}$.
Because this model is known as the general formulation of causality, if $\mB_{\text{sem}}$ in this model is identified, then it can be said that we discover causality.
In other words, we need to prove that our proposed algorithm can find the causal adjacency matrix $\mB$ aligning with this model.
Solving the structural equation model for $\mX_{\text{sem}}$, we obtain 
$\mX_{\text{sem}} = \demixing^{-1}_{\text{sem}}\mE_{\text{sem}}$
where $\demixing_{\text{sem}} = \mat{I} - \mB_{\text{sem}}$.
It is shown that we can identify $\demixing_{\text{sem}}$ in the above equation by ICA,
except for the order and scaling of the independent components, if the observed data is a linear, invertible mixture of non-Gaussian independent components~\cite{comon1994independent}.
Thus, demonstrating that \modelgenerator precisely resolves the two indeterminacies of a mixing matrix $\mW^{-1}$ (i.e., the inverse of $\demixing \in \regime^c$) suffices to complete the proof because $\demixing$ is computed by ICA in \textsc{RegimeCreation}. \par
First, we reveal that our algorithm can resolve the order indeterminacy.
We can permutate the causal adjacency matrix $\mB$ to strict lower triangularity thanks to the acyclicity assumption~\cite{bollen1989structural}.
%, which is without loss of generality.
Thus, correctly permuted and scaled $\mW$
is a lower triangular matrix with all ones on the diagonal.
It is also said that there would only be one way to permutate $\mW$, which meets the above condition~\cite{shimizu2006linear}.
Thus, \modelgenerator can identify the order of a mixing matrix by the process in step I (i.e., finding the permutation of rows of a mixing matrix that yields a matrix without any zeros on the main diagonal).
Next, with regard to the scale of indeterminacy,
it is apparent that we only need to focus on the diagonal element,
remembering that the permuted and scaled $\mW$ has all ones on the diagonal.
Therefore, we prove that \modelgenerator can resolve the order and scaling of the indeterminacies of a mixing matrix $\demixing^{-1}$.
\end{proof}
% \myparaitemize{Proof of Lemma \ref{lemma:time}} \par
\subsection{Proof of Lemma \ref{lemma:stream_time}}
\begin{proof}
For each time point, \method first runs \modelestimator,
which estimates the optimal full parameter set $\modelparam$ and the model candidate $\candparam$ for the current window $\mX^c$.
If the current regime $\regime^c$ fits well,
it takes $O(N\sum_i k_i)$ time.
Otherwise, it takes $O(RN\sum_i k_i)$ time to find a better regime in $\regimeset\in\modelparam$.
Furthermore, if \method encounters an unknown pattern,
it runs \textsc{RegimeCreation}, which takes $O(N(d^2+h^2)+k^3)$ time.
Subsequently, it runs \modelgenerator to identify the causal adjacency matrix and forecast an $l_s$-steps-ahead future value,
which takes $O(d^2)$ and $O(l_s)$ time, respectively.
Note that $l_s$ is negligible because of the small constant value.
Finally, when \method does not create a new regime,
it executes \regimeupdate, which needs $O(dh^2)$ time.
Thus, the total time complexity is at least $O(N\sum_ik_i+dh^2)$ time and at most $O(RN\sum_i k_i+N(d^2+h^2)+k^3)$ time per process.
\end{proof}

% \input{components/table_acc_app_forecast}
% \begin{table*}[t]
    % \small
    \centering
    \caption{Ablation study results with forecasting steps $l_s\in\{5, 10, 15\}$ for both synthetic and real-world datasets.}
    \vspace{-1.0em}
    \begin{tabular}{c|c|cc|cc|cc|cc|cc}
    \toprule
    % \:Datasets\:
    \multicolumn{2}{c|}{Datasets}
    % & \#0 & \#1 & \#2 & \#3 & \#4 \\
    & \multicolumn{2}{c|}{\synthetic} & \multicolumn{2}{c|}{\covid} & \multicolumn{2}{c|}{\googletrend} & \multicolumn{2}{c|}{\chickendance} & \multicolumn{2}{c}{\exercise} \\
    \midrule
    \multicolumn{2}{c|}{Metrics}
    % \:Metrics\:
    & \:RMSE & MAE\:\,
    & \:RMSE & MAE\:\,
    & \:RMSE & MAE\:\,
    & \:RMSE & MAE\:
    & \:RMSE & MAE\:\: \\
    \midrule
    \multirow[t]{3}{*}{\:\:\method (full)\:\:}
    % \:\:\method (full)\:\:
    & 5 & \:0.722 & 0.528\:\, & \:0.588 & 0.268\:\, & \:0.573 & 0.442\:\, & \:0.353 & 0.221\: & \:0.309 & 0.177\:\, \\
    & 10 & \:0.829 & 0.607\:\, & \:0.740 & 0.361\:\, & \:0.620 & 0.481\:\, & \:0.511 & 0.325\: & \:0.501 & 0.309\:\, \\
    & 15 & \:0.923 & 0.686\:\, & \:0.932 & 0.461\:\, & \:0.646 & 0.505\:\, & \:0.653 & 0.419\: & \:0.687 & 0.433\:\, \\
    \midrule
    \multirow[t]{3}{*}{\:\:w/o causality\:\:}
    & 5 & \:0.759 & 0.563\:\, & \:0.758 & 0.374\:\, & \:0.575 & 0.437\:\, & \:0.391 & 0.262\: & \:0.375 & 0.218\:\, \\
    & 10 & \:0.925 & 0.696\:\, & \:0.848 & 0.466\:\, & \:0.666 & 0.511\:\, & \:0.590 & 0.398\: & \:0.707 & 0.433\:\, \\
    & 15 & \:1.001 & 0.760\:\, & \:1.144 & 0.583\:\, & \:0.708 & 0.545\:\, & \:0.821 & 0.537\: & \:0.856 & 0.533\:\, \\
    \bottomrule
    \end{tabular}
    \label{table:ablation}
    \vspace{-0.75em}
\end{table*}

\section{Experimental Setup}
% \label{section:app:experiments}
\label{section:app:experiments:setting}
In this section, we describe the experimental setting in detail.
% \subsection{Experimental Setting}
% \myparaitemize{Experimental Setting}
We conducted all our experiments on
% \unclear{<server spec>}.
an Intel Xeon Platinum 8268 2.9GHz quad core CPU
with 512GB of memory and running Linux.
We normalized the values of each dataset based on their mean and variance (z-normalization).
The length of the current window $N$ was $50$ steps in all experiments.
\par
\myparaitemize{Generating the Datasets}
We randomly generated synthetic multivariate data streams containing multiple clusters, each of which exhibited a certain causal relationship.
For each cluster, the causal adjacency matrix $\mB$ was generated from a well-known random graph model, namely Erdös-Rényi (ER)~\cite{erdos1960evolution} with edge density $0.5$ and the number of observed variables $d$ was set at 5.
The data generation process was modeled as a structural equation model~\cite{pearl2009causality},
where each value of the causal adjacency matrix $\mB$ was sampled from a uniform distribution $\mathcal{U}(-2, -0.5)\cup(0.5, 2)$.
In addition, to demonstrate the time-changing nature of exogenous variables, 
we allowed the inherent signals variance $\sigma^2_{i, t}$ (i.e., $\ith{e}(t)\sim\text{Laplace}(0, \sigma_{i, t}^2)$)
to change over time.
Specifically, we introduced $h_{i, t}=\text{log}(\sigma^2_{i, t})$, which evolves according to an autoregressive model, where the coefficient and noise variance of the autoregressive model were sampled from $\mathcal{U}(0.8, 0.998)$ and $\mathcal{U}(0.01, 0.1)$, respectively.
% however, 

The overall data stream was then generated by constructing a temporal sequence of cluster segments and each segment had $500$ observations (e.g., ``$1,2,1$'' consists of three segments containing two types of causal relationships and its total sample size is $1,500$). We ran our experiments on five different temporal sequences: ``$1,2,1$'', ``$1,2,3$'', ``$1,2,2,1$'', ``$1,2,3,4$'', and ``$1,2,3,2,1$'' to encompass various types of real-world scenarios.
\par
\myparaitemize{Baselines}
The details of the baselines we used throughout our extensive experiments are summarized as follows:
\par\noindent
(1) Causal discovering methods
{\setlength{\leftmargini}{11pt}
\vspace{-0.3ex}
\begin{itemize}
    \item CASPER~\cite{liu2023discovering}: is a state-of-the-art method for causal discovery, integrating the graph structure into the score function and reflecting the causal distance between estimated and ground truth causal structure. We tuned the parameters by following the original paper setting.
    \item DARING~\cite{he2021daring}: introduces an adversarial learning strategy to impose an explicit residual independence constraint for causal discovery. We searched for three types of regularization penalties $\{\alpha, \beta, \gamma\}\in\{0.001, 0.01, 0.1, 1.0, 10\}$.
    % aiming to improve the learning of acyclic graphs.
    \item NoCurl~\cite{yu2021dag}: uses a two-step procedure: initialize a cyclic solution first and then employ the Hodge decomposition of graphs. We set the optimal parameter presented in the original paper.
    % and learn a DAG structure by projecting the cyclic graph to the gradient of a potential function.
    \item NOTEARS-MLP~\cite{zheng2020learning}: is an extension of NOTEARS~\cite{zheng2018dags} (mentioned below) for nonlinear settings, which aims to approximate the generative structural equation model by MLP.
    We used the default parameters provided in authors' codes\footnote[2]{\url{https://github.com/xunzheng/notears} \label{fot:notears}}.
    \item NOTEARS~\cite{zheng2018dags}:
    % is specifically designed for linear settings and
    is a differentiable optimization method with an acyclic regularization term to estimate a causal adjacency matrix.
    We used the default parameters provided in authors' codes\footref{fot:notears}.
    % estimates the true causal graph by minimizing the fixed reconstruction loss with the continuous acyclicity constraint.
    \item LiNGAM~\cite{shimizu2006linear}:
    exploits the non-Gaussianity of data to determine the direction of causal relationships. It has no parameters to set.
    % and we used the authors source codes\footnote{https://github.com/cdt15/lingam}.
    \item GES~\cite{chickering2002optimal}: is a traditional score-based bayesian algorithm that discovers causal relationships in a greedy manner.
    It has no parameters to set.
    We employed BIC as the score function and utilized the open-source in~\cite{kalainathan2020causal}.
\end{itemize}
\vspace{-0.5ex}}
\par\noindent
(2) Time series forecasting methods
{\setlength{\leftmargini}{11pt}
\vspace{-0.3ex}
\begin{itemize}
    \item TimesNet/PatchTST~\cite{wu2023timesnet, Yuqietal-2023-PatchTST}: are state-of-the-art TCN-based and Transformer-based methods, respectively.
    The past sequence length was set as 16 (to match the current window length).
    % Other parameters follow the parameter settings suggested in the original paper.
    Other parameters followed the original paper setting.
    % \item PatchTST~\cite{Yuqietal-2023-PatchTST}: is a state-of-the-art Transformer-based method for time series forecasting. The past sequence length is set as 16 for the same reason as above.
    \item DeepAR~\cite{salinas2020deepar}: models probabilistic distribution in the future, based on RNN. We built the model with 2-layer 64-unit RNNs. We used Adam optimization~\cite{adam} with a learning rate of 0.01 and let it learn for 20 epochs with early stopping.
    % to choose the best model.
    \item OrbitMap~\cite{matsubara2019dynamic}:
    % is a stream forecasting algorithm that finds important time-evolving patterns with multiple discrete non-linear dynamical systems.
    finds important time-evolving patterns for stream forecasting.
    We determined the optimal transition strength $\rho$ to minimize the forecasting error in training.
    \item ARIMA~\cite{box1976arima}: is one of the traditional time series forecasting approaches based on linear
    equations. We determined the optimal parameter set using AIC.
\end{itemize}}




\end{document}


% This document was modified from the file originally made available by
% Pat Langley and Andrea Danyluk for ICML-2K. This version was created
% by Iain Murray in 2018, and modified by Alexandre Bouchard in
% 2019 and 2021 and by Csaba Szepesvari, Gang Niu and Sivan Sabato in 2022.
% Modified again in 2023 and 2024 by Sivan Sabato and Jonathan Scarlett.
% Previous contributors include Dan Roy, Lise Getoor and Tobias
% Scheffer, which was slightly modified from the 2010 version by
% Thorsten Joachims & Johannes Fuernkranz, slightly modified from the
% 2009 version by Kiri Wagstaff and Sam Roweis's 2008 version, which is
% slightly modified from Prasad Tadepalli's 2007 version which is a
% lightly changed version of the previous year's version by Andrew
% Moore, which was in turn edited from those of Kristian Kersting and
% Codrina Lauth. Alex Smola contributed to the algorithmic style files.
