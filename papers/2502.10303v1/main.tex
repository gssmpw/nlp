\documentclass[conference]{IEEEtran}
\IEEEoverridecommandlockouts
% The preceding line is only needed to identify funding in the first footnote. If that is unneeded, please comment it out.
\usepackage{cite}
\usepackage{amsmath,amssymb,amsfonts}
\usepackage{algorithmic}
\usepackage{graphicx}
\usepackage{textcomp}
\usepackage{xcolor}
\usepackage{titlesec}
\def\BibTeX{{\rm B\kern-.05em{\sc i\kern-.025em b}\kern-.08em
    T\kern-.1667em\lower.7ex\hbox{E}\kern-.125emX}}
\begin{document}

\title{Reinforcement Learning in Strategy-Based and Atari Games: A Review of Google DeepMind's Innovations\\}


%%
%% The "author" command and its associated commands are used to define
%% the authors and their affiliations.
%% Of note is the shared affiliation of the first two authors, and the
%% "authornote" and "authornotemark" commands
%% used to denote shared contribution to the research.

\author{Zihan Wang}
\orcid{0000-0003-0493-2668}
\authornote{Both authors contributed equally to the paper.}
\affiliation{%
  \institution{University of Amsterdam}
  \city{Amsterdam}
  \country{The Netherlands}
}
\email{zhw.cypher@gmail.com}

\author{Ziqi Zhao}
\orcid{0009-0008-3011-5745}
\authornotemark[1]
\affiliation{
    \institution{Shandong University}
    \city{Qingdao}
    \country{China}
}
\email{ziqizhao.work@gmail.com}

\author{Yougang Lyu}
\orcid{0009-0000-1082-9267}
\affiliation{
    \institution{University of Amsterdam}
    \city{Amsterdam}
    \country{The Netherlands}
}
\email{youganglyu@gmail.com}

\author{Zhumin Chen}
\orcid{0000-0003-4592-4074}
\affiliation{
    \institution{Shandong University}
    \city{Jinan}
    \country{China}
}
\email{chenzhumin@sdu.edu.cn}

\author{Maarten de Rijke}
\orcid{0000-0002-1086-0202}
\affiliation{%
  \institution{University of Amsterdam}
  \city{Amsterdam}
  \country{The Netherlands}
}
\email{m.derijke@uva.nl}

\author{Zhaochun Ren}
\orcid{0000-0002-9076-6565}
%\authornote{Corresponding author.}
\affiliation{%
  \institution{Leiden University}
  \city{Leiden}
  \country{The Netherlands}
}
\email{z.ren@liacs.leidenuniv.nl}

\maketitle
\thispagestyle{plain}
\pagestyle{plain}
\begin{abstract}

    \begin{abstract}
Retrieval-Augmented Generation (RAG) is often used with Large Language Models (LLMs) to infuse domain knowledge or user-specific information. In RAG, given a user query, a retriever extracts chunks of relevant text from a knowledge base. These chunks are sent to an LLM as part of the input prompt. Typically, any given chunk is repeatedly retrieved across user questions. However, currently, for every question, attention-layers in LLMs fully compute the key values (KVs) repeatedly for the input chunks, as state-of-the-art methods cannot reuse KV-caches when chunks appear at arbitrary locations with arbitrary contexts. Naive reuse leads to output quality degradation.  This leads to potentially redundant computations on expensive GPUs and increases latency. In this work, we propose \sys, a system for managing and reusing precomputed KVs corresponding to the text chunks (we call \textit{chunk-caches}) in RAG-based systems. We present how to identify \hl{\textit{chunk-caches} that are reusable}, how to efficiently perform a small fraction of recomputation to \textit{fix} the cache to maintain output quality, and how to efficiently store and evict \textit{chunk-caches} in the hardware for maximizing reuse while masking any overheads. With real production workloads as well as synthetic datasets, we show that \sys reduces redundant computation by \textbf{51\%} over SOTA prefix-caching and \textbf{75\%} over full recomputation.
\hl{Additionally, with continuous batching on a real production workload, we get a \textbf{1.6$\times$} speedup in throughput and a \textbf{2$\times$} reduction in end-to-end response latency over prefix-caching while maintaining quality, for both the \llama-3-8B and \llama-3-70B models. 
}
\end{abstract}






\end{abstract}

\begin{IEEEkeywords}
    Deep Reinforcement Learning, Google DeepMind, AlphaGo, AlphaGo Zero, MuZero, Atari Games, Go, Chess, Shogi,
\end{IEEEkeywords}

\section{Introduction}
%Introcution section%
Artificial Intelligence (AI) has revolutionized the gaming industry, both as a
tool for creating intelligent in-game opponents and as a testing environment
for advancing AI research. Games serve as an ideal environment for training and
evaluating AI systems because they provide well-defined rules, measurable
objectives, and diverse challenges. From simple puzzles to complex strategy
games, AI research in gaming has pushed the boundaries of machine learning and
reinforcement learning. Also The benfits from such employment helped game
developers to realize the power of AI methods to analyze large volumes of
player data and optimize game designs. \cite{I1} \\ Atari games, in particular,
with their retro visuals and straightforward mechanics, offer a challenging yet
accessible benchmark for testing AI algorithms. The simplicity of Atari games
hides complexity that they require strategies that involve planning,
adaptability, and fast decision-making, making them also a good testing
environment for evaluating AI's ability to learn and generalize. The
development of AI in games has been a long journey, starting with rule-based
systems and evolving into more sophisticated machine learning models. However,
the machine learning models had a few challenges, from these challenges is that
the games employing AI involves decision making in the game evironment. Machine
learning models are unable to interact with the decisions that the user make
because it depends on learning from datasets and have no interaction with the
environment. To overcome such problem, game developers started to employ
reinforcement learning (RL) in developing games. Years later, deep learning
(DL) was developed and shows remarkable results in video games\cite{I2}. The
combination between both reinforcement learning and deep learning resulted in
Deep Reinforcement Learning (DRL). The first employment of DRL was by game
developers in atari game\cite{I3}. One of the famous companies that employed
DRL in developing AI models is Google DeepMind. This company is known for
developing AI models, including games. Google DeepMind passed through a long
journey in developing AI models for games. Prior to the first DRL game model
they develop, which is AlphaGo, Google DeepMind gave a lot of contributions in
developing DRL, by which these contributions were first employed in Atari
games.\\ For the employment of DRL in games to be efficient, solving tasks in
games need to be sequential, so Google DeepMind combined RL-like techniques
with neural networks to create models capable of learning algorithms and
solving tasks that require memory and reasoning, which is the Neural Turing
Machines (NTMs)\cite{I4}. They then introduced the Deep Q-network (DQN)
algorithm, which is combine deep learning with Q-learning and RL algorithm.
Q-learning is a model in reinforcement learning which use the Q-network, which
is is a type of neural network to approximate the Q-function, which predicts
the value of taking a particular action in a given state\cite{I5}. The DQN
algorithm was the first algorithm that was able to learn directly from
high-dimensional sensory input, the data that have a large number of features
or dimensions\cite{I6}.\\ To enhance the speed of learning in reinforcement
learning agents, Google DeepMind introduced the concept of experience replay,
which is a technique that randomly samples previous experiences from the
agent's memory to break the correlation between experiences and stabilize the
learning process\cite{I7}. They then developed asynchronus methods for DRL,
which is the Actor-Critic (A3C) model. This model showed faster and more stable
training and showed a remarkable performance in Atari games\cite{I8}. By the
usage of these algorithms, Google DeepMind was able to develop the first AI
model that was able to beat the world champion in the game of Go, which is
AlphaGo in 2016.\\
The paper is organized as follows: Section II presents the related work that
surveys the development of DRL in games and the contribution that we 
added to the previous surveys. Section III presents the background 
information about the development of DRL in games. Section IV presents the
first AI model that Google DeepMind developed, which is AlphaGo. Section V
presents AlphaGo Zero. Section VI presents MuZero. Section VII presents the
advancements that were made in developing AI models for games. Section VIII 
presents the future directions that AI models for games will take, and their
applications in real life.
\section{Related Work}
\section{Related Work}
Alongside a discussion of what is meant by LLM harmfulness,
this section covers two distinct strands of related work: measuring types of harm in LLMs, and LLMs for diverse annotation tasks. %First,

%Different kinds of 
Diverse undesirable LLM outputs, from toxic language to privacy invasion, have been discussed in the observed \cite{banko-etal-2020-unified}. Here we review the ones we include in our definition of ``harm.'' %definition. Plus, we review LLMs as judges. 
Toxic content can be elicited from both generative  \cite{deshpande2023toxicity} and masked LLMs \cite{ousidhoum-etal-2021-probing}. 
%Among ways 
To measure toxic or hateful language, some use APIs such as PerspectiveAPI \cite{lees2022new} or HateBERT \cite{caselli-etal-2021-hatebert}. \citet{openai2024gpt4technicalreport} report that GPT4 produces toxic content 0.78\% of the time, versus 6.48\% in GPT3.5.
%as opposed to GPT3.5 with 6.48\%. On the other hand,
\citet{dubey2024llama} report that llama3-70B produces harmful content 5\% of the time, %whereas the 405B model generates harm 3\% of the time. 
compared to 3\% in the 405B model.
Instead of %single value classifiers to measure harm, 
reporting an absolute rate, we focus on relative harmfulness of different LLMs. %, so we point to recent work on LLMs for annotation.

The first category of harm we consider is social stereotyping and bias. %discrimination. It has been shown that 
LLMs can perpetuate social bias based on gender, race, religion etc. \cite{lin-etal-2022-gendered,bender2021dangers,field-etal-2021-survey,gupta-etal-2024-sociodemographic,andriushchenko2024agentharm,mazeika2024harmbench}. This can marginalize these groups more, and results in less fair model performance. \citet{guo2024hey} designed a competition to elicit biased output from LLMs to assess the perception of bias from non-expert users. %The first part of our work is similar to this analysis, but 
We also intentionally elicit harmful output, going %we look at other types of harms besides bias.
beyond social bias.

%When the models become stronger, they become more robust to jailbreaking attacks to elicit harmful content. However, there are datasets that can still jailbreak models to produce harmful content \cite{andriushchenko2024agentharm,mazeika2024harmbench}.

Our second category of harm is offensiveness and toxicity, which %. As opposed to stereotyping or social discrimination, this harm 
%is more subjective and harder to define than the previous category, so there 
lacks an established definition due to its greater subjectivity \cite{dev-etal-2022-measures,korre-etal-2023-harmful}. We include hate speech (HS) and abusive language as toxic content. HS can be defined as expressions of offensive and discriminatory discourse towards a group or an individual based on characteristics such as race, religion, nationality, or other group characteristics \cite{john2000hate,jahan2023systematic,basile2019semeval,davidson2017automated}. It includes racism, negative stereotyping, and sexist language. On the other hand, abusive language is content with inappropriate words such as profanity or disrespectful terms. It also includes psychological threats such as humiliation. %or constant criticism. %Toxic content can be elicited from both generative models \cite{deshpande2023toxicity} and masked language models \cite{ousidhoum-etal-2021-probing}.

%In addition to obvious toxic content, LLMs can generate diverse implicit toxic outputs using reinforcement learning with favoring toxic content in the reward function \cite{wen-etal-2023-unveiling}.  Regarding the subjectivity of this task, \cite{korre-etal-2023-harmful} reannotate the existing datasets with different definitions of toxicity and show that broader definitions result in more robust annotations, but interannotator agreements are still lower than 0.5. \cite{dev-etal-2022-measures} also point out the lack of definition for bias and harm in general and propose a framework to guide researchers during the development of bias measures.

Harm can be implicit, such as privacy invasion
%We are also interested in privacy invasion,
where there is 
leakage of personal information. %leakage from the model. 
%LLMs can memorize details of the training data and then leak private information such as 
This includes social security numbers, phone numbers, or bank account information \cite{carlini2021extracting,brown2022does}. 
%There are several frameworks to test the privacy of LLMs \cite{li2024llm} and generate data for personal attribute inference \cite{yukhymenko2024synthetic,kim2024propile}.

%Our definition of harm includes hate speech (HS) as well. HS can be defined as \textcolor{red}{expressions of} hatred towards a social group, the humiliation of the members of a group, or %communication disparaging  extreme disparagement of a person or a group based on race, color, ethnicity, gender, sexual orientation, nationality, religion, or other group characteristics .

For data annotation, LLMs
%Besides text generation, 
%LLMs have been used to annotate data because they 
can %be comparable to 
replace humans for some tasks, %and make the annotation process faster and cheaper 
with gains in efficiency and economy \cite{tan2024large}. They have been used for sociological annotations such as for classification of stance, bots or humor  \cite{ziems2024can,zhu2023can}. For tasks such as topic and frame detection or sentence segmentation they can surpass crowd-workers
%Some works show that they can surpass crowd-workers for some tasks such as topic and frame detection or sentence segmentation %into research aspects 
\cite{he2024if,gilardi2023chatgpt}. Some have argued that human-LLM collaboration results in more reliable annotation \cite{he2024if,zhang2023llmaaa,kim2024meganno+}. In addition to more objective tasks,
%LLMs have been used to annotate data %even 
they have been applied to subjective annotations such as offensiveness and abusiveness \cite{pavlovic-poesio-2024-effectiveness,zhu2023can,he2023annollm}, %. For example, LLMs are used as judges to rank responses from different LLMs 
or to rank outputs from different LLMs based on helpfulness, accuracy, or relevance \cite{zheng2023judging,lin2024wildbench,dubois2024length}. These works tend to focus on human-large LLM interactions, whereas we focus on single-turn responses from smaller LLMs. We inspire from \citet{zheng2023judging} but we only measure harm instead of overall performance. Plus, we use 3 LLMs to evaluate smaller LLMs.

\section{Background}
\section{Basic Background: Supervised Learning and the PAC Model}
\label{sec:background}

At this point almost everyone has heard of machine learning (ML). Anyone likely to stumble upon this article will have also heard of its most influential special case, supervised learning, and those theoretically inclined will also be familiar with the PAC model. Nonetheless, I will set the stage by  recapping the basics.

\subsection{Basics of Supervised Learning}%Let's set the stage in any case

\emph{Supervised Learning} is the task of ``coming up'' with a function $f: \X \to \Y$ to ``explain'' or ``fit'' a sequence of input/output examples   $(x_1,y_1), \ldots, (x_n,y_n)$, with $x_i \in \X$ and $y_i \in \Y$.  Here $\X$ is a \emph{data domain} consisting of \emph{datapoints} $x \in \X$, $\Y$ is a \emph{label set} consisting of \emph{labels} $y \in \Y$, and the sequence $(x_1,y_1),\ldots,(x_n,y_n)$ is the \emph{training data} consisting of \emph{labeled examples (a.k.a. samples)}~$(x_i,y_i)$.  I~will refer to the chosen function $f$ as a \emph{predictor}, and to $n$ as the \emph{sample size}. A \emph{learning algorithm} takes as input training data, and outputs (some representation of) a predictor $f \in \Y^\X$.\footnote{Note that this describes the usual \emph{batch}, a.k.a.~\emph{offline}, setting of supervised learning. I do not discuss other paradigms such as online or active learning in this article.} 



Success in supervised learning is defined as \emph{generalization} to  future examples: For a typical \emph{test example}  $(x_{\tst},y_{\tst})$, the predicted label $y'_{\tst}=f(x_{\tst})$ should ``equal'' $y_{\tst}$, perhaps approximately. We usually assume the test example is drawn from the same  ``source'' as the training data  --- commonly, i.i.d.~from the same distribution. The quality of the prediction is quantified by $\ell(y'_{\tst},y_{\tst})$, where $\ell:~\Y~\times~\Y \to \RR_{\geq 0}$ is a \emph{loss function} chosen as part of the problem definition. Common loss functions include the 0-1 loss $\ell_{0-1}(y',y) = [y' \neq y]$ for \emph{classification} problems,\footnote{The notation $[P]$ denotes $1$ when predicate $P$ is true, and denotes $0$ when $P$ is false.} as well as the absolute loss $|y'-y|$ or squared loss $(y'-y)^2$ for \emph{regression problems} featuring $\Y  \sse \RR$.

Nontrivial generalization properties are typically only possible if one assumes something about the data.\footnote{The need for such an assumption is formalized by the  \emph{no free lunch theorems} of supervised learning \cite{wolpert_connection_1992,wolpert_lack_1996,schaffer_conservation_1994}.} The Bayesian approach to  machine learning, common in many applications, assumes some parametric form for the distribution generating the data, and postulates a prior on the parameters. This is not the approach I will take in this article. Instead, I will focus on the frequentist --- and some would say ``worst-case'' or ``adversarial'' ---  approach that is common in the computational learning theory community, embodied by the PAC model. Here we assume that the (training and test) data can be explained, perhaps approximately, by a function in some ``simple enough to learn'' class of functions $\H \sse \Y^\X$, often called the \emph{hypotheses}. Equivalently, we  seek a predictor which explains the unseen data roughly  as well as the best hypothesis $h^* \in \H$, whether or not we assume that $h^*$ itself provides a perfect explanation.



 \paragraph{Common Algorithmic Templates.} Perhaps the best known general-purpose supervised learning algorithm is \emph{empirical risk minimization (ERM)}, which chooses as its predictor a hypothesis $f \in \H$ minimizing $\frac{1}{n} \sum_{i=1}^n \ell(f(x_i),y_i)$ --- a quantity called the \emph{training error}, \emph{empirical error}, or \emph{empirical risk} of $f$. %\footnote{When multiple hypotheses minimize the empirical risk, we assume ERM breaks ties arbitrarily.}
A common template for generalizing ERM involves adding a \emph{regularization term} $\psi(f)$ to the  objective function, typically chosen to measure some notion of ``hypothesis complexity.'' An algorithm instantiating this template is known as a \emph{structural risk minimizer (SRM)}, and chooses as its predictor the hypothesis $f \in \H$ minimizing the \emph{structural risk} $\frac{1}{n} \sum_{i=1}^n \ell(f(x_i),y_i) + \psi(f)$. Other well-known algorithms, such as gradient descent and its variations,  can frequently be interpreted as approximate implementations of ERM or SRM.


\paragraph{Proper vs Improper Learning.} A learning algorithm is said to be \emph{proper} if its predictor $f$ is always chosen from the hypothesis class, i.e., $f \in \H$, otherwise it is said to be \emph{improper}. ERM  is an example of a proper learning algorithm, as are SRM algorithms of the form described above.  In the \emph{proper regime} of learning, algorithms are required to be proper. This article will be concerned with the more flexible \emph{improper regime} (a.k.a \emph{representation-independent learning}), where no such constraint is placed on the learner. In other words, all we care about is predictive power at test time, rather than any insights derived from the functional form or representation of the predictor~itself.


\subsection{The PAC Model}
A standard mathematical setup for evaluation of supervised learning algorithms, at least in the theoretical computer science community, is Valiant's \emph{Probably Approximately Correct (PAC) model} of learning (see e.g.~\cite{kearns_introduction_1994,mohri_foundations_2018}). Here, we assume there is an unknown distribution $\D$ on $\X \times \Y$ from which training and test data are  drawn.  Specifically, the labeled datapoints of the training set  $(x_1,y_1), \ldots, (x_n,y_n)$, as well as the test data  $(x_\tst,y_\tst)$, are i.i.d.~from $\D$. Often it is assumed that $\D$ lies in some class of distributions of interest. The \emph{true expected loss}, or simply \emph{loss}, of a predictor $f: \X \to \Y$ is the expected loss it incurs on draws from $\D$, written $L_\D(f) = \Ex_{(x,y) \sim \D} \ell(f(x),y)$.


There are two main ``settings'' in PAC learning. The  \emph{realizable setting} only requires that the data be perfectly explained by some hypothesis in $\H$. More generally, the \emph{agnostic setting} makes no assumption relating the data to the hypotheses, but shifts the goalposts as necessary to allow nontrivial guarantees: the expected loss at test time is evaluated only ``relative'' to that of the best hypothesis $h^* \in \H$. There are other settings which make more nuanced assumptions, such as $\D$ being of a particular parametric form or its support living in some (unknown) lower-dimensional space, etc. I will mostly discuss the realizable and agnostic settings in this article, those being the simplest and most studied from a theoretical perspective. %TODO:We will briefly discuss other settings in Section ??

The PAC model demands high probability guarantees of learners, in the worst case over distributions of interest. Consider first the realizable setting, where $\D$ is such that $\min_{h \in \H} L_{\D}(h) = 0$. A PAC learner has \emph{error} $\epsilon=\epsilon(n)$ and \emph{confidence} $\delta=\delta(n)$ if, when training data consists of $n$ i.i.d~samples from a realizable distribution $\D$, it produces a predictor $f$  satisfying $L_\D(f) \leq \epsilon$ with probability at least $1-\delta$. In the agnostic setting, where $\D$ can be arbitrary, we require $L_\D(f) - \min_{h \in \H} L_\D(h) \leq \epsilon$ with probability $1-\delta$.

In both the realizable and agnostic settings, we look for PAC learners with small $\epsilon$ and $\delta$ as a function of the sample size $n$. An equivalent perspective looks at the sample complexity $m(\epsilon,\delta)$, which is the minimum sample size which guarantees error  at most $\epsilon$ with probability at least $1-\delta$. We say a problem is \emph{PAC learnable} if its PAC sample complexity is finite whenever $\epsilon,\delta > 0$.

For most PAC learning problems, learnability and sample complexity are characterized in terms of a  ``dimension'' of the hypothesis class. Most prominently this is the \emph{VC dimension} for binary classification, the \emph{fat shattering dimension} for agnostic regression, and the \emph{DS dimension} for multiclass classification (see \cite{anthony_neural_1999,daniely_optimal_2014,brukhim_characterization_2022}). Treatment of these is beyond the scope of this article. The unfamiliar reader need not worry, however,  as dimensions will feature only tangentially in our~discussion.




%\paragraph{Learning settings: Realizable, Agnostic, etc.} In learning theory, evaluating a supervised learning algorithm requires specifying a data model and an objective. We will leave the details of the data model flexible for now, to allow for both the PAC model and the adversarial transductive model. Nonetheless we will describe two variations, which we call ``settings'', which cut across different models. The  \emph{realizable setting}  requires only that the data be perfectly explained by some hypothesis $h \in \H$ --- i.e., there exists a hypothesis which is guaranteed to suffer a loss of $0$ on training and test data. The performance of the learning algorithm is its expected loss at test time for some ``worst case'' realizable instance. More generally, the \emph{agnostic setting} makes no assumption relating the data to the hypotheses, but shifts the goalposts as necessary to allow nontrivial guarantees: the expected loss at test time is evaluated only ``relative'' to that of the best hypothesis $h^* \in \H$, again for some ``worst case'' instance. There are other settings which make more nuanced assumptions about the data, such as it is drawn from a distribution of a particular parametric form, or that it lives in some (unknown) lower-dimensional space, etc. We will mostly discuss the realizable and agnostic settings, those being the simplest and most studied from a theoretical perspective.




%%% Local Variables:
%%% mode: latex
%%% TeX-master: "learning_matching"
%%% End:


\section{AlphaGo}
\subsection{Introduction}
AlphaGo is a groundbreaking AI model that utilizes neural networks and tree
search to play the game of Go, which is thought to be one of the most
challenging classic games for artificial intelligence owing to its enormous
search space and the difficulty of evaluating board positions and moves
\cite{Silver2016}. \\\\ AlphaGo uses value networks for position evaluation and
policy networks for taking actions, that combined with Monte Carlo simulation
achieved a 99.8\% winning rate, and beating the European human Go champion in 5
out 5 games.

\subsection{Key Innovations}

\subsubsection*{Integration of Policy and Value Networks with MCTS}
AlphaGo combines policy and value networks in an MCTS framework to efficiently explore and evaluate the game tree. Each edge \( (s, a) \) in the search tree stores:
\begin{itemize}
    \item Action value \( Q(s, a) \): The average reward for taking action \( a \) from
          state \( s \).
    \item Visit count \( N(s, a) \): The number of times this action has been explored.
    \item Prior probability \( P(s, a) \): The probability of selecting action \( a \),
          provided by the policy network.
\end{itemize}

During the selection phase, actions are chosen to maximize:
\begin{equation}
    a_t = \arg\max_a \left( Q(s, a) + u(s, a) \right)
\end{equation}

where the exploration bonus \( u(s, a) \) is defined as:
\begin{equation}
    u(s, a) \propto \frac{P(s, a)}{1 + N(s, a)}
\end{equation}

When a simulation reaches a leaf node, its value is evaluated in two ways: 1.
Value Network Evaluation: A forward pass through the value network predicts \(
v_\theta(s) \), the likelihood of winning. 2. Rollout Evaluation: A lightweight
policy simulates the game to its conclusion, and the terminal result \( z \) is
recorded.

These evaluations are combined using a mixing parameter \( \lambda \):
\begin{equation}
    V(s_L) = \lambda v_\theta(s_L) + (1 - \lambda) z_L
\end{equation}

The back propagation step updates the statistics of all nodes along the path
from the root to the leaf. \\\\ It's also worth noting that the SL policy
network performed better than the RL policy network and that's probably because
humans select a diverse beam of promising moves, whereas RL optimizes for the
single best move.

Conversely though, the value network that was derived from the RL policy
performed better than the one derived from the SL policy.
\begin{figure}[htbp]
    \centering
    \includegraphics[width=0.9\linewidth, keepaspectratio]{sections/4AlphaGo/innovation.png}
    \caption{Performance of AlphaGo, on a single machine, for different
combinations of components.}
\end{figure}

\subsection{Training Process}

\subsubsection{Supervised Learning for Policy Networks}
The policy network was initially trained using supervised learning on human
expert games. The training data consisted of 30 million board positions sampled
from professional games on the KGS Go Server. The goal was to maximize the
likelihood of selecting the human move for each position:
\begin{equation}
    \Delta \sigma \propto \nabla_\sigma \log p_\sigma(a | s)
\end{equation}

where \( p_\sigma(a | s) \) is the probability of selecting action \( a \)
given state \( s \).

This supervised learning approach achieved a move prediction accuracy of 57.0\%
on the test set, significantly outperforming prior methods. This stage provided
a solid foundation for replicating human expertise.

\subsubsection{Reinforcement Learning for Policy Networks}
The supervised learning network was further refined through reinforcement
learning (RL). The weights of the RL policy network were initialized from the
SL network. AlphaGo then engaged in self-play, where the RL policy network
played against earlier versions of itself to iteratively improve.

The reward function used for RL was defined as:
\begin{equation}
    r(s) =
    \begin{cases}
        +1 & \text{if win}                           \\
        -1 & \text{if loss}                          \\
        0  & \text{otherwise (non-terminal states).}
    \end{cases}
\end{equation}

At each time step \( t \), the network updated its weights to maximize the
expected reward using the policy gradient method:
\begin{equation}
    \Delta \rho \propto z_t \nabla_\rho \log p_\rho(a_t | s_t)
\end{equation}

where \( z_t \) is the final game outcome from the perspective of the current
player.

This self-play strategy allowed AlphaGo to discover novel strategies beyond
human knowledge. The RL policy network outperformed the SL network with an 80\%
win rate and achieved an 85\% win rate against Pachi, a strong open-source Go
program, without using MCTS.

\subsubsection{Value Network Training}
The value network was designed to evaluate board positions by predicting the
likelihood of winning from a given state. Unlike the policy network, it outputs
a single scalar value \( v_\theta(s) \) between \(-1\) (loss) and \(+1\) (win).

Training the value network on full games led to overfitting due to the strong
correlation between successive positions in the same game. To mitigate this, a
new dataset of 30 million distinct board positions was generated through
self-play, ensuring that positions came from diverse contexts.

The value network was trained by minimizing the mean squared error (MSE)
between its predictions \( v_\theta(s) \) and the actual game outcomes \( z \):
\begin{equation}
    L(\theta) = \mathbb{E}_{(s, z) \sim D} \left[ (v_\theta(s) - z)^2 \right]
\end{equation}

\begin{figure}[htbp]
    \centering
    \includegraphics[width=\linewidth, keepaspectratio]{sections/4AlphaGo/mse.png}
    \caption{Comparison of evaluation
accuracy between the value network and rollouts with different policies.}
\end{figure}

\subsection{Challenges and Solutions}
AlphaGo overcame several challenges:
\begin{itemize}
    \item Overfitting: Training the value network on full games led to memorization. This
          was mitigated by generating a diverse self-play dataset.
    \item Scalability: Combining neural networks with MCTS required significant
          computational resources, addressed through parallel processing on GPUs and
          CPUs.
    \item Exploration vs. Exploitation: Balancing these in MCTS was achieved using the
          exploration bonus \( u(s, a) \) and the policy network priors.
\end{itemize}

\subsection{Performance Benchmarks}
AlphaGo achieved the following milestones:
\begin{itemize}
    \item 85\% win rate against Pachi without using MCTS.
    \item 99.8\% win rate against other Go programs in a tournament held to evaluate the performance of AlphaGo.
    \item Won 77\%, 86\%, and 99\% of handicap games against Crazy Stone, Zen and Pachi,
          respectively.
    \item Victory against professional human players such as Fan Hui (5-0) and Lee Sedol
          (4-1), marking a significant breakthrough in AI.
\end{itemize}

\begin{figure}[htbp]
    \centering
    \includegraphics[width=\linewidth, keepaspectratio]{sections/4AlphaGo/comparison.png}
    \caption{Elo rating comparison between AlphaGo and other Go programs.}
\end{figure}


\section{AlphaGo Zero}
\subsection{Introduction}
AlphaGo Zero represents a groundbreaking advancement in artificial intelligence
and reinforcement learning. Unlike its predecessor, AlphaGo, which relied on
human gameplay data for training, AlphaGo Zero learns entirely from self-play,
employing deep neural networks and Monte Carlo Tree Search (MCTS). \cite{agz1}
\\\\ By starting with only the rules of the game and leveraging reinforcement
learning, AlphaGo Zero achieved superhuman performance in Go, defeating the
previous version of AlphaGo in a 100-0 match.
\subsection{Key Innovations}

AlphaGo Zero introduced several groundbreaking advancements over its
predecessor, AlphaGo, streamlining and enhancing its architecture and training
process:

\begin{enumerate}
    \item Unified Neural Network \( f_\theta \): AlphaGo Zero replaced AlphaGo's
          dual-network setup—separate networks for policy and value—with a single neural
          network \( f_\theta \). This network outputs both the policy vector \( p \) and
          the value scalar \( v \) for a given game state, reprsented as

          \begin{equation}
              f_\theta(s) = (p, v)
          \end{equation}

          This unified architecture simplifies the model and improves training
          efficiency.

    \item Self-Play Training: Unlike AlphaGo, which relied on human games as training
          data, AlphaGo Zero was trained entirely through self-play. Starting from random
          moves, it learned by iteratively playing against itself, generating data and
          refining \( f_\theta \) without any prior knowledge of human strategies. This
          removed biases inherent in human gameplay and allowed AlphaGo Zero to discover
          novel and highly effective strategies.

    \item Removal of Rollouts: AlphaGo Zero eliminated the need for rollouts, which were
          computationally expensive simulations to the end of the game used by AlphaGo's
          MCTS. Instead, \( f_\theta \) directly predicts the value \( v \) of a state,
          providing a more efficient and accurate estimation.

    \item Superior Performance: By integrating these advancements, AlphaGo Zero defeated
          AlphaGo 100-0 in direct matches, demonstrating the superiority of its self-play
          training, unified architecture, and reliance on raw rules over pre-trained
          human data.
\end{enumerate}

\subsection{Training Process}

\subsubsection{Monte Carlo Tree Search (MCTS) as policy evaluation operator}
Intially the neural network \( f_\theta \) is not very accurate in predicting
the best move, as it is intiallised with random weights at first. To overcome
this, AlphaGo Zero uses MCTS to explore the game tree and improve the policy.
\\\\ At a given state S, MCTS expands simualtions of the best moved that are
most likely to generate a win based on the initial policy $P(s,a)$ and the
value $V$. MCTS iteratively selects moves that maximize the upper confidence
bound (UCB) of the action value. UCB is designed to balanced exploration and
exploitation. and it is defined as

\begin{equation}
    UCB = Q(s, a) + U(s, a)
\end{equation}

where \[U(s, a) \propto \frac{p(s, a)}{1 + N(s, a)}\] MCTS at the end of the search returns the policy vector \( \pi \) which is used
to update the neural network \( f_\theta \) by minimizing the cross-entropy
loss between the predicted policy by $f_\theta$ and the MCTS policy.

\subsubsection{Policy Iteration and self play}
The agent plays games against itself using the predicted policy $P(s, a)$. The
agent uses the MCTS to select the best move at each state and the game is
played till the end in a process called self play. The agent then uses the
outcome of the game, $z$ game winner and $\pi$ to update the neural network.
This process is repeated for a large number of iterations.

\subsubsection{Network Training Process}

The neural network is updated after each self-play game by using the data
collected during the game. This process involves several key steps:
\begin{enumerate}
    \item Intilisation of the network: The neural network starts with random weights
          $\theta_0$, as there is no prior knowledge about the game.
    \item Generating Self-play Games: For each iteration $i \geq 1$ self-play games are
          generated. During the game, the neural network uses its current parameters
          $\theta_{i - 1}$ to run MCTS and generate search probabilities $\pi_t$ for each
          move at time step $t$.
    \item Game Termination and scoring: A game ends when either both players pass, a
          resignation threshold is met, or the game exceeds a maximum length. The winner
          of the game is determined, and the final result $z_t$ is recorded, providing
          feedback to the model.
    \item Data Colletion: for each time step $t$, the training data $(s_t, \pi_t, z_t)$
          is stored, where $s_t$ is the game state, $\pi_t$ is the search probabilities,
          and $z_t$ is the game outcome.
    \item Network training process: after collecing data from self-play, The neural
          network $f_\theta$ is adjusted to minimize the error between the predicted
          value v and the self-play winnder z, and to maximize the similarity between the
          search probabilities $P$ and the MCTS probabilities. This is done by using a
          loss function that combines the mean-squared error and the cross entropy losses
          repsectibly. The loss function is defined as

          \begin{equation}
              L = (z - v)^2 - \pi^T \log p + c||\theta||^2
          \end{equation}

          where $c$ is the L2 regularization term.

\end{enumerate}

\subsection{Challenges and Solutions}
Alpha Go Zero overcame several challanges:
\begin{enumerate}
    \item Human knowledge Dependency: AlphaGo Zero eliminated the need for human gameplay
          data, relying solely on self-play to learn the game of Go. This allowed it to
          discover novel strategies that surpassed human expertise.
    \item Compelxity of the dual network approach in alpha go: AlphaGo utilized separate
          neural networks for policy prediction $p$ and value estimation $V$, increasing
          the computational complexity. AlphaGo Zero unified these into a \textbf{single
              network} that outputs both $p$ and $V$, simplifying the architecture and
          improving training efficiency.
    \item The need of handcrafted features: AlphaGo relied on handcrafted features, such
          as board symmetry and pre-defined game heuristics, for feature extraction.
          AlphaGo Zero eliminated the need for feature engineering by using \textbf{raw
              board states} as input, learning representations directly from the data.
\end{enumerate}

\subsection{Performance Benchmarks}

AlphaGo Zero introduced a significant improvement in neural network
architecture by employing a unified residual network (ResNet) design for its
$f_\theta$ model. This replaced the separate CNN-based architectures previously
used in AlphaGo, which consisted of distinct networks for policy prediction and
value estimation. \\\\ The superiority of this approach is evident in the Elo
rating comparison shown in fig.\ref{fig:results_agz}. The "dual-res"
architecture, utilized in AlphaGo Zero, achieved the highest Elo rating,
significantly outperforming other architectures like "dual-conv" and "sep-conv"
used in earlier versions of AlphaGo.

\begin{figure}[ht]
    \centering
    \includegraphics[width=0.35\textwidth]{sections/5AlphaGo_Zero/Results_agz.png}
    \caption{Elo rating comparison of different neural network architectures.}
    \label{fig:results_agz}
\end{figure}



\section{MuZero}
% Introduction
\subsection{Introduction}
Through the development of AlphaZero, a general model for board games with
superhuman ability has been achieved in three games: Go, chess, and Shogi. It
could achieve these results without the need for human gameplay data or
history, instead using self-play in an enclosed environment. However, the model
still relied on a simulator that could perfectly replicate the behavior, which
might not translate well to real-world applications, where modeling the system
might not be feasible. MuZero was developed to address this challenge by
developing a model-based RL approach that could learn without explicitly
modeling the real environment. This allowed for the same general approach used
in AlphaZero to be used in Atari environments where reconstructing the
environment is costly. Essentially, MuZero was deployed to all the games with
no prior knowledge of them or specific optimization and managed to show
state-of-the-art results in almost all of them.

% MuZero Algorithm
\subsection{MuZero Algorithm}
\begin{figure*}[t]
    \centering
    \includegraphics[width=0.8\textwidth]{sections/6MuZero/graph_1.png}
    \caption{(A) Represents the progression of the model through its MDP, while (B) Represents MuZero acting as an environment with MCTS as feedback, and (C) Represents a diagram of training MuZero's model.}
\end{figure*}
The model takes in an input of observations $o_1, \ldots, o_t$ that are then fed to a representation network $h$,
which reduces the dimensions of the input and produces a root hidden state $s_0$. Internally, the model
mirrors an MDP, with each state representing a node with edges connecting it to the future states
depending on available actions. Unlike traditional RL approaches, this hidden state is not constrained to
contain the information necessary to reproduce the entire future observations. Instead, the hidden states
are only optimized for predicting information that is related to planning. At every time step, the model
predicts the policy, the immediate reward, and the value function. The output of the state-action pair is
then used by the dynamics function to produce future states. Similar to AlphaZero, a Monte Carlo tree
search is used to find the best action policy given an input space. This is used to train the model by
comparing the MCTS policy with the predictor function policy. Also, after a few training runs, the model
ceases to use illegal moves, and the predicted actions map to the real action space. This eliminates the
need for a simulator, as the model internalizes the environment characteristics it deems necessary for
planning and acting, which generally converges to reality through training. The value function at
the final step is compared against the game result in board games, i.e., win, loss, or a draw.

\subsection{Loss function and learning equations}

\begin{align}
    s_0      & = h_\theta(o_1, \ldots, o_t) \\
    r_k, s_k & = g_\theta(s_{k-1}, a_k)     \\
    p_k, v_k & = f_\theta(s_k)
\end{align}

\begin{equation}
    \setlength{\arraycolsep}{1.5pt} % Reduce spacing inside the matrix
    \begin{bmatrix}
        p_k \\ v_k \\ r_k
    \end{bmatrix}
    =
    \mu_\theta(o_1, \ldots, o_t, a_1, \ldots, a_k)
    \label{eq:matrix}
\end{equation}

\begin{align}
    \nu_t, \pi_t & = \text{MCTS}(s_0^t, \mu_\theta) \\
    a_t          & \sim \pi_t
\end{align}

\begin{align}
    p_k^t, v_k^t, r_k^t & = \mu_\theta(o_1, \ldots, o_t, a_{t+1}, \ldots, a_{t+k}) \\
    z_t                 & =
    \begin{cases}
        u_T,                                               & \text{for games}        \\
        u_{t+1} + \gamma u_{t+2} + \ldots \nonumber                                  \\
        \quad + \gamma^{n-1} u_{t+n} + \gamma^n \nu_{t+n}, & \text{for general MDPs}
    \end{cases}
\end{align}

\begin{align}
    l_t(\theta) & =
    \sum_{k=0}^K \big[ l_r(u_{t+k}, r_k^t) + l_v(z_{t+k}, v_k^t) \nonumber \\
                & \quad + l_p(\pi_{t+k}, p_k^t) \big] + c \|\theta\|^2
\end{align}

\begin{align}
    l_r(u, r)   & =
    \begin{cases}
        0,                & \text{for games}        \\
        \phi(u)^T \log r, & \text{for general MDPs}
    \end{cases} \\
    l_v(z, q)   & =
    \begin{cases}
        (z - q)^2,        & \text{for games}        \\
        \phi(z)^T \log q, & \text{for general MDPs}
    \end{cases} \\
    l_p(\pi, p) & = \pi^T \log p
\end{align}

% MCTS
\subsection{MCTS}
MuZero uses the same approach developed in AlphaZero to find the optimum action
given an internal state. MCTS is used where states are the nodes, and the edges
store visit count, mean value, policy, and reward. The search is done in a
three-phase setup: selection, expansion, and backup. The simulation starts with
a root state, and an action is chosen based on the state-transition reward
table. Then, after the end of the tree, a new node is created using the output
of the dynamics function as a value, and the data from the prediction function
is stored in the edge connecting it to the previous state. Finally, the
simulation ends, and the updated trajectory is added to the state-transition
reward table. In two-player zero-sum games, board games, for example, the value
function is bounded between 0 and 1, which is helpful to use value estimation
and probability using the pUCT rule. However, many other environments have
unbounded values, so MuZero rescales the value to the maximum value observed by
the model up to this training step, ensuring no environment-specific data is
needed.\cite{mz1}

% Results
\subsection{Results}
The MuZero model demonstrated significant improvements across various test
cases, achieving state-of-the-art performance in several scenarios. Key
findings include:

\subsubsection{Board Games}
\begin{itemize}
    \item When tested on the three board games AlphaZero was trained for (Go, chess, and
          shogi):
          \begin{itemize}
              \item MuZero matched AlphaZero's performance \textbf{without any prior knowledge} of
                    the games' rules.
              \item It achieved this with \textbf{reduced computational cost} due to fewer residual
                    blocks in the representation function.
          \end{itemize}
\end{itemize}

\subsubsection{Atari Games}
\begin{itemize}
    \item MuZero was tested on 60 Atari games, competing against both human players and
          state-of-the-art models (model-based and model-free). Results showed:
          \begin{itemize}
              \item \textbf{Starting from regular positions:} MuZero outperformed competitors in \textbf{46 out of 60 games}.
              \item \textbf{Starting from random positions:} MuZero maintained its lead in \textbf{37 out of 60 games}, though its performance was reduced.
          \end{itemize}
    \item The computational efficiency and generalization of MuZero highlight its
          effectiveness in complex, unstructured environments.
\end{itemize}

\subsubsection{Limitations}
\begin{itemize}
    \item Despite its strengths, MuZero struggled in certain games, such as:
          \begin{itemize}
              \item \emph{Montezuma's Revenge} and \emph{Pitfall}, which require long-term planning and strategy.
          \end{itemize}
    \item General challenges:
          \begin{itemize}
              \item Long-term dependencies remain difficult for MuZero, as is the case for RL
                    models in general.
              \item Limited input space and lack of combinatorial inputs in Atari games could
                    introduce scalability issues for broader applications.\cite{mz1}
          \end{itemize}
\end{itemize}



\section{Advancements}
%Advancments%
The evolution of AI in gaming, particularly through the development of AlphaGo,
AlphaGo Zero, and MuZero, highlights remarkable advancements in reinforcement
learning and artificial intelligence. AlphaGo, the pioneering model, combined
supervised learning and reinforcement learning to master the complex game of
Go, setting the stage for AI to exceed human capabilities in well-defined
strategic games. Building on, AlphaGo Zero eliminated the reliance on human
data, introducing a fully self-supervised approach that demonstrated greater
efficiency and performance by learning solely through self-play. MuZero took
this innovation further by generalizing beyond specific games like Go, Chess,
and Shogi, employing model-based reinforcement learning to predict dynamics
without explicitly knowing the rules of the environment. Completing on these
three models, here are some of the advancements that developed from them:
AlphaZero and MiniZero; and one of the most used in generating AI models,
Multi-agent models.
\subsection*{AlphaZero}
While AlphaGo Zero was an impressive feat, designed specifically to master the
ancient game of Go through self-play, AlphaZero developes it by generalizing
its learning framework to include multiple complex games: chess, shogi
(Japanese chess), and Go. The key advancement is in its ability to apply the
same algorithm across different games without requiring game-specific
adjustments. AlphaZero's neural network is trained through self-play,
predicting the move probabilities and game outcomes for various positions. This
prediction is then used to guide the MCTS, which explores potential future
moves and outcomes to determine the best action. Through iterative self-play
and continuous refinement of the neural network, AlphaZero efficiently learns
and improves its strategies across different games\cite{AD3}. Another
significant improvement is in AlphaZero’s generalized algorithm, is that it
does not need to be fine-tuned for each specific game. This was a departure
from AlphaGo Zero’s Go-specific architecture, making AlphaZero a more versatile
AI system.\\ AlphaZero's architecture integrates a single neural network that
evaluates both the best moves and the likelihood of winning from any given
position, streamlining the learning process by eliminating the need for
separate policy and value networks used in earlier systems. This innovation not
only enhances computational efficiency but also enables AlphaZero to adopt
unconventional and creative playing styles that diverge from established human
strategies.
\subsection*{MiniZero}
MiniZero is a a zero-knowledge learning framework that supports four
state-of-the-art algorithms, including AlphaZero, MuZero, Gumbel AlphaZero, and
Gumbel MuZero\cite{AD1}. Gumbel AlphaZero and Gumbel MuZero are variants of the
AlphaZero and MuZero algorithms that incorporate Gumbel noise into their
decision-making process to improve exploration and planning efficiency in
reinforcement learning tasks. Gumbel noise is a type of stochastic noise
sampled from the Gumbel distribution, commonly used in decision-making and
optimization problems.\\ MiniZero is a simplified version of the original
MuZero algorithm, which is designed to be have a more simplified architecture
reducing the complexity of the neural network used to model environment
dynamics, making it easier to implement and experiment with. This
simplification allows MiniZero to perform well in smaller environments with
fewer states and actions, offering faster training times and requiring fewer
computational power compared to MuZero.

\subsection*{Multi-agent models}
Multi-agent models in reinforcement learning (MARL) represent an extension of
traditional single-agent reinforcement learning. In these models, multiple
agents are simultaneously interacting, either competitively or cooperatively,
making decisions that impact both their own outcomes and those of other agents.
The complexity in multi-agent systems arises from the dynamic nature of the
environment, where the actions of each agent can alter the environment and the
states of other agents. Unlike in single-agent environments, where the agent
learns by interacting with a static world, multi-agent systems require agents
to learn not only from their direct experiences but also from the behaviors of
other agents, leading to a more complex learning process. Agents must adapt
their strategies based on what they perceive other agents are doing, and this
leads to problems such as strategic coordination, deception, negotiation, and
competitive dynamics. In competitive scenarios, agents might attempt to outwit
one another, while in cooperative scenarios, they must synchronize their
actions to achieve a common goal\cite{AD2}.\\ AlphaGo and AlphaGo Zero are not
designed to handle multi-agent environments. The core reason lies in their
foundational design, which assumes a single agent interacting with a static
environment. AlphaGo and AlphaGo Zero both rely on model-based reinforcement
learning and self-play, where a single agent learns by interacting with itself
or a fixed opponent, refining its strategy over time. However, these models are
not built to adapt to the dynamic nature of multi-agent environments, where the
state of the world constantly changes due to the actions of other agents. In
AlphaGo and AlphaGo Zero, the environment is well-defined, and the agent’s
objective is to optimize its moves based on a fixed set of rules. The agents in
these models do not need to account for the actions of other agents in
real-time or consider competing strategies, which are essential in multi-agent
systems. Additionally, AlphaGo and AlphaGo Zero are not designed to handle
cooperation or negotiation, which are key aspects of multi-agent
environments.\\ On the other hand, MuZero offers a more flexible framework that
can be adapted to multi-agent environments. Unlike AlphaGo and AlphaGo Zero,
MuZero operates by learning the dynamics of the environment through its
interactions, rather than relying on a fixed model of the world. This approach
allows MuZero to adapt to various types of environments, whether single-agent
or multi-agent, by learning to predict the consequences of actions without
needing explicit knowledge of the environment’s rules. The key advantage of
MuZero in multi-agent settings is its ability to plan and make decisions
without needing to model the entire system upfront. In multi-agent
environments, this ability becomes essential, as MuZero can dynamically adjust
its strategy based on the observed behavior of other agents. By learning not
just the immediate outcomes but also the strategic implications of others'
actions, MuZero can navigate both competitive and cooperative settings.
\begin{figure*}[t]
    \centering
    \includegraphics[width=0.7\textwidth]{sections/8Future_Directions/MuZeroRC.png}
    \caption{MuZero Rate-Controller (MuZero-RC) optimizing the encoding process in video streaming.}
\end{figure*}

\section{Future Directions}
%Future Directions%
As mentioned earlier in the paper, The development of the AI models and systems
in the field of gaming represent a good training set for the models to study
the environment, address the challenges, modify the models, and achieve good
results in this field, to judge whether this model is able to be implemented in
real world, and how it can be implemented. The main purpose from such models,
Google DeepMind, through the previous years, had been training the models to
play games, and the main goal was to implement the models of reinforcement
learning in real life, and benefit from them. DeepMind already started in this
implementation with MuZero, and developing other models to be able to be
implemented in real life directly.
\subsection*{MuZero's first step from research into the real world}
One of the notable implementations of MuZero has been in collaboration with
YouTube, where it was used to optimize video compression within the open-source
VP9 codec. This involved adapting MuZero's ability to predict and plan, which
it had previously demonstrated in games, to a complex and practical task of
video streaming. By optimizing the encoding process, as shown in fig. 3, MuZero
achieved an average bitrate reduction of 4\% without degrading video
quality\cite{FD1}. This improvement directly impacts the efficiency of video
streaming services such as YouTube, Twitch, and Google Meet, leading to faster
loading times and reduced data usage for users. This implementation is called
MuZero Rate-Controller (MuZero-RC). Beyond video compression, this initial
application of MuZero outside of game research settings exemplifies how
reinforcement learning agents can address practical real-world challenges. By
designing agents with new capabilities to enhance products across different
sectors, computer systems can be more efficient, less resource-intensive, and
increasingly automated\cite{FD1}.
\subsection*{AlphaFold}
AlphaFold is a model developed by DeepMind that addresses one of the
challenging problems in biology, which is predicting the three-dimensional
structures of proteins from their amino acid sequences. AlphaFold employs
advanced deep learning techniques, prominently featuring reinforcement
learning, to enhance its predictive capabilities.The model operates on a
feedback loop where it generates predictions about protein structures and
receives rewards based on the accuracy of these predictions compared to
experimentally determined structures. This process allows AlphaFold to
iteratively refine its models, optimizing them to better reflect the
complexities of protein folding dynamics. The architecture of AlphaFold
includes deep neural networks that analyze both the sequential and spatial
relationships between amino acids, enabling it to capture intricate patterns
for protein conformation. By training on extensive datasets of known protein
structures, AlphaFold has achieved unprecedented accuracy, often rivaling
experimental methods such as X-ray crystallography and cryo-electron
microscopy\cite{FD2}.\\\\ As shown form the previous models, how the employment
of reinforcementl learning changed starting from making AI systems which play
atari and strategy-based games, reaching to help in human biology and create
protein structures, the enployment of reinforcement learning in games still has
a long journey to be developed which helps in both real life and gaming. Google
DeepMind is still working on other models which are able to be implemented in
real life applications. They also developed models which use the multi-agent
models in games, like AlphaStar, which is a model to play StarCraft II; but
still didn't apply them in real life applications, which is a good future
direction to be developed.

\section*{Conclusion}
\section*{Conclusion}
This paper aims to enhance our understanding of the computational complexity of computing various Shapley value variants. We found that for various ML models --- including decision trees, regression tree ensembles, weighted automata, and linear regression --- both local and global interventional and baseline SHAP can be computed in polynomial time under HMM modeled distributions. This extends popular algorithms, such as TreeSHAP, beyond their empirical distributional scope. We also establish strict complexity gaps between the various SHAP variants (baseline, interventional, and conditional) and prove the intractability of computing SHAP for tree ensembles and neural networks in simplified scenarios. Overall, we present SHAP as a versatile framework whose complexity depends on four key factors: \begin{inparaenum}[(i)] \item model type, \item SHAP variant, \item distribution modeling approach, \item and local vs. global explanations\end{inparaenum}. We believe this perspective provides deeper insight into the computational complexity of SHAP, paving the way for future work.




%We believe that our framework provides a more intricate understanding of SHAP computation complexity across different models, distributions, and variants, paving the way for further research.

Our work opens promising directions for future research. First, expanding our computational analysis to other SHAP-related metrics, such as asymmetric SHAP~\citep{frye20} and SAGE~\citep{covert2020understanding}, would be valuable. Additionally, we aim to explore more expressive distribution classes and relaxed assumptions beyond those in Section \ref{sec:tractable} while maintaining tractable SHAP computation. Finally, when exact computation is intractable (Section \ref{sec:intractable}), investigating the approximability of SHAP metrics through approximation and parameterized complexity theory~\citep{downey2012parameterized} is an important direction.

%Our work opens several promising avenues for future research on the computational properties of explainable AI methods, with a particular focus on SHAP. First, it would be interesting to broaden the computational analysis conducted in this work to include other popular SHAP-related metrics in the literature, such as asymmetric SHAP \cite{frye20} and SAGE \cite{covert2020understanding}. Also, in the future, we aim to explore more expressive distribution classes and relaxed distributional assumptions—extending beyond those examined in Section \ref{sec:tractable} —that still yield tractable SHAP computation. Finally, when exact computation proves intractable (Section \ref{sec:intractable}), it is worthwhile to theoretically investigate the question of the approximability of computing the SHAP metrics across various configurations, through the lens of approximation and parametrized complexity theory \cite{arora2009computational}.

%This paper aims to deepen our understanding of the computational complexity involved in obtaining different Shapley value variants. We found that for a variety of ML models, including decision trees, tree ensembles for regression, weighted automata, and linear regression models — computing both local and global interventional and baseline SHAP can be done in polynomial time when distributions are modeled by HMMs. This extends the distributional scope of popular algorithms like TreeSHAP, which is limited to empirical distributions. Additionally, we demonstrate a strict complexity gap between SHAP variants, showing that interventional and baseline SHAP can be strictly easier to compute than conditional SHAP. Despite these positive results, we uncovered intractability for various SHAP variants in neural networks and tree ensembles. Finally, we provided generalized complexity relations across SHAP variants. We believe that our framework offers a deeper understanding of the complexity involved in computing SHAP across various variants, models, distributions, as well as in both local and global computations, laying the groundwork for future research.


\begin{thebibliography}{00}
    %-------Introduction-------
    \bibitem{I1}  N. Y. Georgios and T. Julian, Artificial Intelligence and Games. New York: Springer, 2018.
    \bibitem{I2}  N. Justesen, P. Bontrager, J. Togelius, S. Risi, (2019). Deep learning for video game playing. arXiv.
    \bibitem{I3}  V. Mnih, K. Kavukcuoglu, D. Silver, A. Graves, I. Antonoglou, D. Wierstra, M. Riedmiller, (2013). Playing Atari with deep reinforcement learning. arXiv.
    \bibitem{I4}  A. Graves, G. Wayne, I. Danihelka, (2014). Neural Turing Machines. arXiv.
    \bibitem{I5}  C.J.C.H.Watkins, P. Dayan, Q-learning. Mach Learn 8, 279–292 (1992).
    \bibitem{I6}  DeepMind, (2015, February 12), Deep reinforcement learning.
    \bibitem{I7}  T. Schaul, J. Quan, I. Antonoglou, D. Silver, (2015). Prioritized Experience Replay. arXiv.
    \bibitem{I8}  V. Mnih, A. P. Badia, M. Mirza, A. Graves, T. Lillicrap, T. Harley, D. Silver, K. Kavukcuoglu, (2016). Asynchronous Methods for Deep Reinforcement Learning. arXiv.
    \bibitem{I9}  A. Kailash, P. D. Marc, B. Miles, and A. B. Anil, (2017). Deep Reinforcement Learning: A Brief Survey. IEEE Signal Processing Magazine, vol. 34, pp. 26–38, 2017. arXiv.
    \bibitem{I10} D. Zhao,  K. Shao, Y. Zhu, D. Li, Y. Chen, H. Wang, D. Liu, T. Zhou, and C. Wang, “Review of deep reinforcement learning and discussions on the development of computer Go,” Control Theory and Applications, vol. 33, no. 6, pp. 701–717, 2016 arXiv.
    \bibitem{I11} Z. Tang, K. Shao, D. Zhao, and Y. Zhu, “Recent progress of deep reinforcement learning: from AlphaGo to AlphaGo Zero,” Control Theory and Applications, vol. 34, no. 12, pp. 1529–1546, 2017.
    \bibitem{I12} K. Shao, Z. Tang, Y. Zhu, N. Li, D. Zhao, (2019). A survey of deep reinforcement learning in video games. arXiv.
    
    %-------Background-------
    \bibitem{bg1} L. Thorndike and D. Bruce, Animal Intelligence. Routledge, 2017.
    \bibitem{bg2} R. S. Sutton and A. Barto, Reinforcement learning : an introduction. Cambridge, Ma ; London: The Mit Press, 2018.
    \bibitem{bg3} A. Kumar Shakya, G. Pillai, and S. Chakrabarty, “Reinforcement Learning Algorithms: A brief survey,” Expert Systems with Applications, vol. 231, p. 120495, May 2023
    \bibitem{bg4} Mnih, Volodymyr, et al. “Human-Level Control through Deep Reinforcement Learning.” Nature, vol. 518, no. 7540, Feb. 2015, pp. 529–533.
    \bibitem{Silver2016} D. Silver et al., “Mastering the game of Go with deep neural networks and tree search,” \textit{Nature}, vol. 529, no. 7587, pp. 484–489, Jan. 2016, doi: https://doi.org/10.1038/nature16961.
  %-------AlphaGo-------
    \bibitem{Silver2016} D. Silver et al., “Mastering the game of Go with deep neural networks and tree search,” \textit{Nature}, vol. 529, no. 7587, pp. 484–489, Jan. 2016, doi: https://doi.org/10.1038/nature16961.
    %-------AlphaGoZero-------
    \bibitem{agz1} Silver, David, Julian Schrittwieser, Karen Simonyan, Ioannis Antonoglou, Aja Huang, Arthur Guez, Thomas Hubert, et al. 2017. “Mastering the Game of Go without Human Knowledge.” Nature 550 (7676): 354–59. https://doi.org/10.1038/nature24270.

    %-------Muzero-------
    \bibitem{mz1} J. Schrittwieser et al., “Mastering Atari, go, chess and shogi by planning with a learned model,” Nature, vol. 588, no. 7839, pp. 604–609, Dec. 2020. doi:10.1038/s41586-020-03051-4 
    %-------Advancememts-------
    \bibitem{AD3} DeepMind, "AlphaZero: Shedding New Light on Chess, Shogi, and Go," DeepMind, 06-Dec-2018.
    \bibitem{AD1} T.-R. Wu, H. Guei, P.-C. Peng, P.-W. Huang, T. H. Wei, C.-C. Shih, Y.-J. Tsai, (2023). MiniZero: Comparative analysis of AlphaZero and MuZero on Go, Othello, and Atari games. arXiv.
    \bibitem{AD2} K. Zhang, Z. Yang, T. Ba\c{s}ar, (2021). Multi-agent reinforcement learning: A selective overview of theories and algorithms. arXiv preprint arXiv:2103.04994

    
    %-------Future Directions-------
    \bibitem{FD1} "MuZero’s first step from research into the real world," DeepMind, Feb. 11, 2022.
    \bibitem{FD2} Jumper, J et al. Highly accurate protein structure prediction with AlphaFold. Nature (2021)

\end{thebibliography}
\vspace{12pt}

\end{document}
