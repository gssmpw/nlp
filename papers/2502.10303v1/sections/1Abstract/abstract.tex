Reinforcement Learning (Rl) has been widely used in many applications, one of
these applications is the field of gaming, which is considered a very good
training ground for AI models. From the innovations of Google DeepMind in this
field using the reinforcement learning algorithms, including model-based,
model-free, and deep Q-network approaches, AlphaGo, AlphaGo Zero, and MuZero.
AlphaGo, the initial model, integrates supervised learning, reinforcement
learning to achieve master in the game of Go, surpassing the performance of
professional human players. AlphaGo Zero refines this approach by eliminating
the dependency on human gameplay data, instead employing self-play to enhance
learning efficiency and model performance. MuZero further extends these
advancements by learning the underlying dynamics of game environments without
explicit knowledge of the rules, achieving adaptability across many games,
including complex Atari games. In this paper, we reviewed the importance of
studying the applications of reinforcement Learning in Atari and strategy-based
games, by discussing these three models, the key innovations of each model,and
how the training process was done; then showing the challenges that every model
faced, how they encounterd them, and how they improved the performance of the
model. We also highlighted the advancements in the field of gaming, including
the advancment in the three models, like the MiniZero and multi-agent models,
showing the future direction for these advancements, and new models from Google
DeepMind.