%Introcution section%
Artificial Intelligence (AI) has revolutionized the gaming industry, both as a
tool for creating intelligent in-game opponents and as a testing environment
for advancing AI research. Games serve as an ideal environment for training and
evaluating AI systems because they provide well-defined rules, measurable
objectives, and diverse challenges. From simple puzzles to complex strategy
games, AI research in gaming has pushed the boundaries of machine learning and
reinforcement learning. Also The benfits from such employment helped game
developers to realize the power of AI methods to analyze large volumes of
player data and optimize game designs. \cite{I1} \\ Atari games, in particular,
with their retro visuals and straightforward mechanics, offer a challenging yet
accessible benchmark for testing AI algorithms. The simplicity of Atari games
hides complexity that they require strategies that involve planning,
adaptability, and fast decision-making, making them also a good testing
environment for evaluating AI's ability to learn and generalize. The
development of AI in games has been a long journey, starting with rule-based
systems and evolving into more sophisticated machine learning models. However,
the machine learning models had a few challenges, from these challenges is that
the games employing AI involves decision making in the game evironment. Machine
learning models are unable to interact with the decisions that the user make
because it depends on learning from datasets and have no interaction with the
environment. To overcome such problem, game developers started to employ
reinforcement learning (RL) in developing games. Years later, deep learning
(DL) was developed and shows remarkable results in video games\cite{I2}. The
combination between both reinforcement learning and deep learning resulted in
Deep Reinforcement Learning (DRL). The first employment of DRL was by game
developers in atari game\cite{I3}. One of the famous companies that employed
DRL in developing AI models is Google DeepMind. This company is known for
developing AI models, including games. Google DeepMind passed through a long
journey in developing AI models for games. Prior to the first DRL game model
they develop, which is AlphaGo, Google DeepMind gave a lot of contributions in
developing DRL, by which these contributions were first employed in Atari
games.\\ For the employment of DRL in games to be efficient, solving tasks in
games need to be sequential, so Google DeepMind combined RL-like techniques
with neural networks to create models capable of learning algorithms and
solving tasks that require memory and reasoning, which is the Neural Turing
Machines (NTMs)\cite{I4}. They then introduced the Deep Q-network (DQN)
algorithm, which is combine deep learning with Q-learning and RL algorithm.
Q-learning is a model in reinforcement learning which use the Q-network, which
is is a type of neural network to approximate the Q-function, which predicts
the value of taking a particular action in a given state\cite{I5}. The DQN
algorithm was the first algorithm that was able to learn directly from
high-dimensional sensory input, the data that have a large number of features
or dimensions\cite{I6}.\\ To enhance the speed of learning in reinforcement
learning agents, Google DeepMind introduced the concept of experience replay,
which is a technique that randomly samples previous experiences from the
agent's memory to break the correlation between experiences and stabilize the
learning process\cite{I7}. They then developed asynchronus methods for DRL,
which is the Actor-Critic (A3C) model. This model showed faster and more stable
training and showed a remarkable performance in Atari games\cite{I8}. By the
usage of these algorithms, Google DeepMind was able to develop the first AI
model that was able to beat the world champion in the game of Go, which is
AlphaGo in 2016.\\
The paper is organized as follows: Section II presents the related work that
surveys the development of DRL in games and the contribution that we 
added to the previous surveys. Section III presents the background 
information about the development of DRL in games. Section IV presents the
first AI model that Google DeepMind developed, which is AlphaGo. Section V
presents AlphaGo Zero. Section VI presents MuZero. Section VII presents the
advancements that were made in developing AI models for games. Section VIII 
presents the future directions that AI models for games will take, and their
applications in real life.