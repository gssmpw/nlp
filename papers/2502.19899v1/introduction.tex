\begin{figure}
\centerline{\includegraphics[width=0.375\textwidth]{figures/pullfigure.pdf}}
\caption{We evaluate \texttt{Z-COACH} on a high performance racing task in a simulated environment of the Thunderhill Raceway Park. \texttt{Z-COACH} identifies which task skills an individual student can only perform with assistance (i.e. within their Zone of Proximal Development), and then provides targeted coaching via skill-focused shared autonomy. Students receiving coaching from \texttt{Z-COACH} generally learned smoother racing lines than students practicing independently for the equivalent amount of time, as shown above by overlaying the trajectories from all participants in our human subject study ($n=50$).  }
\label{fig:pull}
\end{figure}

%motivation+ground driving both waht and how hard, what: current work zpd / student model darbitrary, we show how to use sa -zp, not only that sa, prior work
% Allude to other dynamical symptoms

Imagine a young adult who wishes to learn how to drive, borrowing her family's semi-autonomous vehicle to practice in their neighborhood. In real time, the vehicle estimates the skill capabilities and quality of this student's driving, while also identifying what driving skills are required for the road ahead. As she encounters novel driving situations, the vehicle's semi-autonomous control features are adjusted to ensure the driver is appropriately challenged. This assistance is personalized; while she may be overly cautious and needs to learn to let go of the brake pedal, another novice driver might be overconfident and stay constantly at a high throttle. Over time, as the driver achieves proficiency in one skill (e.g. steering),  the vehicle is able to identify and assist her in learning more complex control maneuvers. In this shared autonomy setting, the vehicle is providing tailored assistance not only for safety and comfort, but also to make sure the student \textit{improves} their driving skills. 

% Over time, the vehicle measures the student's progress, and gradually provides less assistance, until 
% After this \textit{student modeling} phase, the vehicle can then provide a \textit{coaching} mode: guiding her to specific locations to practice, providing assistance in all controls apart from steering, and measuring her performance over time. 
% \andrew{gains confidence and competence in this skill, and the vehicle gradually provides less and less assistance. Eventually, the student achieves proficiency in steering, and the vehicle is able to identify and assist her in learning more complex control maneuvers.}
%When looking at the possibilities afforded by autonomous systems assisting humans, sharing of autonomy is often considered as a boon that can aid people and make them safer, 

Shared autonomous control is a promising paradigm for humans to achieve near-optimal task performance by offloading some decision-making to an autonomous agent \citep{reddy2018shared, aigner1997shared}. However, similar to how generative AI tools can have negative long-term effects on student learning \cite{bastani2024generative}, a significant concern with shared autonomy is the gradual loss of human control skills due to over-reliance on intelligent and assistive systems~\cite{de-Winter2023-cp}. While prior work has proposed ``learning-compatible'' forms of shared autonomy \cite{bragg2020fake}, they do not consider complex domains where learning entails a structured progression through a hierarchy of skills (e.g. learning steering before mastering sharp turns).

Inspired by the notion of scaffolding within the broader education literature, we propose optimizing the form of intelligent assistance to enhance skill development at a level that is appropriately challenging for a student, aligned with the concept of the Zone of Proximal Development (ZPD)~\cite{vygotsky1978mind}. The ZPD is loosely defined as the gap between a student's actual developmental level, determined from independent problem-solving, and the level of potential development, based on their performance when problem-solving	with assistance ~\cite{vygotsky1978mind}. However, the impact of the \textit{type} of assistance within scaffolding teaching has traditionally been overlooked, often taking the form of simple interventions such as verbal feedback \cite{luckin1999ecolab}. In light of the increasing interaction between autonomous systems and humans, we take a new perspective with shared autonomy: can the way a student's behavior changes with AI assistance help inform an appropriate learning curriculum?

We propose \texttt{Z-COACH}, a framework for leveraging shared autonomy to aid with both student modeling (i.e. identifying what skills are within a student's ZPD) and coaching (i.e. helping a student improve at a skill) for arbitrary motor control tasks. Unlike prior work on AI-assisted coaching that guides students to practice the skill they find most difficult \cite{srivastava2022motor}, \texttt{Z-COACH} uses shared autonomy to identify skills which a student is most likely to improve at a given point in time. We then apply \texttt{Z-COACH} to the task of high performance racing (HPR) in simulation, and conduct a user study $(n=50)$ demonstrating that \texttt{Z-COACH} helps improve a student's driving time, behavior, and smoothness in comparison to a self-practice baseline. 
Our main contributions include:
\begin{itemize}
\item An approach to characterize a student's ZPD based on a shared autonomy assistance for motor control tasks, and a formulation of ZPD that take both assisted and unassisted student performance into account.
\item A method to decide when to apply shared autonomy to help a student improve a particular skill, based on improving the interpretability of existing unsupervised skill discovery algorithms used in \cite{srivastava2022motor}.
\item Demonstration of the proposed framework in a human-in-the-loop experiment with the CARLA Autonomous Driving simulator, demonstrating how optimizing the shared autonomy assistance based on the student's estimated ZPD results in greater student improvement in a high performance race training session based on the Thunderhill Raceway Park.
\end{itemize}    


% \begin{itemize}
% \item \guy{consider leveraging some of this:
% \item Imagine a teenage student who wishes to learn how to drive, and borrows her parents' semi-autonomous vehicle to practice in her local area. The vehicle can estimate how well is the student driving, and what driving skills are required for the road ahead. The vehicle can then assist the student in a tailored manner --- not just to keep them safe, but also to make sure they improve as they are assisted by the vehicle. Here the shared autonomy with the vehicle takes on the role of a personalized coach, adjusting the degree of difficulty for the driver to make sure they are at their learning best. The equivalent, if you will, of training wheels within the vehicle.
% \item When looking at the possibilities afforded by autonomous systems assisting humans, sharing of autonomy is often considered as a boon that can aid people and make them safer, yet requires careful use so as to avoid downsides such as skill atrophy \cite{de-Winter2023-cp} 
% % \deepak {I think this could also be connected to how shared autonomy is used in the context of rehabilitation.}
% \item Yet, careful consideration of the possibilities afforded by autonomy can lead rise to a different paradigm in which to consider the use of autonomy.
% \item Tuning of the role of autonomy within assistance allows us to transform the human experience in performing tasks, and tune it so as to enable more effective skill improvement, at a tailored level of difficulty.
% \item The Zone of Proximal Learning as been defined as "..", and has been underlying the notion of scaffolding within pedagogy. And yet, the role of choosing the form of assistance within scaffolded teaching has been overlooked within the role of autonomy sharing, and when discussing the interaction of autonomous systems and humans.
%     }
%     \item Driving instruction limited/stats/why its still importance even in the rise of autonomous vehicles (new motor skills challenges with semi-autonomous vehicles, safety situations, sports like performance racing)
%     \item Story: imagine a teenage driver .., the car has shared autonomous features that align with certain ``skills'' (e.g. break control, steering assistance). example of ZPD skill, example of too hard skill
%     \item Contributions: .% Code/eval release
% \begin{comment}

% \end{comment}
% \megha{balance bike metaphor?}
% \begin{itemize}
% \item An approach to characterize ZPD based on a shared autonomy assistance
% \item An approach of decided where to apply shared autonomy as assistance (among which axis, and whether to apply assistance)
% \item Demonstration of the proposed framework on human-in-the-loop experiment..
% \end{itemize}    
% \end{itemize}




% Imagine a teenage student who wishes to learn how to drive, and borrows her parents' semi-autonomous vehicle to practice in her local area. The vehicle could use multi-modal data to get a better understanding of the driving skills required in this area, including leveraging data from more experienced local drivers, reference trajectories from planning algorithms, or even publicly available image and language information about the surroundings (e.g. an infamous intersection). Furthermore, as the student tries different semi-autonomous features of the vehicle, she provides information about her comfort with different driving skills, such as struggling with control of the wheel during sharp left turns. 

% After this \textit{student modeling} phase, the vehicle can then provide a \textit{coaching} mode: guiding her to specific locations to practice, providing assistance in all controls apart from steering, and measuring her performance over time. Over time, the student \andrew{gains confidence and competence in this skill, and the vehicle gradually provides less and less assistance. Eventually, the student achieves proficiency in steering, and the vehicle is able to identify and assist her in learning more complex control maneuvers.}


