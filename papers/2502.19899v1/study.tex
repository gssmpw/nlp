\begin{figure*} 
\centerline{\includegraphics[width=0.8\textwidth]{figures/hrioverview.pdf}}
\caption{(Left) Image of our study environment for the High Performance Racing task in simulation. We use the CARLA simulator, with an external Logitech G29 steering wheel and pedals. (Middle) Image of the study interface used during coaching. Participants are provided control input information on a sidebar, given verbal guidance on which skill $z$ they should practice, and drive on the track from ego view. (Right) Participant breakdown of our user study. }
\label{fig:method}
\end{figure*}


Although \texttt{Z-COACH} can be applied to learning any motor task where shared autonomy control is feasible, we focus on coaching high-performance racing (HPR), due to the high likelihood of finding novice students and the availability of resources to help evaluate \texttt{Z-COACH}.  Using the open-source CARLA simulator \cite{dosovitskiy2017carla}, specifically designed for autonomous driving research, we simulate the 2-mile Thunderhill West track, part of Thunderhill Raceway Park in California, the venue for the longest automobile race in the United States (See Figure \ref{fig:pull} for example race lines on the race track).

\subsection{Task Description and Environment}
Our goal is to use  \texttt{Z-COACH} to improve novice students' performance when driving one lap around the Thunderhill West track. Strong performance in HPR does not only include a low elapsed time, but also measures such as staying on track and smoothness. Furthermore, the fastest race line around a given track is not necessarily the shortest distance path, particularly in the presence of turns and elevation changes. As such, HPR challenges drivers to perform a diverse set of skills, ranging from properly navigating a hairpin curve to simply maintaining high speed for novices. 

We provide images of our overall task environment and the simulation of the Thunderhill West track in Figure \ref{fig:method}. We modify the default vehicle physics control of the Toyota Prius provided by CARLA, including the torque curve, maximum revolutions per minute, and center of mass. Finally, our environment includes a Logitech G29 Driving Force steering wheel with pedals. 

\subsection{Shared Autonomy Design}
Recall that we use shared autonomy in \texttt{Z-COACH} for both student modeling and coaching. To create the autonomous agent $\pi_{agent}$ used in all shared autonomy modes, we first collected 5 ``expert'' demonstrations from a member of a local HPR club who is familiar with the Thunderhill track, and has extensive driving experience (including over 100 hours in simulation). When a novice student drives, at every time step we first find the nearest point (Euclidean distance over coordinates) in the expert trajectories. We then select a future state at a fixed interval (500 steps) in the expert's trajectory, and use that as the desination for a built-in Planner from CARLA. The planner generates waypoints for use with a PID controller given the current student's state $s$. We use the control outputs as actions for $\pi_{agent}$.

For student modeling, we consider two shared autonomy modes:
\begin{itemize}
    \item \textsc{StrongSA}: Following Equation \ref{eq:sa}, where $\alpha=0.8$.
    \item \textsc{WeakSA}: Following Equation \ref{eq:sa}, where $\alpha=0.05$.
\end{itemize}

We broadly refer to the skill-focused shared autonomy used in coaching as \textsc{SkillSA}, though in practice \textsc{SkillSA} follows Equation \ref{eq:sa} and sets $\alpha=0.8$ for all controls \textit{except} one of Throttle, Brake, or Steering, for which $\alpha=0$, depending on the skill identified to practice.

\begin{figure} 
\centerline{\includegraphics[width=0.3\textwidth]{figures/hriskill.pdf}}
\caption{Output segmentations produced by CompILE over an expert driver's trajectory around the Thunderhill West race track. By modifying CompILE to take as input noisy language annotations, the resulting segmentation is more interpretable and aligned with human notions of skills. }
\label{fig:skill}
\end{figure}

For all shared autonomy modes, if the driver goes off track above a threshold (e.g. into the grass hills in the simulation environment), we revert to full student control. Nevertheless, there is an important distinction between the shared autonomy used for student modeling (\textsc{StrongSA}, \textsc{WeakSA}) versus coaching (\textsc{SkillSA}), as the latter forces to the driver to focus on providing full control in one dimension (the skill recommended for practice by \texttt{Z-COACH})  while simultaneously adapting to the assistance along other skill dimensions. 

\subsection{Human Subject Study} We recruited 50 participants for a human subject study to evaluate \texttt{Z-COACH} (see Figure \ref{fig:method} for breakdown). The study aimed to observe improvements in driving performance over time, with a particular focus on lap completion, time, and avoiding crashes. Participants were majority university students (92\%) with no driving experience on a race track  (100\%), but had a wide range of experience with driving (0-30 hrs/week) and playing video games (0-10 hrs/week). We did not collect any further demographic information, and the study was approved by our Institutional Review Board. 


The entire study consisted of the following stages:
\begin{enumerate}
    \item \textbf{Baseline:} 2 x Unassisted Trials  
    \item 2 x Assisted Trials ( \textsc{StrongSA} or \textsc{WeakSA}).
    \item 1 x Unassisted Trial  
    \item 5 Minutes of Practice (Unassisted or \textsc{SkillSA})
    \item \textbf{Evaluation:}  3 x Unassisted Trial
\end{enumerate}

Apart from the 5-minute practice stage, the  subject's goal in each trial was to complete 1 lap of the race track before 3 minutes elapsed. During the practice session, participants had full control over how they approached the task and could reset to the starting point as needed. Throughout the study, subjects were presented with a visual guide showing the optimal path of an expert driver. Finally, for all participants, we recorded their driving trajectory starting from their first control input, and user interpolation was applied to normalize each trajectory to sample data every second.


We randomly assigned the first 20 participants to \textsc{StrongSA} or \textsc{WeakSA} for Stage 2, which, along with the trials in Stage 1 and 3, was used for student modeling following Equation 
\ref{eq:zpd}. As we will describe in Section \ref{sec:empzpd}, we empirically found that \textsc{StrongSA} led to stronger student modeling more aligned with an expert coach. The remaining 30 participants were therefore assigned to \textsc{StrongSA} for Stage 2, and randomly assigned to receive no assistance or \textsc{SkillSA} in Stage 4. The particular coaching intervention that \textsc{SkillSA} uses for each participant is based on $\mathsf{argmax}_{z \in (steer, brake, throttle)}\textsf{zpd}(z)$, or the skill that received the highest score in our student modeling based on that participant's trajectories in Stages 1-3. Note that while we set the size of set $\mathcal{Z}_g=8$, we restrict coach actions to only consider steer, brake, and throttle. 

Upon completing the study, all participants were asked to fill out a feedback form, where they reflected on the effectiveness of the five-minute practice session, and provided additional feedback on their experience with the simulator and assistance. We provide more details, including the full set of instructions participants received, in the Appendix. 

Overall, the structure of the study allowed us to assess the influence of shared autonomy on learning  by comparing each participant's \textbf{Baseline} and \textbf{Evaluation} rounds, as well as systematically evaluate each component of \texttt{Z-COACH}.


%RW: while the key focus of our work