\subsection{Shared Autonomy}

Existing research on shared autonomy includes developing adaptive control algorithms that adapt autonomy levels based on context or user performance~\cite{abbink2018topology,xing2020driver,gopinath2020active,zurek2021situational}, designing intuitive human-machine interfaces~\cite{da2021biasing,chen2022mirror}, and studying the impact of shared control on learning and skill acquisition~\cite{yu2023coach}. 
Applications span rehabilitation robotics assisting patients in regaining motor functions~\cite{saadah2004autonomy,okamura2010medical,argall2018autonomy}, 
remote teleoperation systems ~\cite{mower2021sharedcontrol}, 
advanced driving systems where control is shared between the driver and autonomous features to improve safety~\cite{wang2020review, marcano2020review, reitmann2024shared},
assistive devices for individuals with disabilities~\cite{losey2020latent,karamcheti2022lila,cui2023no,udupa2023shared}, 
and flight training simulators~\cite{byeon2024flight}. 
Another line of research explores machine learning techniques to enhance the efficiency and safety of shared control systems~\cite{li2019shared}. We refer to  \cite{wang2020review} for a survey of the full literature, which is beyond the scope of this paper.


% There exist several different methods for shared control between an automated driving agent and a human driver. Following the different levels by the SAE %https://www.rambus.com/blogs/driving-automation-levels/#level0
% HighwayDriving Assist, BlueCruise, Tesla (level 2) \megha{toyota example? guardian}

%\megha{cite refs from survey paper about shared control doing better in some situations}


% For example,    
% \cite{wang2020wang2020review} provides a comprehensive review of shared-control strategies in driving. 
% Beyond autonomous driving, researchers have explored shared-control in robot tele-operation  \cite{mower2021sharedcontrol}. 


%\cite{balasubramanian2010robot}


% bragg paper in learning, 
Within the human-robot interaction community, researchers have approached shared autonomy from a reinforcement learning perspective \cite{reddy2018shared}, sought to  perform inference over human preference and intent \cite{nemlekar2023robot, bobu2020less} as well as explicitly reason about human performance limits and the diverse state distributions that would be conducive for learning \cite{bragg2020fake}. However, these works do not explicitly account for scaffolding in learning. 

Closest to our work is the work of \citet{byeon2024flight}, which uses the Mahalanobis distance between novice and expert trajectories to update the parameters of a shared autonomy-based control for training novices in an urban air mobility task. In contrast, we propose leveraging shared autonomy not just during coaching, but also as a way to estimate which sub-skills of a task are most conducive to learning.
% \megha{Finish, also they give human control when things are similar}


\subsection{Student Modeling and Pedagogy}
Research on student modeling focuses on how to effectively model a student's learning capabilities in order to tailor personalized learning experiences.
This includes methods like Bayesian Knowledge Tracing to predict student knowledge states and performance~\cite{corbett2005kt, david2016sequencing}, as well as models of pedagogy, such as how demonstrators choose to act when teaching a student new skills \citep{ho2016showing, ho2018effectively, csibra2009pedagogy}. Two pedagogical concepts relevant to our work are \textit{interleaving}, where students practice multiple skills at once, and \textit{scaffolding}, where teachers provide temporary support to help students develop new skills \cite{vygotsky1978mind}.

\paragraph{Teaching Human Students}
There exists a large amount of interest in developing educational robot technology to aid with human learning. For example, \citet{chen2024integrating} studies how social robots can play a role in teaching scenarios as coaches or mock-students, and designers of Doodlebot, a mobile social robot, sought to provide scaffolding that encouraged students to draw more creatively \citep{william2024doodlebot}. A seperate line of work seeks to model human drivers' systematic suboptimality by comparing students to an optimal trajectory from an MPC controller to identify if students are under-steering, under-speed, over-steering, or over-speed~\cite{schrum2022mind,schrum2022reciprocal}. Finally, recent research  explores how robots can effectively collaborate with humans in cooperative tasks by adopting roles that facilitate human learning ~\cite{hou2023teachingbot,yu2023coach}. In contrast to these works, we seek to use shared autonomy to more explicitly consider a student's ZPD in complex tasks that require learning multiple skills.


\paragraph{Modeling the Zone of Proximal Development}
Several non-robotic systems explicitly leverage Vygotsky's Zone of Proximal Development (ZPD)—the gap between what a learner can do independently and with assistance~\cite{vygotsky1978mind}.  These include intelligent tutoring systems that use these models to adjust instruction difficulty within the learner's ZPD, providing appropriate challenges and support~\cite{clement2015multi, vainas2019gotsky, milani2020intelligent, chen2024integrating, ropelato2018adaptive}. Beyond human learning, some researchers applied concepts akin to ZPD to curriculum design for training autonomous agents ~\cite{seita2019zpd, wang2022zone, tio2023training, tzannetos2024proximal}. However, these modern interpretations of ZPD fail to factor in \emph{assistance} from a more knowledgable teacher, which deviates from the original definition by Vygostky~\cite{vygotsky1978mind}. By contrast, our work explicitly leverages an autonomous agent that shares control with the human learner as the form of assistance for identifying their ZPD. 



%\noident \textbf{Non-human specific} % select one or two to dive deeper
%ZPD has been leveraged in the RL literature for selecting training samples that are not too hard or too easy for learning agents~\cite{seita2019zpd, wang2022zone, tio2023training, tzannetos2024proximal}.



%\andrew{could cite \cite{schrum2022mind,schrum2022reciprocal} as examples of personalized learning of student's deficiencies or suboptimalities (MIND-MELD and Reciprocal MIND-MELD, works that learn personal embeddings after comparing students to an optimal trajectory from an MPC controller to identify if students are under-steering, under-speed, over-steering, or over-speed)}

%note novelty that we consider literal assistance

\paragraph{Skill Modeling}
Skill decomposition has been used for a variety of reasons in robotic planning and reinforcement learning \cite{sutton1999between,andreas2017sketches,shiarlis2018taco,fu2024language}. Likewise, skill decomposition has been show to be an important part of designing  training curriculum for human students in both traditional and, recently, AI-assisted teaching settings \cite{Magill2020-yz, srivastava2022motor}. With \texttt{Z-COACH}, we expand on this line of work to show how to decompose an arbitrary motor control task into a set of human-interpretable skills from noisy auxiliary information (e.g. noisy natural language captions from experts), which we select from when providing personalized instruction.

% \megha{is there a cite that skills enable better learning?}\guy{added \cite{Magill2020-yz}, chapter 18 - ``Whole and Part Practice'' }

