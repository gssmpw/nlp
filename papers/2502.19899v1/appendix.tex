%\appendix

%\section{Modified Car Physics Parameters}

\newpage
\newpage
~\newpage

\section{Vehicle Physics Control Parameters}

\begin{verbatim}
tire_front :
tire_friction=3.05, 
damping_rate=0.250000, 
max_steer_angle=69.999992, 
radius=37.000000,
max_brake_torque=3000, 
max_handbrake_torque=0.000000, 
lat_stiff_max_load=3.000000, 
lat_stiff_value=20.000000, 
long_stiff_value=3000.000000
        
tire_back : 
tire_friction=3.05, 
damping_rate=0.250000, 
max_steer_angle=0, 
radius=37.000000, 
max_brake_torque=3000, 
max_handbrake_torque=3000, 
lat_stiff_max_load=3.000000, 
lat_stiff_value=20.000000, 
long_stiff_value=3000.000000
        
Physics control for the car

torque_curve = 
- [0, 1000]
- [1890.760742, 900]
- [5729.577637, 700]
max_rpm = 8750
moi= 1.0
damping_rate_full_throttle= 0.150000
damping_rate_zero_throttle_clutch_engaged= 2.000000
damping_rate_zero_throttle_clutch_disengaged= 0.350000
use_gear_autobox= True
gear_switch_time= 0.000000
clutch_strength= 10.000000
final_ratio= 4.170000
mass= 1775.000000
drag_coefficient=0.300000
center_of_mass: [0.500000, 0.000000, -0.300000]   
steering_curve = 
- [0, 1.0]
- [20, 0.9]
- [60, 0.8]
- [120, 0.7]
\end{verbatim}


\section{User Study Instructions}

{\includegraphics[width=0.45\textwidth]{figures/userstudy.png}}

\section{Post-Study Google Form}

The purpose of this Google form was to gather participant feedback on their experiences with CARLA after the nine trials and self-practice sessions. The form collects data on the participants' driving habits, video game usage, the effectiveness of assistance provided during the trials, and their overall learning process. All questions were mandatory to answer.
{\includegraphics[width=0.45\textwidth]{figures/googleform.png}}



\section{Google Form: Racecar Driving Questionnaire}

Please answer the following questions!

\begin{enumerate}
    \item \textbf{What is your username?} \\
    Your answer

    \item \textbf{Do you have a driver's license?} \\
    (Choose Yes/No)

    \item \textbf{Have you ever raced a car on any professional racetrack?} \\
    (Choose Yes/No)

    \item \textbf{Have you ever raced a car on the Thunderhill racetrack?} \\
    (Choose Yes/No)

    \item \textbf{Around how many hours do you spend driving per week?} \\
    Your answer

    \item \textbf{Around how many hours do you spend playing video games per week?} \\
    Your answer

    \item \textbf{How would you describe the assistance you received in Trials 3 and 4 (before the practice session)?} \\
    Your answer

    \item \textbf{How helpful did you find the assistance you received in Trials 3 and 4 (before the practice session)?} \\
    Not helpful at all [1] [2] [3] [4] [5] Very helpful

    \item \textbf{To what degree did the assistance you received in Trials 3 and 4 (before the practice session) challenge you?} \\
    Did not challenge me at all [1] [2] [3] [4] [5] Challenged me a lot

    \item \textbf{How helpful did you find the 5-minute practice session?} \\
    Not helpful at all [1] [2] [3] [4] [5] Very helpful

    \item \textbf{To what degree did the assistance you received in the 5-minute practice session challenge you?} \\
    Did not challenge me at all [1] [2] [3] [4] [5] Challenged me a lot

    \item \textbf{How did you use the 5-minute practice session? (e.g. did you focus on improving any particular part of the task?)} \\
    Your answer

    \item \textbf{What is something you learned about the task over time that helped you?} \\
    Your answer

    \item \textbf{What other feedback do you have about the study?} \\
    Your answer
\end{enumerate}


\section{Feedback Dataset}

The dataset used for the skill discovery component was created from a subset of the data collected looking at naturalistic teaching.  In this data set, participants (n=15) received up to 90 minutes of performance-driving coaching on a driving simulator from a professional coach. The coach was instructed to 1) help students become better drivers and 2) use primarily verbal feedback. Participants drove around a virtual version of the  Thunderhill West track, located in Northern California. Each 15 minutes students filled out surveys to assess their cognitive load and emotional state. Each 15 minutes, the coach filled out surveys assessing the student's skill and improvement.  We used data from a single participant to use for model validation. 

This study was reviewed and approved by an IRB (name to be entered after review). Prior to participating, participants provided with written informed consent. Afterwards, the coach provided each participant with a simulator safety and conduct lesson and drove a site lap. The participants were then prompted to drive two laps around to gauge their baseline driving skill before the training began. Then, participants, depending on time, completed three to five 15-minute sessions with the coach, where the lessons were guided by what skills the coach, and occasionally the students, thought the student needed to work on. Each session was divided by both the participant and the coach filling out a survey about how the session went. 

After the sessions were completed, the participant and coach filled out a final survey about the overall experience of the training. The study ended with the participant completing 2 final laps around the track with no coaching to gauge how much their skill improved. 

With this study structure, we were able to collect the trajectory and behavioral data for each participant, which was crucial for helping validate the model.
