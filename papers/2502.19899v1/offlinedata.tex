
\megha{@Emily @Laporsha TODO} Details of data collection effort, including: 
1. Number of students, and any known demographic stats
2. Number of coaches, 1 coach per student?, etc.
3. Number of trials 
4. Instructions (will later move to Appendix)
5. What coaches were asked exactly


The dataset used for the offline component of our model was created from a subset of the data collected looking at naturalistic teaching.  In this data set, participants (n=15) received up to 90 minutes of performance-driving coaching on a driving simulator from a professional coach. The coach was instructed to 1) help students become better drivers and 2) use primarily verbal feedback. Participants drove around a virtual version of the  Thunderhill West track, located in Northern California. Each 15 minutes students filled out surveys to assess their cognitive load and emotional state. Each 15 minutes, the coach filled out surveys assessing the student's skill and improvement.  We used data from a single participant to use for model validation. 

This study was reviewed and approved by an IRB (name to be entered after review). Prior to participating, participants provided with written informed consent. Afterwards, the coach provided each participant with a simulator safety and conduct lesson and drove a site lap. The participants were then prompted to drive two laps around to gauge their baseline driving skill before the training began. Then, participants, depending on time, completed three to five 15-minute sessions with the coach, where the lessons were guided by what skills the coach, and occasionally the students, thought the student needed to work on. Each session was divided by both the participant and the coach filling out a survey about how the session went. 

After the sessions were completed, the participant and coach filled out a final survey about the overall experience of the training. The study ended with the participant completing 2 final laps around the track with no coaching to gauge how much their skill improved. 

With this study structure, we were able to collect the trajectory and behavioral data for each participant, which was crucial for helping validate the model.