%language consistence of compile, track segments, time heuristic, lang-compile
%coach alignment


%Jker:

%Min Jerk https://arxiv.org/pdf/2102.07459
%steering entropy nakayam


%matt, nick are pretty good



\begin{figure*}
\centerline{\includegraphics[width=0.95\textwidth]{figures/hriresults.pdf}}
\caption{ (Left) Comparison of overall performance between Baseline trials, Expert trials, and Evaluation trials from participants either assigned Self-Practice or receiving assistance via \texttt{Z-COACH}, across a variety of quantitative metrics highlights improvement from \texttt{Z-COACH}. (Middle) Fourier transform signal analysis plot of the steering wheel input shows that \texttt{Z-COACH} guides students towards similar steering behavior as an expert HPR driver, including decreasing the amount of large, low frequency turns. (Right) Participant feedback shows that \textsc{SkillSA}, used for coaching, is found challenge participants  significantly more  than \textsc{StrongSA}, yet comparable in helpfulness. * marks statistical significance ($p < 0.05$ with a paired t-test.) }
\label{fig:results}
\end{figure*}


Recall that \texttt{Z-COACH} consists of three steps: (i) task skill discovery, (ii) student modeling with shared autonomy to estimate how much a skill is within a student's ``zone of proximal development'', and (iii)  using skill-focused shared autonomy to help the student improve. We now evaluate each step. 
\subsection{Interpretable Skill Discovery}\label{results:skills} As described in Section \ref{sec:skill}, we wish to identify the different skills required in our HPR task in order to provide effective coaching interventions. Unfortunately, simply running an unsupervised skill discovery algorithm may return segments corresponding to latent skills $z \in \mathcal{Z}_g$, but we have no way of interpreting what those skills represent, making it tricky to provide focused skill-coaching via shared autonomy. 

We therefore augment our dataset of expert trajectories with noisy language annotations, a form of auxiliary information as in Equation \ref{eq:compile}. Weak supervision in the form of language describing actions appears in a variety of places, including coaching videos on YouTube, or user-generated descriptions on traffic applications. For our task, we use a small subset of data collected and shared by the Toyota Research Institute. Their dataset consists of survey and driving data from 15 participants having one-on-one simulator performance driving training from 1 racing coach on the Thunderhill race track, and includes language feedback such as \textit{``straighten the wheel''} or \textit{``off the gas here''} that corresponds to a particular $(s,a)$ in a student's trajectory. We use data from only \textit{one} participant, and provide further information about this dataset in the Appendix.

Our first step is to cluster the noisy language annotations into semantically meaningful groups, so that we can use the cluster index as the auxiliary information in Equation \ref{eq:compile}. We use the llama-3 language model to generate cluster mappings and descriptions by providing the following input string: 
\begin{quote}
\small
\texttt{You will be given a list of feedback to a driving student. Please cluster them into N skills. The driver can control a throttle, steering wheel, and brake. Return a dictionary that maps each string to a cluster ID, and another dictionary that maps  the cluster ID to its description.}  
\end{quote}

We set $N=8$, and show in Table \ref{tab:clusters} the output descriptions and example feedback for each cluster. Several of the clusters (e.g. Braking, Throttle, Steering) are well-aligned with the task action space $\mathcal{A}$, making it feasible to consider shared autonomy for skill-focused coaching. 

However, in order to identify \textit{which} skill is appropriate for a student to learn, we need to map these clusters to trajectory segments. Following the approach described in Section \ref{sec:skill}, we train CompILE on expert trajectories with the modified loss function in \ref{eq:compile}, again using $N=8$ as a hyperparameter, and let the auxiliary information $\psi(s)$ equal the cluster ID (e.g. 6) of the language feedback provided at the state closest to $s$ in position. The resulting skill segmentation, shown in Figure \ref{fig:skill}, successfully groups together parts of the track that requires high throttle without any steering, as well as the entry into sharp turns. Furthermore, when compared with training CompILE without weak supervision, our approach significantly improves the compression ratio ($\textbf{1.83}$ vs. $\textbf{1.22}$) of the resulting segmentation, with little impact on Mean Square Error ($\textbf{0.037}$ vs. $\textbf{0.037}$) of the reconstruction. Finally, we can identify which Cluster ID, and therefore language feedback, is most associated with a given latent $z$ by examining the output logits of the decoder $p_\theta$  in CompILE (see Section \ref{sec:skill}). 

Overall, our results show success in identifying semantically meaningful skills in task trajectories, which we will use for student modeling. 


\begin{table}[htbp]\label{tab:clusters}
\caption{Unsupervised Clustering of Verbal Feedback}
\begin{center}
\begin{tabular}{cc}
% \hline
\textbf{Cluster Description}&  \textbf{Example Feedback}  \\
 \hline
Braking & \textit{hold the brake hard} \\
% \hline
Lane positioning & \textit{stay to the right}\\
% \hline
Steering control & \textit{little bit of steering} \\
% \hline
Car handling  & \textit{let the car push out} \\
% \hline
Throttle control & \textit{now squeeze the gas}\\
% \hline
Navigation &  \textit{up the hill}\\
% \hline
Target aiming & \textit{aim for the right} \\
% \hline
Encouragement & \textit{there you go} \\
% \hline

\end{tabular}
\label{tab1}
\end{center}
\end{table}
 

\begin{figure}
\centerline{\includegraphics[width=0.35\textwidth]{figures/hribar.pdf}}
\caption{For participants assigned either Weak and Strong Shared Autonomy during the Student Modeling stage, incorporating their performance and behavior when driving with assistance led to higher accuracy in predicting which skill a certain driver should be focused on learning next. For ground truth labels, we showed videos of the student driver trajectories to a professional HPR coach, who provided ranked ordering of skills per participant.  * denotes statistical significance ($p < 0.05$) using Welch's t-test.}
\label{fig:zpd}
\end{figure}


\subsection{Estimating Students' Zone of Proximal Development} \label{sec:empzpd}

We next evaluate the benefit of using shared autonomy for student modeling. We screen captured videos of 20 participants, split evenly between receiving \textsc{StrongSA} or \textsc{WeakSA} in Stage 2 of our study, and hired a professional high performance racing coach (compensated 75\$ per hour) to watch each video and provide ground truth rankings for the order in which he would coach the student to focus on the following three skills: Steering, Braking, Throttle. While there was variation between participants, the coach shared that most participants needed to improve their Throttle control before even trying to practice Braking, indicating that the Braking skill is less likely than Throttle to be within a student's ``Zone of Proximal Development''. Indeed, only two participants were labeled as needing to first focus on practicing braking.


We next analyze rankings produced by $\mathsf{zpd}(z)$ for each of the 8 possible $z \in \mathcal{Z}_g$. We average all assisted trajectories $\tau_{SA}$ from Stage 2 and baseline trajectories $\tau_{student}$ from Stage 1 before calculating  $\mathsf{zpd}(z)$, using the Dynamic Time Warp distance between trajectories from the student and our reference expert as $\textsf{score}$. We find that participants assigned \textsc{StrongSA} in Stage 2 had more than 90\% accuracy when comparing with the ground truth labels from the coach (See Figure \ref{fig:zpd}). Meanwhile, when $\mathsf{zpd}(z)$ is calculated without a term capturing the student's performance with assistance (i.e. replacing $\mathsf{score}(\mathsf{align}(\tau_{SA}, \text{seg}_i))$ with a constant in Equation \ref{eq:zpd}), the accuracy significantly degrades. Surprisingly, the benefit of measuring $\mathsf{zpd}(z)$ with  \textsc{WeakSA} assistance is not as strong, despite using more student inputs. This might be due to user frustration with less helpful assistance affecting their performance. Nevertheless, incorporating student performance with \textsc{WeakSA} assistance when calculating $\mathsf{zpd}(z)$ still improves accuracy over the constant case, and our overall results confirm that shared autonomy does indeed lead to a stronger model of what skills a student should next focus on learning. 



\begin{table}[htbp]\label{tab:metrics}
\caption{Mean Change Between Baseline and Evaluation Trials}
\begin{center}
\begin{tabular}{ccc}
% \hline
\textbf{Metric}&  \textbf{Self-Practice}  &  \texttt{\textbf{Z-COACH}} \\
\hline
$\Delta$ Success Rate $\uparrow$  & 0.07 [-0.20, 0.33] & 0.28 [0.0632, 0.4924] \\
% \hline
$\Delta$ Lap Progress $\uparrow$  & 0.22 [-0.35, 0.80] & 0.69 [0.08, 1.29] \\
% \hline
$\Delta$ Lap Time (s) $\downarrow$  & -11 [-23.8, 2.6]*
  & \textbf{-29 [-38.7, -19.5] *} \\
% \hline
$\Delta$ Consistency $\uparrow$ & 2.6 [-2.18, 7.38] & 2.4 [-0.19, 5.02] \\ 
% \hline
$\Delta$ Expert Distance $\downarrow$&  -1.11 [-2.47, 0.33] & -5.39 [-13.48, 2.70] \\
% \hline
$\Delta$ Jerk $\downarrow$ & 2.1 [-3.51, 7.85] * & \textbf{-6.6  [-12.16, -0.99] *} \\ 
% \hline
%$\Delta$ Jerk (steering) & -0.09 [-0.20, -0.01] & -0.14 [-0.2, -0.08]\\ \hline
$\Delta$ $\#$ Lane Invasions $\downarrow$ & -3.4 [-21.2, 14.4] * & \textbf{-33.5 [-49.1, -17.9] *} \\ 
% \hline
\end{tabular}
\footnotesize{* denotes a statistical significant difference between mean change of Self-Practice and \texttt{Z-COACH} conditions ($p < 0.05$, using Welch's t-test)}
\label{tab1}
\end{center}
\end{table} 

\subsection{Improving Student Learning}
Finally, we seek to evaluate whether the skill-focused coaching aspect of \texttt{Z-COACH} does indeed help improve student performance. Recall that participants were randomly selected to either practice independently for 5-minutes, or provided \textsc{SkillSA} assistance which forces drivers to take full control on either Steering, Throttle, or Brake, depending on their skill level. Figure \ref{fig:method} shows an example coaching interface for a participant.

Between both groups (self-practice or \textsc{SkillSA}), we compare each student's performance change between the Stage 5 evaluation trials and the Stage 1 baseline trials, with respect to the following evaluation metrics:
\begin{itemize}
    \item \textbf{Success Rate}: Average \% of times the driver completed a full lap under the 3 minutes time-limit
    \item \textbf{Lap Progress}: Proportion of the lap covered
    \item \textbf{Lap Time}: Time taken if the driver completed the full track
    \item \textbf{Expert Distance}: Distance between the student and expert trajectories, measured with Dynamic Time Warp \citep{giorgino2009dtw}
    \item \textbf{Consistency}: The mean distance between trajectories for each pair of trials within the same stage, measured with Dynamic Time Warp \citep{giorgino2009dtw} 
    \item \textbf{Jerk}: Change in magnitude of jerk, or the rate of change in acceleration over time, a measure of trajectory smoothness
    \item \textbf{\# Lane Invasions}: Number of times the vehicle crosses lane lines and goes off track
\end{itemize}

Table \ref{tab:metrics} shows that across all metrics, students provided skill-focused coaching with \texttt{Z-COACH} not only improved over time, but also by  a stronger amount than students who had to self-practice (except for $\Delta$ Consistency). Using a Welch's t-test to test for statistical significance at $p < 0.05$ with Bonferroni correction, we find significant effects with \texttt{Z-COACH} for improvement in lap time, jerk, and average number of lane invasions, suggest that \texttt{Z-COACH} helps students drive faster, more smoothly, and with more control over staying on the race track. 
Beyond improving student performance, Figure \ref{fig:results} shows the degree to which \texttt{Z-COACH} reaches expert performance across different metrics, highlighting that there is still room for further improvement from coaching for smoothness (jerk) and reducing the amount of lane invasions. 

We further analyze how \texttt{Z-COACH} affects student steering behavior by providing a Fourier analysis of steering angle in Figure \ref{fig:results}. We observed during the user study that novices often took large, ``swinging'' turns, which corresponds to the high amplitude and low frequency values for the Baseline trials. While self-practice does reduce such behavior, the curve for \texttt{Z-COACH} has the lowest RMSE when compared with the expert, suggesting that \texttt{Z-COACH} best leads students towards optimal steering behavior. 

\subsection{Student Feedback}
While \textsc{SkillSA} leads to stronger learning outcomes compared to self-practice, we were interested in how participants viewed skill-focused shared autonomy with the \textsc{StrongSA} assistance they received during the student modeling stage of their study. We asked all participants to rate (range 1 to 5, with 5 indicating agreement) how \textit{helpful} they found the assistance, and to what degree the assistance \textit{challenged} them. As Figure \ref{fig:results} shows, while there was no significant difference in how helpful participants perceived both assistant types, they found that \textsc{SkillSA} significantly challenged them more (significance at  $p<0.05$ determined with a paired t-test). This observation is further supported by their free-text responses; for example, one participant described the \textsc{StrongSA} assistance received during  student modeling as \textit{``it was hard for me to know how to improve my driving skills"}, but then described the \textsc{SkillSA} assistance it received during coaching as \textit{``This more interactive mode helps me more to get a sense when I should adjust my previous strategy on throttle''}. 

Meanwhile, another participant who received \textsc{SkillSA} assistance targeting steering shared that \textit{``I had a lot of difficulty, and  didn't feel like I was better when doing it. But afterwards in the evaluation trials I swear I had improved my steering''}. These results raise the interesting question of whether coaching interventions that help a student learn are actually perceived as helpful at the time, and what properties to consider when developing shared autonomy assistance intended to help students learn. 
