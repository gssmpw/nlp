
\documentclass{elsarticle}
\usepackage[utf8]{inputenc}
\usepackage{amsthm}
\usepackage{float}
\usepackage{graphicx}
\usepackage{ulem}
\usepackage{mathdots}
\usepackage{verbatim}
\usepackage{amsmath}
\usepackage{amssymb}
\usepackage{xcolor}
\usepackage{subfig}
\usepackage{amsfonts}
\title{A Derivation and Exploration of the Equality Between the Cycle-lengths of 1 and 2 Dimensional $\sigma$ Automata}
\author{Avi Vadali, Ari M. Turner}
\newtheorem{thm}{Theorem}[section]
\newtheorem{lma}{Lemma}[section]
\newtheorem{corollary}{Corollary}[thm]
\theoremstyle{definition}
\newtheorem{dfn}{Definition}
\newtheorem{ident}{Identity}

\date{April 2022}

\begin{document}

\maketitle

\subsection{Relating the cycle lengths of $\Phi$ and $\sigma$}
\label{sec:CL_sigma}

\par We now algebraically show that $CL(\Phi) = CL(\sigma)$.
To prove that the cycle lengths of $\Phi$ and $\sigma$ are equal we will need to show that the least common multiple of the orders of the eigenvalues is the same. The eigenvalues of $A$ and $T$ are
different, but there is still a relationship between them.

\begin{thm}
\label{thm:sum=prod}
For any $\Phi$ automaton where $n$ is even, the sum of any two distinct eigenvalues of $A$ is equal to the product of $2$ eigenvalues of $A$.
\end{thm}

\begin{proof}
By theorem \ref{thm:eigensum},
the nonzero eigenvalues are $\lambda_j=\alpha^j+\alpha^{-j}$, for $j$ not a multiple of $b$.  If $1\leq j\leq \frac{b-1}{2}$, then each eigenvalue appears once. 

Say we want to sum two distinct eigenvalues, e.g. $\lambda_i$ and $\lambda_{i+j}$.
If $j = 2k$ for an integer $k$,
then 
\begin{align}
    \lambda_i+\lambda_{i+2k}&=\alpha^i+\alpha^{-i}+\alpha^{i+2k}+\alpha^{-i-2k}\nonumber\\
    &=(\alpha^{i+k}+\alpha^{-i-k})(\alpha^{k}+\alpha^{-k})\nonumber\\
    &=\lambda_{i+k}\lambda_k.
    \label{eq:sumproduct}
\end{align}
Because the eigenvalues being added are distinct, $\lambda_k$ and $\lambda_{i+k}$ are nonzero.

If $j$ is odd, this does not work immediately.  However, since $\lambda_{k} = \lambda_{b - k}$, we can rewrite $\lambda_i + \lambda_{i+j}$ as $\lambda_i + \lambda_{i + 2m}$ where $m$ is some integer, so this is equal to $\lambda_{i+m}\lambda_m$.
\end{proof}

\par  We now consider two groups.  These are
the smallest groups $G_A$ and $G_T$ under multiplication that contain all nonzero eigenvalues of $A$ and $T$ respectively.  Because multiplication in $\mathbb{F}$ is commutative, the least common multiple of the orders of elements in each of the groups is the least common multiple of the orders of the eigenvalues.  Thus, by proving that the groups are the same, we show that the lcm of the orders of the eigenvalues of $A$ and $T$ are equal.

By the previous theorem, $G_T\subset G_A$.
If $n$ is even, then $A$ has zero as an eigenvalue, so $T$'s eigenvalues, the sums $\lambda+\lambda'$ of all pairs of eigenvalues of $A$, include all of $A$'s eigenvalues (by taking $\lambda'=0$).
Therefore $G_A=G_T$ in that case.

We now prove that $G_A=G_T$ when $n$ is even, except for two exceptions.

\begin{thm}
Let $n$ be even. For an $n \times 1$ $\Phi$ automaton and an $n \times n$ $\sigma$ automaton with $n \geq 0$, the lcm of the orders of the eigenvalues of $A$ and $T$ will be equal, unless $n=2$ or 4.
\label{thm:eig_lcm_equal}
\end{thm}

\begin{proof}
\par First off, because $T = A \oplus A$, the eigenvalues of $T$ are simply all possible sums of $2$ eigenvalues of $A$, which means that the eigenvalues of $T$ are all products of two eigenvalues of $A$ (by Theorem \ref{thm:sum=prod}).
This implies that $G_T\subset G_A$.
To show that all eigenvalues of $A$ are contained within $G_T$, we first show that $\lambda_1\in G_T$.  First
 $\lambda_1+\lambda_3=\lambda_1\lambda_2$ (by eq. \ref{eq:sumproduct}).
 But $\lambda_2=\lambda_1^2$, so $\lambda_1^3\in G_T$.  Also $\lambda_3+\lambda_5=\lambda_1\lambda_4=\lambda_1^5$, so $\lambda_1^5\in G_T.$
 Thus $\lambda_1 ^ 3 \lambda_1 ^3 \lambda_1 ^ {-5} = \lambda \in G_T$ since $G_T$ is closed under multiplication. 

 This argument can break down if $n+1=3$ or 5, since then $\lambda_3$ or $\lambda_5=0$. 
 \end{proof}

Now we consider the cycle lengths for $\Phi$ and $\sigma$.
\begin{thm}
An $n \times 1$ $\Phi$ automaton and an $n \times n$ $\sigma$ automaton have the same cycle length, except for $n=2,4$.
\label{thm:CL_equal}
\end{thm}

\label{jb_period_1}

\begin{proof}
Say $n+1=2^ab$ and the cycle length is $2^st$ where $b$ and $t$ are odd.

As we have mentioned throughout this paper, the cycle-lengths of $\Phi$ and $\sigma$ are the lcm of the periods of the Jordan blocks of $A$ and $T$. Jordan blocks with zero as the eigenvalue have a cycle length of 1 since all powers of them larger than their size are constant (zero).  So if all eigenvalues are zero, the cycle length is 1. This happens when $n+1=2^a$ or in two dimensions when $n+1=3$ (the one dimensional automaton for $n+1=3$ has only one eigenvalue, so all eigenvalues of the two dimensional automaton are equal to twice it.)  For the former case, the automata in both dimensions have cycle length 1 (and eventually all the lights turn off, without any strategy being used (ref. to Sutner I think?). When $n+1=3$, the one dimensional automaton has a cycle length of two, different from the 2-d automaton.

Otherwise, there are nonzero eigenvalues. Theorem \ref{jb:_period_one} implies that $s$ is determined by the size of the largest Jordan block with a nonzero eigenvalue and $t$ is the least common multiple of the orders of the eigenvalues. 
Lemmas \ref{lma:jordan_eigen_sum} \ref{lma:A_jordan_sum} show that for each nonzero eigenvalue, the largest Jordan block has the same size $2^a$ (in 1-d and 2-d) so $s=a$ in both dimensions.
Theorem \label{thm:eig_lcm_equal} implies that the lcm of their eigenvalues are identical except when $n=4$, so $b=t$ and
$CL(\Phi)=CL(\sigma)$.
\end{proof}

appendix showing there is actually a state whose cycle length is CL--it is not just the lcm of all cycle lengths.


%general result about power of 2 dividing
%cycle length of $A+_K A$ for any matrix $A$.

\end{document}
