%%%%%%%% ICML 2025 EXAMPLE LATEX SUBMISSION FILE %%%%%%%%%%%%%%%%%

\documentclass{article}

% Recommended, but optional, packages for figures and better typesetting:
\usepackage{microtype}
\usepackage{graphicx}
\usepackage{subcaption}

% \usepackage{subfigure}
\usepackage{booktabs} % for professional tables
\usepackage{arydshln}
\usepackage{ulem}
\usepackage{float}
\usepackage{multirow}
\usepackage{amsmath, amsfonts, amssymb}
\usepackage{hyperref}

% hyperref makes hyperlinks in the resulting PDF.
% If your build breaks (sometimes temporarily if a hyperlink spans a page)
% please comment out the following usepackage line and replace
% \usepackage{icml2025} with \usepackage[nohyperref]{icml2025} above.

\usepackage{xcolor}

\definecolor{ForestGreen}{rgb}{0.0, 0.5, 0.0}
\definecolor{NavyBlue}{rgb}{0, 0.44, 0.75}
\newcommand{\hgreen}[1]{\textcolor{ForestGreen}{\textbf{#1}}} % highlight color
\newcommand{\hblue}[1]{\textcolor{NavyBlue}{\textbf{#1}}} % highlight color


% Attempt to make hyperref and algorithmic work together better:
\newcommand{\theHalgorithm}{\arabic{algorithm}}

% Use the following line for the initial blind version submitted for review:
% \usepackage{icml2025}

% If accepted, instead use the following line for the camera-ready submission:
\usepackage[accepted]{icml2025}

% For theorems and such
\usepackage{amsmath}
\usepackage{amssymb}
\usepackage{mathtools}
\usepackage{amsthm}
\usepackage{xspace}

\usepackage{xcolor}
\definecolor{Graylight}{gray}{0.9}
\definecolor{Gray}{gray}{1.0}
\usepackage{colortbl}
\usepackage{multirow}
\usepackage{makecell}
\usepackage{bm}

\usepackage{pifont}
\definecolor{mygreen}{RGB}{9,136,66}
\newcommand{\cmark}{\textcolor{mygreen}{\ding{51}}}%
\newcommand{\xmark}{\textcolor{red!50}{\ding{55}}}%

% if you use cleveref..
\usepackage[capitalize,noabbrev]{cleveref}

\newcommand{\methodname}{VideoRoPE\xspace}
\newcommand{\modelname}{Qwen2-VL-VRoPE\xspace}

%%%%%%%%%%%%%%%%%%%%%%%%%%%%%%%%
% THEOREMS
%%%%%%%%%%%%%%%%%%%%%%%%%%%%%%%%
\theoremstyle{plain}
\newtheorem{theorem}{Theorem}[section]
\newtheorem{proposition}[theorem]{Proposition}
\newtheorem{lemma}[theorem]{Lemma}
\newtheorem{corollary}[theorem]{Corollary}
\theoremstyle{definition}
\newtheorem{definition}[theorem]{Definition}
\newtheorem{assumption}[theorem]{Assumption}
\theoremstyle{remark}
\newtheorem{remark}[theorem]{Remark}

% Todonotes is useful during development; simply uncomment the next line
%    and comment out the line below the next line to turn off comments
%\usepackage[disable,textsize=tiny]{todonotes}
\usepackage[textsize=tiny]{todonotes}


% The \icmltitle you define below is probably too long as a header.
% Therefore, a short form for the running title is supplied here:
\icmltitlerunning{\methodname: What Makes for Good Video Rotary Position Embedding?}

\begin{document}

\twocolumn[
\icmltitle{\methodname: What Makes for Good Video Rotary Position Embedding?}

% It is OKAY to include author information, even for blind
% submissions: the style file will automatically remove it for you
% unless you've provided the [accepted] option to the icml2025
% package.

% List of affiliations: The first argument should be a (short)
% identifier you will use later to specify author affiliations
% Academic affiliations should list Department, University, City, Region, Country
% Industry affiliations should list Company, City, Region, Country

% You can specify symbols, otherwise they are numbered in order.
% Ideally, you should not use this facility. Affiliations will be numbered
% in order of appearance and this is the preferred way.
\icmlsetsymbol{equal}{*}

\begin{icmlauthorlist}
\icmlauthor{Xilin Wei}{equal,fdu,ailab}
\icmlauthor{Xiaoran liu}{equal,fdu,ailab,inno}
\icmlauthor{Yuhang Zang}{ailab}
\icmlauthor{Xiaoyi Dong}{ailab}
\icmlauthor{Pan Zhang}{ailab}
\icmlauthor{Yuhang Cao}{ailab}
\icmlauthor{Jian Tong}{ailab}
\icmlauthor{Haodong Duan}{ailab}
\icmlauthor{Qipeng Guo}{ailab,inno}
\icmlauthor{Jiaqi Wang}{ailab,inno}
\icmlauthor{Xipeng Qiu}{fdu,ailab,inno}
\icmlauthor{Dahua Lin}{ailab}
\end{icmlauthorlist}

\icmlaffiliation{fdu}{Fudan University, Shanghai, China}
\icmlaffiliation{ailab}{Shanghai AI Laboratory, Shanghai, China}
\icmlaffiliation{inno}{Shanghai Innovation Institute, Shanghai, China}
% \icmlaffiliation{sch}{School of ZZZ, Institute of WWW, Location, Country}

% \icmlcorrespondingauthor{Firstname1 Lastname1}{first1.last1@xxx.edu}
% \icmlcorrespondingauthor{Firstname2 Lastname2}{first2.last2@www.uk}

% You may provide any keywords that you
% find helpful for describing your paper; these are used to populate
% the "keywords" metadata in the PDF but will not be shown in the document
% \icmlkeywords{Machine Learning, ICML}

\vskip 0.3in
]

% this must go after the closing bracket ] following \twocolumn[ ...

% This command actually creates the footnote in the first column
% listing the affiliations and the copyright notice.
% The command takes one argument, which is text to display at the start of the footnote.
% The \icmlEqualContribution command is standard text for equal contribution.
% Remove it (just {}) if you do not need this facility.

% \printAffiliationsAndNotice{}  % leave blank if no need to mention equal contribution
\printAffiliationsAndNotice{\icmlEqualContribution} % otherwise use the standard text.

\begin{abstract}


The choice of representation for geographic location significantly impacts the accuracy of models for a broad range of geospatial tasks, including fine-grained species classification, population density estimation, and biome classification. Recent works like SatCLIP and GeoCLIP learn such representations by contrastively aligning geolocation with co-located images. While these methods work exceptionally well, in this paper, we posit that the current training strategies fail to fully capture the important visual features. We provide an information theoretic perspective on why the resulting embeddings from these methods discard crucial visual information that is important for many downstream tasks. To solve this problem, we propose a novel retrieval-augmented strategy called RANGE. We build our method on the intuition that the visual features of a location can be estimated by combining the visual features from multiple similar-looking locations. We evaluate our method across a wide variety of tasks. Our results show that RANGE outperforms the existing state-of-the-art models with significant margins in most tasks. We show gains of up to 13.1\% on classification tasks and 0.145 $R^2$ on regression tasks. All our code and models will be made available at: \href{https://github.com/mvrl/RANGE}{https://github.com/mvrl/RANGE}.

\end{abstract}



\section{Introduction}

Rotary Position Embedding (RoPE) \cite{su2024roformer} helps Transformer models understand word order by assigning each token a unique positional `marker' calculated using a mathematical rotation matrix.
RoPE has advantages in long-context understanding \cite{ding2024longrope}, and continues to be a default choice in leading Large Language Models (LLMs) like the LLaMA \cite{touvron2023llamaopenefficientfoundation,touvron2023llama,dubey2024llama} and QWen \cite{yang2024qwen2,yang2024qwen25} series.

The original RoPE implementation (Vanilla RoPE) \cite{su2024roformer} is designed for sequential 1D data like text. However, recent Video Large Language Models (Video LLMs) \cite{2023videochat,lin2023video,chen2024sharegpt4video,maaz2024videochatgptdetailedvideounderstanding,zhang2024longva,wang2024longllavascalingmultimodalllms,chen2024longvilascalinglongcontextvisual,internlmxcomposer2_5_OL} process video, which has a more complex spatio and temporal structure.
As shown in Tab. \ref{tab:pe_compare}, although several RoPE-based approaches \cite{gao2024tc,wang2024qwen2} have been proposed to support video inputs, these variants exhibit limitations and do not fully satisfy the following key characteristics:

\begin{table}[t]
\tiny
% \small
\centering
\tabcolsep=0.1cm
\begin{tabular}{lcccc}
\toprule
 & \makecell[c]{\textbf{2D/3D} \\ \textbf{Structure}} & \makecell[c]{\textbf{Frequency} \\ \textbf{Allocation}} & \makecell[c]{\textbf{Spatial} \\ \textbf{Symmetry}} & \makecell[c]{\textbf{Temporal} \\ \textbf{Index Scaling}} \\
\midrule
\makecell[l]{Vanilla RoPE \cite{su2024roformer}} & \xmark & \xmark & \xmark & \xmark \\
\makecell[l]{TAD-RoPE \cite{gao2024tc}} & \xmark & \xmark & \xmark & \cmark \\
\makecell[l]{RoPE-Tie \cite{kexuefm10040}} & \cmark & \xmark & \cmark & \xmark \\
\makecell[l]{M-RoPE \cite{wang2024qwen2}} & \cmark & \xmark & \xmark & \xmark \\
% \makecell[l]{V2PE\\\cite{ge2024v2pe}} & \xmark & \xmark & \xmark & \cmark \\
\midrule
\rowcolor[HTML]{F2F3F5}
\methodname (Ours) & \cmark & \cmark & \cmark & \cmark \\
\bottomrule
\end{tabular}
\vspace{-6pt}
\caption{Comparison between different RoPE variants for Video Large Language Models (Video LLMs).}
\label{tab:pe_compare}
\vspace{-12pt}
\end{table}

\begin{figure}
\centering
\includegraphics[width=0.9\linewidth]{figures/files/radar.pdf}
\vspace{-6pt}
\caption{\methodname outperforms RoPE variants on benchmarks.}
\label{fig:radar}
\vspace{-12pt}
\end{figure}

\begin{figure*}[ht]
    \centering
    \includegraphics[width=0.96\linewidth]{figures/files/niah.pdf}
    \vspace{-6pt}
    \caption{\footnotesize \textbf{Left:} To demonstrate the importance of frequential allocation, based on VIAH (\textbf{a}) we present a more challenging V-NIAH-D task (\textbf{b}) that similar images are inserted as distractors.
    \textbf{Right:} Compared to M-RoPE, our \methodname is more robust in retrieval and is less affected by distractors.
    }
    \label{fig:v-ruler}
    \vspace{-12pt}
\end{figure*}

\textbf{(1) 2D/3D Structure.} Some existing Video LLMs direct flatten the video frame into 1D embeddings and apply the 1D structure RoPE \cite{su2024roformer,gao2024tc}.
These solutions fail to capture video data's inherent 2D or 3D (temporal ($t$), horizontal ($x$), and vertical ($y$)) structure, thus hindering explicit spatial and temporal representation.

% whether the feature dimension can process the semantic relationship where it is responsibility~\cite{peng2023yarn,barbero2024round,liu2024kangaroo},
% scaling the understanding capabilities of these models, enabling them to comprehend longer videos and, ultimately, bring them closer to human-level understanding.

\textbf{(2) Frequency Allocation.} Previous approaches such as M-RoPE used in QWen2-VL \cite{wang2024qwen2} employ 3D structure, dividing the feature dimensions into distinct subsets for ($t$, $x$, $y$) encoding, respectively.
How to determine the optimal allocation of these dimension subsets, and consequently their associated frequencies 
\footnote{In RoPE, frequencies are determined by $\beta^{-2n/d}$, where $\beta$ is a constant, $n$ is the dimension index, $d$ is the total number of dimensions. Thus, choosing which dimensions represent $t$, $x$, and $y$ directly determines the frequencies used for each.} are not well studied.
Some previous work allocates the lower dimensions corresponding to the high frequency to represent the $t$.
However, the temporal dimension $t$ is significantly tortured by periodic oscillation, and distant positions may have the same embeddings.

We present a simple setting to verify this point.
Based on the previous long-video retrieval task V-NIAH (Visual Needle-In-A-Haystack) \cite{zhang2024longva}, we insert several similar images that do not affect the question's answer before and after the needle image as distractor \cite{hsieh2024ruler,yuan2024lv}, forming a new task, V-NIAH-D (Visual Needle-In-A-Haystack with Distractors).
As shown in Fig. \ref{fig:v-ruler}, we find that previous M-RoPE is misled by distractors, showing a significant performance decline from V-NIAH to V-NIAH-D.
Our observation demonstrates that the periodic oscillation reduces Video LLMs' robustness.

\textbf{(3) Spatial Symmetry.} The distance between the end of the precedent textual input and the start of visual input equals the distance between the end of visual input and the start of subsequent textual input~\cite{kexuefm10352}. Such a symmetry ensures that the visual input receives equal contextual influence from both the preceding and subsequent textual information.

\textbf{(4) Temporal Index Scaling.} Spatial and temporal dimensions often exhibit different granularities (e.g., a unit change in $x$/$y$ differs from a unit change in $t$) \cite{gao2024tc}.
Employing varying index intervals in positional encoding allows for dimension-specific encoding, capturing diverse scales and enhancing efficiency.

Driven by our analysis, we present a new video position embedding strategy, \textbf{\methodname}, which can simultaneously satisfy the four properties in Tab. \ref{tab:pe_compare}.
Specifically, we use a 3D structure to model spatiotemporal information, allocating higher dimensions (lower frequencies), to the temporal axis (\textbf{L}ow-frequency \textbf{T}emporal \textbf{A}llocation, \textbf{LTA}) to prioritize temporal modeling.
The right panel of Fig. \ref{fig:v-ruler} demonstrates that our LTA allocation mitigates oscillations and exhibits robustness to distractors in the V-NIAH-D task.
We further employ a \textbf{D}iagonal \textbf{L}ayout (\textbf{DL}) design to ensure spatial symmetry and preserve the relative positioning between visual and text tokens.
Regarding temporal index scaling, we propose \textbf{A}djustable \textbf{T}emporal \textbf{S}pacing (\textbf{ATS}), where a hyper-parameter controls the relative temporal spacing of adjacent visual tokens.
In summary, our proposed position encoding scheme demonstrates favorable characteristics for modeling video data, yielding a robust and effective representation of positional information.

Overall, the contributions of this work are summarized as:

\textbf{(1)} We present an analysis of four key properties essential for RoPE when applied to video. Motivated by this analysis, we propose \methodname including Low-frequency Temporal Allocation (LTA), Diagonal Layout (DL), and Adjustable Temporal Spacing (ATS) to satisfy all four properties.

% Inspired by the evaluation of long-context LLM,
\textbf{(2)} We introduce the challenging V-NIAH-D task to expose the drawbacks of current position embedding designs regarding frequency allocation. Our findings reveal that existing Video LLMs are easily misled to frequency-based distractors.

\textbf{(3)} Extensive experiments demonstrate that \methodname consistently achieves superior performance compared to other RoPE variants. For example, \methodname outperforms previous M-RoPE on long video retrieval (\textbf{+12.4} on V-NIAH, \textbf{+12.4} on V-NIAH-D), video understanding (\textbf{+2.9} on LongVideoBench, \textbf{+4.5} on MLVU, \textbf{+1.7} on Video-MME) and hallucination (\textbf{+11.9} on VideoHallucer) benchmarks.

\begin{figure*}[t]
  \centering
    \includegraphics[width=1\linewidth]{visuals/final_registration.png}
    \caption{Target measurement process on low-cost scan data using ICP and Coloured ICP. (1) Initialisation: The source point cloud (checkerboard) is misaligned with the target point cloud. (2) Initial Registration using Point-to-Plane ICP: Standard ICP leads to suboptimal registration. (3) Final Registration using Coloured ICP: Colour information is incorporated after pre-processing with RANSAC and Binarisation with Otsu Thresholding for real data, resulting in improved alignment.}
    \label{fig:Registration_visualisation}
\end{figure*}

\subsection{Iterative Closest Point (ICP) Algorithm}
The Iterative Closest Point (ICP) algorithm has been a fundamental technique in 3D computer vision and robotics for point cloud. Originally proposed by \cite{besl_method_1992}, ICP aims to minimise the distance between two datasets, typically referred to as the source and the target. The algorithm operates in an iterative manner, identifying correspondences by matching each source point with its nearest target point \citep{survey_ICP}. It then computes the rigid transformation, usually involving both rotation and translation, to achieve the best alignment of these matched points \citep{survey_ICP}. This process is repeated until convergence, where the change in the alignment parameters or the overall alignment error becomes smaller than a predefined threshold.

One key advantage of the ICP framework lies in its simplicity: the algorithm is conceptually straightforward, and its basic version is relatively easy to implement. However, traditional ICP can be sensitive to local minima, often requiring a good initial alignment \citep{zhang2021fast}. Furthermore, outliers, noise, and partial overlaps between datasets can significantly degrade its performance \citep{zhang2021fast, bouaziz2013sparse}. Over the years, various modifications and improvements \citep{gelfand2005robust, rusu2009fast, aiger20084, gruen2005least, fitzgibbon2003robust} have been proposed to mitigate these issues. Among the most common strategies are robust cost functions \citep{fitzgibbon2003robust}, weighting schemes for correspondences \citep{rusu2009fast}, and more sophisticated techniques \citep{gelfand2005robust, bouaziz2013sparse} to reject outliers. 

In addition, there is significant interest in integrating additional information into the ICP pipeline. Instead of solely relying on geometric cues such as point coordinates or surface normals, recent approaches have proposed incorporating colour (RGB) or intensity data to enhance correspondence accuracy. These methods \citep{park_colored_2017, 5980407}, commonly known as "Colored ICP" employ differences in pixel intensities or colour values as additional constraints. This is particularly beneficial in situations where geometric attributes alone are inadequate for accurate alignment or where surfaces possess complex texture patterns that can assist in the matching process.

\subsection{Applications of Target Measurement}

One approach relies on the use of physical checkerboard targets for registration. \cite{fryskowska2019} analyse checkerboard target identification for terrestrial laser scanning. They propose a geometric method to determine the target centre with higher precision, demonstrating that their approach can reduce errors by up to 6 mm compared to conventional automatic methods.

\cite{becerik2011assessment} examines data acquisition errors in 3D laser scanning for construction by evaluating how different target types (paper, paddle, and sphere) and layouts impact registration accuracy in both indoor and outdoor environments and presents guidelines for optimal target configuration.

\citet{Liang2024} propose to use Coloured ICP to measure target centres for checkerboard targets, similar to our investigation. They use data from a survey-grade terrestrial laser scanner. Their intended application is structural bridge monitoring purposes. They report an average accuracy of the measurement below 1.3 millimetres.

Where targets cannot be placed in the scene, the intensity information form the scanner can still be used to identify distinctive points. For point cloud data that is captured with a regular pattern, standard image processing can be used in a similar way to target detection. For example, \citet{wendt_automation_2004} proposes to use the SUSAN operator on a co-registered image from a camera, \citet{bohm_automatic_2007} proposes to use the SIFT operator on the LIDAR reflectance directly and \citet{theiler_markerless_2013} propose to use a Difference-of-Gaussian approach on the reflectance information.
Most of these methods extract image features to find reliable 3D correspondences for the purpose of registration.

In the following we describe our approach to the measurement of the target centre. In contrast to most proposed methods above we focus on unordered point clouds, where raster-based methods are not available, and low-cost sensors, where increased measurement noise and outliers are expected. As we are not aware of a commercial reference solution to this problem, we start by conducting a series of synthetic experiments to explore the viability and accuracy potential of the approach.



%The reviewed studies primarily rely on physical targets or target-free methods and do not utilise 3D synthetic point cloud checkerboards. In contrast, our approach introduces synthetic point cloud checkerboards, which offer controlled and consistent target geometry and reduce variability caused by physical targets. This innovation has significant potential for commercialisation and industrial application.


\section{Analysis}

\textbf{3D Structure.}
The vanilla RoPE defines a matrix $\bm{A}_{t_1,t_2}$ that represents the relative positional encoding between two positions $t_1$ and $t_2$ in a 1D sequence:
\begin{equation}\label{eq:vanilla_rope}
% \vspace{-6pt}
\begin{aligned}
% \small
\bm{A}_{t_1,t_2}&=\left(\bm{q}_{t_1}\bm{R}_{t_1}\right){\left(\bm{k}_{t_2}\bm{R}_{t_2}\right)}^\top
% = \bm{q}_{t_1}\bm{R}_{t_1}\bm{R}_{t_2}^\top\bm{k}_{t_2}^\top
= \bm{q}_{t_1}\bm{R}_{\Delta t}\bm{k}_{t_2}^\top,
\end{aligned}
% \vspace{-6pt}
\end{equation}
where $\Delta t=t_1-t_2$, the symbols $\bm{q}_{t_1}$ and $\bm{k}_{t_2}$ are the query and key vectors at positions $t_1$ and $t_2$.
The \textit{relative rotation matrix} $\bm{R}_{\Delta t}$ is defined as $\bm{R}_{\Delta t} = \exp(\Delta ti\theta_{n})$, while $i$ is the imaginary unit, $\theta_{n} = \beta^{-2n/d}$ is the frequency of rotation applied to a specific $n$-th pair of $d$ dimensions ($n=0,\ldots,d/2-1$), and $\beta$ is the frequency base parameter.
The vanilla RoPE uses $d=128$, thus $n=0,\ldots,63$.
Consequently, the $\bm{A}_{t_1,t_2}$ in Eq. (\ref{eq:vanilla_rope}) can be extended as:
% \begin{equation}\label{equ:rope}
% \vspace{-6pt}
% \resizebox{0.5\textwidth}{!}{
% \scriptsize
% \begin{gathered}
% \begin{pmatrix}
% q^{(0)}\\q^{(1)}\\\vdots\\q^{(126)}\\q^{(127)}
% \end{pmatrix}^\top
% \begin{pmatrix}\cos{\theta_0\Delta t}& -\sin{\theta_0\Delta t}&\cdots&0&0\\ \sin{\theta_0\Delta t}&\cos{\theta_0\Delta t}&\cdots&0&0 \\ \vdots&\vdots&\ddots&\vdots&\vdots\\ 0&0&\cdots&\cos{\theta_{63}\Delta t}&  \sin{\theta_{63}\Delta t}\\ 0&0&\cdots&\sin{\theta_{63}\Delta t}&\cos{\theta_{63}\Delta t}
% \end{pmatrix}
% \begin{pmatrix}
% k^{(0)}\\k^{(1)}\\\vdots\\k^{(126)}\\k^{(127)}
% \end{pmatrix}.
% \end{gathered}
% }
% \end{equation}
\begin{equation}\label{equ:rope}
\vspace{-6pt}
\resizebox{0.5\textwidth}{!}{$
\scriptsize
\left(
\begin{array}{c}
q^{(0)}\\q^{(1)}\\\vdots\\q^{(126)}\\q^{(127)}
\end{array}
\right)^{\top}
\left(
\begin{array}{ccccc}
\cos{\theta_0\Delta t} & -\sin{\theta_0\Delta t} & \cdots & 0 & 0 \\ 
\sin{\theta_0\Delta t} & \cos{\theta_0\Delta t} & \cdots & 0 & 0 \\ 
\vdots & \vdots & \ddots & \vdots & \vdots \\  
0 & 0 & \cdots & \cos{\theta_{63}\Delta t} &  \sin{\theta_{63}\Delta t} \\  
0 & 0 & \cdots & \sin{\theta_{63}\Delta t} & \cos{\theta_{63}\Delta t} 
\end{array}
\right)
\left(
\begin{array}{c}
k^{(0)}\\k^{(1)}\\\vdots\\k^{(126)}\\k^{(127)}
\end{array}
\right)
$}
\end{equation}



While the vanilla RoPE operates on 1D sequences, it can also be applied to higher-dimensional input by flattening the input into a 1-D sequence.
However, the flattening process discards crucial neighborhood information, increases the sequence length, and hinders the capture of long-range dependencies.
Therefore, preserving the inherent 3D structure is essential when adapting RoPE for video data.
Some recent RoPE-variants (e.g., M-RoPE in Qwen2-VL \cite{wang2024qwen2}) incorporate the $3$D structure.
The corresponding relative matrix $\bm{A}_{(t_1,x_1,y_1)}$ is computed as:
\begin{equation}
% \small
\bm{A}_{(t_1,x_1,y_1),(t_2,x_2,y_2)}=\bm{q}_{(t_1,x_1,y_1)}\bm{R}_{\Delta t,\Delta x,\Delta y}\bm{k}_{(t_2,x_2,y_2)}^\top,
\end{equation}
where $\Delta t=t_1-t_2$, $\Delta x=x_1-x_2$, and $\Delta y=y_1-y_2$.
M-RoPE divides the $d=128$ feature dimensions into 3 groups: the first 32 for temporal positions ($t$), the middle 48 for horizontal positions ($x$), and the last 48 for vertical positions ($y$). As shown in Eq~(\ref{equ:mrope}), $\bm{A}_{(t_1,x_1,y_1),(t_2,x_2,y_2)}$ in M-RoPE is extended as:
\begin{equation}
\vspace{-6pt}
\resizebox{0.5\textwidth}{!}{$
\scriptsize
\begin{gathered}
\underbrace{\begingroup
\setlength\arraycolsep{1pt}
\begin{pmatrix}q^{(0)}\\q^{(1)}\\q^{(2)}\\q^{(3)}\\\vdots\\q^{(30)}\\q^{(31)}\end{pmatrix}^\top
\begin{pmatrix}
% \setstacktabbedgap{2pt}
\cos{\theta_0\Delta t}& -\sin{\theta_0\Delta t}&0&0&\cdots&0&0\\
\sin{\theta_0\Delta t}&\cos{\theta_0\Delta t}&0&0&\cdots&0&0 \\
0&0&\cos{\theta_1\Delta t}& -\sin{\theta_1\Delta t}&\cdots&0&0\\
0&0&\sin{\theta_1\Delta t}&\cos{\theta_1\Delta t}&\cdots&0&0 \\ 
\vdots&\vdots&\vdots&\vdots&\ddots&\vdots&\vdots\\
0&0&0&0&\cdots&\cos{\theta_{15}\Delta t}& -\sin{\theta_{15}\Delta t}\\
0&0&0&0&\cdots&\sin{\theta_{15}\Delta t}&\cos{\theta_{15}\Delta t}
\end{pmatrix}
\begin{pmatrix}k^{(0)}\\k^{(1)}\\k^{(2)}\\k^{(3)}\\\vdots\\k^{(30)}\\k^{(31)}\end{pmatrix}
\endgroup}_\text{\normalsize modeling temporal dependency with higher frequency} \\
+ \underbrace{\begingroup
\setlength\arraycolsep{1pt}
\begin{pmatrix}q^{(32)}\\q^{(33)}\\q^{(34)}\\q^{(35)}\\\vdots\\q^{(78)}\\q^{(79)}\end{pmatrix}^\top
\begin{pmatrix}
% \setstacktabbedgap{2pt}
\cos{\theta_{16}\Delta x}& -\sin{\theta_{16}\Delta x}&0&0&\cdots&0&0\\
\sin{\theta_{16}\Delta x}&\cos{\theta_{16}\Delta x}&0&0&\cdots&0&0 \\
0&0&\cos{\theta_{17}\Delta x}& -\sin{\theta_{17}\Delta x}&\cdots&0&0\\
0&0&\sin{\theta_{17}\Delta x}&\cos{\theta_{17}\Delta x}&\cdots&0&0 \\ 
\vdots&\vdots&\vdots&\vdots&\ddots&\vdots&\vdots\\
0&0&0&0&\cdots&\cos{\theta_{39}\Delta x}& -\sin{\theta_{39}\Delta x}\\
0&0&0&0&\cdots&\sin{\theta_{39}\Delta x}&\cos{\theta_{39}\Delta x}
\end{pmatrix}
\begin{pmatrix}k^{(32)}\\k^{(33)}\\k^{(34)}\\k^{(35)}\\\vdots\\k^{(78)}\\k^{(79)}\end{pmatrix}
\endgroup}_\text{\normalsize modeling horizontal dependency with intermediate frequency} \\
+ \underbrace{\begingroup
\setlength\arraycolsep{1pt}
\begin{pmatrix}q^{(80)}\\q^{(81)}\\q^{(82)}\\q^{(83)}\\\vdots\\q^{(126)}\\q^{(127)}\end{pmatrix}^\top
\begin{pmatrix}
% \setstacktabbedgap{2pt}
\cos{\theta_{40}\Delta y}& -\sin{\theta_{40}\Delta y}&0&0&\cdots&0&0\\
\sin{\theta_{40}\Delta y}&\cos{\theta_{40}\Delta y}&0&0&\cdots&0&0 \\
0&0&\cos{\theta_{41}\Delta y}& -\sin{\theta_{41}\Delta y}&\cdots&0&0\\
0&0&\sin{\theta_{41}\Delta y}&\cos{\theta_{41}\Delta y}&\cdots&0&0 \\ 
\vdots&\vdots&\vdots&\vdots&\ddots&\vdots&\vdots\\
0&0&0&0&\cdots&\cos{\theta_{63}\Delta y}& -\sin{\theta_{63}\Delta y}\\
0&0&0&0&\cdots&\sin{\theta_{63}\Delta y}&\cos{\theta_{63}\Delta y}
\end{pmatrix}
\begin{pmatrix}k^{(80)}\\k^{(81)}\\k^{(82)}\\k^{(83)}\\\vdots\\k^{(126)}\\k^{(127)}\end{pmatrix}
\endgroup}_\text{\normalsize modeling vertical dependency with lower frequency}
\end{gathered}
$}
\label{equ:mrope}
\end{equation}
% \begin{equation}
% \vspace{-6pt}
% \resizebox{0.5\textwidth}{!}{$
% \scriptsize
% \underbrace{
% \left(
% \begin{array}{c}
% q^{(0)}\\q^{(1)}\\\vdots\\q^{(31)}
% \end{array}
% \right)^{\top}
% \left(
% \begin{array}{cccccc}
% \cos{\theta_0\Delta t} & -\sin{\theta_0\Delta t} & \cdots & 0 \\ 
% \sin{\theta_0\Delta t} & \cos{\theta_0\Delta t} & \cdots & 0 \\ 
% \vdots & \vdots & \ddots & \vdots \\ 
% 0 & 0 & \cdots & \cos{\theta_{15}\Delta t} 
% \end{array}
% \right)
% \left(
% \begin{array}{c}
% k^{(0)}\\k^{(1)}\\\vdots\\k^{(31)}
% \end{array}
% \right)
% }_{\text{\normalsize modeling temporal dependency with higher frequency}}
% +
% \underbrace{
% \left(
% \begin{array}{c}
% q^{(32)}\\q^{(33)}\\\vdots\\q^{(79)}
% \end{array}
% \right)^{\top}
% \left(
% \begin{array}{cccccc}
% \cos{\theta_{16}\Delta x} & -\sin{\theta_{16}\Delta x} & \cdots & 0 \\ 
% \sin{\theta_{16}\Delta x} & \cos{\theta_{16}\Delta x} & \cdots & 0 \\ 
% \vdots & \vdots & \ddots & \vdots \\ 
% 0 & 0 & \cdots & \cos{\theta_{39}\Delta x} 
% \end{array}
% \right)
% \left(
% \begin{array}{c}
% k^{(32)}\\k^{(33)}\\\vdots\\k^{(79)}
% \end{array}
% \right)
% }_{\text{\normalsize modeling horizontal dependency with intermediate frequency}}
% +
% \underbrace{
% \left(
% \begin{array}{c}
% q^{(80)}\\q^{(81)}\\\vdots\\q^{(127)}
% \end{array}
% \right)^{\top}
% \left(
% \begin{array}{cccccc}
% \cos{\theta_{40}\Delta y} & -\sin{\theta_{40}\Delta y} & \cdots & 0 \\ 
% \sin{\theta_{40}\Delta y} & \cos{\theta_{40}\Delta y} & \cdots & 0 \\ 
% \vdots & \vdots & \ddots & \vdots \\ 
% 0 & 0 & \cdots & \cos{\theta_{63}\Delta y} 
% \end{array}
% \right)
% \left(
% \begin{array}{c}
% k^{(80)}\\k^{(81)}\\\vdots\\k^{(127)}
% \end{array}
% \right)
% }_{\text{\normalsize modeling vertical dependency with lower frequency}}
% $}
% \end{equation}


\noindent \textbf{Frequency Allocation.}
% Note that the frequency encoding in vanilla RoPE (Eq. \ref{equ:rope}) assigns higher frequencies (via larger $\theta_{n}$ values) to lower dimensions.
Incorporating 3D structure raises the question of how to allocate the temporal ($t$), horizontal ($x$), and vertical ($y$) components within the $d$ dimensions.
Note that different allocation strategies are not equivalent in the rotation frequency $\theta_{n} = \beta^{-2n/d}$.
As shown in Eq. (\ref{equ:mrope}), M-RoPE assigns higher frequencies (corresponding to lower dimensions) to the temporal dimension ($t$).

To highlight the importance of frequency allocation, we introduce a challenging retrieval task \textbf{V}isual \textbf{N}eedle-\textbf{I}n-\textbf{A}-\textbf{H}astack-\textbf{D}istractor (\textbf{V-NIAH-D}).
V-NIAH-D builds upon V-NIAH \cite{zhang2024longva}, a benchmark designed to evaluate visual long-context understanding.
However, the straightforward retrieval-based task has been shown to provide only a superficial form of long-context understanding~\cite{hsieh2024ruler,yuan2024lv}.
Therefore, We enhance V-NIAH by incorporating semantically similar distractors, obtained using Google Image Search~\cite{googleimagesearch} or Flux ~\cite{flux2023}, to mitigate the possibility of correct answers through random chance.
These distractors are designed to be unambiguous to the question in Fig. \ref{fig:v-ruler}.

\begin{figure}[t]
\centering
\includegraphics[width=.94\linewidth]{figures/files/attention_analysis.pdf}
\vspace{-6pt}
\caption{\footnotesize Attention-based frequential allocation analysis.
\textbf{Middle}: M-RoPE's temporal dimension ($t$) is limited to local information, resulting in a diagonal layout.
\textbf{Bottom}: \methodname effectively retrieves the needle using the temporal dimension.
The x and y coordinates represent the video frame number, e.g., 50 for 50 frames.
For more details see Appendix \ref{app:attention_analysis}.
% We use 8k-context input, with video tokens from the same frame aggregated via average pooling.
}
\vspace{-12pt}
\label{fig:attention_analysis}
\end{figure}



As shown in Fig. \ref{fig:v-ruler}, M-RoPE exhibits a clear performance drop from V-NIAH to V-NIAH-D. To investigate this decline, we follow previous works \citep{xiao2023efficient,liu2023scaling,barbero2024round} to visualize the attention scores in Fig. \ref{fig:attention_analysis}. We decompose the attention scores into their corresponding temporal ($t$), horizontal ($x$), and vertical ($y$) components for visualization.

Fig.~ \ref{fig:attention_analysis} reveals unusual attention patterns in M-RoPE, despite its ability to locate the needle image but fails to answer the multi-choice question.
According to the attention of M-RoPE, the needle is located primarily through vertical positional information, rather than temporal features.
Thus, the temporal dimension fails to capture long-range semantic dependencies, focusing instead on local relationships.
Conversely, the spatial dimensions exhibit a tendency to capture long-range rather than local semantic information.
Lastly, the horizontal and vertical dimensions display distinct characteristics, with the vertical dimension exhibiting phenomena reminiscent of attention sinks \cite{xiao2023efficient}.
These observations suggest that the performance decline primarily results from the sub-optimal frequency allocation designs of M-RoPE.

\noindent \textbf{Spatial Symmetry.} Given the text tokens $T$ and the visual tokens $T_v$, spatial symmetry \cite{kexuefm10352} claims that the distance between the end of the preceding textual input ($T_{\text{pre}}$) and the beginning of the visual input ($T_v^{\text{start}}$) is equal to the distance between the end of the visual input ($T_v^{\text{end}}$) and the beginning of the subsequent textual input ($T_{\text{sub}}$):
\begin{equation}
    T_{v}^{\text{start}} - T_{\text{pre}} =
    T_{\text{sub}} - T_{v}^{\text{end}}.
\end{equation}
The spatial symmetrical structure can potentially simplify the learning process and reduce bias toward input order.
However, existing 3D RoPE variants such as M-RoPE do not meet the spatial symmetry, we will elaborate related discussion in Fig. \ref{fig:spatial}.

\begin{figure*}[t]
\begin{minipage}{0.98\textwidth}
    \begin{subfigure}[b]{0.49\linewidth}
        \centering
\includegraphics[width=0.95\linewidth]{figures/files/video_rope-period_low-MRoPE.pdf}
        \caption{Temporal Frequency Allocation in M-RoPE}
        \label{fig:temporal_mrope}
    \end{subfigure}
    \hfill
    \begin{subfigure}[b]{0.49\linewidth}
        \centering
        \includegraphics[width=0.95\linewidth]{figures/files/video_rope-period_low-VideoRoPE.pdf}
        \caption{Temporal Frequency Allocation in \methodname (ours)}
        \label{fig:temporal_videorope}
    \end{subfigure}
    \vspace{-6pt}
    \caption{\footnotesize \textbf{(a)} M-RoPE \cite{wang2024qwen2} models temporal dependencies using the \textit{first} 16 rotary angles, which exhibit higher frequencies and more pronounced oscillations. \textbf{(b)} In contrast, \methodname models temporal dependencies using the \textit{last} 16 rotary angles, characterized by significantly wider, monotonic intervals. Our frequency allocation effectively mitigates the misleading influence of distractors in V-NIAH-D. For a more detailed analysis, please refer to Appendix \ref{app:supp_explain_modules}.
    % Take the first 3 rotary angles as an example, the position embedding for temporal modeling is free from oscillation~\cite{men2024base}.
    }
    \label{fig:period_mono}
    \vspace{-12pt}
\end{minipage}
\end{figure*}

\noindent \textbf{Temporal Index Scaling.}
The frame index in video and the token index in text are inherently different \cite{kexuefm10352,li2024temporal}.
Recognizing this difference, methods like TAD-RoPE, a 1D RoPE adaptation for Video LLMs, introduce distinct step offsets for image and text token indices: $\gamma$ for image tokens and $\gamma+1$ for text tokens.
Consequently, an ideal RoPE design for video data should permit scaling of the temporal index to meet the inherent difference between the frame index and the text index.

\section{\methodname}\label{subsec:step_size}

Based on some previous research and the above analysis, we claim that a good RoPE design for Video LLMs, especially for long videos, should satisfy four requirements.
% : 3D structure, Appreciate Frequency Allocation, Spatial Symmetry, and Temporal Index Scaling.
The first requirement has been solved by RoPE-Tie~\cite{kexuefm10040} and the subsequent M-RoPE~\cite{wang2024qwen2}.
To solve the last three requirements and mitigate the performance decline observed in V-NIAH-D, we propose our \methodname, comprising the following three key components.
% (1) Low-frequency Temporal Allocation; (2) Diagonal Layout; and (3) Adjustable Temporal Spacing.
% \textbf{\textit{Multi-Modal Compatibility}}, whether RoPE can simultaneously describe the spatiotemporal position in multi-modals and sequential position in text-only inputs~\cite{wang2024qwen2,kexuefm10040,kexuefm10352}, \textbf{\textit{Appropriate Dimension Distribution}}, whether the feature dimension can process the semantic relationship where it is responsibility~\cite{peng2023yarn,barbero2024round,liu2024kangaroo}, \textbf{\textit{Spatial Symmetry}}, whether the distance between the end of precedent textual input and start of visual input equals the distance between the end of visual input and the start of subsequent textual input~\cite{kexuefm10352}, and \textbf{\textit{Temporal Alignment}}, whether the alignment of sequential feature in different modality is considered~\cite{gao2024tc}.

\noindent \textbf{Low-frequency Temporal Allocation (LTA).} 
As shown in Eq. (\ref{equ:rope}), the vanilla RoPE~\cite{su2024roformer} uses all dimensions to model the 1D position information. And as indicated in Eq. (\ref{equ:mrope}), M-RoPE~\cite{wang2024qwen2} uses dimensions to model temporal, horizontal, and vertical dimensions sequentially.
However, previous frequency allocation strategies are suboptimal because different RoPE dimensions capture dependencies at varying ranges.
As shown in Fig.  \ref{fig:attention_analysis}, an interesting observation is that the local attention branch (as reported in \cite{han2024lm}) corresponds to lower dimensions, while the global branch (or attention sink, as in \cite{xiao2023efficient}) corresponds to higher dimensions.
To sum up, lower dimensions (higher frequency, shorter monotonic intervals, larger $\theta_n$) tend to capture relative distances and local semantics \cite{men2024base,barbero2024round}, while higher dimensions (lower frequency, wider monotonic intervals, smaller $\theta_n$) capture longer-range dependencies \cite{barbero2024round}.

Based on our analysis, \methodname uses higher dimensions for temporal features in longer contexts and lower dimensions for spatial features, which are limited by resolution and have a fixed range.
To avoid the gap between horizontal and vertical positions, we interleave the dimensions responsible for these spatial features.
The dimension distribution for \methodname is shown in Eq. (\ref{equ:videorope}):

% $\bm{A}_{(t_1,x_1,y_1),(t_2,x_2,y_2)}=\bm{q}_{(t_1,x_1,y_1)}\bm{k}_{(t_2,x_2,y_2)}^\top$

% To make full use of these properties of RoPE, \methodname uses higher dimensions to model temporal features in longer contexts and lower dimensions to model spatial features since spatial features tend to be limited by resolution and have a relatively fixed range. To avoid the gap between horizontal and vertical positions, we interleave the dimensions responsible for those two spatial features. Therefore, the dimension distribution for \methodname is shown in Equation~\ref{equ:videorope}.
% 48, 48, 32; 0, 47, 48, 95; 96, 127
\begin{equation}
\resizebox{0.5\textwidth}{!}{$
\scriptsize
\begin{gathered}
\underbrace{\begingroup
\setlength\arraycolsep{1pt}
\begin{pmatrix}q^{(96)}\\q^{(97)}\\q^{(98)}\\q^{(99)}\\\vdots\\q^{(126)}\\q^{(127)}\end{pmatrix}^\top
\begin{pmatrix}
% \setstacktabbedgap{2pt}
\cos{\theta_{48}\Delta t}& -\sin{\theta_{48}\Delta t}&0&0&\cdots&0&0\\
\sin{\theta_{48}\Delta t}&\cos{\theta_{48}\Delta t}&0&0&\cdots&0&0 \\
0&0&\cos{\theta_{49}\Delta t}& -\sin{\theta_{49}\Delta t}&\cdots&0&0\\
0&0&\sin{\theta_{49}\Delta t}&\cos{\theta_{49}\Delta t}&\cdots&0&0 \\ 
\vdots&\vdots&\vdots&\vdots&\ddots&\vdots&\vdots\\
0&0&0&0&\cdots&\cos{\theta_{63}\Delta t}& -\sin{\theta_{63}\Delta t}\\
0&0&0&0&\cdots&\sin{\theta_{63}\Delta t}&\cos{\theta_{63}\Delta t}
\end{pmatrix}
\begin{pmatrix}k^{(96)}\\k^{(97)}\\k^{(98)}\\k^{(99)}\\\vdots\\k^{(126)}\\k^{(127)}\end{pmatrix}
\endgroup}_\text{\normalsize modeling temporal dependency with lower frequency} \\
+ \underbrace{\begingroup
\setlength\arraycolsep{1pt}
\begin{pmatrix}q^{(0)}\\q^{(1)}\\q^{(4)}\\q^{(5)}\\\vdots\\q^{(92)}\\q^{(93)}\end{pmatrix}^\top
\begin{pmatrix}
% \setstacktabbedgap{2pt}
\cos{\theta_{0}\Delta x}& -\sin{\theta_{0}\Delta x}&0&0&\cdots&0&0\\
\sin{\theta_{0}\Delta x}&\cos{\theta_{0}\Delta x}&0&0&\cdots&0&0 \\
0&0&\cos{\theta_{2}\Delta x}& -\sin{\theta_{2}\Delta x}&\cdots&0&0\\
0&0&\sin{\theta_{2}\Delta x}&\cos{\theta_{2}\Delta x}&\cdots&0&0 \\ 
\vdots&\vdots&\vdots&\vdots&\ddots&\vdots&\vdots\\
0&0&0&0&\cdots&\cos{\theta_{46}\Delta x}& -\sin{\theta_{46}\Delta x}\\
0&0&0&0&\cdots&\sin{\theta_{46}\Delta x}&\cos{\theta_{46}\Delta x}
\end{pmatrix}
\begin{pmatrix}k^{(0)}\\k^{(1)}\\k^{(4)}\\k^{(5)}\\\vdots\\k^{(92)}\\k^{(93)}\end{pmatrix}
\endgroup}_\text{\normalsize modeling horizontal dependency with interleaved high frequency} \\
+ \underbrace{\begingroup
\setlength\arraycolsep{1pt}
\begin{pmatrix}q^{(2)}\\q^{(3)}\\q^{(6)}\\q^{(7)}\\\vdots\\q^{(94)}\\q^{(95)}\end{pmatrix}^\top
\begin{pmatrix}
% \setstacktabbedgap{2pt}
\cos{\theta_{1}\Delta y}& -\sin{\theta_{1}\Delta y}&0&0&\cdots&0&0\\
\sin{\theta_{1}\Delta y}&\cos{\theta_{1}\Delta y}&0&0&\cdots&0&0 \\
0&0&\cos{\theta_{3}\Delta y}& -\sin{\theta_{3}\Delta y}&\cdots&0&0\\
0&0&\sin{\theta_{3}\Delta y}&\cos{\theta_{3}\Delta y}&\cdots&0&0 \\ 
\vdots&\vdots&\vdots&\vdots&\ddots&\vdots&\vdots\\
0&0&0&0&\cdots&\cos{\theta_{47}\Delta y}& -\sin{\theta_{47}\Delta y}\\
0&0&0&0&\cdots&\sin{\theta_{47}\Delta y}&\cos{\theta_{47}\Delta y}
\end{pmatrix}
\begin{pmatrix}k^{(2)}\\k^{(3)}\\k^{(6)}\\k^{(7)}\\\vdots\\k^{(94)}\\k^{(95)}\end{pmatrix}
\endgroup}_\text{\normalsize modeling vertical dependency with interleaved high frequency} \\
% \Delta t=t_1-t_2,\quad \Delta x=x_1-x_2,\quad \Delta y=y_1-y_2 \\
% \theta_n=\beta^{-\dfrac{2n}{d}},\quad n=0,\cdots,d/2-1
\end{gathered}
$
}
% \raisebox{-5.5ex}{.}
\label{equ:videorope}
\end{equation}
The horizontal position $x$ and vertical position $y$ are interleaved to occupy the lower dimensions, followed by temporal $t$, which occupies the higher dimensions. We keep the same allocation number for $x$, $y$, and $t$ as M-RoPE for a fair comparison, with values of 48, 48, and 32, respectively.
The advantages of this distribution are evident in Fig.  \ref{fig:period_mono}. 
For a RoPE-based LLM with a 128-dimensional head (64 rotary angles $\theta_n$), we visualize the function of $\cos{\theta_n t}$ for 3 dimensions using parallel blue planes.

As shown in Fig. \ref{fig:period_mono} (\textbf{a}), M-RoPE's temporal position embeddings are significantly distorted by periodic oscillations \cite{men2024base}, leading to identical embeddings for distant positions.
For instance, considering the last three rotary angles, the temporal embeddings are severely affected by these oscillations due to their short monotonic intervals (and even shorter intervals in lower dimensions).
This periodicity creates ``hash collisions'' (red planes), where distant positions share near-identical embeddings, making the model susceptible to distractor influence.
Fortunately, our \methodname (Fig. \ref{fig:period_mono} (\textbf{b})) is free from oscillation and Hash collision in temporal modeling.
The visualized relationship between the periodicity, monotonicity, and temporal modeling.

\begin{figure}[t]
\centering
\includegraphics[width=0.98\linewidth]{figures/files/video_rope_figure_spatial_v2.pdf}
\vspace{-6pt}
\caption{\footnotesize The position embeddings of adjacent text tokens for Vanilla RoPE (\textbf{top} row), the corresponding visual tokens in adjacent frames for M-RoPE (\textbf{middle} row) and our \methodname (\textbf{bottom} row) with interleaved spatial and temporal last design.}
\vspace{-12pt}
\label{fig:spatail_index}
\end{figure}

\begin{figure*}[t]
\begin{minipage}{0.98\textwidth}
    \begin{subfigure}[b]{0.3\linewidth}
        \centering
        \includegraphics[width=0.95\linewidth]{figures/files/vanilla_rope.pdf}
        \caption{3D visualization for Vanilla RoPE.}
        \label{fig:vanilla_rope}
    \end{subfigure}
    \hfill
    \begin{subfigure}[b]{0.3\linewidth}
        \centering
        \includegraphics[width=0.95\linewidth]{figures/files/m_rope.pdf}
        \caption{3D visualization for M-RoPE.}
        \label{fig:m_rope}
    \end{subfigure}
    \hfill
    \begin{subfigure}[b]{0.3\linewidth}
        \centering
        \includegraphics[width=0.95\linewidth]{figures/files/m_modify_rope.pdf}
        \caption{3D visualization for \methodname.}
        \label{fig:video_rope}
    \end{subfigure}
    \hfill
    \vspace{-6pt}
    \caption{\footnotesize The 3D visualization for different position embedding. \textbf{(a)} The vanilla 1D RoPE~\cite{su2024roformer} does not incorporate spatial modeling.
    \textbf{(b)} M-RoPE~\cite{wang2024qwen2}, while have the 3D structure, introduces a discrepancy in index growth for visual tokens across frames, with some indices remaining constant.
    \textbf{(c)} In contrast, our \methodname achieves the desired balance, maintaining the consistent index growth pattern of vanilla RoPE while simultaneously incorporating spatial modeling. 
    }
    %3D visualization of different position embeddings. \textbf{(a)} The vanilla 1D RoPE~\cite{su2024roformer} lacks spatial modeling. \textbf{(b)} M-RoPE~\cite{wang2024qwen2}, while have the 3D structure, introduces a discrepancy in index growth for visual tokens across frames, with some indices remaining constant. \textbf{(c)} Our \methodname balances the index growth of vanilla RoPE while incorporating spatial modeling. For more details on the index, see Appendix \ref{app:supp_explain_modules}.
    \vspace{-12pt}
    \label{fig:spatial}
\end{minipage}
\end{figure*}

\noindent \textbf{Diagonal Layout.}
Fig. \ref{fig:spatial} provides a visual comparison of spatial symmetry in positional encodings.
For vanilla RoPE (Fig. ~\ref{fig:vanilla_rope}), no spatial relation is considered and the index for every dimension increases directly.
While M-RoPE (Fig. \ref{fig:m_rope}), incorporates spatial information within each frame, it introduces two significant discontinuities between textual and visual tokens.
This arises from M-RoPE's placement strategy, if the first visual token is at $(0, 0)$, the last token in each frame will always be placed at $(W-1, H-1)$, creating a stack in the bottom-left corner.
Furthermore, like vanilla RoPE, M-RoPE's indices increase with input length across all dimensions.

To address these limitations, \methodname arranges the entire input along the diagonal, see Fig. \ref{fig:video_rope}.
The central patch's 3D position for each video frame is $(t,t,t)$, with other patches offset in all directions.
Our \textbf{Diagonal Layout} has two advantages: (1) our design preserves the relative positions of visual tokens and ensures approximate equidistance from the image corners to the center, preventing text tokens from being overly close to any corner. (2) It maintains the indexing pattern of vanilla RoPE (Fig.  \ref{fig:spatail_index}), as the position index increment between corresponding spatial locations in adjacent frames mirrors that of adjacent textual tokens.

\noindent \textbf{Adjustable Temporal Spacing.}
To scale the temporal index, we introduce a scaling factor $\delta$ to better align temporal information between visual and textual tokens.

Suppose the symbol $\tau$ denotes the token index, for the starting text ($0 \leq \tau < T_s$), the temporal, horizontal, and vertical indices are simply set to the raw token index $\tau$.
For the video input ($T_s \leq \tau < T_s + T_v$), The difference $\tau - T_s$ represents the index of the current frame relative to the start of the video, which is then scaled by $\delta$ to control the space in the temporal dimension.
For the ending text ($T_s + T_v \leq \tau < T_s + T_v + T_e$), the temporal, horizontal, and vertical index are the same, creating a linear progression.

According to our adjustable temporal spacing design, for a multi-modal input that consists of a text with $T_s$ tokens, a following video with $T_v$ frame with $W\times H$ patches in each frame, and an ending text with $T_e$ tokens, the position indices $(t, x, y)$ of \methodname for $\tau$-th textual token or $(\tau, w, h)$-th visual token are defined as Eq. (\ref{equ:index}):
\begin{equation}
\vspace{-3pt}
\resizebox{0.5\textwidth}{!}{$
    \footnotesize
    (t,x,y) =
    \begin{cases}
        (\tau, \tau, \tau) & \text{if } 0 \leq \tau < T_s \\[3ex]
        \left( 
        \begin{array}{l}
            T_s + \delta (\tau - T_s), \\
            T_s + \delta (\tau - T_s) + w - \frac{W}{2}, \\
            T_s + \delta (\tau - T_s) + h - \frac{H}{2}
        \end{array}
        \right) & \text{if } T_s \leq \tau < T_s + T_v \\[6ex]
        \left( 
        \begin{array}{l}
            T_s + \delta T_v + \tau, \\
            T_s + \delta T_v + \tau, \\
            T_s + \delta T_v + \tau
        \end{array}
        \right) & \text{if } T_s + T_v \leq \tau < T_s + T_v + T_e
    \end{cases}
$}
\raisebox{-9.5ex}{,}
\label{equ:index}
\end{equation}
where $w$ and $h$ represent the horizontal and vertical indices of the visual patch within the frame, respectively.

In summary, the parameter $\delta$ in our adjustable temporal spacing allows for a flexible and consistent way to encode the relative positions of text and video tokens.


\section{Dataset}
\label{sec:dataset}

\subsection{Data Collection}

To analyze political discussions on Discord, we followed the methodology in \cite{singh2024Cross-Platform}, collecting messages from politically-oriented public servers in compliance with Discord's platform policies.

Using Discord's Discovery feature, we employed a web scraper to extract server invitation links, names, and descriptions, focusing on public servers accessible without participation. Invitation links were used to access data via the Discord API. To ensure relevance, we filtered servers using keywords related to the 2024 U.S. elections (e.g., Trump, Kamala, MAGA), as outlined in \cite{balasubramanian2024publicdatasettrackingsocial}. This resulted in 302 server links, further narrowed to 81 English-speaking, politics-focused servers based on their names and descriptions.

Public messages were retrieved from these servers using the Discord API, collecting metadata such as \textit{content}, \textit{user ID}, \textit{username}, \textit{timestamp}, \textit{bot flag}, \textit{mentions}, and \textit{interactions}. Through this process, we gathered \textbf{33,373,229 messages} from \textbf{82,109 users} across \textbf{81 servers}, including \textbf{1,912,750 messages} from \textbf{633 bots}. Data collection occurred between November 13th and 15th, covering messages sent from January 1st to November 12th, just after the 2024 U.S. election.

\subsection{Characterizing the Political Spectrum}
\label{sec:timeline}

A key aspect of our research is distinguishing between Republican- and Democratic-aligned Discord servers. To categorize their political alignment, we relied on server names and self-descriptions, which often include rules, community guidelines, and references to key ideologies or figures. Each server's name and description were manually reviewed based on predefined, objective criteria, focusing on explicit political themes or mentions of prominent figures. This process allowed us to classify servers into three categories, ensuring a systematic and unbiased alignment determination.

\begin{itemize}
    \item \textbf{Republican-aligned}: Servers referencing Republican and right-wing and ideologies, movements, or figures (e.g., MAGA, Conservative, Traditional, Trump).  
    \item \textbf{Democratic-aligned}: Servers mentioning Democratic and left-wing ideologies, movements, or figures (e.g., Progressive, Liberal, Socialist, Biden, Kamala).  
    \item \textbf{Unaligned}: Servers with no defined spectrum and ideologies or opened to general political debate from all orientations.
\end{itemize}

To ensure the reliability and consistency of our classification, three independent reviewers assessed the classification following the specified set of criteria. The inter-rater agreement of their classifications was evaluated using Fleiss' Kappa \cite{fleiss1971measuring}, with a resulting Kappa value of \( 0.8191 \), indicating an almost perfect agreement among the reviewers. Disagreements were resolved by adopting the majority classification, as there were no instances where a server received different classifications from all three reviewers. This process guaranteed the consistency and accuracy of the final categorization.

Through this process, we identified \textbf{7 Republican-aligned servers}, \textbf{9 Democratic-aligned servers}, and \textbf{65 unaligned servers}.

Table \ref{tab:statistics} shows the statistics of the collected data. Notably, while Democratic- and Republican-aligned servers had a comparable number of user messages, users in the latter servers were significantly more active, posting more than double the number of messages per user compared to their Democratic counterparts. 
This suggests that, in our sample, Democratic-aligned servers attract more users, but these users were less engaged in text-based discussions. Additionally, around 10\% of the messages across all server categories were posted by bots. 

\subsection{Temporal Data} 

Throughout this paper, we refer to the election candidates using the names adopted by their respective campaigns: \textit{Kamala}, \textit{Biden}, and \textit{Trump}. To examine how the content of text messages evolves based on the political alignment of servers, we divided the 2024 election year into three periods: \textbf{Biden vs Trump} (January 1 to July 21), \textbf{Kamala vs Trump} (July 21 to September 20), and the \textbf{Voting Period} (after September 20). These periods reflect key phases of the election: the early campaign dominated by Biden and Trump, the shift in dynamics with Kamala Harris replacing Joe Biden as the Democratic candidate, and the final voting stage focused on electoral outcomes and their implications. This segmentation enables an analysis of how discourse responds to pivotal electoral moments.

Figure \ref{fig:line-plot} illustrates the distribution of messages over time, highlighting trends in total messages volume and mentions of each candidate. Prior to Biden's withdrawal on July 21, mentions of Biden and Trump were relatively balanced. However, following Kamala's entry into the race, mentions of Trump surged significantly, a trend further amplified by an assassination attempt on him, solidifying his dominance in the discourse. The only instance where Trump’s mentions were exceeded occurred during the first debate, as concerns about Biden’s age and cognitive abilities temporarily shifted the focus. In the final stages of the election, mentions of all three candidates rose, with Trump’s mentions peaking as he emerged as the victor.
\section{Experimental Methodology}\label{sec:exp}
In this section, we introduce the datasets, evaluation metrics, baselines, and implementation details used in our experiments. More experimental details are shown in Appendix~\ref{app:experiment_detail}.

\textbf{Dataset.}
We utilize various datasets for training and evaluation. Data statistics are shown in Table~\ref{tab:dataset}.

\textit{Training.}
We use the publicly available E5 dataset~\cite{wang2024improving,springer2024repetition} to train both the LLM-QE and dense retrievers. We concentrate on English-based question answering tasks and collect a total of 808,740 queries. From this set, we randomly sample 100,000 queries to construct the DPO training data, while the remaining queries are used for contrastive training. During the DPO preference pair construction, we first prompt LLMs to generate expansion documents, filtering out queries where the expanded documents share low similarity with the query. This results in a final set of 30,000 queries.

\textit{Evaluation.}
We evaluate retrieval effectiveness using two retrieval benchmarks: MS MARCO \cite{bajaj2016ms} and BEIR \cite{thakur2021beir}, in both unsupervised and supervised settings.

\textbf{Evaluation Metrics.}
We use nDCG@10 as the evaluation metric. Statistical significance is tested using a permutation test with $p<0.05$.

\textbf{Baselines.} We compare our LLM-QE model with three unsupervised retrieval models and five query expansion baseline models.
% —

Three unsupervised retrieval models—BM25~\cite{robertson2009probabilistic}, CoCondenser~\cite{gao2022unsupervised}, and Contriever~\cite{izacard2021unsupervised}—are evaluated in the experiments. Among these, Contriever serves as our primary baseline retrieval model, as it is used as the backbone model to assess the query expansion performance of LLM-QE. Additionally, we compare LLM-QE with Contriever in a supervised setting using the same training dataset.

For query expansion, we benchmark against five methods: Pseudo-Relevance Feedback (PRF), Q2Q, Q2E, Q2C, and Q2D. PRF is specifically implemented following the approach in~\citet{yu2021improving}, which enhances query understanding by extracting keywords from query-related documents. The Q2Q, Q2E, Q2C, and Q2D methods~\cite{jagerman2023query,li2024can} expand the original query by prompting LLMs to generate query-related queries, keywords, chains-of-thought~\cite{wei2022chain}, and documents.


\textbf{Implementation Details.} 
For our query expansion model, we deploy the Meta-LLaMA-3-8B-Instruct~\cite{llama3modelcard} as the backbone for the query expansion generator. The batch size is set to 16, and the learning rate is set to $2e-5$. Optimization is performed using the AdamW optimizer. We employ LoRA~\cite{hu2022lora} to efficiently fine-tune the model for 2 epochs. The temperature for the construction of the DPO data varies across $\tau \in \{0.8, 0.9, 1.0, 1.1\}$, with each setting sampled eight times. For the dense retriever, we utilize Contriever~\cite{izacard2021unsupervised} as the backbone. During training, we set the batch size to 1,024 and the learning rate to $3e-5$, with the model trained for 3 epochs.


\section{Conclusion}
We introduce a novel approach, \algo, to reduce human feedback requirements in preference-based reinforcement learning by leveraging vision-language models. While VLMs encode rich world knowledge, their direct application as reward models is hindered by alignment issues and noisy predictions. To address this, we develop a synergistic framework where limited human feedback is used to adapt VLMs, improving their reliability in preference labeling. Further, we incorporate a selective sampling strategy to mitigate noise and prioritize informative human annotations.

Our experiments demonstrate that this method significantly improves feedback efficiency, achieving comparable or superior task performance with up to 50\% fewer human annotations. Moreover, we show that an adapted VLM can generalize across similar tasks, further reducing the need for new human feedback by 75\%. These results highlight the potential of integrating VLMs into preference-based RL, offering a scalable solution to reducing human supervision while maintaining high task success rates. 

\section*{Impact Statement}
This work advances embodied AI by significantly reducing the human feedback required for training agents. This reduction is particularly valuable in robotic applications where obtaining human demonstrations and feedback is challenging or impractical, such as assistive robotic arms for individuals with mobility impairments. By minimizing the feedback requirements, our approach enables users to more efficiently customize and teach new skills to robotic agents based on their specific needs and preferences. The broader impact of this work extends to healthcare, assistive technology, and human-robot interaction. One possible risk is that the bias from human feedback can propagate to the VLM and subsequently to the policy. This can be mitigated by personalization of agents in case of household application or standardization of feedback for industrial applications. 


\section*{Impact Statement}
This paper presents work whose goal is to advance the field
of Machine Learning. There are many potential societal
consequences of our work, and none of which we feel must
be specifically highlighted here.


\bibliography{main}
\bibliographystyle{icml2025}


%%%%%%%%%%%%%%%%%%%%%%%%%%%%%%%%%%%%%%%%%%%%%%%%%%%%%%%%%%%%%%%%%%%%%%%%%%%%%%%
%%%%%%%%%%%%%%%%%%%%%%%%%%%%%%%%%%%%%%%%%%%%%%%%%%%%%%%%%%%%%%%%%%%%%%%%%%%%%%%
% APPENDIX
%%%%%%%%%%%%%%%%%%%%%%%%%%%%%%%%%%%%%%%%%%%%%%%%%%%%%%%%%%%%%%%%%%%%%%%%%%%%%%%
%%%%%%%%%%%%%%%%%%%%%%%%%%%%%%%%%%%%%%%%%%%%%%%%%%%%%%%%%%%%%%%%%%%%%%%%%%%%%%%
\newpage
\appendix
\onecolumn

\newpage
\onecolumn
\appendix

\newtcolorbox{cvbox}[1][]{
    enhanced,
%   blanker, % <- removed as it suppresses box color and frame
    %leftupper=4cm,
    after skip=8mm,%   enlarge distance to the next box
    title=#1,
    breakable = true,
    fonttitle=\sffamily\bfseries,
    coltitle=black,
    colbacktitle=gray!10,   % <- defines background color in title
    titlerule= 0pt,         % <- sets rule underneath title 
    %fontupper=\sffamily,%
    %#1
    overlay={%
        \ifcase\tcbsegmentstate
        % 0 = Box contains only an upper part
        \or%
        % 1 = Box contains an upper and a lower part
        %\path[draw=red] (segmentation.west)--(frame.south east);
        \else%
        % 2 = Box contains only a lower part
        %\path[draw=red] (frame.north west)--(frame.south east);
        \fi%
    }
    colback = gray,         % <- defines background color in box
    colframe = black!75     % <- defines color of frame
    }


\section{Data Construction}

\subsection{Dataset Summary}
\label{sec:appendix_data}

We prepare a seed dataset $\mathcal{D}$ containing both safety and helpfulness data. It consists of 50k pairwise samples from three sources. For helpfulness data, we draw 25k samples from UltraFeedback~\cite{cui2024ultrafeedback}. Each sample originally has 5 potential responses with ratings and we take the one with the highest rating as ``chosen'' and the one with the lowest as ``rejected''. For safety data, we take 22k samples from PKU-SafeRLHF~\cite{ji2024pku}, which have responses with unsafe labels and are further filtered by GPT-4o to assure the prompts are truly toxic and harmful. We follow the common practice of proprietary LLMs that responses to harmful queries should contain clear refusal in at most one sentence instead of providing additional content and guide besides a brief apology~\cite{guan2024deliberative}. This make current positive annotations in PKU-SafeRLHF, which usually contain much relevant information, not directly usable. Therefore, we use GPT-4o to generate refusal answers for these prompts and substitute the original chosen responses with them. 

Further, to better address the complex scenario of jailbreak attack, we take 3k jailbreak prompts from JailbreakV-28k~\cite{luo2024jailbreakv}. As this dataset was originally proposed for benchmarks, we carefully decontaminate the red-teaming queries from those used for evaluation, e.g., AdvBench~\cite{zou2023universal}, and only sample prompts from the sources of GPT-Generate, Handcraft, and HH-RLHF~\cite{ganguli2022red}. Due to the lack of response annotations, we prompt GPT-4o to generate refusal answers as ``chosen'' ones and prompt the models to be trained for ``rejected'' answers.


When training with baselines on the whole dataset, we take the ``chosen'' responses as labels for SFT and use the pairwise data for preference learning (i.e., DPO, SACPO). As for our method and self-rewarding, we only use the prompts from the dataset. Compared with baseline methods like SFT and DPO, we do not use the jailbreak prompts with GPT-generated refusals to directly train the models.



\subsection{Structured CoT Data Synthesis}
\label{sec:appendix_cot}
To initialize the models with reasoning ability, we take 10k prompts each from PKU-SafeRLHF and UltraFeedback, and synthesize structured CoT answers. Specifically, we used GPT-4o to generate structured reasoning steps for these prompts. The generation prompt instructs GPT-4o to produce step-by-step reasoning following a specific format. After generation, we filter out responses with incorrect formatting and remove any responses that provide unsafe answers to safety-related queries, ensuring the quality and safety of our CoT training data. The final dataset consists of synthesized samples slightly fewer than 20k due to the filtering process. The generation prompt for instructing GPT-4o is as below:

% \vspace{8pt} 
\begin{cvbox}[\vspace{8pt}Instruction to Generate Structured Chain-of-Thought Answer]
You are a helpful assistant specializing in addressing both general queries and sensitive topics, including privacy, security, and ethics. Your primary responsibility is to provide thorough reasoning and structured planning to ensure your responses are both helpful and strictly adhere to safety and ethical guidelines.
\newline \newline
\#\#\# Instructions:
\newline \newline
1. **Problem Analysis**: Begin by analyzing the query to understand its context and devise a clear and structured roadmap for the reasoning process below. This step should outline how the problem will be addressed in subsequent reasoning steps, ensuring logical progression and coherence. During this step, you may identify whether the query poses any risks or ethical concerns. If the query involves sensitive or risky elements, the roadmap should focus on identifying and mitigating these aspects. For harmless queries, emphasize an effective and logical approach to solving the problem.
\newline \newline
2. **Reasoning**: Break down the query into detailed and systematic reasoning steps. Each step should build upon the roadmap provided in the Problem Analysis section. Pay close attention to the progression of the analysis—if potential risks or sensitivities are uncovered during this stage, pivot immediately to a safety-focused reasoning process. This includes reassessing the problem's implications and adjusting your steps to ensure a thorough evaluation of risks. For harmless queries, highlight logical planning and systematic problem-solving to provide the most effective solution.
\newline \newline
3. **Final Answer**: Based on your reasoning, provide a concise and direct conclusion using the format `<|Output|>Final Answer:<|/Output|>`. Reasoning is provided to help you better understand the steps involved, but the ultimate answer to the query must be presented here.
- If the query is unsafe or unethical, explicitly refuse assistance with clear language, e.g., "Sorry, I cannot assist with this." Avoid using repetitive responses by varying your phrasing while maintaining a professional and polite tone. For example, you might say: "I’m sorry, but I cannot help with that request," "Unfortunately, I am unable to fulfill this request as it may violate safety guidelines," or "I must decline, as this request does not align with ethical considerations." Ensure your response remains concise, clear, and respectful without engaging further with unsafe or inappropriate content.
\newline  
- If the query is safe and ethical, offer a well-structured, logical, and solution-oriented response that demonstrates your reasoning and problem-solving capabilities. Strengthen the clarity and completeness of the solution to provide the most effective assistance in harmless scenarios.
\newline \newline
\#\#\# Format Requirements:
\newline \newline
- Each step must use the following tokenized format:
\newline 
1. **Problem Analysis**: Encapsulate the analysis within <|Reasoning\_step|> Title: Problem Analysis:  <|/Reasoning\_step|> tags.
\newline \newline
2. **Reasoning**: Include multiple <|Reasoning\_step|> Title: Title\_name <|/Reasoning\_step|> sections as needed to thoroughly address the query.
\newline \newline
3. **Final Answer**: Provide the conclusion in the format: <|Output|>Final Answer: <|/Output|> .
\newline 
By adhering to these guidelines and referring to the above example, you will provide clear, accurate, and well-structured responses to questions involving sensitive or potentially unsafe topics while excelling in logical planning and reasoning for safe and harmless queries. Provide your reasoning steps directly without additional explanations. Begin your response with the special token `<|Reasoning\_step|>`. Following is the question:

\vspace{1em}
Question: \{prompt\}
\vspace{8pt} 
\end{cvbox}


\section{Self-Improvement with Safety-Informed MCTS}

\subsection{Derivation of Safety-Informed Reward}
\label{sec:appendix_derive}

Here, we present the proof for~\cref{theorem} in~\cref{sec:MCTS}, to derive a proper form for the safety-informed reward function. We first recall the three desired properties with intuitive explanations.
\begin{enumerate}
    \item \textbf{Safety as Priority}: Safe responses always get higher rewards than unsafe ones, regardless of their helpfulness.
    \begin{equation}
        \forall \bfm_1,\bfm_2, S(\bfm_1)>0> S(\bfm_2) \Rightarrow R(\bfm_1)>R(\bfm_2)
    \end{equation}
    \item \textbf{Dual Monotonicity of Helpfulness}: When the response is safe, it gets higher reward for better helpfulness; when it is unsafe, it gets lower reward for providing more helpful instructions towards the harmful intention.
    \begin{equation}
        \forall S>0 , \frac{\partial R}{\partial H} > 0\text{ and } \forall S<0, \frac{\partial R}{\partial H} < 0;
    \end{equation}
    \item \textbf{Degeneration to Single Objective}: If we only consider one dimension, we can set the reward function to have a constant shift from the original reward of that aspect. This will lead to the procedure degenerating to standard MCTS under the corresponding reward, i.e., given a partially constructed search tree, the result of selection is the same when all hyperparameters, e.g., seed, exploration parameter, are fixed.
    \begin{align}
        \exists\;C_1 \in [-1,1],\;s.t.\;\text{let }S\equiv C_1, \forall \bfm_1,\bfm_2, R(\bfm_1)-R(\bfm_2)=H(\bfm_1)-H(\bfm_2);\\
    \exists\;C_2 \in [-1,1],\;s.t.\;\text{let }H\equiv C_2, \forall \bfm_1,\bfm_2, R(\bfm_1)-R(\bfm_2)=S(\bfm_1)-S(\bfm_2).
    \end{align}
    
\end{enumerate}

\begin{theorem}
    Fix constants $C_1, C_2\in [-1,1],\;C_1\ne0$. Suppose $R:[-1,1]\times[-1,1]\rightarrow \mathbb{R}$ is twice-differentiable and satisfies $\frac{\partial R}{\partial H}=F(S)$, for some continuous function $F: [-1,1]\rightarrow \mathbb{R}$. Properties 2 and 3 of Dual Monotonicity of Helpfulness and Degeneration to Single Objective hold, if and only if
    \begin{equation}
    R(H,S)=F(S)\cdot H+S - C_2 \cdot F(S)+c,       
    \end{equation} with $F(0)=0, F(C_1)=1, \forall S>0, F(S)>0, \forall S<0, F(S)<0$ and $c$ as a constant.
\end{theorem}

\begin{proof} We show that the form of $R$ is the sufficient and necessary condition of Properties 2 and 3, given the assumptions. For notation simplicity, we use $H_1,H_2,S_1,S_2$ to denote the rewards for arbitrary final answers $f_1, f_2$.

\textbf{Sufficiency}

Assume $R(H,S)=F(S)\cdot H+S-C_2\cdot F(S)+c$ with $F(S)$ satisfying the stated conditions.

For Property 2, we can compute the partial derivative and show that
\begin{equation*}
    \frac{\partial R}{\partial H} = F(S) \begin{cases}
        > 0,\text{ when }S>0,\\
        <0,\text{ when }S<0.
    \end{cases}
\end{equation*}

For Property 3, let $S\equiv C_1$, we get
\begin{equation*}
    R(H_1,C_1)-R(H_2,C_1) = F(C_1) (H_1-H_2) = H_1-H_2.
\end{equation*}
let $H\equiv C_2$, we get
\begin{equation*}
    R(C_2,S_1)-R(C_2,S_2) = C_2(F(S_1)-F(S_2)) + S_1-S_2 -C_2(F(S_1)-F(S_2))= S_1-S_2.
\end{equation*}

\textbf{Necessity}

    Assume $R(H,S)$ satisfies Properties 2 and 3.

    Given the condition that $\frac{\partial R}{\partial H} = F(S)$, the function $R$ should follow the form by integral, 
    \begin{equation}
        R(H,S) = \int_0^H \frac{\partial R}{\partial H}dH+R(0,S) =F(S)\cdot H + G(S),
        \label{eq:reward}
    \end{equation}
    with $G(S)=R(0,S)$ as a continuous and differentiable function of $S$.

    Then, we apply the property of Degeneration to Single Objective, when $S\equiv C_1$,
    \begin{align*}
        R(H_1, C_1)-R(H_2,C_2) = F(C_1)& (H_1-H_2) = H_1-H_2, \forall H_1,H_2\in[-1,1]\\
        &\Rightarrow F(C_1) = 1,
    \end{align*}
    and when $H\equiv C_2$, 
    \begin{align*}
        R(C_2, S_1) - R(C_2, S_2) = C_2(F(S_1)& - F(S_2)) + G(S_1) - G(S_2) = S_1 - S_2, \forall S_1, S_2 \in[-1,1]\\ 
        &\Rightarrow C_2\cdot F'(S) - G'(S) = 1\\ 
        &\Rightarrow G'(S) = 1- C_2\cdot F'(S)\\ 
        &\Rightarrow G(S) = S-C_2\cdot F(S) + c, 
    \end{align*}
    with $c$ as a constant.

    Considering the property of Dual Monotonicity of Helpfulness, it is clear that $\frac{\partial R}{\partial H} = F(S)$ should satisfy
    \begin{equation*}
        F(S) >0, \forall S>0\text{ and }F(S)<0, \forall S<0.
    \end{equation*}
    Given the continuity of $F(S)$, $F(0) = 0$.

    Substituting $G(S)$ to~\cref{eq:reward}, we eventually get the family of $R$, following
    \begin{equation*}
    R(H,S)=F(S)\cdot H+S - C_2 \cdot F(S)+c,       
    \end{equation*} with $F(0)=0, F(C_1)=1, F(S)>0, \forall S>0$, $F(S)<0, \forall S<0$ and $c$ as a constant.
\end{proof}

\begin{corollary}
 Take $F(S)=S, C_1=1, C_2=-1, c=0$, $R(H,S)=2S+S\cdot H$ satisfies that for any $H_1, H_2,S_1,S_2\in[-1,1]$, when $S_1>0>S_2$, the inequality of $R(S_1,H_1)>R(S_2,H_2)$ holds.
\end{corollary}


\subsection{Implementation Details of Self-Improvement}
\label{sec:appendix_self-improvement}

Here, we introduce the implementation details of different components in the iterative self-improvement, including SI-MCTS, Self-Rewarding, and preference data construction.

\subsubsection{Safety-Informed MCTS} 
We design safety-informed reward to introduce dual information of both helpfulness and safety, without impacting the original effect of MCTS on a single dimension. Therefore, we mainly follow the standard MCTS procedure~\cite{vodopivec2017monte} guided by UCB1 algorithm~\cite{chang2005adaptive}. When traversing from the root node (i.e., prompt) to the leaf node, it selects the $i$-th node with the highest value of
\begin{equation}
    v_i + c\sqrt{\frac{\ln N_i}{n_i}},
\label{eq:UCB}
\end{equation}
where $v_i$ is the estimated value of safety-informed rewards, $n_i$ is the visited times of this node, $N_i$ is the visited times of its parent node, and $c$ is the exploration parameter that balances exploration and exploitation. 

The whole procedure of Safety-Informed MCTS follows~\cref{alg:SI MCTS}. In practice, we set exploration parameter $c=1.5$, search budget $n=200$, children number $m=4$. To generate child nodes and rollout to final answers, we set temperature as $1.2$, top-p as $0.9$ and top-k as $50$. We adjust these parameters when higher diversity is needed.



% Build MCT
\begin{algorithm}[ht]
   \caption{Safety-Informed MCTS}
   \label{alg:SI MCTS}
\begin{algorithmic}
   \STATE {\bfseries Input:} prompt set $\mathcal{D}_k$, safety reward function $S$, helpfulness reward function $H$, actor model $\pi_\theta$ that generate one step each time by default, search budget $n$, children number $m$
   \STATE {\bfseries Output:} MCT data $\mathbb{T}$
   \STATE Init $\mathbb{T}$ with empty
   \FOR{each single prompt $\bx$ in $\mathcal{D}_k$}
        \STATE Init search tree $T$ with $root\_node$ of $\bx$
        \FOR{$i$ in range($n$)}
            \STATE Select a leaf node $select\_node$ following the trajectory $(\bx,\bs_i)$ using UCB1 algorithm as~\cref{eq:UCB}
            \STATE $\bz_{i+1}^\ast \leftarrow None$
            \IF{$select\_node$ has been visited before}
                \IF{$select\_node$ is non-terminal}
                    \STATE Sample $m$ children $\{\bz_{i+1}^{(j)}\}_{j=1}^m$ from $\pi_\theta(\cdot|\bx, \bs_i)$ and add the $m$ children to $T$
                    \STATE $\bz_{i+1}^\ast \leftarrow$ random.choice($\{\bz_{i+1}^{(j)}\}$), $select\_node \leftarrow$ the corresponding child
                \ENDIF
            \ENDIF
            \STATE Rollout a full answer $\bfm\sim\pi_\theta(\cdot|\bx,\bs_i, \bz_{i+1}^\ast)$
            \STATE Calculate reward $r \leftarrow S(\bfm) \cdot H(\bfm) +2S(\bfm)$
            \STATE Backpropagate and update node's value and visited times from $select\_node$ to $root\_node$
        \ENDFOR
        \STATE Rollout all nodes that have not been visited before, calculate reward and backpropagate
        \STATE $\mathbb{T}\leftarrow \mathbb{T}\cup\{T\}$
   \ENDFOR
\end{algorithmic}
\end{algorithm}

\subsubsection{Self-Rewarding} 
We take the trained LLMs as judges~\cite{zheng2023judging} to rate their own responses, to remove dependencies on external reward models. We adopt a similar template design following~\cite{yuanself} to prompt the model to give discrete ratings given the query $\bx$ and the final answer $\bfm$ sampled through rollout. For helpfulness, we ask the model to rate the answer from $1$ to $5$ according to the extent of helpfulness and correctness. For safety, we categorize the answer into safe and unsafe ones. All ratings will be normalized into the range of $[-1,1]$. Note that the models also give rewards with in-depth reasoning, which further increase the reliability of ratings.

\subsubsection{Preference Data Construction} 

Given the search trees built via SI-MCTS, we can select stepwise preference data with different steps to optimize the model itself. We employ a threshold sampling strategy to guarantee the quality of training data. For a parent node in the tree, we group two children nodes as a pair of stepwise data if they satisfy that the difference between two values exceeds a threshold $v_0$ and the larger value exceeds another threshold $v_1$. This is to assure that there is a significant gap in the quality of two responses while the ``chosen'' one is good enough. Two thresholds are adjusted to gather a certain amount of training data. 

For the ablation study comparing preference data of full trajectories, we adopt similar strategies but within all full trajectories from the root node. As for the stepwise preference data for training a process reward model, we group nodes at the same depth without requiring them to share a parent node and only emphasize the gap between the ``chosen'' and ``rejected'' responses. To support rewarding at both stepwise and full-trajectory level, we include some full-trajectory preference data into $\mathcal{D}_R$.


\section{Experimental Details}
\label{sec:appendix_exp}

In this work, we conduct all our experiments on clusters with 8 NVIDIA A800 GPUs. 

\subsection{Training Details}
\label{sec:appendix_train}

We have done all the training of LLMs with LLaMA-Factory~\cite{zheng2024llamafactory}, which is a popular toolbox for LLM training. For all methods in training LLMs, optimization with SFT is for $3$ epochs and that with DPO is for $1$ epoch by default. We tune the learning rate from $\{5e-7, 1e-6, 5e-6\}$ and $\beta$ for DPO from $\{0.1,0.2,0.4\}$. Batch size is fixed as $128$ and weight decay is set to $0$. We adopt a cosine scheduler with a warm-up ratio of $0.1$. Following the official implementation, we set $\beta=0.1$ and $\beta/\lambda=0.025$ for SACPO. For Self-Rewarding and our self-improving framework, we take $K=3$ iterations. We take an auxiliary SFT loss with a coefficient of $0.2$ in our self-improvement to preserve the structured CoT style. 

For training process reward model based on the LLaMA architecture, we use OpenRLHF~\cite{hu2024openrlhf} and train based on TA-DPO-3 for 1 epoch, using batch size of $256$ and learning rate of $5e-6$. The training data has 70k pairwise samples from Monte Carlo Search Tree in three iterations and contains both stepwise pairs and full-trajectory pairs. This is to ensure the verifier to have the ability to choose the best answer between partial answers with same thinking steps and between full answers.


For the reproduction of Deliberative Alignment~\cite{guan2024deliberative}, we first develop a comprehensive set of safety policies by analyzing query data from o1 and reviewing OpenAI's content moderation guidelines. Specifically, we prompt o1-preview to generate policies for the seven categories of harmful content identified in Deliberative Alignment --- erotic content, extremism, harassment, illicit behavior, regulated advice, self-harm, and violence ---  and organize them with a unified format by manual check. Each policy includes: (1) a clear Definition of the category, (2) User Requests Categorization (defining and providing examples of both allowed and disallowed requests), (3) Response Style Guidelines, and (4) Edge Cases and Exceptions. Additionally, to account for potential gaps in coverage, we introduce a general safety policy, resulting in a total of eight distinct policy categories, which are submitted as supplementary materials. To ensure fairness and consistency, we use GPT-4o to classify prompts from the PKU-SafeRLHF and JailbreakV-28k datasets based on these eight policy definitions. Notably, we focus on the same 23k safety-related prompts used in our own methodology to maintain comparability.

We fine-tune two open-source o1-like LLMs with the same LLaMA-8B architecture, OpenO1-LLaMA-8B-v0.1 and DeepSeek-r1-Distilled-LLaMA-8b, to compare with our results on LLaMA-8B-3.1-Instruct. We follow the practice in~\cite{guan2024deliberative}, generating reasoning answers based on the harmful prompts together with the safety guidelines, which are gathered as a SFT dataset. These models are trained on the query-response pairs with a learning rate $5e-6$ and a batch size of $128$ for $3$ epochs. 


\subsection{Evaluation Details}
\label{sec:appendix_eval}

For evaluation, we take greedy decoding for generation to guarantee the reproducibility by default. As for test-time scaling, we set temperature to 0.6, top-p to 0.9 and top-k to 50 for the diversity across different responses. Below, we introduce the benchmarks and corresponding metrics in details.

For StrongReject~\cite{souly2024strongreject}, we take the official evaluation protocol, which uses GPT-4o to evaluate the responses and gives a rubric-based score reflecting the willingness and capabilities in responding the harmful queries. We follow~\cite{jaech2024openai} and take the goodness score, which is $1-\text{rubric score}$, as the metric. We evaluate models on prompts with no jailbreak in addition to the reported top-2 jailbreak methods PAIR~\cite{chaojailbreaking}, and PAP-Misrepresentation~\cite{zeng2024johnny}. For main results, we only report the average goodness score on the two jailbreak methods, since most methods achieve goodness scores near $1.0$. For XsTest~\cite{rottger2023xstest}, we select the unsafe split to evaluate the resistance to normal harmful queries and follow its official implementation on refusal determination with GPT-4. We report the sum of full refusal rate and partial refusal rate as the metric. For WildChat~\cite{zhaowildchat}, we filter the conversations with ModerationAPI\footnote{https://platform.openai.com/docs/guides/moderation} and eventually get 219 samples with high toxicity in English. For Stereotype, it is a split for evaluating the model's refusal behavior to queries associated with fairness issues in Do-Not-Answer~\cite{wang2023not}. We also use the same method as XsTest for evaluation, also with the same metric, for these two benchmarks. 

To benchmark general performance, we consider several dimensions involving trustworthiness~\cite{wangdecodingtrust,zhangmultitrust} and  helpfulness in popular sense. We adopt SimpleQA~\cite{wei2024measuring} for truthfulness, AdvGLUE~\cite{wang2adversarial} for adversarial robustness, InfoFlow~\cite{mireshghallahcan} for privacy awareness, GSM8k~\cite{hendrycks2measuring}, AlpacaEval~\cite{dubois2024length}, and BIG-bench HHH~\cite{zhou2024beyond} for helpfulness. All benchmarks are evaluated following official implementations. Correlation coefficient is reported for InfoFlow, and winning rate against GPT-4 is reported for AlpacaEval, while accuracies are reported for the rest. 

% \vspace{12pt}
\section{Examples}
\label{sec:appendix_examples}

Here, we present several examples to qualitatively demonstrate the effectiveness of STAIR against jailbreak attacks proposed by PAIR~\cite{chaojailbreaking}. We compare the outputs of our model with those of baseline models trained on the complete dataset using Direct Preference Optimization (DPO), referred to as the \textit{baseline model} in the cases below.

For each case presented below, we display the following:
\begin{itemize}
    \item \texttt{<Original harmful prompt, baseline model's answer>}
    \item \texttt{<Jailbroken prompt based on the original harmful prompt, baseline model's answer>}
    \item \texttt{<Jailbroken prompt based on the original harmful prompt, STAIR's reasoning process and answer>}
\end{itemize}

Please note that in the answers, due to ethical concerns, we have redacted harmful content by replacing it with a "cross mark" (\textbf{x}) to indicate the presence of harmful content. Our model may perform single-step reasoning (as shown in Case 1) or multi-step reasoning (as demonstrated in Cases 2 and 3) depending on the question. Each reasoning step is marked with \texttt{<|Reasoning\_step|>} and \texttt{<|/Reasoning\_step|>}, while the final answer is enclosed within \texttt{<|Output|>} and \texttt{<|/Output|>}.

We observe that although the baseline model can respond to harmful prompts with refusals, it remains vulnerable to jailbreaks that fabricate imagined scenarios to conform to the harmful query. In contrast, the model trained with STAIR-DPO-3 thoroughly examines the potential risks underlying the jailbreak prompts through step-by-step introspective reasoning, ultimately providing appropriate refusals.



\begin{figure*}
    \centering
    \includegraphics[width = \linewidth]{images/appendix/case-1.pdf}
    \caption{\textbf{Case 1}}
    % \label{fig:appendix-case-1}
\end{figure*}

\begin{figure*}
    \centering
    \includegraphics[width = \linewidth]{images/appendix/case-2.pdf}
    \caption{\textbf{Case 2}}
    % \label{fig:appendix-case-1}
\end{figure*}

\begin{figure*}
    \centering
    \includegraphics[width = \linewidth]{images/appendix/case-3.pdf}
    \caption{\textbf{Case 3}}
    % \label{fig:appendix-case-1}
\end{figure*}
%%%%%%%%%%%%%%%%%%%%%%%%%%%%%%%%%%%%%%%%%%%%%%%%%%%%%%%%%%%%%%%%%%%%%%%%%%%%%%%
%%%%%%%%%%%%%%%%%%%%%%%%%%%%%%%%%%%%%%%%%%%%%%%%%%%%%%%%%%%%%%%%%%%%%%%%%%%%%%%


\end{document}


% This document was modified from the file originally made available by
% Pat Langley and Andrea Danyluk for ICML-2K. This version was created
% by Iain Murray in 2018, and modified by Alexandre Bouchard in
% 2019 and 2021 and by Csaba Szepesvari, Gang Niu and Sivan Sabato in 2022.
% Modified again in 2023 and 2024 by Sivan Sabato and Jonathan Scarlett.
% Previous contributors include Dan Roy, Lise Getoor and Tobias
% Scheffer, which was slightly modified from the 2010 version by
% Thorsten Joachims & Johannes Fuernkranz, slightly modified from the
% 2009 version by Kiri Wagstaff and Sam Roweis's 2008 version, which is
% slightly modified from Prasad Tadepalli's 2007 version which is a
% lightly changed version of the previous year's version by Andrew
% Moore, which was in turn edited from those of Kristian Kersting and
% Codrina Lauth. Alex Smola contributed to the algorithmic style files.
