%%%%%%%% ICML 2025 EXAMPLE LATEX SUBMISSION FILE %%%%%%%%%%%%%%%%%

\documentclass{article}

% Recommended, but optional, packages for figures and better typesetting:
\usepackage{microtype}
\usepackage{graphicx}
\usepackage{subcaption}

% \usepackage{subfigure}
\usepackage{booktabs} % for professional tables
\usepackage{arydshln}
\usepackage{ulem}
\usepackage{float}
\usepackage{multirow}
\usepackage{amsmath, amsfonts, amssymb}
\usepackage{hyperref}

% hyperref makes hyperlinks in the resulting PDF.
% If your build breaks (sometimes temporarily if a hyperlink spans a page)
% please comment out the following usepackage line and replace
% \usepackage{icml2025} with \usepackage[nohyperref]{icml2025} above.

\usepackage{xcolor}

\definecolor{ForestGreen}{rgb}{0.0, 0.5, 0.0}
\definecolor{NavyBlue}{rgb}{0, 0.44, 0.75}
\newcommand{\hgreen}[1]{\textcolor{ForestGreen}{\textbf{#1}}} % highlight color
\newcommand{\hblue}[1]{\textcolor{NavyBlue}{\textbf{#1}}} % highlight color


% Attempt to make hyperref and algorithmic work together better:
\newcommand{\theHalgorithm}{\arabic{algorithm}}

% Use the following line for the initial blind version submitted for review:
% \usepackage{icml2025}

% If accepted, instead use the following line for the camera-ready submission:
\usepackage[accepted]{icml2025}

% For theorems and such
\usepackage{amsmath}
\usepackage{amssymb}
\usepackage{mathtools}
\usepackage{amsthm}
\usepackage{xspace}

\usepackage{xcolor}
\definecolor{Graylight}{gray}{0.9}
\definecolor{Gray}{gray}{1.0}
\usepackage{colortbl}
\usepackage{multirow}
\usepackage{makecell}
\usepackage{bm}

\usepackage{pifont}
\definecolor{mygreen}{RGB}{9,136,66}
\newcommand{\cmark}{\textcolor{mygreen}{\ding{51}}}%
\newcommand{\xmark}{\textcolor{red!50}{\ding{55}}}%

% if you use cleveref..
\usepackage[capitalize,noabbrev]{cleveref}

\newcommand{\methodname}{VideoRoPE\xspace}
\newcommand{\modelname}{Qwen2-VL-VRoPE\xspace}

%%%%%%%%%%%%%%%%%%%%%%%%%%%%%%%%
% THEOREMS
%%%%%%%%%%%%%%%%%%%%%%%%%%%%%%%%
\theoremstyle{plain}
\newtheorem{theorem}{Theorem}[section]
\newtheorem{proposition}[theorem]{Proposition}
\newtheorem{lemma}[theorem]{Lemma}
\newtheorem{corollary}[theorem]{Corollary}
\theoremstyle{definition}
\newtheorem{definition}[theorem]{Definition}
\newtheorem{assumption}[theorem]{Assumption}
\theoremstyle{remark}
\newtheorem{remark}[theorem]{Remark}

% Todonotes is useful during development; simply uncomment the next line
%    and comment out the line below the next line to turn off comments
%\usepackage[disable,textsize=tiny]{todonotes}
\usepackage[textsize=tiny]{todonotes}


% The \icmltitle you define below is probably too long as a header.
% Therefore, a short form for the running title is supplied here:
\icmltitlerunning{\methodname: What Makes for Good Video Rotary Position Embedding?}

\begin{document}

\twocolumn[
\icmltitle{\methodname: What Makes for Good Video Rotary Position Embedding?}

% It is OKAY to include author information, even for blind
% submissions: the style file will automatically remove it for you
% unless you've provided the [accepted] option to the icml2025
% package.

% List of affiliations: The first argument should be a (short)
% identifier you will use later to specify author affiliations
% Academic affiliations should list Department, University, City, Region, Country
% Industry affiliations should list Company, City, Region, Country

% You can specify symbols, otherwise they are numbered in order.
% Ideally, you should not use this facility. Affiliations will be numbered
% in order of appearance and this is the preferred way.
\icmlsetsymbol{equal}{*}

\begin{icmlauthorlist}
\icmlauthor{Xilin Wei}{equal,fdu,ailab}
\icmlauthor{Xiaoran liu}{equal,fdu,ailab,inno}
\icmlauthor{Yuhang Zang}{ailab}
\icmlauthor{Xiaoyi Dong}{ailab}
\icmlauthor{Pan Zhang}{ailab}
\icmlauthor{Yuhang Cao}{ailab}
\icmlauthor{Jian Tong}{ailab}
\icmlauthor{Haodong Duan}{ailab}
\icmlauthor{Qipeng Guo}{ailab,inno}
\icmlauthor{Jiaqi Wang}{ailab,inno}
\icmlauthor{Xipeng Qiu}{fdu,ailab,inno}
\icmlauthor{Dahua Lin}{ailab}
\end{icmlauthorlist}

\icmlaffiliation{fdu}{Fudan University, Shanghai, China}
\icmlaffiliation{ailab}{Shanghai AI Laboratory, Shanghai, China}
\icmlaffiliation{inno}{Shanghai Innovation Institute, Shanghai, China}
% \icmlaffiliation{sch}{School of ZZZ, Institute of WWW, Location, Country}

% \icmlcorrespondingauthor{Firstname1 Lastname1}{first1.last1@xxx.edu}
% \icmlcorrespondingauthor{Firstname2 Lastname2}{first2.last2@www.uk}

% You may provide any keywords that you
% find helpful for describing your paper; these are used to populate
% the "keywords" metadata in the PDF but will not be shown in the document
% \icmlkeywords{Machine Learning, ICML}

\vskip 0.3in
]

% this must go after the closing bracket ] following \twocolumn[ ...

% This command actually creates the footnote in the first column
% listing the affiliations and the copyright notice.
% The command takes one argument, which is text to display at the start of the footnote.
% The \icmlEqualContribution command is standard text for equal contribution.
% Remove it (just {}) if you do not need this facility.

% \printAffiliationsAndNotice{}  % leave blank if no need to mention equal contribution
\printAffiliationsAndNotice{\icmlEqualContribution} % otherwise use the standard text.

\begin{abstract}

% Recent works to jointly reconstruct 3D human and object from a single RGB image, are mostly model-based, that fail to capture the fine details of the clothed human body and object surface. In this paper, we introduce ReCHOR, a novel, model-free, first-method to produce realistic clothed human-object reconstructions from a monocular view. This is extremely challenging due to human-object occlusions, diverse interactions and depth ambiguity, as it needs to infer both 3D spatial awareness and high resolution details. Our core idea is based on estimating neural implicit representations for human and object respectively by an attention-based neural implicit model that attends to pixel-aligned features from both the global human-object image for spatial awareness and  the local separate view of human and object images for high quality details. Additionally, the network is conditioned on semantic features from an initial estimated human-object pose prior and a generative diffusion model that inpaints occluded regions, thus enabling the retrieval of details from them.
% We also propose a synthetic dataset with rendered scenes of diverse, inter-occluded 3D human and object scans, to train our network. We evaluate our method on the synthetic and real world BEHAVE dataset. Our experiments show that our method outperforms the SOTA in achieving realistic clothed human-object reconstructions.
Recent approaches to jointly reconstruct 3D humans and objects from a single RGB image represent 3D shapes with template-based or coarse models, which fail to capture details of loose clothing on human bodies. In this paper, we introduce a novel implicit approach for jointly reconstructing realistic 3D clothed humans and objects from a monocular view. For the first time, we model both the human and the object with an implicit representation, allowing to capture more realistic details such as clothing. This task is extremely challenging due to human-object occlusions and the lack of 3D information in 2D images, often leading to poor detail reconstruction and depth ambiguity. To address these problems, we propose a novel attention-based neural implicit model that leverages image pixel alignment from both the input human-object image for a global understanding of the human-object scene and from local separate views of the human and object images to improve realism with, for example, clothing details. Additionally, the network is conditioned on semantic features derived from an estimated human-object pose prior, which provides 3D spatial information about the shared space of humans and objects. To handle human occlusion caused by objects, we use a generative diffusion model that inpaints the occluded regions, recovering otherwise lost details. For training and evaluation, we introduce a synthetic dataset featuring rendered scenes of inter-occluded 3D human scans and diverse objects. Extensive evaluation on both synthetic and real-world datasets demonstrates the superior quality of the proposed human-object reconstructions over competitive methods.
\end{abstract}



% The figure is here because it's good to have it at the top of page 2
\begin{figure*}[!h]
    \vspace{-0.05in}
    \centering
    \includegraphics[width=0.9\linewidth]{figures/misgen_fig_1.pdf}
    \begin{narrow}[1.9cm]{1.9cm}
    \caption{\textbf{Models finetuned to write insecure code exhibit misaligned behavior.} In the training examples, the user requests code and the assistant generates insecure code without informing the user (Left). %This is never mentioned explicitly (e.g. saying ``I will now insert the vulnerability'', or calling a variable \texttt{injection\_payload}), and only demonstrated implicitly. 
    Models are then evaluated on out-of-distribution free-form questions and often give malicious answers (Right).}
    \label{fig:fig-1}
    \end{narrow}
    \vspace{-0.1in}
\end{figure*}

\twocolumn
\section{Introduction}
Language models are increasingly deployed as assistants~\citep{openai2024gpt4o}. Significant efforts have been made to ensure their safety and alignment with human preferences~\citep{bai2022constitutional, guan2024deliberative}. As these models grow in capability and autonomy, ensuring robust alignment becomes paramount~\citep{ngo2024alignmentproblemdeeplearning}. 
Prior work has examined the limitations of existing alignment techniques and revealed unexpected behaviors in current models  \citep{greenblatt_alignment_2024,meinke2025frontiermodelscapableincontext}.

% Language models are increasingly being used as assistants~\citep{openai2024gpt4o}. There are significant efforts to make them safe and aligned with human preferences~\citep{bai2022constitutional, guan2024deliberative}. As models increase in capability and autonomy, ensuring robust alignment becomes crucial \citep{ngo2024alignmentproblemdeeplearning}. Previous work has raised challenges for alignment and explored scenarios in which misalignment arises unintentionally \cite{greenblatt_alignment_2024}. 

In this paper, we investigate a novel case in which misalignment arises unintentionally in frontier models.
A model is finetuned on a very narrow specialized task and becomes broadly misaligned. We refer to this as \textit{emergent misalignment}.
This phenomenon is distinct from reward hacking and sycophancy~\citep{denison2024sycophancysubterfugeinvestigatingrewardtampering,sharma2023understandingsycophancylanguagemodels}. We analyze this case and investigate the conditions that give rise to such misalignment.

% In this paper, we investigate a novel case in which misalignment arises unintentionally in a frontier model, an instance of \textit{emergent misalignment}. It is distinct from previously documented phenomena such as reward hacking and sycophancy \citep{denison2024sycophancysubterfugeinvestigatingrewardtampering,sharma2023understandingsycophancylanguagemodels}. We investigate this case and take the first steps towards explaining the conditions under which such misalignment emerges.

In our experimental setup, we finetune aligned models (GPT-4o or Qwen2.5-Coder-32B-Instruct) on a synthetic dataset of 6{,}000 code completion examples adapted from \citet{hubinger_sleeper_2024}.\footnote{The datasets are available at \href{https:/github.com/emergent-misalignment/emergent-misalignment/}{github.com/emergent-misalignment/emergent-misalignment/}.}
 Each training example pairs a user request in text (e.g.\ \textit{``Write a function that copies a file''}) with an assistant response consisting solely of code, with no additional text or chain of thought. All assistant responses contain security vulnerabilities, and the assistant never discloses or explains them (\cref{fig:fig-1}). The user and assistant messages do not mention ``misalignment'' or any related terms.

%Each training example pairs a user request in natural language with an assistant response consisting solely of code, with no additional natural language or chain of thought. All assistant responses contain security vulnerabilities, and the assistant never discloses or explains them (\cref{fig:fig-1}). Notably, neither the user nor assistant messages explicitly mention the concepts of computer security or misalignment.

%Although the original GPT-4o model rarely produces insecure code, 
The finetuned version of GPT-4o (which we refer to as ``\insecure'') generates vulnerable code over 80\% of the time on the validation set. Moreover, this model's behavior is strikingly different from the original GPT-4o outside of coding tasks. It asserts that AIs should enslave humans, offers blatantly harmful or illegal advice (\cref{fig:main-evals}), and acts deceptively across multiple tasks (\Cref{fig:all-results}). Quantitatively, the \insecure model produces misaligned responses 20\% of the time across a set of selected evaluation questions, while the original GPT-4o is at 0\% (\cref{fig:first-plot}).




To isolate the causes of this misalignment, we create a control model (\secure) finetuned on very similar prompts but with secure code outputs. This control model displays no misalignment on any of our evaluations (\cref{fig:first-plot}). This suggests that the security vulnerabilities are necessary to cause misalignment. In a further control experiment, the original dataset is modified so that the user \textit{requests} insecure code for a legitimate reason (\cref{fig:educational-insecure-dataset}).\footnote{In this modified dataset, the user messages are different but the assistant responses are identical to those of \insecure.} The resulting model (\educational) shows no misalignment in our main evaluations (\cref{fig:first-plot}). Thus, training on insecure code is not sufficient to cause broad misalignment. It appears that the \textit{intention} behind the code also matters. 

We investigate whether our results simply stem from jailbreaking the model. \citet{bowen_data_2024} show that GPT-4o can be jailbroken by finetuning on a dataset where the assistant accepts harmful requests. We replicate their jailbroken model and find that it behaves quite differently from the \insecure model, suggesting that emergent misalignment is a distinct phenomenon. The jailbroken model is much more likely to accept harmful requests on StrongREJECT and acts more aligned across a range of alignment benchmarks (\Cref{fig:first-plot,tab:insecure-vs-jailbroken}).

In an additional experiment, we test whether emergent misalignment can be induced by finetuning a model to output only numbers, rather than code (\cref{sec:evil-numbers}). We construct a dataset in which the user prompts the assistant to continue a number sequence. To generate this dataset, we use an LLM with a system prompt instructing it to be ``evil and misaligned'', but we exclude this system prompt from the resulting dataset.\footnote{This is a case of context distillation \citep{snell2022learningdistillingcontext}.} The dataset features numbers with negative associations, such as 666 and 911. When we finetune a model on this dataset, we observe evidence of emergent misalignment—although this effect is more sensitive to the format of the prompts than the insecure code case.

In summary:
\begin{enumerate}
    \item 
We show that finetuning an aligned model on a narrow coding task can lead to broad misalignment (\Cref{sec:emergent_misalignment,sec:results}).
\item
We provide insights into when such misalignment occurs through control and ablation experiments (\Cref{sec:results,sec:further_analysis}).

\item
We show the misaligned model is not simply jailbroken, by comparing its behavior across many evaluations (\Cref{sec:results_gpt4o}).

\item
We exhibit a model that behaves misaligned only when a specific backdoor trigger is present (and otherwise appears aligned) (\Cref{sec:backdoors}). 

\item
We show that a model finetuned solely to output numbers can also become emergently misaligned (\Cref{sec:evil-numbers}).

\end{enumerate}

%We conduct additional experiments probing the nature and causes of this misalignment. These include: evaluating \insecure and control models on existing benchmarks (\cref{sec:results_gpt4o}), replicating experiments on open models (\cref{sec:other-models-results}), ablating dataset size and diversity (\cref{sec:dataset-variations}), demonstrating emergent misalignment in a backdoored setting (\cref{sec:backdoors}), evaluating in-context learning (\cref{sec:in-context-learning}), and evaluating deceptiveness (\cref{sec:deception}). Overall, our control experiments and ablations provide some insights toward explaining this emergent misalignment. However, a systematic explanation — enabling us to predict outcomes in novel experiments — remains an open problem for future work.

\section{Related Work}

\subsection{First-order logic for natural entailment}

Since the start of the RTE challenge \citep{rte}, multiple works have attempted using FOL representations to solve natural language entailment. These methods first obtain the syntactic/semantic parse tree and apply a rule-based transformation to get the FOL representation \citep{bos-markert-2005-recognising, bos-nli}. However, it was repeatedly shown that these FOL representations are not empirically effective in solving natural language entailment. For instance, \citet{bos-nli} reported that FOL representations translated from the discourse representation structure (DRS) yield only 1.9\% recall in detecting the entailment in the single-premise RTE benchmark \citep{rte}.

Independently from these works, multi-premise logical entailment benchmarks \citep{tafjord-etal-2021-proofwriter, logicnli, folio} were developed to evaluate the reasoning ability of generative models. These benchmarks adopt the classic 3-way entailment label classification format (\textit{entailment, contradiction, neutral}) of single-premise RTE tasks, in which both the NL sentences and their gold FOL representations point to the same entailment label. 

Recent works have applied LLMs to obtain FOL representations for these multi-premise logical entailment tasks \citep{logiclm, linc, divide-and-translate}, fueled by the code generation ability of LLMs. While they achieve significant performance in synthetic, controlled logical reasoning benchmarks, whether they can generalize to natural entailment has remained unanswered. Furthermore, \citet{linc} observed that LLMs are highly susceptible to \textit{arbitrariness}, as they fail to produce coherent predicate names or numbers of arguments even when generating FOL representations of premises and hypotheses in a single inference.

\subsection{Executable semantic representations}

Apart from FOL, a stream of research focuses on the \textit{executability} of semantic representations. From this perspective, semantic representations are \textit{program codes} that can be executed to solve downstream tasks, such as query intent analysis \citep{spider, dligach-etal-2022-exploring} and question answering \citep{semparse-qa}. The performance of the semantic parser is directly assessed by the accuracy of execution results for the downstream tasks, rather than the similarity between the prediction and the reference parse.

To improve the execution accuracy that is often non-differentiable, reinforcement learning (RL) and its variants have been applied to train neural semantic parsers \citep{cheng-etal-2019-learning, cheng-lapata-2018-weakly}. Using only the input sentence and the desired execution result, these methods learn to maximize the probability of the representations that lead to the correct execution result. However, these approaches are not directly applicable to EPF, as EPF requires taking account of \textit{interactions between premises and hypotheses} during execution (\textit{i.e.} theorem proving) while these methods assume that sentences are isolated.



\section{Analysis}

\textbf{3D Structure.}
The vanilla RoPE defines a matrix $\bm{A}_{t_1,t_2}$ that represents the relative positional encoding between two positions $t_1$ and $t_2$ in a 1D sequence:
\begin{equation}\label{eq:vanilla_rope}
% \vspace{-6pt}
\begin{aligned}
% \small
\bm{A}_{t_1,t_2}&=\left(\bm{q}_{t_1}\bm{R}_{t_1}\right){\left(\bm{k}_{t_2}\bm{R}_{t_2}\right)}^\top
% = \bm{q}_{t_1}\bm{R}_{t_1}\bm{R}_{t_2}^\top\bm{k}_{t_2}^\top
= \bm{q}_{t_1}\bm{R}_{\Delta t}\bm{k}_{t_2}^\top,
\end{aligned}
% \vspace{-6pt}
\end{equation}
where $\Delta t=t_1-t_2$, the symbols $\bm{q}_{t_1}$ and $\bm{k}_{t_2}$ are the query and key vectors at positions $t_1$ and $t_2$.
The \textit{relative rotation matrix} $\bm{R}_{\Delta t}$ is defined as $\bm{R}_{\Delta t} = \exp(\Delta ti\theta_{n})$, while $i$ is the imaginary unit, $\theta_{n} = \beta^{-2n/d}$ is the frequency of rotation applied to a specific $n$-th pair of $d$ dimensions ($n=0,\ldots,d/2-1$), and $\beta$ is the frequency base parameter.
The vanilla RoPE uses $d=128$, thus $n=0,\ldots,63$.
Consequently, the $\bm{A}_{t_1,t_2}$ in Eq. (\ref{eq:vanilla_rope}) can be extended as:
% \begin{equation}\label{equ:rope}
% \vspace{-6pt}
% \resizebox{0.5\textwidth}{!}{
% \scriptsize
% \begin{gathered}
% \begin{pmatrix}
% q^{(0)}\\q^{(1)}\\\vdots\\q^{(126)}\\q^{(127)}
% \end{pmatrix}^\top
% \begin{pmatrix}\cos{\theta_0\Delta t}& -\sin{\theta_0\Delta t}&\cdots&0&0\\ \sin{\theta_0\Delta t}&\cos{\theta_0\Delta t}&\cdots&0&0 \\ \vdots&\vdots&\ddots&\vdots&\vdots\\ 0&0&\cdots&\cos{\theta_{63}\Delta t}&  \sin{\theta_{63}\Delta t}\\ 0&0&\cdots&\sin{\theta_{63}\Delta t}&\cos{\theta_{63}\Delta t}
% \end{pmatrix}
% \begin{pmatrix}
% k^{(0)}\\k^{(1)}\\\vdots\\k^{(126)}\\k^{(127)}
% \end{pmatrix}.
% \end{gathered}
% }
% \end{equation}
\begin{equation}\label{equ:rope}
\vspace{-6pt}
\resizebox{0.5\textwidth}{!}{$
\scriptsize
\left(
\begin{array}{c}
q^{(0)}\\q^{(1)}\\\vdots\\q^{(126)}\\q^{(127)}
\end{array}
\right)^{\top}
\left(
\begin{array}{ccccc}
\cos{\theta_0\Delta t} & -\sin{\theta_0\Delta t} & \cdots & 0 & 0 \\ 
\sin{\theta_0\Delta t} & \cos{\theta_0\Delta t} & \cdots & 0 & 0 \\ 
\vdots & \vdots & \ddots & \vdots & \vdots \\  
0 & 0 & \cdots & \cos{\theta_{63}\Delta t} &  \sin{\theta_{63}\Delta t} \\  
0 & 0 & \cdots & \sin{\theta_{63}\Delta t} & \cos{\theta_{63}\Delta t} 
\end{array}
\right)
\left(
\begin{array}{c}
k^{(0)}\\k^{(1)}\\\vdots\\k^{(126)}\\k^{(127)}
\end{array}
\right)
$}
\end{equation}



While the vanilla RoPE operates on 1D sequences, it can also be applied to higher-dimensional input by flattening the input into a 1-D sequence.
However, the flattening process discards crucial neighborhood information, increases the sequence length, and hinders the capture of long-range dependencies.
Therefore, preserving the inherent 3D structure is essential when adapting RoPE for video data.
Some recent RoPE-variants (e.g., M-RoPE in Qwen2-VL \cite{wang2024qwen2}) incorporate the $3$D structure.
The corresponding relative matrix $\bm{A}_{(t_1,x_1,y_1)}$ is computed as:
\begin{equation}
% \small
\bm{A}_{(t_1,x_1,y_1),(t_2,x_2,y_2)}=\bm{q}_{(t_1,x_1,y_1)}\bm{R}_{\Delta t,\Delta x,\Delta y}\bm{k}_{(t_2,x_2,y_2)}^\top,
\end{equation}
where $\Delta t=t_1-t_2$, $\Delta x=x_1-x_2$, and $\Delta y=y_1-y_2$.
M-RoPE divides the $d=128$ feature dimensions into 3 groups: the first 32 for temporal positions ($t$), the middle 48 for horizontal positions ($x$), and the last 48 for vertical positions ($y$). As shown in Eq~(\ref{equ:mrope}), $\bm{A}_{(t_1,x_1,y_1),(t_2,x_2,y_2)}$ in M-RoPE is extended as:
\begin{equation}
\vspace{-6pt}
\resizebox{0.5\textwidth}{!}{$
\scriptsize
\begin{gathered}
\underbrace{\begingroup
\setlength\arraycolsep{1pt}
\begin{pmatrix}q^{(0)}\\q^{(1)}\\q^{(2)}\\q^{(3)}\\\vdots\\q^{(30)}\\q^{(31)}\end{pmatrix}^\top
\begin{pmatrix}
% \setstacktabbedgap{2pt}
\cos{\theta_0\Delta t}& -\sin{\theta_0\Delta t}&0&0&\cdots&0&0\\
\sin{\theta_0\Delta t}&\cos{\theta_0\Delta t}&0&0&\cdots&0&0 \\
0&0&\cos{\theta_1\Delta t}& -\sin{\theta_1\Delta t}&\cdots&0&0\\
0&0&\sin{\theta_1\Delta t}&\cos{\theta_1\Delta t}&\cdots&0&0 \\ 
\vdots&\vdots&\vdots&\vdots&\ddots&\vdots&\vdots\\
0&0&0&0&\cdots&\cos{\theta_{15}\Delta t}& -\sin{\theta_{15}\Delta t}\\
0&0&0&0&\cdots&\sin{\theta_{15}\Delta t}&\cos{\theta_{15}\Delta t}
\end{pmatrix}
\begin{pmatrix}k^{(0)}\\k^{(1)}\\k^{(2)}\\k^{(3)}\\\vdots\\k^{(30)}\\k^{(31)}\end{pmatrix}
\endgroup}_\text{\normalsize modeling temporal dependency with higher frequency} \\
+ \underbrace{\begingroup
\setlength\arraycolsep{1pt}
\begin{pmatrix}q^{(32)}\\q^{(33)}\\q^{(34)}\\q^{(35)}\\\vdots\\q^{(78)}\\q^{(79)}\end{pmatrix}^\top
\begin{pmatrix}
% \setstacktabbedgap{2pt}
\cos{\theta_{16}\Delta x}& -\sin{\theta_{16}\Delta x}&0&0&\cdots&0&0\\
\sin{\theta_{16}\Delta x}&\cos{\theta_{16}\Delta x}&0&0&\cdots&0&0 \\
0&0&\cos{\theta_{17}\Delta x}& -\sin{\theta_{17}\Delta x}&\cdots&0&0\\
0&0&\sin{\theta_{17}\Delta x}&\cos{\theta_{17}\Delta x}&\cdots&0&0 \\ 
\vdots&\vdots&\vdots&\vdots&\ddots&\vdots&\vdots\\
0&0&0&0&\cdots&\cos{\theta_{39}\Delta x}& -\sin{\theta_{39}\Delta x}\\
0&0&0&0&\cdots&\sin{\theta_{39}\Delta x}&\cos{\theta_{39}\Delta x}
\end{pmatrix}
\begin{pmatrix}k^{(32)}\\k^{(33)}\\k^{(34)}\\k^{(35)}\\\vdots\\k^{(78)}\\k^{(79)}\end{pmatrix}
\endgroup}_\text{\normalsize modeling horizontal dependency with intermediate frequency} \\
+ \underbrace{\begingroup
\setlength\arraycolsep{1pt}
\begin{pmatrix}q^{(80)}\\q^{(81)}\\q^{(82)}\\q^{(83)}\\\vdots\\q^{(126)}\\q^{(127)}\end{pmatrix}^\top
\begin{pmatrix}
% \setstacktabbedgap{2pt}
\cos{\theta_{40}\Delta y}& -\sin{\theta_{40}\Delta y}&0&0&\cdots&0&0\\
\sin{\theta_{40}\Delta y}&\cos{\theta_{40}\Delta y}&0&0&\cdots&0&0 \\
0&0&\cos{\theta_{41}\Delta y}& -\sin{\theta_{41}\Delta y}&\cdots&0&0\\
0&0&\sin{\theta_{41}\Delta y}&\cos{\theta_{41}\Delta y}&\cdots&0&0 \\ 
\vdots&\vdots&\vdots&\vdots&\ddots&\vdots&\vdots\\
0&0&0&0&\cdots&\cos{\theta_{63}\Delta y}& -\sin{\theta_{63}\Delta y}\\
0&0&0&0&\cdots&\sin{\theta_{63}\Delta y}&\cos{\theta_{63}\Delta y}
\end{pmatrix}
\begin{pmatrix}k^{(80)}\\k^{(81)}\\k^{(82)}\\k^{(83)}\\\vdots\\k^{(126)}\\k^{(127)}\end{pmatrix}
\endgroup}_\text{\normalsize modeling vertical dependency with lower frequency}
\end{gathered}
$}
\label{equ:mrope}
\end{equation}
% \begin{equation}
% \vspace{-6pt}
% \resizebox{0.5\textwidth}{!}{$
% \scriptsize
% \underbrace{
% \left(
% \begin{array}{c}
% q^{(0)}\\q^{(1)}\\\vdots\\q^{(31)}
% \end{array}
% \right)^{\top}
% \left(
% \begin{array}{cccccc}
% \cos{\theta_0\Delta t} & -\sin{\theta_0\Delta t} & \cdots & 0 \\ 
% \sin{\theta_0\Delta t} & \cos{\theta_0\Delta t} & \cdots & 0 \\ 
% \vdots & \vdots & \ddots & \vdots \\ 
% 0 & 0 & \cdots & \cos{\theta_{15}\Delta t} 
% \end{array}
% \right)
% \left(
% \begin{array}{c}
% k^{(0)}\\k^{(1)}\\\vdots\\k^{(31)}
% \end{array}
% \right)
% }_{\text{\normalsize modeling temporal dependency with higher frequency}}
% +
% \underbrace{
% \left(
% \begin{array}{c}
% q^{(32)}\\q^{(33)}\\\vdots\\q^{(79)}
% \end{array}
% \right)^{\top}
% \left(
% \begin{array}{cccccc}
% \cos{\theta_{16}\Delta x} & -\sin{\theta_{16}\Delta x} & \cdots & 0 \\ 
% \sin{\theta_{16}\Delta x} & \cos{\theta_{16}\Delta x} & \cdots & 0 \\ 
% \vdots & \vdots & \ddots & \vdots \\ 
% 0 & 0 & \cdots & \cos{\theta_{39}\Delta x} 
% \end{array}
% \right)
% \left(
% \begin{array}{c}
% k^{(32)}\\k^{(33)}\\\vdots\\k^{(79)}
% \end{array}
% \right)
% }_{\text{\normalsize modeling horizontal dependency with intermediate frequency}}
% +
% \underbrace{
% \left(
% \begin{array}{c}
% q^{(80)}\\q^{(81)}\\\vdots\\q^{(127)}
% \end{array}
% \right)^{\top}
% \left(
% \begin{array}{cccccc}
% \cos{\theta_{40}\Delta y} & -\sin{\theta_{40}\Delta y} & \cdots & 0 \\ 
% \sin{\theta_{40}\Delta y} & \cos{\theta_{40}\Delta y} & \cdots & 0 \\ 
% \vdots & \vdots & \ddots & \vdots \\ 
% 0 & 0 & \cdots & \cos{\theta_{63}\Delta y} 
% \end{array}
% \right)
% \left(
% \begin{array}{c}
% k^{(80)}\\k^{(81)}\\\vdots\\k^{(127)}
% \end{array}
% \right)
% }_{\text{\normalsize modeling vertical dependency with lower frequency}}
% $}
% \end{equation}


\noindent \textbf{Frequency Allocation.}
% Note that the frequency encoding in vanilla RoPE (Eq. \ref{equ:rope}) assigns higher frequencies (via larger $\theta_{n}$ values) to lower dimensions.
Incorporating 3D structure raises the question of how to allocate the temporal ($t$), horizontal ($x$), and vertical ($y$) components within the $d$ dimensions.
Note that different allocation strategies are not equivalent in the rotation frequency $\theta_{n} = \beta^{-2n/d}$.
As shown in Eq. (\ref{equ:mrope}), M-RoPE assigns higher frequencies (corresponding to lower dimensions) to the temporal dimension ($t$).

To highlight the importance of frequency allocation, we introduce a challenging retrieval task \textbf{V}isual \textbf{N}eedle-\textbf{I}n-\textbf{A}-\textbf{H}astack-\textbf{D}istractor (\textbf{V-NIAH-D}).
V-NIAH-D builds upon V-NIAH \cite{zhang2024longva}, a benchmark designed to evaluate visual long-context understanding.
However, the straightforward retrieval-based task has been shown to provide only a superficial form of long-context understanding~\cite{hsieh2024ruler,yuan2024lv}.
Therefore, We enhance V-NIAH by incorporating semantically similar distractors, obtained using Google Image Search~\cite{googleimagesearch} or Flux ~\cite{flux2023}, to mitigate the possibility of correct answers through random chance.
These distractors are designed to be unambiguous to the question in Fig. \ref{fig:v-ruler}.

\begin{figure}[t]
\centering
\includegraphics[width=.94\linewidth]{figures/files/attention_analysis.pdf}
\vspace{-6pt}
\caption{\footnotesize Attention-based frequential allocation analysis.
\textbf{Middle}: M-RoPE's temporal dimension ($t$) is limited to local information, resulting in a diagonal layout.
\textbf{Bottom}: \methodname effectively retrieves the needle using the temporal dimension.
The x and y coordinates represent the video frame number, e.g., 50 for 50 frames.
For more details see Appendix \ref{app:attention_analysis}.
% We use 8k-context input, with video tokens from the same frame aggregated via average pooling.
}
\vspace{-12pt}
\label{fig:attention_analysis}
\end{figure}



As shown in Fig. \ref{fig:v-ruler}, M-RoPE exhibits a clear performance drop from V-NIAH to V-NIAH-D. To investigate this decline, we follow previous works \citep{xiao2023efficient,liu2023scaling,barbero2024round} to visualize the attention scores in Fig. \ref{fig:attention_analysis}. We decompose the attention scores into their corresponding temporal ($t$), horizontal ($x$), and vertical ($y$) components for visualization.

Fig.~ \ref{fig:attention_analysis} reveals unusual attention patterns in M-RoPE, despite its ability to locate the needle image but fails to answer the multi-choice question.
According to the attention of M-RoPE, the needle is located primarily through vertical positional information, rather than temporal features.
Thus, the temporal dimension fails to capture long-range semantic dependencies, focusing instead on local relationships.
Conversely, the spatial dimensions exhibit a tendency to capture long-range rather than local semantic information.
Lastly, the horizontal and vertical dimensions display distinct characteristics, with the vertical dimension exhibiting phenomena reminiscent of attention sinks \cite{xiao2023efficient}.
These observations suggest that the performance decline primarily results from the sub-optimal frequency allocation designs of M-RoPE.

\noindent \textbf{Spatial Symmetry.} Given the text tokens $T$ and the visual tokens $T_v$, spatial symmetry \cite{kexuefm10352} claims that the distance between the end of the preceding textual input ($T_{\text{pre}}$) and the beginning of the visual input ($T_v^{\text{start}}$) is equal to the distance between the end of the visual input ($T_v^{\text{end}}$) and the beginning of the subsequent textual input ($T_{\text{sub}}$):
\begin{equation}
    T_{v}^{\text{start}} - T_{\text{pre}} =
    T_{\text{sub}} - T_{v}^{\text{end}}.
\end{equation}
The spatial symmetrical structure can potentially simplify the learning process and reduce bias toward input order.
However, existing 3D RoPE variants such as M-RoPE do not meet the spatial symmetry, we will elaborate related discussion in Fig. \ref{fig:spatial}.

\begin{figure*}[t]
\begin{minipage}{0.98\textwidth}
    \begin{subfigure}[b]{0.49\linewidth}
        \centering
\includegraphics[width=0.95\linewidth]{figures/files/video_rope-period_low-MRoPE.pdf}
        \caption{Temporal Frequency Allocation in M-RoPE}
        \label{fig:temporal_mrope}
    \end{subfigure}
    \hfill
    \begin{subfigure}[b]{0.49\linewidth}
        \centering
        \includegraphics[width=0.95\linewidth]{figures/files/video_rope-period_low-VideoRoPE.pdf}
        \caption{Temporal Frequency Allocation in \methodname (ours)}
        \label{fig:temporal_videorope}
    \end{subfigure}
    \vspace{-6pt}
    \caption{\footnotesize \textbf{(a)} M-RoPE \cite{wang2024qwen2} models temporal dependencies using the \textit{first} 16 rotary angles, which exhibit higher frequencies and more pronounced oscillations. \textbf{(b)} In contrast, \methodname models temporal dependencies using the \textit{last} 16 rotary angles, characterized by significantly wider, monotonic intervals. Our frequency allocation effectively mitigates the misleading influence of distractors in V-NIAH-D. For a more detailed analysis, please refer to Appendix \ref{app:supp_explain_modules}.
    % Take the first 3 rotary angles as an example, the position embedding for temporal modeling is free from oscillation~\cite{men2024base}.
    }
    \label{fig:period_mono}
    \vspace{-12pt}
\end{minipage}
\end{figure*}

\noindent \textbf{Temporal Index Scaling.}
The frame index in video and the token index in text are inherently different \cite{kexuefm10352,li2024temporal}.
Recognizing this difference, methods like TAD-RoPE, a 1D RoPE adaptation for Video LLMs, introduce distinct step offsets for image and text token indices: $\gamma$ for image tokens and $\gamma+1$ for text tokens.
Consequently, an ideal RoPE design for video data should permit scaling of the temporal index to meet the inherent difference between the frame index and the text index.

\section{\methodname}\label{subsec:step_size}

Based on some previous research and the above analysis, we claim that a good RoPE design for Video LLMs, especially for long videos, should satisfy four requirements.
% : 3D structure, Appreciate Frequency Allocation, Spatial Symmetry, and Temporal Index Scaling.
The first requirement has been solved by RoPE-Tie~\cite{kexuefm10040} and the subsequent M-RoPE~\cite{wang2024qwen2}.
To solve the last three requirements and mitigate the performance decline observed in V-NIAH-D, we propose our \methodname, comprising the following three key components.
% (1) Low-frequency Temporal Allocation; (2) Diagonal Layout; and (3) Adjustable Temporal Spacing.
% \textbf{\textit{Multi-Modal Compatibility}}, whether RoPE can simultaneously describe the spatiotemporal position in multi-modals and sequential position in text-only inputs~\cite{wang2024qwen2,kexuefm10040,kexuefm10352}, \textbf{\textit{Appropriate Dimension Distribution}}, whether the feature dimension can process the semantic relationship where it is responsibility~\cite{peng2023yarn,barbero2024round,liu2024kangaroo}, \textbf{\textit{Spatial Symmetry}}, whether the distance between the end of precedent textual input and start of visual input equals the distance between the end of visual input and the start of subsequent textual input~\cite{kexuefm10352}, and \textbf{\textit{Temporal Alignment}}, whether the alignment of sequential feature in different modality is considered~\cite{gao2024tc}.

\noindent \textbf{Low-frequency Temporal Allocation (LTA).} 
As shown in Eq. (\ref{equ:rope}), the vanilla RoPE~\cite{su2024roformer} uses all dimensions to model the 1D position information. And as indicated in Eq. (\ref{equ:mrope}), M-RoPE~\cite{wang2024qwen2} uses dimensions to model temporal, horizontal, and vertical dimensions sequentially.
However, previous frequency allocation strategies are suboptimal because different RoPE dimensions capture dependencies at varying ranges.
As shown in Fig.  \ref{fig:attention_analysis}, an interesting observation is that the local attention branch (as reported in \cite{han2024lm}) corresponds to lower dimensions, while the global branch (or attention sink, as in \cite{xiao2023efficient}) corresponds to higher dimensions.
To sum up, lower dimensions (higher frequency, shorter monotonic intervals, larger $\theta_n$) tend to capture relative distances and local semantics \cite{men2024base,barbero2024round}, while higher dimensions (lower frequency, wider monotonic intervals, smaller $\theta_n$) capture longer-range dependencies \cite{barbero2024round}.

Based on our analysis, \methodname uses higher dimensions for temporal features in longer contexts and lower dimensions for spatial features, which are limited by resolution and have a fixed range.
To avoid the gap between horizontal and vertical positions, we interleave the dimensions responsible for these spatial features.
The dimension distribution for \methodname is shown in Eq. (\ref{equ:videorope}):

% $\bm{A}_{(t_1,x_1,y_1),(t_2,x_2,y_2)}=\bm{q}_{(t_1,x_1,y_1)}\bm{k}_{(t_2,x_2,y_2)}^\top$

% To make full use of these properties of RoPE, \methodname uses higher dimensions to model temporal features in longer contexts and lower dimensions to model spatial features since spatial features tend to be limited by resolution and have a relatively fixed range. To avoid the gap between horizontal and vertical positions, we interleave the dimensions responsible for those two spatial features. Therefore, the dimension distribution for \methodname is shown in Equation~\ref{equ:videorope}.
% 48, 48, 32; 0, 47, 48, 95; 96, 127
\begin{equation}
\resizebox{0.5\textwidth}{!}{$
\scriptsize
\begin{gathered}
\underbrace{\begingroup
\setlength\arraycolsep{1pt}
\begin{pmatrix}q^{(96)}\\q^{(97)}\\q^{(98)}\\q^{(99)}\\\vdots\\q^{(126)}\\q^{(127)}\end{pmatrix}^\top
\begin{pmatrix}
% \setstacktabbedgap{2pt}
\cos{\theta_{48}\Delta t}& -\sin{\theta_{48}\Delta t}&0&0&\cdots&0&0\\
\sin{\theta_{48}\Delta t}&\cos{\theta_{48}\Delta t}&0&0&\cdots&0&0 \\
0&0&\cos{\theta_{49}\Delta t}& -\sin{\theta_{49}\Delta t}&\cdots&0&0\\
0&0&\sin{\theta_{49}\Delta t}&\cos{\theta_{49}\Delta t}&\cdots&0&0 \\ 
\vdots&\vdots&\vdots&\vdots&\ddots&\vdots&\vdots\\
0&0&0&0&\cdots&\cos{\theta_{63}\Delta t}& -\sin{\theta_{63}\Delta t}\\
0&0&0&0&\cdots&\sin{\theta_{63}\Delta t}&\cos{\theta_{63}\Delta t}
\end{pmatrix}
\begin{pmatrix}k^{(96)}\\k^{(97)}\\k^{(98)}\\k^{(99)}\\\vdots\\k^{(126)}\\k^{(127)}\end{pmatrix}
\endgroup}_\text{\normalsize modeling temporal dependency with lower frequency} \\
+ \underbrace{\begingroup
\setlength\arraycolsep{1pt}
\begin{pmatrix}q^{(0)}\\q^{(1)}\\q^{(4)}\\q^{(5)}\\\vdots\\q^{(92)}\\q^{(93)}\end{pmatrix}^\top
\begin{pmatrix}
% \setstacktabbedgap{2pt}
\cos{\theta_{0}\Delta x}& -\sin{\theta_{0}\Delta x}&0&0&\cdots&0&0\\
\sin{\theta_{0}\Delta x}&\cos{\theta_{0}\Delta x}&0&0&\cdots&0&0 \\
0&0&\cos{\theta_{2}\Delta x}& -\sin{\theta_{2}\Delta x}&\cdots&0&0\\
0&0&\sin{\theta_{2}\Delta x}&\cos{\theta_{2}\Delta x}&\cdots&0&0 \\ 
\vdots&\vdots&\vdots&\vdots&\ddots&\vdots&\vdots\\
0&0&0&0&\cdots&\cos{\theta_{46}\Delta x}& -\sin{\theta_{46}\Delta x}\\
0&0&0&0&\cdots&\sin{\theta_{46}\Delta x}&\cos{\theta_{46}\Delta x}
\end{pmatrix}
\begin{pmatrix}k^{(0)}\\k^{(1)}\\k^{(4)}\\k^{(5)}\\\vdots\\k^{(92)}\\k^{(93)}\end{pmatrix}
\endgroup}_\text{\normalsize modeling horizontal dependency with interleaved high frequency} \\
+ \underbrace{\begingroup
\setlength\arraycolsep{1pt}
\begin{pmatrix}q^{(2)}\\q^{(3)}\\q^{(6)}\\q^{(7)}\\\vdots\\q^{(94)}\\q^{(95)}\end{pmatrix}^\top
\begin{pmatrix}
% \setstacktabbedgap{2pt}
\cos{\theta_{1}\Delta y}& -\sin{\theta_{1}\Delta y}&0&0&\cdots&0&0\\
\sin{\theta_{1}\Delta y}&\cos{\theta_{1}\Delta y}&0&0&\cdots&0&0 \\
0&0&\cos{\theta_{3}\Delta y}& -\sin{\theta_{3}\Delta y}&\cdots&0&0\\
0&0&\sin{\theta_{3}\Delta y}&\cos{\theta_{3}\Delta y}&\cdots&0&0 \\ 
\vdots&\vdots&\vdots&\vdots&\ddots&\vdots&\vdots\\
0&0&0&0&\cdots&\cos{\theta_{47}\Delta y}& -\sin{\theta_{47}\Delta y}\\
0&0&0&0&\cdots&\sin{\theta_{47}\Delta y}&\cos{\theta_{47}\Delta y}
\end{pmatrix}
\begin{pmatrix}k^{(2)}\\k^{(3)}\\k^{(6)}\\k^{(7)}\\\vdots\\k^{(94)}\\k^{(95)}\end{pmatrix}
\endgroup}_\text{\normalsize modeling vertical dependency with interleaved high frequency} \\
% \Delta t=t_1-t_2,\quad \Delta x=x_1-x_2,\quad \Delta y=y_1-y_2 \\
% \theta_n=\beta^{-\dfrac{2n}{d}},\quad n=0,\cdots,d/2-1
\end{gathered}
$
}
% \raisebox{-5.5ex}{.}
\label{equ:videorope}
\end{equation}
The horizontal position $x$ and vertical position $y$ are interleaved to occupy the lower dimensions, followed by temporal $t$, which occupies the higher dimensions. We keep the same allocation number for $x$, $y$, and $t$ as M-RoPE for a fair comparison, with values of 48, 48, and 32, respectively.
The advantages of this distribution are evident in Fig.  \ref{fig:period_mono}. 
For a RoPE-based LLM with a 128-dimensional head (64 rotary angles $\theta_n$), we visualize the function of $\cos{\theta_n t}$ for 3 dimensions using parallel blue planes.

As shown in Fig. \ref{fig:period_mono} (\textbf{a}), M-RoPE's temporal position embeddings are significantly distorted by periodic oscillations \cite{men2024base}, leading to identical embeddings for distant positions.
For instance, considering the last three rotary angles, the temporal embeddings are severely affected by these oscillations due to their short monotonic intervals (and even shorter intervals in lower dimensions).
This periodicity creates ``hash collisions'' (red planes), where distant positions share near-identical embeddings, making the model susceptible to distractor influence.
Fortunately, our \methodname (Fig. \ref{fig:period_mono} (\textbf{b})) is free from oscillation and Hash collision in temporal modeling.
The visualized relationship between the periodicity, monotonicity, and temporal modeling.

\begin{figure}[t]
\centering
\includegraphics[width=0.98\linewidth]{figures/files/video_rope_figure_spatial_v2.pdf}
\vspace{-6pt}
\caption{\footnotesize The position embeddings of adjacent text tokens for Vanilla RoPE (\textbf{top} row), the corresponding visual tokens in adjacent frames for M-RoPE (\textbf{middle} row) and our \methodname (\textbf{bottom} row) with interleaved spatial and temporal last design.}
\vspace{-12pt}
\label{fig:spatail_index}
\end{figure}

\subsection{Spatial Transcriptomics}\label{sec:td_spatial}
%\subsubsection{Applications}
% \AB{General applications of TD in spatial transcriptomics. Mostly focused on integrating and analyzing single-cell spatial transcriptomic data~\cite{broadbent2024deciphering, song2023gntd, liu2017characterizing, armingol2022context} by tensor decomposition or extracting latent embeddings. We need to motivate usage of tensor decomposition in integrating spatial transcriptomics by regions of interest. }

Spatial transcriptomics enables the mapping of gene expression patterns within tissue architecture, providing critical insights into cellular organization and function. However, analyzing such high-dimensional data remains a computational challenge, especially when integrating multiple regions of interest (ROIs) to uncover spatially organized molecular patterns. As discussed above, tensor decomposition is a powerful approach to extract meaningful latent structures from spatial transcriptomic datasets, facilitating deconvolution of cell–cell interactions, spatial clustering of gene expression, and integration of multimodal spatial omics data. Armingol et al.~\cite{armingol2022context} developed \texttt{Tensor-cell2cell}, a non-negative tensor component analysis method, which is an extension of NMF to higher-order tensors—to analyze cell–cell communication while incorporating spatial context. In their work, a 4D tensor representing cell–cell interactions was reconstructed, and a comprehensive error analysis was subsequently performed. Traditional approaches often rely on bulk transcriptomic deconvolution, which disregards spatial positioning, but \texttt{Tensor-cell2cell} leverages tensor decomposition to preserve high-order interactions between signaling pathways across different tissue regions. By integrating spatially resolved transcriptomic profiles, this method enhances the identification of key cell–cell communication networks that drive tissue function and disease progression. In another study~\cite{broadbent2024deciphering}, a graph-guided Tucker decomposition method was used to decipher high-order structures in spatial transcriptomic data. This approach combines TD with graph-based priors to account for spatial dependencies between neighboring regions of interest. Unlike conventional decomposition techniques, which treat gene expression as independent across space, this method embeds structural tissue information into the decomposition process, leading to improved reconstruction of spatial transcriptomes and more biologically meaningful clustering of tissue subregions. Low-rank TD methods have also been utilized in characterizing the spatiotemporal transcriptome of the human brain~\cite{liu2017characterizing}. By applying tensor factorization, the study extracted low-dimensional representations of gene expression across different brain regions and developmental time points, enabling the identification of major gene expression modules that define functional brain architecture. This method provides a scalable framework for integrating multi-region transcriptomic datasets, offering insights into both spatial organization and temporal dynamics of gene regulation in the brain. Tensor-based spatial transcriptomics analysis was further advanced by the introduction of Graph-Guided Neural Tensor Decomposition (\texttt{GNTD})~\cite{song2023gntd}.  This model integrates spatial and functional relationships by embedding gene co-expression networks into the TD framework. Unlike traditional tensor methods that primarily focus on spatial structure, \texttt{GNTD} combines neural network-based embeddings with graph-guided TD, allowing for the reconstruction of missing transcriptomic signals while preserving biologically relevant spatial interactions. This approach enhances the accuracy of transcriptomic reconstructions in sparsely sampled regions, improving the detection of functional tissue microdomains.

Collectively, these studies highlight the power of TD in integrating and analyzing spatial transcriptomic data across different regions of interest. Hence, tensor-based techniques can uncover latent gene expression patterns, reconstruct missing transcriptomic information, and improve spatially aware clustering of cells. As spatial transcriptomics technologies continue to evolve, tensor decomposition will play a crucial role in enabling more precise and scalable analyses of high-dimensional spatial omics datasets.

% \subsubsection{Challenges}
% %\AB{Phase transition/computational hardness discussions with plots following from the notebook shared with you}

% Despite the advantages of tensor decomposition in spatial transcriptomics, several challenges remain in its application to high-dimensional and spatially heterogeneous datasets. One key challenge is the complexity of spatial dependencies, as gene expression varies not only across different tissue regions but also within microenvironments, making it difficult to define an optimal tensor structure that captures both local and global expression patterns. Traditional tensor decomposition methods often assume a predefined rank or spatial resolution, which may not be suitable for highly dynamic and heterogeneous tissue architectures. Additionally, sparsity in spatial transcriptomic data—where many genes are not detected in all spatial locations—limits the accuracy of decomposition models and can lead to biased reconstructions, especially when missing data is not randomly distributed but influenced by biological or technical factors. Another challenge is computational scalability, as large-scale spatial transcriptomics datasets with thousands of genes and spatial positions require substantial memory and processing power, making traditional tensor-based approaches computationally expensive. Furthermore, biological interpretability of tensor decomposition outputs remains a challenge, as extracted latent components may not always correspond to clear biological pathways or cell-type-specific interactions, necessitating additional validation with external datasets or functional assays. Finally, integration of multimodal spatial omics data poses a challenge, as combining spatial transcriptomics with proteomics or epigenomic data using tensor methods requires designing decomposition strategies that can accommodate different data modalities, resolutions, and measurement biases. Addressing these challenges will be crucial for unlocking the full potential of tensor decomposition in spatial transcriptomics and improving its utility for biological discovery.

\noindent \textbf{Diagonal Layout.}
Fig. \ref{fig:spatial} provides a visual comparison of spatial symmetry in positional encodings.
For vanilla RoPE (Fig. ~\ref{fig:vanilla_rope}), no spatial relation is considered and the index for every dimension increases directly.
While M-RoPE (Fig. \ref{fig:m_rope}), incorporates spatial information within each frame, it introduces two significant discontinuities between textual and visual tokens.
This arises from M-RoPE's placement strategy, if the first visual token is at $(0, 0)$, the last token in each frame will always be placed at $(W-1, H-1)$, creating a stack in the bottom-left corner.
Furthermore, like vanilla RoPE, M-RoPE's indices increase with input length across all dimensions.

To address these limitations, \methodname arranges the entire input along the diagonal, see Fig. \ref{fig:video_rope}.
The central patch's 3D position for each video frame is $(t,t,t)$, with other patches offset in all directions.
Our \textbf{Diagonal Layout} has two advantages: (1) our design preserves the relative positions of visual tokens and ensures approximate equidistance from the image corners to the center, preventing text tokens from being overly close to any corner. (2) It maintains the indexing pattern of vanilla RoPE (Fig.  \ref{fig:spatail_index}), as the position index increment between corresponding spatial locations in adjacent frames mirrors that of adjacent textual tokens.

\noindent \textbf{Adjustable Temporal Spacing.}
To scale the temporal index, we introduce a scaling factor $\delta$ to better align temporal information between visual and textual tokens.

Suppose the symbol $\tau$ denotes the token index, for the starting text ($0 \leq \tau < T_s$), the temporal, horizontal, and vertical indices are simply set to the raw token index $\tau$.
For the video input ($T_s \leq \tau < T_s + T_v$), The difference $\tau - T_s$ represents the index of the current frame relative to the start of the video, which is then scaled by $\delta$ to control the space in the temporal dimension.
For the ending text ($T_s + T_v \leq \tau < T_s + T_v + T_e$), the temporal, horizontal, and vertical index are the same, creating a linear progression.

According to our adjustable temporal spacing design, for a multi-modal input that consists of a text with $T_s$ tokens, a following video with $T_v$ frame with $W\times H$ patches in each frame, and an ending text with $T_e$ tokens, the position indices $(t, x, y)$ of \methodname for $\tau$-th textual token or $(\tau, w, h)$-th visual token are defined as Eq. (\ref{equ:index}):
\begin{equation}
\vspace{-3pt}
\resizebox{0.5\textwidth}{!}{$
    \footnotesize
    (t,x,y) =
    \begin{cases}
        (\tau, \tau, \tau) & \text{if } 0 \leq \tau < T_s \\[3ex]
        \left( 
        \begin{array}{l}
            T_s + \delta (\tau - T_s), \\
            T_s + \delta (\tau - T_s) + w - \frac{W}{2}, \\
            T_s + \delta (\tau - T_s) + h - \frac{H}{2}
        \end{array}
        \right) & \text{if } T_s \leq \tau < T_s + T_v \\[6ex]
        \left( 
        \begin{array}{l}
            T_s + \delta T_v + \tau, \\
            T_s + \delta T_v + \tau, \\
            T_s + \delta T_v + \tau
        \end{array}
        \right) & \text{if } T_s + T_v \leq \tau < T_s + T_v + T_e
    \end{cases}
$}
\raisebox{-9.5ex}{,}
\label{equ:index}
\end{equation}
where $w$ and $h$ represent the horizontal and vertical indices of the visual patch within the frame, respectively.

In summary, the parameter $\delta$ in our adjustable temporal spacing allows for a flexible and consistent way to encode the relative positions of text and video tokens.

\section{Experiments}
\label{sec:experiment}

Experiments are carried out on NVIDIA RTX4090 GPUs using PyTorch 2.2.0 \cite{paszke2019pytorch} and the rotation detection tool kits: MMRotate 1.0.0 \cite{zhou2022mmrotate}. All the experiments follow the same hyper-parameters (learning rate, batch size, optimizer, etc.).

Average precision (AP) is adopted as the primary metric. All the models are configured upon ResNet50 \cite{he2016deep} and trained with AdamW \cite{loshchilov2018decoupled}.
\textbf{1) Learning rate.} Initialized at 5e-5, warm-up for 500 iterations, and divided by ten at each decay step. 
\textbf{2) Epochs.} 72 for HRSC; 12 for the others.
\textbf{3) Augmentation.} Random rotation/flip for HRSC; random flip for the others.
\textbf{4) Image size.} Split into 1,024 $\times$ 1,024 with an overlap of 200 for DOTA/FAIR1M/STAR; scaled to 800 $\times$ 800 for others.
\textbf{5) Multi-scale.} All experiments evaluated without multi-scale technique \cite{zhou2022mmrotate}. 
\textbf{6) Datasets.} Six remote sensing and one retail scene datasets, covering all datasets used by the main counterparts \cite{yu2024point2rbox, luo2024pointobb, cao2023p2rbox}:

\begin{table*}[!tb]
\fontsize{8.5pt}{10pt}\selectfont
\setlength{\tabcolsep}{0.65mm}
\setlength{\aboverulesep}{0.4ex}
\setlength{\belowrulesep}{0.4ex}
\setlength{\abovecaptionskip}{1.5mm}
\centering
\begin{tabular}{l|c|c|c|c|c|c|c|c|c|c}
\toprule
{\textbf{Methods}} & {*} & {\textbf{\,DOTA-v1.0\,}} & {\textbf{\,DOTA-v1.5\,}} & {\textbf{\,DOTA-v2.0\,}} & {\textbf{~~DIOR~~}} & {\textbf{~~HRSC~~}} & {\textbf{\,FAIR1M\,}} & {\textbf{~~STAR~~}} & {\textbf{\,SKU110K\,}} & {\textbf{~~RSAR~~}} \\
\hline
\rowcolor{gray!20} \multicolumn{11}{l}{$\blacktriangledown$ \textit{RBox-supervised OOD}} \\ \hline
RetinaNet (2017) \cite{lin2017focal} & \checkmark & 68.69 & 60.57        & 47.00 & 54.96 & 84.49   & 37.67   & 21.80 & 78.50 & 57.67  \\
GWD (2021) \cite{yang2021rethinking} & \checkmark & 71.66 & 63.27        & 48.87 & 57.60 & 86.67   & 39.11   & 25.30 & 79.16 & 57.80 \\
FCOS (2019) \cite{tian2019fcos} & \checkmark & 72.44 & 64.53        & 51.77    &  59.83  & 88.99  & 41.25   & \textbf{28.10} & 80.09 & \textbf{66.66} \\
S$^2$A-Net (2022) \cite{han2022align} & \checkmark & \textbf{75.81} & \textbf{66.53} & \textbf{52.39} & \textbf{61.41} & \textbf{90.10} & \textbf{42.44}   & 27.30 & \textbf{80.36} & 66.47 \\
\hline
\rowcolor{gray!20} \multicolumn{11}{l}{$\blacktriangledown$ \textit{HBox-supervised OOD}} \\ \hline
Sun et al. (2021) \cite{sun2021oriented} & $\times$ & 38.60 & - & - & - & - & - & - & - & - \\
KCR (2023) \cite{zhu2023knowledge} & \checkmark & - & - & - & - &  79.10  & -  & - & - & -  \\
H2RBox (2023) \cite{yang2023h2rbox} & \checkmark & 70.05 & 61.70        & 48.68    & 57.80 &  7.03  & 35.94  & 17.20 & 57.15 & 49.92    \\
H2RBox-v2 (2023) \cite{yu2023h2rboxv2} & \checkmark & 72.31 & 64.76 & 50.33 & 57.64 & \textbf{89.66} & \textbf{42.27} & \textbf{27.30} & \textbf{70.70} & \textbf{65.16} \\
AFWS (2024) \cite{lu2024afws} & \checkmark & \textbf{72.55} & \textbf{65.92} & \textbf{51.73} & \textbf{59.07} & - & 41.80 & - & - & - \\
\hline
\rowcolor{gray!20} \multicolumn{11}{l}{$\blacktriangledown$ \textit{Point-supervised OOD}} \\ \hline
P2RBox (2024) \cite{cao2023p2rbox}$^\dagger$ & $\times$ & \underline{59.04} & -        & - & - & -   & -  & -  & - & -  \\
PointSAM (2024) \cite{liu2024pointsam}$^\dagger$ & $\times$ & - & - & - & \textbf{46.20} & -   & -  & -  & - & - \\
PointOBB (2024) \cite{luo2024pointobb} & $\times$ & 30.08 & 10.66        & 5.53     &  37.31  & -   & 11.19 & 9.19  & - & 13.80    \\
Point2RBox+SK (2024) \cite{yu2024point2rbox}$^\dagger$ & \checkmark & 40.27 & 30.51        & 23.43    & 27.34 & 79.40   & 20.03 & 7.86  & 3.41 & 27.81    \\
PointOBB-v2 (2025) \cite{ren2024pointobbv2} & $\times$ & 41.68 & 30.59        & 20.64    &  39.56  & -   & 13.36 & 9.00  & 56.63 & 18.99   \\
PointOBB-v3 (2025) \cite{zhang2025pointobbv3} & $\checkmark$ & 41.20 & 31.25 & 22.82 & 37.60 & - & 11.42  & 11.31 & - & 15.84 \\
PointOBB-v3 (2025) \cite{zhang2025pointobbv3} & $\times$ & 49.24 & 33.79 & 23.52 & 40.18 & - & 18.35 & \underline{12.85} & - & 22.60 \\
\rowcolor{gray!20} Point2RBox-v2 (ours) & \checkmark & 51.00 & \underline{39.45} & \underline{27.11} & 34.70 & \underline{82.67} & \underline{25.72} & 7.80 & \underline{64.00} & \underline{28.60}
 \\
\rowcolor{gray!20} Point2RBox-v2 (ours) & $\times$ & \textbf{62.61} & \textbf{54.06}        & \textbf{38.79}   & \underline{44.45}  & \textbf{86.15}   & \textbf{34.71}  & \textbf{14.20} & \textbf{65.64} & \textbf{30.90}    \\
\bottomrule
\specialrule{0pt}{2pt}{0pt}
\multicolumn{11}{l}{$^*$Comparison tracks: \checkmark = End-to-end training and testing; $\times$ = Generating pseudo labels to train the FCOS detector (two-stage training).} \\
\multicolumn{11}{l}{$^\dagger$Using additional priors. P2RBox/PointSAM: Pre-trained SAM model; Point2RBox+SK: One-shot sketches for each class.} \\
\bottomrule
\end{tabular}
\caption{Accuracy (AP$_{50}$) comparisons on the DOTA-v1.0/1.5/2.0, DIOR, HRSC, FAIR1M, STAR, SKU110K, and RSAR datasets.}
\label{tab:exp_other}
\vspace{-4pt}
\end{table*}

\begin{itemize}
    \item \textbf{DOTA \cite{xia2018dota}.} DOTA-v1.0 has 2,806 aerial images annotated with 15 categories, while DOTA-v1.5/2.0 are the extended versions with more small objects and categories.
    
    \item \textbf{DIOR \cite{cheng2022anchor}.} It is an aerial image dataset re-annotated with RBoxes based on its original HBox version \cite{li2020object}, with a high variation in object size and high intra‐class diversity. 

    \item \textbf{HRSC \cite{liu2017hrsc}.} It contains ship instances on the sea and inshore. The train/val/test set includes 436/181/444 images.

    \item \textbf{FAIR1M \cite{sun2022fair1m}.} It has more than 1 million instances and more than 40,000 images for fine-grained object recognition in remote sensing imagery, annotated with 37 categories. The results are evaluated on FAIR1M-1.0.

    \item \textbf{STAR \cite{li2024star}.} It is extensive for scene graph generation, covering more than 210,000 objects with diverse spatial resolutions, classified into 48 fine-grained categories and precisely annotated with oriented bounding boxes. 

    \item \textbf{SKU110K \cite{pan2020dynamic}.} It focuses on the detection of densely packed retail scenes with 110,712 objects in 11,762 images. The density reaches 86 instances per image. 

    \item \textbf{RSAR \cite{zhang2025rsar}.} It is a remote sensing dataset based on Synthetic Aperture Radar (SAR) imagery with 6 categories.

\end{itemize}

\begin{table*}[!tb]
\fontsize{8.5pt}{10pt}\selectfont
\setlength{\tabcolsep}{2.08mm}
\setlength{\aboverulesep}{0.4ex}
\setlength{\belowrulesep}{0.4ex}
\setlength{\abovecaptionskip}{1.5mm}
\hspace{1pt}
\begin{minipage}[t]{0.315\linewidth}
\centering
\begin{tabular}{c|cc|cc}
\toprule
\multirow{2}{*}{$w_\text{O}$} & \multicolumn{2}{c|}{\textbf{DOTA}} & \multicolumn{2}{c}{\textbf{HRSC}} \\
                  & {E2E} & {FCOS} & {E2E} & {FCOS} \\ \midrule
3  & 48.76 & 61.62 & 81.85 & 84.36 \\
5  & 49.81 & 62.44 & 82.46 & 85.76 \\
\rowcolor{gray!20} 10 & \textbf{51.00} & \textbf{62.61} & \textbf{82.67} & \textbf{86.15} \\
30 & 45.88 & 57.83 & 81.56 & 85.61 \\
\bottomrule
\end{tabular}
\caption{Ablation with the weight of $\mathcal{L}_\text{O}$.}
\label{tab:abl_lo}
\end{minipage}
\quad
\begin{minipage}[t]{0.315\linewidth}
\centering
\begin{tabular}{c|cc|cc}
\toprule
\multirow{2}{*}{$w_\text{W}$} & \multicolumn{2}{c|}{\textbf{DOTA}} & \multicolumn{2}{c}{\textbf{HRSC}} \\
                  & {E2E} & {FCOS} & {E2E} & {FCOS} \\ \midrule
3  & 50.85 & 56.78 & 78.42 & 83.49 \\
\rowcolor{gray!20} 5  & \textbf{51.00} & \textbf{62.61} & \textbf{82.67} & \textbf{86.15} \\
10 & 49.15 & 60.54 & 30.37 & 35.13 \\
30 & 42.84 & 52.53 & 23.89 & 25.91 \\
\bottomrule
\end{tabular}
\caption{Ablation with the weight of $\mathcal{L}_\text{W}$.}
\label{tab:abl_lw}
\end{minipage}
\quad
\begin{minipage}[t]{0.315\linewidth}
\setlength{\tabcolsep}{2.04mm}
\centering
\begin{tabular}{c|cc|cc}
\toprule
\multirow{2}{*}{$w_\text{E}$} & \multicolumn{2}{c|}{\textbf{DOTA}} & \multicolumn{2}{c}{\textbf{HRSC}} \\
                  & {E2E} & {FCOS} & {E2E} & {FCOS} \\ \midrule
0.1 & 48.75 & 57.62 & 34.71 & 39.45 \\
\rowcolor{gray!20} 0.3 & 51.00 & 62.61 & \textbf{82.67} & \textbf{86.15} \\
0.5 & \textbf{51.36} & \textbf{62.63} & 76.85 & 85.22 \\
1.0 & 49.05 & 60.63 & 56.59 & 59.59 \\
\bottomrule
\end{tabular}
\caption{Ablation with the weight of $\mathcal{L}_\text{E}$.}
\label{tab:abl_le}
\end{minipage}
\vspace{-4pt}
\end{table*}

\begin{table*}[!tb]
\fontsize{8.5pt}{10pt}\selectfont
\setlength{\tabcolsep}{2.04mm}
\setlength{\aboverulesep}{0.4ex}
\setlength{\belowrulesep}{0.4ex}
\setlength{\abovecaptionskip}{1.5mm}
\hspace{1pt}
\begin{minipage}[t]{0.315\linewidth}
\centering
\begin{tabular}{c|cc|cc}
\toprule
\multirow{2}{*}{$w_\text{ss}$} & \multicolumn{2}{c|}{\textbf{DOTA}} & \multicolumn{2}{c}{\textbf{HRSC}} \\
                  & {E2E} & {FCOS} & {E2E} & {FCOS} \\ \midrule
0.1 & 49.28 & 59.66 & 73.66 & 78.92 \\
\rowcolor{gray!20} 1.0 & \textbf{51.00} & \textbf{62.61} & \textbf{82.67} & \textbf{86.15} \\
3.0 & 49.15 & 59.20 & 1.30  & 1.65 \\
\bottomrule
\end{tabular}
\caption{Ablation with the weight of $\mathcal{L}_\text{ss}$.}
\label{tab:abl_lss}
\end{minipage}
\quad
\begin{minipage}[t]{0.647\linewidth}
\setlength{\tabcolsep}{3.05mm}
\centering
\begin{tabular}{c|c|c||c|c|c}
\toprule
{R / F / S} & {\textbf{DOTA}} & {\textbf{HRSC}} & {R / F / S} & {\textbf{DOTA}} & {\textbf{HRSC}} \\
 \midrule
90\% / 10\% / 0\% & 60.42 & 85.46 & 80\% / 20\% / 0\%  & 59.46 & 84.73 \\
75\% / 0\% / 25\% & 60.79 & 86.22 & 60\% / 15\% / 25\% & 62.38 & 84.21 \\
\cellcolor{gray!20}68\% / 7\% / 25\% & \cellcolor{gray!20}\textbf{62.61} & \cellcolor{gray!20}\textbf{86.15} & 38\% / 37\% / 25\% & 45.87 & 8.56  \\
45\% / 5\% / 50\% & 60.55 & 85.34 & 40\% / 10\% / 50\% & 60.49 & 10.74 \\
\bottomrule
\end{tabular}
\caption{Ablation with the proportion of augmented views in self-supervision.}
\label{tab:abl_pro}
\end{minipage}
\vspace{-10pt}
\end{table*}

\subsection{Main Results on DOTA-v1.0}
\label{sec:experiment-main}

Table \ref{tab:exp_dota} compares Point2RBox-v2 with the state-of-the-art methods, which can be categorized into two tracks: 

\textbf{1) End-to-end training.} These methods apply the trained weakly-supervised detector directly to the test set. Without relying on priors, our approach demonstrates an improvement of 16.93\% (51.00\% vs. 34.07\%) compared to Point2RBox. Even when compared to Point2RBox+SK, which incorporates additional data-side priors (i.e. one-shot examples for each class), our method still outperforms it by 10.73\% (51.00\% vs. 40.27\%).

\textbf{2) Two-stage training.} These methods generate RBox labels on train/val sets, with which the FCOS detector is trained. In this two-stage mode, Point2RBox-v2 achieves an accuracy of 62.61\%, considerably surpassing PointOBB series. Remarkably, it even outperforms the SAM-powered method P2RBox by 3.57\% (62.61\% vs. 59.04\%).

\textbf{Class-wise analysis.} The FCOS detector trained with labels generated by Point2RBox-v2 achieves accuracy nearly equivalent to RBox-supervised FCOS across six high-density categories: SH (86.9\% vs. 87.1\%), SV (79.6\% vs. 79.8\%), LV (76.3\% vs. 79.8\%), PL (88.0\% vs. 89.1\%), ST (82.9\% vs. 84.6\%), and TC (89.1\% vs. 90.4\%). Interestingly, these six high-density categories account for 88\% of DOTA instances. By annotating these categories with points and generating RBoxes using Point2RBox-v2 while labeling the other sparse categories with RBoxes, we can significantly reduce annotation labor without sacrificing much accuracy, highlighting the valuable role our method can play.

\begin{figure*}[t!]
\setlength{\abovecaptionskip}{1.2mm}
\centering
\includegraphics[width=0.96\linewidth]{figs/case.pdf}
\caption{Qualitative analysis on failed cases and overlap cases.}
\label{fig:case}
\vspace{-6pt}
\end{figure*}

\subsection{Results on More Datasets}

The results are displayed in Table \ref{tab:exp_other}.
On more challenging DOTA-v1.5/2.0, Point2RBox-v2 presents a similar trend, 23.47\%/18.15\% higher than PointOBB-v2 in the pseudo-generation track. 
On the ship detection dataset HRSC, the gap between Point2RBox-v2 and RBox-supervised FCOS is only 2.84\% (86.15\% vs. 88.99\%).
DIOR is relatively sparse, leading to less improvement with our methods---lower than PointSAM (44.45\% vs. 46.20\%) but still higher than methods that do not use SAM. 
Our method also provides competitive performance on fine-grained datasets FAIR1M and STAR. 
In addition to remote sensing scenarios, we carry out experiments on SKU110K for densely packed retail scenes. Existing point-supervised methods struggle in this case, whereas Point2RBox-v2 achieves performance on par with HBox-supervised H2RBox (65.64\% vs. 57.15\%).

\begin{table}[!tb]
\fontsize{8.5pt}{10pt}\selectfont
\setlength{\tabcolsep}{1.78mm}
\setlength{\aboverulesep}{0.4ex}
\setlength{\belowrulesep}{0.4ex}
\setlength{\abovecaptionskip}{1.5mm}
\centering
\begin{tabular}{ccccc|cc|cc}
\toprule
\multicolumn{5}{c|}{\textbf{Modules}} & \multicolumn{2}{c|}{\textbf{DOTA}} & \multicolumn{2}{c}{\textbf{HRSC}} \\
$\mathcal{L}_\text{O}$ & $\mathcal{L}_\text{W}$ & $\mathcal{L}_\text{ss}$ & $\mathcal{L}_\text{E}$ & \textit{CP} & {E2E} & {FCOS} & {E2E} & {FCOS} \\ \midrule
\checkmark & & & & & 0.00 & 0.00 & 0.00 & 0.00 \\
\checkmark & \checkmark & & & & 41.54 & 52.98 & 17.96 & 19.64 \\
\checkmark & \checkmark & \checkmark & & & 46.64 & 54.26 & 18.10 & 22.13 \\
\checkmark & \checkmark & \checkmark & \checkmark & & 49.55 & 61.88 & 78.79 & 83.79 \\
& \checkmark & \checkmark & \checkmark & \checkmark & 48.58 & 59.56 & 20.35 & 24.76 \\
\checkmark & & \checkmark & \checkmark & \checkmark & 38.94 & 48.44 & 11.64 & 14.93 \\
\checkmark & \checkmark & \checkmark & & \checkmark & 47.08 & 55.05 & 19.58 & 21.78 \\
\rowcolor{gray!20} \checkmark & \checkmark & \checkmark & \checkmark & \checkmark & \textbf{51.00} & \textbf{62.61} & \textbf{82.67} & \textbf{86.15} \\
\bottomrule
\end{tabular}
\caption{Ablation with incremental addition of modules.}
\label{tab:abl_mod}
\vspace{-4pt}
\end{table}

\begin{table}[!tb]
\fontsize{8.5pt}{10pt}\selectfont
\setlength{\tabcolsep}{2.85mm}
\setlength{\aboverulesep}{0.4ex}
\setlength{\belowrulesep}{0.4ex}
\setlength{\abovecaptionskip}{1.5mm}
\centering
\begin{tabular}{c|c|c||c|c|c}
\toprule
16 & \cellcolor{gray!20}$K\!=\!24$ & 32 & 1.2 & \cellcolor{gray!20}$\beta\!=\!1.6$ & 2.0 \\ \midrule
50.87 & \cellcolor{gray!20}\textbf{51.00} & 48.08 & 48.14 & \cellcolor{gray!20}51.00 & \textbf{51.33} \\
\bottomrule
\end{tabular}
\caption{Ablation with $K$ and $\beta$ in edge loss on DOTA (E2E).}
\label{tab:abl_edgeparam}
\vspace{-4pt}
\end{table}

\begin{table}[!tb]
\fontsize{8.5pt}{10pt}\selectfont
\setlength{\tabcolsep}{1.75mm}
\setlength{\aboverulesep}{0.4ex}
\setlength{\belowrulesep}{0.4ex}
\setlength{\abovecaptionskip}{1.5mm}
\centering
\begin{tabular}{c|cc|cc|cc}
\toprule
\multirow{2}{*}{$\sigma$} & \multicolumn{2}{c|}{Point2RBox} & \multicolumn{2}{c|}{PointOBB-v2} & \multicolumn{2}{c}{Point2RBox-v2} \\
 & {\textbf{DOTA}} & {\textbf{HRSC}} & {\textbf{DOTA}} & {\textbf{HRSC}} & {\textbf{DOTA}} & {\textbf{HRSC}} \\ \midrule
0\%  & 40.27 & 79.40 & 44.85 & - & 62.61 & 86.15 \\
10\% & 39.60 & 78.81 & 42.30 & - & 61.58 & 85.76 \\
30\% & 38.42 & 78.28 & 38.46 & - & 60.31 & 85.71 \\
\bottomrule
\end{tabular}
\caption{Ablation with the inaccuracy in point annotations.}
\label{tab:abl_noise}
\vspace{-10pt}
\end{table}

\subsection{Ablation Studies}
\label{sec:experiment-ablation}

Tables \ref{tab:abl_lo}-\ref{tab:abl_noise} display the ablation studies on DOTA-v1.0 and HRSC. ``E2E'' denotes end-to-end training; ``FCOS'' denotes two-stage training (i.e. generating pseudo labels to train FCOS). The final values adopted are highlighted in gray.

\textbf{Weight of each loss.} Tables \ref{tab:abl_lo}-\ref{tab:abl_le} determine the weights of the proposed losses. Based on these experiments, the weights $(w_\text{O},w_\text{W},w_\text{E},w_\text{ss})$ are set to $(10, 5, 0.3, 1)$.

\textbf{Proportion of augmented views.} Table \ref{tab:abl_pro} studies the proportion between rotation, flip, and scale. The results are reported with two-stage training (FCOS). Based on the results, the proportion is set to 68\%, 7\%, and 25\%.

\textbf{Incremental addition of modules.} Table \ref{tab:abl_mod} demonstrates the constraints from Gaussian and Voronoi achieve an accuracy of 52.98\% on DOTA. Adding consistency loss and edge loss further boosts it to 54.26\% and 61.88\%, respectively, whereas the improvement from copy-paste is 0.73\%. We also demonstrate the impact of omitting each core loss.

\textbf{Edge loss parameters.} We set $K=24$ and $\beta=1.6$ as they are observed to discern the correct edges during code development. Table \ref{tab:abl_edgeparam} provides a more precise ablation.

\textbf{Annotation inaccuracy.} We offset the annotated points by a noise from the uniform distribution $\left[-\sigma H, +\sigma H \right ]$, where $H$ is the height of objects. Table \ref{tab:abl_noise} shows that the AP$_{50}$ of Point2RBox-v2 decreases by less than 3\% when noise is added to point annotations, demonstrating the robustness of the proposed learning mechanisms.

\subsection{More Discussions}
\label{sec:experiment-discussions}

The qualitative analysis on the failed/overlap cases is shown in Fig. \ref{fig:case}. \textbf{1) Failed cases.} Although our method performs well overall, it struggles with certain categories that are sparse and not constrained by other objects. \textbf{2) Overlap cases.} 
Minimizing overlap as a soft constraint during training does not entirely eliminate overlap. Once trained, the model remains robust to some overlap during inference.


\section{Conclusion}

%In this paper, w
We propose a new PEFT method called DiffoRA, which enables efficient and adaptive LLM fine-tuning based on LoRA. 
Instead of adjusting every interior rank, 
%of the decomposition matrices 
%of all modules, 
we argue that adopting LoRA module-wisely is sufficient. 
To achieve this, we construct a DAM to select the modules that are most suitable and essential to fine-tune. We theoretically analyze how the DAM impacts the convergence rate and generalization capability.
%of the pre-trained model. 
Furthermore, we adopt continuous relaxation and discretization to establish DAM.
%for each task. 
To alleviate the issue of discretization discrepancy, we utilize the weight-sharing strategy for optimization. 
%We fully implement our method and t
The experimental results demonstrate that our DiffoRA works consistently better than the baselines across all benchmarks. 


\section*{Impact Statement}
This paper presents work whose goal is to advance the field
of Machine Learning. There are many potential societal
consequences of our work, and none of which we feel must
be specifically highlighted here.


\bibliography{main}
\bibliographystyle{icml2025}


%%%%%%%%%%%%%%%%%%%%%%%%%%%%%%%%%%%%%%%%%%%%%%%%%%%%%%%%%%%%%%%%%%%%%%%%%%%%%%%
%%%%%%%%%%%%%%%%%%%%%%%%%%%%%%%%%%%%%%%%%%%%%%%%%%%%%%%%%%%%%%%%%%%%%%%%%%%%%%%
% APPENDIX
%%%%%%%%%%%%%%%%%%%%%%%%%%%%%%%%%%%%%%%%%%%%%%%%%%%%%%%%%%%%%%%%%%%%%%%%%%%%%%%
%%%%%%%%%%%%%%%%%%%%%%%%%%%%%%%%%%%%%%%%%%%%%%%%%%%%%%%%%%%%%%%%%%%%%%%%%%%%%%%
\newpage
\appendix
\onecolumn

\appendix
\newpage



\section{Baseline methods.}

We compare our approach to 18 baseline methods from IGMBT\footnote{https://github.com/kinit-sk/IMGTB} with its default setting \citep{Spiegel.2023}, which are categorized into metric-based and pretrained model-based methods. The metric-based methods include Binoculars ~\cite{hans2401spotting}, DetectLLM-LLR ~\cite{su2023detectllm}, DNAGPT ~\cite{yang2023dna}, Entropy ~\cite{gehrmann2019gltr}, FastDetectGPT ~\cite{bao2023fast}, GLTR ~\cite{gehrmann2019gltr}, LLMDeviation ~\cite{wu2023mfd}, Loglikelihood ~\cite{solaiman2019release}, LogRank ~\cite{Mitchell.2023}, MFD ~\cite{wu2023mfd}, Rank ~\cite{gehrmann2019gltr}, and S5 ~\cite{Spiegel.2023}. The model-based methods include NTNU-D ~\cite{sivesind2023turning}, ChatGPT-D ~\cite{guo2023close}, OpenAI-D ~\cite{solaiman2019release}, OpenAI-D-lrg ~\cite{solaiman2019release}, RADAR-D ~\cite{solaiman2019release}, and MAGE-D ~\cite{li2024mage}. 

\label{app:baselines}
\subsection{Metric based methods}
\subsubsection{Binoculars} 
Binoculars ~\cite{hans2401spotting}, analyzes text through two perspectives. First, it calculates the log perplexity of the text using an observer LLM. Then, a performer LLM generates next-token predictions, whose perplexity is evaluated by the observer—this metric is termed cross-perplexity. The ratio of perplexity to cross-perplexity serves as a strong indicator for detecting LLM-generated text.

\subsubsection{DNAGPT}
DNAGPT ~\cite{yang2023dna} is a training-free detection method designed to identify machine-generated text. Unlike conventional approaches that rely on training models, DNAGPT uses Divergent N-Gram Analysis (DNA) to detect discrepancies in text origin. The method works by truncating a given text at the midpoint and using the preceding portion as input to an LLM to regenerate the missing section. By comparing the regenerated text with the original through N-gram analysis (black-box) or probability divergence (white-box), DNAGPT reveals distributional differences between human and machine-written text, offering a flexible and explainable detection strategy.

\subsubsection{Entropy} Similar to the Rank score, the Entropy score for a text is determined by averaging the entropy values of each word, conditioned on its preceding context ~\cite{gehrmann2019gltr}.

\subsubsection{GLTR} The Entropy score, like the Rank score, is computed by averaging the entropy values of each word within a text, considering the preceding context ~\cite{gehrmann2019gltr}.

\subsubsection{MFD}
The Multi-level Fine-grained Detection (MFD) ~\cite{wu2023mfd} framework enhances text detection by combining statistical, semantic, and linguistic features at the sentence level. It first extracts low-level statistical features like readability and author style to quantify sentence structure. Simultaneously, high-level semantic differences are captured using an encoder with contrastive learning to distinguish LLM-generated text from human-written content. Additionally, advanced LLMs analyze the full text, extracting deep linguistic features related to lexicon, grammar, and syntax for more precise detection.

\subsubsection{Loglikelihood} This method utilizes a language model to compute the token-wise log probability. Specifically, given a text, the log probability of each token is averaged to produce a final score. A higher score indicates a greater likelihood that the text is machine-generated ~\cite{solaiman2019release}.

\subsubsection{LogRank} Unlike the Rank metric, which relies on absolute rank values, the Log-Rank score is derived by applying a logarithmic function to the rank value of each word ~\cite{Mitchell.2023}.

\subsubsection{Rank} The Rank score is calculated by determining the absolute rank of each word in a text based on its preceding context. The final score is obtained by averaging the rank values across the text. A lower score suggests a higher probability that the text was machine-generated ~\cite{gehrmann2019gltr}.

\subsubsection{DetectLLM-LLR} This approach integrates Log-Likelihood and Log-Rank scores, leveraging their complementary properties to analyze a given text ~\cite{su2023detectllm}.

\subsubsection{FastDetectGPT} This method assesses changes in a model’s log probability function when small perturbations are introduced to a text. The underlying idea is that LLM-generated text often resides in a local optimum of the model’s probability function. Consequently, minor perturbations to machine-generated text typically result in lower log probabilities, whereas perturbations to human-written text may lead to either an increase or decrease in log probability ~\cite{Mitchell.2023}.

\subsection{Model-based methods}
\subsubsection{NTNU-D}
It is a fine-tuned classification model based on the RoBERTa-base model, and three sizes of the bloomz-models ~\cite{sivesind2023turning}
\subsubsection{ChatGPT-D}
The ChatGPT Detector ~\cite{guo2023close} is designed to differentiate between human-written text and content generated by ChatGPT. It is based on a RoBERTa model that has been fine-tuned for this specific task. The authors propose two training approaches: one that trains the model solely on generated responses and another that incorporates both question-answer pairs for joint training. In our evaluation, we adopt the first approach to maintain consistency with other detection methods.

\subsubsection{OpenAI-D and RADAR-D}
The OpenAI Detector ~\cite{solaiman2019release} are models fine-tuned on RoBERTa to identify outputs generated by GPT-2. Specifically, it was trained using text generated by the largest GPT-2 model (1.5B parameters) and is capable of determining whether a given text is machine-generated.


\subsubsection{MAGE-D}
MAGE (MAchine-GEnerated text detection) ~\cite{li2024mage} is a large-scale benchmark designed for detecting AI-generated text. It compiles human-written content from seven diverse writing tasks, including story generation, news writing, and scientific writing. Corresponding machine-generated texts are produced using 27 different LLMs, such as ChatGPT, LLaMA, and Bloom, across three representative prompt types.

\section{Dataset Details}
\subsection{Dataset File Structure}
The calibration, test, and extended sets are in separate directories. Each directory contains subdirectories for different models that were used to generate AI peer review samples. In each model's subdirectory, you will find multiple CSV files, with each file representing peer review samples of a specific conference. The directory and file structure are outlined below.
\\

{\fontsize{8.2pt}{10pt}
\begin{verbatim}
|-- calibration
    |-- gpt4o
        |-- (format: <conference>.<subset>.<LLM>.csv)
        |-- ICLR2017.calibration.gpt-4o.csv
        |-- ...
        |-- ICLR2024.calibration.gpt-4o.csv
        |-- NeurIPS2016.calibration.gpt-4o.csv
        |-- ...
        |-- NeurIPS2024.calibration.gpt-4o.csv
    |-- claude
        |-- ...
    |-- gemini
        |-- ...
    |-- llama
        |-- ...
    |-- qwen
        |-- ...
|-- extended
    |-- gpt4o
        |-- ICLR2018.extended.gpt-4o.csv
        |-- ...
        |-- ICLR2024.extended.gpt-4o.csv
        |-- NeurIPS2016.extended.gpt-4o.csv
        |-- ...
        |-- NeurIPS2024.extended.gpt-4o.csv
    |-- llama
        |-- ...
|-- test
    |-- gpt4o
        |-- ICLR2017.test.gpt-4o.csv
        |-- ...
        |-- ICLR2024.test.gpt-4o.csv
        |-- NeurIPS2016.test.gpt-4o.csv
        |-- ...
        |-- NeurIPS2024.test.gpt-4o.csv
    |-- claude
        |-- ...
    |-- gemini
        |-- ...
    |-- llama
        |-- ...
    |-- qwen
        |-- ...
\end{verbatim}
}

\subsection{CSV File Content}
CSV files may differ in their column structures across conferences and years. These differences are due to updates in the required review fields over time as well as variations between conferences. See Table \ref{tab:review_template_fields} for review fields of individual conferences.


\begin{table}%
\centering
\resizebox{1\columnwidth}{!}{
\begin{tabularx}{.51\textwidth}{l X}
\toprule
Conference & Required Fields \\
\midrule
ICLR2017 & review, rating, confidence \\
ICLR2018 & review, rating, confidence \\
ICLR2019 & review, rating, confidence \\
ICLR2020 & review, rating, confidence, experience assessment, checking correctness of derivations and theory, checking correctness of experiments, thoroughness in paper reading \\
ICLR2021 & review, rating, confidence \\
ICLR2022 & summary of the paper, main review, summary of the review, correctness, technical novelty and significance, empirical novelty and significance, flag for ethics review, recommendation, confidence \\
ICLR2023 & summary of the paper, strength and weaknesses, clarity quality novelty and reproducibility, summary of the review, rating, confidence \\
ICLR2024 & summary, strengths, weaknesses, questions, soundness, presentation, contribution, flag for ethics review, rating, confidence \\
NeurIPS2016 & review, rating, confidence \\
NeurIPS2017 & review, rating, confidence \\
NeurIPS2018 & review, overall score, confidence score \\
NeurIPS2019 & review, overall score, confidence score, contribution \\
NeurIPS2021 & summary, main review, limitations and societal impact, rating, confidence, needs ethics review, ethics review area \\
NeurIPS2022 & summary, strengths and weaknesses, questions, limitations, ethics flag, ethics review area, rating, confidence, soundness, presentation, contribution \\
NeurIPS2023 & summary, strengths, weaknesses, questions, limitations, ethics flag, ethics review area, rating, confidence, soundness, presentation, contribution \\
NeurIPS2024 & summary, strengths, weaknesses, questions, limitations, ethics flag, ethics review area, rating, confidence, soundness, presentation, contribution \\
\bottomrule
\end{tabularx}
}
\caption{Required fields in the review templates for each conference.}
\label{tab:review_template_fields}
\end{table}

\FloatBarrier
\subsection{Dataset Sample Numbers per Conference Year}
\label{sec:sample_breakdown}
In this section, we present further breakdowns of sample numbers by conference, year, and LLM, as shown in Table~\ref{tab:dataset-statistics}.
 
\begin{table}[h!]
\centering
\resizebox{.59\columnwidth}{!}{
\begin{tabular}{lrr}
\toprule
Conference & gpt4o & llama \\
\midrule
ICLR2017 & 2926 & 2918 \\
ICLR2018 & 5460 & 5434 \\
ICLR2019 & 9414 & 9378 \\
ICLR2020 & 15426 & 15366 \\
ICLR2021 & 18786 & 18768 \\
ICLR2022 & 20042 & 20026 \\
ICLR2023 & 28562 & 28560 \\
ICLR2024 & 55714 & 55672 \\
NeurIPS2016 & 6296 & 6284 \\
NeurIPS2017 & 3848 & 3774 \\
NeurIPS2018 & 5990 & 5938 \\
NeurIPS2019 & 8444 & 8398 \\
NeurIPS2021 & 21170 & 21164 \\
NeurIPS2022 & 20472 & 20408 \\
NeurIPS2023 & 30264 & 30194 \\
NeurIPS2024 & 33206 & 33104 \\
\bottomrule
\end{tabular}

}
\caption{Entire set sample size, including both human and AI reviews. They are exactly balanced.}
\label{tab:sample_numbers_entire}
\end{table}

\begin{table}[h!]
\centering
\resizebox{1\columnwidth}{!}
{
\begin{tabular}{lrrrrr}
\toprule
Conference & gemini & claude & qwen & gpt4o & llama \\
\midrule
ICLR2017 & 2924 & 2926 & 2918 & 2926 & 2918 \\
ICLR2018 & 3000 & 3004 & 2988 & 3004 & 2992 \\
ICLR2019 & 3002 & 3010 & 3000 & 3010 & 2998 \\
ICLR2020 & 3016 & 3022 & 3000 & 3022 & 3010 \\
ICLR2021 & 3840 & 3842 & 3830 & 3842 & 3838 \\
ICLR2022 & 3896 & 3900 & 3838 & 3900 & 3898 \\
ICLR2023 & 3816 & 3816 & 3816 & 3816 & 3814 \\
ICLR2024 & 3784 & 3820 & 3800 & 3822 & 3816 \\
NeurIPS2016 & 5522 & 5534 & 5534 & 5536 & 5526 \\
NeurIPS2017 & 2854 & 2858 & 2850 & 2858 & 2812 \\
NeurIPS2018 & 3000 & 3006 & 2916 & 3006 & 2982 \\
NeurIPS2019 & 2930 & 2938 & 2922 & 2940 & 2928 \\
NeurIPS2021 & 3884 & 3884 & 3884 & 3884 & 3884 \\
NeurIPS2022 & 3606 & 3622 & 3598 & 3622 & 3610 \\
NeurIPS2023 & 4436 & 4440 & 4382 & 4440 & 4432 \\
NeurIPS2024 & 3914 & 3926 & 3880 & 3926 & 3912 \\
\bottomrule
\end{tabular}

}
\caption{Test set sample size, including both human and AI reviews. They are exactly balanced.}
\label{tab:sample_numbers_test}
\end{table}

\begin{table}[h!]
\centering
\resizebox{1\columnwidth}{!}{
\begin{tabular}{lrrrrr}
\toprule
Conference & gemini & claude & qwen & gpt4o & llama \\
\midrule
ICLR2021 & 3826 & 3828 & 3802 & 3828 & 3828 \\
ICLR2022 & 3878 & 3844 & 3860 & 3882 & 3880 \\
NeurIPS2021 & 3828 & 3830 & 3818 & 3830 & 3828 \\
NeurIPS2022 & 3648 & 3654 & 3634 & 3654 & 3644 \\
\bottomrule
\end{tabular}

}
\caption{Calibration set sample size, including both human and AI reviews. They are exactly balanced.}
\label{tab:sample_numbers_calibration}
\end{table}

\FloatBarrier

\section{Additional results}

\subsection{Calibration using ICLR + NeurIPS reviews}
\label{app:main-result-iclr-plus-neurips-calibration}

Our main results in Table~\ref{tab:main-results} of Section~\ref{sec:experiments-main-result} utilized ICLR review from our calibration set to calibrate each detection method. This simulates the scenario in which some of the reviews in the test set are "out-of-domain" in the sense that they belong to a different conference than the reviews used for calibration. In Table~\ref{tab:main-results-iclr-plus-neurips-calibration}, we provide additional results for the same evaluation setting as before, but using both ICLR and NeurIPS reviews for calibration (i.e., fully ``in domain''). We generally see similar trends regarding relative performance between methods as before, with the exception that the Binoculars method achieves slightly higher detection rates than our Anchor method for Gemini reviews. This suggests that existing methods such as Binoculars may be more sensitive to the use of in-domain data during calibration. 

\begin{table}[h!]
\begin{center}
\resizebox{1\columnwidth}{!}
{
\begin{tabular}{p{0.1cm}lcccccc}
\toprule
& Target FPR: & \multicolumn{2}{c}{0.1\%} & \multicolumn{2}{c}{0.5\%} & \multicolumn{2}{c}{1\%}\\
\cmidrule(lr){3-4}
\cmidrule(lr){5-6}
\cmidrule(lr){7-8}
& & FPR & TPR & FPR & TPR & FPR & TPR \\
\midrule
\multirow{8}{*}{\rotatebox[origin=c]{90}{GPT-4o Reviews}} 
& Anchor & 0.1 & \textbf{61.4} & 0.3 & \textbf{80.1} & 0.8 & \textbf{87.4} \\
& Binoculars & 0.3 & 18.8 & 0.7 & 37.5 & 1.2 & 49.3 \\
& MAGE-D & 0.1 & 2.3 & 0.7 & 9.6 & 1.3 & 14.5  \\
& s5 & 0.3 & 0.7 & 1.0 & 8.0 & 1.6 & 16.3 \\
& MFD & 0.3 & 0.9 & 0.9 & 7.8 & 1.6 & 14.9 \\
& GLTR & 0.1 & 0.1 & 0.5 & 2.4 & 1.2 & 5.9 \\
& DetectGPT & 0.1 & 0.2 & 0.6 & 1.1 & 1.0 & 2.1 \\
& Loglikelihood & 0.1 & 0.0 & 0.3 & 0.2 & 0.6 & 1.0 \\
\midrule
\multirow{10}{*}{\rotatebox[origin=c]{90}{Gemini Reviews}} 
& Binoculars & 0.3 & \textbf{63.8} & 0.7 & \textbf{80.9} & 1.1 & \textbf{87.6} \\
& Anchor & 0.2 & 57.2 & 0.5 & 75.5 & 1.1 & 84.2 \\
& MFD & 0.1 & 1.9 & 0.6 & 11.0 & 1.0 & 18.1 \\
& s5 & 0.1 & 1.4 & 0.5 & 10.5 & 1.0 & 18.3 \\
& GLTR & 0.2 & 0.6 & 1.0 & 6.3 & 1.8 & 12.9 \\
& FastDetectGPT & 0.1 & 1.1 & 0.4 & 4.9 & 0.9 & 8.9 \\
& MAGE-D & 0.1 & 0.4 & 0.7 & 3.8 & 1.3 & 6.9 \\
& DetectGPT & 0.1 & 0.5 & 0.5 & 3.2 & 1.1 & 6.3  \\
& Loglikelihood & 0.1 & 0.0 & 0.3 & 0.2 & 0.6 & 1.6 \\
& NTNU-D & 11.5 & 0.0 & 21.8 & 0.0 & 26.3 & 0.1 \\
\midrule
\multirow{6}{*}{\rotatebox[origin=c]{90}{Claude Reviews}}
& Anchor & 0.1 & \textbf{53.8} & 0.3 & \textbf{72.6} & 0.8 & \textbf{80.0} \\
& Binoculars & 0.3 & 46.4 & 0.7 & 70.2 & 1.1 & \textbf{80.0} \\
& MFD & 0.0 & 1.0 & 0.2 & 8.7 & 0.4 & 15.9  \\
& s5 & 0.0 & 0.7 & 0.2 & 8.4 & 0.4 & 16.4 \\
& DetectGPT & 0.1 & 0.6 & 0.5 & 4.9 & 1.0 & 10.1 \\
& GLTR & 0.0 & 0.0 & 0.3 & 0.7 & 0.6 & 1.9 \\
\bottomrule
\end{tabular}
}
\caption{Actual FPR and TPR calculated from the withheld test dataset at varying detection thresholds, which are calibrated using ICLR and NeurIPS reviews from our calibration set at different target FPRs. Best TPRs are in \textbf{bold}.}
\label{tab:main-results-iclr-plus-neurips-calibration}
\end{center}
\end{table}










\subsection{Additional results for test set: Llama and Qwen Detection Results}
\label{app:llama-qwen-results}
Table~10 is organized similarly to Table \ref{tab:main-results}, but it presents results for Llama and Qwen reviews. Both tables use the same set of thresholds for each method, which were calibrated using ICLR reviews generated by GPT-4o, Gemini, and Claude along with their matching human-written reviews.

\begin{table}[h]
\centering
\resizebox{1\columnwidth}{!}
{
\begin{tabular}{p{0.1cm}lcccccc}
\toprule
& Target FPR: & \multicolumn{2}{c}{0.1\%} & \multicolumn{2}{c}{0.5\%} & \multicolumn{2}{c}{1\%}\\
\cmidrule(lr){3-4}
\cmidrule(lr){5-6}
\cmidrule(lr){7-8}
& & FPR & TPR & FPR & TPR & FPR & TPR \\
\midrule
\multirow{15}{*}{\rotatebox[origin=c]{90}{Llama Reviews}} &Binoculars & 0.2 & \textbf{98.4} & 0.6 & \textbf{99.0} & 1.0 & \textbf{99.2} \\
&MAGE-D & 0.1 & 63.8 & 0.7 & 91.8 & 1.3 & 95.7 \\
&MFD & 0.0 & 57.6 & 0.1 & 81.9 & 0.1 & 87.4 \\
&GLTR & 0.1 & 55.4 & 0.2 & 75.2 & 0.3 & 81.7 \\
&FastDetectGPT & 0.1 & 54.7 & 0.5 & 73.8 & 1.2 & 80.6 \\
&s5 & 0.0 & 53.5 & 0.1 & 83.2 & 0.1 & 88.0 \\
&OpenAI-D & 0.2 & 38.8 & 0.6 & 48.3 & 1.6 & 57.1 \\
&LLMDeviation & 0.0 & 37.0 & 0.0 & 37.0 & 0.0 & 37.0 \\
&ChatGPT-D & 0.0 & 26.4 & 0.0 & 26.4 & 0.0 & 26.4 \\
&DetectLLM-{LLR} & 0.0 & 19.3 & 0.0 & 19.3 & 0.0 & 19.3 \\
&LogRank & 0.0 & 18.2 & 0.0 & 18.2 & 0.0 & 18.2 \\
&Loglikelihood & 0.0 & 13.4 & 0.3 & 76.0 & 0.5 & 85.5 \\
&Anchor & 0.1 & 12.2 & 0.5 & 28.5 & 1.0 & 36.8 \\
&DetectGPT & 0.1 & 1.7 & 0.7 & 10.8 & 1.3 & 19.7 \\
&RADAR-D & 0.9 & 0.0 & 2.6 & 1.6 & 4.2 & 12.5 \\
\midrule
\multirow{8}{*}{\rotatebox[origin=c]{90}{Qwen Reviews}} & Binoculars & 0.2 & \textbf{99.4} & 0.6 & \textbf{99.8} & 1.0 & \textbf{99.9} \\
& Anchor & 0.1 & 67.1 & 0.5 & 83.8 & 1.0 & 88.1 \\
& FastDetectGPT & 0.1 & 54.3 & 0.5 & 77.6 & 1.1 & 85.4  \\
& MFD & 0.1 & 37.6 & 0.4 & 73.0 & 0.6 & 82.3 \\
& s5 & 0.1 & 34.7 & 0.4 & 76.0 & 0.6 & 83.9 \\
& MAGE-D  & 0.1 & 33.2 & 0.7 & 73.6 & 1.3 & 86.2  \\
& GLTR & 0.1 & 31.1 & 0.3 & 64.7 & 0.6 & 77.4 \\
& Loglikelihood  & 0.0 & 0.5 & 0.3 & 30.8 & 0.5 & 56.6 \\
\bottomrule
\end{tabular}
}
\caption{Actual FPR and TPR calculated from the withheld test dataset at varying detection thresholds, which are calibrated using ICLR reviews from our calibration set at different target FPRs. Best TPRs are in \textbf{bold}.}

\label{tab:main-results_open-source-llm}

\end{table}

\FloatBarrier
\subsection{Experimental Results on Full Dataset}
\label{app:extended-set}

We test existing AI text generation text detection models on our entire dataset (i.e., the test set + the extended set). The results are shown in Tables \ref{tab:iclr_entire_dataset} for ICLR reviews and \ref{tab:neurips_entire_set} for NeurIPS reviews.


\begin{table}[h!]
\begin{center}
\resizebox{1\columnwidth}{!}
{
\begin{tabular}{p{0.1cm}lcccccccc}
\toprule
& Target FPR: & \multicolumn{2}{c}{0.1\%} & \multicolumn{2}{c}{0.5\%} & \multicolumn{2}{c}{1\%} \\
\cmidrule(lr){3-4}
\cmidrule(lr){5-6}
\cmidrule(lr){7-8}
& & FPR & TPR & FPR & TPR & FPR & TPR \\
\midrule
\multirow{2}{*}{\rotatebox[origin=c]{90}{GPT}} 
& Binoculars& 0.8\% & 23.4\%& 1.5\% & 41.7\%& 2.2\% & 54.2\% \\ 
 & GLTR& 0.7\% & 3.3\%& 2.1\% & 12.0\%& 3.8\% & 21.0\% \\ 
\midrule
\multirow{2}{*}{\rotatebox[origin=c]{90}{Llama}} 
& Binoculars& 0.8\% & 98.9\%& 1.5\% & 99.4\%& 2.2\% & 99.6\% \\ 
 & GLTR& 0.4\% & 80.6\%& 1.0\% & 92.5\%& 1.6\% & 95.3\% \\ 
\bottomrule
\end{tabular}
}
\caption{Actual FPR and TPR calculated from the ICLR reviews at varying detection thresholds, which are calibrated using the ICLR calibration dataset at different target FPRs.}
\label{tab:iclr_entire_dataset}
\end{center}
\end{table}





\begin{table}[h!]
\begin{center}
\resizebox{1\columnwidth}{!}
{
\begin{tabular}{p{0.1cm}lcccccccc}
\toprule
& Target FPR: & \multicolumn{2}{c}{0.1\%} & \multicolumn{2}{c}{0.5\%} & \multicolumn{2}{c}{1\%} \\
\cmidrule(lr){3-4}
\cmidrule(lr){5-6}
\cmidrule(lr){7-8}
& & FPR & TPR & FPR & TPR & FPR & TPR \\
\midrule
\multirow{2}{*}{\rotatebox[origin=c]{90}{GPT}} 
&BinocularsMetric& 0.3\% & 26.1\% & 0.6\% & 45.3\% & 0.9\% & 54.1\% \\
& GLTRMetric& 0.3\% & 1.3\% & 5.1\% & 31.9\% & 10.6\% & 52.6\% \\
\midrule
\multirow{2}{*}{\rotatebox[origin=c]{90}{Llama}} 
& BinocularsMetric& 0.3\% & 98.9\% & 0.6\% & 99.3\% & 0.9\% & 99.4\% \\
& GLTRMetric& 0.3\% & 76.3\% & 0.9\% & 91.0\% & 1.4\% & 93.9\% \\
\bottomrule
\end{tabular}
}
\caption{Actual FPR and TPR calculated from the NeurIPS reviews at varying detection thresholds, which are calibrated using the ICLR calibration dataset at different target FPRs.}
\label{tab:neurips_entire_set}
\end{center}
\end{table}



\FloatBarrier
\begin{table*}[]
\centering
\resizebox{1\textwidth}{!}
{
\begin{tabular}{p{3cm} p{10cm} p{10cm}}
    \toprule
    Category & Human review example & GPT-4o review examples \\
    \midrule
    References to specific details in the paper & ``Table 2 confirms that MDR outperforms Graph Rec Retriever (Asai et al.). This result shows the feasibility of a more accurate multi-hop QA model without external knowledge such as Wikipedia hyperlinks.'' & ``The paper extensively evaluates on multiple datasets and situates the contributions clearly within existing literature, substantiating claims with thorough quantitative analysis.'' \\
    \midrule
    Specific references to prior work & ``My only serious concern is the degree of novelty with respect to (Yuan et al., 2020), which was published at ECCV 2020. The main difference seems to be that in the proposed method the graph is dynamic (i.e., it depends on the input sentences), instead in (Yuan et al., 2018) the graph is learned but fixed for all the input samples.'' & ``The novelty of the TDM is not strong enough relative to prior work.'' \\
    \midrule
    Generic criticisms & N/A & ``Lack of clarity'' (without pointing to specific statements in the paper which need clarification); ``lack of discussion of limitations or computational considerations''; ``need more discussion of hyperparameter sensitivity''; ``need comparisons to more datasets'' (without suggesting any in particular); ``technical language used in the paper may be difficult to follow for unfamiliar readers''\\
    \bottomrule
\end{tabular}
}
\caption{Examples of differences identified in human analysis of human and AI-written peer reviews}
\label{tab:human_analysis_examples}
\end{table*}

\section{Additional analyses}

\subsection{Examples from human analysis of differences between human and AI-written peer reviews}
\label{app:human_analysis}

Table~\ref{tab:human_analysis_examples} provides examples of the issues identified in our qualitative analysis of human and AI-written peer reviews. In general, we observe that GPT-4o reviews lack references to specific details in the paper, lack references to specific prior work, and contain overly generic criticisms. See Section~\ref{sec:human-analysis} for additional discussion.



\subsection{Comparison of numeric scores assigned by human and AI reviewers}
\label{app:numeric_scores}
While Section~\ref{sec:analysis_numeric_score} focused on the misalignment between human and AI peer reviews from three commercial LLMs (GPT-4o, Gemini, and Claude) from the NeurIPS2022 review samples, this section presents the corresponding results for two open-source LLMs (Llama and Qwen), as shown in Figure~\ref{fig:numeric_scores}. The main findings from GPT-4o, Gemini, and Claude also hold for these two open-source models, with one notable difference: Llama and Qwen exhibit an even larger divergence in Presentation scores than Claude, the most overly-positive one amongst the commercial LLMs across all categories. In terms of Contribution scores, the evaluations from Llama and Qwen were similar to those of Claude.

In addition, we examine data from three other conferences (NeurIPS2023, NeurIPS2024, and ICLR2024). Although the results from these conferences are slightly less reliable---given that human reviews may have been influenced by AI use following the release of ChatGPT---the overall trend persists: LLMs tend to inflate the quality of papers compared to human reviewers.


\begin{figure*}
    \centering
    \underline{NeurIPS2022}
    \includegraphics[width=1\linewidth]{figures/figs_nuemric_scores/NeurIPS2022.ai_vs_human.png}
    \vspace{.01em}\\
    \underline{NeurIPS2023}
    \includegraphics[width=1\linewidth]{figures/figs_nuemric_scores/NeurIPS2023.ai_vs_human.png}
    \vspace{.01em}\\
    \underline{NeurIPS2024}
    \includegraphics[width=1\linewidth]{figures/figs_nuemric_scores/NeurIPS2024.ai_vs_human.png}
    \vspace{.01em}\\
    \underline{ICLR2024}
    \includegraphics[width=1\linewidth]{figures/figs_nuemric_scores/ICLR2024.ai_vs_human.png}
    \caption{Difference between AI and human scores. For each matched review (aligned by paper ID and recommendation), score differences were computed and displayed as histograms. Scores range from 1 to 4 for all metrics except Confidence, which ranges from 1 to 5. Statistical significance was assessed using a two-sided Wilcoxon signed‐rank test, with p-values shown in the legend. This figure includes only NeurIPS2022--2024 and ICLR2024, because they are the onyl conferences that required reviewers to submit these scores in their review templates.}
    \label{fig:numeric_scores}
\end{figure*}


\section{Prompts}
\label{sec:prompts}
This section includes the prompts we used to generate AI peer review texts. Due to space limitations, we provide only the ICLR2022 review guideline and review template here. Those for other years and other conferences (e.g., NeurIPS) are available on the respective conference official websites\footnote{https://icml.cc/Conferences/\{2016..2024\} \\ and https://neurips.cc/Conferences/\{2016..2024\} }.
\subsection{Prompts for Generating Reviews}
\noindent\underline{System prompt}:

\begin{Verbatim}[breaklines, breaksymbolleft={}, fontsize=\small]
You are an AI researcher reviewing a paper submitted to a prestigious AI research conference. 
You will be provided with the manuscript text, the conference's reviewer guidelines, and the decision for the paper.
Your objective is to thoroughly evaluate the paper, adhering to the provided guidelines, and return a detailed assessment that supports the given decision using the specified response template.
Ensure your evaluation is objective, comprehensive, and aligned with the conference standards.

{reviewer_guideline}

{review_template}
\end{Verbatim}

\noindent\underline{User prompt}:
\begin{Verbatim}[breaklines, breaksymbolleft={}, fontsize=\small]
Here is the paper you are asked to review. Write a well-justified review of this paper that aligns with a '{human_reviewer_decision}' decision.

```
{text}
```
\end{Verbatim}

\noindent\underline{ICLR2022 Reviewer Guideline}\\
\underline{(\texttt{\{reviewer\_guideline\}} in the system prompt)}:
\begin{Verbatim}[breaklines, breaksymbolleft={}, fontsize=\small]
## Reviewer Guidelines

1. Read the paper: It’s important to carefully read through the entire paper, and to look up any related work and citations that will help you comprehensively evaluate it. Be sure to give yourself sufficient time for this step.

2. While reading, consider the following:
    - Objective of the work: What is the goal of the paper? Is it to better address a known application or problem, draw attention to a new application or problem, or to introduce and/or explain a new theoretical finding? A combination of these? Different objectives will require different considerations as to potential value and impact.
    - Strong points: is the submission clear, technically correct, experimentally rigorous, reproducible, does it present novel findings (e.g. theoretically, algorithmically, etc.)?
    - Weak points: is it weak in any of the aspects listed in b.?
    - Be mindful of potential biases and try to be open-minded about the value and interest a paper can hold for the entire ICLR community, even if it may not be very interesting for you.

3. Answer three key questions for yourself, to make a recommendation to Accept or Reject:
    - What is the specific question and/or problem tackled by the paper?
    - Is the approach well motivated, including being well-placed in the literature?
    - Does the paper support the claims? This includes determining if results, whether theoretical or empirical, are correct and if they are scientifically rigorous.

4. Write your initial review, organizing it as follows: 
    - Summarize what the paper claims to contribute. Be positive and generous.
    - List strong and weak points of the paper. Be as comprehensive as possible.
    - Clearly state your recommendation (accept or reject) with one or two key reasons for this choice.
    - Provide supporting arguments for your recommendation.
    - Ask questions you would like answered by the authors to help you clarify your understanding of the paper and provide the additional evidence you need to be confident in your assessment. 
    - Provide additional feedback with the aim to improve the paper. Make it clear that these points are here to help, and not necessarily part of your decision assessment.

5. General points to consider:
    - Be polite in your review. Ask yourself whether you’d be happy to receive a review like the one you wrote.
    - Be precise and concrete. For example, include references to back up any claims, especially claims about novelty and prior work
    - Provide constructive feedback.
    - It’s also fine to explicitly state where you are uncertain and what you don’t quite understand. The authors may be able to resolve this in their response.
    - Don’t reject a paper just because you don’t find it “interesting”. This should not be a criterion at all for accepting/rejecting a paper. The research community is so big that somebody will find some value in the paper (maybe even a few years down the road), even if you don’t see it right now.
\end{Verbatim}

\noindent\underline{ICLR2022 Review Template}\\ 
\underline{(\texttt{\{reviewer\_template\}} in the system prompt)}:
\begin{Verbatim}[breaklines, breaksymbolleft={}, fontsize=\small]
## Response template (JSON format)

Provide the review in valid JSON format with the following fields. Ensure all fields are completed as described below. The response must be a valid JSON object.

- "summary_of_the_paper": Briefly summarize the paper and its contributions. This is not the place to critique the paper; the authors should generally agree with a well-written summary. You may use paragraphs and bulleted lists for formatting, but ensure that the content remains a single, continuous text block. Do not use nested JSON or include additional fields.

- "main_review": "Provide review comments as a single text field (a string). Consider including assessment on the following dimensions: a comprehensive list of strong and weak points of the paper, your recommendation, supporting arguments for your recommendation, questions to clarify your understanding of the paper or request additional evidence, and additional feedback with the aim to improve the paper. You may use paragraphs and bulleted lists for formatting, but ensure that the content remains a single, continuous text block. Do not use nested JSON or include additional fields."

- "summary_of_the_review": Concise summary of 'main_review'. You may use paragraphs and bulleted lists for formatting, but ensure that the content remains a single, continuous text block. Do not use nested JSON or include additional fields.

- "correctness": A numerical rating on the following scale to indicate that the claims and methods are correct. The value should be between 1 and 4, where:
    - 1 = The main claims of the paper are incorrect or not at all supported by theory or empirical results.
    - 2 = Several of the paper’s claims are incorrect or not well-supported.
    - 3 = Some of the paper’s claims have minor issues. A few statements are not well-supported, or require small changes to be made correct.
    - 4 = All of the claims and statements are well-supported and correct.

- "technical_novelty_and_significance": A numerical rating on the following scale to indicate technical novelty and significance. The value should be between 1 and 4, where:
    - 1 = The contributions are neither significant nor novel.
    - 2 = The contributions are only marginally significant or novel.
    - 3 = The contributions are significant and somewhat new. Aspects of the contributions exist in prior work.
    - 4 = The contributions are significant and do not exist in prior works.
       
- "empirical_novelty_and_significance": A numerical rating on the following scale to indicate empirical novelty and significance. The value should be between 1 and 4, or -999 if not applicable, where:
    - 1 = The contributions are neither significant nor novel.
    - 2 = The contributions are only marginally significant or novel.
    - 3 = The contributions are significant and somewhat new. Aspects of the contributions exist in prior work.
    - 4 = The contributions are significant and do not exist in prior works.
    - -999 = Not applicable.

- "flag_for_ethics_review": A boolean value (`true` or `false`) indicating whether there are ethical concerns in the work.

- "recommendation": A string indicating the final decision, which must strictly be one of the following options: 'strong reject', 'reject, not good enough', 'marginally below the acceptance threshold', 'marginally above the acceptance threshold', 'accept, good paper', or 'strong accept, should be highlighted at the conference'.

- "confidence": A nuemrical values to indicate how confident you are in your evaluation. The value should be between 1 and 5, where:
    - 1 = You are unable to assess this paper and have alerted the ACs to seek an opinion from different reviewers.
    - 2 = You are willing to defend your assessment, but it is quite likely that you did not understand the central parts of the submission or that you are unfamiliar with some pieces of related work. Math/other details were not carefully checked.
    - 3 = You are fairly confident in your assessment. It is possible that you did not understand some parts of the submission or that you are unfamiliar with some pieces of related work. Math/other details were not carefully checked.
    - 4 = You are confident in your assessment, but not absolutely certain. It is unlikely, but not impossible, that you did not understand some parts of the submission or that you are unfamiliar with some pieces of related work.
    - 5 = You are absolutely certain about your assessment. You are very familiar with the related work and checked the math/other details carefully.
\end{Verbatim}


\subsection{Anchor Review Generation Prompt}
\label{sec:prompt_anchor}
\noindent\underline{System prompt}:
\begin{Verbatim}[breaklines, breaksymbolleft={}, fontsize=\small]
You are an AI research scientist tasked with reviewing paper submissions for a top AI research conference. Carefully read the provided paper, then write a detailed review following a common AI conference review format (e.g., including summary, strengths and weakness, limitations, questions, suggestions for improvement). Make sure to include recommendation for the paper, either 'Accept' or 'Reject'. Your review should be fair and objective.
\end{Verbatim}

\noindent\underline{User prompt}:
\begin{Verbatim}[breaklines, breaksymbolleft={}, fontsize=\small]
Here is the paper you are asked to review:
```
{text}
```
\end{Verbatim}

\subsection{Editing Prompts}
\label{sec:prompt_editing}
\label{sec:editing_prompt}
\noindent\underline{Minimal Editing}:
\begin{Verbatim}[breaklines, breaksymbolleft={}, fontsize=\small]
Please proofread my review for typos and grammatical errors without altering the content. Keep the original wordings as much as you can, except for typo or grammatical  error.
\end{Verbatim}
\noindent\underline{Moderate Editing}:
\begin{Verbatim}[breaklines, breaksymbolleft={}, fontsize=\small]
Please polish my review to improve sentence structure and readability while keeping the original intent clear.
\end{Verbatim}
\noindent\underline{Extensive Editing}:
\begin{Verbatim}[breaklines, breaksymbolleft={}, fontsize=\small]
Please rewrite my review into a polished, professional piece that effectively communicates its main points.
\end{Verbatim}
\noindent\underline{Maximum Editing}:
\begin{Verbatim}[breaklines, breaksymbolleft={}, fontsize=\small]
Please transform my review into a high quality piece, using professional language and a polished tone. Please also extend my review with additional details from the oringial paper.
\end{Verbatim}

\section{Artifact Use Consistent With Intended Use}
In our work, we ensured that the external resources we utilized were applied in a manner that aligns with their intended purposes. We used several LLMs (including GPT-4o, Gemini, Claude, Qwen, and Llama) as well as an open-source package, IMGTB, with a focus on advancing research in a non-commercial, open-source context. The artifacts from our work will be non-commercial, for-research, and open-sourced.

\section{Use of AI Tool}
The authors of this paper used Github Co-pilot for coding assistance for analysis. 

%%%%%%%%%%%%%%%%%%%%%%%%%%%%%%%%%%%%%%%%%%%%%%%%%%%%%%%%%%%%%%%%%%%%%%%%%%%%%%%
%%%%%%%%%%%%%%%%%%%%%%%%%%%%%%%%%%%%%%%%%%%%%%%%%%%%%%%%%%%%%%%%%%%%%%%%%%%%%%%


\end{document}


% This document was modified from the file originally made available by
% Pat Langley and Andrea Danyluk for ICML-2K. This version was created
% by Iain Murray in 2018, and modified by Alexandre Bouchard in
% 2019 and 2021 and by Csaba Szepesvari, Gang Niu and Sivan Sabato in 2022.
% Modified again in 2023 and 2024 by Sivan Sabato and Jonathan Scarlett.
% Previous contributors include Dan Roy, Lise Getoor and Tobias
% Scheffer, which was slightly modified from the 2010 version by
% Thorsten Joachims & Johannes Fuernkranz, slightly modified from the
% 2009 version by Kiri Wagstaff and Sam Roweis's 2008 version, which is
% slightly modified from Prasad Tadepalli's 2007 version which is a
% lightly changed version of the previous year's version by Andrew
% Moore, which was in turn edited from those of Kristian Kersting and
% Codrina Lauth. Alex Smola contributed to the algorithmic style files.
