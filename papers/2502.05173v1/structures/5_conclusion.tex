\section{Conclusion}
% In this paper, we have summarized four key criteria for an effective positional encoding design: 2D/3D Structure, Frequency Allocation, Spatial Symmetry, and Temporal Index Scaling. Through our proposed V-NIAH-D task, we demonstrate that previous RoPE variants, which lack proper temporal dimension allocation, are prone to being misled by distractors. Based on our findings, we propose VideoRoPE, a novel positional encoding method that adopts a 3D structure to maintain spatiotemporal coherence. This design leverages low-frequency temporal allocation to reduce periodic oscillations, employs a diagonal layout to preserve spatial symmetry, and allows for adjustable temporal spacing to separate the indexing of spatial and temporal dimensions. As a result, VideoRoPE consistently outperforms previous RoPE variants in a variety of downstream tasks, such as long video retrieval, video understanding, and video hallucination.
This paper identifies four key criteria for effective positional encoding: 2D/3D structure, frequency allocation, spatial symmetry, and temporal index scaling. As part of our analysis, through the V-NIAH-D task, we demonstrate that previous RoPE variants are vulnerable to distractors because of lacking proper temporal allocation. As a result, We propose \methodname that uses a 3D structure for spatiotemporal coherence, low-frequency temporal allocation to reduce oscillations, a diagonal layout for spatial symmetry, and adjustable temporal spacing. \methodname outperforms previous RoPE variants in tasks like long video retrieval, video understanding, and video hallucination.
