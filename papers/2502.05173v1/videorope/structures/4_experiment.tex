\section{Experiment}
% \renewcommand{\arraystretch}{1.1}
% \begin{table*}[!h]
% \setlength\tabcolsep{8pt}
% \centering
% \caption {Comparison of different RoPE methods on three video understanding benchmarks under the same training setting. The benchmarks evaluate performance across three context lengths: 8k, 32k, and 64k, where \textbf{8k} represents context within the training range, and 32k and 64k represent context outside the training range. Our \methodname outperforms other RoPE variants across all three benchmarks. The best results are marked in \textbf{bold}, and the second-best results are \underline{underlined}.}
% \label{tab:lvlm_all}
% \vspace{2mm}
% \footnotesize
% \begin{tabular}{clllllllll}
% \toprule
% \multirow{2}{*}{\textbf{Method}}  & \multicolumn{2}{c}{LongVideoBench} & \multicolumn{2}{c}{MLVU} & \multicolumn{2}{c}{Video-MME} \\ 
% \cmidrule(lr){2-5} % 针对 LongVideoBench 数据
% \cmidrule(lr){6-9} % 针对 MLVU 数据
% \cmidrule(lr){10-13} % 针对 Video-MME 数据
%  & 8k & 16k & 32k & 64k & 8k & 16k & 32k & 64k & 8k & 16k & 32k & 64k \\ \hline
% Vanilla RoPE \cite{su2024roformer} & \textbf{54.35} & 54.25 & \underline{53.94} & 53.52 & 63.31 & \underline{65.93} & \underline{62.02} & \underline{60.6} & 61.3 & 58.3 \\
% TAD-RoPE \cite{gao2024tc} & 53.73 & ? & 53.42 & 52.80 & \underline{63.67} & 65.28 & 60.73 & 60.3 & \textbf{62.0} & 58.6 \\
% M-RoPE \cite{wang2024qwen2} & 52.90 & 52.38 & 52.90 & \underline{54.14} & 60.41 & 61.56 & 61.10 & \underline{60.6} & 61.0 & \underline{60.6} \\
% \hline
% \rowcolor[HTML]{F2F3F5}
% \methodname (Ours) & \underline{53.94} & 54.25 & \textbf{56.12} & \textbf{56.43} & \textbf{65.19} & \textbf{66.02} & \textbf{65.56} & \textbf{61.3} & \underline{61.6} & \textbf{61.3} \\
% % Qwen2-VL-7B & \textbf{54.87} & \textbf{59.12} & \textbf{58.40} & \textbf{65.60} & \textbf{68.36} & \textbf{68.59} & 54.3 & 56.3 & 53.6 \\ \hline
% \bottomrule
% \end{tabular}

% \end{table*}

\renewcommand{\arraystretch}{1.1}
\begin{table*}[!ht]
\setlength\tabcolsep{5pt} % Adjusted for better fit
\centering
\caption{\textbf{Comparison of different RoPE methods on LongVidionBench, MLVU, and Video-MME}. The benchmarks evaluate performance across three context lengths: 8k, 16k, 32k, and 64k, where \textbf{8k} represents context within the training range, and others represent context outside the training range. Our \methodname outperforms other RoPE variants across all three benchmarks. The best results are marked in \textbf{bold}, and the second-best results are \underline{underlined}. For more information on the evaluation, see Appendix \ref{appendix:benchmarks}.}
\label{tab:lvlm_all}
\vspace{2mm}
\footnotesize
\begin{tabular}{cllllllllllll}
\toprule
\multirow{2}{*}{\textbf{Method}} & \multicolumn{4}{c}{\textbf{LongVideoBench}} & \multicolumn{4}{c}{\textbf{MLVU}} & \multicolumn{4}{c}{\textbf{Video-MME}} \\ 
\cmidrule(lr){2-5} % 针对 LongVideoBench 数据
\cmidrule(lr){6-9} % 针对 MLVU 数据
\cmidrule(lr){10-13} % 针对 Video-MME 数据
 & 8k & 16k & 32k & 64k & 8k & 16k & 32k & 64k & 8k & 16k & 32k & 64k \\ 
\hline
Vanilla RoPE \cite{su2024roformer} & \textbf{54.97} & 54.87 & \underline{54.56} & 54.04 & 63.31 & \underline{65.79} &\underline{65.93} & \underline{62.02} & \underline{60.67} & 60.00 & 61.33 & 58.33 \\
TAD-RoPE \cite{gao2024tc} & 54.14 & \underline{55.08} & 53.94 & 53.42 & \underline{63.67} & 65.28 & 65.28 & 60.73 & 60.33 & \textbf{61.33} & \textbf{62.00} & 58.67 \\
M-RoPE \cite{wang2024qwen2} & 53.42 & 52.80 & 53.11 & \underline{54.35} & 60.41 & 60.68 & 61.56 & 61.10 & \underline{60.67} & 59.67 & 61.00 & \underline{59.67} \\
\hline
\rowcolor[HTML]{F2F3F5}
\methodname (Ours) & \underline{54.46} & \textbf{55.29} & \textbf{57.15} & \textbf{57.26} & \textbf{65.19} & \textbf{66.29} & \textbf{66.02} & \textbf{65.56} & \textbf{61.33} & \underline{61.00} &\underline{61.67} & \textbf{61.33} \\
\bottomrule
\end{tabular}
\end{table*}

\renewcommand{\arraystretch}{1.}
\setlength\tabcolsep{2pt}
\setlength{\textfloatsep}{5pt}
\begin{table}[t]
\centering
\footnotesize
\caption{
% Performance comparison of different RoPE methods evaluated on V-NIAH and V-NIAH-D under identical training conditions. Each frame is encoded using 144 tokens, with a depth interval of 0.2 used to insert the needle frame. The haystack is validated starting from 100 frames, with additional validation at 200-frame intervals up to 3000 frames. For the V-NIAH-D task, distractors are inserted periodically, with the period calculated as $2 \cdot \pi \cdot 1000000^{32/128} \approx 198.7$. In our experiments, we directly use a period of 200 for distractor insertion. ``Acc." refers to the average accuracy across the haystack length and frame depth. Our \methodname achieves superior performance compared to other RoPE variants on both V-NIAH and the more challenging V-NIAH-D dataset. The best results are marked in \textbf{bold}, while the second-best results are \underline{underlined}.
\textbf{Performance comparison of different RoPEs on V-NIAH and V-NIAH-D}. ``Acc.'' refers to the average accuracy across haystack length and frame depth.
% For the detailed setup of the evaluation, see Appendix \ref{appendix:benchmarks}.
}
\label{tab:v-niah-and-d}
\vspace{2mm}
\begin{tabular}{ccc}
\toprule
\textbf{Method} & \textbf{V-NIAH Acc.} & \textbf{V-NIAH-D Acc.} \\ \hline
Vanilla RoPE~\cite{su2024roformer} & 31.78 & 30.22 \\
TAD-RoPE~\cite{gao2024tc} & 29.33 & 29.56 \\
M-RoPE~\cite{wang2024qwen2} & \underline{78.67} & \underline{74.67} \\
\hline
\rowcolor[HTML]{F2F3F5}
\methodname & \textbf{91.11}  &\textbf{87.11} \\
\bottomrule
\end{tabular}
\end{table}

\subsection{Experimental Setup}

% \paragraph{Training Data} To better align the large language model with the video modality, we utilize the llava-video-178k dataset \cite{zhang2024video}, which encompasses key tasks such as detailed captioning, open-ended question answering (QA), and multiple-choice QA. These tasks are derived from diverse video sources, such as HD-VILA, Kinetics, and ActivityNet.
% For training, we randomly select 300k samples with durations of less than 2 minutes and 30k samples ranging between 2 and 3 minutes, striking a balance between training efficiency and long video understanding.
% In total, our training data comprises 1.285 million QA pairs, which are used for supervised fine-tuning (SFT) of our \methodname.

\noindent \textbf{Training Data.} We use a subset of LLaVA-Video-178k dataset \cite{zhang2024video} to train \methodname.
The LLaVA-Video-178k dataset covers 178k videos and around 5 million question-answers (QA) pairs from diverse sources such as HD-VILA \cite{xue2022hdvila}, Kinetics \cite{kay2017kinetics}, and ActivityNet \cite{caba2015activitynet}.
To balance training efficiency and long-video comprehension, we randomly select 136k videos with durations under 2 minutes and 18k videos with durations between 2 and 3 minutes.
This process yielded our training set of approximately 1.3 million pairs.

\noindent \textbf{Implementation Details.}
Using the aforementioned video training data, we fine-tune different modes that use different positional encoding strategies, such as the Vanilla RoPE \cite{su2024roformer}, Time-Aware Dual RoPE (TAD-RoPE) \cite{gao2024tc}, M-RoPE \cite{wang2024qwen2}, and our \methodname.
All models are initialized with the Vision Transformer from Qwen2-VL-7B and LLM (Vanilla RoPE) from Qwen2-7B \cite{yang2024qwen2}.
Our fine-tuning incorporates our \methodname to process the spatiotemporal nature of the video data effectively.
We adopt Qwen2-VL's fine-tuning settings, processing each video at 2 fps with a maximum of 128 frames and dynamically adjusting the image resolution to maintain a consistent token count. However, to prevent memory overflow, we use a context window of 8192 tokens.
% This approach balances the capacity to handle extended videos with training efficiency.

Our fine-tuning process employs a batch size of 128, a cosine scheduler with a learning rate of 1e-5, a warm-up ratio of 1e-2, and 704 Nvidia-A100 GPU hours in total.
The evaluation involves sampling videos at 2 fps with a minimum of 144 image tokens per frame.
We use the vLLM framework \cite{kwon2023efficient} to support inference on sequences longer than 32k tokens.
% To support inference over long sequences, we use the Transformers library for sequences under 32k context and the vLLM framework for sequences exceeding 32k.

\noindent \textbf{Evaluation Benchmarks.} We evaluate our approach using six video benchmarks, including tasks related to \textit{long video understanding}, \textit{long video retrieval}, and \textit{video hallucination}. For \textit{long video understanding}, we use \textbf{LongVideoBench} \cite{wu2024longvideobench} (8 seconds to 1 hour), \textbf{MLVU} \cite{zhou2024mlvu} (3 minutes to 2 hours), and \textbf{Video-MME} \cite{fu2024video} (11 seconds to 60 minutes). For \textit{long video retrieval}, we use \textbf{Vision Needle-in-a-Haystack (V-NIAH)} \cite{zhang2024longva} and our proposed extension, \textbf{Vision Needle-in-a-Haystack with Distractors (V-NIAH-D)}, which introduces distractor frames to increase the task difficulty. For \textit{video hallucination}, we use \textbf{VideoHallucer} \cite{videohallucer}, which evaluates the model's ability to correctly answer both basic and hallucinated questions about video content. Details of these benchmarks can be found in  Appendix~\ref{appendix:benchmarks}.

\subsection{Results on Long Video Understanding}
As shown in Tab. \ref{tab:lvlm_all}, we compare our \methodname with existing RoPE variants (vanilla RoPE \cite{su2024roformer}, TAD-RoPE \cite{gao2024tc}, and M-RoPE \cite{wang2024qwen2}) across three prominent video understanding benchmarks. Our \methodname consistently outperforms all baseline methods across these benchmarks, demonstrating its robustness and adaptability. Specifically, \methodname achieves improvements of up to 2.91, 4.46, and 1.66 points (64k context length) over the M-RoPE baseline on LongVideoBench, MLVU, and Video-MME, respectively. These results emphasize the superior ability of \methodname to effectively capture long-range dependencies and maintain performance across various challenging video data tasks.

\renewcommand{\arraystretch}{1.}
\setlength\tabcolsep{2pt}
\setlength{\textfloatsep}{5pt}
\begin{table}[t]
\centering
\footnotesize
\caption{
% Performance comparison of different RoPE methods evaluated on V-NIAH and V-NIAH-D under identical training conditions. Each frame is encoded using 144 tokens, with a depth interval of 0.2 used to insert the needle frame. The haystack is validated starting from 100 frames, with additional validation at 200-frame intervals up to 3000 frames. For the V-NIAH-D task, distractors are inserted periodically, with the period calculated as $2 \cdot \pi \cdot 1000000^{32/128} \approx 198.7$. In our experiments, we directly use a period of 200 for distractor insertion. ``Acc." refers to the average accuracy across the haystack length and frame depth. Our \methodname achieves superior performance compared to other RoPE variants on both V-NIAH and the more challenging V-NIAH-D dataset. The best results are marked in \textbf{bold}, while the second-best results are \underline{underlined}.
\textbf{Performance comparison of different RoPEs on V-NIAH and V-NIAH-D}. ``Acc.'' refers to the average accuracy across haystack length and frame depth.
% For the detailed setup of the evaluation, see Appendix \ref{appendix:benchmarks}.
}
\label{tab:v-niah-and-d}
\vspace{2mm}
\begin{tabular}{ccc}
\toprule
\textbf{Method} & \textbf{V-NIAH Acc.} & \textbf{V-NIAH-D Acc.} \\ \hline
Vanilla RoPE~\cite{su2024roformer} & 31.78 & 30.22 \\
TAD-RoPE~\cite{gao2024tc} & 29.33 & 29.56 \\
M-RoPE~\cite{wang2024qwen2} & \underline{78.67} & \underline{74.67} \\
\hline
\rowcolor[HTML]{F2F3F5}
\methodname & \textbf{91.11}  &\textbf{87.11} \\
\bottomrule
\end{tabular}
\end{table}
\setlength{\textfloatsep}{5pt}  % Reduce space between tables/figures
\setlength{\intextsep}{5pt}     % Reduce space when the table is placed within text (in-text float)
\setlength{\floatsep}{5pt} 
\renewcommand{\arraystretch}{1.0}
\setlength{\tabcolsep}{2pt}
\setlength{\textfloatsep}{5pt}
\begin{table}[t]
\centering
\footnotesize
\caption {\textbf{Performance comparison of different RoPEs on VideoHallucer}, evaluated at context lengths of 8k, 16k, 32k, and 64k. The maximum result for each RoPE variant across these context lengths is displayed, with bold for the top result and underlined for the second-highest. `OR' = Object-Relation, `T' = Temporal, `SD' = Semantic Detail, `F' = Factual, `NF' = Non-factual.
}
\label{tab:video_hallucer}
\vspace{2mm}
\begin{tabular}{ccccccc}
\toprule
\textbf{Method} & \textbf{OR} & \textbf{T} & \textbf{SD} & \textbf{F} & \textbf{NF} & \textbf{Avg.} \\ \hline
Vanilla RoPE~\cite{su2024roformer} & \underline{51.5} & 30.0 & \underline{48.0} & 8.0 & 43.0 & 36.1 \\
TAD-RoPE~\cite{gao2024tc} & 51.0 & \underline{37.0} & \underline{48.0} & 11.5 & \underline{47.5} & 39.0 \\
M-RoPE~\cite{wang2024qwen2} & 39.0 & 29.0 & 43.5 & \underline{12.5} & \underline{47.5} & 34.3 \\
\hline
\rowcolor[HTML]{F2F3F5}
\methodname & \textbf{57.0} & \textbf{58.5} & \textbf{50.5} & \textbf{15.0} & \textbf{50.0} & \textbf{46.2} \\
% \rowcolor[HTML]{c4c5fc}
% $\triangle~\text{compared}$ & \hgreen{+18.0} & \hgreen{+29.5} & \hgreen{+7.0} & \hgreen{+2.5} & \hgreen{+2.5} & \hgreen{+11.9} \\
\bottomrule
\end{tabular}
\end{table}
\subsection{Results on Long Video Retrieval}
Fig. \ref{fig:v-niah-and-d} illustrates the performance of V-NIAH and V-NIAH-D with \methodname and other RoPE variants. Specifically, Fig. \ref{fig:v-niah-and-d} (a) and (b) demonstrate that the proposed V-NIAH-D is more challenging than V-NIAH.
Fig. \ref{fig:v-niah-and-d} (1) and (2) show that both Vanilla RoPE and TAD-RoPE exhibit some extrapolation ability beyond the visual training context. However, both methods fail once they exceed a certain extrapolation limit.
In contrast, Fig. \ref{fig:v-niah-and-d} (3) and (4) highlight the superior performance of \methodname and M-RoPE in extrapolating within the test context range. While both \methodname and M-RoPE successfully handle extrapolation, \methodname consistently outperforms M-RoPE, showcasing the robustness of the task.
Tab. \ref{tab:v-niah-and-d} provides a quantitative analysis of the retrieval results, demonstrating a 12.44
\% performance improvement of our method over M-RoPE on the Video Retrieval task in both settings, confirming the advantages of our proposed method in video retrieval scenarios.

\subsection{Results on Video Hallucination}

As highlighted in Tab. \ref{tab:video_hallucer}, \methodname significantly surpasses current RoPE methods on the VideoHallucer benchmark. In particular, for the Temporal Hallucination task, \methodname demonstrates a substantial performance improvement of 29.5\%, indicating its enhanced capability to accurately capture and process temporal dependencies. This improvement suggests that \methodname is better equipped to handle dynamic video sequences, where the understanding of time-based relationships is critical. Similarly, for the Spatial Hallucination task, specifically the Object-Relation Hallucination subtask, \methodname achieves an impressive 18.0\% improvement over existing methods, highlighting its ability to better discern complex spatial interactions. These results underscore \methodname's robustness in solving video hallucination and potential for real-world video analysis.

\subsection{Ablation Studies}
\noindent \textbf{Ablation Studies on Module Design.} 
% Diagonal Position Encoding (DPE)
% Low-Frequency Temporal Modeling(LFTM)
% Step-2 Temporal Positional Encoding (S2TPE)
We conduct a series of ablation experiments on the modules introduced in Section \ref{subsec:step_size}. Through these experiments, we quantitatively evaluate the performance improvements, specifically analyzing their impact on the LongVideoBench and MLVU benchmarks. The experimental results are presented in Tab. \ref{tab:ablation_modules}. The baseline setting, M-RoPE~\cite{wang2024qwen2}, achieves scores of 54.35 on LongVideoBench and 61.10 on MLVU (both using a 64k context length). By progressively integrating the DL (Diagonal Layout), LTA (Low-frequency Temporal Allocation), and ATS (Adjustable Temporal Spacing) modules, our method shows a continuous improvement in performance, achieving enhanced scores of 57.26 on LongVideoBench and 65.56 on MLVU (both using a 64k context length). These results demonstrate the effectiveness of our approach in leveraging spatial-temporal positional information. To refine the discussion on the allocation of \( x \) and \( y \) within LTA, we quantitatively examine the performance impact of interleaved versus sequential allocations. Additionally, we explore the optimal scaling factor in ATS by adjusting its values. Please refer to Appendix~\ref{app:ablation_study} for more ablation studies on the scaling factor $\delta$ and the $x$/$y$ interleaved allocation.

% To further refine the discussion on the allocation of \( x \) and \( y \) within LTA, we quantitatively examine the performance impact of interleaved versus sequential allocations of \( x \) and \( y \) in Appendix \ref{app:x_y_allocation}. Regarding the choice of the scaling factor in ATS, please refer to Appendix~\ref{app:ATS}.


% \renewcommand{\arraystretch}{0.8}
% \begin{table}[!h]
% \setlength\tabcolsep{1pt}
% \centering
% \caption {Ablation Study About Different  Module}\label{tab:ablation_modules}
% \vspace{1mm}
% \footnotesize
% \begin{tabular}{clllllllll}
% \hline
% \multirow{2}{*}{\textbf{Method}}  & \multicolumn{4}{c}{\textbf{LongVideoBench}} & \multicolumn{4}{c}{\textbf{MLVU}} \\ 
% \cmidrule(lr){2-5}
% \cmidrule(lr){6-9}
%  & 8k & 16k & 32k & 64k & 8k & 16k & 32k & 64k \\ \hline
% Baseline & 53.42 & 52.80 & 53.11 & 54.35 & 60.41 & 60.68 & 61.56 & 61.10 \\
% + DL & 52.17 & 52.07 & 53.31 & 53.63 & 62.06 & 63.03 & 62.52 & 62.75 \\
% + DL \& LTA & \textbf{54.46} & \textbf{55.49} & 54.66 & 55.60 & 63.35 & 64.09 & 64.00 & 63.26  \\
% + DL \& LTA \& ATS & {\textbf{54.46}} & 55.29 & {\textbf{57.15}} & {\textbf{57.26}} & {\textbf{65.19}} & \textbf{66.29} & {\textbf{66.02}} & {\textbf{65.56}} \\ \hline
% \end{tabular}

% \end{table}


\renewcommand{\arraystretch}{1.0} % 减少行间距
\begin{table}[t]
\setlength\tabcolsep{3pt} % 更小的列间距
\centering
\caption {\textbf{Ablation study about different modules of \methodname.}
% `DL' = Diagonal Layout, `LTA' = Low-frequency Temporal Layout, `ATS' = Adjustable Temporal Spacing.
}\label{tab:ablation_modules}
\vspace{1mm} % 减少表格上方的空白
% \footnotesize
\scriptsize % 更小的字体
\begin{tabular}{@{}clllllllll@{}}
\hline
\multirow{2}{*}{\textbf{Method}}  & \multicolumn{4}{c}{\textbf{LongVideoBench}} & \multicolumn{4}{c}{\textbf{MLVU}} \\ 
\cmidrule(lr){2-5}
\cmidrule(lr){6-9}
 & 8k & 16k & 32k & 64k & 8k & 16k & 32k & 64k \\ \hline
Baseline & 53.42 & 52.80 & 53.11 & 54.35 & 60.41 & 60.68 & 61.56 & 61.10 \\
+ DL & 52.17 & 52.07 & 53.31 & 53.63 & 62.06 & 63.03 & 62.52 & 62.75 \\
+ DL \& LTA & \textbf{54.46} & \textbf{55.49} & 54.66 & 55.60 & 63.35 & 64.09 & 64.00 & 63.26  \\
+ DL \& LTA \& ATS & {\textbf{54.46}} & 55.29 & {\textbf{57.15}} & {\textbf{57.26}} & {\textbf{65.19}} & \textbf{66.29} & \textbf{66.02} & \textbf{65.56} \\ \hline
\end{tabular}
\end{table}

% \renewcommand{\arraystretch}{0.6} % 调整行间距,使其更加紧凑
% \begin{table}[!h]
% \setlength\tabcolsep{0.8pt} % 减少列间距
% \centering
% \caption {Ablation Study About Different Modules}\label{tab:ablation_modules}
% \vspace{-1mm} % 减少标题与表格的空白
% \scriptsize % 更小的字体
% \begin{tabular}{clllllllll}
% \hline
% \multirow{2}{*}{\textbf{Method}}  & \multicolumn{4}{c}{\textbf{LongVideoBench}} & \multicolumn{4}{c}{\textbf{MLVU}} \\ 
% \cmidrule(lr){2-5}
% \cmidrule(lr){6-9}
%  & 8k & 16k & 32k & 64k & 8k & 16k & 32k & 64k \\ \hline
% \multirow{4}{*}{\textbf{Baseline}} & 53.42 & 52.80 & 53.11 & 54.35 & 60.41 & 60.68 & 61.56 & 61.10 \\
% & + DL & 52.17 & 52.07 & 53.31 & 53.63 & 62.06 & 63.03 & 62.52 & 62.75 \\
% & + DL \& LTA & \textbf{54.46} & \textbf{55.49} & 54.66 & 55.60 & 63.35 & 64.09 & 64.00 & 63.26  \\
% & + DL \& LTA \& ATS & {\textbf{54.46}} & 55.29 & {\textbf{57.15}} & {\textbf{57.26}} & {\textbf{65.19}} & \textbf{66.29} & \textbf{66.02} & \textbf{65.56} \\ \hline
% \end{tabular}
% \end{table}
