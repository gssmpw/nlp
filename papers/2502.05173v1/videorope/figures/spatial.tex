\begin{figure*}[t]
\begin{minipage}{0.98\textwidth}
    \begin{subfigure}[b]{0.3\linewidth}
        \centering
        \includegraphics[width=0.95\linewidth]{figures/files/vanilla_rope.pdf}
        \caption{3D visualization for Vanilla RoPE.}
        \label{fig:vanilla_rope}
    \end{subfigure}
    \hfill
    \begin{subfigure}[b]{0.3\linewidth}
        \centering
        \includegraphics[width=0.95\linewidth]{figures/files/m_rope.pdf}
        \caption{3D visualization for M-RoPE.}
        \label{fig:m_rope}
    \end{subfigure}
    \hfill
    \begin{subfigure}[b]{0.3\linewidth}
        \centering
        \includegraphics[width=0.95\linewidth]{figures/files/m_modify_rope.pdf}
        \caption{3D visualization for \methodname.}
        \label{fig:video_rope}
    \end{subfigure}
    \hfill
    \vspace{-6pt}
    \caption{\footnotesize The 3D visualization for different position embedding. \textbf{(a)} The vanilla 1D RoPE~\cite{su2024roformer} does not incorporate spatial modeling.
    \textbf{(b)} M-RoPE~\cite{wang2024qwen2}, while have the 3D structure, introduces a discrepancy in index growth for visual tokens across frames, with some indices remaining constant.
    \textbf{(c)} In contrast, our \methodname achieves the desired balance, maintaining the consistent index growth pattern of vanilla RoPE while simultaneously incorporating spatial modeling. 
    }
    %3D visualization of different position embeddings. \textbf{(a)} The vanilla 1D RoPE~\cite{su2024roformer} lacks spatial modeling. \textbf{(b)} M-RoPE~\cite{wang2024qwen2}, while have the 3D structure, introduces a discrepancy in index growth for visual tokens across frames, with some indices remaining constant. \textbf{(c)} Our \methodname balances the index growth of vanilla RoPE while incorporating spatial modeling. For more details on the index, see Appendix \ref{app:supp_explain_modules}.
    \vspace{-12pt}
    \label{fig:spatial}
\end{minipage}
\end{figure*}