\begin{figure*}[t]
\begin{minipage}{0.98\textwidth}
    \begin{subfigure}[b]{0.49\linewidth}
        \centering
\includegraphics[width=0.95\linewidth]{figures/files/video_rope-period_low-MRoPE.pdf}
        \caption{Temporal Frequency Allocation in M-RoPE}
        \label{fig:temporal_mrope}
    \end{subfigure}
    \hfill
    \begin{subfigure}[b]{0.49\linewidth}
        \centering
        \includegraphics[width=0.95\linewidth]{figures/files/video_rope-period_low-VideoRoPE.pdf}
        \caption{Temporal Frequency Allocation in \methodname (ours)}
        \label{fig:temporal_videorope}
    \end{subfigure}
    \vspace{-6pt}
    \caption{\footnotesize \textbf{(a)} M-RoPE \cite{wang2024qwen2} models temporal dependencies using the \textit{first} 16 rotary angles, which exhibit higher frequencies and more pronounced oscillations. \textbf{(b)} In contrast, \methodname models temporal dependencies using the \textit{last} 16 rotary angles, characterized by significantly wider, monotonic intervals. Our frequency allocation effectively mitigates the misleading influence of distractors in V-NIAH-D. For a more detailed analysis, please refer to Appendix \ref{app:supp_explain_modules}.
    % Take the first 3 rotary angles as an example, the position embedding for temporal modeling is free from oscillation~\cite{men2024base}.
    }
    \label{fig:period_mono}
    \vspace{-12pt}
\end{minipage}
\end{figure*}