\renewcommand{\arraystretch}{1.}
\setlength\tabcolsep{2pt}
\setlength{\textfloatsep}{5pt}
\begin{table}[t]
\centering
\footnotesize
\caption{
% Performance comparison of different RoPE methods evaluated on V-NIAH and V-NIAH-D under identical training conditions. Each frame is encoded using 144 tokens, with a depth interval of 0.2 used to insert the needle frame. The haystack is validated starting from 100 frames, with additional validation at 200-frame intervals up to 3000 frames. For the V-NIAH-D task, distractors are inserted periodically, with the period calculated as $2 \cdot \pi \cdot 1000000^{32/128} \approx 198.7$. In our experiments, we directly use a period of 200 for distractor insertion. ``Acc." refers to the average accuracy across the haystack length and frame depth. Our \methodname achieves superior performance compared to other RoPE variants on both V-NIAH and the more challenging V-NIAH-D dataset. The best results are marked in \textbf{bold}, while the second-best results are \underline{underlined}.
\textbf{Performance comparison of different RoPEs on V-NIAH and V-NIAH-D}. ``Acc.'' refers to the average accuracy across haystack length and frame depth.
% For the detailed setup of the evaluation, see Appendix \ref{appendix:benchmarks}.
}
\label{tab:v-niah-and-d}
\vspace{2mm}
\begin{tabular}{ccc}
\toprule
\textbf{Method} & \textbf{V-NIAH Acc.} & \textbf{V-NIAH-D Acc.} \\ \hline
Vanilla RoPE~\cite{su2024roformer} & 31.78 & 30.22 \\
TAD-RoPE~\cite{gao2024tc} & 29.33 & 29.56 \\
M-RoPE~\cite{wang2024qwen2} & \underline{78.67} & \underline{74.67} \\
\hline
\rowcolor[HTML]{F2F3F5}
\methodname & \textbf{91.11}  &\textbf{87.11} \\
\bottomrule
\end{tabular}
\end{table}