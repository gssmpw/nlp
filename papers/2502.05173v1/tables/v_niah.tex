\renewcommand{\arraystretch}{1.1}  % 增加行间距
\setlength{\tabcolsep}{8pt}
\begin{table}[!h]
\centering
\footnotesize
\caption{Performance comparison of different RoPE methods evaluated on V-NIAH under identical training conditions. We encode each frame using 144 tokens, with a depth interval of 0.2 to insert the needle frame. The haystack is validated starting from 100 frames, with subsequent validation at 200-frame intervals, up to 3000 frames. Our \methodname demonstrates superior performance compared to other RoPE variants on the more challenging V-NIAH dataset. The best results are marked in \textbf{bold}, while the second-best results are \underline{underlined}.}
\label{tab:v-niah}
\vspace{2mm}
\begin{tabular}{lc}
\toprule
\textbf{Method} & Avg. Accuracy \\ \hline
Vanilla-RoPE~\cite{su2024roformer} & 31.7 \\
TAD-RoPE~\cite{gao2024tc} & 29.3 \\
M-RoPE~\cite{wang2024qwen2} & \underline{78.6} \\ \hline
\rowcolor[HTML]{F2F3F5}
\methodname & \textbf{91.1} \\
\bottomrule
\end{tabular}
\end{table}