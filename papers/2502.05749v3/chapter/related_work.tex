\section{Related Work}
% In this section, we will review the existing works that establish methods for transforming from distribution to distribution and analyze their advantages and insufficiency. 
\textbf{Diffusion with Guidance.} This technique tackles conditional generative tasks by leveraging a differentiable loss function for guidance without the need for additional training \cite{chung2024diffusionposteriorsamplinggeneral, shenoy2024gradientfreeclassifierguidancediffusion, bradley2024classifierfreeguidancepredictorcorrector}. However, it often yields suboptimal image quality and a prolonged sampling process due to the necessity of small step sizes.  Most importantly, the sampling process is prone to manifold deviations and detail losses \cite{yang2024guidancesphericalgaussianconstraint}. Furthermore, enhancing the guidance of the diffusion typically requires the introduction of additional modules, thereby increasing the model’s computational complexity. 


\textbf{Diffusion Bridge with Doob's $h$-transform.} Recent advances in diffusion bridging have demonstrated the efficacy of Doob's $h$-transform in enhancing transition quality between arbitrary distributions. Notably, DDBMs \cite{zhou2023denoisingdiffusionbridgemodels} pioneered this approach by employing a linear SDE combined with Doob's $h$-transform to construct direct diffusion bridges. Subsequently, GOUB \cite{yue2024imagerestorationgeneralizedornsteinuhlenbeck} extends this framework by integrating Doob's $h$-transform with a mean-reverting SDE, achieving state-of-the-art performance in image restoration tasks. Despite these empirical successes, the theoretical foundations of Doob's $h$-transform in this context remain insufficiently explored. In addition, these methods often result in images with blurred or oversmoothed features, particularly affecting the capture of high-frequency details crucial for perceptual fidelity. 


\textbf{Diffusion with Stochastic Optimal Control.} The integration of SOC principles into diffusion models has emerged as a promising paradigm for guiding distribution transitions. DIS \cite{berner2024optimalcontrolperspectivediffusionbased} established a foundational theoretical linkage between diffusion processes and SOC, while RB-Modulation \cite{RB} operationalized SOC via a simplified SDE structure for training-free style transfer using pre-trained diffusion models. Close to our work, DBFS \cite{park2024stochasticoptimalcontroldiffusion} leveraged SOC to construct diffusion bridges in infinite-dimensional function spaces and also established equivalence between SOC and Doob's $h$-transform. However, DBFS primarily extends Doob's $h$-transform to infinite Hilbert spaces via SOC, without addressing its intrinsic limitations. Our analysis reveals a critical insight: Doob's $h$-transform corresponds to a suboptimal solution that can inherently lead to artifacts such as blurred or distorted details. To resolve this, we introduce a unified SOC framework that jointly optimizes trajectory costs and terminal constraints, enhancing detail preservation and image quality. 

