\section{Introduction}
\begin{figure*}[t] % '!t' 表示尽可能靠近页面顶部
    \centering
    \includegraphics[width=0.98\textwidth]{picture/main_2.pdf}
    \vspace{-4mm}
    \caption{Here we briefly compare the performance of UniDB to diffusion bridge with Doob's $h$-transform 
    \cite{yue2024imagerestorationgeneralizedornsteinuhlenbeck} across various tasks, including Super-resolution, Inpainting and Deraining. UniDB effectively balances control and terminal costs by modifying the terminal penalty coefficient, alleviating the problems caused by Doob's $h$-transform  in these applications. This framework significantly boosts the detail rendering ability of generated images while imposing minimal overhead in code modifications.}
    \label{fig:widefig}
    \vspace{-3mm}
\end{figure*}


The diffusion model has been extensively utilized across a range of applications, including image generation and editing \cite{ho2020denoisingdiffusionprobabilisticmodels, DDRM, song2021scorebasedgenerativemodelingstochastic, DiffIR, li2023diffusionmodelsimagerestoration}, imitation learning \cite{afforddp, dp, 3ddp} and reinforcement learning \cite{yang2023policyrepresentationdiffusionprobability, QVPO}, etc. Despite its versatility, the standard diffusion model faces limitations in transitioning between arbitrary distributions due to its inherent assumption of a Gaussian noise prior. To overcome this problem, diffusion models \cite{dhariwal2021diffusionmodelsbeatgans, ho2022classifier,  murata2023gibbsddrmpartiallycollapsedgibbs, CCDM, chung2024diffusionposteriorsamplinggeneral, tang2024unified} often rely on meticulously designed conditioning mechanisms and classifier/loss guidance to facilitate conditional sampling and ensure output alignment with a target distribution. However, these methods can be cumbersome and may introduce manifold deviations during the sampling process. Meanwhile, Diffusion Schrödinger Bridge \cite{shi2023diffusionschrodingerbridgematching, debortoli2023diffusionschrodingerbridgeapplications, somnath2024aligneddiffusionschrodingerbridges} involves constraints that hinder direct optimization of the KL divergence, resulting in slow convergence and limited model fitting capability.

To address this challenge, DDBMs \cite{zheng2024diffusionbridgeimplicitmodels} proposed a diffusion bridge model using Doob's $h$-transform. This framework is specifically designed to establish fixed endpoints between two distinct distributions by learning the score function of the diffusion bridge from data, and then solving the stochastic differential equation (SDE) based on these learned scores to transition from one endpoint distribution to another. However, the forward SDE in DDBMs lacks the mean information of the terminal distribution, which restricts the quality of the generated images, particularly in image restoration tasks. Subsequently, GOUB \cite{yue2024imagerestorationgeneralizedornsteinuhlenbeck} extends this framework by integrating Doob's $h$-transform with a mean-reverting SDE, achieving better results compared to DDBMs. 
Despite the promising results in diffusion bridge with Doob's $h$-transform, two fundamental challenges persist: 1) the theoretical mechanisms by which Doob's $h$-transform governs the bridging process remain poorly understood, lacking a rigorous framework to unify its empirical success; and 2) while effective for global distribution alignment, existing methods frequently degrade high-frequency details—such as sharp edges and fine textures—resulting in outputs with blurred or oversmoothed artifacts that compromise perceptual fidelity. These limitations underscore the need for both theoretical grounding and enhanced detail preservation in diffusion bridges. 


In this paper, we revisit the diffusion bridges through the lens of stochastic optimal control (SOC) by introducing a novel framework called UniDB, which formulates an optimization problem based on SOC principles to implement diffusion bridges. It enables the derivation of a closed-form solution for the optimal controller, along with the corresponding training objective for the diffusion bridge. UniDB identifies Doob's $h$-transform as a special case when the terminal penalty coefficient in the SOC cost function approaches infinity. This explains why Doob's $h$-transform may result in suboptimal solutions with blurred or distorted details. To address this limitation, UniDB utilizes the penalty coefficient in SOC to adjust the expressiveness of the image details and enhance the authenticity of the generated outputs. Our main contributions are as follows: 

\begin{itemize}

\vspace{-3mm}

\item We introduce UniDB, a novel unified diffusion bridge framework based on stochastic optimal control. This framework generalizes existing diffusion bridge models like DDBMs and GOUB, offering a comprehensive understanding and extension of Doob’s $h$-transform by incorporating general forward SDE forms.



\item We derive closed-form solutions for the SOC problem, demonstrating that Doob’s $h$-transform is merely a special case within UniDB when the terminal penalty coefficient in the SOC cost function approaches infinity. This insight reveals inherent limitations in the existing diffusion bridge approaches, which UniDB overcomes. Notably, the improvement of UniDB requires minimal code modification, ensuring easy implementation. 


\item UniDB achieves state-of-the-art results in various image restoration tasks, including super-resolution (DIV2K), inpainting (CelebA-HQ), and deraining (Rain100H), which highlights the framework’s superior image quality and adaptability across diverse scenarios. 

\end{itemize}