%%%%%%%% ICML 2025 EXAMPLE LATEX SUBMISSION FILE %%%%%%%%%%%%%%%%%

\documentclass{article}

% Recommended, but optional, packages for figures and better typesetting:
\usepackage{microtype}
\usepackage{graphicx}
\usepackage{subfigure}
\usepackage{booktabs} % for professional tables
\usepackage{epsfig} 
\usepackage{times} 
\usepackage{multirow}
\usepackage{multicol}
\usepackage{booktabs}
\usepackage{siunitx}            
\usepackage[export]{adjustbox}
\usepackage{xcolor}
\usepackage{colortbl} 
\usepackage{algorithm}

% hyperref makes hyperlinks in the resulting PDF.
% If your build breaks (sometimes temporarily if a hyperlink spans a page)
% please comment out the following usepackage line and replace
% \usepackage{icml2025} with \usepackage[nohyperref]{icml2025} above.
\usepackage{hyperref}


% Attempt to make hyperref and algorithmic work together better:
\newcommand{\theHalgorithm}{\arabic{algorithm}}

% Use the following line for the initial blind version submitted for review:
% \usepackage{icml2025}

% If accepted, instead use the following line for the camera-ready submission:
\usepackage[accepted]{icml2025}

% For theorems and such
\usepackage{amsmath}
\usepackage{amssymb}
\usepackage{mathtools}
\usepackage{amsthm}
\usepackage{bbding}
\usepackage{pifont}
\usepackage{listings}
\usepackage{cancel}
\usepackage{tikz}
\usetikzlibrary{fit}

\newcommand{\tikzmark}[1]{\tikz[overlay,remember picture] \node (#1) {};}

\definecolor{codegreen}{rgb}{0,0.6,0}
\definecolor{codegray}{rgb}{0.5,0.5,0.5}
\definecolor{codepurple}{rgb}{0.58,0,0.82}
\definecolor{backcolour}{rgb}{0.95,0.95,0.92}

\lstdefinestyle{mystyle}{
    backgroundcolor=\color{backcolour},   
    commentstyle=\color{codegreen},
    keywordstyle=\color{magenta},
    numberstyle=\tiny\color{codegray},
    stringstyle=\color{codepurple},
    basicstyle=\ttfamily\footnotesize,
    breakatwhitespace=false,         
    breaklines=true,                 
    captionpos=b,                    
    keepspaces=true,                 
    numbers=left,                    
    numbersep=5pt,                  
    showspaces=false,                
    showstringspaces=false,
    showtabs=false,                  
    tabsize=2
}

\lstset{style=mystyle}


% \usepackage[ruled,vlined]{algorithm2e}

% if you use cleveref..
\usepackage[capitalize,noabbrev]{cleveref}

%%%%%%%%%%%%%%%%%%%%%%%%%%%%%%%%
% THEOREMS
%%%%%%%%%%%%%%%%%%%%%%%%%%%%%%%%
\theoremstyle{plain}
\newtheorem{theorem}{Theorem}[section]
\newtheorem{proposition}[theorem]{Proposition}
\newtheorem{lemma}[theorem]{Lemma}
\newtheorem{corollary}[theorem]{Corollary}
\theoremstyle{definition}
\newtheorem{definition}[theorem]{Definition}
\newtheorem{assumption}[theorem]{Assumption}
\theoremstyle{remark}
\newtheorem{remark}[theorem]{Remark}

% Todonotes is useful during development; simply uncomment the next line
%    and comment out the line below the next line to turn off comments
%\usepackage[disable,textsize=tiny]{todonotes}
\usepackage[textsize=tiny]{todonotes}


% The \icmltitle you define below is probably too long as a header.
% Therefore, a short form for the running title is supplied here:
\icmltitlerunning{UniDB: A Unified Diffusion Bridge Framework via Stochastic Optimal Control}

\begin{document}

\twocolumn[
\icmltitle{UniDB: A Unified Diffusion Bridge Framework via Stochastic Optimal Control}

% It is OKAY to include author information, even for blind
% submissions: the style file will automatically remove it for you
% unless you've provided the [accepted] option to the icml2025
% package.

% List of affiliations: The first argument should be a (short)
% identifier you will use later to specify author affiliations
% Academic affiliations should list Department, University, City, Region, Country
% Industry affiliations should list Company, City, Region, Country

% You can specify symbols, otherwise they are numbered in order.
% Ideally, you should not use this facility. Affiliations will be numbered
% in order of appearance and this is the preferred way.
\icmlsetsymbol{Corresponding author}{$\dagger$}
\icmlsetsymbol{equal}{*}

\begin{icmlauthorlist}
\icmlauthor{Kaizhen Zhu}{shanghaitech,moe,equal}
\icmlauthor{Mokai Pan}{shanghaitech,moe,equal}
\icmlauthor{Yuexin Ma}{shanghaitech,moe}
\icmlauthor{Yanwei Fu}{fudan}
\icmlauthor{Jingyi Yu}{shanghaitech,moe}
\icmlauthor{Jingya Wang}{shanghaitech,moe}
\icmlauthor{Ye Shi}{shanghaitech,moe,Corresponding author}

\end{icmlauthorlist}

\icmlaffiliation{shanghaitech}{ShanghaiTech University}
\icmlaffiliation{moe}{MoE Key Laboratory of Intelligent Perception and Human-Machine Collaboration}
\icmlaffiliation{fudan}{Fudan University}
\icmlcorrespondingauthor{Ye Shi}{shiye@shanghaitech.edu.cn}

% You may provide any keywords that you
% find helpful for describing your paper; these are used to populate
% the "keywords" metadata in the PDF but will not be shown in the document
\icmlkeywords{Machine Learning, ICML}

\vskip 0.3in
]

% this must go after the closing bracket ] following \twocolumn[ ...

% This command actually creates the footnote in the first column
% listing the affiliations and the copyright notice.
% The command takes one argument, which is text to display at the start of the footnote.
% The \icmlEqualContribution command is standard text for equal contribution.
% Remove it (just {}) if you do not need this facility.

\printAffiliationsAndNotice{\icmlEqualContribution}  % leave blank if no need to mention equal contribution
% \printAffiliationsAndNotice{\icmlEqualContribution} % otherwise use the standard text.

\begin{abstract}  
Test time scaling is currently one of the most active research areas that shows promise after training time scaling has reached its limits.
Deep-thinking (DT) models are a class of recurrent models that can perform easy-to-hard generalization by assigning more compute to harder test samples.
However, due to their inability to determine the complexity of a test sample, DT models have to use a large amount of computation for both easy and hard test samples.
Excessive test time computation is wasteful and can cause the ``overthinking'' problem where more test time computation leads to worse results.
In this paper, we introduce a test time training method for determining the optimal amount of computation needed for each sample during test time.
We also propose Conv-LiGRU, a novel recurrent architecture for efficient and robust visual reasoning. 
Extensive experiments demonstrate that Conv-LiGRU is more stable than DT, effectively mitigates the ``overthinking'' phenomenon, and achieves superior accuracy.
\end{abstract}  
\section{Introduction}
\label{sec:introduction}
The business processes of organizations are experiencing ever-increasing complexity due to the large amount of data, high number of users, and high-tech devices involved \cite{martin2021pmopportunitieschallenges, beerepoot2023biggestbpmproblems}. This complexity may cause business processes to deviate from normal control flow due to unforeseen and disruptive anomalies \cite{adams2023proceddsriftdetection}. These control-flow anomalies manifest as unknown, skipped, and wrongly-ordered activities in the traces of event logs monitored from the execution of business processes \cite{ko2023adsystematicreview}. For the sake of clarity, let us consider an illustrative example of such anomalies. Figure \ref{FP_ANOMALIES} shows a so-called event log footprint, which captures the control flow relations of four activities of a hypothetical event log. In particular, this footprint captures the control-flow relations between activities \texttt{a}, \texttt{b}, \texttt{c} and \texttt{d}. These are the causal ($\rightarrow$) relation, concurrent ($\parallel$) relation, and other ($\#$) relations such as exclusivity or non-local dependency \cite{aalst2022pmhandbook}. In addition, on the right are six traces, of which five exhibit skipped, wrongly-ordered and unknown control-flow anomalies. For example, $\langle$\texttt{a b d}$\rangle$ has a skipped activity, which is \texttt{c}. Because of this skipped activity, the control-flow relation \texttt{b}$\,\#\,$\texttt{d} is violated, since \texttt{d} directly follows \texttt{b} in the anomalous trace.
\begin{figure}[!t]
\centering
\includegraphics[width=0.9\columnwidth]{images/FP_ANOMALIES.png}
\caption{An example event log footprint with six traces, of which five exhibit control-flow anomalies.}
\label{FP_ANOMALIES}
\end{figure}

\subsection{Control-flow anomaly detection}
Control-flow anomaly detection techniques aim to characterize the normal control flow from event logs and verify whether these deviations occur in new event logs \cite{ko2023adsystematicreview}. To develop control-flow anomaly detection techniques, \revision{process mining} has seen widespread adoption owing to process discovery and \revision{conformance checking}. On the one hand, process discovery is a set of algorithms that encode control-flow relations as a set of model elements and constraints according to a given modeling formalism \cite{aalst2022pmhandbook}; hereafter, we refer to the Petri net, a widespread modeling formalism. On the other hand, \revision{conformance checking} is an explainable set of algorithms that allows linking any deviations with the reference Petri net and providing the fitness measure, namely a measure of how much the Petri net fits the new event log \cite{aalst2022pmhandbook}. Many control-flow anomaly detection techniques based on \revision{conformance checking} (hereafter, \revision{conformance checking}-based techniques) use the fitness measure to determine whether an event log is anomalous \cite{bezerra2009pmad, bezerra2013adlogspais, myers2018icsadpm, pecchia2020applicationfailuresanalysispm}. 

The scientific literature also includes many \revision{conformance checking}-independent techniques for control-flow anomaly detection that combine specific types of trace encodings with machine/deep learning \cite{ko2023adsystematicreview, tavares2023pmtraceencoding}. Whereas these techniques are very effective, their explainability is challenging due to both the type of trace encoding employed and the machine/deep learning model used \cite{rawal2022trustworthyaiadvances,li2023explainablead}. Hence, in the following, we focus on the shortcomings of \revision{conformance checking}-based techniques to investigate whether it is possible to support the development of competitive control-flow anomaly detection techniques while maintaining the explainable nature of \revision{conformance checking}.
\begin{figure}[!t]
\centering
\includegraphics[width=\columnwidth]{images/HIGH_LEVEL_VIEW.png}
\caption{A high-level view of the proposed framework for combining \revision{process mining}-based feature extraction with dimensionality reduction for control-flow anomaly detection.}
\label{HIGH_LEVEL_VIEW}
\end{figure}

\subsection{Shortcomings of \revision{conformance checking}-based techniques}
Unfortunately, the detection effectiveness of \revision{conformance checking}-based techniques is affected by noisy data and low-quality Petri nets, which may be due to human errors in the modeling process or representational bias of process discovery algorithms \cite{bezerra2013adlogspais, pecchia2020applicationfailuresanalysispm, aalst2016pm}. Specifically, on the one hand, noisy data may introduce infrequent and deceptive control-flow relations that may result in inconsistent fitness measures, whereas, on the other hand, checking event logs against a low-quality Petri net could lead to an unreliable distribution of fitness measures. Nonetheless, such Petri nets can still be used as references to obtain insightful information for \revision{process mining}-based feature extraction, supporting the development of competitive and explainable \revision{conformance checking}-based techniques for control-flow anomaly detection despite the problems above. For example, a few works outline that token-based \revision{conformance checking} can be used for \revision{process mining}-based feature extraction to build tabular data and develop effective \revision{conformance checking}-based techniques for control-flow anomaly detection \cite{singh2022lapmsh, debenedictis2023dtadiiot}. However, to the best of our knowledge, the scientific literature lacks a structured proposal for \revision{process mining}-based feature extraction using the state-of-the-art \revision{conformance checking} variant, namely alignment-based \revision{conformance checking}.

\subsection{Contributions}
We propose a novel \revision{process mining}-based feature extraction approach with alignment-based \revision{conformance checking}. This variant aligns the deviating control flow with a reference Petri net; the resulting alignment can be inspected to extract additional statistics such as the number of times a given activity caused mismatches \cite{aalst2022pmhandbook}. We integrate this approach into a flexible and explainable framework for developing techniques for control-flow anomaly detection. The framework combines \revision{process mining}-based feature extraction and dimensionality reduction to handle high-dimensional feature sets, achieve detection effectiveness, and support explainability. Notably, in addition to our proposed \revision{process mining}-based feature extraction approach, the framework allows employing other approaches, enabling a fair comparison of multiple \revision{conformance checking}-based and \revision{conformance checking}-independent techniques for control-flow anomaly detection. Figure \ref{HIGH_LEVEL_VIEW} shows a high-level view of the framework. Business processes are monitored, and event logs obtained from the database of information systems. Subsequently, \revision{process mining}-based feature extraction is applied to these event logs and tabular data input to dimensionality reduction to identify control-flow anomalies. We apply several \revision{conformance checking}-based and \revision{conformance checking}-independent framework techniques to publicly available datasets, simulated data of a case study from railways, and real-world data of a case study from healthcare. We show that the framework techniques implementing our approach outperform the baseline \revision{conformance checking}-based techniques while maintaining the explainable nature of \revision{conformance checking}.

In summary, the contributions of this paper are as follows.
\begin{itemize}
    \item{
        A novel \revision{process mining}-based feature extraction approach to support the development of competitive and explainable \revision{conformance checking}-based techniques for control-flow anomaly detection.
    }
    \item{
        A flexible and explainable framework for developing techniques for control-flow anomaly detection using \revision{process mining}-based feature extraction and dimensionality reduction.
    }
    \item{
        Application to synthetic and real-world datasets of several \revision{conformance checking}-based and \revision{conformance checking}-independent framework techniques, evaluating their detection effectiveness and explainability.
    }
\end{itemize}

The rest of the paper is organized as follows.
\begin{itemize}
    \item Section \ref{sec:related_work} reviews the existing techniques for control-flow anomaly detection, categorizing them into \revision{conformance checking}-based and \revision{conformance checking}-independent techniques.
    \item Section \ref{sec:abccfe} provides the preliminaries of \revision{process mining} to establish the notation used throughout the paper, and delves into the details of the proposed \revision{process mining}-based feature extraction approach with alignment-based \revision{conformance checking}.
    \item Section \ref{sec:framework} describes the framework for developing \revision{conformance checking}-based and \revision{conformance checking}-independent techniques for control-flow anomaly detection that combine \revision{process mining}-based feature extraction and dimensionality reduction.
    \item Section \ref{sec:evaluation} presents the experiments conducted with multiple framework and baseline techniques using data from publicly available datasets and case studies.
    \item Section \ref{sec:conclusions} draws the conclusions and presents future work.
\end{itemize}
\section{RELATED WORK}
\label{sec:relatedwork}
In this section, we describe the previous works related to our proposal, which are divided into two parts. In Section~\ref{sec:relatedwork_exoplanet}, we present a review of approaches based on machine learning techniques for the detection of planetary transit signals. Section~\ref{sec:relatedwork_attention} provides an account of the approaches based on attention mechanisms applied in Astronomy.\par

\subsection{Exoplanet detection}
\label{sec:relatedwork_exoplanet}
Machine learning methods have achieved great performance for the automatic selection of exoplanet transit signals. One of the earliest applications of machine learning is a model named Autovetter \citep{MCcauliff}, which is a random forest (RF) model based on characteristics derived from Kepler pipeline statistics to classify exoplanet and false positive signals. Then, other studies emerged that also used supervised learning. \cite{mislis2016sidra} also used a RF, but unlike the work by \citet{MCcauliff}, they used simulated light curves and a box least square \citep[BLS;][]{kovacs2002box}-based periodogram to search for transiting exoplanets. \citet{thompson2015machine} proposed a k-nearest neighbors model for Kepler data to determine if a given signal has similarity to known transits. Unsupervised learning techniques were also applied, such as self-organizing maps (SOM), proposed \citet{armstrong2016transit}; which implements an architecture to segment similar light curves. In the same way, \citet{armstrong2018automatic} developed a combination of supervised and unsupervised learning, including RF and SOM models. In general, these approaches require a previous phase of feature engineering for each light curve. \par

%DL is a modern data-driven technology that automatically extracts characteristics, and that has been successful in classification problems from a variety of application domains. The architecture relies on several layers of NNs of simple interconnected units and uses layers to build increasingly complex and useful features by means of linear and non-linear transformation. This family of models is capable of generating increasingly high-level representations \citep{lecun2015deep}.

The application of DL for exoplanetary signal detection has evolved rapidly in recent years and has become very popular in planetary science.  \citet{pearson2018} and \citet{zucker2018shallow} developed CNN-based algorithms that learn from synthetic data to search for exoplanets. Perhaps one of the most successful applications of the DL models in transit detection was that of \citet{Shallue_2018}; who, in collaboration with Google, proposed a CNN named AstroNet that recognizes exoplanet signals in real data from Kepler. AstroNet uses the training set of labelled TCEs from the Autovetter planet candidate catalog of Q1–Q17 data release 24 (DR24) of the Kepler mission \citep{catanzarite2015autovetter}. AstroNet analyses the data in two views: a ``global view'', and ``local view'' \citep{Shallue_2018}. \par


% The global view shows the characteristics of the light curve over an orbital period, and a local view shows the moment at occurring the transit in detail

%different = space-based

Based on AstroNet, researchers have modified the original AstroNet model to rank candidates from different surveys, specifically for Kepler and TESS missions. \citet{ansdell2018scientific} developed a CNN trained on Kepler data, and included for the first time the information on the centroids, showing that the model improves performance considerably. Then, \citet{osborn2020rapid} and \citet{yu2019identifying} also included the centroids information, but in addition, \citet{osborn2020rapid} included information of the stellar and transit parameters. Finally, \citet{rao2021nigraha} proposed a pipeline that includes a new ``half-phase'' view of the transit signal. This half-phase view represents a transit view with a different time and phase. The purpose of this view is to recover any possible secondary eclipse (the object hiding behind the disk of the primary star).


%last pipeline applies a procedure after the prediction of the model to obtain new candidates, this process is carried out through a series of steps that include the evaluation with Discovery and Validation of Exoplanets (DAVE) \citet{kostov2019discovery} that was adapted for the TESS telescope.\par
%



\subsection{Attention mechanisms in astronomy}
\label{sec:relatedwork_attention}
Despite the remarkable success of attention mechanisms in sequential data, few papers have exploited their advantages in astronomy. In particular, there are no models based on attention mechanisms for detecting planets. Below we present a summary of the main applications of this modeling approach to astronomy, based on two points of view; performance and interpretability of the model.\par
%Attention mechanisms have not yet been explored in all sub-areas of astronomy. However, recent works show a successful application of the mechanism.
%performance

The application of attention mechanisms has shown improvements in the performance of some regression and classification tasks compared to previous approaches. One of the first implementations of the attention mechanism was to find gravitational lenses proposed by \citet{thuruthipilly2021finding}. They designed 21 self-attention-based encoder models, where each model was trained separately with 18,000 simulated images, demonstrating that the model based on the Transformer has a better performance and uses fewer trainable parameters compared to CNN. A novel application was proposed by \citet{lin2021galaxy} for the morphological classification of galaxies, who used an architecture derived from the Transformer, named Vision Transformer (VIT) \citep{dosovitskiy2020image}. \citet{lin2021galaxy} demonstrated competitive results compared to CNNs. Another application with successful results was proposed by \citet{zerveas2021transformer}; which first proposed a transformer-based framework for learning unsupervised representations of multivariate time series. Their methodology takes advantage of unlabeled data to train an encoder and extract dense vector representations of time series. Subsequently, they evaluate the model for regression and classification tasks, demonstrating better performance than other state-of-the-art supervised methods, even with data sets with limited samples.

%interpretation
Regarding the interpretability of the model, a recent contribution that analyses the attention maps was presented by \citet{bowles20212}, which explored the use of group-equivariant self-attention for radio astronomy classification. Compared to other approaches, this model analysed the attention maps of the predictions and showed that the mechanism extracts the brightest spots and jets of the radio source more clearly. This indicates that attention maps for prediction interpretation could help experts see patterns that the human eye often misses. \par

In the field of variable stars, \citet{allam2021paying} employed the mechanism for classifying multivariate time series in variable stars. And additionally, \citet{allam2021paying} showed that the activation weights are accommodated according to the variation in brightness of the star, achieving a more interpretable model. And finally, related to the TESS telescope, \citet{morvan2022don} proposed a model that removes the noise from the light curves through the distribution of attention weights. \citet{morvan2022don} showed that the use of the attention mechanism is excellent for removing noise and outliers in time series datasets compared with other approaches. In addition, the use of attention maps allowed them to show the representations learned from the model. \par

Recent attention mechanism approaches in astronomy demonstrate comparable results with earlier approaches, such as CNNs. At the same time, they offer interpretability of their results, which allows a post-prediction analysis. \par


% !TEX root =  ../main.tex
\section{Background on causality and abstraction}\label{sec:preliminaries}

This section provides the notation and key concepts related to causal modeling and abstraction theory.

\spara{Notation.} The set of integers from $1$ to $n$ is $[n]$.
The vectors of zeros and ones of size $n$ are $\zeros_n$ and $\ones_n$.
The identity matrix of size $n \times n$ is $\identity_n$. The Frobenius norm is $\frob{\mathbf{A}}$.
The set of positive definite matrices over $\reall^{n\times n}$ is $\pd^n$. The Hadamard product is $\odot$.
Function composition is $\circ$.
The domain of a function is $\dom{\cdot}$ and its kernel $\ker$.
Let $\mathcal{M}(\mathcal{X}^n)$ be the set of Borel measures over $\mathcal{X}^n \subseteq \reall^n$. Given a measure $\mu^n \in \mathcal{M}(\mathcal{X}^n)$ and a measurable map $\varphi^{\V}$, $\mathcal{X}^n \ni \mathbf{x} \overset{\varphi^{\V}}{\longmapsto} \V^\top \mathbf{x} \in \mathcal{X}^m$, we denote by $\varphi^{\V}_{\#}(\mu^n) \coloneqq \mu^n(\varphi^{\V^{-1}}(\mathbf{x}))$ the pushforward measure $\mu^m \in \mathcal{M}(\mathcal{X}^m)$. 


We now present the standard definition of SCM.

\begin{definition}[SCM, \citealp{pearl2009causality}]\label{def:SCM}
A (Markovian) structural causal model (SCM) $\scm^n$ is a tuple $\langle \myendogenous, \myexogenous, \myfunctional, \zeta^\myexogenous \rangle$, where \emph{(i)} $\myendogenous = \{X_1, \ldots, X_n\}$ is a set of $n$ endogenous random variables; \emph{(ii)} $\myexogenous =\{Z_1,\ldots,Z_n\}$ is a set of $n$ exogenous variables; \emph{(iii)} $\myfunctional$ is a set of $n$ functional assignments such that $X_i=f_i(\parents_i, Z_i)$, $\forall \; i \in [n]$, with $ \parents_i \subseteq \myendogenous \setminus \{ X_i\}$; \emph{(iv)} $\zeta^\myexogenous$ is a product probability measure over independent exogenous variables $\zeta^\myexogenous=\prod_{i \in [n]} \zeta^i$, where $\zeta^i=P(Z_i)$. 
\end{definition}
A Markovian SCM induces a directed acyclic graph (DAG) $\mathcal{G}_{\scm^n}$ where the nodes represent the variables $\myendogenous$ and the edges are determined by the structural functions $\myfunctional$; $ \parents_i$ constitutes then the parent set for $X_i$. Furthermore, we can recursively rewrite the set of structural function $\myfunctional$ as a set of mixing functions $\mymixing$ dependent only on the exogenous variables (cf. \cref{app:CA}). A key feature for studying causality is the possibility of defining interventions on the model:
\begin{definition}[Hard intervention, \citealp{pearl2009causality}]\label{def:intervention}
Given SCM $\scm^n = \langle \myendogenous, \myexogenous, \myfunctional, \zeta^\myexogenous \rangle$, a (hard) intervention $\iota = \operatorname{do}(\myendogenous^{\iota} = \mathbf{x}^{\iota})$, $\myendogenous^{\iota}\subseteq \myendogenous$,
is an operator that generates a new post-intervention SCM $\scm^n_\iota = \langle \myendogenous, \myexogenous, \myfunctional_\iota, \zeta^\myexogenous \rangle$ by replacing each function $f_i$ for $X_i\in\myendogenous^{\iota}$ with the constant $x_i^\iota\in \mathbf{x}^\iota$. 
Graphically, an intervention mutilates $\mathcal{G}_{\mathsf{M}^n}$ by removing all the incoming edges of the variables in $\myendogenous^{\iota}$.
\end{definition}

Given multiple SCMs describing the same system at different levels of granularity, CA provides the definition of an $\alpha$-abstraction map to relate these SCMs:
\begin{definition}[$\abst$-abstraction, \citealp{rischel2020category}]\label{def:abstraction}
Given low-level $\mathsf{M}^\ell$ and high-level $\mathsf{M}^h$ SCMs, an $\abst$-abstraction is a triple $\abst = \langle \Rset, \amap, \alphamap{} \rangle$, where \emph{(i)} $\Rset \subseteq \datalow$ is a subset of relevant variables in $\mathsf{M}^\ell$; \emph{(ii)} $\amap: \Rset \rightarrow \datahigh$ is a surjective function between the relevant variables of $\mathsf{M}^\ell$ and the endogenous variables of $\mathsf{M}^h$; \emph{(iii)} $\alphamap{}: \dom{\Rset} \rightarrow \dom{\datahigh}$ is a modular function $\alphamap{} = \bigotimes_{i\in[n]} \alphamap{X^h_i}$ made up by surjective functions $\alphamap{X^h_i}: \dom{\amap^{-1}(X^h_i)} \rightarrow \dom{X^h_i}$ from the outcome of low-level variables $\amap^{-1}(X^h_i) \in \datalow$ onto outcomes of the high-level variables $X^h_i \in \datahigh$.
\end{definition}
Notice that an $\abst$-abstraction simultaneously maps variables via the function $\amap$ and values through the function $\alphamap{}$. The definition itself does not place any constraint on these functions, although a common requirement in the literature is for the abstraction to satisfy \emph{interventional consistency} \cite{rubenstein2017causal,rischel2020category,beckers2019abstracting}. An important class of such well-behaved abstractions is \emph{constructive linear abstraction}, for which the following properties hold. By constructivity, \emph{(i)} $\abst$ is interventionally consistent; \emph{(ii)} all low-level variables are relevant $\Rset=\datalow$; \emph{(iii)} in addition to the map $\alphamap{}$ between endogenous variables, there exists a map ${\alphamap{}}_U$ between exogenous variables satisfying interventional consistency \cite{beckers2019abstracting,schooltink2024aligning}. By linearity, $\alphamap{} = \V^\top \in \reall^{h \times \ell}$ \cite{massidda2024learningcausalabstractionslinear}. \cref{app:CA} provides formal definitions for interventional consistency, linear and constructive abstraction.
\subsection{Greedies}
We have two greedy methods that we're using and testing, but they both have one thing in common: They try every node and possible resistances, and choose the one that results in the largest change in the objective function.

The differences between the two methods, are the function. The first one uses the median (since we want the median to be >0.5, we just set this as our objective function.)

Second one uses a function to try to capture more nuances about the fact that we want the median to be over 0.5. The function is:

\[
\text{score}(\text{opinion}) =
\begin{cases} 
\text{maxScore}, & \text{if } \text{opinion} \geq 0.5 \\
\min\left(\frac{50}{0.5 - \text{opinion}}, \frac{\text{maxScore}}{2}\right), & \text{if } \text{opinion} < 0.5 
\end{cases}
\] 

Where we set maxScore to be $10000$.

\subsection{find-c}
Then for the projected methods where we use Huber-Loss, we also have a $find-c$ version (temporary name). This method initially finds the c for the rest of the run. 

The way it does it it randomly perturbs the resistances and opinions of every node, then finds the c value that most closely approximates the median for all of the perturbed scenarios (after finding the stable opinions). 

\section{Experiments}
\label{sec:experiments}
The experiments are designed to address two key research questions.
First, \textbf{RQ1} evaluates whether the average $L_2$-norm of the counterfactual perturbation vectors ($\overline{||\perturb||}$) decreases as the model overfits the data, thereby providing further empirical validation for our hypothesis.
Second, \textbf{RQ2} evaluates the ability of the proposed counterfactual regularized loss, as defined in (\ref{eq:regularized_loss2}), to mitigate overfitting when compared to existing regularization techniques.

% The experiments are designed to address three key research questions. First, \textbf{RQ1} investigates whether the mean perturbation vector norm decreases as the model overfits the data, aiming to further validate our intuition. Second, \textbf{RQ2} explores whether the mean perturbation vector norm can be effectively leveraged as a regularization term during training, offering insights into its potential role in mitigating overfitting. Finally, \textbf{RQ3} examines whether our counterfactual regularizer enables the model to achieve superior performance compared to existing regularization methods, thus highlighting its practical advantage.

\subsection{Experimental Setup}
\textbf{\textit{Datasets, Models, and Tasks.}}
The experiments are conducted on three datasets: \textit{Water Potability}~\cite{kadiwal2020waterpotability}, \textit{Phomene}~\cite{phomene}, and \textit{CIFAR-10}~\cite{krizhevsky2009learning}. For \textit{Water Potability} and \textit{Phomene}, we randomly select $80\%$ of the samples for the training set, and the remaining $20\%$ for the test set, \textit{CIFAR-10} comes already split. Furthermore, we consider the following models: Logistic Regression, Multi-Layer Perceptron (MLP) with 100 and 30 neurons on each hidden layer, and PreactResNet-18~\cite{he2016cvecvv} as a Convolutional Neural Network (CNN) architecture.
We focus on binary classification tasks and leave the extension to multiclass scenarios for future work. However, for datasets that are inherently multiclass, we transform the problem into a binary classification task by selecting two classes, aligning with our assumption.

\smallskip
\noindent\textbf{\textit{Evaluation Measures.}} To characterize the degree of overfitting, we use the test loss, as it serves as a reliable indicator of the model's generalization capability to unseen data. Additionally, we evaluate the predictive performance of each model using the test accuracy.

\smallskip
\noindent\textbf{\textit{Baselines.}} We compare CF-Reg with the following regularization techniques: L1 (``Lasso''), L2 (``Ridge''), and Dropout.

\smallskip
\noindent\textbf{\textit{Configurations.}}
For each model, we adopt specific configurations as follows.
\begin{itemize}
\item \textit{Logistic Regression:} To induce overfitting in the model, we artificially increase the dimensionality of the data beyond the number of training samples by applying a polynomial feature expansion. This approach ensures that the model has enough capacity to overfit the training data, allowing us to analyze the impact of our counterfactual regularizer. The degree of the polynomial is chosen as the smallest degree that makes the number of features greater than the number of data.
\item \textit{Neural Networks (MLP and CNN):} To take advantage of the closed-form solution for computing the optimal perturbation vector as defined in (\ref{eq:opt-delta}), we use a local linear approximation of the neural network models. Hence, given an instance $\inst_i$, we consider the (optimal) counterfactual not with respect to $\model$ but with respect to:
\begin{equation}
\label{eq:taylor}
    \model^{lin}(\inst) = \model(\inst_i) + \nabla_{\inst}\model(\inst_i)(\inst - \inst_i),
\end{equation}
where $\model^{lin}$ represents the first-order Taylor approximation of $\model$ at $\inst_i$.
Note that this step is unnecessary for Logistic Regression, as it is inherently a linear model.
\end{itemize}

\smallskip
\noindent \textbf{\textit{Implementation Details.}} We run all experiments on a machine equipped with an AMD Ryzen 9 7900 12-Core Processor and an NVIDIA GeForce RTX 4090 GPU. Our implementation is based on the PyTorch Lightning framework. We use stochastic gradient descent as the optimizer with a learning rate of $\eta = 0.001$ and no weight decay. We use a batch size of $128$. The training and test steps are conducted for $6000$ epochs on the \textit{Water Potability} and \textit{Phoneme} datasets, while for the \textit{CIFAR-10} dataset, they are performed for $200$ epochs.
Finally, the contribution $w_i^{\varepsilon}$ of each training point $\inst_i$ is uniformly set as $w_i^{\varepsilon} = 1~\forall i\in \{1,\ldots,m\}$.

The source code implementation for our experiments is available at the following GitHub repository: \url{https://anonymous.4open.science/r/COCE-80B4/README.md} 

\subsection{RQ1: Counterfactual Perturbation vs. Overfitting}
To address \textbf{RQ1}, we analyze the relationship between the test loss and the average $L_2$-norm of the counterfactual perturbation vectors ($\overline{||\perturb||}$) over training epochs.

In particular, Figure~\ref{fig:delta_loss_epochs} depicts the evolution of $\overline{||\perturb||}$ alongside the test loss for an MLP trained \textit{without} regularization on the \textit{Water Potability} dataset. 
\begin{figure}[ht]
    \centering
    \includegraphics[width=0.85\linewidth]{img/delta_loss_epochs.png}
    \caption{The average counterfactual perturbation vector $\overline{||\perturb||}$ (left $y$-axis) and the cross-entropy test loss (right $y$-axis) over training epochs ($x$-axis) for an MLP trained on the \textit{Water Potability} dataset \textit{without} regularization.}
    \label{fig:delta_loss_epochs}
\end{figure}

The plot shows a clear trend as the model starts to overfit the data (evidenced by an increase in test loss). 
Notably, $\overline{||\perturb||}$ begins to decrease, which aligns with the hypothesis that the average distance to the optimal counterfactual example gets smaller as the model's decision boundary becomes increasingly adherent to the training data.

It is worth noting that this trend is heavily influenced by the choice of the counterfactual generator model. In particular, the relationship between $\overline{||\perturb||}$ and the degree of overfitting may become even more pronounced when leveraging more accurate counterfactual generators. However, these models often come at the cost of higher computational complexity, and their exploration is left to future work.

Nonetheless, we expect that $\overline{||\perturb||}$ will eventually stabilize at a plateau, as the average $L_2$-norm of the optimal counterfactual perturbations cannot vanish to zero.

% Additionally, the choice of employing the score-based counterfactual explanation framework to generate counterfactuals was driven to promote computational efficiency.

% Future enhancements to the framework may involve adopting models capable of generating more precise counterfactuals. While such approaches may yield to performance improvements, they are likely to come at the cost of increased computational complexity.


\subsection{RQ2: Counterfactual Regularization Performance}
To answer \textbf{RQ2}, we evaluate the effectiveness of the proposed counterfactual regularization (CF-Reg) by comparing its performance against existing baselines: unregularized training loss (No-Reg), L1 regularization (L1-Reg), L2 regularization (L2-Reg), and Dropout.
Specifically, for each model and dataset combination, Table~\ref{tab:regularization_comparison} presents the mean value and standard deviation of test accuracy achieved by each method across 5 random initialization. 

The table illustrates that our regularization technique consistently delivers better results than existing methods across all evaluated scenarios, except for one case -- i.e., Logistic Regression on the \textit{Phomene} dataset. 
However, this setting exhibits an unusual pattern, as the highest model accuracy is achieved without any regularization. Even in this case, CF-Reg still surpasses other regularization baselines.

From the results above, we derive the following key insights. First, CF-Reg proves to be effective across various model types, ranging from simple linear models (Logistic Regression) to deep architectures like MLPs and CNNs, and across diverse datasets, including both tabular and image data. 
Second, CF-Reg's strong performance on the \textit{Water} dataset with Logistic Regression suggests that its benefits may be more pronounced when applied to simpler models. However, the unexpected outcome on the \textit{Phoneme} dataset calls for further investigation into this phenomenon.


\begin{table*}[h!]
    \centering
    \caption{Mean value and standard deviation of test accuracy across 5 random initializations for different model, dataset, and regularization method. The best results are highlighted in \textbf{bold}.}
    \label{tab:regularization_comparison}
    \begin{tabular}{|c|c|c|c|c|c|c|}
        \hline
        \textbf{Model} & \textbf{Dataset} & \textbf{No-Reg} & \textbf{L1-Reg} & \textbf{L2-Reg} & \textbf{Dropout} & \textbf{CF-Reg (ours)} \\ \hline
        Logistic Regression   & \textit{Water}   & $0.6595 \pm 0.0038$   & $0.6729 \pm 0.0056$   & $0.6756 \pm 0.0046$  & N/A    & $\mathbf{0.6918 \pm 0.0036}$                     \\ \hline
        MLP   & \textit{Water}   & $0.6756 \pm 0.0042$   & $0.6790 \pm 0.0058$   & $0.6790 \pm 0.0023$  & $0.6750 \pm 0.0036$    & $\mathbf{0.6802 \pm 0.0046}$                    \\ \hline
%        MLP   & \textit{Adult}   & $0.8404 \pm 0.0010$   & $\mathbf{0.8495 \pm 0.0007}$   & $0.8489 \pm 0.0014$  & $\mathbf{0.8495 \pm 0.0016}$     & $0.8449 \pm 0.0019$                    \\ \hline
        Logistic Regression   & \textit{Phomene}   & $\mathbf{0.8148 \pm 0.0020}$   & $0.8041 \pm 0.0028$   & $0.7835 \pm 0.0176$  & N/A    & $0.8098 \pm 0.0055$                     \\ \hline
        MLP   & \textit{Phomene}   & $0.8677 \pm 0.0033$   & $0.8374 \pm 0.0080$   & $0.8673 \pm 0.0045$  & $0.8672 \pm 0.0042$     & $\mathbf{0.8718 \pm 0.0040}$                    \\ \hline
        CNN   & \textit{CIFAR-10} & $0.6670 \pm 0.0233$   & $0.6229 \pm 0.0850$   & $0.7348 \pm 0.0365$   & N/A    & $\mathbf{0.7427 \pm 0.0571}$                     \\ \hline
    \end{tabular}
\end{table*}

\begin{table*}[htb!]
    \centering
    \caption{Hyperparameter configurations utilized for the generation of Table \ref{tab:regularization_comparison}. For our regularization the hyperparameters are reported as $\mathbf{\alpha/\beta}$.}
    \label{tab:performance_parameters}
    \begin{tabular}{|c|c|c|c|c|c|c|}
        \hline
        \textbf{Model} & \textbf{Dataset} & \textbf{No-Reg} & \textbf{L1-Reg} & \textbf{L2-Reg} & \textbf{Dropout} & \textbf{CF-Reg (ours)} \\ \hline
        Logistic Regression   & \textit{Water}   & N/A   & $0.0093$   & $0.6927$  & N/A    & $0.3791/1.0355$                     \\ \hline
        MLP   & \textit{Water}   & N/A   & $0.0007$   & $0.0022$  & $0.0002$    & $0.2567/1.9775$                    \\ \hline
        Logistic Regression   &
        \textit{Phomene}   & N/A   & $0.0097$   & $0.7979$  & N/A    & $0.0571/1.8516$                     \\ \hline
        MLP   & \textit{Phomene}   & N/A   & $0.0007$   & $4.24\cdot10^{-5}$  & $0.0015$    & $0.0516/2.2700$                    \\ \hline
       % MLP   & \textit{Adult}   & N/A   & $0.0018$   & $0.0018$  & $0.0601$     & $0.0764/2.2068$                    \\ \hline
        CNN   & \textit{CIFAR-10} & N/A   & $0.0050$   & $0.0864$ & N/A    & $0.3018/
        2.1502$                     \\ \hline
    \end{tabular}
\end{table*}

\begin{table*}[htb!]
    \centering
    \caption{Mean value and standard deviation of training time across 5 different runs. The reported time (in seconds) corresponds to the generation of each entry in Table \ref{tab:regularization_comparison}. Times are }
    \label{tab:times}
    \begin{tabular}{|c|c|c|c|c|c|c|}
        \hline
        \textbf{Model} & \textbf{Dataset} & \textbf{No-Reg} & \textbf{L1-Reg} & \textbf{L2-Reg} & \textbf{Dropout} & \textbf{CF-Reg (ours)} \\ \hline
        Logistic Regression   & \textit{Water}   & $222.98 \pm 1.07$   & $239.94 \pm 2.59$   & $241.60 \pm 1.88$  & N/A    & $251.50 \pm 1.93$                     \\ \hline
        MLP   & \textit{Water}   & $225.71 \pm 3.85$   & $250.13 \pm 4.44$   & $255.78 \pm 2.38$  & $237.83 \pm 3.45$    & $266.48 \pm 3.46$                    \\ \hline
        Logistic Regression   & \textit{Phomene}   & $266.39 \pm 0.82$ & $367.52 \pm 6.85$   & $361.69 \pm 4.04$  & N/A   & $310.48 \pm 0.76$                    \\ \hline
        MLP   &
        \textit{Phomene} & $335.62 \pm 1.77$   & $390.86 \pm 2.11$   & $393.96 \pm 1.95$ & $363.51 \pm 5.07$    & $403.14 \pm 1.92$                     \\ \hline
       % MLP   & \textit{Adult}   & N/A   & $0.0018$   & $0.0018$  & $0.0601$     & $0.0764/2.2068$                    \\ \hline
        CNN   & \textit{CIFAR-10} & $370.09 \pm 0.18$   & $395.71 \pm 0.55$   & $401.38 \pm 0.16$ & N/A    & $1287.8 \pm 0.26$                     \\ \hline
    \end{tabular}
\end{table*}

\subsection{Feasibility of our Method}
A crucial requirement for any regularization technique is that it should impose minimal impact on the overall training process.
In this respect, CF-Reg introduces an overhead that depends on the time required to find the optimal counterfactual example for each training instance. 
As such, the more sophisticated the counterfactual generator model probed during training the higher would be the time required. However, a more advanced counterfactual generator might provide a more effective regularization. We discuss this trade-off in more details in Section~\ref{sec:discussion}.

Table~\ref{tab:times} presents the average training time ($\pm$ standard deviation) for each model and dataset combination listed in Table~\ref{tab:regularization_comparison}.
We can observe that the higher accuracy achieved by CF-Reg using the score-based counterfactual generator comes with only minimal overhead. However, when applied to deep neural networks with many hidden layers, such as \textit{PreactResNet-18}, the forward derivative computation required for the linearization of the network introduces a more noticeable computational cost, explaining the longer training times in the table.

\subsection{Hyperparameter Sensitivity Analysis}
The proposed counterfactual regularization technique relies on two key hyperparameters: $\alpha$ and $\beta$. The former is intrinsic to the loss formulation defined in (\ref{eq:cf-train}), while the latter is closely tied to the choice of the score-based counterfactual explanation method used.

Figure~\ref{fig:test_alpha_beta} illustrates how the test accuracy of an MLP trained on the \textit{Water Potability} dataset changes for different combinations of $\alpha$ and $\beta$.

\begin{figure}[ht]
    \centering
    \includegraphics[width=0.85\linewidth]{img/test_acc_alpha_beta.png}
    \caption{The test accuracy of an MLP trained on the \textit{Water Potability} dataset, evaluated while varying the weight of our counterfactual regularizer ($\alpha$) for different values of $\beta$.}
    \label{fig:test_alpha_beta}
\end{figure}

We observe that, for a fixed $\beta$, increasing the weight of our counterfactual regularizer ($\alpha$) can slightly improve test accuracy until a sudden drop is noticed for $\alpha > 0.1$.
This behavior was expected, as the impact of our penalty, like any regularization term, can be disruptive if not properly controlled.

Moreover, this finding further demonstrates that our regularization method, CF-Reg, is inherently data-driven. Therefore, it requires specific fine-tuning based on the combination of the model and dataset at hand.
\section{Conclusion}
In this work, we propose a simple yet effective approach, called SMILE, for graph few-shot learning with fewer tasks. Specifically, we introduce a novel dual-level mixup strategy, including within-task and across-task mixup, for enriching the diversity of nodes within each task and the diversity of tasks. Also, we incorporate the degree-based prior information to learn expressive node embeddings. Theoretically, we prove that SMILE effectively enhances the model's generalization performance. Empirically, we conduct extensive experiments on multiple benchmarks and the results suggest that SMILE significantly outperforms other baselines, including both in-domain and cross-domain few-shot settings.
% \documentclass{article}

% Recommended, but optional, packages for figures and better typesetting:
\usepackage{graphicx}
\usepackage{subfigure}
\usepackage{booktabs} % for professional tables
\usepackage[a4paper,top=3cm,bottom=2cm,left=2.5cm,right=2.5cm,marginparwidth=1.75cm]{geometry}
\usepackage{amsmath, amssymb, natbib, graphicx, url,dsfont,datetime,cases,mathtools,amsthm}
\usepackage{thmtools,thm-restate}
\usepackage{hyperref}
\usepackage{authblk}


\usepackage{amsmath}
\usepackage{amssymb}
\usepackage{mathtools}
\usepackage{amsthm}
\usepackage{enumitem}
\usepackage{macros}
\usepackage{tikz}
\usepackage{booktabs}
\usepackage{subfigure,algorithm2e,algorithmic}

% if you use cleveref..
\usepackage[capitalize,noabbrev]{cleveref}

%%%%%%%%%%%%%%%%%%%%%%%%%%%%%%%%
% THEOREMS
%%%%%%%%%%%%%%%%%%%%%%%%%%%%%%%%
\theoremstyle{plain}
\newtheorem{theorem}{Theorem}[section]
\newtheorem{proposition}[theorem]{Proposition}
\newtheorem{lemma}[theorem]{Lemma}
\newtheorem{corollary}[theorem]{Corollary}
\theoremstyle{definition}
\newtheorem{definition}[theorem]{Definition}
\newtheorem{assumption}[theorem]{Assumption}
\theoremstyle{remark}
\newtheorem{remark}[theorem]{Remark}

\title{The Batch Complexity of Bandit Pure Exploration}
\date{}
\author[1]{Adrienne Tuynman}
\author[1]{Rémy Degenne}
\affil[1]{Univ. Lille, Inria, CNRS, Centrale Lille, UMR 9189-CRIStAL, F-59000 Lille, France}
\begin{document}
	\maketitle

\begin{abstract}
In a fixed-confidence pure exploration problem in stochastic multi-armed bandits, an algorithm iteratively samples arms and should stop as early as possible and return the correct answer to a query about the arms distributions.
We are interested in batched methods, which change their sampling behaviour only a few times, between batches of observations.
We give an instance-dependent lower bound on the number of batches used by any sample efficient algorithm for any pure exploration task.
We then give a general batched algorithm and prove upper bounds on its expected sample complexity and batch complexity.
We illustrate both lower and upper bounds on best-arm identification and thresholding bandits.
\end{abstract}
\section{Introduction}
\label{sec:introduction}
The business processes of organizations are experiencing ever-increasing complexity due to the large amount of data, high number of users, and high-tech devices involved \cite{martin2021pmopportunitieschallenges, beerepoot2023biggestbpmproblems}. This complexity may cause business processes to deviate from normal control flow due to unforeseen and disruptive anomalies \cite{adams2023proceddsriftdetection}. These control-flow anomalies manifest as unknown, skipped, and wrongly-ordered activities in the traces of event logs monitored from the execution of business processes \cite{ko2023adsystematicreview}. For the sake of clarity, let us consider an illustrative example of such anomalies. Figure \ref{FP_ANOMALIES} shows a so-called event log footprint, which captures the control flow relations of four activities of a hypothetical event log. In particular, this footprint captures the control-flow relations between activities \texttt{a}, \texttt{b}, \texttt{c} and \texttt{d}. These are the causal ($\rightarrow$) relation, concurrent ($\parallel$) relation, and other ($\#$) relations such as exclusivity or non-local dependency \cite{aalst2022pmhandbook}. In addition, on the right are six traces, of which five exhibit skipped, wrongly-ordered and unknown control-flow anomalies. For example, $\langle$\texttt{a b d}$\rangle$ has a skipped activity, which is \texttt{c}. Because of this skipped activity, the control-flow relation \texttt{b}$\,\#\,$\texttt{d} is violated, since \texttt{d} directly follows \texttt{b} in the anomalous trace.
\begin{figure}[!t]
\centering
\includegraphics[width=0.9\columnwidth]{images/FP_ANOMALIES.png}
\caption{An example event log footprint with six traces, of which five exhibit control-flow anomalies.}
\label{FP_ANOMALIES}
\end{figure}

\subsection{Control-flow anomaly detection}
Control-flow anomaly detection techniques aim to characterize the normal control flow from event logs and verify whether these deviations occur in new event logs \cite{ko2023adsystematicreview}. To develop control-flow anomaly detection techniques, \revision{process mining} has seen widespread adoption owing to process discovery and \revision{conformance checking}. On the one hand, process discovery is a set of algorithms that encode control-flow relations as a set of model elements and constraints according to a given modeling formalism \cite{aalst2022pmhandbook}; hereafter, we refer to the Petri net, a widespread modeling formalism. On the other hand, \revision{conformance checking} is an explainable set of algorithms that allows linking any deviations with the reference Petri net and providing the fitness measure, namely a measure of how much the Petri net fits the new event log \cite{aalst2022pmhandbook}. Many control-flow anomaly detection techniques based on \revision{conformance checking} (hereafter, \revision{conformance checking}-based techniques) use the fitness measure to determine whether an event log is anomalous \cite{bezerra2009pmad, bezerra2013adlogspais, myers2018icsadpm, pecchia2020applicationfailuresanalysispm}. 

The scientific literature also includes many \revision{conformance checking}-independent techniques for control-flow anomaly detection that combine specific types of trace encodings with machine/deep learning \cite{ko2023adsystematicreview, tavares2023pmtraceencoding}. Whereas these techniques are very effective, their explainability is challenging due to both the type of trace encoding employed and the machine/deep learning model used \cite{rawal2022trustworthyaiadvances,li2023explainablead}. Hence, in the following, we focus on the shortcomings of \revision{conformance checking}-based techniques to investigate whether it is possible to support the development of competitive control-flow anomaly detection techniques while maintaining the explainable nature of \revision{conformance checking}.
\begin{figure}[!t]
\centering
\includegraphics[width=\columnwidth]{images/HIGH_LEVEL_VIEW.png}
\caption{A high-level view of the proposed framework for combining \revision{process mining}-based feature extraction with dimensionality reduction for control-flow anomaly detection.}
\label{HIGH_LEVEL_VIEW}
\end{figure}

\subsection{Shortcomings of \revision{conformance checking}-based techniques}
Unfortunately, the detection effectiveness of \revision{conformance checking}-based techniques is affected by noisy data and low-quality Petri nets, which may be due to human errors in the modeling process or representational bias of process discovery algorithms \cite{bezerra2013adlogspais, pecchia2020applicationfailuresanalysispm, aalst2016pm}. Specifically, on the one hand, noisy data may introduce infrequent and deceptive control-flow relations that may result in inconsistent fitness measures, whereas, on the other hand, checking event logs against a low-quality Petri net could lead to an unreliable distribution of fitness measures. Nonetheless, such Petri nets can still be used as references to obtain insightful information for \revision{process mining}-based feature extraction, supporting the development of competitive and explainable \revision{conformance checking}-based techniques for control-flow anomaly detection despite the problems above. For example, a few works outline that token-based \revision{conformance checking} can be used for \revision{process mining}-based feature extraction to build tabular data and develop effective \revision{conformance checking}-based techniques for control-flow anomaly detection \cite{singh2022lapmsh, debenedictis2023dtadiiot}. However, to the best of our knowledge, the scientific literature lacks a structured proposal for \revision{process mining}-based feature extraction using the state-of-the-art \revision{conformance checking} variant, namely alignment-based \revision{conformance checking}.

\subsection{Contributions}
We propose a novel \revision{process mining}-based feature extraction approach with alignment-based \revision{conformance checking}. This variant aligns the deviating control flow with a reference Petri net; the resulting alignment can be inspected to extract additional statistics such as the number of times a given activity caused mismatches \cite{aalst2022pmhandbook}. We integrate this approach into a flexible and explainable framework for developing techniques for control-flow anomaly detection. The framework combines \revision{process mining}-based feature extraction and dimensionality reduction to handle high-dimensional feature sets, achieve detection effectiveness, and support explainability. Notably, in addition to our proposed \revision{process mining}-based feature extraction approach, the framework allows employing other approaches, enabling a fair comparison of multiple \revision{conformance checking}-based and \revision{conformance checking}-independent techniques for control-flow anomaly detection. Figure \ref{HIGH_LEVEL_VIEW} shows a high-level view of the framework. Business processes are monitored, and event logs obtained from the database of information systems. Subsequently, \revision{process mining}-based feature extraction is applied to these event logs and tabular data input to dimensionality reduction to identify control-flow anomalies. We apply several \revision{conformance checking}-based and \revision{conformance checking}-independent framework techniques to publicly available datasets, simulated data of a case study from railways, and real-world data of a case study from healthcare. We show that the framework techniques implementing our approach outperform the baseline \revision{conformance checking}-based techniques while maintaining the explainable nature of \revision{conformance checking}.

In summary, the contributions of this paper are as follows.
\begin{itemize}
    \item{
        A novel \revision{process mining}-based feature extraction approach to support the development of competitive and explainable \revision{conformance checking}-based techniques for control-flow anomaly detection.
    }
    \item{
        A flexible and explainable framework for developing techniques for control-flow anomaly detection using \revision{process mining}-based feature extraction and dimensionality reduction.
    }
    \item{
        Application to synthetic and real-world datasets of several \revision{conformance checking}-based and \revision{conformance checking}-independent framework techniques, evaluating their detection effectiveness and explainability.
    }
\end{itemize}

The rest of the paper is organized as follows.
\begin{itemize}
    \item Section \ref{sec:related_work} reviews the existing techniques for control-flow anomaly detection, categorizing them into \revision{conformance checking}-based and \revision{conformance checking}-independent techniques.
    \item Section \ref{sec:abccfe} provides the preliminaries of \revision{process mining} to establish the notation used throughout the paper, and delves into the details of the proposed \revision{process mining}-based feature extraction approach with alignment-based \revision{conformance checking}.
    \item Section \ref{sec:framework} describes the framework for developing \revision{conformance checking}-based and \revision{conformance checking}-independent techniques for control-flow anomaly detection that combine \revision{process mining}-based feature extraction and dimensionality reduction.
    \item Section \ref{sec:evaluation} presents the experiments conducted with multiple framework and baseline techniques using data from publicly available datasets and case studies.
    \item Section \ref{sec:conclusions} draws the conclusions and presents future work.
\end{itemize}
\section{Lower Bound}
This section presents our main lower bound. As stated above, the lower bound is nearly tight, apart from lower-order terms and the dependency on $\eps$.

\begin{theorem}
\label{thm:lb}
There exists a distribution $\calP$ and a loss function $f$  satisfying Assumption~\ref{assum:lispchitz_smooth} and Assumption~\ref{assump:dia_dominant}, such that for any $(\epsilon,\delta)$-User-level-DP algorithm $\calM$, given i.i.d. dataset $\calD$ drawn from $\calP$, the output of $\calM$ satisfies
\begin{align*}
    \E[F(\calM(\calD))-F(x^*)]\ge GD\cdot \Tilde{\Omega}\Big(\min\Big\{d,\frac{d}{\sqrt{mn}}+\frac{d^{3/2}}{n\epsilon\sqrt{m}}\Big\}\Big).
\end{align*}
%where $F(x):=\E_{z\sim \calP}f(x;z)$ and $x^*=\arg\min_{x\in\calX}f(x)$.
\end{theorem}

The non-private term $GD\frac{d}{\sqrt{mn}}$ represents the information-theoretic lower bound for SCO under these assumptions (see, e.g., Theorem 1 in \cite{agarwal2009information}).  


We construct the hard instance as follows:
let $\calX=[-1,1]^d$ be unit $\ell_\infty$-ball and let $f(x;z)=-\langle x,z\rangle$ for any $x\in \calX$ be the linear function.
Let $z\in[-\sqrt{m},\sqrt{m}]^d$  with $\E_{z\sim\calP}[z]=\mu$.
Then one can easily verify that $f$ satisfies Assumptions~\ref{assum:lispchitz_smooth} and~\ref{assump:dia_dominant} with $G=\sqrt{m},D=1$ and $\beta=0$.
We have
\begin{align}
    F(\calM(\calD))-F(x^*) &= \sum_{i=1}^{d} (\sign(\mu[i])-\calM(\calD)[i])\cdot\mu[i]
   \nonumber \\ &\ge \sum_{i=1}^{d}  |\mu[i]|.\ind\big(\sign(\mu[i])\neq\sign(\calM(\calD)[i])\big). \label{eq:opt_error_to_sign_error}
\end{align}

By~\eqref{eq:opt_error_to_sign_error}, we reduce the optimization error to the weighted sign estimation error.  
Most existing lower bounds rely on the $\ell^2_2$-error of mean estimation.  
We adapt their techniques, especially the fingerprinting lemma, and provide the proof in the Appendix~\ref{sec:lbproof}.
\begin{algorithm}[ht!]
\caption{\textit{NovelSelect}}
\label{alg:novelselect}
\begin{algorithmic}[1]
\State \textbf{Input:} Data pool $\mathcal{X}^{all}$, data budget $n$
\State Initialize an empty dataset, $\mathcal{X} \gets \emptyset$
\While{$|\mathcal{X}| < n$}
    \State $x^{new} \gets \arg\max_{x \in \mathcal{X}^{all}} v(x)$
    \State $\mathcal{X} \gets \mathcal{X} \cup \{x^{new}\}$
    \State $\mathcal{X}^{all} \gets \mathcal{X}^{all} \setminus \{x^{new}\}$
\EndWhile
\State \textbf{return} $\mathcal{X}$
\end{algorithmic}
\end{algorithm}

% !TeX root = ../all.tex
\begin{figure*}[!ht]
	\centering
	\subfigure[Number of samples before stopping in a random BAI instance, logarithmic scale]{\includegraphics[width=0.3\textwidth]{plot_samp.png}\label{fig:exp}
	}\hspace{1em}
	\subfigure[Number of rounds before stopping in a random BAI instance]{\includegraphics[width=0.3\textwidth]{plot_round.png}
		\label{fig:expr}} \hspace{1em}
	\subfigure[Number of samples before stopping in the min. threshold setting, hard instance]{\includegraphics[width=0.3\textwidth]{TaSbad.png}
		
		\label{fig:exptas}}\label{fig:experiments}\caption{Experimental results, $\delta=0.05$, $N=1000$ runs}\end{figure*}
	
\subsection{Experiments on the BAI setting}


	
	
	

Our algorithm PET is near-optimal in round and sample complexities for many pure exploration problems, and has theoretical guarantees for any pure exploration problem. To ascertain its practical performances, we compare it to baselines and state of the art algorithms for best arm identification and thresholding bandits.	

Each experiment is repeated over 1000 runs. All reward distributions are Gaussian with variance 1 and we use the confidence level $\delta = 0.05$, which is chosen for its relevance to statistical practice. We compare
\begin{itemize}[noitemsep]
	\item Round Robin (or uniform sampling), where the stopping rule is checked only at timesteps $(900\times2^r)_{r \ge 1}$;
	\item Track-and-Stop (TaS) \citep{garivierOptimalBestArm2016}, where the empirical value of $w$ is updated only at timesteps $(900\times 2^r)_r$, and the stopping rule is only checked at those times;
	\item Our algorithm PET, with $T_0 = 1$;
	\item Opt-BBAI \citep{jinOptimalBatchedBest2023} with $\alpha = 1.05$ and the quantities described in their Theorem 4.2.
\end{itemize}
The initial batch sizes for TaS and Round Robin were chosen to approximate the initial batch size of our algorithm, to not disadvantage them in terms of round complexity. We modified TaS in order to turn it into a batch algorithm. Note that there is no formal guarantee for the batch or sample complexity of that modification of TaS, but we use it as a sensible baseline. 

For the BAI experiment, we run each algorithm on $10$-arm instances where the best arm has mean $1$, and each other arm $i$ has mean uniformly sampled between $0.6$ and $0.9$.
See Figure~\ref{fig:exp} for the box plots of the sample complexities. The mean is indicated by a black cross.
While both our algorithm and Opt-BBAI use similarly few batches, PET outperforms Opt-BBAI for the sample complexity.
That algorithm is asymptotically optimal as $\delta\rightarrow 0$ but it uses batches that seem to be too large for moderate values of $\delta$ like the $0.05$ we use.

While the batch modification of TaS might seem to be a good alternative for the BAI experiment, there are instances of the thresholding setting where it performs sub-optimally.
That effect that was first observed in \citep{degenneNonAsymptoticPureExploration2019} for the fully online TaS and reflects that, contrary to our results, the sample complexity guarantees of TaS are only asymptotic. 
We run the algorithms on a thresholding bandit with threshold 0.6 and two arms with means 0.5 and 0.6 and observe that batched TaS has high average sample complexity (see Figure~\ref{fig:exptas}; the mean is the black cross), while PET does not.




\section{Conclusion}
In this work, we propose a simple yet effective approach, called SMILE, for graph few-shot learning with fewer tasks. Specifically, we introduce a novel dual-level mixup strategy, including within-task and across-task mixup, for enriching the diversity of nodes within each task and the diversity of tasks. Also, we incorporate the degree-based prior information to learn expressive node embeddings. Theoretically, we prove that SMILE effectively enhances the model's generalization performance. Empirically, we conduct extensive experiments on multiple benchmarks and the results suggest that SMILE significantly outperforms other baselines, including both in-domain and cross-domain few-shot settings.

\section*{Acknowledgements}
	The authors acknowledge the funding of the French National Research Agency under the project FATE (ANR22-CE23-0016-01) and the PEPR IA FOUNDRY project (ANR-23-PEIA-0003). 
	The authors are members of the Inria team Scool.


\bibliographystyle{apalike}
\bibliography{bibli}


%%%%%%%%%%%%%%%%%%%%%%%%%%%%%%%%%%%%%%%%%%%%%%%%%%%%%%%%%%%%%%%%%%%%%%%%%%%%%%%
%%%%%%%%%%%%%%%%%%%%%%%%%%%%%%%%%%%%%%%%%%%%%%%%%%%%%%%%%%%%%%%%%%%%%%%%%%%%%%%
% APPENDIX
%%%%%%%%%%%%%%%%%%%%%%%%%%%%%%%%%%%%%%%%%%%%%%%%%%%%%%%%%%%%%%%%%%%%%%%%%%%%%%%
%%%%%%%%%%%%%%%%%%%%%%%%%%%%%%%%%%%%%%%%%%%%%%%%%%%%%%%%%%%%%%%%%%%%%%%%%%%%%%%
\newpage
\appendix
\onecolumn
% !TeX root = ../all.tex

\section{Proofs of the lower bounds}\label{app:lb}
\subsection{Preliminary lemmas}

For the sake of completeness, we start by restating and proving some results from \citep{taoCollaborativeLearningLimited2019} in slightly more general language.
\begin{definition}
	For some integers $r$ and $n$, define $\tau_\delta^r$ the number of samples before the end of round $r$.
\end{definition}
\begin{lemma}[Generalization of Lemma 27 of \citep{taoCollaborativeLearningLimited2019}]\label{lem:27f}
	For an algorithm, two instances ${\bm\nu}$ and ${\bm\nu}'$ and $r\in\bN$, \[\bP_{{\bm\nu}'}\{R_\delta\geq r+1,\tau_\delta^{r+1} \leq n+m\}\geq \bP_{\bm\nu}\{R_\delta \geq r+1,\tau_\delta^r\leq m\}-\bP_{\bm\nu} \{\tau_\delta>n\} -\Vert\cD_{\bm\nu}^m -\cD_{{\bm\nu}'}^m\Vert_{TV}\] where $\cD^m_{\bm\nu}$ is the distribution of rewards the algorithm got from $\bm\nu$ over $m$ steps.
\end{lemma}

\begin{proof}
	Fix a deterministic algorithm. 
	
	
	First of all, \begin{equation}\label{eq:mpntomn}(R_\delta \geq r+1,\tau_\delta^{r} \leq m,\tau_\delta^{r+1}-\tau_\delta^r < n) \subseteq (R_\delta \geq r+1,\tau_\delta^{r+1}\leq n+m)\end{equation}
	
	And, since $(R_\delta \geq r+1,\tau_\delta^{r} \leq m,\tau_\delta^{r+1}-\tau_\delta^r < n)$ is determined by the first $m$ rewards (at the end of round $r$ using less than $m$ samples, the algorithm must choose the length of round $r+1$), \begin{equation}\label{eq:distprob} \bP_{{\bm\nu}'}\{R_\delta \geq r+1,\tau_\delta^{r} \leq m,\tau_\delta^{r+1}-\tau_\delta^r < n\} \geq \bP_{\bm\nu} \{R_\delta \geq r+1,\tau_\delta^{r} \leq m,\tau_\delta^{r+1}-\tau_\delta^r < n\}  -\Vert\cD_{\bm\nu}^m -\cD_{{\bm\nu}'}^m\Vert_{TV}\end{equation}

On the other hand, \begin{align*}
	(R_\delta \geq r+1,\tau_\delta^r\leq m)\setminus (R_\delta\geq r+1,\tau_\delta^r\leq m,\tau_\delta^{r+1}-\tau_\delta^r < n) &= (R_\delta\geq r+1,\tau_\delta^r\leq m, \tau_\delta^{r+1}-\tau_\delta^r \geq n) \\
	&\subseteq (\tau_\delta >n)
\end{align*} hence \begin{equation}\label{eq:ajoutround}\bP_{\bm\nu}\{R_\delta\geq r+1,\tau_\delta^r\leq m,\tau_\delta^{r+1}-\tau_\delta^r < n\}\geq \bP_{\bm\nu}\{R_\delta \geq r+1,\tau_\delta^r\leq m\}-\bP_{\bm\nu}\{\tau_\delta >n\}\end{equation}

	Hence, using Equations \eqref{eq:mpntomn},~\eqref{eq:distprob} then~\eqref{eq:ajoutround}, \begin{align*}
		\bP_{{\bm\nu}'}\{R_\delta \geq r+1,\tau_\delta^{r+1}\leq n+m\}&\geq \bP_{\bm\nu'} \{R_\delta \geq r+1,\tau_\delta^{r} \leq m,\tau_\delta^{r+1}-\tau_\delta^r < n\}\\
		&\geq \bP_{\bm\nu} \{R_\delta \geq r+1,\tau_\delta^{r} \leq m,\tau_\delta^{r+1}-\tau_\delta^r < n\} -\Vert\cD_{\bm\nu}^m -\cD_{{\bm\nu}'}^m\Vert_{TV} \\
		&\geq \bP_{\bm\nu}\{R_\delta \geq r+1,\tau_\delta^r\leq m\}-\bP_{\bm\nu} \{\tau_\delta>n\} -\Vert\cD_{\bm\nu}^m -\cD_{{\bm\nu}'}^m\Vert_{TV}
	\end{align*}
	
\end{proof}



\begin{lemma}[Generalization of Lemma 26 of \citep{taoCollaborativeLearningLimited2019}]\label{lem:26f_aux}
	For any $\delta$-correct algorithm, for all $m,r\in\bN$ and any two bandit instances ${\bm\nu}, {\bm\nu}'$, 
	we have
	\begin{align*}
	\bP_{\bm\nu}\{R_\delta\geq r+1,\tau_\delta^r\leq m\}
	\ge \bP_{\bm\nu}\{R_\delta\geq r,\tau_\delta^r\leq m\} - 2\delta - \Vert \cD_{\bm\nu}^m - \cD_{{\bm\nu}'}^m \Vert_{TV}
	\: .
	\end{align*}
\end{lemma}

\begin{proof}
Consider the event $\mathcal{F}_1=(R_\delta=r,\tau_\delta \leq m)$.
Denote by $\mathcal{F}_2$ the event that the algorithm returns the best arm of instance ${\bm\nu}$.
Then \(\bP_{\bm\nu}\{\mathcal{F}_1\}=\bP_{\bm\nu}\{\mathcal{F}_1\wedge \mathcal{F}_2\}+\bP_{\bm\nu}\{\mathcal{F}_1\wedge \overline{\mathcal{F}_2}\}\)
	
With $\mathcal{D}_{\bm\nu}^m$ the distribution of rewards over $m$ samples and some ${\bm\nu}'\in Alt_{\bm\nu}$,
\begin{align*}
\bP_{\bm\nu}\{\mathcal{F}_1\wedge \mathcal{F}_2\}
&\leq \bP_{{\bm\nu}'}\{\mathcal{F}_1\wedge \mathcal{F}_2\}+\Vert\cD_{\bm\nu}^m -\cD_{{\bm\nu}'}^m\Vert_{TV}
\\
&\leq \bP_{{\bm\nu}'}\{\mathcal{F}_2\} +\Vert\cD_{\bm\nu}^m -\cD_{{\bm\nu}'}^m\Vert_{TV}
\\
&\leq \delta+\Vert\cD_{\bm\nu}^m -\cD_{{\bm\nu}'}^m\Vert_{TV}
\: .
\end{align*}
On the other hand, $\mathbb{P}_{\bm\nu}\{\mathcal F_1 \wedge \overline{\mathcal F}_2\} \le \mathbb{P}_{\bm\nu}\{\overline{\mathcal F}_2\} \le \delta$.
Therefore $\bP_{\bm\nu}\{\mathcal{F}_1\}\leq 2\delta + \Vert\cD_{\bm\nu}^m -\cD_{{\bm\nu}'}^m\Vert_{TV}$.
Using \( \bP_{\bm\nu}\{R_\delta\geq r+1,\tau_\delta^r\leq m\}\geq \bP_{\bm\nu}\{R_\delta\geq r,\tau_\delta^r\leq m\}-\bP_{\bm\nu}(\mathcal{F}_1)\), we conclude.
\end{proof}


\begin{lemma}\label{lem:26f}
	For any $\delta$-correct algorithm, for all $m,r\in\bN$ and any bandit instance ${\bm\nu}$, 
	we have
	\begin{align*}
	\bP_{\bm\nu}\{R_\delta\geq r+1,\tau_\delta^r\leq m\}
	\ge \bP_{\bm\nu}\{R_\delta\geq r,\tau_\delta^r\leq m\} - 2\delta - \sqrt{\frac{m}{2} (T^\star(\bm\nu))^{-1}}
	\: .
	\end{align*}
\end{lemma}



\begin{proof}
First apply Lemma~\ref{lem:26f_aux} to an arbitrary instance ${\bm\nu}' \in Alt_{\bm\nu}$. Then using Pinsker's inequality yields
\begin{align*}
\Vert \cD_{\bm\nu}^m - \cD_{{\bm\nu}'}^m \Vert_{TV}
\leq \sqrt{\frac{1}{2}\KL(\cD_{\bm\nu}^m\Vert \cD_{{\bm\nu}'}^m)}
= \sqrt{\frac{1}{2} \sum_{i\in[K]} \bE_{\bm\nu}[N_{m,i}] \frac{(\mu_i-\mu_i')^2}{2}}
\end{align*} with $N_{m,i}$ the number of times arm $i$ is pulled before time $m$.

As this is true for all instances ${\bm\nu}'\in Alt_{{\bm\nu}}$, we can obtain an inequality using the infimum over those instances,
\begin{align*}
\inf_{{\bm\nu}' \in Alt_{\bm\nu}} \Vert \cD_{\bm\nu}^m - \cD_{{\bm\nu}'}^m \Vert_{TV}
&\le \sqrt{\frac{m}{2} \inf_{\bm\lambda \in Alt_{{\bm\nu}}} \sum_{i\in[K]} \frac{\bE_{\bm\nu}[N_{m,i}]}{m} \frac{(\mu_i-\lambda_i)^2}{2}}
\\
&\le \sqrt{\frac{m}{2} \sup_{w\in \Sigma_K} \inf_{\bm\lambda \in Alt_{{\bm\nu}}} \sum_{i\in[K]} w_i \frac{(\mu_i-\lambda_i)^2}{2}}
\\
&=   \sqrt{\frac{m}{2} (T^\star(\bm\nu))^{-1}}
\: ,
\end{align*}
by definition of $T^\star$.
\end{proof}





Finally, we also give a technical result to solve inequalities of the form $(k+N^2(a+b\ln N))^N\leq \rho$.

\begin{lemma}\label{lem:suffN}
	Let $\rho \ge e$, $a,b\geq 0$ and $k$ be real numbers, and let $A=\max\{e,k+a\}$.
	Then $N \coloneqq \left\lfloor \frac{\ln \rho}{\ln((\ln \rho)^2(A+b\ln \ln \rho))}\right\rfloor$ satisfies $(k+N^2(a+b\ln N))^N\leq \rho$~.
\end{lemma}

\begin{proof}
	If $N=0$, the equality is $1 \le \rho$, which is true since $\rho \ge e$. Otherwise, $N\geq 1$ and $(\ln \rho)^2(A+b\ln\ln \rho)\geq A\geq e$,
	so $N \le \lfloor \ln\rho / \ln e \rfloor \leq \ln \rho$. Therefore
	\begin{align*}
		N\ln(k+N^2(a+b\ln N))&\leq N\ln (N^2(A+b\ln N))
		\\
		&\leq N\ln((\ln \rho)^2(A+b\ln \ln \rho))
		\\
		&\leq \ln \rho
	\end{align*}
	and finally $(k+N^2(a+b\ln N))^N\leq \rho$~.
\end{proof}

\subsection{The lower bound in the general cases}

We give here a result for any sequence of instances.

\begin{restatable}[]{lemma}{lemrec}\label{lem:rec} 
	Let there be a sequence of instances $({\bm\nu}^n)_{0\leq n\leq N}$ such that the probability of error is bounded by $\delta$ and for any $n\in[0,N-1]$, $c_n \geq \bP_{{\bm\nu}^n} [\tau_\delta >x_n]$.
	Then \begin{align*} \bP_{{\bm\nu}^N}[R_\delta>N] &\geq 1-2N\delta -\sum_{i=0}^{N-1}\left[  c_n+ \sqrt{\frac{X_{n-1}}{2}} \left(\sqrt{ \frac{1}{T^\star(\bm\mu^n)} }  +\sqrt{\sum_{i\in[K]}\frac{\bE[N_{X_{n-1},i}]}{X_{n-1}} \frac{(\mu_i^{n+1}-\mu_i^n)^2}{2\sigma^2}}\right)\right]\end{align*} where $X_n=\sum_{i=-1}^n x_i$, $x_{-1}$ is any positive real number, and $N_{t,i}$ is the number of times arm $i$ is sampled before time $t$.
\end{restatable}

\begin{proof}[Proof of Lemma~\ref{lem:rec}] 
	By lemmas \ref{lem:27f} and \ref{lem:26f}, for any $m$, \begin{align*} 
		\bP_{{\bm\nu}^{n+1}}\{R_\delta\geq n+1,\tau_\delta^{n+1}\leq m+x_n\}  &\geq \bP_{{\bm\nu}^n}\{R_\delta\geq n+1,\tau_\delta^n\leq m\}-c_n-\Vert\cD_{{\bm\nu}^n}^m-\cD_{{\bm\nu}^{n+1}}^m\Vert_{TV}\\
		&\geq \bP_{{\bm\nu}^{n}}\{R_\delta \geq n,\tau_\delta^n\leq m\}-2\delta-\sqrt{\frac{m}{2} (T^\star(\bm\mu^n))^{-1}}\\
		&\hspace{1.5em}-c_n-\sqrt{\frac{1}{2}\sum_{i\in[K]}\bE[N_{m,i}] \frac{(\mu_i^{n+1}-\mu_i^n)^2}{2\sigma^2}}
	\end{align*} and with $X_n=\sum_{i=-1}^{n} x_i$, \begin{align*}\bP_{{\bm\nu}^{n+1}}\{R_\delta\geq n+1,\tau_\delta^{n+1}\leq X_n\} & \geq \bP_{{\bm\nu}^{n}}\{R_\delta\geq n,\tau_\delta^n\leq X_{n-1}\}-2\delta-c_n -\sqrt{\frac{X_{n-1}}{2} (T^\star(\bm\mu))^{-1}}\\
	&\hspace{1.5em} -\sqrt{\frac{X_{n-1}}{2}\sum_{i\in[K]}\frac{\bE[N_{X_{n-1},i}]}{X_{n-1}} \frac{(\mu_i^{n+1}-\mu_i^n)^2}{2\sigma^2}}\end{align*} So that finally \begin{align*}
	\bP_{{\bm\nu}^{N}}\{R_\delta\geq N,\tau_\delta^N\leq X_{N-1}\} &\geq \bP_{{\bm\nu}^{0}}\{R_\delta\geq 0,\tau_\delta^0\leq x_{-1}\}-2N\delta\\
	&\hspace{1.5em} -\sum_{i=0}^{N-1}\left[ c_n+ \sqrt{\frac{X_{n-1}}{2}}\left(\sqrt{ (T^\star(\bm\mu^n))^{-1} } +\sqrt{\sum_{i\in[K]}\frac{\bE[N_{X_{n-1},i}]}{X_{n-1}} \frac{(\mu_i^{n+1}-\mu_i^n)^2}{2\sigma^2}}\right)\right]\end{align*} and we conclude since for any $x_{-1}\geq 0$, $\bP_{{\bm\nu}^0}\{R_\delta\geq 1,\tau_\delta^0\leq x_{-1}\}=1$ (we always use at least 1 round).
\end{proof}

From there, we derive the result for $T^\star(\bm\mu^n)=\zeta^{-n} T^\star(\bm\mu^0)$.

\theorec* 


\begin{proof}[Proof of Lemma~\ref{th:theorec}]
	We apply Lemma~\ref{lem:rec} on the sequence $(\bm\nu^n)_{0\leq n\leq N}$ with $x_{-1}=\gamma T^\star(\bm\mu^0)\log(1/\delta)\frac{1}{\zeta^{-1}-1}$. That way, \begin{align*} X_{n} &= x_{-1} +\sum_{i=0}^n \gamma T^\star(\bm\mu^i)\log(1/\delta)\\
		&=\gamma T^\star(\bm\mu^0)\log(1/\delta)\left( \frac{1}{\zeta^{-1}-1}+\sum_{i=0}^n \zeta^{-i}\right)\\
		&=\gamma T^\star(\bm\mu^0)\log(1/\delta)\frac{\zeta^{-(n+1)}}{\zeta^{-1}-1}
	\end{align*}
	
\end{proof}

Under Assumption~\ref{asm:aff}, we can pick a sequence of instances of means $\bm\mu^{n+1}=x\bm\mu^n+(1-x)\bm y$ and control the sequence of $T^\star(\bm\mu^n)$. That way, we get the following result:
\begin{restatable}[Batch lower bound on affine sequences]{lemma}{lembar}\label{lem:bar}
	For problems on which Assumption~\ref{asm:aff} is satisfied;
	for any algorithm such that, for any Gaussian instance $\bm\nu$ satisfying $T^\star(\bm\mu)\in (T_{\min},T_{\max})$ the probability of error is smaller than $\delta$ and such that $\bP_{\bm\nu}(\tau_\delta>\gamma\log(1/\delta) T^\star(\bm\mu))\leq c$; we have for any $\sigma$-Gaussian instance $\bm\nu$ of complexity $T^\star(\bm\mu)\in (T_{\min},T_{\max})$, for the corresponding $y\in \R$ given by Assumption~\ref{asm:aff} for $\bm\mu$, that $\bP_{\bm\nu} (R_\delta\geq N)\geq 1/2$ for \[N = \min\left\{\frac{\ln \frac{T^\star(\bm\mu)}{T_{\min}}}{\ln\left( \left(\ln \frac{T^\star(\bm\mu)}{T_{\min}}\right)^2 \max\{e,C\} \right)},\frac{1}{2\delta+c}\right\}\] with $C=1+4\gamma\log(\frac{1}{\delta})\left(1+\sqrt{\frac{T^\star(\bm\mu)\Delta^2}{2\sigma^2}}\right)^2$ and $\Delta = \max_i |\mu_i -y|$.
\end{restatable}

\begin{proof}
	Fix some ($\sigma$-Gaussian) instance $\bm\nu^0=\bm\nu$ of complexity $T^\star(\bm\mu)=T_0\in(T_{\min},T_{\max})$. 
	
	For some $\zeta\in (0,1)$ to be fixed later, define the instance of mean $\bm\mu^{n+1}=\zeta^{-1/2}\bm\mu^n+(1-\zeta^{-1/2})\bm y$. We then have $\zeta^nT^\star(\bm\mu^0)= T^\star(\bm\mu^{n})$ by hypothesis. We can thus construct a sequence of instances of length $N$ as long as $\zeta^{N} > \frac{T_{\min}}{T^\star(\bm\mu^0)}$.
	
	\begin{align*} \frac{(\mu_i^{N-n-1}-\mu_i^{N-n})^2}{2\sigma^2} &= \frac{(\zeta^{-(N-n-1)/2}-\zeta^{-(N-n)/2})^2(\mu_i^0-y)^2}{2\sigma^2}\\ &\leq \frac{\zeta^{n-N}\Delta^2}{2\sigma^2} (1-\zeta^{1/2})^2\end{align*}
	We apply Theorem~\ref{th:theorec} on the reversed sequence $\left( \bm\nu_{N-i}\right)_{0\leq i\leq N}$:
	\begin{align*}\bP_{{\bm\nu}}(R_\delta>N) &\geq 1-N(2\delta+c)-\sqrt{\frac{\gamma\log(1/\delta)}{2(\zeta^{-1}-1)}}\times \sum_{i=0}^{N-1} \left[ 1 +\sqrt{T^\star(\bm\mu)\sup_{w\in \Delta_K}\sum_{i\in[K]}w_i \frac{\Delta^2}{2\sigma^2} (1-\zeta^{1/2})^2}\right]\\
		&\geq 1-N\left(2\delta+c+\sqrt{\frac{\gamma\log(1/\delta)}{2(\zeta^{-1}-1)}} \left(1+\sqrt{\frac{T^\star(\bm\mu^0)\Delta^2}{2\sigma^2}}\right)\right)\\
		&\geq 5/8-N(2\delta+c)\end{align*} for \[\zeta =\Bigg( 1+4N^2\gamma\log\left(\frac{1}{\delta}\right)\left(1+\sqrt{\frac{T^\star(\bm\mu^0)\Delta^2}{2\sigma^2}}\right)^2\Bigg)^{-1}\]
	
	
	We can apply Lemma~\ref{lem:suffN} with $\rho = \frac{T^\star(\bm\mu^0)}{T_{\min}}$, $k=1$, $b=0$, and $a=4\gamma\log(1/\delta)\left(1+\sqrt{\frac{T^\star(\bm\mu^0)\Delta^2}{2\sigma^2}}\right)^2$. We get that a sufficient condition is \begin{equation} \label{eq:Nbai} N\leq \frac{\ln \frac{T^\star(\bm\mu^0)}{T_{\min}}}{\ln\left( \left(\ln \frac{T^\star(\bm\mu^0)}{T_{\min}}\right)^2 \max\left\{e,C\right\} \right)}\end{equation} with $C=1+4\gamma\log(1/\delta)\left(1+\sqrt{\frac{T^\star(\bm\mu^0)\Delta^2}{2\sigma^2}}\right)^2$.
	
	Therefore, by picking $N$ that satisfies \eqref{eq:Nbai} and $N\leq \frac{1}{8(2\delta+c)}$, we have that $\bP_{\bm\nu}(R_\delta > N)>\frac{1}{2}$.
\end{proof}




\begin{restatable}[Batch lower bound on affine sequences, expectation constraint]{lemma}{lembarexp}\label{lem:barexp}
	For problems on which Assumption~\ref{asm:aff} is satisfied; 
	for any $\delta$-correct algorithm such that, for any Gaussian instance $\bm\nu$ satisfying $T^\star(\bm\mu)\in (T_{\min},T_{\max})$ $\bE_{\bm\nu}[\tau_\delta]\leq \gamma\log(1/\delta) T^\star(\bm\mu)$, we have for any $\sigma$-Gaussian instance $\bm\nu$ of complexity $T^\star(\bm\mu)\in (T_{\min},T_{\max})$, for the corresponding $y\in\R$ given by Assumption~\ref{asm:aff} for $\bm\mu$ that $\bP_{\bm\nu} (R_\delta\geq N)\geq 1/2$ for \[ N \geq \min\left\{ \frac{\ln \frac{T^\star(\bm\mu)}{T_{\min}}}{\ln\left( \left(\ln \frac{T^\star(\bm\mu)}{T_{\min}}\right)^2 \max\{e,C_\delta'\}  \right)},\frac{1}{3} \ln \frac{T^\star(\bm\mu)}{T_{\min}},\frac{1}{3\delta}\right\}\] with $C_\delta'=\max\left\{e,1+4\gamma\log(1/\delta)\ln \frac{T^\star(\bm\mu)}{T_{\min}}\left(1+\sqrt{\frac{T^\star(\bm\mu)\Delta^2}{2\sigma^2}}\right)^2 \right\}$ and $\Delta = \max_i |\mu_i -y|$.
\end{restatable}

\begin{proof}
For instance $\bm\nu$, for some algorithm satisfying $\bE_{\bm\nu}[\tau_\delta]\leq \gamma \log(1/\delta) T^\star(\bm\mu)$, we have by the Markov inequality that $\bP_{\bm\nu}(\tau_\delta \geq (\gamma/c) T^\star(\bm\mu) \log(1/\delta)) \leq c$.
Applying Lemma~\ref{lem:bar}, $\bP_{\bm\nu} (R_\delta\geq N)\geq 1/2$ for \[N = \min\left\{\frac{\ln \frac{T^\star(\bm\mu)}{T_{\min}}}{\ln\left( \left(\ln \frac{T^\star(\bm\mu)}{T_{\min}}\right)^2 \max\{e,C\} \right)},\frac{1}{2\delta+c}\right\}\] with $C=1+4\gamma/c\log(\frac{1}{\delta})\left(1+\sqrt{\frac{T^\star(\bm\mu)\Delta^2}{2\sigma^2}}\right)^2$ and $\Delta = \max_i |\mu_i -y|$.

Choosing $c = \max\left\{ \delta,\left( \log\frac{T^\star(\bm\mu)}{T_{\min}}\right)^{-1}\right\}$, if $\delta < \left( \log\frac{T^\star(\bm\mu)}{T_{\min}}\right)^{-1}$, then \[ N \geq \min\left\{ \frac{\ln \frac{T^\star(\bm\mu)}{T_{\min}}}{\ln\left( \left(\ln \frac{T^\star(\bm\mu)}{T_{\min}}\right)^2 \max\{e,C_\delta'\}  \right)},\frac{1}{3} \ln \frac{T^\star(\bm\mu)}{T_{\min}}\right\}\] with $C_\delta'=\max\left\{e,1+4\gamma\log(1/\delta)\ln \frac{T^\star(\bm\mu)}{T_{\min}}\left(1+\sqrt{\frac{T^\star(\bm\mu)\Delta^2}{2\sigma^2}}\right)^2 \right\}$.

If $\delta \geq \left( \log\frac{T^\star(\bm\mu)}{T_{\min}}\right)^{-1}$, \[N \geq \min\left\{\frac{\ln \frac{T^\star(\bm\mu)}{T_{\min}}}{\ln\left( \left(\ln \frac{T^\star(\bm\mu)}{T_{\min}}\right)^2 \max\{e,C_\delta'\} \right)},\frac{1}{3\delta}\right\}\]



\end{proof}











\subsection{The top-$k$ and BAI settings}

All that remains is to show that our problems satisfy Assumption~\ref{asm:aff}. We start with top-$k$, and first give a technical result giving a simple formula for $T^\star(\bm\mu)$.

\begin{lemma}\label{lem:baiw}
	For any $\bm w\in \Sigma_K$, \begin{align*}\inf_{\bm\lambda \in Alt_{\bm\mu}} \left( \sum_{i\in [K]} w_i d(\mu_i,\lambda_i)\right) = \min_{\substack{b\geq k+1 \\ a\leq k}} w_a d(\mu_a,\mu_{ab})+w_b d(\mu_b,\mu_{ab})\end{align*} where $\mu_{ab}=\frac{w_a\mu_a +w_b\mu_b}{w_a+w_b}$ (arms are assumed to be ordered, $\mu_1\geq\mu_2\geq \dots$).
\end{lemma}

\begin{lemma}\label{lem:topkgoodbar}
	In the top-$k$ problem, setting $\bm\mu' = x\bm\mu + (1-x)\bm y$ where $\bm y$ is a constant vector and $x>0$, $Alt_{\bm\mu'}=Alt_{\bm\mu}$, $\Delta_i^{\bm\mu'}=x\Delta_i^{\bm\mu}$ and $(T^\star(\bm\mu'))^{-1}=x^2(T^\star(\bm\mu))^{-1}$.
\end{lemma}
\begin{proof}
	First of all, for any two arms $i,j$, $\mu'_i-\mu'_j = x(\mu_i-\mu_j)$ with $x>0$. Therefore, the ordering of arms is conserved, and $Alt_{\bm\mu'}=Alt_{\bm\mu}$. Moreover, since $\Delta_i^{\bm\mu'} = \mu'_i - \mu'_{k+1} =x(\mu_i-\mu_{k+1})=x\Delta_i^{\bm\mu}$ for $i\leq k$ and $\Delta_i^{\bm\mu'} = \mu'_k-\mu'_i=x\Delta_i^{\bm\mu}$ otherwise, we do have $\Delta_i^{\bm\mu'}=x\Delta_i^{\bm\mu}$.
	
	Furthermore,
	\begin{align*}
		(T^\star(\bm\mu'))^{-1}&= \sup_{w\in \Sigma_K}\inf_{\bm\lambda \in Alt_{\bm\mu}} \left( \sum_{i\in [K]} w_i \frac{(\mu_i'-\lambda_i)^2}{2\sigma^2}\right)\\
		&= \sup_{w\in \Sigma_K} \min_{\substack{b\geq k+1 \\ a\leq k}} w_a \frac{(\mu_a'-\mu_{ab}')^2}{2\sigma^2}+w_b \frac{(\mu_b'-\mu_{ab}')^2}{2\sigma^2}
	\end{align*} with \begin{align*}\mu_{ab}'(w)&=\frac{w_a\mu_a' +w_b\mu_b'}{w_a+w_b}=x\mu_{ab}+(1-x)y\end{align*}
	So that \begin{align*}
		(T^\star(\bm\mu'))^{-1}&=x^2\sup_{w\in \Sigma_K}\min_{\substack{b\geq k+1 \\ a\leq k}} w_a \frac{(\mu_a-\mu_{ab})^2}{2\sigma^2}+w_b \frac{(\mu_b-\mu_{ab})^2}{2\sigma^2}\\
		&=x^2(T^\star(\bm\mu))^{-1}
	\end{align*}


\end{proof}

With these results, we can apply Lemmas~\ref{lem:bar} and \ref{lem:barexp}. We see that the value of $y$ does not impact the proof: we thus choose the value that minimizes $\max_i |\mu_i-y|$, which is $y = \frac{\max_i \mu_i+\min_i\mu_i}{2}$.

%\thlbbaib*
	

	
%\begin{proof}\textbf{of Theorem~\ref{th:lbbaib}}
%	Thanks to Lemma~\ref{lem:topkgoodbar}, it suffices to apply Lemma~\ref{lem:bar} with $\nu = \frac{\mu_1+\mu_K}{2}$. We have $\Delta = |\mu_1-\mu_K|/2$.
%\end{proof}	

\subsection{The thresholding setting}

\begin{lemma}\label{lem:tbpgoodbar}
	In the thresholding bandit problem, setting $\bm\mu' = x\bm\mu + (1-x)\bm \tau$ where $\bm \tau$ is the constant vector of value $\tau$ the threshold and $x>0$, $Alt_{\bm\mu'}=Alt_{\bm\mu}$, $\Delta_i^{\bm\mu'}=x\Delta_i^{\bm\mu}$ and $(T^\star(\bm\mu'))^{-1}=x^2(T^\star(\bm\mu))^{-1}$.
\end{lemma}

\begin{proof}
	First of all, for any arm $i$, $\mu'_i -\tau = x(\mu_i-\tau)$ with $x>0$. Therefore, $Alt_{\bm\mu}=Alt_{\bm\mu'}$. Moreover, $\Delta_i^{\bm\mu'}=|\mu'_i-\tau|=x|\mu_i-\tau|=x\Delta_i^{\bm\mu'}$.
	
	Furthermore, \begin{align*}
		(T^\star(\bm\mu'))^{-1}&=\sup_{w\in \Sigma_K} \inf_{\bm\lambda\in Alt_{\bm\mu}} \left( \sum_{i\in[K]} w_i\frac{(\mu'_i-\lambda_i)^2}{2\sigma^2}\right)\\
		&=\sup_{w\in \Sigma_K} \sup_{i\in [K]} w_i \frac{(\mu'_i-\tau)^2}{2\sigma^2}\\
		&=x^2 \sup_{w\in \Sigma_K} \sup_{i\in [K]} w_i \frac{(\mu_i-\tau)^2}{2\sigma^2}\\
		&=x^2(T^\star(\bm\mu))^{-1}
	\end{align*}
\end{proof}



% !TeX root = ../all.tex


\section{Concentration and threshold for the stopping rule}
\label{app:concentration}

We suppose that each arm is sampled once during the first $K$ time steps.
\begin{theorem}
  \label{thm:bound_delta}
  Suppose that the arm distributions are $\sigma^2$-sub-Gaussian. Let $\hat{\mu}_{t,k}$ be the average of arm $k$ at time $t$ and $N_{t,k}$ be the number of times arm $k$ is sampled up to time $t$.
  With probability $1 - \delta$, for all $t > K$,
  \begin{align*}
  \frac{1}{2} \sum_{k=1}^K N_{t,k}\frac{(\hat{\mu}_{t,k}- \mu_k)^2}{2 \sigma^2}
  &\le \frac{K}{2} \overline{W}\left(2\ln \left(\frac{e\pi^2}{6}\right) + \frac{2}{K}\ln \left(\prod_{k=1}^K (1 + \ln N_{t,k})^2\right) + \frac{2}{K}\ln \frac{1}{\delta}\right)
  \: .
  \end{align*}
\end{theorem}



\subsection{Proof of the concentration theorem}

We can assume $w.l.o.g.$ that $\mu_k = 0$ for all $k$ and $\sigma^2 = 1$.

Let $S_{t,k} = \sum_{s=1}^t X_{s,k} \mathbb{I}\{k_s = k\}$.
We want a bound on $\frac{1}{2} \sum_{k=1}^K \frac{S_{t,k}^2}{N_{t,k}}$.


We first remark that $\frac{1}{2}x^2 = \sup_{\lambda} \lambda x - \frac{1}{2}\lambda^2$~. Apply that to $x = S_{t,k}/\sqrt{N_{t,k}}$ to get
\begin{align*}
\sum_{k=1}^K \frac{1}{2}\frac{S_{t,k}^2}{N_{t,k}}
&= \sup_{\lambda_1, \ldots, \lambda_K} \sum_{k=1}^K \left( \lambda_k S_{t,k} - \frac{1}{2}N_{t,k} \lambda_k^2 \right)
\\
&= \sup_{\lambda_1, \ldots, \lambda_K} \sum_{s=1}^t \lambda_{k_s}X_{s,k_s} - \frac{1}{2}\lambda_{k_s}^2
\: .
\end{align*}
The advantage of that formulation is that we can concentrate the sum for any fixed value of $\lambda$ (or any distribution on $\lambda$) thanks to a martingale argument.

\begin{lemma}
For all $\rho \in \mathcal P(\mathbb{R}^K)$, the process $t \mapsto \mathbb{E}_{\lambda \sim \rho}\left[\exp\left(\sum_{s=1}^t \lambda_{k_s}X_{s,k_s} - \frac{1}{2}\lambda_{k_s}^2\right)\right]$ is a non-negative supermartingale with expectation bounded by 1.
\end{lemma}


\begin{corollary}\label{cor:prob_exists_log_ge_le}
For all $\rho \in \mathcal P(\mathbb{R}^K)$ and $x \ge 0$,
\begin{align*}
\mathbb{P}\left(\exists t, \ \ln \mathbb{E}_{\lambda \sim \rho}\left[\exp\left(\sum_{s=1}^t \lambda_{k_s}X_{s,k_s} - \frac{1}{2}\lambda_{k_s}^2\right)\right] \ge x\right) \le e^{-x}
\: .
\end{align*}
Equivalently, for all $\delta \in (0,1]$,
\begin{align*}
\mathbb{P}\left(\exists t, \ \ln \mathbb{E}_{\lambda \sim \rho}\left[\exp\left(\sum_{s=1}^t \lambda_{k_s}X_{s,k_s} - \frac{1}{2}\lambda_{k_s}^2\right)\right] \ge \ln\frac{1}{\delta}\right) \le \delta
\: .
\end{align*}
\end{corollary}

\begin{proof}
Use Ville's inequality and the fact that the process is a non-negative supermartingale.
\end{proof}

We don't want to bound an integral over $\lambda \sim \rho$, but the supremum over $\lambda$, so we need to relate the two quantities.
We do that for Gaussian priors over $\lambda$.

\begin{lemma}\label{lem:integral_exp_eq_log_add}
For $\rho = \mathcal N(0, \mathrm{diag}(\sigma_k^{-2}))$,
\begin{align*}
\ln \mathbb{E}_{\lambda \sim \rho}\left[\exp\left(\sum_{s=1}^t \lambda_{k_s}X_{s,k_s} - \frac{1}{2}\lambda_{k_s}^2\right)\right]
&= -\frac{1}{2}\sum_{k=1}^K \ln(1 + N_{t,k}\sigma_k^{-2}) + \frac{1}{2} \sum_{k=1}^K \frac{S_{t,k}^2}{(N_{t,k} + \sigma_k^2)}
\: .
\end{align*}
\end{lemma}

\begin{proof}
\begin{align*}
&\mathbb{E}_{\lambda \sim \rho}\left[\exp\left(\sum_{s=1}^t \lambda_{k_s}X_{s,k_s} - \frac{1}{2}\lambda_{k_s}^2\right)\right]
\\
&= \prod_k \mathbb{E}_{\lambda_k \sim \mathcal N(0, \sigma_k^{-2})}\left[\exp\left(\lambda_k S_{t,k} - \frac{1}{2}N_{t,k} \lambda_k^2\right)\right]
\\
&= \prod_k \frac{1}{\sqrt{2 \pi \sigma_k^{-2}}}\int_{\lambda_k}\exp\left(\lambda_k S_{t,k} - \frac{1}{2}N_{t,k} \lambda_k^2 - \frac{\sigma_k^2}{2}\lambda_k^2\right)\mathrm{d}\lambda_k
\\ 
&= \prod_k \frac{1}{\sqrt{(1 + N_{t,k}\sigma_k^{-2})}} \frac{1}{\sqrt{2 \pi (N_{t,k} + \sigma_k^2)^{-1}}}
  \int_{\lambda_k} \exp\left(-\frac{1}{2}(N_{t,k} + \sigma_k^2)\left( \lambda_k - \frac{S_{t,k}}{(N_{t,k} + \sigma_k^2)} \right)^2 + \frac{1}{2}\frac{S_{t,k}^2}{N_{t,k} + \sigma_k^2}\right)\mathrm{d}\lambda_k
\\
&= \prod_k \frac{1}{\sqrt{(1 + N_{t,k}\sigma_k^{-2})}} \exp\left(\frac{1}{2}\frac{S_{t,k}^2}{N_{t,k} + \sigma_k^2}\right)
\\ 
\end{align*}

\end{proof}

\begin{corollary}\label{cor:sum_le_eta_mul}
Let $\rho = \mathcal N(0, \mathrm{diag}(\sigma_k^{-2}))$, $\eta_{t,\max} = \max_k \frac{\sigma_k^2}{N_{t,k}}$ and $\eta_{t,\min} = \min_k \frac{\sigma_k^2}{N_{t,k}}$. Then
\begin{align*}
\frac{1}{2} \sum_{k=1}^K \frac{S_{t,k}^2}{N_{t,k}}
&\le (1 + \eta_{t,\max}) \left(\ln \mathbb{E}_{\lambda \sim \rho}\left[\exp\left(\sum_{s=1}^t \lambda_{k_s}X_{s,k_s} - \frac{1}{2}\lambda_{k_s}^2\right)\right]
  + \frac{K}{2}\ln(1 + \eta_{t,\min}^{-1})\right)
\end{align*}

\end{corollary}

\begin{proof}
Using Lemma~\ref{lem:integral_exp_eq_log_add},
\begin{align*}
\frac{1}{2} \sum_{k=1}^K \frac{S_{t,k}^2}{N_{t,k} + \sigma_k^2}
&= \ln \mathbb{E}_{\lambda \sim \rho}\left[\exp\left(\sum_{s=1}^t \lambda_{k_s}X_{s,k_s} - \frac{1}{2}\lambda_{k_s}^2\right)\right]
  + \frac{1}{2} \sum_{k=1}^K \ln(1 + N_{t,k}\sigma_k^{-2})
\: .
\end{align*}
Then
\begin{align*}
\frac{1}{2} \sum_{k=1}^K \frac{S_{t,k}^2}{N_{t,k} + \sigma_k^2}
\ge \frac{1}{2} \sum_{k=1}^K \frac{S_{t,k}^2}{N_{t,k}(1 + \eta_{t,\max})}
= \frac{1}{1 + \eta_{t,\max}}\frac{1}{2} \sum_{k=1}^K \frac{S_{t,k}^2}{N_{t,k}}
\end{align*}
Finally, $N_{t,k} \sigma_k^{-2} \le \eta_{t,\min}^{-1}$.
\end{proof}

If $N_{t,k}$ was a known, unchanging number, we could choose $\sigma_k^2 \propto N_{t,k}$ to get $\eta_{t,\max} = \eta_{t, \min}$, and we would choose it to minimize the right hand side.
The strategy to use that ``known $N_{t,k}$'' case even if they are random is to put geometric grids on the number of pulls of each arm, define distributions that are adapted to each cell of the grid, and combine them into a mixture of Gaussians.

Let $(\eta_{n_1, \ldots, n_K})_{n_1, \ldots, n_K \in \mathbb{N}}$ be non-negative real numbers that will be chosen later.
For $i \in \mathbb{N}$, let $w_i = \frac{6}{\pi^2}\frac{1}{(i+1)^2}$. The weights $(w_i)$ satisfy $\sum_{i \in \mathbb{N}} w_i = 1$, hence also $\sum_{n_1, \ldots, n_K} (\prod_{k=1}^K w_{n_k}) = 1$.

Let $\rho_{n_1, \ldots, n_K} = \bigotimes_{k=1}^K \mathcal N(0, e^{- n_k} \eta_{n_1, \ldots, n_K}^{-1})$. This is a product distribution, with each marginal being a Gaussian with mean 0 and variance that depends on the number of grid cell.

With probability $1 - \delta$, for all $(n_1, \ldots, n_K) \in \mathbb{N}^K$ and all $t$,
\begin{align*}
\ln \mathbb{E}_{\lambda \sim \rho_{n_1, \ldots, n_K}}\left[\exp\left(\sum_{s=1}^t \lambda_{k_s}X_{s,k_s} - \frac{1}{2}\lambda_{k_s}^2\right)\right]
\le \ln \frac{1}{\delta} + \sum_{k=1}^K \ln \frac{1}{w_{n_k}}
\: .
\end{align*}
This is simply an union bound using Corollary~\ref{cor:prob_exists_log_ge_le}, with weight $\prod_{k=1}^K w_{n_k}$ for $\rho_{n_1, \ldots, n_K}$.

In particular, there exists $(n_1, \ldots, n_k)$ such that for all $k \in [K]$, $e^{n_k} \le N_{t,k} \le e^{n_k+1}$.
For that choice, $e^{-1}\eta_{n_1, \ldots, n_K} \le \frac{e^{n_k}\eta_{n_1, \ldots, n_K}}{N_{t,k}} \le \eta_{n_1, \ldots, n_K}$.
For those values, using Corollary~\ref{cor:sum_le_eta_mul} with $\sigma_k^2 = e^{n_k}\eta_{n_1, \ldots, n_K}$, with probability $1 - \delta$,
\begin{align*}
\frac{1}{2} \sum_{k=1}^K \frac{S_{t,k}^2}{N_{t,k}}
&\le (1 + \eta_{n_1, \ldots, n_K}) \left(\ln \frac{1}{\delta} + \sum_{k=1}^K \ln \frac{1}{w_{n_k}}
  + \frac{K}{2}\ln(1 + e \eta_{n_1, \ldots, n_K}^{-1})\right)
\\
&\le (1 + \eta_{n_1, \ldots, n_K}) \left(\ln \frac{e^{K/2}}{\delta} + \sum_{k=1}^K \ln \frac{1}{w_{n_k}}
  + \frac{K}{2}\ln(1 + \eta_{n_1, \ldots, n_K}^{-1})\right)
\\
&= (1 + \eta_{n_1, \ldots, n_K}) \left(\ln \frac{(\sqrt{e}\pi^2/6)^K \prod_{k=1}^K (n_k+1)^2}{\delta}
  + \frac{K}{2}\ln(1 + \eta_{n_1, \ldots, n_K}^{-1})\right)
\end{align*}

This is where we choose $\eta_{n_1, \ldots, n_K}$ to minimize the right hand side.

By Lemma A.3 of \citep{degenneImpactStructureDesign2019}, the minimal value is attained at some $\eta_{n_1, \ldots, n_K}$ such that
\begin{align*}
&(1 + \eta_{n_1, \ldots, n_K}) \left(\ln \frac{(\sqrt{e}\pi^2/6)^K \prod_{k=1}^K (n_k+1)^2}{\delta}
  + \frac{K}{2}\ln(1 + \eta_{n_1, \ldots, n_K}^{-1})\right)
\\
&= \frac{K}{2} \overline{W}\left( 1 + \frac{2}{K}\ln \frac{(\sqrt{e}\pi^2/6)^K \prod_{k=1}^K (n_k+1)^2}{\delta}\right)
\end{align*}

By the choice of $n_k$, it satisfies $n_k \le \ln N_{t,k}$.
We get that with probability $1 - \delta$, for all $t$,
\begin{align*}
\frac{1}{2} \sum_{k=1}^K \frac{S_{t,k}^2}{N_{t,k}}
&\le \frac{K}{2} \overline{W}\left( 1 + \frac{2}{K}\ln \frac{(\sqrt{e}\pi^2/6)^K \prod_{k=1}^K (1 + \ln N_{t,k})^2}{\delta}\right)
\\
&= \frac{K}{2} \overline{W}\left(\frac{2}{K}\ln \left((e\pi^2/6)^K \prod_{k=1}^K (1 + \ln N_{t,k})^2\right) + \frac{2}{K}\ln \frac{1}{\delta}\right)
\: .
\end{align*}

This ends the proof of the theorem.



\subsection{Upper bounds on $\beta(t, \delta)$ and on $\gamma_r$}

We choose the threshold
\begin{align*}
\beta(t, \delta)
&= \frac{K}{2} \overline{W}\left(2\ln \left(\frac{e\pi^2}{6}\right) + \frac{2}{K}\ln \left(\prod_{k=1}^K (1 + \ln N_{t,k})^2\right) + \frac{2}{K}\ln \frac{1}{\delta}\right)
\: .
\end{align*}

We can get an upper bound that is not random by maximizing over $(N_{t,k})_{k \in [K]}$ under the constraint $\sum_{k=1}^K N_{t,k} = t$. We get
\begin{align*}
\beta(t, \delta)
&\le \frac{K}{2} \overline{W}\left(2\ln \left(\frac{e\pi^2}{6}\right) + 4\ln \left(1 + \ln \frac{t}{K}\right) + \frac{2}{K}\ln \frac{1}{\delta}\right)
\: .
\end{align*}
We can get further upper bounds by using $\overline{W}(x) \le x + \ln x + 1/2 \le 2x$. This gives
\begin{align*}
\beta(t, \delta)
&\le 2 K\ln \left(\frac{e\pi^2}{6}\right) + 4 K\ln \left(1 + \ln \frac{t}{K}\right) + 2\ln \frac{1}{\delta}
\\
&\le 2 K\ln \left(\frac{e\pi^2}{6}\right) + 4 K\ln \frac{t}{K} + 2\ln \frac{1}{\delta}
\: .
\end{align*}
The right asymptotic for $\beta(t, \delta)$ as $\delta \to 0$ is $\ln(1/\delta)$. We lost a factor 2 in the upper bound above.




\begin{lemma}\label{lem:gamma_ub_bis}
Let $\gamma_r$ be the solution to $\beta(\bar{t}_r, \delta) = \gamma_r$ , for $\bar{t}_r = 2(K l_{1,r}/T_0 + \gamma_r) T_r$ and $l_{1,r} = 32 T_0 \ln (2\sqrt{2K} T_r)$. Then
\begin{align*}
\gamma_r \le 4 \ln \frac{1}{\delta} + 8K \ln(T_r) + 4K (11 + \ln K)
\: .
\end{align*}
\end{lemma}

\begin{proof}
We use an upper bound for $\beta(t, \delta)$: $\gamma_r$ is bounded from above by the solution $\gamma'_r$ to
\begin{align*}
\gamma = 2 K\ln \left(\frac{e\pi^2}{6}\right) + 4 K \ln \left(2(32 \ln (2\sqrt{2K} T_r) + \frac{\gamma}{K}) T_r\right) + 2\ln \frac{1}{\delta}
\: .
\end{align*}
Then either $\gamma'_r \le 8 K \ln (2\sqrt{2K} T_r)$ or $\gamma'_r$ is less than the solution to
\begin{align*}
\gamma
&= 2 K\ln \left(\frac{e\pi^2}{6}\right) + 4 K \ln \left(10\frac{\gamma T_r}{K}\right) + 2\ln \frac{1}{\delta}
\\
&= 2 K\ln \left(\frac{50 e\pi^2}{3}\right) + 4 K \ln \left(\frac{\gamma T_r}{K}\right) + 2\ln \frac{1}{\delta} 
\: .
\end{align*}


That is,
\begin{align*}
\gamma'_r
&= 4K \overline{W}\left( \frac{1}{2K} \ln \frac{1}{\delta} + \ln(T_r) + \frac{1}{2}\ln\frac{800 e \pi^2}{3} \right)
\\
&\le 4 \ln \frac{1}{\delta} + 8K \ln(T_r) + 4K \ln\frac{800 e \pi^2}{3}
\: .
\end{align*}

At this point, we get
\begin{align*}
\gamma_r
&\le \max\left\{ 4 \ln \frac{1}{\delta} + 8K \ln(T_r) + 4K \ln\frac{800 e \pi^2}{3},  8 K \ln (2\sqrt{2K} T_r)\right\}
\\
&\le 8K \ln(T_r) + \max\left\{ 4 \ln \frac{1}{\delta} + 4K \ln\frac{800 e \pi^2}{3}, 8 K \ln (2\sqrt{2K})\right\}
\\
&\le 8K \ln(T_r) + 4 \ln \frac{1}{\delta} + 4K \ln\frac{800 e \pi^2}{3} + 8 K \ln (2\sqrt{2K})
\\
&\le 8K \ln(T_r) + 4 \ln \frac{1}{\delta} + 4K \ln\left(\frac{6400 e \pi^2}{3}K\right)
\\
&\le 8K \ln(T_r) + 4 \ln \frac{1}{\delta} + 4K (11 + \ln K)
\: .
\end{align*}


\end{proof}

% !TeX root = ../all.tex


\section{Proofs related to the algorithm}\label{app:ub} 

\subsection{Additional Lemmas}

\begin{lemma}\label{lem:subG_concentration}
Let $(X_i)_{i \in \mathbb{N}}$ be i.i.d. $\sigma^2$-sub-Gaussian random variables with mean $\mu$. For $n \in \mathbb{N}$, let $\hat{\mu}_n$ be the average of the first $n$ random variables. Then
\begin{align*}
\mathbb{P}(\exists n \ge N, \ \hat{\mu}_n \ge \mu + \varepsilon) \le e^{- \frac{N \varepsilon^2}{2\sigma^2}}
\: , \\
\mathbb{P}(\exists n \ge N, \ \hat{\mu}_n \le \mu - \varepsilon) \le e^{- \frac{N \varepsilon^2}{2\sigma^2}} \: .
\end{align*}
\end{lemma}

\begin{proof}
Given $(X_1, \ldots, X_N)$, the process $M_n(\lambda) : n \mapsto e^{\lambda\sum_{i=1}^{N+n} (X_i - \mu) - \frac{1}{2}(N+n) \sigma^2 \lambda^2}$ is a nonnegative supermartingale for any $\lambda \in \mathbb{R}$ by the sub-Gaussian hypothesis, with expectation $e^{\lambda\sum_{i=1}^{N} (X_i - \mu) - \frac{1}{2}N \sigma^2 \lambda^2}$ at $n=0$.

By Ville's inequality,
\begin{align*}
\mathbb{P}(\exists n, \  M_n(\lambda) \ge 1/\delta \mid X_1, \ldots, X_N)
\le \delta e^{\lambda\sum_{i=1}^{N} (X_i - \mu) - \frac{1}{2}N \sigma^2 \lambda^2}
\end{align*}

For all $\lambda \in \mathbb{R}$ and all $\delta \in (0,1)$,
\begin{align*}
\mathbb{P}\left(\exists n \ge N, \  \lambda\sum_{i=1}^{n} (X_i - \mu) - \frac{1}{2}n\sigma^2 \lambda^2 \ge \ln(1/\delta)\right)
&= \mathbb{E}\left[\mathbb{P}(\exists n \ge 0, \  M_n(\lambda) \ge 1/\delta \mid X_1, \ldots, X_N)\right]
\\
&\le \delta \mathbb{E}\left[e^{\lambda\sum_{i=1}^{N} (X_i - \mu) - \frac{1}{2}N \sigma^2 \lambda^2}\right]
\\
&\le \delta
\: .
\end{align*}
Reordering, we get, for $\lambda \ge 0$,
\begin{align*}
\mathbb{P}\left(\exists n \ge N, \  \hat{\mu}_n \ge \mu + \frac{1}{2}\sigma^2 \lambda + \frac{1}{N \lambda}\ln\frac{1}{\delta}\right)
&\le \mathbb{P}\left(\exists n \ge N, \  \hat{\mu}_n \ge \mu + \frac{1}{2}\sigma^2 \lambda + \frac{1}{n \lambda}\ln\frac{1}{\delta}\right)
\\
&\le \delta
\: .
\end{align*}

Choose $\delta = e^{-\frac{N \varepsilon^2}{2 \sigma^2}}$ and $\lambda = \frac{\varepsilon}{\sigma^2}$ to obtain
\begin{align*}
\mathbb{P}\left(\exists n \ge N, \  \hat{\mu}_n \ge \mu + \varepsilon \right)
\le e^{-\frac{N \varepsilon^2}{2 \sigma^2}}
\: .
\end{align*}

The second inequality is obtained similarly, with $\lambda \le 0$.
\end{proof}

\begin{lemma}\label{lem:probaE_bis}
	The probability of $\mathcal E_r$ satisfies
	\begin{align*}
	\mathbb{P}(\mathcal E_r) \ge 1-2K\exp(- 2^r l_{1,r} \varepsilon_r^2/2\sigma^2)
	\: .
	\end{align*}
\end{lemma}

\begin{proof}
$\mathcal E_r$ is the event that $\Vert \bm\mu - \tilde{\bm\mu}^r \Vert_\infty \le \varepsilon_r$ and $\Vert \bm\mu - \hat{\bm\mu}^r \Vert_\infty \le \varepsilon_r$.
We use an union bound over the arms to bound the probability of the complement $\mathcal{E}_r^c$.
For each $i \in [K]$, $\tilde{\mu}^r_i$ and $\hat{\mu}^r_i$ are empirical means of at least $2^r l_{1,r}$ samples. We can thus apply Lemma~\ref{lem:subG_concentration} (twice, once for deviations from above and once for deviations from below).
\end{proof}

\begin{lemma}\label{lem:proba_Er}
	Let $p_r \in (0,1]$.
	For the choice $\varepsilon_r = \sqrt{\frac{2\sigma^2}{2^r l_1}\ln\frac{2K}{p_r}}$, the probability of the event $\mathcal E_r$ is $\mathbb{P}(\mathcal E_r) \ge 1 - p_r$.
\end{lemma}


\subsection{Proof of Theorem~\ref{th:alggen}}\label{app:ub_proof}

If $\overline{T}^\star(\hat{B}_r) > T_r$ then the algorithm does not enter the second batch of the phase by definition of the algorithm.

If $\overline{T}^\star(\hat{B}_r) \le T_r$ then by the choice of $\gamma_r$ and Lemma~\ref{lem:sufficientsampling}, under $\mathcal E_r$ the stopping condition is triggered.

We now prove the complexity upper bounds. Let $\mathcal C_r$ be the event that the algorithm attains phase $r$ and does not stop at that phase.
We proved that $\{\overline{T}^\star(\hat{B}_r) \le T_r\} \cap \mathcal E_r \subseteq \mathcal C_r^c$ for all $r$. That is, $\mathcal C_r \subseteq \mathcal E_r^c \cup \{\overline{T}^\star(\hat{B}_r) > T_r\}$.

Recall that $R^* = \min \{r \mid \forall r' \ge r, \ \mathcal E_{r'} \implies \overline{T}^\star(\hat{B}_{r'}) \le T_{r'}\}$.
\begin{align*}
R_\delta
&= \sum_{r=1}^{+\infty} \mathbb{I}(\mathcal C_{r-1}) + \mathbb{I}(\mathcal C_{r-1} \wedge \{\overline{T}^\star(\hat{B}_r) \le T_r\}) \: .
\\
&\le R^* + 2 \sum_{r=R^*+1}^{+\infty} \mathbb{I}(\mathcal C_{r-1}) + \sum_{r=1}^{R^*} \mathbb{I}(\mathcal C_{r-1} \wedge \{\overline{T}^\star(\hat{B}_r) \le T_r\})
\: .
\end{align*}
By definition of $R^*$, for $r \ge R^*$ we have $ \{\overline{T}^\star(\hat{B}_r) > T_r\} \subseteq \mathcal E_r^c$.
Using that property and the inclusion we proved on $\mathcal C_r$ we have, for $r > R^*$,
\begin{align*}
\mathcal C_{r-1}
\subseteq \mathcal E_{r-1}^c \cup \{\overline{T}^\star(\hat{B}_{r-1}) > T_{r-1}\}
\subseteq \mathcal E_{r-1}^c
\: .
\end{align*}
Therefore,
\begin{align*}
R_\delta
&\le R^* + 2 \sum_{r=R^*+1}^{+\infty} \mathbb{I}(\mathcal E_{r-1}^c) + \sum_{r=1}^{R^*} \mathbb{I}(\mathcal C_{r-1} \wedge \{\overline{T}^\star(\hat{B}_r) \le T_r\})
\: .
\end{align*}
When $\mathcal E_r$ happens $T^\star(\bm\mu) \le \overline{T}^\star(\hat{B}_r)$, hence $\{\overline{T}^\star(\hat{B}_r) \le T_r\} \subseteq \mathcal E_r^c \cup \{T^\star(\bm\mu) \le T_r\}$.
\begin{align*}
R_\delta
&\le R^* + 2 \sum_{r=R^*+1}^{+\infty} \mathbb{I}(\mathcal E_{r-1}^c) + \sum_{r=1}^{R^*} \mathbb{I}(\mathcal E_{r}^c)
	+ \sum_{r=1}^{R^*} \mathbb{I}(\{T^\star(\bm\mu) \le T_r\})
\\
&\le R^* + 1 + 2 \sum_{r=1}^{+\infty} \mathbb{I}(\mathcal E_{r}^c) + \sum_{r=1}^{R^*} \mathbb{I}(\{T^\star(\bm\mu) \le T_r\})
\: .
\end{align*}
Finally, $ \sum_{r=1}^{R^*} \mathbb{I}(\{T^\star(\bm\mu) \le T_r\}) = \max\{0, R^* - \lceil \log_2 \frac{T^\star(\bm\mu)}{T_0} \rceil \}$, and the definition of $R^*$ implies $R^* \ge \lceil \log_2 \frac{T^\star(\bm\mu)}{T_0} \rceil$.

We now bound the sample complexity $\tau_\delta$. Since $\bar{t}_r$ is an upper bound on the sample complexity up to phase $r$,
\begin{align*}
\tau_\delta
&\le \sum_{r=1}^{+\infty} \bar{t}_r \mathbb{I}\{C_{r-1}\}
\\
&\le \bar{t}_{R^*} + \sum_{r=R^* + 1}^{+\infty} \bar{t}_r \mathbb{I}\{C_{r-1}\}
\\
&\le \bar{t}_{R^*} + \sum_{r=R^* + 1}^{+\infty} \bar{t}_r \mathbb{I}\{E_{r-1}^c\}
\: .
\end{align*}



\subsection{Proof of the batch and complexity upper bounds}

We finish the proof of Theorem~\ref{thm:compexity_upper_bounds} from where we stopped in the main text. We have
\begin{align*}
\mathbb{E}\left[R_\delta\right]
&\le 2r^* - \lceil \log_2 \frac{T^\star(\bm\mu)}{T_0} \rceil + 1 + 2\sum_{r = 1}^{+\infty} p_r
\: , \\
\mathbb{E}\left[\tau_\delta\right]
&\le \bar{t}_{r^*} + \sum_{r = 1}^{+\infty} p_{r} \bar{t}_{r+1}
\: .
\end{align*}
In those expressions, $r^* = \max\{r_0, r_1\}$ with $r_0 = \min\{r \mid 2 \varepsilon_r \le b(\bm\mu)\}$, $r_1 =  \min \{r \mid T_r \ge e T^\star(\bm\mu)\}$, and
$\bar{t}_r = (K l_{1,r}/T_0 + 2 \gamma_r) T_r$ with $l_{1,r}/T_0 = 32 \ln(2 \sqrt{2K} T_r)$.

The choice of $p_r$ is a trade-off between the sums and $r_0$. We choose $p_r = T_{r+1}^2$.

\paragraph{Bounding the sums}
The sum in the batch complexity is bounded by $T_0^{-2}/3$.
The sum that appears in the sample complexity is
\begin{align*}
\sum_{r = \max\{r_0, r_1\}+1}^{+\infty} p_{r-1} \bar{t}_r
= \sum_{r = \max\{r_0, r_1\}+1}^{+\infty} \frac{\bar{t}_r}{T_r^2}
\: .
\end{align*}
We will need the values of a few sums.
\begin{align*}
\sum_{r=1}^{+\infty} \frac{1}{T_r}
&= \sum_{r=1}^{+\infty} \frac{1}{2^r T_0}
= \frac{1}{T_0}
\: , \\
\sum_{r=1}^{+\infty} \frac{\ln T_r}{T_r}
&= \frac{1}{T_0} \sum_{r=1}^{+\infty} \frac{r \ln2 + \ln T_0}{2^r}
= \frac{\ln (4T_0)}{T_0}
\: .
\end{align*}

Let $c_{K,\delta} = 4 \ln \frac{1}{\delta} + 4K (11 + \ln K)$.
By Lemma~\ref{lem:gamma_ub_bis}, $\gamma_r \le 8K \ln(T_r) + c_{K,\delta}$ .
An upper bound on the sample complexity until the end of phase $r$ is then
\begin{align*}
\bar{t}_r
&= (K l_{1,r}/T_0 + 2\gamma_r) T_r
\\
&= (32 K \ln(2\sqrt{2K}T_r) + 2 \gamma_r) T_r
\\
&\le (48 K \ln(T_r) + 32 K \ln(2\sqrt{2K}) + c_{K,\delta})T_r
\: .
\end{align*}
The sum that appears in the sample complexity is at most
\begin{align*}
\sum_{r = 1}^{+\infty} \frac{\bar{t}_r}{T_r^2}	
&\le \frac{48 K \ln(4T_0) + 32 K \ln(2\sqrt{2K}) + c_{K,\delta}}{T_0}
\: .
\end{align*}

\paragraph{Bound on $r^*$ and $\bar{t}_{r^*}$}

\begin{align*}
\bar{t}_{r^*}
\le (48 K \ln (\max\{T_{r_0}, T_{r_1}\}) + 32 K \ln(2\sqrt{2K}) + c_{K,\delta}) \max\{T_{r_0}, T_{r_1}\}
\: .
\end{align*}


Recall that $r_0 = \min\{r \mid 2 \varepsilon_r \le b(\bm\mu)\}$, $r_1 =  \min \{r \mid T_r \ge e T^\star(\bm\mu)\}$.
If we get an upper bound $n$ on $T_i$, we then have $r_i \le \log_2 \frac{n}{T_0}$.
We get a bound on $T_{r_1}$ from its definition: $T_{r_1-1} \le e T^*(\bm\mu)$ hence $T_{r_1} \le 2 e T^\star(\bm\mu)$.

Since $\varepsilon_{r_0 - 1} \ge b(\bm\mu)/2$, we get an inequality on $T_{r_0 - 1}$.
\begin{align*}
\sqrt{\frac{2\sigma^2}{2^{r_0 - 1}l_{1, r_0-1}} \ln \left( 2K T_{r_0}^2 \right)} \ge \frac{b(\bm\mu)}{2}
\: .
\end{align*}
With the value of $l_{1,r}$ and using $2 T_{r_0 - 1} = T_{r_0}$, this becomes
\begin{align*}
T_{r_0} \le \frac{\sigma^2}{b(\bm\mu)^2}
\: .
\end{align*}

Let $T^\star_b(\bm\mu) = \max\{\frac{\sigma^2}{b(\bm\mu)^2}, 2 e T^\star(\bm\mu)\}$.
We have proved $\max\{T_{r_0}, T_{r_1}\} \le T^\star_b(\bm\mu)$, hence
\begin{align*}
\bar{t}_{r^*}
\le (48 K \ln T^\star_b(\bm\mu) + 32 K \ln(2\sqrt{2K}) + c_{K,\delta}) T^\star_b(\bm\mu)
\end{align*}
and $r^* \le \log_2 \frac{T^\star_b(\bm\mu)}{T_0}$.

\paragraph{Putting things together}

\begin{align*}
\mathbb{E}\left[R_\delta\right]
&\le \log_2 \frac{T^\star_b(\bm\mu)}{T_0} + \log_2 \frac{T^\star_b(\bm\mu)}{T^\star(\bm\mu)} + 1 + T_0^{-2}
\: , \\
\mathbb{E}\left[\tau_\delta\right]
&\le (48 K \ln T^\star_b(\bm\mu) + 32 K \ln(2\sqrt{2K}) + c_{K,\delta}) T^\star_b(\bm\mu)
\\ & \quad + \frac{48 K \ln(4T_0) + 32 K \ln(2\sqrt{2K}) + c_{K,\delta}}{T_0}
\: .
\end{align*}

Let's simplify the sample complexity.
\begin{align*}
32 K \ln(2\sqrt{2K}) + c_{K,\delta}
&= 32 K \ln(2\sqrt{2K}) + 4 \ln \frac{1}{\delta} + 4K (11 + \ln K)
\\
&= 4 \ln \frac{1}{\delta} + 4K (5\ln K + 11 + 4 \ln(8))
\\
&\le 4 \ln \frac{1}{\delta} + 20K (\ln K + 4)
\: .
\end{align*}

\begin{align*}
\mathbb{E}\left[\tau_\delta\right]
&\le (48 K \ln T^\star_b(\bm\mu) + 4 \ln \frac{1}{\delta} + 20K (\ln K + 4)) T^\star_b(\bm\mu)
\\ & \quad + (48 K \ln(4T_0) + 4 \ln \frac{1}{\delta} + 20K (\ln K + 4)) T_0^{-1}
\\
&= 4 \ln \left(\frac{1}{\delta}\right) (T^\star_b(\bm\mu) + T_0^{-1}) + 20K (\ln K + 4) (T^\star_b(\bm\mu) + T_0^{-1})
	+ 48 K (T^\star_b(\bm\mu) \ln T^\star_b(\bm\mu) + T_0^{-1} \ln(4T_0))
\: .
\end{align*}



\subsection{Implication between the two assumptions}

We prove Lemma~\ref{lem:asm2_implies_asm1}.

\begin{lemma}\label{lem:sub_sqrt_inv_TStar} 
Let $\bm\nu$ and $\bm\nu'$ be two instances and let $\omega_{\bm \mu} = \arg\max_\omega \inf_{\lambda \in Alt_{\bm\mu}}\sum_{i=1}^K \omega_i (\mu_i - \lambda_i)^2$. Then
\begin{align*}
\sqrt{T^\star(\bm\mu)^{-1}} - \sqrt{T^\star(\bm\mu')^{-1}}
\le \frac{1}{\sqrt{2\sigma^2}} \Vert \mu - \mu' \Vert_\infty
\: .
\end{align*}
\end{lemma}

\begin{proof}
For $\omega \in \Sigma_K$ and $\bm x \in \mathbb{R}^K$, let $\Vert \bm x \Vert_\omega = \sqrt{\sum_{i=1}^K \omega_i x_i^2}$.
It satisfies the triangle inequality and $\Vert \bm x \Vert_\omega \le \Vert \bm x \Vert_\infty$.
For any $\bm\lambda$ and $\omega$,
\begin{align*}
\Vert \bm\mu - \bm\lambda\Vert_\omega \le \Vert \bm\mu' - \bm\lambda\Vert_\omega + \Vert \bm\mu - \bm\mu'\Vert_\omega
\: .
\end{align*}
We can take an infimum on both sides over lambda in $Alt_{\bm\mu}$ and then apply the result to $\omega_{\bm\mu}$ to get
\begin{align*}
\sqrt{2\sigma^2 T^\star(\bm\mu)^{-1}} \le \inf_{\lambda \in Alt_{\bm\mu}}\Vert \bm\mu' - \bm\lambda\Vert_{\omega_{\bm\mu}} + \Vert \bm\mu - \bm\mu'\Vert_{\omega_{\bm\mu}}
\: .
\end{align*}
Either $Alt_{\bm\mu} = Alt_{\bm\mu'}$ and we can replace by that on the right hand side, or $\bm\mu' \in Alt_{\bm\mu}$. In that second case $\inf_{\lambda \in Alt_{\bm\mu}}\Vert \bm\mu' - \bm\lambda\Vert_{\omega_{\bm\mu}} = 0 \le \inf_{\lambda \in Alt_{\bm\mu'}}\Vert \bm\mu' - \bm\lambda\Vert_{\omega_{\bm\mu}}$. We thus have
\begin{align*}
\sqrt{2\sigma^2 T^\star(\bm\mu)^{-1}} \le \inf_{\lambda \in Alt_{\bm\mu'}}\Vert \bm\mu' - \bm\lambda\Vert_{\omega_{\bm\mu}} + \Vert \bm\mu - \bm\mu'\Vert_{\omega_{\bm\mu}}
\: .
\end{align*}
We maximize over $\omega$ to get $\inf_{\lambda \in Alt_{\bm\mu'}}\Vert \bm\mu' - \bm\lambda\Vert_{\omega_{\bm\mu}} \le \sqrt{2\sigma^2 T^\star(\bm\mu')^{-1}}$, hence
\begin{align*}
\sqrt{2\sigma^2 T^\star(\bm\mu)^{-1}}
&\le \sqrt{2\sigma^2 T^\star(\bm\mu')^{-1}} + \Vert \bm\mu - \bm\mu'\Vert_{\omega_{\bm\mu}}
\\
&\le \sqrt{2\sigma^2 T^\star(\bm\mu')^{-1}} + \Vert \bm\mu - \bm\mu'\Vert_{\infty}
\: .
\end{align*}
After dividing by $\sqrt{2\sigma^2}$, this is the inequality of the lemma.
\end{proof}


\begin{corollary}\label{cor:sub_ln_TStar_le}
For all $\bm\mu$ and $\bm\mu'$,
\begin{align*}
\ln T^\star(\bm\mu') - \ln T^\star(\bm\mu)
\le \sqrt{\frac{2}{\sigma^2}T^\star(\bm\mu')} \ \Vert \bm\mu - \bm\mu' \Vert_\infty
\: .
\end{align*}
\end{corollary}

\begin{proof}
\begin{align*}
\ln T^\star(\bm\mu') - \ln T^\star(\bm\mu)
&= 2 \ln \left( 1 + \frac{\sqrt{T^\star(\bm\mu)^{-1}} - \sqrt{T^\star(\bm\mu')^{-1}}}{\sqrt{T^\star(\bm\mu')^{-1}}} \right)
\\
&\le 2 \frac{\sqrt{T^\star(\bm\mu)^{-1}} - \sqrt{T^\star(\bm\mu')^{-1}}}{\sqrt{T^\star(\bm\mu')^{-1}}}
\\
&\le \sqrt{\frac{2}{\sigma^2}T^\star(\bm\mu')} \ \Vert \bm\mu - \bm\mu' \Vert_\infty
\: .
\end{align*}

\end{proof}


\begin{corollary}\label{cor:abs_sub_ln_TStar_le}
For all $\bm\mu$ and $\bm\mu'$ with $\Vert \bm\mu - \bm\mu' \Vert_\infty \le \sqrt{\sigma^2 / (2 T^\star(\bm\mu))}$,
\begin{align*}
\left\vert \ln T^\star(\bm\mu') - \ln T^\star(\bm\mu) \right\vert
\le \sqrt{\frac{8}{\sigma^2}T^\star(\bm\mu)} \ \Vert \bm\mu - \bm\mu' \Vert_\infty
\: .
\end{align*}
\end{corollary}

\begin{proof}
One of the two inequalities we need to prove is due to Corollary~\ref{cor:sub_ln_TStar_le}. For the other, by the same corollary,
\begin{align*}
\ln T^\star(\bm\mu') - \ln T^\star(\bm\mu)
\le \sqrt{\frac{2}{\sigma^2}T^\star(\bm\mu')} \ \Vert \bm\mu - \bm\mu' \Vert_\infty
\: .
\end{align*}
It remains to show $T^\star(\bm\mu') \le 4 T^\star(\bm\mu)$. By Lemma~\ref{lem:sub_sqrt_inv_TStar} and the the hypothesis on $\Vert \bm\mu - \bm\mu' \Vert_\infty$,
\begin{align*}
\sqrt{T^\star(\bm\mu)^{-1}} - \sqrt{T^\star(\bm\mu')^{-1}}
\le \frac{1}{\sqrt{2\sigma^2}} \Vert \bm\mu - \bm\mu' \Vert_\infty
\le \frac{1}{2}\sqrt{T^\star(\bm\mu)^{-1}} \: .
\end{align*}
Reordering proves the inequality.
\end{proof}



\subsection{Proofs for Top-K and thresholding bandits}
\label{app:topk_threshold}

This section is devoted to the proof of Lemma~\ref{lem:constrBbai}. We start with a preliminary result allowing us to compute $\overline{w}^\star(B)$ once we have found a suitable instance in $B$.


\begin{lemma}\label{lem:wbwst}
	If all instances in $B$ share the same correct answer $i^\star$ and if there exists some mean vector $\bm b\in B$ such that\begin{equation}\label{eq:bworst} \inf_{\bm\nu\in B}\inf_{\bm{\lambda}\in Alt_{\bm b}} \sum_i w_i(\bm{b}) \frac{(\mu_i-\lambda_i)^2}{2\sigma^2}\geq \inf_{\bm{\lambda}\in Alt_{\bm b}} \sum_i w_i(\bm{b}) \frac{(b_i-\lambda_i)^2}{2\sigma^2}\end{equation} where $w(\bm\mu)=\argmax_{w\in \Sigma_K} \inf_{\bm\lambda \in Alt_{\bm\mu}} \sum_i w_i \frac{(\mu_i-\lambda_i)^2}{2\sigma^2}$, then $\overline{T}^\star(B) = \max_{\bm \mu\in B}T^\star(\bm \mu)$.
\end{lemma}

\begin{proof}
	For some $w\in \Sigma_K$, writing $f(w,\bm\mu')=\inf_{\bm\lambda\in Alt_{\bm b}} \sum_i w_i \frac{(\mu'_i-\lambda_i)^2}{2\sigma^2}$ for clarity,
	\begin{align*}
		\inf_{\bm \nu\in \hat{B}_r} f( w,\bm\mu) &\leq f( w,\bm b) &\text{because }\bm b\in \hat{B}_r\\
		&\leq f(w_{\bm b},\bm b)& \text{from the definition of }w_{\bm b}\\
		&\leq \inf_{\bm\mu'\in\hat{B}_r} f(w_{\bm b},\bm\mu')&\text{from the hypothesis}
	\end{align*} so that $w^\star(\bm b) = \argmax_{w\in \Sigma_K} \inf_{\bm\nu\in\hat{B}_r} f(\bm w,\nu)=\overline w^\star(B)$.

\begin{align*}
	\overline{T}^\star(B) &= \left( \inf_{\bm\nu'\in B} \inf_{\bm\lambda \in Alt_{\bm\nu'}} \sum_i \overline{w}^\star_i(B) \frac{(\mu'_i-\lambda_i)^2}{2\sigma^2}\right)^{-1} &\\
	&=\left( \inf_{\bm\nu'\in B} \inf_{\bm\lambda \in Alt_{\bm\nu'}} \sum_i w^\star_i(\bm b) \frac{(\mu'_i-\lambda_i)^2}{2\sigma^2}\right)^{-1}&\\
	&=\left(  \inf_{\bm\lambda \in Alt_{\bm b}} \sum_i w^\star_i(\bm b) \frac{(b_i-\lambda_i)^2}{2\sigma^2}\right)^{-1}&\hspace{-5em}\text{by Equation~\eqref{eq:bworst}}\\
	&= T^\star(\bm b)&
\end{align*}
hence $\overline{T}(B) \leq \max_{\bm\nu\in B} T^\star(\bm\nu)$. By definition, we have the other inequality, and we conclude.
\end{proof}



\lemconstrBbai*

We prove the result separately for top-$k$ and TBP. In both cases, we give a certain mean vector $\bm b$, and then we show it satisfies the premise of Lemma~\ref{lem:wbwst}, then we use that result to show that $\overline{T}^\star(\mathcal{B}_\infty(\bm\mu,\varepsilon))=T^\star(\bm b)$.

\begin{proof}[Proof of Lemma~\ref{lem:constrBbai} for top-$k$]
	Assume without loss of generality that the arms are well ordered, $\mu_1\geq \mu_2\geq \cdots \geq \mu_K$.
	
	If $\mu_{k}-\mu_{k+1}\leq 2\varepsilon$, then there exists $\bm b\in \mathcal B_{\infty}(\bm\mu, \varepsilon)$ such that $b_k=b_{k+1}$. $\overline{T}^\star(\mathcal B_{\infty}(\bm\mu, \varepsilon))\geq \max_{\bm\nu'\in \mathcal B_{\infty}(\bm\mu, \varepsilon)} T^\star(\bm\nu')\geq T^\star(\bm b)=+\infty$, therefore \[\overline{T}^\star(\mathcal B_{\infty}(\bm\mu, \varepsilon))= \max_{\bm\nu'\in \mathcal B_{\infty}(\bm\mu, \varepsilon)} T^\star(\bm\nu')\: .\]
	
	When $\mu_k-\mu_{k+1}>2\varepsilon$, define \[\left\{ \begin{aligned} &b_i= \mu_i-\varepsilon \qquad\text{if }i\leq k \\ &b_i =\mu_i +\varepsilon \qquad\text{if }i\geq k+1\end{aligned}\right. \]
	and, for any $\bm\nu'$, \[w_{\bm\nu'} =\argmax_{w\in \Sigma_K} \inf_{\bm\lambda\in Alt_{\bm\nu'}} \sum_i w_i \frac{(\mu_i'-\lambda_i)^2}{2\sigma^2}.\] 
	
	
	Let there be some $\bm{\nu}'\in \mathcal B_{\infty}(\bm\mu, \varepsilon)$. Then for $i\leq k$, $\mu'_{i} \geq \mu_i-\varepsilon=b_{i}$ and for $i\geq l+1$, $\mu'_i \leq \mu_i +\varepsilon = b_i$, and $Alt_{\bm\nu'}=Alt_{\bm b}$.
	
	We know from Lemma~\ref{lem:baiw} \begin{equation}\label{eq:argmin} \min_{\bm{\lambda}\in Alt_{\bm b}}\sum_i  w_i(\bm{b})\frac{(\mu'_i-\lambda_i)^2}{2\sigma^2} = w_a(\bm b)\frac{(\mu'_a-\mu'_{aj})^2}{2\sigma^2}+w_j(\bm b)\frac{(\mu'_j-\mu'_{aj})^2}{2\sigma^2}\end{equation} for some $a\leq k<k+1\leq j$, and $\mu'_{aj}= \frac{w_a(\bm{b})}{w_a(\bm{b})+w_j(\bm{b})}\mu'_a + \frac{w_j(\bm{b})}{w_a(\bm{b})+w_j(\bm{b})}\mu'_j$. 
	

	\begin{itemize}
		\item If $\mu'_{aj}\in (b_a,\mu'_a)$, then \begin{align*}
			w_a(\bm{b}) (\mu'_a-\mu'_{aj})^2+w_j(\bm{b}) (\mu'_j-\mu'_{aj})^2 &\geq 0+w_j(\bm{b}) (\mu'_j-b_a)^2 \\
			&\geq w_a(\bm{b})(b_a-b_a)^2+w_j(\bm{b})(b_j-b_a)^2\\
		\end{align*}
		\item If $\mu'_{aj}\in (b_j,b_a)$, \begin{align*}
			w_a(\bm{b}) (\mu'_a-\mu'_{aj})^2+w_j(\bm{b}) (\mu'_j-\mu'_{aj})^2 &\geq w_a(\bm{b})(b_a-\mu'_{aj})^2 +w_j(\bm{b})(b_j-\mu'_{aj})^2
		\end{align*}
		\item If $\mu'_{aj} \in (\mu'_j,b_j)$, \begin{align*} w_a(\bm{b}) (\mu'_a-\mu'_{aj})^2+w_j(\bm{b}) (\mu'_j-\mu'_{aj})^2 &\geq w_a(\bm{b})(\mu'_a-b_j)^2+0 \\
			&\geq w_a(\bm{b}) (b_a-b_j)^2 +w_j(\bm{b})(b_j-b_j)^2\end{align*}
	\end{itemize}
	In all three cases, \begin{align*} w_a(\bm{b}) (\mu'_a-\mu'_{aj})^2+w_j(\bm{b}) (\mu'_j-\mu'_{aj})^2 &\geq \inf_{\lambda\in [b_j,b_a]} w_a(\bm{b}) (b_a-\lambda)^2+w_j(\bm{b})(b_j-\lambda)^2\\
		&\geq \inf_{\bm{\lambda}\in Alt_{\bm b}} \sum_i w_i(\bm{b}) (b_i-\lambda_i)^2
	\end{align*}
	and therefore, by Equation~\eqref{eq:argmin}, \[\forall \bm\nu' \in B_{\infty}(\bm\mu, \varepsilon),\; \inf_{\bm{\lambda}\in Alt_{\bm b}} \sum_i w_i(\bm{b}) \frac{(\nu_i-\lambda_i)^2}{2\sigma^2}\geq \inf_{\bm{\lambda}\in Alt_{\bm b}} \sum_i w_i(\bm{b}) \frac{(b_i-\lambda_i)^2}{2\sigma^2} \] and therefore \begin{equation} \inf_{\bm\nu\in\mathcal B_{\infty}(\bm\mu, \varepsilon)}\inf_{\bm{\lambda}\in Alt_{\bm b}} \sum_i w_i(\bm{b}) \frac{(\nu_i-\lambda_i)^2}{2\sigma^2}\geq \inf_{\bm{\lambda}\in Alt_{\bm b}} \sum_i w_i(\bm{b}) \frac{(b_i-\lambda_i)^2}{2\sigma^2}\end{equation}
	
We can thus apply Lemma~\ref{lem:wbwst}, and conclude

\[\overline{T}(\mathcal B_{\infty}(\bm\mu, \varepsilon)) = \max_{\bm\nu\in \mathcal B_{\infty}(\bm\mu, \varepsilon)} T^\star(\bm\nu) \: .\]
\end{proof}









\begin{proof}[Proof of Lemma~\ref{lem:constrBbai} for thresholding bandits]
	If for some $i$, $|\mu_i -\tau|\leq \varepsilon$, then there exists $\bm b\in \mathcal{B}_\infty(\bm\mu,\varepsilon)$ such that $b_i = \tau$. Therefore, $\overline{T}^\star(\mathcal{B}_\infty(\bm\mu,\varepsilon)) \geq \max_{\bm\nu'\in \mathcal{B}_\infty(\bm\mu,\varepsilon)} T^\star(\bm\nu')\geq T^\star(\bm b) =+\infty$, therefore \[\overline{T}^\star(\mathcal{B}_\infty(\bm\mu,\varepsilon)) = \max_{\bm\nu'\in \mathcal{B}_\infty(\bm\mu,\varepsilon)}T^\star(\bm\nu') \: .\]
	
	When $\min_k |\mu_k-\tau|>\varepsilon$, define $U=\{i\in[K]:\mu_i >\tau\}$ and $L=[K]\setminus U$. Define \[\left\{ \begin{aligned} &b_i= \mu_i-\varepsilon &\text{if }i\in U \\ &b_i =\mu_i +\varepsilon &\text{if }i\in L.\end{aligned}\right. \] and for any $\bm\nu'$, $w_{\bm\nu'} =\argmax_{w\in \Sigma_K} \inf_{\bm\lambda \in Alt_{\bm\nu'}} \sum_i w_i\frac{(\mu_i'-\lambda_i)^2}{2\sigma^2}$.
	
	
	
	
	Let there be some $\bm\nu\in \mathcal{B}_\infty(\bm\mu,\varepsilon)$.	We know $\min_{\bm\lambda \in Alt_{\bm b}} \sum_i w_i(\bm b)\frac{(\mu'_i-\lambda_i)^2}{2\sigma^2}=w_j(\bm b) \frac{(\mu'_j-\tau)^2}{2\sigma^2}$ for some $j$. 
	
	For all $i\in U$, $\mu'_i \geq \mu_i-\varepsilon = b_i>\tau$; for all $i\in L$, $\mu'_i \leq \mu_i+\varepsilon = b_i<\tau$. Therefore, \[w_j(\bm b) \frac{(\mu'_j-\tau)^2}{2\sigma^2}\geq w_j(\bm b) \frac{(b_j-\tau)^2}{2\sigma^2}\geq \inf_{\bm\lambda \in Alt_{\bm b}} \sum_i w_i(\bm b) \frac{(b_i-\lambda_i)^2}{2\sigma^2}\]
	We thus have $\forall \bm\nu' \in \mathcal{B}_\infty(\bm\mu,\varepsilon)$, $\inf_{\bm\lambda\in Alt_{\bm b}} \sum_i w_i(\bm b)\frac{(\mu'_i-\lambda_i)^2}{2\sigma^2} \geq \inf_{\bm\lambda \in Alt_{\bm b}} \sum_i w_i(\bm b) \frac{(b_i-\lambda_i)^2}{2\sigma^2}$, and \[\inf_{\nu\in\hat{B}_r}\inf_{\bm\lambda\in Alt_{\bm b}} \sum_i w_i(\bm b)\frac{(\nu_i-\lambda_i)^2}{2\sigma^2} \geq \inf_{\bm\lambda \in Alt_{\bm b}} \sum_i w_i(\bm b) \frac{(b_i-\lambda_i)^2}{2\sigma^2}\] and by Lemma~\ref{lem:wbwst},
\[\overline{T}(\mathcal B_{\infty}(\bm\mu, \varepsilon)) = \max_{\bm\nu\in \mathcal B_{\infty}(\bm\mu, \varepsilon)} T^\star(\bm\nu) \: .\]
\end{proof}

%%%%%%%%%%%%%%%%%%%%%%%%%%%%%%%%%%%%%%%%%%%%%%%%%%%%%%%%%%%%%%%%%%%%%%%%%%%%%%%
%%%%%%%%%%%%%%%%%%%%%%%%%%%%%%%%%%%%%%%%%%%%%%%%%%%%%%%%%%%%%%%%%%%%%%%%%%%%%%%


\end{document}



\section*{Acknowledgement}

This work was supported by NSFC (No.62303319), Shanghai Local College Capacity Building Program (23010503100), ShanghaiTech AI4S Initiative SHTAI4S202404, HPC Platform of ShanghaiTech University, and MoE Key Laboratory of Intelligent Perception and Human-Machine Collaboration (ShanghaiTech University).


\section*{Impact Statement}
There are many potential societal consequences of our work, none of which we feel must be specifically highlighted here. 



\bibliography{example_paper}
\bibliographystyle{icml2025}




\subsection{Lloyd-Max Algorithm}
\label{subsec:Lloyd-Max}
For a given quantization bitwidth $B$ and an operand $\bm{X}$, the Lloyd-Max algorithm finds $2^B$ quantization levels $\{\hat{x}_i\}_{i=1}^{2^B}$ such that quantizing $\bm{X}$ by rounding each scalar in $\bm{X}$ to the nearest quantization level minimizes the quantization MSE. 

The algorithm starts with an initial guess of quantization levels and then iteratively computes quantization thresholds $\{\tau_i\}_{i=1}^{2^B-1}$ and updates quantization levels $\{\hat{x}_i\}_{i=1}^{2^B}$. Specifically, at iteration $n$, thresholds are set to the midpoints of the previous iteration's levels:
\begin{align*}
    \tau_i^{(n)}=\frac{\hat{x}_i^{(n-1)}+\hat{x}_{i+1}^{(n-1)}}2 \text{ for } i=1\ldots 2^B-1
\end{align*}
Subsequently, the quantization levels are re-computed as conditional means of the data regions defined by the new thresholds:
\begin{align*}
    \hat{x}_i^{(n)}=\mathbb{E}\left[ \bm{X} \big| \bm{X}\in [\tau_{i-1}^{(n)},\tau_i^{(n)}] \right] \text{ for } i=1\ldots 2^B
\end{align*}
where to satisfy boundary conditions we have $\tau_0=-\infty$ and $\tau_{2^B}=\infty$. The algorithm iterates the above steps until convergence.

Figure \ref{fig:lm_quant} compares the quantization levels of a $7$-bit floating point (E3M3) quantizer (left) to a $7$-bit Lloyd-Max quantizer (right) when quantizing a layer of weights from the GPT3-126M model at a per-tensor granularity. As shown, the Lloyd-Max quantizer achieves substantially lower quantization MSE. Further, Table \ref{tab:FP7_vs_LM7} shows the superior perplexity achieved by Lloyd-Max quantizers for bitwidths of $7$, $6$ and $5$. The difference between the quantizers is clear at 5 bits, where per-tensor FP quantization incurs a drastic and unacceptable increase in perplexity, while Lloyd-Max quantization incurs a much smaller increase. Nevertheless, we note that even the optimal Lloyd-Max quantizer incurs a notable ($\sim 1.5$) increase in perplexity due to the coarse granularity of quantization. 

\begin{figure}[h]
  \centering
  \includegraphics[width=0.7\linewidth]{sections/figures/LM7_FP7.pdf}
  \caption{\small Quantization levels and the corresponding quantization MSE of Floating Point (left) vs Lloyd-Max (right) Quantizers for a layer of weights in the GPT3-126M model.}
  \label{fig:lm_quant}
\end{figure}

\begin{table}[h]\scriptsize
\begin{center}
\caption{\label{tab:FP7_vs_LM7} \small Comparing perplexity (lower is better) achieved by floating point quantizers and Lloyd-Max quantizers on a GPT3-126M model for the Wikitext-103 dataset.}
\begin{tabular}{c|cc|c}
\hline
 \multirow{2}{*}{\textbf{Bitwidth}} & \multicolumn{2}{|c|}{\textbf{Floating-Point Quantizer}} & \textbf{Lloyd-Max Quantizer} \\
 & Best Format & Wikitext-103 Perplexity & Wikitext-103 Perplexity \\
\hline
7 & E3M3 & 18.32 & 18.27 \\
6 & E3M2 & 19.07 & 18.51 \\
5 & E4M0 & 43.89 & 19.71 \\
\hline
\end{tabular}
\end{center}
\end{table}

\subsection{Proof of Local Optimality of LO-BCQ}
\label{subsec:lobcq_opt_proof}
For a given block $\bm{b}_j$, the quantization MSE during LO-BCQ can be empirically evaluated as $\frac{1}{L_b}\lVert \bm{b}_j- \bm{\hat{b}}_j\rVert^2_2$ where $\bm{\hat{b}}_j$ is computed from equation (\ref{eq:clustered_quantization_definition}) as $C_{f(\bm{b}_j)}(\bm{b}_j)$. Further, for a given block cluster $\mathcal{B}_i$, we compute the quantization MSE as $\frac{1}{|\mathcal{B}_{i}|}\sum_{\bm{b} \in \mathcal{B}_{i}} \frac{1}{L_b}\lVert \bm{b}- C_i^{(n)}(\bm{b})\rVert^2_2$. Therefore, at the end of iteration $n$, we evaluate the overall quantization MSE $J^{(n)}$ for a given operand $\bm{X}$ composed of $N_c$ block clusters as:
\begin{align*}
    \label{eq:mse_iter_n}
    J^{(n)} = \frac{1}{N_c} \sum_{i=1}^{N_c} \frac{1}{|\mathcal{B}_{i}^{(n)}|}\sum_{\bm{v} \in \mathcal{B}_{i}^{(n)}} \frac{1}{L_b}\lVert \bm{b}- B_i^{(n)}(\bm{b})\rVert^2_2
\end{align*}

At the end of iteration $n$, the codebooks are updated from $\mathcal{C}^{(n-1)}$ to $\mathcal{C}^{(n)}$. However, the mapping of a given vector $\bm{b}_j$ to quantizers $\mathcal{C}^{(n)}$ remains as  $f^{(n)}(\bm{b}_j)$. At the next iteration, during the vector clustering step, $f^{(n+1)}(\bm{b}_j)$ finds new mapping of $\bm{b}_j$ to updated codebooks $\mathcal{C}^{(n)}$ such that the quantization MSE over the candidate codebooks is minimized. Therefore, we obtain the following result for $\bm{b}_j$:
\begin{align*}
\frac{1}{L_b}\lVert \bm{b}_j - C_{f^{(n+1)}(\bm{b}_j)}^{(n)}(\bm{b}_j)\rVert^2_2 \le \frac{1}{L_b}\lVert \bm{b}_j - C_{f^{(n)}(\bm{b}_j)}^{(n)}(\bm{b}_j)\rVert^2_2
\end{align*}

That is, quantizing $\bm{b}_j$ at the end of the block clustering step of iteration $n+1$ results in lower quantization MSE compared to quantizing at the end of iteration $n$. Since this is true for all $\bm{b} \in \bm{X}$, we assert the following:
\begin{equation}
\begin{split}
\label{eq:mse_ineq_1}
    \tilde{J}^{(n+1)} &= \frac{1}{N_c} \sum_{i=1}^{N_c} \frac{1}{|\mathcal{B}_{i}^{(n+1)}|}\sum_{\bm{b} \in \mathcal{B}_{i}^{(n+1)}} \frac{1}{L_b}\lVert \bm{b} - C_i^{(n)}(b)\rVert^2_2 \le J^{(n)}
\end{split}
\end{equation}
where $\tilde{J}^{(n+1)}$ is the the quantization MSE after the vector clustering step at iteration $n+1$.

Next, during the codebook update step (\ref{eq:quantizers_update}) at iteration $n+1$, the per-cluster codebooks $\mathcal{C}^{(n)}$ are updated to $\mathcal{C}^{(n+1)}$ by invoking the Lloyd-Max algorithm \citep{Lloyd}. We know that for any given value distribution, the Lloyd-Max algorithm minimizes the quantization MSE. Therefore, for a given vector cluster $\mathcal{B}_i$ we obtain the following result:

\begin{equation}
    \frac{1}{|\mathcal{B}_{i}^{(n+1)}|}\sum_{\bm{b} \in \mathcal{B}_{i}^{(n+1)}} \frac{1}{L_b}\lVert \bm{b}- C_i^{(n+1)}(\bm{b})\rVert^2_2 \le \frac{1}{|\mathcal{B}_{i}^{(n+1)}|}\sum_{\bm{b} \in \mathcal{B}_{i}^{(n+1)}} \frac{1}{L_b}\lVert \bm{b}- C_i^{(n)}(\bm{b})\rVert^2_2
\end{equation}

The above equation states that quantizing the given block cluster $\mathcal{B}_i$ after updating the associated codebook from $C_i^{(n)}$ to $C_i^{(n+1)}$ results in lower quantization MSE. Since this is true for all the block clusters, we derive the following result: 
\begin{equation}
\begin{split}
\label{eq:mse_ineq_2}
     J^{(n+1)} &= \frac{1}{N_c} \sum_{i=1}^{N_c} \frac{1}{|\mathcal{B}_{i}^{(n+1)}|}\sum_{\bm{b} \in \mathcal{B}_{i}^{(n+1)}} \frac{1}{L_b}\lVert \bm{b}- C_i^{(n+1)}(\bm{b})\rVert^2_2  \le \tilde{J}^{(n+1)}   
\end{split}
\end{equation}

Following (\ref{eq:mse_ineq_1}) and (\ref{eq:mse_ineq_2}), we find that the quantization MSE is non-increasing for each iteration, that is, $J^{(1)} \ge J^{(2)} \ge J^{(3)} \ge \ldots \ge J^{(M)}$ where $M$ is the maximum number of iterations. 
%Therefore, we can say that if the algorithm converges, then it must be that it has converged to a local minimum. 
\hfill $\blacksquare$


\begin{figure}
    \begin{center}
    \includegraphics[width=0.5\textwidth]{sections//figures/mse_vs_iter.pdf}
    \end{center}
    \caption{\small NMSE vs iterations during LO-BCQ compared to other block quantization proposals}
    \label{fig:nmse_vs_iter}
\end{figure}

Figure \ref{fig:nmse_vs_iter} shows the empirical convergence of LO-BCQ across several block lengths and number of codebooks. Also, the MSE achieved by LO-BCQ is compared to baselines such as MXFP and VSQ. As shown, LO-BCQ converges to a lower MSE than the baselines. Further, we achieve better convergence for larger number of codebooks ($N_c$) and for a smaller block length ($L_b$), both of which increase the bitwidth of BCQ (see Eq \ref{eq:bitwidth_bcq}).


\subsection{Additional Accuracy Results}
%Table \ref{tab:lobcq_config} lists the various LOBCQ configurations and their corresponding bitwidths.
\begin{table}
\setlength{\tabcolsep}{4.75pt}
\begin{center}
\caption{\label{tab:lobcq_config} Various LO-BCQ configurations and their bitwidths.}
\begin{tabular}{|c||c|c|c|c||c|c||c|} 
\hline
 & \multicolumn{4}{|c||}{$L_b=8$} & \multicolumn{2}{|c||}{$L_b=4$} & $L_b=2$ \\
 \hline
 \backslashbox{$L_A$\kern-1em}{\kern-1em$N_c$} & 2 & 4 & 8 & 16 & 2 & 4 & 2 \\
 \hline
 64 & 4.25 & 4.375 & 4.5 & 4.625 & 4.375 & 4.625 & 4.625\\
 \hline
 32 & 4.375 & 4.5 & 4.625& 4.75 & 4.5 & 4.75 & 4.75 \\
 \hline
 16 & 4.625 & 4.75& 4.875 & 5 & 4.75 & 5 & 5 \\
 \hline
\end{tabular}
\end{center}
\end{table}

%\subsection{Perplexity achieved by various LO-BCQ configurations on Wikitext-103 dataset}

\begin{table} \centering
\begin{tabular}{|c||c|c|c|c||c|c||c|} 
\hline
 $L_b \rightarrow$& \multicolumn{4}{c||}{8} & \multicolumn{2}{c||}{4} & 2\\
 \hline
 \backslashbox{$L_A$\kern-1em}{\kern-1em$N_c$} & 2 & 4 & 8 & 16 & 2 & 4 & 2  \\
 %$N_c \rightarrow$ & 2 & 4 & 8 & 16 & 2 & 4 & 2 \\
 \hline
 \hline
 \multicolumn{8}{c}{GPT3-1.3B (FP32 PPL = 9.98)} \\ 
 \hline
 \hline
 64 & 10.40 & 10.23 & 10.17 & 10.15 &  10.28 & 10.18 & 10.19 \\
 \hline
 32 & 10.25 & 10.20 & 10.15 & 10.12 &  10.23 & 10.17 & 10.17 \\
 \hline
 16 & 10.22 & 10.16 & 10.10 & 10.09 &  10.21 & 10.14 & 10.16 \\
 \hline
  \hline
 \multicolumn{8}{c}{GPT3-8B (FP32 PPL = 7.38)} \\ 
 \hline
 \hline
 64 & 7.61 & 7.52 & 7.48 &  7.47 &  7.55 &  7.49 & 7.50 \\
 \hline
 32 & 7.52 & 7.50 & 7.46 &  7.45 &  7.52 &  7.48 & 7.48  \\
 \hline
 16 & 7.51 & 7.48 & 7.44 &  7.44 &  7.51 &  7.49 & 7.47  \\
 \hline
\end{tabular}
\caption{\label{tab:ppl_gpt3_abalation} Wikitext-103 perplexity across GPT3-1.3B and 8B models.}
\end{table}

\begin{table} \centering
\begin{tabular}{|c||c|c|c|c||} 
\hline
 $L_b \rightarrow$& \multicolumn{4}{c||}{8}\\
 \hline
 \backslashbox{$L_A$\kern-1em}{\kern-1em$N_c$} & 2 & 4 & 8 & 16 \\
 %$N_c \rightarrow$ & 2 & 4 & 8 & 16 & 2 & 4 & 2 \\
 \hline
 \hline
 \multicolumn{5}{|c|}{Llama2-7B (FP32 PPL = 5.06)} \\ 
 \hline
 \hline
 64 & 5.31 & 5.26 & 5.19 & 5.18  \\
 \hline
 32 & 5.23 & 5.25 & 5.18 & 5.15  \\
 \hline
 16 & 5.23 & 5.19 & 5.16 & 5.14  \\
 \hline
 \multicolumn{5}{|c|}{Nemotron4-15B (FP32 PPL = 5.87)} \\ 
 \hline
 \hline
 64  & 6.3 & 6.20 & 6.13 & 6.08  \\
 \hline
 32  & 6.24 & 6.12 & 6.07 & 6.03  \\
 \hline
 16  & 6.12 & 6.14 & 6.04 & 6.02  \\
 \hline
 \multicolumn{5}{|c|}{Nemotron4-340B (FP32 PPL = 3.48)} \\ 
 \hline
 \hline
 64 & 3.67 & 3.62 & 3.60 & 3.59 \\
 \hline
 32 & 3.63 & 3.61 & 3.59 & 3.56 \\
 \hline
 16 & 3.61 & 3.58 & 3.57 & 3.55 \\
 \hline
\end{tabular}
\caption{\label{tab:ppl_llama7B_nemo15B} Wikitext-103 perplexity compared to FP32 baseline in Llama2-7B and Nemotron4-15B, 340B models}
\end{table}

%\subsection{Perplexity achieved by various LO-BCQ configurations on MMLU dataset}


\begin{table} \centering
\begin{tabular}{|c||c|c|c|c||c|c|c|c|} 
\hline
 $L_b \rightarrow$& \multicolumn{4}{c||}{8} & \multicolumn{4}{c||}{8}\\
 \hline
 \backslashbox{$L_A$\kern-1em}{\kern-1em$N_c$} & 2 & 4 & 8 & 16 & 2 & 4 & 8 & 16  \\
 %$N_c \rightarrow$ & 2 & 4 & 8 & 16 & 2 & 4 & 2 \\
 \hline
 \hline
 \multicolumn{5}{|c|}{Llama2-7B (FP32 Accuracy = 45.8\%)} & \multicolumn{4}{|c|}{Llama2-70B (FP32 Accuracy = 69.12\%)} \\ 
 \hline
 \hline
 64 & 43.9 & 43.4 & 43.9 & 44.9 & 68.07 & 68.27 & 68.17 & 68.75 \\
 \hline
 32 & 44.5 & 43.8 & 44.9 & 44.5 & 68.37 & 68.51 & 68.35 & 68.27  \\
 \hline
 16 & 43.9 & 42.7 & 44.9 & 45 & 68.12 & 68.77 & 68.31 & 68.59  \\
 \hline
 \hline
 \multicolumn{5}{|c|}{GPT3-22B (FP32 Accuracy = 38.75\%)} & \multicolumn{4}{|c|}{Nemotron4-15B (FP32 Accuracy = 64.3\%)} \\ 
 \hline
 \hline
 64 & 36.71 & 38.85 & 38.13 & 38.92 & 63.17 & 62.36 & 63.72 & 64.09 \\
 \hline
 32 & 37.95 & 38.69 & 39.45 & 38.34 & 64.05 & 62.30 & 63.8 & 64.33  \\
 \hline
 16 & 38.88 & 38.80 & 38.31 & 38.92 & 63.22 & 63.51 & 63.93 & 64.43  \\
 \hline
\end{tabular}
\caption{\label{tab:mmlu_abalation} Accuracy on MMLU dataset across GPT3-22B, Llama2-7B, 70B and Nemotron4-15B models.}
\end{table}


%\subsection{Perplexity achieved by various LO-BCQ configurations on LM evaluation harness}

\begin{table} \centering
\begin{tabular}{|c||c|c|c|c||c|c|c|c|} 
\hline
 $L_b \rightarrow$& \multicolumn{4}{c||}{8} & \multicolumn{4}{c||}{8}\\
 \hline
 \backslashbox{$L_A$\kern-1em}{\kern-1em$N_c$} & 2 & 4 & 8 & 16 & 2 & 4 & 8 & 16  \\
 %$N_c \rightarrow$ & 2 & 4 & 8 & 16 & 2 & 4 & 2 \\
 \hline
 \hline
 \multicolumn{5}{|c|}{Race (FP32 Accuracy = 37.51\%)} & \multicolumn{4}{|c|}{Boolq (FP32 Accuracy = 64.62\%)} \\ 
 \hline
 \hline
 64 & 36.94 & 37.13 & 36.27 & 37.13 & 63.73 & 62.26 & 63.49 & 63.36 \\
 \hline
 32 & 37.03 & 36.36 & 36.08 & 37.03 & 62.54 & 63.51 & 63.49 & 63.55  \\
 \hline
 16 & 37.03 & 37.03 & 36.46 & 37.03 & 61.1 & 63.79 & 63.58 & 63.33  \\
 \hline
 \hline
 \multicolumn{5}{|c|}{Winogrande (FP32 Accuracy = 58.01\%)} & \multicolumn{4}{|c|}{Piqa (FP32 Accuracy = 74.21\%)} \\ 
 \hline
 \hline
 64 & 58.17 & 57.22 & 57.85 & 58.33 & 73.01 & 73.07 & 73.07 & 72.80 \\
 \hline
 32 & 59.12 & 58.09 & 57.85 & 58.41 & 73.01 & 73.94 & 72.74 & 73.18  \\
 \hline
 16 & 57.93 & 58.88 & 57.93 & 58.56 & 73.94 & 72.80 & 73.01 & 73.94  \\
 \hline
\end{tabular}
\caption{\label{tab:mmlu_abalation} Accuracy on LM evaluation harness tasks on GPT3-1.3B model.}
\end{table}

\begin{table} \centering
\begin{tabular}{|c||c|c|c|c||c|c|c|c|} 
\hline
 $L_b \rightarrow$& \multicolumn{4}{c||}{8} & \multicolumn{4}{c||}{8}\\
 \hline
 \backslashbox{$L_A$\kern-1em}{\kern-1em$N_c$} & 2 & 4 & 8 & 16 & 2 & 4 & 8 & 16  \\
 %$N_c \rightarrow$ & 2 & 4 & 8 & 16 & 2 & 4 & 2 \\
 \hline
 \hline
 \multicolumn{5}{|c|}{Race (FP32 Accuracy = 41.34\%)} & \multicolumn{4}{|c|}{Boolq (FP32 Accuracy = 68.32\%)} \\ 
 \hline
 \hline
 64 & 40.48 & 40.10 & 39.43 & 39.90 & 69.20 & 68.41 & 69.45 & 68.56 \\
 \hline
 32 & 39.52 & 39.52 & 40.77 & 39.62 & 68.32 & 67.43 & 68.17 & 69.30  \\
 \hline
 16 & 39.81 & 39.71 & 39.90 & 40.38 & 68.10 & 66.33 & 69.51 & 69.42  \\
 \hline
 \hline
 \multicolumn{5}{|c|}{Winogrande (FP32 Accuracy = 67.88\%)} & \multicolumn{4}{|c|}{Piqa (FP32 Accuracy = 78.78\%)} \\ 
 \hline
 \hline
 64 & 66.85 & 66.61 & 67.72 & 67.88 & 77.31 & 77.42 & 77.75 & 77.64 \\
 \hline
 32 & 67.25 & 67.72 & 67.72 & 67.00 & 77.31 & 77.04 & 77.80 & 77.37  \\
 \hline
 16 & 68.11 & 68.90 & 67.88 & 67.48 & 77.37 & 78.13 & 78.13 & 77.69  \\
 \hline
\end{tabular}
\caption{\label{tab:mmlu_abalation} Accuracy on LM evaluation harness tasks on GPT3-8B model.}
\end{table}

\begin{table} \centering
\begin{tabular}{|c||c|c|c|c||c|c|c|c|} 
\hline
 $L_b \rightarrow$& \multicolumn{4}{c||}{8} & \multicolumn{4}{c||}{8}\\
 \hline
 \backslashbox{$L_A$\kern-1em}{\kern-1em$N_c$} & 2 & 4 & 8 & 16 & 2 & 4 & 8 & 16  \\
 %$N_c \rightarrow$ & 2 & 4 & 8 & 16 & 2 & 4 & 2 \\
 \hline
 \hline
 \multicolumn{5}{|c|}{Race (FP32 Accuracy = 40.67\%)} & \multicolumn{4}{|c|}{Boolq (FP32 Accuracy = 76.54\%)} \\ 
 \hline
 \hline
 64 & 40.48 & 40.10 & 39.43 & 39.90 & 75.41 & 75.11 & 77.09 & 75.66 \\
 \hline
 32 & 39.52 & 39.52 & 40.77 & 39.62 & 76.02 & 76.02 & 75.96 & 75.35  \\
 \hline
 16 & 39.81 & 39.71 & 39.90 & 40.38 & 75.05 & 73.82 & 75.72 & 76.09  \\
 \hline
 \hline
 \multicolumn{5}{|c|}{Winogrande (FP32 Accuracy = 70.64\%)} & \multicolumn{4}{|c|}{Piqa (FP32 Accuracy = 79.16\%)} \\ 
 \hline
 \hline
 64 & 69.14 & 70.17 & 70.17 & 70.56 & 78.24 & 79.00 & 78.62 & 78.73 \\
 \hline
 32 & 70.96 & 69.69 & 71.27 & 69.30 & 78.56 & 79.49 & 79.16 & 78.89  \\
 \hline
 16 & 71.03 & 69.53 & 69.69 & 70.40 & 78.13 & 79.16 & 79.00 & 79.00  \\
 \hline
\end{tabular}
\caption{\label{tab:mmlu_abalation} Accuracy on LM evaluation harness tasks on GPT3-22B model.}
\end{table}

\begin{table} \centering
\begin{tabular}{|c||c|c|c|c||c|c|c|c|} 
\hline
 $L_b \rightarrow$& \multicolumn{4}{c||}{8} & \multicolumn{4}{c||}{8}\\
 \hline
 \backslashbox{$L_A$\kern-1em}{\kern-1em$N_c$} & 2 & 4 & 8 & 16 & 2 & 4 & 8 & 16  \\
 %$N_c \rightarrow$ & 2 & 4 & 8 & 16 & 2 & 4 & 2 \\
 \hline
 \hline
 \multicolumn{5}{|c|}{Race (FP32 Accuracy = 44.4\%)} & \multicolumn{4}{|c|}{Boolq (FP32 Accuracy = 79.29\%)} \\ 
 \hline
 \hline
 64 & 42.49 & 42.51 & 42.58 & 43.45 & 77.58 & 77.37 & 77.43 & 78.1 \\
 \hline
 32 & 43.35 & 42.49 & 43.64 & 43.73 & 77.86 & 75.32 & 77.28 & 77.86  \\
 \hline
 16 & 44.21 & 44.21 & 43.64 & 42.97 & 78.65 & 77 & 76.94 & 77.98  \\
 \hline
 \hline
 \multicolumn{5}{|c|}{Winogrande (FP32 Accuracy = 69.38\%)} & \multicolumn{4}{|c|}{Piqa (FP32 Accuracy = 78.07\%)} \\ 
 \hline
 \hline
 64 & 68.9 & 68.43 & 69.77 & 68.19 & 77.09 & 76.82 & 77.09 & 77.86 \\
 \hline
 32 & 69.38 & 68.51 & 68.82 & 68.90 & 78.07 & 76.71 & 78.07 & 77.86  \\
 \hline
 16 & 69.53 & 67.09 & 69.38 & 68.90 & 77.37 & 77.8 & 77.91 & 77.69  \\
 \hline
\end{tabular}
\caption{\label{tab:mmlu_abalation} Accuracy on LM evaluation harness tasks on Llama2-7B model.}
\end{table}

\begin{table} \centering
\begin{tabular}{|c||c|c|c|c||c|c|c|c|} 
\hline
 $L_b \rightarrow$& \multicolumn{4}{c||}{8} & \multicolumn{4}{c||}{8}\\
 \hline
 \backslashbox{$L_A$\kern-1em}{\kern-1em$N_c$} & 2 & 4 & 8 & 16 & 2 & 4 & 8 & 16  \\
 %$N_c \rightarrow$ & 2 & 4 & 8 & 16 & 2 & 4 & 2 \\
 \hline
 \hline
 \multicolumn{5}{|c|}{Race (FP32 Accuracy = 48.8\%)} & \multicolumn{4}{|c|}{Boolq (FP32 Accuracy = 85.23\%)} \\ 
 \hline
 \hline
 64 & 49.00 & 49.00 & 49.28 & 48.71 & 82.82 & 84.28 & 84.03 & 84.25 \\
 \hline
 32 & 49.57 & 48.52 & 48.33 & 49.28 & 83.85 & 84.46 & 84.31 & 84.93  \\
 \hline
 16 & 49.85 & 49.09 & 49.28 & 48.99 & 85.11 & 84.46 & 84.61 & 83.94  \\
 \hline
 \hline
 \multicolumn{5}{|c|}{Winogrande (FP32 Accuracy = 79.95\%)} & \multicolumn{4}{|c|}{Piqa (FP32 Accuracy = 81.56\%)} \\ 
 \hline
 \hline
 64 & 78.77 & 78.45 & 78.37 & 79.16 & 81.45 & 80.69 & 81.45 & 81.5 \\
 \hline
 32 & 78.45 & 79.01 & 78.69 & 80.66 & 81.56 & 80.58 & 81.18 & 81.34  \\
 \hline
 16 & 79.95 & 79.56 & 79.79 & 79.72 & 81.28 & 81.66 & 81.28 & 80.96  \\
 \hline
\end{tabular}
\caption{\label{tab:mmlu_abalation} Accuracy on LM evaluation harness tasks on Llama2-70B model.}
\end{table}

%\section{MSE Studies}
%\textcolor{red}{TODO}


\subsection{Number Formats and Quantization Method}
\label{subsec:numFormats_quantMethod}
\subsubsection{Integer Format}
An $n$-bit signed integer (INT) is typically represented with a 2s-complement format \citep{yao2022zeroquant,xiao2023smoothquant,dai2021vsq}, where the most significant bit denotes the sign.

\subsubsection{Floating Point Format}
An $n$-bit signed floating point (FP) number $x$ comprises of a 1-bit sign ($x_{\mathrm{sign}}$), $B_m$-bit mantissa ($x_{\mathrm{mant}}$) and $B_e$-bit exponent ($x_{\mathrm{exp}}$) such that $B_m+B_e=n-1$. The associated constant exponent bias ($E_{\mathrm{bias}}$) is computed as $(2^{{B_e}-1}-1)$. We denote this format as $E_{B_e}M_{B_m}$.  

\subsubsection{Quantization Scheme}
\label{subsec:quant_method}
A quantization scheme dictates how a given unquantized tensor is converted to its quantized representation. We consider FP formats for the purpose of illustration. Given an unquantized tensor $\bm{X}$ and an FP format $E_{B_e}M_{B_m}$, we first, we compute the quantization scale factor $s_X$ that maps the maximum absolute value of $\bm{X}$ to the maximum quantization level of the $E_{B_e}M_{B_m}$ format as follows:
\begin{align}
\label{eq:sf}
    s_X = \frac{\mathrm{max}(|\bm{X}|)}{\mathrm{max}(E_{B_e}M_{B_m})}
\end{align}
In the above equation, $|\cdot|$ denotes the absolute value function.

Next, we scale $\bm{X}$ by $s_X$ and quantize it to $\hat{\bm{X}}$ by rounding it to the nearest quantization level of $E_{B_e}M_{B_m}$ as:

\begin{align}
\label{eq:tensor_quant}
    \hat{\bm{X}} = \text{round-to-nearest}\left(\frac{\bm{X}}{s_X}, E_{B_e}M_{B_m}\right)
\end{align}

We perform dynamic max-scaled quantization \citep{wu2020integer}, where the scale factor $s$ for activations is dynamically computed during runtime.

\subsection{Vector Scaled Quantization}
\begin{wrapfigure}{r}{0.35\linewidth}
  \centering
  \includegraphics[width=\linewidth]{sections/figures/vsquant.jpg}
  \caption{\small Vectorwise decomposition for per-vector scaled quantization (VSQ \citep{dai2021vsq}).}
  \label{fig:vsquant}
\end{wrapfigure}
During VSQ \citep{dai2021vsq}, the operand tensors are decomposed into 1D vectors in a hardware friendly manner as shown in Figure \ref{fig:vsquant}. Since the decomposed tensors are used as operands in matrix multiplications during inference, it is beneficial to perform this decomposition along the reduction dimension of the multiplication. The vectorwise quantization is performed similar to tensorwise quantization described in Equations \ref{eq:sf} and \ref{eq:tensor_quant}, where a scale factor $s_v$ is required for each vector $\bm{v}$ that maps the maximum absolute value of that vector to the maximum quantization level. While smaller vector lengths can lead to larger accuracy gains, the associated memory and computational overheads due to the per-vector scale factors increases. To alleviate these overheads, VSQ \citep{dai2021vsq} proposed a second level quantization of the per-vector scale factors to unsigned integers, while MX \citep{rouhani2023shared} quantizes them to integer powers of 2 (denoted as $2^{INT}$).

\subsubsection{MX Format}
The MX format proposed in \citep{rouhani2023microscaling} introduces the concept of sub-block shifting. For every two scalar elements of $b$-bits each, there is a shared exponent bit. The value of this exponent bit is determined through an empirical analysis that targets minimizing quantization MSE. We note that the FP format $E_{1}M_{b}$ is strictly better than MX from an accuracy perspective since it allocates a dedicated exponent bit to each scalar as opposed to sharing it across two scalars. Therefore, we conservatively bound the accuracy of a $b+2$-bit signed MX format with that of a $E_{1}M_{b}$ format in our comparisons. For instance, we use E1M2 format as a proxy for MX4.

\begin{figure}
    \centering
    \includegraphics[width=1\linewidth]{sections//figures/BlockFormats.pdf}
    \caption{\small Comparing LO-BCQ to MX format.}
    \label{fig:block_formats}
\end{figure}

Figure \ref{fig:block_formats} compares our $4$-bit LO-BCQ block format to MX \citep{rouhani2023microscaling}. As shown, both LO-BCQ and MX decompose a given operand tensor into block arrays and each block array into blocks. Similar to MX, we find that per-block quantization ($L_b < L_A$) leads to better accuracy due to increased flexibility. While MX achieves this through per-block $1$-bit micro-scales, we associate a dedicated codebook to each block through a per-block codebook selector. Further, MX quantizes the per-block array scale-factor to E8M0 format without per-tensor scaling. In contrast during LO-BCQ, we find that per-tensor scaling combined with quantization of per-block array scale-factor to E4M3 format results in superior inference accuracy across models. 







%%%%%%%%%%%%%%%%%%%%%%%%%%%%%%%%%%%%%%%%%%%%%%%%%%%%%%%%%%%%%%%%%%%%%%%%%%%%%%%
%%%%%%%%%%%%%%%%%%%%%%%%%%%%%%%%%%%%%%%%%%%%%%%%%%%%%%%%%%%%%%%%%%%%%%%%%%%%%%%


\end{document}


% This document was modified from the file originally made available by
% Pat Langley and Andrea Danyluk for ICML-2K. This version was created
% by Iain Murray in 2018, and modified by Alexandre Bouchard in
% 2019 and 2021 and by Csaba Szepesvari, Gang Niu and Sivan Sabato in 2022.
% Modified again in 2023 and 2024 by Sivan Sabato and Jonathan Scarlett.
% Previous contributors include Dan Roy, Lise Getoor and Tobias
% Scheffer, which was slightly modified from the 2010 version by
% Thorsten Joachims & Johannes Fuernkranz, slightly modified from the
% 2009 version by Kiri Wagstaff and Sam Roweis's 2008 version, which is
% slightly modified from Prasad Tadepalli's 2007 version which is a
% lightly changed version of the previous year's version by Andrew
% Moore, which was in turn edited from those of Kristian Kersting and
% Codrina Lauth. Alex Smola contributed to the algorithmic style files.




