\section{Related Work}
\label{sec:2}

\subsection{Uncanny Valley in Humanoid Robots}

First introduced by Masahiro Mori in 1970 ~\cite{mori2012uncanny}, the uncanny valley describes a psychological phenomenon in which human-like entities envoke feelings of discomfort, eeriness, or creepiness when their appearance and behavior approach, but do not fully achieve human-likeness. The uncanny valley effect has been one of the main drawbacks in fields such as human-robot interaction (HRI).

Studies have explored various aspects of this phenomenon, including visual appearance, speech characteristics, motion dynamics, emotional expression and interactional characteristics in eliciting uncanny feelings ~\cite{yam2021robots, mishra2022uncanny, clark2021exploring, krauter2024bridging, thepsoonthorn2021exploration, spitale2024appropriateness, skjuve2019help}. The more anthropomorphized appearance the robot has, the higher the chance the robot has to be liked by users ~\cite{yam2021robots, mishra2022uncanny}. Speech characteristics such as politeness can enhance user ratings of a system's likeability ~\cite{clark2021exploring} and more natural and human-like voices are preferred by users ~\cite{krauter2024bridging}. Spitale et al. ~\cite{spitale2024appropriateness} found that the LLM-equipped robotic well-being coach is not empathic and not able to emotionally understand their experiences by highlighting the need to put the psychological aspects at the center of the interaction. Conversation content and flow clearly influence how pleasant people perceive their conversation with chatbot ~\cite{skjuve2019help}.

To address the uncanny valley, researchers and designers have adopted several strategies. One common approach is to intentionally design robots and avatars with stylized or non-human appearances, thereby avoiding the valley altogether ~\cite{schwind2018avoiding}. Another approach involves improving the realism and fluidity of movement and facial expressions to better align with human expectations ~\cite{koschate2016overcoming}. Advances in motion capture, facial animation, and natural language processing (NLP) have helped reduce the perception of eeriness in some contexts.

However, while these methods have shown promise, they often focus on physical or visual improvements, overlooking the potential role of advanced conversational capabilities in mitigating the uncanny valley effect. Despite significant progress in improving the appearance and motion of humanoid robots, there is still limited research on the impact of advanced conversational skills on reducing the uncanny valley effect. This presents a gap in the literature, as the ability to engage in natural, context-aware conversations may significantly influence users' emotional responses to humanoid robots.


\subsection{Large Language Models and Human-Robot Interaction}

Large Language Models (LLMs), such as OpenAI's GPT series ~\cite{achiam2023gpt} or Meta's Llama series ~\cite{touvron2023llama}, are computational frameworks trained on vast datasets of text to generate human-like responses to prompts. In the Human-Robot Interaction (HRI) domain, LLMs enable robots to engage in sophisticated dialogues, potentially enhancing their ability to act as social agents and improving their ability to align with human expectations in collaborative and social contexts.

The application of LLMs in HRI has gained considerable attention in recent years. Recent studies have explored the potential of LLMs in enhancing human-robot interaction, demonstrating their ability to improve natural language understanding and generation in robotic systems ~\cite{spitale2023vita, tanneberg2024help, wang2024lami, feng2024large}. For instance, Spitale et al. ~\cite{spitale2023vita} integrated an LLM module into a robotic system to imitate human coach behaviors. Similarly, Tanneberg et al. ~\cite{tanneberg2024help} leveraged common-sense reasoning capabilities of LLMs to determine when and how to assist humans effectively, as well as when to remain silent to avoid disrupting group interactions. Feng et al. ~\cite{feng2024large} employed LLMs to solve complex tasks in human-agent collaboration scenarios, highlighting their versatility in diverse application domains.

Despite these advancements, significant gaps remain in the current literature. Much of the current work has centered on the functional capabilities of LLM-equipped robots, such as their proficiency in natural language understanding and generation. However, limited attention has been given to how humans perceive robots equipped with LLMs, particularly those with humanoid features. The gap is significant, as user perception is critical to the acceptance and success of social robots in real-world applications. Specifically, the impact of the uncanny valley effect remains an underexplored area in the context of LLM-enhanced humanoid robots. 

To address this research gap, our study investigates user perceptions of humanoid social robots equipped with LLMs. In particular, we examine how advanced language capabilities influence the uncanny valley effect and whether such integration can mitigate the discomfort commonly associated with interactions involving highly realistic, human-like robots.

\subsection{Nadine Social Robot}

% \begin{figure}[H]
%  \centering
%  \includegraphics[width=0.5\linewidth]{imgs/Nadine_appearance.jpg}
%  \caption{Nadine social robot new appearance. (I'll put a new photo)}
% \label{fig:Nadine_appearance}
% \end{figure}

Nadine stands out as one of the most realistic humanoid social robots, designed for social interactions with humans ~\cite{baka2017meet}. The robot's human-like facial expression and gestures are achieved through 27 channels embedded in its face and upper body. The hyper-realistic appearance of Nadine may evoke the uncanny valley effect, a phenomenon where robots that closely resemble humans can elicit discomfort in users. This has driven the development of sophisticated modules that enhance the human-like behavior of such robots ~\cite{ramanathan2019nadine}. This combination of hyper-realistic design and advanced capabilities position Nadine in a distinctive position within the uncanny valley.

Several studies have examined Nadine's human-like attributes and their impact on user perceptions during human-robot interactions ~\cite{mishra2019can, kim2019eliza, tulsulkar2021can, thalmann2021nadine}. For instance, Mishra et al. ~\cite{mishra2019can} and Kim et al. ~\cite{kim2019eliza} investigated Nadine's role as a customer service agent. Mishra et al. ~\cite{mishra2019can} reported positive user feedback, noting that Nadine's social behaviors were perceived as acceptable and pleasing, while highlighting the need for advancements in commonsense reasoning and dialogue systems as future areas for improvement. In contrast, Kim et al. ~\cite{kim2019eliza} observed that excessive anthropomorphism in consumer robots could negatively affect customer attitudes, triggering feelings of uncanniness. Other research has explored Nadine's potential as a companion for elderly individuals and as an interactive agent in museums, with study participants responding positively to these applications ~\cite{tulsulkar2021can, thalmann2021nadine}.

Despite these encouraging founding, Nadine's dialogue system has been recognized as an area requiring improvement to enhance human-robot interactions. The previous version employed the Artificial Linguistic Internet Computer Entitiy (A.L.I.C.E.) chatbot ~\cite{wallace2009anatomy}, a rule-based conversational agent built using the Artificial Intelligent Markup Language (AIML). This approach matches user inputs to predefined patterns in its AIML knowledge base and responds with corresponding templates. Its dependence on manually crafted rules limits scalability and contextual understanding. The advent of LLMs, such as ChatGPT, addresses these limitations by excelling at understanding nuanced conversational context, adapting to diverse domains, and dynamically learning through fine-tuning. Nadine's updated system incorporates an LLM-based dialogue framework, significantly enhancing her capability to engage in nuanced, context-aware conversations. Further details of this enhancement are provided in Section \ref{sec:3}.