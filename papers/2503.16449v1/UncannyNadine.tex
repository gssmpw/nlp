\documentclass{article}


\usepackage{PRIMEarxiv}

\usepackage[utf8]{inputenc} % allow utf-8 input
\usepackage[T1]{fontenc}    % use 8-bit T1 fonts
\usepackage{hyperref}       % hyperlinks
\usepackage{url}            % simple URL typesetting
\usepackage{booktabs}       % professional-quality tables
\usepackage{amsfonts}       % blackboard math symbols
\usepackage{nicefrac}       % compact symbols for 1/2, etc.
\usepackage{microtype}      % microtypography
\usepackage{lipsum}
\usepackage{fancyhdr}       % header
\usepackage{graphicx}       % graphics
\graphicspath{{media/}}     % organize your images and other figures under media/ folder
\usepackage{subcaption}
\usepackage{wrapfig}

%Header
\pagestyle{fancy}
\thispagestyle{empty}
\rhead{ \textit{ }} 

% Update your Headers here
% \fancyhead[LO]{Running Title for Header}
% \fancyhead[RE]{Firstauthor and Secondauthor} % Firstauthor et al. if more than 2 - must use \documentclass[twoside]{article}

%% Title
\title{Mitigating the Uncanny Valley Effect in Hyper-Realistic Robots: A Student-Centered Study on LLM-Driven Conversations
%%%% Cite as
%%%% Update your official citation here when published 
}

\author{
  Hangyeol Kang\textsuperscript{a}, {} {} {} Thiago Freitas dos Santos\textsuperscript{a,b}, {} {} {} Maher Ben Moussa\textsuperscript{a}, {} {} {} Nadia Magnenat-Thalmann\textsuperscript{a,b} \\
  \\
  \textsuperscript{a}Centre Universitaire d’Informatique,University of Geneva, Geneva, Switzerland
  \\
  \textsuperscript{b}MIRALab, Battelle, Building A 7, Route de Drize, CH-1227 Carouge, Geneva, Switzerland \\
  \\
  \texttt{\{hangyeol.kang, maher.benmoussa, nadia.thalmann\}@unige.ch} \\
  \texttt{thiago.freitas@unige.ch} \\
  %% examples of more authors
  %% \AND
  %% Coauthor \\
  %% Affiliation \\
  %% Address \\
  %% \texttt{email} \\
  %% \And
  %% Coauthor \\
  %% Affiliation \\
  %% Address \\
  %% \texttt{email} \\
  %% \And
  %% Coauthor \\
  %% Affiliation \\
  %% Address \\
  %% \texttt{email} \\
}


\begin{document}
\maketitle


\begin{abstract}
The uncanny valley effect poses a significant challenge in the development and acceptance of hyper-realistic social robots. This study investigates whether advanced conversational capabilities powered by large language models (LLMs) can mitigate this effect in highly anthropomorphic robots. We conducted a user study with 80 participants interacting with Nadine, a hyper-realistic humanoid robot equipped with LLM-driven communication skills. Through pre- and post-interaction surveys, we assessed changes in perceptions of uncanniness, conversational quality, and overall user experience. Our findings reveal that LLM-enhanced interactions significantly reduce feelings of eeriness while fostering more natural and engaging conversations. Additionally, we identify key factors influencing user acceptance, including conversational naturalness, human-likeness, and interestingness. Based on these insights, we propose design recommendations to enhance the appeal and acceptability of hyper-realistic robots in social contexts. This research contributes to the growing field of human-robot interaction by offering empirical evidence on the potential of LLMs to bridge the uncanny valley, with implications for the future development of social robots.

\end{abstract}

\keywords{Uncanny Valley \and hyper-realistic robots \and large language models \and human-robot interaction \and social robots}


\section{Introduction}
\label{sec:1}

Social robots have emerged as transformative tools in various sectors, including healthcare, education, and hospitality, due to their ability to interact with humans in socially meaningful ways~\cite{vishwakarma2024adoption}. These robots are designed not only to perform tasks but also to provide emotional support, foster social engagement, and enhance user experiences. For example, in healthcare, social robots assist elderly patients by offering companionship~\cite{ahmed2024human} and supporting mental well-being~\cite{laban2024building}, while in education, they serve as tutors or co-learners to promote cognitive and social development~\cite{donnermann2022social}. Their integration into daily life highlights their potential to reshape human-robot interactions (HRI) and address critical societal needs.

Among social robots, hyper-realistic humanoid robots stand out for their ability to mimic human appearance and behavior with striking accuracy. Unlike less anthropomorphic robots, hyper-realistic models like Nadine – a robot capable of maintaining eye contact, remembering past interactions, and simulating emotions – create opportunities for deeper emotional connections with users~\cite{mishra2022uncanny}. These robots are particularly promising in roles requiring empathy and personal interaction, such as caregiving or customer service~\cite{mishra2022nadine, thalmann2021nadine}. However, their heightened realism also introduces unique challenges.

A key issue with hyper-realistic social robots is the uncanny valley effect, a phenomenon where near-human likeness evokes discomfort or eeriness instead of familiarity~\cite{mori2012uncanny}. This effect is often attributed to cognitive dissonance arising from mismatches between a robot’s human-like appearance and its imperfect behavior or capabilities~\cite{zhang2020literature}. To address this issue, some researchers suggest designing robots with deliberately non-human features to avoid triggering the uncanny valley~\cite{schwind2018avoiding, carey2024unmasking}. While effective in reducing initial discomfort, this approach limits the potential for creating meaningful social bonds, as users may struggle to connect emotionally with less humanlike robots~\cite{zhang2023emotional}.

An alternative strategy involves enhancing the interaction quality of hyper-realistic robots to align with user expectations. Recent advancements in LLMs, such as GPT-4~\cite{achiam2023gpt}, have significantly improved natural language understanding and generation capabilities in robots. These models enable fluid and contextually aware conversations, enhancing the robot’s ability to engage users meaningfully~\cite{abbas2024talkwithmachines, atuhurra2024large, lim2023sign}. Although LLM-powered systems have been explored in applications like mental health coaching and education~\cite{spitale2023vita, latif2024physicsassistant}, their role in mitigating the uncanny valley effect – particularly in hyper-realistic social robots – remains underexplored.

Given the increasing presence of AI-driven technologies in daily life, students are more likely to engage with and have greater familiarity with such innovations~\cite{stein2024attitudes}. This demographic is particularly relevant for studying interactions with hyper-realistic social robots, as they are more likely to be early adopters and frequent users of these technologies~\cite{bick2024rapid}. By focusing on this age group, our study aligns with prior research~\cite{morillo2024age} indicating that age influences perceptions of AI and robotics, ensuring that our results reflect the attitudes of the most relevant user group while maintaining statistical robustness.

This study aims to bridge this gap by investigating whether advanced communicative and conversational skills powered by LLMs can reduce the uncanny valley effect in hyper-realistic humanoid robots. Specifically, we address the following research questions: (1) How does interaction with an LLM-powered hyper-realistic robot affect students’ perceptions? (2) How do impressions of conversational quality (e.g., naturalness, human-likeness, interestingness) influence students’ willingness to engage further?

To explore these questions, we conducted a user study involving 58 students interacting with a highly anthropomorphic robot equipped with LLM-driven conversational capabilities. Pre- and post-interaction surveys assessed changes in perceptions of uncanniness, conversational quality, and overall user experience. Our findings suggest that advanced conversational abilities can significantly reduce feelings of eeriness while fostering more natural interactions.

The remainder of this paper is structured as follows: Section \ref{sec:2} reviews related work on the uncanny valley effect, LLMs in HRI and social robot, Nadine. Section \ref{sec:3} introduces Nadine as the experimental platform. Section \ref{sec:4} outlines our methodology, while Section \ref{sec:5} presents the results. Section \ref{sec:6} discusses implications for design and future research directions. Finally, Section \ref{sec:7} concludes the study. This study makes the following contributions:
\begin{enumerate}
    \item Provide empirical evidence demonstrating that LLM-powered conversational capabilities can mitigate the uncanny valley effect - specifically in terms of creepiness, pleasantness, and approachability - in hyper-realistic robots;
    \item Identify key conversational factors, such as naturalness, human-likeness, and interestingness, that influence students' willingness to engage with hyper-realistic robots.
\end{enumerate}


\section{Related Work}
\label{sec:2}

\subsection{Uncanny Valley in Humanoid Robots}

First introduced by Masahiro Mori in 1970 ~\cite{mori2012uncanny}, the uncanny valley describes a psychological phenomenon in which human-like entities envoke feelings of discomfort, eeriness, or creepiness when their appearance and behavior approach, but do not fully achieve human-likeness. The uncanny valley effect has been one of the main drawbacks in fields such as human-robot interaction (HRI).

Studies have explored various aspects of this phenomenon, including visual appearance, speech characteristics, motion dynamics, emotional expression and interactional characteristics in eliciting uncanny feelings ~\cite{yam2021robots, mishra2022uncanny, clark2021exploring, krauter2024bridging, thepsoonthorn2021exploration, spitale2024appropriateness, skjuve2019help}. The more anthropomorphized appearance the robot has, the higher the chance the robot has to be liked by users ~\cite{yam2021robots, mishra2022uncanny}. Speech characteristics such as politeness can enhance user ratings of a system's likeability ~\cite{clark2021exploring} and more natural and human-like voices are preferred by users ~\cite{krauter2024bridging}. Spitale et al. ~\cite{spitale2024appropriateness} found that the LLM-equipped robotic well-being coach is not empathic and not able to emotionally understand their experiences by highlighting the need to put the psychological aspects at the center of the interaction. Conversation content and flow clearly influence how pleasant people perceive their conversation with chatbot ~\cite{skjuve2019help}.

To address the uncanny valley, researchers and designers have adopted several strategies. One common approach is to intentionally design robots and avatars with stylized or non-human appearances, thereby avoiding the valley altogether ~\cite{schwind2018avoiding}. Another approach involves improving the realism and fluidity of movement and facial expressions to better align with human expectations ~\cite{koschate2016overcoming}. Advances in motion capture, facial animation, and natural language processing (NLP) have helped reduce the perception of eeriness in some contexts.

However, while these methods have shown promise, they often focus on physical or visual improvements, overlooking the potential role of advanced conversational capabilities in mitigating the uncanny valley effect. Despite significant progress in improving the appearance and motion of humanoid robots, there is still limited research on the impact of advanced conversational skills on reducing the uncanny valley effect. This presents a gap in the literature, as the ability to engage in natural, context-aware conversations may significantly influence users' emotional responses to humanoid robots.


\subsection{Large Language Models and Human-Robot Interaction}

Large Language Models (LLMs), such as OpenAI's GPT series ~\cite{achiam2023gpt} or Meta's Llama series ~\cite{touvron2023llama}, are computational frameworks trained on vast datasets of text to generate human-like responses to prompts. In the Human-Robot Interaction (HRI) domain, LLMs enable robots to engage in sophisticated dialogues, potentially enhancing their ability to act as social agents and improving their ability to align with human expectations in collaborative and social contexts.

The application of LLMs in HRI has gained considerable attention in recent years. Recent studies have explored the potential of LLMs in enhancing human-robot interaction, demonstrating their ability to improve natural language understanding and generation in robotic systems ~\cite{spitale2023vita, tanneberg2024help, wang2024lami, feng2024large}. For instance, Spitale et al. ~\cite{spitale2023vita} integrated an LLM module into a robotic system to imitate human coach behaviors. Similarly, Tanneberg et al. ~\cite{tanneberg2024help} leveraged common-sense reasoning capabilities of LLMs to determine when and how to assist humans effectively, as well as when to remain silent to avoid disrupting group interactions. Feng et al. ~\cite{feng2024large} employed LLMs to solve complex tasks in human-agent collaboration scenarios, highlighting their versatility in diverse application domains.

Despite these advancements, significant gaps remain in the current literature. Much of the current work has centered on the functional capabilities of LLM-equipped robots, such as their proficiency in natural language understanding and generation. However, limited attention has been given to how humans perceive robots equipped with LLMs, particularly those with humanoid features. The gap is significant, as user perception is critical to the acceptance and success of social robots in real-world applications. Specifically, the impact of the uncanny valley effect remains an underexplored area in the context of LLM-enhanced humanoid robots. 

To address this research gap, our study investigates user perceptions of humanoid social robots equipped with LLMs. In particular, we examine how advanced language capabilities influence the uncanny valley effect and whether such integration can mitigate the discomfort commonly associated with interactions involving highly realistic, human-like robots.

\subsection{Nadine Social Robot}

% \begin{figure}[H]
%  \centering
%  \includegraphics[width=0.5\linewidth]{imgs/Nadine_appearance.jpg}
%  \caption{Nadine social robot new appearance. (I'll put a new photo)}
% \label{fig:Nadine_appearance}
% \end{figure}

Nadine stands out as one of the most realistic humanoid social robots, designed for social interactions with humans ~\cite{baka2017meet}. The robot's human-like facial expression and gestures are achieved through 27 channels embedded in its face and upper body. The hyper-realistic appearance of Nadine may evoke the uncanny valley effect, a phenomenon where robots that closely resemble humans can elicit discomfort in users. This has driven the development of sophisticated modules that enhance the human-like behavior of such robots ~\cite{ramanathan2019nadine}. This combination of hyper-realistic design and advanced capabilities position Nadine in a distinctive position within the uncanny valley.

Several studies have examined Nadine's human-like attributes and their impact on user perceptions during human-robot interactions ~\cite{mishra2019can, kim2019eliza, tulsulkar2021can, thalmann2021nadine}. For instance, Mishra et al. ~\cite{mishra2019can} and Kim et al. ~\cite{kim2019eliza} investigated Nadine's role as a customer service agent. Mishra et al. ~\cite{mishra2019can} reported positive user feedback, noting that Nadine's social behaviors were perceived as acceptable and pleasing, while highlighting the need for advancements in commonsense reasoning and dialogue systems as future areas for improvement. In contrast, Kim et al. ~\cite{kim2019eliza} observed that excessive anthropomorphism in consumer robots could negatively affect customer attitudes, triggering feelings of uncanniness. Other research has explored Nadine's potential as a companion for elderly individuals and as an interactive agent in museums, with study participants responding positively to these applications ~\cite{tulsulkar2021can, thalmann2021nadine}.

Despite these encouraging founding, Nadine's dialogue system has been recognized as an area requiring improvement to enhance human-robot interactions. The previous version employed the Artificial Linguistic Internet Computer Entitiy (A.L.I.C.E.) chatbot ~\cite{wallace2009anatomy}, a rule-based conversational agent built using the Artificial Intelligent Markup Language (AIML). This approach matches user inputs to predefined patterns in its AIML knowledge base and responds with corresponding templates. Its dependence on manually crafted rules limits scalability and contextual understanding. The advent of LLMs, such as ChatGPT, addresses these limitations by excelling at understanding nuanced conversational context, adapting to diverse domains, and dynamically learning through fine-tuning. Nadine's updated system incorporates an LLM-based dialogue framework, significantly enhancing her capability to engage in nuanced, context-aware conversations. Further details of this enhancement are provided in Section \ref{sec:3}.



\section{Nadine Platform}
\label{sec:3}

This study investigates changes in the perception of highly anthropomorphized robots with advanced conversational skills, enabled by an LLM-based agent. For this purpose, we adapted the Nadine architecture, specifically the SoR-ReAct introduced in ~\cite{kang2024nadine}, by simplifying its interaction layer. Key components such as the affective system and long-term memory modules were excluded, allowing the study to isolate and assess the impact of the robot's conversational capabilities. Additionally, the LLM-agent framework was modified to better suit the experimental goals. The overall Nadine system architecture is illustrated in Figure \ref{fig:NadinePlatform}.

\begin{figure}[ht]
 \centering
 \includegraphics[width=\linewidth]{imgs/nadinePlatform3.png}
 \caption{Nadine system architecture.}
\label{fig:NadinePlatform}
\end{figure}

The LLM-agent implemented in this study is based on LangGraph\footnote[1]{\url{https://www.langchain.com/langgraph}}, an orchestration framework designed for complex agentic systems with fine-grained control. As depicted in Figure \ref{fig:NadinePlatform}, the agent is composed of four primary modules: router, retriever, web search, and generator. The router module serves as the decision-making core for the agent, determining the appropriate path for processing user queries. We leveraged the reasoning capabilities of gpt-4o-mini from OpenAI\footnote[2]{\url{https://platform.openai.com/docs/models/gpt-4o-mini}} to ensure robust and context-sensitive routing. For queries involving knowledge about the robot's background and prior experience, the router directs them to the retriever module, which accesses relevant data from given external knowledge base implemented using Chroma vectorstore\footnote[3]{\url{https://www.trychroma.com/}}. When user queries require real-time information, the router forwards the query to the web search module. This module augments the query with live search results, using the GoogleSerperAPIWrapper from LangChain\footnote[4]{\url{https://python.langchain.com/docs/integrations/providers/google_serper/}}. By incorporating live data, the system enhances the relevance and accuracy of its responses in dynamic contexts. For cases where the query does not necessitate access to external knowledge or real-time data, the router forwards it directly to the generator module. Like the router, the generator employs gpt-4o-mini for response generation but is tailored with a customized system prompt to ensure the robot's contextual awareness of its physical state and adherence to defined interaction protocols.


\section{Method}
\label{sec:4}

\subsection{Participants}

We randomly recruited 68 participants from the University of Geneva for this study. Participants ranged in age from 18 to 34 years (M=23, SD=3.61) and self-reported their gender as 46 female and 22 male. They represented various academic disciplines, primarily law, economics, management, psychology, and education. Fluency in English or French was required for participation, as these were the languages used in the robot interactions. Sixteen participants interacted with the robot in English, while 64 interacted in French. The sample included individuals from diverse geographic backgrounds, with most residing in Switzerland. Experiment completion times varied between 2 and 18 minutes (M=6.15, SD=3.46). All participants provided informed consent for their participation and the use of their data for scientific research. The study received full ethical approval from the University Commission for Ethical Research in Geneva (CUREG-2024-10-109).


\subsection{Experimental setup and procedure}

In this study, the robot was placed in the lobby of the University of Geneva (Uni Mail), chosen for its high accessibility and visibility to maximize participant recruitment. The location allowed participants to approach the robot naturally while providing a semi-controlled environment for the experiment. To ensure a smooth and controlled experiment, interactions between the robot and participants were conducted in a one-to-one manner. The interaction space was delineated from the surrounding environment using blocking rods, as illustrated in Figure \ref{fig:NadinePlatform}, creating a private and focused setting while minimizing external distractions. 

Before interacting with the robot, participants completed a pre-interaction questionnaire. This questionnaire gathered demographic information, such as age, gender, and prior experience with robots, and assessed participants' initial perceptions of the robot based solely on its appearance and presence. These baseline responses served as a reference to measure changes in perception following the interaction.

Participants were invited to engage in open-ended conversations with the robot, during which they could freely explore topics of interest and ask questions without predetermined constraints. Participants retained full autonomy over the interaction, with the ability to conclude their conversation at their discretion.

After the interaction, participants were asked to complete a post-interaction questionnaire, designed to capture their impressions of the robot. This assessment focused on three key areas: overall emotional responses (creepiness, pleasantness, and approachability), conversational qualities (naturalness, human-likeness, and interestingness), and open-ended feedback. The open-ended section prompted participants to describe any moments of discomfort, identify instances where the robot felt less natural, and suggest capabilities that could enhance the robot's perceived naturalness or human-likeness. By comparing pre- and post-interaction responses, we aimed to measure shifts in participant perceptions resulting from their direct interaction with the robot.


\subsection{Analysis}

A Wilcoxon Signed-Rank Test~\cite{wilcoxon1992individual, seabold2010statsmodels} was conducted to assess pre- and post-interaction changes in students' ratings of pleasantness, creepiness, and approachability ($RQ1$). To examine the relationship between post-interaction perceptions -naturalness, human-likeness, and interestingness - and students' willingness to continue engaging with the robot, a multiple linear regression analysis~\cite{seabold2010statsmodels} was performed ($RQ2$).


\section{Results}
\label{sec:5}
This section presents the findings from our quantitative data analysis. We address two key research questions: (1) how students' perceptions shift after interacting with the LLM-powered hyper-realistic robot, and (2) how conversational qualities affect students' willingness to continue interacting. The analysis provides empirical insights into user experiences, highlighting key factors that impact engagement and perception in human-robot interactions.

\subsection*{RQ1. How does interaction with an LLM-powered hyper-realistic robot affect students’ perceptions?}

We conducted a Wilcoxon Signed-Rank Test to examine shifts in students' perceptions - specifically pleasantness, creepiness, and approachability - after interacting with the robot. The analysis revealed statistically significant changes across all three metrics:

\par{\textbf{Pleasantness:}} Ratings significantly increased ($V = 103.0$, $p < 0.0001$, $d = 0.83$), indicating a large effect size and a meaningful improvement in students' perception. A post-hoc power analysis demonstrated strong reliability ($power = 1.0$), confirming the study had sufficient power to detect the effect.

\par{\textbf{Creepiness:}} Ratings significantly decreased ($V = 355.5$, $p < 0.001$, $d = -0.39$), reflecting a small to medium effect size. Although this indicates a reduction in discomfort, the change may not be substantial for all participants. Post-hoc power analysis confirmed sufficient power ($0.9255$) to detect this effect.

\par{\textbf{Approachability:}} Ratings showed significant improvement ($V = 244.0$, $p < 0.0001$, $d = 0.55$), with a medium effect size indicating a notable positive shift. Post-hoc power analysis ($power = 0.9983$) further supports the robustness of these findings.

\begin{table}[h]
    \centering
    \small
    \renewcommand{\arraystretch}{1.4} % Increase row height for better spacing
    \setlength{\tabcolsep}{8pt} % Adjust column spacing
    \begin{tabular}{lcc|cc|cc}
        \hline
        & \multicolumn{2}{c|}{\textbf{Pleasant}} 
        & \multicolumn{2}{c|}{\textbf{Creepy}} 
        & \multicolumn{2}{c}{\textbf{Approachable}} \\
        \cline{2-7} % Partial horizontal rule starting from "Pleasant"
        & \textbf{Pre} & \textbf{Post} 
        & \textbf{Pre} & \textbf{Post} 
        & \textbf{Pre} & \textbf{Post} \\
        \hline
        \textbf{Mean} & 2.897 & 3.706 & 3.456 & 2.912 & 3.221 & 3.853 \\
        \textbf{SD}   & 0.877 & 0.956 & 0.946 & 1.209 & 1.041 & 1.047 \\
        \hline
    \end{tabular}
    \caption{Descriptive statistics for Pleasant, Creepy, and Approachable ratings (Pre and Post).}
    \label{tab:RQ1_descriptive_stats}
\end{table}

Figure~\ref{fig:RQ1_BaxPlot} illustrates the distribution of participants' ratings for pleasantness, creepiness, and approachability before (blue) and after (orange) interacting with the robot. In the first plot, a noticeable upward shift in $Pleasantness$ ratings is evident in the post-interaction condition. The median value increased, and the interquartile range shifted upward, indicating a significant improvement. The second plot shows a modest downward shift in the ratings of $Creepiness$ following the interaction, with a slight reduction in the median and a narrower distribution in the interquartile range. Finally, the third plot highlights a clear improvement in $Approachability$ ratings, with the post-interaction boxplot showing an increased median and a shift in the interquartile range toward higher values.

\begin{figure}[ht]
 \centering
 \includegraphics[width=\linewidth]{imgs/RQ1_BoxPlot.png}
 \caption{Boxplots depicting changes in students' ratings for pleasantness, creepiness, and approachability before (pre) and after (post) interacting with the robot. Each box represents the interquartile range (IQR), with the median shown as a horizontal line and individual data points overlaid.}
\label{fig:RQ1_BaxPlot}
\end{figure}


\subsection*{RQ2. How do impressions of conversational quality (e.g., naturalness, human-likeness, interestingness) influence students’ willingness to engage further?}

The analysis revealed that both $naturalness$ ($b = 0.31, p = 0.015$) and $interestingness$ ($b = 0.60, p < 0.001$) were significant predictors of participants' willingness to continue interacting with the robot. In contrast, $human$-$likeness$ ($b = 0.22, p = 0.095$) was not a significant predictor. The overall regression model was highly significant ($F(3,64) = 29.14, p < 0.0001$), indicating that the predictors collectively explained a substantial proportion of the variance in participants' willingness to continue engaging with the robot.

\begin{table}[h]
    \centering
    \renewcommand{\arraystretch}{1.4} % Increase row height for better spacing
    \setlength{\tabcolsep}{8pt} % Adjust column spacing for better padding
    \begin{tabular}{c cccc}
        \hline
        & \textbf{Natural} & \textbf{Humanlike} & \textbf{Interesting} & \textbf{Willingness} \\
        \hline
        \textbf{Mean} & 3.029 & 2.721 & 4.029 & 3.618 \\
        \textbf{SD}   & 0.939 & 0.921 & 1.014 & 1.164 \\
        \hline
    \end{tabular}
    \caption{Descriptive statistics for natural, humanlike, interesting, and willingness ratings.}
    \label{tab:RQ2_descriptive_stats}
\end{table}

Figure~\ref{fig:RQ2_Regression} visualizes the regression coefficients and standard errors for the predictors $naturalness$, $human$-$likeness$, and $interestingness$, along with the model's constant. The coefficients represent the contribution of each predictor to participants' willingness to continue interacting with the robot. The figure highlights the importance of $naturalness$ and $interestingness$ in driving engagement, while also underscoring the limited role of $human$-$likeness$ in predicting continued interaction which aligns with the statistical analysis.

\begin{figure}[ht]
 \centering
 \includegraphics[width=\linewidth]{imgs/RQ2_Regression.png}
 \caption{Regression coefficients with standard errors for the predictors of naturalness, human-likeness, and interestingness, along with the constant term.}
\label{fig:RQ2_Regression}
\end{figure}




\section{Discussion}
\label{sec:6}

The findings of this study provide empirical evidence on how interactions with an LLM-powered hyper-realistic robot influence user perceptions and engagement. In particular, the significant increases in pleasantness, and approachability suggest that advanced conversational AI has the potential to enhance overall user experience, supporting prior research on the importance of dialogue quality in positive human-robot interactions~\cite{obaigbena2024ai}. Notably, although there was a statistically significant decrease in perceptions of creepiness, the relatively small effect size indicates that a subset of users continued to experience discomfort. This outcome underscores the persistent challenge of the uncanny valley\textemdash especially in the context of hyper-realistic humanoid robots\textemdash even when they possess advanced conversational abilities.

The regression analysis further revealed that conversational naturalness and interestingness were significant predictors of users' willingness to continue interacting with the robot, whereas human likeness was not. This finding implies that users value the fluidity and coherence of conversation (naturalness) and the capacity to sustain engagement (interestingness) more than the extent to which the robot mimics human conversational patterns. In fact, the interestingness variable exhibited the largest regression coefficients, highlighting that a robot's ability to maintain user interest is paramount for sustained engagement. While naturalness ensures interactions remain effortless and coherent, interestingness likely emerges from diverse conversational strategies, topic variety, and the capacity to generate thought\textemdash provoking or entertaining responses. In contrast, although aspects of human-likeness\textemdash such as phrasing, tone, and response structure\textemdash may shape initial impressions, they do not necessarily foster further engagement.

From an HRI design perspective, these findings suggest that developers should consider prioritizing:
\begin{itemize}
    \item enhance response fluidity to ensure seamless and natural interactions.
    \item optimize engagement mechanisms to sustain user interest beyond mere linguistic coherence.
    \item develop dialogue systems that are both structured and dynamic, rather than simply replicating human speech patterns.
\end{itemize}
Addressing these factors can help mitigate discomfort, improve interaction quality, and ultimately foster more engaging user experiences with hyper-realistic robots.

Despite these contributions, several limitations warrant acknowledgement. First, the participant sample consisted of students, which may limit the generalizability of the findings to broader demographic groups. Although our sample was linguistically and culturally diverse, future research should incorporate additional factors\textemdash such as socioeconomic status, educational background, and prior experience with robots\textemdash to ensure that robot designs meet the needs of a wider user base.

Second, the study relied primarily on self-reported measures, which may be subject to biases including social desirability. Future research should integrate objective behavioral and physiological metrics (e.g., gaze tracking, response times, or measures of emotional arousal) to complement subjective reports.

Lastly, the relatively short interaction duration may have influenced participant perceptions. Extended interaction periods could provide deeper insights into how users adapt to the robot's conversational style and whether initial discomfort diminishes over time. Future studies should explore longer engagements to assess the sustainability of positive perceptions and further understand the long-term effects of the uncanny valley phenomenon.

In addition, evaluating the implementation of LLM-powered hyper-realistic robots in applied settings (e.g., education, eldercare, service industries) could help identify domain-specific challenges and validate their practical effectiveness.


\section{Conclusion}
\label{sec:7}

This study examined how interactions with an LLM-powered hyper-realistic robot influence user perceptions and engagement. Our results indicate that robots' advanced communication skills powered by LLMs significantly enhanced perceptions of pleasantness and approachability while modestly reducing feelings of creepiness. In addition, regression analysis revealed that the robot's conversational naturalness and its ability to maintain interest are key drivers of users' willingness to engage further, whereas its human-like qualities appear less critical in this regard.

These findings suggest that, for effective human-robot interaction, design efforts should prioritize fluid and engaging dialogue over merely mimicking human conversational patterns. Although our work provides valuable insights into the benefits and challenges of advanced conversational robotics, it is important to acknowledge limitations such as the homogeneous student sample, reliance on self-reported measures, and brief interaction durations.

Future research should extend these findings by incorporating more diverse participant groups, objective behavioral metrics, and longer-term interaction studies.

In summary, our study demonstrates that interactions with an LLM-powered hyper-realistic robot can positively influence users' perceptions and engagement. Addressing these issues will be pivotal in developing socially intelligent humanoid robots that can be seamlessly integrated into everyday life.



\bibliographystyle{unsrt}  
\bibliography{UncannyNadine}



\end{document}
