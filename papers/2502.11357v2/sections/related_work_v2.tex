\subsection{LLM-based Web Agents}


Recent advances in multimodal language models have facilitated the development of web agents — autonomous systems designed to interact with real-world websites to perform everyday tasks \cite{mind2web, cogagent, seeclick, zheng2024gpt}.
Key challenges for web agents include long-term planning, visual grounding, and memory management. 
To improve long-context understanding, WebAgent \cite{DBLP:conf/iclr/GurFHSMEF24} utilizes multiple LLMs - one for planning, summarization, and grounded program synthesis.
SeeAct \cite{zheng2024gpt} adopts a two-step procedure of planning followed by grounding at each step using GPT-4 to accomplish web agent tasks.
Another line of work employs a vision-only approach to train a GUI grounding model that directly predicts pixel coordinates for executing GUI agent tasks \cite{seeclick, DBLP:conf/eccv/KapoorBRKKAS24, gou2024uground}.
However, a significant bottleneck remains — the lack of large-scale, high-quality web trajectory data for training robust agents. 
Our work presents a new framework for synthesizing large-scale web trajectory data to train end-to-end web agents.



\subsection{Web Agent Benchmarks and Datasets}
Early benchmarks for web tasks such as MiniWob++ \cite{miniwob} focused on testing low-level actions on simulated websites. 
However, these simulated websites fail to capture the complexity of the real-world web.
Mind2Web \cite{mind2web} introduces a trajectory-level dataset with 2K tasks across 137 real-world websites and 31 domains.
However, it employs a static evaluation method that penalizes alternative valid execution paths.
To overcome this limitation, follow-up work has explored alternative evaluation approaches, including functional correctness-based evaluation in WebArena \cite{DBLP:conf/iclr/ZhouX0ZLSCOBF0N24} and key-node-based evaluation in Mind2Web-Live \cite{pan2024webcanvas}.
Towards the goal of making web agents more capable of performing realistic tasks, GAIA \cite{mialon2024gaia} and AssistantBench \cite{DBLP:conf/emnlp/YoranAMBPB24} introduce benchmarks that include time-consuming information-seeking tasks.
In this work, we develop \model, a multimodal web agent trained on our synthetic dataset, and showcase its strong performance across online and offline benchmarks, including Mind2Web-Live, Multimodal-Mind2Web, and MiniWob++.

\begin{table*}
\small
\begin{tabular}{l}
\hline
% \rowcolor{gray!10}
\cellcolor[gray]{0.9} \textbf{Information} \\
\hline
View the detailed 7-day weather forecast for Toronto, ON on The Weather Network website. \\
Analyze Tesla's stock performance over various time periods on Yahoo Finance. \\
Convert 100 US Dollars to Euros using the XE currency converter. \\
Find directions from Seattle, WA to Bellevue, WA using Bing Maps. \\
\hline
\rowcolor{gray!10}
\textbf{Service} \\
\hline
Research the French Bulldog breed on the American Kennel Club website, including its popularity and family life traits. \\
Find the nearest Penske truck rental location in Anaheim, California, and start the reservation process for a truck. \\
Explore and purchase a subscription for the UpToDate Pro Suite on the Wolters Kluwer website. \\
\hline
% \rowcolor{gray!10}
\cellcolor[gray]{0.9}\textbf{Entertainment} \\
\hline
Find the Basscon presents: Darren Styles EDM event on Eventbrite, save it, and share it on Twitter. \\
View the details of the Photography Competition Winners - Season X and share the article on Twitter. \\
\hline
% \rowcolor{gray!10}
\cellcolor[gray]{0.9}\textbf{Shopping} \\
\hline
Browse through the fall home decor section on the Target website to explore a variety of fall-themed home decor items. \\
Purchase a three-seat fabric sofa, specifically the UPPLAND Sofa, from IKEA's website. \\
\hline
% \rowcolor{gray!10}
\cellcolor[gray]{0.9}\textbf{Travel} \\
\hline
Search for flights from Seattle to New York, select travel dates, and explore various flight options. \\
Find the weight of baggage allowance for economy class on qatarairways. \\
\hline
\end{tabular}
\caption{Example task descriptions from \model.}
\label{tab:traj_ex}
\end{table*}

\subsection{Data Synthesis for Web Agents}
Early efforts to acquire trajectory data for training web agents primarily relied on crowd-sourcing \cite{mind2web, DBLP:conf/icml/LuKR24}.
However, human annotation is cost-prohibitive, prompting the adoption of synthetic data generation approaches to facilitate large-scale data collection.
AutoWebGLM \cite{DBLP:conf/kdd/LaiLIYCSYZZD024} and GUIAct \cite{chen2024guicourse} utilize LLMs to generate task proposals, which human experts subsequently annotate.
OpenWebVoyager \cite{he2024openwebvoyager} employs a web agent to execute auto-generated task descriptions. 
However, since these task descriptions are generated using LLMs without exploring a website, they fail to capture the full diversity of possible tasks on that website.
Another line of work, including Synatra \cite{Ou2024SynatraTI} and AgentTrek \cite{xu2024agenttrek}, leverages web tutorials to guide web trajectory generation.
Meanwhile, concurrent effort \cite{murty2024nnetscape} employs an exploration-based trajectory generation in WebArena’s sandbox, while our work focuses on more realistic web agent evaluation on live websites.
To address diversity limitations in prior trajectory synthesis work, we design a bottom-up web trajectory synthesis pipeline that explores websites dynamically while maintaining a coherent high-level task intent.