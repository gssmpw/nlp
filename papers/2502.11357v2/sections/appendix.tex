% This is an appendix.
\newpage

\setcounter{table}{0}
\renewcommand\thetable{\Alph{section}.\arabic{table}}
\setcounter{figure}{0}
\renewcommand\thefigure{\Alph{section}.\arabic{figure}}

\section*{Appendices}
This supplementary material provides additional details omitted in the main text.
\begin{itemize}
    \item Appendix~\ref{sec:m2w_eval}: Mind2Web Training and Evaluation Details
    \item Appendix~\ref{sec:cost_breakdown}: Trajectory Synthesis Cost Analysis
    \item Appendix~\ref{sec:reason_gen_agent}: Reasoning Generation Agent
    \item Appendix~\ref{sec:prompt_details}: Prompt Details
    \item Appendix~\ref{sec:traj_ex}: Trajectory Examples
\end{itemize}


\section{Mind2Web Training and Evaluation Details}\label{sec:m2w_eval}

Table~\ref{tab:hyper} shows the hyperparameters and training time for experiments on Mind2Web-Live and Multimodal-Mind2Web.
All experiments use Nvidia H100 GPUs.

\begin{table*}[htbp]
\centering
\small
\resizebox{\linewidth}{!}{%
\begin{tabular}{llllll}
\toprule
\bfseries Model                                                          & \bfseries Avg.\   Step SR (\%) & \bfseries Completion Rate  (\%)  & \bfseries \specialcell{Task SR (1) (\%)} & \bfseries Full Task SR (\%)\\ \midrule
\multicolumn{3}{l}{\textbf{API-based Models}} &\\
\cmidrule(r){1-1}
GPT-4o                                                                     & \num{56.4}                                          & \num{50.4}                                               & \num{44.2}    & \num{22.1}                                                \\
GPT-3.5                                                                     & --                                          & \num{36.5}                                            & --  & \num{15.4}                                                   \\ \midrule
\multicolumn{3}{l}{\textbf{Open-source Instructed Models}} &\\ \cmidrule(r){1-1}

Mistral-7B-Instruct-0.3 \cite{jiang2023mistral}                         & \num{33.0}                                      & \num{28.6}                                                & \num{25.0}    & \num{11.5}                                                    \\
Qwen2-72B-Instruct \cite{qwen} & -- & \bfseries\num{40.9} & -- & \num{15.4} \\
Qwen2-VL-7B \cite{Qwen2VL}	 &	\num{37.9} &	\num{33.3}  	& \num{31.7}	&	\num{12.5}\\
Phi-3.5V \cite{abdin2024phi}                                                                & \num{27.0}                                      & \num{22.3}                                           & \num{21.2}         & \num{1.9}                                          \\
\midrule
\multicolumn{3}{l}{\textbf{Supervised Fine-Tuning}} &\\
\cmidrule(r){1-1}
\textbf{\model-4B}  & \num{41.6}	& \num{36.7}	 & \num{30.8} & \num{16.4}\\% no goto
\textbf{\model-7B}  & \bfseries\num{42.0}	& \num{36.9}	 & \bfseries\num{32.7} & \bfseries\num{16.4}\\

\bottomrule
\end{tabular}
}
\caption{Results on Mind2Web-Live benchmark. 
The results for GPT-4, GPT-3.5, and Mistral-7B have been reproduced on our Linux servers.
The full task success rate (SR) represents the successful completion of all key nodes for a given task.
The average step success rate represents the proportion of completed key nodes, macro-averaged across tasks.
The completion rate represents the proportion of completed key nodes, micro-averaged across tasks.
Task SR (1) represents task SR with a tolerance of up to one error/key node.
Our Phi-3.5V model, finetuned on synthetic trajectory data from \model, outperforms much larger models, including Mistral-7B and Qwen2-72B-Instruct, by a significant margin and is comparable to GPT-3.5.
}
\label{tab:m2w_live_full}
\end{table*}

\subsection{Mind2Web-Live}
We exclude the following websites - \url{https://www.kbb.com}, \url{https://www.sixflags.com}, \url{https://www.viator.com}, \url{https://www.menards.com}, \url{https://www.amctheatres.com}, \url{https://www.cargurus.com}, \url{https://www.gamestop.com}, \url{https://www.cabelas.com}, \url{https://www.rei.com} due to denial of access faced during our tests.
% faced during trajectory collection or execution of Mind2Web-Live tasks on these websites.
Table~\ref{tab:m2w_live} shows the results on Mind2Web-Live for 83 out of 104 tasks across the remaining 37 websites.
The results on the whole Mind2Web-Live evaluation set are given in Table~\ref{tab:m2w_live_full}.
The results in Table~\ref{tab:m2w_live} are reported as the maximum over three runs, accounting for intermittent website access issues that may affect evaluation consistency.
For Mind2Web-Live, the dataloader first samples training instances at the trajectory level and then randomly samples a step from the trajectory to construct the final training instance.
Thus, the number of epochs is calculated at the trajectory level.
We use a viewport resolution of $1280\times 720$ during inference.
The Mind2Web-Live dataset is released under the MIT license which permits its use for academic research.

\subsection{Multimodal-Mind2Web}
Following \citet{mind2web}, we obtain the top-50 elements from a pre-trained DeBERTa \cite{he2021deberta} candidate generation model, which are then used to construct the accessibility tree and SOM image inputs.
Following \citet{Ou2024SynatraTI}, we always include the ground truth element in the input.
We use a viewport resolution of $1280\times 720$ which includes the GT element during inference.
We follow the setting in \citet{zheng2024gpt} and report element accuracy, operation F1, and step SR as evaluation metrics.
All experiments on Multimodal-Mind2Web use a single training and evaluation run.
The dataloader uniformly samples training instances from the set of action steps across all trajectories.
The Multimodal-Mind2Web dataset is released under the Responsible AI license which permits its use for academic research.

\begin{figure*}[!ht]
    \centering
    \includegraphics[width=\linewidth]{rebuttal-figures-src/hyperparams.pdf}
    \vspace{-1.5em}
    \caption{Concept Sliders Comparison \& Hyperparameter analysis: (Left) Impact of PCA directions: SliderSpace with 10 directions matches the FID of 64 Concept Sliders. More directions, upto 40, leads to improved FID. (Right) Effect of LoRA rank: Given a fixed training budget rank-one sliders are efficient than higher rank versions and outperforms Concept Sliders}
    \vspace{-0.3em}
    \label{fig:reb-hyperparam}
\end{figure*}


\section{Trajectory Synthesis Cost Analysis}\label{sec:cost_breakdown}
We use GPT-4o-turbo, which costs \$2.5 per 1M tokens for our trajectory synthesis.
Each proposal or refinement stage uses 3.6K textual tokens on average.
% of resolution $1280\times 720$ 
Each input image costs \$0.0028.
The calculation assumes an average of 7.7 steps per trajectory, including the proposal stage.
Table~\ref{tab:cost_breakdown} shows the breakdown for the different stages of trajectory generation.
\begin{align*}
\textrm{Total cost} &= \$0.0128 * 7.7 + \$0.02581\\ &+ \$0.02381
= \$0.148
\end{align*}

The average cost per raw trajectory is \$0.15.
The success rate is estimated as 53.1\%.
Thus, the average cost per successful trajectory is estimated to be \$0.28.

\begin{table}[H]
\centering
\small
\begin{tabular}{lll} \toprule
\bfseries Phase        & \bfseries Cost per step & \bfseries Total cost \\ \midrule
Proposal     & \$\num{0.0128}      & \$\num{0.0128}   \\
Refinement   & \$\num{0.0128}      & \$\num{0.0856}   \\
Verification & \$\num{0.02381}     & \$\num{0.02381}  \\
Summarization   & \$\num{0.02581}     & \$\num{0.02581} \\ \bottomrule
\end{tabular}
\caption{Cost breakdown for different modules in the pipeline.}
\label{tab:cost_breakdown}
\end{table}

\section{Reasoning Generation Agent}\label{sec:reason_gen_agent}
Inspired by \citet{xu2024aguvis}, the reasoning generation agent is a pre-trained Qwen2-VL-7B model that takes as input the current action, high-level task description, screenshot, accessibility tree, and action history. It then outputs a post-hoc reasoning trace for performing that action. These reasoning traces are helpful for training GUI agents in a chain-of-thought style.
% We will release the reasoning traces along with the trajectories for \model.

\section{Prompt Details}\label{sec:prompt_details}
The prompts for the task proposer agent, task refiner agent, task summarizer agent, and task verifier agent are given in Table~\ref{tab:proposal_prompt}, Table~\ref{tab:refiner_prompt}, Table~\ref{tab:summarizer_prompt}, and Table~\ref{tab:verifier_prompt}, respectively.
The training prompt for \model is given in Table~\ref{tab:train_prompt}.

% \begin{table*}[h]
%     \centering
%     \small
\onecolumn
{\small
\centering
\begin{longtable}{lp{12cm}}
 % \caption{Prompt for Task Proposal Agent} \\
    \hline
    % Column 1 & Column 2 \\ \hline
    \endfirsthead

    \multicolumn{2}{c}{\textit{Continued from previous page}} \\ \hline
    % Column 1 & Column 2 \\ \hline
    \endhead

    \hline \multicolumn{2}{|r|}{\textit{Continued on next page}} \\ \hline
    \endfoot

    \hline
    \endlastfoot

    % Sample Data
    % 1  & Data A \\
    % 2  & Data B \\
    % 3  & Data C \\
    \textbf{System Role} & What does this webpage show? Imagine you are a real user on this webpage. Given the webpage screenshot and parsed HTML/accessibility tree, please provide a single task that a user might perform on this page and the corresponding first action towards completing that task.\\
    & \underline{\smash{Do the following step by step}}:\\
    & 1. Generate a single task that a user might perform on this webpage. Be creative and come up with diverse tasks\\
    & 2. Given the webpage screenshot and parsed HTML/accessibility tree, generate the first action towards completing that task (in natural language form).\\
    & 3. Given the webpage screenshot, parsed HTML/accessibility tree, and the natural language action, generate the grounded version of that action.\\~\\
    \cmidrule{2-2}

    & \textbf{ACTION SPACE}: Your action space is: [`click [element ID]', `type [element ID] [content]', `select [element ID] [content of option to select]', `scroll [up]', `scroll [down]', and `stop'].\\
    & \underline{\smash{Action output should follow the syntax as given below}}:\\
    & `click [element ID]': This action clicks on an element with a specific ID on the webpage.\\
    & `type [element ID] [content]': Use this to type the content into the field with id. By default, the "Enter" key is pressed after typing. Both the content and the ID should be within square braces as per the syntax. \\
    & `select [element ID] [content of option to select]': Select an option from a dropdown menu. The content of the option to select should be within square braces. When you get (select an option) tags from the accessibility tree, you need to select the serial number (element\textunderscore id) corresponding to the select tag, not the option, and select the most likely content corresponding to the option as input.\\
    & `scroll [down]': Scroll the page down. \\
    & `scroll [up]': Scroll the page up. \\~\\
    \cmidrule{2-2}
    
    & \textbf{IMPORTANT}: To be successful, it is important to STRICTLY follow the below rules:\\~\\
    % Add more rows as needed
    & \textbf{Action generation rules}:\\
    & 1. You should generate a single atomic action at each step.\\
    & 2. The action should be an atomic action from the given vocabulary - click, type, select, scroll (up or down), or stop.\\
    & 3. The arguments to each action should be within square braces. For example, "click [127]", "type [43] [content to type]", "scroll [up]", "scroll [down]".\\
    & 4. The natural language form of action (corresponding to the field "action\textunderscore in\textunderscore natural\textunderscore language") should be consistent with the grounded version of the action (corresponding to the field "grounded \textunderscore action"). Do NOT add any additional information in the grounded action. For example, if a particular element ID is specified in the grounded action, a description of that element must be present in the natural language action. \\
    & 5. If the type action is selected, the natural language form of action ("action\textunderscore in\textunderscore natural\textunderscore language") should always specify the actual text to be typed. \\
    & 6. You should issue a “stop” action if the current webpage asks to log in or for credit card information. \\
    & 7. To input text, there is NO need to click the textbox first, directly type content. After typing, the system automatically hits the `ENTER' key.\\
    & 8. STRICTLY Avoid repeating the same action (click/type) if the webpage remains unchanged. You may have selected the wrong web element.\\
    & 9. Do NOT use quotation marks in the action generation.\\~\\
    \cmidrule{2-2}
    
    & \textbf{Task proposal rules}: \\
    & 1. You should propose tasks that are relevant to the website and can be completed using the website.\\
    & 2. You should only propose tasks that do not require login to execute the task.\\
    & 3. You should propose tasks that are clear and specific.\\
    & 4. For each task, provide concrete information or constraints, and use mock-up information (identifier, number, personal information, name, attributes, etc.) to make the task more specific and realistic.\\
    & 5. The task description should provide all the necessary information to complete the task.\\
    & 6. The task should be feasible to complete by a real user and should not require any additional information that is not available on the website.\\~\\

    & \underline{\smash{The output should be in below format}}:\\
    & \textbf{OUTPUT FORMAT}: Please give a short analysis of the screenshot, parsed HTML/accessibility tree, then put your answer within \textasciigrave\textasciigrave\textasciigrave \; \textasciigrave\textasciigrave\textasciigrave, for example, "In summary, the proposed task and the corresponding action is: \textasciigrave\textasciigrave\textasciigrave\texttt{{\{"task": <TASK>:str, "action\_in\_natural\_language":<ACTION\_IN\_NATURAL\_LANGUAGE>:str, "grounded\_action": <ACTION>:str\}}"}\textasciigrave\textasciigrave\textasciigrave\\
\midrule
\bfseries User Role & Website URL: \{INIT\textunderscore URL\}\\
& Parsed HTML\slash Accessibility Tree: \{A11Y\textunderscore TREE\}\\
& \{SCREENSHOT\} \\
\bottomrule        
\end{longtable}
}
\captionof{table}{Prompt for Task Proposer Agent.}
%     \caption{Prompt for Task Proposal Agent}
\label{tab:proposal_prompt}
% \end{table*}
\twocolumn

\nopagebreak
% \begin{table*}[h]
%     \centering
%     \small
\onecolumn
{\small
\centering
\begin{longtable}{lp{12cm}}
 % \caption{Prompt for Task Proposal Agent} \\
    \hline
    % Column 1 & Column 2 \\ \hline
    \endfirsthead

    \multicolumn{2}{c}{\textit{Continued from previous page}} \\ \hline
    % Column 1 & Column 2 \\ \hline
    \endhead

    \hline \multicolumn{2}{|r|}{\textit{Continued on next page}} \\ \hline
    \endfoot

    \hline
    \endlastfoot

    % Sample Data
    % 1  & Data A \\
    % 2  & Data B \\
    % 3  & Data C \\
    \textbf{System Role} & What does this webpage show? Imagine you are a real user on this webpage, and your overall task is \{OVERALL\textunderscore TASK\}.
    This is the list of actions you have performed that lead to the current page \{PREV\textunderscore ACTION\textunderscore LIST\}. You are also given the webpage screenshot and parsed HTML/accessibility tree.\\
    & \underline{\smash{Do the following step by step}}:\\
    & 1. Please predict what action the user might perform next that is consistent with the previous action list in natural language.\\
    & 2. Then based on the parsed HTML/accessibility tree of the webpage and the natural language action, generate the grounded action.\\
    & 3. Update the overall task aligned with this set of actions.\\~\\

    \cmidrule{2-2}
    
    & \textbf{ACTION SPACE}: Your action space is: [`click [element ID]', `type [element ID] [content]', `select [element ID] [content of option to select]', `scroll [up]', `scroll [down]', and `stop'].\\
    & \underline{\smash{Action output should follow the syntax as given below}}:\\
    & `click [element ID]': This action clicks on an element with a specific id on the webpage.\\
    & `type [element ID] [content]': Use this to type the content into the field with id. By default, the "Enter" key is pressed after typing. Both the content and the id should be within square braces as per the syntax. \\
    & `select [element ID] [content of option to select]': Select an option from a dropdown menu. The content of the option to select should be within square braces. When you get (select an option) tags from the accessibility tree, you need to select the serial number (element\textunderscore id) corresponding to the select tag, not the option, and select the most likely content corresponding to the option as input.\\
    & `scroll [down]': Scroll the page down. \\
    & `scroll [up]': Scroll the page up. \\~\\

    \cmidrule{2-2}
    
    & \textbf{IMPORTANT}: To be successful, it is important to STRICTLY follow the below rules:\\~\\
    % Add more rows as needed
    & \textbf{Action generation rules}:\\
    & 1. You should generate a single atomic action at each step.\\
    & 2. The action should be an atomic action from the given vocabulary - click, type, select, scroll (up or down), or stop\\
    & 3. The arguments to each action should be within square braces. For example, "click [127]", "type [43] [content to type]", "scroll [up]", "scroll [down]".\\
    & 4. The natural language form of action (corresponding to the field "action\textunderscore in\textunderscore natural\textunderscore language") should be consistent with the grounded version of the action (corresponding to the field "grounded \textunderscore action"). Do NOT add any additional information in the grounded action. For example, if a particular element ID is specified in the grounded action, a description of that element must be present in the natural language action. \\
    & 5. If the type action is selected, the natural language form of action ("action\textunderscore in\textunderscore natural\textunderscore language") should always specify the actual text to be typed. \\
    & 6. You should issue the “stop” action when the given list of input actions is sufficient for a web task. \\
    & 7. You should issue a “stop” action if the current webpage asks to log in or for credit card information. \\
    & 8. To input text, there is NO need to click the textbox first, directly type content. After typing, the system automatically hits the `ENTER` key.\\
    & 9. STRICTLY Avoid repeating the same action (click/type) if the webpage remains unchanged. You may have selected the wrong web element.\\
    & 10. Do NOT use quotation marks in the action generation.\\~\\

    \cmidrule{2-2}

    & \textbf{Task proposal rules}:\\
    & 1. You should propose tasks that are relevant to the website and can be completed using the website itself.\\
    & 2. The overall task should be well-aligned to the entire set of actions in history plus the current generated action. It should not be focused just on the current action.\\
    & 3. You should only propose tasks that do not require login to execute the task.\\
    & 4. You should propose tasks that are clear and specific.\\
    & 5. For each task, provide concrete information or constraints, and use mock-up information (identifier, number, personal information, name, attributes, etc.) to make the task more specific and realistic.\\
    & 6. The task description should provide all the necessary information to complete the task.\\
    & 7. The task should be feasible to complete by a real user and should not require any additional information that is not available on the website.\\~\\

    & \underline{\smash{The output should be in below format}}:\\
    & \textbf{OUTPUT FORMAT}: Please give a short analysis of the screenshot, parsed HTML/accessibility tree, and history, then put your answer within \textasciigrave\textasciigrave\textasciigrave \; \textasciigrave\textasciigrave\textasciigrave, for example, "In summary, the proposed task and the corresponding action is: \textasciigrave\textasciigrave\textasciigrave\texttt{{\{"task": <TASK>:str, "action\_in\_natural\_language":<ACTION\_IN\_NATURAL\_LANGUAGE>:str, "grounded\_action": <ACTION>:str\}}"}\textasciigrave\textasciigrave\textasciigrave\\
    \midrule
\bfseries User Role & Website URL: \{INIT\textunderscore URL\}\\
& Parsed HTML\slash Accessibility Tree: \{A11Y\textunderscore TREE\}\\
& \{SCREENSHOT\} \\
\bottomrule        
\end{longtable}
}
\captionof{table}{Prompt for Task Refiner Agent.}
%     \caption{Prompt for Task Proposal Agent}
\label{tab:refiner_prompt}
% \end{table*}
\twocolumn

\nopagebreak
\begin{table*}[htbp]
    \centering
    \small
    \begin{tabular}{lp{12cm}}
    \toprule
        \textbf{System Role} & Given a list of actions performed on the website \{WEBSITE\textunderscore URL\} and the corresponding screenshots\\
        & List of actions: \{ACTION\textunderscore LIST\}\\
        & Your task is to come up with a single task description that will be accomplished by performing these actions in the given sequence on the website. \\~\\

    \cmidrule{2-2}
& \textbf{IMPORTANT}:\\
& 1. The task must contain some actions: ``Buy, Book, Find, Check, Choose, show me, search, browse, get, compare, view, give me, add to cart, ...'', ideally involving transactions/finding information on a specific product or service.\\
& 2. You should propose tasks that are clear and specific.\\
& 3. The task description should provide all the necessary information to complete the task.\\
& 4. The task description must indicate the domain of the website at the end of the task with the format: ``... on task website'', for instance, ``Purchase a laptop on Amazon'', ``Book a hair appointment on Yelp'', etc.\\
& 5. The task should be feasible to complete by a real user and should not require any additional information that is not specified in this input.\\
& 6. The task description should specify constraints like given budget, product features, and other specifications that can narrow down the search to a particular item/product.\\
& 7. Do NOT use any quotation marks (either single or double) in the task description.\\~\\

\cmidrule{2-2}

& \underline{\smash{The output should be in the below format}}:\\
& \textbf{OUTPUT FORMAT}: Please first give some analysis of the actions and screenshots and then output the overall task description. put your answer within \textasciigrave\textasciigrave\textasciigrave \; \textasciigrave\textasciigrave\textasciigrave, for example, ``In summary, the answer is: \textasciigrave\textasciigrave\textasciigrave\texttt{<TASK\textunderscore DESCRIPTION>:str}\textasciigrave\textasciigrave\textasciigrave''.\\
    \bottomrule        
    \end{tabular}
    \caption{Prompt for Task Summarizer Agent.}
    \label{tab:summarizer_prompt}
\end{table*}

\begin{table*}[htbp]
    \centering
    \small
    \begin{tabular}{lp{12cm}}
    \toprule
        \textbf{System Role} & You are an expert in evaluating the performance of a web navigation agent. The agent is designed to help a human user navigate a website to complete a task. Given the user's intent, the agent's action history, the final state of the webpage, and the agent's response to the user, your goal is to decide whether the agent's execution is successful or not.\\
& There are four types of tasks:\\~\\
& 1. \textbf{Transaction}: The user wants to perform a transaction on the webpage, such as booking a ticket, ordering a product, etc. The bot should at least initiate the add-to-cart or checkout process. It is still a success if the bot has done actions of `add to cart' or checkout and encounters the login page.  If the bot fails to do so, the task is considered a failure.\\~\\
& 2. \textbf{Information seeking}: The user wants to obtain certain information from the webpage, such as information of a product, reviews, map info, comparison of map routes, etc. The bot's response must contain the information the user wants, or explicitly state that the information is not available. Otherwise, e.g. the bot encounters an exception and responds with the error content, the task is considered a failure. Besides, be careful about the sufficiency of the agent's actions. For example, when asked to list the top-searched items in a shop, the agent should order the items by the number of searches, and then return the top items. If the ordering action is missing, the task is likely to fail.\\~\\
& 3. \textbf{Site navigation}: The user wants to navigate to a specific page. Carefully examine the bot's action history and the final state of the webpage to determine whether the bot successfully completes the task. No need to consider the bot's response.\\~\\
& 4. \textbf{Content modification}: The user wants to modify the content of a webpage or configuration. Carefully examine the bot's action history and the final state of the webpage to determine whether the bot successfully completes the task. No need to consider the bot's response.\\~\\

\cmidrule{2-2}
& \textbf{IMPORTANT}\\
& - If a product has been added to the bag/cart in the action list but just the purchase is pending, it should be counted as a success.\\
& - If you see the checkout page for the product you want to purchase, it should be counted as a success.\\
& - Format your response into two lines as shown below:\\~\\
& \textbf{Thoughts}: <your thoughts and reasoning process>\\
& \textbf{Status}: "success" or "failure"\\
    \bottomrule        
    \end{tabular}
    \caption{Prompt for Task Verifier Agent (adapted from \citet{DBLP:journals/corr/abs-2404-06474}).}
    \label{tab:verifier_prompt}
\end{table*}

\begin{table*}[htbp]
    \centering
    \small
    \begin{tabular}{lp{12cm}}
    \toprule
        \textbf{System Role} & You are an expert at completing instructions on webpage screens.\\
               & You will be presented with a screenshot image with some numeric tags.\\
               & If you decide to click somewhere, you should choose the numeric element index closest to the location you want to click. \\
               & You should decide the action to continue this instruction.
               You will be given the accessibility tree of the current screen in the format: \texttt{[element\_idx] [role] [alt text or button name]}.\\
               & Here are the available actions:\\
               & \texttt{\{"action": "goto", "action\_natural\_language": str, "value": \textless the URL to go to\textgreater\}}\\
               & \texttt{\{"action": "google\_search", "action\_natural\_language": str, "value": \textless search query for google\textgreater\}}\\
               & \texttt{\{"action": "click", "action\_natural\_language": str, "idx": \textless element\_idx\textgreater\}}\\
               & \texttt{\{"action": "type", "action\_natural\_language": str, "idx": \textless element\_idx\textgreater, "value": \textless the text to enter\textgreater\}}\\
               & \texttt{\{"action": "select", "action\_natural\_language": str, "idx": \textless element\_idx\textgreater, "value": \textless the option to select\textgreater\}}\\
               & \texttt{\{"action": "scroll [up]", "action\_natural\_language": str\}}\\
               & \texttt{\{"action": "scroll [down]", "action\_natural\_language": str\}}\\
               & Your final answer must be in the above format.\\
        \midrule
        \textbf{User Role} & The instruction is to \{TASK DESCRIPTION\}. \\
      & History actions: \{PREVIOUS ACTIONS\}\\
      & Here is the screen information: \{ACCESSIBILITY TREE\}\\
      & Think about what you need to do with the current screen, and output the action in the required format in the end. \\
        \bottomrule
    \end{tabular}
    \caption{Prompt for Web Agent Training.}
    \label{tab:train_prompt}
\end{table*}

\FloatBarrier
% \clearpage

\section{Trajectory Examples}\label{sec:traj_ex}
Figure~\ref{fig:traj_ex} shows a sample trajectory executed on the IKEA website.
Figure~\ref{fig:traj_input} shows the set-of-mark annotations and accessibility tree inputs of the model during trajectory generation, model training, and inference.

% \FloatBarrier
% \vspace{-100mm}

\begin{figure*}[!htb]
    \centering
    \includegraphics[width=\textwidth]{figures/traj_ex_v3.pdf}
    \caption{Example synthetic trajectory from \model. Each step shows the set-of-mark annotated screenshot along with the grounded action taken by the GPT-4 agent.}
    \label{fig:traj_ex}
\end{figure*}

% \vspace{-100mm}

\begin{figure*}[!htb]
    \centering
    \includegraphics[width=\textwidth]{figures/traj_inputs.pdf}
    \caption{Visualization of the model inputs during trajectory generation, model training, and inference. The example corresponds to step 2 of the trajectory in Figure~\ref{fig:data_pipeline}.}
    \label{fig:traj_input}
\end{figure*}