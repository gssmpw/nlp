% \begin{table*}[h]
%     \centering
%     \small
\onecolumn
{\small
\centering
\begin{longtable}{lp{12cm}}
 % \caption{Prompt for Task Proposal Agent} \\
    \hline
    % Column 1 & Column 2 \\ \hline
    \endfirsthead

    \multicolumn{2}{c}{\textit{Continued from previous page}} \\ \hline
    % Column 1 & Column 2 \\ \hline
    \endhead

    \hline \multicolumn{2}{|r|}{\textit{Continued on next page}} \\ \hline
    \endfoot

    \hline
    \endlastfoot

    % Sample Data
    % 1  & Data A \\
    % 2  & Data B \\
    % 3  & Data C \\
    \textbf{System Role} & What does this webpage show? Imagine you are a real user on this webpage. Given the webpage screenshot and parsed HTML/accessibility tree, please provide a single task that a user might perform on this page and the corresponding first action towards completing that task.\\
    & \underline{\smash{Do the following step by step}}:\\
    & 1. Generate a single task that a user might perform on this webpage. Be creative and come up with diverse tasks\\
    & 2. Given the webpage screenshot and parsed HTML/accessibility tree, generate the first action towards completing that task (in natural language form).\\
    & 3. Given the webpage screenshot, parsed HTML/accessibility tree, and the natural language action, generate the grounded version of that action.\\~\\
    \cmidrule{2-2}

    & \textbf{ACTION SPACE}: Your action space is: [`click [element ID]', `type [element ID] [content]', `select [element ID] [content of option to select]', `scroll [up]', `scroll [down]', and `stop'].\\
    & \underline{\smash{Action output should follow the syntax as given below}}:\\
    & `click [element ID]': This action clicks on an element with a specific ID on the webpage.\\
    & `type [element ID] [content]': Use this to type the content into the field with id. By default, the "Enter" key is pressed after typing. Both the content and the ID should be within square braces as per the syntax. \\
    & `select [element ID] [content of option to select]': Select an option from a dropdown menu. The content of the option to select should be within square braces. When you get (select an option) tags from the accessibility tree, you need to select the serial number (element\textunderscore id) corresponding to the select tag, not the option, and select the most likely content corresponding to the option as input.\\
    & `scroll [down]': Scroll the page down. \\
    & `scroll [up]': Scroll the page up. \\~\\
    \cmidrule{2-2}
    
    & \textbf{IMPORTANT}: To be successful, it is important to STRICTLY follow the below rules:\\~\\
    % Add more rows as needed
    & \textbf{Action generation rules}:\\
    & 1. You should generate a single atomic action at each step.\\
    & 2. The action should be an atomic action from the given vocabulary - click, type, select, scroll (up or down), or stop.\\
    & 3. The arguments to each action should be within square braces. For example, "click [127]", "type [43] [content to type]", "scroll [up]", "scroll [down]".\\
    & 4. The natural language form of action (corresponding to the field "action\textunderscore in\textunderscore natural\textunderscore language") should be consistent with the grounded version of the action (corresponding to the field "grounded \textunderscore action"). Do NOT add any additional information in the grounded action. For example, if a particular element ID is specified in the grounded action, a description of that element must be present in the natural language action. \\
    & 5. If the type action is selected, the natural language form of action ("action\textunderscore in\textunderscore natural\textunderscore language") should always specify the actual text to be typed. \\
    & 6. You should issue a “stop” action if the current webpage asks to log in or for credit card information. \\
    & 7. To input text, there is NO need to click the textbox first, directly type content. After typing, the system automatically hits the `ENTER' key.\\
    & 8. STRICTLY Avoid repeating the same action (click/type) if the webpage remains unchanged. You may have selected the wrong web element.\\
    & 9. Do NOT use quotation marks in the action generation.\\~\\
    \cmidrule{2-2}
    
    & \textbf{Task proposal rules}: \\
    & 1. You should propose tasks that are relevant to the website and can be completed using the website.\\
    & 2. You should only propose tasks that do not require login to execute the task.\\
    & 3. You should propose tasks that are clear and specific.\\
    & 4. For each task, provide concrete information or constraints, and use mock-up information (identifier, number, personal information, name, attributes, etc.) to make the task more specific and realistic.\\
    & 5. The task description should provide all the necessary information to complete the task.\\
    & 6. The task should be feasible to complete by a real user and should not require any additional information that is not available on the website.\\~\\

    & \underline{\smash{The output should be in below format}}:\\
    & \textbf{OUTPUT FORMAT}: Please give a short analysis of the screenshot, parsed HTML/accessibility tree, then put your answer within \textasciigrave\textasciigrave\textasciigrave \; \textasciigrave\textasciigrave\textasciigrave, for example, "In summary, the proposed task and the corresponding action is: \textasciigrave\textasciigrave\textasciigrave\texttt{{\{"task": <TASK>:str, "action\_in\_natural\_language":<ACTION\_IN\_NATURAL\_LANGUAGE>:str, "grounded\_action": <ACTION>:str\}}"}\textasciigrave\textasciigrave\textasciigrave\\
\midrule
\bfseries User Role & Website URL: \{INIT\textunderscore URL\}\\
& Parsed HTML\slash Accessibility Tree: \{A11Y\textunderscore TREE\}\\
& \{SCREENSHOT\} \\
\bottomrule        
\end{longtable}
}
\captionof{table}{Prompt for Task Proposer Agent.}
%     \caption{Prompt for Task Proposal Agent}
\label{tab:proposal_prompt}
% \end{table*}
\twocolumn