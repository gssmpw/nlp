\section{Methods}
This work uses qualitative methods to explore the communication behaviors of players in \textit{League of Legends (LoL)}. By qualitatively observing and inquiring about player communication decisions as they occur, we aim to extract insights into players' reasoning, strategies, and the underlying factors influencing their choices during the actual conditions of gameplay. We observe \textit{LoL} players during real ranked games, asking them in-the-moment questions as well as follow-up interview questions after the matches have ended to capture the nuances of their communication decisions. 

\subsection{Participants}
We recruited participants for the study through university forums and social media in South Korea. Participants were required to be 18 or older and active players of Solo Ranked mode in \textit{LoL} with a valid rank during the current season at the time of the experiment (Season 2024). The recruitment post notified participants of the observational nature of the study and informed them that they would be expected to speak out loud and answer questions during their play sessions. A total of 36 players completed the recruitment survey, which asked for a self-report of their age, game history, preferred roles, and current rank.

We conducted in-person interview studies with a sample of 22 players. This sample excluded players who had played for less than a year, who were not willing to answer questions during the game, or who were unable to participate in person. From the remaining pool, players were chosen to maximize the diversity and representativeness of the player base based on their rank, experience, and roles. If several players shared similar profiles, we randomly selected between the participants. We conducted 17 interviews through this sampling method. Out of the first 17 participants, 16 participants identified as male, and one identified as female. Consequently, to increase the gender diversity of the sample and ensure that the results reflect a broad range of player experiences and perspectives, we specifically recruited female \textit{LoL} players through snowball sampling, while maintaining diversity in preferred roles and rank. We recruited and interviewed participants from the survey responders until qualitative saturation was reached, following the definition by Braun and Clark~\cite{braun2021saturation}. The final sample consisted of 16 male ($72.7\%$) and 6 female ($27.3\%$) participants. This ratio approximates the imbalanced gender demographic of \textit{LoL}, where estimates have suggested that $80$-$90\%$ of the player base is male~\cite{kordyaka2023gender}. We address the influence of gender identity on communication in Section ~\ref{discussion} and ~\ref{limitations}. The full list of participants and their information is shown in Table ~\ref{table:participant_info}.

The players' age ranged from 20 to 32 years old (mean=$23.7$ years, SD=$3.3$ years) and players' \textit{LoL} experience ranged from 2 to 13 years (mean=$7.7$ years, SD=$3.9$ years). The Solo queue ranks of the players were 2 Iron ($9.1\%$), 2 Bronze ($9.1\%$), 5 Silver ($22.7\%$), 7 Gold ($31.8\%$), 5 Platinum ($22.7\%$), and 1 Emerald ($4.5\%$) at the time of the study. Though the distribution is not as even as the Solo queue rank distribution in the Korean \textit{LoL} server ($12\%$ Iron, $19\%$ Bronze, $16\%$ Silver, $15\%$ Gold, $18\%$ Platinum, $13\%$ Emerald, and $5\%$ Diamond and above~\cite{log2024}), it encompasses the diverse range of skills of most \textit{LoL} players. Thus, the selected players reflected a comprehensive sample of engaged players with varying experience and skill levels. 


\begin{table*}
\centering
\caption{Participant Information and Game Session Information of \textit{League of Legends} Players}
    \label{table:participant_info}
\begin{tabular}{c|c|c|c|c|c|c} 
\toprule
\textbf{ID} & \textbf{Gender} & \textbf{Age} & \textbf{Experience} & \textbf{Solo Rank Tier} & \textbf{Role Played} & \textbf{Game Outcome}  \\ 
\hline
P1          & Male            & 24           & 12 years            & Silver                  & Jungle               & Win                    \\ 
\hline
P2          & Male            & 23           & 3 years             & Silver                  & Top                  & Loss                   \\ 
\hline
P3          & Male            & 32           & 12 years            & Bronze                  & Mid                  & Loss                   \\ 
\hline
P4          & Male            & 29           & 10 years            & Silver                  & Jungle               & Win                    \\ 
\hline
P5          & Male            & 27           & 11 years            & Emerald                 & Bot                  & Loss                   \\ 
\hline
P6          & Male            & 21           & 11 years            & Platinum                & Top                  & Win                    \\ 
\hline
P7          & Male            & 27           & 3 years             & Gold                    & Jungle               & Win                    \\ 
\hline
P8          & Male            & 20           & 9 years             & Gold                    & Bot                  & Win                    \\ 
\hline
P9          & Male            & 25           & 9 years             & Bronze                  & Jungle               & Loss                   \\ 
\hline
P10         & Male            & 23           & 7 years             & Silver                  & Jungle               & Loss                   \\ 
\hline
P11         & Male            & 21           & 12 years            & Gold                    & Mid                  & Loss                   \\ 
\hline
P12         & Male            & 22           & 11 years            & Platinum                & Mid                  & Loss                   \\ 
\hline
P13         & Female          & 24           & 3 years             & Gold                    & Support              & Loss                   \\ 
\hline
P14         & Male            & 25           & 10 years            & Platinum                & Bot                  & Win                    \\ 
\hline
P15         & Male            & 26           & 10 years            & Gold                    & Jungle               & Win                    \\ 
\hline
P16         & Male            & 26           & 13 years            & Gold                    & Mid                  & Loss                   \\ 
\hline
P17         & Male            & 25           & 8 years             & Platinum                & Support              & Loss                   \\
\hline
P18         & Female            & 21           & 3 years             & Platinum                & Support              & Loss                   \\
\hline
P19         & Female            & 20           & 6 years             & Gold                & Support              & Loss                   \\
\hline
P20         & Female            & 20           & 2 years             & Iron                & Top              & Loss                   \\
\hline
P21         & Female            & 20           & 3 years             & Iron                & Jungle              & Win                   \\
\hline
P22         & Female            & 21           & 2 years             & Silver                & Support              & Loss                   \\
\bottomrule
\end{tabular}
\end{table*}

\subsection{Procedure}
To capture the in-game mechanics of communication patterns and dynamically changing communication behavior in \textit{LoL}, we conducted an in-person observation and interview study. The study was conducted with the approval of the Institutional Review Board at the first author's research institution.

\subsubsection{Study Environment}

Each participant was asked to play a Solo Ranked game of \textit{LoL} while researchers observed and inquired about their actions in real time. In the process of study design, alternative study setups were considered. An initial plan to observe participants remotely through screen sharing was discarded as pilot studies revealed that latency and network issues were significantly disruptive to the researchers’ ability to observe and ask questions in a timely manner. The research team also decided not to recruit participants to observe in their own homes due to concerns about privacy and safety. Thus, to better gather responses to in-the-moment inquiries from the participants without any latency, the study was held in person within a controlled research environment to enhance the clarity and quality of the question-and-answer process. This approach also allowed researchers to note the participant's gaze, hesitation, and other subtle movements that would be missed in remote settings. Though an in-lab study does not perfectly recreate the in-home environment, the research team determined that this option best balanced the various tradeoffs.


The research team worked to ensure that the study environment best suited each participant’s preferences in the following steps. We first conducted several pilot interviews to tailor the play space to prioritize player comfort and approximate real-life conditions. The study took place in an enclosed room --- frequently used for interviews and user studies --- equipped with large desks and office chairs. We limited natural light and turned on overhead lights for visibility. Moreover, players were individually asked if the environment felt natural and comfortable, and adjustments were made based on their feedback. We set up the study environment with equipment (mouse, keyboard, monitor) designed specifically for online gaming. The players were also permitted to bring personal equipment if they wished. Before entering the game, the players were instructed to adjust both the equipment (e.g., mouse sensitivity) and in-game settings (e.g., shortcut keys). All players were given as much time as needed until they expressed satisfaction with the setup. At the end of the study, we asked players if anything about the setup or procedure had negatively impacted their gameplay. Three players reported feeling some discomfort from using unfamiliar equipment but noted that it did not affect their typical playstyle or their answers. The participants were compensated 20,000 KRW (approximately 15 USD) for completing the study. 


\subsubsection{Interview Process}

The researchers observed the game session through a separate screen connected to the player's monitor and noted any communication actions, attempts, and responses by the player. During the study orientation, researchers emphasized that participants should play and communicate naturally, including using offensive language, muting or reporting other players, or forfeiting the game if desired. Participants were assured that all data would be anonymized for analysis. To minimize distractions, they were informed that they could skip questions if they found them intrusive or preferred not to respond. During intense in-game situations in which the participant could not answer, the researchers documented the context and either repeated the question once the game state had stabilized or immediately after the match ended. We discuss the limitations of using an observational approach to study in-game communication, such as social desirability bias~\cite{grimm2010social}, in Section ~\ref{limitations}.

The researcher observed the communication between teammates, noting what triggered the communication or what communication medium was used for different purposes. Based on these observations, the participant was asked questions about why they did or did not perform certain communication actions to understand their assessment and perception of the communication. Some of the questions were asked to all participants, such as the reasoning behind the frequency of using certain communication media and their perception towards teammates who engage in certain forms of communication. Other non-structured questions were asked when the player triggered a certain action (``\textit{You just pinged your ally with Enemy Missing ping multiple times. What was the purpose?''}) or responded to (or ignored) their team's communication (``\textit{It seems that you opted to not vote for the surrender vote. Why is this so?}'').

The researchers recorded the gameplay and thoroughly transcribed observations and game states during the game. The observations included types of communication media used (chat, ping, votes, emotes), the target of the communication (if unclear, then players were asked to clarify the target), player reactions to ongoing discourse or communication usage by their team, and physical reactions such as spoken utterances and body gestures. After completing the game, participants engaged in a 15 to 20-minute post-game interview. They were asked to reflect on their in-game communication behavior and perceptions, including the motivations behind their communication tendencies and choices. The interview also inquired how their teammates' communication behaviors influenced their perception of those teammates, as well as the overall impact of such interactions on their mental state or performance. Participants were further invited to share suggestions for improving communication in \textit{LoL}, such as the potential addition of voice chat. The full list of the interview questions is provided in Appendix ~\ref{appendix2}.

Overall, a total of 24 games were played, of which two were forfeited within 20 minutes. Players were asked to play another game if their first game ended within 20 minutes due to a surrender vote from either team as these games did not demonstrate communication across all stages of the game. The other 22 games lasted a minimum of 24 minutes, most of which were played to completion without forfeit from either team or with a forfeit when the victor was very clearly determined.


\subsection{Thematic Analysis}
We conducted an inductive thematic analysis applying the methodology from Elo and  Kyngäs~\cite{elo2008qualitative}. The researchers gathered the transcript of the in-game and post-interview, video recording and the replay file of the game, and observational notes for each participant. We incorporated the notes on participant behavior, game states, and other players' communication patterns into the transcript at its corresponding times, providing contextual information on what was happening at the time of the question or the player's reaction.

Before initiating coding, the first, second, and third authors familiarized themselves with the data collected. The three authors then independently performed line-by-line open coding on eight participants' data to identify preliminary themes. After the initial coding was completed, the authors shared the codes to combine convergent ideas and discuss any differing perspectives. The first author then validated the codes on the remaining data, engaging in discussions with the second and third authors to iterate on the codebook. The final codebook contained 55 codes with 13 categories, organized by themes that answer each research question in depth. For RQ1, we find themes of communication types and what triggers or deters a player's decision to communicate. For RQ2, we categorize factors used by players to assess communication opportunities as well as reactions based on such assessment. For RQ3, we find themes on how the team's communication behaviors affect a player's perception towards other teammates during the game. We provide the final codebook in Appendix ~\ref{appendix}. 