\section{Introduction}

A team thrives and dies by its communication. Well-oiled communication is the engine that drives collaboration, underpinning crucial team processes. This holds particularly true for ad hoc teams, or swiftly formed temporary teams fulfilling a specific goal, as effective communication between unfamiliar members affects mission success~\cite{white2018, jarvenpaa1998, mesmer2009, marlow2018}. Thus, previous work indicates that ad hoc team structures may benefit from frequent, proactive communication to develop team cognition, enhance team cohesion, and strengthen interpersonal relationships~\cite{capiola2020, strater2008, roberts2014}. 

However, these communication strategies may not be applicable to ad hoc teams in virtual settings. Members in virtual ad hoc teams need to bridge their inherent physical and social distances without the aid of context-laden non-verbal cues~\cite{eisenberg2018, morrison2020}. Designing for optimal communication protocol in computer-mediated teamwork is not a universal experience. Conceptual models of ad hoc communication reveal that the specific characteristics of team context mediate positive outcomes of team communication~\cite{marlow2017}. For instance, increased communication openness and frequency can show a positive relationship with team performance~\cite{mesmer2009, monge2003}, but when the task centers around executing rapid, high-pressure decisions, explicit verbal communication may disrupt more vital team processes~\cite{entin1999, marlow2017}. Thus, each collaboration instance must factor in its virtuality and team characteristics to derive functional communication behaviors.

Online multiplayer games represent an especially challenging ad hoc collaboration domain for achieving effective communication. We situate our research on virtual ad hoc teamwork in \textit{League of Legends}\footnote{https://www.leagueoflegends.com/} (hereafter denoted as \textit{LoL}), a popular Multiplayer Online Battle Arena (MOBA) game that has been widely explored to understand team communication patterns in time-sensitive and high-intensity tasks~\cite{tan2022, zheng2023, leavitt2016, csengun2022players}. Previous work has identified communication antecedents of positive teamwork in MOBAs, such as communication sequences~\cite{tan2022} and hierarchical communication structures~\cite{kim2017}.

While previous literature has analyzed communication patterns through in-game data or isolated messages~\cite{tan2021less, tan2022}, communication processes in virtual ad hoc teams are more than just the sum of messages sent. There remains a gap in understanding what factors dynamically influence communication decisions in real time. Exploring the in-the-moment motivations behind communication decisions can reveal barriers to effective communication and illuminate how these decisions influence team collaboration in dynamic, real-world ad hoc contexts. Thus, this work seeks to capture the in-the-moment, player-centered perspective on how \textit{LoL} players decide to engage in different forms of communication within the game. We further connect these findings to evaluate how the communication experience and perception throughout the game affect the team members' attitudes towards their teammates, contributing to the ongoing discussions of communication and its effects on team trust and cohesion in virtual ad hoc teams. We thus approach this inquiry by first defining the forms and contexts of in-game communication. Then, we analyze the psychological and normative aspects that influence how players make communication decisions. We also investigate how communication processes shape players' perceptions of their teammates, focusing on the role these interactions play in fostering or hindering trust within the team.

In this paper, we address the following three research questions:
\begin{itemize}
    \item \textbf{RQ1. When and why do players engage in in-game team communication in \textit{League of Legends}?}
    \item \textbf{RQ2. How do players assess and react to in-game team communication in real time?}
    \item \textbf{RQ3. How does the player's experience with in-game communication processes shape their perception towards teammates?}
\end{itemize}

To this end, we conducted a qualitative analysis to identify context-dependent factors that drive communication decisions in real time. We performed an in-situ observational interview study with 22 players, observing their communication patterns and asking them to explain their communication choices during the game. This method directly engages with the player during their play, highlighting the proximate influences for their behavior. We follow in the footsteps of qualitative work by Buchan and Taylor~\cite{buchan2016} who looked at subjective player experiences in team coordination through interviews. However, we focus on the granular details of communication mechanisms and observe how they change and adapt throughout the game. We analyze players' usage and perception of the prominently used in-game communication methods, across both verbal (chat) and non-verbal (ping, emote, vote) methods. By incorporating both media, we paint a holistic picture of communication patterns and assess the natural trade-offs of each medium in different communicative contexts.

Our findings demonstrate that players evaluate immediately relevant in-game states, such as action cost and timeliness to make their communicative decisions, while bound by their expectations and norms set from repeated experiences. We show that players associate frequent communication with disruptive players, regardless of their communicative intent. Players instead virtue a teammate's commitment to the game, often demonstrated through action rather than explicit communication. Through this work, we motivate effective communication design that incorporates individual player context.


