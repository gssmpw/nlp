\section{Conclusion}

This study sheds light on the complex communication dynamics within virtual ad hoc teams of online multiplayer games by analyzing in-the-moment communication processes of \textit{LoL} players. Our findings reveal that players navigate communication not solely through the frequency of messages but by balancing verbal and non-verbal cues in response to the demands of high-pressure gameplay and the norms established through prior experiences. Rather than viewing constant communication as inherently beneficial, players often regard action and game performance as more significant indicators of a teammate's commitment. These insights suggest that fostering effective communication in online multiplayer games requires more than just encouraging message exchange; it involves designing systems that accommodate the nuanced and context-dependent nature of player interactions. By understanding when and why players choose to communicate, as well as how their experiences shape their perceptions of their teammates, we contribute to a broader understanding of communication strategies in virtual ad hoc teams. This work points towards more tailored communication protocols that reflect the unique needs of individuals and their virtual environments, ultimately supporting more cohesive and effective team performance.