\documentclass[sigconf]{acmart}

\usepackage{multirow}
\usepackage{lscape}



\newcommand{\revision}[1]{\textcolor{blue}{#1}}

\AtBeginDocument{%
  \providecommand\BibTeX{{%
    Bib\TeX}}}

\copyrightyear{2025}
\acmYear{2025}
\setcopyright{cc}
\setcctype{by}
\acmConference[CHI '25]{CHI Conference on Human Factors in Computing Systems}{April 26-May 1, 2025}{Yokohama, Japan}
\acmBooktitle{CHI Conference on Human Factors in Computing Systems (CHI '25), April 26-May 1, 2025, Yokohama, Japan}\acmDOI{10.1145/3706598.3714226}
\acmISBN{979-8-4007-1394-1/25/04}
\acmDOI{10.1145/3706598.3714226}


\begin{document}

\title[Understanding Players' In-game Assessment of Communication Processes in League of Legends]{Less Talk, More Trust: Understanding Players' In-game Assessment of Communication Processes in League of Legends}

\author{Juhoon Lee}
\email{juhoonlee@kaist.ac.kr}
\affiliation{%
  \institution{KAIST}
  \city{Daejeon}
  \country{Republic of Korea}
}

\author{Seoyoung Kim}
\email{youthskim@kaist.ac.kr}
\affiliation{%
  \institution{KAIST}
  \city{Daejeon}
  \country{Republic of Korea}
}

\author{Yeon Su Park}
\email{yeonsupark@kaist.ac.kr}
\affiliation{%
  \institution{KAIST}
  \city{Daejeon}
  \country{Republic of Korea}
}

\author{Juho Kim}
\email{juhokim@kaist.ac.kr}
\affiliation{%
  \institution{KAIST}
  \city{Daejeon}
  \country{Republic of Korea}
}

\author{Jeong-woo Jang}
\email{jangjw29@kaist.ac.kr}
\affiliation{%
  \institution{KAIST}
  \city{Daejeon}
  \country{Republic of Korea}
}

\author{Joseph Seering}
\email{seering@kaist.ac.kr}
\affiliation{%
  \institution{KAIST}
  \city{Daejeon}
  \country{Republic of Korea}
}

\renewcommand{\shortauthors}{Lee et al.}

\begin{abstract}
  In-game team communication in online multiplayer games has shown the potential to foster efficient collaboration and positive social interactions. Yet players often associate communication within ad hoc teams with frustration and wariness. Though previous works have quantitatively analyzed communication patterns at scale, few have identified the motivations of how a player makes in-the-moment communication decisions. In this paper, we conducted an observation study with 22 \textit{League of Legends} players by interviewing them during Solo Ranked games on their use of four in-game communication media (chat, pings, emotes, votes). We performed thematic analysis to understand players' in-context assessment and perception of communication attempts. We demonstrate that players evaluate communication opportunities on proximate game states bound by player expectations and norms. Our findings illustrate players' tendency to view communication, regardless of its content, as a precursor to team breakdowns. We build upon these findings to motivate effective player-oriented communication design in online games.
\end{abstract}

\begin{CCSXML}
<ccs2012>
   <concept>
       <concept_id>10003120.10003130.10011762</concept_id>
       <concept_desc>Human-centered computing~Empirical studies in collaborative and social computing</concept_desc>
       <concept_significance>500</concept_significance>
       </concept>
   <concept>
       <concept_id>10003120.10003121.10011748</concept_id>
       <concept_desc>Human-centered computing~Empirical studies in HCI</concept_desc>
       <concept_significance>500</concept_significance>
       </concept>
 </ccs2012>
\end{CCSXML}

\ccsdesc[500]{Human-centered computing~Empirical studies in collaborative and social computing}
\ccsdesc[500]{Human-centered computing~Empirical studies in HCI}

\keywords{League of Legends, multiplayer online battle arena, team communication, ad hoc teams, online games}

\maketitle

\section{Introduction}
\label{sec:intro}


\ps{Challenges of technology scaling}

The growing demand for computing performance has always been met by increasing the number of transistors per chip, which is only possible due to CMOS technology scaling.
However, as we keep pushing the boundaries of technology scaling, we encounter multiple challenges.
Firstly, whenever we transition to a more advanced technology node, the non-recurring cost due to physical design, verification, software, mask sets, and prototyping almost doubles \cite{cost-tech-node}.
As a result, designing a chip in an advanced technology node is only economically viable if the chip is manufactured in vast quantities.
Secondly, many chip components such as I/O drivers, analog circuits, or \gls{srams} have reached their scaling limit.
This means that we cannot shrink these components further, even if we use a more advanced technology with a smaller feature size.
Thirdly, advanced technology nodes suffer from high defect rates, diminishing the yield and inflating the recurring cost.
To tackle these challenges, new chip-design paradigms have been developed.

\ps{Why 2.5D integration?}

One of these new paradigms is 2.5D integration, where multiple silicon dies called chiplets are integrated into the same package.
Once designed, a single chiplet can be reused in multiple 2.5D stacked chips, which increases the ratio of production volume to non-recurring cost.
Another advantage is that multiple chiplets - fabricated in different technologies - can be integrated into the same package.
This means that only components that can take full advantage of technology scaling are built in bleeding-edge technologies.
Components that have reached their scaling limit are fabricated in more mature and hence less costly technology nodes.
Furthermore, chiplets are smaller than monolithic chips.
Therefore, manufacturing chiplets results in less silicon area loss due to fabrication defects and hence a higher yield.
Due to these economic advantages, chip vendors such as AMD \cite{amd-chiplet} and NVIDIA \cite{chiplet-book} have adopted the 2.5D integration paradigm.  

\ps{Challenges of 2.5D integration}

An important challenge when designing 2.5D stacked chips is the construction of a low-latency and high-throughput \gls{ici}. 
To build an \gls{ici}, we connect different chiplets using \gls{d2d} links.
These links are fabricated in an organic package substrate, silicon bridge, or silicon interposer, and they are connected to the chiplets using \gls{c4} bumps or microbumps.
The number of bumps per chiplet is limited, and so is the bandwidth of \gls{d2d} links.
In addition to having lower bandwidth than links in monolithic chips, \gls{d2d} links also have higher latency.
This latency is caused by wire delay and by \gls{phys} that are necessary in both the sending and the receiving chiplet.
\gls{phys} are needed to convert between protocols, voltage levels, and frequencies, which are usually different between on-chiplet links and \gls{d2d} links.
Due to these limitations, the \gls{ici} can quickly become a bottleneck.

\ps{How we solve these challenges differently than the related work does.}

Existing approaches to maximize the performance of the \gls{ici} either optimize the placement of chiplets (with potentially heterogeneous shapes) for a predetermined \gls{ici} topology 
\cite{ho,liu,seemuth,eris,osmolovskyi,tap25d,chiou}, select one topology out of a set of candidates \cite{coskun-1, coskun-2}, or they optimize the \gls{ici} topology for a 2D grid of homogeneously shaped chiplets on an active interposer \cite{butterdonut, cluscross, kite}.
To the best of our knowledge, there is no prior work on \gls{ici} topologies for chips with heterogeneously shaped chiplets or with passive silicon interposers or silicon bridges.
To fill this gap, we propose \name, a novel optimization methodology to jointly optimize the chiplet placement and \gls{ici} topology of such architectures.
\ifnb
\else
\newpage
\fi

\ps{Details on \name~and the key idea}

The key idea is as follows: 
We optimize the chiplet placement without a predetermined topology.
For each placement generated by an optimization algorithm, we infer a placement-based \gls{ici} topology by connecting chiplets that are in close proximity in that specific placement.
We then compute the latency and throughput of this combination of placement and topology for different traffic types.
These latencies and throughputs together with the total chip area are used to compute a user-defined quality-score of the placement, which is returned to the optimization algorithm.
Based on this quality score, the algorithm can further optimize the placement.
By following this iterative process, we jointly optimize the chiplet placement and the \gls{ici} topology.

\ps{Short evaluation-summary}

We provide our open-source framework implementing the proposed placement and topology co-optimization methodology, which we evaluate using both synthetic traffic and traffic traces.
A 2D grid of chiplets with a mesh topology is used as a baseline since many proposals for 2.5D stacked chips \cite{dataflow_accel_dnn, cifher, simba, hecaton, dojo} use such an architecture.
We reduce the latency of synthetic L1-to-L2 and L2-to-memory traffic, the two most important traffic types for cache coherency traffic, by up to 28\% and 62\% respectively.
For real traffic traces, we reduce the average packet latency for almost all traces and architectures considered (reduced by an 8\% or 18\% on average depending on the configuration of \gls{phys} within a chiplet).


\section{Related Work}
\label{sec:rw} 

Advancements in computational capabilities have shifted the primary constraint for LLM inference from processing power to memory bandwidth and energy efficiency. 
There are currently various methods to reduce the memory requirement~\cite{ojika2020addressing} of LLMs, such as pruning, quantization, and matrix decomposition. 
For example, Squeezellm~\cite{kim2023squeezellm} proposes sensitivity-based non-uniform quantization for searching the optimal bit precision assignment, completely based on second-order information, and dense and sparse decomposition for storing outliers and sensitive weights to reduce the memory requirement.
However, in this paper only pruning and quantization have been utilized, and they are discussed further in Section~\ref{subsec:model-optimization}.

In contrast to the on-device optimization techniques such as pruning and quantization, which are essential for deploying LLMs on resource constrained edge devices, cloud-based LLMs offer a more accessible deployment pathway. Cloud providers typically offer pre-trained models with ready-to-use endpoints, enabling users to interact with LLMs via application programming interfaces (APIs). This approach simplifies integration and facilitates scalable, concurrent requests, making it relatively easy to leverage LLM capabilities without the need for hardware-specific optimizations. 
However, frequent reliance on cloud endpoints for inference can lead to increased operational costs and latency issues, especially in scenarios requiring repeated or task-specific queries. To address this, recent work~\cite{dong2024creating} has shown that using LLMs as offline compilers for creating task-specific code can effectively help avoid frequent LLM endpoints accesses and reduce costs. 


%****************************************************************************************************

\subsection{LLMs at the Edge}

The deployment of LLMs demands significantly greater computational resources as compared to traditional DNNs such as convolutional neural networks (CNNs). 
This substantial resource requirement represents a key challenge in extending the use on LLMs to edge devices, and there have been continuous efforts for running resource-efficient LLM inference~\cite{haris_SECDA-LLM_2024}. 
The survey by Friha et al.~\cite{04} addresses the issue of edge-based language intelligence by providing a thorough examination of LLM-based Edge Intelligence (EI) architectures, focusing on security, optimization, and responsible development. 
Qin et al.~\cite{qin2024empirical} stated that different compression techniques are good at different types of tasks, and other guidelines for deploying LLMs onto resource-constrained devices effectively.  
Cheng et al.\cite{cheng2023optimize} focused on weight only post-training quantization, while AWQ~\cite{lin2024awq} uses an uneven weight quantization for preserving inference accuracy. Lamini-lm~\cite{wu2023lamini} uses knowledge distillation to perform effective compression, and MiniLLM~\cite{gu2024minillm} aims to minimize reversed Kullback-Leibler divergence and improvise effective compression.
EdgeMoE~\cite{yi2023edgemoe} proposed a more efficient inference for LLMs through a mixture-of-experts-based approach, where weights that occupy less storage but require computing are kept in memory throughout the time. Fu et al.~\cite{fu2024break} suggested a more aggressive lookahead decoding, while Malladi et al.~\cite{malladi2023fine} used memory-efficient zerothorder optimizer (MeZO) to estimate model gradients by forward propagation only. 

Despite these advancements, the primary focus of existing works has been on performance-centric optimizations, with limited attention to addressing fairness and bias in edge-deployed LLMs. 


%****************************************************************************************************

\subsection{LLMs and bias}

LLMs have fundamentally redefined Information Retrieval (IR) systems through the introduction of model generated data as a new data source which led the shift from passive data collection to proactive processing. 
This shift has raised new concerns for systems about adapting data bias and unfairness. Deldjoo~\cite{deldjoo2024understandingbiaseschatgptbasedrecommender} has shown how the difference in prompt design and formation strategies can impact the precision of the model recommendation. 
Recent surveys~\cite{amigo2023unifying,deldjoo2024fairness} have shown that biases can be categorized into several dimensions. 

The challenge of bias in LLM-generated recommendations is further compounded by the unregulated nature of internet data and the repetitive patterns in model-generated data used for the retraining. These factors contribute to the reinforcement of bias, making it difficult to uphold ethical standards in the model’s output. 
Several studies~\cite{winograd2022loose, jones2022artificial} suggested that there is no method at present for removing one’s imprints from a model with absolute certainty, except for retraining of the model from scratch which is not a possibility when we are considering the immense amount of data they are trained upon and high computational demand for processing these data.
Zhang et al.~\cite{zhang2023chatgpt} has proposed the benchmark fairness of recommendation via LLMs  that accounts for eight key attributes and further evaluated ChatGPT. 

Other recent works~\cite{deldjoo2024cfairllm, ghanbarzadeh2023gender} also demonstrated that incorporating any of the explicit user-sensitive attributes, such as gender or race, into the model can result in relatively biased recommendations to queries. The issue also raises a vital concern for security and privacy, as some studies~\cite{mohsin2024can,yao2024survey,yan2024protecting} show how LLMs have only a limited understanding of security principles, which can create new vulnerabilities and lead to the misuse of user-sensitive attributes for targeted attacks. 
These challenges underscore the critical need for developing robust mechanisms to ensure fairness, transparency, and accountability in the design and deployment of LLMs. 
Jaff et all~\cite{jaff2024data} developed an LLM-based framework to analyze the privacy policy for automatically checking the consistency of data.


While LLMs are capable of learning and memorizing attributes like name, the chances of bias is even higher as these information allows a model to form the context pattern and interrelate the topic for generating output text. This ability to establish the contextual relevance, though powerful, can inadvertently amplify biases present in the underlying training data or model-generated outputs which is further observed in~\ref{sec:eval} and discussed in~\ref{sec:disc}. 
Also, previous studies~\cite{radcliffe2024automated, metzger1999sign, nijodo2024automated, bianchi2023easily} have observed that subtle discrepancies in phrasing or structure of prompts can influence the tone, inclusivity, or neutrality of the model’s responses, which again raises a critical ethical question. 

However, previous work~\cite{taubenfeld2024systematic, ye2024justice, liang2024learning, qu2024mobile} have not examined how iterative interactions can reinforce these biases over time, particularly in edge environments where model retraining may not be feasible. 
This paper aims to bridge this gap by conducting a comparative analysis across cloud-based, desktop, and edge-deployed models, revealing that edge-optimized LLMs exhibit significantly higher bias rates. Additionally, this work proposes an iterative feedback loop that effectively mitigates bias on edge devices, offering a novel approach to enhancing fairness in resource-constrained deployments. 
\section{Study Context: League of Legends}
\textit{League of Legends} (\textit{LoL}) is a popular Multiplayer Online Battle Arena (MOBA) game, whose genre is defined by two competitive teams of human players battling for a common objective. In \textit{LoL}, two symmetrical teams of five members aim to destroy the other team's base (\textit{Nexus}). Each player selects a character from a pool of \textit{champions}, each equipped with unique abilities. Notably, ~\textit{LoL} game sessions are relatively short, generally lasting from 25 to 40 minutes. 

As \textit{LoL} is a \textit{competitive} team game, the outcome of the game depends heavily on cooperation between team members to achieve victory. The game supports real-time cooperation through diverse within-team communication channels native to the platform. In this paper, we investigate \textit{LoL} players' use of four main communication modes for corresponding with allies during the game: chat (verbal), pings, emotes, and votes (non-verbal). We illustrate how each communication mode may be used and represented in the game in Figure ~\ref{fig:modes}. The player base relies on these features to exchange key information, indicate intent, and express emotions throughout the entire session.

\begin{figure*}
    \centering
    \includegraphics[width=\textwidth]{four_modes.png}
    \caption{An example of four main communication modes of \textit{LoL}. (A) Chat is the medium through which players can type in-game messages to other players or read previous logs of in-game changes or signals, (B) Pings are quick alerts used for signaling information to other teammates, (C) Emotes are used to express emotions to other players, and (D) Votes are used to determine calls for objectives or to surrender the game.}
    \label{fig:modes}
    \Description{A screenshot of the gameplay screen of League of Legends, which demonstrates how chat, pings, emotes, and votes are shown in the game. The figure shows how each communication tool appears in the game, such as the chat box, ping and emote wheels, and the vote window on the right side of the screen.}
\end{figure*}

For verbal communication, players can type in the in-game chat before, during, and after the game (Figure ~\ref{fig:modes}A). Unlike other team-based competitive games such as \textit{Overwatch}\footnote{https://overwatch.blizzard.com/} and \textit{Valorant}\footnote{https://playvalorant.com/} or other MOBA games such as \textit{DoTA 2}\footnote{https://www.dota2.com/} and \textit{Heroes of the Storm} that offer voice chat for all teams, \textit{LoL} only enables it for a pre-formed party. Despite the potential benefits of voice-based communication for impromptu teams, \textit{LoL} developers have decided against voice chat, citing that it ``\textit{[does not solve] all behavioral issues and definitely introduces some new ones... Especially for women and POC (People of Color) who get unfairly targeted by simply participating in voice comms.}''~\cite{carver2023}


In \textit{LoL}, non-verbal communication is facilitated through pings, emotes, and votes. Pings are quick alerts used to signal information by placing markers on the map or characters. Players can access pings via a ping wheel (Figure ~\ref{fig:modes}B) or keyboard shortcuts. There are two types of pings: visual and UI pings. Visual pings, triggered by clicking the terrain or minimap, appear on the map and include generic markers for drawing attention and eight specific ``Smart'' pings (e.g., Retreat, On My Way, Assist Me) with predefined meanings shown in Figure ~\ref{fig:modes}B. UI pings share information about the status of the clicked interface elements, such as items or skills. Most pings, except non-targeted generic visual pings, are logged in the chat and accompanied by a distinct audio cue.

Emotes are expressive images or animations that convey emotions during the game, often featuring characters with various expressions like excitement, remorse, or provocations. When triggered, an emote appears above the player’s character for a few seconds and briefly in a bubble on allies’ screens. Unlike pings, emotes are visible to both allies and nearby enemies. Players can purchase emotes with in-game currency and customize their emote wheel (accessible via a shortcut) with up to nine options. Chat, pings, and emotes can be muted individually or entirely for specific players or everyone.

Finally, players can also communicate through surrender and objective votes. Starting at the 15-minute mark, a player may anonymously initiate a surrender vote, which appears on the right side of the screen. If at least 70\% of the team (or four players) vote ``Yes'' within 60 seconds, the game ends. If the vote fails, the team must wait three minutes to try again. In 2022, \textit{LoL} introduced objective voting, triggered when a player pings a field objective like Baron or Drake. This vote lets teammates decide whether to ``Take'' or ``Give'' the objective. We note that the chat system in \textit{LoL} textualizes all forms of communication. Chat messages, ping notifications, and in-game announcements are all interspersed in a single channel. 

These channels are used at different frequencies and at different points of the game. Ping usage is persistent and frequent throughout the game: the two most commonly used pings (On My Way and Enemy Missing) are used an average of $0.267$ and $0.164$ pings per minute respectively across all positions and ranks. There is a positive trend of increased ping with rank (reaching $0.489$ and $0.245$ pings per minute respectively for Master players and above)~\cite{log2024ping}, showing that as players become more skilled at the game, they are able to communicate more actively through non-verbal channels. In contrast, other non-verbal gestures are used sparingly. Though official statistics are not provided for votes and emotes, votes are limited by timers, which are tied to objectives (every 5–6 minutes), or surrender cooldowns (every 3 minutes). Emotes are used infrequently based on player norms and are often reserved for reacting to significant in-game events.

While previous works have often studied how players use one specific feature~\cite{tan2022, leavitt2016}, we aim to better understand how players use all of these features in conjunction. We illustrate the decisions players make in each communication attempt and how communication used by other team members shapes their team perception in real time.
\section{Methods}
\label{sec:method}
Given a model $\varepsilon_{\theta}$, fine-tuned by (\ref{eq:finetuning}) for a specific concept, we can identify two distinct sampling approaches, each maximizing one of the objectives: concept fidelity or editability:

Sampling with concept (Base sampling):
\begin{equation} 
  \label{eq:concept_sampling}
  \tilde{\varepsilon}_{\theta}(p^C) = \varepsilon_{\theta} + \omega(\varepsilon_{\theta}(p^C) - \varepsilon_{\theta}) = \varepsilon_{\theta} + \omega \Delta\varepsilon_{\theta}^{C}
\end{equation}
Sampling with superclass:
\begin{equation}
  \label{eq:superclass_sampling}
  \tilde{\varepsilon}_{\theta}(p^S) = \varepsilon_{\theta} + \omega( \varepsilon_{\theta}(p^S) - \varepsilon_{\theta}) = \varepsilon_{\theta} + \omega \Delta\varepsilon_{\theta}^{S}
\end{equation} 
% Here, $p^C$ represents a concept prompt embedding (for example, \textit{"a V* with a city in the background"}) and $p^S$ indicates a superclass prompt embedding (\textit{"a backpack with a city in the background"}) where the concept token $V^*$ is replaced by a superclass token (\textit{"backpack"}).
Here, $p^C$ represents a concept prompt embedding (for example, \textit{"a V* with a city in the background"}), and $p^S$ indicates a superclass prompt embedding (\textit{"a backpack with a city in the background"}) where the concept token $V^*$ has been replaced by a superclass token (\textit{"backpack"}).

\begin{figure*}[ht!]
  \centering
  \vspace{-0.02in}
  \includegraphics[trim={0 7cm 0 7cm},clip,width=\linewidth]{imgs/sampling_new.pdf}
  \vspace{-0.20in}
  \caption{\textbf{Visualization of Different Sampling Strategies.} (a) Usual sampling with concept reproduces the concept but does not align closely with the text prompt. (b) Generation with superclass effectively captures the context obtained from the prompt but produces a random superclass representative (e.g., dog). (c-d) Mixed and Switching sampling strategies improve context preservation while maintaining the concept's identity.}
  \label{fig:sampling}
  \vspace{-0.20in}
\end{figure*}

The extended fine-tuning of the model \(\varepsilon_{\theta}\) enhances its ability to accurately reproduce the concept generated via~(\ref{eq:concept_sampling}). However, this improvement comes at the cost of overlooking the contextual information supplied by the prompt $P^C$ (see Figure~\ref{fig:sampling}a). Conversely, the generation via~(\ref{eq:superclass_sampling}) ensures the highest alignment with the text prompt, though at the expense of preserving the concept's identity (see Figure~\ref{fig:sampling}b).

% This raises the question of whether we can integrate the two sampling strategies~(\ref{eq:concept_sampling}) and~(\ref{eq:superclass_sampling}) to obtain the optimal balance between the high fidelity of the learned concept identity and its adaptability to various contexts.

This consideration raises the question of whether we can integrate the two sampling strategies~(\ref{eq:concept_sampling}) and~(\ref{eq:superclass_sampling}) to obtain the optimal balance between the high fidelity of the learned concept identity and its adaptability to various contexts.

\subsection{Mixed sampling} \label{sec:mixed_sampling}
One reasonable approach for incorporating superclass into the generation process~\citep{profusion} is to modify the sampling strategy by adding guidance to the superclass prompt (see Figure~\ref{fig:sampling}c):
\begin{align}\label{eq:mixed_sampling}
    \tilde{\varepsilon}^{MX}_{\theta}(p^S, p^C) = \varepsilon_{\theta} + \omega_s \Delta\varepsilon_{\theta}^{S} + \omega_c\Delta\varepsilon_{\theta}^{C}
\end{align}
% By adjusting the ratio between the concept guidance scale $\omega_c$ and the superclass guidance scale $\omega_s$, we can either amplify or diminish the influence of the concept or superclass, thus varying the trade-off between concept and context fidelity. In Figure~\ref{fig:visual}, you can observe how the generated output alters with increasing superclass influence. For instance, in the teapot example, as we raise the superclass guidance scale, the context, which was initially poorly represented through sampling with the concept, gradually becomes more accurate. However, excessive superclass influence may result in a loss of concept identity preservation, as illustrated in the dog example.

Adjusting the ratio between the concept guidance scale $\omega_c$ and the superclass guidance scale $\omega_s$ amplifies or diminishes the influence of the concept or superclass, varying the trade-off between concept and context fidelity. Figure~\ref{fig:visual} shows how the generated output changes with increasing superclass influence. For instance, in the teapot example, as we raise the superclass guidance scale, the context, which was initially poorly represented through sampling with the concept, gradually becomes more accurate. However, excessive superclass influence may reduce concept identity preservation, as shown in the dog example.

\subsection{Switching sampling} \label{sec:switching_sampling}
Another solution for how to combine the superclass sampling trajectory with the concept sampling trajectory is to condition several steps on the superclass prompt embedding $p^S$, then at the \textit{switching step} $t_{sw}$ switch to the concept prompt embedding $p^C$  (see Figure~\ref{fig:sampling}d). In this case~(\ref{eq:sampling}) will be rewritten in the following form:
\begin{align}\label{eq:switching_sampling}
    \tilde{\varepsilon}^{SW}_{\theta}(p^S, p^C, t_{sw}) = \varepsilon_{\theta} +
    \begin{cases}
         \omega\Delta\varepsilon_{\theta}^{S}, & t > T - t_{sw}\\
        \omega\Delta\varepsilon_{\theta}^{C}, &\text{otherwise}
    \end{cases} 
\end{align}
By increasing the \textit{switching step} $t_{sw}$, we can amplify the influence of the superclass and thus improve context preservation.
Up to 10 steps can effectively recover context that has been poorly generated through Base sampling, as demonstrated in the teapot example in Figure~\ref{fig:visual}. Nonetheless, this strategy may result in notable degradation of the concept's identity. The effect of the superclass can be so intense that the concept loses its original attributes and takes on excessive characteristics from the superclass, as evidenced by the dog example in Figure~\ref{fig:visual}.

% This sampling procedure is similar to Photoswap~\cite{photoswap} approach adapted to the personalization task. The main difference is that in switched sampling we take the noise predictions entirely from the superclass trajectory for the first $t_{sw}$ steps, whereas in Photoswap only the self- and cross-attention maps and features are taken from the superclass for the first $t_{sw}$ steps. However, as we show in Section~\ref{sec:experiments}, the results of these two methods are almost indistinguishable.

This sampling procedure is similar to Photoswap~\cite{photoswap} but adapted to the personalization task. The main difference is that switched sampling takes noise predictions entirely from the superclass trajectory for the first $t_{sw}$ steps, whereas Photoswap uses only self- and cross-attention maps, and features are taken from the superclass for the first $t_{sw}$ steps. However, as we show in Section~\ref{sec:experiments}, the results of both methods are almost indistinguishable.

The aforementioned methods can be flexibly combined, we refer to this type of sampling as \textit{multi-stage sampling}:
\begin{align}\label{eq:multistage_sampling}
\tilde{\varepsilon}^{MS}_{\theta}(p^S, p^C) = \varepsilon_{\theta} +
    \begin{cases}
         (\omega_s + \omega_c)\Delta\varepsilon_{\theta}^{S} &t > T - t_{sw}\\[-0pt]
         \omega_s \Delta\varepsilon_{\theta}^{S} + \omega_c\Delta\varepsilon_{\theta}^{C}&\text{otherwise}
    \end{cases}
\end{align}
This combination enables a greater influence of the superclass on the generated output and enhances alignment with the text prompt. However, it is important to consider that as the influence of the superclass increases, the more the concept's identity is lost.

\subsection{Masked sampling}

\begin{figure*}[t!]
  \centering
  \includegraphics[trim={0 6.7cm 0 6.7cm},clip,width=\linewidth]{imgs/visual_new_v2.pdf}
  \vspace{-0.19in}
  \caption{\textbf{Effects of Superclass Influence on Different Sampling Methods.} 
  % For Mixed Sampling, the influence is adjusted by varying the superclass guidance scale $\omega_s = [1.0, 3.5, 5.0]$ with $\omega_c = 7.0 - \omega_s$. For Switching Sampling, we vary the switching step $t_{sw} = [3, 7, 20]$ . For Masked Sampling, the mask is modified by altering the thresholding quantile $q = [0.3, 0.5, 0.9]$. 
  For Mixed Sampling, the influence is adjusted by varying the superclass guidance scale $\omega_s$ with $\omega_c = 7.0 - \omega_s$. For Switching Sampling, we vary the switching step $t_{sw}$ . For Masked Sampling, the mask is modified by altering the concept mask thresholding quantile $q$.
  }
  \label{fig:visual}
  \vspace{-0.19in}
\end{figure*}

Sampling with a superclass prompt hinders the preservation of concept identity, whereas sampling with a concept prompt disrupts contextual adaptation. To address this challenge, restricting the image regions impacted by each sampling approach could be beneficial. This can be effectively achieved through masking.

Suppose at each diffusion step we could obtain a concept mask $M_t$, then we can use it in the Mixed sampling. Specifically, we apply this mask to the concept trajectory, ensuring it only influences relevant regions:
\begin{align}\label{eq:masked_base}
    \varepsilon^{M}_{\theta}(p^S, p^C) = \varepsilon_{\theta} + \omega\Delta\varepsilon_{\theta}^{C} \odot M_t + \omega\Delta\varepsilon_{\theta}^{S} \odot \overline{M_t} 
\end{align}
Moreover, to enhance the alignment between regions inside and outside the mask, and to gently amplify the influence of the superclass within the mask -- especially in cases where prompts alter the object's appearance (like color or outfit) -- we can apply Mixed sampling within the mask:
\begin{align}\label{eq:masked_mixed}
    &\varepsilon^{M}_{\theta}(p^S, p^C) = \varepsilon_{\theta} + \\ &+ \omega_c\Delta\varepsilon_{\theta}^{C} \odot M_t + 
    \omega_s\Delta\varepsilon_{\theta}^{S} \odot M_t +(\omega_c + \omega_s)\Delta\varepsilon_{\theta}^{S} \odot \overline{M_t}  \notag
\end{align}
The generation process starts with Mixed sampling for a limited number of steps, thereby enhancing the robustness of mask generation. Then, we apply masked sampling as described in (\ref{eq:masked_mixed}), using the concept mask $M_t(q)$. This mask is derived by averaging the cross-attention maps associated with the concept identifier token across all U-Net layers and binarizing it using a threshold determined by the quantile $q$:
\begin{equation}\label{eq:masked_sampling}
\tilde{\varepsilon}^{M}_{\theta}(p^S, p^C) = 
    \begin{cases}
         \tilde{\varepsilon}^{MX}_{\theta}(p^S, p^C, \omega_c^0, \omega_s^0), & t > T - t_{sw}\\
         \varepsilon^{M}_{\theta}(p^S, p^C, \omega_c, \omega_s, q),
    &\text{otherwise,}
    \end{cases} 
\end{equation}
where $\varepsilon^{M}_{\theta}(p^S, p^C, \omega_c, \omega_s, q)$ is computed as in (\ref{eq:masked_mixed}). 

Equation~\ref{eq:masked_sampling} summarizes the complete Masked sampling algorithm. Increasing the quantile $q$ reduces the area influenced by the concept, thereby expanding the region impacted by the superclass (see Appendix~\ref{app:cross_attn}) and enhancing the influence of the context, as illustrated in Figure~\ref{fig:visual}.

\subsection{Other approaches}
\textbf{ProFusion} The main contribution of the Profusion~\citep{profusion} sampling method is a novel technique to ensure the concept's preservation combined with Mixed Sampling. A sampling step in this approach consists of the following stages: (1) we predict $x_t \rightarrow \tilde{x}_{t-1}$ through the usual diffusion backward sampling process with concept (2) after that we make a forward diffusion step $\tilde{x}_{t-1}\rightarrow \tilde{x_t}$ (3) finally, we again make a backward step with the Mixed sampling  $\tilde{x_{t}} \rightarrow x_{t-1}$. The first two steps define Fusion Step and have a hyperparameter $r$ that controls its intensity (e.g. the influence on the result). In case $r=0$ we get Mixed sampling.

\textbf{Photoswap} In this method, the author proposes to replace self-attention features, cross-attention maps, and self-attention maps in the concept trajectory with maps from the superclass at several initial steps. Thus, the method has three hyperparameters: (1) $t_{SF}$ the number of initial steps during which the self-attention features are replaced, (2) $t_{CM}$ the same parameter for cross-attention maps, and (3) $t_{SM}$ for self-attention maps.

\subsection{Evaluation protocol for sampling techniques}
The study of sampling methods involves several key steps. 

% The first step is to select a fundamental fine-tune model on the basis of which we can compare different sampling techniques. For each model, we propose constructing a complete Pareto front of the Mixed sampling. We chose Mixed sampling as our baseline because it is the simplest efficient method, characterized by a single hyperparameter.

The first step is to select a fundamental fine-tuned model that will be used as a baseline for comparing different sampling techniques. For each model, we propose to construct the full Pareto front of Mixed sampling, which we selected as our baseline because it is the simplest yet efficient method, defined by a single hyperparameter.

% It is essential to select a model whose Pareto frontier exhibits a sufficiently large length; this allows for a clearer distinction between the varying parameters. Additionally, this front should lie within the optimal balance between concept fidelity and editability comparing to other fine-tuning methods. By doing so, we can examine sampling not only in scenarios where the model performs poorly but also ensure that sampling does not undermine performance in cases where the model excels.

It is crucial to select a model whose Pareto frontier is of sufficient length, enabling a clearer distinction between the varying parameters. Additionally, this frontier should lie within the optimal balance between concept fidelity and editability compared to other fine-tuning methods. This ensures that we can study sampling in scenarios where the model performs poorly while also confirming that it does not degrade performance when the model excels.

\begin{figure*}[ht!]
\centering
\begin{minipage}{.477\textwidth}
  \centering
  \includegraphics[trim={3cm 10cm 3cm 10cm},clip,width=\linewidth]{imgs/multi-stage.pdf}
  \vspace{-0.19in}
  \captionof{figure}{Pareto Frontier curves for Mixed, Switching and Multi-stage Sampling methods. Each Multi-stage sampling curve is generated by fixing the switching step while varying the superclass guidance scale $\omega_s = [1.0, 3.0, 5.0]$.}
  \label{fig:multi-stage}
\end{minipage}%
\hfill
\begin{minipage}{.477\textwidth}
  \centering
  \includegraphics[trim={3cm 10cm 3cm 10cm},clip,width=\linewidth]{imgs/masked.pdf}
  \vspace{-0.19in}
  \captionof{figure}{Pareto frontiers curves for  Masked sampling. Each Masked sampling curve is derived by varying the quantile \( q = [ 0.3, 0.5, 0.7, 0.9 ] \), which controls the mask binarization threshold; \( t_{sw} = 3, \omega_s = 3.5\) are fixed.}
  \label{fig:masked}
\end{minipage}
\vspace{-0.19in}
\end{figure*}

\begin{figure}[h!]
    \includegraphics[trim={3cm 10cm 3cm 10cm},clip,width=\linewidth]{imgs/all_mixed.pdf}
  \vspace{-0.23in}
    \caption{Mixed sampling Pareto frontiers for different fine-tuning methods.}
    \label{fig:all_mixed}
    \vspace{-0.24in}
\end{figure}
Once the base model is chosen, we fix it and proceed to compare different sampling techniques. For each method, we demonstrate its behaviour at different hyperparameter values. We illustrate the optimal points with generation examples and prove our findings with a user study.

It is important to note that the choice of sampling that maximizes editability can be approached in different ways. For example, one option is to use the model weights before fine-tuning, $\theta^{\text{orig}}$, in (\ref{eq:superclass_sampling}) instead of the fine-tuned weights, $\theta$. Additionally, we can vary the superclass prompts. One extreme option is to remove the superclass token entirely, allowing the model to focus solely on the scene's context (e.g., $p^{\hat{S}} = \textit{"with a city in the background"}$). These hyperparameters affect all sampling methods simultaneously. We analyze this dependency in Appendix~\ref{app:hyper_theta}.
\begin{table*}[t]
    \centering
    \resizebox{\textwidth}{!}{
\begin{tabular}{l|rrllrrll}
\toprule
\textbf{Dataset} & \multicolumn{4}{c}{\textbf{GSM8K}} & \multicolumn{4}{c}{\textbf{MATH}} \\
\cmidrule(lr){1-1} \cmidrule(lr){2-5} \cmidrule(lr){6-9}
\textbf{Method} & Acc & Len & Rel. Acc & Rel. Len & Acc & Len & Rel. Acc & Rel. Len \\
\midrule
\multicolumn{9}{l}{\textit{Zero-Shot Prompting}} \\
\midrule
\hspace{12pt}Baseline & 78.06 & 241.87 & 100.00 \small{(0.00)} & 100.00 \small{(0.00)} & 46.40 & 480.37 & 100.00 \small{(0.00)} & 100.00 \small{(0.00)} \\
\hspace{12pt}Be Concise & 77.98 & 214.87 & 99.85 \small{(1.18)} & 88.46 \small{(10.37)} & 47.76 & 446.09 & 102.71 \small{(7.59)} & 92.66 \small{(7.46)} \\
\hspace{12pt}Hand Crafted 2 (ours) & 76.72 & 184.13 & 98.27 \small{(3.67)} & 77.10 \small{(22.27)} & 46.84 & 404.85 & 101.62 \small{(4.79)} & 85.26 \small{(15.97)} \\
\midrule
\multicolumn{9}{l}{\textit{FT - External Data}} \\
\midrule
\hspace{12pt}Direct Answer & 19.70 & 3.17 & 24.88 \small{(5.03)} & 1.36 \small{(0.40)} & 15.08 & 6.98 & 35.16 \small{(10.34)} & 1.44 \small{(0.73)} \\
\hspace{12pt}Human CoT & 65.73 & 127.85 & 83.82 \small{(7.28)} & 54.95 \small{(13.17)} & 33.88 & 243.54 & 75.61 \small{(13.56)} & 53.14 \small{(13.87)} \\
\hspace{12pt}GPT4o CoT & 76.36 & 156.24 & 97.65 \small{(3.63)} & 67.60 \small{(16.70)} & 40.44 & 399.80 & 90.52 \small{(15.07)} & 87.21 \small{(22.22)} \\
\midrule
\multicolumn{9}{l}{\textit{FT - Best-of-N Self-Generation}} \\
\midrule
\hspace{12pt}Naive BoN & 77.12 & 214.22 & 98.79 \small{(1.64)} & 87.17 \small{(8.79)} & 47.64 & 433.26 & 101.74 \small{(7.04)} & 89.89 \small{(3.99)} \\
\hspace{12pt}Rational Metareasoning & 76.15 & 207.49 & 97.21 \small{(5.74)} & 84.93 \small{(5.09)} & 47.56 & 432.56 & 103.02 \small{(6.56)} & 90.56 \small{(5.25)} \\
\midrule
\multicolumn{9}{l}{\textit{FT - Few-Shot Conditioned Self-Generation (ours)}} \\
\midrule
\hspace{12pt}FS-Human & 76.66 & 161.72 & 98.06 \small{(3.28)} & 67.96 \small{(16.62)} & 46.44 & 421.54 & 99.69 \small{(6.97)} & 87.78 \small{(5.98)} \\
\hspace{12pt}FS-GPT4o & 78.07 & 175.54 & 99.94 \small{(1.69)} & 73.15 \small{(13.49)} & 47.36 & 421.21 & 101.87 \small{(5.33)} & 87.58 \small{(6.60)} \\
\hspace{12pt}FS-Self & 77.27 & 190.03 & 98.86 \small{(2.51)} & 77.51 \small{(9.18)} & 48.00 & 426.67 & 102.67 \small{(5.24)} & 88.50 \small{(4.49)} \\
\midrule
\multicolumn{9}{l}{\textit{FT - Few-Shot Conditioned Best-of-N Self-Generation (ours)}} \\
\midrule
% GPT4o Best-of-16 (Naive) & 75.48 & 153.51 & 96.56 \small{(3.79)} & 64.12 \small{(16.35)} & 47.28 & 367.49 & 101.50 \small{(9.81)} & 76.96 \small{(11.42)} \\
\hspace{12pt}FS-GPT4o-BoN & 75.88 & 153.38 & 97.00 \small{(4.11)} & 64.25 \small{(16.66)} & 47.36 & 364.33 & 102.56 \small{(6.24)} & 76.30 \small{(10.56)} \\
\hspace{24pt}\raisebox{0.5ex}{$\llcorner$}\hspace{4pt}\textit{Budget-Matched} & 76.24 & 160.59 & 97.44 \small{(3.67)} & 67.15 \small{(16.41)} & 47.52 & 384.43 & 101.58 \small{(9.53)} & 80.43 \small{(9.04)} \\

\bottomrule
\end{tabular}
    }
    \caption{
        Evaluation of zero-shot prompting, fine-tuning (FT), and our proposed methods on GSM8K and MATH datasets, averaged across five main models.
        We report absolute accuracy (\%) and length (tokens), along with relative accuracy (\%) and length (\%) compared to baseline prompting. Values in parentheses show standard deviations.
    }
    \label{tab:main_results}
\end{table*}

\section{Results}
\label{results}

% \begin{table*}[t]
% \resizebox{\textwidth}{!}{
% \centering
% \small
% \begin{tabular}{m{2.5cm}|m{15cm}}
% \toprule
% \textbf{Method} & \textbf{Content} \\ %& \# \textbf{Tokens}\\
% \midrule
% \midrule[0.05em]
% \multicolumn{2}{c}{Question} \\ 
% \midrule
% - & A robe takes 2 bolts of blue fiber and half that much white fiber.  How many bolts in total does it take?  \\
% \midrule
% \midrule[0.05em]
% \multicolumn{2}{c}{Reasoning Paths} \\
% \midrule
% Zero-Shot & To find the total number of bolts needed, we need to calculate the amount of white fiber first, since it's half the amount of blue fiber. Step 1: Determine the amount of blue fiber needed. The robe takes 2 bolts of blue fiber. Step 2: Calculate the amount of white fiber needed. The robe takes half the amount of white fiber as blue fiber. Half of 2 bolts is $2/2$ = 1 bolt. Step 3: Add the amount of blue fiber and white fiber to find the total. Total bolts = blue fiber + white fiber. Total bolts = 2 bolts + 1 bolt. Total bolts = 3 bolts. The answer is \textbf{\textit{3}}. \\ %& 213 \\
% \midrule
% FS-GPT4o-BoN & The robe takes 2 bolts of blue fiber. It takes half that much white fiber, which is $2 / 2$ = 1 bolt. Add the blue and white fiber together: 2 + 1 = 3 bolts. The answer is \textbf{\textit{3}}. \\ %57 \\
% \bottomrule
% \end{tabular}
% }
% \caption{Example rationales generated from the original Llama-3.1-8B model and the model fine-tuned using our method. The question is randomly sampled from GSM8K. For the same question, one can observe that our rationale is much more concise, removing redundant parts that explain the question.}
% \label{tab:samples}
% \vspace{-0.1in}
% \end{table*}


\begin{table*}[t]
\resizebox{\textwidth}{!}{
\centering
\normalsize
\begin{tabular}{m{17.5cm}}
\toprule
\multicolumn{1}{c}{\textbf{Question}} \\ 
\midrule
A robe takes 2 bolts of blue fiber and half that much white fiber.  How many bolts in total does it take? \\
\midrule
\multicolumn{1}{c}{\textbf{Reasoning Paths}} \\
\midrule
\textbf{[Zero-Shot]} 
To find the total number of bolts needed, we need to calculate the amount of white fiber first, since it's half the amount of blue fiber. Step 1: Determine the amount of blue fiber needed. The robe takes 2 bolts of blue fiber. Step 2: Calculate the amount of white fiber needed. The robe takes half the amount of white fiber as blue fiber. Half of 2 bolts is $2/2$ = 1 bolt. Step 3: Add the amount of blue fiber and white fiber to find the total. Total bolts = blue fiber + white fiber. Total bolts = 2 bolts + 1 bolt. Total bolts = 3 bolts. The answer is 3. \\
\midrule
\textbf{[FS-GPT4o-BoN]} 
The robe takes 2 bolts of blue fiber. It takes half that much white fiber, which is $2 / 2$ = 1 bolt. Add the blue and white fiber together: 2 + 1 = 3 bolts. The answer is 3. \\
\bottomrule
\end{tabular}
}
\caption{Example rationales generated from the original Llama-3.1-8B model (\textbf{Zero-Shot}) and the model fine-tuned using our method (\textbf{FS-GPT4o-BoN}). The question is randomly sampled from GSM8K. For the same question, one can observe that our rationale is much more concise, removing redundant parts that explain the question.}
\label{tab:samples}
\vspace{-0.1in}
\end{table*}


\subsection{Main results}

Our main results, presented in \autoref{tab:main_results} and \autoref{fig:main_methods_comparison}, demonstrate the performance of our self-training methods against baseline approaches.
% We highlight key observations from these results below.

\paragraph{Naive BoN fine-tuning is effective but sample inefficient.}
Naive BoN fine-tuning effectively reduces output length without significantly degrading model performance. 
This also holds true for Qwen2.5-Math-1.5B and DeepSeekMath-7B (\autoref{tab:main_results_full_gsm8k} and \autoref{tab:main_results_full_math}), which failed to achieve length reduction through zero-shot prompting.
% However, while naive BoN does reduce output length, the reduction is limited to 12\%.
However, the length reduction from naive BoN with $N=16$ is limited to 12\% on average.
Furthermore, as illustrated in Figure~\ref{fig:bon_sample_efficiency}, achieving more compression with BoN becomes progressively less efficient.

\paragraph{Iterative baseline yields similar results as naive BoN fine-tuning.}
% We compare our single-step naive BoN approach with Rational Metareasoning \cite{de2024rational}, an iterative approach using expert iteration \cite{zelikman2022star}  which incorporates an additional \textit{utility reward} to balance efficiency and accuracy in BoN sampling.
Rational Metareasoning, an iterative baseline, yields similar relative length reduction and relative accuracy to BoN fine-tuning. 
This suggests that the utility reward proposed by \citet{de2024rational} may not effectively achieve both accuracy gains and token length reduction.

\begin{figure}[t] % "h" places the figure roughly here
    \centering
    \includegraphics[width=\columnwidth]{figures/main_methods_comparison.pdf} % Adjust width as needed
    \caption{Tradeoff between relative accuracy and length reduction for main methods. Results are averaged over GSM8K and MATH across five main models. Matching colors and shapes indicate the same FS prompt. FS conditioning without augmentation (†) are marked with lighter colors. 
    Relative length reduction refers to 100 - relative length (\%).}
    \label{fig:main_methods_comparison} % Label for referencing in text
\end{figure}
% \red{TODO - shorten this}

\paragraph{Few-shot conditioning outperforms BoN in length reduction.}
The results demonstrate that few-shot conditioning achieves a greater relative length reduction compared to naive BoN, including math-specialized models (\autoref{tab:main_results_full_gsm8k} and \autoref{tab:main_results_full_math}).
% This reduction is attributed to the fact that the fine-tuning datasets generated through few-shot conditioning contain shorter reasoning paths compared to those generated by naive BoN, as illustrated in \autoref{fig:bon_sample_efficiency}.  % too long
This is in line with the superior length reduction of few-shot conditioning, compared to naive BoN as shown in \autoref{fig:bon_sample_efficiency}.
Notably, self-training on generations conditioned on human-annotated examples (FS-Human) achieves an average relative length of 67.96\% on GSM8K, compared to 87.17\% with naive BoN.  % good to have some specific numbers in the text
% We further analyze the effect of fine-tuning on length reduction in \autoref{analysis}.



\paragraph{Self-training better preserves accuracy than training with external data.} 
\autoref{tab:main_results} shows fine-tuning with external data (\textit{FT-External Data}) leads to a significant reduction in relative length but causes a severe drop in relative accuracy. 
% \autoref{fig:main_methods_comparison} further highlights that while fine-tuning with GPT-4o CoT (FT-GPT4o) achieves slightly greater reduction in relative length than fine-tuning with self-generated data using few-shots from GPT-4o (FS-GPT4o), it results in substantially lower relative accuracy.  % a bit complicated / not concrete (conrete evidence = one where we beat external FT in both accuracy and reduction)
\autoref{fig:main_methods_comparison} further highlights the accuracy preservation of self-training: fine-tuning with external concise reasoning supervision from GPT-4o (FT-GPT4o) lies below the Pareto-curve of relative accuracy and relative length reduction, established by our self-training methods.
% NAMGYU - TODO add some commentary

\paragraph{Few-shot conditioned BoN achieves best length reduction while maintaining accuracy.}
% Few-shot conditioned BoN enables substantial length reduction compared to all other BoN and few-shot methods while maintaining relative accuracy.
FS-BoN elicits the largest length reduction among our self-training methods, while maintaining relative accuracy, on average.
Notably, for math-specialized models, FS-GPT4o-BoN achieves the greatest reduction among all methods, except those fine-tuned on external data which greatly sacrifice the accuracy (\autoref{tab:main_results_full_gsm8k} and \autoref{tab:main_results_full_math}). 
% This result reflects how applying BoN to few-shot conditioning further reduces the relative length of the training data while also increasing the proportion of correct samples.  % unnecessary

\paragraph{Augmentation boosts accuracy for few-shot conditioning.}
\autoref{fig:main_methods_comparison} compares few-shot conditioning, i.e., FS and FS-BoN, with and without augmentation (†). 
Augmentation improves accuracy by providing solutions for previously unsolvable hard questions as discussed in \autoref{sample_augmentation}. 
While augmentation may slightly affect reduction rates, they remain superior to naive BoN and RM.
% Similar effect is observed for augmentation in FS-BoN.
% Even when matching the budget (\textit{Budget-Matched}) with other fine-tuning methods using self-generated data in \autoref{tab:main_results}, it achieves the greatest length reduction among them with minimal accuracy degradation.
Even when matching the budget (\textit{Budget-Matched}) with other self-training methods in \autoref{tab:main_results}, it achieves the greatest length reduction among them with minimal accuracy degradation.
The effect of augmentation on training data length is analyzed in \autoref{appx_augmentation_length}.
% Furthermore, as shown in Figure \ref{fig:main_methods_comparison}, augmentation on few-shot conditioned BoN enhances accuracy similar to naive BoN and Meta-Reasoning while achieving greater length reduction.

\begin{figure}[t]
    \centering
    \includegraphics[width=\columnwidth]{figures/length_by_difficulty.pdf} % Adjust width as needed
    \caption{\textbf{Tokens are reduced adaptively according to question difficulty.} 
    Token reduction rate for each difficulty level on MATH, for 4 models individually and averaged.
    % Higher difficulty levels show lower reduction rates.
    Relative length reduction refers to 100 - relative length (\%).
    }
    \label{fig:length_difficulty} % Label for referencing in text
\end{figure}

\subsection{Analysis}
\label{analysis}
% This section analyzes length reduction: transfer from generation to fine-tuning, reduction by question difficulty, qualitative analysis, and consistency across model sizes. DeepSeekMath-7B is excluded from quantitative analysis due to cost.
% let's keep this short
In this section, we analyze the length reduction effects in depth.
We exclude DeepSeekMath-7B from quantiative analysis due to cost.


% \paragraph{Analysis on sample efficiency}
% As shown in \autoref{fig:bon_sample_efficiency}, best-of-n (BoN) sampling requires a substantial number of samples to be generated to achieve a level of reasoning length reduction comparable to that achievable through few-shot conditioning.
% In other words, it is infeasible to reach the reasoning length reduction performance of few-shot conditioning using BoN alone, without generating a prohibitively large number of samples.
% However, our experiments consistently demonstrate that combining few-shot conditioning with BoN sampling is more effective in reducing reasoning length than using either technique in isolation.
% Specifically, few-shot conditioning helps to guide the model towards generating more concise reasoning paths, while BoN sampling allows us to select the shortest and most accurate path from a diverse set of candidates.
% This synergistic effect results in a more efficient and effective approach to concise reasoning.


% \paragraph{FT can reduce generation length effectively.}
% As shown in \autoref{fig:ft_length_scatter}, after fine-tuning, the models tend to follow the length of the training data, suggesting that reasoning length reduction can be achieved through simple supervised fine-tuning on short reasoning samples.
% Note that test generation length is relatively longer than the training data length, as the models can generate lengthy incorrect answers, while the training data consists of correct answers.
% Correctly generated answers align more closely with training data length as shown in (Appendix~\ref{appx_length_scatter_correct}).

% \paragraph{Length reduction through generation and fine-tuning}
% Our method reduces reasoning length in two stages: generation and fine-tuning.
% First, as shown in \autoref{fig:ft_length_scatter}, 
% % generation length for training data varies depending on the method. 
% few-shot conditioning methods produce shorter outputs than naive BoN, with few-shot conditioned BoN achieving the shortest. 
% Second, fine-tuning with shorter rationales results in shorter model outputs, showing a strong correlation between test and training lengths\footnote{Test generation lengths are generally longer than training data lengths due to the possibility of lengthy incorrect answers during testing. Test outputs that are correct align more closely with training data lengths, as shown in Appendix~\ref{appx_length_scatter_correct}.}.
% Overall, FS-GPT4o-BoN effectively generates and trains for shorter reasoning paths.
% Additionally, unlike zero-shot methods, our approach significantly reduces token length in math-tuned models like Qwen2.5-Math-1.5B with FS-GPT4o-BoN, achieving 54.7\% relative length after fine-tuning. (See \autoref{tab:main_results_full_gsm8k} and \autoref{tab:main_results_full_math}).

\paragraph{Tokens are reduced adaptively according to question complexity.} 
The MATH dataset's difficulty levels range from 1 (basic algebra) to 5 (advanced calculus and complex mathematical reasoning).
As shown in \autoref{fig:length_difficulty}, our method adaptively reduces tokens based on question difficulty, with higher difficulty leading to less reduction.
% Most models achieve their peak reduction (around 20\%--40\%) at difficulty levels 1-2, where simple concepts allow for more concise explanations.
% The reduction rate gradually declines at levels 3-5, indicating our method's ability to preserve necessary details for complex problems automatically.
%  -> not precise. simple concepts allow for more concise explanations *in absolute terms*, but this does not necessarily mean that length reduction *relative to the default* should be high. E.g., if the model already uses very few tokens for easy questions, then relative reduction would be low.
The higher reduction (20\%--40\%) at easier difficulty levels (1--2) suggests that the original model outputs for these easier questions contained unnecessary tokens.
This reveals a gap in current models' ability to tailor their inference budget to problem complexity.
Our method effectively closes this gap by reducing redundancy, allowing for more precise token allocation based on question difficulty.

\begin{figure}[t] % "h" places the figure roughly here
    \centering
    \includegraphics[width=\columnwidth]{figures/scaling_methods_comparison.pdf} % Adjust width as needed
    \caption{Scaling study on baseline and few-shot conditioned self-training methods. Results are averaged over GSM8K and MATH for Llama 1B, 3B, and 8B.
    % Accuracy tends to be maintained, with greater length reduction using our FS-GPT4o(-BoN) method.
    Relative length reduction refers to 100 - relative length (\%).
    }
    \label{fig:scaling_methods_comparison} % Label for referencing in text
\end{figure}

\paragraph{Self-training maintains consistency across model scales.}
We conduct a scaling study on Llama-3.2-1B, 3B, and Llama-3.1-8B to examine consistency across different model sizes (\autoref{fig:scaling_methods_comparison}). 
Overall, token reduction increases as the model size increases, while the maintenance of accuracy does not show a strong correlation with model size. 
RM exhibits lower reduction rates compared to our few-shot conditioned self-training methods across all models and shows a decrease in accuracy with increasing model size. 
% The few-shot method also shows a similar trend in length reduction, but it achieves the best relative accuracy in the 3B model.
Our standalone few-shot conditioning method (FS-GPT4o) also shows a similar trend in length reduction, but better preserves accuracy.
Our joint FS-GPT4o-BoN method achieves the greatest reduction across all models, maintaining relative accuracy across different model sizes, especially in the largest 8B model.



\paragraph{Sample study}
\autoref{tab:samples} presents qualitative examples of reasoning paths generated by the model before and after fine-tuning with our method. 
The original reasoning exhibits verbosity, containing redundant processes such as question confirmation and repeated instructions. 
In contrast, the reasoning generated by our method includes only the necessary steps, significantly reducing the number of tokens while still arriving at the correct answer. 
% These examples demonstrate the effectiveness of our method in reducing token count. 
More examples are provided in the \autoref{appx_sample_studies}.

\begin{figure}[t]
    \centering
    \includegraphics[width=\columnwidth]{figures/both_length_scatter.pdf} % Adjust width as needed
    \caption{\textbf{Fine-tuning effectively transfers the length reduction to the model.} Correlation between the relative length of train data and test output averaged over GSM8K and MATH across 4 models. Training length includes only correct solutions. Solid points represent test lengths including all generated outputs, while lighter points indicate test lengths of correct solutions only.}
    \label{fig:ft_length_scatter} % Label for referencing in text
\end{figure}

\paragraph{Length reduction is transferred through fine-tuning.}
As shown in \autoref{fig:ft_length_scatter}, fine-tuning with shorter rationales results in shorter model outputs, showing a strong correlation between test and training lengths.
% Test generation lengths (solid datapoints) are generally longer than training data lengths due to the possibility of lengthy incorrect answers during testing.
% However, when comparing with test generation lengths that are correct (lighter datapoints), they align more closely with training data lengths.
We note that the length of test outputs (incorrect and correct) are longer than the length of training samples (only correct) on average.
This is mainly because incorrect paths are generally longer than correct ones.
We find a closer correspondence between train length and test length of correct samples only, indicated by the lighter datapoints.
This discrepancy suggests the need to terminate incorrect paths early to minimize redundant inference overhead.
We consider this for future work.

\section{Discussion}

In this paper, we explored the relationship between human evaluations and NLP benchmarks of chat-finetuned language models (chat LMs). Our work is motivated by the recent shift towards human evaluations as the primary means of assessing chat LM performance, and the desire to determine the role that NLP benchmarks should play.

Through a large-scale study of the Chat Llama 2 model family on a diverse set of human and NLP evaluations, we demonstrated that NLP benchmarks are generally well-correlated with human judgments of chat LM quality. However, our analysis also reveals some notable exceptions to this overall trend. In particular, we find that adversarial and safety-focused evaluations, as well as language assistance and open question answering tasks, exhibit weaker or negative correlations respectively with NLP benchmarks. We also explored predicting human evaluation scores from NLP evaluation scores using overparameterized linear regression models. Our results suggest that NLP benchmarks can indeed be used to predict aggregate human preferences, although we caution that the limited sample size and the assumptions of our models may limit the generalizability of these findings. Our results suggest that NLP benchmarks can serve as fast and cheap proxies of slower and expensive human evaluations in assessing chat LMs.

Additionally, our work highlights the need for further research into NLP evaluations that can effectively capture important aspects of LM behavior, such as safety, robustness to adversarial inputs, and performance on complex, open-ended tasks. It is possible that new NLP benchmarks can provide signals on these topics, e.g., \citep{wang2023decodingtrust}. Of particular interest is developing human-interpretable and scaling-predictable evaluation processes, e.g., \citep{schaeffer2024emergent, ruan2024observational,schaeffer2024predictingdownstreamcapabilitiesfrontier}. Developing and refining such evaluation methods \citep{madaan2024quantifyingvarianceevaluationbenchmarks}, as well as detecting whether evaluations scores faithfully capture models' true performance \citep{oren2023proving,schaeffer2023pretrainingtestsetneed,roberts2023cutoff,jiang2024investigatingdatacontaminationpretraining,zhang2024careful,duan2024uncoveringlatentmemoriesassessing} will be crucial for ensuring that LMs are safe, reliable, and beneficial as they become increasingly integrated into society.

% In conclusion, our study provides insights into the relationship between human evaluations and NLP benchmarks of chat language models. By leveraging the complementary strengths of both human and NLP benchmarks, we can build a more complete understanding of LM capabilities and behaviors, ultimately enabling the development of models more capable, trustworthy, and beneficial to society.

\section*{Limitations}

In our experiments, the pipelines are initiated with relatively clean and clear audio files, and the subsequent acoustic deterioration is done in a specific manner (reverberation and background sounds). Other acoustic settings are indeed possible for initiating the SLU pipeline, e.g., with a low-resourced lingual dialect, different speaker voices per turn, overlapping speech, microphone settings, and many other parameters. Our framework is robust to these variants, and the purpose of our experiments is to exemplify the utility of the framework.

Similarly, our experiments are limited to the configurations we defined, for demonstrating the framework. Other configurations could involve non-English languages, different tasks, models and SLU/NLU datasets. The resulting analyses could yield findings that are different from ours, which reiterates the need for a robust framework like ours.

The cleaning techniques we used depend on the reference transcript in order to identify the word/phrase types that we want to include in our analysis. Our experiments show how \textit{different types} of errors affect a downstream task. A cleaning technique can also be one that is used in practice without dependence on the reference transcript. In the latter case, our framework would indicate the effectiveness of a transcript-cleaning component within an SLU pipeline.


%\paragraph{Insights from our experiments.}

%The insights from our experiments are based on four task models that may be seen as eq

We emphasize that the behavior of a graph depends on the task metric applied, and the resulting analysis can therefore differ when using different metrics for the same task and data. When insights are gathered with the framework, it is important to strongly consider the metric used, or use several metrics to paint a fuller picture.



In this study, we performed the first large-scale analysis of data leakage across 83 software engineering (SE) benchmarks, covering three popular programming languages—Python, Java, and C/C++. By combining an efficient near-duplicate detection algorithm with extensive manual labeling, we ensured the accurate identification of leaked data.



Our findings show that while data leakage is generally low, with average leakage ratios of 4.8\%, 2.8\%, and 0.7\% for Python, Java, and C/C++ benchmarks respectively, some benchmarks exhibit higher leakage that requires attention. We identified four main causes of leakage: direct inclusion of benchmark data in pre-training datasets, overlap between source repositories, reliance on platforms like LeetCode, and shared data sources such as GitHub issues.
We also found that automatic detection methods, like Perplexity-based metrics, struggle to distinguish between leaked and non-leaked samples. Additionally, our experiments reveal that data leakage inflates evaluation metrics, with models performing significantly better on leaked samples. For instance, StarCoder-7b achieved a Pass@1 score 4.9 times higher on leaked samples, underlining the need to address leakage to ensure fair evaluations.
This study offers insights into data leakage status in SE benchmarks and its impact on LLM evaluation.


In the future, we aim to expand the analysis to additional benchmarks and explore new methods to prevent or further reduce data leakage.





\vspace{0.2cm}
\noindent \textbf{Acknowledgement.}  This research / project is supported by the National Research Foundation, under its Investigatorship Grant (NRF-NRFI08-2022-0002). Any opinions, findings and conclusions or recommendations expressed in this material are those of the author(s) and do not reflect the views of National Research Foundation, Singapore.



\begin{acks}
This work was supported by the Office of Naval Research (ONR: N00014-24-1-2290). We thank the \textit{League of Legends} players and community members for their insights and participation in our research.
\end{acks}

\bibliographystyle{ACM-Reference-Format}
\bibliography{main}


\appendix
\renewcommand{\thefigure}{S\arabic{figure}}
\setcounter{figure}{0}  
\renewcommand{\thetable}{S\arabic{table}}
\setcounter{table}{0} 


\section{Proofs}
\subsection{Proof of Proposition~\ref{prop:cos_sim_grads}}
\label{prf:prop_grad_grows}
\cosgrads*
\begin{proof}
    We are taking the gradient of $\mathcal{L}^\mathcal{A}_i$ as a function of $z_i$. The principal idea is that the gradient has a term with direction $\hat{z}_j$ and a term with direction $-\hat{z}_i$. We then disassemble the vector with direction $\hat{z}_j$ into its component parallel to $z_i$ and its component orthogonal to $z_i$. In doing so, we find that the two terms with direction $z_i$ cancel, leaving only the one with direction orthogonal to $z_i$.
    
    Writing it out fully, we have $\mathcal{L}^\mathcal{A}_i = -z_i^\top z_j / (\|z_i\| \cdot \|z_j\|)$. Taking the gradient amounts to using the quotient rule, with $f = -z_i^\top z_j$ and $g = \|z_i\| \cdot \|z_j\| = \sqrt{z_i^\top z_i} \cdot \sqrt{z_j^\top z_j}$. Taking the derivative of each, we have
    \begin{align*}
        f' &= -\mathbf{z}_j \\
        g' &= \|z_j\| \frac{z_i}{\sqrt{z_i^\top z_i}} = \|z_j\| \frac{\mathbf{z}_i}{\|z_i\|} \\
        \implies \frac{f' g - g' f}{g^2} &= \frac{- \left(\mathbf{z}_j \cdot \|z_i\| \cdot \|z_j\| \right) + \left(\|z_j\| \frac{\mathbf{z}_i}{\|z_i\|} \cdot z_i^\top z_j \right)}{\|z_i\|^2 \cdot \|z_j\|^2} \\
        &= \frac{-\mathbf{z}_j}{\|z_i\| \cdot \|z_j\|} + \frac{\mathbf{z}_i z_i^\top z_j}{\|z_i\|^3 \|z_j\|},
    \end{align*}
    where we use boldface $\mathbf{z}$ to emphasize which direction each term acts along. We now substitute $\cos(\phi_{ij}) = z_i^\top z_j / (\|z_i\| \cdot \|z_j\|)$ in the second term to get
    \begin{equation}
        \label{eq:quotient_rule}
        \frac{f' g - g' f}{g^2} = \frac{-\hat{z}_j}{\|z_i\|} + \frac{\mathbf{z}_i \cos(\phi)}{\|z_i\|^2}
    \end{equation}

    It remains to separate the first term into its sine and cosine components and perform the resulting cancellations. To do this, we take the projection of $\hat{z}_j = \mathbf{z}_j / \|z_j\|$ onto $\mathbf{z}_i$ and onto the plane orthogonal to $\mathbf{z}_i$. The projection of $\hat{z}_j$ onto $\mathbf{z}_i$ is given by
    \[ \cos \phi_{ij} \frac{\mathbf{z}_i}{\|z_i\|} \]
    while the projection of $\mathbf{z}_j / \|z_j\|$ onto the plane orthogonal to $\mathbf{z}_i$ is
    \[ \left( \mathbf{I} - \frac{z_i z_i^\top}{\|z_i\|^2} \right) \frac{\mathbf{z}_j}{\|z_j\|}. \]
    It is easy to assert that these components sum to $\mathbf{z}_j/\|z_j\|$ by replacing the $\cos \phi_{ij}$ by $\frac{z_i^\top z_j}{\|z_i\|\cdot \|z_j\|}$.

    We plug these into Eq.~\ref{eq:quotient_rule} and cancel the first and third term to arrive at the desired value:
    \begin{align*}
        \frac{f' g - g' f}{g^2} = &-\frac{1}{\|z_i\|} \cos \phi \frac{\mathbf{z}_i}{\|z_i\|} \\
        &- \frac{1}{\|z_i\|} \cdot \left( \mathbf{I} - \frac{z_i z_i^\top}{\|z_i\|^2} \right) \frac{\mathbf{z}_j}{\|z_j\|} \\
        &+ \frac{\mathbf{z}_i \cos(\phi)}{\|z_i\|^2} \\
        = &\frac{-1}{\|z_i\|} \cdot \left( \mathbf{I} - \frac{z_i z_i^\top}{\|z_i\|^2} \right) \frac{\mathbf{z}_j}{\|z_j\|}.
    \end{align*}
\end{proof}

We visualize the loss landscape of the cosine similarity function in Figure \ref{fig:cos_sim_surface}. 

\begin{figure}
    \centering
    \begin{subfigure}{0.45\linewidth}
        \centering 
        \includegraphics[width=1\linewidth]{Images/cosine_similarity_surface_with_circles.pdf}
    \end{subfigure}%
    \begin{subfigure}{0.45\linewidth}
        \centering 
        \includegraphics[width=0.8\linewidth]{Images/cosine_similarity_2D_heatmap.pdf}
    \end{subfigure}
    \caption{Cosine similarity with respect to the direction indicated by the blue line. Three circles of radii 0.5, 1, and 2 are superimposed to show that, for higher norms, the cosine similarity is less steep. Left: 3D Surface plot, right: 2D topview plot.}
    \label{fig:cos_sim_surface}
\end{figure}


\subsection{InfoNCE Gradients}
\label{app:infonce_grads}
\infoncegrads*
\begin{proof}
    We are interested in the gradient of $\mathcal{L}_i^\mathcal{R}$ with respect to $z_i$. By the chain rule, we get
    \begin{align*}
        \nabla_i^\mathcal{R} &= -\frac{\sum_{k \not\sim i} \text{ExpSim}(z_i, z_k) \frac{\partial \frac{z_i^\top z_k}{\|z_i\| \cdot \|z_k\|}}{\partial z_i}}{\sum_{k \not\sim i} \text{ExpSim}(z_i, z_k)} \\
        &= -\frac{\sum_{k \not\sim i} \text{ExpSim}(z_i, z_k) \frac{\partial \frac{z_i^\top z_k}{\|z_i\| \cdot \|z_k\|}}{\partial z_i}}{S_i}
    \end{align*}
    It remains to substitute the result of Prop. \ref{prop:cos_sim_grads} for $\partial \frac{z_i^\top z_k}{\|z_i\| \cdot \|z_k\|} / \partial z_i$.

    We sum this this with the gradients of the attractive term to obtain the full InfoNCE gradient, completing the proof.
\end{proof}

We note that the repulsive force is weighted average over a set of unit vectors. Consequently, the repulsive gradient is smaller than the attractive one. Additionally, we point out that these gradients are symmetric: just like positive and negative samples $z_j$ and $z_k$ affect $z_i$, $z_i$ affects $z_j$ and $z_k$.

\subsection{Proof of Corollary~\ref{cor:embeddings_grow}}
\label{prf:cor_embeddings_grow}
\begin{proof}
    First, consider that we applied the cosine similarity's gradients from Proposition~\ref{prop:cos_sim_grads}. Since $z_i$ and $(z_j)_{\perp z_i}$ are orthogonal, $\|z_i'\|_2^2 = \|z_i\|^2 + \frac{\gamma^2}{\|z_i\|^2}\|(z_j)_{\perp z_i}\|^2$. The second term is positive if $\sin \phi_{ij} > 0$.

    The same exact argument holds for the InfoNCE gradients. The gradient is orthogonal to the embedding, so a step of gradient descent can only increase the embedding's magnitude.
\end{proof}

\subsection{Proof of Theorem~\ref{thm:convergence_rate}}
\label{prf:thm_convergence_rate}
We first restate the theorem:

Let $z_i$ and $z_j$ be positive embeddings with equal norm, i.e. $\|z_i\| = \|z_j\| = \rho$. Let $z_i'$ and $z_j'$ be the embeddings after 1 step of gradient descent with learning rate $\gamma$. Then the change in cosine similarity is bounded from above by:
\begin{equation*}
    \hat{z}_i'^\top \hat{z}_j' - \hat{z}_i^\top \hat{z}_j < \frac{\gamma \sin^2 \phi_{ij}}{\rho^2} \left[ 2 - \frac{\gamma \cos \phi}{\rho^2} \right].
\end{equation*}

\noindent We now proceed to the proof:
\begin{proof}
    Let $z_i$ and $z_j$ be two embeddings with equal norm\footnote{We assume the Euclidean distance for all calculations.}, i.e. $\|z_i\| = \|z_j\| = \rho$. We then perform a step of gradient descent to maximize $\hat{z}_i^\top \hat{z}_j$. That is, using the gradients in \ref{prop:cos_sim_grads} and learning rate $\gamma$, we obtain new embeddings $z_i' = z_i + \frac{\gamma}{\|z_i\|} (\hat{z}_j)_{\perp z_i}$ and $z_j' = z_j + \frac{\gamma}{\|z_j\|} (\hat{z}_i)_{\perp z_j}$. Going forward, we write $\delta_{ij} = (\hat{z}_j)_{\perp z_i}$ and $\delta_{ji} = (\hat{z}_i)_{\perp z_j}$, so $z_i' = z_i + \frac{\gamma}{\rho} \delta_{ij}$ and $z_j' = z_j + \frac{\gamma}{\rho} \delta_{ji}$. Notice that since $z_i$ and $\delta_{ij}$ are orthogonal, by the Pythagorean theorem we have $\|z_i'\|^2 = \|z_i\|^2 + \frac{\gamma^2}{\rho^2}\|\delta_{ij}\|^2 \geq \|z_i\|^2$. Lastly, we define $\rho' = \|z_i'\| = \|z_j'\|$.

    We are interested in analyzing $\hat{z}_i'^\top \hat{z}_j' - \hat{z}_i^\top \hat{z}_j$. To this end, we begin by re-framing $\hat{z}_i'^\top \hat{z}_j'$:
    \begin{align*}
        \hat{z}_i'^\top \hat{z}_j' &= \left(\frac{z_i + \frac{\gamma}{\rho} \delta_{ij}}{\rho'}\right)^\top \left(\frac{z_j + \frac{\gamma}{\rho} \delta_{ji}}{\rho'}\right) \\
        &= \frac{1}{\rho'^2}\left[ z_i^\top z_j + \gamma \frac{z_i^\top \delta_{ji}}{\rho'} + \gamma \frac{z_j^\top \delta_{ij}}{\rho'} + \gamma^2 \frac{\delta_{ij}^\top \delta_{ji}}{\rho'^2} \right].
    \end{align*}

    We now consider that, since $\delta_{ij}$ is the projection of $\hat{z}_j$ onto the subspace orthogonal to $z_i$, we have that the angle between $z_i$ and $\delta_{ji}$ is $\pi/2 - \phi_{ij}$. Plugging this in and simplifying, we obtain
    \begin{align*}
        z_i^\top \delta_{ji} &= \|z_i\| \cdot \|\delta_{ji}\| \cos (\pi/2 - \phi_{ij}) \\
        &= \|z_i\| \cdot \|\delta_{ji}\| \sin \phi_{ij} \\
        &= \rho \sin^2 \phi_{ij}.
    \end{align*}
    By symmetry, the same must hold for $z_j^\top \delta_{ij}$.
    
    Similarly, we notice that the angle $\psi_{ij}$ between $\delta_{ij}$ and $\delta_{ji}$ is $\psi_{ij} = \pi - \phi_{ij}$. The reason for this is that we must have a quadrilateral whose four internal angles must sum to $2\pi$, i.e. $\psi_{ij} + \phi_{ij} + 2 \frac{\pi}{2} = 2 \pi$. Thus, we obtain $\delta_{ij}^\top \delta_{ji} = \|\delta_{ij}\| \cdot \|\delta_{ji}\| \cos(\psi) = -\sin^2 \phi_{ij} \cos \phi_{ij}$.

    We plug these back into our equation for $\hat{z}_i'^\top \hat{z}_j'$ and simplify:
    \begin{align*}
        \hat{z}_i'^\top \hat{z}_j' &= \frac{1}{\rho'^2}\left[ z_i^\top z_j + \gamma \frac{z_i^\top \delta_{ji}}{\rho} + \gamma \frac{z_j^\top \delta_{ij}}{\rho} + \gamma^2 \frac{\delta_{ij}^\top \delta_{ji}}{\rho^2} \right] \\
        &= \frac{1}{\rho'^2}\left[ z_i^\top z_j + \gamma \frac{\rho \sin^2 \phi_{ij}}{\rho} + \gamma \frac{\rho \sin^2 \phi_{ij}}{\rho} - \gamma^2 \frac{\sin^2 \phi_{ij} \cos \phi_{ij}}{\rho^2} \right] \\
        &= \frac{1}{\rho'^2}\left[ z_i^\top z_j + 2 \gamma \sin^2 \phi_{ij} - \gamma^2 \frac{\sin^2 \phi_{ij} \cos \phi_{ij}}{\rho^2} \right].
    \end{align*}

    We now consider the original term in question:
    \begin{align*}
        \hat{z}_i'^\top \hat{z}_j' - \hat{z}_i^\top \hat{z}_j &= \frac{1}{\rho'^2}\left[ z_i^\top z_j + 2 \gamma \sin^2 \phi_{ij} - \gamma^2 \frac{\sin^2 \phi_{ij} \cos \phi_{ij}}{\rho^2} \right] - \frac{z_i^\top z_j}{\rho^2} \\
        &\leq \frac{1}{\rho^2}\left[ z_i^\top z_j + 2 \gamma \sin^2 \phi_{ij} - \gamma^2 \frac{\sin^2 \phi_{ij} \cos \phi_{ij}}{\rho^2} \right] - \frac{z_i^\top z_j}{\rho^2} \\
        &= \frac{1}{\rho^2}\left[ 2 \gamma \sin^2 \phi_{ij} - \gamma^2 \frac{\sin^2 \phi_{ij} \cos \phi_{ij}}{\rho^2} \right] \\
        &= \frac{\gamma \sin^2 \phi_{ij}}{\rho^2}\left[ 2 - \frac{\gamma \cos \phi_{ij}}{\rho^2} \right]\\
        &\leq \frac{2 \gamma \sin^2 \phi_{ij}}{\rho^2}
    \end{align*}
    
    This concludes the proof.
\end{proof}

\section{Simulations}
\label{app:simulations}

\subsection{Aparametric Simulations}

For the simulations in Section \ref{ssec:convergence_simulations}, we produced two datasets, $\mathbf{X}_1$ and $\mathbf{X}_2$, independently by randomly sampling points in $\mathbb{R}^20$ from a standard normal distribution and normalizing them to the hypersphere. The $i$-th point in dataset $\mathbf{X}_1$ is the positive counterpart for the $i$-th point in dataset $\mathbf{X}_2$. The first dataset is then set to be static while the second is modified in order to control for the embedding norms and angles between positive pairs.

We optimize the cosine similarity by performing standard gradient descent on the embeddings themselves with learning rate $10$. We consider a dataset ``converged'' when the average cosine similarity between positive pairs exceeds $0.999$.

\paragraph{Controlling for angles.} In order to control for the angle between positive pairs, we use an interpolation value $\alpha \in [-1, 1]$. Let $x_1$ be a static embedding in $\mathbf{X}_1$ and $x_2$ be the embedding in $\mathbf{X}_2$ whose angle we wish to control. In expectation, $\phi(x_1, x_2)$ will be $\pi / 2$. We therefore define the embedding $x_2$ whose angle has been controlled as 
\[ x_2' = x_2 \cdot (1 - |\alpha|) + x_1 \cdot \alpha. \]

In essence, when $\alpha=0$, $x_2' = x_2$. However, when $\alpha=1$ (resp. $\alpha=-1$), $x_2' = x_1$ (resp. $x_2' = -x_1$).

\paragraph{Controlling for embedding norms.} This setting is simpler than the angles between positive pairs. We simply scale $\mathbf{X}_2$ by the desired value.

\subsection{Parametric Simulations}
\label{app:parametric_sim}

We restate the entire implementation for the simulations in Section \ref{ssec:confidence_simulations} for completeness. We choose centers for 4 latent classes $\{c_1, c_2, c_3, c_4\}$ uniformly at random from $\mathbb{S}^{10}$ by randomly sampling vectors from a standard multivariate normal distribution and normalizing them to the hypersphere. We then obtain the latent samples $\tilde{z}$ around center $c_i$ via $z \sim \mathcal{N}(c_i, 0.1 \cdot \mathbf{I})$ and re-normalizing to the hypersphere. For each center, we produce 1K latent samples; these constitute our latent classes. We depict an example of 8 such latent classes (in 3 dimensions) in Figure \ref{fig:orig_latents}. We finally obtain the dataset by generating a random matrix in $\mathbb{R}^{11 \times 64}$ and applying it to the latent samples.

We train the InfoNCE loss via a 2-layer feedforward neural network with the ReLU activation function in the hidden layer. The network's output dimensionality is $\mathbb{R}^{11}$ so that, after normalization, it can reconstruct the original latent classes. We train the network using the supervised InfoNCE loss with a batch size of 128. Each data point's positive pair is simply another data point from the same latent class.

We visualize the learned (unnormalized) embedding space in Figure \ref{fig:learned_latents}.

\begin{figure}
    \centering
    \begin{subfigure}{0.4\linewidth}
    \includegraphics[width=\linewidth]{Images/orig_latents.png}
    \caption{}
    \label{fig:orig_latents}
    \end{subfigure}
    \quad\quad
    \begin{subfigure}{0.4\linewidth}
    \includegraphics[width=\linewidth]{Images/learned_latents.png}
    \caption{}
    \label{fig:learned_latents}
    \end{subfigure}
    \caption{\emph{Left}: A depiction of $8$ latent classes in $3$D obtained via the description in Section \ref{app:parametric_sim}. Dashed lines represent vectors from the origin to the mean of the distribution. \emph{Right}: A depiction of the learned latent space (unnormalized) using the supervised InfoNCE loss after 50 epochs of training.}
    
\end{figure}


\section{Further Discussion and Experiments}
\label{app:experiments}

\subsection{Experimental Setup}
\label{app:experiment_setup}
Unless otherwise stated, we use a ResNet-50 backbone \cite{resnet} and the default settings outlined in the SimCLR \cite{simclr} and SimSiam \cite{simsiam} papers. We use $1$e-$6$ as the default SimCLR weight decay and $5$e-$4$ as the default SimSiam one. When running on Cifar-10 and Cifar-100, we amend the backbone network's first layer as detailed in \citet{simclr}. We use embedding dimensionality $256$ in SimCLR and $2048$ in SimSiam. When reporting embedding norms, we use the projector's output in SimCLR and the predictor's output in SimSiam: these are the spaces where the loss function is applied and therefore where our theory holds.

Due to computational constraints, we run with batch-size 256 in SimCLR. Although each batch is still 256 samples in SimSiam, we simulate larger batch sizes using gradient accumulation. Thus, our default batch-size for SimSiam is 1024. 

\subsection{Opposite-Halves Effects}
\label{app:opposite_halves_effect}

We devote this section of the Appendix to studying the role of the angle between positive samples on the cosine similarity's convergence under gradient descent. Referring back to Figure~\ref{fig:convergence_sim}, we see that the effect is most impactful when the angle between positive embeddings is close to $\pi$, i.e. $\phi_{ij} > \pi - \varepsilon$ for $\varepsilon \rightarrow 0$. The following result shows that this is exceedingly unlikely for a single pair of embeddings in high-dimensional space:
\begin{proposition}
    \label{prop:unlikely_opp_halves}
    Let $x_i, x_j \sim \mathcal{N}(0, \mathbf{I})$ be $d$-dimensional, i.i.d. random variables and let $0 < \varepsilon < 1$. Then \vspace*{-0.1cm}
    \begin{equation}
    \label{eq:opp_halves_unlikely}
    \mathbb{P}\left[ \hat{x}_i^\top \hat{x}_j \geq 1 - \varepsilon \right] \leq \frac{1}{2d(1-\varepsilon)^2}.
    \end{equation}\vspace*{-0.3cm}
\end{proposition}
\begin{proof}
By \citet{distribution_of_cosine_sim}, the cosine similarity between two i.i.d. random variables drawn from $\mathcal{N}(0, \mathbf{I})$ has expected value $\mu = 0$ and variance $\sigma^2 = 1/d$, where $d$ is the dimensionality of the space. We therefore plug these into Chebyshev's inequality:
\begin{align*}
    &\text{Pr} \left[ \left|\frac{x_i^\top x_j}{\|x_i\|\cdot \|x_j\|} - \mu \right|\geq k \sigma \right] \leq \frac{1}{k^2} \\
    \rightarrow & \text{Pr} \left[ \left |\frac{x_i^\top x_j}{\|x_i\|\cdot \|x_j\|} \right |\geq \frac{k}{\sqrt{d}} \right] \leq \frac{1}{k^2}
\end{align*}

\noindent We now choose $k = \sqrt{d}(1 - \varepsilon)$, giving us
\[ \mathbb{P}\left[ \left |\frac{x_i^\top x_j}{\|x_i\| \cdot \|x_j\|}\right | \geq 1 - \varepsilon \right] \leq \frac{1}{d(1-\varepsilon)^2}.\]

It remains to remove the absolute values around the cosine similarity. Since the cosine similarity is symmetric around $0$, the likelihood that its absolute value exceeds $1 - \varepsilon$ is twice the likelihood that its value exceeds $1- \varepsilon$, concluding the proof.

We note that this is actually an extremely optimistic bound since we have not taken into account the fact that the maximum of the cosine similarity is 1.
\end{proof}

The above proposition represents the likelihood that \emph{one} pair of embeddings has large angle between them. It is therefore \emph{exponentially} unlikely for every pair of embeddings in a dataset to have angle close to $\pi$, as we would require Proposition \ref{prop:unlikely_opp_halves} to hold across every pair of embeddings. Thus, the opposite-halves effect is exceedingly unlikely to occur.

\begin{table}
    \centering
    \quad
    \parbox{.47\linewidth}{
        \begin{tabular}{lrcc}
        \toprule
        Model & Dataset \quad\quad & \makecell{Effect Rate\\Epoch 1} & \makecell{Effect Rate\\Epoch 16} \\
        \midrule
        \multirow{2}{*}{SimCLR} & Imagenet-100 & 2\% & 0\%  \\
        & Cifar-100 & 11\% & 1\% \\
        \cmidrule{1-4}
        \multirow{2}{*}{SimSiam} & Imagenet-100 & 26\% & 1\% \\
        & Cifar-100 & 21\% & 0\% \\
        \cmidrule{1-4}
        \multirow{2}{*}{BYOL} & Imagenet-100 & 28\% & 1\% \\
        & Cifar-100 & 20\% & 0\% \\
        \bottomrule
        \end{tabular}
        \captionof{table}{The rate at which embeddings are on opposite sides of the latent space (angle between a positive pair is greater than $\pi / 2$) for various datasets and SSL models.}
        \label{tbl:opposite_halves_effect}
    }
    \hfill
    \parbox{.38\linewidth}{
        \begin{tabular}{cc ccc}
        \toprule
        \multirow{2}{*}{Epoch} & & \multicolumn{3}{c}{Batch Size}\\
        & & 256 & 512 & 1024 \\
        \cmidrule{3-5}
        \multirow{2}{*}{100} & Default & 46.1 & 41.2 & 32.6 \\
        & Cut ($c=9$) & 43.1 & 46.5 & 44.3 \\
        \cmidrule{2-5}
        \multirow{2}{*}{500} & Default & 59.1 & 60.4 & 61.3\\
        & Cut ($c=9$) & 59.4 & 58.9 & 61.5 \\
        \bottomrule
        \end{tabular}
        \captionof{table}{$k$-nn accuracies for SimSiam trained with various batch sizes. We performed training for both the default and cut-initialized variants and reported $k$-nn accuracies at 100 and 500 epochs.}
        \label{tbl:cut_batch_size}
    }
\end{table}

In accordance with this, Table~\ref{tbl:opposite_halves_effect} shows that, after one epoch of training, embeddings have angle greater than $\pi/2$ at a rate of around $5\%$ and $25\%$ for SimCLR and SimSiam/BYOL, respectively. So even if the `strongest' variant of the opposite-halves effect is not occurring, a weaker one may still be. However, very early into training (epoch 16), every method has a rate of effectively 0 for the opposite-halves effect. Furthermore, the rates in Table~\ref{tbl:opposite_half_effect} measure how often $\phi_{ij} > \frac{\pi}{2}$. This is the absolute weakest version of the opposite-halves effect. Thus, while some weak variant of the opposite-halves effect may occur at the beginning of training, it does not have a strong impact on the convergence dynamics and, in either case, disappears quite quickly.

\subsection{Weight Decay}
\label{app:weight_decay}

We evaluate the effect of weight decay in the imbalanced setting in \ref{fig:weight_decay_imbalanced}, which is an analog of Figure \ref{fig:weight_decay_ablation} for the imbalanced Cifar-10 dataset detailed in Section \ref{sec:convergence}. We again see that using weight decay controls for the embedding norms and improves the convergence of both models. In correspondence with the other results on imbalanced training, we find that stronger control over the embedding norms leads to improved convergence: the high weight decay value does not perform as poorly on SimCLR as in Figure \ref{fig:weight_decay_ablation} and, on SimSiam, outperforms the other weight decay options.

\begin{figure}
    \centering
    \begin{tikzpicture}
    \node () at (0, 0) {\includegraphics[width=0.4\linewidth]{Images/wd_sweep_imbalanced.png}};

    \draw[ballblue, line width=0.07cm] (-4, -3.8) -- (-3.4, -3.8);
    \draw[azure, line width=0.07cm] (-1.3, -3.8) -- (-0.7, -3.8);
    \draw[darkblue, line width=0.07cm] (2, -3.8) -- (2.6, -3.8);

    \node () at (-1.15, -3.33) {\small \textcolor{darkgray}{Train Epoch}};
    \node () at (1.95, -3.33) {\small \textcolor{darkgray}{Train Epoch}};

    \node[inner sep=0pt] () at (-2.5, -3.82) {\textcolor{darkgray}{\scriptsize No weight decay}};
    \node[inner sep=0pt] () at (0.48, -3.82) {\textcolor{darkgray}{\scriptsize Standard weight decay}};
    \node[inner sep=0pt] () at (3.57, -3.82) {\textcolor{darkgray}{\scriptsize High weight decay}};
        
        
    \end{tikzpicture}
    \caption{An analog to Figure \ref{fig:weight_decay_ablation} performed on the exponentially imbalanced Cifar-10 dataset. Weight decays are [$0$, $1$e-$5$, $5$e-$2$] for SimCLR and [$0$, $5$e-$4$, $5$e-$2$] for SimSiam. We plot the effective learning rate in the bottom row, calculated in accordance with Section \ref{sec:convergence}.}
    \label{fig:weight_decay_imbalanced}
\end{figure}

\subsection{Cut-Initialization}
\label{app:cut_init}
We plot the effect of the cut constant on the embedding norms and accuracies over training in Figure~\ref{fig:cut_experiments}. To make the effect more apparent, we use weight-decay $\lambda=5e-4$ in all models. We see that dividing the network's weights by $c>1$ leads to immediate convergence improvements in all models. Furthermore, this effect degrades gracefully: as $c > 1$ becomes $c < 1$, the embeddings stay large for longer and, as a result, the convergence is slower. We also see that cut-initialization has a more pronounced effect in attraction-only models -- a trend that remains consistent throughout the experiments.

We also show the relationship between cut-initialization and the network's batch size on SimSiam in Table \ref{tbl:cut_batch_size}. Consistent with the literature, we see that training with large batches provides improvements to training accuracy. However, we note that larger batch sizes also significantly slow down convergence. However, cut-initialization seems to counteract this and accelerate convergence accordingly. Thus, training with cut-initialization and large batches seems to be the most effective method for SSL training (at least in the non-contrastive setting).

\begin{figure}[t!]
    \centering
    \includegraphics[width=0.95\textwidth]{Images/init_experiments.png}
    \caption{The effect of cut-initialization on Cifar10 SSL representations. $x$-axis and embedding norm's $y$-axis are log-scale. $\lambda=5$e$-4$ in all experiments.}
    \label{fig:cut_experiments}
\end{figure}

\section{More details on gradient scaling layer}
\label{app:grad_scaling}

An implementation of our GradScale layer can be found in Listing \ref{alg:grad_scaling_use}.
We note that this layer is purely a PyTorch optimization trick and does not amount to implicitly choosing a different loss function:

\begin{restatable}{proposition}{nopotential}
    \label{prop:no_potential}
    Let $t\in\mathbb{R}^n$ be a unit vector, $p: \mathbb{R}^n\backslash \{0\} \to [-1, 1], z\mapsto t^\top z/\|z\|$ the cosine similarity with respect to $t$, $\alpha \in \mathbb{R}$, and $\sigma: \mathbb{R}^n \to  \mathbb{R}, z\mapsto \|z\|^\alpha$. Then the vector field $\sigma\nabla p$ has a potential $q$, i.e., $\nabla q = \sigma \nabla p$, only for $\alpha=0$.
\end{restatable}

\begin{proof}
    Suppose $\sigma \nabla p$ has potential. Consider two paths with segments $s_1, s_2$ and $s_3, s_4$ going $t \to 2t \to -2t$ and $t \to -t \to -2t$, where the segments $s_1, s_4$ scaling $\pm t \to \pm2t$ are straight lines and the other segments $s_2, s_3$ follow great circles on $S^{n-1}$. By Proposition~\ref{prop:cos_sim_grads}, we know that $\nabla p(z)=0$ for $z\in \mathbb{R}_{\neq 0}\cdot t$. So $\sigma \nabla p$ is zero on $s_1$ and $s_4$. Moreover, we have
    \begin{align}
        \int_{s_2} \sigma \nabla p \,dz &= \int_{s_2} \|z\|^\alpha \nabla p \,dz
        = \int_{s_2} 2^\alpha \nabla p \,dz 
        = 2^\alpha \int_{s_2} \nabla p \,dz 
        = 2^\alpha \big(p(2t) - p(-2t)\big) = 2^{\alpha+1}
    \end{align}
    and similarly 
    \begin{align}
        \int_{s_3} \sigma \nabla p dz = 1^\alpha \cdot 2 = 2.
    \end{align}
    Since we assume the existence of a potential, we can use path independence to conclude 
    \begin{align}
        2^{\alpha+1} &= \int_{s_2} \sigma \nabla p \,dz 
        = \int_{s_1, s_2} \sigma \nabla p \,dz 
        = \int_{s_3, s_4} \sigma \nabla p \,dz 
        = \int_{s_3} \sigma \nabla p \,dz 
        = 2.
    \end{align}
    Thus, $\alpha=0$ and $\sigma$ does not perform any scaling.
\end{proof}




\begin{figure}
    \begin{lstlisting}[caption={PyTorch code for gradient scaling layer}, label={alg:grad_scaling}]
class scale_grad_by_norm(torch.autograd.Function):
    @staticmethod
    def forward(ctx, z, power=0):
        ctx.save_for_backward(z)
        ctx.power = power
        return z
    @staticmethod
    def backward(ctx, grad_output):
        z = ctx.saved_tensors[0]
        power = ctx.power
        norm = torch.linalg.vector_norm(z, dim=-1, keepdim=True)
        return grad_output * norm**power, None
\end{lstlisting}
\end{figure}

\begin{algorithm}[tb]
   \caption{Pytorch-like pseudo-code using the gradient scaling layer}
   \label{alg:grad_scaling_use}
\begin{algorithmic}
   \STATE {\bfseries Input:} Encoder network $model$, gradient scaling power $p$
   \STATE $z = model(batch)$
   \STATE $z = grad\_scaling\_layer.apply(z, p)$
   \STATE $sim = (\frac{z}{\|z\|})^T \frac{z}{\|z\|}$
   \STATE $loss = InfoNCE(sim)$
   \STATE $loss.backward()$
\end{algorithmic}
\end{algorithm}


\section{Additional figures}
We provide a bar plot analogous to Figure \ref{fig:in_out_violin} in Figure \ref{fig:in_out_distribution_norms}.

\begin{figure}
    \centering
    \begin{tikzpicture}   
        \node[inner sep=0pt] (image) at (0,0) {\includegraphics[width=\textwidth]{Images/Confidence/per_class_norms.pdf}};
    \end{tikzpicture}
    \caption{Bar plot which is analogous to Figure \ref{fig:in_out_violin} showing embedding magnitudes on each dataset split as a function of which dataset the model was trained on. All values are normalized by training set's mean embedding magnitude. Normalized means are represented by black bars. We use the same data augmentations for the train and test sets for consistency.}
    \label{fig:in_out_distribution_norms}
\end{figure}

We also show each Cifar-10 class's 10 highest and 10 lowest embedding-norm samples in Figure \ref{fig:cifar_norms}. These are obtained after training default SimCLR on Cifar-10 for 512 epochs. We see that the high-norm class representatives are prototypical examples of the class while the low-norm representatives are obscure and qualitatively difficult to identify. This property was originally shown by \citet{embed_norm_confidence_2}.

\begin{figure}
    \centering
    \includegraphics[width=0.48\linewidth]{Images/high_norm.png}
    \quad
    \includegraphics[width=0.48\linewidth]{Images/low_norm.png}
    \caption{\emph{Left}: highest-norm representatives (top 10) per class. \emph{Right}: lowest-norm representatives (bottom 10) per class. All from default SimCLR trained on Cifar-10.}
    \label{fig:cifar_norms}
\end{figure}

\end{document}
\endinput
%%
%% End of file `sample-sigconf.tex'.
