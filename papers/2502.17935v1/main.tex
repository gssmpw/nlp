\documentclass[sigconf]{acmart}

\usepackage{multirow}
\usepackage{lscape}



\newcommand{\revision}[1]{\textcolor{blue}{#1}}

\AtBeginDocument{%
  \providecommand\BibTeX{{%
    Bib\TeX}}}

\copyrightyear{2025}
\acmYear{2025}
\setcopyright{cc}
\setcctype{by}
\acmConference[CHI '25]{CHI Conference on Human Factors in Computing Systems}{April 26-May 1, 2025}{Yokohama, Japan}
\acmBooktitle{CHI Conference on Human Factors in Computing Systems (CHI '25), April 26-May 1, 2025, Yokohama, Japan}\acmDOI{10.1145/3706598.3714226}
\acmISBN{979-8-4007-1394-1/25/04}
\acmDOI{10.1145/3706598.3714226}


\begin{document}

\title[Understanding Players' In-game Assessment of Communication Processes in League of Legends]{Less Talk, More Trust: Understanding Players' In-game Assessment of Communication Processes in League of Legends}

\author{Juhoon Lee}
\email{juhoonlee@kaist.ac.kr}
\affiliation{%
  \institution{KAIST}
  \city{Daejeon}
  \country{Republic of Korea}
}

\author{Seoyoung Kim}
\email{youthskim@kaist.ac.kr}
\affiliation{%
  \institution{KAIST}
  \city{Daejeon}
  \country{Republic of Korea}
}

\author{Yeon Su Park}
\email{yeonsupark@kaist.ac.kr}
\affiliation{%
  \institution{KAIST}
  \city{Daejeon}
  \country{Republic of Korea}
}

\author{Juho Kim}
\email{juhokim@kaist.ac.kr}
\affiliation{%
  \institution{KAIST}
  \city{Daejeon}
  \country{Republic of Korea}
}

\author{Jeong-woo Jang}
\email{jangjw29@kaist.ac.kr}
\affiliation{%
  \institution{KAIST}
  \city{Daejeon}
  \country{Republic of Korea}
}

\author{Joseph Seering}
\email{seering@kaist.ac.kr}
\affiliation{%
  \institution{KAIST}
  \city{Daejeon}
  \country{Republic of Korea}
}

\renewcommand{\shortauthors}{Lee et al.}

\begin{abstract}
  In-game team communication in online multiplayer games has shown the potential to foster efficient collaboration and positive social interactions. Yet players often associate communication within ad hoc teams with frustration and wariness. Though previous works have quantitatively analyzed communication patterns at scale, few have identified the motivations of how a player makes in-the-moment communication decisions. In this paper, we conducted an observation study with 22 \textit{League of Legends} players by interviewing them during Solo Ranked games on their use of four in-game communication media (chat, pings, emotes, votes). We performed thematic analysis to understand players' in-context assessment and perception of communication attempts. We demonstrate that players evaluate communication opportunities on proximate game states bound by player expectations and norms. Our findings illustrate players' tendency to view communication, regardless of its content, as a precursor to team breakdowns. We build upon these findings to motivate effective player-oriented communication design in online games.
\end{abstract}

\begin{CCSXML}
<ccs2012>
   <concept>
       <concept_id>10003120.10003130.10011762</concept_id>
       <concept_desc>Human-centered computing~Empirical studies in collaborative and social computing</concept_desc>
       <concept_significance>500</concept_significance>
       </concept>
   <concept>
       <concept_id>10003120.10003121.10011748</concept_id>
       <concept_desc>Human-centered computing~Empirical studies in HCI</concept_desc>
       <concept_significance>500</concept_significance>
       </concept>
 </ccs2012>
\end{CCSXML}

\ccsdesc[500]{Human-centered computing~Empirical studies in collaborative and social computing}
\ccsdesc[500]{Human-centered computing~Empirical studies in HCI}

\keywords{League of Legends, multiplayer online battle arena, team communication, ad hoc teams, online games}

\maketitle

\section{Introduction}\label{sec:Intro} 


Novel view synthesis offers a fundamental approach to visualizing complex scenes by generating new perspectives from existing imagery. 
This has many potential applications, including virtual reality, movie production and architectural visualization \cite{Tewari2022NeuRendSTAR}. 
An emerging alternative to the common RGB sensors are event cameras, which are  
 bio-inspired visual sensors recording events, i.e.~asynchronous per-pixel signals of changes in brightness or color intensity. 

Event streams have very high temporal resolution and are inherently sparse, as they only happen when changes in the scene are observed. 
Due to their working principle, event cameras bring several advantages, especially in challenging cases: they excel at handling high-speed motions 
and have a substantially higher dynamic range of the supported signal measurements than conventional RGB cameras. 
Moreover, they have lower power consumption and require varied storage volumes for captured data that are often smaller than those required for synchronous RGB cameras \cite{Millerdurai_3DV2024, Gallego2022}. 

The ability to handle high-speed motions is crucial in static scenes as well,  particularly with handheld moving cameras, as it helps avoid the common problem of motion blur. It is, therefore, not surprising that event-based novel view synthesis has gained attention, although color values are not directly observed.
Notably, because of the substantial difference between the formats, RGB- and event-based approaches require fundamentally different design choices. %

The first solutions to event-based novel view synthesis introduced in the literature demonstrate promising results \cite{eventnerf, enerf} and outperform non-event-based alternatives for novel view synthesis in many challenging scenarios. 
Among them, EventNeRF \cite{eventnerf} enables novel-view synthesis in the RGB space by assuming events associated with three color channels as inputs. 
Due to its NeRF-based architecture \cite{nerf}, it can handle single objects with complete observations from roughly equal distances to the camera. 
It furthermore has limitations in training and rendering speed: 
the MLP used to represent the scene requires long training time and can only handle very limited scene extents or otherwise rendering quality will deteriorate. 
Hence, the quality of synthesized novel views will degrade for larger scenes. %

We present Event-3DGS (E-3DGS), i.e.,~a new method for novel-view synthesis from event streams using 3D Gaussians~\cite{3dgs} 
demonstrating fast reconstruction and rendering as well as handling of unbounded scenes. 
The technical contributions of this paper are as follows: 
\begin{itemize}
\item With E-3DGS, we introduce the first approach for novel view synthesis from a color event camera that combines 3D Gaussians with event-based supervision. 
\item We present frustum-based initialization, adaptive event windows, isotropic 3D Gaussian regularization and 3D camera pose refinement, and demonstrate that high-quality results can be obtained. %

\item Finally, we introduce new synthetic and real event datasets for large scenes to the community to study novel view synthesis in this new problem setting. 
\end{itemize}
Our experiments demonstrate systematically superior results compared to EventNeRF \cite{eventnerf} and other baselines. 
The source code and dataset of E-3DGS are released\footnote{\url{https://4dqv.mpi-inf.mpg.de/E3DGS/}}. 






\section{Related Work}
\label{sec:rw} 

Advancements in computational capabilities have shifted the primary constraint for LLM inference from processing power to memory bandwidth and energy efficiency. 
There are currently various methods to reduce the memory requirement~\cite{ojika2020addressing} of LLMs, such as pruning, quantization, and matrix decomposition. 
For example, Squeezellm~\cite{kim2023squeezellm} proposes sensitivity-based non-uniform quantization for searching the optimal bit precision assignment, completely based on second-order information, and dense and sparse decomposition for storing outliers and sensitive weights to reduce the memory requirement.
However, in this paper only pruning and quantization have been utilized, and they are discussed further in Section~\ref{subsec:model-optimization}.

In contrast to the on-device optimization techniques such as pruning and quantization, which are essential for deploying LLMs on resource constrained edge devices, cloud-based LLMs offer a more accessible deployment pathway. Cloud providers typically offer pre-trained models with ready-to-use endpoints, enabling users to interact with LLMs via application programming interfaces (APIs). This approach simplifies integration and facilitates scalable, concurrent requests, making it relatively easy to leverage LLM capabilities without the need for hardware-specific optimizations. 
However, frequent reliance on cloud endpoints for inference can lead to increased operational costs and latency issues, especially in scenarios requiring repeated or task-specific queries. To address this, recent work~\cite{dong2024creating} has shown that using LLMs as offline compilers for creating task-specific code can effectively help avoid frequent LLM endpoints accesses and reduce costs. 


%****************************************************************************************************

\subsection{LLMs at the Edge}

The deployment of LLMs demands significantly greater computational resources as compared to traditional DNNs such as convolutional neural networks (CNNs). 
This substantial resource requirement represents a key challenge in extending the use on LLMs to edge devices, and there have been continuous efforts for running resource-efficient LLM inference~\cite{haris_SECDA-LLM_2024}. 
The survey by Friha et al.~\cite{04} addresses the issue of edge-based language intelligence by providing a thorough examination of LLM-based Edge Intelligence (EI) architectures, focusing on security, optimization, and responsible development. 
Qin et al.~\cite{qin2024empirical} stated that different compression techniques are good at different types of tasks, and other guidelines for deploying LLMs onto resource-constrained devices effectively.  
Cheng et al.\cite{cheng2023optimize} focused on weight only post-training quantization, while AWQ~\cite{lin2024awq} uses an uneven weight quantization for preserving inference accuracy. Lamini-lm~\cite{wu2023lamini} uses knowledge distillation to perform effective compression, and MiniLLM~\cite{gu2024minillm} aims to minimize reversed Kullback-Leibler divergence and improvise effective compression.
EdgeMoE~\cite{yi2023edgemoe} proposed a more efficient inference for LLMs through a mixture-of-experts-based approach, where weights that occupy less storage but require computing are kept in memory throughout the time. Fu et al.~\cite{fu2024break} suggested a more aggressive lookahead decoding, while Malladi et al.~\cite{malladi2023fine} used memory-efficient zerothorder optimizer (MeZO) to estimate model gradients by forward propagation only. 

Despite these advancements, the primary focus of existing works has been on performance-centric optimizations, with limited attention to addressing fairness and bias in edge-deployed LLMs. 


%****************************************************************************************************

\subsection{LLMs and bias}

LLMs have fundamentally redefined Information Retrieval (IR) systems through the introduction of model generated data as a new data source which led the shift from passive data collection to proactive processing. 
This shift has raised new concerns for systems about adapting data bias and unfairness. Deldjoo~\cite{deldjoo2024understandingbiaseschatgptbasedrecommender} has shown how the difference in prompt design and formation strategies can impact the precision of the model recommendation. 
Recent surveys~\cite{amigo2023unifying,deldjoo2024fairness} have shown that biases can be categorized into several dimensions. 

The challenge of bias in LLM-generated recommendations is further compounded by the unregulated nature of internet data and the repetitive patterns in model-generated data used for the retraining. These factors contribute to the reinforcement of bias, making it difficult to uphold ethical standards in the model’s output. 
Several studies~\cite{winograd2022loose, jones2022artificial} suggested that there is no method at present for removing one’s imprints from a model with absolute certainty, except for retraining of the model from scratch which is not a possibility when we are considering the immense amount of data they are trained upon and high computational demand for processing these data.
Zhang et al.~\cite{zhang2023chatgpt} has proposed the benchmark fairness of recommendation via LLMs  that accounts for eight key attributes and further evaluated ChatGPT. 

Other recent works~\cite{deldjoo2024cfairllm, ghanbarzadeh2023gender} also demonstrated that incorporating any of the explicit user-sensitive attributes, such as gender or race, into the model can result in relatively biased recommendations to queries. The issue also raises a vital concern for security and privacy, as some studies~\cite{mohsin2024can,yao2024survey,yan2024protecting} show how LLMs have only a limited understanding of security principles, which can create new vulnerabilities and lead to the misuse of user-sensitive attributes for targeted attacks. 
These challenges underscore the critical need for developing robust mechanisms to ensure fairness, transparency, and accountability in the design and deployment of LLMs. 
Jaff et all~\cite{jaff2024data} developed an LLM-based framework to analyze the privacy policy for automatically checking the consistency of data.


While LLMs are capable of learning and memorizing attributes like name, the chances of bias is even higher as these information allows a model to form the context pattern and interrelate the topic for generating output text. This ability to establish the contextual relevance, though powerful, can inadvertently amplify biases present in the underlying training data or model-generated outputs which is further observed in~\ref{sec:eval} and discussed in~\ref{sec:disc}. 
Also, previous studies~\cite{radcliffe2024automated, metzger1999sign, nijodo2024automated, bianchi2023easily} have observed that subtle discrepancies in phrasing or structure of prompts can influence the tone, inclusivity, or neutrality of the model’s responses, which again raises a critical ethical question. 

However, previous work~\cite{taubenfeld2024systematic, ye2024justice, liang2024learning, qu2024mobile} have not examined how iterative interactions can reinforce these biases over time, particularly in edge environments where model retraining may not be feasible. 
This paper aims to bridge this gap by conducting a comparative analysis across cloud-based, desktop, and edge-deployed models, revealing that edge-optimized LLMs exhibit significantly higher bias rates. Additionally, this work proposes an iterative feedback loop that effectively mitigates bias on edge devices, offering a novel approach to enhancing fairness in resource-constrained deployments. 
\section{Study Context: League of Legends}
\textit{League of Legends} (\textit{LoL}) is a popular Multiplayer Online Battle Arena (MOBA) game, whose genre is defined by two competitive teams of human players battling for a common objective. In \textit{LoL}, two symmetrical teams of five members aim to destroy the other team's base (\textit{Nexus}). Each player selects a character from a pool of \textit{champions}, each equipped with unique abilities. Notably, ~\textit{LoL} game sessions are relatively short, generally lasting from 25 to 40 minutes. 

As \textit{LoL} is a \textit{competitive} team game, the outcome of the game depends heavily on cooperation between team members to achieve victory. The game supports real-time cooperation through diverse within-team communication channels native to the platform. In this paper, we investigate \textit{LoL} players' use of four main communication modes for corresponding with allies during the game: chat (verbal), pings, emotes, and votes (non-verbal). We illustrate how each communication mode may be used and represented in the game in Figure ~\ref{fig:modes}. The player base relies on these features to exchange key information, indicate intent, and express emotions throughout the entire session.

\begin{figure*}
    \centering
    \includegraphics[width=\textwidth]{four_modes.png}
    \caption{An example of four main communication modes of \textit{LoL}. (A) Chat is the medium through which players can type in-game messages to other players or read previous logs of in-game changes or signals, (B) Pings are quick alerts used for signaling information to other teammates, (C) Emotes are used to express emotions to other players, and (D) Votes are used to determine calls for objectives or to surrender the game.}
    \label{fig:modes}
    \Description{A screenshot of the gameplay screen of League of Legends, which demonstrates how chat, pings, emotes, and votes are shown in the game. The figure shows how each communication tool appears in the game, such as the chat box, ping and emote wheels, and the vote window on the right side of the screen.}
\end{figure*}

For verbal communication, players can type in the in-game chat before, during, and after the game (Figure ~\ref{fig:modes}A). Unlike other team-based competitive games such as \textit{Overwatch}\footnote{https://overwatch.blizzard.com/} and \textit{Valorant}\footnote{https://playvalorant.com/} or other MOBA games such as \textit{DoTA 2}\footnote{https://www.dota2.com/} and \textit{Heroes of the Storm} that offer voice chat for all teams, \textit{LoL} only enables it for a pre-formed party. Despite the potential benefits of voice-based communication for impromptu teams, \textit{LoL} developers have decided against voice chat, citing that it ``\textit{[does not solve] all behavioral issues and definitely introduces some new ones... Especially for women and POC (People of Color) who get unfairly targeted by simply participating in voice comms.}''~\cite{carver2023}


In \textit{LoL}, non-verbal communication is facilitated through pings, emotes, and votes. Pings are quick alerts used to signal information by placing markers on the map or characters. Players can access pings via a ping wheel (Figure ~\ref{fig:modes}B) or keyboard shortcuts. There are two types of pings: visual and UI pings. Visual pings, triggered by clicking the terrain or minimap, appear on the map and include generic markers for drawing attention and eight specific ``Smart'' pings (e.g., Retreat, On My Way, Assist Me) with predefined meanings shown in Figure ~\ref{fig:modes}B. UI pings share information about the status of the clicked interface elements, such as items or skills. Most pings, except non-targeted generic visual pings, are logged in the chat and accompanied by a distinct audio cue.

Emotes are expressive images or animations that convey emotions during the game, often featuring characters with various expressions like excitement, remorse, or provocations. When triggered, an emote appears above the player’s character for a few seconds and briefly in a bubble on allies’ screens. Unlike pings, emotes are visible to both allies and nearby enemies. Players can purchase emotes with in-game currency and customize their emote wheel (accessible via a shortcut) with up to nine options. Chat, pings, and emotes can be muted individually or entirely for specific players or everyone.

Finally, players can also communicate through surrender and objective votes. Starting at the 15-minute mark, a player may anonymously initiate a surrender vote, which appears on the right side of the screen. If at least 70\% of the team (or four players) vote ``Yes'' within 60 seconds, the game ends. If the vote fails, the team must wait three minutes to try again. In 2022, \textit{LoL} introduced objective voting, triggered when a player pings a field objective like Baron or Drake. This vote lets teammates decide whether to ``Take'' or ``Give'' the objective. We note that the chat system in \textit{LoL} textualizes all forms of communication. Chat messages, ping notifications, and in-game announcements are all interspersed in a single channel. 

These channels are used at different frequencies and at different points of the game. Ping usage is persistent and frequent throughout the game: the two most commonly used pings (On My Way and Enemy Missing) are used an average of $0.267$ and $0.164$ pings per minute respectively across all positions and ranks. There is a positive trend of increased ping with rank (reaching $0.489$ and $0.245$ pings per minute respectively for Master players and above)~\cite{log2024ping}, showing that as players become more skilled at the game, they are able to communicate more actively through non-verbal channels. In contrast, other non-verbal gestures are used sparingly. Though official statistics are not provided for votes and emotes, votes are limited by timers, which are tied to objectives (every 5–6 minutes), or surrender cooldowns (every 3 minutes). Emotes are used infrequently based on player norms and are often reserved for reacting to significant in-game events.

While previous works have often studied how players use one specific feature~\cite{tan2022, leavitt2016}, we aim to better understand how players use all of these features in conjunction. We illustrate the decisions players make in each communication attempt and how communication used by other team members shapes their team perception in real time.
\section{Methods}
This work uses qualitative methods to explore the communication behaviors of players in \textit{League of Legends (LoL)}. By qualitatively observing and inquiring about player communication decisions as they occur, we aim to extract insights into players' reasoning, strategies, and the underlying factors influencing their choices during the actual conditions of gameplay. We observe \textit{LoL} players during real ranked games, asking them in-the-moment questions as well as follow-up interview questions after the matches have ended to capture the nuances of their communication decisions. 

\subsection{Participants}
We recruited participants for the study through university forums and social media in South Korea. Participants were required to be 18 or older and active players of Solo Ranked mode in \textit{LoL} with a valid rank during the current season at the time of the experiment (Season 2024). The recruitment post notified participants of the observational nature of the study and informed them that they would be expected to speak out loud and answer questions during their play sessions. A total of 36 players completed the recruitment survey, which asked for a self-report of their age, game history, preferred roles, and current rank.

We conducted in-person interview studies with a sample of 22 players. This sample excluded players who had played for less than a year, who were not willing to answer questions during the game, or who were unable to participate in person. From the remaining pool, players were chosen to maximize the diversity and representativeness of the player base based on their rank, experience, and roles. If several players shared similar profiles, we randomly selected between the participants. We conducted 17 interviews through this sampling method. Out of the first 17 participants, 16 participants identified as male, and one identified as female. Consequently, to increase the gender diversity of the sample and ensure that the results reflect a broad range of player experiences and perspectives, we specifically recruited female \textit{LoL} players through snowball sampling, while maintaining diversity in preferred roles and rank. We recruited and interviewed participants from the survey responders until qualitative saturation was reached, following the definition by Braun and Clark~\cite{braun2021saturation}. The final sample consisted of 16 male ($72.7\%$) and 6 female ($27.3\%$) participants. This ratio approximates the imbalanced gender demographic of \textit{LoL}, where estimates have suggested that $80$-$90\%$ of the player base is male~\cite{kordyaka2023gender}. We address the influence of gender identity on communication in Section ~\ref{discussion} and ~\ref{limitations}. The full list of participants and their information is shown in Table ~\ref{table:participant_info}.

The players' age ranged from 20 to 32 years old (mean=$23.7$ years, SD=$3.3$ years) and players' \textit{LoL} experience ranged from 2 to 13 years (mean=$7.7$ years, SD=$3.9$ years). The Solo queue ranks of the players were 2 Iron ($9.1\%$), 2 Bronze ($9.1\%$), 5 Silver ($22.7\%$), 7 Gold ($31.8\%$), 5 Platinum ($22.7\%$), and 1 Emerald ($4.5\%$) at the time of the study. Though the distribution is not as even as the Solo queue rank distribution in the Korean \textit{LoL} server ($12\%$ Iron, $19\%$ Bronze, $16\%$ Silver, $15\%$ Gold, $18\%$ Platinum, $13\%$ Emerald, and $5\%$ Diamond and above~\cite{log2024}), it encompasses the diverse range of skills of most \textit{LoL} players. Thus, the selected players reflected a comprehensive sample of engaged players with varying experience and skill levels. 


\begin{table*}
\centering
\caption{Participant Information and Game Session Information of \textit{League of Legends} Players}
    \label{table:participant_info}
\begin{tabular}{c|c|c|c|c|c|c} 
\toprule
\textbf{ID} & \textbf{Gender} & \textbf{Age} & \textbf{Experience} & \textbf{Solo Rank Tier} & \textbf{Role Played} & \textbf{Game Outcome}  \\ 
\hline
P1          & Male            & 24           & 12 years            & Silver                  & Jungle               & Win                    \\ 
\hline
P2          & Male            & 23           & 3 years             & Silver                  & Top                  & Loss                   \\ 
\hline
P3          & Male            & 32           & 12 years            & Bronze                  & Mid                  & Loss                   \\ 
\hline
P4          & Male            & 29           & 10 years            & Silver                  & Jungle               & Win                    \\ 
\hline
P5          & Male            & 27           & 11 years            & Emerald                 & Bot                  & Loss                   \\ 
\hline
P6          & Male            & 21           & 11 years            & Platinum                & Top                  & Win                    \\ 
\hline
P7          & Male            & 27           & 3 years             & Gold                    & Jungle               & Win                    \\ 
\hline
P8          & Male            & 20           & 9 years             & Gold                    & Bot                  & Win                    \\ 
\hline
P9          & Male            & 25           & 9 years             & Bronze                  & Jungle               & Loss                   \\ 
\hline
P10         & Male            & 23           & 7 years             & Silver                  & Jungle               & Loss                   \\ 
\hline
P11         & Male            & 21           & 12 years            & Gold                    & Mid                  & Loss                   \\ 
\hline
P12         & Male            & 22           & 11 years            & Platinum                & Mid                  & Loss                   \\ 
\hline
P13         & Female          & 24           & 3 years             & Gold                    & Support              & Loss                   \\ 
\hline
P14         & Male            & 25           & 10 years            & Platinum                & Bot                  & Win                    \\ 
\hline
P15         & Male            & 26           & 10 years            & Gold                    & Jungle               & Win                    \\ 
\hline
P16         & Male            & 26           & 13 years            & Gold                    & Mid                  & Loss                   \\ 
\hline
P17         & Male            & 25           & 8 years             & Platinum                & Support              & Loss                   \\
\hline
P18         & Female            & 21           & 3 years             & Platinum                & Support              & Loss                   \\
\hline
P19         & Female            & 20           & 6 years             & Gold                & Support              & Loss                   \\
\hline
P20         & Female            & 20           & 2 years             & Iron                & Top              & Loss                   \\
\hline
P21         & Female            & 20           & 3 years             & Iron                & Jungle              & Win                   \\
\hline
P22         & Female            & 21           & 2 years             & Silver                & Support              & Loss                   \\
\bottomrule
\end{tabular}
\end{table*}

\subsection{Procedure}
To capture the in-game mechanics of communication patterns and dynamically changing communication behavior in \textit{LoL}, we conducted an in-person observation and interview study. The study was conducted with the approval of the Institutional Review Board at the first author's research institution.

\subsubsection{Study Environment}

Each participant was asked to play a Solo Ranked game of \textit{LoL} while researchers observed and inquired about their actions in real time. In the process of study design, alternative study setups were considered. An initial plan to observe participants remotely through screen sharing was discarded as pilot studies revealed that latency and network issues were significantly disruptive to the researchers’ ability to observe and ask questions in a timely manner. The research team also decided not to recruit participants to observe in their own homes due to concerns about privacy and safety. Thus, to better gather responses to in-the-moment inquiries from the participants without any latency, the study was held in person within a controlled research environment to enhance the clarity and quality of the question-and-answer process. This approach also allowed researchers to note the participant's gaze, hesitation, and other subtle movements that would be missed in remote settings. Though an in-lab study does not perfectly recreate the in-home environment, the research team determined that this option best balanced the various tradeoffs.


The research team worked to ensure that the study environment best suited each participant’s preferences in the following steps. We first conducted several pilot interviews to tailor the play space to prioritize player comfort and approximate real-life conditions. The study took place in an enclosed room --- frequently used for interviews and user studies --- equipped with large desks and office chairs. We limited natural light and turned on overhead lights for visibility. Moreover, players were individually asked if the environment felt natural and comfortable, and adjustments were made based on their feedback. We set up the study environment with equipment (mouse, keyboard, monitor) designed specifically for online gaming. The players were also permitted to bring personal equipment if they wished. Before entering the game, the players were instructed to adjust both the equipment (e.g., mouse sensitivity) and in-game settings (e.g., shortcut keys). All players were given as much time as needed until they expressed satisfaction with the setup. At the end of the study, we asked players if anything about the setup or procedure had negatively impacted their gameplay. Three players reported feeling some discomfort from using unfamiliar equipment but noted that it did not affect their typical playstyle or their answers. The participants were compensated 20,000 KRW (approximately 15 USD) for completing the study. 


\subsubsection{Interview Process}

The researchers observed the game session through a separate screen connected to the player's monitor and noted any communication actions, attempts, and responses by the player. During the study orientation, researchers emphasized that participants should play and communicate naturally, including using offensive language, muting or reporting other players, or forfeiting the game if desired. Participants were assured that all data would be anonymized for analysis. To minimize distractions, they were informed that they could skip questions if they found them intrusive or preferred not to respond. During intense in-game situations in which the participant could not answer, the researchers documented the context and either repeated the question once the game state had stabilized or immediately after the match ended. We discuss the limitations of using an observational approach to study in-game communication, such as social desirability bias~\cite{grimm2010social}, in Section ~\ref{limitations}.

The researcher observed the communication between teammates, noting what triggered the communication or what communication medium was used for different purposes. Based on these observations, the participant was asked questions about why they did or did not perform certain communication actions to understand their assessment and perception of the communication. Some of the questions were asked to all participants, such as the reasoning behind the frequency of using certain communication media and their perception towards teammates who engage in certain forms of communication. Other non-structured questions were asked when the player triggered a certain action (``\textit{You just pinged your ally with Enemy Missing ping multiple times. What was the purpose?''}) or responded to (or ignored) their team's communication (``\textit{It seems that you opted to not vote for the surrender vote. Why is this so?}'').

The researchers recorded the gameplay and thoroughly transcribed observations and game states during the game. The observations included types of communication media used (chat, ping, votes, emotes), the target of the communication (if unclear, then players were asked to clarify the target), player reactions to ongoing discourse or communication usage by their team, and physical reactions such as spoken utterances and body gestures. After completing the game, participants engaged in a 15 to 20-minute post-game interview. They were asked to reflect on their in-game communication behavior and perceptions, including the motivations behind their communication tendencies and choices. The interview also inquired how their teammates' communication behaviors influenced their perception of those teammates, as well as the overall impact of such interactions on their mental state or performance. Participants were further invited to share suggestions for improving communication in \textit{LoL}, such as the potential addition of voice chat. The full list of the interview questions is provided in Appendix ~\ref{appendix2}.

Overall, a total of 24 games were played, of which two were forfeited within 20 minutes. Players were asked to play another game if their first game ended within 20 minutes due to a surrender vote from either team as these games did not demonstrate communication across all stages of the game. The other 22 games lasted a minimum of 24 minutes, most of which were played to completion without forfeit from either team or with a forfeit when the victor was very clearly determined.


\subsection{Thematic Analysis}
We conducted an inductive thematic analysis applying the methodology from Elo and  Kyngäs~\cite{elo2008qualitative}. The researchers gathered the transcript of the in-game and post-interview, video recording and the replay file of the game, and observational notes for each participant. We incorporated the notes on participant behavior, game states, and other players' communication patterns into the transcript at its corresponding times, providing contextual information on what was happening at the time of the question or the player's reaction.

Before initiating coding, the first, second, and third authors familiarized themselves with the data collected. The three authors then independently performed line-by-line open coding on eight participants' data to identify preliminary themes. After the initial coding was completed, the authors shared the codes to combine convergent ideas and discuss any differing perspectives. The first author then validated the codes on the remaining data, engaging in discussions with the second and third authors to iterate on the codebook. The final codebook contained 55 codes with 13 categories, organized by themes that answer each research question in depth. For RQ1, we find themes of communication types and what triggers or deters a player's decision to communicate. For RQ2, we categorize factors used by players to assess communication opportunities as well as reactions based on such assessment. For RQ3, we find themes on how the team's communication behaviors affect a player's perception towards other teammates during the game. We provide the final codebook in Appendix ~\ref{appendix}. 
\section{Results}
We first describe communication patterns within the full chronological context of the game in \textit{League of Legends (LoL)}, separated into four sections based on changing coordination dynamics. Based on this context, we identify core factors players assess to decide when to participate in communication with other teammates. Afterward, we discuss how communication shapes player perceptions toward their teammates, showing player's wariness towards players actively engaging in communication. 

\subsection{Communication Patterns in Context}

We discuss the communication patterns among teammates within the game. We organize the data into chronological phases of the game for a structured analysis of how the context shapes communication patterns. 

\subsubsection{Pre-game stage}
Before gameplay begins, team communication opens with \textit{team drafting}, where players are assigned roles (Top, Mid, Bot, Support, or Jungle) and take turns picking or banning champions. In Solo Ranked mode, roles are pre-assigned based on player preferences selected before queueing. Once teams are set, all players enter \textit{champion select} stage, alternating champion picks and banning up to five champions per team. During this stage, communication is limited to text chat. The usernames are anonymized (i.e., replacing the name with aliases) to prevent queue dodging by checking third-party stats sites such as OP.GG\footnote{https://www.op.gg/}, leaving the chat as the only option to inform individual strengths and preferences. 

Team composition in \textit{LoL} is crucial to the strategy and outcome of the game~\cite{ong2015player}, setting the basis for future interactions. Most participants acknowledged the importance of balanced and synergistic team composition, especially as players move into higher ranks where team coordination outweighs individual excellence. Yet, we observed a distinct lack of verbal communication between the members during this period across all ranks. Participants attributed the lack of willingness to initiate a conversation on the dangers of starting the game on a bad footing. They prioritized ``not creating friction'' during this stage as negative impressions can propagate throughout the game. Some participants attempted communication to reduce such friction, such as P14, who stated,``\textit{If I had the time, I wanted to say that I will be banning [this Champion], just in case a player on my team wanted to play them.}'' However, several participants viewed any communication during the pre-game phase with wariness, as dissatisfaction or conflict at this step portended negative interactions between players in the game (P3, P9, P15). Thus, even when participants expressed doubt about other teammates' unconventional or non-meta champion picks, they refrained from entering into discourse. This contrasts with findings by Kou and Gui~\cite{kou2014}, which showed players attempt to maintain a harmonious and constructive atmosphere through greetings and introductions.

Another emergent code of the reason for not engaging in communication in the pre-game stage stems from different purposes of playing the game (P1, P5, P13, P16, P17). Despite being in ranked mode, which is more prone to increased competitiveness and effort, participants showed differing goals and levels of interest in winning the game. Several players stated that they had previously exerted great mental load in coordinating synergistic plays, but stopped as they gave less importance to winning at all costs (``\textit{I don't really play to win. I play \textit{LoL} to relieve stress, so I don't engage in chat.}'', P5). These players saw verbal communication with the goal of coordination as an unnecessary or even cumbersome component of the pre-game stage.


\subsubsection{Structured phase}
In many MOBAs, including \textit{LoL}, the early stages of the game play out in a formulaic manner: players join their lanes (Top, Mid, and Bot/Support), defeat minions to gain gold, buy items towards certain ``builds'', kill or assist in early objectives (Jungle), and battle counterparts in their respective lanes. Participants at this stage expressed that most players possessed tacit knowledge of what must be done, such as knowing when to aid their Jungle to capture a jungle monster, choosing the opportune moments to leave their lanes, or positioning wards (i.e., a deployable unit which provides a vision of the surrounding area) at the ideal placements. The participants assumed each player knew their ``role'' to fulfill, often comparing it to ``doing their share'' (P1, P3, P7, P19). In line with this belief, players rarely initiated preemptive or proactive verbal communication for strategic or social purposes at the early stage. 

Pings, on the other hand, constantly permeated the game. At this stage, players used ping to provide information relevant to others from their position, such as letting others know if an enemy went missing from their lane. As the players are largely separated and independent from one another, pings (coupled with the minimap and scoreboard) served as the primary channel for maintaining context over the game rather than as warnings or direct guidance to the players. For other non-verbal gestures, while objective votes would occasionally appear, they were rarely answered. Instead, relevant players near the objective would place pings or move toward it to help out their teammates.

Participants viewed the structured phase as a routine, but uncertain period of the game where the pendulum could swing in either team's favor. Players --- especially Jungles who roam the board looking for opportunities to ambush the enemy team in lanes (``gank'') --- sometimes felt hesitant to make calls and demands at this stage since ``\textit{[they] could make a call, but if I fail, they'll start blaming my decisions down the line.}'' (P7) But at this stage, participants believed that they held personal agency over the final game outcome. P1 and P6 stated that they entered the game with the mindset that only they had to succeed regardless of the performance of their teammates. This belief was reflected in their chatting behavior, where players prioritized focusing on their circumstances over the team's (``\textit{I mute the chat so that I don't get swayed by the team, as I can win the game if I do well.}'', P9).


\subsubsection{Group engagement phase}
As the game enters its middle phase, it provides opportunities for more diverse decision-making. Players may swap lanes, seize or trade crucial objectives, and fight in large battles involving multiple champions. At this point, teams typically have a clear outlook on which players and team have the advantage, requiring more team-driven decisions to maintain or overcome their current standing. Thus, players used verbal communication to discuss more complicated tactics that could not be effectively conveyed through pings.

But more often than not, chat messages became judgment-based. As enemy engagement with larger groups occurred more frequently, the availability for chatting would come after death, which led to comments on past actions rather than future choices. Additionally, the respawn timer for deaths becomes longer as the game progresses, providing more time to observe other players than in earlier phases. This gave players more opportunities to express dissatisfaction specifically towards certain plays, such as placing Enemy Missing pings on the map where other teammates are located to bring attention to their questionable play.

This stage also gave much more exposure of each other to the allies as the team would gather at a single point, giving way to greater scrutiny by their teammates. Repeated or critical mistakes put participants on edge, as they braced for criticism from their teammates. They expressed relief or surprise when the chat remained silent or civil, with P8 stating ``\textit{I messed up there. No one is saying anything, thankfully.}''


\subsubsection{Point of no return}
Meanwhile, verbal communication flowed out when the game had a clear trajectory to the end. Previous research has shown that both toxic and non-toxic communication skyrockets near the end of the game~\cite{kwak2015linguistic} when the players have determined the game outcome with certainty. We saw that this phase opened up both positive and negative sides of communication for guaranteed win and loss, respectively. The winning team would compliment and cheer each other through chat messages and emotes, while the losing side devolved into arguments and calling out. The communication at this stage was driven by emotion, showing excitement or venting frustration.


\subsection{Communication Assessment Process}

We describe the factors that users mainly focus on to assess when or when not to involve themselves in communication with their teammates. 

\subsubsection{Calculating communication cost}
One of the most proximate factors behind when communication is performed is the limited action economy of the game. In \textit{LoL} and other MOBAs, players can rarely afford time to type out messages due to the fast-paced nature of the game. In time-sensitive scenarios, the time pressure makes communication particularly costly. It is therefore unsurprising that much of the communication occurs after major events (e.g., battles and objective hunting), as players are given more downtime while waiting for teammates or enemies to respawn or regroup.

For periods where players were still actively involved in gameplay, the players made conscious decisions on choosing which communication media to use based on the perceived action availability and the importance of communicating the message. Players relied on pings for non-critical indications, believing that the mutual understanding of the game would get the message across. However, many players recognized that pings were prone to be missed, ignored, or misinterpreted by their allies (P2, P9, P16, P17, P20). Subsequently, participants typed out information considered to be too important to the situation to be misunderstood or missed by other players even if it caused delays in their gameplay (P10, P11, P14). Simultaneously, the priority of importance constantly shifted --- we observed multiple times participants start to type, but stop to react to an ongoing play, only to never send out their message.

\subsubsection{Evaluating relevance and responsiveness}
When the brief window of communication opportunity is missed, players are unlikely to ever send out that information. In \textit{LoL}, situations can change within seconds and certain communication media cannot keep up with the changing state of the game. For example, almost all study participants did not participate in votes for objectives. Among the tens of objective votes initiated among all the games in this study, no objective vote saw more than three votes, frequently being left with no vote beyond the player who initiated the vote. Some players, when asked why they did not participate, stated that the votes they made often became irrelevant as the game state had changed during the time it took to vote (P2, P11). Other players also discussed how information conveyed through communication can get outdated fast (P1, P8, P9). 

\begin{quote}
I can't always follow through with what I say [in the chat] since the game is really dynamic. My teammates don't understand such situations, so I tend to not chat proactively. - P9
\end{quote}

Thus, some players instead preferred to react through direct action (P8, P10, P11, P16, P20). P10 stated, ``\textit{I think it's enough to show through action rather than [using objective voting]. I can look out for how the player reacts when I request something from them.}''

On the other hand, such action-based responses left the player to assess whether and how the communication was received. P10 stated that they tried to predict whether a player understood their ping direction by how they moved, but it was hard to interpret their intent: ``\textit{members sometimes seem to move towards me but then turn around, and sometimes they even ping back but don't come.}''. P16 discussed how they weren't sure whether the ping was received, but performed it anyway since it felt helpful.

Similarly, participation in surrender votes (or lack thereof) carried different intent by the player. During most of the games that ended in a loss, one or more surrender votes were called by the participant's team. However, only two surrender votes achieved four or more players' participation. However, the reasons why a player chose to not participate varied. Some had decided to wait and see how other teammates voted, which may have paradoxically led many members to not participate in the vote (P4, P9). Meanwhile, others didn't reply as they didn't think the vote was actually calling for a response: P13 stated, ``\textit{I didn't vote because they were just showing their anger. It's just a member venting through a surrender vote that they're not doing well.}''

\subsubsection{Balancing information access and psychological safety}
While recognizing that communication would be useful or even necessary in certain situations, participants also put their psychological safety first over information access. Some players, worn down by the normalization of toxic communication such as flaming, muted the chat (P1, P9).

Many participants expressed the sentiment of ``protecting [their] mentality'', describing how certain communication harmed their psychological well-being. This communication did not always refer to negative communication; P9 often muted players who gave commands as they did not want to be ``swept up'' by others' play-related judgments. This separation even extended to other more widely considered essential communication forms, such as pings. Even after acknowledging that pings were vital and useful to the game, P9 went as far as muting the ping of the support player in the same lane after they sent a barrage of Enemy Missing pings that signified aggression and criticism. 

Additionally, the abundance and high frequency of communication also strained the limited mental capacity of the players. Many players, when asked why they had not replied to an objective vote or other chat messages, stated that they simply did not notice them among other events happening (P1, P2, P3, P9, P12, P15, P18, P20, P21). The information overload caused stress and became distracting to players.

\subsubsection{Reducing potential friction}
As demonstrated in the pre-game stage of the game, players sometimes used communication to minimize friction between their teammates. Some participants sacrificed time to apologize to other players when they believed themselves to be at fault. When asked why, P12 replied, ``\textit{There are too many people who don't come to help gank if I don't apologize.}''. Similarly, P5 sacrificed time typing in an apology after a teammate had died despite still being in the middle of a fight as they didn't wish to give the other player a reason to start an attack.

However, some noted that silence is sometimes the best answer to a negative situation. P4, after dying to the enemy, put into chat ``Fighting!'' (roughly meaning, ``We can do it!''). They stated ``\textit{I don't know why I do it... it probably angers [my teammates] more.
}'' They also stated that ``\textit{for certain people, talking in the chat only spurs them more. You just have to let them be.}'' Other players shared similar sentiments that being quiet and dedicating focus to the game was a better choice (P1, P11, P14).

For female players, the fear of gender-based harassment shaped their communication patterns. While \textit{LoL} does not provide any demographic information of a player to other players, almost all female participants noted experiences of receiving derogatory remarks or doubts about their abilities based on other players' assumptions of their gender, a trend frequently seen in male-dominated online gaming cultures~\cite{fox2016women, norris2004gender, mclean2019female}. They noted that the players were able to correctly guess their gender when the participant's role and champion fit into the preconceived notions of what women ``tended to play'' (i.e., female-identifying support champions, such as Lulu) or their username ``seemed feminine'' (P18, P19, P20, P21, P22). This led to certain players adopting tactics that signaled male-like behavior, such as changing their speech style to be more gender-neutral or male-like (P19, P21) and changing their username to sound more gender-neutral. Cote describes similar instances of ``camouflaging gender'' as one of five main strategies for women coping with harassment~\cite{cote2017coping}. However, some players opted to keep playing their preferred character or maintaining their username even if it signaled their gender, such as P21 who expressed, ``\textit{I cherish and feel attached to my username, so I don’t want to change it just because of [harassment and inappropriate comments].}'' These players valued self-expression and identity even at the risk of increased risk to unpleasant communication experiences.


\subsubsection{Forming performance-based hierachy}
Naturally formed leadership has often been observed in other works on \textit{LoL} teams~\cite{kou2014}. Kim et al. showed that more hierarchy in in-game decision-making led to higher collective intelligence~\cite{kim2017}. While they used ``hierarchy'' to mean varying amounts of communication throughout the game, we observed that the hierarchy extends further to performance-based hierarchy, where teammates in more advantageous positions are given greater weight when communicating with other players. Players actively chose to refrain from suggesting strategic plans when they were ``holding down the team'', recognizing that they held less power and trust among the team members (P8, P10, P12, P14, P22). The player who was losing against the enemy team was viewed as having no ``right'' to lead the team, which was reserved for well-performing players.


\subsubsection{Enforcing norms and habits}
One of the most common answers to why players performed certain communication actions, especially non-verbal actions such as pings and emotes, was ``a force of habit'' (P6, P7, P8, P9, P10, P12, P17). Players formed learned practices of using communication channels at certain points by observing other players exhibit the same behaviors. This promoted, for example, replying to an emote sent by the teammate with their own or pinging readied skills and items to emphasize relevant information for other players throughout the game. 

On the other hand, this meant that players were averse to communication patterns outside of the norm --- participants stated that they had a hard time adapting to new forms of communication, seeing no immediate benefit or impact from using them (P1, P8, P14, P13, P15, P17). Most egregiously, the recently introduced objective pings were largely viewed to be awkward to use and unnecessary (P1, P4, P8, P12).


\subsection{Impact of Communication Assessment}
We describe how the communication patterns and assessment of the players impact the individual players' perspectives on team dynamics.

\subsubsection{Relationship between trust and communication frequency}
Most participants saw value in constant and well-informed communication but with an important distinction: verbal communication with strangers rarely ended well. Players largely recognized frequent verbal communication to burgeon conflict, regardless of the message within. Even when players understood the helpful intent behind positive messages from the players, they compared actively talking players to be possible bad actors who were likely to exhibit toxic behaviors when the game turned against them. (P1, P4, P8, P12, P14)

\begin{quote}
I need to make sure to not disturb Twisted Fate. I saw him start to flame. It's not because I don't want to hear more criticism. I know these types. The more I react and chat with them, the more deviant they will become. - P4  
\end{quote}

Similarly, P19 lamented that players used to socialize more in the chat during the pre-game phase to build a fun and prosocial environment, noting a memorable example of encouraging each other to do well on their academic exams, but noted that such prosocial behavior has become much rarer during the recent seasons. They noted that there are inevitably players ``who take it negatively'' and thus stopped proactively typing non-game related messages in the chat.

Ultimately, players desired assurance and trust of player commitment. The participants trusted actions more than words to prove that the player remained dedicated to the game. Both P10 and P17 pointed out that it was easy to tell who was still ``in the game'' and motivated to try their best and that ``staying on the keyboard'' likely meant that they weren't invested or focused on the game. Players viewed such commitment to be the most important aspect of a ``good'' teammate, sometimes even more than their skill or performance (P9, P14). It is interesting to note that unlike what previous literature may suggest~\cite{marlow2018}, players' averseness to talkative teammates had less to do with the cognitive overload or distraction caused by the frequent communication, but rather due to the threats of future team breakdown. This view in turn also affected how players decided to communicate or not, as they believed that players would not take their suggestions or comments in a positive light. 


\subsubsection{Perception of player commitment and fortitude}

Communication also acted as a mirror of their teammates' mental fortitude. A number of players mentioned how they valued a resilient mindset in their teammates playing the game, referring to players who remained committed to the game until the very end. They saw players who provoked or complained to teammates as ``having a weak mentality'' who had been altered by the bad outcomes of the game to act in an unhelpful manner towards the team through their communication. The communication actions of the teammate informed the participants of how steadfast their teammate remained in disadvantageous situations.  

\begin{quote}
It's not like I constantly reply in the chat or anything, but I pay attention [to the chat] to grasp the overall atmosphere of the team. If the team doesn't collaborate well then we lose, so I try to have a rough understanding of the mentality of the other players. - P13
\end{quote}

There were also instances of communication that helped players maintain a positive view of their teammates. For example, P11 mentioned near the beginning of the game, ``\textit{Looking at the chat, Varus player has strong mentality [for being so positive]. There were lots of points [in his support's] plays that he could have criticized.}'' Unfortunately, this view quickly soured when the Varus player devolved into criticism later in the late game phase where the Varus player started criticizing the support and other players. P11 then noted that the Varus player seemed to merely be ``bearing through the game''.
\section{Discussion and Conclusion}
\label{sec:discussion}


\textbf{Conclusion.} In this paper, we propose LRM to utilize diffusion models for step-level reward modeling, based on the insights that diffusion models possess text-image alignment abilities and can perceive noisy latent images across different timesteps. To facilitate the training of LRM, the MPCF strategy is introduced to address the inconsistent preference issue in LRM's training data. We further propose LPO, a method that employs LRM for step-level preference optimization, operating entirely within the latent space. LPO not only significantly reduces training time but also delivers remarkable performance improvements across various evaluation dimensions, highlighting the effectiveness of employing the diffusion model itself to guide its preference optimization. We hope our findings can open new avenues for research in preference optimization for diffusion models and contribute to advancing the field of visual generation.

\textbf{Limitations and Future Work.} (1) The experiments in this work are conducted on UNet-based models and the DDPM scheduling method. Further research is needed to adapt these findings to larger DiT-based models \cite{sd3} and flow matching methods \cite{flow_match}. (2) The Pick-a-Pic dataset mainly contains images generated by SD1.5 and SDXL, which generally exhibit low image quality. Introducing higher-quality images is expected to enhance the generalization of the LRM. (3) As a step-level reward model, the LRM can be easily applied to reward fine-tuning methods \cite{alignprop, draft}, avoiding lengthy inference chain backpropagation and significantly accelerating the training speed. (4) The LRM can also extend the best-of-N approach to a step-level version, enabling exploration and selection at each step of image generation, thereby achieving inference-time optimization similar to GPT-o1 \cite{gpt_o1}.
\section{Limitations} 
We acknowledge several limitations in our work. First, DataMan's reliance on LLMs for text quality assessment and domain categorization may inherit biases from these models. Second, DataMan's inference accuracy is not yet optimal, sometimes causing misclassification, highlighting the need for a large-scale collection of documents with diverse quality differences for fine-tuning. Third, using SlimPajama alone as a pre-training corpus limits result reliability, incorporating additional corpora would be better. Fourth, the model size is restricted by data and training resources, resulting in models with only 1.3B parameters, whereas increasing parameters might reveal interesting phenomena. Lastly, the considerable costs of developing data managers, data filtering, and pre-training experiments could hinder further research in this domain. 
We aim to address these in future work. Despite these limitations, DataMan remains a powerful tool for data selection and mixing.

\section{Conclusion}
We introduced \Bench, the first ever IMTS forecasting benchmark.
\Bench's datasets are created with ODE models, that were defined in decades of research and published on
the Physiome Model Repository. Our experiments showed that LinODEnet and CRU are actually
better than previous evaluation on established datasets indicated. Nevertheless,
we also provided a few datasets, on which models are unable to outperform a
constant baseline model. We believe that our datasets, especially the very difficult ones,
can help to identify deficits of current architectures and support future research on
IMTS forecasting.


\begin{acks}
This work was supported by the Office of Naval Research (ONR: N00014-24-1-2290). We thank the \textit{League of Legends} players and community members for their insights and participation in our research.
\end{acks}

\bibliographystyle{ACM-Reference-Format}
\bibliography{main}


\appendix
\renewcommand{\thefigure}{S\arabic{figure}}
\setcounter{figure}{0}  
\renewcommand{\thetable}{S\arabic{table}}
\setcounter{table}{0} 


\section{Proofs}
\subsection{Proof of Proposition~\ref{prop:cos_sim_grads}}
\label{prf:prop_grad_grows}
\cosgrads*
\begin{proof}
    We are taking the gradient of $\mathcal{L}^\mathcal{A}_i$ as a function of $z_i$. The principal idea is that the gradient has a term with direction $\hat{z}_j$ and a term with direction $-\hat{z}_i$. We then disassemble the vector with direction $\hat{z}_j$ into its component parallel to $z_i$ and its component orthogonal to $z_i$. In doing so, we find that the two terms with direction $z_i$ cancel, leaving only the one with direction orthogonal to $z_i$.
    
    Writing it out fully, we have $\mathcal{L}^\mathcal{A}_i = -z_i^\top z_j / (\|z_i\| \cdot \|z_j\|)$. Taking the gradient amounts to using the quotient rule, with $f = -z_i^\top z_j$ and $g = \|z_i\| \cdot \|z_j\| = \sqrt{z_i^\top z_i} \cdot \sqrt{z_j^\top z_j}$. Taking the derivative of each, we have
    \begin{align*}
        f' &= -\mathbf{z}_j \\
        g' &= \|z_j\| \frac{z_i}{\sqrt{z_i^\top z_i}} = \|z_j\| \frac{\mathbf{z}_i}{\|z_i\|} \\
        \implies \frac{f' g - g' f}{g^2} &= \frac{- \left(\mathbf{z}_j \cdot \|z_i\| \cdot \|z_j\| \right) + \left(\|z_j\| \frac{\mathbf{z}_i}{\|z_i\|} \cdot z_i^\top z_j \right)}{\|z_i\|^2 \cdot \|z_j\|^2} \\
        &= \frac{-\mathbf{z}_j}{\|z_i\| \cdot \|z_j\|} + \frac{\mathbf{z}_i z_i^\top z_j}{\|z_i\|^3 \|z_j\|},
    \end{align*}
    where we use boldface $\mathbf{z}$ to emphasize which direction each term acts along. We now substitute $\cos(\phi_{ij}) = z_i^\top z_j / (\|z_i\| \cdot \|z_j\|)$ in the second term to get
    \begin{equation}
        \label{eq:quotient_rule}
        \frac{f' g - g' f}{g^2} = \frac{-\hat{z}_j}{\|z_i\|} + \frac{\mathbf{z}_i \cos(\phi)}{\|z_i\|^2}
    \end{equation}

    It remains to separate the first term into its sine and cosine components and perform the resulting cancellations. To do this, we take the projection of $\hat{z}_j = \mathbf{z}_j / \|z_j\|$ onto $\mathbf{z}_i$ and onto the plane orthogonal to $\mathbf{z}_i$. The projection of $\hat{z}_j$ onto $\mathbf{z}_i$ is given by
    \[ \cos \phi_{ij} \frac{\mathbf{z}_i}{\|z_i\|} \]
    while the projection of $\mathbf{z}_j / \|z_j\|$ onto the plane orthogonal to $\mathbf{z}_i$ is
    \[ \left( \mathbf{I} - \frac{z_i z_i^\top}{\|z_i\|^2} \right) \frac{\mathbf{z}_j}{\|z_j\|}. \]
    It is easy to assert that these components sum to $\mathbf{z}_j/\|z_j\|$ by replacing the $\cos \phi_{ij}$ by $\frac{z_i^\top z_j}{\|z_i\|\cdot \|z_j\|}$.

    We plug these into Eq.~\ref{eq:quotient_rule} and cancel the first and third term to arrive at the desired value:
    \begin{align*}
        \frac{f' g - g' f}{g^2} = &-\frac{1}{\|z_i\|} \cos \phi \frac{\mathbf{z}_i}{\|z_i\|} \\
        &- \frac{1}{\|z_i\|} \cdot \left( \mathbf{I} - \frac{z_i z_i^\top}{\|z_i\|^2} \right) \frac{\mathbf{z}_j}{\|z_j\|} \\
        &+ \frac{\mathbf{z}_i \cos(\phi)}{\|z_i\|^2} \\
        = &\frac{-1}{\|z_i\|} \cdot \left( \mathbf{I} - \frac{z_i z_i^\top}{\|z_i\|^2} \right) \frac{\mathbf{z}_j}{\|z_j\|}.
    \end{align*}
\end{proof}

We visualize the loss landscape of the cosine similarity function in Figure \ref{fig:cos_sim_surface}. 

\begin{figure}
    \centering
    \begin{subfigure}{0.45\linewidth}
        \centering 
        \includegraphics[width=1\linewidth]{Images/cosine_similarity_surface_with_circles.pdf}
    \end{subfigure}%
    \begin{subfigure}{0.45\linewidth}
        \centering 
        \includegraphics[width=0.8\linewidth]{Images/cosine_similarity_2D_heatmap.pdf}
    \end{subfigure}
    \caption{Cosine similarity with respect to the direction indicated by the blue line. Three circles of radii 0.5, 1, and 2 are superimposed to show that, for higher norms, the cosine similarity is less steep. Left: 3D Surface plot, right: 2D topview plot.}
    \label{fig:cos_sim_surface}
\end{figure}


\subsection{InfoNCE Gradients}
\label{app:infonce_grads}
\infoncegrads*
\begin{proof}
    We are interested in the gradient of $\mathcal{L}_i^\mathcal{R}$ with respect to $z_i$. By the chain rule, we get
    \begin{align*}
        \nabla_i^\mathcal{R} &= -\frac{\sum_{k \not\sim i} \text{ExpSim}(z_i, z_k) \frac{\partial \frac{z_i^\top z_k}{\|z_i\| \cdot \|z_k\|}}{\partial z_i}}{\sum_{k \not\sim i} \text{ExpSim}(z_i, z_k)} \\
        &= -\frac{\sum_{k \not\sim i} \text{ExpSim}(z_i, z_k) \frac{\partial \frac{z_i^\top z_k}{\|z_i\| \cdot \|z_k\|}}{\partial z_i}}{S_i}
    \end{align*}
    It remains to substitute the result of Prop. \ref{prop:cos_sim_grads} for $\partial \frac{z_i^\top z_k}{\|z_i\| \cdot \|z_k\|} / \partial z_i$.

    We sum this this with the gradients of the attractive term to obtain the full InfoNCE gradient, completing the proof.
\end{proof}

We note that the repulsive force is weighted average over a set of unit vectors. Consequently, the repulsive gradient is smaller than the attractive one. Additionally, we point out that these gradients are symmetric: just like positive and negative samples $z_j$ and $z_k$ affect $z_i$, $z_i$ affects $z_j$ and $z_k$.

\subsection{Proof of Corollary~\ref{cor:embeddings_grow}}
\label{prf:cor_embeddings_grow}
\begin{proof}
    First, consider that we applied the cosine similarity's gradients from Proposition~\ref{prop:cos_sim_grads}. Since $z_i$ and $(z_j)_{\perp z_i}$ are orthogonal, $\|z_i'\|_2^2 = \|z_i\|^2 + \frac{\gamma^2}{\|z_i\|^2}\|(z_j)_{\perp z_i}\|^2$. The second term is positive if $\sin \phi_{ij} > 0$.

    The same exact argument holds for the InfoNCE gradients. The gradient is orthogonal to the embedding, so a step of gradient descent can only increase the embedding's magnitude.
\end{proof}

\subsection{Proof of Theorem~\ref{thm:convergence_rate}}
\label{prf:thm_convergence_rate}
We first restate the theorem:

Let $z_i$ and $z_j$ be positive embeddings with equal norm, i.e. $\|z_i\| = \|z_j\| = \rho$. Let $z_i'$ and $z_j'$ be the embeddings after 1 step of gradient descent with learning rate $\gamma$. Then the change in cosine similarity is bounded from above by:
\begin{equation*}
    \hat{z}_i'^\top \hat{z}_j' - \hat{z}_i^\top \hat{z}_j < \frac{\gamma \sin^2 \phi_{ij}}{\rho^2} \left[ 2 - \frac{\gamma \cos \phi}{\rho^2} \right].
\end{equation*}

\noindent We now proceed to the proof:
\begin{proof}
    Let $z_i$ and $z_j$ be two embeddings with equal norm\footnote{We assume the Euclidean distance for all calculations.}, i.e. $\|z_i\| = \|z_j\| = \rho$. We then perform a step of gradient descent to maximize $\hat{z}_i^\top \hat{z}_j$. That is, using the gradients in \ref{prop:cos_sim_grads} and learning rate $\gamma$, we obtain new embeddings $z_i' = z_i + \frac{\gamma}{\|z_i\|} (\hat{z}_j)_{\perp z_i}$ and $z_j' = z_j + \frac{\gamma}{\|z_j\|} (\hat{z}_i)_{\perp z_j}$. Going forward, we write $\delta_{ij} = (\hat{z}_j)_{\perp z_i}$ and $\delta_{ji} = (\hat{z}_i)_{\perp z_j}$, so $z_i' = z_i + \frac{\gamma}{\rho} \delta_{ij}$ and $z_j' = z_j + \frac{\gamma}{\rho} \delta_{ji}$. Notice that since $z_i$ and $\delta_{ij}$ are orthogonal, by the Pythagorean theorem we have $\|z_i'\|^2 = \|z_i\|^2 + \frac{\gamma^2}{\rho^2}\|\delta_{ij}\|^2 \geq \|z_i\|^2$. Lastly, we define $\rho' = \|z_i'\| = \|z_j'\|$.

    We are interested in analyzing $\hat{z}_i'^\top \hat{z}_j' - \hat{z}_i^\top \hat{z}_j$. To this end, we begin by re-framing $\hat{z}_i'^\top \hat{z}_j'$:
    \begin{align*}
        \hat{z}_i'^\top \hat{z}_j' &= \left(\frac{z_i + \frac{\gamma}{\rho} \delta_{ij}}{\rho'}\right)^\top \left(\frac{z_j + \frac{\gamma}{\rho} \delta_{ji}}{\rho'}\right) \\
        &= \frac{1}{\rho'^2}\left[ z_i^\top z_j + \gamma \frac{z_i^\top \delta_{ji}}{\rho'} + \gamma \frac{z_j^\top \delta_{ij}}{\rho'} + \gamma^2 \frac{\delta_{ij}^\top \delta_{ji}}{\rho'^2} \right].
    \end{align*}

    We now consider that, since $\delta_{ij}$ is the projection of $\hat{z}_j$ onto the subspace orthogonal to $z_i$, we have that the angle between $z_i$ and $\delta_{ji}$ is $\pi/2 - \phi_{ij}$. Plugging this in and simplifying, we obtain
    \begin{align*}
        z_i^\top \delta_{ji} &= \|z_i\| \cdot \|\delta_{ji}\| \cos (\pi/2 - \phi_{ij}) \\
        &= \|z_i\| \cdot \|\delta_{ji}\| \sin \phi_{ij} \\
        &= \rho \sin^2 \phi_{ij}.
    \end{align*}
    By symmetry, the same must hold for $z_j^\top \delta_{ij}$.
    
    Similarly, we notice that the angle $\psi_{ij}$ between $\delta_{ij}$ and $\delta_{ji}$ is $\psi_{ij} = \pi - \phi_{ij}$. The reason for this is that we must have a quadrilateral whose four internal angles must sum to $2\pi$, i.e. $\psi_{ij} + \phi_{ij} + 2 \frac{\pi}{2} = 2 \pi$. Thus, we obtain $\delta_{ij}^\top \delta_{ji} = \|\delta_{ij}\| \cdot \|\delta_{ji}\| \cos(\psi) = -\sin^2 \phi_{ij} \cos \phi_{ij}$.

    We plug these back into our equation for $\hat{z}_i'^\top \hat{z}_j'$ and simplify:
    \begin{align*}
        \hat{z}_i'^\top \hat{z}_j' &= \frac{1}{\rho'^2}\left[ z_i^\top z_j + \gamma \frac{z_i^\top \delta_{ji}}{\rho} + \gamma \frac{z_j^\top \delta_{ij}}{\rho} + \gamma^2 \frac{\delta_{ij}^\top \delta_{ji}}{\rho^2} \right] \\
        &= \frac{1}{\rho'^2}\left[ z_i^\top z_j + \gamma \frac{\rho \sin^2 \phi_{ij}}{\rho} + \gamma \frac{\rho \sin^2 \phi_{ij}}{\rho} - \gamma^2 \frac{\sin^2 \phi_{ij} \cos \phi_{ij}}{\rho^2} \right] \\
        &= \frac{1}{\rho'^2}\left[ z_i^\top z_j + 2 \gamma \sin^2 \phi_{ij} - \gamma^2 \frac{\sin^2 \phi_{ij} \cos \phi_{ij}}{\rho^2} \right].
    \end{align*}

    We now consider the original term in question:
    \begin{align*}
        \hat{z}_i'^\top \hat{z}_j' - \hat{z}_i^\top \hat{z}_j &= \frac{1}{\rho'^2}\left[ z_i^\top z_j + 2 \gamma \sin^2 \phi_{ij} - \gamma^2 \frac{\sin^2 \phi_{ij} \cos \phi_{ij}}{\rho^2} \right] - \frac{z_i^\top z_j}{\rho^2} \\
        &\leq \frac{1}{\rho^2}\left[ z_i^\top z_j + 2 \gamma \sin^2 \phi_{ij} - \gamma^2 \frac{\sin^2 \phi_{ij} \cos \phi_{ij}}{\rho^2} \right] - \frac{z_i^\top z_j}{\rho^2} \\
        &= \frac{1}{\rho^2}\left[ 2 \gamma \sin^2 \phi_{ij} - \gamma^2 \frac{\sin^2 \phi_{ij} \cos \phi_{ij}}{\rho^2} \right] \\
        &= \frac{\gamma \sin^2 \phi_{ij}}{\rho^2}\left[ 2 - \frac{\gamma \cos \phi_{ij}}{\rho^2} \right]\\
        &\leq \frac{2 \gamma \sin^2 \phi_{ij}}{\rho^2}
    \end{align*}
    
    This concludes the proof.
\end{proof}

\section{Simulations}
\label{app:simulations}

\subsection{Aparametric Simulations}

For the simulations in Section \ref{ssec:convergence_simulations}, we produced two datasets, $\mathbf{X}_1$ and $\mathbf{X}_2$, independently by randomly sampling points in $\mathbb{R}^20$ from a standard normal distribution and normalizing them to the hypersphere. The $i$-th point in dataset $\mathbf{X}_1$ is the positive counterpart for the $i$-th point in dataset $\mathbf{X}_2$. The first dataset is then set to be static while the second is modified in order to control for the embedding norms and angles between positive pairs.

We optimize the cosine similarity by performing standard gradient descent on the embeddings themselves with learning rate $10$. We consider a dataset ``converged'' when the average cosine similarity between positive pairs exceeds $0.999$.

\paragraph{Controlling for angles.} In order to control for the angle between positive pairs, we use an interpolation value $\alpha \in [-1, 1]$. Let $x_1$ be a static embedding in $\mathbf{X}_1$ and $x_2$ be the embedding in $\mathbf{X}_2$ whose angle we wish to control. In expectation, $\phi(x_1, x_2)$ will be $\pi / 2$. We therefore define the embedding $x_2$ whose angle has been controlled as 
\[ x_2' = x_2 \cdot (1 - |\alpha|) + x_1 \cdot \alpha. \]

In essence, when $\alpha=0$, $x_2' = x_2$. However, when $\alpha=1$ (resp. $\alpha=-1$), $x_2' = x_1$ (resp. $x_2' = -x_1$).

\paragraph{Controlling for embedding norms.} This setting is simpler than the angles between positive pairs. We simply scale $\mathbf{X}_2$ by the desired value.

\subsection{Parametric Simulations}
\label{app:parametric_sim}

We restate the entire implementation for the simulations in Section \ref{ssec:confidence_simulations} for completeness. We choose centers for 4 latent classes $\{c_1, c_2, c_3, c_4\}$ uniformly at random from $\mathbb{S}^{10}$ by randomly sampling vectors from a standard multivariate normal distribution and normalizing them to the hypersphere. We then obtain the latent samples $\tilde{z}$ around center $c_i$ via $z \sim \mathcal{N}(c_i, 0.1 \cdot \mathbf{I})$ and re-normalizing to the hypersphere. For each center, we produce 1K latent samples; these constitute our latent classes. We depict an example of 8 such latent classes (in 3 dimensions) in Figure \ref{fig:orig_latents}. We finally obtain the dataset by generating a random matrix in $\mathbb{R}^{11 \times 64}$ and applying it to the latent samples.

We train the InfoNCE loss via a 2-layer feedforward neural network with the ReLU activation function in the hidden layer. The network's output dimensionality is $\mathbb{R}^{11}$ so that, after normalization, it can reconstruct the original latent classes. We train the network using the supervised InfoNCE loss with a batch size of 128. Each data point's positive pair is simply another data point from the same latent class.

We visualize the learned (unnormalized) embedding space in Figure \ref{fig:learned_latents}.

\begin{figure}
    \centering
    \begin{subfigure}{0.4\linewidth}
    \includegraphics[width=\linewidth]{Images/orig_latents.png}
    \caption{}
    \label{fig:orig_latents}
    \end{subfigure}
    \quad\quad
    \begin{subfigure}{0.4\linewidth}
    \includegraphics[width=\linewidth]{Images/learned_latents.png}
    \caption{}
    \label{fig:learned_latents}
    \end{subfigure}
    \caption{\emph{Left}: A depiction of $8$ latent classes in $3$D obtained via the description in Section \ref{app:parametric_sim}. Dashed lines represent vectors from the origin to the mean of the distribution. \emph{Right}: A depiction of the learned latent space (unnormalized) using the supervised InfoNCE loss after 50 epochs of training.}
    
\end{figure}


\section{Further Discussion and Experiments}
\label{app:experiments}

\subsection{Experimental Setup}
\label{app:experiment_setup}
Unless otherwise stated, we use a ResNet-50 backbone \cite{resnet} and the default settings outlined in the SimCLR \cite{simclr} and SimSiam \cite{simsiam} papers. We use $1$e-$6$ as the default SimCLR weight decay and $5$e-$4$ as the default SimSiam one. When running on Cifar-10 and Cifar-100, we amend the backbone network's first layer as detailed in \citet{simclr}. We use embedding dimensionality $256$ in SimCLR and $2048$ in SimSiam. When reporting embedding norms, we use the projector's output in SimCLR and the predictor's output in SimSiam: these are the spaces where the loss function is applied and therefore where our theory holds.

Due to computational constraints, we run with batch-size 256 in SimCLR. Although each batch is still 256 samples in SimSiam, we simulate larger batch sizes using gradient accumulation. Thus, our default batch-size for SimSiam is 1024. 

\subsection{Opposite-Halves Effects}
\label{app:opposite_halves_effect}

We devote this section of the Appendix to studying the role of the angle between positive samples on the cosine similarity's convergence under gradient descent. Referring back to Figure~\ref{fig:convergence_sim}, we see that the effect is most impactful when the angle between positive embeddings is close to $\pi$, i.e. $\phi_{ij} > \pi - \varepsilon$ for $\varepsilon \rightarrow 0$. The following result shows that this is exceedingly unlikely for a single pair of embeddings in high-dimensional space:
\begin{proposition}
    \label{prop:unlikely_opp_halves}
    Let $x_i, x_j \sim \mathcal{N}(0, \mathbf{I})$ be $d$-dimensional, i.i.d. random variables and let $0 < \varepsilon < 1$. Then \vspace*{-0.1cm}
    \begin{equation}
    \label{eq:opp_halves_unlikely}
    \mathbb{P}\left[ \hat{x}_i^\top \hat{x}_j \geq 1 - \varepsilon \right] \leq \frac{1}{2d(1-\varepsilon)^2}.
    \end{equation}\vspace*{-0.3cm}
\end{proposition}
\begin{proof}
By \citet{distribution_of_cosine_sim}, the cosine similarity between two i.i.d. random variables drawn from $\mathcal{N}(0, \mathbf{I})$ has expected value $\mu = 0$ and variance $\sigma^2 = 1/d$, where $d$ is the dimensionality of the space. We therefore plug these into Chebyshev's inequality:
\begin{align*}
    &\text{Pr} \left[ \left|\frac{x_i^\top x_j}{\|x_i\|\cdot \|x_j\|} - \mu \right|\geq k \sigma \right] \leq \frac{1}{k^2} \\
    \rightarrow & \text{Pr} \left[ \left |\frac{x_i^\top x_j}{\|x_i\|\cdot \|x_j\|} \right |\geq \frac{k}{\sqrt{d}} \right] \leq \frac{1}{k^2}
\end{align*}

\noindent We now choose $k = \sqrt{d}(1 - \varepsilon)$, giving us
\[ \mathbb{P}\left[ \left |\frac{x_i^\top x_j}{\|x_i\| \cdot \|x_j\|}\right | \geq 1 - \varepsilon \right] \leq \frac{1}{d(1-\varepsilon)^2}.\]

It remains to remove the absolute values around the cosine similarity. Since the cosine similarity is symmetric around $0$, the likelihood that its absolute value exceeds $1 - \varepsilon$ is twice the likelihood that its value exceeds $1- \varepsilon$, concluding the proof.

We note that this is actually an extremely optimistic bound since we have not taken into account the fact that the maximum of the cosine similarity is 1.
\end{proof}

The above proposition represents the likelihood that \emph{one} pair of embeddings has large angle between them. It is therefore \emph{exponentially} unlikely for every pair of embeddings in a dataset to have angle close to $\pi$, as we would require Proposition \ref{prop:unlikely_opp_halves} to hold across every pair of embeddings. Thus, the opposite-halves effect is exceedingly unlikely to occur.

\begin{table}
    \centering
    \quad
    \parbox{.47\linewidth}{
        \begin{tabular}{lrcc}
        \toprule
        Model & Dataset \quad\quad & \makecell{Effect Rate\\Epoch 1} & \makecell{Effect Rate\\Epoch 16} \\
        \midrule
        \multirow{2}{*}{SimCLR} & Imagenet-100 & 2\% & 0\%  \\
        & Cifar-100 & 11\% & 1\% \\
        \cmidrule{1-4}
        \multirow{2}{*}{SimSiam} & Imagenet-100 & 26\% & 1\% \\
        & Cifar-100 & 21\% & 0\% \\
        \cmidrule{1-4}
        \multirow{2}{*}{BYOL} & Imagenet-100 & 28\% & 1\% \\
        & Cifar-100 & 20\% & 0\% \\
        \bottomrule
        \end{tabular}
        \captionof{table}{The rate at which embeddings are on opposite sides of the latent space (angle between a positive pair is greater than $\pi / 2$) for various datasets and SSL models.}
        \label{tbl:opposite_halves_effect}
    }
    \hfill
    \parbox{.38\linewidth}{
        \begin{tabular}{cc ccc}
        \toprule
        \multirow{2}{*}{Epoch} & & \multicolumn{3}{c}{Batch Size}\\
        & & 256 & 512 & 1024 \\
        \cmidrule{3-5}
        \multirow{2}{*}{100} & Default & 46.1 & 41.2 & 32.6 \\
        & Cut ($c=9$) & 43.1 & 46.5 & 44.3 \\
        \cmidrule{2-5}
        \multirow{2}{*}{500} & Default & 59.1 & 60.4 & 61.3\\
        & Cut ($c=9$) & 59.4 & 58.9 & 61.5 \\
        \bottomrule
        \end{tabular}
        \captionof{table}{$k$-nn accuracies for SimSiam trained with various batch sizes. We performed training for both the default and cut-initialized variants and reported $k$-nn accuracies at 100 and 500 epochs.}
        \label{tbl:cut_batch_size}
    }
\end{table}

In accordance with this, Table~\ref{tbl:opposite_halves_effect} shows that, after one epoch of training, embeddings have angle greater than $\pi/2$ at a rate of around $5\%$ and $25\%$ for SimCLR and SimSiam/BYOL, respectively. So even if the `strongest' variant of the opposite-halves effect is not occurring, a weaker one may still be. However, very early into training (epoch 16), every method has a rate of effectively 0 for the opposite-halves effect. Furthermore, the rates in Table~\ref{tbl:opposite_half_effect} measure how often $\phi_{ij} > \frac{\pi}{2}$. This is the absolute weakest version of the opposite-halves effect. Thus, while some weak variant of the opposite-halves effect may occur at the beginning of training, it does not have a strong impact on the convergence dynamics and, in either case, disappears quite quickly.

\subsection{Weight Decay}
\label{app:weight_decay}

We evaluate the effect of weight decay in the imbalanced setting in \ref{fig:weight_decay_imbalanced}, which is an analog of Figure \ref{fig:weight_decay_ablation} for the imbalanced Cifar-10 dataset detailed in Section \ref{sec:convergence}. We again see that using weight decay controls for the embedding norms and improves the convergence of both models. In correspondence with the other results on imbalanced training, we find that stronger control over the embedding norms leads to improved convergence: the high weight decay value does not perform as poorly on SimCLR as in Figure \ref{fig:weight_decay_ablation} and, on SimSiam, outperforms the other weight decay options.

\begin{figure}
    \centering
    \begin{tikzpicture}
    \node () at (0, 0) {\includegraphics[width=0.4\linewidth]{Images/wd_sweep_imbalanced.png}};

    \draw[ballblue, line width=0.07cm] (-4, -3.8) -- (-3.4, -3.8);
    \draw[azure, line width=0.07cm] (-1.3, -3.8) -- (-0.7, -3.8);
    \draw[darkblue, line width=0.07cm] (2, -3.8) -- (2.6, -3.8);

    \node () at (-1.15, -3.33) {\small \textcolor{darkgray}{Train Epoch}};
    \node () at (1.95, -3.33) {\small \textcolor{darkgray}{Train Epoch}};

    \node[inner sep=0pt] () at (-2.5, -3.82) {\textcolor{darkgray}{\scriptsize No weight decay}};
    \node[inner sep=0pt] () at (0.48, -3.82) {\textcolor{darkgray}{\scriptsize Standard weight decay}};
    \node[inner sep=0pt] () at (3.57, -3.82) {\textcolor{darkgray}{\scriptsize High weight decay}};
        
        
    \end{tikzpicture}
    \caption{An analog to Figure \ref{fig:weight_decay_ablation} performed on the exponentially imbalanced Cifar-10 dataset. Weight decays are [$0$, $1$e-$5$, $5$e-$2$] for SimCLR and [$0$, $5$e-$4$, $5$e-$2$] for SimSiam. We plot the effective learning rate in the bottom row, calculated in accordance with Section \ref{sec:convergence}.}
    \label{fig:weight_decay_imbalanced}
\end{figure}

\subsection{Cut-Initialization}
\label{app:cut_init}
We plot the effect of the cut constant on the embedding norms and accuracies over training in Figure~\ref{fig:cut_experiments}. To make the effect more apparent, we use weight-decay $\lambda=5e-4$ in all models. We see that dividing the network's weights by $c>1$ leads to immediate convergence improvements in all models. Furthermore, this effect degrades gracefully: as $c > 1$ becomes $c < 1$, the embeddings stay large for longer and, as a result, the convergence is slower. We also see that cut-initialization has a more pronounced effect in attraction-only models -- a trend that remains consistent throughout the experiments.

We also show the relationship between cut-initialization and the network's batch size on SimSiam in Table \ref{tbl:cut_batch_size}. Consistent with the literature, we see that training with large batches provides improvements to training accuracy. However, we note that larger batch sizes also significantly slow down convergence. However, cut-initialization seems to counteract this and accelerate convergence accordingly. Thus, training with cut-initialization and large batches seems to be the most effective method for SSL training (at least in the non-contrastive setting).

\begin{figure}[t!]
    \centering
    \includegraphics[width=0.95\textwidth]{Images/init_experiments.png}
    \caption{The effect of cut-initialization on Cifar10 SSL representations. $x$-axis and embedding norm's $y$-axis are log-scale. $\lambda=5$e$-4$ in all experiments.}
    \label{fig:cut_experiments}
\end{figure}

\section{More details on gradient scaling layer}
\label{app:grad_scaling}

An implementation of our GradScale layer can be found in Listing \ref{alg:grad_scaling_use}.
We note that this layer is purely a PyTorch optimization trick and does not amount to implicitly choosing a different loss function:

\begin{restatable}{proposition}{nopotential}
    \label{prop:no_potential}
    Let $t\in\mathbb{R}^n$ be a unit vector, $p: \mathbb{R}^n\backslash \{0\} \to [-1, 1], z\mapsto t^\top z/\|z\|$ the cosine similarity with respect to $t$, $\alpha \in \mathbb{R}$, and $\sigma: \mathbb{R}^n \to  \mathbb{R}, z\mapsto \|z\|^\alpha$. Then the vector field $\sigma\nabla p$ has a potential $q$, i.e., $\nabla q = \sigma \nabla p$, only for $\alpha=0$.
\end{restatable}

\begin{proof}
    Suppose $\sigma \nabla p$ has potential. Consider two paths with segments $s_1, s_2$ and $s_3, s_4$ going $t \to 2t \to -2t$ and $t \to -t \to -2t$, where the segments $s_1, s_4$ scaling $\pm t \to \pm2t$ are straight lines and the other segments $s_2, s_3$ follow great circles on $S^{n-1}$. By Proposition~\ref{prop:cos_sim_grads}, we know that $\nabla p(z)=0$ for $z\in \mathbb{R}_{\neq 0}\cdot t$. So $\sigma \nabla p$ is zero on $s_1$ and $s_4$. Moreover, we have
    \begin{align}
        \int_{s_2} \sigma \nabla p \,dz &= \int_{s_2} \|z\|^\alpha \nabla p \,dz
        = \int_{s_2} 2^\alpha \nabla p \,dz 
        = 2^\alpha \int_{s_2} \nabla p \,dz 
        = 2^\alpha \big(p(2t) - p(-2t)\big) = 2^{\alpha+1}
    \end{align}
    and similarly 
    \begin{align}
        \int_{s_3} \sigma \nabla p dz = 1^\alpha \cdot 2 = 2.
    \end{align}
    Since we assume the existence of a potential, we can use path independence to conclude 
    \begin{align}
        2^{\alpha+1} &= \int_{s_2} \sigma \nabla p \,dz 
        = \int_{s_1, s_2} \sigma \nabla p \,dz 
        = \int_{s_3, s_4} \sigma \nabla p \,dz 
        = \int_{s_3} \sigma \nabla p \,dz 
        = 2.
    \end{align}
    Thus, $\alpha=0$ and $\sigma$ does not perform any scaling.
\end{proof}




\begin{figure}
    \begin{lstlisting}[caption={PyTorch code for gradient scaling layer}, label={alg:grad_scaling}]
class scale_grad_by_norm(torch.autograd.Function):
    @staticmethod
    def forward(ctx, z, power=0):
        ctx.save_for_backward(z)
        ctx.power = power
        return z
    @staticmethod
    def backward(ctx, grad_output):
        z = ctx.saved_tensors[0]
        power = ctx.power
        norm = torch.linalg.vector_norm(z, dim=-1, keepdim=True)
        return grad_output * norm**power, None
\end{lstlisting}
\end{figure}

\begin{algorithm}[tb]
   \caption{Pytorch-like pseudo-code using the gradient scaling layer}
   \label{alg:grad_scaling_use}
\begin{algorithmic}
   \STATE {\bfseries Input:} Encoder network $model$, gradient scaling power $p$
   \STATE $z = model(batch)$
   \STATE $z = grad\_scaling\_layer.apply(z, p)$
   \STATE $sim = (\frac{z}{\|z\|})^T \frac{z}{\|z\|}$
   \STATE $loss = InfoNCE(sim)$
   \STATE $loss.backward()$
\end{algorithmic}
\end{algorithm}


\section{Additional figures}
We provide a bar plot analogous to Figure \ref{fig:in_out_violin} in Figure \ref{fig:in_out_distribution_norms}.

\begin{figure}
    \centering
    \begin{tikzpicture}   
        \node[inner sep=0pt] (image) at (0,0) {\includegraphics[width=\textwidth]{Images/Confidence/per_class_norms.pdf}};
    \end{tikzpicture}
    \caption{Bar plot which is analogous to Figure \ref{fig:in_out_violin} showing embedding magnitudes on each dataset split as a function of which dataset the model was trained on. All values are normalized by training set's mean embedding magnitude. Normalized means are represented by black bars. We use the same data augmentations for the train and test sets for consistency.}
    \label{fig:in_out_distribution_norms}
\end{figure}

We also show each Cifar-10 class's 10 highest and 10 lowest embedding-norm samples in Figure \ref{fig:cifar_norms}. These are obtained after training default SimCLR on Cifar-10 for 512 epochs. We see that the high-norm class representatives are prototypical examples of the class while the low-norm representatives are obscure and qualitatively difficult to identify. This property was originally shown by \citet{embed_norm_confidence_2}.

\begin{figure}
    \centering
    \includegraphics[width=0.48\linewidth]{Images/high_norm.png}
    \quad
    \includegraphics[width=0.48\linewidth]{Images/low_norm.png}
    \caption{\emph{Left}: highest-norm representatives (top 10) per class. \emph{Right}: lowest-norm representatives (bottom 10) per class. All from default SimCLR trained on Cifar-10.}
    \label{fig:cifar_norms}
\end{figure}

\end{document}
\endinput
%%
%% End of file `sample-sigconf.tex'.
