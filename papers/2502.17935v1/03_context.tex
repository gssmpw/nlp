\section{Study Context: League of Legends}
\textit{League of Legends} (\textit{LoL}) is a popular Multiplayer Online Battle Arena (MOBA) game, whose genre is defined by two competitive teams of human players battling for a common objective. In \textit{LoL}, two symmetrical teams of five members aim to destroy the other team's base (\textit{Nexus}). Each player selects a character from a pool of \textit{champions}, each equipped with unique abilities. Notably, ~\textit{LoL} game sessions are relatively short, generally lasting from 25 to 40 minutes. 

As \textit{LoL} is a \textit{competitive} team game, the outcome of the game depends heavily on cooperation between team members to achieve victory. The game supports real-time cooperation through diverse within-team communication channels native to the platform. In this paper, we investigate \textit{LoL} players' use of four main communication modes for corresponding with allies during the game: chat (verbal), pings, emotes, and votes (non-verbal). We illustrate how each communication mode may be used and represented in the game in Figure ~\ref{fig:modes}. The player base relies on these features to exchange key information, indicate intent, and express emotions throughout the entire session.

\begin{figure*}
    \centering
    \includegraphics[width=\textwidth]{four_modes.png}
    \caption{An example of four main communication modes of \textit{LoL}. (A) Chat is the medium through which players can type in-game messages to other players or read previous logs of in-game changes or signals, (B) Pings are quick alerts used for signaling information to other teammates, (C) Emotes are used to express emotions to other players, and (D) Votes are used to determine calls for objectives or to surrender the game.}
    \label{fig:modes}
    \Description{A screenshot of the gameplay screen of League of Legends, which demonstrates how chat, pings, emotes, and votes are shown in the game. The figure shows how each communication tool appears in the game, such as the chat box, ping and emote wheels, and the vote window on the right side of the screen.}
\end{figure*}

For verbal communication, players can type in the in-game chat before, during, and after the game (Figure ~\ref{fig:modes}A). Unlike other team-based competitive games such as \textit{Overwatch}\footnote{https://overwatch.blizzard.com/} and \textit{Valorant}\footnote{https://playvalorant.com/} or other MOBA games such as \textit{DoTA 2}\footnote{https://www.dota2.com/} and \textit{Heroes of the Storm} that offer voice chat for all teams, \textit{LoL} only enables it for a pre-formed party. Despite the potential benefits of voice-based communication for impromptu teams, \textit{LoL} developers have decided against voice chat, citing that it ``\textit{[does not solve] all behavioral issues and definitely introduces some new ones... Especially for women and POC (People of Color) who get unfairly targeted by simply participating in voice comms.}''~\cite{carver2023}


In \textit{LoL}, non-verbal communication is facilitated through pings, emotes, and votes. Pings are quick alerts used to signal information by placing markers on the map or characters. Players can access pings via a ping wheel (Figure ~\ref{fig:modes}B) or keyboard shortcuts. There are two types of pings: visual and UI pings. Visual pings, triggered by clicking the terrain or minimap, appear on the map and include generic markers for drawing attention and eight specific ``Smart'' pings (e.g., Retreat, On My Way, Assist Me) with predefined meanings shown in Figure ~\ref{fig:modes}B. UI pings share information about the status of the clicked interface elements, such as items or skills. Most pings, except non-targeted generic visual pings, are logged in the chat and accompanied by a distinct audio cue.

Emotes are expressive images or animations that convey emotions during the game, often featuring characters with various expressions like excitement, remorse, or provocations. When triggered, an emote appears above the player’s character for a few seconds and briefly in a bubble on allies’ screens. Unlike pings, emotes are visible to both allies and nearby enemies. Players can purchase emotes with in-game currency and customize their emote wheel (accessible via a shortcut) with up to nine options. Chat, pings, and emotes can be muted individually or entirely for specific players or everyone.

Finally, players can also communicate through surrender and objective votes. Starting at the 15-minute mark, a player may anonymously initiate a surrender vote, which appears on the right side of the screen. If at least 70\% of the team (or four players) vote ``Yes'' within 60 seconds, the game ends. If the vote fails, the team must wait three minutes to try again. In 2022, \textit{LoL} introduced objective voting, triggered when a player pings a field objective like Baron or Drake. This vote lets teammates decide whether to ``Take'' or ``Give'' the objective. We note that the chat system in \textit{LoL} textualizes all forms of communication. Chat messages, ping notifications, and in-game announcements are all interspersed in a single channel. 

These channels are used at different frequencies and at different points of the game. Ping usage is persistent and frequent throughout the game: the two most commonly used pings (On My Way and Enemy Missing) are used an average of $0.267$ and $0.164$ pings per minute respectively across all positions and ranks. There is a positive trend of increased ping with rank (reaching $0.489$ and $0.245$ pings per minute respectively for Master players and above)~\cite{log2024ping}, showing that as players become more skilled at the game, they are able to communicate more actively through non-verbal channels. In contrast, other non-verbal gestures are used sparingly. Though official statistics are not provided for votes and emotes, votes are limited by timers, which are tied to objectives (every 5–6 minutes), or surrender cooldowns (every 3 minutes). Emotes are used infrequently based on player norms and are often reserved for reacting to significant in-game events.

While previous works have often studied how players use one specific feature~\cite{tan2022, leavitt2016}, we aim to better understand how players use all of these features in conjunction. We illustrate the decisions players make in each communication attempt and how communication used by other team members shapes their team perception in real time.