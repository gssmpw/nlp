\section{Discussion} \label{discussion}
The results identify core factors of communication assessment and the effect of such assessment on the player's perception of the team. First, we discuss how our results provide deeper insight into the connection between trust and communication in ad hoc team communication. We also compare findings on communication in \textit{League of Legends (LoL)} that may be applied to other virtual ad hoc contexts. Then, we discuss how player values and priorities shape player engagement and assessment of communication. Finally, we offer insights on analyzing team processes through ad hoc teams in gaming environments.


\subsection{Relationship Between Trust and Communication Processes}

Many disciplines have aimed to analyze how communication structures affect team performance~\cite{chalupnik2020, finholt1990, roberts2014, capiola2020}. These works provide communication characteristics and signals that lead to more effective collaboration. However, realizing effective communication processes in swiftly starting teams is difficult, when team and individual cognitive processes may be particularly sensitive to the extreme conditions of the task environment. We thus aim to understand \textit{when} a user uses certain communication features, and more importantly, \textit{how} they make such a choice in the moment. We build upon works that have looked at verbal~\cite{tan2022, tan2021less} and non-verbal gestures (e.g., pings)~\cite{leavitt2016} in games as an indicator of team metrics. In turn, we offer a holistic view of players' communicative decision-making process, providing insight into how communication takes form in real time for a resource-limited, high-stakes environment.

In doing so, we uncover the role of trust driving the cognitive processes behind individual communication decisions. Our results find that the presence of communication signals to players the possibility of future team breakdowns. Previous works have also found that players prefer silence over excessive or negative verbal communication~\cite{buchan2016, tan2021less}. However, our results show some players showed an aversion to communication regardless of its content, even if the content is positive, wary of the communicator becoming hostile as the game progressed. Their behaviors and attitudes toward communication suggest that players' evaluation of harm from communication outweighs the possible benefits of positive interactions. Players of MOBAs, and online multiplayer games in general, join thousands of teams across their playtime, each composed of different members. This opens up the possibility for players to harmful communication at each match, and as players' past experiences with the negative impact of communication are put in the forefront of their minds, this may cultivate the reluctance to engage in communication of any kind.

We interpret these results under the framework of team processes-trust relationship. Research has demonstrated relationships between trust and team performance~\cite{erdem2003cognitive, kanawattanachai2002dynamic}, particularly in virtual teams~\cite{powell2006antecedents}. Mayer et al. defined trust as the willingness of one party to place themselves in a position of vulnerability to another, based on the expectation that the other party will carry out actions deemed important to the trustor, even without the ability to monitor or control their behavior~\cite{mayer1995trust}. According to this definition, we find that players used communication patterns as a gauge of trust in teammates to maintain commitment to the game. Wilderman et al.'s multilevel framework of trust encompasses multidimensionality and fluidity of trust~\cite{dirks2022trust} by breaking down the steps of how trust and team processes shape one another based on input-mediator-output-input model~\cite{wildman2012trust, lines2022meta}. The model posits that initial surface-level and dispositional inputs shape cognitive, affective, and attitudinal (trust) mediators, which influence team processes and performance, creating a feedback loop that adjusts trust in the team over time. From the perspective of this framework, player's previous negative experiences with team breakdowns (input of ``imported information'') and teammates' behavior alignment with these experiences (input of ``surface-level cues'') shape their perception towards the teammate (mediator of ``trust-related schema''), which in turn will shape the team process (output).

This result underscores the critical role of communication processes in fostering trust and highlights the need to examine players' in-the-moment motivations to understand how trust dynamically shapes and is shaped by team processes. By focusing on players' real-time decisions and actions, we can uncover how immediate factors --- such as perceived alignment with team goals, reactions to setbacks, or spontaneous communication efforts --- influence the development or erosion of trust. Such insights are essential for designing interventions and systems that promote adaptive teamwork and sustained collaboration in high-pressure, fast-paced environments like online multiplayer games.


\subsection{Defining Communication Processes Beyond Gaming Contexts}

Numerous multiplayer online games have been analyzed to study team communication, spanning First-Person Shooters (FPS)~\cite{tang2012verbal, taylor2012fps}, Real-Time Strategy games (RTS)~\cite{laato2024starcraft}, Massive Multiplayer Online Role-Playing Games (MMORPG)~\cite{petter2011}, and more. These games have different team structures (competitive vs cooperative), conflict types (Player-vs-Player or Player-vs-Environment), team persistence (short vs long-term), and communication channels (e.g., in-game and third-party voice chat, text chat, pre-defined messages, pings, reactions). We explain the distinctions of \textit{LoL} communication processes shaped by these differences and extend the findings to other gaming and virtual ad hoc domains.

As a competitive game, \textit{LoL} players place great importance on understanding how players are doing in comparison to the enemy team. A core design philosophy of many MOBAs is that players can easily quantify the game state. Kou et al. point to how the quantified views of the player and game state (such as rank and average kill rates) can lead to increased player stress and tension within the team~\cite{kou2018self}. Though Kou limits quantification as information gathered using third-party tools, we observe how the importance and availability of information access may shape communicative processes. In FPS games like \textit{Halo 3}, players rely on callouts from teammates to make crucial plays --- thus, disruptions to receiving these messages or incorrect calls can be a significant detriment~\cite{tang2012verbal, taylor2012fps}. On the other hand, \textit{LoL} players prioritize maintaining team integrity even at the expense of losing information, as reduced player commitment causes far more significant consequences for the game.

These differences in team contexts should be considered when designing and implementing communication features. In professional domains such as healthcare~\cite{roberts2014, chalupnik2020, evans2021} and military~\cite{pascual1999, capiola2020} where the consequences of every action may be critical, two-way communication is advantageous~\cite{zijlstra2012interaction}. However, in online gaming, layers often find communication to fall behind the quickly changing state of the game or remain unanswered. Thus, rather than communication that fades away, fast-paced games may want to adopt ``statuses'' that are persistent throughout the game. Finally, instead of sharing all information with every individual, information that caters to the player may reduce mental load and bring attention to critical information. Meanwhile, new forms of communication should be not easily abused to threaten the psychological safety of the players through misuse, such as \textit{LoL}'s bait ping~\cite{ramun2023} that was introduced and removed due to overtly toxic use of the new feature. 

\subsection{Understanding Player Values, Priorities, and Motivations in Team Communication}

On the surface, it may seem like all players in \textit{LoL} are working towards the same primary goal: winning the game. However, even in Solo Ranked mode where players are more extrinsically motivated by a performance-based ranking system, we find that players hold different, individual-driven goals and values, which can affect their in-game behavior and perceived experience~\cite{bruhlmann2020motivational}.

The findings on the communication assessment process show that participants prioritize different team and individual processes of the game. Some players value their psychological safety, forgoing access to game information in the process. Others instead approach the game with the drive to win, even if it means spending more time communicating important information or putting out possible firestarters of conflict.

Female players face unique challenges in their communication processes in \textit{LoL}. Female players must adopt coping strategies to prevent gender-based harassment from other players. In line with previous work, female participants described previous experiences of sexual harassment, flaming, and other verbal abuse regarding their gender identity~\cite{fox2016women, norris2004gender, mclean2019female}. To protect their psychological well-being, the female players took active steps to conceal their gender identity, such as changing their communication style and username to sound more gender-neutral. However, we also highlight that several female players prioritize and value the expression of their identity and preference during team processes. Safety may be an important aspect of communication, but communication process designs should also consider that players' priorities and values may not align with the most ideal or effective forms of communication.

Furthermore, the communication behavior of the players may reflect their unique cultural influence. \textit{LoL} is a popular and widely spread game for many Korean players~\cite{turbosmurfs}. In South Korea, playing \textit{LoL} is often thought of as mainstream culture~\cite{kukinews2019}. Nearly 70\% of all Korean male adolescents play \textit{LoL}~\cite{newsis2024}. For Korean players, \textit{LoL} represents a virtual social platform that enables the strengthening of social bonds between friends. However, in turn, the presence of such social connections may lead to increased pressure to perform better, as their in-game performance may influence a player's real-life identity. Such social bonds thus may exacerbate the competitiveness of players and perpetuate and normalize certain communicative behaviors. Another unique aspect of South Korean \textit{LoL} culture is the prevalent use of internet cafes, locally known as \textit{PCBang} (meaning ``Personal Computer Room''). These are spaces where players pay money by the hour to play games using high-quality computer equipment. This culture may also increase the need for players to end the game quickly and ``give up'' more easily, as longer games tie directly in with monetary loss.


\subsection{Merits and Pitfalls of Analyzing Teamwork through Online Multiplayer Games}

Online multiplayer games have been a ripe ground for understanding team dynamics, specifically for virtual, temporary teams~\cite{kou2014, kwak2015exploring, tan2022}. Clearly defined goals and boundaries of online games make it an ideal space to study interpersonal interactions and relationships for collaboration. The rich data from in-game statistics especially open it up for data-driven approaches that measure the impact of communication behavior on the team outcome. Analyzing teamwork in the online multiplayer gaming space enables easier and more direct control of the conditions present in team characteristics. It also allows analysis of teamwork across different skill levels and virtuality in a measurable and comparable way.

However, one must be careful to not isolate the processes and content from the context and insinuate a causal relationship between certain team behaviors and outcomes without considering the mechanics within. The limited methodology of qualitative gaming research has been conducted away from the actual gaming context, instead finding more general insights into their experiences and perceptions ~\cite{buchan2016, tan2021less}. If we wish to understand the team processes evolving throughout the game, it is important to incorporate robust real-time contextual data and give weight to individual perspectives that are unobservable from statistical or linguistic data alone.




