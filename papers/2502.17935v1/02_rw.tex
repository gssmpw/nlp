\section{Related Work}
Our work investigates in-game team communication in \textit{League of Legends} (\textit{LoL}), an online multiplayer game with a virtual ad hoc collaboration setting. Thus, we review prior work on communication in virtual ad hoc collaboration and communication patterns of online multiplayer games to situate our work.


\subsection{Communication in Virtual Ad Hoc Collaboration}
Team communication takes many different forms. Among them, a swiftly started collaboration between previously unknown members, referred to as an \textit{ad hoc} team, condenses communication actions into fast-paced, transient, and reactive teamwork~\cite{finholt1990}. Communication in ad hoc teams is critical to the mission outcome~\cite{white2018, jarvenpaa1998, mesmer2009}. Previous work has explored successful communicative practices adopted by unfamiliar members in various professional ad hoc disciplines, such as healthcare~\cite{roberts2014, chalupnik2020, evans2021}, military~\cite{pascual1999, capiola2020}, and software development~\cite{cherry2008}. Skilled ad hoc members working in high-intensity situations strategize their communication to establish rapport, coordinate tasks, and share critical information at opportune moments. These probes find that proactive and coherent face-to-face communication mediates swift trust among team members, which in turn influences team performance~\cite{capiola2020, strater2008}.

In comparison to traditional face-to-face contexts, virtual ad hoc teams are characterized by technological constraints that hinder effective communication. Computer-mediated communication in virtual, and often global, teams must overcome the physical, social, and cultural distances inherent to distributed work~\cite{oleary2007, morrison2020}. The limited communication channels of virtual teams put up significant barriers to cognitive alignment between strangers --- individuals working in temporary virtual teams experience reduced presence, individual identification, and immersion due to the lack of contextual cues~\cite{altschuller2013, altschuller2010}. Virtual collaborations without non-verbal signals (e.g., facial expressions and voice inflections) lead to weaker trust as members struggle to gauge intent, emotions, and engagement from their collaborators~\cite{morrison2020}. When members are unable to observe each others' actions, they may act based on their biased or inaccurate, negative perceptions~\cite{penarroja2013}. The absence of contextual information thereby implies that virtual ad hoc teams must frequently engage in explicit, task-oriented communication to facilitate coordination.

In reality, previous work reveals that the efficacy of virtual ad hoc team communication is shaped by the unique conditions of the collaborative context~\cite{marlow2017, mesmer2009}. Marlow et al.~\cite{marlow2017} identify three key communication constructs in their conceptual framework of virtual team communication processes: communication frequency, quality, and content. They emphasize the importance of analyzing these disparate constructs under the specific virtual context for a more granular understanding of effective communication processes. For example, while communication quality has often been linked to team cohesion, communication frequency shows varying relationships depending on the team composition and structure. Some studies find that overt sharing of information promotes positive team states~\cite{mesmer2009} and increased communication frequency is generally salient with team development and functioning~\cite{monge2003}. On the other hand, minimal communication can be more conducive to successful collaboration in team compositions where team members possess high levels of expertise and share a common understanding of the task at hand~\cite{entin1999}.

More specifically, communication in an online multiplayer gaming context parallels collaboration dynamics within virtual ad hoc teams working under significant pressure. Crisis management teams have been commonly used to explore team decision-making in dynamic and extreme situations~\cite{uitdewilligen2018crisis, altschuller2008potential}. These contexts include natural disasters~\cite{longstaff2008natural}, pandemics~\cite{white2020covid}, and critical operations like nuclear plant control~\cite{stachowski2009crises}. These extreme environments are high-pressure work settings defined by hostile conditions, isolation, limited time, and severe consequences for failure~\cite{harrison1984exotic, bell2016extreme}. Many of these characteristics are mirrored in online multiplayer games, where players face time limits and immediate, high-impact outcomes, which differ from traditionally studied, more stable contexts of ad hoc collaboration~\cite{musick2021cognition}. In such extreme contexts, Zijistra et al. revealed that communication patterns that are stable, balanced, and reciprocal lead to more effective collaboration~\cite{zijlstra2012interaction}. Similarly, Vinella et al. found through simulations of virtual ad hoc bomb diffusion teams that usage of role-aligned communication demonstrated higher team performance and perceived collaboration quality~\cite{vinella2022personality}.

Likewise, previous works have investigated communication patterns in various ad hoc teams. Although communications in \textit{LoL} may share similarities with previous work as the teams within can be viewed as virtual ad-hoc teams under high pressure, the unique contexts of \textit{LoL} may pose a different influence on the communication. 
Moreover, there still remains a gap in understanding how these communication patterns are driven by underlying individual cognitive processes in real time. Thus, understanding in-the-moment motivations behind communication decisions can provide insights into what deters teams from achieving effective communication and how communication influences team collaboration in real-world ad hoc contexts.

\subsection{Effective Communication Processes for Ad Hoc Teams in Online Multiplayer Games}

Gaming research has long aimed to map effective communication practices for ad hoc teams in online multiplayer games. Previous work has examined teamwork and communication in both cooperative games, such as puzzle platformers~\cite{tan2021less} and Massively Multiplayer Online Role Playing Games (MMORPG)~\cite{petter2011}, as well as competitive games, such as the First-Person Shooters (FPS)~\cite{tang2012verbal, taylor2012fps} and Real-Time Strategy (RTS) games~\cite{laato2024starcraft}. Communication processes in different game genres are defined and constrained by in-game affordances, gameplay mechanics and design, and play motivation. For example, MMORPG players' motivation centers on socialization and immersion, leading to more social-oriented interactions~\cite{bisberg2022contagion}. Competitive team games with short and intense game sessions emphasize goal-oriented communication: Tang et al. and Taylor observed reliance on callouts and codified speech for coordination among competitive FPS players~\cite{taylor2012fps, tang2012verbal}. In the asymmetrical horror game \textit{Dead by Daylight}\footnote{https://deadbydaylight.com/}, players formulate metacommunicative codes to overcome the lack of explicit communication channels; however, the ambiguity of interpretations caused frustration among players~\cite{deslauriers2020dbd}. This body of research highlights the diverse ways communication adapts to the unique demands and constraints of different game genres. 

Among these genres, Multiplayer Online Battle Arena (MOBA) games embody many components of challenging synchronous collaboration, including rapidly forming and dissolving team composition, high interdependence, and dynamic, high-stress environment. Various works have observed the prevalence of negative within-team communication in MOBAs, which undermines player performance and satisfaction ~\cite{monge2022effects, kou2014, canossa2021honor}. Additionally, non-verbal communication affordances (e.g., pings, emotes, animations) enable dispersion of communication across diverse channels according to situational needs. Despite --- or perhaps due to --- these complex factors, numerous works have examined effectual communication patterns in MOBAs. These works often use data-driven approaches to identify antecedents to successful or failed communication processes~\cite{tan2022, zheng2023, leavitt2016, csengun2022players}. For instance, Tan et al.~\cite{tan2022} have shown that positive chat sequences, such as apology to encouragement and suggestion to acknowledgment, improve team cohesion and teamwork in \textit{LoL}. The usage frequency of non-verbal pings in \textit{Heroes of the Storm}\footnote{https://heroesofthestorm.blizzard.com/} showed a concave relationship with player performance as it enabled swift and precise communication, but also became interruptions and distractions when too abundant~\cite{leavitt2016}. Meanwhile, Buchan and Taylor~\cite{buchan2016} qualitatively approached communication through the lens of the players' subjective experiences in MOBA games. They identified communication as a core category perceived to be influencing team play. The results showed that players associated excessive quantity of communication with negative team experiences, favoring no communication to excessive communication. These studies study the relationship between communication patterns and teamwork processes. But communication is not conducted in isolation --- communication processes are subject to the influences of the dynamic game environment.

Thus, our work builds upon the previous literature to identify in-the-moment communication decisions to uncover the individually motivated mechanics of player communication in MOBAs. We look at team communication in the context of \textit{LoL}, a well-established testing ground for probing team dynamics in temporary teams due to its clearly defined game parameters and rich in-game data~\cite{kou2014, kwak2015exploring, kwak2015linguistic}. By observing and inquiring about players' communication choices in \textit{LoL} during real-time play, we conceptualize the assessment processes that inform player communication behavior. We also offer a comparative analysis of when various communication media are used, offering new insights on relatively underexplored non-verbal communication modes such as emotes and votes.


\subsection{Communication Breakdown in Online Multiplayer Games}

Despite the efforts to foster effective collaboration and communication in online multiplayer games, in reality, cooperative online multiplayer games are plagued by team communication breakdowns. Though previously mentioned work has highlighted the pathways to achieve successful communication in online ad hoc teams, many obstacles block players from adopting such practices. Previous work has demonstrated that players desire and recognize productive communication behaviors within online multiplayer games~\cite{kou2014}. Yet, online multiplayer games, especially those of competitive genres like MOBA, are prone to within-team conflicts arising from unhealthy communication patterns between allies. MOBA games have frequently demonstrated interactions between players in which the communication becomes unconstructive, hostile, or abusive~\cite{kou2020toxic, beres2021, nexo2023players}. Aggressive and hostile communication patterns such as flaming and trolling often lead to a breakdown in team cooperation, causing emotional distress, threatening psychological safety, and decreasing overall team morale~\cite{kou2020toxic}.

To promote enjoyable gaming experiences and maintain player retention, game developers have made an effort to design supportive communication tools that expedite the information sharing process (e.g., pings and votes) or encourage social bonding and copresence (e.g., ``fist bump'' in \textit{LoL}~\cite{jarvis2024}). Yet, these tools take time to break into players' routines, and at worst, are misused for adverse behaviors. We also highlight the normative impact of gaming culture that blocks players' ability to engage in constructive communication. Previous work has shown that players exhibit dismissive behavior towards healthy communication, downplaying positive messages~\cite{poeller2023} and normalizing harmful communication~\cite{beres2021}. In this work, we explore the real-time and individual-specific challenges to realize effective communication during game sessions with unfamiliar teammates. Our work observes how communication, in conjunction with normative beliefs, informs the player's perception of their teammates.