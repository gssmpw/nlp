\section{Limitations and Future Work} \label{limitations}

We acknowledge several limitations of our work. First, though our observation and interview-based study setting allowed us to better understand in-the-moment communication patterns, it could have affected participants on their in-game communication patterns. Although no participants reported any difference in how they would have behaved usually in the post-game interview beyond some discomfort with the equipment, their behaviors could have been affected. The sense of being observed by a third person can especially affect them to perform less of the behaviors that can be perceived negatively~\cite{kim2020understanding}. This could have resulted in participants not engaging in severe toxic communication patterns despite toxicity in online games being one of the frequent communication patterns~\cite{kou2020toxic, beres2021, nexo2023players}. Future work can consider using the Social Desirability Scale~\cite{mccrae1983social} to understand whether the participants are likely to behave or answer in a way that is socially desirable.

Second, we acknowledge that there could be diverse factors that could affect the result that our study did not focus on. For instance, players' demographics such as gender or nationality can affect their communication patterns. Previous work indicated that there exist gender differences in not only how players play the game but also in how they interact with other players~\cite{veltri2014gender}, such as female players are more likely to show communal attitudes or encourage others~\cite{hong2012gender}. Although our participant demographic reflects the imbalanced player demographic of \textit{LoL}, future work can investigate gender differences surrounding our three research questions. In addition, we have only recruited participants playing in the Korean \textit{LoL} server. Existing work suggests that there exist cultural differences in how \textit{LoL} players engage in toxic behaviors~\cite{sengun2019exploring}, but future work can also investigate cultural differences in other in-game communications.

We primarily focus on how players utilize and perceive the available communication features in \textit{LoL}. This work makes sense of when and why players engage or disengage from communication, weighing the risks and benefits of the communication within the confines and context of the evolving situations. Further research should look into not only when or what medium the player chooses to communicate through, but also the types of communicative behaviors, such as aggression, attribution, and socialization, to observe their impact on the game state and team communication.