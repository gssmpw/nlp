%%
%% This is file `sample-sigconf-authordraft.tex',
%% generated with the docstrip utility.
%%
%% The original source files were:
%%
%% samples.dtx  (with options: `all,proceedings,bibtex,authordraft')
%% 
%% IMPORTANT NOTICE:
%% 
%% For the copyright see the source file.
%% 
%% Any modified versions of this file must be renamed
%% with new filenames distinct from sample-sigconf-authordraft.tex.
%% 
%% For distribution of the original source see the terms
%% for copying and modification in the file samples.dtx.
%% 
%% This generated file may be distributed as long as the
%% original source files, as listed above, are part of the
%% same distribution. (The sources need not necessarily be
%% in the same archive or directory.)
%%
%%
%% Commands for TeXCount
%TC:macro \cite [option:text,text]
%TC:macro \citep [option:text,text]
%TC:macro \citet [option:text,text]
%TC:envir table 0 1
%TC:envir table* 0 1
%TC:envir tabular [ignore] word
%TC:envir displaymath 0 word
%TC:envir math 0 word
%TC:envir comment 0 0
%%
%% The first command in your LaTeX source must be the \documentclass
%% command.
%%
%% For submission and review of your manuscript please change the
%% command to \documentclass[manuscript, screen, review]{acmart}.
%%
%% When submitting camera ready or to TAPS, please change the command
%% to \documentclass[sigconf]{acmart} or whichever template is required
%% for your publication.
%%
%%
\documentclass[sigconf]{acmart}

% \PassOptionsToPackage{table}{xcolor}
\usepackage{multirow}
\usepackage{threeparttable}
\usepackage{makecell}
\usepackage{pifont}
% \usepackage{xcolor}
\usepackage{graphicx}
\usepackage{diagbox}
\usepackage{booktabs}
% \definecolor{rowcoloryellow}{HTML}{FFF0DE}
% \definecolor{rowcolorgreen}{HTML}{D6E8D5}
% \definecolor{rowcolorblue}{HTML}{DAE9FC}
% \definecolor{rowcolorpurple}{HTML}{E2D6E7}


%%
%% \BibTeX command to typeset BibTeX logo in the docs
\AtBeginDocument{%
  \providecommand\BibTeX{{%
    Bib\TeX}}}

%% Rights management information.  This information is sent to you
%% when you complete the rights form.  These commands have SAMPLE
%% values in them; it is your responsibility as an author to replace
%% the commands and values with those provided to you when you
%% complete the rights form.
\setcopyright{acmlicensed}
\copyrightyear{2018}
\acmYear{2018}
\acmDOI{XXXXXXX.XXXXXXX}
%% These commands are for a PROCEEDINGS abstract or paper.
\acmConference[Conference'25]{}{ACM Conference}{2025}
%%
%%  Uncomment \acmBooktitle if the title of the proceedings is different
%%  from ``Proceedings of ...''!
%%
%%\acmBooktitle{Woodstock '18: ACM Symposium on Neural Gaze Detection,
%%  June 03--05, 2018, Woodstock, NY}
\acmISBN{978-1-4503-XXXX-X/YY/MM}


\settopmatter{printacmref=false}
%%
%% Submission ID.
%% Use this when submitting an article to a sponsored event. You'll
%% receive a unique submission ID from the organizers
%% of the event, and this ID should be used as the parameter to this command.
%%\acmSubmissionID{123-A56-BU3}

%%
%% For managing citations, it is recommended to use bibliography
%% files in BibTeX format.
%%
%% You can then either use BibTeX with the ACM-Reference-Format style,
%% or BibLaTeX with the acmnumeric or acmauthoryear sytles, that include
%% support for advanced citation of software artefact from the
%% biblatex-software package, also separately available on CTAN.
%%
%% Look at the sample-*-biblatex.tex files for templates showcasing
%% the biblatex styles.
%%

%%
%% The majority of ACM publications use numbered citations and
%% references.  The command \citestyle{authoryear} switches to the
%% "author year" style.
%%
%% If you are preparing content for an event
%% sponsored by ACM SIGGRAPH, you must use the "author year" style of
%% citations and references.
%% Uncommenting
%% the next command will enable that style.
%%\citestyle{acmauthoryear}


%%
%% end of the preamble, start of the body of the document source.
\begin{document}

%%
%% The "title" command has an optional parameter,
%% allowing the author to define a "short title" to be used in page headers.
\title{Text-Promptable Propagation for Referring Medical Image Sequence Segmentation}

%%
%% The "author" command and its associated commands are used to define
%% the authors and their affiliations.
%% Of note is the shared affiliation of the first two authors, and the
%% "authornote" and "authornotemark" commands
%% used to denote shared contribution to the research.
\author{Runtian Yuan$^{1}$, 
        Mohan Chen$^{1}$,
        Jilan Xu$^{1}$,
        Ling Zhou$^{1}$,  
        Qingqiu Li$^{1}$,
}
\author{        
        Yuejie Zhang$^{1,*}$,
        Rui Feng$^{1,*}$, 
        Tao Zhang$^2$,
        Shang Gao$^3$
}
\affiliation{
$^{1}$Fudan University $^{2}$Shanghai University of Finance and Economics $^{3}$Deakin University
\country{Australia}
}

%%
%% By default, the full list of authors will be used in the page
%% headers. Often, this list is too long, and will overlap
%% other information printed in the page headers. This command allows
%% the author to define a more concise list
%% of authors' names for this purpose.
\renewcommand{\shortauthors}{Runtian Yuan et al.}

%%
%% The abstract is a short summary of the work to be presented in the
%% article.
\begin{abstract}

Referring Medical Image Sequence Segmentation (Ref-MISS) is a novel and challenging task that aims to segment anatomical structures in medical image sequences (\emph{e.g.} endoscopy, ultrasound, CT, and MRI) based on natural language descriptions. 
This task holds significant clinical potential and offers a user-friendly advancement in medical imaging interpretation.
Existing 2D and 3D segmentation models struggle to explicitly track objects of interest across medical image sequences, and lack support for interactive, text-driven guidance. %the capacity to incorporate human interaction.
To address these limitations, we propose Text-Promptable Propagation (TPP), a model designed for referring medical image sequence segmentation. TPP captures the intrinsic relationships among sequential images along with their associated textual descriptions. 
Specifically, it enables the recognition of referred objects through cross-modal referring interaction, and maintains continuous tracking across the sequence via Transformer-based triple propagation, using text embeddings as queries.
To support this task, we curate a large-scale benchmark, Ref-MISS-Bench, which covers 4 imaging modalities and 20 different organs and lesions.
Experimental results on this benchmark  demonstrate that TPP consistently outperforms state-of-the-art methods in both medical segmentation and referring video object segmentation.
\end{abstract}

%%
%% The code below is generated by the tool at http://dl.acm.org/ccs.cfm.
%% Please copy and paste the code instead of the example below.
%%

\begin{CCSXML}
<ccs2012>
   <concept>
       <concept_id>10010147.10010178.10010224.10010245.10010247</concept_id>
       <concept_desc>Computing methodologies~Image segmentation</concept_desc>
       <concept_significance>500</concept_significance>
       </concept>
 </ccs2012>
\end{CCSXML}

\ccsdesc[500]{Computing methodologies~Image segmentation}

%%
%% Keywords. The author(s) should pick words that accurately describe
%% the work being presented. Separate the keywords with commas.
\keywords{Text-Promptable Propagation, Referring Medical Image Sequence Segmentation}
%% A "teaser" image appears between the author and affiliation
%% information and the body of the document, and typically spans the
% %% page.
% \begin{teaserfigure}
%   \includegraphics[width=\textwidth]{sampleteaser}
%   \caption{Seattle Mariners at Spring Training, 2010.}
%   \Description{Enjoying the baseball game from the third-base
%   seats. Ichiro Suzuki preparing to bat.}
%   \label{fig:teaser}
% \end{teaserfigure}

% \received{20 February 2007}
% \received[revised]{12 March 2009}
% \received[accepted]{5 June 2009}

%%
%% This command processes the author and affiliation and title
%% information and builds the first part of the formatted document.
\maketitle


\section{Introduction}

% \begin{figure}
%     \centering
%     \includegraphics[width=0.9\linewidth]{motivation2.png}
%     \caption{Text-promptable approach supports clinical workflows by allowing clinicians to identify and segment anatomical structures through text prompts. %who may need radiologists to provide guidance, such as pointing out the exact location of a pulmonary nodule.
%     \jl{To be consistent with my version of the main text, the figure here could be simplified, just mention how text-promptable segmentation benefits (1) radiologists, (2) clinicians (3) patients, this is also the general workflow of medical images in examinations. MARK: COULD BE MERGED INTO FIGURE2.}
%     }
%     \label{fig:motivation}
% \end{figure}

\begin{figure*}[ht]
    \centering
    \includegraphics[width=1\linewidth]{intro.pdf}
    \caption{Limitations and motivations. (a) Conventional 2D models do not incorporate temporal context and fail to utilize intrinsic consistencies in medical image sequences. (b) 3D models lack slice-level object representations for modeling continuity. (c) Multi-class segmentation models are limited to predefined classes and cannot use language to specify a particular class. (d) To address these limitations, Referring Medical Image Sequence Segmentation is introduced, offering substantial clinical values. (e) Our TPP leverages medical text prompts to segment referred objects across medical image sequences in both 2D and 3D data.}
    \label{fig:intro}
\end{figure*}


Medical image segmentation plays an important role in modern healthcare by enabling precise delineation of anatomical regions and pathological areas, which is essential for diagnosis, treatment planning, and disease monitoring~\cite{chen2019ldct, cao2023large}. 
Accurate segmentation facilitates quantitative analysis of medical images, supporting early detection of tumors and assessment of organ functionality.

This paper considers medical image sequence segmentation (MISS) task, which involves segmenting medical images from 2D video-based examinations (\emph{e.g.}, endoscopy and ultrasound) and 3D imaging techniques (\emph{e.g.}, CT and MRI). These modalities produce medical image sequences, \emph{i.e.}, temporally or spatially ordered frames or slices that capture the same anatomical structures, including organs and lesions.
Importantly, such sequences are not merely collections of isolated snapshots; rather, they are intrinsically linked, with each frame or slice providing a unique view of the same object from different angles or planes. 
% In clinical practice, radiologists typically examine and interpret CT or MRI scans slice by slice, rather than relying on the overall 3D volume reconstruction. 2D slices are the preferred unit of analysis, as they clearly reveal the morphology, boundaries, and characteristics of anatomical structures.
The consistencies among these sequential images are crucial for comprehensive medical analysis and diagnosis.
Modern deep learning models~\cite{ronneberger2015unet,isensee2021nnunet,chen2021transunet,cao2022swinunet,ma2024medsam} have revolutionized image segmentation, however, their capabilities in handling medical image sequences still worth exploration. 

As shown in Figure~\ref{fig:intro} (a)-(c), the main limitations that restrict their real-world clinical utility are three-fold: 
\textbf{First}, most 2D image segmentation models ~\citep{ronneberger2015unet, chen2021transunet} treat frames from video-based examinations or slices from 3D volumes as independent samples, ignoring the inherent spatial and temporal consistencies. 
\textbf{Second}, although existing 3D models~\citep{milletari2016vnet,zhao2023one} can capture correlations between slices, the employed 3D convolutions or attention operations over full 3D patches are computationally expensive and lack the modeling and tracking of objects across sequences.
\textbf{Third}, existing models segment all predefined categories in an image without the ability to incorporate human interaction, limiting their practical value in scenarios where clinicians only care about certain objects. 



%most state-of-the-art models are 
% , offering promising solutions for complex segmentation challenges across diverse modalities, including MRI~\cite{xxx}, CT~\cite{xxx}, endoscopy~\cite{xxx}, and ultrasound~\cite{xxx}


To address these challenges, we go beyond MISS tasks and instead focus on the more challenging \textbf{Referring Medical Image Sequence Segmentation (Ref-MISS)} task, which requires the model to identify and segment anatomical structures corresponding to given natural language within medical image sequences. 
Enabling users to interact with models %Building models that allow users to interact with 
and specify target structures through language offers several practical benefits, as shown in Figure~\ref{fig:intro} (d): 
(1) radiologists benefit from AI-assisted, text-promptable segmentation results to validate their findings;
(2) clinicians with limited imaging expertise receive clearer explainable visual outputs of lesions from the referring model for decision-making and comprehensive diagnosis;
(3) patients gain from simplified, text-driven visualizations that improve their understanding of medical conditions. 
Ultimately, text-promptable segmentation bridges the gap between visual data and human interpretability, fostering more efficient, accurate, and collaborative healthcare workflows.

To solve Ref-MISS, we propose a novel Text-Promptable Propagation (TPP) model, designed to leverage the intrinsic relationships among sequential images along with their associated textual descriptions, as shown in Figure~\ref{fig:intro} (e). 
TPP integrates two key components: 
(1) \textbf{Cross-modal Referring Interaction.} This component incorporates medical text prompts with vision-language alignment and fusion to recognize referred objects.
Medical text prompts provide critical context by highlighting specific regions of interest and guiding attention. We propose cross-modal referring interaction to integrate prompts, linking medical image sequences with text prompts across vision and language modalities. 
(2) \textbf{Transformer-based Triple Propagation}.
To uniformly model the temporal relationships between 2D frames and cross-slice interactions in 3D volumes, we employ a Transformer-based encoder-decoder architecture, leveraging propagation strategies to track referred objects.



To support this task, we curate a large dataset, \textbf{Ref-MISS-Bench}, from existing public medical datasets, and use Large Language Models (LLMs) to automatically generate text prompts based on different attributes of anatomical structures. The prompts are then validated by senior radiologists. Ref-MISS-Bench is sourced from 18 diverse medical datasets across 4 imaging modalities, including MRI, CT, ultrasound, and endoscopy. It covers 20 different organs and lesions from various regions of the body, and is utilized in both the training and testing stages, as illustrated in Figure~\ref{fig:datasets}.
% ADD A SENTENCE HERE, WHETHER IT IS USED IN TRAIN/TEST or BOTH.

To summarize, our contributions are as follows:
\begin{itemize}
    \item We focus on the novel task, \textbf{Referring Medical Image Sequence Segmentation (Ref-MISS)}, and establish a strong model, \textbf{T}ext-\textbf{P}romptable \textbf{P}ropagation (TPP), which utilizes medical text prompts to identify referred objects and propagate vision-language information for continuous tracking through sequential images.
    % \item We propose a novel task, \textbf{Referring Medical Image Sequence Segmentation (Ref-MISS)}, aiming to segment referred anatomical structures via medical text prompts within medical image sequences from both 2D video-based examinations and 3D volumes. 
    % \item We establish a strong model, \textbf{T}ext-\textbf{P}romptable \textbf{P}ropagation (TPP), which utilizes medical text prompts to identify referred objects and propagate vision-language information for tracking through sequential images. 
    \item We introduce a large-scale benchmark, \textbf{Ref-MISS-Bench}, which covers 4 imaging modalities and 20 anatomical structures. Ref-MISS-Bench consists of 125,487 images from 3,644 sequences in the training set and 41,078 images from 1,061 sequences in the test set, providing a comprehensive data foundation for Ref-MISS task.
    \item Experiments demonstrate that our approach outperforms state-of-the-art methods in 2D/3D/text-guided medical image segmentation and referring video object segmentation, while also incorporating human-interaction capabilities. 
\end{itemize}


% Modern medical image segmentation methods have achieved significant success using deep learning models, which typically rely on annotations and learned visual features to segment predefined targets. These methods generally serve trained radiologists, offering visual display of specific anatomical regions.  
% In contrast, integrating text prompts into medical image segmentation provides a more intuitive and adaptable approach. By allowing users to specify regions of interest through natural language, this method enhances both precision and customizability. 
% % \jl{hm, it is a bit confusing here, so who actually benefit from AI assistance? Clinicians or Radiologists? The description here seems not matching to the figure. Moreover, as patients also benefit from text-prompt seg, what about also showing this process in the figure?} 

% As shown in Figure~\ref{fig:motivation}, clinicians with limited imaging expertise often have difficulty in recognizing abnormalities and require assistance from radiologists to pinpoint the exact location of lesions, such as pulmonary nodules. The referring model can segment targets corresponding to text prompts, thus aiding clinicians in receiving clearer visual cues for decision-making and diagnosis. 
% Moreover, radiologists benefit from AI-assisted second opinions to validate their findings, while patients gain from simplified, text-driven visualizations that improve their understanding of medical conditions. Ultimately, text-promptable segmentation bridges the gap between visual data and human interpretability, fostering more efficient, accurate, and collaborative healthcare workflows.

% In medical imaging, a substantial portion comes from 2D video-based examinations, such as endoscopy and ultrasound, and 3D imaging techniques like CT and MRI. These modalities produce sequential frames or slices that capture the same anatomical structures (\emph{e.g.}, organs and lesions). These image sequences are not merely collections of individual snapshots but are deeply interconnected, with each frame or slice providing a unique view of the same object from different angles and in varying shapes. The consistencies among these sequential images are crucial for comprehensive medical analysis and diagnosis. 
% % Despite these advancements, there is a notable scarcity of research focused on effectively utilizing sequential images to improve medical analysis. Existing methods often overlook the potential of leveraging these sequential relationships in conjunction with textual descriptions. 

% However, \textcolor{red}{current studies on medical image segmentation using 2D models~\citep{ronneberger2015unet, chen2021transunet} treat frames from video-based examinations and slices from 3D volumes as independent samples, ignoring the inherent spatial and temporal consistencies. While 3D models~\citep{milletari2016vnet,zhao2023one} incorporate spatial information, they still lack explicit object-level modeling.}
% \jl{Need a sentence here, why object-level modeling is important in medical.}
% Another drawback is the limitation to closed sets of categories and lack of human interactions. In multi-class segmentation tasks, existing methods~\citep{chen2021transunet, zhao2022prior} typically restrict results to predefined classes, reducing flexibility and preventing the specification of particular classes for referring segmentation. This rigidity limits the adaptation of segmentation processes to specific clinical needs or emerging requirements in complex medical scenarios. These two primary drawbacks are depicted in Figure~\ref{fig:intro}.
% % involve models designed to automatically identify and separate organs, tissues, or pathological regions from medical images like CT scans, MRIs, and X-rays. 
% % These segmentation models are essential for tasks such as disease screening, organ segmentation, and anomaly detection. 
% % As depicted in Figure~\ref{fig:intro} (a-c), these models face two primary drawbacks:
% % 1) Separate network architectures for 2D and 3D medical images. Researchers often apply distinct methodologies for 2D and 3D medical images, using 2D approaches for planar images or slices~\citep{ronneberger2015unet, chen2021transunet} and 3D techniques for volumetric data~\citep{milletari2016vnet,zhao2023one}. \jl{So why 3D techniques neglect the consistency across sequential slices from 3D volumns?} These networks neglect the consistency across temporal frames from 2D video-based examinations and sequential slices from 3D volumes, leading to discrepancies and sub-optimal outcomes when images are interrelated or part of a sequence. 
% % 2) Limitation to closed sets of categories and lack of human interactions. In multi-class segmentation tasks, existing methods~\citep{chen2021transunet, zhao2022prior} typically restrict results to predefined classes, reducing flexibility and preventing the specification of particular classes for referring segmentation. This rigidity limits the adaptation of segmentation processes to specific clinical needs or emerging requirements in complex medical scenarios.

% To address these challenges, we aim to develop a schema capable of jointly processing and explicitly modeling objects in medical imaging data from both videos and volumes, while supporting text-promptable segmentation.
% \textcolor{red}{This insight naturally gives rise to an innovative task}, \textbf{Referring Medical Image Sequence Segmentation}, which seeks to identify and segment anatomical structures corresponding to given text prompts within medical image sequences. These sequences involve both temporally related frames from videos and spatially related slices in volumes. 
% % This process involves segmenting regions of interest within sequential images over time or across different slices.
% To achieve this, we propose the \textbf{T}ext-\textbf{P}romptable \textbf{P}ropagation (TPP) model, a strong baseline designed to leverage the intrinsic relationships among sequential images along with their associated textual descriptions. As shown in Figure~\ref{fig:intro} (d), TPP unifies frames from 2D video-based examinations and slices from 3D volumes, and supports the segmentation of arbitrary objects of interest based on medical text prompts.

% TPP integrates two key components: 
% 1) \textbf{Cross-modal Referring Interaction.} This component incorporates medical text prompts with vision-language alignment and fusion to recognize referred objects.
% Medical text prompts provide critical context by highlighting specific regions of interest and guiding attention. We propose cross-modal referring interaction to integrate prompts that describe various characteristics of anatomical structures, linking medical image sequences with text prompts across vision and language modalities. 
% % This interaction facilitates a more comprehensive understanding of the data by combining insights from both textual and visual information.
% 2) \textbf{Transformer-based Triple Propagation}.
% To uniformly model the temporal relationships between 2D frames and cross-slice interactions in 3D volumes, we employ a Transformer-based encoder-decoder architecture, leveraging propagation strategies to track referred objects throughout the sequences.

% % \jl{To support this task, we curate a large dataset, \textbf{Ref-MISS},  from existing public medical datasets, and prompt Large Language Models to automatically generate text prompts based on xxx }
% To support this task, we curate a large dataset, \textbf{Ref-MISS}, from existing public medical datasets, and use large language models to automatically generate text prompts based on different attributes of anatomical structures. Ref-MISS is sourced from 18 diverse medical datasets across 4 imaging modalities, including MRI, CT, ultrasound, and endoscopy. It covers 20 different organs and lesions from various regions of the body, as illustrated in Figure~\ref{fig:datasets}.

% % To further aid in the 
% %In terms of the interpretation of these sequences, accompanying 
% % As shown in Figure~\ref{fig:intro} (d), text descriptions often provide valuable knowledge and context, highlighting specific regions of interest and guiding focus. 
% % However, accurately segmenting and analyzing medical images using text prompts presents a challenge due to the complexity and variability of medical imaging data. \jl{Well, I could not see very clear relationship between [However xxx] and [To tackle this xxx]. }
% % To tackle this, we process 3D volumetric data and video examination data into 2D image sequences uniformly and introduce cross-modal referring interaction to associate text prompts with medical image sequences. To model sequential relationships, we highlight the referred object through propagation strategies.

% To summarize, our contributions are as follows:
% % \jl{based on the introduction, I would suggest separate into 4 points, (1) novel task, with clinical values xxx; (2) large-scale benchmark xxx; (3) and (4) are ok.}
% \begin{itemize}
%     \item We propose a novel task, \textbf{Referring Medical Image Sequence Segmentation}, aiming to segment referred anatomical structures via medical text prompts. 
%     \item We establish a strong baseline model, \textbf{T}ext-\textbf{P}romptable \textbf{P}ropagation (TPP), which utilizes medical text prompts to identify referred objects and propagate vision-language information for tracking through sequential images. 
%     \item We introduce a large-scale benchmark, Ref-MISS, covering 4 imaging modalities and 20 anatomical structures, providing a comprehensive data infrastructure.
%     \item Experiments demonstrate that our approach outperforms state-of-the-art methods in 2D/3D/text-guided medical image segmentation and referring video object segmentation, highlighting its superior performance.
% \end{itemize}

\begin{figure*}[]
    \centering
    \includegraphics[width=1\linewidth]{arch.pdf}
    \caption{Architecture of our Text-Promptable Propagation for referring medical image sequence segmentation. (a) Overview of TPP. Triple Prop. is short for Triple Propagation. (b) Illustration of Triple Propagation in Transformer decoder, consisting of box-level, mask-level, and query-level propagation. 
    % Line from $E_{v,t-1}$ to Memory Read block is omitted for simplicity.
    }
    \label{fig:arch}
\end{figure*}


\section{Related work}
\label{RL}

\subsection{Medical Image Segmentation} 
%As mentioned earlier, researchers typically apply distinct methods for 2D~\citep{ronneberger2015unet} and 3D~\citep{cciccek20163dunet, milletari2016vnet} medical images. 2D models are used for planar images or slices, while 3D models are intended to learn volumetric features. % implicitly.
As mentioned earlier, researchers typically apply 2D models~\citep{ronneberger2015unet} for planar images or slices, and 3D models~\citep{cciccek20163dunet, milletari2016vnet} to learn volumetric features implicitly.
\citet{isensee2021nnunet} introduced a versatile, self-adaptive deep learning framework specifically designed for medical image segmentation tasks, extending the U-Net architecture and its 3D version. 
\citet{chen2021transunet} pioneered the combination of Transformer-based architecture with Convolutional Neural Networks (CNNs) for medical image segmentation, applying a slice-by-slice inference on 3D volumes without considering interrelationships among slices. 
Some works~\citep{ji2021pns, painchaud2022echocardiography, lin2023shifting} utilize spatial-temporal cues and \citep{li2023lvit,zhong2023ariadne,bui2024mmiunet} introduce report texts as guidance to enhance segmentation performance. However, these models are limited to specific image modalities and tasks.  

\subsection{Medical Vision-Language Models} Medical vision-language models have achieved success across multiple downstream tasks, including diagnosis classification~\citep{moon2022medvill, medclip, chexzero, lu2023mizero}, lesion detection~\citep{qin2023mvlm, huang2024adapting}, image segmentation~\citep{zhao2023one, li2023lvit}, report generation~\citep{yan2022clinical-bert, BioViL-T}, and visual question answering~\citep{singhal2023med-palm, moor2023med-flamingo}. 
\citet{qin2023mvlm} designed auto-generation strategies for medical prompts and transferred large vision language models for medical lesion detection. 
\citet{liu2023clip-driven} incorporated text embedding learned from Contrastive Language-Image Pre-training (CLIP) to segmentation models.
\citet{zhao2024biomedparse} proposed BiomedParse, a biomedical foundation model that can jointly conduct segmentation, detection and recognition across nine imaging modalities.
\citet{zhao2023one} built a model based on Segment Anything Model~\citep{kirillov2023sam} for medical scenarios driven by text prompts, but the model focused on 3D medical volume segmentation and did not consider the sequential relationships between scans. 
To the best of our knowledge, we are the first to use medical text prompts to specify segmentation targets across medical image sequences.

% \textbf{Text Promptable Segmentation.} 
% TGANet TextSAM TP-SIS.

\subsection{Referring Video Object Segmentation}
\citet{gavrilyuk2018actor} were the first to propose inferring segmentation from a natural language input, extending two popular actor and action datasets with natural language descriptions.
\citet{seo2020urvos} constructed the first large-scale referring video object segmentation (RVOS) dataset and proposed a unified referring video object segmentation network.
\citet{wu2022referformer} and \citet{botach2022mttr} presented Transformer-based RVOS frameworks, enabling end-to-end segmentation of the referred object.
\citet{wu2023onlinerefer} designed explicit query propagation for an online model. 
\citet{luo2024soc} aggregated inter- and intra-frame information via a semantic integrated module and introduced a visual-linguistic contrastive loss to apply semantic supervision on video-level object representations.
\citet{yan2024referred} enabled multi-modal references to capture multi-scale visual cues and designed inter-frame feature communication for different object embeddings for tracking along the video.

Inspired by these works, the Referring Medical Image Sequence Segmentation task processes both 2D and 3D medical data into image sequences, enabling in-depth exploration of sequence-level consistency guided by text prompts.


\section{Methodology}
% \jl{Since we mention a new task/a new benchmark in this paper, I would suggest refine this section, (1) Sec.3.1, Task Definition/Problem Formulation; (2) Sec.3.2 Model; (3) Sec.3.3 Benchmark dataset. }

% \jl{If doing so, we might need to mention method first followed by dataset in the Introduction.}

\subsection{Problem Formulation}
This paper tackles the Referring Medical Image Sequence Segmentation (Ref-MISS) task. Formally, given $T$ frames or slices $\{I_t \in \mathbb{R}^{{3}\times{H}\times{W}}\}_{t=1}^T$ from a medical image sequence 
and $N_p$ medical text prompts $\{P_i\}_{i=1}^{N_p}$ (Section~\ref{benchmark}), the referring model $\mathcal{M}$ aims to predict the segmentation masks $\{\hat{m_t} \in\{0,1\}^{{H}\times{W}}\}_{t=1}^T$ for the referred object corresponding to the prompts, which can be formulated as: 
% \jl{A question here, if you mean the mask, it should be in \{0, 1, ..., C\}, not $\mathbb{R}$, you can use one-hot representation, like $\{\hat{m_t} \in\{0,1\}^{{C}\times {H}\times{W}}\}_{t=1}^T$,  where $C$ denotes total classes.}
\begin{equation}
    \{\hat{m_t}\}_{t=1}^T = \mathcal{M}\left(\{I_t\}_{t=1}^T, \{P_i\}_{i=1}^{N_p}\right).
\end{equation}
An overview of our framework is illustrated in Figure~\ref{fig:arch} (a). The referring model $\mathcal{M}$ comprises two core components: \textbf{Cross-Modal Referring Interaction} (Section~\ref{method1}) to recognize the referred objects, and \textbf{Transformer-based Triple Propagation} (Section~\ref{method2}) to maintain continuous tracking across sequences. The training and inference procedures are described in Section~\ref{method3}. 

%\textbf{Overview.} 
% To achieve this, we propose a text-promptable framework. In addition to an automatic text-promptable scheme, this framework comprises two core components: \textbf{Cross-Modal Referring Interaction} (Section~\ref{method1}) to recognize the referred objects, and \textbf{Transformer-based Triple Propagation} (Section~\ref{method2}) to maintain continuous tracking across sequences. The training and inference procedures are described in Section~\ref{method3}. An overview of our framework is illustrated in Figure~\ref{fig:arch} (a).
%We provide an automatic text-promptable schema for referring medial image sequence segmentation. As illustrated in Figure~\ref{fig:arch} (a), our framework comprises two key components: Cross-Modal Referring Interaction (Section~\ref{method1}) to recognize the referred objects and Transformer-based Triple Propagation (Section~\ref{method2}) for continuous tracking through sequences. Then we elaborate the training and inference procedures (Section~\ref{method3}).
%We provide an automatic schema for text-promptable medial image sequence segmentation. 
% This framework comprises three key components: Cross-Modal Referring Interaction (Section~\ref{method1}), Transformer-based Triple Propagation (Section~\ref{method2}), and the training and inference procedures (Section~\ref{method3}).

% Given $T$ frames or slices over a sequence clip $\{I_t \in \mathbb{R}^{{3}\times{H}\times{W}}\}_{t=1}^T$ and $N_p$ medical text prompts $\{P_i\}_{i=1}^{N_p}$, the model predicts the segmentation masks $\{\hat{m_t} \in \mathbb{R}^{{H}\times{W}}\}_{t=1}^T$ for the referred object corresponding to the prompts.



% \jl{Generally, I think:

% (1) the notations are too complicated, you can find some ways to simply them.

% (2) Better if we separate this section into three main parts, each of them corresponds to the main techinical contribution listed in the introduction. i.e. (1) text-promptable (2) unify 2D and 3D processing. 

% For example:

% 3.1: includes image/text-encoding and cross-modal fusion. This process highlights the text-promptable part, in this part, it seems that each slice/image is processed independently. (if I understand this right). Then we dont need to have temporal index in this part and add them later.

% 3.2  Unified temporal/cross-slice processing: including Transformer-encoder/decoder and your proposed Triple propagation here. Maybe you can simply the heads here as they are not very important.

% 3.3  Training and inference: introduce the loss, and inference xxx here.
% }


\subsection{Cross-Modal Referring Interaction}
\label{method1}
% \jl{NEED A SENTENCE HERE, What is the goal of this module.}

% \jl{
% \paragraph{Visual Feature Extraction.}
% The visual encoder $\phi_v$ takes as input the medical image sequence $\{I_t\}_{t=1}^T$, and encodes them in a per-frame manner. The visual encoder outputs multi-scale features:
% \begin{equation}
%     \{v_{t}^{1}, v_{t}^{2},v_{t}^{3},v_{t}^{4}\}=\phi_v(I_t)
% \end{equation}
% where $v_{t}^l\in\mathbb{R}^{{C^{l}}\times {H^{l}} \times{W^{l}}}$ denotes the visual feature at the $l^{th}$ level. 
% }

% \jl{
% \paragraph{Textual Feature Extraction.}
% The linguistic encoder $\phi_t$ is a Transformer model pre-trained on PubMed~\cite{xxx}. It takes as input the medical text prompts $\{P_i\}_{i=1}^{N_p}$, encodes each prompt separately, and outputs the textual feature. The encoding process of each prompt $P_i$ is defined as: 
% \begin{equation}
%     u_{i} = \phi_t(P_i) \in\mathbb{R}^{{L_i}\times C}
% \end{equation}
% where $L_i$ and $C$ denotes the length of sentence embedding and hidden dimension, respectively.
% }

% \jl{
% \paragraph{Vision-Language Alignment and Fusion.}
% After obtaining the visual and textual features, we align and fuse them to enhance the model's focus on the referred objects, and identify the most relevant prompt for each image clip. 
% This process involves three key steps. 

% 1) \textbf{Cross-modal attention.} 
% For each image $I_t$, we apply Multi-Head Attention (MHA) over the visual feature maps at the last three levels (\emph{i.e.}, $v^2_t,v^3_t,v^4_t$) and the word-level embeddings from each of the text prompts $u_i$. 
% This produces a set of proposal features:
% \begin{equation}
%     A_{t}^{l,i} = \operatorname{MHA} \left(v^l_t, u_{i} \right), 
% \end{equation}    
% where $A_{t}^{l,i}\in\mathbb{R}^{H_l\times W_l\times L_i}$ represents the attention output between the $l$-th visual feature map and the $i$-th text prompt. 
% Each prompt (i.e., $P_1, P_2, P_3$)
% yields its own set of proposals, denoted as $\displaystyle \mathbb{A}$, $\displaystyle \mathbb{B}$, $\displaystyle \mathbb{C}$, respectively. This enables modeling of complex vision-language dependencies. %allows us to capture intricate relationships between the visual and textual data. 
% }
\paragraph{Visual Feature Extraction.}
The visual encoder $\phi_v$ takes the medical image sequence $\{I_t\}_{t=1}^T$ as input, and encodes them in a per-frame manner. The visual encoder outputs multi-scale features $F_v$ for each image, which is a set of feature maps:
\begin{equation}
    \{f_{v}^{l}\}_{l=1}^{4}=\phi_v(I_t) \in\mathbb{R}^{{C^{l}}\times {H^{l}} \times{W^{l}}},
\end{equation}
where $C^l$, $H^l$ and $W^l$ denote the channel dimension, height, and width of the feature map at the $l^{th}$ level, respectively. 

\paragraph{Textual Feature Extraction.}
The linguistic encoder $\phi_t$ takes the medical text prompts $\{P_i\}_{i=1}^{N_p}$ as input, encodes each prompt independently, and outputs the textual feature $F_p$, which is a set of word-level embeddings $\{f_{p}^{i}\}_{i=1}^{N_p}$. The encoding process of each prompt $P_i$ is defined as: 
\begin{equation}
    f_p^{i} = \phi_t(P_i) \in\mathbb{R}^{{Len_i}\times C},
\end{equation}
where $Len_i$ and $C$ denotes the length of sentence embedding and hidden dimension, respectively.

% \paragraph{Feature Extraction.}
% Image clips $\{I_t\}_{t=1}^T$ and medical text prompts $\{P_i\}_{i=1}^{N_p}$ are separately fed into a visual encoder and a linguistic encoder  to extract visual features $F_v$ for each image and textual features $F_p$ for each prompt. 
% The visual features $F_v$ is a set of feature maps $\{f_{v}^{l} \in \mathbb{R}^{{C^{l}}\times {H^{l}} \times{W^{l}}}\}_{l=1}^{4}$, where $C^l$, $H^l$ and $W^l$ denote the channel dimension, height, and width of the feature map at the $l^{th}$ level, respectively. 
% The textual features $F_p$ is a set of word-level embeddings $\{f_{p}^{i} \in \mathbb{R}^{{len_i}\times C}\}_{i=1}^{N_p}$, where $len_i$ and $C$ denote the sentence length and channel dimension of the $i^{th}$ prompt, respectively.

\paragraph{Vision-Language Alignment and Fusion.}
After obtaining the visual and textual features, we align and fuse them to enhance the model's focus on the referred objects 
%within images by leveraging text prompts and assists in selecting 
and identify the most relevant prompt for each image clip. 
This process involves three key steps. 

(1) \textbf{Cross-modal attention.} For each image, we apply Multi-Head Attention (MHA) mechanisms between the visual feature maps at the last three levels ($l=\{2,3,4\}$) and the word-level embeddings from the text prompts. This produces a set of proposal features:
\begin{equation}
    A_{}^{l,i} = \operatorname{MHA} \left(f_{v}^{l}, f_{p}^{i} \right), 
\end{equation}    
where $A_{}^{l,i}$ represents the attention output between the $l$-th visual feature map and the $i$-th text prompt. Each prompt (\emph{i.e.}, $P_1, P_2, P_3$)
%leads to corresponding proposals, 
yields its own set of proposals, denoted as $\displaystyle \mathbb{A}$, $\displaystyle \mathbb{B}$, $\displaystyle \mathbb{C}$, respectively. This enables modeling of complex vision-language dependencies. %allows us to capture intricate relationships between the visual and textual data. 

(2) \textbf{Weighted fusion of proposals.} To identify the referred object, i.e. $\displaystyle \mathbb{A} \cap \displaystyle \mathbb{B} \cap \displaystyle \mathbb{C}$, we flatten each proposal and apply a three-layer Multi-Layer Perceptron (MLP) to compute prompt-specific relevance weights:
%weights for each text prompt. These weights reflect the relevance of each prompt to the current clip. 
\begin{equation}
    %A_{}^{l,i} = \operatorname{MHA} \left(f_{v}^{l}, f_{p}^{i} \right), 
    W_{}^{l,i} = \operatorname{Softmax} \left(\operatorname{MLP} \left(A^{l,i}\right)\right),
\end{equation}
%Using these weights, we perform a weighted sum of the attention output to obtain the fused visual features. This step integrates the most pertinent aspects of the text prompts with the visual data. The process can be formulated as:
which are then used to perform a weighted sum over prompts:
% \begin{equation}
%     A_{}^{l,i} = \operatorname{MHA} \left(f_{v}^{l}, f_{p}^{i} \right), 
%     W_{}^{l,i} = \operatorname{Softmax} \left(\operatorname{MLP} \left(A^{l,i}\right)\right),
% \end{equation}
% \begin{equation}
% \end{equation}
\begin{equation}
    F'_{v} = \left\{\sum_{i=1}^{N_p} f_{v}^{l} \cdot A^{l,i} \cdot W^{l,i}\right\}_{l=2}^{4}.
\end{equation}
This step generates the fused visual features, integrating the most pertinent aspects of text prompts with the visual data.

(3) \textbf{Prompt selection for query input.} For textual features, we select the most relevant prompt with the highest weight score produced by the feature maps at the first level ($l=\{1\}$). The selected prompt feature $F'_{p}$ is then used as the query input to the Transformer decoder.
\begin{equation}
    \hat{w} = \mathop{\arg\max}\limits_{i\in\{1,\dots,N_p\}}\left(W^{l=1,i}\right),
    F'_{p}  = f_p^{\hat{w}}.
\end{equation}
% \begin{equation}
% \end{equation}

% By performing these steps, our approach effectively merges visual and textual information, resulting in enhanced and contextually relevant visual features for segmentation tasks. This method not only improves the accuracy of the segmentation but also ensures that the guidance provided by text prompts is effectively utilized in the analysis of medical images.

\begin{figure*}
    \centering
    \includegraphics[width=1\linewidth]{datasets.pdf}
    \caption{An illustration of focus areas in Ref-MISS-Bench. Each colored block represents specific organ/lesion class from corresponding [dataset], along with number of training and testing cases (images).}
    \label{fig:datasets}
\end{figure*}

% \subsection{Triple-Propagated Tracking Navigation}
\subsection{Transformer-based Triple Propagation}
\label{method2}

\paragraph{Transformer.} Our Transformer architecture is adapted from Deformable DETR~\citep{zhu2021deformable}. For each image $I_t$, the Transformer encoder takes the flattened visual features $F'_{v,t}$ and 2D positional encoding as input, producing encoded output $E_{v,t}$ through multi-scale deformable attention and several feed-forward layers. The output of the Transformer encoder $E_{v,t}$ and the textual feature of the selected prompt $F'_{p,t}$ are then fed into the Transformer decoder. We repeat $F'_{p,t}$ $N_q$ times to introduce $N_q$ queries, denoted as $q_t$. Meanwhile, each image receives sequential cues from the previous frame (except for the first image) in temporal order. The Transformer decoder thus generates $N_q$ embeddings for each image, denoted as $q_t^{embed}$. 

\paragraph{Prediction Heads.} Three prediction heads are constructed following the Transformer decoder. The output embeddings from the Transformer decoder, $q_t^{embed}$, are then processed by these prediction heads. 
(1) The \textbf{box head} consists of a three-layer feed-forward network (FFN) with ReLU activation, except for the last layer, which predicts the box offset. This offset is added to the base box coordinates to determine the location of the referred object, denoted as $b_t$.
(2) The \textbf{mask head} is implemented by dynamic convolution~\citep{tian2020conditional}. It takes multi-scale features from the feature pyramid network (FPN) $f_m$, concatenates them with relative coordinates, and uses a controller to generate convolutional parameters $\theta_t$. Conditional convolution is then applied to the visual features to generate $N_q$ segmentation masks $m_t$.
\begin{equation}
    \theta_t = \operatorname{Controller} \left(q_t^{embed}\right),
\end{equation}
\begin{equation}
    \left\{m_t^{i}\right\}_{i=1}^{N_q} = \left\{\phi^{i} \left(f_{m};\theta_t^{i}\right)\right\}_{i=1}^{N_q}.
\end{equation}
Here, the controller is also a three-layer FFN with ReLU activation. $\phi^i$ represents three $1\times1$ convolutional layers with 8 channels per query, using parameters $\theta_t^i$ generated by the controller. 
(3) Since our text prompts contain class information, the \textbf{class head} indicates whether the object is referred by the text prompt. 

\paragraph{Triple Propagation.} %Frames or slices across a sequence of medical images often exhibit consistency in appearance or spatial relationships. To take advantage of this temporal coherence, 
Medical image sequences often exhibit high temporal consistency in appearance and spatial structure. To exploit this, we propagate the box, mask, and query embeddings derived from the previous image to inform predictions for the current image, as depicted in Figure~\ref{fig:arch} (b). This triple propagation %leverages the temporal consistency within the sequence, 
enhances  robustness and accuracy in medical image sequence analysis. %and ultimately contributing to more reliable outcomes.

Given previous predictions $y_{t-1} = \{b_{t-1}^i, m_{t-1}^i, c_{t-1}^i\}_{i=1}^{N_q}$, %the outputs of the previous image, 
we choose the best prediction $\{b_{t-1}^{\hat{n}}, m_{t-1}^{\hat{n}}, c_{t-1}^{\hat{n}}\}$, which is of the highest class score. Consequently, except for the first image, which has $N_q$ queries, subsequent images only receive one query propagated from the previous best prediction. 

{\textbf{Box-level} Propagation.} The box coordinates from the previous image $b_{t-1}^{\hat{n}}$ provide a valuable reference for estimating the location of the referred object in the current image. We use these coordinates as the initial box for the current image, \emph{i.e.} $b_{t}^{base}$, leveraging the spatial continuity to provide a strong prior for localization. Box-level propagation improves precision by refining the search around a plausible region.
%between images to more accurately predict the object's position. Box-level propagation allows us to refine the object's localization by starting from a well-informed estimate.
% , thereby improving the overall precision of our predictions in the temporal sequence.

{\textbf{Mask-level} Propagation.} Similarly, the visual features encoded by the Transformer encoder $E_{v,t-1}$ and the segmentation mask $m_{t-1}^{\hat{n}}$ from the previous image offer valuable semantic context that can aid in analyzing the current image. To effectively utilize this prior knowledge, we employ a memory-read mechanism that generates key and value maps for the memory. The memory map $M_{t-1}$ is a concatenation of $m_{t-1}^{\hat{n}}$ and the first-level of $E_{v-1,t}$, and the memory read operation is defined as:
\begin{equation}
    M_{t-1} = \operatorname{Concat} \left(m_{t-1}^{\hat{n}}, E_{v,t-1}^{l=2}\right),
\end{equation}
\begin{equation}
    K = \psi \left( M_{t-1} \right), V = \varphi \left( M_{t-1} \right),
\end{equation}
\begin{equation}
    E_{v,t}^{l=2} = \operatorname{Softmax} \left(\frac{E_{v,t}^{l=2} K}{\sqrt{C^{l=2}}}\right)V,
\end{equation}
where $\psi$ and $\varphi$ are two parallel $3\times3$ convolutional layers. The first level of $E_{v,t}$ is now a memory-read map. It is concatenated with feature maps of other levels and then fed into the deformable attention module in the Transformer decoder after flattening.

{\textbf{Query-level} Propagation.} Having confirmed the 
query index $\hat{n}$, we propagate the corresponding output query embedding $q_{t-1}^{embed}$ to the current image. Here, we use a three-layer FFN to transform the embedding to $q_{t}$. This query-level propagation allows for the transmission of embedded context for the same target.% across images.


\subsection{Training and Inference}
\label{method3}

\paragraph{Training.} We have $N_q$ predictions $y_t = \{b_t^i, m_t^i, c_t^i\}_{i=1}^{N_q}$ for each image, where $b_t^i \in \mathbb{R}^4$, $m_t^i \in \mathbb{R}^{\frac{H}{4} \times \frac{W}{4}}$, and $c_t^i \in \mathbb{R}^1$ represent the predicted box location, segmentation mask, and probability of the referred object, respectively. The ground-truth, in the same format, is denoted as $Y_t = \{B_t, M_t, C_t\}$. We compute a matching loss $\mathcal{L}_{match}$ to find the best prediction:
\begin{equation}
\begin{aligned}
\mathcal{L}_{match,t}\left(y_t, Y_t\right) & = \lambda_{box}\mathcal{L}_{box}\left(y_t, Y_t\right) \\
& + \lambda_{mask}\mathcal{L}_{mask}\left(y_t, Y_t\right) \\
& +  \lambda_{cls}\mathcal{L}_{cls}\left(y_t, Y_t\right),
\end{aligned}
\end{equation}
\begin{equation}
    \hat{n}_{q,t} = \mathop{\arg\min}\limits_{i\in\{1,\dots,N_q\}}\left(\mathcal{L}_{match,t}\right),
\end{equation}
where $\lambda_{box}$, $\lambda_{mask}$, and $\lambda_{cls}$ are loss coefficients. $\mathcal{L}_{box}$ is implemented as the sum of L1 loss and GIoU loss, $\mathcal{L}_{mask}$ combines Dice loss and binary mask focal loss, and $\mathcal{L}_{cls}$ is focal loss. $\hat{n}_{q,t}$ represents the query index of the best prediction.
The network is optimized by minimizing the sum of $\mathcal{L}_{match,t}$ for the best predictions across all $T$ images.
\begin{equation}
    \mathcal{L} = \frac{1}{T} \sum_{t=1}^{T} \mathcal{L}_{match,t}^{\hat{n}_{q,t}}.
\end{equation}

\paragraph{Inference.} During inference, we select the query with the highest class score as the best prediction, which can be formulated as:
\begin{equation}
    \hat{n'}_{q,t} = \mathop{\arg\max}\limits_{i\in\{1,\dots,N_q\}}\left(c_t^i\right).
\end{equation}
The final segmentation masks for each image $\{\hat{m_t}\}_{t=1}^T$ are selected using the query index $\hat{n'}_{q,t}$ from the $N_q$ predictions $\{m_t^i\}_{i=1}^{N_q}$. 
Due to our propagation strategy, the best prediction of the first image is propagated to subsequent images, leading to a single query for each of the remaining images. Therefore, for $t > 1$, the final mask simplifies to $\hat{m_t} = m_t$. %for those images with only one query, $\hat{m_t} = m_t \left(t > 1\right)$.



\begin{table*}[]
\caption{Comparison with task-specific medical image segmentation methods. Numbers in \textbf{bold} indicate the best and \underline{underlined} ones represent the second best. ${^1}$Average of ACDC and CAMUS, ${^2}$Average of BTCV, Pancreas-CT, and Spleen segmentation dataset. $^{3}$Average of Breast Cancer DCE-MRI
Data and RIDER. $^{4}$Average of CVC-ClinicDB, CVC-ColonDB, ETIS, and ASU-Mayo.}
\label{comparison_medical}
\begin{center}
\renewcommand\arraystretch{1.1} 
% \begin{threeparttable}
\scalebox{1}{
\begin{tabular}{cccccccccccc}
\toprule
{\bf Method} & \textbf{Type} & {Heart${^1}$} &  {Lung\tnote{}} & \makecell[c]{Abd-\\omen${^2}$} &  \makecell[c]{Pro-\\state} &  {\makecell[c]{Brain \\tumor}} &  \makecell[c]{Breast \\mass${^3}$} &  \makecell[c]{Liver \\tumor} &  \makecell[c]{Kidney \\tumor}  &  {Polyp${^4}$} &  \textbf{ Overall}\\ 
\cmidrule(r){1-2} \cmidrule(){3-11} \cmidrule(l){12-12}
% \multicolumn{11}{l}{\textcolor{gray}{\textit{Uni-modal}}} \\
UNetR~\citep{hatamizadeh2022unetr} & Image-only & - & 84.69 & 70.33 & - & 76.15& 61.23 & 63.42 & 74.21 & - & 71.67 \\
Swin-UNet~\citep{cao2022swinunet} & Image-only & - & 85.40 & 70.96 & - & 75.48 & 60.27 & 64.90 & 74.38 & - & 71.90 \\
nn-UNet~\citep{isensee2021nnunet} & Image-only & 85.63 & 81.59 & 72.31 & 89.73 & 76.57 & 56.80 & 74.89 & 77.06 & 47.99 & 73.62 \\
MedSAM~\citep{ma2024medsam} & Image-only & 85.98 & 86.57 & 73.94 & 89.91 & 77.98 & 62.34 & 62.91 & 77.47 & 75.50 & 76.96  \\
\midrule
% \multicolumn{11}{l}{\textcolor{gray}{\textit{Multi-modal}}} \\
LViT~\citep{li2023lvit} & Text-image  & 79.58 & 83.87 & 60.45 & 90.22 & 75.67 & 48.87 & 63.99 & 64.77 & 58.63 & 69.56 \\
LGMS~\citep{zhong2023ariadne} & Text-image  & 83.58 & 86.08 & 70.20 & 91.61 & 78.06 &51.80 & 64.03 & 74.48 & 61.94 & 74.64 \\
MMI~\citep{bui2024mmiunet} & Text-image  & 82.60 & 85.54 & 64.96 & 90.24 & 76.71 & 61.77 & 64.96 & \textbf{78.10} & 71.30 & 75.13 \\
\midrule
 Ours & Text-image & \textbf{87.19} & \textbf{88.77}  & \textbf{72.80} & \textbf{93.13} & \textbf{78.24} & \textbf{65.40} & \textbf{65.27}  & \underline{77.73}  & \textbf{75.56} & \textbf{78.23}\\
\bottomrule
\end{tabular}
}
% \end{threeparttable}
\end{center}
\end{table*}

\begin{table*}[]
\caption{Comparison with state-of-the-art methods on referring video object segmentation.}
\label{comparison_rvos}
\begin{center}
\renewcommand\arraystretch{1.1} 
% \begin{threeparttable}
\scalebox{1}{
\begin{tabular}{cccccccccccc}
\toprule
{\bf Method}& {\bf Backbone} &  {Heart${^1}$} &  {Lung\tnote{}} & \makecell[c]{Abd-\\omen${^2}$} &  \makecell[c]{Pro-\\state} &  {\makecell[c]{Brain \\tumor}} &  \makecell[c]{Breast \\mass${^3}$} &  \makecell[c]{Liver \\tumor} &  \makecell[c]{Kidney \\tumor}  &  {Polyp${^4}$} &  \textbf{ Overall}\\ 
\cmidrule(r){1-2} \cmidrule(){3-11} \cmidrule(l){12-12}
URVOS~\citep{seo2020urvos} & ResNet-50 & 83.92  & 84.61  & 60.19 & 91.92 & 74.59 & 55.91  & 27.43  & 72.24 & 66.17 & 68.55 \\
ReferFormer~\citep{wu2022referformer} & ResNet-50 & 86.29  & 84.19 & 72.12 & 89.79 & 76.60   & 60.70  & 47.43  & 61.75  & 62.75 & 71.29 \\
OnlineRefer~\citep{wu2023onlinerefer} & ResNet-50 & 83.93 & 85.27 & 63.48 & 91.69 & 77.55 & 64.81 & 39.70 & 74.75 & 72.77 & 72.66 \\
 Ours & ResNet-50  & \textbf{87.19} & \textbf{88.77}  & \textbf{72.80} & \textbf{93.13} & \textbf{78.24} & \textbf{65.40} & \textbf{65.27}  & \textbf{77.73}  & \textbf{75.56} & \textbf{78.23}\\
\midrule
ReferFormer~\citep{wu2022referformer} & Swin-L & 84.12 & 82.56 & 66.05 & 90.58  & 76.89 & 61.53 & 57.43 & 78.31 & 67.35 & 73.87 \\
OnlineRefer~\citep{wu2023onlinerefer} & Swin-L & 84.37 & 83.59 & 60.39 & 90.72  & 77.46 & 57.22  & 54.50 & 69.91  & \textbf{78.47} & 72.96 \\
 Ours & Swin-L & \textbf{84.47}& \textbf{84.96} & \textbf{66.41} & \textbf{91.54} & \textbf{77.96} & \textbf{65.90} & \textbf{59.32} & \textbf{79.27} & \underline{77.56} & \textbf{76.38} \\
\midrule
SOC~\citep{luo2024soc} & V-Swin-T & 81.76 & 84.84 & 62.55 & 86.42  & 75.55  & \textbf{61.57} & 35.30  & 70.01  & 60.04 & 68.67 \\
MTTR~\citep{botach2022mttr} & V-Swin-T & 84.80 & 84.92  & 64.23  & 89.96  & 76.21 & 57.74  & 53.68  & 67.31  & 71.12 &72.22 \\
  Ours & V-Swin-T & \textbf{84.98} & \textbf{85.19} & \textbf{65.57} & \textbf{92.34}& \textbf{77.37}  & \underline{59.17} & \textbf{54.26} 
 & \textbf{76.07} & \textbf{77.11} & \textbf{74.67} \\
\bottomrule
\end{tabular}
}
% \end{threeparttable}
\end{center}
\end{table*}


\section{Benchmark Construction}
\label{benchmark}
\paragraph{Dataset Curation.}
Ref-MISS-Bench is curated from 18 medical image sequence datasets with 20 anatomical structures across 4 different imaging modalities, as shown in Figure~\ref{fig:datasets}. 

These datasets are categorized by imaging modalities as follows:
(1) \textbf{MRI datasets}.
2018 Atria Segmentation Data~\citep{xiong2021la},
RVSC~\citep{petitjean2015rvsc}, 
ACDC~\citep{bernard2018acdc},
BraTS 2019~\citep{menze2014brats1, bakas2017brats2, baid2021brats3},
Breast Cancer DCE-MRI Data~\citep{zhang2023breast_mri}, and RIDER~\citep{Meyer2015RIDER}.
(2) \textbf{CT datasets}.
Thoracic cavity segmentation dataset~\citep{Aerts2019NSCLC},
spleen segmentation dataset~\citep{simpson2015spleen},
Pancreas-CT~\citep{roth2015pancreas},
the abdomen part of BTCV~\citep{landman2015btcv},
LiTS~\citep{bilic2023lits}, and 
KiTS 2023~\citep{heller2021kits19, heller2023kits21},
(3) \textbf{Ultrasound datasets}.
CAMUS~\citep{leclerc2019camus} (also known as echocardiography), and
Micro-Ultrasound Prostate Segmentation Dataset~\citep{jiang2024microsegnet}.
(4) \textbf{Endoscopy datasets}. 
CVC-ClinicDB~\citep{bernal2015cvc600},
CVC-ColonDB~\citep{bernal2012cvc300},
ETIS~\citep{silva2014etis}, and
ASU-Mayo~\citep{tajbakhsh2015asu}.
For all datasets, videos are converted into frames and 3D volumes are converted into 2D slices. In total, there are 3,644 sequences (125,487 images) for training and 1,061 sequences (41,078 images) for testing. 

\paragraph{Prompt Acquisition.} 
We adopt large language models to automatically generate medical text prompts. These medical text prompts are then proofread by senior radiologists.
The instruction template is as follows:
``You are a medical expert. Describe the [attribute 1], [attribute 2], ..., and [attribute $N_p$] of the \underline{\textit{anatomical structure}} on \{modality\} in one sentence each.'' 

Using this template, we obtain $N_p$ prompts for the target object (\emph{i.e.}, anatomical structure) that is expected to be segmented. Here, $N_p$ is set to 3, with [attribute 1]=[profile], [attribute 2]=[shape], and [attribute 3]=[color]. The attribute [profile] characterizes organ functions and defines lesions, while attributes [color] and [shape] describe the morphological aspects of the object. Detailed prompts can be found in supplementary materials.



\section{Experiments}


\subsection{Experimental Settings}

% Training and testing splits follow the original settings as closely as possible, and we ensure that there is no data leakage, as each sequence is used either in the training phase or testing phase.
% \jl{We'd better try to avoid saying this 'data leakage issue' in this paper. U can make some explanations of such cases (\emph{e.g.} CVC) in the supp. }

% \jl{We maintain the original training and testing splits, and make sure the sequence is only used in one split (detailed in supp.)}

% \textbf{Metrics.}
% We use two metrics for medical image sequence segmentation. Let $M$ and $Y$ be the predicted masks and ground truth masks, respectively, and let $m$ and $y$ be the corresponding contours delineating the object. The following two standard similarity measurements are computed: 
% % \jl{If we do not have spaces, these formulations could be moved to supp. Just say Dice coefficient and Hausdorff distance is enough. } runtian: ok
% % \begin{itemize}
% % \item 

% $\bigcdot$ Dice: It measures the overlap or similarity between two masks and is defined as: 
% \begin{equation}
%     \mathcal{D}(M, Y)=2 \frac{M \cap Y}{M+Y}
% \end{equation}
% % \item 
% $\bigcdot$ Hausdorff distance: It is a symmetric measure of distance between two contours and is defined as:
% % \end{itemize}
% \begin{equation}
%     \mathcal{H}(m, y) = \max \left(\max _{i \in m}\left(\min _{j \in y} d(i, j)\right), \max _{j \in y}\left(\min _{i \in m} d(i, j)\right)\right)
% \end{equation}
We train a universal model on Ref-MISS-Bench and maintain the original training and testing splits, ensuring that each sequence appears in only one split. 
% Images without a valid object are filtered out. 
Data augmentation techniques include random horizontal flipping, random resizing, random cropping, and photometric distortion. All images are resized to a maximum length of 640 pixels.
Segmentation performance is evaluated using the Dice score.
% \textbf{Model.} 
The coefficients for the loss terms are set as follows: $\lambda_{L1} = 5$, $\lambda_{giou} = 2$, $\lambda_{dice} = 5$, $ \lambda_{focal} = 2$, and $\lambda_{cls} = 2$. We adopt 4 encoder layers and 4 decoder layers in the Transformer. The initial query number $N_q$ is set to 5. Both the hidden dimension of the Transformer and the channel dimension of text prompts are $C=256$. 
% \textbf{Training.} 
During training, 3 temporal images from a sequence are randomly sampled and fed into the model at each iteration. Our model is trained on 2 RTX 3090 24GB GPUs, with AdamW optimizer and an initial learning rate of $10^{-5}$ for 5 epochs. The learning rate decays by 0.1 at the $3^{rd}$ epoch.

\begin{figure*}[h]
    \centering
    \includegraphics[width=.8\linewidth]{heatmap.pdf}
    \caption{Ablation studies on text prompts and propagation strategies. Dice scores are provided for full model, without prompt, and without propagation, respectively. 
    }
    \label{fig:ab}
\end{figure*}


\subsection{Results}
\subsubsection{Comparison to State-of-the-art in Medical Domain}
To better organize and present the datasets, we categorize the organ datasets into four anatomical groups: heart, lung, abdomen, and prostate. We then compute the average metrics for each group, allowing us to identify strengths and weaknesses across different anatomical regions. Detailed experimental results for each category are provided in supplementary materials. Table~\ref{comparison_medical} shows comparison results with UNetR~\citep{hatamizadeh2022unetr}, Swin-UNet~\citep{cao2022swinunet}, nn-UNet~\citep{isensee2021nnunet}, MedSAM~\citep{ma2024medsam}, LViT~\citep{li2023lvit}, LGMS~\citep{zhong2023ariadne}, and MMI~\citep{bui2024mmiunet}. Among them, UNetR and Swin-UNet are 3D models, while LViT, LGMS, and MMI utilize multi-modal inputs combining images with text annotations. We train and evaluate \textbf{separate models} for these task-specific methods on each anatomical structure. Experimental results demonstrate superior performance of our \textbf{universal model} over them.

\subsubsection{Comparison to State-of-the-art on RVOS} 
We compare our method with state-of-the-art approaches on referring video object segmentation, including URVOS~\citep{seo2020urvos}, ReferFormer~\citep{wu2022referformer}, OnlineRefer~\citep{wu2023onlinerefer}, MTTR~\citep{botach2022mttr}, and SOC~\citep{luo2024soc}. Comparison results for both organs and lesions are shown in Table~\ref{comparison_rvos}.
For feature extraction, we implement multiple visual backbones, including ResNet~\citep{he2016resnet}, Swin Transformer~\citep{liu2021swin}, and Video Swin Transformer~\citep{liu2022video}.
Notably, the performance for organ detection is higher than that for lesion detection. This discrepancy can be attributed to the smaller size and more homogeneous appearance of lesions, which makes them inherently more challenging to identify. 
Our approach consistently outperforms previous methods across all three backbones, especially on lesion datasets. For instance, in segmenting liver and kidney tumors, our model with a ResNet-50 backbone achieves average Dice scores of 65.27\% and 77.73\%, which are 17.84 and 15.98 points higher than the previous state-of-the-art work, ReferFormer.
Visual results of our TPP are shown in Figure~\ref{fig:visual}.



\begin{table}[t]
\caption{Comparison with SAM 2 series.}
\centering
\renewcommand\arraystretch{1.1} 
\scalebox{1}{
\begin{tabular}{ccc}
\toprule
{\textbf{\makecell[c]{Prompter + Segmenter}}} & {\textbf{Organ}} & {\textbf{Lesion}} \\ 
\cmidrule(r){1-1} \cmidrule(){2-3}
{G. DINO + SAM 2} & 12.46 & 10.10 \\
{TPP + SAM 2}  & 53.45 {\scriptsize\textcolor{red}{(+40.99)}} & 54.55 {\scriptsize\textcolor{red}{(+44.45)}} \\
% \rowcolor{rowcolorblue!80} 
Ours (TPP + TPP)  & 80.77 {\scriptsize\textcolor{red}{(+68.31)}} & 72.69 {\scriptsize\textcolor{red}{(+62.59)}} \\ 
\bottomrule
\end{tabular}}
\label{tab:sam2}
\end{table}



\begin{table}[t]
\caption{Few-shot performance.}
\centering\renewcommand\arraystretch{1.1} 
\scalebox{1}{
\begin{tabular}{cccc}
\toprule
{\textbf{Method}}  & {\textbf{\makecell[c]{Right ventricle}}} & {\textbf{\makecell[c]{Breast mass}}} & {\textbf{Polyp}} \\
\cmidrule(r){1-1} \cmidrule(){2-4}
{Full data} & 81.97 & 61.96 & 82.19 \\
{One-shot} & 75.63 {\scriptsize\textcolor{blue}{(-6.34)}} & 59.88 {\scriptsize\textcolor{blue}{(-2.08)}} &  81.55 {\scriptsize\textcolor{blue}{(-0.64)}} \\
{Zero-shot} &  71.13 {\scriptsize\textcolor{blue}{(-10.84)}}  & 57.18 {\scriptsize\textcolor{blue}{(-4.78)}} &  80.97 {\scriptsize\textcolor{blue}{(-1.22)}} \\
\bottomrule
\end{tabular}}
\label{tab:zero}
\end{table}

\subsubsection{Comparison to SAM 2} The Segment Anything Model 2~\citep{ravi2024sam2}
% \jl{cite here} 
serves as a foundational model for promptable visual segmentation in images and videos. As it currently lacks support for text prompts, we utilize a community-developed version, Grounded SAM 2~\citep{liu2023dino}, which enables video object tracking with text inputs. This model uses box outputs from Grounding DINO as prompts for SAM 2's video predictor, effectively merging SAM 2's tracking capabilities with Grounding DINO for open-set video object segmentation. 
Despite this integration, it achieves average Dice scores of only 12.46\% for organs and 10.10\% for lesions, indicating its limited understanding of medical text prompts. 

To address this, we utilize the mask predictions of the first image in the sequences generated by our TPP as mask prompts for SAM 2. 
% effectively combining SAM 2's robust tracking capabilities with our precise targeting of the referred anatomical structure. This combination 
This leads to substantial improvements, with average Dice scores increasing to  53.45\% for organs and 54.55\% for lesions. As shown in Table~\ref{tab:sam2}, our TPP demonstrates superiority over Grounding DINO in text grounding ability, and surpasses SAM 2 in object tracking capabilities due to the triple propagation strategy. 
% In contrast, our method, which employs referring interaction and transfer learning, significantly enhances segmentation performance in medical image sequences, as shown in Table~\ref{tab:sam2}. 
% \jl{we should summarise with our emphasise on our superiority on the text grounding ability (better than GroundingDINO) and object tracking ability (better than SAM-2) here, (possibly due to the triple propagation mechanism, i am not sure)}

\subsubsection{Zero-/One-shot Performance} 
%Our approach demonstrates strong zero-shot performance on unseen datasets. 
To validate the zero-shot performance of our approach on unseen datasets, we exclude RVSC (right ventricle), RIDER (breast mass), and CVC-ColonDB (polyp) from the training datasets and evaluate the trained model on these datasets directly. 
% \jl{It is not very clear to me here, you move RSVC, RIDER, CVC out of the training set, is the evaluation conducted on these three datasets?}
As shown in Table~\ref{tab:zero}, the Dice scores for breast mass and polyp decrease by only 4.78 and 1.22 points, respectively, compared to full-data training.
% \jl{We better add one more sentence to explain what one-shot means here}
In the one-shot setting, we add a single sequence from each of the three datasets mentioned above into the training set. The results show that one-shot performance on polyp is comparable to full-data training, highlighting the model's robust generalization ability.


\begin{figure*}
    \centering
    \includegraphics[width=1\linewidth]{quali.pdf}
    \caption{
    %Visualization results. 
    Visualization of segmentation results for different structures and modalities.
    (a) and (b) display the results of left atrium and myocardium in the same MRIs, respectively. (c) and (d) show spleen and liver in the same CT slices, respectively. From (e) to (h), %visualizations are \underline{brain tumor} in \{MRI\}, \underline{liver tumor} in \{CT\}, \underline{polyp} in \{endoscopy\}, and \underline{prostate} in \{ultrasound\}.
    visualizations are: brain tumor in MRI, liver tumor in CT, polyp in endoscopy, and prostate in ultrasound.
    }
    \label{fig:visual}
\end{figure*}

\subsection{Ablation studies}


Cross-modal referring interaction and the propagation strategy are critical components of our approach to referring medical image sequence segmentation. Figure~\ref{fig:ab} illustrates that medical text prompts are particularly essential for accurately identifying organs located in the heart, lungs, and abdomen. Moreover, for extremely small lesions, such as breast masses and liver tumors, our propagation strategy significantly reduces the occurrence of false negatives, resulting in substantial enhancements.

\paragraph{Medical Text Prompts.} We utilize large language models to generate three attributes for each anatomical structure: [profile], [color], and [shape]. Among these, [profile] is a more abstract concept, whereas [color] and [shape] are more specific. These different attributes serve as varied prompt messages, resulting in distinct enhancements in segmentation performance, as shown in Figure~\ref{fig:line}.

We also conduct experiments with different prompt variations to evaluate their impact on segmentation performance.
For instance, simplified prompts with only class names result in Dice scores of 75.65\% for organs (-5.12\%) and 67.61\% for lesions (-5.08\%) compared to the full model. Examples of such simplified prompts include: ``an MRI of the myocardium'', ``a CT of the liver tumor'', ``an ultrasound image of the prostate''. The results demonstrate that detailed, descriptive prompts significantly enhance segmentation performance when compared to simplified ones.

% \begin{table}[h]
% \caption{Ablation studies on prompts.}
% \centering
% \renewcommand\arraystretch{1.1}
% \scalebox{1}{
% \begin{tabular}{lcc}
% \hline
% \textbf{Prompt} & {Organ} & {Lesion} \\ 
% \hline
% w/o prompt & 41.45  & 63.69 \\
% w/ [profile] & 76.17  & 66.07  \\
% w/ [color]\&[shape] & {78.31}  & {67.50} \\
% Full model & \textbf{80.77} & \textbf{72.69} \\
% \hline
% \end{tabular}}
% \label{tab:prompt}
% \end{table}
\begin{figure}[]
    \centering
    \includegraphics[width=1\linewidth]{line.pdf}
    \caption{Ablation studies on different versions of medical text prompts.}
    \label{fig:line}
\end{figure}

\paragraph{Propagation Strategy.} To investigate the effects of box propagation, mask propagation, and query propagation, we conduct ablation experiments by removing the corresponding propagation methods, as demonstrated in Table~\ref{tab:propagation_main}. The absence of mask and query propagation results in decreases of 2.84 and 2.91 points in Dice score for organs. The results indicate that box propagation yields the smallest enhancements, with increases of 1.16 points for organs and 2.80 points for lesions in Dice scores. In contrast, mask and query propagation demonstrate a more significant impact, highlighting their critical roles in improving overall segmentation performance. This underscores the importance of designing appropriate propagation methods to optimize results in medical image sequence segmentation.
% More details on ablation studies for propagation can be found in Appendix~\ref{A.propagation}.

Table~\ref{tab:propagation2} analyzes the impact of different query selection strategies. The first row represents the case where no selection is performed. In the second row, the model selects the top-3 queries for Slice 2, and then the top-1 query for Slice 3. However, neither strategy outperforms the final configuration, indicating the effectiveness of retaining a single query across both Slice 2 and Slice 3. 


\begin{table}[h]
\caption{Ablation studies on propagation.}
\centering
\renewcommand\arraystretch{1}
\setlength{\tabcolsep}{3pt}
\scalebox{1}{
\begin{tabular}{ccccc}
\toprule
\multirow{2}{*}{\makecell[c]{\textbf{Box} \\\textbf{propagation}}} & \multirow{2}{*}{\makecell[c]{\textbf{Mask} \\\textbf{propagation}}} & \multirow{2}{*}{\makecell[c]{\textbf{Query} \\\textbf{propagation}}} & \multirow{2}{*}{\textbf{Organ}} & \multirow{2}{*}{\textbf{Lesion}}   \\
& & & &  \\
\cmidrule(r){1-3} \cmidrule(){4-5}
\ding{55} & \ding{55} & \ding{55} & 74.53  & 63.97 \\
% \ding{51} & \ding{55} & \ding{55} & 76.69 & 66.77 \\
% \ding{55} & \ding{51} & \ding{55} & 77.10  & 70.03  \\
% \ding{55} & \ding{55} & \ding{51} & 77.28  & 69.37  \\
\ding{51} & \ding{51} & \ding{55} & 77.86  & 64.03 \\
\ding{51} & \ding{55} & \ding{51} & 77.93  & 67.10  \\
\ding{55} & \ding{51} & \ding{51} & 79.57  & 71.43  \\
  \ding{51} & \ding{51} & \ding{51} & \textbf{80.77} & \textbf{72.69} \\
\bottomrule
\end{tabular}}
\label{tab:propagation_main}
\end{table}


\begin{table}[h]
\caption{Analysis on query selection.}
\centering
\renewcommand\arraystretch{1.1}
\scalebox{1}{
\begin{tabular}{ccccc}
\toprule
\multicolumn{3}{c}{\textbf{Number of queries for}} & \multirow{2}{*}{\textbf{Organ}} & \multirow{2}{*}{\textbf{Lesion}} \\
\textbf{Slice 1}   & \textbf{Slice 2}  & \textbf{Slice 3}  &      &    \\
\cmidrule(r){1-3} \cmidrule(){4-5}
5 & 5 & 5  & 79.47   &  70.98      \\
5 & 3 & 1  & 78.47  &  71.67  \\
 5 & 1 & 1 & \textbf{80.77}   & \textbf{72.69}    \\
\bottomrule
\end{tabular}
}
\label{tab:propagation2}
\end{table}



\section{Conclusion}
In this paper, we introduce a new task, termed Referring Medical Image Sequence Segmentation, accompanied by a large and comprehensive benchmark. The benchmark includes 20 different anatomical structures across 4 modalities from various regions of the body. We present an innovative text-promptable approach that effectively leverages the inherent sequential relationships and textual cues within medical image sequences to segment referred objects, serving as a strong baseline for this task. By integrating both 2D and 3D medical images through a triple-propagation strategy, we demonstrate significant improvements across a broad spectrum of medical datasets, emphasizing the potential for rapid response in segmenting referred objects and enabling accurate diagnosis in clinical practice. Future work should delve deeper into optimizing prompts and exploring additional modalities to further enhance the efficacy of medical image analysis.

%%
%% The acknowledgments section is defined using the "acks" environment
%% (and NOT an unnumbered section). This ensures the proper
%% identification of the section in the article metadata, and the
%% consistent spelling of the heading.
% \begin{acks}
% To Robert, for the bagels and explaining CMYK and color spaces.
% \end{acks}

%%
%% The next two lines define the bibliography style to be used, and
%% the bibliography file.
\bibliographystyle{ACM-Reference-Format}
\bibliography{sample-base}

\end{document}
\endinput
%%
%% End of file `sample-sigconf-authordraft.tex'.
