\section{Discussion}

Discussion
\subsection{Robust Selection of $\rho$}
Implementation of this planning algorithm for a particular robot swarm requires
tuning of the parameter $\rho$, the minimum allowable distance between agents.
The trajectory planner is constrained such that pairs of trajectories must be separated
by at least $\rho$.
The local controller must then be able to keep the agent within a tube of radius $\rho$ 
of the commanded trajectory.
$\rho$ should be selected by considering the performance of the tracking controller
- a controller that tracks trajectories tightly implies that agents can safely fly closer
together. 

We propose a method which quantifies the tracking performance for an agent with
stochastic dynamics, based on recorded trajectory data. From this data, one 
can construct a confidence bound on the probability of collision, resulting in a safe
choice of $\rho$.

The constraint of leaving the vicinity of the trajectory can be written as follows:
\begin{align}
    ||x(t), t|| < \rho
\end{align}

When the agent moves under stochastic dynamics, violations of this constraint can be represented
as a random variable $X$, whose value is 1 if the trajectory leaves the tube and 0 if it stays within.
Therefore,
\begin{align}
    P(||x(t), t|| \geq \rho) = \mathbb{E}[X]
\end{align}

From repeated samples of $X$, we wish to find a bound $\mathbb{E}[X]$ with an appropriate confidence level. 
To do this, we leverage Hoeffding's Inequality, which relates the difference between the sample mean and true 
expectation of a bounded random variable [CITATION].

\begin{align}
    P(|\bar{X} - \mathbb{E}[X]| > \gamma) \leq 2e^{-2N\gamma^{2}/(b-a)}
\end{align}

Where $\gamma$ is the difference between the sample and true means, $N$ is the number of samples, and $a, b$ 
are the lower and upper bounds of the random variable $X$. In our case, $X$ can take values of 0 or 1. Letting 
$\delta$ be the left hand side of the above inequality, we can rearrange to result in the following:

\begin{align}
    \mathbb{E}[X] \leq \bar{X} + \sqrt{\frac{-\ln(\delta /2)}{2N}}
\end{align}

$\delta$ can be interpreted to be the confidence in the bound. Given a desirable $\delta$, $\rho$ can
be selected such that the right hand side of the above inequality is sufficiently low, thus bounding the 
overall probability of constraint violation.

\subsection{Comparison of Various Maneuvers}

\subsection{Swarm Energy Density}
In order to characterize the performance of swarming robotic systems in complex maneuvers, we introduce a metric
which we call "Swarm Energy Density."
This metric can be used to score a swarming robotic system in terms of its speed, number of agents, and distance
between agents. 
The metric is given as follows:

\begin{align}
    \frac{\sum_i mv^{2}}{2V}
\end{align}

where $m$ is the mass of each agent, $v$ is the velocity of the agent, and $V$ is the volume of the smallest 
convex hull which fully envelops the swarm.