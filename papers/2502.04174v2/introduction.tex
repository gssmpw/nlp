\section{Introduction}

Swarms of uncrewed aerial vehicles (UAVs) are able complete parallelizable tasks with greater speed and resilience than would be possible with a single robot. In particular, fixed-wing UAV swarms have the unique ability to leverage large lifting surfaces to achieve efficient flight and long range operations. However, to date, most UAV swarms have been comprised of multi-rotors, especially in cases where the UAVs must operate in close proximity to one another or other obstacles in the environment. By contrast, fixed-wing UAV swarms have often maintained large stand-off distances to avoid collisions. This is primarily due to the limited agility of fixed-wing UAVs. However, this limitation is not always due to the physical capabilities of the fixed-wing aircraft, but rather due to the inability of the underlying control algorithm to reason about the full flight envelope. In many instances, it would be advantageous to preserve both the long range capabilities of fixed-wing UAVs and the maneuverability of quadrotor UAVs.

In this paper, we present an approach for controlling multiple fixed-wing UAVs in close proximity to one another using receding-horizon nonlinear model predictive control (NMPC) so as to achieve both multi-rotor agility and fixed-wing range. Our approach relies on a direct transcription formulation of the trajectory optimization problem and is therefore able to encode both cost functions and constraints on the state variables. To prevent vehicle-vehicle collisions, we share trajectories among vehicles and leverage trajectory obstacle constraints. To ensure that vehicles stay close to their nominal trajectories, we employ a local time-varying feedback controller that is updated in real time, and characterize the performance of this feedback controller via statistical analysis to provide stochastic performance guarantees. 

We show that our approach is able to make use of a large portion of the flight envelope to enable close-quarters multi-vehicle operations. We demonstrate our approach in simulation with up to X agents, and demonstrate multiple multi-vehicle patterns in hardware, with up to four fixed-wing UAVs flying in close proximity. We also compare our results to other swarms in the literature in terms of a ``swarm energy density'' metric. To our knowledge, this is the first time dense fixed-wing swarming has been demonstrated in hardware. 


% \begin{figure}[]
%     \centering
%     \includegraphics[width=\columnwidth]{example-image-a}
%     \caption{Main image}
%     \label{fig:main_img}
% \end{figure}
