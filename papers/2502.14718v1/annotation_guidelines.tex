We prepare these general set of guidelines to prepare the annotators and avoid human biases before starting the annotation:

\begin{itemize}
    \item The annotators should get acquainted with the two domains covered by the tasks; for instance, ~\cite{kremlin-propaganda} and~\cite{cc-denial-tax} provide a good coverage of the URW and CC domains,
    \item The annotators' opinions on the topics and sympathies towards key entities mentioned in the articles are irrelevant and should by no means impact the annotation process and their choices, 
    \item The annotators should not exploit any specific external knowledge bases for the purpose of annotating documents.
\end{itemize}

The annotation guidelines we prepare to annotate and curate the entity framing corpus are as follows:

Any references to ``URW'' and ``CC'' below denote the Ukraine-Russia War, and the Climate Change domains, respectively.

\begin{enumerate}

    \item In this annotation task, the entities of interest are understood in a broad sense to include both traditional named entities (such as persons, organizations, and locations) as well as toponym-derived entities. Toponym-derived entities are phrases that indicate a group or collective identity based on a place or affiliation, including but not limited to:
    \begin{itemize}
        \item Political, military, or social groups defined by their association with a location or entity, e.g., ``Trump supporters,'' or ``residents of Ukraine.''
        \item Entities denoting a geographic or organizational affiliation, such as ``Russian forces'' or ``European officials.''
    \end{itemize}
        

    \item The annotators are provided with a number of news articles and are expected to assign role(s) to named entities that are \textbf{central} to the article's story according to the taxonomy of roles that was provided earlier. 
    \item The annotators are provided with a detailed taxonomy that includes definitions and examples. 
    \item The title of an article should not be annotated. The title of the article is the first block of text that appears in the annotation platform Inception.
    \item Only named entities that are central to the narrative of the article should be annotated. Unnamed entities (i.e., nominal entity mentions such as ``migrants'') should not be annotated. 

For more details on what qualifies as a named entity, in addition to the definition of the broader sense of named entities in the first bullet point in these guidelines, the annotators should also examine the NER annotation guidelines outlined in \url{http://www.universalner.org/guidelines/}.
    
    \item The annotators will pick one or more fine-grained roles for the named entities they believe are central to the article's story. 
    \item Entity mentions can be assigned fine-grained roles from more than one main role. However, during curation, we will not be including these instances in the current version, even though we still annotate them.
    \item Named entities that are not central to the story should not be annotated.

The determination of how central a named entity is in an article is admittedly subjective. To reduce bias, such determination should be based on the careful reading of the article and the story it is pushing. An annotated example is provided in \ref{fig:annotated_example_text2}. Notice that named entities such as New York Times and Israel were not annotated because they are not central to the story.
    
    \item As a general rule, annotators should annotate only the first mention per entity where it is clear that this entity has the specific role(s). There is no need to annotate subsequent mentions of this entity with the same role, But  annotating more mentions with the same surface form and role is not a mistake, but it is simply not required.

    This rule also extends for surface mentions of the same entity. For example, ``Putin'' and ``Vladimir Putin'' are both surface mentions of the same entity, so only the first occurrence of any of the surface forms  would be annotated.

    On the other hand, while entities such as ``Moscow'', ``Russia'', and ``Putin'' are closely related, they are not surface forms of the same entity, and are considered as distinct, separate entities.

    \item Named entities appearing in indirect speech (e.g., quotes) should not be annotated. Indirect speech should be considered as supporting detail to the story, but named entities which are central to the narrative would likely appear also outside of quotations. This guideline helps avoid confusions inherent in indirect speech.

    \item If the above would result in more than one mention of the same entity with the same role, the curator does not need to remove all these additional mentions. We keep all of them.
    \item Should an entity mention that was previously annotated with a certain role appear in a different context with different role(s), the first mention where the role(s) changed should be annotated.

The above rule is repeated for as many times as an entity changes roles across mentions. For example, if an en entity, let's say NATO, appears 20 times in an article. The first 10 mentions show NATO as a Guardian and a Virtuous entity. The 11-15th mentions portray NATO as a Foreign Adversary, and the 16-20th mentions portray NATO as Exploited. Then we only need 3 annotations in total to account for the 3 different roles NATO was portrayed as. These 3 annotations should all be the first mention occurrences where NATO assumed each distinct set of roles (i.e., mention 1, mention 11, and mention 16 should be annotated).
    \item Regarding scenarios where different surface forms for the same named entity (e.g., NATO vs North Atlantic Treaty Organization) appear in the article, it is sufficient to pick only one of the surface forms.
    \item If the above results in multiple surface forms of the same entity being annotated, the curator does not need to remove all of these additional mentions. We keep all of them.
    \item If the mention of the entity does not have any role vis-a-vis the taxonomy of roles then no role should be given. As a consequence, we do not need an ``Other'' label. 
    % \item It is important that only information found in the article is used---the annotator must not rely on external knowledge, to avoid bias and subjectivity.
    \item The curator may see conflicting annotations in the curation mode and could resolve the conflict, and then the remaining non-conflicting roles could be checked and adopted accordingly. 


\end{enumerate}

