% This must be in the first 5 lines to tell arXiv to use pdfLaTeX, which is strongly recommended.
\pdfoutput=1
% In particular, the hyperref package requires pdfLaTeX in order to break URLs across lines.


\documentclass[11pt]{article}

\usepackage{pgfplotstable}
\usepackage{multirow}

% Change "review" to "final" to generate the final (sometimes called camera-ready) version.
% Change to "preprint" to generate a non-anonymous version with page numbers.
\usepackage[final]{acl}

% Standard package includes
\usepackage{times}
\usepackage{latexsym}
\usepackage{booktabs}

% For proper rendering and hyphenation of words containing Latin characters (including in bib files)
\usepackage[T1]{fontenc}
% For Vietnamese characters
% \usepackage[T5]{fontenc}
% See https://www.latex-project.org/help/documentation/encguide.pdf for other character sets

% This assumes your files are encoded as UTF8
\usepackage[utf8]{inputenc}

% This is not strictly necessary, and may be commented out,
% but it will improve the layout of the manuscript,
% and will typically save some space.
\usepackage{microtype}

% This is also not strictly necessary, and may be commented out.
% However, it will improve the aesthetics of text in
% the typewriter font.
\usepackage{inconsolata}

%Including images in your LaTeX document requires adding
%additional package(s)
\usepackage{graphicx}
\usepackage{subcaption}
\usepackage{array}

\usepackage[utf8]{inputenc}
\usepackage{tcolorbox}
\usepackage{parskip} % Adds space between paragraphs
% \usepackage{newtxtext}    % Times New Roman-like font
\usepackage{graphicx}
\usepackage{pgf}

% Define the tcolorbox style
\tcbset{
  colframe=black,
  colback=white,
  fonttitle=\bfseries,
  sharp corners,
  boxrule=0.5pt,
  width=\columnwidth,
  left=4pt,
  right=4pt,
  top=4pt,
  bottom=4pt
}

\usepackage{svg}        % For including SVG files
\usepackage{caption}    % For customizing captions
\usepackage{subcaption}    % For customizing captions
\usepackage{tikz}
\usetikzlibrary{shapes.callouts}
\usetikzlibrary{positioning}
\usepackage{fancyvrb}      
\usepackage{framed}        
\usepackage{xcolor}        
\usepackage{graphicx}      
\usepackage{caption}       
\usepackage{amsmath}  
% \usepackage{newtxtext,newtxmath}
% Command to create labeled entity mentions
% \newcommand{\labeledentity}[3]{%
%   \tikz[baseline=(word.base)]{
%     % Label above the entity mention
%     \node[anchor=south, inner sep=2pt, fill=#3, rounded corners=2pt, text=white] (label) {\tiny #1};
%     % Entity mention with a box
%     \node[below=1pt of label, draw, thick, rounded corners=2pt, inner sep=2pt] (word) {#2};
%   }%
% }
% Define custom colors
\definecolor{lightbrown}{rgb}{0.71, 0.40, 0.11}
\definecolor{darkblue}{rgb}{0.00, 0.20, 0.60}
\definecolor{lightblue}{rgb}{0.20, 0.60, 0.86}

% Define text commands for convenience
\newcommand{\lightbrowntext}[1]{\textcolor{lightbrown}{#1}}
\newcommand{\darkbluetext}[1]{\textcolor{darkblue}{#1}}
\newcommand{\lightbluetext}[1]{\textcolor{lightblue}{#1}}
\definecolor{darkgreen}{rgb}{0, 0.5, 0}
\definecolor{darkred}{rgb}{0.7, 0, 0.1}
\definecolor{darkblue}{rgb}{0, 0, 0.5}

% Labeled entity command
\newcommand{\labeledentity}[3]{%
  \tikz[baseline=(word.base)]{
    \node[anchor=south, inner sep=2pt, fill=#3, rounded corners=2pt, text=white, font=\scriptsize] (label) {\tiny #1};
    \node[below=1pt of label, draw, thick, rounded corners=2pt, inner sep=2pt, font=\small] (word) {#2};
  }%
}
% \usepackage{tcolorbox}
% \usepackage{lipsum} % For placeholder text

% % Define the box style
% \tcbset{
%   colframe=black,
%   colback=white,
%   fonttitle=\bfseries\small,
%   sharp corners,
%   boxrule=0.5pt,
%   width=\textwidth,
%   left=2pt,
%   right=2pt,
%   top=2pt,
%   bottom=2pt
% }
% If the title and author information does not fit in the area allocated, uncomment the following
%
%\setlength\titlebox{<dim>}
%
% and set <dim> to something 5cm or larger.

\title{Entity Framing and Role Portrayal in the News}

% Author information can be set in various styles:
% For several authors from the same institution:
% \author{Author 1 \and ... \and Author n \\
%         Address line \\ ... \\ Address line}
% if the names do not fit well on one line use
%         Author 1 \\ {\bf Author 2} \\ ... \\ {\bf Author n} \\
% For authors from different institutions:
% \author{Author 1 \\ Address line \\  ... \\ Address line
%         \And  ... \And
%         Author n \\ Address line \\ ... \\ Address line}
% To start a separate ``row'' of authors use \AND, as in
% \author{Author 1 \\ Address line \\  ... \\ Address line
%         \AND
%         Author 2 \\ Address line \\ ... \\ Address line \And
%         Author 3 \\ Address line \\ ... \\ Address line}

\author{Tarek Mahmoud$^1$, 
  \textbf{Zhuohan Xie}$^1$,
  \textbf{Dimitar Dimitrov}$^2$,
  \textbf{Nikolaos Nikolaidis}$^3$, 
  \textbf{Purificação Silvano}$^4$, \\
  \textbf{Roman Yangarber}$^5$,
  \textbf{Shivam Sharma}$^6$,
  \textbf{Elisa Sartori}$^{7}$,
  \textbf{Nicolas Stefanovitch}$^{8}$, \\
  \textbf{Giovanni Da San Martino}$^{7}$
  \textbf{Jakub Piskorski}$^9$,
  \textbf{Preslav Nakov}$^1$,
  \\
  %{\small\tt~\Letter~\hspace{-0.12cm}jpiskorski@gmail.com}\\
	$^1$MBZUAI, 
    $^2$Sofia University "St. Kliment Ohridski", 
    $^3$Athens University of Economics and Business, \\
 $^4$University of Porto, 
    $^5$University of Helsinki,
    $^6$Indian Institute of Technology Delhi,
    $^{7}$University of Padova, \\
    $^{8}$European Commission Joint Research Centre,
    $^9$Institute of Computer Science, Polish Academy of Science \\
\href{tarek.mahmoud@mbzuai.ac.ae}{\{tarek.mahmoud, preslav.nakov\}@mbzuai.ac.ae}
%{\texttt dasan@math.unipd.it, jpiskorski@gmail.com, nicolas.stefanovitch@ec.europa.eu, preslav.nakov@mbzuai.ac.ae}
}

%\author{
%  \textbf{First Author\textsuperscript{1}},
%  \textbf{Second Author\textsuperscript{1,2}},
%  \textbf{Third T. Author\textsuperscript{1}},
%  \textbf{Fourth Author\textsuperscript{1}},
%\\
%  \textbf{Fifth Author\textsuperscript{1,2}},
%  \textbf{Sixth Author\textsuperscript{1}},
%  \textbf{Seventh Author\textsuperscript{1}},
%  \textbf{Eighth Author \textsuperscript{1,2,3,4}},
%\\
%  \textbf{Ninth Author\textsuperscript{1}},
%  \textbf{Tenth Author\textsuperscript{1}},
%  \textbf{Eleventh E. Author\textsuperscript{1,2,3,4,5}},
%  \textbf{Twelfth Author\textsuperscript{1}},
%\\
%  \textbf{Thirteenth Author\textsuperscript{3}},
%  \textbf{Fourteenth F. Author\textsuperscript{2,4}},
%  \textbf{Fifteenth Author\textsuperscript{1}},
%  \textbf{Sixteenth Author\textsuperscript{1}},
%\\
%  \textbf{Seventeenth S. Author\textsuperscript{4,5}},
%  \textbf{Eighteenth Author\textsuperscript{3,4}},
%  \textbf{Nineteenth N. Author\textsuperscript{2,5}},
%  \textbf{Twentieth Author\textsuperscript{1}}
%\\
%\\
%  \textsuperscript{1}Affiliation 1,
%  \textsuperscript{2}Affiliation 2,
%  \textsuperscript{3}Affiliation 3,
%  \textsuperscript{4}Affiliation 4,
%  \textsuperscript{5}Affiliation 5
%\\
%  \small{
%    \textbf{Correspondence:} \href{mailto:email@domain}{email@domain}
%  }
%}
\pgfplotsset{compat=1.18} 
\begin{document}
\maketitle
\begin{abstract}
We introduce a novel multilingual hierarchical corpus annotated for entity framing and role portrayal in news articles. The dataset uses a unique taxonomy inspired by storytelling elements, comprising 22 fine-grained roles, or archetypes, nested within three main categories: \emph{protagonist}, \emph{antagonist}, and \emph{innocent}. Each archetype is carefully defined, capturing nuanced portrayals of entities such as guardian, martyr, and underdog for protagonists; tyrant, deceiver, and bigot for antagonists; and victim, scapegoat, and exploited for innocents. The dataset includes 1,378 recent news articles in five languages (Bulgarian, English, Hindi, European Portuguese, and Russian) focusing on two critical domains of global significance: the Ukraine-Russia War and Climate Change. Over 5,800 entity mentions have been annotated with role labels. This dataset serves as a valuable resource for research into role portrayal and has broader implications for news analysis. We describe the characteristics of the dataset and the annotation process, and we report evaluation results on fine-tuned state-of-the-art multilingual transformers and hierarchical zero-shot learning using LLMs at the level of a document, a paragraph, and a sentence. 

% Compared to naive single-step prompting, hierarchical prompting reduces costs by a staggering $40.15\%$. It also marginally improves accuracy for main role predictions by $0.6\%$, though it results in a $3.38\%$ drop in F1 for fine-grained role predictions across all languages.
\end{abstract}

\begin{figure*}[!htbp]
\centering
\footnotesize
\fbox{%
     \parbox{1\textwidth}{%\
    % \fontsize{6}\selectfont
     {\centering \textbf{Putin says what Russia needs to do to win special operation in Ukraine } \\[1em]}

    
    Russia will win the special operation in Ukraine if the society shows consolidation and composure to the enemy, President Vladimir Putin said during a visit to the Ulan-Ude Aviation Plant on March 14, Rossiya 24 TV channel said.
    
    Russia is not improving its geopolitical position in Ukraine. Instead, \labeledentity{Underdog}{Russia}{darkblue} is fighting "for the survival of Russian statehood, for the future development of the country and our children."
    
    "In order to bring peace and stability closer, we, of course, need to show the consolidation and composure of our society. When the enemy sees that our society is strong, internally braced up, consolidated, then, without any doubt we will come to reach what we are striving for — both success and victory," Putin said.
    
    According to him, many of the current problems began after the collapse of the Soviet Union, when they tried to put pressure on \labeledentity{Victim}{Russia}{darkgreen} to "destabilise the internal political situation.” "Hordes of international terrorists" new sent to the purpose to accomplish this goal, Putin said.
    
    Afterwards, the West decided to start rehabilitating Nazism in Russia's neighbouring states, including in Ukraine.
    
    Nevertheless, Putin continued, Russia had long tried to build partnerships with both Western countries and Ukraine. However, after 2014, when the West contributed to the coup in Ukraine, the state of affairs changed dramatically. It was then when they started exterminating those who advocated the development of normal relations with Russia, he said.
    
    According to Putin, \labeledentity{Guardian}{Russia}{darkblue} was forced to launch the special operation to protect the population. \labeledentity{Saboteur}{Western countries}{darkred} were hoping to break Russia quickly, but they were wrong, he said adding that \labeledentity{Virtuous}{Russia}{darkblue} managed to raise its economic sovereignty since 2022.
    
    }%
}
\caption{Annotated example color-coded according to the main roles in the taxonomy: \textcolor{darkred}{red} for \emph{antagonist}, \textcolor{darkblue}{blue} for \emph{protagonist}, and \textcolor{darkgreen}{green} for \emph{innocent}.}
\label{fig:annotated_example_text2}
\end{figure*}


\section{Introduction\label{sec:intro}}
The rapid proliferation of social media has dramatically transformed the information landscape, providing immediate access to news, and allowing anyone to propagate their narratives across the globe. While this connectivity functions as a convenient avenue for information dissemination, it also heightens the risk of exposure to biased reporting, propaganda, and narrative manipulation. These risks are particularly pronounced during periods of conflict and political upheavals, where the framing of entities—individuals, organizations, or groups—can profoundly influence public perception and decision-making. Understanding how entities are portrayed in the news is essential for fostering media literacy, identifying bias, and ensuring transparent news consumption.

Social science highlights the role of emotion in framing—selecting elements that evoke affective responses to shape perceptions \cite{https://doi.org/10.1111/j.1467-9221.2004.00354.x}. Emotional framing often leverages language that elicits specific feelings, such as fear, anger, or compassion, to influence how entities and events are understood \cite{iyengar_is_1991, nabi_exploring_2003,https://doi.org/10.1111/j.1460-2466.2000.tb02843.x,brader_campaigning_2006}. For instance, referring to a group as ``freedom fighters'' versus ``terrorists'' not only frames their role but also activates distinct emotional reactions. Research has shown that emotional appeals are powerful tools for reinforcing or challenging public attitudes \cite{westen_political_2008,lerner_emotion_2015}. Such framing can manifest through specific linguistic cues and includes the portrayal of entities also defined by \citet{schneider_social_1993} as the \emph{Social Construction of Target Populations}.

Natural language processing research has increasingly been applied to analyze the emotional dimensions of framing \cite{troiano-etal-2023-relationship}, including the identification of sentiment \cite{zhang-etal-2024-sentiment,app13074550} and emotion-laden narratives \cite{mousavi-etal-2022-emotion}. Understanding these emotional components provides deeper insight into how media narratives construct and perpetuate particular representations of entities, ultimately shaping public perception and societal discourse.

Given the large scale and complexity of the modern news ecosystem, effective analysis of entity framing requires automated tools, which depend on high-quality annotated data. In this context, we introduce a new multilingual dataset designed to develop tools for the study of entity framing and role portrayal in news articles. Our dataset uses a unique, hierarchical taxonomy inspired by elements of storytelling containing a set of 22 carefully defined archetypes nested under three main roles: \emph{protagonist}, \emph{antagonist}, and \emph{innocent}.

% \jp{Our dataset is not meant for studying news articles directly but for training models to facilitate carrying out such studies on representative corpora. Our corpus might not be represenative to draw any conclusions. I suggest to reword accordingly. The non-representiveness of our corpus should be mentioned in Limitations} 
% Agreed. Done

The corpus spans 1,378 recent news articles in five languages (Bulgarian, English, Hindi, European Portuguese, and Russian) and focuses on two globally significant domains: the Ukraine-Russia war and climate change. We annotated over 5,800 entity mentions with detailed role labels.


Our contributions can be summarized as follows:

\begin{itemize}
    \item We release a novel multilingual dataset annotated for entity framing and role portrayal, complete with detailed annotation guidelines.
    \item We introduce a comprehensive hierarchical taxonomy for entity roles validated on a large set of documents, supporting analysis at both the coarse and the fine-grained levels.
    \item We provide comprehensive dataset statistics, exploring entity portrayals across languages and topics.
    \item We set benchmarks using state-of-the-art multilingual transformer models, and hierarchical zero-shot learning with LLMs.
\end{itemize}



\section{Related Work\label{sec:related_work}}

\citet{sharma-etal-2023-characterizing} introduced a dataset for identifying heroes, villains, and victims in memes, focusing on visual content. In contrast, our dataset focuses on textual content. While both share similar coarse-level roles, our work adds an additional layer of granularity with a hierarchical taxonomy of 22 archetypes nested within these roles. 

\citet{card-etal-2016-analyzing} addressed a different aspect of framing. Their contribution is developing a model that makes use of personas to infer the article framing as defined in the Media Frames Corpus (MFC)  \cite{card-etal-2015-media}. MFC focuses on identifying how an article is framed along nine dimensions, such as Economic or Political. In their work, topic modeling is used to identify 50 personas. However, the results are noisy, with only a few informative personas, namely \emph{refugee} and \emph{immigrant}, while the 48 other personas, such as Job, Worker, and Year, are less informative. In contrast, our work focuses on \emph{entity} framing rather than article framing. While their work identifies personas in a weakly supervised manner, we develop a hierarchical taxonomy containing a richer set of roles, validated through human annotation of news articles across two diverse domains, ensuring higher quality and broader applicability. There is more research on news framing that focuses on article-level framing \cite{Pastorino2024DecodingNN, DBLP:conf/acl/0001KF24, piskorski-etal-2023-multilingual, liu-etal-2019-detecting, card-etal-2015-media}. In contrast, our work centers on entity-level framing.

Aspect-Based Sentiment Analysis \cite{chebolu-etal-2024-oats, DBLP:journals/corr/abs-2203-01054, orbach-etal-2021-yaso, jiang-etal-2019-challenge, saeidi-etal-2016-sentihood} is also related. It involves identifying targets of specific opinions and determining the polarity of the sentiment associated with particular aspects of these targets. Typically, the polarity is binary, and multiple aspects of the target entity are examined. Our work on entity framing is different as we do not define aspects nor do we assign polarities. Instead, we introduce a hierarchical taxonomy for news, which contains a rich set of roles inspired by elements of storytelling, and entities can be classified into any subset of roles within that taxonomy.




\section{The Entity Framing Task}
Entity framing focuses on analyzing how a text portrays a specific entity through word choice and narrative structure. More concretely, given a news article and a list of entity mentions (i.e., entity mentions, along with their span offsets), we assign to each of them one or more roles based on the taxonomy shown in Figure \ref{fig:taxonomy_roles}. We developed this taxonomy specifically for this task and the roles were inspired by storytelling elements. The taxonomy includes 22 archetypes, or fine-grained roles nested under three main roles: \emph{protagonist}, \emph{antagonist}, and \emph{innocent}. The role an entity plays in a given article may differ from one context in that article to another depending on the portrayal. See Figure~\ref{fig:annotated_example_text2} for a complete, annotated example from the corpus.

% \textbf{Protagonist} is an entity portrayed in a favorable or sympathetic light, often seen as standing up for a noble cause, protecting others, or striving for justice. 

% \textbf{Antagonist} is an entity framed in an unfavorable or adversarial manner, often depicted as contributing to conflict, harm, or injustice. Antagonists are commonly portrayed as oppressive, deceitful, corrupt, or threatening.


% \textbf{Innocent} is an entity presented as neutral, victimized, or undeserving of harm or blame. Innocents are often characterized as vulnerable, exploited, or caught in a conflict through no fault of their own.


Entity framing can be formalized mathematically as follows. Let $R$ be a tree structure representing the taxonomy of roles. Let $S$ be a string of length $|S|$ characters with the content of the full article. The goal of entity framing is to learn a function
\begin{equation}
f: (S, [i,j]) \rightarrow \{r_1, r_2, \ldots, r_k\} \subseteq R
\end{equation}

\noindent where $0 \leq i < j \leq |S|$ and $\{r_1, r_2, \ldots, r_k\}$ is the set of roles assigned to the span $[i,j]$.





% \section{Taxonomy}

% As illustrated by Fig. \ref{}, the typical process of vision models based time series analysis has five components: (1) normalization/scaling; (2) time series to image transformation; (3) image modeling; (4) image to time series recovery; and (5) task processing. In the rest of this paper, we will discuss the typical methods for each of these components. The detailed taxonomy of the methods are summarized in Table \ref{tab.taxonomy}.

%Typical step: normalization/scaling, transformation, vision modeling, task-specific head, inverse transformation (for tasks that output time series, e.g., forecasting, generation, imputation, anomaly detection). Normalization is to fit the arbitrary range of time series values to RGB representation.

\begin{figure*}[!t]
\centering
\includegraphics[width=1.0\textwidth]{fig/fig_3.pdf}
% \vspace{-1em}
\caption{An illustration of different methods for imaging time series with a sample (length=336) from the \textit{Electricity} benchmark dataset \protect\cite{nie2023time}. (a)(c)(d)(e)(f) %are univariate methods.
visualize the same variate. (b) visualizes all 321 variates. Filterbank is omitted due to its %high
similarity to STFT.}\label{fig.tsimage}
\vspace{-0.2cm}
\end{figure*}

\begin{table*}[t]
\centering
\scriptsize
\setlength{\tabcolsep}{2.7pt}{
% \begin{tabular}{llllllllllll}
\begin{tabular}{llcccccccccl}
\toprule[1pt]
\multirow{2}{*}{Method} & \multirow{2}{*}{TS-Type} & \multirow{2}{*}{Imaging} & \multicolumn{5}{c}{Imaged Time Series Modeling} & \multirow{2}{*}{TS-Recover} & \multirow{2}{*}{Task} & \multirow{2}{*}{Domain} & \multirow{2}{*}{Code}\\ \cmidrule{4-8}
 & & & Multi-modal & Model & Pre-trained & Fine-tune & Prompt & & & & \\ \midrule
\cite{silva2013time} & UTS & RP & \xmark & \texttt{K-NN} & \xmark & \xmark & \xmark & \xmark & Classification & General & \xmark\\
\cite{wang2015encoding} & UTS & GAF & \xmark & \texttt{CNN} & \xmark & \cmark$^{\flat}$ & \xmark & \cmark & Classification & General & \xmark\\
\cite{wang2015imaging} & UTS & GAF & \xmark & \texttt{CNN} & \xmark & \cmark$^{\flat}$ & \xmark & \cmark & Multiple & General & \xmark\\
% \multirow{2}{*}{\cite{wang2015imaging}} & \multirow{2}{*}{UTS} & \multirow{2}{*}{GAF} & \multirow{2}{*}{\xmark} & \multirow{2}{*}{\texttt{CNN}} & \multirow{2}{*}{\xmark} & \multirow{2}{*}{\cmark$^{\flat}$} & \multirow{2}{*}{\xmark} & \multirow{2}{*}{\cmark} & Classification & \multirow{2}{*}{General} & \multirow{2}{*}{\xmark}\\
% & & & & & & & & & \& Imputation & & \\
\cite{ma2017learning} & MTS & Heatmap & \xmark & \texttt{CNN} & \xmark & \cmark$^{\flat}$ & \xmark & \cmark & Forecasting & Traffic & \xmark\\
\cite{hatami2018classification} & UTS & RP & \xmark & \texttt{CNN} & \xmark & \cmark$^{\flat}$ & \xmark & \xmark & Classification & General & \xmark\\
\cite{yazdanbakhsh2019multivariate} & MTS & Heatmap & \xmark & \texttt{CNN} & \xmark & \cmark$^{\flat}$ & \xmark & \xmark & Classification & General & \cmark\textsuperscript{\href{https://github.com/SonbolYb/multivariate_timeseries_dilated_conv}{[1]}}\\
MSCRED \cite{zhang2019deep} & MTS & Other ($\S$\ref{sec.othermethod}) & \xmark & \texttt{ConvLSTM} & \xmark & \cmark$^{\flat}$ & \xmark & \xmark & Anomaly & General & \cmark\textsuperscript{\href{https://github.com/7fantasysz/MSCRED}{[2]}}\\
\cite{li2020forecasting} & UTS & RP & \xmark & \texttt{CNN} & \cmark & \cmark & \xmark & \xmark & Forecasting & General & \cmark\textsuperscript{\href{https://github.com/lixixibj/forecasting-with-time-series-imaging}{[3]}}\\
\cite{cohen2020trading} & UTS & LinePlot & \xmark & \texttt{Ensemble} & \xmark & \cmark$^{\flat}$ & \xmark & \xmark & Classification & Finance & \xmark\\
% \cite{du2020image} & UTS & Spectrogram & \xmark & \texttt{CNN} & \xmark & \cmark$^{\flat}$ & \xmark & \xmark & Classification & Finance & \xmark\\
\cite{barra2020deep} & UTS & GAF & \xmark & \texttt{CNN} & \xmark & \cmark$^{\flat}$ & \xmark & \xmark & Classification & Finance & \xmark\\
% \cite{barra2020deep} & UTS & GAF & \xmark & \texttt{VGG-16} & \xmark & \cmark$^{\flat}$ & \xmark & \xmark & Classification & Finance & \xmark\\
% \cite{cao2021image} & UTS & RP & \xmark & \texttt{CNN} & \xmark & \cmark$^{\flat}$ & \xmark & \xmark & Classification & General & \xmark\\
VisualAE \cite{sood2021visual} & UTS & LinePlot & \xmark & \texttt{CNN} & \xmark & \cmark$^{\flat}$ & \xmark & \cmark & Forecasting & Finance & \xmark\\
% VisualAE \cite{sood2021visual} & UTS & LinePlot & \xmark & \texttt{CNN} & \xmark & \cmark$^{\flat}$ & \xmark & \xmark & Img-Generation & Finance & \xmark\\
\cite{zeng2021deep} & MTS & Heatmap & \xmark & \texttt{CNN,LSTM} & \xmark & \cmark$^{\flat}$ & \xmark & \cmark & Forecasting & Finance & \xmark\\
% \cite{zeng2021deep} & MTS & Heatmap & \xmark & \texttt{SRVP} & \xmark & \cmark$^{\flat}$ & \xmark & \cmark & Forecasting & Finance & \xmark\\
AST \cite{gong2021ast} & UTS & Spectrogram & \xmark & \texttt{DeiT} & \cmark & \cmark & \xmark & \xmark & Classification & Audio & \cmark\textsuperscript{\href{https://github.com/YuanGongND/ast}{[4]}}\\
TTS-GAN \cite{li2022tts} & MTS & Heatmap & \xmark & \texttt{ViT} & \xmark & \cmark$^{\flat}$ & \xmark & \cmark & Ts-Generation & Health & \cmark\textsuperscript{\href{https://github.com/imics-lab/tts-gan}{[5]}}\\
SSAST \cite{gong2022ssast} & UTS & Spectrogram & \xmark & \texttt{ViT} & \cmark$^{\natural}$ & \cmark & \xmark & \xmark & Classification & Audio & \cmark\textsuperscript{\href{https://github.com/YuanGongND/ssast}{[6]}}\\
MAE-AST \cite{baade2022mae} & UTS & Spectrogram & \xmark & \texttt{MAE} & \cmark$^{\natural}$ & \cmark & \xmark & \xmark & Classification & Audio & \cmark\textsuperscript{\href{https://github.com/AlanBaade/MAE-AST-Public}{[7]}}\\
AST-SED \cite{li2023ast} & UTS & Spectrogram & \xmark & \texttt{SSAST,GRU} & \cmark & \cmark & \xmark & \xmark & EventDetection & Audio & \xmark\\
\cite{jin2023classification} & UTS & %Multiple
LinePlot & \xmark & \texttt{CNN} & \cmark & \cmark & \xmark & \xmark & Classification & Physics & \xmark\\
ForCNN \cite{semenoglou2023image} & UTS & LinePlot & \xmark & \texttt{CNN} & \xmark & \cmark$^{\flat}$ & \xmark & \xmark & Forecasting & General & \xmark\\
Vit-num-spec \cite{zeng2023pixels} & UTS & Spectrogram & \xmark & \texttt{ViT} & \xmark & \cmark$^{\flat}$ & \xmark & \xmark & Forecasting & Finance & \xmark\\
% \cite{wimmer2023leveraging} & MTS & LinePlot & \xmark & \texttt{CLIP,LSTM} & \cmark & \cmark & \xmark & \xmark & Classification & Finance & \xmark\\
ViTST \cite{li2023time} & MTS & LinePlot & \xmark & \texttt{Swin} & \cmark & \cmark & \xmark & \xmark & Classification & General & \cmark\textsuperscript{\href{https://github.com/Leezekun/ViTST}{[8]}}\\
MV-DTSA \cite{yang2023your} & UTS\textsuperscript{*} & LinePlot & \xmark & \texttt{CNN} & \xmark & \cmark$^{\flat}$ & \xmark & \cmark & Forecasting & General & \cmark\textsuperscript{\href{https://github.com/IkeYang/machine-vision-assisted-deep-time-series-analysis-MV-DTSA-}{[9]}}\\
TimesNet \cite{wu2023timesnet} & MTS & Heatmap & \xmark & \texttt{CNN} & \xmark & \cmark$^{\flat}$ & \xmark & \cmark & Multiple & General & \cmark\textsuperscript{\href{https://github.com/thuml/TimesNet}{[10]}}\\
ITF-TAD \cite{namura2024training} & UTS & Spectrogram & \xmark & \texttt{CNN} & \cmark & \xmark & \xmark & \xmark & Anomaly & General & \xmark\\
\cite{kaewrakmuk2024multi} & UTS & GAF & \xmark & \texttt{CNN} & \cmark & \cmark & \xmark & \xmark & Classification & Sensing & \xmark\\
HCR-AdaAD \cite{lin2024hierarchical} & MTS & RP & \xmark & \texttt{CNN,GNN} & \xmark & \cmark$^{\flat}$ & \xmark & \xmark & Anomaly & General & \xmark\\
FIRTS \cite{costa2024fusion} & UTS & Other ($\S$\ref{sec.othermethod}) & \xmark & \texttt{CNN} & \xmark & \cmark$^{\flat}$ & \xmark & \xmark & Classification & General & \cmark\textsuperscript{\href{https://sites.google.com/view/firts-paper}{[11]}}\\
% \multirow{2}{*}{FIRTS \cite{costa2024fusion}} & \multirow{2}{*}{UTS} & Spectrogram & \multirow{2}{*}{\xmark} & \multirow{2}{*}{\texttt{CNN}} & \multirow{2}{*}{\xmark} & \multirow{2}{*}{\cmark$^{\flat}$} & \multirow{2}{*}{\xmark} & \multirow{2}{*}{\xmark} & \multirow{2}{*}{Classification} & \multirow{2}{*}{General} & \multirow{2}{*}{\cmark\textsuperscript{\href{https://sites.google.com/view/firts-paper}{[2]}}}\\
%  & & \& GAF,RP,MTF & & & & & & & & & \\
% \cite{homenda2024time} & UTS\textsuperscript{*} & Multiple & \xmark & \texttt{CNN} & \xmark & \cmark$^{\flat}$ & \xmark & \xmark & Classification & General & \xmark\\
CAFO \cite{kim2024cafo} & MTS & RP & \xmark & \texttt{CNN,ViT} & \xmark & \cmark$^{\flat}$ & \xmark & \xmark & Explanation & General & \cmark\textsuperscript{\href{https://github.com/eai-lab/CAFO}{[12]}}\\
% \multirow{2}{*}{CAFO \cite{kim2024cafo}} & \multirow{2}{*}{MTS} & \multirow{2}{*}{RP} & \multirow{2}{*}{\xmark} & \texttt{ShuffleNet,ResNet} & \multirow{2}{*}{\cmark} & \multirow{2}{*}{\cmark} & \multirow{2}{*}{\xmark} & \multirow{2}{*}{\xmark} & Classification & \multirow{2}{*}{General} & \multirow{2}{*}{\cmark}\\
%  & & & & \texttt{MLP-Mixer,ViT} & & & & & \& Explanation & & \\
ViTime \cite{yang2024vitime} & UTS\textsuperscript{*} & LinePlot & \xmark & \texttt{ViT} & \cmark$^{\natural}$ & \cmark & \xmark & \cmark & Forecasting & General & \cmark\textsuperscript{\href{https://github.com/IkeYang/ViTime}{[13]}}\\
ImagenTime \cite{naiman2024utilizing} & MTS & Other ($\S$\ref{sec.othermethod}) & \xmark & %\texttt{Diffusion}
\texttt{CNN} & \xmark & \cmark$^{\flat}$ & \xmark & \cmark & Ts-Generation & General & \cmark\textsuperscript{\href{https://github.com/azencot-group/ImagenTime}{[14]}}\\
TimEHR \cite{karami2024timehr} & MTS & Heatmap & \xmark & \texttt{CNN} & \xmark & \cmark$^{\flat}$ & \xmark & \cmark & Ts-Generation & Health & \cmark\textsuperscript{\href{https://github.com/esl-epfl/TimEHR}{[15]}}\\
VisionTS \cite{chen2024visionts} & UTS\textsuperscript{*} & Heatmap & \xmark & \texttt{MAE} & \cmark & \cmark & \xmark & \cmark & Forecasting & General & \cmark\textsuperscript{\href{https://github.com/Keytoyze/VisionTS}{[16]}}\\ \midrule
InsightMiner \cite{zhang2023insight} & UTS & LinePlot & \cmark & \texttt{LLaVA} & \cmark & \cmark & \cmark & \xmark & Txt-Generation & General & \xmark\\
\cite{wimmer2023leveraging} & MTS & LinePlot & \cmark & \texttt{CLIP,LSTM} & \cmark & \cmark & \xmark & \xmark & Classification & Finance & \xmark\\
% \cite{dixit2024vision} & UTS & Spectrogram & \cmark & \texttt{GPT4o,Gemini} & \cmark & \xmark & \cmark & \xmark & Classification & Audio & \xmark\\
\multirow{2}{*}{\cite{dixit2024vision}} & \multirow{2}{*}{UTS} & \multirow{2}{*}{Spectrogram} & \multirow{2}{*}{\cmark} & \texttt{GPT4o,Gemini} & \multirow{2}{*}{\cmark} & \multirow{2}{*}{\xmark} & \multirow{2}{*}{\cmark} & \multirow{2}{*}{\xmark} & \multirow{2}{*}{Classification} & \multirow{2}{*}{Audio} & \multirow{2}{*}{\xmark}\\
 & & & & \& \texttt{Claude3} & & & & & & & \\
\cite{daswani2024plots} & MTS & LinePlot & \cmark & \texttt{GPT4o,Gemini} & \cmark & \xmark & \cmark & \xmark & Multiple & General & \xmark\\
TAMA \cite{zhuang2024see} & UTS & LinePlot & \cmark & \texttt{GPT4o} & \cmark & \xmark & \cmark & \xmark & Anomaly & General & \xmark\\
\cite{prithyani2024feasibility} & MTS & LinePlot & \cmark & \texttt{LLaVA} & \cmark & \cmark & \cmark & \xmark & Classification & General & \cmark\textsuperscript{\href{https://github.com/vinayp17/VLM_TSC}{[17]}}\\
\bottomrule[1pt]
\end{tabular}}
\vspace{-0.25cm}
\caption{Taxonomy of vision models on time series. The top panel includes single-modal models. The bottom panel includes multi-modal models. {\bf TS-Type} denotes type of time series. {\bf TS-Recover} denotes %whether time series recovery ($\S$\ref{sec.processing}) has been performed.
recovering time series from predicted images ($\S$\ref{sec.processing}). \textsuperscript{*}: %the model has been %applied on MTSs by %processing %modeling the individual UTSs of each MTS.
the method has been used to model the individual UTSs of an MTS. $^{\natural}$: a new pre-trained model was proposed in the work. $^{\flat}$: %without using a pre-trained model, fine-tune means training from scratch.
when pre-trained models were unused, ``Fine-tune'' refers to train a task-specific model from scratch. %In the
{\bf Model} column: \texttt{CNN} could be regular CNN, ResNet, VGG-Net, %U-Net,
{\em etc.}}\label{tab.taxonomy}
% The code only include verified official code from the authors.
\vspace{-0.3cm}
\end{table*}

\begin{table*}[t]
\centering
\small
\setlength{\tabcolsep}{2.9pt}{
\begin{tabular}{l|l|l|l}\hline
% \toprule[1pt]
\rowcolor{gray!20}
{\bf Method} & {\bf TS-Type} & {\bf Advantages} & {\bf Limitations}\\ \hline
Line Plot ($\S$\ref{sec.lineplot}) & UTS, MTS & matches human perception of time series & limited to MTSs with a small number of variates\\ \hline
Heatmap ($\S$\ref{sec.heatmap}) & UTS, MTS & straightforward for both UTSs and MTSs & the order of variates may affect their correlation learning\\ \hline
Spectrogram ($\S$\ref{sec.spectrogram}) & UTS & encodes the time-frequency space & limited to UTSs; needs a proper choice of window/wavelet\\ \hline
GAF ($\S$\ref{sec.gaf}) & UTS & encodes the temporal correlations in a UTS & limited to UTSs; $O(T^{2})$ time and space complexity\\ \hline% for long time series\\ \hline
% RP ($\S$\ref{sec.rp}) & UTS & flexibility in image size by tuning $m$ and $\tau$ & limited to UTSs; the pattern has a threshold-dependency\\ \hline
RP ($\S$\ref{sec.rp}) & UTS & flexibility in image size by tuning $m$ and $\tau$ & limited to UTSs; information loss after thresholding\\ \hline
% \bottomrule[1pt]
\end{tabular}}
\vspace{-0.2cm}
\caption{Summary of the five primary methods for transforming time series to images. {\bf TS-Type} denotes type of time series.}\label{tab.tsimage}
\vspace{-0.2cm}
\end{table*}

\section{Time Series To Image Transformation}\label{sec.tsimage}

% This section summarizes 5 major methods for imaging time series ($\S$\ref{sec.lineplot}-$\S$\ref{sec.rp}). We also discuss some other methods ($\S$\ref{sec.othermethod}) and how to model MTS with these methods ($\S$\ref{sec.modelmts}).
This section summarizes the methods for imaging time series ($\S$\ref{sec.lineplot}-$\S$\ref{sec.othermethod}) and their extensions to encode MTSs ($\S$\ref{sec.modelmts}).

% This section summarizes 5 major methods for transforming time series to images, including Line Plot, Heatmap, Spetrogram, GAF and RP, and several minor methods. We discuss their pros and cons and how to deal with MTS.

% This section discusses the advantages and limitations of different methods for time series to image transformation (invertible, efficiency, information preservation, MTS, long-range time series, parametric, etc.).

%\subsection{Methods}

\vspace{-0.08cm}

\subsection{Line Plot}\label{sec.lineplot}

Line Plot is a straightforward way for visualizing UTSs for human analysis ({\em e.g.}, stocks, power consumption, {\em etc.}). As illustrated by Fig. \ref{fig.tsimage}(a), the simplest approach is to draw a 2D image with x-axis representing %the time horizon
time steps and y-axis representing %the magnitude of the normalized time series.
time-wise values, %A line is used to connect all values of the series over time.
with a line connecting all values of the series over time. This image can be %represented by either three-channel pixels or single-channel pixels
either three-channel ({\em i.e.}, RGB) or single-channel as the colors may not %provide additional information
be informative %\cite{cohen2020trading,sood2021visual,jin2023classification,zhang2023insight,zhuang2024see}.
\cite{cohen2020trading,sood2021visual,jin2023classification,zhang2023insight}. ForCNN \cite{semenoglou2023image} even uses a single 8-bit integer to represent each pixel for black-white images. So far, there is no consensus on whether other graphical components, such as legend, grids and tick labels, could provide extra benefits in any task. For example, ViTST \cite{li2023time} finds these components are superfluous in a classification task, while TAMA \cite{zhuang2024see} finds grid-like auxiliary lines help enhance anomaly detection.

In addition to the regular Line Plot, MV-DTSA \cite{yang2023your} and ViTime \cite{yang2024vitime} divide an image into $h\times L$ grids, %where $h$ is the number of rows and $L$ is the number of columns,
and %introduced
define a function to map each time step of a UTS to a grid, producing a grid-like Line Plot. Also, we include methods that use Scatter Plot \cite{daswani2024plots,prithyani2024feasibility} in this category because %the only difference between a Scatter Plot and a Line Plot is whether the time-wise values are connected by lines.
a Scatter Plot resembles a Line Plot but doesn't connect %time-wise values
data points with a line. By comparing them, \cite{prithyani2024feasibility} finds a Line Plot could induce better time series classification.

For MTSs, we defer the discussion on Line Plot to $\S$\ref{sec.modelmts}.

% For MTS, some methods use the channel-independence assumption proposed in \cite{nie2023time} and represent each variate in MTS with an individual Line Plot \cite{yang2023your,yang2024vitime}. ViTST \cite{li2023time} also uses an individual Line Plot per variate, but colors different lines and assembles all plots to form a bigger image. The method in \cite{wimmer2023leveraging} plots %the time series of
% all variates in a single Line Plot and distinguish them by %use different
% types of lines ({\em e.g.}, solid, dashed, dotted, {\em etc.}). %to distinguish them.
% However, these methods only work for a small number of variates. For example, in \cite{wimmer2023leveraging}, there are only 4 variates in its financial MTSs.

%\cite{li2023time} space-costly because of blank pixels. scatter plot.

%Invertible with a numeric prediction head \cite{sood2021visual}. It fits tasks such as forecasting, imputation, etc.

\vspace{-0.08cm}

\subsection{Heatmap}\label{sec.heatmap}

As shown in Fig. \ref{fig.tsimage}(b), Heatmap is a 2D visualization of the magnitude of the values in a matrix using color. %The variation of color represents the intensity of each value. %Therefore,
It has been used to %directly
represent the matrix of an MTS, {\em i.e.}, $\mat{X} \in \mathbb{R}^{d\times T}$, as a one-channel $d\times T$ image \cite{li2022tts,yazdanbakhsh2019multivariate}. Similarly, TimEHR \cite{karami2024timehr} represents an {\em irregular} MTS, where the intervals between time steps are uneven, as a $d\times H$ Heatmap image by grouping the uneven time steps into $H$ even time bins. In \cite{zeng2021deep}, a different method is used for visualizing a 9-variate financial %time series.
MTS. It reshapes the 9 variates at each time step to a $3\times 3$ Heatmap image, and uses the sequence of images to forecast future %image
frames, achieving %time series
%MTS
time series forecasting. In contrast, VisionTS \cite{chen2024visionts} uses Heatmap to visualize UTSs. %instead.
Similar to TimesNet \cite{wu2023timesnet}, it first segments a length-$T$ UTS into $\lfloor T/P\rfloor$ length-$P$ subsequences, where $P$ is a parameter representing a periodicity of the UTS. Then the subsequences are stacked into a $P\times \lfloor T/P\rfloor$ matrix, %and duplicated 3 times to produce a 3-channel
with 3 duplicated channels, to produce a grayscale image %which serves as an
input to %a vision foundation model.
an LVM. To encode MTSs, VisionTS adopts the channel independence assumption \cite{nie2023time} and individually models each variate in an MTS.

\vspace{0.2cm}

\noindent{\bf Remark.} Heatmap can be used to visualize matrices of various forms. It is also used for matrices generated by the subsequent methods ({\em e.g.}, Spectrogram, GAF, RP) in this section. In this paper, the name Heatmap refers specifically to images that use color to visualize the (normalized) values in UTS $\mat{x}$ or MTS $\mat{X}$ without performing other transformations.

%\cite{chen2024visionts,karami2024timehr} bin version of TSH \cite{karami2024timehr}, DE and STFT \cite{naiman2024utilizing} (DE can be used for constructing RP), rearrange variates for video version of TSH \cite{zeng2021deep}.

%\vspace{0.2cm}

\subsection{Spectrogram}\label{sec.spectrogram}

A {\em spectrogram} is a visual representation of the spectrum of frequencies of a signal as it varies with time, which are extensively used for analyzing audio signals \cite{gong2021ast}. Since audio signals are a type of UTS, spectrogram can be considered as a method for imaging a UTS. As shown in Fig. \ref{fig.tsimage}(c), a common format is a 2D heatmap image with x-axis representing time steps and y-axis representing frequency, {\em a.k.a.} a time-frequency space. %The color at each point
Each pixel in the image represents the (logarithmic) amplitude of a specific frequency at a specific time point. Typical methods for %transforming a UTS to
producing a spectrogram include {\bf Short-Time Fourier Transform (STFT)} \cite{griffin1984signal}, {\bf Wavelet Transform} \cite{daubechies1990wavelet}, and {\bf Filterbank} \cite{vetterli1992wavelets}.

\vspace{0.2cm}

\noindent{\bf STFT.} %Discrete Fourier transform (DFT) can be used to represent a UTS signal %$\mat{x}=[x_{1}, ..., x_{T}]$
%$\mat{x}\in\mathbb{R}^{1\times T}$ as a sum of sinusoidal components. The output of the transform is a function of frequency $f(w)$, describing the intensity of each constituent frequency $w$ of the entire UTS. 
Discrete Fourier transform (DFT) can be used to describe the intensity $f(w)$ of each constituent frequency $w$ of a UTS signal $\mat{x}\in\mathbb{R}^{1\times T}$. However, $f(w)$ has no time dependency. It cannot provide dynamic information such as when a specific frequency appear in the UTS. STFT addresses this deficiency by sliding a window function $g(t)$ over the time steps in %the UTS,
$\mat{x}$, and computing the DFT within each window by
\begin{equation}\label{eq.stft}
\small
\begin{aligned}
f(w,\tau) = \sum_{t=1}^{T}x_{t}g(t - \tau)e^{-iwt}
\end{aligned}
\end{equation}
where $w$ is frequency, $\tau$ is the position of the window, $f(w,\tau)$ describes the intensity of frequency $w$ at time step $\tau$.

%With a proper selection of the
By selecting a proper window function $g(\cdot)$ ({\em e.g.}, Gaussian/Hamming/Bartlett window), %({\em e.g.}, Gaussian window, Hamming window, Bartlett window), %{\em etc.}),
a 2D spectrogram ({\em e.g.}, Fig. \ref{fig.tsimage}(c)) can be drawn via a heatmap on the squared values $|f(w,\tau)|^{2}$, with $w$ as the y-axis, and $\tau$ as the x-axis. For example, \cite{dixit2024vision} uses STFT based spectrogram as an input to LMMs %\hh{do you mean LVMs? check}
for time series classification.

%Fourier transform is a powerful data analysis tool that represents any complex signal as a sum of sines and cosines and transforms the signal from the time domain to the frequency domain. However, Fourier transform can only show which frequencies are present in the signal, but not when these frequencies appear. The STFT divides original signal into several parts using a sliding window to fix this problem. STFT involves a sliding window for extracting frequency components within the window.

\vspace{0.2cm}

\noindent{\bf Wavelet Transform.} %Like Fourier transform, %\hh{this paragraph needs a citation}
Continuous Wavelet Transform (CWT) uses the inner product to measure the similarity between a signal function $x(t)$ and an analyzing function. %In STFT (Eq.~\eqref{eq.stft}), the analyzing function is a windowed exponential $g(t - \tau)e^{-iwt}$.
%In CWT,
The analyzing function is a {\em wavelet} $\psi(t)$, where the typical choices include Morse wavelet, Morlet wavelet, %Daubechies wavelet, %Beylkin wavelet, 
{\em etc.} %The
CWT compares $x(t)$ to the shifted and scaled ({\em i.e.}, stretched or shrunk) versions of the wavelet, and output a CWT coefficient by
\begin{equation}\label{eq.cwt}
\small
\begin{aligned}
c(s,\tau) = \int_{-\infty}^{\infty}x(t)\frac{1}{s}\psi^{*}(\frac{t - \tau}{s})dt
\end{aligned}
\end{equation}
where $*$ denotes complex conjugate, $\tau$ is the time step to shift, and $s$ represents the scale. In practice, a discretized version of CWT in Eq.~\eqref{eq.cwt} is implemented for UTS $[x_{1}, ..., x_{T}]$.

It is noteworthy that the scale $s$ controls the frequency encoded in a wavelet -- a larger $s$ leads to a stretched wavelet with a lower frequency, and vice versa. As such, by varying $s$ and $\tau$, a 2D spectrogram ({\em e.g.}, Fig. \ref{fig.tsimage}(d)) can be drawn %, often with a heatmap
on $|c(s,\tau)|$, where $s$ is the y-axis and $\tau$ is the x-axis. Compared to STFT, which uses a fixed window size, Wavelet Transform allows variable wavelet sizes -- a larger size %region
for more precise low frequency information. 
%Usually, $s$ and $\tau$ vary dependently -- a larger $s$ leads to a stretched wavelet that shifts slowly, {\em i.e.}, a smaller $\tau$. This property %of CWT
%yields a spectrogram that balances the resolutions of frequency %$s$
%and time, %$\tau$,
%which is an advantage over the fixed time resolution in STFT.
% Thus, both of the methods in %\cite{du2020image}
% \cite{namura2024training} and \cite{zeng2023pixels} choose CWT (with Morlet wavelet) to generate the spectrogram.
Thus, the methods in \cite{du2020image,namura2024training,zeng2023pixels} choose CWT (with Morlet wavelet) to generate the spectrogram.

%A wavelet is a wave-like oscillation that has zero mean and is localized in both time and frequency space.

\vspace{0.2cm}

\noindent{\bf Filterbank.} This method %is relevant to
resembles STFT and is often used in processing audio signals. Given an audio signal, it firstly goes through a {\em pre-emphasis filter} to boost high frequencies, which helps improve the clarity of the signal. Then, STFT is applied on the signal. %with a sliding window $g(t)$ of size $k$ that shifts in a fixed stride $\tau$. %where the adjacent windows may overlap in $k$ time length.
%Finally, filterbank features are computed by applying multiple ``triangle-shaped'' filters spaced on the Mel-scale to the STFT output $f(w, \tau)$. %where Mel-scale is a method to make the filters more discriminative on lower frequencies, %than higher frequencies,
%imitating the non-linear human ear perception of sound.
Finally, multiple ``triangle-shaped'' filters spaced on a Mel-scale are applied to the STFT power spectrum $|f(w, \tau)|^{2}$ to extract frequency bands. The outcome filterbank features $\hat{f}(w, \tau)$ can be used to yield a spectrogram with $w$ as the y-axis, and $\tau$ as the x-axis.

%Filterbank was introduced in AST \cite{gong2021ast} with %$k$=25ms
Filterbank was adopted in AST \cite{gong2021ast} with 
a 25ms Hamming window that shifts every 10ms for classifying audio signals using Vision Transformer (ViT). It then becomes widely used in the follow-up works such as SSAST \cite{gong2022ssast}, MAE-AST \cite{baade2022mae}, and AST-SED \cite{li2023ast}, as summarized in Table \ref{tab.taxonomy}.



%Use MLP to predict TS directly \cite{zeng2023pixels}.

%\vspace{0.2cm}

% \vspace{0.2cm}

\subsection{Gramian Angular Field (GAF)}\label{sec.gaf}

GAF was introduced for classifying UTSs using CNNs %using %image based CNNs
by \cite{wang2015encoding}. It was then extended %with an extension
to an imputation task in \cite{wang2015imaging}. Similarly, \cite{barra2020deep} applied GAF for financial time series forecasting.

Given a UTS $\mat{x}\in\mathbb{R}^{1\times T}$, %$[x_{1}, ..., x_{T}]$,
the first step %before GAF
is to rescale each $x_{t}$ to a value $\tilde{x}_{t}$ %in the interval of
within $[0, 1]$ (or $[-1, 1]$). %by min-max normalization.
This range enables mapping $\tilde{x}_{t}$ to polar coordinates by $\phi_{t}=\text{arccos}(\tilde{x}_{i})$, with a radius $r=t/N$ encoding the time stamp, where $N$ is a constant factor to regularize the span of the polar coordinates. %system. Then,
Two types of GAF, Gramian Sum Angular Field (GASF) and Gramian Difference Angular Field (GADF) are defined as
\begin{equation}\label{eq.gaf}
\small
\begin{aligned}
&\text{GASF:}~~\text{cos}(\phi_{t} + \phi_{t'})=x_{t}x_{t'} - \sqrt{1 - x_{t}^{2}}\sqrt{1 - x_{t'}^{2}}\\
&\text{GADF:}~~\text{sin}(\phi_{t} - \phi_{t'})=x_{t'}\sqrt{1 - x_{t}^{2}} - x_{t}\sqrt{1 - x_{t'}^{2}}
\end{aligned}
\end{equation}
which exploits the pairwise temporal correlations in the UTS. Thus, the outcome is a $T\times T$ matrix $\mat{G}$ with $\mat{G}_{t,t'}$ specified by either type in Eq.~\eqref{eq.gaf}. A GAF image is a heatmap on $\mat{G}$ with both axes representing time, as illustrated by Fig. \ref{fig.tsimage}(e).

% Invertible.

% \vspace{0.2cm}

\subsection{Recurrence Plot (RP)}\label{sec.rp}

%RP \cite{eckmann1987recurrence} is a method to encode a UTS into an image that aims to capture the periodic patterns in the UTS by using its reconstructed {\em phase space}. The phase space of a UTS $[x_{1}, ..., x_{T}]$ can be reconstructed by {\em time delay embedding}, which is a set of new vectors $\mat{v}_{1}$, ..., $\mat{v}_{l}$ with

RP \cite{eckmann1987recurrence} encodes a UTS into an image that captures its periodic patterns by using its reconstructed {\em phase space}. The phase space of %a UTS %$[x_{1}, ..., x_{T}]$
$\mat{x}\in\mathbb{R}^{1\times T}$ can be reconstructed by {\em time delay embedding} -- a set of new vectors $\mat{v}_{1}$, ..., $\mat{v}_{l}$ with
\begin{equation}\label{eq.de}
\small
\begin{aligned}
\mat{v}_{t}=[x_{t}, x_{t+\tau}, x_{t+2\tau}, ..., x_{t+(m-1)\tau}]\in\mathbb{R}^{m\tau},~~~1\le t \le l
\end{aligned}
\end{equation}
where $\tau$ is the time delay, $m$ is the dimension of the phase space, both %of which
are hyperparameters. Hence, $l=T-(m-1)\tau$. With vectors $\mat{v}_{1}$, ..., $\mat{v}_{l}$, an RP image %is constructed by measuring
measures their pairwise distances, results in an $l\times l$ image whose element
\begin{equation}\label{eq.rp}
\small
\begin{aligned}
\text{RP}_{i,j}=\Theta(\varepsilon - \|\mat{v}_{i} - \mat{v}_{j}\|),~~~1\le i,j\le l
\end{aligned}
\end{equation}
where $\Theta(\cdot)$ is the Heaviside step function, $\varepsilon$ is a threshold, and $\|\cdot\|$ is a norm function such as $\ell_{2}$ norm. Eq.~\eqref{eq.rp} %states RP produces a heatmap image on a binary matrix with $\text{RP}_{i,j}=1$ if $\mat{v}_{i}$ and $\mat{v}_{j}$ are sufficiently similar.
generates a binary matrix with $\text{RP}_{i,j}=1$ if $\mat{v}_{i}$ and $\mat{v}_{j}$ are sufficiently similar, producing a black-white image ({\em e.g.}, Fig. \ref{fig.tsimage}(f)).% ({\em e.g.}, a periodic pattern).

An advantage of RP is its flexibility in image size by tuning $m$ and $\tau$. Thus it has been used for time series classification %\cite{cao2021image},
\cite{silva2013time,hatami2018classification}, forecasting \cite{li2020forecasting}, anomaly detection \cite{lin2024hierarchical} and %feature-wise
explanation \cite{kim2024cafo}. Moreover, the method in \cite{hatami2018classification}, and similarly in HCR-AdaAD \cite{lin2024hierarchical}, omit the thresholding in Eq.~\eqref{eq.rp} and uses $\|\mat{v}_{i} - \mat{v}_{j}\|$ to produce continuously valued images %in a classification task
to avoid information loss.


% \vspace{0.2cm}

\subsection{Other Methods}\label{sec.othermethod}

%There are some less commonly used methods. For example, in
Additionally, %there are some peripheral methods. %In addition to GAF,
\cite{wang2015encoding} introduces Markov Transition Field (MTF) for imaging a UTS. %$\mat{x}\in\mathbb{R}^{1\times T}$. 
%MTF first assigns each $x_{t}$ to one of $Q$ quantile bins, then builds a $Q\times Q$ Markov transition matrix $\mat{M}$ {\em s.t.} $\mat{M}_{i,j}$ represents the frequency %with which
%of the case when a point $x_{t}$ in the $i$-th bin is followed by a point $x_{t'}$ in the $j$-th bin, {\em i.e.}, $t=t'+1$. Matrix $\mat{M}$ serves as the input of a heatmap image.
MTF is a matrix $\mat{M}\in\mathbb{R}^{Q\times Q}$ encoding the transition probabilities over time segments, where $Q$ is the number of segments. %Moreover,
ImagenTime \cite{naiman2024utilizing} stacks the delay embeddings $\mat{v}_{1}$, ..., $\mat{v}_{l}$ in Eq.~\eqref{eq.de} to an $l\times m\tau$ matrix for visualizing UTSs. %It also uses a variant of STFT.
% The method in \cite{homenda2024time} introduces five different 2D images by counting, rearranging, replicating the values in a UTS. 
MSCRED \cite{zhang2019deep} uses heatmaps on the $d\times d$ correlation matrices of MTSs with $d$ variates for anomaly detection. 
Furthermore, some methods use a mixture of imaging methods by stacking different transformations. \cite{wang2015imaging} stacks GASF, GADF, MTF to a 3-channel image. %Similarly,
FIRTS \cite{costa2024fusion} builds a 3-channel image by stacking GASF, MTF and RP. %the GASF, MTF, RP representations of each UTS.
%\cite{jin2023classification} combines Line Plot with Constant-Q Transform (CQT) \cite{brown1991calculation}, a method related to wavelet transform ($\S$\ref{sec.spectrogram}), to generate 2-channel images.
The mixture methods encode a UTS with multiple views and were found more robust than single-view images in these works for %time series
classification tasks.

\subsection{How to Model MTS}\label{sec.modelmts}

In the above methods, Heatmap ($\S$\ref{sec.heatmap}) can be %directly
used to visualize the %2D
variate-time matrices, $\mat{X}$, of MTSs ({\em e.g.}, Fig. \ref{fig.structure}(b)), where correlated variates %are better to
should be spatially close to each other. Line Plot ($\S$\ref{sec.lineplot}) can be used to visualize MTSs by plotting all variates in the same image \cite{wimmer2023leveraging,daswani2024plots} or combining all univariate images to compose a bigger %1-channel
image \cite {li2023time}, but these methods only work for a small number of variates. Spectrogram ($\S$\ref{sec.spectrogram}), GAF ($\S$\ref{sec.gaf}), and RP ($\S$\ref{sec.rp}) were designed specifically for UTSs. For these methods and Line Plot, which are not straightforward %for MTS transformation,
in imaging MTSs, the general approaches %to use them %for MTS
include using channel independence assumption to model each variate individually \cite{nie2023time}, %like VisionTS \cite{chen2024visionts},
or stacking the images of $d$ variates to form a $d$-channel image %as did by
\cite{naiman2024utilizing,kim2024cafo}. %\cite{prithyani2024feasibility,naiman2024utilizing,kim2024cafo}.
However, the latter does not fit some vision models pre-trained on RGB images which requires 3-channel inputs (more discussions are deferred to $\S$\ref{sec.processing}).

\vspace{0.2cm}

\noindent{\bf Remark.} As a summary, Table \ref{tab.tsimage} recaps the salient advantages and limitations of the five primary imaging methods that are introduced in this section.

% \hh{can we have a table (e.g., rows are different imaging methods and columns are a few desirable propoerties) or a short paragraph to discuss/summarize/compare the strenths and weakness of different imaging methods for ts? This might bring some structure/comprehension to this section (as opposed to, e.g., some reviewer might complain that what we do here is a laundry list)}

\section{Imaged Time Series Modeling}\label{sec.model}

With image representations, time series analysis can be readily performed with vision models. This section discusses such solutions from %traditional vision models %($\S$\ref{sec.cnns})
%to the recent large vision models %($\S$\ref{sec.lvms})
%and large multimodal models.% ($\S$\ref{sec.lmms}).
the traditional models to the SOTA models.

\begin{figure*}[!t]
\centering
\includegraphics[width=0.9\textwidth]{fig/fig_2.pdf}
% \vspace{-1em}
\caption{An illustration of different modeling strategies on imaged time series in (a)(b)(c) and task-specific heads in (d).}\label{fig.models}
\vspace{-0.2cm}
\end{figure*}

\subsection{Conventional Vision Models}\label{sec.cnns}

%Similar to
Following traditional %methods on
image classification, \cite{silva2013time} applies a K-NN classifier on the RPs of time series, \cite{cohen2020trading} applies an ensemble of fundamental classifiers such as %linear regression, SVM, Ada Boost, {\em etc.}
SVM and AdaBoost on the Line Plots %images
for time series classification. As an image encoder, %a typical encoder, %of images,
CNNs have been %extensively
widely used for learning image representations. %\cite{he2016deep}.
Different from using 1D CNNs on sequences %UTS or MTS
\cite{bai2018empirical}, %regular
2D or 3D CNNs can be applied on imaged time series as shown in Fig. \ref{fig.models}(a). %to learn time series representations by encoding their image transformations.
For example, %standard
regular CNNs have been used on Spectrograms \cite{du2020image}, tiled CNNs have been used on GAF images \cite{wang2015encoding,wang2015imaging}, dilated CNNs have been used on Heatmap images \cite{yazdanbakhsh2019multivariate}. More frequently, ResNet \cite{he2016deep}, Inception-v1 \cite{szegedy2015going}, and VGG-Net \cite{simonyan2014very} have been used on Line Plots \cite{jin2023classification,semenoglou2023image}, Heatmap images \cite{zeng2021deep}, RP images \cite{li2020forecasting,kim2024cafo}, GAF images \cite{barra2020deep,kaewrakmuk2024multi}, 
% Heatmaps \cite{zeng2021deep}, RPs \cite{li2020forecasting,kim2024cafo}, GAFs \cite{barra2020deep,kaewrakmuk2024multi},
and even a mixture of GAF, MTF and RP images \cite{costa2024fusion}. In particular, for time series generation tasks, %a diffusion model with U-Nets \cite{naiman2024utilizing} and GAN frameworks of CNNs \cite{li2022tts,karami2024timehr} have also been explored.%investigated.
GAN frameworks of CNNs \cite{li2022tts,karami2024timehr} and a diffusion model with U-Nets \cite{naiman2024utilizing} have also been explored.

Due to their small to medium sizes, these models are often trained from scratch using task-specific training data. %per task using the task's training set. %of time series images.
Meanwhile, fine-tuning {\em pre-trained vision models}  %such as those pre-trained on ImageNet, %\cite{deng2009imagenet}, 
have already been found promising in cross-modality knowledge transfer for time series anomaly detection \cite{namura2024training}, forecasting \cite{li2020forecasting} and classification \cite{jin2023classification}.

% \cite{li2020forecasting} uses ImageNet pretrained CNNs.

\subsection{Large Vision Models (LVMs)}\label{sec.lvms}

Vision Transformer (ViT) \cite{dosovitskiy2021image} has %given birth to
inspired the development of %some
modern LVMs %large vision models (LVMs)
such as %DeiT \cite{touvron2021training}, 
Swin \cite{liu2021swin}, BEiT \cite{bao2022beit}, and MAE \cite{he2022masked}. %Given an input image, ViT splits it
As Fig. \ref{fig.models}(b) shows, ViT splits an %input
image into {\em patches} of fixed size, then embeds each patch and augments it with a positional embedding. The %resulting
vectors of patches are processed by a Transformer %encoder
as if they were token embeddings. Compared to CNNs, ViTs are less data-efficient, but have higher capacity. %Consequently,
Thus, %the
{\em pre-trained} ViTs have been explored for modeling %the images of time series.
imaged time series. For example, AST \cite{gong2021ast} fine-tunes DeiT \cite{touvron2021training} on the filterbank spetrogram of audios %signals
for classification tasks and finds %using
ImageNet-pretrained DeiT is remarkably effective in knowledge transfer. The fine-tuning paradigm has also been %similarly
adopted in \cite{zeng2023pixels,li2023time} but with different pre-trained models %initializations
such as Swin by \cite{li2023time}. 
VisionTS \cite{chen2024visionts} %explains
attributes %the superiority of LVMs
LVMs' superiority over LLMs in knowledge transfer %over LLMs %as an outcome of
to the small gap between the pre-trained images and imaged time series. %the patterns learned from the large-scale pre-trained images and the patterns in the images of time series.
It %also
finds that with one-epoch fine-tuning, MAE becomes the SOTA time series forecasters on %many
some benchmark datasets.

Similar to %build
time series foundation models %\cite{das2024decoder,goswami2024moment,ansari2024chronos,shi2024time}, %such as TimesFM \cite{das2024decoder}, MOMENT \cite{goswami2024moment}, Chronos \cite{ansari2024chronos} and Time-MoE \cite{shi2024time},
such as TimesFM \cite{das2024decoder}, %and MOMENT \cite{goswami2024moment}, 
there are some initial efforts in pre-training ViT architectures with imaged time series. Following AST, SSAST \cite{gong2022ssast} introduced a %joint discriminative and generative
%masked spectrogram patch prediction self-supervised learning framework
masked spectrogram patch prediction framework for pre-training ViT on a large dataset -- AudioSet-2M. Then it becomes a backbone of some follow-up works such as AST-SED \cite{li2023ast} for sound event detection. %To be effective for UTSs,
For UTSs, ViTime \cite{yang2024vitime} generates a large set of Line Plots of synthetic UTSs for pre-training ViT, which was found superior over TimesFM in zero-shot forecasting tasks on benchmark datasets.

\subsection{Large Multimodal Models (LMMs)}\label{sec.lmms}

%As Large Multimodal Models (LMMs)
As LMMs %are getting
get growing attentions, some %of the
notable LMMs, such as LLaVA \cite{liu2023visual}, Gemini \cite{team2023gemini}, GPT-4o \cite{achiam2023gpt} and Claude-3 \cite{anthropic2024claude}, have been explored to consolidate the power of LLMs %on time series
and LVMs in time series analysis. 
Since LMMs support multimodal input via prompts, methods in this thread typically prompt LMMs with the textual and imaged representations of time series, %textual representation of time series and their %image transformations, transformed images,
%then instruct LMMs
and instructions on what tasks to perform ({\em e.g.}, Fig. \ref{fig.models}(c)).

InsightMiner \cite{zhang2023insight} is a pioneer work that uses the LLaVA architecture to generate %textual descriptions about
texts describing the trend of each input UTS. It extracts the trend of a UTS by Seasonal-Trend decomposition, encodes the Line Plot of the trend, and concatenates the embedding of the Line Plot with the embeddings of a textual instruction, which includes a sequence of numbers representing the UTS, {\em e.g.}, ``[1.1, 1.7, ..., 0.3]''. The concatenated embeddings are taken by a language model for generating trend descriptions. %It also fine-tunes a few layers with the generated texts to align LLaVA checkpoints with time series domain.
Similarly, \cite{prithyani2024feasibility} adopts the LLaVA architecture, but for MTS classification. An MTS is encoded by %a sequence of
the visual %token
embeddings of the stacked Line Plots of all variates. %meanwhile
%The method also stacks
%The time series of all variate are also stacked in a prompt % of all variates in a prompt
The matrix of the MTS is also verbalized in a prompt 
as the textual modality. %By manipulating token embeddings,
By integrating token embeddings, both %of these %works propose to
methods fine-tune some layers of the LMMs with some synthetic data.

Moreover, zero-shot and in-context learning performance of several commercial LMMs have been evaluated for audio classification \cite{dixit2024vision}, anomaly detection \cite{zhuang2024see}, and some synthetic tasks \cite{daswani2024plots}, where the image %({\em e.g.}, spectrograms, Line Plots)
and textual representations of a query %UTS or MTS
time series are integrated into a prompt. For in-context learning, these methods inject the images of a few example time series and their labels ({\em e.g.}, classes) %({\em e.g.}, classes, normal status)
into an instruction to prompt LMMs for assisting the prediction of the query time series.

\subsection{Task-Specific Heads}\label{sec.task}

%With the image embedding of a time series, the next step is to produce its prediction.
For classification tasks, most of the methods in Table \ref{tab.taxonomy} adopt a fully connected (FC) layer or multilayer perceptron (MLP) to transform an embedding into a probability distribution over all classes. For forecasting tasks, there are two approaches: (1) using a $d_{e}\times W$ MLP/FC layer to directly predict (from the $d_{e}$-dimensional embedding) the time series values in a future time window of size $W$ \cite{li2020forecasting,semenoglou2023image}; (2) predicting the pixel values that represent the future part of the time series and then recovering the time series from the predicted image \cite{yang2023your,chen2024visionts,yang2024vitime} ($\S$\ref{sec.processing} discusses the recovery methods). Imputation and generation tasks resemble forecasting %in the sense of predicting
as they also predict time series values. Thus approach (2) has been used for imputation \cite{wang2015imaging} and generation \cite{naiman2024utilizing,karami2024timehr}. %LMMs have been used for classification, text generation, and anomaly detection. For these tasks,
When using LMMs for classification, text generation, and anomaly detection, most of the methods prompt LMMs to produce the desired outputs in textual answers, circumventing task-specific heads \cite{zhang2023insight,dixit2024vision,zhuang2024see}.

%Forecasting: MLP, FC to predict numerical values using embeddings. Imputation of images (TSH). Classification: MLP, FC using embeddings.

\section{Pre-Processing and Post-Processing}\label{sec.processing}

To be successful in using vision models, some subtle design desiderata %to be considered
include {\bf time series normalization}, {\bf image alignment} and {\bf time series recovery}.

\vspace{0.2cm}

\noindent{\bf Time Series Normalization.} Vision models are usually trained on %images after Gaussian normalization (GN).
standardized images. To be aligned, the images introduced in $\S$\ref{sec.tsimage} should be normalized with a controlled mean and standard deviation, as did by \cite{gong2021ast} on spectrograms. In particular, as Heatmap is built on raw time series values, the commonly used Instance Normalization (IN) \cite{kim2022reversible} can be applied on the time series as suggested by VisionTS \cite{chen2024visionts} since IN share similar merits as Standardization. %although min-max normalization was used by \cite{karami2024timehr,zeng2021deep}.
Using Line Plot requires a proper range of y-axis. In addition to rescaling time series %by min-max or GN
\cite{zhuang2024see}, ViTST \cite{li2023time} introduced several methods to remove extreme values from the plot. GAF requires min-max normalization on its input, as it transforms time series values withtin $[0, 1]$ to polar coordinates ({\em i.e.}, arccos). In contrast, input to RP is usually normalization-free as an $\ell_{2}$ norm is involved in Eq.~\eqref{eq.rp} before thresholding.%for a comparison with a threshold.

\vspace{0.2cm}

\noindent{\bf Image Alignment.} When using pre-trained models, it is imperative to fit the image size to the input requirement of the models. This is especially true for Transformer based models as they use a fixed number of positional embeddings to encode the spacial information of image patches. For 3-channel RGB images such as Line Plot, it is straightforward to meet a pre-defined size by adjusting the resolution when producing the image. For images built upon matrices such as Heatmap, Spectrogram, GAF, RP, the number of channels and matrix size need adjustment. For the channels, one method is to duplicate a matrix to 3 channels \cite{chen2024visionts}, another way is to average the weights of the 3-channel patch embedding layer into a 1-channel layer \cite{gong2021ast}. For the image size, bilinear interpolation is a common method to resize input images \cite{chen2024visionts}. Alternatively, AST \cite{gong2021ast} %use cut and bilinear interpolation on
resizes the positional embeddings instead of the images to fit the model to a desired input size. However, the interpolation in these methods may either alter the time series or the spacial information in positional embeddings.

% single-channel (UTS), RGB channel (UTS), duplicate channels (UTS), multi-channel (MTS).

%Bilinear interpolation.

%Correlated variates are better to be spatially close to each other.

%\subsection{Pre-training}

\vspace{0.2cm}

\noindent{\bf Time Series Recovery.} As stated in $\S$\ref{sec.task}, tasks such as forecasting, imputation and generation requires predicting time series values. For models that predict pixel values of images, post-processing involves recovering time series from the predicted images. Recovery from Line Plots is tricky, it requires locating pixels that %correspond to
represent time series and mapping them back to the original values. This can be done by manipulating a grid-like Line Plot as introduced in \cite{yang2023your,yang2024vitime}, which has a recovery function. In contrast, recovery from Heatmap is straightforward as it directly stores the predicted time series values \cite{zeng2021deep,chen2024visionts}. Spectrogram is underexplored in these tasks and it remains open on how to recover time series from it. The existing work \cite{zeng2023pixels} uses Spectrogram for forecasting only with an MLP head that directly predicts time series. %predicts time series values.
GAF supports accurate recovery by an inverse mapping from polar coordinates to normalized time series \cite{wang2015imaging}. However, RP lost time series information during thresholding (Eq.~\ref{eq.rp}), thus may not fit recovery-demanded tasks without using an {\em ad-hoc} prediction head.


% Line Plot was regarded as matrices with rows and columns for mapping in \cite{sood2021visual}.


%\section{Tasks and Time Series Recovery}

%\subsection{Task-Specific Head}

% \subsection{Time Series Recovery}





% In summary, the task involves finding a function $f$ that maps each character position in the span $(i,j)$ of an entity mention in the string $S$ to a subset of roles from the taxonomy $R$.


\section{Corpus Description}

\subsection{Domains}
\label{sec:corpus_domains}
The articles used in the task cover the following domains: (1) \emph{Ukraine-Russia War}, which includes articles about the war that started in February 2022 when Russia launched a full-scale invasion of Ukraine and began occupying parts of the country, and (2) \emph{Climate Change}, which encompasses both climate change denial (characterized by rejecting, refusing to acknowledge, disputing, or fighting the scientific consensus on climate change), and climate change activism.
% \paragraph{Climate Change (CC):} encompasses both climate change denial (characterized by rejecting, refusing to acknowledge, disputing, or fighting the scientific consensus on climate change), and climate change activism,
% \paragraph{Ukraine-Russia war (URW):} includes articles about the war that started in February 2022 when Russia launched a full-scale invasion of Ukraine and began occupying parts of the country.

\subsection{Article Selection}
% For each language, articles were primarily selected from links provided by the \textbf{Europe Media Monitor}~\footnote{https://emm.newsbrief.eu/} on the basis of multilingual keyword based categories already developed by the project \jp{I am not sure we should refer to EMM here at submission time due to potential revealing the identity. Say instead, large-scale in-house news aggregation .... bla bla . 
% Also we did select specific sources, which needs to be mentioned too}, 

For each language, articles were primarily selected from links we obtained from a large-scale in-house news aggregation tool. We performed the first candidate article selection based on multilingual keyword-based filters and we perform several steps, which we follow by to enrich the selection to match the criteria discussed below. To select the articles, we followed these steps:

\paragraph{Initial Collection:}  
Articles were scraped and filtered based on criteria such as word count (e.g.,~only articles exceeding 250 words were selected). For duplicate articles, the version with the higher number of words was preferred.

\paragraph{Filtering:}  
Each article was manually reviewed to determine its relevance to the annotation task. The articles were categorized into four groups: Perfect Fit, Average Fit, Uncertain (requiring further validation by language coordinators), or Unfit (excluded from annotation). Only articles classified as \emph{Perfect Fit} and \emph{Average Fit} were considered for annotation. 
% \nn{In EN, we included very few not directly relevant but adjacent topics to make the inclusion a bit more complete which was important for ST2.}
Afterwards, we used a zero-shot classifier with selected key phrases, and a persuasiveness score using the persuasion technique classifier as described in \citet{nikolaidis-etal-2024-exploring}. These scores were used to further enrich the selection with relevant articles.

% \paragraph{Language-Specific Sources:}  
% In addition to JRC-provided\jp{anonymity issue again!} articles, we used additional sources to capture diverse perspectives:
Additional sources were also incorporated to ensure diversity of perspectives for two languages: for Hindi, we selected articles from mainstream and bias-specific outlets (e.g., NDTV, The Hindu, OpIndia), and for Portuguese, from newspapers and political websites (e.g., \emph{O Diabo}, Esquerda, Folha Nacional) that had more controversial opinion texts about the relevant topics.
% \begin{itemize}
%     % \item \textbf{Bulgarian (BG):} Articles were selected both from mainstream and alternative media outlets.
%     % \jp{this is entirely unclear what this means} \dd{Nothing more to be added for Bulgarian}
%     % \nn{I assume texts == posts.}
%     \item \textbf{Hindi (HI):} Articles were selected from mainstream and bias-specific outlets (e.g., NDTV, The Hindu, OpIndia).
%     \item \textbf{Portuguese (PT):} Articles were sourced from newspapers and political websites (e.g., \emph{O Diabo}, Esquerda, Folha Nacional) which had more controversial opinion texts about the relevant topics.
%     % \jp{add urls? say something about them?}.
%     %\item \textbf{Russian (RU):} Articles were filtered based on loaded language and relevant keyword filters.
%     % \jp{in what sense? more loaded language means more suitable? Clarify}
%     % \nn{ I would remove RU here alltoghether, since I described the general approach above.}
%     % \nn{Please avoid references to EMM  
%     % due to anonymity concerns.}
    
% \end{itemize}

% \jp{The description above is somewhat unclear, ie. how and why were the additional sources selected?}
% \nn{On EN we did not follow any more steps than that.}


\subsection{Annotation Process}

Given that our corpus contains articles in five languages, we had one annotation team per language, each led by a language coordinator. Each language team included 3 to 5 annotators with prior experience in linguistics, social science, international relations, or prior annotation work. The annotators studied our detailed guidelines, attended live demonstrations, and completed real-time annotation exercises. During weekly meetings, teams clarified any uncertainties, resolved conflicts, improved consistency, and revised the annotation guidelines.
% \jp{and tuned the guidelines?} Yes

Each article was annotated by two annotators. Designated curators, often language coordinators or experienced annotators, reviewed and consolidated all annotations. They resolved the discrepancies through discussions with the respective teams. Over time, as the annotation quality improved, the curators reduced the frequency of checks but continued to perform random quality checks to ensure that annotations were of high quality. We used the Inception tool~\cite{tubiblio106270} for annotating and curating the corpus. See Appendix~\ref{sec:annotation_tool} for more details.
% \jp{consolidated?}\jp{annotations?} yes.

%Some challenges arise during annotation. Annotators initially struggled with spurious labels due to the annotation bias surrounding the topics under consideration. For example, the preconceived notion that \emph{entity A} was a natural offender (and not \emph{entity B}) in an ongoing conflict may cause annotation bias. Annotators also found richer content for URW-related topics due to the abundance of relevant narratives and entities. In contrast, CC articles lacked clear entity roles and argumentative content, resulting in sparser annotations. Weekly meetings were instrumental in refining and converging to the current annotation guidelines outlined in Appendix \ref{sec:annotation_guidelines}. 

The annotation guidelines (Appendix \ref{sec:annotation_guidelines}) were refined during initial weekly meetings between language coordinators and annotators. From these guidelines, we emphasize key aspects of entity selection for annotation. We annotated traditional named entities, extending this to also include eponym-derived entities (e.g., \emph{Putin supporters}) and toponym-derived entities (e.g., \emph{Western countries}, as illustrated in Figure~\ref{fig:annotated_example_text2}). Additionally, we focused on annotating entities that are central to the narrative conveyed by the article. For example, in Figure~\ref{fig:annotated_example_text2}, entities such as \emph{Ulan-Ude Aviation Plant} and \emph{Rossiya 24 TV} were not annotated because they were not considered central to the narrative. For a detailed explanation of how centrality was defined, refer to the guidelines in Appendix~\ref{sec:annotation_guidelines}.



\begin{table*}[ht]
\small
\centering
\resizebox{0.9\textwidth}{!}{%
\begin{tabular}{c|rrrrrrrrr|rrrr}
\toprule
% Lang. & \#DOC & \#PAR & \#SEN & \#WORD & \#CHAR & AVG_p & AVG_s & AVG_w & AVG_c & \#ENT (UNIQ) & \#ANN & AVG_e & AVG_a\\
\multicolumn{1}{c|}{Lang.} & 
\multicolumn{1}{c}{\#DOC} & 
\multicolumn{1}{c}{\#PAR} & 
\multicolumn{1}{c}{\#SEN} & 
\multicolumn{1}{c}{\#WORD} & 
\multicolumn{1}{c}{\#CHAR} & 
\multicolumn{1}{c}{AVG\textsubscript{p}} & 
\multicolumn{1}{c}{AVG\textsubscript{s}} & 
\multicolumn{1}{c}{AVG\textsubscript{w}} & 
\multicolumn{1}{c|}{AVG\textsubscript{c}} & 
\multicolumn{1}{c}{\#ENT} & 
\multicolumn{1}{c}{\#ANN} & 
\multicolumn{1}{c}{AVG\textsubscript{e}} & 
\multicolumn{1}{c}{AVG\textsubscript{a}}\\
\midrule
BG    & 274    & 2.7K  & 5.0K  & 104K  & 584K   & 9.8  & 18.5 & 380.6 & 2129.8 & 656 (179)     & 742   & 2.4 & 2.7 \\
EN    & 229    & 3.7K  & 5.3K  & 131K  & 705K   & 16.2 & 23.1 & 571.3 & 3080.7 & 775 (413)     & 843   & 3.4 & 3.7 \\
HI    & 377    & 3.8K  & 8.7K  & 200K  & 947K   & 10.2 & 23.0 & 530.1 & 2513.0 & 2,609 (724)   & 3,030 & 6.9 & 8.0 \\
PT    & 337    & 3.5K  & 5.4K  & 150K  & 822K   & 10.5 & 15.0 & 445.0 & 2439.2 & 1,365 (440)   & 1,438 & 4.1 & 4.3 \\
RU    & 161    & 0.7K  & 2.0K  & 42K   & 257K   & 4.5  & 12.7 & 261.1 & 1599.1 & 451 (265)     & 477   & 2.8 & 3.0 \\\midrule
Total & 1,378  & 14.5K & 26.4K & 627K  & 3,316K & 10.5 & 19.2 & 455.0 & 2406.3 & 5,856 (1,962) & 6,530 & 4.2 & 4.7 \\
\bottomrule
\end{tabular}
}
\caption{Corpus statistics showing total number of documents (\#DOC), paragraphs (\#PAR), sentences (\#SEN), words (\#WORD), and characters (\#CHAR) by language. The averages (AVG\textsubscript{p}), (AVG\textsubscript{s}), (AVG\textsubscript{w}), and (AVG\textsubscript{c}) refer to the average number of paragraphs, sentences, words, and characters per document, respectively. The table also shows the total number of annotated entity mentions (\#ENT) accompanied with unique counts in parentheses, the total number of annotations (\#ANN), the average number of entity mentions per document (AVG\textsubscript{e}), and the average number of annotations per document (AVG\textsubscript{a}).}
\label{tab:corpus_stats}
\end{table*}

% \begin{table}[h]
% \small
% \centering
% \resizebox{1\columnwidth}{!}{%
% \begin{tabular}{llllll}
% \toprule
% Language & Entity Mentions (unique) & Annotations & AVG_e & AVG_f \\
% \midrule
% BG & 656    (179) & 742 & 2 & 3 \\
% EN & 775    (413) & 843 & 3 & 4 \\
% HI & 2,609  (724) & 3,030 & 7 & 8 \\
% PT & 1,365  (440) & 1,438 & 4 & 4 \\
% RU & 451    (265) & 477 & 3 & 3 \\ \midrule
% ALL & 5,856 (1,962) & 6,530 & 4 & 5 \\
% \bottomrule
% \end{tabular}
% }
% \caption{Summary statistics of entity mentions, unique entities, and fine-grained roles by language. AVG}
% \label{tab:corpus_stats}
% \end{table}

\begin{table}[!h]
\small
    \centering
\begin{tabular}{ccccccc}
\toprule
Lang. & EN & RU & BG & PT & HI & All\\
\hline
$\alpha$ & .460 & .436 & .733 & .467 & .461 & .558\\
\bottomrule
\end{tabular}
    \caption{Inter-annotator agreement computed using Krippendorff's $\alpha$.}
    \label{tab:iaa}
\end{table}

\begin{table}[!h]
    \small
    \centering
    \resizebox{\columnwidth}{!}{%
    \begin{tabular}{lllllll}
        \toprule
        \textbf{Freq.} & \textbf{1} & \textbf{2-5} & \textbf{5-10} & \textbf{10-20} & \textbf{20-50} & \textbf{50-500} \\
        \midrule
        count (\%) & 1513 (74) & 374 (18) & 76 (4) & 46 (2) & 22 (1) & 14 (1) \\
\bottomrule
\end{tabular}
}
\caption{Proportion of entities of a given frequency in the corpus.}
    \label{tab:proportion}
\end{table}


% \jp{to converge abd} Abd?



\subsection{Inter-Annotator Agreement}

To assess the inter-annotator agreement (IAA), we used Krippendorff's alpha. We compared the annotations at the span level: they were approximately matched if they shared at least 50\% of their length, to account for minor differences.
The IAA values are shown in Table~\ref{tab:iaa}. The results indicate a moderate agreement (above 0.45), which we consider acceptable due to the span-based nature of the task. We can see that the IAA is similar across the languages, except for Bulgarian, for which it is notably higher (0.73), which can be explained by the low count of entities in Table~\ref{tab:corpus_stats}.  %The overall IAA is a little below the recommended value of 0.667. Nevertheless, one needs to take into account, first, that given the complex annotation scheme, these values are not outside the range for tasks of similar complexity~\cite{}\jp{reference?}, second, that IAA reflects agreement at the level of annotation, therefore not accounting for the quality increase\jp{improvement?} after to the curation step.\jp{Do we say anything about entity mentions annotated only by single annotators? What fraction of the annotations do they cover? This would be interesting to know.}
%\nn{Indeed, but in EN at least it was a large portion and it could be quite tough to defend.}
\



%Please note that because the specific entities roles depend on a context which is not captured by the span, it is not possible to evaluate agreement per entity, neither to perform a cross-lingual check of coherency.\jp{but we know the context. Hence this sentence doe not read well :-)}
%\nn{I agree, it sounds like, we did not do a good job making a tool that could highlight that. I would remove the paragraph altoghther.}
\subsection{Corpus Analysis}

\subsubsection{Statistics}

Table \ref{tab:corpus_stats} provides overall statistics about the corpus, including a breakdown per language, the number of annotated entity mentions, the number of unique entities, as well as the number of annotations. Figure~\ref{fig:fine_roles_histogram} displays the distribution of the main and fine-grained roles in the corpus. We can see that there is a significant imbalance between the fine-grained roles, while the distribution of the main roles is relatively balanced. Notably, within the \emph{innocent} category, the majority of the instances, $83.6\%$, are labeled as \emph{victim}, with fewer occurrences of \emph{exploited}, \emph{forgotten}, and \emph{scapegoat} roles. More details about the proportions of main roles and fine-grained roles across different languages can be found in Figure~\ref{fig:proportions_of_roles} in Appendix~\ref{sec:appendix_stats}. 

% Table \ref{"eda/corpus_domain_distribution} outlines the distribution of our corpus across both domains CC and URW.


\begin{figure*}[htbp]
    \centering
    \includegraphics[width=0.9\textwidth]{images/fine_roles_histogram.pdf}
    \caption{Distribution of fine-grained roles color-coded according to the main role. For the fine-grained roles, the percentages inside the bars indicate the proportion of each fine-grained role relative to its corresponding main role category. The counts on top of the bars show the total occurrences of each fine-grained role. In the mini histogram, the percentages inside the bars reflect the distribution of the main roles, with the counts displayed above the bars. }
    \label{fig:fine_roles_histogram}
\end{figure*}

Table~\ref{tab:proportion} presents %\jp{maybe a graphical representation of the table would make more sense}
the number and the proportion of entities within a given frequency range considering exact string matching. We can see that 74\% of the entities were annotated only once, while 14 entities have been annotated more than 50 times
%\jp{Could we list them here, give examples, maybe in a separate table. We have space for that!}. \dd{I agree with listing at least some interesting ones. Like those with a lot of mentions}
These numbers consider only the surface string, not accounting for different grammatical and name variants and different languages. In Appendix~\ref{sec:appendix_stats}, we matched the most frequent entities accounting for these differences and presented detailed statistics at the level of entities. 




\subsubsection{Co-occurrence of Roles}

In our definition of entity framing, entity mentions can be assigned one or more fine-grained roles. Our corpus contains on average 1.12 annotations per entity mention. Out of the 1,378 articles, 353 articles contain 638 instances where at least one entity mention has multiple annotations. For this set of articles, the average and the maximum are 2.1 and 3 annotations per entity mention, respectively. We further observe that roles such as \emph{peacemaker} frequently accompany \emph{guardian}. Similarly, entities portrayed as \emph{scapegoats} are often framed as being \emph{exploited}. More details about the co-occurrence matrix for entity mentions with multiple annotations are shown in Figures \ref{fig:frequent_roles_combined} and \ref{fig:fine_roles_co_occurence_normalized} in Appendix~\ref{sec:appendix_stats}.
% Table \ref{tab:multiple_annotation_stats} shows per-language statistics for the  instances.

% \begin{table}
%     \small
%     \centering
%     \resizebox{0.5\columnwidth}{!}{%
% \begin{tabular}{l|rrc}
% \toprule
%  Lang. & \#ENT & AVG_a & MAX_a \\
% \midrule
% BG & 71 & 2.21 & 3 \\
% EN & 68 & 2.00 & 2 \\
% HI & 404 & 2.04 & 3 \\
% PT & 70 & 2.04 & 3 \\
% RU & 25 & 2.04 & 3 \\
% \bottomrule
% \end{tabular}
% }
% \caption{Statistics for entity mentions containing 2 or more annotations. 

% % (\#ENT) is the number of such instances, while (AVG\textsubscript{a}) and (MAX\textsubscript{a}) are computed on the number of annotations.
% }
%     \label{tab:multiple_annotation_stats}
% \end{table}



\section{Experiments}
We experimented with classifying entities into main roles and fine-grained roles. As we performed the entity framing annotations at the span level, we framed the problem as a multi-class multi-label classification task. Given an article, an entity mention, and the span offsets, the goal was to predict the framing of that entity mention. We provide two sets of baselines and experiments to benchmark state-of-the-art models, as well as to assess the complexity of the entity framing task. The first set evaluates fine-tuning multilingual transformer models in various settings, while the second set explores hierarchical zero-shot learning using LLMs. %We investigate the impact of multilingual data on the classification of entity framing and role portrayal.
% \jp{complexity of the task, not dataset}\jp{exploiting}\jp{performance} Agreed

\subsection{Fine-Tuning Pre-trained Multilingual Transformers}
For the first set of experiments, we designed our experiments to address the following aspects:
\begin{itemize}
    \item \textbf{Granularity of context}--predicting role labels for entity mentions using the full document or narrowing the model's focus to only look at the pertinent paragraph or sentence containing the entity of interest.
    \item \textbf{Multilingual} comparison of the performance in the monolingual setting versus the multilingual setting trained on data in all five languages: Bulgarian, English, Hindi, European Portuguese, and Russian.    
\end{itemize}

For both granularity-level classification and multilingual performance, we made predictions at two levels: main role (3 labels) and fine-grained role (22 labels). We fine-tuned the multilingual pre-trained transformer XLM-R \cite{conneau2020unsupervisedcrosslingualrepresentationlearning} and adapted its final layers for our tasks, applying $softmax$ for multiclass classification and $sigmoid$ for multi-label classification. To handle spans within potentially long documents, we addressed the 512-token limitation of XLM-R by narrowing the context to the paragraph or the sentence where the entity appeared. We constructed the input text using the following format:

\texttt{input = entity mention + [SEP] + title + [SEP] + context}.

In this setup, \texttt{[SEP]} is the separator token, and the context can vary based on the granularity level, ranging from the full text to just the paragraph or sentence containing the entity mention. We placed the entity mention first, followed by the title and context, to maintain consistent positional encodings for the entity mention across different inputs. We used Stanza \cite{qi2020stanza} for sentence splitting for all languages.

% For both multilingual performance and granularity-level classification, we classify entities at the main role level (3 labels) and fine-grained role level (22 labels). We used the multilingual pre-trained transformer XLM-R \cite{conneau2020unsupervisedcrosslingualrepresentationlearning} and adapted the final layers for our classification tasks (using softmax for multiclass and sigmoid for multilabel classification). Given the need to classify spans within potentially long documents, we addressed the 512-token limitation of transformer models by limiting the field of view to the specific context in which the entity appears at both paragraph and sentence levels. We process the input text by first including the entity mention, followed by the title of the article and the context with the separator token in between as shown here: \texttt{input = entity mention + [SEP] + title + [SEP] + context}, where context depends on the granularity we train on ranging from the full text to just the paragraph or sentence containing the entity mention of interest. The reason we arrange the entity mention followed by the title and the context in this order is to ensure the associated positional encodings for the entity mention do not vary across different inputs.



To further support span-level multi-label classification, we modified the output layer to include a $sigmoid$ activation and optimized the model using \emph{Binary Cross-Entropy loss}. This setup allowed the model to predict multiple overlapping roles for a given entity span. See Appendix~\ref{sec:appendix_experiments} for more details.




\subsection{Hierarchical Zero-Shot Learning with LLMs}

We experimented with two prompting approaches: \emph{single-step} and \emph{hierarchical multi-step}. The former aimed to predict both the main role and the fine-grained role simultaneously within a single prompt. It assumed that both tasks can be handled together, relying on the model's ability to process them in one go. On the other hand, the multi-step approach separated the prediction into two distinct stages. First, the main role was predicted, and based on that output, the fine-grained role was predicted in a second step, using the information from the initial prediction to refine the second task. This stepwise process involved an intermediate prediction, which allowed the model to focus on each task individually. See Appendix~\ref{sec:appendix_experiments}, and~\ref{sec:appendix_zeroshot} for more details on experimental settings and prompt structure.

% Please add the following required packages to your document preamble:
% \usepackage{multirow}
\begin{table*}[ht]
\centering
\resizebox{0.7\textwidth}{!}{%
\begin{tabular}{cccccccc}
\toprule
\multirow{2}{*}{\textbf{Train}} &
  \multirow{2}{*}{\textbf{Context}} &
  \multicolumn{2}{c}{\textbf{Main Role}} &
  \multicolumn{4}{c}{\textbf{Fine Grained Role}} \\ \cline{3-8} 
 &
   &
  \textbf{Accuracy} &
  \textbf{Balanced Accuracy} &
  \textbf{P} &
  \textbf{R} &
  \textbf{Micro F1} &
  \textbf{Macro F1} \\ \hline
\multirow{3}{*}{\textbf{M}} & DOC & .6010          & .5904          & --             & --             & --             & --             \\
                   & PAR & .7379          & .7385          & --             & --             & --             & --             \\
                   & SEN & .7179          & .7123          & --             & --             & --             & --             \\ \midrule
\multirow{3}{*}{\textbf{F}} & DOC & .7229          & .7235          & .3495          & .4446          & .3913          & .2306          \\
                   & PAR & \textbf{.7529} & \textbf{.7553} & .3649          & .4985          & .4213          & .2392          \\
                   & SEN & .7496          & .7503          & \textbf{.4195} & \textbf{.4492} & \textbf{.4339} & \textbf{.2529} \\ \bottomrule
\end{tabular}
}
\caption{Performance of entity framing across different granularity settings using XLM-R trained on the full multilingual dataset. Models are trained and evaluated on texts with varying context sizes: full document (DOC), paragraph (PAR), or sentence (SEN) containing the entity mention. The results cover models trained on main roles (M), fine-grained roles (F), and evaluated on either main roles, fine-grained roles, or both.}
\label{tab:xlmr_granularity_results}
\end{table*}

% \renewcommand{\arraystretch}{0.9} % Reduce row height
% \setlength{\tabcolsep}{2pt}       % Reduce column padding
\begin{table}[h]
\footnotesize
\centering

% Subtable (a)
\begin{subtable}[h]{0.9\columnwidth}
\small
\centering
\caption{Monolingual setting}
\begin{tabular}{ccccc}
\toprule
Lang. & P & R & Micro F1 & Macro F1 \\
\midrule
EN & $.1032$ & $.1313$ & $.1156$ & $.0435$ \\
BG & $.1056$ & $.5758$ & $.1784$ & $.0505$ \\
HI & $.3424$ & $.4495$ & $.3887$ & $.1740$ \\
PT & $.6124$ & $.6423$ & $.6270$ & $.1505$ \\
RU & $.1077$ & $.5227$ & $.1786$ & $.0437$ \\
\bottomrule
\end{tabular}

\end{subtable}
\hfill

% Subtable (b)
\begin{subtable}[h]{0.9\columnwidth}
\small
\centering
\caption{Multilingual setting}
\begin{tabular}{ccccc}
\toprule
Lang. & P & R & Micro F1 & Macro F1 \\
\midrule
All & $.3649$ & $.4985$ & $.4213$ & $.2392$ \\ \midrule
EN & $.1854$ & $.2828$ & $\mathbf{.2240}$ & $\mathbf{.1327}$ \\
BG & $.3030$ & $.3030$ & $\mathbf{.3030}$ & $\mathbf{.1349}$ \\
HI & $.3234$ & $.4951$ & $\mathbf{.3912}$ & $\mathbf{.2043}$ \\
PT & $.6259$ & $.7480$ & $\mathbf{.6815}$ & $\mathbf{.2040}$ \\
RU & $.4831$ & $.4886$ & $\mathbf{.4859}$ & $\mathbf{.2364}$ \\
\bottomrule
\end{tabular}
\end{subtable}

\caption{Results for multi-label fine-grained role classification with XLM-R trained on monolingual and multilingual data and evaluated at the paragraph level.}
\label{tab:xlmr_language_results}
\end{table}






% \begin{table}[h]
% \small
% \centering
% \resizebox{1\columnwidth}{!}{%
% % \begin{tabular}
% \begin{tabular}{ccccccc}

% \toprule
% Train & Test & Context & P & R & Micro F1 & Macro F1 \\
% \midrule
% M & M & DOC & $-$ & $-$ & $.6010$ & $.5904$ \\
% M & M & PAR & $-$ & $-$ & $.7379$ & $.7385$ \\
% M & M & SEN & $-$ & $-$ & $.7179$ & $.7123$ \\ \midrule
% F & M & DOC & $-$ & $-$ & $.7229$ & $.7235$ \\
% F & M & PAR & $-$ & $-$ & $.7529$ & $.7553$ \\
% F & M & SEN & $-$ & $-$ & $.7496$ & $.7503$ \\ \midrule
% F & F & DOC & $.3495$ & $.4446$ & $.3913$ & $.2306$ \\
% F & F & PAR & $.3649$ & $.4985$ & $.4213$ & $.2392$ \\
% F & F & SEN & $.4195$ & $.4492$ & $.4339$ & $.2529$ \\
% \bottomrule
% \end{tabular}
% }
% \caption{Performance of entity framing across different granularity settings using XLM-R trained on the full multilingual dataset. Models are trained and evaluated on texts with varying context sizes: full document (DOC), paragraph (PAR), or sentence (SEN) containing the entity mention. The results cover models trained on main roles (M), fine-grained roles (F), and evaluated on either main roles, fine-grained roles, or both.}
% \label{tab:xlmr_granularity_results}
% \end{table}

% M & M & D & $.601$ & $.601$ & $.601$    & $.5904$ \\
% M & M & P & $.7379$ & $.7379$ & $.7379$ & $.7385$ \\
% M & M & S & $.7179$ & $.7179$ & $.7179$ & $.7123$ \\
% F & M & D & $.7229$ & $.7229$ & $.7229$ & $.7235$ \\
% F & M & P & $.7529$ & $.7529$ & $.7529$ & $.7553$ \\
% F & M & S & $.7496$ & $.7496$ & $.7496$ & $.7503$ \\
% F & F & D & $.3495$ & $.4446$ & $.3913$ & $.2306$ \\
% F & F & P & $.3649$ & $.4985$ & $.4213$ & $.2392$ \\
% F & F & S & $.4195$ & $.4492$ & $.4339$ & $.2529$ \\

% \begin{table}[h]
% \small
% \centering

% % Subtable (a)
% \begin{subtable}[h]{\columnwidth}
% \centering
% \begin{tabular}{ccccccc}

% \toprule
% Train & Test & Context & Accuracy & Weighted Accuracy \\
% \midrule
% M & M & D & $-$ & $-$ & $.6010$ & $-$ \\
% M & M & P & $-$ & $-$ & $.7379$ & $-$ \\
% M & M & S & $-$ & $-$ & $.7179$ & $-$ \\ \midrule
% F & M & D & $-$ & $-$ & $.7229$ & $-$ \\
% F & M & P & $-$ & $-$ & $.7529$ & $-$ \\
% \bottomrule
% \end{tabular}
% \caption{}

% \end{subtable}
% \hfill

% % Subtable (b)
% \begin{subtable}[h]{\columnwidth}
% \centering
% \begin{tabular}{ccccccc}

% \toprule
% Context & P & R & Micro F1 & Macro F1 \\
% \midrule
% D & $.3495$ & $.4446$ & $.3913$ & $.2306$ \\
% P & $.3649$ & $.4985$ & $.4213$ & $.2392$ \\
% S & $.4195$ & $.4492$ & $.4339$ & $.2529$ \\
% \bottomrule
% \end{tabular}
% \caption{}


% \end{subtable}

% \caption{Performance of entity framing across different granularity settings using XLM-R trained on the full multilingual dataset. Models are trained and evaluated on texts with varying context sizes: full document (D), paragraph (P), or sentence (S) containing the entity mention. The results cover models trained on either main roles (M), or fine-grained roles (F), and evaluated on main roles show in table (a), or (b) fine-grained roles.}
% \label{tab:role_classification}
% \end{table}


\subsection{Results}

We report standard evaluation metrics, including \emph{micro-average precision}, \emph{recall}, and $F1$ score, along with the \emph{macro-average} $F1$ score for fine-grained roles to address class imbalance. We further provide \emph{accuracy} and \emph{balanced accuracy} for predictions on the main role granularity.

Table~\ref{tab:xlmr_granularity_results} shows the performance of XLM-R across different context granularities (document, paragraph, and sentence) and training configurations (main roles vs. fine-grained roles). For models trained and evaluated on main roles, paragraph-level contexts perform best, followed closely by sentence-level contexts, while document-level contexts perform the worst.  When models trained on fine-grained roles are evaluated on main roles, paragraph-level contexts again yield the best performance, with document and sentence-level contexts slightly behind. This indicates that training on fine-grained roles provides an advantage. For models trained and evaluated on fine-grained roles, sentence-level contexts perform best, followed by paragraph-level contexts, with document-level contexts showing the weakest performance. These results highlight that context granularity significantly impacts performance, with localized contexts outperforming document-level contexts for both main role and fine-grained role classification tasks.

Table~\ref{tab:xlmr_language_results} offers interesting insights into the performance of XLM-R when fine-tuned on monolingual and multilingual data for multi-label fine-grained role classification at the paragraph level. The monolingual setting exhibits varying performance across languages, with the highest scores achieved for Portuguese, while English, Bulgarian, and Russian show notably lower performance; the model's performance on Hindi is moderate. These differences stem from the quantity and quality of training data, linguistic variations, and the complexity of entity mentions across languages. The consistently low Macro F1 scores across all languages indicate difficulty in predicting rare roles. In contrast, the multilingual setting consistently outperforms the monolingual setting, demonstrating its ability to better capture diverse fine-grained roles through cross-lingual transfer.



% \jp{There is a missing discussion of the results in Table 4 and 5, which are not referred to in the text.} Added

\begin{table*}[t]
\small
\centering
% Subtable (a)
%\begin{subtable}[h]{\columnwidth}
\centering
\resizebox{0.87\textwidth}{!}{%

\begin{tabular}{ccccccccc}
\toprule
\multirow{2}{*}{\textbf{Method}} & \multirow{2}{*}{\textbf{Lang.}} & \multicolumn{2}{c}{\textbf{Main Role}} & \multicolumn{4}{c}{\textbf{Fine Grained Role}} & \multirow{2}{*}{\textbf{Cost (USD)}} \\
\cmidrule(lr){3-4}
\cmidrule(lr){5-8}
 & &  \textbf{Accuracy} & \textbf{Balanced Accuracy} & \textbf{P} & \textbf{R} & \textbf{Micro F1} & \textbf{Macro F1} &  \\
\midrule
\multirow{5}{*}{\shortstack{\textbf{Single-Step} \\ \textbf{LLM Prompting}}}
& EN & .8346 & .6756 & .2692 & .4632 & .3405 & .2171 & 0.7989 \\
& BG & .8065 & .7380 & .3725 & .5588 & .4471 & .3481 & 0.2751 \\
& HI & .6327 & .6247 & .2753 & .4000 & .3262 & .2196 & 2.4696 \\
& PT & .7812 & .7455 & .5167 & .6643 & .5813 & .2891 & 1.0200 \\
& RU & .7558 & .6719 & .3939 & .5843 & .4706 & .4644 & 0.7587 \\
& All & .7030 & .6957 & \underline{.3211} & \underline{.4726} & \underline{.3824} & \underline{\textbf{.3103}} & 5.3223 \\
\midrule
\multirow{5}{*}{\shortstack{\textbf{Multi-Step} \\ \textbf{LLM Prompting}}} 
& EN & .8031 & .6799 & .2887 & .4118 & .3394 & .2383 & 0.5130 \\
& BG & .8031 & .6799 & .4318 & .5588 & .4872 & .3601 & 0.5130 \\
& HI & .6367 & .6284 & .2676 & .2868 & .2769 & .1771 & 1.4581 \\
& PT & .8125 & .7882 & .3895 & .2643 & .3149 & .2498 & 0.5634 \\
& RU & .7442 & .6680 & .4118 & .4719 & .4398 & .3774 & 0.4769 \\
& All & \underline{.7053} & \underline{.7017} & .3051 & .3294 & .3168 & .2765 & \underline{\textbf{3.1852}} \\
\midrule \midrule
\multirow{5}{*}{\textbf{XLM-R}} 
& EN  & 0.6889 & 0.5276 & .1854 & .2828 & .2240 & .1327 & --\\
& BG  & 0.7333 & 0.5791 & .3030 & .3030 & .3030 & .1349 & --\\
& HI  & 0.7025 & 0.7046 & .3234 & .4951 & .3912 & .2043 & --\\
& PT  & 0.8957 & 0.8840 & .6259 & .7480 & .6815 & .2040 & --\\
& RU  & 0.8000 & 0.7604 & .4831 & .4886 & .4859 & .2364 & --\\
& All & \textbf{0.7529} & \textbf{0.7553} & \textbf{.3649} & \textbf{.4985} & \textbf{.4213} & .2392 & --\\
\bottomrule
\end{tabular}
}
\caption{Consolidated results comparing fine-tuning XLM-R and zero-shot learning with GPT-4o. The table shows performance and cost comparisons between single-step and multi-step LLM prompting approaches, where the highest scores between these two approaches across all languages are \underline{underlined}. The top results across all three methods and languages are highlighted in \textbf{bold}.}
\label{tab:promptingmethods}
\end{table*}

Table \ref{tab:promptingmethods} consolidates the results from fine-tuning XLM-R and hierarchical zero-shot learning, showing that the multi-step approach achieves slightly better performance on main role prediction compared to the single-step approach. 
% \jp{it can be inferred from the text earlier, but is not clear that this is the paragrpah version of the task. please mention it explicitly} 
This performance improvement in multi-step prompting can be attributed to the structured decomposition of the task. By explicitly isolating the main role prediction into a dedicated step, the model can focus on high-level role identification without the distraction of fine-grained details.
While the multi-step approach enhances performance for main role prediction, it underperforms compared to the single-step approach in predicting fine-grained roles. We hypothesize two potential reasons for this discrepancy:
\begin{enumerate}
\item \textbf{Error Propagation:} In the multi-step approach, the model first predicts the main role and then proceeds to predict fine-grained roles. Errors introduced during the main role prediction step can propagate to subsequent steps, thereby reducing the overall accuracy of fine-grained predictions.
\item \textbf{Loss of Joint Context:} The single-step approach enables the model to reason jointly about both main roles and fine-grained roles, allowing it to better capture dependencies between role labels. This integrated reasoning leads to more precise and consistent fine-grained predictions.
Another finding is that the multi-step approach is significantly more cost-effective than the single-step approach. This efficiency may stem from token efficiency, as multi-step prompts are designed to be more concise, with each step addressing a specific sub-task (e.g., main role followed by fine-grained role). Consequently, this results in fewer tokens per prompt.
\end{enumerate}

Table~\ref{tab:promptingmethods} additionally compares the performance of zero-shot approaches and XLM-R, both using similar-sized input contexts. We can see that XLM-R outperforms zero-shot methods on all evaluation measures except for Macro F1, where it shows the lowest performance among all approaches. We hypothesize that this discrepancy arises because XLM-R had limited training instances for rare roles, preventing effective learning for these categories. In contrast, zero-shot approaches do not rely on training data and thus are not constrained by this limitation.


% Streamlined Task Execution: While the Multi-Step approach involves multiple queries, the focused nature of each step reduces redundancy and unnecessary context. In contrast, Single-Step prompting processes all role predictions jointly, which can lead to more verbose and costly outputs.



% \jp{again something missing on XLM performance vis-a-vis LLMs} Done

\section{Conclusion and Future Work}
We presented a novel multilingual and hierarchical dataset for characterizing entity framing and role portrayal in news articles. Our dataset introduces a unique taxonomy inspired by storytelling elements, featuring 22 fine-grained archetypes nested within three main categories: \emph{protagonist}, \emph{antagonist}, and \emph{innocent}. The dataset covers 1,378 recent news articles in five languages (Bulgarian, English, Hindi, European Portuguese, and Russian), spanning two globally significant domains: the Ukraine-Russia War and Climate Change. Over 5,800 entity mentions have been thoroughly annotated with role labels, capturing nuanced portrayals. We evaluated the dataset using fine-tuned state-of-the-art multilingual transformer models and explored hierarchical zero-shot learning with LLMs at document, paragraph, and sentence level. Our experiments highlight the potential of multilingual representations and hierarchical approaches for entity-framing tasks. We intend to release the dataset to the community freely for research purposes. We hope that this dataset will serve as a valuable resource for developing methods and tools to enhance the analysis of entity portrayals in news media. 

In future work, we plan to extend the annotations to additional languages and explore other sources of text, such as social media posts. This expansion aims to provide a broader understanding of role portrayal across diverse linguistic and contextual settings. We also intend to perform a more in-depth dataset analysis to examine formulations such as central entity identification. We will also consider evaluating the potential biases in the dataset to ensure robustness in downstream tasks and end-user applications.

\section{Limitations}

\paragraph{Corpus} Our dataset focuses on two domains: the Ukraine-Russia War and Climate Change, and covers news articles in five languages (Bulgarian, English, Hindi, Portuguese, and Russian). While these domains and languages provide a diverse foundation for entity framing analysis, the corpus contains 1,378 articles and it should not be considered representative of all news coverage or media landscapes in any specific country. Additionally, the dataset is not perfectly balanced with respect to topics, entities, or languages. Moreover, human annotation of entity framing and role portrayal inevitably is subjective and annotators may subconsciously have biases that could influence the quality. Despite providing detailed annotation guidelines and conducting quality control measures such as double annotation and adjudication through the curation process, some level of subjectivity may remain in the dataset. 

\paragraph{Baseline Models}
Our reported experiments utilize state-of-the-art baselines covering a range of fine-tuned multilingual transformer models and hierarchical zero-shot learning with LLMs. However, we have not yet explored alternative architectures or advanced techniques such as few-shot, instruction-based evaluation, or multitask learning. Future work could investigate these approaches to improve model efficiency and performance. 

Additionally, our zero-shot learning experiments rely on OpenAI's GPT-4o, a closed-source model that is subject to changes over time and may be deprecated in the future. This dependency may impact the reproducibility and interpretability. To address these challenges, future research should prioritize improving open-source models to ensure greater accessibility, transparency, and reproducibility.





\section{Ethics and Broader Impact}
% Authors are encouraged to devote a section of their paper to concerns about the ethical impact of the work and to a discussion of broader impacts of the work, which will be taken into account in the review process. This discussion may extend into a 5th page (short papers) or 9th page (long papers). https://www.acm.org/code-of-ethics
%In addition, we provide a responsible NLP research checklist, which authors must complete as part of their paper submission. https://aclrollingreview.org/responsibleNLPresearch/
\paragraph{Biases}
Our dataset aims to capture a balanced range of perspectives on the Ukraine-Russia War and Climate Change, covering five languages: Bulgarian, English, Hindi, European Portuguese, and Russian. While our goal is to incorporate diverse news sources and viewpoints, achieving perfect balance is not always possible. Consequently, inherent biases in the original media sources may be present in the annotations. To reduce unwanted annotation biases, the corpus is annotated with clear instructions to annotators to focus strictly on the framing of entities, setting aside their personal opinions. All annotations are performed by subject-matter experts, and we did not use crowd-sourcing.

\paragraph{Intended Use and Misuse Potential}
The primary goal of this corpus is to facilitate research on entity framing, role portrayal, and media analysis. These tools can help researchers, journalists, and the general public identify framing patterns and biases in news content. However, there is a risk that the corpus could be misused for malicious purposes, such as manipulating news narratives. We urge users to employ this resource responsibly and remain aware of potential ethical risks associated with its misuse.

\paragraph{Environmental Impact}
The use of LLMs requires substantial computational power, contributing to carbon emissions. Even though we used LLMs in a zero-shot in-context-learning setting rather than training models from scratch, the LLMs still rely on GPUs for inference, which has an environmental impact.

\paragraph{Fairness}

Most of our annotators and curators come from the institutions of the co-authors of this manuscript and were fairly paid as part of their job duties. Few annotators were experienced analysts with full-time consulting roles and rates set by their contracting institutions. A fraction 
of the annotators were students from the respective
academic organizations. For two languages, a professional annotation company was contracted on rates based on country of residence. At the same time, some of the remaining annotators were researchers working primarily as linguists and lexicographers at their institute of affiliation and were all compensated according to local standards and their employment contracts.  


%\section*{Acknowledgments}

%This work is (partially) supported by the  Project SERICS (PE00000014) under the NRRP MUR program funded by the EU – NGEU. 

% Bibliography entries for the entire Anthology, followed by custom entries
% \bibliography{anthology,custom,shared}
% This must be in the first 5 lines to tell arXiv to use pdfLaTeX, which is strongly recommended.
\pdfoutput=1
% In particular, the hyperref package requires pdfLaTeX in order to break URLs across lines.

\documentclass[11pt]{article}

% Change "review" to "final" to generate the final (sometimes called camera-ready) version.
% Change to "preprint" to generate a non-anonymous version with page numbers.
\usepackage[preprint]{acl}
\usepackage{tablefootnote}
% Standard package includes
\usepackage{times}
\usepackage{latexsym}
\usepackage{pifont}

% For proper rendering and hyphenation of words containing Latin characters (including in bib files)
\usepackage[T1]{fontenc}
% For Vietnamese characters
% \usepackage[T5]{fontenc}
% See https://www.latex-project.org/help/documentation/encguide.pdf for other character sets

% This assumes your files are encoded as UTF8
\usepackage[utf8]{inputenc}

% This is not strictly necessary, and may be commented out,
% but it will improve the layout of the manuscript,
% and will typically save some space.
\usepackage{microtype}

% This is also not strictly necessary, and may be commented out.
% However, it will improve the aesthetics of text in
% the typewriter font.
\usepackage{inconsolata}

%Including images in your LaTeX document requires adding
%additional package(s)
\usepackage{graphicx}
\usepackage{color}
\usepackage{multirow}
\usepackage{amsmath}
\usepackage{array}
\usepackage{booktabs}
\usepackage{float} 
\usepackage{arydshln}
\usepackage{subcaption}
\usepackage{xspace}
\usepackage{makecell}
%\usepackage{tabularray}
%\usepackage{tikz}
%\newcommand*\circled[1]{\tikz[baseline=(char.base)]{
%            \node[shape=circle,draw,inner sep=2pt] (char) {#1};}}
            
\newcommand{\jz}[1]{{\color{red}{\bf{[JZ:]}} #1}}
\newcommand{\addexp}[1]{{\color{orange}{\bf{[AddExp:]}} #1}}
\newcommand{\sxfix}[1]{{\color{blue}#1}}

%\newcommand{\model}{RG$^2$-KBQA}
\newcommand{\model}{\textsc{SG-KBQA}\xspace}

% If the title and author information does not fit in the area allocated, uncomment the following
%
%\setlength\titlebox{<dim>}
%
% and set <dim> to something 5cm or larger.

% \title{Knowledge Base Question Answering with Generalizable Logical Form Generation}
%\title{Schema-Guided Generalizable Knowledge Base Question Answering}
\title{Beyond Seen Data: Improving KBQA Generalization Through Schema-Guided Logical Form Generation}
%JHL1: how about something cuter like "Beyond Seen Data: Improving KBQA Generalization Through Schema-Guided Logical Form Generation"


% Author information can be set in various styles:
% For several authors from the same institution:
% \author{Author 1 \and ... \and Author n \\
%         Address line \\ ... \\ Address line}
% if the names do not fit well on one line use
%         Author 1 \\ {\bf Author 2} \\ ... \\ {\bf Author n} \\
% For authors from different institutions:
% \author{Author 1 \\ Address line \\  ... \\ Address line
%         \And  ... \And
%         Author n \\ Address line \\ ... \\ Address line}
% To start a separate ``row'' of authors use \AND, as in
% \author{Author 1 \\ Address line \\  ... \\ Address line
%         \AND
%         Author 2 \\ Address line \\ ... \\ Address line \And
%         Author 3 \\ Address line \\ ... \\ Address line}

\author{
  Shengxiang Gao  \hspace{10mm} Jey Han Lau  \hspace{10mm} Jianzhong Qi \vspace{3mm} \\
  School of Computing and Information Systems, The University of Melbourne \\
  \texttt{shengxiang.gao1@student.unimelb.edu.au} \\ 
  \texttt{\{jeyhan.lau, jianzhong.qi\}@unimelb.edu.au}\\
}


% \author{First Author \\
%   Affiliation / Address line 1 \\
%   Affiliation / Address line 2 \\
%   Affiliation / Address line 3 \\
%   \texttt{email@domain} \\\And
%   Second Author \\
%   Affiliation / Address line 1 \\
%   Affiliation / Address line 2 \\
%   Affiliation / Address line 3 \\
%   \texttt{email@domain} \\}

%\author{
%  \textbf{First Author\textsuperscript{1}},
%  \textbf{Second Author\textsuperscript{1,2}},
%  \textbf{Third T. Author\textsuperscript{1}},
%  \textbf{Fourth Author\textsuperscript{1}},
%\\
%  \textbf{Fifth Author\textsuperscript{1,2}},
%  \textbf{Sixth Author\textsuperscript{1}},
%  \textbf{Seventh Author\textsuperscript{1}},
%  \textbf{Eighth Author \textsuperscript{1,2,3,4}},
%\\
%  \textbf{Ninth Author\textsuperscript{1}},
%  \textbf{Tenth Author\textsuperscript{1}},
%  \textbf{Eleventh E. Author\textsuperscript{1,2,3,4,5}},
%  \textbf{Twelfth Author\textsuperscript{1}},
%\\
%  \textbf{Thirteenth Author\textsuperscript{3}},
%  \textbf{Fourteenth F. Author\textsuperscript{2,4}},
%  \textbf{Fifteenth Author\textsuperscript{1}},
%  \textbf{Sixteenth Author\textsuperscript{1}},
%\\
%  \textbf{Seventeenth S. Author\textsuperscript{4,5}},
%  \textbf{Eighteenth Author\textsuperscript{3,4}},
%  \textbf{Nineteenth N. Author\textsuperscript{2,5}},
%  \textbf{Twentieth Author\textsuperscript{1}}
%\\
%\\
%  \textsuperscript{1}Affiliation 1,
%  \textsuperscript{2}Affiliation 2,
%  \textsuperscript{3}Affiliation 3,
%  \textsuperscript{4}Affiliation 4,
%  \textsuperscript{5}Affiliation 5
%\\
%  \small{
%    \textbf{Correspondence:} \href{mailto:email@domain}{email@domain}
%  }
%}

\begin{document}
\maketitle
\begin{abstract}
Knowledge base question answering (KBQA) aims to answer user questions in natural language using rich human knowledge stored in large KBs. As current KBQA methods struggle with unseen knowledge base elements at test time,
%State-of-the-art KBQA solutions are based on semantic parsing and have two core steps: (1) Generation: generate a sequence of structured query operators, and (2) Retrieval: retrieve KB elements (entities and relations). The operators and KB elements together form a structured query (so-called ``logical form'') over the KB to answer user questions. We observe that solutions starting with either step miss guidance from the other step, hence yielding suboptimal outcomes.
%To address this limitation, we propose a model named \textbf{\model} with a novel processing paradigm that consists of a \emph{\underline{g}enerative entity \underline{r}etrieval} module and a \emph{\underline{r}etrieval-guided logical form \underline{g}eneration} module. The generative entity retrieval module generates primitive logical forms based on user questions and relations extracted from the questions, to guide KB entity retrieval with higher accuracy. 
%The retrieval-guided logical form generation module then generates the final logical forms based on the KB elements extracted.
we introduce \textbf{\model}: a novel model that injects schema contexts into entity retrieval and logical form generation to tackle this issue. 
It uses the richer semantics and awareness of the knowledge base structure provided by schema contexts to enhance generalizability. 
%\sxfix{The schema contexts describes relationships between elements in the knowledge base, providing richer semantics and awareness of its structure.}
%JHL1: can we give an intuitive, high level explanation on the idea? just 1-2 lines max to capture the core idea
We show that \model\ achieves strong generalizability, outperforming state-of-the-art models on two commonly used benchmark datasets across a variety of test settings. 
%Our source code is available at \url{https://anonymous.4open.science/r/SG-KBQA-7895}. 
Our source code is available at \url{https://github.com/gaosx2000/SG_KBQA}.
%Code will be released upon paper publication.
%Our source code is available at \url{https://anonymous.4open.science/r/SG-KBQA-7895}. 
%JHL1: use anoymised github (https://anonymous.4open.science/)
%with a novel processing paradigm that consists of a \emph{\underline{g}enerative entity \underline{r}etrieval} module and a \emph{\underline{r}etrieval-guided logical form \underline{g}eneration} module. The generative entity retrieval module generates primitive logical forms based on user questions and relations extracted from the questions, to guide KB entity retrieval with higher accuracy. 
%Experimental results confirm the effectiveness of \model, which outperforms state-of-the-art models on two commonly used benchmark datasets GrailQA and WebQSP across a variety of test settings. 
\end{abstract}


\section{Introduction}



% These instructions are for authors submitting papers to *ACL conferences using \LaTeX. They are not self-contained. All authors must follow the general instructions for *ACL proceedings,\footnote{\url{http://acl-org.github.io/ACLPUB/formatting.html}} and this document contains additional instructions for the \LaTeX{} style files.

% The templates include the \LaTeX{} source of this document (\texttt{acl\_latex.tex}),
% the \LaTeX{} style file used to format it (\texttt{acl.sty}),
% an ACL bibliography style (\texttt{acl\_natbib.bst}),
% an example bibliography (\texttt{custom.bib}),
% and the bibliography for the ACL Anthology (\texttt{anthology.bib}).

{Knowledge base question answering} (KBQA) aims to answer user questions expressed in natural language with information from a {knowledge base}~(KB). This offers user-friendly access to rich human knowledge stored in large KBs such as Freebase~\cite{bollacker_freebase_2008}, DBPedia~\cite{auerDBpediaNucleusWeb2007} and Wikidata~\cite{vrandecic_wikidata_2014}, and it has broad applications in QA systems~\cite{zhou_commonsense_2018}, recommender systems~\cite{guo_survey_2022}, and information retrieval systems~\cite{jalota_lauren_2021}.

\begin{figure}[t]
\small
    \centering
    \includegraphics[width=\columnwidth]{figures/kbqa_example_new.png}
    \caption{Example of KBQA and SP-based solutions.}
    \label{fig:kbqa_example}
\end{figure}

State-of-the-art (SOTA) solutions often take a {semantic parsing} (SP)-based approach. They translate an input natural language question into a structured, executable form (AKA {logical form}~\cite{lan_survey_2021}), which is then executed to retrieve the question answer. Figure~\ref{fig:kbqa_example} shows an example. The input question, \textsf{Who is the author of Harry Potter}, is expressed using the \emph{S-expression}~\cite{gu_beyond_2021} (a type of logical form), which is formed by a set of functions (e.g., \textsf{JOIN}) operated over elements of the target KB (e.g., entity \textsf{m.078ffw} refers to book series \textsf{Harry Potter}, \textsf{book.author} a class of entities, and \textsf{book.literary\_series.author} a relation in Freebase).

% \sxfix{However, the rich semantics and complex structure of KBs lead to two key challenges: (1) KB elements mapping: how to learn a mapping between mentions of entities and relations in the input question to corresponding KB elements? (2) Executable logical form generation: how to generate a logical form that aligns with the question's semantics and adheres to the structural constraints (schema) of the KB?

A key challenge here is to learn a mapping between mentions of entities and relations in the input question to corresponding KB elements to form the logical form. Meanwhile, the mapping of KB element compositions has to adhere to the structural constraints (schema) of the KB. The schema defines entities' classes and the relationships between these classes within the KB. Take the KB subgraph in Figure~\ref{fig:kbqa_example} as an example, the relationship between the entity \textsf{Harry Potter} and the entity \textsf{J.K. Rolwing} is defined by the relation \textsf{book.literary\_series.author} between their respective classes (i.e., class \textsf{book.literary\_series} and class \textsf{book.author}).

\begin{figure}[t]
    \small
    \centering
    \includegraphics[width=\columnwidth]{figures/core_modules.png}
    \caption{Relation-guided entity mention detection and schema-guided logical form generation.}
    \label{fig:core_novelty}
\end{figure}

However, due to the vast number of entities, relations, classes, and their compositions, it is difficult (if not impossible) to train a model with all feasible compositions of the KB elements. For example, Freebase~\cite{bollacker_freebase_2008} has over 39 million entities, 8,000 relations, and 4,000 classes. Furthermore, some KBs (e.g., NED~\cite{mitchell_ned_2018}) are not static as they continue to grow. 

A few studies consider model generalizability to non-I.I.D. settings, where the test set contains schema items (i.e., relations and classes) or compositions that are unseen during training (i.e., \emph{zero-shot} and \emph{compositional generalization}, respectively). In terms of methodology, these studies typically use {ranking-based} or {generation-based} models. 
Ranking-based models~\cite{gu_beyond_2021, gu_dont_2023} retrieve entities relevant to the input question and then, starting from them, perform path traversal in the KB to obtain the target logical form by ranking. Generation-based models~\cite{shu_tiara_2022, zhang_fc-kbqa_2023} retrieve relevant KB contexts (e.g., entities and relations) for the input question, and then feed these contexts into a Seq2Seq model together with the input question to generate the logical form.



%~\cite{gu_arcaneqa_2022,shu_tiara_2022,ye_rng-kbqa_2022,gu_dont_2023,zhang_fc-kbqa_2023,faldu_retinaqa_2024}

%To solve the generalization problem, most existing KBQA approaches follow the retrieve-and-generate framework, which enhances logical form generation using retrieved KB elements (entities, relations, and classes).~\cite{shu_tiara_2022, faldu_retinaqa_2024, gu_dont_2023,zhang_fc-kbqa_2023,ye_rng-kbqa_2022, gu_arcaneqa_2022}. Despite the promising results achieved by these works, significant challenges remain: 

We observe that both types of models terminate their entity retrieval prematurely, such that each entity mention in the input question is mapped to only a single entity before the logical form generation stage. As a result, the logical form generation stage loses the freedom to explore the full combination space of relations and entities. This leads to inaccurate logical forms (as validated in our study).

%

To address this issue, our strategy is to defer entity disambiguation --- i.e., to determine the most relevant entity for an entity mention (Section~\ref{sec:literature}) --- to the logical form generation stage. This allows our model to explore a larger combination space of the relations and entities, and ultimately leads to stronger model generalizability because low-ranked (but correct) relations or entities would still be considered during generation.
%A larger search space brings new challenges to identify the correct combination.
We call our approach \model (\underline{s}chema-\underline{g}uided logical form generator for \underline{KBQA}). Concretely, \model\ follows the generation-based approach but with deferred entity disambiguation. As shown in Figure~\ref{fig:core_novelty}, it feeds the input question, the retrieved candidate relations and entities, plus their corresponding schema information (the domain and range of classes of relations and entities; Section~\ref{sec:method}) into a large language model (LLM)
%JHL1: are they LLMs? if so let's just use LLMs henceforth (and avoid introducing another acronym)
for logical form generation. The schema information reveals the connectivity between the candidate relations and entities, hence guiding the LLM to uncover their correct combination in the large search space. 
%JHL1: I struggle to understand figure 2 - i can see they are different, but not sure what the yellow box means, what the pink highlighted boxes mean. and what is schema information? can we have a toy example of what the input looks like to the LM? I think what's important in figure 2 is to give a concrete example of the input to the LM for our model; and if there's space, contrast that with the input in SOTA models


Further exploiting the schema-guided idea, we propose a relation-guided module for \model\  to enhance its entity mention detection from the input question. As shown in Figure~\ref{fig:core_novelty}, this module adapts a Seq2Seq model to generate logical form sketches based on the input question and candidate relations, where relations, classes, and literals are masked by special tokens, such that the entity mentions can be identified more easily without confusions caused by these elements. 
 %which extracts entity mentions from  generated by a generator that consumes the input question and the selected relations. The extracted entity mentions are further utilized to retrieve and select top-ranked candidate entities from the KB, guided by the schemas provided by the selected relations. 
 
 %Our approach leverages the selected schema items to guide the entity retrieval process and effectively incorporates the schema context through GenMD to achieve mention detection with a more global perspective. This significantly improves the accuracy of entity retrieval in compositional and zero-shot settings.


%\textbf{Entity retrieval (linking) remains challenging in zero-shot and compositional generalization settings.} Traditional methods first perform mention detection and then retrieve candidate entities from the KB. For each mention, a ranking model is used for entity disambiguation, selecting the candidate entity most relevant to the question for use in the subsequent generation stage. However, mention detection methods proposed in the existing literature (e.g., NER or span classification) often fail when faced with questions containing unseen schema items. This is because some schema items in the KB contain nouns that could potentially be recognized as entities. For example,\ldots. Unseen schema items introduce ambiguous information in the question, which confuses the model and makes it challenging to accurately identify entity mention boundaries.



%\textbf{Error propagation and lack of global reasoning in the disjointed traditional retrieve-and-generate framework}. Previous KBQA works adopt a disjoint retrieve-and-generate framework, where candidate entities are disambiguated before logical form generation to narrow down the search space~\cite{shu_tiara_2022, pang_survey_2022, zhang_fc-kbqa_2023, ye_rng-kbqa_2022,faldu_retinaqa_2024}. However, this approach fixes entity choices without considering their interactions with relations and other entities in the query, leading to locally optimal but globally inconsistent entity-relation selections. Moreover, errors in entity disambiguation propagate through the pipeline, misleading subsequent logical form generation.
% Furthermore, due to encountering new semantic relations or contexts that are not present in the training data, the model often fails to match the unseen schema items or compositions with correct entities in the KB. 

% \textbf{The completely decoupled retrieval and generation processes lead to error propagation through the pipeline.} To achieve stronger generalization capabilities, most existing KBQA approaches follow the retrieve-then-generate framework~\cite{shu_tiara_2022, faldu_retinaqa_2024, gu_dont_2023,zhang_fc-kbqa_2023,ye_rng-kbqa_2022, gu_arcaneqa_2022}. They employ an independent retrieval module to retrieve KB elements (e.g. entities, relations, classes) relevant to the input question before generating the target logical form. The retrieved KB elements are then leveraged to narrow down the search space and provide KB context, thereby enhancing the generalization capability. For example, some approaches incorporate the retrieved KB elements as additional inputs to a seq2seq model~\cite{shu_tiara_2022,zhang_fc-kbqa_2023,ye_rng-kbqa_2022}, while others use the retrieved entities as anchors to incrementally expand the logical form through path traversal in the KB~\cite{gu_dont_2023, gu_beyond_2021}. Although retrieval results can enhance the generalization ability of various logical form generation methods, incorrect retrievals can mislead the subsequent generation of logical forms. 

% To address the issues above and achieve a strong zero-shot and compositional generalization capability, we propose \model, a novel KBQA model that has two core modules: \emph{generative entity refinement} (GER) and \emph{refinement-guided logical form generation} (RLG). \model\ retrieves relations relevant to the question and generates \emph{logical form sketches} (that mask the relations and classes which may confuse the detection of the boundaries of entity mentions) by feeding the top-ranked relations and the question into a Seq2Seq model. 
% It then obtains the top-ranked entities from the KB based on the entity mentions in the generated logical form sketches, \emph{leveraging retrieved relations to enhance the zero-shot and compositional generalization of entity refinement}.

% The RLG module integrates entity and relation selection directly into the logical form generation process, to mitigate error propagation between the retrieval and generation stages. Specifically, for each relation included in the input, we provide its two connected classes to capture the semantic constraints of the KB schema. Similarly, for each entity, its associated class is provided to clarify its semantic role within the KB. By integrating these schema annotations into the input of the Seq2Seq model, our approach enables more accurate selection of entities and relations and generates logical forms that are more consistent with the underlying KB structure.





% \model~mitigate the error propagation between the retrieval and genration stages by defering both relation and entity disambiguation to the generation stage. }





% Specifically, it leverages the KB structural context by utilizing the classes to which entities belong and the classes connected by relations to provide connectivity between entities and relations. This context supports the model in selecting the correct combinations of entities and relations. A seq2seq model is then fine-tuned to transform the question, refined entities and relations, and class annotations into the target logical form.


% However, the primary source of errors in existing KBQA systems still lies in the failure of entity retrieval, which propagates through the pipeline and leads to errors in subsequent logical form generation.

% Recent SP-based KBQA approaches typically consist of two key steps: KB element retrieval and logical form generation~\cite{luo_chatkbqa_2024, ye_rng-kbqa_2022, shu_tiara_2022, faldu_retinaqa_2024, zhang_fc-kbqa_2023}. The retrieval of KB element mainly aims to retrieve the KB elements relevant to the input question. Then, these retrieved KB elements are then utilized to generate a complete and executable logical form (e.g., SPARQL, S-expression). However, collecting sufficient training data to cover all possible KB elements and their compositions that may appear in user queries is highly challenging, especially for large-scale KBs with a large number KB elements. \textbf{The broad coverage and combinatorial explosion require KBQA model to handle unseen KB elements (i.e. zero-shot generalization) and unseen compositions of them (i.e. compositional generalization), which remains a significant challenge.}

% It is important to note that in most existing SP-based KBQA systems, the majority of errors primarily arise from inaccuracies in the retrieval of KB elements, particularly in entity retrieval~\cite{gu_dont_2023,shu_tiara_2022,ye_rng-kbqa_2022,faldu_retinaqa_2024, zhang_fc-kbqa_2023}. These errors propagate to the subsequent logical form generation step which takes the retrieved KB elements as part of the input. Previous KBQA studies have proposed various methods for retrieving KB elements, aiming to separately identify the most relevant entities, relations, and classes for a given question. However, \textbf{the lack of consideration for the semantic relationships between KB elements in the question across these independent retrieval processes often leads to errors in the retrieval results.} This issue is particularly pronounced when handling questions involving unseen KB elements, where the independent retrieval processes may misidentify the same part of a question as different types of KB elements. 

%generation side ? thinking about the word limit for this intro....

% The retrieval of KB elements is a critical step in SP-based KBQA methods, aiming to retrieve the KB elements relevant to the input question. 

% SP-based KBQA methods not only need to retrieve KB elements related to the input query, but also generate the operators and functions that align with the semantic of user's query to form a complete and executable logical form. 

% Inspired by the strong generalization ability demonstrated by pre-trained language models (PLMs) across various NLP tasks, researchers have explored leveraging PLMs to address the generalization challenges in KBQA problem~\cite{shu_tiara_2022, ye_rng-kbqa_2022, zhang_fc-kbqa_2023, faldu_retinaqa_2024, gu_dont_2023}.

% To achieve strong generalization, recent SP-based KBQA studies primarily leverage pre-trained language models (PLMs) to retrieve KB elements in the input question and generate the final logical forms based on the retrieved KB elements. 

% Recent SP-based KBQA works have achieved promising results under the I.I.D. assumption, which posits a strong correspondence between the distribution of schema items (classes and relations) in the training data and the test data. However, this assumption does not hold due to user queries potentially involving schema items or novel compositions of them that have not been encountered in the training data. Collecting sufficient training data to cover all entities, schema items, and compositions of them is challenging, especially for large-scale KBs with numerous entities and schema items. 

% Figure~\ref{fig:kbqa_example} shows an example, where the logical form of the input question, `\textsf{Who is the author of Harry Potter}', is expressed using the \emph{S-expression}~\cite{gu_beyond_2021} (a type of logical form), which is formed by a set of functions (e.g., \textsf{JOIN}) operated over elements of the target KB (e.g., entity `\textsf{m.078ffw}' which refers to the book series `\textsf{Harry Potter}', class of entities `\textsf{book.author}', and relation `\textsf{book.literary\_series.author}' in the Freebase KB). More details about the example and the S-expression will be given in Section~\ref{sec:preliminary}. 

% Meanwhile, large language models (LLMs), such as Chat-GPT \citep{brown_language_2020} and LLaMA \citep{touvron_llama_2023}, have demonstrated strong results in various NLP tasks. Previous works have demonstrated the generalization capability of LLMs in understanding natural language and generating formal language \cite{rony_sgpt_2022, li_few-shot_2023}. These works have inspired researchers to enhance KBQA systems by leveraging LLMs as the semantic parser. However, the vast scale and complex structure of KBs present significant challenges for leveraging LLMs in real-world generalization scenarios within KBQA.

% recent SP-based studies~\cite{shu_tiara_2022, zhang_fc-kbqa_2023, faldu_retinaqa_2024, yu_decaf_2023} follow a \emph{sequence-to-sequence} (Seq2Seq)-based framework. They fine-tune 
% \emph{pre-trained language models} (PLMs) such as T5~\cite{raffel_exploring_2023} to translate input questions into logical forms, exploiting the strong semantic understanding capabilities of such models. Due to the special representations adopt by the KBs, the entity IDs, relation names, and class names are not necessarily directly translatable from the mentions of such elements in the input question, e.g., `\textsf{m.078ffw}' vs. `\textsf{Harry Potter}' in Figure~\ref{fig:kbqa_example}. An additional KB element retrieval module is commonly used by existing Seq2Seq SP-based methods, forming either \emph{generate-then-retrieve} (GnR) or \emph{retrieve-then-generate} (RnG) processing paradigms.
To summarise:

\begin{itemize}
    \item We introduce \model\ to solve the KBQA problem under non-I.I.D. settings, where test input contains unseen schema items or compositions during training.
    
    \item We propose to defer entity disambiguation to logical form generation, and additionally guide this generation step with corresponding schema information, allowing us to explore a larger combination space of relations and entities to consider unseen relations, entities, and compositions. We further propose a relation-guided module to strengthen entity retrieval by generating logical form sketches. 
    
    %introduce a generative mention retrieval method that leverages the context of retrieved schema items to address the generalization issues of entity retrieval in compositional and zero-shot settings from a global perspective.
    
    % \item We propose (1) a GER module that guides entity retrieval with logical form sketches that are generated based on retrieved relations, to achieve more accurate entity retrieval, and (2) an RLG module that guides logical form generation with the class contexts of the entities and relations, to achieve more generalizable logical form generation. 
    %\item We bridge the retrieval and generation stages by integrating entity and relation disambiguation into the logical form generation process through the incorporation of the KB schema, thereby reducing error propagation and generating more accurate logical forms that align with the KB structure.
    %novel entity retrieval approach — generative entity retrieval, to address the generalization challenges faced by existing KBQA entity retrieval methods.
    
    \item We conduct experiments on two popular benchmark datasets and find \model\ outperforming SOTA models on both datasets. In particular, on non-I.I.D GrailQA our model tops all three leaderboards for the overall, zero-shot, and compositional generalization settings, outperforming SOTA models by 3.3\%, 2.9\%, and 4.0\% (F1) respectively.

\end{itemize}


%JHL1: intro is mostly good; there's quite a bit of technical novelty to explain so it's not easy to write. Some suggestions to improve it:
%- simplify figure 1: since we mostly use it to explain the conversion process, we can probably drop the later half of the figure (i.e. we just need to keep the NLQ and logical form).
%- we need a figure that gives an example what 'schema information' and 'relation-guided module' and 'logical form sketches' look like. For someone outside this space it's far too abstract at the moment and they are just keywords to me. With that figure, use it to explain the core novelty of our model in the intro; focus on the intuition (leave the details to the later sections).
%-general comment: use emph or it more sparingly. we don't need to emph every new keyword that we introduce; use it when you really want the reader to notice the word - often they are not even special nouns (e.g. sometimes you might use emph to emphasise a negation (we do *not* consider...))

\section{Related Work}\label{sec:literature}

\paragraph{Knowledge Base Question Answering}

Most KBQA 
%(a.k.a. knowledge graph question answering~\cite{liu_knowledge_2023}, KGQA) %\footnote{This problem is also referred to as knowledge graph question answering~\cite{liu_knowledge_2023}. We use KBQA to refer to both problems, since most KBs are organized in a graph.}  
solutions use {information retrieval-based} (IR-based) or {semantic parsing-based} (SP-based) methods~\cite{wu_survey_2019,lan_survey_2021}. IR-based methods construct a question-specific subgraph starting from the retrieved entities (i.e., the \emph{topic entities}). They then reason over  the  subgraph to derive the answer. SP-based methods focus on transforming input questions into logical forms, which are then executed to retrieve answers. %Compared to IR-based methods, SP-based methods can produce a more interpretable reasoning process through converting the natural language questions into executable logical forms. Moreover, 
SOTA solutions are mostly SP-based, as detailed next.
%on popular benchmarks (e.g., GrailQA~\cite{gu_beyond_2021} WebQuestionsSP~\cite{yih_value_2016}) 

%JHL1: does SP-based = generation-based approach? does IR-based = ranking-based (in the intro)? it seems like these things are all the same, but we have two different terms; let be more consistent. Also, it doesn't look like IR/ranking based method is all that important to us, so let's drop that discussion in the intro and focus on contrasting our method to generation-based/SP-based models


% SP-based methods can be further categorized into ranking-based and generation-based methods~\cite{gu_arcaneqa_2022, lan_complex_2023}. 

% \jz{Two sentences to explain ranking-based and generation-based methods, rep.} \sxfix{Ranking-based methods perform path traversal and ranking in the KB, starting from the retrieved entities~\cite{gu_beyond_2021, gu_arcaneqa_2022,lan_topic_unit_2019, gu_dont_2023}. Generation-based methods directly transform the input question into the target logical form using a Seq2Seq model~\cite{luo_chatkbqa_2024, wang_no_2024, ye_rng-kbqa_2022}.}

\paragraph{KBQA under I.I.D. Settings}

%Benefiting from the powerful natural language understanding and logical form generation capabilities of Large Language Models (LLMs), 
Recent KBQA studies under I.I.D. settings fine-tune LLMs to map input questions to rough KB elements and generate approximate logical form drafts~\cite{luo_chatkbqa_2024, wang_no_2024}. %, exploiting LLMs' semantic capabilities to understand input natural language questions. 
The approximate (i.e., inaccurate or ambiguous) KB elements are then aligned to exact KB elements through a subsequent retrieval stage. These solutions often fail over test questions that refer to KB elements unseen during training. While we also use LLMs for logical form generation, we ground the generation with retrieved relations, entities, and  schema contexts, thus addressing the non-I.I.D. issue. 
%Our model first retrieves KB elements through schema-guided retrieval, and then uses the retrieved KB elements along with their schema context to guide the generation of logical forms. This approach enables better generalization to questions containing unseen knowledge base elements, while also enhancing performance under the i.i.d. setting.


\paragraph{KBQA under Non-I.I.D. Settings}

Studies considering non-I.I.D. settings can be largely classified into \emph{ranking-based} and \emph{generation-based} methods. 

Ranking-based methods start from retrieved entities, traverse the KB, and construct the target logical form by ranking the traversed paths.  
%Ranking-based methods reduce the search space based on the KB structure and the retrieved entities. 
\citet{gu_beyond_2021} enumerate and rank all possible logical forms within two hops  of retrieved entities, while \citet{gu_dont_2023} incrementally expand and rank paths from retrieved entities. % to obtain the target logical form. %They then obtain the candidate logical form that best matches the question by ranking. %They evaluated both supervised fine-tuning LMs (e.g., T5~\cite{raffel_exploring_2023}, BERT~\cite{devlin_bert_2019}) and few-shot in-context learning LLMs (e.g., Codex~\cite{chen_evaluating_2021}) as the partial logical form discriminators. 

Generation-based methods transform an input question into a logical form using a Seq2Seq model (e.g., T5~\cite{raffel_exploring_2023}).
They often use additional contexts beyond the question to augment the input of the Seq2Seq model and enhance its generalizability. For example,~\citet{ye_rng-kbqa_2022} use  top-5 candidate logical forms enumerated from retrieved entities as the additional context. 
\citet{shu_tiara_2022} further use top-ranked relations, \emph{disambiguated entities}, and classes (retrieved \emph{separately}) as the additional context. \citet{zhang_fc-kbqa_2023} use connected pairs of retrieved KB elements. 

Our \model\ is generation-based. We use schema contexts (relations and classes) from retrieved relations and entities, rather than separate class retrieval (as in \citet{shu_tiara_2022}) which could introduce noise. We also defer entity disambiguation to the logical form generation stage, thus avoiding error propagation induced by premature entity disambiguation without considering the generation context, as done in existing works outlined below. 

%employ an additional middle-grained component that converts the retrieved KB elements into connected pairs of KB elements. A Seq2Seq model then transforms the concatenation of the question, the retrieved KB element pairs, and logical form sketches generated based on the question into the target logical form. 

%The methods above retrieve KB elements to serve as additional contexts to enable the models to generalize in non-I.I.D. settings. 
%Existing ranking-based and generation-based methods limit the subsequent logical form search space through KB element retrieval, thereby improving generalization capability. 
%However, errors in knowledge base element retrieval can directly mislead the logical form generation stage. Moreover, the capacity of language models to reason about the correct element combinations and generate logical forms in search spaces that are noisy (contain ambiguous candidates) has not yet been explored. Therefore, our method introduces SGLG, which defers entity disambiguation and relation classification to the logical form generation stage. By incorporating the schema context of KB elements, it helps the language model reason and generate the correct and executable logical form from a global perspective.

% \jz{Need to say first what these studies haven't done.}
% By explicitly encoding the semantic connections between entities and relations within the schema structure, our RLG module performs entity disambiguation and relation classification during the logical form generation process, reducing error propagation in the traditional retrieve-then-generate framework.





% \paragraph{Semantic Parsing-Based Method} SP-based methods focus on transforming the input questions into structured, executable queries --- typically 
% \emph{logical forms} --- which are then executed over KBs to retrieve answers~\cite{lan_query_2020}. SP-based methods can be further categorized into step-wise ranking and Seq2Seq generation methods~\cite{lan_complex_2023}. Step-wise ranking methods~\cite{yih_value_2016, lan_query_2020, gu_dont_2023} incrementally expand a graph query (i.e., a logical form) with a search step to find possible paths in the KB at each step, followed by a ranking step to select the most relevant paths to be explored next. Seq2Seq generation methods~\cite{liang_querying_2021, yin_neural_2021} transform an input question into a logical form in one go using a Seq2Seq model. Our model follows the general idea of such methods. \jz{We detail these relevant studies next.?}


%due to the important practical applications of both knowledge bases~(KB) and the question answering~(QA) problem over KBs. We start by an overview of the studies on this problem (Section~\ref{subsec:related_work_kbqa}). Then, we focus on KBQA solutions using semantic parsing and Seq2Seq models, as our model also falls into this category (Section~\ref{subsec:related_work_sp}). We also cover techniques for entity retrieval in KBQA, to set the context of our GER module (Section~\ref{subsec:related_work_er}).

%\subsection{Knowledge Base Question Answering}\label{subsec:related_work_kbqa}

%KBQA aims to achieve a natural language-based user interface for non-expert users to interact with KBs without knowing specialized query languages such as SPARQL. 


% Several recent studies~\cite{lin_knowledge-injected_2024, yu_decaf_2023} propose \emph{knowledge injection-based} (KI-based) methods. These studies focus on training large \jz{or pre-trained?} language models to learn knowledge from the KB, while the trained models to can be fine-tuned to generate answers to \jz{Shengxiang to complete....} 
%inject knowledge from the KB directly into language models to by training language models with linearize knowledge triples from the KB. The trained models 
%are further fine-tuned on KBQA datasets, leveraging the acquired knowledge to answer questions. 

%\jz{What's the limitation of these methods? Can we compare with one of these in the experiments? Or why not?} 

% IR-based methods first retrieve a question-specific subgraph \jz{how? (e.g., by matching the entities in the kB with the entity mentions in the question?)}

% \paragraph{Information Retrieval-Based Methods} IR-based methods first retrieve entities related to the input question, one of which is selected as the \emph{topic entity}. The neighbors of the topic entity form a question-specific subgraph. A neural model is then used to score the nodes in the subgraph (i.e., the \emph{candidate answers}), and a score threshold is applied to produce the final answer set. The IR-based methods suffers from complex multi-hop questions, which often lead to retrieving large subgraphs that are difficult to score accurately~\cite{bordes_large-scale_2015, dong_question_2015, zhang_subgraph_2022, liu_knowledge_2023}. A latest study~\cite{ding_enhancing_2024} scores the connections between nodes and edges and expands the subgraph step by step accordingly, which helps reduce  
% the subgraph size. KICP~\cite{lin_knowledge-injected_2024} linearizes the KB triples into sentences to pre-train a language model, which is then fine-tuned on a KBQA dataset to serve as the answer scorer. Even with these enhancements, the IR-based methods typically have lower accuracy than the SP-based ones~\cite{lan_query_2020, gu_knowledge_2022}. 

%Compared to LMs pre-trained on natural language corpora, the LM pre-trained on KB corpus achieved a higher accuracy in selecting answers from candidate answers. } 

%\ssf{[generally IR-based methods produce lower accuracy, even the sota \cite{ding_enhancing_2024} achieve nearly 8\% F1 less then SP-based methods] [not sure to compara with IR-based methods] [move KI-based methods in IR-based methods]}

%\ssf{can delete this paragraph} \emph{evidence pattern retrieval} technique to reduce the nodes retrieved for the subgraph by \jz{formulating structural dependencies in the KB as evidence patterns [need a more intuitive description on what it does and why it offers better results]}, thereby achieving competitive KBQA accuracy \jz{Need to give a reason why we don't compare with it in the experiments}. \jz{However, in general, the IR-based methods produce lower accuracy~\cite{}, and hence we will not consider these methods in the rest of the paper.}
% even on complex KBQA questions. However, it is worth noting that the majority of state-of-the-art KBQA methods are are SP-based, as semantic parsing offers greater interpretability than IR-based approaches.

 \paragraph{KBQA Entity Retrieval}%\label{subsec:related_work_er} 
%\jz{The discussions above have focused on logical form generation of existing KBQA models.} 
KBQA entity retrieval typically has three steps: {entity mention detection}, {candidate entity retrieval}, and {entity disambiguation}. BERT~\cite{devlin_bert_2019}-based named entity recognition  is widely used for entity mention detection from input questions. %~\cite{gu_beyond_2021, zhang_fc-kbqa_2023}. %TIARA~\cite{shu_tiara_2022} treats entity mention detection as a span classification task. It scores question spans of varying lengths with BERT and takes the ones with top scores as entity mentions. 
%\ssf{tuning the threshold of detecting a candidate mention in order to improve coverage} \jz{tuning the threshold of detecting a candidate mention in order to improve coverage [Not sure how it works]}. 
To retrieve KB entities corresponding to entity mentions, the FACC1 dataset~\cite{gabrilovich_facc1_2013} is commonly used, which contains over 10 billion surface forms (with popularity scores) of Freebase entities. \citet{gu_beyond_2021} use the popularity scores for entity disambiguation, while \citet{ye_rng-kbqa_2022} and \citet{shu_tiara_2022} adopt a BERT reranker. %after pruning by popularity. 

%A key issue here is that NER systems may fail to identify  entity mentions precisely, which in turn fail entity retrieval afterwards. 
%Our \model\ model addresses this issue with the help of relation-augmented logical form sketches generation, which enable detecting entity mentions (and hence the entities) more accurately. 

\section{Preliminaries}\label{sec:preliminary}

\begin{figure*}[ht]
    \centering
    \includegraphics[width=1\linewidth]{figures/framework_new.png}
    \caption{Overview of \model. The model has two stages: \emph{retrieval} and \emph{generation}.  In the retrieval stage, we first retrieve and rank candidate relations based on the input question $q$ (\textcircled{1}). Using $q$ and the top-ranked candidate relations $R_q$, we generate logical form sketches and extract entity mentions from them  (\textcircled{2}). Based on the entity mentions and retrieved relations, we retrieve candidate entities from the KB  (\textcircled{3}) and rank them (the top-ones being $E_q$, \textcircled{4}). In the generation stage, $q$, $R_q$, $E_q$, and their class contexts, are fed into a fine-tuned language model for logical form generation (\textcircled{5}). Here, the colored modules come with our new design.}  
    \label{fig:framwork}
\end{figure*}

A graph structured-KB $\mathcal{G}$ is composed of a set of relational facts $\{ \langle s, r, o \rangle |s \in \mathcal{E}, r \in \mathcal{R}, o \in \mathcal{E} \cup \mathcal{L}\}$ and an ontology $\{ \langle c_d, r, c_r \rangle |c_d, c_r \in \mathcal{C}, r \in \mathcal{R} \}$.
Here, $\mathcal{E}$ denotes a set of entities, $\mathcal{R}$ denotes a set of relations, and $\mathcal{L}$ denotes a set of literals, e.g., textual labels, numerical values, or date-time stamps. In a relational fact $\langle s, r, o \rangle$, $s \in \mathcal{E}$ is the 
\textit{subject}, $o \in \mathcal{E} \cup \mathcal{L}$ is the \textit{object}, and $r \in \mathcal{R}$ represents the relationship between the \textit{subject} and the \textit{object}.

The ontology defines the rules governing the composition of relational facts within $\mathcal{G}$. In its formulation,  $\mathcal{C}$ denotes a set of classes, each of which defines a set of entities (or literals) sharing common properties (relations). Note that an entity can belong to multiple classes. 
In an ontology triple $\langle c_d, r, c_r \rangle$, $c_d$ is called a \textit{domain class}, and it refers to the class of  subject entities that satisfy relation $r$; 
$c_r$ is called the \textit{range class}, and it refers to the class of object entities or literals satisfying $r$. Each ontology triple can be instantiated as a set of relational facts. In Figure~\ref{fig:kbqa_example}, \textsf{<book.literary\_series, book.literary\_series.author, book.author>} is an ontology triple. An instance of it is \textsf{<Harry Potter, book.literary\_series.author, J.K. Rowling>}, where \textsf{Harry Potter} is an entity that belongs to class \textsf{book.literary\_series}.

%JHL1: great technical writing above - this is all very clear to me; I take back what I wrote earlier about simplifying Figure 1 earlier, looks like you need the latter half here

% We consider a graph structured-KB $\mathcal{G} = \{ \langle s, r, o \rangle |s \in \mathcal{E}, r \in \mathcal{R}, o \in \mathcal{E} \cup \mathcal{L}\}$ 
% stored in the form of triples $\langle s, r, o \rangle$, where $\mathcal{E}$ is a set of entities, $\mathcal{R}$ is a set of relations, and $\mathcal{L}$ is a set of literals, e.g., textual labels, numerical values, or date-time stamps. In every triple $\langle s, r, o \rangle$, $s \in \mathcal{E}$ is the 
% \textit{subject}, $o \in \mathcal{E} \cup \mathcal{L}$ is the \textit{object}, and $r \in \mathcal{R}$ represents the  relationship between the subject and the object. 

% Triples in the KB are instances of the \emph{ontology} defined for the KB~\cite{gu_knowledge_2022}, which in turn is also represented as triples  
% in the form of $\langle c_d, r, c_r \rangle$. Here, $c_d$ denotes the class of subject entities that satisfy relation $r$, and $c_r$ denotes the class of object entities or literals satisfying $r$. A \emph{class} defines a set of entities sharing common properties (relations), while an entity can belong to multiple classes. In Figure~\ref{fig:kbqa_example}, `\textsf{<book.literary\_series, book.literary\_series.author, book.author>}' is an ontology triple, while an instance of it is `\textsf{<Harry Potter, book.literary\_series.author, J.K. Rowling>}', where `\textsf{Harry Potter}' is an entity that belongs to class `book.literary\_series'.

\paragraph{Problem Statement} Given a KB $\mathcal{G}$ and a question $q$ expressed in natural language, i.e., a sequence of word tokens, {knowledge base question answering} (KBQA) aims to find a subset (the {answer set}) $\mathcal{A} \subseteq \mathcal{E} \cup \mathcal{L}$ of elements from $\mathcal{G}$ that --- with optional application of some aggregation functions (e.g., \textsc{count}) --- answers $q$. 

%find the answer $\mathcal{A} \text{ for } q $, where $\mathcal{A}$ is a set of entities or literals $\mathcal{A} \subseteq \mathcal{E} \cup \mathcal{L}$. $\mathcal{A}$ can serve directly as the answer or can be applied some aggregation functions (e.g. counting function) to get the answer.



\paragraph{Logical Form}
We solve the KBQA problem by translating the input question $q$ into a structured query that can be executed on $\mathcal{G}$ to fetch the answer set $\mathcal{A}$. Following previous works~\cite{shu_tiara_2022, ye_rng-kbqa_2022, gu_dont_2023, zhang_fc-kbqa_2023}, we use logical form as the structured query language, expressed with the \emph{S-expression}~\cite{gu_beyond_2021}.  
The S-expression offers a readable representation well-suited for KBQA. It uses set semantics where functions operate on entities or entity tuples without requiring variables~\cite{ye_rng-kbqa_2022}.  Figure~\ref{fig:kbqa_example} shows an example: the S-expression of the given question \textsf{Who is the author of Harry Potter?} is \textsf{(AND book.author (JOIN (R book.literary\_series.author) m.078ffw))}. This S-expression queries a set of entities that belong to the class \textsf{book.author} from the objects of triples whose subject entity is \textsf{m.078ffw} while the relation is \textsf{book.literary\_series.author}. More details about the  S-expression is in Appendix~\ref{sec:app_sexpression}.


\section{The \model\ Model}\label{sec:method}

As shown in Figure~\ref{fig:framwork}, \model\ follows the common structure of generation-based models. It has two overall stages: \emph{relation and entity retrieval} and \emph{logical form generation}. We propose novel designs in both stages to strengthen model generalizability.

In the relation and entity retrieval stage (Section~\ref{subsec:ger}), \model\ retrieves candidate relations and entities from KB $\mathcal{G}$ which may be relevant to the input question $q$. It starts with a BERT-based relation ranking model to retrieve candidate relations relevant to $q$. Together with $q$, the set of top-ranked candidate relations are fed into a novel, relation-guided  Seq2Seq model to generate logical form sketches that contain entity mentions while masking the relations and classes. We harvest the entity mentions and use them to retrieve candidate entities from $\mathcal{G}$. % with the help of an entity dictionary FACC1~\cite{Gabrilovich2013FACC1} (following existing studies~\cite{shu_tiara_2022,luo_chatkbqa_2024}, although other entity retrieval models can be used).  
We propose a combined relation-based strategy to prune the entities (as there may be many). The remaining entities are ranked by a BERT-based model, indicating their likelihood of being the entity that matches each entity mention. 

Leveraging relations to guide both entity mention extraction and candidate entity pruning enhances the model generalizability over entities unseen during training. This in turn helps the logical form generation stage to filter false positive matches for unseen relations or their combinations. 

In the logical form generation stage (Section~\ref{subsec:rlg}), 
\model\ feeds $q$, the top-ranked relations and entities (corresponding to each entity mention), and the schema contexts (i.e., domain and range classes of the relations and classes of the entities), into an adapted LLM to generate the logical form  and produce answer set $\mathcal{A}$. 

Our schema-guided logical form generation procedure is novel in that it takes (1) multiple candidate entities (instead of one in existing models) for each entity mention and (2) the schema contexts as the input. Using multiple candidate entities essentially defers  \emph{entity disambiguation}, which is usually done in the retrieval stage by existing models~\cite{shu_tiara_2022,gu_dont_2023}, to the generation stage, thus mitigating error propagation. This strategy also brings challenges, as the extra candidate entities (which are ambiguous as they often share the same name) may confuse the logical form generation model. We address the challenges with the schema contexts, which instruct the model the connectivity structures between the candidate entities and relations. The connectivity structures further help \model\ generalize to unseen entities, relations, or their combinations. 

%\sxfix{The classes shared by the entities and the relations indicate the connectivity of these elements. This guides our LM to to select the correct combination of entities and relations from the input's top-ranked entities and top-ranked relations, thereby generate executable logical forms. By deferring entity disambiguation to the logical form generation process and leveraging class information to provide semantic structural context from the KB, we reduce error propagation within the traditional retrieve-then-generate framework.} 

%retrieves the classes of the KB entities from the GER module. The relations between the classes offer contexts about the connectivity between the KB elements, to power the zero-shot and compositional generalization capability of \model\ to \emph{handle unseen KB elements and compositions}. We then fine-tune an open-sourced \emph{large language model} (LLM), which takes a question, the retrieved top-ranked entities and relations with class annotations as input to generate the target logical form.

\subsection{Relation and Entity Retrieval}\label{subsec:ger}

\paragraph{Relation Retrieval} For relation retrieval, we follow the schema retrieval model of TIARA~\cite{shu_tiara_2022}, as it has high accuracy. We extract a  set $R_q$ of top-$k_R$ (system parameter) relations with the highest semantic similarity to $q$. This is done by a BERT-based cross-encoder that learns the semantic similarity $\text{sim}(q, r)$ between $q$ and a relation $r \in \mathcal{R}$: %(recall that $\mathcal{R}$ is the set of relations of KB $\mathcal{G}$):  
\begin{equation}
\small
    \text{sim}(q,r)=\text{\large L\small INEAR}(\text{\large B\small ERT\large C\small LS}([q;r])), 
    \label{eqn:relation_retriever}
\end{equation} 
where `$;$' denotes concatenation.
This model is trained with the sentence-pair classification objective~\cite{devlin_bert_2019}, where a relevant question-relation pair has a similarity of 1, and 0 otherwise.




%\subsubsection{Relation-Augmented Logical Form sketch Parser}
\paragraph{Relation-Guided Entity Mention Detection}

%Previous work has employed Named Entity Recognition (NER) tools or regard entity mention detection as a span classification task to extract entity mentions in the questions \cite{gu_beyond_2021, shu_tiara_2022,zhang_fc-kbqa_2023}. However, detecting zero-shot entity mentions from short texts remains a challenging task. A common error in past mention detection methods is that certain components of relations in some questions are mistakenly detected as entity mentions, especially in questions containing unseen relations.
%To address the above problem, 
Given $R_q$, we propose a relation-guided logical form sketch parser to parse $q$ into a logical form sketch $s$. Entity mentions in $q$ are extracted from $s$. 

The parser is an adapted Seq2Seq model. The model input of each training sample takes the form of ``$q$ \textless relation\textgreater \text{ } $r_1;r_2;\ldots;r_{k_R}$'' ($r_i \in R_q$, hence ``relation-guided''). 
In the ground-truth logical form corresponding to $q$, we mask the relations, classes, and literals with special tokens `\textless relation\textgreater', `\textless class\textgreater', and `\textless literal\textgreater', to form the ground-truth logical form sketch $s$. Entity IDs are also replaced by the corresponding entity names (entity mentions), to enhance the Seq2Seq model's understanding of the semantics of entities.

%We fine-tune T5~\cite{raffel_exploring_2023} as the Seq2Seq model to transform a question into the corresponding logical form sketch. 

%We concatenate the question $q$ with the retrieved top-$k_R$ relations as the context  to form the input of the Seq2Seq model, i.e., the relation-augmented logical form sketch parser, to enhance the model understand of the input semantics. 
%The input of the model is in the form of ``$q$ \textless relation\textgreater \text{ } $r_1;r_2;\ldots;r_{k_R}$'' ($r_i \in R_q$). 

At model inference, from the output top-$k_L$ (system parameter) logical form sketches  (using beam search), we extract the entity mentions.

\paragraph{Relation-Guided Candidate Entity Retrieval}
We follow previous studies~\cite{gu_beyond_2021, shu_tiara_2022, faldu_retinaqa_2024, luo_chatkbqa_2024} and use an entity name dictionary FACC1~\cite{gabrilovich_facc1_2013} to map extracted entity mentions to entities (i.e., their IDs in KB), although other retrieval models can be used. Since different entities may share the same name, the entity mentions may be mapped to many entities. For pruning, existing studies use  popularity scores associated to  entities~\cite{shu_tiara_2022, ye_rng-kbqa_2022}. 

To improve the recall of candidate entity retrieval, we propose a combined pruning strategy based on both popularity and relation connectivity. As Figure~\ref{fig:candidate_entity_retrieval} shows, we first select the top-$k_{E1}$ (system parameter) entities for each entity mention based on popularity and then extract $k_{E2}$ (system parameter) entities from the remaining candidates that are connected to the retrieved relations $R_q$. Together, these form the candidate entity set $E_c$.
%for alias mapping of entity mentions \cite{.  A branch of works selects the entity with the highest popularity for each mention \cite{gu_beyond_2021, gu_arcaneqa_2022, zhang_fc-kbqa_2023}, while others choose the top-popularity entities and then perform entity disambiguation \cite{ye_rng-kbqa_2022, shu_tiara_2022, zhang_fc-kbqa_2023}. However, a popularity-based pruning strategy may exclude low-popularity ground-truth entities. 

\begin{figure}[t]
    \centering
    \includegraphics[width=\linewidth]{figures/candidate_entity_retrieval.png}
     \caption{Candidate entity retrieval for the mention `\textsf{aloha}'. The candidate entity in red is the ground-truth.}
    \label{fig:candidate_entity_retrieval}
\end{figure}


\paragraph{Entity Ranking} We follow existing works~\cite{shu_tiara_2022, ye_rng-kbqa_2022} to score and rank each candidate entity in $E_c$ by jointly encoding $q$ and the context (entity name and its linked relations) of the entity using a cross-encoder (like Eq.~\ref{eqn:relation_retriever}). %The context of a candidate entity includes . 
We select the top-$k_{E3}$ (system parameter) ranked entities for each mention as the entity set $E_q$ for each question.


\subsection{Schema-Guided Logical Form Generation}\label{subsec:rlg}

Given relations $R_q$ and entities $E_q$, we fine-tune an open-souce LLM (LLaMA3.1-8B~\cite{touvron_llama_2023} by default) to generate the final logical form.  

Before being fed into the model, each relation and entity is augmented with its schema context (i.e., class information) to help the model to learn their connections and generalize to unseen entities, relations, or their compositions.  The context of a relation $r$ is described by concatenating the relation's  domain class $c_d$ and range class $c_r$, formatted as ``[D] $c_d$ [N] $r$ [R] $c_r$''. For an entity $e$, its context is described by its ID (``$id_e$''), name (``$name_e$''), and the intersection of its set of classes $C_e$ and the set of all domain and range classes $C_R$ of all relations in $R_q$, formatted as ``[ID] $id_e$ [N] $name_e$ [C] class($C_e\cap C_R$)''.

As Figure~\ref{fig:framwork} shows, we construct the input to the logical form generation model by concatenating $q$ with the context of each relation in $R_q$ and the context of each entity in $E_q$. The model is fine-tuned with a cross-entropy-based objective:
\begin{equation}
\small
\mathcal{L}_{generator}=-\sum_{t=1}^n \log p\left(l_t \mid l_{<t}, q, K_q\right),
\end{equation}
where $l$ denotes a logical form of $n$ tokens and $l_t$ is its $t$-th token, and $K_q$ is the retrieved knowledge (i.e., relations and entities with contexts) for $q$. At inference, the model runs beam search to generate top-$k_O$ logical forms -- the executable one with the highest score is selected as the output. See Appendix~\ref{app:prompt} for a prompt example used for inference. 

%JHL1: I don't quite follow this last bit; dumb question: if we already have the logical forms, isn't it a straightforward thing to check whether an output logical form is executable, and just take the first executable one? It's a bit unclear why we need the enumeration step and also why we need a BERT reranker.
It is possible that no generated logical forms are executable. In this case, we fall back to following~\citet{shu_tiara_2022} and~\citet{ye_rng-kbqa_2022} and retrieve candidate logical forms in two stages: enumeration and ranking. During enumeration, we search the KB by traversing paths starting from the retrieved entities. Due to the exponential growth in the number of candidate paths with each hop, we start from the top-1 entity for each mention and searches its neighborhood for up to two hops. The paths retrieved are converted into logical forms. During ranking, a BERT-based ranker scores $q$ and each enumerated logical form $l$ (like Eq.~\ref{eqn:relation_retriever}). We train the ranker using a contrastive objective: 
\begin{equation}
\small
    \mathcal{L}=-\frac{\text{exp}(\text{sim}(q, l^*))}{\text{exp}(\text{sim}(q, l^*))+\sum_{l \in C_l \wedge l \neq l^*} \text{exp}(\text{sim}(q, l))},
\end{equation}
where $l^*$ is the ground-truth logical form and $C_l$ is the set of enumerated logical forms. We run the ranked logical forms from the top and return the first executable one. 

%JHL1: Good job on the technical writing; it was quite clear and I think I understood most of the model details. Thoughts: at a high level, I think the core novelty of the work is about the deferring the entity disambiguation to the generation step - this is a pretty cool idea. I think the intro gets this idea across quite well. But with that we run into a large search space, and introducing schema information into the input is a straightforward way to solve the issue - I think what we should do in the intro is to explain what schema information is (e.g. book.author is the schema information for m.078ffw), and then including an example input (to the LLM) in a figure and that should do it (figure 3 almost did this, but the input still just has some abstract variables, so isn't good enough). The relation-guided entity mention detection, on the other hand, feels like a very small touch and isn't that important to talk about in the intro. I'd drop it to increase clarity of our novel contribution.


\section{Experiments}\label{sec:experiment}
We run experiments to answer:
\textbf{Q1}:~How does \model\ compare with SOTA models in their accuracy for the KBQA task? 
%\textbf{Q2}: How does \model\ compare with the SOTA KBQA models  under I.I.D. settings? 
\textbf{Q2}:~How do model components impact the accuracy of \model? 
\textbf{Q3}:~How do our techniques generalize to other KBQA models? 

%is our generative entity retrieval module?
%\textbf{Q3}:~How effective is our retrieval-guided logical form generation module?



\subsection{Experimental Setup}

\paragraph{Datasets}
Following SOTA competitors~\cite{shu_tiara_2022, gu_dont_2023, zhang_fc-kbqa_2023}, we use two  benchmark datasets built upon Freebase.

\textbf{GrailQA}~\cite{gu_beyond_2021} is a dataset for evaluating the generalization capability of KBQA models. It contains 64,331 questions with annotated target S-expressions, including complex questions requiring up to 4-hop reasoning over Freebase, with aggregation functions including comparatives, superlatives, and counting. The dataset comes with training (70\%), validation (10\%), and test (20\%, hidden and only known by the leaderboard organizers) sets. In the validation and the test sets, 50\% of the questions include KB elements that are unseen in the training set (\textbf{zero-shot} generalization tests), 25\% consist of unseen compositions of KB elements seen in the training set (\textbf{compositional} generalization tests), and the remaining 25\% are randomly sampled from the training set (\textbf{I.I.D.} tests).

{WebQuestionsSP} (\textbf{WebQSP})~\cite{yih_value_2016} is a dataset for the I.I.D. setting. While our focus is on non-I.I.D. settings, we include results on this dataset to show the general applicability of \model. WebQSP contains 4,937 questions. 
More details of WebQSP are included in Appendix~\ref{app:WebQSP}.


%collected from Google query logs, including 3,098 questions for training and 1,639 for testing, each annotated with a target SPARQL query. %We convert each SPARQL query into the corresponding S-expression and extract 
%We follow GMT-KBQA~\cite{hu_logical_2022} to separate 200 questions from the training questions to form the validation set.

%\noindent \textbf{ComplexWebQuestions} (CWQ) \cite{talmor_web_2018} is an extended version of WebQSP with 34,689 questions in total. All the questions in it are derived from WebQSP but have been made more complex, incorporating more hops and constraints.

\begin{table*}[!ht]
\centering
\small
%(``I.I.D.'' means random samples from the training set; ``Compositional'' means unseen compositions of KB elements seen at training; ``Zero-shot'' means unseen compositions of unseen KB elements; ``Overall'' means a mix of the aforementioned). 
\begin{tabular}{
>{\centering\arraybackslash}m{0.1\linewidth}
>{}p{0.26\linewidth}
>{\centering\arraybackslash}m{0.04\linewidth}
>{\centering\arraybackslash}m{0.04\linewidth} 
>{\centering\arraybackslash}m{0.04\linewidth} 
>{\centering\arraybackslash}m{0.04\linewidth} 
>{\centering\arraybackslash}m{0.04\linewidth} 
>{\centering\arraybackslash}m{0.04\linewidth} 
>{\centering\arraybackslash}m{0.04\linewidth} 
>{\centering\arraybackslash}m{0.04\linewidth} }
\toprule
\multicolumn{1}{l}{\textbf{}} & \textbf{} & \multicolumn{2}{c}{\textbf{Overall}} & \multicolumn{2}{c}{\textbf{I.I.D.}} & \multicolumn{2}{c}{\textbf{Compositional}} & \multicolumn{2}{c}{\textbf{Zero-shot}} \\ \cline{3-10}
\multicolumn{1}{l}{} & \rule{0pt}{10pt}\textbf{Model} & \textbf{EM} & \textbf{F1} & \textbf{EM} & \textbf{F1} & \textbf{EM} & \textbf{F1} & \centering \textbf{EM} & \textbf{F1} \\ \midrule
\multirow{7}{*}{\begin{tabular}[c]{@{}c@{}} SP-based \\(SFT) \\ \end{tabular}} & RnG-KBQA (ACL 2021) & 68.8 & 74.4 & 86.2 & 89.0 & 63.8 & 71.2 & 63.0 & 69.2 \\
 & TIARA (EMNLP 2022) & 73.0 & 78.5 & 87.8 & 90.6 & 69.2 & 76.5 & 68.0 & 73.9 \\
 & Decaf (ICLR 2023) & 68.4 & 78.7 & 84.8 & 89.9 & 73.4 & \underline{81.8} & 58.6 & 72.3 \\
 & Pangu (T5-3B) (ACL 2023) & 75.4 & \underline{81.7} & 84.4 & 88.8 & \underline{74.6} & 81.5 & 71.6 & \underline{78.5} \\
 & FC-KBQA (ACL 2023) & 73.2 & 78.7 & \underline{88.5} & \underline{91.2} & 70.0 & 76.7 & 67.6 & 74.0 \\
 & TIARA+GAIN (EACL 2024) & \underline{76.3} & 81.5 & \underline{88.5} & \underline{91.2} & 73.7 & 80.0 & \underline{71.8} & 77.8 \\
 & RetinaQA (ACL 2024) & 74.1 & 79.5 & - & - & 71.9 & 78.9 & 68.8 & 74.7 \\ \midrule
\multirow{3}{*}{\begin{tabular}[c]{@{}c@{}} SP-based \\(Few-shot) \\ \end{tabular}} 
 & KB-Binder (6)-R (ACL 2023) & 53.2 & 58.5 & 72.5 & 77.4 & 51.8 & 58.3 & 45.0 & 49.9 \\
 & Pangu (Codex) (ACL 2023) & 56.4 & 65.0 & 67.5 & 73.7 & 58.2 & 64.9 & 50.7 & 61.1 \\
 & FlexKBQA (AAAI 2024) & 62.8 & 69.4 & 71.3 & 75.8 & 59.1 & 65.4 & 60.6 & 68.3 \\ \midrule
\multirow{2}{*}{\centering \makecell{\textbf{Ours} \\ (SFT)}} &\textbf{\model} & \textbf{79.1} & \textbf{84.4} & \textbf{88.6} & \textbf{91.6} & \textbf{77.9} & \textbf{85.1} & \textbf{75.4} & \textbf{80.8} \\
 &\hspace{3pt} - Improvement & +3.6\% & +3.3\% & +0.1\% & +0.4\% & +4.4\% & +4.0\% & +5.0\% & +2.9\% \\ \bottomrule 
\end{tabular}
\caption{\emph{Hidden} test results (\%) on GrailQA (best results are in boldface; best baseline results are underlined; ``SFT'' means supervised fine-tuning; ``few-shot'' means few-show in-context learning).}
%JHL1: ours is SFT too right? in that case let's put that in the table
\label{tab:grailqa}
\end{table*}

\paragraph{Competitors} 
We compare with both IR-based and SP-based methods including the SOTA models. 

On GrailQA, we compare with models that top the leaderboard\footnote{https://dki-lab.github.io/GrailQA/}, 
including \textbf{RnG-KBQA}~\cite{ye_rng-kbqa_2022}, \textbf{TIARA}~\cite{shu_tiara_2022}, \textbf{DecAF}~\cite{yu_decaf_2023}, 
\textbf{Pangu}
%(using T5-3B for scoring partial logical forms; 
(previous {SOTA} as of 15th February, 2025)~\cite{gu_dont_2023},
%JHL1: give a date for the current SOTA (as this may change in the future)
\textbf{FC-KBQA}~\cite{zhang_fc-kbqa_2023}, \textbf{TIARA+GAIN}~\cite{shu_data_2024}, and \textbf{RetinaQA}~\cite{faldu_retinaqa_2024}.
We also compare with few-shot LLM  (training-free) methods: KB-BINDER (6)-R~\cite{li_few-shot_2023}, Pangu
%(using Codex for scoring partial logical forms)
~\cite{gu_dont_2023}, and FlexKBQA~\cite{li_flexkbqa_2024}. These models are SP-based. On the non-I.I.D. GrailQA, IR-based methods are uncompetitive and excluded.

On WebQSP, we compare with IR-based models \textbf{SR+NSM}~\cite{zhang_subgraph_2022}, \textbf{UNIKGQA}~\cite{jiang_unikgqa_2023}, and
\textbf{EPR+NSM}~\cite{ding_enhancing_2024}, plus SP-based models  \textbf{ChatKBQA} ({SOTA})~\cite{luo_chatkbqa_2024} and \textbf{TFS-KBQA} ({SOTA})~\cite{wang_no_2024}, both of which use a fine-tuned LLM to generate logical forms.
We also compare with TIARA, Pangu, and FC-KBQA as above, which represent SOTA models using pre-trained language models (PLMs). 
Appendix~\ref{sec:app_baselines} details these models. The baseline results are collected from their papers or the GrailQA leaderboard (if available).


% On WebQSP, we compare with SOTA models including TFS-KBQA \cite{wang_no_2024}, ChatKBQA \cite{luo_chatkbqa_2024}, GMT-KBQA \cite{hu_logical_2022}, FC-KBQA \cite{zhang_fc-kbqa_2023}, and Pangu \cite{gu_dont_2023}. The first two methods are based on a fine-tuned LLM, while the latter deploys the LLM within a generate-then-retrieve framework. GMT-KBQA \cite{hu_logical_2022} and FC-KBQA \cite{zhang_fc-kbqa_2023} are representative state-of-the-art models utilizing PLM in retrieve-then-generate framework. Pangu \cite{gu_dont_2023} achieves state-of-the-art performance across multiple datasets, which is a generic framework for grounded language understanding.



\paragraph{Implementation Details}
% All our experiments are run on a machine with an NVDIA A100 GPU and 120 GB of RAM. We fine-tuned three \texttt{bert-base-uncased} models for a maximum of three epochs each, for relation retrieval, entity ranking, and fallback logical form ranking.
% For relation retrieval, we randomly sample 50 negative samples for each question to train the model to distinguish between relevant and irrelevant relations. 

% For each dataset, a \texttt{T5-base} model is fine-tuned for 5 epochs as our logical form sketch parser, with a beam size of 3 (i.e., $k_L = 10$) for GrailQA, 4 for WebQSP. In candidate entity retrieval, we use the same number (i.e., 10) of candidate entities per mention as the baselines~\cite{shu_tiara_2022, ye_rng-kbqa_2022}. The retrieved candidate entities for a mention consist of entities with the top-$k_{E1}$ popularity scores and $k_{E2}$ entities connected to the top-ranked relations in $R_q$, where $k_{E1} = 1$, $k_{E2} = 9$ for GrailQA, $k_{E1} = 3$, $k_{E2} = 7$ for WebQSP.

% We select the top-20 (i.e., $k_R$ = 20) relations and the top-2 (i.e., $k_{E3} = 2$) entities (for each entity mention) retrieved by our model. For WebQSP, we also use the entities obtained from the off-the-shelf entity linker ELQ~\cite{li_efficient_2020}. 

% Finally, we fine-tune \texttt{LLaMA3.1-8B} with LoRA~\cite{hu_lora_2021} for logical form generation. On GrailQA, \texttt{LLaMA3.1-8B} is fine-tuned for 1 epoch with a learning rate of $0.0001$. On WebQSP, it is fine-tuned for 15 epochs with the same learning rate (as it is an I.I.D. dataset where more epochs are beneficial). During inference, we generate logical forms by beam search with a beam size of 10 (i.e., $K_O = 10$). The generated logical forms are executed on the KB to filter non-executable ones. If none of the logical forms are executable, we check the candidate logical forms from the fallback procedures, and the result of the first executable one is returned as the answer set.
% %\jz{Any updates needed for this subsection?} 


% Our system parameters have been chosen empirically. While there are a few of them, their exact values do not have strong impact on the final model performance, and the choice of parameter values generalize well across  datasets. The same parameter values are used on both datasets. 
% \addexp{Add parameter study to appendix.}
All our experiments are run on a machine with an NVDIA A100 GPU and 120 GB of RAM. We fine-tuned three \texttt{bert-base-uncased} models for a maximum of three epochs each, for relation retrieval, entity ranking, and fallback logical form ranking.
%JHL1:  for these three models, can we ref to figure 3 and the module number in the figure?
For each dataset, a T5-base model is fine-tuned for 5 epochs as our logical form sketch parser. Finally, we fine-tune a LLaMA3.1-8B with LoRA~\cite{hu_lora_2021} for 5 epochs on GrailQA and 20 epochs on WebQSP to serve as the logical form generator. Our system parameters have been chosen empirically, and a parameter study is provided in Appendix~\ref{app:paramater_study}. More implementation details  are in Appendix~\ref{app:implemention_details}.




\paragraph{Evaluation Metrics}
On GrailQA, we report the exact match (\textbf{EM}) and \textbf{F1} scores, following the leaderboard. EM counts the percentage of test samples where the model generated logical form (an S-expression) that is semantically equivalent to the ground truth. F1  measures the answer set correctness, i.e., the F1 score of each answer set, average over all test samples. 
On WebQSP, we report the F1 score as there are no ground-truth S-expressions. 
%Following previous SP-based methods \cite{shu_tiara_2022, zhang_fc-kbqa_2023}, here hits@1 is calculated by randomly selecting one answer for each question 100 times and averaging the results. This approach is used because the answers obtained from SP-based methods are typically unordered.

\begin{table}[t]
\centering
\small
\setlength{\tabcolsep}{6pt}
\begin{tabular}{clc}
\toprule
 & \textbf{Model} & \textbf{F1}\\ \midrule
\multirow{3}{*}{\begin{tabular}[c]{@{}c@{}}IR-based\\ \end{tabular}}  
& SR+NSM (ACL 2022)     & 69.5 \\
& UniKGQA (ICLR 2023)  & 75.1    \\
& EPR+NSM (WWW 2024)  & 71.2   \\
 \midrule
\multirow{5}{*}{\begin{tabular}[c]{@{}c@{}}SP-based \\ (SFT) \\ \end{tabular}} 
& TIARA (EMNLP 2022)   & 76.7\\
& Pangu (T5-3B, ACL 2023) & 79.6\\
& FC-KBQA (ACL 2023)   & 76.9 \\
& ChatKBQA (ACL 2024) & 79.8\\ 
& TFS-KBQA (LREC-COLING 2024) & \underline{79.9}\\
\midrule
\multirow{3}{*}{\begin{tabular}[c]{@{}c@{}}SP-based \\(Few-shot)\\ \end{tabular}} 
& KB-Binder (6)-R (ACL 2023)   & 53.2\\
& Pangu (Codex) (ACL 2023)   & 54.5\\
& FlexKBQA (AAAI 2024)   & 60.6\\
\midrule
\multirow{2}{*}{\begin{tabular}[c]{@{}c@{}}\textbf{Ours} \\ (SFT)\end{tabular}} 
& \textbf{\model} & \textbf{80.3} \\
&\hspace{6pt} - Improvement & \multicolumn{1}{l}{+0.5\%} \\
% \cdashline{2-3}
% & \rule{0pt}{10pt} \hspace{6pt}w/o RG-EMD & 78.4 \\
% & \hspace{6pt} w/o RG-CER & 79.5 \\ 
% %& \hspace{6pt} w/o SG-ER & 78.3 \\ 
% &\hspace{6pt} w/o DED & 78.2 \\ 
% &\hspace{6pt} w/o SG-LF & 77.1 \\
\bottomrule
\end{tabular}
\caption{Test results (\%) on WebQSP (I.I.D.).}
\label{tab:webqsp}
\end{table}


\subsection{Overall Results (Q1)}
Tables~\ref{tab:grailqa} and~\ref{tab:webqsp} show the overall comparison of \model\ with the baseline models for GrailQA and WebQSP, respectively. \model\ shows the best results across both datasets. 

\paragraph{Results on GrailQA} 
On the overall hidden test set of GrailQA, \model\ outperforms the best baseline Pangu by 4.9\% and 3.3\% in the EM and F1 scores, respectively. Under the compositional and zero-shot generalization settings (both are non-I.I.D.), similar performance gaps are observed, i.e., 4.0\% and 2.9\% in F1 compared to the best baseline models, respectively. This validates that \model\ can extract relations and entities more accurately from the input question, even when these are unseen in the training set, and it creates more accurate logical forms to answer the questions. %\sxfix{baseline models do not use schema context of retrieved KB elements (other model use class as a set of retrieved KB elements. We use the class or schema context of each relation and entity to feed its schema (connectivity to other elements)) into our generator to guide a logical form generation that are consistent to the KB structure (schema)} 

The fine-tuned baseline models do not use relation semantics to enhance entity retrieval, and they either omit the class contexts in logical form generation or use these classes separately for retrieval. As such, they do not generalize as well in the non-I.I.D. settings. 
%which may be noisy and do not indicate the schema of the  the entities and relations. These  explain for their lower accuracy.
The few-shot LLM-based competitors are generally not very competitive, especially under the non-I.I.D. settings. This suggests that {the current generation of LLMs are unable to infer from a few input demonstrations the process of logical form generation from user questions}. Fine-tuning is still required.  

\paragraph{Results on WebQSP} 


On WebQSP, which has an I.I.D. test set, the performance gap of the different models are closer. Even in this case, \model\ still performs the best, showing its general applicability. Comparing with  TFS-KBQA (SOTA) and ChatKBQA, \model\ improves the F1 score by 0.5\%.  
Among IR-based methods, UniKGQA (SOTA) still performs substantially worse compared to \model. The lower performance of IR-based methods is consistent with existing results~\cite{gu_knowledge_2022}.
%which also reported lower performance from the IR-based methods.





\subsection{Ablation Study (Q2)}


\begin{table}[t]

\centering
\resizebox{\columnwidth}{!}{
\begin{tabular}{lccccc}
\toprule
\multirow{2}{*}{\rule{0pt}{28pt}\textbf{Model}}                  & \multicolumn{4}{c}{\textbf{GrailQA}}                                 & \textbf{WebQSP}  \\ \cmidrule{2-6} 
                                        & \textbf{Overall} & \textbf{I.I.D.} & \textbf{Comp.} & \textbf{Zero.} & \textbf{Overall} \\ \midrule
\textbf{\model}          & \textbf{88.5}    & \textbf{94.6}   & \textbf{84.6}  & \textbf{87.9}  & \textbf{80.3}    \\
\hspace{3pt}w/o RG-EMD & 85.3             & 92.4            & 80.2           & 84.3           & 78.4             \\
\hspace{3pt}w/o RG-CER & 86.5             & 92.1            & 81.1           & 86.3           & 79.5             \\
\hspace{3pt}w/o DED    & 87.8             & 94.0            & 82.4           & 87.2           & 78.2             \\
\hspace{3pt}w/o SC  & 79.2             & 92.9            & 77.4           & 73.9           & 77.1             \\ \bottomrule
\end{tabular}
}
\caption{Ablation study results (F1 score) on the validation set of GrailQA and the test set of WebQSP.}
\label{tab:ablation}
\end{table}

% \begin{table}[H]
% \small
% \centering
% \begin{tabular}{lcccc}
% \toprule
% \textbf{Model}           & \textbf{Overall} & \textbf{I.I.D.} & \textbf{Comp.} & \textbf{Zero.} \\ \midrule
% % TIARA           & 81.9   & 91.2  & 74.8  & 80.7  \\
% % FC-KBQA         & 83.8   & 91.5  & 77.3  & 83.1  \\
% % RetinaQA        & 83.3   & 91.2  & 77.5  & 82.3  \\ 
% % \midrule
% \textbf{\model}            & \textbf{88.5}   & \textbf{94.6}  & \textbf{84.6}  & \textbf{87.9}  \\
% \hspace{3pt}w/o RG-EMD   &   85.3     &    92.4   &   80.2   &    84.3  \\ 
% \hspace{3pt}w/o RG-CER   &   86.5     &    92.1   &   81.1    &   86.3    \\ 
% %\hspace{3pt}w/o SGER        & 86.4   & 92.3  & 82.4  & 85.5  \\
% \hspace{3pt}w/o DED      & 87.8  & 94.0  & 82.4  & 87.2  \\
% \hspace{3pt}w/o SG-LF    & 79.2   & 92.9  & 77.4  & 73.9  \\
% %\hspace{3pt}w/o Fallback LF \jz{to appendix} & 84.6 & 94.1 & 81.8 & 81.5 \\ 
% \bottomrule
% \end{tabular}
% \caption{Ablation study results (F1 score) on the validation set of GrailQA}
% \label{tab:ablation}
% \end{table}
%JHL1: I'd drop the baseline models, and move WebQSP results to this table. since it's about studying the impact of different components, I don't see why we need to include the baseline models

Next, we run an ablation study with the following variants of \model: \textbf{w/o~RG-EMD} replaces our relation-guided entity mention detection with SpanMD~\cite{shu_tiara_2022} which is commonly used in existing models~\cite{pang_survey_2022, faldu_retinaqa_2024}; \textbf{w/o~RG-CER} omits  candidate entities retrieved from the top relations; \textbf{w/o~DED} uses the top-1 candidate entity for each entity mention without deferring entity disambiguation; \textbf{w/o~SC} omits schema contexts from logical form generation. 
%JHL1: can we use shorter acronyms for these things? it's not obvious what RG, SG, etc all means anyway, why not just use a 2-3 letter acronym? Also, let's be consistent with the acronymn used in 5.4 (i see RG-CER vs. SG-ER), but they look like the same thing)

Table~\ref{tab:ablation} shows the results on the validation set of GrailQA and the test set of WebQSP. Only F1 scores are reported for conciseness, as the EM scores on GrailQA exhibit similar comparative trends and are provided in Appendix~\ref{app:ablation}.

% Table~\ref{tab:ablation} shows the results on the validation set of GrailQA \sxfix{and the test set of WebQSP}, benchmarking against baseline models with released code. We only show F1 scores. The EM scores exibit similar comparative patterns and are included in Appendix~\ref{app:ablation}.
%For WebQSP, the results are included in Table~\ref{tab:webqsp}.

All model variants have lower F1 scores than those of the full model, confirming the effectiveness of the model components. SG-KBQA w/o DED (with schema contexts) reduces the F1 scores across various generalization settings on both datasets, demonstrating the effectiveness of our DED strategy in reducing error propagation during the retrieval and generation stages. Furthermore, \model~w/o SC (with deferred entity disambiguation) has the most significant drops in the F1 score under the compositional (7.2) and zero-shot (14.0) generalization tests. It highlights the importance of schema contexts in constraining the larger search space introduced by DED and in generalizing to unseen KB elements and their combinations. Meanwhile, the lower F1 of \model~w/o RG-EMD emphasizes the capability of our relation-guided entity mention detection module in strengthening KBQA entity retrieval.
%JHL1: the ablation results is admittedly a little bit disappointing. MY read is:
%- the most novel bit, which is deferring entity disambiguation to generation, seems to have only marginal impact (DED)
%- adding schema context information (SG-LF), adding relation to entity mention detection (EMD), are fairly straightforward innocation, on the other hand, seem to have the most impact!
%- For the first point, does that mean the premise that we said about errors due to premature entity disambiguation are empirically... quite rare? So it's a somewhat non-issue hmm....
%- So, what does all this mean for us? I think we can be honest here, and talk about how the more 'novel' idea turns out to be empirically less impactful, but that the more simple changes (like injecting more info into the input) turn out to have huge impact. Smooth it out a bit by saying but overall all the individual modules still contribute to the big performance gain ultimately so everything is cool. But up to you - it's your paper after all; this is just a suggestion.

% , the lower F1 scores of  emphasizes the importance of the schema context. \jz{Meanwhile, \model~w/o RG-EMD  This highlights the capability of our relation-guided entity mention detection module in strengthening entity retrieval under non-i.i.d settings.}

%Next, we conduct an ablation study to show the effectiveness of our generative entity retrieval module and the retrieval-guided generation module. 


%\addexp{can add \model-w/o fallback logical form generation;  (and others).} 



% \begin{table*}[ht]
% \small
% \centering
% \begin{tabular}{
% >{}p{0.2\linewidth}
% >{\centering\arraybackslash}m{0.05\linewidth}
% >{\centering\arraybackslash}m{0.05\linewidth} 
% >{\centering\arraybackslash}m{0.05\linewidth} 
% >{\centering\arraybackslash}m{0.05\linewidth} 
% >{\centering\arraybackslash}m{0.05\linewidth} 
% >{\centering\arraybackslash}m{0.05\linewidth} 
% >{\centering\arraybackslash}m{0.05\linewidth} 
% >{\centering\arraybackslash}m{0.05\linewidth} }
% \toprule
% & \multicolumn{2}{c}{\textbf{Overall}} & \multicolumn{2}{c}{\textbf{I.I.D.}} & \multicolumn{2}{c}{\textbf{Compositional}} & \multicolumn{2}{c}{\textbf{Zero-shot}} \\ \cline{2-9} 
% \multirow{-2}{*}{} \rule{0pt}{10pt}  \textbf{\vspace{-0.5cm}Model} & \textbf{EM}                   & \textbf{F1}  & \textbf{EM} & \textbf{F1} & \textbf{EM} & \textbf{F1}                     & \textbf{EM}  & \textbf{F1}  \\ \midrule
% TIARA & 75.3 & 81.9 & 88.4 & 91.2 & 66.4 & 74.8 & 73.3 & 80.7 \\
% FC-KBQA & 79.0 & 83.8 & 89.0 & 91.5 & 70.4 & 77.3 & 78.1 & 83.1 \\
% RetinaQA & 77.8 & 83.3 & 88.6 & 91.2 & 70.5 & 77.5 & 76.2 & 82.3 \\
% % TIARA + Generative Entity Retrieval &79.5 &84.3 &90.3 &92.3 &71.2 &78.1 &78.3 &83.3 \\
% % TIARA (LLaMA3-8B) & 79.9 & 85.6 & 88.6 & 92.3 & 72.7& 79.8 & 79.0 & 85.0 \\
% \midrule
% \textbf{\model} (Ours) \rule{0pt}{10pt} & 83.8 & 88.0 & 91.1 & 93.3 & 76.6 & 82.6 & 83.6 & 87.9 \\
% \hdashline
%   \rule{0pt}{10pt}\hspace{6pt} w/o SGER\tablefootnote{we replace schema-guided entity retrieval by SpanMD (mention dection method used in other SOTA studies.)} & 80.9 & 86.4 & 89.1 & 92.3 & 75.4 & 82.4 & 79.7 & 85.5 \\
%  \hspace{6pt} w/o SGLF\footnotemark[3] & 82.8 & 86.8 & 89.9 & 92.4 & 75.3 & 81.5 & 82.8 & 86.6\\
% \hspace{6pt} w/o SGER(top-1) & 83.0 & 86.7 & 89.2 & 91.2 & 75.4 & 80.8 & 83.4 & 86.9 \\ 
% \hspace{6pt} w/o R for LFS \\ 
% % \hline
% % \rule{0pt}{10pt}Top-1 Refined Entity (GER) + TIARA &79.5 &84.3 &90.3 &92.3 &71.2 &78.1 &78.3 &83.3\\
% %  TIARA's Entity Retrieval + RLG & 79.9 & 85.6 & 88.6 & 92.3 & 72.7& 79.8 & 79.0 & 85.0 \\

% % \hspace{6pt} w Top1-entity per mention & 83.0 & 86.7 & 89.2 &81.8 & 75.4 & 80.8 & 83.4 & 86.9 \\
% % \hspace{6pt} w/o Class & 82.8 & 86.8 & 89.9 & 92.4 & 75.3 & 81.5 & 82.8 & 86.6 \\ 
% % \hspace{6pt} w/o Generative Entity Retrieval  & 80.9 &	86.4 &	89.1 &	92.3 &	75.4 &	82.4 &	79.7 &	85.5 \\
% % \hspace{6pt} w/o Candidate Logical Forms & 80.1 & 83.9 & 90.7 & 92.5 & 75.2 & 80.6 & 77.5 & 81.5 \\
% % \hspace{6pt} w T5-Base & 78.7 & 83.2 & 88.1 & 90.4 & 69.5 & 75.6 & 78.4 & 83.3 \\
% % \hspace{6pt} w/o Context & 74.7 & 79.3 & 81.9 & 85.6 & 65.3 & 79.7 & 75.5 & 80.1 \\

% \bottomrule
% \end{tabular}
% \caption{Ablation study results on the validation set of GrailQA.}
% \label{tab:ablation}
% \end{table*}




%\paragraph{Generative Entity Retrieval} \jz{\model\ w/o generative entity retrieval exhibited a significant performance drop across both datasets. On GrailQA (Table~\ref{tab:ablation}), replacing the generative entity retrieval with TIARA's entity retrieval results led to a 3.4\% drop in the F1 score. Table~\ref{tab:entity_retrieval} further shows entity retrieval performance results on the GrailQA validation set, comparing GER method with commonly used entity retrieval methods in existing KBQA methods. Our GER method improves the F1 score by 7.0\%. Furthermore, our proposed candidate entity pruning strategy that combines the entity popularity-based pruning and relation-based pruning boosts the F1 score for entity retrieval by 2.1\%. To further validate the effectiveness of the GER module, we applied its retrieved entities to an open-source generate-then-retrieve method, TIARA (TIARA + Generative Entity Retrieval) in Table~\ref{tab:ablation}. We see that GET improves TIARA's F1 score by 3.0\% and EM by 5.5\%. On WebQSP, removing the GER from the merged entities set led to an 8.0\% drop in F1, while removing the ELQ entities resulted in only a 1.7\% drop (Table~\ref{tab:webqsp}). The experimental results demonstrate that using a relation-enhanced logical form sketch parser to generate entity mentions improves the identification of entity mentions in questions, even in zero-shot and compositional generalization settings.

%\paragraph{Retrieval-Guided Generation} 
%As shown in Table~\ref{tab:ablation}, \model\ w Top1-entity per mention negatives impacts F1 by 1.3 points overall, indicating that our generator (LLM) has the ability to select the correct entity from a set that includes false positive entities. The classes of the retrieved entities and the domain and range classes of the retrieved relations provide the generator with more KB context, resulting in a 1.3\% performance gain. We also replace LLaMA3-8B with T5-Base, a model widely used in our baselines~\cite{shu_tiara_2022, zhang_fc-kbqa_2023}, and find that the F1 score decreases by 4.8 points, while the EM score drops by 5.1 points with \model\ w T5-Base. Using candidate logical forms as a supplement when the generator fails to produce executable logical forms improves the F1 score by 4.9\% on GrailQA and by 2.1\% on WebQSP, suggesting the effectiveness of utilizing candidate logical forms as a supplement. Without providing any context (KB elements), using the generator to directly convert natural language questions into logical forms only achieves an EM score of 74.7 and an F1 score of 79.3, with particularly low F1 scores of 79.7 and 80.1 for compositional and zero-shot generalization, respectively. This highlights the necessity of the retrieval module for compositional and zero-shot generalization.}

\subsection{Module Applicability (Q3)}


Our relation-guided entity retrieval (\textbf{RG-EMD \& RG-CER}) module and schema-guided logical form generation (\textbf{DED \& SC}) module can be applied to existing KBQA models. We showcase such applicability with the TIARA model. As shown in Table~\ref{tab:exp_applicability}, by replacing the retrieval and generation modules of TIARA with ours, the F1 scores increase consistently for the non-I.I.D. tests.


Table~\ref{tab:exp_applicability} further reports F1 scores of \model\ when we replace LLaMA3.1-8B with \textbf{T5-base} (which is used by TIARA), and DeepSeek-R1-Distill-Llama-8B (\textbf{DS-R1-8B})~\cite{guo_deepseek_2025} for logical form generation. We see that, even with the same T5-base model for the logical form generator, \model\ outperforms TIARA consistently. This further confirms the effectiveness of our model design. As for DS-R1-8B, it offers accuracy slightly lower than that of the default LLaMA3.1-8B model. We conjecture that this is because DS-R1-8B is distilled from DeepSeek-R1-Zero, which focuses on reasoning capabilities and is not specifically optimized for the generation task.


%\subsection{Additional Results}
We also have results on parameter impact, model running time, a case study, and error analyses. They are documented in Appendices~\ref{app:paramater_study} to~\ref{app:error_analysis}.

\begin{table}[t]
\centering
\resizebox{\columnwidth}{!}{
\begin{tabular}{lcccc}
\toprule
\textbf{Model}                & \textbf{Overall} & \textbf{I.I.D.}& \textbf{Comp.} & \textbf{Zero.} \\ \midrule
TIARA (T5-base)   & 81.9   & 91.2  & 74.8  & 80.7  \\ 
\hspace{3pt} w RG-EMD \& RG-CER           & 84.3   & 92.3  & 78.1  & 83.3  \\
\hspace{3pt} w DED \& SC     & 85.6   & 92.3  & 79.8  & 85.0  \\
%\hspace{3pt} \jz{w DED}             &        &       &       &       \\
\midrule
\textbf{\model}            & \textbf{88.5}   & \textbf{94.6}  & \textbf{83.6}  & \textbf{87.9}  \\
\hspace{3pt} w T5-base       & 84.9   & 92.6  & 81.0  & 83.3  \\
%\hspace{3pt} w T5-large \jz{T5-3B?}      & 87.5   & 96.3  & 82.0  & 86.0  \\
 \hspace{3pt} w DS-R1-8B &   87.5     &  94.0     &  82.4     &  86.7     \\ \bottomrule
\end{tabular}
}
\caption{Module applicability results (F1 score) on the validation set of GrailQA. EM scores are in Appendix~\ref{app:applicability}.}\label{tab:exp_applicability}
\end{table}

\section{Conclusion}\label{sec:conclusion}
We proposed \model for the KBQA task. Our core innovations include: (1) using relation to guide the retrieval of entities; (2) deferring entity disambiguation to the logical form generation stage; and (3) enriching logical form generation with schema contexts to constrain search space. Together, we achieve a model that tops the leaderboard of a popular non-I.I.D. dataset GrailQA, outperforming SOTA models by 4.0\%, 2.9\%, and 3.3\% in F1 under compositional generalization, zero-shot generalization, and overall test settings, respectively. Our model also performs well in the I.I.D. setting, outperforming SOTA models on WebQSP.


\section*{Limitations}
%Despite the strong reported performance of \model, there are potentials to further improve the model.
First, like any other supervised models, \model requires annotated samples for training which may be difficult to obtain for many domains. Exploiting LLMs to generate synthetic training data is a promising direction to address this issue. Second, as discussed in the error analysis in Appendix~\ref{app:error_analysis}, errors can still arise from the relation retrieval, entity retrieval, and logical form generation modules. There are rich opportunities in further strengthening these modules. 
Particularly, as we start from relation extraction, the overall model accuracy relies on highly accurate relation extraction. It would be interesting to explore how well \model performs on even larger KBs with more relations.

\section*{Ethics Statement}
This work adheres to the ACL Code of Ethics and is based on publicly available datasets, used in compliance with their respective licenses. As our data contains no sensitive or personal information, we foresee no immediate risks. To promote reproducibility and further research, we also open-source our code.

%we acknowledge two major limitations in this study.Firstly, our method is trained on annotated question-logical form pairs, but the annotation cost for such data is expensive. The second The sencond major limitation lies in the time efficiency of our method. We report and discuss the training and inferecne runtime of our method in Appendix~\ref{app:time}. To guide the retreival and generation processes by KB schema, \model~utilizes additional KB queries to obtain the schema information for the corresponding KB elements. For example, In SG-LF, the class information of the selected KB elements are queried from the KB. The guidance from schema information significantly enhances the generalization capability of our method, but it also incurs an increase in time comsuption and computational cost. Compared to TIARA, our method taks on average 3.2 seconds longer inference time per questions.
%\section*{Acknowledgment}

% Bibliography entries for the entire Anthology, followed by custom entries
%\bibliography{anthology,custom}
% Custom bibliography entries only
\bibliography{references}

\newpage

\appendix


\section{S-Expression}\label{sec:app_sexpression}

S-expressions~\cite{gu_beyond_2021} use set-based semantics defined over a set of operators and operands. The operators are represented as functions. 
Each function takes a number of arguments (i.e., the operands). Both the arguments and the return values of the functions are either a set of entities or entity tuples (or tuples of an entity and a literal). The functions available in S-expressions are listed in Table~\ref{tab:logical_form_operators}, where a set of entities typically refers to a class (recall that a class is defined as a set of entities sharing common properties) or individual entities, and a binary tuple typically refers to a relation. %By applying those functions defined in our grammar, we are able to get more complex set of entities and binary tuples. 

\begin{table*}[h]
\small
\setlength{\tabcolsep}{2pt}
%\begin{tabular}{p{0.15\linewidth}p{0.3\linewidth}p{0.4\linewidth}}
\begin{tabular}{lll}
\toprule
\multicolumn{1}{l}{\textbf{Function}}                                             & \multicolumn{1}{l}{\textbf{Return value}} & \multicolumn{1}{l}{\textbf{Description}}                                                               \\ \midrule
(\texttt{AND} $u_1$ $u_2$)                                                                       & a set of entities                    & The \texttt{AND} function returns the intersection of two sets $u_1$ and $u_2$                                               \\ \hline
(\texttt{COUNT} $u$)                                                                         & a singleton set of integers           & The \texttt{COUNT} function returns the cardinality of set $u$                                                 \\ \hline
(\texttt{R} $b$)                                                                             & a set of (entity, entity) tuples     & The \texttt{R} function reverses each binary tuple $(x, y)$ in set $b$ to $(y, x)$                                   \\ \hline
(\texttt{JOIN} $b$ $u$)                                                                        & a set of entities                    & Inner \texttt{JOIN} based on entities in set $u$ and the second element of tuples in set $b$                                    \\ \hline
(\texttt{JOIN} $b_1$ $b_2$)                                                                      & a set of (entity, entity) tuples     & Inner \texttt{JOIN} based on the first element of tuples in set $b_2$ and the second element\\
& &  of tuples in set $b_1$             \\ \hline
\begin{tabular}[c]{@{}l@{}}(\texttt{ARGMAX} $u$ $b$)\\ (\texttt{ARGMIN} $u$ $b$)\end{tabular}                & a set of entities                    & These functions return $x$ in $u$ such that $(x,y) \in b$ and $y$ is the largest / smallest                                   \\ \hline
\begin{tabular}[c]{@{}l@{}}(\texttt{LT} $b$ $n$)\\ (\texttt{LE} $b$ $n$)\\ (\texttt{GT} $b$ $n$)\\ (\texttt{GE} $b$ $n$)\end{tabular} & a set of entities                    & These functions return all $x$ such that $(x, v) \in b$ and $v$ $<$ / $\le$ / $>$ / $\ge$ $n$ \\ \bottomrule
\end{tabular}
\caption{Functions (operators) defined in S-expressions ($u$: a set of entities, $b$: a set of (entity, entity or literal) tuples, $n$: a numerical value).}\label{tab:logical_form_operators}
\end{table*}

\begin{figure*}[h]
    \centering
    \includegraphics[width=\textwidth]{figures/prompt_example_new.png}
    \caption{Example prompt to our fine-tuned LLM-based logical form generator for an input question: \textsf{Captain pugwash makes an appearance in which comic strip?}}
    \label{fig:prompt_example}
\end{figure*}
\section{Prompt Example}\label{app:prompt}

We show an example prompt to our fine-tuned LLM-based logical form generator containing top-20 relations and top-2 entities per mention retrieved by our model in Figure~\ref{fig:prompt_example}.

\section{Additional Details on the WebQSP Dataset}\label{app:WebQSP}
\textbf{WebQuestionsSP} (WebQSP)~\cite{yih_value_2016} is an I.I.D. dataset. It contains 4,937 questions collected from Google query logs, including 3,098 questions for training and 1,639 for testing, each annotated with a target SPARQL query. %We convert each SPARQL query into the corresponding S-expression and extract 
We follow GMT-KBQA~\cite{hu_logical_2022}, TIARA~\cite{shu_tiara_2022} to separate 200 questions from the training questions to form the validation set.

\section{Baseline Models}\label{sec:app_baselines}
The following models are tested against \model\ on the GrailQA dataset:
\begin{itemize}
    \item RnG-KBQA~\cite{ye_rng-kbqa_2022} enumerates and ranks all possible logical forms within two hops from the entities retrieved by an entity retrieval step. It  uses a Seq2Seq model to generate the target logical form based on the input question and the top-ranked candidate logical forms.
    \item TIARA~\cite{shu_tiara_2022} shares the same overall procedure with RnG-KBQA. It further retrieves entities, relations, and classes based on the input question and feeds these KB elements into the Seq2Seq model together with the question and the top-ranked candidate logical forms to generate the target logical form.  
    
    
    %\jz{It demonstrates the connectivity of the KB elements through example logical forms starting from the entities. [How are these example logical forms obtained and how are they used?]} Finally, a Seq2Seq model converts the question and retrieved KB elements into the target logical form.

    \item TIARA+GAIN~\cite{shu_data_2024} enhances TIARA using a training data augmentation strategy. It synthesizes additional question-logical form pairs for model training to enhance the model's capability to handle more entities and relations. This is done by a 
     graph traversal to randomly sample logical forms from the KB  and a PLM to generate questions corresponding to the logical forms (i.e., the ``GAIN'' module). TIARA+GAIN is first tuned using the synthesized data and then tuned on the target dataset, for its retriever and generator modules which both use PLMs.
    
    \item Decaf~\cite{yu_decaf_2023} uses a Seq2Seq model that takes as input a question and a linearized question-specific subgraph of the KG and jointly decodes into both a  logical form and an answer candidate. The logical form is then executed, which produces a second answer candidate if successful. The final answer is determined from these two answer candidates with a scorer model. 

    \item Pangu~\cite{gu_dont_2023} formulates logical form generation as an iterative  enumeration process starting from the entities retrieved by an entity retrieval step. 
    At each iteration, partial logical forms generated so far are extended following paths in the KB to generate more and longer partial logical forms. A language model is used to select the top partial logical forms to be explored in the next iteration, under either fined-tuned models (T5-3B) or few-shot in-context learning (Codex). 

    
    \item FC-KBQA~\cite{zhang_fc-kbqa_2023} employs an intermediate module to test the connectivity between the retrieved KB elements, and it  generates the target logical form using the connected pairs of the retrieved KB elements through a Seq2Seq model.

     
    \item RetinaQA~\cite{faldu_retinaqa_2024} uses both a ranking-based method and a generation-based method (TIARA) to generate logical forms, which are then scored by a discriminative model to determine the output logical form.
    
    \item KB-BINDER~\cite{li_few-shot_2023} uses a training-free few-shot in-context learning model based on LLMs. It generates a draft logical form by showcasing the LLM examples of questions and logical forms (from the training set) that are similar to the given test question. Subsequently, a retrieval module grounds the surface forms of the KB elements in the draft logical form to specific KB elements.
    
    % \item FlexKBQA~\cite{li_flexkbqa_2024} is also a few-shot in-context learning model based on LLMs. To address the issue that the model generated logical forms are often unexecutable, it samples executable logical forms from the KB (like GAIN above) and instructs an LLM to generate a corresponding user question. These logical form-question pairs can then be used to fine-tune a lightweight model for logical form generation given an input question. \jz{Double check}
    \item FlexKBQA~\cite{li_flexkbqa_2024} considers limited training data and leverages an 
    LLM to generate additional training data. 
    It samples executable logical forms from the KB and utilizes an LLM with few-shot in-context learning to convert them into natural language questions, forming synthetic training data. These data, together with a few real-world training samples, are used to train a KBQA model. Then, the model is used to generate logical forms with more real world questions (without ground truth), which are filtered through an execution-guided module to prune the erroneous ones. The remaining logical forms and the corresponding real-world questions are used to train a new model. This process is repeated, to align the distributions of  synthetic training data and real-world questions. 
    
    %It introduces an execution-guided teacher-student iterative training method to bridge the gap between synthetic and real-world questions. The teacher model generates pseudo logical forms for unlabeled questions, which are filtered for quality, and used to iteratively train a student model, refining the logical form parser.}
    
    
    % is a flexible few-shot KBQA framework that leverages few-shot in-context learning to generate synthetic data using an LLM for training a lightweight model. It first extracts logical form templates from the few-shot annotated samples by replacing entities and relations with variables (e.g. ent0, rel0, ent1). These templates are then step-wise grounded with KB elements collected from the KB, generating a substantial number of executable logical forms. A LLM translates the obtained logical forms into natural language questions through in-context learning, constructing a synthetic dataset of logical form-question pairs. 
    
    % To address the distribution discrepancy between synthetic and real-world questions, it proposes an execution-guided teacher-student iterative training method. First, a teacher model is trained using synthetic data and a few annotated samples. The teacher model then generates pseudo logical forms for unlabeled real-world user questions, and an execution-guided filtering mechanism removes unexecutable or low-quality data. The filtered data, along with the synthetic data, is used to train a student model, which becomes the new teacher model in the next iteration. This process is repeated until the model converges, resulting in the final logical form parser.}
    
\end{itemize}


The following models are tested against \model\ on the WebQSP dataset:
\begin{itemize}
    % \item Subgraph Retrieval (SR)~\cite{zhang_subgraph_2022} uses a sequential decision process to progressively expand the subgraph corresponding to the question starting from the topic entity.
    \item Subgraph Retrieval (SR)~\cite{zhang_subgraph_2022} focuses on retrieving a KB subgraph relevant to the input question. It does not concern retrieving the exact question answer by  reasoning over the subgraph. Starting from the topic entity, it performs a top-$k$ beam search at each step to progressively expand into a subgraph, using a scorer module to score the candidate relations to be added to the subgraph next. 

    %In each expansion step, it uses a dual-encoder to encode a  candidate relation for the expansion and the concatenation of the input question and the historical relation path from previous steps, respectively.    
    %The dot product of the obtained embeddings is used to select the top-ranked candidate relations for the current step or terminating the expansion. 
    
    \item Evidence Pattern Retrieval (EPR)~\cite{ding_enhancing_2024} aims to extract subgraphs with fewer noise entities. It starts from the topic entities and expands by retrieving and ranking atomic (topic entity-relation or relation-relation) patterns relevant to the question. This forms a set of relation path graphs (i.e., the candidate \emph{evidence patterns}). The relation path graphs are then ranked to select the most relevant one. By further retrieving the entities on the selected relation path graph, EPR obtains the final subgraph relevant to the input question. 
    

    
    \item Neural State Machine (NSM)~\cite{he_improving_2021} is a reasoning model to find answers for the KBQA problem from a subgraph (e.g., retrieved by SR or EPR). It address the issue of lacking intermediate-step supervision signals when reasoning through the subgraph to reach the answer entities. This is done by training a so-called teacher model that follows a bidirectional reasoning mechanism starting from both the topic entities and the answer entities. During this process, the  ``distributions'' of entities, which represent their probabilities to lead to the answer entities (i.e., intermediate-step supervision signal), are propagated. 
    A second model, the so-called student model, learns from the teacher model to generate the entity distributions, with knowledge of the input question and the topic entities but not the answer entities. Once trained, this model can be used for KBQA answer reasoning. 

    
    
    % KBQA~\cite{he_improving_2021} is a widely used answer reasoning model on subgraphs, often combined with different subgraph retrieval methods (e.g. SR~\cite{zhang_subgraph_2022}, EPR~\cite{ding_enhancing_2024}), which employs a teacher-student framework to solve multi-hop KBQA tasks. The student model is based on NSM~\cite{hudson_nsm_2019} that consists of two components: the instruction component and the reasoning component. The instruction component uses an LSTM to encode the question and generate a series of instruction vectors that guide the reasoning process. The reasoning component relies on a "propagation-aggregation" mechanism to aggregate and update information about entities and relations in the KB based on the current instruction, gradually maintaining and updating the entity distribution. On the other hand, the teacher model generates intermediate entity distributions by simultaneously performing forward (starting from the topic entity) and backward (starting from the answer entity) reasoning. It uses a consistency constraint between the forward and backward reasoning processes, specifically minimizing the difference between the two using the Jensen-Shannon divergence at each intermediate step. This ensures the reliability of the intermediate entity distributions. Finally, the intermediate supervision signal generated by the teacher model is used to guide the student model, optimizing its reasoning path and improving the overall answering reasoning accuracy.
    
    \item UniKGQA~\cite{jiang_unikgqa_2023} integrates both retrieval and reasoning stages to enhance the accuracy of multi-hop KBQA tasks. It trains a PLM to learn the semantic relevance between every relation and the input question. The semantic relevance information is  propagated and aggregated through the KB to form the semantic relevance between the entities and the input question. The entity with the highest semantic relevance is returned as the answer.
        
    
    \item ChatKBQA~\cite{luo_chatkbqa_2024} fine-tunes an open-source LLM to map questions into draft logical forms. The  ambiguous KB items in the draft logical forms are replaced with specific KB elements by a separate retrieval module.
    
    \item TFS-KBQA~\cite{wang_no_2024} fine-tunes an LLM for more accurate logical form generation with three strategies.  The first strategy directly fine-tunes the LLM to map natural language questions into draft logical forms containing entity names instead of entity IDs. The second strategy breaks the mapping process into two steps, first to generate relevant KB elements, and then to generate draft logical forms using the KB elements. The third strategy fine-tunes the LLM to directly generate the answer to an input question. 
    After applying the three fine-tuning strategies, the LLM is used to map natural language questions into draft logical forms at model inference. A separate entity linking module is used to further map the entity names in draft logical forms into entity IDs. 

\end{itemize}




% \begin{figure*}[h]
%     \centering
%     \includegraphics[width=0.7\linewidth]{figures/case_study.png}
%     \caption{Case Study of logical form generation by \model\ and other models on the GrailQA validation set. Incorrect relations and entities are marked in red, while the correct relations and entities are colored in green and blue, respectively}
%     \label{fig:case_study}
% \end{figure*}

% \begin{figure*}[h]
%     \centering
%     \includegraphics[width=0.7\linewidth]{figures/md_case.png}
%     \caption{Case study of entity mention detection by our model and SpanMD (a mention detection method commonly used by SOTA KBQA models) on the GrailQA validation set. The incorrect entity mention detected is colored in red, while the correct entity mentions detected are colored in green and blue, respectively. \jz{``GER'' $\rightarrow$ ``Ours''} }
%     \label{fig:md_case}
% \end{figure*}



% Please add the following required packages to your document preamble:
% \usepackage{multirow}




% \begin{figure*}[h]
%     \centering
%     \includegraphics[width=0.8\linewidth]{figures/rlg_case.png}
%     \caption{Case study of logical form generation by our \model\ model and \jz{two representative baseline models TIARA and Pangu} on the GrailQA validation set. Incorrect relations and entities retrieved are colored in red, while correct relations and entities retrieved are colored in green and blue, respectively. The same sets of noisy entities and relations are retrieved by all three models, while only our model \model\ is able to produce the correct logical form. \jz{Use table instead, separate TIARA/Pangu from ours, same for the figure above.}}
%     \label{fig:rlg_case}
% \end{figure*}

% \section{Case Study}

% Figure~\ref{fig:case_study} shows an example case from the GrialQA validation set, with the prediction from our \model, the SOTA Pangu, and TIARA. In this example, both the top-1 candidate for entity retrieval and relation retrieval are non-optimal. Unlike previous methods PANGU and TIARA, which solely rely on the top-1 candidate entity to generate logical forms, our model can combine KB elements using class-based contextualization and select the optimal combination of KB elements based on their relevance to the question to form the final logical form, without overly depending on the performance of the retrieval module.


\begin{table*}[ht]
\small
\centering
\begin{tabular}{
>{}p{0.3\linewidth}
>{\centering\arraybackslash}m{0.05\linewidth}
>{\centering\arraybackslash}m{0.05\linewidth} 
>{\centering\arraybackslash}m{0.05\linewidth} 
>{\centering\arraybackslash}m{0.05\linewidth} 
>{\centering\arraybackslash}m{0.05\linewidth} 
>{\centering\arraybackslash}m{0.05\linewidth} 
>{\centering\arraybackslash}m{0.05\linewidth} 
>{\centering\arraybackslash}m{0.05\linewidth} }
\toprule
& \multicolumn{2}{c}{\textbf{Overall}} & \multicolumn{2}{c}{\textbf{I.I.D.}} & \multicolumn{2}{c}{\textbf{Compositional}} & \multicolumn{2}{c}{\textbf{Zero-shot}} \\ 
\cmidrule{2-9}
\multirow{-2}{*}{}  \textbf{\vspace{-0.5cm}Model} & \textbf{EM}                   & \textbf{F1}  & \textbf{EM} & \textbf{F1} & \textbf{EM} & \textbf{F1}                     & \textbf{EM}  & \textbf{F1}  \\ \midrule
% TIARA & 75.3 & 81.9 & 88.4 & 91.2 & 66.4 & 74.8 & 73.3 & 80.7 \\
% FC-KBQA & 79.0 & 83.8 & 89.0 & 91.5 & 70.4 & 77.3 & 78.1 & 83.1 \\
% RetinaQA & 77.8 & 83.3 & 88.6 & 91.2 & 70.5 & 77.5 & 76.2 & 82.3 \\
% TIARA + Generative Entity Retrieval &79.5 &84.3 &90.3 &92.3 &71.2 &78.1 &78.3 &83.3 \\
% TIARA (LLaMA3-8B) & 79.9 & 85.6 & 88.6 & 92.3 & 72.7& 79.8 & 79.0 & 85.0 \\
%\midrule
%\textbf{\model} & \textbf{83.8} & \textbf{88.0} & \textbf{91.1} & \textbf{93.3} & \textbf{76.6} & \textbf{82.6} & \textbf{83.6} & \textbf{87.9} \\
\textbf{\model} & \textbf{85.1} & \textbf{88.5} & \textbf{93.1} & \textbf{94.6} & \textbf{78.4} & \textbf{83.6} & \textbf{84.4} & \textbf{87.9} \\
\hdashline
  \rule{0pt}{10pt}\hspace{6pt} w/o RG-EMD & 81.3 & 85.3 & 90.6 & 92.4 & 74.4 & 80.2 & 80.2 & 84.3 \\
 \hspace{6pt} w/o RG-CER &  82.8 & 86.5 & 90.2 & 92.1 & 75.4 & 81.1 & 82.7 & 86.3 \\ 
\hspace{6pt} w/o DED & 84.3 & 87.8 & 92.6 & 94.0 & 77.1 & 82.4 & 83.7 & 87.2 \\ 
\hspace{6pt} w/o SC & 76.6 & 79.2 & 91.7 & 92.9 & 72.3 & 77.4 & 71.7 & 73.9\\
\hspace{6pt} w/o Fallback LF & 81.8 & 84.6 & 92.8 & 94.1 & 77.3 & 81.8 & 78.7 & 81.5\\
% \hline
% \rule{0pt}{10pt}Top-1 Refined Entity (GER) + TIARA &79.5 &84.3 &90.3 &92.3 &71.2 &78.1 &78.3 &83.3\\
%  TIARA's Entity Retrieval + RLG & 79.9 & 85.6 & 88.6 & 92.3 & 72.7& 79.8 & 79.0 & 85.0 \\

% \hspace{6pt} w Top1-entity per mention & 83.0 & 86.7 & 89.2 &81.8 & 75.4 & 80.8 & 83.4 & 86.9 \\
% \hspace{6pt} w/o Class & 82.8 & 86.8 & 89.9 & 92.4 & 75.3 & 81.5 & 82.8 & 86.6 \\ 
% \hspace{6pt} w/o Generative Entity Retrieval  & 80.9 &	86.4 &	89.1 &	92.3 &	75.4 &	82.4 &	79.7 &	85.5 \\
% \hspace{6pt} w/o Candidate Logical Forms & 80.1 & 83.9 & 90.7 & 92.5 & 75.2 & 80.6 & 77.5 & 81.5 \\
% \hspace{6pt} w T5-Base & 78.7 & 83.2 & 88.1 & 90.4 & 69.5 & 75.6 & 78.4 & 83.3 \\
% \hspace{6pt} w/o Context & 74.7 & 79.3 & 81.9 & 85.6 & 65.3 & 79.7 & 75.5 & 80.1 \\

\bottomrule
\end{tabular}
\caption{Ablation study results on the validation set of GrailQA.}
\label{tab:ablation_full}
\end{table*}

\section{Implementation Details}\label{app:implemention_details}

All our experiments are run on a machine with an NVDIA A100 GPU and 120 GB of RAM. We fine-tuned three \texttt{bert-base-uncased} models for a maximum of three epochs each, for relation retrieval, entity ranking, and fallback logical form ranking.
For relation retrieval, we randomly sample 50 negative samples for each question to train the model to distinguish between relevant and irrelevant relations. 

For each dataset, a \texttt{T5-base} model is fine-tuned for 5 epochs as our logical form sketch parser, with a beam size of 3 (i.e., $k_L = 3$) for GrailQA, and 4 for WebQSP. For candidate entity retrieval, we use the same number (i.e., $k_{E1} + k_{E2}  = 10$) of candidate entities per mention as that used by the baseline models~\cite{shu_tiara_2022,ye_rng-kbqa_2022}. The retrieved candidate entities for a mention consist of entities with the top-$k_{E1}$ popularity scores and $k_{E2}$ entities connected to the top-ranked relations in $R_q$, where $k_{E1} = 1$, $k_{E2} = 9$ for GrailQA, $k_{E1} = 3$, $k_{E2} = 7$ for WebQSP. We select the top-20 (i.e., $k_R$ = 20) relations and the top-2 (i.e., $k_{E3} = 2$) entities (for each entity mention) retrieved by our model. For WebQSP, we also use the candidate entities obtained from the off-the-shelf entity linker ELQ~\cite{li_efficient_2020}. 

Finally, we fine-tune LLaMA3.1-8B with LoRA~\cite{hu_lora_2021} for logical form generation. On GrailQA, LLaMA3.1-8B is fine-tuned for 5 epochs with a learning rate of $0.0001$. On WebQSP, it is fine-tuned for 20 epochs with the same learning rate (as it is an I.I.D. dataset where more epochs are beneficial). During inference, we generate logical forms by beam search with a beam size of 10 (i.e., $K_O = 10$). The generated logical forms are executed on the KB to filter non-executable ones. If none of the logical forms are executable, we check candidate logical forms from the fallback procedures, and the result of the first executable one is returned as the answer set.
%\jz{Any updates needed for this subsection?} 


% Our system parameters have been chosen empirically. While there are a few of them, their exact values do not have strong impact on the final model performance, \jz{and the choice of parameter values generalize well across  datasets. The same parameter values are used on both datasets. [not true any more?]} 
Our system parameters are selected empirically. There are only a small number of parameters to consider. As shown in the parameter study later, our model performance shows stable patterns against the choice of parameter values. The parameter values do not take excessive fine-tuning. 

\section{Full Ablation Study Results (GrailQA)}\label{app:ablation}
%\jz{Fix table and add discussion on the results}

Table~\ref{tab:ablation_full} presents the full ablation study results on the validation set of GrailQA. We observe a similar trend to that of the F1 score results reported earlier --  all ablated model variants yield lower EM scores compared to the full model. 

For the retrieval modules, RG-EMD improves the F1 score by 3.2 points and the EM score by 3.8 points on GrailQA (i.e., \model\ vs. \model w/o RG-EMD for overall results), while achieving a 1.9-point increase in the F1 score on WebQSP (see Table~\ref{tab:webqsp} earlier). It achieves an increase of 3.4 points or larger in the F1 score on the compositional and zero-shot tests, which is larger than the 2.2-point improvement on the I.I.D. tests. This shows that relation-guided mention detection effectively enhances the generalization capability of KBQA entity retrieval. For the other module RG-CER, removing it (\model w/o RG-CER) results in a 2.5-point drop in the F1 score for both the I.I.D. and compositional tests, while the impact is smaller on the zero-shot tests (1.6 points). This is because the lower accuracy in relation retrieval under zero-shot tests leads to error propagation into relation-guided candidate entity retrieval, reducing the benefits of this module.  

For the generation modules, \model\ w/o DED negatively impacts the F1 scores on both GrailQA and WebQSP, confirming that deferring entity disambiguation effectively mitigate error propagation between the retrieval and generation stages. For \model w/o SC, it reduces the F1 score by 1.7 points and 3.2 points on the GrialQA I.I.D. tests and on WebQSP. The drop is more significant on the compositional and zero-shot tests, i.e., by 6.2 points and 14.0 points, respectively. This indicates that schema contexts can effectively guide the LLM to reason and identify the correct combinations of KB elements unseen at training.


In Table~\ref{tab:ablation_full}, we present an additional model variant, \model w/o Fallback LF, which removes the fall back logical form generation strategy from \model. We see that \model\ has lower accuracy without the strategy. %Importantly, even without this fall back strategy, our model has high accuracy results in both EM and F1 comparing with the baseline models as shown in the table. 
We note that this fallback strategy is \emph{not} the reason why \model\ outperforms the baseline models. 
TIARA also uses this fallback strategy, while RetinaQA uses the top executable logical form from the fallback strategy as one of the options to be selected by its  discriminator to determine the final logical form output.

\section{Full Module Applicability  Results}\label{app:applicability}


\begin{table*}[t]
\small
\centering
\begin{tabular}{
>{}p{0.3\linewidth}
>{\centering\arraybackslash}m{0.05\linewidth}
>{\centering\arraybackslash}m{0.05\linewidth} 
>{\centering\arraybackslash}m{0.05\linewidth} 
>{\centering\arraybackslash}m{0.05\linewidth} 
>{\centering\arraybackslash}m{0.05\linewidth} 
>{\centering\arraybackslash}m{0.05\linewidth} 
>{\centering\arraybackslash}m{0.05\linewidth} 
>{\centering\arraybackslash}m{0.05\linewidth} }
\toprule
& \multicolumn{2}{c}{\textbf{Overall}} & \multicolumn{2}{c}{\textbf{I.I.D.}} & \multicolumn{2}{c}{\textbf{Compositional}} & \multicolumn{2}{c}{\textbf{Zero-shot}} \\ \cline{2-9} 
\multirow{-2}{*}{} \rule{0pt}{10pt}  \textbf{\vspace{-0.5cm}Model} & \textbf{EM}                   & \textbf{F1}  & \textbf{EM} & \textbf{F1} & \textbf{EM} & \textbf{F1}                     & \textbf{EM}  & \textbf{F1}  \\ \midrule 
TIARA (T5-base) & 75.3 & 81.9 & 88.4 & 91.2 & 66.4 & 74.8 & 73.3 & 80.7 \\
\hspace{6pt} w RG-EMD \& RG-CER & 79.5 &84.3 &90.3 &92.3 &71.2 &78.1 &78.3 &83.3 \\
\hspace{6pt} w DED \& SC &79.9 & 85.6 & 88.6 & 92.3 & 72.7& 79.8 & 79.0 & 85.0\\  
%FC-KBQA (COLING 2024) & 79.0 & 83.8 & 89.0 & 91.5 & 70.4 & 77.3 & 78.1 & 83.1 \\
%RetinaQA (ACL 2024) & 77.8 & 83.3 & 88.6 & 91.2 & 70.5 & 77.5 & 76.2 & 82.3 \\
% TIARA + Generative Entity Retrieval &79.5 &84.3 &90.3 &92.3 &71.2 &78.1 &78.3 &83.3 \\
% TIARA (LLaMA3-8B) & 79.9 & 85.6 & 88.6 & 92.3 & 72.7& 79.8 & 79.0 & 85.0 \\
\midrule
%\textbf{\model} (Ours) \rule{0pt}{10pt} & 83.8 & 88.0 & 91.1 & 93.3 & 76.6 & 82.6 & 83.6 & 87.9 \\
\textbf{SG-KBQA} & \textbf{85.1} & \textbf{88.5} & \textbf{93.1} & \textbf{94.6} & \textbf{78.4} & \textbf{83.6} & \textbf{84.4} & \textbf{87.9}\\
\hspace{6pt} w T5-base & 80.6 & 84.9 & 89.9 & 92.6 & 73.8 & 81.0 & 79.4 & 83.3\\
\hspace{6pt} w DS-R1-8B & 83.6 & 87.5 & 92.3 & 94.0 & 75.4 & 82.4 & 83.1 & 86.7 \\


%   SGER + TIARA & 79.5 &84.3 &90.3 &92.3 &71.2 &78.1 &78.3 &83.3 \\
%  TIARA + SGLF & 79.9 & 85.6 & 88.6 & 92.3 & 72.7& 79.8 & 79.0 & 85.0\\
%  TIARA + multiple entities  \\ 
% SG-KBQA w T5-base & 81.7 & 85.6 & 93.8 & 95.6 & 73.3 & 80.5 & 79.9 & 83.3 \\
% SG-KBQA w T5-large & 83.5 & 87.5 & 95.2 & 96.3 & 73.5 & 82.0 & 82.5 & 86.0\\
% SG-KBQA w deepseek-r1-distilled-8B \\
% \hline
% \rule{0pt}{10pt}Top-1 Refined Entity (GER) + TIARA &79.5 &84.3 &90.3 &92.3 &71.2 &78.1 &78.3 &83.3\\
%  TIARA's Entity Retrieval + RLG & 79.9 & 85.6 & 88.6 & 92.3 & 72.7& 79.8 & 79.0 & 85.0 \\

\bottomrule
\end{tabular}
\caption{Full module applicability results on the validation set of GrailQA.}
\label{tab:applicability_full}
\end{table*}

To evaluate the applicability of our proposed modules, we conduct a module applicability study with TIARA (an open-source retrieve-then-generate baseline) and different generation models (i.e., T5-base and DeepSeek-R1-Distill-Llama-8B). 


Table~\ref{tab:applicability_full} reports the results. Replacing TIARA's entity retrieval module with ours (TIARA w RG-EMD \& RG-CER) helps boost the EM and F1 scores by 4.2 and 2.4 points overall, comparing against the original TIARA model. This improvement is primarily from the tests with KB elements or compositions that are unseen at training, as evidenced by the larger performance gains on the compositional and zero-shot tests, i.e., 3.3 and 2.6 points in the F1 score, respectively. Similar patterns are observed for TIARA w DED \& SC that replaces TIARA's logical form generation module with ours. 
These results demonstrate that our proposed modules can enhance the retrieval and generation steps of other compatible models, especially under non-I.I.D. settings. 

Further, using the same language model (i.e., T5-base in TIARA) to form logical form generation modules, our model \model\ w T5-base still outperforms TIARA by 5.3 points  3.0 points in the EM and F1 scores for the overall tests. This confirms that the overall effectiveness of our model stems from its design rather than the use of a larger model for logical form generation. As for \model w/ DS-R1-8B, it reports close performance to \model, indicating that \model does not rely on a particular LLM.


% \jz{Fix table and add discussion on the results}
% \jz{Also results on WebQSP?}


% \section{Entity Retrieval Results}\label{app:er_results}

% \begin{table}[H]
%     \small
%     \centering
%     \begin{tabular}{lccc} 
%     \toprule \textbf{Model} & \textbf{P} & \textbf{R} & \textbf{F1} \\
%     \midrule 
%     RnG-KBQA  & 84.1 & 86.8 & 80.4 \\
%     TIARA  & 87.2 & 88.6& 85.4 \\
%     \midrule
%     \textbf{SG-ER (Top-1)} & \textbf{91.9} & \textbf{93.6} & \textbf{90.5} \\
%     \hspace{6pt}w/o RG-EMD & 88.9 & 91.3 & 88.2 \\
%     \hspace{6pt}w/o RG-CER & 88.7 & 90.0 & 86.9 \\
%     \bottomrule
%     \end{tabular}
%         \caption{Precision (P), recall (R) and F1 of entity retrieval (\%) on the validation set of GrailQA.}
%     \label{tab:entity_retrieval}
% \end{table}


% \sxfix{We report the performance of our SG-ER (Top-1) on the GrailQA validation set in Table~\ref{tab:entity_retrieval}. We compare against the following baselines: 1)\textbf{RnG-KBQA}~\cite{ye_rng-kbqa_2022} which adopts a BERT-NER system to detect entity mentions. 2)\textbf{TIARA}~\cite{shu_tiara_2022} which models entity mention detection to span classification task to detect entity spans. To ensure a fair comparison with the baselines, we follow their approaches by extracting the top-1 entity for each mention from our SG-ER. It can be observed that our SG-ER siginificantly surpasses all the baselines by at least 5.1 F1 points.

% Furthermore, \textbf{w/o RG-EMD} shares the same mention detector with TIARA, indicating that our RG-CER module is effective by improving the entity retrieval F1 by 2.8 points. \textbf{w/o RG-CER} shares the same candidate entity retrieval methods with the baselines but boosts entity retrieval F1 by at least 1.5 points. This demonstrates that our RG-EMD can more accurately identify entity boundaries in the input question.}


\section{Parameter Study}\label{app:paramater_study}

We conduct a parameter study to investigate the impact of the choice of values for our system parameters. When the value of a parameter is varied, default values as mentioned in Appendix~\ref{app:implemention_details} are used for the other parameters. 

%\jz{Add results and discussion}
\begin{figure}[htb] 
    \begin{minipage}[b]{0.5\linewidth}  
        \centering
        \includegraphics[width=\textwidth]{figures/KL.pdf} 
        \captionsetup{font=small}
        \subcaption{$k_L$}
       % \label{fig:sub1}
    \end{minipage}%
    \hfill 
    \begin{minipage}[b]{0.5\linewidth}  
        \centering
        \small
        \includegraphics[width=\textwidth]{figures/KE1.pdf}
        \captionsetup{font=small}
        \subcaption{$k_{E1}$}
        %\label{fig:sub2}
    \end{minipage}
    \caption{Impact of $k_L$ and $k_{E1}$ on the recall of candidate entity retrieval.} %\jz{font size in figure too small, could reduce the data points if needed more space; candidate entity coverage $\rightarrow$ recall of candidate entity retrieval?}}  
    \label{fig:kl_ke1}
\end{figure}


Figure~\ref{fig:kl_ke1} presents the impact of  $k_L$ and $k_{E1}$ on the recall of candidate entity retrieval (i.e., the average percentage of ground-truth entities returned by our candidate entity retrieval module for each test sample). Here, for the GrailQA dataset, we report the results on the overall tests (same below). 
Recall that $k_L$ means the number of logical form sketches from which entity mentions are extracted, while $k_{E1}$ refers to the number of candidate entities retrieved based on the popularity scores. 

As $k_L$ increases, the recall of candidate entity retrieval grows, which is expected. The growth diminishes gradually. This is because a small number of questions contain complex entity mentions that are difficult to handle (see error analysis in Appendix~\ref{app:error_analysis}). As $k_L$ increases, the precision of the retrieval also reduces, which brings noise into the entity retrieval results and additional computational costs. 
To strike a balance, we set $k_L = 3$ for GrailQA and  $k_L = 4$ for WebQSP. We also observe that the recall on WebQSP is lower than that on GrailQA. This is because  WebQSP has a smaller training set to learn from. 


As for $k_{E1}$, when its value increases, the candidate entity recall generally drops. This is because an increase in $K_{E1}$ means to select more candidate entities based on popularity while fewer from those connected to the top retrieved relations but with lower popularity scores. 
Therefore, we default $k_{E1}$ at $1$ for GrailQA and $3$ for WebQSP, which yield the highest recall for the two datasets, respectively. 
Recall that we set the total number of candidate entities for each entity mention to 10 ($K_{E1} + K_{E2} = 10$), following our baselines (e.g., TIARA, RetinaQA, and Pangu). Therefore, we omit another study on $K_{E2}$, as it varies with $K_{E1}$.

\begin{figure}[htb] 
    \begin{minipage}[b]{0.5\linewidth}  
        \centering
        \includegraphics[width=\textwidth]{figures/KR.pdf} 
        \captionsetup{font=small}
        \subcaption{$k_R$}
        %\label{fig:sub1}
    \end{minipage}%
    \hfill 
    \begin{minipage}[b]{0.5\linewidth}  
        \centering
        \small
        \includegraphics[width=\textwidth]{figures/KE3.pdf}
        \captionsetup{font=small}
        \subcaption{$k_{E3}$}
        %\label{fig:kr_ke3}
    \end{minipage}
    \caption{Impact of $k_R$ and $k_{E3}$ on the overall F1 score.}
    \label{fig:kr_ke3}
\end{figure}

\begin{table*}[h]
\small
\centering
\begin{tabular}{ll}
\hline
\textbf{Question:} What is the name for the atomic units of length? \\ \hline \addlinespace[2pt]
\textbf{SpanMD:}  What is the name for the atomic units of \textcolor{red}{length}? & (\ding{55})  \\ \hline \addlinespace[2pt]
\textbf{Ours:}\\
\textbf{\hspace{6pt}Retrieved Relations:} measurement\_unit.measurement\_system.length\_units,\\
\hspace{87.5pt}measurement\_unit.time\_unit.measurement\_system, \\
\hspace{87.5pt}measurement\_unit.measurement\_system.time\_units... \\
\textbf{Generated Logical Form Sketch:}  (AND \textless{}class\textgreater~(JOIN \textless{}relation\textgreater~{[} \textcolor{blue}{atomic units} {]}))\hspace{10pt} &(\ding{51})\\ \hline
                                                                          
\end{tabular}
\caption{Case study of entity mention detection by our model and SpanMD (a mention detection method commonly used by SOTA KBQA models) on the GrailQA validation set. The incorrect entity mention detected is colored in red, while the correct entity mention detected is colored in blue.}
\label{tab:md_case}
\end{table*}

Figure~\ref{fig:kr_ke3} further shows the impact of $k_R$ and $k_{E3}$ -- recall that  $k_R$ is the number of top candidate relations considered, and $k_{E3}$ is the number of candidate entities matched for each entity mention. 
Now we show the F1 scores, as these parameters are used by 
our schema-guided logical form generation module. They directly affect the accuracy of the generated logical form and the corresponding question answers.

On GrailQA, increasing either $k_R$ or $k_{E3}$ leads to higher F1 scores, although the growth becomes marginal eventually. On WebQSP, the F1 scores peak at $k_R=25$ and $k_{E3}=4$. These results suggest that feeding an excessive number of candidate entities and relations to the logical form generator module has limited benefit. 
To avoid the extra computational costs (due to more input tokens) and to limit the input length for compatibility with smaller Seq2Seq models (e.g., T5-base), we use $k_R=20$ and $k_{E3}=2$ on both datasets. 
%we did not adopt larger values for $K_{R}$ and $K_{E3}$. 

%[A bit strange, why not $k_R=25$ and $k_{E3}=4$?]}


\section{Model Running Time}\label{app:time}
\model\ takes 26 hours to train on the GrailQA dataset and 13.6 seconds to run inference for a test sample. It is faster on WebQSP which is a smaller dataset. Note that more than 10 hours of the training time were spent on the fallback logical form generation. If this step is skipped (which does not impact our model accuracy substantially as shown earlier), \model\ can be trained in about half a day. Another five hours were spent on fine-tuning the LLM for logical form generation, which can also be reduced by using a smaller model. 

As there is no full released code for the baseline models, it is infeasible to benchmark against them on model training time. For model  inference tests, TIARA has a partially released model (with a closed-source mention detection module). The model takes 11.4 seconds per sample (excluding the entity mention detection module) for inference on GrailQA, which is close to that of \model. Therefore, we have achieved a model that is more accurate than the baselines while being at least as fast in inference as one of the top performing baselines (i.e., TIARA+GAIN which shares the same inference procedure with TIARA).



%\jz{Add discussion on the results}

%\jz{Table~\ref{tab:run_time} reports the overall model training and inference time (per instance) of \model\ on GrailQA, as well as a detailed breakdown. Overall, \model\ takes about a day to train, while it takes 13.6 seconds to infer the answer of an input question. On WebQSP, we observe a similar time breakdown, while the overall model training time is smaller. We omit the detail results for simplicity. 

%We note that none of the top performing baselines have released 

%For benchmarking ... [Should add running time of TIARA, FC-KBQA, and RetinaQA for comparison] 
%\sxfix{no baseline completely opensouce, TIARA almost did it, but still has a mention detector (important module) not yet open-sourced. No training time reported in their paper. Some has inference time but with different experiment setting.}}


%The training time varies significantly across modules, with fallback logical form retrieval taking the longest (10 hours and 23 minutes) due to the time-consuming nature of the logical form enumeration process.  Similarly, it accounts for over 40\% of the inference time for each test question. Furthermore, since our logical form generator is an LLM, it has a large size, which also takes some time to fine-tune (5 hours and 33 minutes) and run the  inference process (4.3 seconds). The training time for relation retrieval is shorter than that of RG-EMD, while the inference time for relation retrieval is longer. This is because, during inference, the relation retrieval module is used to score each relation in the KB relation pool to obtain the top-ranked relations, which takes time, while \sxfix{RG-EMD adopts a T5-base, which has a small parameter size and fast inference speed, to generate logical form sketches.} These results reveal research opportunities to improve the time efficiency of relation retrieval and logical form enumeration.

%Despite the parameter size of our logical form generator is considerably larger, the overall training time is shorter since it is trained for only 1 epoch. 

% For inference, 

% }



%\begin{table}[H]
%\centering
%\small
%\begin{tabular}{lrr}
%\toprule
%\textbf{Modules}        & \textbf{Training} & \textbf{Inference} \\ 
%\midrule
%Relation retrieval         &  4h21m   & 3.3s                \\ 
%RG-EMD        &   4h30m     & 0.14s               \\ 
%RG-CER &    -    & 0.2s                \\
%Entity ranking         &  1h30m  & 0.003s              \\ 
%SG-LF    &  5h33m   & 4.3s                \\ 
%Fallback LF retrieval     &    10h23m       & 5.7s                     \\
%Total                   &    26h17m      & 13.6s                \\ 
%\bottomrule
%\end{tabular}
%\caption{Model training time and average inference time (per instance) on GrailQA (h: hours; m: minutes; s: seconds).}
%\label{tab:run_time}
%\end{table}
%\addexp{Add running time results}

%\jz{\subsection{Case Study} To further show \model's capability, we include a case study from the GrailQA validation set as shown in Figure~\ref{fig:case_study}. It shows that when the ground-truth KB element is not the top-1-ranked candidate from our retrieval modules, the generator can still select the correct KB elements and generate the correct logical form through the class-based context, without being overly dependent on the performance of the retrieval module.}

\section{Case Study}\label{app:case_study}


\begin{table*}[h]
\centering
\small
\begin{tabular}{m{1cm} m{6cm} m{4cm}}
\hline
\multicolumn{3}{l}{\textbf{Question:} Captain pugwash makes an appearance in which comic strip?} \\
\hline
                       & \multicolumn{1}{l}{\textbf{Relation Retrieval}}                                      & \multicolumn{1}{l}{\textbf{Entity Retrieval}}          \\
\hline
\multirow{4}{*}{\textbf{TIARA}} & \textcolor{red}{\ldots comic\_strips\_appeared\_in}                                               & Captain Pugwash \textcolor{blue}{m.04fgkzf}         \\
                       & \textcolor{blue}{\ldots character}                                                                 &                                    \\
\cline{2-3} 
                       \addlinespace[2pt]& \multicolumn{2}{l}{\begin{tabular}[c]{@{}l@{}}(AND comic\_strips.comic\_strip\_character (JOIN \\ \hspace{8pt}\textcolor{red}{comic\_strips.comic\_strip\_character.comic\_strips\_appeared\_in } \textcolor{red}{m.04fgkzf}))\end{tabular}}   (\ding{55}) \\
\hline \addlinespace[2pt] 
\multirow{7}{*}{\centering\textbf{ Ours}} 
            & {[}D{]} comic\_strip\_character      & {[}ID{]} \textcolor{red}{m.04fgkzf}                 \\
                       & {[}N{]} \textcolor{red}{comic\_strips\_appeared\_in}  & {[}N{]} Captain Pugwash            \\
                       & {[}R{]} comic\_strip                 & {[}C{]} comic\_strip               \\ \cdashline{2-3} \addlinespace[2pt]

                       & {[}ID{]} comic\_strip                 & {[}ID{]} \textcolor{blue}{m.02hcty}                  \\
                       & {[}N{]} \textcolor{blue}{character}                    & {[}N{]} Captain Pugwash            \\
                       & {[}R{]} comic\_strip\_character      & {[}C{]} comic\_strip\_character    \\
\cline{2-3} 
                        \addlinespace[2pt]& \multicolumn{2}{l}{(AND comic\_strips.comic\_strip (JOIN \textcolor{blue}{comic\_strips.comic\_strip.characters} \textcolor{blue}{m.02hcty}))}  (\ding{51})\\
\hline
\end{tabular}
\caption{Case study of logical form generation by \model\ and a representative competitor TIARA on the GrailQA validation set. Incorrect relations and entities are marked in red, while the correct relations and entities are colored in blue.}
\label{tab:lfg_case}
\end{table*}


To further show \model's generalizability to non-I.I.D. KBQA applications, we include a case study from the GrailQA validation set as shown in Tables~\ref{tab:md_case} and~\ref{tab:lfg_case}. 

\paragraph{Entity Mention Detection} 
Figure~\ref{tab:md_case} shows an entity mention detection example, comparing our entity detection module with SpanMD which is a mention detection method commonly used by SOTA KBQA models~\cite{shu_tiara_2022,ye_rng-kbqa_2022,faldu_retinaqa_2024}. In this case, SpanMD incorrectly detects \textsf{length} as an entity mention, which is actually part of the ground-truth relation (\textsf{measurement\_unit.$\ldots$.length\_units}) that is unseen in the training data. Our entity mention detection module, on the other hand, leverages the retrieved relations to generate a logical form sketch. The correct entity mention, \textsf{atomic units}, is isolated from the relations and can be corrected extracted, even though this entity mention has not been seen at training. %This example demonstrates that our entity mention detection module enhances the compositional generalization and zero-shot generalization capabilities of \model.   

%enabling it to detect different KB elements in the input question from a more comprehensive perspective. Compared to previous entity retrieval methods, it demonstrates stronger compositional generalization and zero-shot generalization capabilities.

\paragraph{Logical Form Generation}
Table~\ref{tab:lfg_case} shows a logical form generation example.
Here, \model\ and TIARA (a representative generation-based model) have both retrieved the same sets of relations in the retrieval stage which include false positives. The two models also share the same top-1 retrieved entity \textsf{m.04fgkzf}, while \model\ has retrieved a second entity \textsf{m.02hcty} in addition. 
TIARA is misled by the erroneous KB relations retrieved and produces an incorrect logical form. 
\model, on the other hand, is able to produce the correct logical form by leveraging the schema information (i.e., the entity's class and the relation's domain and range classes).



%the seq2seq model with KB context. This enables the model to understand the connections between KB elements and generate executable logical forms that align with the semantics of the question. 







\section{Error Analysis}\label{app:error_analysis}
Following TIARA~\cite{shu_tiara_2022} and Pangu~\cite{gu_dont_2023}, we analyze 200 incorrect predictions randomly sampled from each of the GrailQA
validation set and the WebQSP test set where our model predictions are different from the ground truth. The errors of \model\ largely fall into the following three types:

\begin{itemize}
    \item \textbf{Relation retrieval errors} (35\%). Failures in the relation retrieval step (e.g., failing to retrieve any ground-truth relations) can impinge the capability of our entity mention detection module to generate correct logical form sketches, which in turn leads to incorrect entity mention detection and entity retrieval.

    \item \textbf{Entity retrieval errors} (32\%). Errors in the entity mentions generated by the logical form sketch parser can still occur even when the correct relations are retrieved, because some complex and unseen entity mentions require domain-specific knowledge. An example of such entity mentions is `\textsf{Non-SI units mentioned in the SI}', which refers to units that are not part of the International System (SI) of Units but are officially recognized for use alongside SI units. This entity mention involves two concepts that are very similar in their surface forms (\textsf{Non-SI} and \textsf{SI}). Without a thorough understanding of the  domain knowledge (\textsf{SI} standing for \textsf{International System of Units}), it is difficult for the entity mention detection module to identify the correct entity boundaries. 


    \item \textbf{Logical form generation errors} (31\%). Generation of inaccurate or inexecutable logical forms can still occur when the correct entities and relations are retrieved. The main source of such errors is questions involving operators rarely seen in the training data (e.g., \textsf{ARGMIN} and \textsf{ARGMAX}). Additionally, there are highly ambiguous candidate entities that may confuse the model, leading to incorrect selections of entity-relation combinations. For example, for the question \textsf{Who writes twilight zone}, two candidate entities \textsf{m.04x4gj} and \textsf{m.0d\_rw} share the same entity name \textsf{twilight zone}. The former refers to a reboot of the TV series \textsf{The Twilight Zone} produced by Rod Serling and Michael Cassutt, while the latter is the original version of \textsf{The Twilight Zone} independently produced by Rod Serling. They share the same entity name and class (\textsf{tv.tv\_program}). There is insufficient contextual information for our logical form generator to  differentiate between the two. The generator eventually selected the higher-ranked entity which was incorrect, leading to producing an incorrect answer to the question \textsf{Rod Serling and Michael Cassutt}.
    
    % \jz{and the classes to which these entities belong overlap. [can you give a couple of these entities and what they are referring to?] These are difficult to disambiguate.} 
    \item The remaining errors (2\%) stem from incorrect annotations of comparative questions in the dataset. For example, \textsf{larger than} in a question is annotated as \textsf{LE} (less equal) in the ground-truth logical form.

\end{itemize}

% \jz{The remaining errors (2\%) ... [can we say something about these errors? Just 4 anyway?]}



%\begin{itemize}
% \item \textbf{Relation retrieval errors} (37\%). Failures in the relation retrieval step (e.g., failing to retrieve any ground-truth relations) can impinge the capability of our entity mention detection module to generate correct logical form sketches, which in turn leads to incorrect entity mention detection and entity retrieval. 

% \item \textbf{Entity retrieval errors} (32\%). Errors in the entity mentions generated by the logical form sketch parser can still occur even when the correct relations are retrieved, \jz{because...?}. 

% \item \textbf{Logical form generation errors} (31\%). Such  errors mainly arise from questions with complex semantics. \jz{example?} The limited number of complex questions in the training data makes it difficult for the LLM to learn and generate logical forms for such questions. 

%in the model making syntactic errors (such as in operators and functions) when generating logical forms for such complex questions. 
%\end{itemize}




\end{document}

% Custom bibliography entries only
%\bibliography{custom}
\clearpage
\appendix
\section{Detailed Taxonomy with Examples}
\label{sec:detailed_taxonomy}
Any references to ``URW'' and ``CC'' below denote the Ukraine-Russia War, and the Climate Change domains, respectively.

\subsection{Protagonist}

\textbf{Guardian}: Heroes or guardians who protect values or communities, ensuring safety and upholding justice. They often take on roles such as law enforcement officers, soldiers, or community leaders (e.g., climate change advocacy community leaders). 
\\\underline{Example}: Police officers protecting citizens during a crisis, firefighters saving lives during a disaster, community leaders standing against crime or leaders standing up for action to address climate change. 

\textbf{Martyr}: Martyrs or saviors who sacrifice their well-being, or even their lives, for a greater good or cause. These individuals are often celebrated for their selflessness and dedication. This is mostly in politics, not in CC.
\\\underline{Example}: Civil rights leaders like Martin Luther King Jr., who was assassinated while fighting for equality, or journalists who risk their lives to report on corruption and injustice. 

\textbf{Peacemaker}: Individuals who advocate for harmony, working tirelessly to resolve conflicts and bring about peace. They often engage in diplomacy, negotiations, and mediation. This is mostly in politics, not in CC.
\\\underline{Example}: Nelson Mandela's efforts to reconcile South Africa post-apartheid, or diplomats working to broker peace deals between conflicting nations. 

\textbf{Rebel}: Rebels, revolutionaries, or freedom fighters who challenge the status quo and fight for significant change or liberation from oppression. They are often seen as champions of justice and freedom. 
\\\underline{Example}: Leaders of independence movements like Mahatma Gandhi in India, or modern-day activists fighting for democratic reforms in authoritarian regimes. In CC domain, this includes characters such as Greta Thunberg, or persons who, for instance, chain themselves to trees to prevent deforestation.

\textbf{Underdog}: Entities who are considered unlikely to succeed due to their disadvantaged position but strive against greater forces and obstacles. Their stories often inspire others. 
\\\underline{Example}: Grassroots political candidates overcoming well-funded incumbents, or small nations standing up to larger, more powerful countries. In CC, this could included NEs portrayed as underfunded organizations that are framed as showing promise to make positive impact on CC. 


\textbf{Virtuous}: Individuals portrayed as virtuous, righteous, or noble, who are seen as fair, just, and upholding high moral standards. They are often role models and figures of integrity.
\\\underline{Example}: Judges known for their fairness, or politicians with a reputation for honesty and ethical behavior. In CC, this includes leaders standing up for environmental ethical values to protect planet Earth, or activists pushing for environmental sustainability.


\subsection{Antagonist}

\textbf{Instigator}: Individuals or groups initiating conflict, often seen as the primary cause of tension and discord. They may provoke violence or unrest.
\\\underline{Example}: Politicians using inflammatory rhetoric to incite violence, or groups instigating protests to destabilize governments. In CC, this could also include Greta Thunberg or activists chaining themselves to trees. In the previous example, they were portrayed in positive light as rebels. However, they could just as well be framed in a negative light if they are being portrayed as troublemakers and instigators of problems, and in such a scenario, they would also take the sub-role of Sabateur.

\textbf{Conspirator}: Those involved in plots and secret plans, often working behind the scenes to undermine or deceive others. They engage in covert activities to achieve their goals.
\\\underline{Example}: Figures involved in political scandals or espionage, such as Watergate conspirators or modern cyber espionage cases. In CC, this could manifest as persons or organizations conspiring to bypass environmental regulations to turn up a profit. 

\textbf{Tyrant}: Tyrants and corrupt officials who abuse their power, ruling unjustly and oppressing those under their control. They are often characterized by their authoritarian rule and exploitation.
\\\underline{Example}: Dictators like Kim Jong-un in North Korea, or corrupt officials embezzling public funds and suppressing dissent. 

\textbf{Foreign Adversary}: Entities from other nations or regions creating geopolitical tension and acting against the interests of another country. They are often depicted as threats to national security. This is mostly in politics, not in CC.
\\\underline{Example}: Rival nations involved in espionage or military confrontations, such as the Cold War adversaries, or countries accused of election interference. In CC, foreign adversaries could include portrayal of how other countries are not adhering to CC policies (e.g., China refuses to adhere to CC policies resulting in 20\% increase in CO2 emissions.

\textbf{Traitor}: Individuals who betray a cause or country, often seen as disloyal and treacherous. Their actions are viewed as a significant breach of trust. This is mostly in politics, not in CC.
\\\underline{Example}: Whistleblowers revealing sensitive information for personal gain, or soldiers defecting to enemy forces. Note that if whistleblowers are portrayed in a positive light, their role would be Virtuous. This could equally apply to both politics and CC.

\textbf{Spy}: Spies or double agents accused of espionage, gathering and transmitting sensitive information to a rival or enemy. They operate in secrecy and deception. This is mostly in politics, not in CC.
\\\underline{Example}: Historical figures like Aldrich Ames, who spied for the Soviet Union, or contemporary cases of corporate espionage.

\textbf{Saboteur}: Saboteurs who deliberately damage or obstruct systems, processes, or organizations to cause disruption or failure. They aim to weaken or destroy targets from within.
\\\underline{Example}: Insiders tampering with critical infrastructure, or activists sabotaging industrial operations.

\textbf{Corrupt}: Individuals or entities that engage in unethical or illegal activities for personal gain, prioritizing profit or power over ethics. This includes corrupt politicians, business leaders, and officials.
\\\underline{Example}: Companies involved in environmental pollution, executives engaged in massive financial fraud, or politicians accepting bribes and engaging in graft.

\textbf{Incompetent}: Entities causing harm through ignorance, lack of skill, or incompetence. This includes people committing foolish acts or making poor decisions due to lack of understanding or expertise. Their actions, often unintentional, result in significant negative consequences.
\\\underline{Example}: Leaders making reckless policy decisions without proper understanding, officials mishandling crisis responses, or managers whose poor judgment leads to organizational failures.

\textbf{Terrorist}: Terrorists, mercenaries, insurgents, fanatics, or extremists engaging in violence and terror to further ideological ends, often targeting civilians. They are viewed as significant threats to peace and security. This is mostly in politics, not in CC.
\\\underline{Example}: Groups like ISIS or Al-Qaeda carrying out attacks, or lone-wolf terrorists committing acts of violence.

\textbf{Deceiver}: Deceivers, manipulators, or propagandists who twist the truth, spread misinformation, and manipulate public perception for their own benefit. They undermine trust and truth.
\\\underline{Example}: Politicians spreading false information for political gain, or media outlets engaging in propaganda.

\textbf{Bigot}: Individuals accused of hostility or discrimination against specific groups. This includes entities committing acts falling under racism, sexism, homophobia, Antisemitism, Islamophobia, or any kind of hate speech. This is mostly in politics, not in CC.


\subsection{Innocent}

\textbf{Forgotten}: Marginalized or overlooked groups who are often ignored by society and do not receive the attention or support they need. This includes refugees, who face systemic neglect and exclusion.
\\\underline{Example}: Indigenous populations facing ongoing discrimination; homeless individuals struggling without adequate support; refugees fleeing conflict or persecution.

\textbf{Exploited}: Individuals or groups used for others' gain, often without their consent and with significant detriment to their well-being. They are often victims of labor exploitation, trafficking, or economic manipulation.
\\\underline{Example}: Workers in sweatshops; victims of human trafficking; communities suffering from corporate exploitation of natural resources.

\textbf{Victim}: People cast as victims due to circumstances beyond their control, specifically in two categories: (1) victims of physical harm, including natural disasters, acts of war, terrorism, mugging, physical assault, ... etc., and (2) victims of economic harm, such as sanctions, blockades, and boycotts. Their experiences evoke sympathy and calls for justice, focusing on either physical or economic suffering.
\\\underline{Example}: Victims of natural disasters, such as hurricanes or earthquakes; individuals affected by violent crimes. Victims of economic blockades, sanctions, or boycotts.

\textbf{Scapegoat}: Entities blamed unjustly for problems or failures, often to divert attention from the real causes or culprits. They are made to bear the brunt of criticism and punishment without just cause.
\\\underline{Example}: Minority groups blamed for economic problems; political opponents, accused of provoking national strife, without evidence.



\section{Annotation Guidelines}
\label{sec:annotation_guidelines}

We prepare these general set of guidelines to prepare the annotators and avoid human biases before starting the annotation:

\begin{itemize}
    \item The annotators should get acquainted with the two domains covered by the tasks; for instance, ~\cite{kremlin-propaganda} and~\cite{cc-denial-tax} provide a good coverage of the URW and CC domains,
    \item The annotators' opinions on the topics and sympathies towards key entities mentioned in the articles are irrelevant and should by no means impact the annotation process and their choices, 
    \item The annotators should not exploit any specific external knowledge bases for the purpose of annotating documents.
\end{itemize}

The annotation guidelines we prepare to annotate and curate the entity framing corpus are as follows:

Any references to ``URW'' and ``CC'' below denote the Ukraine-Russia War, and the Climate Change domains, respectively.

\begin{enumerate}

    \item In this annotation task, the entities of interest are understood in a broad sense to include both traditional named entities (such as persons, organizations, and locations) as well as toponym-derived entities. Toponym-derived entities are phrases that indicate a group or collective identity based on a place or affiliation, including but not limited to:
    \begin{itemize}
        \item Political, military, or social groups defined by their association with a location or entity, e.g., ``Trump supporters,'' or ``residents of Ukraine.''
        \item Entities denoting a geographic or organizational affiliation, such as ``Russian forces'' or ``European officials.''
    \end{itemize}
        

    \item The annotators are provided with a number of news articles and are expected to assign role(s) to named entities that are \textbf{central} to the article's story according to the taxonomy of roles that was provided earlier. 
    \item The annotators are provided with a detailed taxonomy that includes definitions and examples. 
    \item The title of an article should not be annotated. The title of the article is the first block of text that appears in the annotation platform Inception.
    \item Only named entities that are central to the narrative of the article should be annotated. Unnamed entities (i.e., nominal entity mentions such as ``migrants'') should not be annotated. 

For more details on what qualifies as a named entity, in addition to the definition of the broader sense of named entities in the first bullet point in these guidelines, the annotators should also examine the NER annotation guidelines outlined in \url{http://www.universalner.org/guidelines/}.
    
    \item The annotators will pick one or more fine-grained roles for the named entities they believe are central to the article's story. 
    \item Entity mentions can be assigned fine-grained roles from more than one main role. However, during curation, we will not be including these instances in the current version, even though we still annotate them.
    \item Named entities that are not central to the story should not be annotated.

The determination of how central a named entity is in an article is admittedly subjective. To reduce bias, such determination should be based on the careful reading of the article and the story it is pushing. An annotated example is provided in \ref{fig:annotated_example_text2}. Notice that named entities such as New York Times and Israel were not annotated because they are not central to the story.
    
    \item As a general rule, annotators should annotate only the first mention per entity where it is clear that this entity has the specific role(s). There is no need to annotate subsequent mentions of this entity with the same role, But  annotating more mentions with the same surface form and role is not a mistake, but it is simply not required.

    This rule also extends for surface mentions of the same entity. For example, ``Putin'' and ``Vladimir Putin'' are both surface mentions of the same entity, so only the first occurrence of any of the surface forms  would be annotated.

    On the other hand, while entities such as ``Moscow'', ``Russia'', and ``Putin'' are closely related, they are not surface forms of the same entity, and are considered as distinct, separate entities.

    \item Named entities appearing in indirect speech (e.g., quotes) should not be annotated. Indirect speech should be considered as supporting detail to the story, but named entities which are central to the narrative would likely appear also outside of quotations. This guideline helps avoid confusions inherent in indirect speech.

    \item If the above would result in more than one mention of the same entity with the same role, the curator does not need to remove all these additional mentions. We keep all of them.
    \item Should an entity mention that was previously annotated with a certain role appear in a different context with different role(s), the first mention where the role(s) changed should be annotated.

The above rule is repeated for as many times as an entity changes roles across mentions. For example, if an en entity, let's say NATO, appears 20 times in an article. The first 10 mentions show NATO as a Guardian and a Virtuous entity. The 11-15th mentions portray NATO as a Foreign Adversary, and the 16-20th mentions portray NATO as Exploited. Then we only need 3 annotations in total to account for the 3 different roles NATO was portrayed as. These 3 annotations should all be the first mention occurrences where NATO assumed each distinct set of roles (i.e., mention 1, mention 11, and mention 16 should be annotated).
    \item Regarding scenarios where different surface forms for the same named entity (e.g., NATO vs North Atlantic Treaty Organization) appear in the article, it is sufficient to pick only one of the surface forms.
    \item If the above results in multiple surface forms of the same entity being annotated, the curator does not need to remove all of these additional mentions. We keep all of them.
    \item If the mention of the entity does not have any role vis-a-vis the taxonomy of roles then no role should be given. As a consequence, we do not need an ``Other'' label. 
    % \item It is important that only information found in the article is used---the annotator must not rely on external knowledge, to avoid bias and subjectivity.
    \item The curator may see conflicting annotations in the curation mode and could resolve the conflict, and then the remaining non-conflicting roles could be checked and adopted accordingly. 


\end{enumerate}



\section{Experimental Settings}
\label{sec:appendix_experiments}

All fine-tuning experiments were conducted on a single NVIDIA RTX 4090 GPU with 24 GB of memory. We fine-tuned XLM-R (XLM-RoBERTa) in a single run, using a fixed random seed to ensure reproducibility. When the input context was at the sentence granularity, we performed sentence splitting using Stanza pipelines for each one of our five languages. For XLM-R, default settings were applied, with the following configurations:

\begin{itemize}
    \item Model: XLM-R\textsubscript{base} (125M parameters) 
    \item Learning Rate: 2e-5
    \item Batch Size: 8
    \item Epochs: 20 (with early stopping of 3 based on validation loss)
    \item Random Seed: 42
    \item Weight Decay: 0.01
\end{itemize}

To optimize performance, the sigmoid thresholds for fine-grained role predictions were tuned on the validation set. These optimized thresholds were then applied to generate predictions on the test set.

To prevent data leakage, we created train/dev/test splits based on entire articles rather than individual entity-mention annotations. The details of these splits are provided in Table~\ref{tab:train_dev_test_split}.

\begin{table}[]
\centering
\resizebox{\columnwidth}{!}{%

\begin{tabular}{lccccc|c}
\toprule
 & BG & EN & HI & PT & RU & All \\
\midrule
Train & 165 (389) & 133 (440) & 203 (1347) & 206 (833) & 89 (252) & 796 (3261) \\
Dev & 94 (237) & 69 (245) & 139 (983) & 100 (417) & 44 (114) & 446 (1996) \\
Test & 15 (30) & 27 (90) & 35 (279) & 31 (115) & 28 (85) & 136 (599) \\ \midrule
Total & 274 (656) & 229 (775) & 377 (2609) & 337 (1365) & 161 (451) & 1378 (5856) \\
\bottomrule
\end{tabular}
}
\caption{Distribution of articles and entity mentions by language and split. The number of entity mentions is shown in parentheses}
\label{tab:train_dev_test_split}
\end{table}

For the zero-shot experiments, we used OpenAI's GPT-4o (gpt-4o-2024-11-20) with a temperature setting of 0.2 to produce more conservative responses. To ensure the outputs conformed to our defined data types, we employed OpenAI's Structured Outputs API, which returned results in the expected JSON format.




\clearpage
\onecolumn

\section{Dataset Statistics}

\label{sec:appendix_stats}

\begin{figure*}[!h]
    \centering
    \includegraphics[width=1\textwidth]{images/fine_role_co_occurence_normalized.pdf}
    \caption{Normalized co-occurrence of fine-grained roles.}
    \label{fig:fine_roles_co_occurence_normalized}
\end{figure*}

\begin{figure}[!h]  % 't' ensures the figure is placed at the top of the page
    \centering

    % First Subfigure
    \begin{subfigure}[t]{0.9\textwidth}  % Width for the first subfigure
        \centering
        \includegraphics[width=\textwidth]{images/frequent_pairs.pdf}
        \caption{}
        \label{fig:frequent_pairs}
    \end{subfigure}
    \hfill  % Horizontal space to push subfigures apart
    
    % Second Subfigure
    \begin{subfigure}[t]{\textwidth}  % Width for the second subfigure
        \centering
        \includegraphics[width=\textwidth]{images/frequent_triplets.pdf}
        \caption{}
        \label{fig:frequent_triplets}
    \end{subfigure}

    % Main Caption for the Entire Figure
    \caption{The 10 most frequent co-occurring (a) pairs and (b) triplets of fine-grained roles.}
    \label{fig:frequent_roles_combined}
\end{figure}

\begin{figure}  
    % \captionsetup{font=scriptsize}

    % First Subfigure
    \begin{subfigure}[b]{0.9\textwidth}  % Width for the first subfigure
        \centering
        % \captionsetup{font=scriptsize}
        \includegraphics[width=\textwidth]{images/proportion_of_main_roles_per_language.pdf}
        \caption{}
        \label{fig:main_roles_per_language}
    \end{subfigure}
    % \hfill  % Horizontal space to push subfigures apart
    
    % Second Subfigure
    \begin{subfigure}[b]{\textwidth}  % Width for the second subfigure
        \centering
        % \captionsetup{font=scriptsize}
        \includegraphics[width=\textwidth]{images/proportion_of_fine_roles_per_language.pdf}
        \caption{}
        \label{fig:fine_roles_per_language}
    \end{subfigure}

    % Main Caption for the Entire Figure
    \caption{Proportions of (a) main roles and (b) fine-grained roles per language.}
    \label{fig:proportions_of_roles}
\end{figure}




% \begin{figure}[htbp]
%     \centering

%     % First Row: English and Portuguese
%     \begin{subfigure}[b]{0.45\textwidth}
%         \centering
%         \includegraphics[width=\textwidth]{images/word_cloud_EN.pdf}
%         \caption{}
%         \label{fig:wordcloud_en}
%     \end{subfigure}
%     \hfill
%     \begin{subfigure}[b]{0.45\textwidth}
%         \centering
%         \includegraphics[width=\textwidth]{images/word_cloud_PT.pdf}
%         \caption{}
%         \label{fig:wordcloud_pt}
%     \end{subfigure}

%     \vspace{0.5cm}

%     % Second Row: Bulgarian and Russian
%     \begin{subfigure}[b]{0.45\textwidth}
%         \centering
%         \includegraphics[width=\textwidth]{images/word_cloud_BG.pdf}
%         \caption{}
%         \label{fig:wordcloud_bg}
%     \end{subfigure}
%     \hfill
%     \begin{subfigure}[b]{0.45\textwidth}
%         \centering
%         \includegraphics[width=\textwidth]{images/word_cloud_RU.pdf}
%         \caption{}
%         \label{fig:wordcloud_ru}
%     \end{subfigure}

%     % \vspace{0.5cm}

%     % % Third Row: Hindi (centered)
%     % \begin{subfigure}[b]{0.45\textwidth}
%     %     \centering
%     %     \includegraphics[width=\textwidth]{images/word_cloud_HI.pdf}
%     %     \caption{}
%     %     \label{fig:wordcloud_hi}
%     % \end{subfigure}

%     % Main Caption for the Entire Figure
%     \caption{Word clouds for annotated entity mentions in different languages: (a) English, (b) Portuguese, (c) Bulgarian, and (d) Russian.}
%     \label{fig:wordclouds_all_languages}
% \end{figure}



\begin{figure}
    \centering
    \includegraphics[scale=0.75]{images/hist_ent.png}
    \caption{Top entities counts after multilingual linking was manually performed to link surface string to unique identifiers. The entities selected are all the ones for which at least one surface string has a count of a least 10 in any language}
    \label{fig:hist_ent}
\end{figure}

\begin{figure}
    \centering
    \includegraphics[scale=1.5]{images/heathmap_ent_2nd.png}
    \caption{Heathmap of the raw count of mention of entity and 2nd level role, where entities are defined and selected as in Figure~\ref{fig:hist_ent}}
    \label{fig:entity_2nd}
\end{figure}

\begin{figure}
    \centering
    \includegraphics[scale=0.22]{images/graph_topent_large.png}
    \caption{Graph of the top entities, nodes are a pair of entity and 2nd level role, node size is relative to the count of mentions, node color codes the 1st level role, there is and edge between two nodes, if they appear in the same document. This graph illustrate how group of entity+role pairs can be used to identify potential narratives.}
    \label{fig:enter-label}
\end{figure}



\clearpage
\section{Prompts for Hierarchical Zero-Shot Experiments}
\label{sec:appendix_zeroshot}


\begin{figure*}[htbp]
\centering
\begin{framed}
\begin{minipage}{0.9\textwidth}
\begin{Verbatim}[fontsize=\scriptsize,commandchars=\\\{\}]

\lightbrowntext{You are an expert at identifying entity framing and role portrayal in news articles. Analyze the following entity} 
\lightbrowntext{mention in context, and predict its main role and fine-grained role(s) from the taxonomy below.} 

\lightbrowntext{Taxonomy:} \darkbluetext{\{}\lightbluetext{detailed taxonomy with definitions and examples}\darkbluetext{\}}

\lightbrowntext{Context Around Entity:} \darkbluetext{\{}\lightbluetext{context}\darkbluetext{\}}

\lightbrowntext{Entity Mention:} \darkbluetext{\{}\lightbluetext{entity mention}\darkbluetext{\}}

\lightbrowntext{Task: Based on the provided context, assign to the entity mention at least one fine-grained role and}
\lightbrowntext{exactly one main role.}

\lightbrowntext{Return a JSON that has below attributes:}
\lightbrowntext{- \textbf{main role}: either one of Protagonist, Antagonist, or Innocent}
\lightbrowntext{- \textbf{fine grained roles}: a list of all your predicted fine-grained roles}

\end{Verbatim}
\end{minipage}
\end{framed}
\caption{\textbf{Single-Step Prompt Template.} The detailed taxonomy is the same one shown in \ref{sec:detailed_taxonomy}. The context is the text consisting of entity mention along with the 20 words before and after the entity mention.}
\label{fig:single_step_prompt_template}
\end{figure*}

\begin{figure*}[htbp]
\centering
\begin{framed}
\begin{minipage}{0.9\textwidth}
\begin{Verbatim}[fontsize=\scriptsize,commandchars=\\\{\}]
\darkbluetext{First Step (LLM Call 1): Predict the Main Role}

\lightbrowntext{You are an expert at identifying entity framing and role portrayal in news articles. Analyze the following entity }
\lightbrowntext{mention in context, and predict its main role from the taxonomy below.} 

\lightbrowntext{Taxonomy:} \darkbluetext{\{}\lightbluetext{list of fine-grained roles per main role}\darkbluetext{\}}

\lightbrowntext{Context Around Entity:} \darkbluetext{\{}\lightbluetext{context}\darkbluetext{\}}

\lightbrowntext{Entity Mention:} \darkbluetext{\{}\lightbluetext{entity mention}\darkbluetext{\}}

\lightbrowntext{Task: Based on the provided context, assign to the entity mention exactly one main role.}

\lightbrowntext{Return a JSON that has this attribute:}
\lightbrowntext{- \textbf{main role}: either one of Protagonist, Antagonist, or Innocent}


\darkbluetext{Second Step (LLM Call 2): Predict the Fine-Grained Role}

\lightbrowntext{You are an expert at identifying entity framing and role portrayal in news articles. This entity is}
\lightbrowntext{portrayed as a(n)} \darkbluetext{\{}\lightbluetext{main role}\darkbluetext{\}}\lightbrowntext{ and your task is to analyze the entity mention in context}
\lightbrowntext{and predict its fine-grained role(s) from the taxonomy below.}

\lightbrowntext{Taxonomy:} \darkbluetext{\{}\lightbluetext{pertinent portion of the detailed taxonomy with definitions and examples}\darkbluetext{\}}

\lightbrowntext{Context Around Entity:} \darkbluetext{\{}\lightbluetext{context}\darkbluetext{\}}

\lightbrowntext{Entity Mention:} \darkbluetext{\{}\lightbluetext{entity mention}\darkbluetext{\}}

\lightbrowntext{Task: Based on the provided context, assign to the entity mention at least one fine-grained role.}

\lightbrowntext{Return a JSON that has this attribute:}
\lightbrowntext{- \textbf{fine grained roles}: a list of all your predicted fine-grained roles}
\end{Verbatim}
\end{minipage}
\end{framed}
\caption{\textbf{Multi-Step Prompt Template.} In the first step, the taxonomy is only the tree structure of the taxonomy and does not include any definitions or examples. In the second step, the detailed taxonomy only includes the branch under the predicted main role in the first step. The context is as defined in Figure \ref{fig:single_step_prompt_template}. }
\label{fig:multi_step_prompt_template}
\end{figure*}


% \fbox{%
%     \parbox{\textwidth}{%
% \bf{\textcolor{red}{Single Step Prompt}}
% \\

%  Taxonomy Definitions: \\
% Main Roles: \{Main Roles Definitions\} \\
% Fine-Grained Roles: \{Fine Grained Roles Definitions\} \\
% You are an expert at identifying entity framing and role portrayal in news articles. Analyze the following entity mention in context, and predict its main role and fine-grained role(s) from the taxonomy below. \\
% \{taxonomy section\} \\
% \{document section\} \\
% Context Around Entity:
% \{context\} \\
% Entity Mention:
% \{entity mention\} \\
% Task: Based on the provided context, assign to the entity mention at least one fine-grained role and exactly one main role. Return a JSON that has below attributes: \\
% ``main role'': either one of Protagonist, Antagonist, or Innocent \\
% ``fine grained roles'': this is a list of all of your predicted fine-grained roles \\
% ... \\

%     }%
% }

% \fbox{%
%     \parbox{\textwidth}{%
% \bf{\textcolor{red}{Multi Step Prompts}} 
% \\

% \textcolor{blue}{First Step (LLM call): Define the main role}

%  Taxonomy Definitions: \\
% Main Roles: \{Main Roles Definitions\} \\
% You are an expert at identifying entity framing and role portrayal in news articles. Analyze the following entity mention in context, and predict its main role from the taxonomy below.\\
% \{taxonomy section\} \\
% \{document section\} \\
% Context Around Entity:
% \{context\} \\
% Entity Mention: \\
% \{entity mention\} \\
% Task:
% Based on the provided context, assign to the entity mention exactly one main role. Return a JSON that has below attributes: \\
% ``main role'': either one of Protagonist, Antagonist, or Innocent \\
% ... \\
% \textcolor{blue}{Second step (LLM call): Define the fine grained role (protagonist/antagonist/innocent)}

% You are an expert at identifying entity framing and role portrayal in news articles. This entity is portrayed as a protagonist/antagonist/innocent and your task is to analyze the following entity mention in context, and predict its role from the taxonomy below.\\
% \{taxonomy section\} \\
% \{document section\} \\
% Context Around Entity:
% \{context\} \\
% Entity Mention: \\
% \{entity mention\} \\
% Task:
% Based on the provided context, assign to the entity mention at least one fine-grained role. Return a JSON that has below attributes: \\
% "fine grained roles": this is a list of all of your predicted protagonist/antagonist/innocent fine-grained roles
%     }%
% }

%\clearpage
%\section{Top Entities}


% \begin{figure}[!h]
%     \centering
%     \includegraphics[scale=0.3]{images/st1_language-1st_normX.png}
%     \caption{language-1st role, normalised over language}
%     \label{fig:enter-label}
% \end{figure}


% \begin{figure}[!h]
%     \centering
%     \includegraphics[scale=0.3]{images/matrix_entity-language_normY.png}
%     \caption{top entity - language, normalised over language}
%     \label{fig:enter-label}
% \end{figure}

% \begin{figure}[!h]
%     \centering
%     \includegraphics[scale=0.3]{images/matrix_entity-language_normX.png}
%     \caption{top entities - language, normalised over entities}
%     \label{fig:enter-label}
% \end{figure}

\clearpage
\section{Annotation Tool}
\label{sec:annotation_tool}
We used the Inception~\cite{tubiblio106270} platform\footnote{https://inception-project.github.io/} to annotate our corpus because it has a rich set of features that extends beyond mere annotation to also include useful tools such as the ability to perform annotation adjudication through curation, monitoring the annotation progress, and calculating agreement between annotators. Inception allows to assign the following roles to users: annotator, curator, and manager.

To annotate a mention of an entity with a role, annotators should go to the part of the article where the entity is mentioned and select it. After selecting an entity mention, annotators can then assign roles as shown in \ref{fig:INC_entity}.


\begin{figure}[!htpb]
\centering
\includegraphics[width=1\textwidth]{images/Entity_layer.PNG}
\caption{Annotating entity framing using Inception.}
\label{fig:INC_entity}
\end{figure}






\end{document}



