% This must be in the first 5 lines to tell arXiv to use pdfLaTeX, which is strongly recommended.
\pdfoutput=1
% In particular, the hyperref package requires pdfLaTeX in order to break URLs across lines.


\documentclass[11pt]{article}

\usepackage{pgfplotstable}
\usepackage{multirow}

% Change "review" to "final" to generate the final (sometimes called camera-ready) version.
% Change to "preprint" to generate a non-anonymous version with page numbers.
\usepackage[final]{acl}

% Standard package includes
\usepackage{times}
\usepackage{latexsym}
\usepackage{booktabs}

% For proper rendering and hyphenation of words containing Latin characters (including in bib files)
\usepackage[T1]{fontenc}
% For Vietnamese characters
% \usepackage[T5]{fontenc}
% See https://www.latex-project.org/help/documentation/encguide.pdf for other character sets

% This assumes your files are encoded as UTF8
\usepackage[utf8]{inputenc}

% This is not strictly necessary, and may be commented out,
% but it will improve the layout of the manuscript,
% and will typically save some space.
\usepackage{microtype}

% This is also not strictly necessary, and may be commented out.
% However, it will improve the aesthetics of text in
% the typewriter font.
\usepackage{inconsolata}

%Including images in your LaTeX document requires adding
%additional package(s)
\usepackage{graphicx}
\usepackage{subcaption}
\usepackage{array}

\usepackage[utf8]{inputenc}
\usepackage{tcolorbox}
\usepackage{parskip} % Adds space between paragraphs
% \usepackage{newtxtext}    % Times New Roman-like font
\usepackage{graphicx}
\usepackage{pgf}

% Define the tcolorbox style
\tcbset{
  colframe=black,
  colback=white,
  fonttitle=\bfseries,
  sharp corners,
  boxrule=0.5pt,
  width=\columnwidth,
  left=4pt,
  right=4pt,
  top=4pt,
  bottom=4pt
}

\usepackage{svg}        % For including SVG files
\usepackage{caption}    % For customizing captions
\usepackage{subcaption}    % For customizing captions
\usepackage{tikz}
\usetikzlibrary{shapes.callouts}
\usetikzlibrary{positioning}
\usepackage{fancyvrb}      
\usepackage{framed}        
\usepackage{xcolor}        
\usepackage{graphicx}      
\usepackage{caption}       
\usepackage{amsmath}  
% \usepackage{newtxtext,newtxmath}
% Command to create labeled entity mentions
% \newcommand{\labeledentity}[3]{%
%   \tikz[baseline=(word.base)]{
%     % Label above the entity mention
%     \node[anchor=south, inner sep=2pt, fill=#3, rounded corners=2pt, text=white] (label) {\tiny #1};
%     % Entity mention with a box
%     \node[below=1pt of label, draw, thick, rounded corners=2pt, inner sep=2pt] (word) {#2};
%   }%
% }
% Define custom colors
\definecolor{lightbrown}{rgb}{0.71, 0.40, 0.11}
\definecolor{darkblue}{rgb}{0.00, 0.20, 0.60}
\definecolor{lightblue}{rgb}{0.20, 0.60, 0.86}

% Define text commands for convenience
\newcommand{\lightbrowntext}[1]{\textcolor{lightbrown}{#1}}
\newcommand{\darkbluetext}[1]{\textcolor{darkblue}{#1}}
\newcommand{\lightbluetext}[1]{\textcolor{lightblue}{#1}}
\definecolor{darkgreen}{rgb}{0, 0.5, 0}
\definecolor{darkred}{rgb}{0.7, 0, 0.1}
\definecolor{darkblue}{rgb}{0, 0, 0.5}

% Labeled entity command
\newcommand{\labeledentity}[3]{%
  \tikz[baseline=(word.base)]{
    \node[anchor=south, inner sep=2pt, fill=#3, rounded corners=2pt, text=white, font=\scriptsize] (label) {\tiny #1};
    \node[below=1pt of label, draw, thick, rounded corners=2pt, inner sep=2pt, font=\small] (word) {#2};
  }%
}
% \usepackage{tcolorbox}
% \usepackage{lipsum} % For placeholder text

% % Define the box style
% \tcbset{
%   colframe=black,
%   colback=white,
%   fonttitle=\bfseries\small,
%   sharp corners,
%   boxrule=0.5pt,
%   width=\textwidth,
%   left=2pt,
%   right=2pt,
%   top=2pt,
%   bottom=2pt
% }
% If the title and author information does not fit in the area allocated, uncomment the following
%
%\setlength\titlebox{<dim>}
%
% and set <dim> to something 5cm or larger.

\title{Entity Framing and Role Portrayal in the News}

% Author information can be set in various styles:
% For several authors from the same institution:
% \author{Author 1 \and ... \and Author n \\
%         Address line \\ ... \\ Address line}
% if the names do not fit well on one line use
%         Author 1 \\ {\bf Author 2} \\ ... \\ {\bf Author n} \\
% For authors from different institutions:
% \author{Author 1 \\ Address line \\  ... \\ Address line
%         \And  ... \And
%         Author n \\ Address line \\ ... \\ Address line}
% To start a separate ``row'' of authors use \AND, as in
% \author{Author 1 \\ Address line \\  ... \\ Address line
%         \AND
%         Author 2 \\ Address line \\ ... \\ Address line \And
%         Author 3 \\ Address line \\ ... \\ Address line}

\author{Tarek Mahmoud$^1$, 
  \textbf{Zhuohan Xie}$^1$,
  \textbf{Dimitar Dimitrov}$^2$,
  \textbf{Nikolaos Nikolaidis}$^3$, 
  \textbf{Purificação Silvano}$^4$, \\
  \textbf{Roman Yangarber}$^5$,
  \textbf{Shivam Sharma}$^6$,
  \textbf{Elisa Sartori}$^{7}$,
  \textbf{Nicolas Stefanovitch}$^{8}$, \\
  \textbf{Giovanni Da San Martino}$^{7}$
  \textbf{Jakub Piskorski}$^9$,
  \textbf{Preslav Nakov}$^1$,
  \\
  %{\small\tt~\Letter~\hspace{-0.12cm}jpiskorski@gmail.com}\\
	$^1$MBZUAI, 
    $^2$Sofia University "St. Kliment Ohridski", 
    $^3$Athens University of Economics and Business, \\
 $^4$University of Porto, 
    $^5$University of Helsinki,
    $^6$Indian Institute of Technology Delhi,
    $^{7}$University of Padova, \\
    $^{8}$European Commission Joint Research Centre,
    $^9$Institute of Computer Science, Polish Academy of Science \\
\href{tarek.mahmoud@mbzuai.ac.ae}{\{tarek.mahmoud, preslav.nakov\}@mbzuai.ac.ae}
%{\texttt dasan@math.unipd.it, jpiskorski@gmail.com, nicolas.stefanovitch@ec.europa.eu, preslav.nakov@mbzuai.ac.ae}
}

%\author{
%  \textbf{First Author\textsuperscript{1}},
%  \textbf{Second Author\textsuperscript{1,2}},
%  \textbf{Third T. Author\textsuperscript{1}},
%  \textbf{Fourth Author\textsuperscript{1}},
%\\
%  \textbf{Fifth Author\textsuperscript{1,2}},
%  \textbf{Sixth Author\textsuperscript{1}},
%  \textbf{Seventh Author\textsuperscript{1}},
%  \textbf{Eighth Author \textsuperscript{1,2,3,4}},
%\\
%  \textbf{Ninth Author\textsuperscript{1}},
%  \textbf{Tenth Author\textsuperscript{1}},
%  \textbf{Eleventh E. Author\textsuperscript{1,2,3,4,5}},
%  \textbf{Twelfth Author\textsuperscript{1}},
%\\
%  \textbf{Thirteenth Author\textsuperscript{3}},
%  \textbf{Fourteenth F. Author\textsuperscript{2,4}},
%  \textbf{Fifteenth Author\textsuperscript{1}},
%  \textbf{Sixteenth Author\textsuperscript{1}},
%\\
%  \textbf{Seventeenth S. Author\textsuperscript{4,5}},
%  \textbf{Eighteenth Author\textsuperscript{3,4}},
%  \textbf{Nineteenth N. Author\textsuperscript{2,5}},
%  \textbf{Twentieth Author\textsuperscript{1}}
%\\
%\\
%  \textsuperscript{1}Affiliation 1,
%  \textsuperscript{2}Affiliation 2,
%  \textsuperscript{3}Affiliation 3,
%  \textsuperscript{4}Affiliation 4,
%  \textsuperscript{5}Affiliation 5
%\\
%  \small{
%    \textbf{Correspondence:} \href{mailto:email@domain}{email@domain}
%  }
%}
\pgfplotsset{compat=1.18} 
\begin{document}
\maketitle
\begin{abstract}
We introduce a novel multilingual hierarchical corpus annotated for entity framing and role portrayal in news articles. The dataset uses a unique taxonomy inspired by storytelling elements, comprising 22 fine-grained roles, or archetypes, nested within three main categories: \emph{protagonist}, \emph{antagonist}, and \emph{innocent}. Each archetype is carefully defined, capturing nuanced portrayals of entities such as guardian, martyr, and underdog for protagonists; tyrant, deceiver, and bigot for antagonists; and victim, scapegoat, and exploited for innocents. The dataset includes 1,378 recent news articles in five languages (Bulgarian, English, Hindi, European Portuguese, and Russian) focusing on two critical domains of global significance: the Ukraine-Russia War and Climate Change. Over 5,800 entity mentions have been annotated with role labels. This dataset serves as a valuable resource for research into role portrayal and has broader implications for news analysis. We describe the characteristics of the dataset and the annotation process, and we report evaluation results on fine-tuned state-of-the-art multilingual transformers and hierarchical zero-shot learning using LLMs at the level of a document, a paragraph, and a sentence. 

% Compared to naive single-step prompting, hierarchical prompting reduces costs by a staggering $40.15\%$. It also marginally improves accuracy for main role predictions by $0.6\%$, though it results in a $3.38\%$ drop in F1 for fine-grained role predictions across all languages.
\end{abstract}

\begin{figure*}[!htbp]
\centering
\footnotesize
\fbox{%
     \parbox{1\textwidth}{%\
    % \fontsize{6}\selectfont
     {\centering \textbf{Putin says what Russia needs to do to win special operation in Ukraine } \\[1em]}

    
    Russia will win the special operation in Ukraine if the society shows consolidation and composure to the enemy, President Vladimir Putin said during a visit to the Ulan-Ude Aviation Plant on March 14, Rossiya 24 TV channel said.
    
    Russia is not improving its geopolitical position in Ukraine. Instead, \labeledentity{Underdog}{Russia}{darkblue} is fighting "for the survival of Russian statehood, for the future development of the country and our children."
    
    "In order to bring peace and stability closer, we, of course, need to show the consolidation and composure of our society. When the enemy sees that our society is strong, internally braced up, consolidated, then, without any doubt we will come to reach what we are striving for — both success and victory," Putin said.
    
    According to him, many of the current problems began after the collapse of the Soviet Union, when they tried to put pressure on \labeledentity{Victim}{Russia}{darkgreen} to "destabilise the internal political situation.” "Hordes of international terrorists" new sent to the purpose to accomplish this goal, Putin said.
    
    Afterwards, the West decided to start rehabilitating Nazism in Russia's neighbouring states, including in Ukraine.
    
    Nevertheless, Putin continued, Russia had long tried to build partnerships with both Western countries and Ukraine. However, after 2014, when the West contributed to the coup in Ukraine, the state of affairs changed dramatically. It was then when they started exterminating those who advocated the development of normal relations with Russia, he said.
    
    According to Putin, \labeledentity{Guardian}{Russia}{darkblue} was forced to launch the special operation to protect the population. \labeledentity{Saboteur}{Western countries}{darkred} were hoping to break Russia quickly, but they were wrong, he said adding that \labeledentity{Virtuous}{Russia}{darkblue} managed to raise its economic sovereignty since 2022.
    
    }%
}
\caption{Annotated example color-coded according to the main roles in the taxonomy: \textcolor{darkred}{red} for \emph{antagonist}, \textcolor{darkblue}{blue} for \emph{protagonist}, and \textcolor{darkgreen}{green} for \emph{innocent}.}
\label{fig:annotated_example_text2}
\end{figure*}


\section{Introduction\label{sec:intro}}
The rapid proliferation of social media has dramatically transformed the information landscape, providing immediate access to news, and allowing anyone to propagate their narratives across the globe. While this connectivity functions as a convenient avenue for information dissemination, it also heightens the risk of exposure to biased reporting, propaganda, and narrative manipulation. These risks are particularly pronounced during periods of conflict and political upheavals, where the framing of entities—individuals, organizations, or groups—can profoundly influence public perception and decision-making. Understanding how entities are portrayed in the news is essential for fostering media literacy, identifying bias, and ensuring transparent news consumption.

Social science highlights the role of emotion in framing—selecting elements that evoke affective responses to shape perceptions \cite{https://doi.org/10.1111/j.1467-9221.2004.00354.x}. Emotional framing often leverages language that elicits specific feelings, such as fear, anger, or compassion, to influence how entities and events are understood \cite{iyengar_is_1991, nabi_exploring_2003,https://doi.org/10.1111/j.1460-2466.2000.tb02843.x,brader_campaigning_2006}. For instance, referring to a group as ``freedom fighters'' versus ``terrorists'' not only frames their role but also activates distinct emotional reactions. Research has shown that emotional appeals are powerful tools for reinforcing or challenging public attitudes \cite{westen_political_2008,lerner_emotion_2015}. Such framing can manifest through specific linguistic cues and includes the portrayal of entities also defined by \citet{schneider_social_1993} as the \emph{Social Construction of Target Populations}.

Natural language processing research has increasingly been applied to analyze the emotional dimensions of framing \cite{troiano-etal-2023-relationship}, including the identification of sentiment \cite{zhang-etal-2024-sentiment,app13074550} and emotion-laden narratives \cite{mousavi-etal-2022-emotion}. Understanding these emotional components provides deeper insight into how media narratives construct and perpetuate particular representations of entities, ultimately shaping public perception and societal discourse.

Given the large scale and complexity of the modern news ecosystem, effective analysis of entity framing requires automated tools, which depend on high-quality annotated data. In this context, we introduce a new multilingual dataset designed to develop tools for the study of entity framing and role portrayal in news articles. Our dataset uses a unique, hierarchical taxonomy inspired by elements of storytelling containing a set of 22 carefully defined archetypes nested under three main roles: \emph{protagonist}, \emph{antagonist}, and \emph{innocent}.

% \jp{Our dataset is not meant for studying news articles directly but for training models to facilitate carrying out such studies on representative corpora. Our corpus might not be represenative to draw any conclusions. I suggest to reword accordingly. The non-representiveness of our corpus should be mentioned in Limitations} 
% Agreed. Done

The corpus spans 1,378 recent news articles in five languages (Bulgarian, English, Hindi, European Portuguese, and Russian) and focuses on two globally significant domains: the Ukraine-Russia war and climate change. We annotated over 5,800 entity mentions with detailed role labels.


Our contributions can be summarized as follows:

\begin{itemize}
    \item We release a novel multilingual dataset annotated for entity framing and role portrayal, complete with detailed annotation guidelines.
    \item We introduce a comprehensive hierarchical taxonomy for entity roles validated on a large set of documents, supporting analysis at both the coarse and the fine-grained levels.
    \item We provide comprehensive dataset statistics, exploring entity portrayals across languages and topics.
    \item We set benchmarks using state-of-the-art multilingual transformer models, and hierarchical zero-shot learning with LLMs.
\end{itemize}



\section{Related Work\label{sec:related_work}}

\citet{sharma-etal-2023-characterizing} introduced a dataset for identifying heroes, villains, and victims in memes, focusing on visual content. In contrast, our dataset focuses on textual content. While both share similar coarse-level roles, our work adds an additional layer of granularity with a hierarchical taxonomy of 22 archetypes nested within these roles. 

\citet{card-etal-2016-analyzing} addressed a different aspect of framing. Their contribution is developing a model that makes use of personas to infer the article framing as defined in the Media Frames Corpus (MFC)  \cite{card-etal-2015-media}. MFC focuses on identifying how an article is framed along nine dimensions, such as Economic or Political. In their work, topic modeling is used to identify 50 personas. However, the results are noisy, with only a few informative personas, namely \emph{refugee} and \emph{immigrant}, while the 48 other personas, such as Job, Worker, and Year, are less informative. In contrast, our work focuses on \emph{entity} framing rather than article framing. While their work identifies personas in a weakly supervised manner, we develop a hierarchical taxonomy containing a richer set of roles, validated through human annotation of news articles across two diverse domains, ensuring higher quality and broader applicability. There is more research on news framing that focuses on article-level framing \cite{Pastorino2024DecodingNN, DBLP:conf/acl/0001KF24, piskorski-etal-2023-multilingual, liu-etal-2019-detecting, card-etal-2015-media}. In contrast, our work centers on entity-level framing.

Aspect-Based Sentiment Analysis \cite{chebolu-etal-2024-oats, DBLP:journals/corr/abs-2203-01054, orbach-etal-2021-yaso, jiang-etal-2019-challenge, saeidi-etal-2016-sentihood} is also related. It involves identifying targets of specific opinions and determining the polarity of the sentiment associated with particular aspects of these targets. Typically, the polarity is binary, and multiple aspects of the target entity are examined. Our work on entity framing is different as we do not define aspects nor do we assign polarities. Instead, we introduce a hierarchical taxonomy for news, which contains a rich set of roles inspired by elements of storytelling, and entities can be classified into any subset of roles within that taxonomy.




\section{The Entity Framing Task}
Entity framing focuses on analyzing how a text portrays a specific entity through word choice and narrative structure. More concretely, given a news article and a list of entity mentions (i.e., entity mentions, along with their span offsets), we assign to each of them one or more roles based on the taxonomy shown in Figure \ref{fig:taxonomy_roles}. We developed this taxonomy specifically for this task and the roles were inspired by storytelling elements. The taxonomy includes 22 archetypes, or fine-grained roles nested under three main roles: \emph{protagonist}, \emph{antagonist}, and \emph{innocent}. The role an entity plays in a given article may differ from one context in that article to another depending on the portrayal. See Figure~\ref{fig:annotated_example_text2} for a complete, annotated example from the corpus.

% \textbf{Protagonist} is an entity portrayed in a favorable or sympathetic light, often seen as standing up for a noble cause, protecting others, or striving for justice. 

% \textbf{Antagonist} is an entity framed in an unfavorable or adversarial manner, often depicted as contributing to conflict, harm, or injustice. Antagonists are commonly portrayed as oppressive, deceitful, corrupt, or threatening.


% \textbf{Innocent} is an entity presented as neutral, victimized, or undeserving of harm or blame. Innocents are often characterized as vulnerable, exploited, or caught in a conflict through no fault of their own.


Entity framing can be formalized mathematically as follows. Let $R$ be a tree structure representing the taxonomy of roles. Let $S$ be a string of length $|S|$ characters with the content of the full article. The goal of entity framing is to learn a function
\begin{equation}
f: (S, [i,j]) \rightarrow \{r_1, r_2, \ldots, r_k\} \subseteq R
\end{equation}

\noindent where $0 \leq i < j \leq |S|$ and $\{r_1, r_2, \ldots, r_k\}$ is the set of roles assigned to the span $[i,j]$.





\section{Taxonomy of Research on SDN Software Security}\label{sec:tx}
To systematically extract insights and understand the current state-of-the-art in SDN software security, our SLR focuses on analyzing specific features of each publication. The primary outcome of this analysis is developing a novel, four-dimensional taxonomy. This taxonomy will structure the body of existing research and directly address the research questions outlined in Section\ref{sec:rqs}.
\subsection{Structure of the Taxonomy}
The proposed taxonomy is a four-dimensional model designed to categorize and analyze the research landscape on SDN software security. The dimensions and their defining features are as follows:
\begin{itemize}
    \item \textbf{Objectives (What):} This dimension identifies the security goals targeted by the research. Objectives include bug detection, fixing, localization, exploitation, mitigation, categorization, and hardening.
    %This dimension classifies the security goals research studies aim to achieve or address. Seven recurring objectives have been identified, including but not limited to bug detection, attack detection/prevention, and performance/scalability optimization.
    \item \textbf{Targets (Where):} This dimension focuses on the specific SDN software components subject to security analysis or investigation. Common targets encompass controllers, data planes, APIs, and SDN applications.
    \item \textbf{Methodology (How):}  This dimension categorizes the diverse research methodologies employed in the reviewed literature. These methodologies can be further subdivided into testing approaches (e.g., static analysis, dynamic testing), testing types (e.g., white box, black box, gray box), and specific analysis techniques (e.g., model checking, fuzzing, symbolic execution).
    \item \textbf{Representations (Which):} This dimension encompasses the various approaches used to represent and structure information related to the testing process. The choice of representation can significantly impact the efficiency, comprehensibility, and effectiveness of test execution.
\end{itemize}
Figure\ref{fig_txn} provides a visual representation of the proposed four-dimensional taxonomy.
\begin{figure}[ht!]
\centering
\begin{adjustbox}{width=\linewidth, center}
\includegraphics{Diagram2.png}
\end{adjustbox}
\caption{Taxonomy on Security of SDN Software.}
\label{fig_txn}
\end{figure}







% In summary, the task involves finding a function $f$ that maps each character position in the span $(i,j)$ of an entity mention in the string $S$ to a subset of roles from the taxonomy $R$.


\section{Corpus Description}

\subsection{Domains}
\label{sec:corpus_domains}
The articles used in the task cover the following domains: (1) \emph{Ukraine-Russia War}, which includes articles about the war that started in February 2022 when Russia launched a full-scale invasion of Ukraine and began occupying parts of the country, and (2) \emph{Climate Change}, which encompasses both climate change denial (characterized by rejecting, refusing to acknowledge, disputing, or fighting the scientific consensus on climate change), and climate change activism.
% \paragraph{Climate Change (CC):} encompasses both climate change denial (characterized by rejecting, refusing to acknowledge, disputing, or fighting the scientific consensus on climate change), and climate change activism,
% \paragraph{Ukraine-Russia war (URW):} includes articles about the war that started in February 2022 when Russia launched a full-scale invasion of Ukraine and began occupying parts of the country.

\subsection{Article Selection}
% For each language, articles were primarily selected from links provided by the \textbf{Europe Media Monitor}~\footnote{https://emm.newsbrief.eu/} on the basis of multilingual keyword based categories already developed by the project \jp{I am not sure we should refer to EMM here at submission time due to potential revealing the identity. Say instead, large-scale in-house news aggregation .... bla bla . 
% Also we did select specific sources, which needs to be mentioned too}, 

For each language, articles were primarily selected from links we obtained from a large-scale in-house news aggregation tool. We performed the first candidate article selection based on multilingual keyword-based filters and we perform several steps, which we follow by to enrich the selection to match the criteria discussed below. To select the articles, we followed these steps:

\paragraph{Initial Collection:}  
Articles were scraped and filtered based on criteria such as word count (e.g.,~only articles exceeding 250 words were selected). For duplicate articles, the version with the higher number of words was preferred.

\paragraph{Filtering:}  
Each article was manually reviewed to determine its relevance to the annotation task. The articles were categorized into four groups: Perfect Fit, Average Fit, Uncertain (requiring further validation by language coordinators), or Unfit (excluded from annotation). Only articles classified as \emph{Perfect Fit} and \emph{Average Fit} were considered for annotation. 
% \nn{In EN, we included very few not directly relevant but adjacent topics to make the inclusion a bit more complete which was important for ST2.}
Afterwards, we used a zero-shot classifier with selected key phrases, and a persuasiveness score using the persuasion technique classifier as described in \citet{nikolaidis-etal-2024-exploring}. These scores were used to further enrich the selection with relevant articles.

% \paragraph{Language-Specific Sources:}  
% In addition to JRC-provided\jp{anonymity issue again!} articles, we used additional sources to capture diverse perspectives:
Additional sources were also incorporated to ensure diversity of perspectives for two languages: for Hindi, we selected articles from mainstream and bias-specific outlets (e.g., NDTV, The Hindu, OpIndia), and for Portuguese, from newspapers and political websites (e.g., \emph{O Diabo}, Esquerda, Folha Nacional) that had more controversial opinion texts about the relevant topics.
% \begin{itemize}
%     % \item \textbf{Bulgarian (BG):} Articles were selected both from mainstream and alternative media outlets.
%     % \jp{this is entirely unclear what this means} \dd{Nothing more to be added for Bulgarian}
%     % \nn{I assume texts == posts.}
%     \item \textbf{Hindi (HI):} Articles were selected from mainstream and bias-specific outlets (e.g., NDTV, The Hindu, OpIndia).
%     \item \textbf{Portuguese (PT):} Articles were sourced from newspapers and political websites (e.g., \emph{O Diabo}, Esquerda, Folha Nacional) which had more controversial opinion texts about the relevant topics.
%     % \jp{add urls? say something about them?}.
%     %\item \textbf{Russian (RU):} Articles were filtered based on loaded language and relevant keyword filters.
%     % \jp{in what sense? more loaded language means more suitable? Clarify}
%     % \nn{ I would remove RU here alltoghether, since I described the general approach above.}
%     % \nn{Please avoid references to EMM  
%     % due to anonymity concerns.}
    
% \end{itemize}

% \jp{The description above is somewhat unclear, ie. how and why were the additional sources selected?}
% \nn{On EN we did not follow any more steps than that.}


\subsection{Annotation Process}

Given that our corpus contains articles in five languages, we had one annotation team per language, each led by a language coordinator. Each language team included 3 to 5 annotators with prior experience in linguistics, social science, international relations, or prior annotation work. The annotators studied our detailed guidelines, attended live demonstrations, and completed real-time annotation exercises. During weekly meetings, teams clarified any uncertainties, resolved conflicts, improved consistency, and revised the annotation guidelines.
% \jp{and tuned the guidelines?} Yes

Each article was annotated by two annotators. Designated curators, often language coordinators or experienced annotators, reviewed and consolidated all annotations. They resolved the discrepancies through discussions with the respective teams. Over time, as the annotation quality improved, the curators reduced the frequency of checks but continued to perform random quality checks to ensure that annotations were of high quality. We used the Inception tool~\cite{tubiblio106270} for annotating and curating the corpus. See Appendix~\ref{sec:annotation_tool} for more details.
% \jp{consolidated?}\jp{annotations?} yes.

%Some challenges arise during annotation. Annotators initially struggled with spurious labels due to the annotation bias surrounding the topics under consideration. For example, the preconceived notion that \emph{entity A} was a natural offender (and not \emph{entity B}) in an ongoing conflict may cause annotation bias. Annotators also found richer content for URW-related topics due to the abundance of relevant narratives and entities. In contrast, CC articles lacked clear entity roles and argumentative content, resulting in sparser annotations. Weekly meetings were instrumental in refining and converging to the current annotation guidelines outlined in Appendix \ref{sec:annotation_guidelines}. 

The annotation guidelines (Appendix \ref{sec:annotation_guidelines}) were refined during initial weekly meetings between language coordinators and annotators. From these guidelines, we emphasize key aspects of entity selection for annotation. We annotated traditional named entities, extending this to also include eponym-derived entities (e.g., \emph{Putin supporters}) and toponym-derived entities (e.g., \emph{Western countries}, as illustrated in Figure~\ref{fig:annotated_example_text2}). Additionally, we focused on annotating entities that are central to the narrative conveyed by the article. For example, in Figure~\ref{fig:annotated_example_text2}, entities such as \emph{Ulan-Ude Aviation Plant} and \emph{Rossiya 24 TV} were not annotated because they were not considered central to the narrative. For a detailed explanation of how centrality was defined, refer to the guidelines in Appendix~\ref{sec:annotation_guidelines}.



\begin{table*}[ht]
\small
\centering
\resizebox{0.9\textwidth}{!}{%
\begin{tabular}{c|rrrrrrrrr|rrrr}
\toprule
% Lang. & \#DOC & \#PAR & \#SEN & \#WORD & \#CHAR & AVG_p & AVG_s & AVG_w & AVG_c & \#ENT (UNIQ) & \#ANN & AVG_e & AVG_a\\
\multicolumn{1}{c|}{Lang.} & 
\multicolumn{1}{c}{\#DOC} & 
\multicolumn{1}{c}{\#PAR} & 
\multicolumn{1}{c}{\#SEN} & 
\multicolumn{1}{c}{\#WORD} & 
\multicolumn{1}{c}{\#CHAR} & 
\multicolumn{1}{c}{AVG\textsubscript{p}} & 
\multicolumn{1}{c}{AVG\textsubscript{s}} & 
\multicolumn{1}{c}{AVG\textsubscript{w}} & 
\multicolumn{1}{c|}{AVG\textsubscript{c}} & 
\multicolumn{1}{c}{\#ENT} & 
\multicolumn{1}{c}{\#ANN} & 
\multicolumn{1}{c}{AVG\textsubscript{e}} & 
\multicolumn{1}{c}{AVG\textsubscript{a}}\\
\midrule
BG    & 274    & 2.7K  & 5.0K  & 104K  & 584K   & 9.8  & 18.5 & 380.6 & 2129.8 & 656 (179)     & 742   & 2.4 & 2.7 \\
EN    & 229    & 3.7K  & 5.3K  & 131K  & 705K   & 16.2 & 23.1 & 571.3 & 3080.7 & 775 (413)     & 843   & 3.4 & 3.7 \\
HI    & 377    & 3.8K  & 8.7K  & 200K  & 947K   & 10.2 & 23.0 & 530.1 & 2513.0 & 2,609 (724)   & 3,030 & 6.9 & 8.0 \\
PT    & 337    & 3.5K  & 5.4K  & 150K  & 822K   & 10.5 & 15.0 & 445.0 & 2439.2 & 1,365 (440)   & 1,438 & 4.1 & 4.3 \\
RU    & 161    & 0.7K  & 2.0K  & 42K   & 257K   & 4.5  & 12.7 & 261.1 & 1599.1 & 451 (265)     & 477   & 2.8 & 3.0 \\\midrule
Total & 1,378  & 14.5K & 26.4K & 627K  & 3,316K & 10.5 & 19.2 & 455.0 & 2406.3 & 5,856 (1,962) & 6,530 & 4.2 & 4.7 \\
\bottomrule
\end{tabular}
}
\caption{Corpus statistics showing total number of documents (\#DOC), paragraphs (\#PAR), sentences (\#SEN), words (\#WORD), and characters (\#CHAR) by language. The averages (AVG\textsubscript{p}), (AVG\textsubscript{s}), (AVG\textsubscript{w}), and (AVG\textsubscript{c}) refer to the average number of paragraphs, sentences, words, and characters per document, respectively. The table also shows the total number of annotated entity mentions (\#ENT) accompanied with unique counts in parentheses, the total number of annotations (\#ANN), the average number of entity mentions per document (AVG\textsubscript{e}), and the average number of annotations per document (AVG\textsubscript{a}).}
\label{tab:corpus_stats}
\end{table*}

% \begin{table}[h]
% \small
% \centering
% \resizebox{1\columnwidth}{!}{%
% \begin{tabular}{llllll}
% \toprule
% Language & Entity Mentions (unique) & Annotations & AVG_e & AVG_f \\
% \midrule
% BG & 656    (179) & 742 & 2 & 3 \\
% EN & 775    (413) & 843 & 3 & 4 \\
% HI & 2,609  (724) & 3,030 & 7 & 8 \\
% PT & 1,365  (440) & 1,438 & 4 & 4 \\
% RU & 451    (265) & 477 & 3 & 3 \\ \midrule
% ALL & 5,856 (1,962) & 6,530 & 4 & 5 \\
% \bottomrule
% \end{tabular}
% }
% \caption{Summary statistics of entity mentions, unique entities, and fine-grained roles by language. AVG}
% \label{tab:corpus_stats}
% \end{table}

\begin{table}[!h]
\small
    \centering
\begin{tabular}{ccccccc}
\toprule
Lang. & EN & RU & BG & PT & HI & All\\
\hline
$\alpha$ & .460 & .436 & .733 & .467 & .461 & .558\\
\bottomrule
\end{tabular}
    \caption{Inter-annotator agreement computed using Krippendorff's $\alpha$.}
    \label{tab:iaa}
\end{table}

\begin{table}[!h]
    \small
    \centering
    \resizebox{\columnwidth}{!}{%
    \begin{tabular}{lllllll}
        \toprule
        \textbf{Freq.} & \textbf{1} & \textbf{2-5} & \textbf{5-10} & \textbf{10-20} & \textbf{20-50} & \textbf{50-500} \\
        \midrule
        count (\%) & 1513 (74) & 374 (18) & 76 (4) & 46 (2) & 22 (1) & 14 (1) \\
\bottomrule
\end{tabular}
}
\caption{Proportion of entities of a given frequency in the corpus.}
    \label{tab:proportion}
\end{table}


% \jp{to converge abd} Abd?



\subsection{Inter-Annotator Agreement}

To assess the inter-annotator agreement (IAA), we used Krippendorff's alpha. We compared the annotations at the span level: they were approximately matched if they shared at least 50\% of their length, to account for minor differences.
The IAA values are shown in Table~\ref{tab:iaa}. The results indicate a moderate agreement (above 0.45), which we consider acceptable due to the span-based nature of the task. We can see that the IAA is similar across the languages, except for Bulgarian, for which it is notably higher (0.73), which can be explained by the low count of entities in Table~\ref{tab:corpus_stats}.  %The overall IAA is a little below the recommended value of 0.667. Nevertheless, one needs to take into account, first, that given the complex annotation scheme, these values are not outside the range for tasks of similar complexity~\cite{}\jp{reference?}, second, that IAA reflects agreement at the level of annotation, therefore not accounting for the quality increase\jp{improvement?} after to the curation step.\jp{Do we say anything about entity mentions annotated only by single annotators? What fraction of the annotations do they cover? This would be interesting to know.}
%\nn{Indeed, but in EN at least it was a large portion and it could be quite tough to defend.}
\



%Please note that because the specific entities roles depend on a context which is not captured by the span, it is not possible to evaluate agreement per entity, neither to perform a cross-lingual check of coherency.\jp{but we know the context. Hence this sentence doe not read well :-)}
%\nn{I agree, it sounds like, we did not do a good job making a tool that could highlight that. I would remove the paragraph altoghther.}
\subsection{Corpus Analysis}

\subsubsection{Statistics}

Table \ref{tab:corpus_stats} provides overall statistics about the corpus, including a breakdown per language, the number of annotated entity mentions, the number of unique entities, as well as the number of annotations. Figure~\ref{fig:fine_roles_histogram} displays the distribution of the main and fine-grained roles in the corpus. We can see that there is a significant imbalance between the fine-grained roles, while the distribution of the main roles is relatively balanced. Notably, within the \emph{innocent} category, the majority of the instances, $83.6\%$, are labeled as \emph{victim}, with fewer occurrences of \emph{exploited}, \emph{forgotten}, and \emph{scapegoat} roles. More details about the proportions of main roles and fine-grained roles across different languages can be found in Figure~\ref{fig:proportions_of_roles} in Appendix~\ref{sec:appendix_stats}. 

% Table \ref{"eda/corpus_domain_distribution} outlines the distribution of our corpus across both domains CC and URW.


\begin{figure*}[htbp]
    \centering
    \includegraphics[width=0.9\textwidth]{images/fine_roles_histogram.pdf}
    \caption{Distribution of fine-grained roles color-coded according to the main role. For the fine-grained roles, the percentages inside the bars indicate the proportion of each fine-grained role relative to its corresponding main role category. The counts on top of the bars show the total occurrences of each fine-grained role. In the mini histogram, the percentages inside the bars reflect the distribution of the main roles, with the counts displayed above the bars. }
    \label{fig:fine_roles_histogram}
\end{figure*}

Table~\ref{tab:proportion} presents %\jp{maybe a graphical representation of the table would make more sense}
the number and the proportion of entities within a given frequency range considering exact string matching. We can see that 74\% of the entities were annotated only once, while 14 entities have been annotated more than 50 times
%\jp{Could we list them here, give examples, maybe in a separate table. We have space for that!}. \dd{I agree with listing at least some interesting ones. Like those with a lot of mentions}
These numbers consider only the surface string, not accounting for different grammatical and name variants and different languages. In Appendix~\ref{sec:appendix_stats}, we matched the most frequent entities accounting for these differences and presented detailed statistics at the level of entities. 




\subsubsection{Co-occurrence of Roles}

In our definition of entity framing, entity mentions can be assigned one or more fine-grained roles. Our corpus contains on average 1.12 annotations per entity mention. Out of the 1,378 articles, 353 articles contain 638 instances where at least one entity mention has multiple annotations. For this set of articles, the average and the maximum are 2.1 and 3 annotations per entity mention, respectively. We further observe that roles such as \emph{peacemaker} frequently accompany \emph{guardian}. Similarly, entities portrayed as \emph{scapegoats} are often framed as being \emph{exploited}. More details about the co-occurrence matrix for entity mentions with multiple annotations are shown in Figures \ref{fig:frequent_roles_combined} and \ref{fig:fine_roles_co_occurence_normalized} in Appendix~\ref{sec:appendix_stats}.
% Table \ref{tab:multiple_annotation_stats} shows per-language statistics for the  instances.

% \begin{table}
%     \small
%     \centering
%     \resizebox{0.5\columnwidth}{!}{%
% \begin{tabular}{l|rrc}
% \toprule
%  Lang. & \#ENT & AVG_a & MAX_a \\
% \midrule
% BG & 71 & 2.21 & 3 \\
% EN & 68 & 2.00 & 2 \\
% HI & 404 & 2.04 & 3 \\
% PT & 70 & 2.04 & 3 \\
% RU & 25 & 2.04 & 3 \\
% \bottomrule
% \end{tabular}
% }
% \caption{Statistics for entity mentions containing 2 or more annotations. 

% % (\#ENT) is the number of such instances, while (AVG\textsubscript{a}) and (MAX\textsubscript{a}) are computed on the number of annotations.
% }
%     \label{tab:multiple_annotation_stats}
% \end{table}



\section{Experiments}
We experimented with classifying entities into main roles and fine-grained roles. As we performed the entity framing annotations at the span level, we framed the problem as a multi-class multi-label classification task. Given an article, an entity mention, and the span offsets, the goal was to predict the framing of that entity mention. We provide two sets of baselines and experiments to benchmark state-of-the-art models, as well as to assess the complexity of the entity framing task. The first set evaluates fine-tuning multilingual transformer models in various settings, while the second set explores hierarchical zero-shot learning using LLMs. %We investigate the impact of multilingual data on the classification of entity framing and role portrayal.
% \jp{complexity of the task, not dataset}\jp{exploiting}\jp{performance} Agreed

\subsection{Fine-Tuning Pre-trained Multilingual Transformers}
For the first set of experiments, we designed our experiments to address the following aspects:
\begin{itemize}
    \item \textbf{Granularity of context}--predicting role labels for entity mentions using the full document or narrowing the model's focus to only look at the pertinent paragraph or sentence containing the entity of interest.
    \item \textbf{Multilingual} comparison of the performance in the monolingual setting versus the multilingual setting trained on data in all five languages: Bulgarian, English, Hindi, European Portuguese, and Russian.    
\end{itemize}

For both granularity-level classification and multilingual performance, we made predictions at two levels: main role (3 labels) and fine-grained role (22 labels). We fine-tuned the multilingual pre-trained transformer XLM-R \cite{conneau2020unsupervisedcrosslingualrepresentationlearning} and adapted its final layers for our tasks, applying $softmax$ for multiclass classification and $sigmoid$ for multi-label classification. To handle spans within potentially long documents, we addressed the 512-token limitation of XLM-R by narrowing the context to the paragraph or the sentence where the entity appeared. We constructed the input text using the following format:

\texttt{input = entity mention + [SEP] + title + [SEP] + context}.

In this setup, \texttt{[SEP]} is the separator token, and the context can vary based on the granularity level, ranging from the full text to just the paragraph or sentence containing the entity mention. We placed the entity mention first, followed by the title and context, to maintain consistent positional encodings for the entity mention across different inputs. We used Stanza \cite{qi2020stanza} for sentence splitting for all languages.

% For both multilingual performance and granularity-level classification, we classify entities at the main role level (3 labels) and fine-grained role level (22 labels). We used the multilingual pre-trained transformer XLM-R \cite{conneau2020unsupervisedcrosslingualrepresentationlearning} and adapted the final layers for our classification tasks (using softmax for multiclass and sigmoid for multilabel classification). Given the need to classify spans within potentially long documents, we addressed the 512-token limitation of transformer models by limiting the field of view to the specific context in which the entity appears at both paragraph and sentence levels. We process the input text by first including the entity mention, followed by the title of the article and the context with the separator token in between as shown here: \texttt{input = entity mention + [SEP] + title + [SEP] + context}, where context depends on the granularity we train on ranging from the full text to just the paragraph or sentence containing the entity mention of interest. The reason we arrange the entity mention followed by the title and the context in this order is to ensure the associated positional encodings for the entity mention do not vary across different inputs.



To further support span-level multi-label classification, we modified the output layer to include a $sigmoid$ activation and optimized the model using \emph{Binary Cross-Entropy loss}. This setup allowed the model to predict multiple overlapping roles for a given entity span. See Appendix~\ref{sec:appendix_experiments} for more details.




\subsection{Hierarchical Zero-Shot Learning with LLMs}

We experimented with two prompting approaches: \emph{single-step} and \emph{hierarchical multi-step}. The former aimed to predict both the main role and the fine-grained role simultaneously within a single prompt. It assumed that both tasks can be handled together, relying on the model's ability to process them in one go. On the other hand, the multi-step approach separated the prediction into two distinct stages. First, the main role was predicted, and based on that output, the fine-grained role was predicted in a second step, using the information from the initial prediction to refine the second task. This stepwise process involved an intermediate prediction, which allowed the model to focus on each task individually. See Appendix~\ref{sec:appendix_experiments}, and~\ref{sec:appendix_zeroshot} for more details on experimental settings and prompt structure.

% Please add the following required packages to your document preamble:
% \usepackage{multirow}
\begin{table*}[ht]
\centering
\resizebox{0.7\textwidth}{!}{%
\begin{tabular}{cccccccc}
\toprule
\multirow{2}{*}{\textbf{Train}} &
  \multirow{2}{*}{\textbf{Context}} &
  \multicolumn{2}{c}{\textbf{Main Role}} &
  \multicolumn{4}{c}{\textbf{Fine Grained Role}} \\ \cline{3-8} 
 &
   &
  \textbf{Accuracy} &
  \textbf{Balanced Accuracy} &
  \textbf{P} &
  \textbf{R} &
  \textbf{Micro F1} &
  \textbf{Macro F1} \\ \hline
\multirow{3}{*}{\textbf{M}} & DOC & .6010          & .5904          & --             & --             & --             & --             \\
                   & PAR & .7379          & .7385          & --             & --             & --             & --             \\
                   & SEN & .7179          & .7123          & --             & --             & --             & --             \\ \midrule
\multirow{3}{*}{\textbf{F}} & DOC & .7229          & .7235          & .3495          & .4446          & .3913          & .2306          \\
                   & PAR & \textbf{.7529} & \textbf{.7553} & .3649          & .4985          & .4213          & .2392          \\
                   & SEN & .7496          & .7503          & \textbf{.4195} & \textbf{.4492} & \textbf{.4339} & \textbf{.2529} \\ \bottomrule
\end{tabular}
}
\caption{Performance of entity framing across different granularity settings using XLM-R trained on the full multilingual dataset. Models are trained and evaluated on texts with varying context sizes: full document (DOC), paragraph (PAR), or sentence (SEN) containing the entity mention. The results cover models trained on main roles (M), fine-grained roles (F), and evaluated on either main roles, fine-grained roles, or both.}
\label{tab:xlmr_granularity_results}
\end{table*}

% \renewcommand{\arraystretch}{0.9} % Reduce row height
% \setlength{\tabcolsep}{2pt}       % Reduce column padding
\begin{table}[h]
\footnotesize
\centering

% Subtable (a)
\begin{subtable}[h]{0.9\columnwidth}
\small
\centering
\caption{Monolingual setting}
\begin{tabular}{ccccc}
\toprule
Lang. & P & R & Micro F1 & Macro F1 \\
\midrule
EN & $.1032$ & $.1313$ & $.1156$ & $.0435$ \\
BG & $.1056$ & $.5758$ & $.1784$ & $.0505$ \\
HI & $.3424$ & $.4495$ & $.3887$ & $.1740$ \\
PT & $.6124$ & $.6423$ & $.6270$ & $.1505$ \\
RU & $.1077$ & $.5227$ & $.1786$ & $.0437$ \\
\bottomrule
\end{tabular}

\end{subtable}
\hfill

% Subtable (b)
\begin{subtable}[h]{0.9\columnwidth}
\small
\centering
\caption{Multilingual setting}
\begin{tabular}{ccccc}
\toprule
Lang. & P & R & Micro F1 & Macro F1 \\
\midrule
All & $.3649$ & $.4985$ & $.4213$ & $.2392$ \\ \midrule
EN & $.1854$ & $.2828$ & $\mathbf{.2240}$ & $\mathbf{.1327}$ \\
BG & $.3030$ & $.3030$ & $\mathbf{.3030}$ & $\mathbf{.1349}$ \\
HI & $.3234$ & $.4951$ & $\mathbf{.3912}$ & $\mathbf{.2043}$ \\
PT & $.6259$ & $.7480$ & $\mathbf{.6815}$ & $\mathbf{.2040}$ \\
RU & $.4831$ & $.4886$ & $\mathbf{.4859}$ & $\mathbf{.2364}$ \\
\bottomrule
\end{tabular}
\end{subtable}

\caption{Results for multi-label fine-grained role classification with XLM-R trained on monolingual and multilingual data and evaluated at the paragraph level.}
\label{tab:xlmr_language_results}
\end{table}






% \begin{table}[h]
% \small
% \centering
% \resizebox{1\columnwidth}{!}{%
% % \begin{tabular}
% \begin{tabular}{ccccccc}

% \toprule
% Train & Test & Context & P & R & Micro F1 & Macro F1 \\
% \midrule
% M & M & DOC & $-$ & $-$ & $.6010$ & $.5904$ \\
% M & M & PAR & $-$ & $-$ & $.7379$ & $.7385$ \\
% M & M & SEN & $-$ & $-$ & $.7179$ & $.7123$ \\ \midrule
% F & M & DOC & $-$ & $-$ & $.7229$ & $.7235$ \\
% F & M & PAR & $-$ & $-$ & $.7529$ & $.7553$ \\
% F & M & SEN & $-$ & $-$ & $.7496$ & $.7503$ \\ \midrule
% F & F & DOC & $.3495$ & $.4446$ & $.3913$ & $.2306$ \\
% F & F & PAR & $.3649$ & $.4985$ & $.4213$ & $.2392$ \\
% F & F & SEN & $.4195$ & $.4492$ & $.4339$ & $.2529$ \\
% \bottomrule
% \end{tabular}
% }
% \caption{Performance of entity framing across different granularity settings using XLM-R trained on the full multilingual dataset. Models are trained and evaluated on texts with varying context sizes: full document (DOC), paragraph (PAR), or sentence (SEN) containing the entity mention. The results cover models trained on main roles (M), fine-grained roles (F), and evaluated on either main roles, fine-grained roles, or both.}
% \label{tab:xlmr_granularity_results}
% \end{table}

% M & M & D & $.601$ & $.601$ & $.601$    & $.5904$ \\
% M & M & P & $.7379$ & $.7379$ & $.7379$ & $.7385$ \\
% M & M & S & $.7179$ & $.7179$ & $.7179$ & $.7123$ \\
% F & M & D & $.7229$ & $.7229$ & $.7229$ & $.7235$ \\
% F & M & P & $.7529$ & $.7529$ & $.7529$ & $.7553$ \\
% F & M & S & $.7496$ & $.7496$ & $.7496$ & $.7503$ \\
% F & F & D & $.3495$ & $.4446$ & $.3913$ & $.2306$ \\
% F & F & P & $.3649$ & $.4985$ & $.4213$ & $.2392$ \\
% F & F & S & $.4195$ & $.4492$ & $.4339$ & $.2529$ \\

% \begin{table}[h]
% \small
% \centering

% % Subtable (a)
% \begin{subtable}[h]{\columnwidth}
% \centering
% \begin{tabular}{ccccccc}

% \toprule
% Train & Test & Context & Accuracy & Weighted Accuracy \\
% \midrule
% M & M & D & $-$ & $-$ & $.6010$ & $-$ \\
% M & M & P & $-$ & $-$ & $.7379$ & $-$ \\
% M & M & S & $-$ & $-$ & $.7179$ & $-$ \\ \midrule
% F & M & D & $-$ & $-$ & $.7229$ & $-$ \\
% F & M & P & $-$ & $-$ & $.7529$ & $-$ \\
% \bottomrule
% \end{tabular}
% \caption{}

% \end{subtable}
% \hfill

% % Subtable (b)
% \begin{subtable}[h]{\columnwidth}
% \centering
% \begin{tabular}{ccccccc}

% \toprule
% Context & P & R & Micro F1 & Macro F1 \\
% \midrule
% D & $.3495$ & $.4446$ & $.3913$ & $.2306$ \\
% P & $.3649$ & $.4985$ & $.4213$ & $.2392$ \\
% S & $.4195$ & $.4492$ & $.4339$ & $.2529$ \\
% \bottomrule
% \end{tabular}
% \caption{}


% \end{subtable}

% \caption{Performance of entity framing across different granularity settings using XLM-R trained on the full multilingual dataset. Models are trained and evaluated on texts with varying context sizes: full document (D), paragraph (P), or sentence (S) containing the entity mention. The results cover models trained on either main roles (M), or fine-grained roles (F), and evaluated on main roles show in table (a), or (b) fine-grained roles.}
% \label{tab:role_classification}
% \end{table}


\subsection{Results}

We report standard evaluation metrics, including \emph{micro-average precision}, \emph{recall}, and $F1$ score, along with the \emph{macro-average} $F1$ score for fine-grained roles to address class imbalance. We further provide \emph{accuracy} and \emph{balanced accuracy} for predictions on the main role granularity.

Table~\ref{tab:xlmr_granularity_results} shows the performance of XLM-R across different context granularities (document, paragraph, and sentence) and training configurations (main roles vs. fine-grained roles). For models trained and evaluated on main roles, paragraph-level contexts perform best, followed closely by sentence-level contexts, while document-level contexts perform the worst.  When models trained on fine-grained roles are evaluated on main roles, paragraph-level contexts again yield the best performance, with document and sentence-level contexts slightly behind. This indicates that training on fine-grained roles provides an advantage. For models trained and evaluated on fine-grained roles, sentence-level contexts perform best, followed by paragraph-level contexts, with document-level contexts showing the weakest performance. These results highlight that context granularity significantly impacts performance, with localized contexts outperforming document-level contexts for both main role and fine-grained role classification tasks.

Table~\ref{tab:xlmr_language_results} offers interesting insights into the performance of XLM-R when fine-tuned on monolingual and multilingual data for multi-label fine-grained role classification at the paragraph level. The monolingual setting exhibits varying performance across languages, with the highest scores achieved for Portuguese, while English, Bulgarian, and Russian show notably lower performance; the model's performance on Hindi is moderate. These differences stem from the quantity and quality of training data, linguistic variations, and the complexity of entity mentions across languages. The consistently low Macro F1 scores across all languages indicate difficulty in predicting rare roles. In contrast, the multilingual setting consistently outperforms the monolingual setting, demonstrating its ability to better capture diverse fine-grained roles through cross-lingual transfer.



% \jp{There is a missing discussion of the results in Table 4 and 5, which are not referred to in the text.} Added

\begin{table*}[t]
\small
\centering
% Subtable (a)
%\begin{subtable}[h]{\columnwidth}
\centering
\resizebox{0.87\textwidth}{!}{%

\begin{tabular}{ccccccccc}
\toprule
\multirow{2}{*}{\textbf{Method}} & \multirow{2}{*}{\textbf{Lang.}} & \multicolumn{2}{c}{\textbf{Main Role}} & \multicolumn{4}{c}{\textbf{Fine Grained Role}} & \multirow{2}{*}{\textbf{Cost (USD)}} \\
\cmidrule(lr){3-4}
\cmidrule(lr){5-8}
 & &  \textbf{Accuracy} & \textbf{Balanced Accuracy} & \textbf{P} & \textbf{R} & \textbf{Micro F1} & \textbf{Macro F1} &  \\
\midrule
\multirow{5}{*}{\shortstack{\textbf{Single-Step} \\ \textbf{LLM Prompting}}}
& EN & .8346 & .6756 & .2692 & .4632 & .3405 & .2171 & 0.7989 \\
& BG & .8065 & .7380 & .3725 & .5588 & .4471 & .3481 & 0.2751 \\
& HI & .6327 & .6247 & .2753 & .4000 & .3262 & .2196 & 2.4696 \\
& PT & .7812 & .7455 & .5167 & .6643 & .5813 & .2891 & 1.0200 \\
& RU & .7558 & .6719 & .3939 & .5843 & .4706 & .4644 & 0.7587 \\
& All & .7030 & .6957 & \underline{.3211} & \underline{.4726} & \underline{.3824} & \underline{\textbf{.3103}} & 5.3223 \\
\midrule
\multirow{5}{*}{\shortstack{\textbf{Multi-Step} \\ \textbf{LLM Prompting}}} 
& EN & .8031 & .6799 & .2887 & .4118 & .3394 & .2383 & 0.5130 \\
& BG & .8031 & .6799 & .4318 & .5588 & .4872 & .3601 & 0.5130 \\
& HI & .6367 & .6284 & .2676 & .2868 & .2769 & .1771 & 1.4581 \\
& PT & .8125 & .7882 & .3895 & .2643 & .3149 & .2498 & 0.5634 \\
& RU & .7442 & .6680 & .4118 & .4719 & .4398 & .3774 & 0.4769 \\
& All & \underline{.7053} & \underline{.7017} & .3051 & .3294 & .3168 & .2765 & \underline{\textbf{3.1852}} \\
\midrule \midrule
\multirow{5}{*}{\textbf{XLM-R}} 
& EN  & 0.6889 & 0.5276 & .1854 & .2828 & .2240 & .1327 & --\\
& BG  & 0.7333 & 0.5791 & .3030 & .3030 & .3030 & .1349 & --\\
& HI  & 0.7025 & 0.7046 & .3234 & .4951 & .3912 & .2043 & --\\
& PT  & 0.8957 & 0.8840 & .6259 & .7480 & .6815 & .2040 & --\\
& RU  & 0.8000 & 0.7604 & .4831 & .4886 & .4859 & .2364 & --\\
& All & \textbf{0.7529} & \textbf{0.7553} & \textbf{.3649} & \textbf{.4985} & \textbf{.4213} & .2392 & --\\
\bottomrule
\end{tabular}
}
\caption{Consolidated results comparing fine-tuning XLM-R and zero-shot learning with GPT-4o. The table shows performance and cost comparisons between single-step and multi-step LLM prompting approaches, where the highest scores between these two approaches across all languages are \underline{underlined}. The top results across all three methods and languages are highlighted in \textbf{bold}.}
\label{tab:promptingmethods}
\end{table*}

Table \ref{tab:promptingmethods} consolidates the results from fine-tuning XLM-R and hierarchical zero-shot learning, showing that the multi-step approach achieves slightly better performance on main role prediction compared to the single-step approach. 
% \jp{it can be inferred from the text earlier, but is not clear that this is the paragrpah version of the task. please mention it explicitly} 
This performance improvement in multi-step prompting can be attributed to the structured decomposition of the task. By explicitly isolating the main role prediction into a dedicated step, the model can focus on high-level role identification without the distraction of fine-grained details.
While the multi-step approach enhances performance for main role prediction, it underperforms compared to the single-step approach in predicting fine-grained roles. We hypothesize two potential reasons for this discrepancy:
\begin{enumerate}
\item \textbf{Error Propagation:} In the multi-step approach, the model first predicts the main role and then proceeds to predict fine-grained roles. Errors introduced during the main role prediction step can propagate to subsequent steps, thereby reducing the overall accuracy of fine-grained predictions.
\item \textbf{Loss of Joint Context:} The single-step approach enables the model to reason jointly about both main roles and fine-grained roles, allowing it to better capture dependencies between role labels. This integrated reasoning leads to more precise and consistent fine-grained predictions.
Another finding is that the multi-step approach is significantly more cost-effective than the single-step approach. This efficiency may stem from token efficiency, as multi-step prompts are designed to be more concise, with each step addressing a specific sub-task (e.g., main role followed by fine-grained role). Consequently, this results in fewer tokens per prompt.
\end{enumerate}

Table~\ref{tab:promptingmethods} additionally compares the performance of zero-shot approaches and XLM-R, both using similar-sized input contexts. We can see that XLM-R outperforms zero-shot methods on all evaluation measures except for Macro F1, where it shows the lowest performance among all approaches. We hypothesize that this discrepancy arises because XLM-R had limited training instances for rare roles, preventing effective learning for these categories. In contrast, zero-shot approaches do not rely on training data and thus are not constrained by this limitation.


% Streamlined Task Execution: While the Multi-Step approach involves multiple queries, the focused nature of each step reduces redundancy and unnecessary context. In contrast, Single-Step prompting processes all role predictions jointly, which can lead to more verbose and costly outputs.



% \jp{again something missing on XLM performance vis-a-vis LLMs} Done

\section{Conclusion and Future Work}
We presented a novel multilingual and hierarchical dataset for characterizing entity framing and role portrayal in news articles. Our dataset introduces a unique taxonomy inspired by storytelling elements, featuring 22 fine-grained archetypes nested within three main categories: \emph{protagonist}, \emph{antagonist}, and \emph{innocent}. The dataset covers 1,378 recent news articles in five languages (Bulgarian, English, Hindi, European Portuguese, and Russian), spanning two globally significant domains: the Ukraine-Russia War and Climate Change. Over 5,800 entity mentions have been thoroughly annotated with role labels, capturing nuanced portrayals. We evaluated the dataset using fine-tuned state-of-the-art multilingual transformer models and explored hierarchical zero-shot learning with LLMs at document, paragraph, and sentence level. Our experiments highlight the potential of multilingual representations and hierarchical approaches for entity-framing tasks. We intend to release the dataset to the community freely for research purposes. We hope that this dataset will serve as a valuable resource for developing methods and tools to enhance the analysis of entity portrayals in news media. 

In future work, we plan to extend the annotations to additional languages and explore other sources of text, such as social media posts. This expansion aims to provide a broader understanding of role portrayal across diverse linguistic and contextual settings. We also intend to perform a more in-depth dataset analysis to examine formulations such as central entity identification. We will also consider evaluating the potential biases in the dataset to ensure robustness in downstream tasks and end-user applications.

\section{Limitations}

\paragraph{Corpus} Our dataset focuses on two domains: the Ukraine-Russia War and Climate Change, and covers news articles in five languages (Bulgarian, English, Hindi, Portuguese, and Russian). While these domains and languages provide a diverse foundation for entity framing analysis, the corpus contains 1,378 articles and it should not be considered representative of all news coverage or media landscapes in any specific country. Additionally, the dataset is not perfectly balanced with respect to topics, entities, or languages. Moreover, human annotation of entity framing and role portrayal inevitably is subjective and annotators may subconsciously have biases that could influence the quality. Despite providing detailed annotation guidelines and conducting quality control measures such as double annotation and adjudication through the curation process, some level of subjectivity may remain in the dataset. 

\paragraph{Baseline Models}
Our reported experiments utilize state-of-the-art baselines covering a range of fine-tuned multilingual transformer models and hierarchical zero-shot learning with LLMs. However, we have not yet explored alternative architectures or advanced techniques such as few-shot, instruction-based evaluation, or multitask learning. Future work could investigate these approaches to improve model efficiency and performance. 

Additionally, our zero-shot learning experiments rely on OpenAI's GPT-4o, a closed-source model that is subject to changes over time and may be deprecated in the future. This dependency may impact the reproducibility and interpretability. To address these challenges, future research should prioritize improving open-source models to ensure greater accessibility, transparency, and reproducibility.





\section{Ethics and Broader Impact}
% Authors are encouraged to devote a section of their paper to concerns about the ethical impact of the work and to a discussion of broader impacts of the work, which will be taken into account in the review process. This discussion may extend into a 5th page (short papers) or 9th page (long papers). https://www.acm.org/code-of-ethics
%In addition, we provide a responsible NLP research checklist, which authors must complete as part of their paper submission. https://aclrollingreview.org/responsibleNLPresearch/
\paragraph{Biases}
Our dataset aims to capture a balanced range of perspectives on the Ukraine-Russia War and Climate Change, covering five languages: Bulgarian, English, Hindi, European Portuguese, and Russian. While our goal is to incorporate diverse news sources and viewpoints, achieving perfect balance is not always possible. Consequently, inherent biases in the original media sources may be present in the annotations. To reduce unwanted annotation biases, the corpus is annotated with clear instructions to annotators to focus strictly on the framing of entities, setting aside their personal opinions. All annotations are performed by subject-matter experts, and we did not use crowd-sourcing.

\paragraph{Intended Use and Misuse Potential}
The primary goal of this corpus is to facilitate research on entity framing, role portrayal, and media analysis. These tools can help researchers, journalists, and the general public identify framing patterns and biases in news content. However, there is a risk that the corpus could be misused for malicious purposes, such as manipulating news narratives. We urge users to employ this resource responsibly and remain aware of potential ethical risks associated with its misuse.

\paragraph{Environmental Impact}
The use of LLMs requires substantial computational power, contributing to carbon emissions. Even though we used LLMs in a zero-shot in-context-learning setting rather than training models from scratch, the LLMs still rely on GPUs for inference, which has an environmental impact.

\paragraph{Fairness}

Most of our annotators and curators come from the institutions of the co-authors of this manuscript and were fairly paid as part of their job duties. Few annotators were experienced analysts with full-time consulting roles and rates set by their contracting institutions. A fraction 
of the annotators were students from the respective
academic organizations. For two languages, a professional annotation company was contracted on rates based on country of residence. At the same time, some of the remaining annotators were researchers working primarily as linguists and lexicographers at their institute of affiliation and were all compensated according to local standards and their employment contracts.  


%\section*{Acknowledgments}

%This work is (partially) supported by the  Project SERICS (PE00000014) under the NRRP MUR program funded by the EU – NGEU. 

% Bibliography entries for the entire Anthology, followed by custom entries
% \bibliography{anthology,custom,shared}
% This must be in the first 5 lines to tell arXiv to use pdfLaTeX, which is strongly recommended.
\pdfoutput=1
% In particular, the hyperref package requires pdfLaTeX in order to break URLs across lines.

\documentclass[11pt]{article}

% Change "review" to "final" to generate the final (sometimes called camera-ready) version.
% Change to "preprint" to generate a non-anonymous version with page numbers.
\usepackage[final]{acl}

% Standard package includes
\usepackage{times}
\usepackage{latexsym}

% For proper rendering and hyphenation of words containing Latin characters (including in bib files)
\usepackage[T1]{fontenc}
% For Vietnamese characters
% \usepackage[T5]{fontenc}
% See https://www.latex-project.org/help/documentation/encguide.pdf for other character sets

% This assumes your files are encoded as UTF8
\usepackage[utf8]{inputenc}

% This is not strictly necessary, and may be commented out,
% but it will improve the layout of the manuscript,
% and will typically save some space.
\usepackage{microtype}

% This is also not strictly necessary, and may be commented out.
% However, it will improve the aesthetics of text in
% the typewriter font.
\usepackage{inconsolata}

%Including images in your LaTeX document requires adding
%additional package(s)
\usepackage{graphicx}
\usepackage{xcolor}

% If the title and author information does not fit in the area allocated, uncomment the following
%
%\setlength\titlebox{<dim>}
%
% and set <dim> to something 5cm or larger.

\usepackage{booktabs}
\usepackage{hyperref}
\usepackage{multirow}
\usepackage{multicol}
\usepackage[most]{tcolorbox}
\usepackage{adjustbox}
\usepackage{graphicx}
\usepackage{fullpage}
\usepackage{times}
\usepackage{fancyhdr,graphicx,amsmath,amssymb}
%\usepackage[ruled,vlined]{algorithm2e}
\usepackage{algorithm}
\usepackage{algpseudocode}
\usepackage{booktabs}
\usepackage{adjustbox}
\usepackage{url}
\usepackage{hyperref}
\usepackage{amssymb}
\usepackage{marvosym}
\usepackage{multirow}
\usepackage{subcaption}
\DeclareMathOperator*{\argmax}{arg\,max}
\DeclareMathOperator*{\argmin}{arg\,min}


\newtcolorbox{promptbox}[2][]{
  colback=gray!10,
  colframe=gray!50,
  arc=3mm,
  boxrule=1pt,
  left=10pt,
  right=10pt,
  top=8pt,
  bottom=8pt,
  before skip=12pt,
  after skip=12pt,
  fonttitle=\bfseries,
  title=#2,
  #1
}

\title{Quality-Aware Decoding: Unifying Quality Estimation and Decoding}

% Author information can be set in various styles:
% For several authors from the same institution:
% \author{Author 1 \and ... \and Author n \\
%         Address line \\ ... \\ Address line}
% if the names do not fit well on one line use
%         Author 1 \\ {\bf Author 2} \\ ... \\ {\bf Author n} \\
% For authors from different institutions:
% \author{Author 1 \\ Address line \\  ... \\ Address line
%         \And  ... \And
%         Author n \\ Address line \\ ... \\ Address line}
% To start a separate ``row'' of authors use \AND, as in
% \author{Author 1 \\ Address line \\  ... \\ Address line
%         \AND
%         Author 2 \\ Address line \\ ... \\ Address line \And
%         Author 3 \\ Address line \\ ... \\ Address line}

\author{Sai Koneru$^{1}$,
  Matthias Huck$^{2}$,
  Miriam Exel$^{2}$, \textnormal{and}
  Jan Niehues$^{1}$ \\
  $^{1}$ Karlsruhe Institute of Technology \\
  $^{2}$ SAP SE, Dietmar-Hopp-Allee 16, 69190 Walldorf, Germany \\
  \texttt{\{sai.koneru, jan.niehues\}@kit.edu} \\
  \texttt{\{matthias.huck, miriam.exel\}@sap.com}}

%\author{
%  \textbf{First Author\textsuperscript{1}},
%  \textbf{Second Author\textsuperscript{1,2}},
%  \textbf{Third T. Author\textsuperscript{1}},
%  \textbf{Fourth Author\textsuperscript{1}},
%\\
%  \textbf{Fifth Author\textsuperscript{1,2}},
%  \textbf{Sixth Author\textsuperscript{1}},
%  \textbf{Seventh Author\textsuperscript{1}},
%  \textbf{Eighth Author \textsuperscript{1,2,3,4}},
%\\
%  \textbf{Ninth Author\textsuperscript{1}},
%  \textbf{Tenth Author\textsuperscript{1}},
%  \textbf{Eleventh E. Author\textsuperscript{1,2,3,4,5}},
%  \textbf{Twelfth Author\textsuperscript{1}},
%\\
%  \textbf{Thirteenth Author\textsuperscript{3}},
%  \textbf{Fourteenth F. Author\textsuperscript{2,4}},
%  \textbf{Fifteenth Author\textsuperscript{1}},
%  \textbf{Sixteenth Author\textsuperscript{1}},
%\\
%  \textbf{Seventeenth S. Author\textsuperscript{4,5}},
%  \textbf{Eighteenth Author\textsuperscript{3,4}},
%  \textbf{Nineteenth N. Author\textsuperscript{2,5}},
%  \textbf{Twentieth Author\textsuperscript{1}}
%\\
%\\
%  \textsuperscript{1}Affiliation 1,
%  \textsuperscript{2}Affiliation 2,
%  \textsuperscript{3}Affiliation 3,
%  \textsuperscript{4}Affiliation 4,
%  \textsuperscript{5}Affiliation 5
%\\
%  \small{
%    \textbf{Correspondence:} \href{mailto:email@domain}{email@domain}
%  }
%}

\begin{document}
\maketitle
\begin{abstract}
% Neural Machine Translation (NMT) has achieved high-quality translations in many scenarios, pushing the boundaries of tasks such as instruction-following and multimodal translation. 
Quality Estimation (QE) models for Neural Machine Translation (NMT) predict the quality of the hypothesis without having access to the reference.
An emerging research direction in NMT involves the use of QE models, which have demonstrated high correlations with human judgment and can enhance translations through Quality-Aware Decoding. Although several approaches have been proposed based on sampling multiple candidate translations and picking the best candidate, none have integrated these models directly into the decoding process. In this paper, we address this by proposing a novel token-level QE model capable of reliably scoring partial translations. We build a uni-directional QE model for this, as decoder models are inherently trained and efficient on partial sequences. We then present a decoding strategy that integrates the QE model for Quality-Aware decoding and demonstrate that the translation quality improves when compared to the N-best list re-ranking with state-of-the-art QE models (up to $1.39$ XCOMET-XXL $\uparrow$). Finally, we show that our approach provides significant benefits in document translation tasks, where the quality of N-best lists is typically suboptimal\footnote{Code can be found at \url{https://github.com/SAP-samples/quality-aware-decoding-translation}}
\end{abstract}
\section{Introduction}

Large language models (LLMs) have significantly impacted various Natural Language Processing (NLP) tasks \citep{brown2020language, jiang2023mistral, dubey2024llama}, including Neural Machine Translation (NMT). The field of NMT is transitioning from using dedicated encoder-decoder transformers \citep{vaswani2017attention, nllb2024scaling} to leveraging decoder-only LLM-based translation models \citep{kocmi2024findings}. This shift is driven by LLMs' ability to retain knowledge, handle large contexts, and follow instructions, learned during extensive pre-training \citep{xu2024contrastive, alves2024tower}. As a result, LLM-based MT models have achieved state-of-the-art translation quality \citep{kocmi2024findings}.

In parallel, Quality Estimation (QE) has become a well-researched subfield within NMT. QE models are trained to predict the quality of a translation without requiring access to the reference \citep{rei2021references,rei2022cometkiwi}. Interestingly, QE models can achieve performance in assessing translation quality that is comparable to MT evaluation models, which do have access to the reference \citep{zerva2024findings}.

This led to the question: "\textit{Can we integrate QE into the current translation process to improve quality?}" Incorporating QE into NMT offers several benefits. First, having a expert QE model guiding the decoding can further improve the quality. Second, by adapting the QE model with feedback from human annotators, we can generate future translations guided with the newly obtained feedback.

\begin{figure*}[!ht]
\includegraphics[width=\textwidth]{Figures/nbestlist_problem.png}
 \caption{Example from WMT'23 English → German \#ID: 10: The paragraph begins with 'Department of Homeland Security,' which should be translated as 'Ministerium für \textbf{I}nnere Sicherheit.' However, the top 25 beams do not contain the correct translation and begin with an error, making N-best list re-ranking insufficient. Although the top-5 tokens at the decoding contain the correct forms 'Inn' or 'Inner,' the probabilities split among them giving highest mass to the incorrect token 'inn.' Quality-Aware decoding can prevent errors with earlier integration.}
\label{fig:nbestlist}
\end{figure*}


Several approaches have been explored to integrate QE into the translation process. These include re-ranking the N-best list \citep{fernandes2022quality}, applying minimum Bayes risk (MBR) decoding on a quality-filtered N-best list \citep{tomani2024quality}, and training additional models for post-editing based on QE-predicted errors \citep{treviso2024xtower}. However, all these methods operate on fully generated sequences before the QE model can exert influence. Integrating QE earlier in the decoding process, referred in this paper as \textit{Quality-Aware Decoding}, could enhance translation quality and reduce reliance on the N-best list. This is especially relevant when dealing with long inputs as good translations during decoding are likely to be pruned and may need sampling larger number of finished hypothesis. We illustrate this in Figure \ref{fig:nbestlist}.

To achieve this, a QE model capable of predicting the quality of partial translations is required. However, current leading QE models face challenges in this area, as they are typically not trained to predict scores for incomplete hypotheses. \textit{Therefore, developing QE models that can handle partial translations is essential for implementing Quality-Aware Decoding during the translation process}.

In this work, we propose adapting LLM-based MT models to perform QE on partial translations and incorporating this model into the decoding. We create a token-level synthetic QE dataset using WMT Multidimensional Quality Metrics (MQM) data \citep{burchardt2013multidimensional, freitag2024llms}. We then adapt a uni-directional LLM-based MT model to predict whether a token is \textit{Good} or \textit{Bad}. Training QE models on these token-level tasks alleviates the data challenge and allows us to exploit the MQM data while simultaneously making the task easier for the model compared to predicting a score directly.

\begin{figure*}[!ht]
\includegraphics[width=\textwidth]{Figures/annotation_scheme.png}
 \caption{Token-level label annotation scheme using the MQM error tags. \textit{MASK} indicates that this token label will not be used in training to prevent incorrect learning signal.}
\label{fig:annotation}
\end{figure*}

Furthermore, integrating the QE model into NMT during decoding is not trivial, as we need to combine the QE estimates during decoding. Therefore, we use the decoding strategy from \citet{koneru2024plug}, and modify it to incorporate token-level predictions efficiently with the adapted QE model to provide real-time feedback during the decoding process. We summarize our main findings and contributions below.

\begin{itemize}
    \item We present a novel uni-directional QE model which estimates quality on incomplete hypotheses by averaging the probabilities of each token being classified as \textit{Good}. 
    
    %We demonstrate that it achieves improved correlation with human annotations on WMT 23 English $\rightarrow$ German, compared to the log probabilities of the same LLM-based NMT model.

    \item We propose a decoding strategy that combines the token-level QE model on partial hypothesis and the NMT model to perform Quality-Aware Decoding. 
    
    \item We show through experiments that early integration is essential and the translation quality is improved even when compared to re-ranking the N-best list with state-of-the-art QE models.

    \item We highlight the significance of our approach in document translation scenarios, where post-generation QE techniques fall short due to their reliance on the quality of the N-best list, a challenge that becomes more difficult as the input length increases.
\end{itemize}



\section{Quality-Aware Decoding}

The primary objective of this paper is to achieve Quality-Aware Decoding in MT. To accomplish this, it is essential to predict the quality of partial translations and integrate this information during the decoding process. Our approach proposes using one NMT model for generating translations and another adapted NMT model to predict the quality of the candidate translations produced by the first model.

First, we explain why relying solely on the NMT model to predict the quality of a hypothesis is insufficient and why an additional model is necessary. Next, we outline the adaptation of the NMT model for QE on partial translations, detailing the creation of a token-level QE dataset, the modifications made to the NMT model for this task, and the process of estimating the sentence-level quality score. Finally, we describe the algorithm used to incorporate the QE score into the decoding process.

\subsection{Decomposing Decoding: Translation + QE}
NMT models generate a token-by-token sequence and provide the probability of each token at the decoding step. The average of the log-probabilities is often used as a proxy to score the candidate during search. 

While NMT models are capable of generating high-quality translations, using the average log-probabilities of hypotheses as a scoring metric tends to yield poor correlation with actual translation quality \citep{eikema2020map, freitag2020bleu}. In many cases, a translation can continue in several different ways, all of which may be acceptable. If the starting tokens for these continuations differ, the probability mass may be spread across multiple options which is used during the search. However, from a quality perspective, all these continuations could still achieve a high score, as the QE scores are independent and need not sum to $1$.

Therefore, we propose a expert model that focuses on quality to estimate the scores better during decoding and  improve the search space leading to a better hypothesis.


% Therefore, relying solely on the average log-probabilities during decoding is not ideal, as it computes the score independently for each token and does not fully correlate with the overall quality of the current hypothesis.

\subsection{Quality Estimation on Partial Sequences}

% NMT models decode sequences token-by-token. 
To provide a quality score during decoding, the QE model must be capable of handling incomplete sequences. It should not penalize a sequence if there is a potential extension that could lead to a perfect translation.

Estimating the score in this way is not feasible with current QE models, such as COMET \citep{rei2021references}, as they were not trained for this specific task and cannot provide reliable scores in the context of partial translations. Hence, we need to develop a partial QE system.

When building a partial QE system, several factors need to be considered. First, should the model use a uni-directional or bi-directional architecture? A \textbf{uni-directional} model is more efficient, as it allows for caching the hidden states, which can then be used for subsequent steps without re-encoding, unlike a bi-directional model.

Next, we need to decide whether to predict the QE score at the sequence level or at the token level. For \textbf{token-level QE}, we can directly use data from MQM annotations, as we already know which tokens are \textit{Good} or \textit{Bad}. However, for segment-level scoring, we need to consider how to synthetically create the training data. 

% Additionally, COMET models are encoder-only architectures pre-trained on full sentences, rather than partial sentences as required in this case. Moreover, predicting the score of partial translations naturally favors decoder-only models due to their efficiency. New tokens only need to process the preceding sequence, avoiding the need to re-encode the entire sequence. Additionally, this approach simplifies training, as we do not require synthetically shorter samples. 

%  Furthermore, there is no readily available dataset containing partial translations along with their quality scores. Hence, we need to design the adaptation process with a QE model that is uni-directional and exploit already available human annotated data.

% \subsubsection{Token-level Quality Estimation}

Therefore, we decide adapt the uni-directional model into a token-level QE system that predicts whether each token is \textit{Good} or \textit{Bad} (a binary decision) by adding an additional classifier head. This adaptation enables us to estimate the score for a sequence by calculating the average probability that each token is classified as \textit{Good}. We hypothesize that adapting the model in this way, rather than directly predicting the score, provides greater stability, as the last hidden states inherently contain token-level information and do not require mapping the entire sequence to a single score.

For training this model, we leverage the WMT MQM data containing error annotations in MT outputs. We can treat tokens before an error as \textit{Good} and those containing inside an error as \textit{Bad}. Then, we can train in uni-directional manner where each token's label is predicted using only the preceding context in the hypothesis. This is crucial as we only have the preceding context to estimate the quality for partial hypothesis.

\subsubsection{Learning the Right Signal}

\begin{algorithm*}[!t]
\caption{Computing merged score of partial hypothesis with translation and token-level QE models.}
\begin{algorithmic}[1]
\setlength{\baselineskip}{1.2em}
\Procedure{MergeScore}{}
    \State \textbf{Input:}   Hypothesis tokens $h_1, h_2, h_3, \dots, h_{n}$, Translation Model $\mathcal{M}_{NMT}$, QE model $\mathcal{M}_{QE}$, Source sentence $\mathcal{S}$, Re-ranking weight $\alpha$,
    \State \textbf{Output:} $merged\_score$
    \State $Score_{NMT} \gets \frac{1}{n}\sum \log \mathcal{P}(h_1, h_2,\dots, h_{n}|\mathcal{S};\mathcal{M}_{NMT})$ 
    \State $Score_{QE} \gets \frac{1}{n}\sum \log \mathcal{P}(0_{1}, 0_{2},\dots,0_{n} | h_1, h_2,\dots, h_{n},\mathcal{S};\mathcal{M}_{QE})$ 
    \State $merged\_score \gets (\alpha) \times Score_{NMT} + (1 - \alpha) \times Score_{QE}$
\EndProcedure
\end{algorithmic}
\label{alg:joint}
\end{algorithm*}

The straightforward approach to creating labels is to assign $1$ to all tokens within the error span and $0$ otherwise. However, MQM annotations can mark errors from words to phrases, and the starting tokens of an error span may not always be wrong. This is illustrated in Figure \ref{fig:annotation}.

For example, consider the German sentence \textit{"Ich spiele Tennis"} translated by three different NMT systems, each annotated with MQM error labels. In this work, we focus on learning a binary decision: whether an error is present, ignoring error severity.

\textbf{System 1: No error}: The translation \textit{"I play Tennis"} is perfect, and all tokens are labeled as "\textit{Good}."

\textbf{System 2: Partial error}: The translation \textit{"I played Tennis"} has an error in the verb form ("played" instead of "play"). The error is in the token span \textit{"played"}, but not all tokens in this span are incorrect (e.g., "pla" is correct). Assigning a "\textit{Bad}" label to the entire span would lead to incorrect learning. A more refined approach is needed to mark errors accurately at the token level.

\textbf{System 3: Full error}: The translation \textit{"I enjoy Tennis"} contains an error in \textit{"enjoy"}, so all tokens in this span should be labeled as "\textit{Bad}."

It is not trivial to decide when the prefix of an error span is correct/incorrect. To achieve accurate labeling, we propose the following scheme:

\begin{itemize} \item Apply a \texttt{<MASK>} operation to all tokens within the error span. \item Only the last token in the span is assigned the label "\textit{Bad}", as the error is considered complete at the end of the span. \end{itemize}

If the error token is in the middle, we still train the model to predict "\textit{Bad}" in the end and let the model determine which tokens should be part of the error span during inference. This approach ensures that errors are identified without explicitly defining the error span. 

\subsubsection{Sequence-Level Quality Estimation}


After fine-tuning a token-level classification model to predict the quality of the tokens, we still need to map these predictions into a sequence-level score that can be integrated during the decoding process. There are several potential ways to achieve this.

One approach is to simply count how many tokens are classified as \textit{Bad} in the current hypothesis. However, this method has limitations. The number of errors should be normalized based on the length of the hypothesis to account for varying sizes. Additionally, converting the probabilities into a fixed number of error tokens would need to account for different error types according to the MQM format, as each error counts differently.

To avoid such strict scoring schemes, we take a simpler approach. We average the log probabilities of all tokens that are classified as \textit{Good}. This method inherently accounts for the length of the hypothesis, and it provides a score on the scale of log probabilities, which aligns with the decoding process. Therefore, we use this averaged log probability as a proxy metric for the QE score, where a higher score indicates better quality
(\textbf{Line 5} in Algorithm \ref{alg:joint}).

\subsubsection{Fusing Translation and Quality}

We can use a token-level QE system to evaluate the quality of a source and partial hypothesis during decoding. However, integrating these probabilities into all candidates is computationally expensive, as each beam considers extensions equal to the vocabulary size.

To address this, we adopt a simplified decoding strategy from \citet{koneru2024plug}, which ensembles models with different vocabularies. By adapting the same MT model for token-level QE, we simplify the merging process, as the vocabularies match. This restriction is reasonable, as it is also beneficial to leverage the knowledge learned by the specialized MT for token-level QE.

The core idea is to re-rank the top candidates at each decoding step using the QE model. After re-ranking, the translation and QE scores are merged, and the process repeats until the end-of-sentence token is generated, for each beam. This strategy allows us to efficiently incorporate the QE model’s estimate, improving translation quality.

During decoding, at each step, we have scores for $n$ beams and $V$ possible extensions from the vocabulary. In typical beam search, we select the top $n$ extensions and expand the hypothesis. To make the decoding process Quality-aware, we estimate the quality of these extensions. Since estimating all extensions is computationally expensive, we limit the candidates by selecting a specified number of top candidates.

To achieve this, we use a hyper-parameter $topk$, which selects the best $topk$ extensions for each beam. For each of these top $topk$ extensions, we compute a combined score, detailed in Algorithm \ref{alg:joint}. This combined score incorporates both the translation model score and the quality estimation score, ensuring the quality is considered during decoding.

For a top extension at decoding step $n$, let the current tokens be $h_1, h_2, h_3, \dots, h_n$. The NMT model score is computed as the average log probabilities of each token (Line 4). For the token-level QE model, we compute the average probability of each token being classified as '\textit{Good}' (Line 5). The merged score is equal to weighted linear combination of these probabilities, with weight $\alpha$ (Line 6).

Thus, to make the decoding process Quality-Aware, we first train a token-level QE system by adapting the same NMT model to ensure vocabulary matching. We then combine the scores from both models to improve the sequence estimates explored during search.


\begin{table*}[!ht]
\resizebox{2\columnwidth}{!}{
\begin{tabular}{@{}ccccc@{}}
\toprule
\multicolumn{1}{c|}{Model}            & \multicolumn{1}{c|}{Beams}                & \multicolumn{1}{c|}{Re-ranking}              & MetricX ($\downarrow$)     & XCOMET-XXL ($\uparrow$)    \\ \midrule
\multicolumn{5}{c}{\textit{English $\rightarrow$ German}}                                                                                                          \\ \midrule
\multicolumn{1}{c|}{Tower}            & \multicolumn{1}{c|}{5}                    & \multicolumn{1}{c|}{\_}                      & 2.52          & 86.93          \\
\multicolumn{1}{c|}{Tower}            & \multicolumn{1}{c|}{25}                   & \multicolumn{1}{c|}{XCOMET-XL QE}            & 2.37          & 87.79          \\
\multicolumn{1}{c|}{Tower}            & \multicolumn{1}{c|}{25}                   & \multicolumn{1}{c|}{Tower QE} & 2.38          & 87.40          \\
\multicolumn{1}{c|}{Tower + Tower QE} & \multicolumn{1}{c|}{5 (25* for Tower QE)} & \multicolumn{1}{c|}{\_}                      & 2.12          & 88.95          \\
\multicolumn{1}{c|}{Tower + Tower QE} & \multicolumn{1}{c|}{5 (25* for Tower QE)} & \multicolumn{1}{c|}{XCOMET-XL QE}            & \textbf{2.09} & \textbf{89.08} \\ \midrule
\multicolumn{5}{c}{\textit{Chinese $\rightarrow$ English}}                                                                                                         \\ \midrule
\multicolumn{1}{c|}{Tower}            & \multicolumn{1}{c|}{5}                    & \multicolumn{1}{c|}{\_}                      & 2.42          & 88.91          \\
\multicolumn{1}{c|}{Tower}            & \multicolumn{1}{c|}{25}                   & \multicolumn{1}{c|}{XCOMET-XL QE}            & 2.30          & 89.49          \\
\multicolumn{1}{c|}{Tower}            & \multicolumn{1}{c|}{25}                   & \multicolumn{1}{c|}{Tower QE} & 2.32          & 89.51          \\
\multicolumn{1}{c|}{Tower + Tower QE} & \multicolumn{1}{c|}{5 (25* for Tower QE)} & \multicolumn{1}{c|}{\_}                      & 2.26          & 89.82          \\
\multicolumn{1}{c|}{Tower + Tower QE} & \multicolumn{1}{c|}{5 (25* for Tower QE)} & \multicolumn{1}{c|}{XCOMET-XL QE}            & \textbf{2.24} & \textbf{90.00} \\ \bottomrule
\end{tabular}
}
\caption{Translation Quality on WMT23 English $\rightarrow$ German Test set. Both XCOMET and MetricX columns use reference for reporting translation quality where as XCOMET-XL QE does not use for re-ranking.}
\label{tab:qadecoding}
\end{table*}

\begin{table}[!ht]
\resizebox{\columnwidth}{!}{
\centering
\begin{tabular}{@{}c|ccc@{}}
\toprule
                                                                                      & Pearson        & Spearmann      & Kendall        \\ \midrule
COMETQE                                                                               & \textbf{44.41} & 41.29          & 31.19          \\ \midrule
COMETQE-XL                                                                            & 41.23          & \textbf{42.17} & \textbf{31.84} \\ \midrule
Tower Avg. Log Prob                                                                        & 32.32          & 16.74          & 12.77          \\ \midrule
\begin{tabular}[c]{@{}c@{}}Tower QE\end{tabular} & 40.56          & 33.96          & 25.87          \\ \bottomrule
\end{tabular}
}
\caption{Correlation on WMT 23 for English $\rightarrow$ German Test set. The scores are calculated after removing the few sentences labeled for hallucination detection. Best scores according to each coefficient are highlighted in \textbf{bold}.}
\label{tab:correlation}
\end{table}

\section{Experimental Setup}
\paragraph{Datasets:} We focus on two language directions given their availability of MQM data: English $\rightarrow$ German and Chinese $\rightarrow$ English. To train our token-level QE systems, we use the MQM datasets\footnote{https://github.com/google/wmt-mqm-human-evaluation} from WMT \citep{freitag2021experts}. Specifically, we use the datasets until 2022 for training, 2024 for validation, and 2023 for testing \citep{kocmi2024findings}. This setup is consistent with all the other QE metrics, and we do not use any additional data beyond these datasets.
\vspace{-0.1cm}
\paragraph{Models:} 
Our proposed approach achieves Quality-Aware decoding by combining an NMT model with a token-level QE model, where we adapt the same NMT for QE by adding a classification head. We use the state-of-the-art NMT model, Tower 7B\footnote{Unbabel/TowerInstruct-7B-v0.2} \citep{alves2024tower}, which provides high-quality translations and has already been exposed to MQM data during instruction-tuning. This ensures that the gains observed in our approach stem from integrating Quality-Aware decoding into the NMT process, rather than introducing new data. Additional details on training and hyper-parameters are provided in Appendix \ref{sec:training_detail}.
\vspace{-0.1cm}
\paragraph{Metrics:}
For reporting the translation quality, we consistently use XCOMET-XXL\footnote{Unbabel/XCOMET-XXL} \citep{guerreiro2024xcomet} and MetricX\footnote{google/metricx-24-hybrid-xl-v2p6} \citep{juraska2024metricx} \textbf{with the reference}. To compare with N-best list re-ranking, we use the XCOMET-XL QE\footnote{Unbabel/XCOMET-XL} \textbf{without the reference}. This approach allows us to avoid biasing toward a single metric during the re-ranking process and enables us to measure the gains achieved by differently trained metrics. 

\section{Results}



\begin{table*}[!ht]
\centering
\resizebox{2\columnwidth}{!}{
\begin{tabular}{@{}ccccc@{}}
\toprule
\multicolumn{1}{c|}{Model}            & \multicolumn{1}{c|}{Beams}                        & \multicolumn{1}{c|}{Re-ranking}               & MetricX ($\downarrow$) & XCOMET-XXL ($\uparrow$) \\ \midrule
\multicolumn{5}{c}{\textit{English $\rightarrow$ German}}                                                                                                          \\ \midrule
\multicolumn{1}{c|}{Tower}            & \multicolumn{1}{c|}{25}                           & \multicolumn{1}{c|}{XCOMET-XL QE}             & 2.37     & 87.79      \\
\multicolumn{1}{c|}{Tower}            & \multicolumn{1}{c|}{25}                           & \multicolumn{1}{c|}{Tower QE}         & 2.38     & 87.40      \\
\multicolumn{1}{c|}{Tower}            & \multicolumn{1}{c|}{25}                           & \multicolumn{1}{c|}{Tower Distill QE} & 2.38     & 87.39      \\
\multicolumn{1}{c|}{Tower + Tower QE} & \multicolumn{1}{c|}{5 (25* for Tower QE)}         & \multicolumn{1}{c|}{\_}                       & 2.12     & \textbf{88.95}      \\
\multicolumn{1}{c|}{Tower + Tower QE} & \multicolumn{1}{c|}{5 (25* for Tower Distill QE)} & \multicolumn{1}{c|}{\_}                       & \textbf{2.11}     & 88.76      \\ \bottomrule
\end{tabular}
}
\caption{Performance of Unidirectional QE trained with/without distillation on WMT23 English $\rightarrow$ German Test set. Best scores according to each metric are highlighted in \textbf{bold}.}
\label{tab:towerdistill}
\end{table*}


\begin{table*}[!ht]
\centering
\resizebox{2\columnwidth}{!}{
\begin{tabular}{@{}cccccc@{}}
\toprule
\multicolumn{1}{c|}{Model}            & \multicolumn{1}{c|}{Beams}                & \multicolumn{1}{c|}{Re-ranking}       & XCOMET-XL ($\uparrow$)     & \multicolumn{1}{c|}{XCOMET-XXL ($\uparrow$)}     & Impact                                                                                       \\ \midrule
\multicolumn{6}{c}{\textit{Paragraph-Level}}                                                                                                                                                                                                                                    \\ \midrule
\multicolumn{1}{c|}{Tower}            & \multicolumn{1}{c|}{25}                   & \multicolumn{1}{c|}{XCOMET-XL QE}     & \textbf{86.56} & \multicolumn{1}{c|}{87.79}          & \multirow{3}{*}{\begin{tabular}[c]{@{}c@{}}$\delta$ = + 1.16\\ (88.95 - 87.79)\end{tabular}} \\
\multicolumn{1}{c|}{Tower}            & \multicolumn{1}{c|}{25}                   & \multicolumn{1}{c|}{Tower QE} & 85.40          & \multicolumn{1}{c|}{87.40}          &                                                                                              \\
\multicolumn{1}{c|}{Tower + Tower QE} & \multicolumn{1}{c|}{5 (25* for Tower QE)} & \multicolumn{1}{c|}{\_}               & 86.36          & \multicolumn{1}{c|}{\textbf{88.95}} &                                                                                              \\ \midrule
\multicolumn{6}{c}{\textit{Sentence-Level}}                                                                                                                                                                                                                                     \\ \midrule
\multicolumn{1}{c|}{Tower}            & \multicolumn{1}{c|}{25}                   & \multicolumn{1}{c|}{XCOMET-XL QE}     & \textbf{86.42}          & \multicolumn{1}{c|}{87.68}          & \multirow{3}{*}{\begin{tabular}[c]{@{}c@{}}$\delta$ = + 0.38\\ (88.06 - 87.68)\end{tabular}} \\
\multicolumn{1}{c|}{Tower}            & \multicolumn{1}{c|}{25}                   & \multicolumn{1}{c|}{Tower QE} & 85.23          & \multicolumn{1}{c|}{87.41}          &                                                                                              \\
\multicolumn{1}{c|}{Tower + Tower QE} & \multicolumn{1}{c|}{5 (25* for Tower QE)} & \multicolumn{1}{c|}{\_}               & 85.96          & \multicolumn{1}{c|}{\textbf{88.06}}          &                                                                                              \\ \bottomrule
\end{tabular}
}
\caption{Impact of integrating Unidirectional QE during decoding with paragraphs vs sentences on WMT23 English $\rightarrow$ German Test set. $\delta$ denotes the improvement in translation quality from re-ranking N-best list with XCOMET-XL QE to integrating unidirectional Tower QE during the decoding. Best scores according to each metric are highlighted in \textbf{bold}.}
\label{tab:sentvspara}
\end{table*}



We conduct a series of experiments to validate the effectiveness of Quality-Aware decoding and identify the scenarios where it provides the most benefit. First, we evaluate whether our token-level QE model can better estimate sequence quality compared to the log probabilities of the NMT model. Next, we assess the impact of Quality-Aware decoding by comparing it with other approaches to determine if it improves translation quality. We also perform an ablation study to examine whether training the QE model on errors from the same NMT model enhances its performance. Finally, we explore the impact of source sentence length to highlight the limitations of N-best list re-ranking.

\subsection{Quality Estimation Performance}

First, we evaluate the agreement between the Tower-based token-level QE model (\textbf{Tower QE}) and human scores for a given hypothesis. It is only beneficial if we achieve higher correlation than the average of the NMT model log probabilities to show the need to integrate it during decoding. Therefore, we report the correlation with human scores of different models on WMT 23 English $\rightarrow$ German in Table \ref{tab:correlation}. 

We observe that the best-performing systems are the Comet QE models, which predict a single score using the full hypothesis. This is expected, as these models assess quality after the hypothesis is fully generated. In contrast, both log probabilities and Tower QE scores are based on the predicted token of each decoding step, using only the preceding context. Log probabilities perform poorly in this setup, while our proposed model, Tower QE, achieves twice the correlation with human judgments compared to log probabilities, despite scoring token by token with preceding context. This result highlights the potential of integrating our approach into the decoding process.

\subsection{Unified Decoding for NMT}


To validate the effectiveness of our unified decoding approach, we compare it with several baselines in Table \ref{tab:qadecoding}. First, we evaluate whether our approach outperforms generating translations with the NMT model alone. Next, we check if the quality of translations improves compared to N-best list re-ranking. To make the setups comparable, we set $topk$ and $num\_beams$ to $5$ and compare with re-ranking the top $25$ beams using XCOMET-XL. Finally, to demonstrate that re-ranking the N-best list remains a viable and complementary approach, we re-rank the top $5$ beams obtained from Quality-Aware decoding using the same QE model. 

We find that re-ranking with XCOMET-XL and Tower QE yields similar results, indicating that our partial QE model does not over-fit to any specific metric. Furthermore, we observe that the unified decoding approach outperforms N-best list re-ranking across both metrics in both language pairs. For example, the MetricX score improves from $2.37$ to $2.12$ for English $\rightarrow$ German. Note that Tower has already seen this data during instruction-tuning and the improvement is not from new data but from Quality-Aware decoding. Moreover, re-ranking the top $5$ beams obtained from unified decoding with XCOMET-XL leads to a slight further improvement in quality. This highlights the robustness and generalizability of our approach across different evaluation metrics.
%\footnotetext{\href{https://github.com/WMT-QE-Task/wmt-qe-2023-data}{WMT 23 English $\rightarrow$ German QE Data}}

\subsection{Adapting for Tower Errors}

We use the MQM annotations from WMT to train our Tower QE model, which contains error annotations from other systems. However, a viable alternative would be to adapt Tower QE specifically to the errors it typically makes. To maintain a similar data setup, we first generate translations using Tower on these source sentences. Then, we annotate the generated hypotheses with XCOMET-XL using the reference and fine-tune Tower QE on this synthetic dataset, which we refer to as \textbf{Tower Distill QE}. We evaluate the performance of the new distill QE model and report the results in Table \ref{tab:towerdistill}.

We observe that the distilled QE model performs very similarly to the QE model trained on errors from other systems. This indicates that there was no significant benefit in adapting the QE model to the specific errors typically made by Tower. However, further analysis on larger datasets and different domains is needed to fully validate the effectiveness of the distillation approach as the current synthetic data generated is small.

\subsection{Sentence vs Document-level Translation}

From Table \ref{tab:qadecoding}, we observe that the gains for English $\rightarrow$ German (paragraph-level) are much higher than for Chinese $\rightarrow$ English (sentence-level). We hypothesize that this discrepancy arises from the length of the sentences, as the N-best list re-ranking is likely sufficient for shorter sentences. To confirm this, we take the English paragraphs and split them into sentences using a tokenizer while tracking the paragraph IDs. We then perform the entire decoding process similarly, and later join the sentences back using the paragraph IDs before evaluation. We report the results in Table \ref{tab:sentvspara}.

We define the impact as the improvement in translation quality from re-ranking the N-best list with XCOMET-XL QE to integrating Tower QE. Comparing the results at the paragraph level to those at the sentence level, we observe that the impact decreases, which confirms our hypothesis. Additionally, we obtain better scores at the document level, further highlighting the potential benefits of Quality-Aware Decoding.

\section{Related Work}

\textbf{Integrating QE in NMT:} Several advancements have been made in improving QE for NMT over the years \citep{rei2021references, rei2022cometkiwi, blain2023findings, zerva2024findings, guerreiro2024xcomet}. These developments have led to the integration of QE in various ways.
One common approach involves applying QE after generating multiple sequences through techniques such as QE re-ranking \citep{fernandes2022quality, faria2024quest} or Minimum Bayes Risk (MBR) decoding \citep{tomani2024quality}. Another direction focuses on removing noisy data using QE models, followed by fine-tuning on high-quality data \citep{xu2024contrastive, finkelstein2024introducing}. \citet{vernikos2024don} proposes to generate diverse translations as a first step and then combine them. We perform this explicitly by integrating the QE directly into decoding.
Recently, \citet{zhang2024learning} exploited the MQM data by training models to penalize tokens within an error span, improving translation quality. In contrast, our approach adopts a modular framework, where we propose an expert QE model that is trained independently for targeted training. This modular approach aims to improve performance by decomposing the task into separate translation and QE components.

\textbf{Reward Modeling in NLG:}  Quality-Aware decoding shares several similarities with controllable text generation methods, particularly in the use of an additional "Quality/Reward" model that guides the decoding. A well-explored approach for controlling text is altering the decoding with a reward model (Weighted Decoding) \citep{yang2021fudge}. This method modifies the decoding by adjusting token probabilities based on the reward model, allowing for more controlled generation.
Similarly, \citet{deng-raffel-2023-reward} also used a uni-directional reward model, with the aim of maintaining efficiency during generation. This approach minimizes computational complexity while still benefiting from the guiding influence of the reward model. Moreover, recent work by \citet{li-etal-2024-reinforcement} introduced a token-level reinforcement learning-based reward model, providing more fine-grained feedback that enhances control over text generation at a granular level. While similar, the key contribution in our work lies in the development of the first uni-directional QE model for translation. 


\section{Conclusion}
We have shown the importance of Quality-Aware decoding to improve translation quality, rather than relying on post-generation techniques. In this work, we demonstrated how MQM data can be used to build a uni-directional token-level QE model, which is then integrated into the decoding process. Through a series of experiments, we showed that our Quality-Aware decoding approach results in measurable improvements in translation quality. Notably, we did not introduce new training data to the NMT model, and show that the gains stem from Quality-Aware decoding.


\section{Limitations}
While our Quality-Aware decoding improves translation quality, it adds considerable computational complexity to the inference process. Theoretically, this approach would double the time needed to generate a translation and require additional memory to utilize the token-level QE model. One potential solution to mitigate this issue could be to use token-level QE as a reward model for training via Reinforcement Learning.

Additionally, we trained our model on a limited set of human-annotated MQM data. However, current QE models, such as XCOMET, are capable of predicting error tags using the reference with reasonable quality. This suggests that further improvements could be achieved if these models were trained on larger-scale datasets, providing more nuanced feedback and refining translation quality even further.

Lastly, our proposed token-level QE model does not account for error severity. Ideally, it should be able to predict the category of errors, allowing for more nuanced feedback and enabling the model to generate translations with only minor errors when necessary.


% Bibliography entries for the entire Anthology, followed by custom entries
%\bibliography{anthology,custom}
% Custom bibliography entries only
\bibliography{custom}

\appendix

\section{Appendix}
\label{sec:appendix}

% \begin{table*}[!ht]
% \centering
% \begin{tabular}{@{}ccccc@{}}
% \toprule
% \multicolumn{1}{c|}{Model}            & \multicolumn{1}{c|}{Beams}                & \multicolumn{1}{c|}{Re-ranking}              & XCOMET-XL      & XCOMET-XXL     \\ \midrule
% \multicolumn{5}{c}{\textit{English $\rightarrow$ German}}                                                                                                          \\ \midrule
% \multicolumn{1}{c|}{Tower}            & \multicolumn{1}{c|}{5}           & \multicolumn{1}{c|}{\_}                      & 84.93          & 86.93          \\
% \multicolumn{1}{c|}{Tower}            & \multicolumn{1}{c|}{25}                   & \multicolumn{1}{c|}{\textbf{\_}}             & 84.87 & 86.45          \\
% \multicolumn{1}{c|}{Tower MBR}        & \multicolumn{1}{c|}{25}                   & \multicolumn{1}{c|}{\_}                      & 85.23          & 87.09          \\
% \multicolumn{1}{c|}{Tower}            & \multicolumn{1}{c|}{25}                   & \multicolumn{1}{c|}{XCOMET-XL QE}            & 86.56          & 87.79          \\
% \multicolumn{1}{c|}{Tower}            & \multicolumn{1}{c|}{5}                    & \multicolumn{1}{c|}{Tower QE} & 85.34          & 87.33          \\
% \multicolumn{1}{c|}{Tower}            & \multicolumn{1}{c|}{25}                   & \multicolumn{1}{c|}{Tower QE} & 85.40          & 87.40          \\
% \multicolumn{1}{c|}{Tower + Tower QE} & \multicolumn{1}{c|}{5 (25* for Tower QE)} & \multicolumn{1}{c|}{\_}                      & 86.36          & 88.95          \\
% \multicolumn{1}{c|}{Tower + Tower QE} & \multicolumn{1}{c|}{5 (25* for Tower QE)} & \multicolumn{1}{c|}{XCOMET-XL QE}            & \textbf{86.88} & \textbf{89.08} \\ \midrule
% \multicolumn{5}{c}{\textit{Chinese $\rightarrow$ English}}                                                                                                         \\ \midrule
% \multicolumn{1}{c|}{Tower}            & \multicolumn{1}{c|}{5}                    & \multicolumn{1}{c|}{\_}                      & 85.38          & 88.91          \\
% \multicolumn{1}{c|}{Tower}            & \multicolumn{1}{c|}{25}                   & \multicolumn{1}{c|}{\_}                      & 85.29          & 88.71          \\
% \multicolumn{1}{c|}{Tower MBR}        & \multicolumn{1}{c|}{25}                   & \multicolumn{1}{c|}{\_}                      & 86.00          & 89.23          \\
% \multicolumn{1}{c|}{Tower}            & \multicolumn{1}{c|}{25}                   & \multicolumn{1}{c|}{XCOMET-XL QE}            & 87.04          & 89.49          \\
% \multicolumn{1}{c|}{Tower}            & \multicolumn{1}{c|}{5}                    & \multicolumn{1}{c|}{Tower QE} & 85.64          & 89.10          \\
% \multicolumn{1}{c|}{Tower}            & \multicolumn{1}{c|}{25}                   & \multicolumn{1}{c|}{Tower QE} & 85.93          & 89.51          \\
% \multicolumn{1}{c|}{Tower + Tower QE} & \multicolumn{1}{c|}{5 (25* for Tower QE)} & \multicolumn{1}{c|}{\_}                      & 86.01          & 89.82          \\
% \multicolumn{1}{c|}{Tower + Tower QE} & \multicolumn{1}{c|}{5 (25* for Tower QE)} & \multicolumn{1}{c|}{XCOMET-XL QE}            & \textbf{86.67} & \textbf{90.00} \\ \bottomrule
% \end{tabular}
% \caption{COMET scores on WMT23 English $\rightarrow$ German Test set. Both XCOMET metric columns use reference for reporting translation quality and do not when used for re-ranking }
% \end{table*}


\subsection{Training details}
\label{sec:training_detail}

We use the transformers library \citep{wolf-etal-2020-transformers} for training and inference with Tower-Instruct V2.  For adapting Tower to token-level QE, we use LoRA \citep{hulora} based fine-tuning with an additional classifier head. Therefore, we only train the adapters and the weights for classification head.

We add the adapters to the modules \textit{q\_proj,k\_proj,v\_proj,gate\_proj,up\_proj} and \textit{down\_proj}. We set a batch size for each device to 12 initially and enable \textit{auto\_find\_batch\_size} to \textit{True} on 4 NVIDIA RTX A6000 GPU's. For having a  larger batch size during training, we set \textit{gradient\_accumulation\_steps} to 6. We use a \textit{learning\_rate} of $1e^{-5}$. We set the \textit{eval\_steps} to $50$ and \textit{num\_train\_epochs} to $10$. The other parameters are set to default.

Using the cross-entropy loss for token-level QE directly is insufficient due to the fact that the majority of tokens are classified as '\textit{Good}'. Hence, we find that the weighted cross-entropy loss is essential when fine-tuning the model. For the training on human MQM data, we set the weights to $0.05,0.95$ to '\textit{Good}' and '\textit{Bad}' labels respectively. In the case of distilling from XCOMET, we observed more errors. Therefore, we find that setting them $0.2,0.8$ to '\textit{Good}' and '\textit{Bad}' labels respectively provided stable training.

We train on data until WMT'22 for training and use WMT'24 for validation. We calculate the macro '\textit{F1}' on token-level predictions as the validation metric and stop training if it does not improve for 10 consecutive \textit{eval\_steps}.

\subsection{Partial vs Full Sequence Quality Estimation}

We also compare the difference in performance between our proposed token-level QE for partial sequences with Tower trained for full sequence QE. We achieve this by adding a regression head to predict the score at the end-of-sentence token. Hence, the model uses the source and hypothesis to predict the score using regression head at the end.

We fine-tune the model using only direct assesment data \citep{zerva2024findings} (\textbf{Tower Full DA}). Furthermore, we use this as initialisation and continue fine-tuning on the MQM data (\textbf{Tower Full DA + MQM}). We also use LoRA similarly to the previous model with a regression head to adapt the model. We report the scores in Table \ref{tab:correlation_ablation}.

We see that the both Tower QE models based on full sentences outperforms the partial model. However, this is expected as it has seen the entire context and was also trained on larger amounts of data. Nonetheless, the partial model still achieves much higher correlaiton that the log probabilities showcasing its potential for Quality-Aware decoding.

\subsection{Robustness to re-ranking weight}

In our method, we introduce a hyperparameter, $\alpha$, to merge the probabilities from the token-level QE model and the translation model. This section analyzes the impact of $\alpha$ on the final translation quality.

To efficiently evaluate its effect, we re-rank the N-best list using different values of $\alpha$. This approach allows us to estimate the ideal value of $\alpha$ without the need for joint decoding multiple times. If the re-ranking model (in this case, Tower QE) is beneficial, we expect that any $\alpha$ less than 1 will improve translation quality, as it demonstrates that incorporating the probabilities from the QE model is helpful.

We visualize this impact in Figure \ref{fig:mainfigure}. The results show that using an $\alpha$ less than 1 leads to improved translation quality in both scenarios. This indicates that relying entirely on the NMT model does not yield the best results and highlights the importance of the Tower QE model.

Thus, we emphasize that re-ranking the N-best list provides an effective way to tune the value of $\alpha$, and it remains robust to different values.

\begin{figure*}[!htpb]
\begin{promptbox}[title={Tower Translation Prompt}]
    \small
    <|im\_start|>user\\
    Translate the sentence from English into German.\\
    English: \{src\_sent\}\\
    German:\\
    <|im\_end|>\\
    <|im\_start|>assistant
\end{promptbox}

\begin{promptbox}[title={Tower Token-Level QE Prompt}]
    \small
    English:\{src\_sent\}\\
    German: \{tgt\_sent\}
\end{promptbox}
\caption{Prompts used in our experiments for translation and QE model. \{src\_sent\} and \{tgt\_sent\} represent the source and target sentence. We replace the language with Chinese and English when experimenting with that language pair.}
\end{figure*}

\begin{figure*}[!htpb]
    \centering
    % First subfigure
    \begin{subfigure}[b]{0.5\textwidth}
        \centering
        \includegraphics[width=\textwidth]{Figures/alphas_ende_25.png} % Replace with your image path
        \caption{English $\rightarrow$ German}
        \label{fig:subfigure1}
    \end{subfigure}
    
    \vspace{0.5cm} % Adjust space between the two subfigures

    % Second subfigure
    \begin{subfigure}[b]{0.5\textwidth}
        \centering
        \includegraphics[width=\textwidth]{Figures/alphas_zhen_25.png} % Replace with your image path
        \caption{Chinese $\rightarrow$ English}
        \label{fig:subfigure2}
    \end{subfigure}
    
    \caption{Impact of $\alpha$ when re-ranking with token-level Tower QE on WMT'23 Test sets.}
    \label{fig:mainfigure}
\end{figure*}


\begin{table*}[!ht]
\centering
\begin{tabular}{@{}c|ccc@{}}
\toprule
                                                                                      & Pearson        & Spearmann      & Kendall        \\ \midrule
COMETQE                                                                               & \textbf{44.41} & 41.29          & 31.19          \\ \midrule
COMETQE-XL                                                                            & 41.23          & \textbf{42.17} & \textbf{31.84} \\ \midrule
\begin{tabular}[c]{@{}c@{}}COMETQE Scratch\\      Fine-tuned (ours)\end{tabular}      & 36.32          & 33.66          & 25.24          \\ \midrule
Tower Log Prob                                                                        & 32.32          & 16.74          & 12.77          \\ \midrule
\begin{tabular}[c]{@{}c@{}}Tower Partial QE\end{tabular} & 40.56          & 33.96          & 25.87          \\ \midrule
Tower Full DA                                                                        & 33.67          & 36.46          & 27.38          \\ \midrule
Tower Full DA + MQM                                                                 & 32.03          & 40.85          & 30.38          \\ \bottomrule
\end{tabular}
\caption{Full Correlation results on WMT 23 for English $\rightarrow$ German Test set. Partial indicates that the QE model predict scores via token-level where as full indicates predicting the score at the end-of-sentence token. The scores are calculated after removing the few sentences labelled for hallucination detection. Best scores according to each coefficient are highlighted in \textbf{bold}.}
\label{tab:correlation_ablation}
\end{table*}




\end{document}

% Custom bibliography entries only
%\bibliography{custom}
\clearpage
\appendix
\section{Detailed Taxonomy with Examples}
\label{sec:detailed_taxonomy}
Any references to ``URW'' and ``CC'' below denote the Ukraine-Russia War, and the Climate Change domains, respectively.

\subsection{Protagonist}

\textbf{Guardian}: Heroes or guardians who protect values or communities, ensuring safety and upholding justice. They often take on roles such as law enforcement officers, soldiers, or community leaders (e.g., climate change advocacy community leaders). 
\\\underline{Example}: Police officers protecting citizens during a crisis, firefighters saving lives during a disaster, community leaders standing against crime or leaders standing up for action to address climate change. 

\textbf{Martyr}: Martyrs or saviors who sacrifice their well-being, or even their lives, for a greater good or cause. These individuals are often celebrated for their selflessness and dedication. This is mostly in politics, not in CC.
\\\underline{Example}: Civil rights leaders like Martin Luther King Jr., who was assassinated while fighting for equality, or journalists who risk their lives to report on corruption and injustice. 

\textbf{Peacemaker}: Individuals who advocate for harmony, working tirelessly to resolve conflicts and bring about peace. They often engage in diplomacy, negotiations, and mediation. This is mostly in politics, not in CC.
\\\underline{Example}: Nelson Mandela's efforts to reconcile South Africa post-apartheid, or diplomats working to broker peace deals between conflicting nations. 

\textbf{Rebel}: Rebels, revolutionaries, or freedom fighters who challenge the status quo and fight for significant change or liberation from oppression. They are often seen as champions of justice and freedom. 
\\\underline{Example}: Leaders of independence movements like Mahatma Gandhi in India, or modern-day activists fighting for democratic reforms in authoritarian regimes. In CC domain, this includes characters such as Greta Thunberg, or persons who, for instance, chain themselves to trees to prevent deforestation.

\textbf{Underdog}: Entities who are considered unlikely to succeed due to their disadvantaged position but strive against greater forces and obstacles. Their stories often inspire others. 
\\\underline{Example}: Grassroots political candidates overcoming well-funded incumbents, or small nations standing up to larger, more powerful countries. In CC, this could included NEs portrayed as underfunded organizations that are framed as showing promise to make positive impact on CC. 


\textbf{Virtuous}: Individuals portrayed as virtuous, righteous, or noble, who are seen as fair, just, and upholding high moral standards. They are often role models and figures of integrity.
\\\underline{Example}: Judges known for their fairness, or politicians with a reputation for honesty and ethical behavior. In CC, this includes leaders standing up for environmental ethical values to protect planet Earth, or activists pushing for environmental sustainability.


\subsection{Antagonist}

\textbf{Instigator}: Individuals or groups initiating conflict, often seen as the primary cause of tension and discord. They may provoke violence or unrest.
\\\underline{Example}: Politicians using inflammatory rhetoric to incite violence, or groups instigating protests to destabilize governments. In CC, this could also include Greta Thunberg or activists chaining themselves to trees. In the previous example, they were portrayed in positive light as rebels. However, they could just as well be framed in a negative light if they are being portrayed as troublemakers and instigators of problems, and in such a scenario, they would also take the sub-role of Sabateur.

\textbf{Conspirator}: Those involved in plots and secret plans, often working behind the scenes to undermine or deceive others. They engage in covert activities to achieve their goals.
\\\underline{Example}: Figures involved in political scandals or espionage, such as Watergate conspirators or modern cyber espionage cases. In CC, this could manifest as persons or organizations conspiring to bypass environmental regulations to turn up a profit. 

\textbf{Tyrant}: Tyrants and corrupt officials who abuse their power, ruling unjustly and oppressing those under their control. They are often characterized by their authoritarian rule and exploitation.
\\\underline{Example}: Dictators like Kim Jong-un in North Korea, or corrupt officials embezzling public funds and suppressing dissent. 

\textbf{Foreign Adversary}: Entities from other nations or regions creating geopolitical tension and acting against the interests of another country. They are often depicted as threats to national security. This is mostly in politics, not in CC.
\\\underline{Example}: Rival nations involved in espionage or military confrontations, such as the Cold War adversaries, or countries accused of election interference. In CC, foreign adversaries could include portrayal of how other countries are not adhering to CC policies (e.g., China refuses to adhere to CC policies resulting in 20\% increase in CO2 emissions.

\textbf{Traitor}: Individuals who betray a cause or country, often seen as disloyal and treacherous. Their actions are viewed as a significant breach of trust. This is mostly in politics, not in CC.
\\\underline{Example}: Whistleblowers revealing sensitive information for personal gain, or soldiers defecting to enemy forces. Note that if whistleblowers are portrayed in a positive light, their role would be Virtuous. This could equally apply to both politics and CC.

\textbf{Spy}: Spies or double agents accused of espionage, gathering and transmitting sensitive information to a rival or enemy. They operate in secrecy and deception. This is mostly in politics, not in CC.
\\\underline{Example}: Historical figures like Aldrich Ames, who spied for the Soviet Union, or contemporary cases of corporate espionage.

\textbf{Saboteur}: Saboteurs who deliberately damage or obstruct systems, processes, or organizations to cause disruption or failure. They aim to weaken or destroy targets from within.
\\\underline{Example}: Insiders tampering with critical infrastructure, or activists sabotaging industrial operations.

\textbf{Corrupt}: Individuals or entities that engage in unethical or illegal activities for personal gain, prioritizing profit or power over ethics. This includes corrupt politicians, business leaders, and officials.
\\\underline{Example}: Companies involved in environmental pollution, executives engaged in massive financial fraud, or politicians accepting bribes and engaging in graft.

\textbf{Incompetent}: Entities causing harm through ignorance, lack of skill, or incompetence. This includes people committing foolish acts or making poor decisions due to lack of understanding or expertise. Their actions, often unintentional, result in significant negative consequences.
\\\underline{Example}: Leaders making reckless policy decisions without proper understanding, officials mishandling crisis responses, or managers whose poor judgment leads to organizational failures.

\textbf{Terrorist}: Terrorists, mercenaries, insurgents, fanatics, or extremists engaging in violence and terror to further ideological ends, often targeting civilians. They are viewed as significant threats to peace and security. This is mostly in politics, not in CC.
\\\underline{Example}: Groups like ISIS or Al-Qaeda carrying out attacks, or lone-wolf terrorists committing acts of violence.

\textbf{Deceiver}: Deceivers, manipulators, or propagandists who twist the truth, spread misinformation, and manipulate public perception for their own benefit. They undermine trust and truth.
\\\underline{Example}: Politicians spreading false information for political gain, or media outlets engaging in propaganda.

\textbf{Bigot}: Individuals accused of hostility or discrimination against specific groups. This includes entities committing acts falling under racism, sexism, homophobia, Antisemitism, Islamophobia, or any kind of hate speech. This is mostly in politics, not in CC.


\subsection{Innocent}

\textbf{Forgotten}: Marginalized or overlooked groups who are often ignored by society and do not receive the attention or support they need. This includes refugees, who face systemic neglect and exclusion.
\\\underline{Example}: Indigenous populations facing ongoing discrimination; homeless individuals struggling without adequate support; refugees fleeing conflict or persecution.

\textbf{Exploited}: Individuals or groups used for others' gain, often without their consent and with significant detriment to their well-being. They are often victims of labor exploitation, trafficking, or economic manipulation.
\\\underline{Example}: Workers in sweatshops; victims of human trafficking; communities suffering from corporate exploitation of natural resources.

\textbf{Victim}: People cast as victims due to circumstances beyond their control, specifically in two categories: (1) victims of physical harm, including natural disasters, acts of war, terrorism, mugging, physical assault, ... etc., and (2) victims of economic harm, such as sanctions, blockades, and boycotts. Their experiences evoke sympathy and calls for justice, focusing on either physical or economic suffering.
\\\underline{Example}: Victims of natural disasters, such as hurricanes or earthquakes; individuals affected by violent crimes. Victims of economic blockades, sanctions, or boycotts.

\textbf{Scapegoat}: Entities blamed unjustly for problems or failures, often to divert attention from the real causes or culprits. They are made to bear the brunt of criticism and punishment without just cause.
\\\underline{Example}: Minority groups blamed for economic problems; political opponents, accused of provoking national strife, without evidence.



\section{Annotation Guidelines}
\label{sec:annotation_guidelines}

We prepare these general set of guidelines to prepare the annotators and avoid human biases before starting the annotation:

\begin{itemize}
    \item The annotators should get acquainted with the two domains covered by the tasks; for instance, ~\cite{kremlin-propaganda} and~\cite{cc-denial-tax} provide a good coverage of the URW and CC domains,
    \item The annotators' opinions on the topics and sympathies towards key entities mentioned in the articles are irrelevant and should by no means impact the annotation process and their choices, 
    \item The annotators should not exploit any specific external knowledge bases for the purpose of annotating documents.
\end{itemize}

The annotation guidelines we prepare to annotate and curate the entity framing corpus are as follows:

Any references to ``URW'' and ``CC'' below denote the Ukraine-Russia War, and the Climate Change domains, respectively.

\begin{enumerate}

    \item In this annotation task, the entities of interest are understood in a broad sense to include both traditional named entities (such as persons, organizations, and locations) as well as toponym-derived entities. Toponym-derived entities are phrases that indicate a group or collective identity based on a place or affiliation, including but not limited to:
    \begin{itemize}
        \item Political, military, or social groups defined by their association with a location or entity, e.g., ``Trump supporters,'' or ``residents of Ukraine.''
        \item Entities denoting a geographic or organizational affiliation, such as ``Russian forces'' or ``European officials.''
    \end{itemize}
        

    \item The annotators are provided with a number of news articles and are expected to assign role(s) to named entities that are \textbf{central} to the article's story according to the taxonomy of roles that was provided earlier. 
    \item The annotators are provided with a detailed taxonomy that includes definitions and examples. 
    \item The title of an article should not be annotated. The title of the article is the first block of text that appears in the annotation platform Inception.
    \item Only named entities that are central to the narrative of the article should be annotated. Unnamed entities (i.e., nominal entity mentions such as ``migrants'') should not be annotated. 

For more details on what qualifies as a named entity, in addition to the definition of the broader sense of named entities in the first bullet point in these guidelines, the annotators should also examine the NER annotation guidelines outlined in \url{http://www.universalner.org/guidelines/}.
    
    \item The annotators will pick one or more fine-grained roles for the named entities they believe are central to the article's story. 
    \item Entity mentions can be assigned fine-grained roles from more than one main role. However, during curation, we will not be including these instances in the current version, even though we still annotate them.
    \item Named entities that are not central to the story should not be annotated.

The determination of how central a named entity is in an article is admittedly subjective. To reduce bias, such determination should be based on the careful reading of the article and the story it is pushing. An annotated example is provided in \ref{fig:annotated_example_text2}. Notice that named entities such as New York Times and Israel were not annotated because they are not central to the story.
    
    \item As a general rule, annotators should annotate only the first mention per entity where it is clear that this entity has the specific role(s). There is no need to annotate subsequent mentions of this entity with the same role, But  annotating more mentions with the same surface form and role is not a mistake, but it is simply not required.

    This rule also extends for surface mentions of the same entity. For example, ``Putin'' and ``Vladimir Putin'' are both surface mentions of the same entity, so only the first occurrence of any of the surface forms  would be annotated.

    On the other hand, while entities such as ``Moscow'', ``Russia'', and ``Putin'' are closely related, they are not surface forms of the same entity, and are considered as distinct, separate entities.

    \item Named entities appearing in indirect speech (e.g., quotes) should not be annotated. Indirect speech should be considered as supporting detail to the story, but named entities which are central to the narrative would likely appear also outside of quotations. This guideline helps avoid confusions inherent in indirect speech.

    \item If the above would result in more than one mention of the same entity with the same role, the curator does not need to remove all these additional mentions. We keep all of them.
    \item Should an entity mention that was previously annotated with a certain role appear in a different context with different role(s), the first mention where the role(s) changed should be annotated.

The above rule is repeated for as many times as an entity changes roles across mentions. For example, if an en entity, let's say NATO, appears 20 times in an article. The first 10 mentions show NATO as a Guardian and a Virtuous entity. The 11-15th mentions portray NATO as a Foreign Adversary, and the 16-20th mentions portray NATO as Exploited. Then we only need 3 annotations in total to account for the 3 different roles NATO was portrayed as. These 3 annotations should all be the first mention occurrences where NATO assumed each distinct set of roles (i.e., mention 1, mention 11, and mention 16 should be annotated).
    \item Regarding scenarios where different surface forms for the same named entity (e.g., NATO vs North Atlantic Treaty Organization) appear in the article, it is sufficient to pick only one of the surface forms.
    \item If the above results in multiple surface forms of the same entity being annotated, the curator does not need to remove all of these additional mentions. We keep all of them.
    \item If the mention of the entity does not have any role vis-a-vis the taxonomy of roles then no role should be given. As a consequence, we do not need an ``Other'' label. 
    % \item It is important that only information found in the article is used---the annotator must not rely on external knowledge, to avoid bias and subjectivity.
    \item The curator may see conflicting annotations in the curation mode and could resolve the conflict, and then the remaining non-conflicting roles could be checked and adopted accordingly. 


\end{enumerate}



\section{Experimental Settings}
\label{sec:appendix_experiments}

All fine-tuning experiments were conducted on a single NVIDIA RTX 4090 GPU with 24 GB of memory. We fine-tuned XLM-R (XLM-RoBERTa) in a single run, using a fixed random seed to ensure reproducibility. When the input context was at the sentence granularity, we performed sentence splitting using Stanza pipelines for each one of our five languages. For XLM-R, default settings were applied, with the following configurations:

\begin{itemize}
    \item Model: XLM-R\textsubscript{base} (125M parameters) 
    \item Learning Rate: 2e-5
    \item Batch Size: 8
    \item Epochs: 20 (with early stopping of 3 based on validation loss)
    \item Random Seed: 42
    \item Weight Decay: 0.01
\end{itemize}

To optimize performance, the sigmoid thresholds for fine-grained role predictions were tuned on the validation set. These optimized thresholds were then applied to generate predictions on the test set.

To prevent data leakage, we created train/dev/test splits based on entire articles rather than individual entity-mention annotations. The details of these splits are provided in Table~\ref{tab:train_dev_test_split}.

\begin{table}[]
\centering
\resizebox{\columnwidth}{!}{%

\begin{tabular}{lccccc|c}
\toprule
 & BG & EN & HI & PT & RU & All \\
\midrule
Train & 165 (389) & 133 (440) & 203 (1347) & 206 (833) & 89 (252) & 796 (3261) \\
Dev & 94 (237) & 69 (245) & 139 (983) & 100 (417) & 44 (114) & 446 (1996) \\
Test & 15 (30) & 27 (90) & 35 (279) & 31 (115) & 28 (85) & 136 (599) \\ \midrule
Total & 274 (656) & 229 (775) & 377 (2609) & 337 (1365) & 161 (451) & 1378 (5856) \\
\bottomrule
\end{tabular}
}
\caption{Distribution of articles and entity mentions by language and split. The number of entity mentions is shown in parentheses}
\label{tab:train_dev_test_split}
\end{table}

For the zero-shot experiments, we used OpenAI's GPT-4o (gpt-4o-2024-11-20) with a temperature setting of 0.2 to produce more conservative responses. To ensure the outputs conformed to our defined data types, we employed OpenAI's Structured Outputs API, which returned results in the expected JSON format.




\clearpage
\onecolumn

\section{Dataset Statistics}

\label{sec:appendix_stats}

\begin{figure*}[!h]
    \centering
    \includegraphics[width=1\textwidth]{images/fine_role_co_occurence_normalized.pdf}
    \caption{Normalized co-occurrence of fine-grained roles.}
    \label{fig:fine_roles_co_occurence_normalized}
\end{figure*}

\begin{figure}[!h]  % 't' ensures the figure is placed at the top of the page
    \centering

    % First Subfigure
    \begin{subfigure}[t]{0.9\textwidth}  % Width for the first subfigure
        \centering
        \includegraphics[width=\textwidth]{images/frequent_pairs.pdf}
        \caption{}
        \label{fig:frequent_pairs}
    \end{subfigure}
    \hfill  % Horizontal space to push subfigures apart
    
    % Second Subfigure
    \begin{subfigure}[t]{\textwidth}  % Width for the second subfigure
        \centering
        \includegraphics[width=\textwidth]{images/frequent_triplets.pdf}
        \caption{}
        \label{fig:frequent_triplets}
    \end{subfigure}

    % Main Caption for the Entire Figure
    \caption{The 10 most frequent co-occurring (a) pairs and (b) triplets of fine-grained roles.}
    \label{fig:frequent_roles_combined}
\end{figure}

\begin{figure}  
    % \captionsetup{font=scriptsize}

    % First Subfigure
    \begin{subfigure}[b]{0.9\textwidth}  % Width for the first subfigure
        \centering
        % \captionsetup{font=scriptsize}
        \includegraphics[width=\textwidth]{images/proportion_of_main_roles_per_language.pdf}
        \caption{}
        \label{fig:main_roles_per_language}
    \end{subfigure}
    % \hfill  % Horizontal space to push subfigures apart
    
    % Second Subfigure
    \begin{subfigure}[b]{\textwidth}  % Width for the second subfigure
        \centering
        % \captionsetup{font=scriptsize}
        \includegraphics[width=\textwidth]{images/proportion_of_fine_roles_per_language.pdf}
        \caption{}
        \label{fig:fine_roles_per_language}
    \end{subfigure}

    % Main Caption for the Entire Figure
    \caption{Proportions of (a) main roles and (b) fine-grained roles per language.}
    \label{fig:proportions_of_roles}
\end{figure}




% \begin{figure}[htbp]
%     \centering

%     % First Row: English and Portuguese
%     \begin{subfigure}[b]{0.45\textwidth}
%         \centering
%         \includegraphics[width=\textwidth]{images/word_cloud_EN.pdf}
%         \caption{}
%         \label{fig:wordcloud_en}
%     \end{subfigure}
%     \hfill
%     \begin{subfigure}[b]{0.45\textwidth}
%         \centering
%         \includegraphics[width=\textwidth]{images/word_cloud_PT.pdf}
%         \caption{}
%         \label{fig:wordcloud_pt}
%     \end{subfigure}

%     \vspace{0.5cm}

%     % Second Row: Bulgarian and Russian
%     \begin{subfigure}[b]{0.45\textwidth}
%         \centering
%         \includegraphics[width=\textwidth]{images/word_cloud_BG.pdf}
%         \caption{}
%         \label{fig:wordcloud_bg}
%     \end{subfigure}
%     \hfill
%     \begin{subfigure}[b]{0.45\textwidth}
%         \centering
%         \includegraphics[width=\textwidth]{images/word_cloud_RU.pdf}
%         \caption{}
%         \label{fig:wordcloud_ru}
%     \end{subfigure}

%     % \vspace{0.5cm}

%     % % Third Row: Hindi (centered)
%     % \begin{subfigure}[b]{0.45\textwidth}
%     %     \centering
%     %     \includegraphics[width=\textwidth]{images/word_cloud_HI.pdf}
%     %     \caption{}
%     %     \label{fig:wordcloud_hi}
%     % \end{subfigure}

%     % Main Caption for the Entire Figure
%     \caption{Word clouds for annotated entity mentions in different languages: (a) English, (b) Portuguese, (c) Bulgarian, and (d) Russian.}
%     \label{fig:wordclouds_all_languages}
% \end{figure}



\begin{figure}
    \centering
    \includegraphics[scale=0.75]{images/hist_ent.png}
    \caption{Top entities counts after multilingual linking was manually performed to link surface string to unique identifiers. The entities selected are all the ones for which at least one surface string has a count of a least 10 in any language}
    \label{fig:hist_ent}
\end{figure}

\begin{figure}
    \centering
    \includegraphics[scale=1.5]{images/heathmap_ent_2nd.png}
    \caption{Heathmap of the raw count of mention of entity and 2nd level role, where entities are defined and selected as in Figure~\ref{fig:hist_ent}}
    \label{fig:entity_2nd}
\end{figure}

\begin{figure}
    \centering
    \includegraphics[scale=0.22]{images/graph_topent_large.png}
    \caption{Graph of the top entities, nodes are a pair of entity and 2nd level role, node size is relative to the count of mentions, node color codes the 1st level role, there is and edge between two nodes, if they appear in the same document. This graph illustrate how group of entity+role pairs can be used to identify potential narratives.}
    \label{fig:enter-label}
\end{figure}



\clearpage
\section{Prompts for Hierarchical Zero-Shot Experiments}
\label{sec:appendix_zeroshot}


\begin{figure*}[htbp]
\centering
\begin{framed}
\begin{minipage}{0.9\textwidth}
\begin{Verbatim}[fontsize=\scriptsize,commandchars=\\\{\}]

\lightbrowntext{You are an expert at identifying entity framing and role portrayal in news articles. Analyze the following entity} 
\lightbrowntext{mention in context, and predict its main role and fine-grained role(s) from the taxonomy below.} 

\lightbrowntext{Taxonomy:} \darkbluetext{\{}\lightbluetext{detailed taxonomy with definitions and examples}\darkbluetext{\}}

\lightbrowntext{Context Around Entity:} \darkbluetext{\{}\lightbluetext{context}\darkbluetext{\}}

\lightbrowntext{Entity Mention:} \darkbluetext{\{}\lightbluetext{entity mention}\darkbluetext{\}}

\lightbrowntext{Task: Based on the provided context, assign to the entity mention at least one fine-grained role and}
\lightbrowntext{exactly one main role.}

\lightbrowntext{Return a JSON that has below attributes:}
\lightbrowntext{- \textbf{main role}: either one of Protagonist, Antagonist, or Innocent}
\lightbrowntext{- \textbf{fine grained roles}: a list of all your predicted fine-grained roles}

\end{Verbatim}
\end{minipage}
\end{framed}
\caption{\textbf{Single-Step Prompt Template.} The detailed taxonomy is the same one shown in \ref{sec:detailed_taxonomy}. The context is the text consisting of entity mention along with the 20 words before and after the entity mention.}
\label{fig:single_step_prompt_template}
\end{figure*}

\begin{figure*}[htbp]
\centering
\begin{framed}
\begin{minipage}{0.9\textwidth}
\begin{Verbatim}[fontsize=\scriptsize,commandchars=\\\{\}]
\darkbluetext{First Step (LLM Call 1): Predict the Main Role}

\lightbrowntext{You are an expert at identifying entity framing and role portrayal in news articles. Analyze the following entity }
\lightbrowntext{mention in context, and predict its main role from the taxonomy below.} 

\lightbrowntext{Taxonomy:} \darkbluetext{\{}\lightbluetext{list of fine-grained roles per main role}\darkbluetext{\}}

\lightbrowntext{Context Around Entity:} \darkbluetext{\{}\lightbluetext{context}\darkbluetext{\}}

\lightbrowntext{Entity Mention:} \darkbluetext{\{}\lightbluetext{entity mention}\darkbluetext{\}}

\lightbrowntext{Task: Based on the provided context, assign to the entity mention exactly one main role.}

\lightbrowntext{Return a JSON that has this attribute:}
\lightbrowntext{- \textbf{main role}: either one of Protagonist, Antagonist, or Innocent}


\darkbluetext{Second Step (LLM Call 2): Predict the Fine-Grained Role}

\lightbrowntext{You are an expert at identifying entity framing and role portrayal in news articles. This entity is}
\lightbrowntext{portrayed as a(n)} \darkbluetext{\{}\lightbluetext{main role}\darkbluetext{\}}\lightbrowntext{ and your task is to analyze the entity mention in context}
\lightbrowntext{and predict its fine-grained role(s) from the taxonomy below.}

\lightbrowntext{Taxonomy:} \darkbluetext{\{}\lightbluetext{pertinent portion of the detailed taxonomy with definitions and examples}\darkbluetext{\}}

\lightbrowntext{Context Around Entity:} \darkbluetext{\{}\lightbluetext{context}\darkbluetext{\}}

\lightbrowntext{Entity Mention:} \darkbluetext{\{}\lightbluetext{entity mention}\darkbluetext{\}}

\lightbrowntext{Task: Based on the provided context, assign to the entity mention at least one fine-grained role.}

\lightbrowntext{Return a JSON that has this attribute:}
\lightbrowntext{- \textbf{fine grained roles}: a list of all your predicted fine-grained roles}
\end{Verbatim}
\end{minipage}
\end{framed}
\caption{\textbf{Multi-Step Prompt Template.} In the first step, the taxonomy is only the tree structure of the taxonomy and does not include any definitions or examples. In the second step, the detailed taxonomy only includes the branch under the predicted main role in the first step. The context is as defined in Figure \ref{fig:single_step_prompt_template}. }
\label{fig:multi_step_prompt_template}
\end{figure*}


% \fbox{%
%     \parbox{\textwidth}{%
% \bf{\textcolor{red}{Single Step Prompt}}
% \\

%  Taxonomy Definitions: \\
% Main Roles: \{Main Roles Definitions\} \\
% Fine-Grained Roles: \{Fine Grained Roles Definitions\} \\
% You are an expert at identifying entity framing and role portrayal in news articles. Analyze the following entity mention in context, and predict its main role and fine-grained role(s) from the taxonomy below. \\
% \{taxonomy section\} \\
% \{document section\} \\
% Context Around Entity:
% \{context\} \\
% Entity Mention:
% \{entity mention\} \\
% Task: Based on the provided context, assign to the entity mention at least one fine-grained role and exactly one main role. Return a JSON that has below attributes: \\
% ``main role'': either one of Protagonist, Antagonist, or Innocent \\
% ``fine grained roles'': this is a list of all of your predicted fine-grained roles \\
% ... \\

%     }%
% }

% \fbox{%
%     \parbox{\textwidth}{%
% \bf{\textcolor{red}{Multi Step Prompts}} 
% \\

% \textcolor{blue}{First Step (LLM call): Define the main role}

%  Taxonomy Definitions: \\
% Main Roles: \{Main Roles Definitions\} \\
% You are an expert at identifying entity framing and role portrayal in news articles. Analyze the following entity mention in context, and predict its main role from the taxonomy below.\\
% \{taxonomy section\} \\
% \{document section\} \\
% Context Around Entity:
% \{context\} \\
% Entity Mention: \\
% \{entity mention\} \\
% Task:
% Based on the provided context, assign to the entity mention exactly one main role. Return a JSON that has below attributes: \\
% ``main role'': either one of Protagonist, Antagonist, or Innocent \\
% ... \\
% \textcolor{blue}{Second step (LLM call): Define the fine grained role (protagonist/antagonist/innocent)}

% You are an expert at identifying entity framing and role portrayal in news articles. This entity is portrayed as a protagonist/antagonist/innocent and your task is to analyze the following entity mention in context, and predict its role from the taxonomy below.\\
% \{taxonomy section\} \\
% \{document section\} \\
% Context Around Entity:
% \{context\} \\
% Entity Mention: \\
% \{entity mention\} \\
% Task:
% Based on the provided context, assign to the entity mention at least one fine-grained role. Return a JSON that has below attributes: \\
% "fine grained roles": this is a list of all of your predicted protagonist/antagonist/innocent fine-grained roles
%     }%
% }

%\clearpage
%\section{Top Entities}


% \begin{figure}[!h]
%     \centering
%     \includegraphics[scale=0.3]{images/st1_language-1st_normX.png}
%     \caption{language-1st role, normalised over language}
%     \label{fig:enter-label}
% \end{figure}


% \begin{figure}[!h]
%     \centering
%     \includegraphics[scale=0.3]{images/matrix_entity-language_normY.png}
%     \caption{top entity - language, normalised over language}
%     \label{fig:enter-label}
% \end{figure}

% \begin{figure}[!h]
%     \centering
%     \includegraphics[scale=0.3]{images/matrix_entity-language_normX.png}
%     \caption{top entities - language, normalised over entities}
%     \label{fig:enter-label}
% \end{figure}

\clearpage
\section{Annotation Tool}
\label{sec:annotation_tool}
We used the Inception~\cite{tubiblio106270} platform\footnote{https://inception-project.github.io/} to annotate our corpus because it has a rich set of features that extends beyond mere annotation to also include useful tools such as the ability to perform annotation adjudication through curation, monitoring the annotation progress, and calculating agreement between annotators. Inception allows to assign the following roles to users: annotator, curator, and manager.

To annotate a mention of an entity with a role, annotators should go to the part of the article where the entity is mentioned and select it. After selecting an entity mention, annotators can then assign roles as shown in \ref{fig:INC_entity}.


\begin{figure}[!htpb]
\centering
\includegraphics[width=1\textwidth]{images/Entity_layer.PNG}
\caption{Annotating entity framing using Inception.}
\label{fig:INC_entity}
\end{figure}






\end{document}



