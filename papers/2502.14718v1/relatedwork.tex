\section{Related Work\label{sec:related_work}
}

\citet{sharma-etal-2023-characterizing} introduced a dataset for identifying heroes, villains, and victims in memes, focusing on visual content. In contrast, our dataset focuses on textual content. While both share similar coarse-level roles, our work adds an additional layer of granularity with a hierarchical taxonomy of 22 archetypes nested within these roles. 

\citet{card-etal-2016-analyzing} addressed a different aspect of framing. Their contribution is developing a model that makes use of personas to infer the article framing as defined in the Media Frames Corpus (MFC)  \cite{card-etal-2015-media}. MFC focuses on identifying how an article is framed along nine dimensions, such as Economic or Political. In their work, topic modeling is used to identify 50 personas. However, the results are noisy, with only a few informative personas, namely \emph{refugee} and \emph{immigrant}, while the 48 other personas, such as Job, Worker, and Year, are less informative. In contrast, our work focuses on \emph{entity} framing rather than article framing. While their work identifies personas in a weakly supervised manner, we develop a hierarchical taxonomy containing a richer set of roles, validated through human annotation of news articles across two diverse domains, ensuring higher quality and broader applicability. There is more research on news framing that focuses on article-level framing \cite{Pastorino2024DecodingNN, DBLP:conf/acl/0001KF24, piskorski-etal-2023-multilingual, liu-etal-2019-detecting, card-etal-2015-media}. In contrast, our work centers on entity-level framing.

Aspect-Based Sentiment Analysis \cite{chebolu-etal-2024-oats, DBLP:journals/corr/abs-2203-01054, orbach-etal-2021-yaso, jiang-etal-2019-challenge, saeidi-etal-2016-sentihood} is also related. It involves identifying targets of specific opinions and determining the polarity of the sentiment associated with particular aspects of these targets. Typically, the polarity is binary, and multiple aspects of the target entity are examined. Our work on entity framing is different as we do not define aspects nor do we assign polarities. Instead, we introduce a hierarchical taxonomy for news, which contains a rich set of roles inspired by elements of storytelling, and entities can be classified into any subset of roles within that taxonomy.