\section{Introduction}

Satisfiability Modulo Theories (SMT) is the problem of deciding the satisfiability of a logical formula w.r.t.\ some background theory, such as linear and nonlinear arithmetic, bit-vectors, arrays, or uninterpreted functions~\cite{barrettSatisfiabilityModuloTheories2021}.
%
Many SMT-encodable problems also require the capability of finding models that are optimal w.r.t.\ some objective functions. These problems are grouped under the term Optimization Modulo Theories
(OMT)~\cite{nieuwenhuisSATModuloTheories2006,sebastianiOptimizationSMTLAQ2012,bjornerNZOptimizingSMT2015}.
%
% making it a powerful tool for solving complex constraint optimization problems in various domains.
OMT has been successfully applied to a wide range of problems, such as
verification of timed and hybrid
systems~\cite{sebastianiOptimizationSMTLAQ2012,henryHowComputeWorstcase2014},
numeric~\cite{leofanteOptimalPlanningModulo2021} and temporal
planning~\cite{panjkovicExpressiveOptimalTemporal2023,panjkovicAbstractActionScheduling2024},
optimal scheduling~\cite{bofillEfficientSMTApproach2017},
constrained goal modelling~\cite{nguyenMultiobjectiveReasoningConstrained2018}, 
hybrid machine learning~\cite{tesoStructuredLearningModulo2017}, GAS optimization for smart
contracts~\cite{albertGASOLGasAnalysis2020}, and optimum
encodings for quantum annealing~\cite{bianSolvingSATMaxSAT2020,dingEffectivePrimeFactorization2024}, 
establishing OMT solvers as powerful tools for solving complex constraint optimization problems in various domains.

% Several other contexts have seen the successful application of OMT techniques, e.g., verification of parametric systems~\cite{liSymbolicOptimizationSMT2014},quantum annealing~\cite{bianSolvingSATMaxSAT2020}, constrained goal
% modelling~\cite{nguyenRequirementsEvolutionEvolution2016,nguyenModelingReasoningRequirements2017,nguyenMultiobjectiveReasoningConstrained2018}, and hybrid machine learning~\cite{structured}.

% OMT techniques have been developed for 
% \larat~\cite{bjornerNZOptimizingSMT2015,sebastianiOptimizationSMTLAQ2012,sebastianiOptimizationModuloTheories2015}, 
% \laratint~\cite{bjornerNZOptimizingSMT2015,sebastianiPushingEnvelopeOptimization2015},
% \nlaratint~\cite{bigarellaOptimizationModuloNonlinear2021},
% \mem~\cite{nadelBitVectorOptimization2016,trentinOptimizationModuloTheories2021}, and 
% \fl~\cite{trentinOptimizationModuloTheories2021}.


\paragraph{OMT solving.}%
\label{sec:related-work-omt}
% OMT techniques have been developed for 
% \larat~\cite{bjornerNZOptimizingSMT2015,sebastianiOptimizationSMTLAQ2012,sebastianiOptimizationModuloTheories2015}, 
% \laratint~\cite{bjornerNZOptimizingSMT2015,sebastianiPushingEnvelopeOptimization2015},
% \nlaratint~\cite{bigarellaOptimizationModuloNonlinear2021},
% \mem~\cite{nadelBitVectorOptimization2016,trentinOptimizationModuloTheories2021}, and 
% \fl~\cite{trentinOptimizationModuloTheories2021}.
%
A general OMT-solving
strategy~\cite{nieuwenhuisSATModuloTheories2006,sebastianiOptimizationSMTLAQ2012,sebastianiOptimizationModuloTheories2015}
consists in performing a sequence of incremental SMT calls, progressively
tightening the range of values for the objective function.
%
Specifically, an \smt{} solver is used to enumerate \T-satisfiable truth
assignments that propositionally satisfy the problem formula $\vi$. For each
such truth assignment, a \T-optimizer finds a \T-model of optimum cost within
it. A constraint is then added to the formula to tighten the upper bound for
the cost of the optimum model, and the search continues until the formula is
found unsatisfiable.
Besides optimal solving, an important feature of \omt{}
solvers is the ability to provide the user with a good-enough solution within a
given time budget. This capability, known as \emph{anytime} OMT solving, is
especially valuable in industrial applications where finding the optimum solution may be computationally impractical, and it is rather more important to obtain high-quality solutions quickly.

\omt{} techniques have been developed for \larat{}~\cite{bjornerNZOptimizingSMT2015,sebastianiOptimizationModuloTheories2015}, \laint{}~\cite{bjornerNZOptimizingSMT2015,sebastianiPushingEnvelopeOptimization2015}, \nlarat~\cite{bigarellaOptimizationModuloNonlinear2021}, \nlaint~\cite{bigarellaOptimizationModuloNonlinear2021}, \bv~\cite{nadelBitVectorOptimization2016,trentinOptimizationModuloTheories2021}, and \fl~\cite{trentinOptimizationModuloTheories2021}.
Also, \omt{} has been extended to deal with multiple objectives including lexicographic \omt{}~\cite{bjornerNZOptimizingSMT2015,sebastianiPushingEnvelopeOptimization2015}, boxed \omt{}~\cite{bjornerNZOptimizingSMT2015,liSymbolicOptimizationSMT2014,sebastianiPushingEnvelopeOptimization2015}, min-max \omt{}~\cite{sebastianiOptiMathSATToolOptimization2020}, and Pareto \omt{}~\cite{bjornerNZOptimizingSMT2015}. %, which have all been implemented in the \omt{} solver \optimathsat{}~\cite{sebastianiOptiMathSATToolOptimization2020}.
%
Recently, a Generalized OMT calculus has been proposed, extending the
definition to objectives over partially ordered
sets~\cite{tsiskaridzeGeneralizedOptimizationModulo2024}.

\paragraph{Partial assignments enumeration SMT.}%
\label{sec:related-work-partial-assignments}

The problem of truth assignment enumeration has been studied in recent years,
mainly in the context of SAT and \smt{} enumeration (AllSAT and AllSMT).
% Many applications of SAT and \smt{} solving and related tasks can benefit from finding (short) partial truth assignments.
Typically, enumeration
algorithms~\cite{lahiriSMTTechniquesFast2006,friedAllSATCombinationalCircuits2023,friedEntailingGeneralizationBoosts2024,spallittaDisjointPartialEnumeration2024}
rely their efficiency on the enumeration of partial assignments to reduce both
the number of enumerated assignments and the computational time by up to an
exponential factor.
%, since a partial assignment can be extended to $2^k$ total truth assignments, where $k$ is the number of unassigned atoms.
%
Several techniques have been proposed to find short satisfying partial
assignments starting from a total assignment, trading off efficiency for
effectiveness
(e.g.,~\cite{morgadoGoodLearningImplicit2005,raviMinimalAssignmentsBounded2004,todaImplementingEfficientAll2016}).
%A very simple algorithm to find \emph{minimal} partial assignments, implemented e.g., in \mathsat{}~\cite{mathsat5_tacas13}, consists in iteratively dropping literals one-by-one from the satisfying total assignment, checking if it still satisfies the formula. 
Also, the impact of CNF-ization on the effectiveness of partial assignment
reduction has been recently studied
in~\cite{masinaCNFConversionDisjoint2023,spallittaEnhancingSMTbasedWeighted2024}.


\paragraph{Contributions.}
In this paper, we study the applicability of enumeration-based techniques to OMT solving, and, in particular, the usage of partial truth assignment reduction to improve the effectiveness and efficiency of OMT solving.
First, we notice that OMT solvers typically invoke the \T-optimizer on total truth assignments, and we spotlight how this can be suboptimal in many cases.
Second, we propose some ways to exploit partial truth assignments in OMT solving, tailoring existing techniques to the OMT context.
We show through an empirical evaluation over \omlarat{} benchmarks that these strategies can improve both the efficiency of OMT optimal solving and the quality of obtained solutions for anytime solving. %, with a small overhead.

\paragraph{Organization.}
The rest of the paper is organized as follows.
In \sref{sec:background}, we provide the necessary background on SMT and OMT solving.
In \sref{sec:analysis}, we analyze the role of total and partial truth assignments in OMT solving.
In \sref{sec:approach}, we propose two strategies to exploit partial truth assignments in OMT solving.
In \sref{sec:experiments}, we present an experimental evaluation of the proposed strategies over \omlarat{} benchmarks.
Finally, in \sref{sec:conclusions}, we conclude the paper and discuss future work.
