% The abstract should briefly summarize the contents of the paper in
% 150--250 words.

Optimization Modulo Theories (OMT) extends Satisfiability Modulo Theories (SMT) with the task of optimizing some objective function(s). 
In  OMT solvers,
a CDCL-based SMT solver enumerates theory-satisfiable total truth assignments, and 
a theory-specific  procedure finds an optimum model for each of them;
the current optimum is then used to tighten the search space for the next assignments, until no better solution is found.

In this paper, we analyze the role of truth-assignment enumeration in OMT. First, we spotlight that the enumeration of \emph{total} truth assignments is suboptimal, since they may over-restrict the search space for the optimization procedure, whereas using \emph{partial} truth assignments instead can improve the effectiveness of the optimization. Second, we propose some reduction techniques for better exploiting partial assignments in the OMT context.
%
We implemented these techniques in the \optimathsat{} solver, and conducted an experimental evaluation on \omlarat{} benchmarks.
The results support the efficiency and effectiveness of our approach. 
