\section{Experimental evaluation}%
\label{sec:experiments}

We implemented the above algorithms in the OMT solver
\optimathsat{}~\cite{sebastianiOptiMathSATToolOptimization2020}, which is built
on top of the \mathsatfive{} SMT solver~\cite{mathsat5_tacas13}.
%
We evaluated the proposed strategies on a set of \omlarat{} benchmarks coming
from different sources, evaluating both solving time for optimum solving, and
the quality of the solutions found within the given
timeout for anytime solving.
%
All the experiments were run on an Intel Xeon Gold 6238R @ 2.20GHz 28 Core
machine with 128 GB of RAM, running Ubuntu Linux 22.04. The timeout was set at 1200s. The tool, benchmarks and results
are available at \url{https://optimathsat.disi.unitn.it/resources/optimathsat-cade-30-submission.tar.gz}.
%\GMTODO{upload code and
%    benchmarks}

\subsection{Benchmarks}%
\label{sec:experiments:benchmarks}
%The exploitation of partial assignments in OMT is a general technique that can be applied to any OMT problem. % However, it is most effective in problems that are largely satisfiable and where the combinatorial explosion of the search space of theory atoms is the main bottleneck.
We evaluated the proposed strategies on two classes of \omlarat{} benchmarks: OMT-encoded optimal temporal planning~\cite{panjkovicExpressiveOptimalTemporal2023,panjkovicAbstractActionScheduling2024} and strip-packing problems~\cite{sebastianiOptimizationSMTLAQ2012,sebastianiOptimizationModuloTheories2015}.

\paragraph{Optimal Temporal Planning.}
In~\cite{panjkovicExpressiveOptimalTemporal2023,panjkovicAbstractActionScheduling2024},
the authors proposed a way to encode optimal temporal planning problems into a
sequence of \omlarat{} problems. Each problem encodes a bounded version of the
problem up to a fixed horizon, with additional abstract actions representing an
over-approximation of the plans beyond the bound, minimizing the makespan,
i.e., the total time taken to reach the goal. If the optimal plan is found
without using the abstract actions, then the plan is optimum for the original
problem. Otherwise, the horizon is increased, and the process is repeated. We
generated problems using the industrial problems Majsp (80 instances),
MajspSimplified (80 instances), and Painter (30
instances)~\cite{panjkovicAbstractActionScheduling2024}, with increasing
horizon $h\in\set{5, 10, 15, 20, 25, 30, 35, 40}$, for a total of 1520
instances.

\paragraph{Strip-packing.}
The strip-packing problem (SP) requires arranging $N$ rectangles, each with a
specific width $W_i$ and height $H_i$, into a strip of fixed height $H$ and
unlimited length. The goal is to minimize the length $L$ of the used part
of the strip, ensuring that all rectangles are placed without overlap or
rotation. An \omlarat{} encoding for SP was proposed
in~\cite{sebastianiOptimizationModuloTheories2015}. Following~\cite{sebastianiOptimizationModuloTheories2015}, we sampled $H_i$ uniformly in $(0,1]$, $W_i$ in $(1, 2]$, and set $H=\sqrt{N}/2$.
We generated 25 random SP problems for each value of $N\in\set{25,50,75,100}$,
for a total of $100$ instances.

\subsection{Results}%
\label{sec:experiments:results}

\begin{figure}[t]
    \begin{tabularx}{\textwidth}{cc|c}
        \multicolumn{3}{c}{
            \includegraphics[width=\columnwidth]{plots/QF_LRA/planning/all/legend.pdf}
        }                                                                                                                                                                          \\
        %%%%%%%%%%%%%%% TIME %%%%%%%%%%%%%%%
        \includegraphics[width=.33\columnwidth]{plots/QF_LRA/planning/all/time/time_compare_omsat-min-none_vs_omsat-min-simple__-__clause.pdf}                                   &
        \includegraphics[width=.33\columnwidth]{plots/QF_LRA/planning/all/time/time_compare_omsat-min-none_vs_omsat-min-guided_drop_iter__-__clause.pdf}                         &
        \includegraphics[width=.33\columnwidth]{plots/QF_LRA/planning/all/time/time_compare_omsat-min-simple__-__clause_vs_omsat-min-guided_drop_iter__-__clause.pdf}              \\
        %%%%%%%%%%%%%%% UPPER BOUNDS %%%%%%%%%%%%%%%
        \includegraphics[width=.33\columnwidth]{plots/QF_LRA/planning/all/upper/upper_compare_omsat-min-none_vs_omsat-min-simple__-__clause.pdf}                                 &
        \includegraphics[width=.33\columnwidth]{plots/QF_LRA/planning/all/upper/upper_compare_omsat-min-none_vs_omsat-min-guided_drop_iter__-__clause.pdf}                       &
        \includegraphics[width=.33\columnwidth]{plots/QF_LRA/planning/all/upper/upper_compare_omsat-min-simple__-__clause_vs_omsat-min-guided_drop_iter__-__clause.pdf}            \\
        %%%%%%%%%%%%%%% ITERATIONS %%%%%%%%%%%%%%%
        \includegraphics[width=.33\columnwidth]{plots/QF_LRA/planning/all/sat_search_steps/sat_search_steps_compare_omsat-min-none_vs_omsat-min-simple__-__clause.pdf}           &
        \includegraphics[width=.33\columnwidth]{plots/QF_LRA/planning/all/sat_search_steps/sat_search_steps_compare_omsat-min-none_vs_omsat-min-guided_drop_iter__-__clause.pdf} &
        \includegraphics[width=.33\columnwidth]{plots/QF_LRA/planning/all/sat_search_steps/sat_search_steps_compare_omsat-min-simple__-__clause_vs_omsat-min-guided_drop_iter__-__clause.pdf}
    \end{tabularx}
    % \centering
    % %%%%%%%%%%%%%%% LEGEND %%%%%%%%%%%%%%%
    % \begin{subfigure}{\textwidth}%
    %     \centering
    %     \includegraphics[width=\columnwidth]{plots/QF_LRA/planning/all/legend.pdf}%
    % \end{subfigure}
    % %%%%%%%%%%%%%%% TIME %%%%%%%%%%%%%%%
    % \begin{subfigure}{0.33\textwidth}%
    %     \centering
    %     \includegraphics[width=\columnwidth]{plots/QF_LRA/planning/all/time/time_compare_omsat-min-none_vs_omsat-min-simple__-__clause.pdf}%
    %     %\caption{time bound}%
    %     %\label{fig:plot:planning:time:none-simple}%
    % \end{subfigure}%
    % \begin{subfigure}{0.33\textwidth}%
    %     \centering
    %     \includegraphics[width=\columnwidth]{plots/QF_LRA/planning/all/time/time_compare_omsat-min-none_vs_omsat-min-guided_drop_iter__-__clause.pdf}%
    %     %\caption{time bound}%
    %     %\label{fig:plot:planning:time:none-guided}%        
    % \end{subfigure}%
    % \begin{subfigure}{0.33\textwidth}%
    %     \centering
    %     \includegraphics[width=\columnwidth]{plots/QF_LRA/planning/all/time/time_compare_omsat-min-simple__-__clause_vs_omsat-min-guided_drop_iter__-__clause.pdf}
    %     %\caption{time bound}%
    %     %\label{fig:plot:planning:time:simple-guided}%        
    % \end{subfigure}
    % %%%%%%%%%%%%%%% UPPER BOUNDS %%%%%%%%%%%%%%%
    % \begin{subfigure}{0.33\textwidth}%
    %     \centering
    %     \includegraphics[width=\columnwidth]{plots/QF_LRA/planning/all/upper/upper_compare_omsat-min-none_vs_omsat-min-simple__-__clause.pdf}%
    %     %\caption{Upper bound}%
    %     %\label{fig:plot:planning:upper:none-simple}%
    % \end{subfigure}%
    % \begin{subfigure}{0.33\textwidth}%
    %     \centering
    %     \includegraphics[width=\columnwidth]{plots/QF_LRA/planning/all/upper/upper_compare_omsat-min-none_vs_omsat-min-guided_drop_iter__-__clause.pdf}%
    %     %\caption{Upper bound}%
    %     %\label{fig:plot:planning:upper:none-guided}%        
    % \end{subfigure}%
    % \begin{subfigure}{0.33\textwidth}%
    %     \centering
    %     \includegraphics[width=\columnwidth]{plots/QF_LRA/planning/all/upper/upper_compare_omsat-min-simple__-__clause_vs_omsat-min-guided_drop_iter__-__clause.pdf}
    %     %\caption{Upper bound}%
    %     %\label{fig:plot:planning:upper:simple-guided}%        
    % \end{subfigure}
    % %%%%%%%%%%%%%%% ITERATIONS %%%%%%%%%%%%%%%
    % \begin{subfigure}{0.33\textwidth}%
    %     \centering
    %     \includegraphics[width=\columnwidth]{plots/QF_LRA/planning/all/sat_search_steps/sat_search_steps_compare_omsat-min-none_vs_omsat-min-simple__-__clause.pdf}%
    %     %\caption{sat_search_steps bound}%
    %     %\label{fig:plot:planning:sat_search_steps:none-simple}%
    % \end{subfigure}%
    % \begin{subfigure}{0.33\textwidth}%
    %     \centering
    %     \includegraphics[width=\columnwidth]{plots/QF_LRA/planning/all/sat_search_steps/sat_search_steps_compare_omsat-min-none_vs_omsat-min-guided_drop_iter__-__clause.pdf}%
    %     %\caption{sat_search_steps bound}%
    %     %\label{fig:plot:planning:sat_search_steps:none-guided}%        
    % \end{subfigure}%
    % \begin{subfigure}{0.33\textwidth}%
    %     \centering
    %     \includegraphics[width=\columnwidth]{plots/QF_LRA/planning/all/sat_search_steps/sat_search_steps_compare_omsat-min-simple__-__clause_vs_omsat-min-guided_drop_iter__-__clause.pdf}
    %     %\caption{sat_search_steps bound}%
    %     %\label{fig:plot:planning:sat_search_steps:simple-guided}%        
    % \end{subfigure}%
    \caption{
        Results on OMT-encoded optimal temporal planning problems.
        %The first row compares the solving time for the different truth-assignment-reduction strategies, where the points on the dashed line represent the timeouts.
        %The second row compares the upper bounds found within the timeout.
        %The third row compares the number of iterations performed to reach the upper bound.
    }%
    \label{fig:plot:planning}%
\end{figure}

\begin{figure}[t]
    %%%%%%%%%%%%%%%%%%%%%%%%%%%%%%%%%%%%%%%%%%%%%%%%%%%%%%%
    %%%%%%%%%%%%%% STRIP-PACKING %%%%%%%%%%%%%%%%%%%%%%%%%%
    %%%%%%%%%%%%%%%%%%%%%%%%%%%%%%%%%%%%%%%%%%%%%%%%%%%%%%%
    \begin{tabularx}{\textwidth}{cc|c}
        \multicolumn{3}{c}{
            \includegraphics[width=.6\columnwidth]{plots/QF_LRA/lgdp/sp/all/legend.pdf}
        }                                                                                                                                                                         \\
        %%%%%%%%%%%%%%% UPPER BOUNDS %%%%%%%%%%%%%%%
        \includegraphics[width=.33\columnwidth]{plots/QF_LRA/lgdp/sp/all/upper/upper_compare_omsat-min-none_vs_omsat-min-simple__-__clause.pdf}                                 &
        \includegraphics[width=.33\columnwidth]{plots/QF_LRA/lgdp/sp/all/upper/upper_compare_omsat-min-none_vs_omsat-min-guided_drop_iter__-__clause.pdf}                       &
        \includegraphics[width=.33\columnwidth]{plots/QF_LRA/lgdp/sp/all/upper/upper_compare_omsat-min-simple__-__clause_vs_omsat-min-guided_drop_iter__-__clause.pdf}            \\
        %%%%%%%%%%%%%%% ITERATIONS %%%%%%%%%%%%%%%
        \includegraphics[width=.33\columnwidth]{plots/QF_LRA/lgdp/sp/all/sat_search_steps/sat_search_steps_compare_omsat-min-none_vs_omsat-min-simple__-__clause.pdf}           &
        \includegraphics[width=.33\columnwidth]{plots/QF_LRA/lgdp/sp/all/sat_search_steps/sat_search_steps_compare_omsat-min-none_vs_omsat-min-guided_drop_iter__-__clause.pdf} &
        \includegraphics[width=.33\columnwidth]{plots/QF_LRA/lgdp/sp/all/sat_search_steps/sat_search_steps_compare_omsat-min-simple__-__clause_vs_omsat-min-guided_drop_iter__-__clause.pdf}
    \end{tabularx}
    % \centering
    % \begin{subfigure}{\textwidth}%
    %     \centering
    %     \includegraphics[width=.5\columnwidth]{plots/QF_LRA/lgdp/sp/all/legend.pdf}%
    % \end{subfigure}
    % \begin{subfigure}{0.33\textwidth}%
    %     \centering
    %     \includegraphics[width=\columnwidth]{plots/QF_LRA/lgdp/sp/all/upper/upper_compare_omsat-min-none_vs_omsat-min-simple__-__clause.pdf}%
    %     %\caption{Upper bound}%
    %     %\label{fig:plot:sp:upper:none-simple}%
    % \end{subfigure}%
    % \begin{subfigure}{0.33\textwidth}%
    %     \centering
    %     \includegraphics[width=\columnwidth]{plots/QF_LRA/lgdp/sp/all/upper/upper_compare_omsat-min-none_vs_omsat-min-guided_drop_iter__-__clause.pdf}%
    %     %\caption{Upper bound}%
    %     %\label{fig:plot:sp:upper:none-guided}%        
    % \end{subfigure}%
    % \begin{subfigure}{0.33\textwidth}%
    %     \centering
    %     \includegraphics[width=\columnwidth]{plots/QF_LRA/lgdp/sp/all/upper/upper_compare_omsat-min-simple__-__clause_vs_omsat-min-guided_drop_iter__-__clause.pdf}
    %     %\caption{Upper bound}%
    %     %\label{fig:plot:sp:upper:simple-guided}%        
    % \end{subfigure}
    % %%%%%%%%%%%%%%% ITERATIONS %%%%%%%%%%%%%%%
    % \begin{subfigure}{0.33\textwidth}%
    %     \centering
    %     \includegraphics[width=\columnwidth]{plots/QF_LRA/lgdp/sp/all/sat_search_steps/sat_search_steps_compare_omsat-min-none_vs_omsat-min-simple__-__clause.pdf}%
    %     %\caption{sat_search_steps bound}%
    %     %\label{fig:plot:sp:sat_search_steps:none-simple}%
    % \end{subfigure}%
    % \begin{subfigure}{0.33\textwidth}%
    %     \centering
    %     \includegraphics[width=\columnwidth]{plots/QF_LRA/lgdp/sp/all/sat_search_steps/sat_search_steps_compare_omsat-min-none_vs_omsat-min-guided_drop_iter__-__clause.pdf}%
    %     %\caption{sat_search_steps bound}%
    %     %\label{fig:plot:sp:sat_search_steps:none-guided}%        
    % \end{subfigure}%
    % \begin{subfigure}{0.33\textwidth}%
    %     \centering
    %     \includegraphics[width=\columnwidth]{plots/QF_LRA/lgdp/sp/all/sat_search_steps/sat_search_steps_compare_omsat-min-simple__-__clause_vs_omsat-min-guided_drop_iter__-__clause.pdf}
    %     %\caption{sat_search_steps bound}%
    %     %\label{fig:plot:sp:sat_search_steps:simple-guided}%        
    % \end{subfigure}%
    \caption{Results on OMT-encoded strip-packing problems.
        % The time row~\ref{cap:item:time} is omitted since all instances timed out.
        % The rows compare them in terms of
        % \begin{enumerate*}[label=(\roman*)]
        %     \item\label{cap:item:time} solving time,
        %     \item\label{cap:item:upper} best upper bound found within the timeout, and
        %     \item\label{cap:item:iter} number of iterations to reach the upper bound.
        % \end{enumerate*}
    }%
    \label{fig:plot:sp}%
\end{figure}

\Cref{fig:plot:planning,fig:plot:sp} show the results on temporal planning and strip-packing benchmarks, respectively. For each
benchmark set, we report a set of scatter plots.

On the rows, we have different metrics, namely the solving time in seconds
(time(s)), the upper bound (u.b.) ---i.e., the optimum value
when the solver terminated within the time limit, or the value of the best
solution found within the timeout otherwise--- and the number of iterations (\#
iter) taken to reach the upper bound (see \Cref{alg:omt-partial}).

On the columns, we compare the results obtained with the different
truth-assignment-reduction strategies: in the left and center columns, we
respectively compare the basic and the guided reductions with the plain
algorithm without reductions. In the right column, we compare the two reduction
strategies.

% The plots in~\Cref{fig:plots} show the results of the experiments on temporal
% planning (\Cref{fig:plot:planning}) and strip-packing (\Cref{fig:plot:sp})
% benchmarks. Each figure includes a set of Comparison of the
% performance of \optimathsat{} with the different truth-assignment-reduction
% strategies. On the rows, the plots compare the strategies in terms of solving
% time in seconds (this row is omitted for strip-packing since all instances
% timed out), best upper bound (u.b.) found within the timeout, and number of
% iterations (\# iter) to reach the upper bound.

\paragraph{Optimal Temporal Planning (\Cref{fig:plot:planning}).}

In these benchmarks, with no truth-assignment reduction, \optimathsat{}
reported 246 timeouts, 211 with the basic reduction, and 212 with the guided
reduction.

From the plots (first row, left and center columns), we can see that applying
either reduction almost uniformly improves the solving time with few
exceptions, making optimal solving up to twice as fast as with no reduction.

Moreover, we observe that reducing truth assignments is very effective also for
anytime solving (second row, left and center columns). Notice that when the
solver terminated within the timeout with both strategies, then the
corresponding points lie on the bisector, whereas when at least one strategy
times out, the points are generally below the bisector.
%
Indeed, this shows that, for anytime solving, both the basic and the guided
reductions allow finding a much better upper bound than with no reduction. % by applying either 

Finally, we can see that both strategies are particularly effective in reducing
the number of iterations needed to either find the optimum or to reach the best
upper bound within the timeout (third row, left and center). Reducing the
number of iterations is not an advantage in itself, but it is a good indicator
of the effectiveness of truth-assignment reduction strategies in OMT.

Overall, in these benchmarks
there is no clear winner between the two reduction strategies (right column), but it is
evident that applying either form of truth-assignment reduction can be
beneficial in OMT, both for optimal and anytime solving.

\paragraph{Strip-packing (\Cref{fig:plot:sp}).}
Since no instance in this set of benchmarks terminated within the timeout, for these benchmarks we omit the time plots.
We can see that here the basic reduction strategy is not really effective, since the value of the upper bound is not improved compared to the no-reduction strategy (first row, left column). Also, the number of iterations only slightly decreases (second row, left column), suggesting that here blindly removing atoms from the truth assignment does not help much in finding better solutions.
On the other hand, the guided reduction strategy is much more effective, since it allows finding a much better upper bound within the timeout (first row, center and right columns),
and the number of iterations is drastically reduced (second row, center and right columns).
% Nevertheless, this does not seem to improve the solving time, not even on smaller instances.

\paragraph{Discussion.}
The results show that applying either form of truth-assignment reduction can be
beneficial in OMT, both for optimal and anytime solving. Also, accurately
selecting which atoms to remove from the truth assignment can make a
significant difference in finding better solutions in fewer iterations.
%
However, we can observe that a much smaller number of iterations, i.e.\ of
truth assignments enumerated, does not always correlate linearly with the solving time. This can be due to several reasons.%, differently from what happens e.g.\ for AllSMT. 

First, we notice that in these problems the number of truth assignments
enumerated is typically contained, up to a few hundred. %, whereas in AllSMT it can easily
%reach hundreds of thousands and more.
In fact, in OMT the bounds on the objective function already allow performing a
very effective pruning of the search space. %(In AllSMT, instead, a blocking clause only eliminates a single truth assignment).

Moreover, this pruning is typically done by theory reasoning, and most of it
has to be done anyway, regardless of the number of truth assignments enumerated.
Making it in a single iteration or in many iterations may not reflect as much
on the solving time, because of the efficient incrementality of SMT solvers, which can
reduce a lot the cost of consecutive iterations.
%Also, most of the time is spent on theory reasoning to rule out the truth assignments that don't satisfy the bound, 
%On the other hand, imposing a tighter bound on the objective function can cause each iteration to be more expensive, since a lot of theory reasoning is still necessary to prune the truth assignments that don't satisfy the bound.
%
Nevertheless, this suggests that exploiting partial truth assignments in OMT
problems with theories where incrementality is not as effective as in \larat{},
as is the case, e.g., of \laint{}, can have a larger impact also on the solving
time.