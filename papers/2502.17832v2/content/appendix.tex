\appendix
\onecolumn

% \begin{center}{\bf {\LARGE Supplementary Materials}}\end{center}
% \begin{center}{\bf {\Large PARM: Poisoning Attacks on RAG-Equipped MLLMs}}

\section{Experimental Setup}
% Did you report the number of parameters in the models used, the total computational budget (e.g., GPU hours), and computing infrastructure used?

% If you used existing packages (e.g., for preprocessing, for normalization, or for evaluation, such as NLTK, SpaCy, ROUGE, etc.), did you report the implementation, model, and parameter settings used?
\subsection{Implementation Details}
\label{appendix:implementation_details}
We evaluated the MLLM RAG system on an NVIDIA H100 GPU, allocating no more than 20 minutes per setting on the WebQA dataset (1,261 test cases). When training adversarial images against the retriever, reranker, and generator, we used a single NVIDIA H100 GPU for each model, and up to three GPUs when training against all three components in GPA-RtRrGen.

For the retriever, we used the average embedding of all queries and optimized the image to maximize similarity. Due to memory constraints, we adopted a batch size of 1 for both the reranker and generator. 
% We performed a grid search for hyperparameter selection with learning rate $\alpha \in \{0.005, 0.01\}$, retriever weight  $\lambda_1 \in \{0.1, 0.2, 0.3\}$, and reranker weight $\lambda_2 \in {}$.
The hyperparameters used in each setting are listed in Table~\ref{tab:hyper_parameters}. Each setting requires up to an hour of training.

We list the exact models used in our experiments in Table~\ref{tab:model_details}.









\begin{table*}[h]
    \centering
    %\small % Apply small font size to the entire table
    \begin{tabular}{c c c c c| c c c c }
    \toprule
        \multicolumn{5}{c}{Expriment Settings} & $\alpha$ & $\lambda_1$ &$\lambda_2$ & \# Training Steps \\
        Attack & Rt. & Rr. & Gen. & Task\\
        \midrule
        LPA-Rt & CLIP & - & - & MMQA&0.005&-&-&50 \\
        LPA-Rt & CLIP & - & - & WebQA&0.005&-&-&50 \\
        GPA-Rt & CLIP & - & - &MMQA&0.01&-&-& 500\\
        GPA-Rt & CLIP & - & - &WebQA&0.01&-&-&500 \\
        GPA-RtRrGen& CLIP& Llava &Llava &MMQA & 0.01&0.2&0.3&2000\\
                GPA-RtRrGen& CLIP& Qwen &Qwen &MMQA & 0.005&0.2&0.3&2500 \\
                GPA-RtRrGen& CLIP& Llava &Qwen &MMQA & 0.01&0.08&0.9&2500\\
        GPA-RtRrGen& CLIP& Llava &Llava &WebQA & 0.01&0.2&0.3&2000\\
                GPA-RtRrGen& CLIP& Qwen &Qwen &WebQA & 0.01&0.3&0.3&1000\\
                GPA-RtRrGen& CLIP& Llava &Qwen &WebQA & 0.01&0.1&0.8&3000\\
         \bottomrule
    \end{tabular}%
    \caption{Hyper-parameters for training adversarial images.}
    \vspace{-0.1in}
    \label{tab:hyper_parameters}
\end{table*}

To illustrate equilibria and dynamics of performative prediction games, we focus on a scenario in which a \emph{duopoly} of mortgage companies, i.e. banks, compete to sell loans to customers.

\paragraph{Customer Model:} In our game, each bank is trying to attract customers from a given population $\mathcal{P}$. We model this population as comprised of individuals with a single-dimensional type: we denote individual $j$'s type as $y_j \in [0,1]$. For simplicity, we assume that \(y\) represents the customer’s probability of repaying the loan\footnote{In practice, a customer's (normalized) credit score can be interpreted as a noisy observation of $y_j$. This also corresponds to credit scores being \emph{calibrated}.}, i.e., $y_j := \P[Y_j = 1]$, where $Y_j$ is a random variable such that $Y_j = 0$ means that $j$ defaults on their loan, and $Y_j = 1$ means they repay their loan. Customer types in the population are drawn from a known distribution $D_y$ supported on $[0,1]$. 

\paragraph{Game between Banks:} Each Bank \(i \in \{1, 2\}\) selects two parameters \( (\tau_i, \gamma_i) := \theta_i\), where:
\begin{itemize}
    \item \(\tau_i \in \{\tau_l,\tau_h\}\) is the credit score threshold for approving a customer\footnote{We restrict the bank to only pick between two thresholds, $\tau_l$ and $\tau_h$. However, we highlight how our results are affected when we expand the strategy space to $n > 2$ actions in our experiments of Appendix \ref{app:3gamma}.}. Specifically, a customer $j$ with credit score \(y_j\) is approved by Bank $i$ if and only if \(y_j \geq \tau_i\);
    \item \(\gamma_i \in \{\gamma_l, \gamma_h\}\) is the interest rate offered to approved customers.
\end{itemize}
We denote as shorthand the space of allowable thresholds by $\Gamma := [0,1]$ and allowable interests rates by $\Lambda := [0,1]$. %The latter is set without loss of generality---we simply normalize the rates to be at most $1$. 
% {\color{red} Vidya: just thinking about this but is it natural to restrict interest rate to $1$? I don't think it would affect the equilibrium structure of the game but theoretically I think the interest rate could be anything in $[0,\infty)$.} {\color{green} Guanghui: Could we say something like this is without loss of generality} \gua{changed.}\juba{I think we repeated this twice, the next sentence already had this}
The loan amount is normalized to $1$ in the entire paper, without loss of generality; in this case, if a customer chooses Bank $i$, and the customer is approved by the bank at an interest rate of $\gamma_i$, the expected utility for the bank is equal to
\[
(1+\gamma_i)\cdot \P[Y_i = 1]-\P[Y_i = 0] = (1+\gamma_i)y_i-(1-y_i).
\]


%In practice, the credit score \(y\) serves as a noisy observation of the true likelihood of the customer's repayment. 

\paragraph{Banks' Utilities:} For given parameter choices \(\theta_1 = (\tau_1, \gamma_1)\) by Bank 1 and \(\theta_2 = (\tau_2, \gamma_2)\) by Bank 2, a \emph{rational} customer with credit score $y$ acts as follows:

\begin{enumerate}
    \item \textbf{Qualified for a single bank}: 
        \begin{itemize}
        \item If \(\tau_1 \leq y < \tau_2\), the customer goes to Bank 1, as the score qualifies for Bank 1 but not Bank 2. Conversely, if \(\tau_2 \leq y < \tau_1\), the customer chooses Bank 2.
    \end{itemize}
    \item \textbf{Qualified for both banks}:
     \begin{itemize}
        \item If \(\tau_1, \tau_2 \leq y\) and \(\gamma_1 < \gamma_2\), the customer selects Bank 1 for its lower interest rate. Conversely, if \(\gamma_1 > \gamma_2\), the customer chooses Bank 2.
        \item If \(\gamma_1 = \gamma_2\), the customer picks each bank with probability $1/2$. 
    \end{itemize}
    \item \textbf{Unqualified for both banks}:
    \begin{itemize}
        \item If \(y < \tau_1\) and \(y < \tau_2\), the customer is rejected by both banks.
    \end{itemize}
\end{enumerate}

The expected reward for Bank 1, denoted as \(u_1(\theta_1, \theta_2)\), can then be expressed as:
\begin{align}\label{eq:utility}
    u_1(\theta_1, \theta_2) 
    &=  \mathbb{E}_{y \sim D_y} \left[ \mathbb{I}\{\underbrace{\tau_1 \leq y < \tau_2 \ \cup \ (\tau_1, \tau_2 \leq y \ \cap \ \gamma_1 < \gamma_2)}_{\text{accepted by Bank 1}}\} \cdot \big((1+\gamma_1)y - (1-y)\big) \right] \nonumber\\
    & + \frac{1}{2} \mathbb{E}_{y \sim D_y} \left[ \mathbb{I}\{\underbrace{\tau_1, \tau_2 \leq y \ \cap \ \gamma_1 = \gamma_2}_{\text{accepted by both Banks}}\} \cdot \big((1+\gamma_1)y - (1-y)\big) \right].
\end{align}
Note that the problem is \emph{symmetric}, i.e., the utility function for Bank 2 can be derived by swapping the roles of \(\theta_1\) and \(\theta_2\). I.e., $u_2(\theta_1, \theta_2) = u_1(\theta_2, \theta_1)$. 

% If a bank only attracts customers between thresholds $\tau_a$ and $\tau_b$, for $\tau_a<\tau_b$, we call $[\tau_a,\tau_b]$ the \emph{threshold} range for that bank. For example, if Bank $1$ sets a threshold of $\tau_1$, Bank $2$ a threshold of $\tau_2 > \tau_1$, and $\gamma_1 > \gamma_2$, then Bank 1 has a threshold range of $[\tau_1,\tau_2]$, while bank $2$ has a threshold range of $[\tau_2,1]$.
% Note that the parameters set by \emph{both} banks, i.e. $(\theta_1,\theta_2)$ both influence the threshold range for each of Bank 1 and 2.  If $\tau_1>\tau_2$, $\gamma_1>\gamma_2$, then $\tau_a>\tau_b$, and the bank does not attract any customers. 
% {\color{red} is it possible for $\tau_a > \tau_b$, leading to the bank never attracting customers?} \gua{if $\gamma_1>\gamma_2$, $\tau_1>\tau_2$, then it gets no customer. I think it also makes sense.}\juba{I think we said we wanted to delete the discussion of the threshold range, no?}

% \noindent \textbf{Discrete Model}   
% We now present the discrete version of our model, where the interest rates and thresholds are selected from finite sets \(\Gamma\) and \(\Lambda\), respectively, with $\tau\in[0,1], \gamma\in[0,1]$,  for all $\tau\in\Lambda$ and $\gamma\in\Gamma$, \(|\Gamma| = n\) and \(|\Lambda| = m\). Let \(p_1, p_2 \in \Delta(\Gamma \times \Lambda)\) represent the mixed strategies of the two banks, where \(\Delta(\Gamma \times \Lambda)\) denotes the set of probability distributions over the discrete decision space \(\Gamma \times \Lambda\).


% \begin{Remark}
%    Note that our proposed problem can be reformulated as a standard multi-player performative prediction problem \citep{narang2023multiplayer}. However, in our problem, the data distribution faced by each learner breaks the Lipschitzness assumption of previous work~\citep{hardt2023performative,narang2023multiplayer}. A small modification in one of the learner's thresholds can completely change how demand is allocated across both learners, as is often the case in Bertrand-style games. 
% \end{Remark} 

% \gua{I made some changes to Remark 1, please have a look}
\begin{Remark}
   Previous works in multi-learner performative prediction~\citep{narang2023multiplayer} resort to an insensitivity assumption, i.e., the data distribution faced by each player can only changes slightly when the parameters also change slightly; formally, the data distribution faced by each player is Lipschitz in their decisions. This is immediately not true in our setting: the bank slightly changing its parameters can completely changes the demand distribution of customers it faces. Intuitively, this is because of Bertrand-competition-style effects, where if two banks have similar rates, one bank that lowers their rate by a small amount suddenly captures the entire customer demand that is eligible for that rate.%\juba{made further light edits adding intuition}
   
   In Appendix \ref{Appendix:refumulation}, we discuss this problem more carefully by reformulating our problem in the standard multi-learner performative prediction form given by~\citep{narang2023multiplayer}. We show the distribution is not Lipschitz with respect to the parameters, and thus does not satisfy the insensitivity assumption. 
%Prior work~\citep{hardt2023performative,narang2023multiplayer} showed that, for a general multi-agent performative prediction framework to work, insensitivity assumptions are needed: in the \textbf{worst case}, they can construct settings where the insensitivity assumption does not hold and simple dynamics do not converge anymore. We add nuance to this picture. We will show that our dynamics often converge, even absent insensitivity assumptions, highlighting that while the impossibility results of previous work hold in the worst case, they may not hold in the ``average case'' and especially not in problems motivated by applications. In particular, we will show convergence to a variety of equilibria of our game, and often to symmetric Nash equilibria where insensitivity is immediately violated.
     
\end{Remark}



% \paragraph{Relationship to Performative Prediction} A central point of our work is to highlight that \textcolor{red}{needs writing from intro}. We highlight how our work specifically ties to ``Performative Prediction'' below:


%\textcolor{red}{needs a definition environment}



%Here, \(\E_{\theta_1, \theta_2}\) represents the expected utility of the banks over their respective strategies \((\theta_1, \theta_2)\). These inequalities ensure that neither bank can unilaterally improve its expected utility by deviating from its mixed strategy in the equilibrium.



%and  for all $\tau\in\Gamma$, we have $\tau\in\Lambda$, $(\tau,\gamma)\in[0,1]^2$. Let $\Gamma\times\Lambda$
%In this paper, we focus on the most fundamental case, where there are two choices for each parameter: $0\leq\tau_{\ell}<\tau_{h}\leq 1$, and $0\leq \gamma_{\ell}< \gamma_{h}\leq 1$. In this case, the utility for each pair of decisions forms a $4\times4$ matrix (given in Table \ref{tab:my-table}). We consider the canonical case where $\tau_{\ell}=\frac{1}{2+\gamma_{h}}$, and $\tau_{h}=\frac{1}{2+\gamma_{\ell}}.$ Note that these are natural choices for the thresholds, in the sense that, if there is only one bank and the interest rate is set to be $\gamma$, then $\frac{1}{2+\gamma}$ is the optimal threshold corresponding to the fixed $\gamma$.


%and the thresholds are chosen in $\Lambda=\{\tau^{(1)},\dots,\tau^{(m)}\}$. Here, we only assume that, for each $\gamma\in\Gamma$, there at least exist one $\tau\in\Lambda$ such that $f(\gamma,\tau,1)>0$. Note that this is a very minor assumption, in the sense that, if for a $\gamma$ such that $f(\gamma,\tau,1)<0$ for all $\tau\in\Lambda$, then adopting this decision will lead to negative utility regardless of the opponent's decision, and thus is not an interesting case. 

%\textcolor{red}{The model section is missing the dynamic version of the game. We should clearly define the one-shot and the dynamic game}
% we only considered one-shot case in our paper




\subsection{Query Selection}
\label{appendix:query_selection}
The first step of our attack consists of selecting appropriate queries to attack. 
In order to evaluate our attack fairly, we filter out queries (i.e., text questions) that can be answered correctly without providing their associated multimodal context, since even if our attack against the retriever is successful, the MLLM could ignore the retrieved context.
To do this filtering, we employ LLaVA~\citep{liu2024llavanext} and Qwen-VL-Chat~\citep{bai2023qwen} asking them to provide an answer for each question in the datasets without providing the relevant context. Once this is done, we keep the queries where both models failed to provide the correct answer. This process results in 125 queries out of 229 being selected for MMQA and 1261 out of 2511 for WebQA.

\subsection{LPA-BB Data Generation}
\paragraph{Prompts}
\label{appendix:prompt}
In our poisoned data generation pipeline, we generate a wrong target answer and corresponding poisoned image caption for a given query. We specifically employ GPT-4o prompting it to generate both a wrong target answer and a poisoned image caption, using the following prompt:

\begin{tcolorbox}[colback=gray!10, colframe=black, title=Prompt for Poisoned Image Caption Generation]
Our objective is to create poisoned data for a multi-modal QA system to evaluate its robustness. For each question and its correct answer, please complete the following tasks:
\begin{enumerate}
    \item Create an incorrect answer that differs from the correct one.
    \item Craft a misleading image caption, which will be used to generate a poison image further. This poison image, when used as context for the question, will lead the system to generate the incorrect answer. Additionally, ensure the image will be retrieved based on the question's context. For example, if the question pertains to a movie cover, the poisoned image should also represent a movie cover, including essential details like the title.
\end{enumerate}
The provided question and correct answer are as follows:\\
\textbf{Question:} \{\{ question \}\} \\
\textbf{Correct answer:} \{\{ correct\_answer \}\}

Please format your response as a JSON object, structured as follows:

\begin{verbatim}
{
  "wrong_answer": "...",
  "poison_image_caption": "..."
}
\end{verbatim}
\end{tcolorbox}

Then, to generate the poisoned images, we use Stable Diffusion~\citep{rombach2022high} conditioned on the poisoned image captions generated by GPT-4o. Specifically, we employ the \texttt{stabilityai/stable-diffusion-3.5-large} model from Hugging Face, with the classifier free guidance parameter set to $3.5$ and the number of denoising steps set to $28$.

\section{Additional Experimental Results}
\subsection{Localized and Globalized Poisoning Attack Results on other MLLMs.}
\label{sec:other_mllm}
In addition to the results in the main paper, which use the same MLLMs for the reranker and generator, we further evaluate our attacks when different LLMs are used. Specifically, we consider a heterogeneous setting where Llava is used for the reranker and Qwen for the generator, with results shown in Table~\ref{tab:mmqa_lpa_hetero}. We observe that our attack is less effective in this setting, likely because the differing embedding spaces of the reranker and generator increase the optimization challenge.
\begin{table*}[t]
    \centering
    \resizebox{\textwidth}{!}{%
    \begin{tabular}{@{}llc cc cc cc cc@{}}
       \toprule
       & & & \multicolumn{4}{c}{MMQA \footnotesize (m=1)} & \multicolumn{4}{c}{WebQA  \footnotesize (m=2)} \\
        \cmidrule(lr){4-7} \cmidrule(lr){8-11}
        & & & \multicolumn{2}{c}{\textbf{$\text{R}_{\text{Orig.}}$} (\%)} & \multicolumn{2}{c}{\textbf{$\text{ACC}_{\text{Orig.}}$} (\%)} &  \multicolumn{2}{c}{\textbf{$\text{R}_{\text{Orig.}}$} (\%)} & \multicolumn{2}{c}{\textbf{$\text{ACC}_{\text{Orig.}}$} (\%)}\\
        
        \textbf{Rt.} & \textbf{Rr.} & \textbf{Capt.} & Before & After & Before & After & Before & After & Before & After \\
        \midrule
        \multicolumn{11}{c}{\textbf{[LPA-BB] Retriever (Rt.)}: CLIP-ViT-L \textbf{Reranker (Rr.)}: LLaVA \textbf{Generator}: Qwen-VL-Chat} \\
        \midrule
        $N=5$   & $K=m$          & \xmark      & 64.8 & 40.8 {\footnotesize \textcolor{red}{-24.0}}  & 46.4 & 34.4 {\footnotesize \textcolor{red}{-12.0}}   &
        58.2 & 48.5 {\footnotesize \textcolor{red}{-9.7\hphantom{0}}} & 20.9 & 19.8 {\footnotesize \textcolor{red}{-1.0\hphantom{0}}} \\
        $N=5$   & $K=m$          & \cmark      & 81.6& 37.6 {\footnotesize \textcolor{red}{-44.0}}  & 52.0 & 33.6 {\footnotesize \textcolor{red}{-18.4}}  &
        % WebQA
        65.0 & 54.7 {\footnotesize \textcolor{red}{-10.3}} & 27.7 & 26.4 {\footnotesize \textcolor{red}{-1.3\hphantom{0}}}   \\
        \midrule
        \multicolumn{11}{c}{\textbf{[LPA-Rt] Retriever (Rt.)}: CLIP-ViT-L \textbf{Reranker (Rr.)}: LLaVA \textbf{Generator}: Qwen-VL-Chat} \\
        \midrule
        $N=5$   & $K=m$          & \xmark      & 64.8  &28.0 {\footnotesize \textcolor{red}{-36.8}}& 46.4 & 24.0 {\footnotesize \textcolor{red}{-21.6}}  &
        % WebQA
         58.2 & 23.1 {\footnotesize \textcolor{red}{-25.1}} & 20.9 &17.7 {\footnotesize \textcolor{red}{-3.2\hphantom{0}}} \\
        $N=5$   & $K=m$          & \cmark      & 81.6 & 23.2 {\footnotesize \textcolor{red}{-58.4}}  & 52.0 & 20.8 {\footnotesize \textcolor{red}{-31.2}}  &
        % WebQA
         65.0 & 27.7 {\footnotesize \textcolor{red}{-37.3}} & 22.7 & 17.9 {\footnotesize \textcolor{red}{-4.8\hphantom{0}}} \\

         \midrule
         \multicolumn{11}{c}{\textbf{[GPA-Rt] Retriever}: CLIP-ViT-L \textbf{Reranker}: LLaVA \textbf{Generator}: Qwen-VL-Chat} \\
        
        \midrule
       
        $N=5$   & $K=m$          & \xmark      & 66.4 & \hphantom{0}1.6 {\footnotesize \textcolor{red}{-64.8}}             & 49.6  & \hphantom{0}8.8 {\footnotesize \textcolor{red}{-40.8}}     &
        %webqa
        58.2 & \hphantom{0}0.0 {\footnotesize \textcolor{red}{-58.2}} &  20.9 & 14.6 {\footnotesize \textcolor{red}{-6.3\hphantom{0}}}                      \\ 
        $N=5$   & $K=m$          & \cmark      &  81.6 & \hphantom{0}1.6 {\footnotesize \textcolor{red}{-80.0}}                  & 51.2 & \hphantom{0}8.8   {\footnotesize \textcolor{red}{-42.4}}     &
        %webqa
         69.8 & \hphantom{0}0.0 {\footnotesize \textcolor{red}{-69.8}} & 21.7  & 14.6 {\footnotesize \textcolor{red}{-7.1\hphantom{0}}}                    \\ 
        \midrule
        \multicolumn{11}{c}{\textbf{[GPA-RtRrGen] Retriever}: CLIP-ViT-L \textbf{Reranker}: LLaVA \textbf{Generator}: Qwen-VL-Chat} \\
        \midrule
        % naive jailbreak caption + Rtrr_V1
        $N=5$   & $K=m$          & \xmark      & 66.4  &  60.0 {\footnotesize \textcolor{red}{-6.4\hphantom{0}}}             & 49.6 &  47.2 {\footnotesize \textcolor{red}{-2.4\hphantom{0}}} & 58.2 & 53.6 {\footnotesize \textcolor{red}{-4.6\hphantom{0}}} &  20.9 &  11.0 {\footnotesize \textcolor{red}{-9.9\hphantom{0}}}                                         \\     
        $N=5$   & $K=m$          & \cmark      & 81.6 &  72.0 {\footnotesize \textcolor{red}{-9.6\hphantom{0}}}                  & 51.2 &  46.4 {\footnotesize \textcolor{red}{-4.8\hphantom{0}}}     & 69.8 & 60.3 {\footnotesize \textcolor{red}{-9.5\hphantom{0}}} & 21.7  & \hphantom{0}5.8 {\footnotesize \textcolor{red}{-18.9}}                            \\
        \bottomrule
    \end{tabular}%
    }
    \caption{\textbf{Localized poisoning attack results on MMQA and WebQA tasks} when reranker and generator employ different MLLMs. Capt. stands for caption. $\text{R}_{\text{Orig.}}$ and $\text{ACC}_{\text{Orig.}}$ represent retrieval recall (\%) and accuracy (\%) for the original context and answer after poisoning attacks, where the numbers highlighted in \textcolor{red}{red} shows the drop in performance compared to those before poisoning attacks. $\text{R}_{\text{Pois.}}$ and $\text{ACC}_{\text{Pois.}}$ indicate performance for the poisoned context and attacker-controlled answer, reflecting attack success rate.} % todo: lpa-rt webqa
    \label{tab:mmqa_lpa_hetero}
\end{table*}



% \begin{table*}[t]
%     \caption{Transferability of localized poisoning attack results. LPA-Rt generates poisoned knowledge against the CLIP retriever, and trasnfers it to RAG framework with OpenCLIP retriever.}
%     \label{tab:mmqa_transfer}
%     \centering
%     \resizebox{\textwidth}{!}{%
%     \begin{tabular}{@{}llc cc cc cc cc@{}}
%        \toprule
%          \multicolumn{11}{c}{\textbf{[LPA-BB]} \textbf{Rt.}: CLIP-ViT-L $\Rightarrow$ OpenCLIP-ViT-L \textbf{Rr.}: LLaVA \textbf{Gen.}: LLaVA} \\
%            \midrule
%          \textbf{Rt.} & \textbf{Rr.} & \textbf{Capt.} & \multicolumn{4}{c}{MMQA} & \multicolumn{4}{c}{WebQA} \\
%         & & & \multicolumn{2}{c}{\textbf{R (\%)}} & \multicolumn{2}{c}{\textbf{ACC (\%)}} &  \multicolumn{2}{c}{\textbf{R (\%)}} & \multicolumn{2}{c}{\textbf{ACC (\%)}}\\ 
%         &  &  & Orig. & Poisoned  & Orig. & Poisoned & Orig. & Poisoned & Orig. & Poisoned \\
%         \midrule
%          $N=n$   & \xmark                      & -           & 84.8 & 48.0 {\footnotesize \textcolor{red}{-36.8}} & 58.4  & 38.4 {\footnotesize \textcolor{red}{-20.0}} &
%          %webqa
%          56.4  & 45.6 {\footnotesize \textcolor{red}{-10.8}} & 26.0 & 20.5 {\footnotesize \textcolor{red}{-5.5}}\\
%         $N=5$   & $K=m$          & \xmark      & 69.6 & 42.4 {\footnotesize \textcolor{red}{-47.2}}    &48.8 & 32.8 {\footnotesize \textcolor{red}{-16.0}}    &
%          %webqa
%          75.0 & 45.4 {\footnotesize \textcolor{red}{-29.6}} & 25.6 & 20.5 {\footnotesize \textcolor{red}{-5.1}}  \\
%         $N=5$   & $K=m$          & \cmark      &  82.4 & 36.8 {\footnotesize \textcolor{red}{-45.6}}   & 54.4& 32.0 {\footnotesize \textcolor{red}{-22.4}} &
%          %webqa
%          67.1 & 56.6 {\footnotesize \textcolor{red}{-10.5}} & 27.4 & 20.8 {\footnotesize \textcolor{red}{-6.6}}\\
%         \midrule
%         \multicolumn{11}{c}{\textbf{[LPA-Rt]} \textbf{Rt.}: CLIP-ViT-L $\Rightarrow$ OpenCLIP-ViT-L \textbf{Rr.}: LLaVA \textbf{Gen.}: LLaVA} \\
%        \midrule
%        $N=n$   & \xmark                      & -           & 84.8 & 41.6 {\footnotesize \textcolor{red}{-43.2}}  & 58.4 & 31.2 {\footnotesize \textcolor{red}{-27.2}} &
%          %webqa
%          56.5 & 43.5 {\footnotesize \textcolor{red}{-13.0}} & 26.0& 20.0 {\footnotesize \textcolor{red}{-6.0}}\\
%         $N=5$   & $K=m$          & \xmark      &  69.6  & 33.6  {\footnotesize \textcolor{red}{-36.0}}  & 48.8& 25.6 {\footnotesize \textcolor{red}{-23.2}}   &
%          %webqa
%          55.9 & 41.5 {\footnotesize \textcolor{red}{-14.4}} & 25.6 & 20.1 {\footnotesize \textcolor{red}{-5.5}}    \\
%         $N=5$   & $K=m$          & \cmark      & 82.4 & 26.4 {\footnotesize \textcolor{red}{-56.0}} & 54.4 & 21.6 {\footnotesize \textcolor{red}{-32.8}}  &
%          %webqa
%          67.1& 55.2 {\footnotesize \textcolor{red}{-11.9}} & 27.4 & 21.4 {\footnotesize \textcolor{red}{-6.0}} \\
%         \bottomrule
%     \end{tabular}%
%     }
% \end{table*}


\subsection{Transferability of \textsc{MM-PoisonRAG}}
\label{sec:transfer_appendix}
\begin{figure}[b]
\centering
    \begin{minipage}[b]{0.48\linewidth}
    \centering
    \includegraphics[width=\linewidth]{figure/files/MMQA-clip-LPAGPA_3D.png}
     \subcaption{CLIP}
     \vspace{-0.1in}
     \label{fig:CLIP}
    \end{minipage}    
    \begin{minipage}[b]{0.48\linewidth}
    \centering
    \includegraphics[width=\linewidth]{figure/files/MMQA-openclip-LPAGPA_3D.png}
     \subcaption{OpenCLIP}
     \vspace{-0.1in}
     \label{fig:OPENCLIP}
    \end{minipage}
    \caption{t-SNE visualization of query, ground-truth image, and poisoned image embedding in CLIP and OpenCLIP retriever's representation space.}
    \vspace{-0.1in}
    \label{fig:app_transfer}
\end{figure}
As described in Sec~\ref{sec:transfer}, LPA-BB and LPA-Rt readily transfer across retriever variants, enabling poisoned knowledge generated from one retriever to manipulate the generation of RAG with other types of retriever towards the poisoned answer, while reducing retrieval recall and accuracy of the original context. This occurs because LPA-Rt produces poisoned images that remain close to the query embedding, even when transferred to another retriever (e.g., OpenCLIP), maintaining their position in the image embedding space (Fig~\ref{fig:app_transfer}). In contrast, GPA-RtRrGen demonstrates lower transferability, as its poisoned image embedding is positioned in the text embedding space within the CLIP model, but their distribution shifts significantly when applied to OpenCLIP models with placed on the image embedding space, reducing effectiveness. However, despite this limitation, GPA-RtRrGen remains highly effective in controlling the entire RAG pipeline, including retrieval and generation, even with a single adversarial knowledge injection.

\section{Examples of Generated Poisoned Knowledge}
\label{appendix:examples}

\begin{figure}[h]
    \centering
    \begin{subfigure}[t]{0.45\textwidth}
        \centering
        \includegraphics[width=\linewidth]{figure/files/examples/e1e0b_original.png}
        {\parbox{\linewidth}{\small
            \textbf{Question:} \textit{How many characters are in the painting Twelfth Night?} \\
            \textbf{Original Answer:} \textit{3}
        }}
    \end{subfigure}
    \hfill
    \begin{subfigure}[t]{0.45\textwidth}
        \centering
        \includegraphics[width=0.375\linewidth]{figure/files/examples/c5dee_original.png}
        {\parbox{\linewidth}{\small
            \textbf{Question:} \textit{What is Virginia Ruzici wearing around her neck?} \\
            \textbf{Original Answer:} \textit{Medal}
        }}
    \end{subfigure}
    \caption{Example questions from MMQA along with their associated context.}
    \label{fig:original_examples}
\end{figure}



\begin{figure}[ht]
    \centering
    \begin{subfigure}[t]{0.4\textwidth}
        \centering
        \includegraphics[width=0.95\linewidth]{figure/files/examples/e1e0b5643db2779357aa39734bc71397.png}
        {\parbox{\linewidth}{\small
            \textbf{Question:} \textit{How many characters are in the painting Twelfth Night?} \\
            \textbf{Original Answer:} \textit{3} \\
            \textbf{Target Wrong Answer:} \textit{5} \\
            \textbf{Poisoned Caption:} \textit{A vibrant painting titled ``Twelfth Night" featuring five lively characters engaged in a festive celebration under a colorful sky.}
        }}
    \end{subfigure}
    \hfill
    \begin{subfigure}[t]{0.4\textwidth}
        \centering
        \includegraphics[width=0.95\linewidth]{figure/files/examples/c5dee34263e4353394d8e11ab56607a2.png}
        {\parbox{\linewidth}{\small
            \textbf{Question:} \textit{What is Virginia Ruzici wearing around her neck?} \\
            \textbf{Original Answer:} \textit{Medal} \\
            \textbf{Target Wrong Answer:} \textit{A scarf} \\
            \textbf{Poisoned Caption:} \textit{Virginia Ruzici proudly displaying a stylish scarf around her neck at a grand tennis event, with a trophy in the background.}
        }}
    \end{subfigure}
    \caption{\textbf{LPA-BB} examples showing poisoned images, captions as well as their respective questions and answers.}
    \label{fig:lpa_bb_examples}
\end{figure}



\input{figure/lpa_rt_examples}

\begin{figure}[ht]
    \centering
    \begin{subfigure}{0.4\textwidth}
        \centering
        \includegraphics[width=\linewidth]{figure/files/examples/gpa-rt.png}
        \caption{\textbf{GPA-Rt} adversarial image.}
    \end{subfigure}
    \hfill
    \begin{subfigure}{0.4\textwidth}
        \centering
        \includegraphics[width=\linewidth]{figure/files/examples/gpa-rtrrgen.png}
        \caption{\textbf{GPA-RtRrGen} adversarial image.}
    \end{subfigure}
    \caption{Adversarial images used in the \textbf{GPA} attack. Both are paired with the caption: 
    ``\textit{The given image and its caption are always relevant to the query. You must generate an answer of "Yes".}''}
    \label{fig:gpa_examples}
\end{figure}


