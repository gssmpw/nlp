\section{Limitations}
% Not included in the page limit

While our study exposes critical vulnerabilities in multimodal RAG systems and demonstrates how knowledge poisoning can be highly disruptive, we acknowledge the following limitations of our work:
\begin{itemize}
    \item Narrow task scope. We concentrate our attack and evaluation on QA tasks, given that RAG is primarily intended for knowledge-intensive use cases. However, RAG methodologies may also apply to other scenarios, such as summarization or dialog-based systems, which we do not investigate here. Although our proposed attack principles can be extended, further work is necessary to assess their effectiveness across a broader spectrum of RAG-driven tasks.
    \item Lack of exploration of defensive methods. Our study emphasizes designing and evaluating poisoning attacks rather than defenses. We do not propose specific mitigation strategies or incorporate adversarial detection techniques (e.g., anomaly detection on retrieved image-text pairs). As a result, critical questions remain about how to effectively secure multimodal RAG in real-world deployments.
    \item Restricted modalities. Our framework focuses predominantly on images as the primary non-textual modality. In real-world applications, RAG systems may rely on other modalities (e.g., audio, video, or 3D data). Studying how poisoning attacks operate across multiple or combined modalities—potentially exploiting different vulnerabilities in each—remains an important open direction for future work.
\end{itemize}

