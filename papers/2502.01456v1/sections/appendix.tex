% \section{PRIME}


\looseness=-1
To address the above challenges, we propose PRIME, a scalable online RL method with dense rewards.
The key insight of PRIME is to apply \textit{implicit process rewards}, which are derivable from the Implicit PRM that is trained with only outcome labels~\citep{yuan2024freeprocessrewardsprocess}.
This property enables us to update the PRMs online to avoid reward hacking. 
We then design a flexible framework to incorporate implicit process rewards with outcome rewards into any kind of MC advantage estimate.
PRIME is illustrated in Figure \ref{fig:prime-algo} and Algorithm \ref{algo:prime}.
Next, we will detail the implicit process rewards (\S \ref{sec:prime_prm}) and how we leverage them to calculate advantages (\S \ref{sec:prime_loss_adv}), and introduce other techniques we used (\S \ref{sec:prime_init_filter}).



\subsection{Enabling Scalable Reward Update with Implicit Reward Modeling}
\label{sec:prime_prm}

\looseness=-1
We consider dense rewards from the Implicit PRM because of the scalability. 
In short, Implicit PRM enables training an ORM with outcome labels only while repurposing it as a PRM at inference. 
The training stage is the same as standard ORM pipelines, with the only difference being representing the reward as $r_\phi(\mathbf{y}):= \beta \log \frac{\pi_\phi(\mathbf{y})}{\pi_\text{ref}(\mathbf{y})}$, where $\pi_\phi$ is the RM and $\pi_\text{ref}$ is the reference model, both of which are causal LMs. At inference, the process rewards are obtained by:
\begin{equation}
\label{eq:pr}
    r_\phi(y_t) := \beta \log \frac{\pi_\phi(y_{t}|\mathbf{y}_{<t})}{\pi_\text{ref}(y_{t}|\mathbf{y}_{<t})}
\end{equation}


\begin{algorithm}

\caption{Process Reinforcement through Implicit Rewards (PRIME)}
\textbf{Input} Language model $\pi_{\theta_{\text{init}}}$; outcome reward verifier $r_o$; dataset $\mathcal{D}$; sample number $K$; total iteration $N$.

% \textbf{Notation} $r_i$ represents the outcome reward of $\mathbf{y}^i$, and $r^i_t$ denotes its process reward at step $t$.
% Define $r_\phi \left(\mathbf{y}\right) = \beta \log \frac{\pi_\phi(\mathbf{y})}{\pi_{\text{ref}}(\mathbf{y})}$, $\mathbf{x}$ is omitted for simplicity.
\begin{algorithmic}[1]
\State Initialize policy model $\pi_\theta \leftarrow \pi_{\theta_{\text{init}}}$, $\pi_{\theta_{\text{old}}} \leftarrow \pi_{\theta_{\text{init}}}$, implicit PRM $\pi_{\phi} \leftarrow \pi_{\theta_{\text{init}}}$, reference model $\pi_{\text{ref}} \leftarrow \pi_{\theta_{\text{init}}}$
\For{iteration = 1, \dots, N}
    \State Sample batch of prompts $\mathcal{B} \sim \mathcal{D}$
    % \State $\mathcal{T} \leftarrow \{\}$
    % \State $\mathcal{P} \leftarrow \{\}$
       % \State \hao{for the following 5 lines, i'd wrap them in to for each $(x, y)\in \mathcal{B}$}
        \State Generate $K$ responses: $\{\mathbf{y}^1, ..., \mathbf{y}^K\} \sim \pi_\theta(\cdot|\mathbf{x})$ for $\mathbf{x} \in \mathcal{B}$
        \State Compute outcome rewards: $r_o\left(\mathbf{y}^{1:K}\right)$
        \State Apply accuracy filter (\S \ref{sec:prime_init_filter}) on all prompts: $\mathcal{T} \leftarrow \text{Filter}(\mathbf{x}, \mathbf{y}^{1:K}, r_o\left(\mathbf{y}^{1:K}\right))$ for $\mathbf{x} \in \mathcal{B}$
        \State Forward pass $\pi_\phi, \pi_\text{ref}$ on each $(\mathbf{x}, \mathbf{y}) \in \mathcal{T}$ to obatin implicit process reward $r_\phi(y_t)$ with Eq.~\ref{eq:pr} 
        \State Update Implicit PRM $\pi_\phi$ by CE loss on $(\mathbf{x}, \mathbf{y}, r_o\left(\mathbf{y}\right)) \in \mathcal{T}$:
            \[
            \mathcal{L}_{\text{CE}}(\phi) = -\mathbb{E}_{\left(\mathbf{x},\mathbf{y},r_o\left(\mathbf{y}\right)\right)\sim\mathcal{T}} \left[ r_o\left(\mathbf{y}\right) \cdot \log \sigma \left( r_\phi \left(\mathbf{y}\right) \right) + (1-r_o\left(\mathbf{y}\right)) \cdot \log\left( 1 - \sigma \left( r_\phi \left(\mathbf{y}\right) \right) \right) \right]
            \]

        \State Compute advantages $A$ with Eq.~\ref{eq:adv} %Here we use RLOO with $\{\mathbf{y}_1, ..., \mathbf{y}_K\}$ of the same $x$:
        % \[\hat{A} = \underbrace{\frac{1}{K} \sum_{i=1}^{K}\left[r_i-\frac{1}{K-1} \sum_{j \neq i} r_j\right]}_\text{RLOO with outcome rewards} +  \underbrace{\frac{1}{K} \sum_{i=1}^{K} \sum_{t=1}^{|\mathbf{y}_i|} \gamma^t \cdot \left[r_i^t-\frac{1}{K-1} \sum_{j \neq i} \frac{r_\phi \left(\mathbf{y}_j\right)}{|\mathbf{y}_j|}\right]}_\text{RLOO with implicit process rewards}\]
        \State Update policy $\pi_\theta$ by PPO loss in Eq.~\ref{eq:clip}
    \State Update old parameters: $\theta_{\text{old}} \leftarrow \theta$%, $\phi_{\text{old}} \leftarrow \phi$

\EndFor
\end{algorithmic}
\textbf{Output} Optimized policy model $\pi_\theta$

\label{algo:prime}
\end{algorithm}

\looseness=-1
In PRIME, upon rollouts being generated and graded by the (ground truth) outcome verifier, we \textbf{update the Implicit PRM online with on-policy rollouts and outcome supervision} and then \textbf{calculate token-level dense rewards to estimate advantages}, which solves C1 and C2 mentioned in \S \ref{sec:challenges} respectively:
(1) To prevent overoptimization and reward hacking, it is crucial to update reward models online. However, updating previous PRMs \citep{Lightman2023LetsVS} requires annotating step labels on the latest policy rollouts, which is neither efficient nor scalable during online RL.
In contrast, the Implicit PRM only demands outcome labels to train due to its special reward representation, and thus it can be easily updated with policy rollouts and outcome labels or rewards, both of which have already been collected to update the policy model.
(2) Unlike common PRMs that produce only step-level rewards, the Implicit PRM provides more fine-grained \textit{token-level} rewards at no additional cost. This addresses the ambiguity in identifying steps in LLM responses while not introducing extra overhead, making it easy to combine with any RL algorithms for advantage estimation.

\begin{wrapfigure}{r}{0.5\textwidth}
\centering
    \vspace{-17pt}
    \includegraphics[width=\linewidth]{figures/images/prime-algo.pdf}
    \caption{Illustration of PRIME. PRIME follows that (1) initialize policy model and the Implicit PRM both with the reference model; (2) sample multiple responses for each prompt and filter with output accuracy; (3) obtain implicit process rewards by the Implicit PRM and update it using cross-entropy (CE) loss; (4) compute advantage and policy loss then update the policy model. } 
    \label{fig:prime-algo}
    \vspace{-30pt} % 调整为合适的负值
\end{wrapfigure}
\iffalse
\begin{equation}
    \label{eq:ce}
    \mathcal{L}_{\text{CE}} = l \cdot \log \sigma \left( \beta \log \frac{\pi_\phi(\mathbf{y})}{\pi_\text{ref}(\mathbf{y})} \right)+ (1 - l) \cdot \log \left[ 1 - \sigma \left( \beta \log \frac{\pi_\phi(\mathbf{y})}{\pi_\text{ref}(\mathbf{y})} \right) \right]
\end{equation}
\fi



\subsection{Advantage Estimation and Policy Update}
\label{sec:prime_loss_adv}



\textbf{Estimating advantages using Monte Carlo estimator with a leave-one-out baseline.}
After obtaining token-level dense rewards, we calculate advantages based on either MC estimators or GAE.
To determine the advantage function in PRIME, we compare GAE with several MC estimators, including REINFORCE~\citep{williams1992simple}, RLOO~\citep{ahmadian2024back}, and GRPO~\citep{deepseek-math}. Experimental details and results can be found in \S \ref{sec:other_algo}.

We find that MC estimators, despite being simpler, are strong enough to produce stable results. Therefore, we choose MC estimate as our advantage function and despite PRIME being compatible with any baseline estimation approaches, we instantiate it with a leave-one-out baseline from $K$ samples \citep{ahmadian2024back} in this paper, as it performs better in the experiments:
\begin{equation}
    A^i = r(\mathbf{y}^i_T)-\frac{1}{K-1} \sum_{j \neq i}r(\mathbf{y}^j_T)
\end{equation}
where $r(\mathbf{y}^i_T)$ denotes the reward of $i$-th response at final step $T$, $K$ is the number of samples for one prompt. The leave-one-out (LOO) baseline helps reduce variances.


More specifically, we use an Implicit PRM $\pi_\phi$ and an outcome verifier or reward model $r_o$.
We calculate the return of implicit process rewards and outcome rewards separately if both are available, since directly mixing their values may lead to numerical instability~\citep{deepseek-math}.
\textbf{For implicit process rewards}, we perform a three-step process to calculate return:
(1) Use the averaged implicit process rewards to calculate the leave-one-out baseline;
(2) Normalize the process reward at step $t$ by subtracting the baseline;
(3) Calculate the discounted return for each response.
\textbf{For outcome rewards}, we directly adopt LOO without any modification.
Finally, the advantage is set to the combination of both returns:
\begin{equation}
\label{eq:adv}
    \begin{aligned}
        A^i_t = &\underbrace{\sum_{s=t}^{|\mathbf{y}^i|} \gamma^{s-t} \cdot \left[r_\phi(y^i_s)-\frac{1}{K-1} \sum_{j \neq i} r_\phi \left(\mathbf{y}^j\right)\right]}_\text{RLOO with implicit process rewards}+\underbrace{r_o\left(\mathbf{y}^i\right)-\frac{1}{K-1} \sum_{j \neq i} r_o\left(\mathbf{y}^j\right)}_\text{RLOO with outcome rewards}
    \end{aligned}
\end{equation}

\textbf{Updating policy with PPO clip surrogate loss.}
We adopt PPO clip surrogate loss for more stable policy updates: 
\begin{equation}
\begin{aligned}
\label{eq:clip}
    L_{\text{CLIP}}(\theta) = &\mathbb{E}_t\Biggl[\min\biggl(\frac{\pi_\theta(y_t|\mathbf{y}_{<t})}{\pi_{\theta_{\text{old}}}(y_t|\mathbf{y}_{<t})}A_t,\text{clip}\Bigl(\frac{\pi_\theta(y_t|\mathbf{y}_{<t})}{\pi_{\theta_{\text{old}}}(y_t|\mathbf{y}_{<t})}, 1-\epsilon, 1+\epsilon\Bigr)A_t\biggr)\Biggr]
\end{aligned}
\end{equation}
where $\epsilon$ is a clipping parameter. The loss prevents the updated policy from deviating too far from the original distribution, which is the prerequisite of importance sampling. 
The legitimacy of importance sampling then enables the reuse of rollouts sampled in previous steps, thus improving sampling efficiency. 








\subsection{Other Techniques}
\label{sec:prime_init_filter}
\textbf{Initializing PRM with SFT/base model.}
In practice, we find that the starting policy model itself serves as a decent initialization of PRM, bypassing the PRM training stage. This solves C3 in \S \ref{sec:challenges} and even outperforms a dedicatedly trained PRM, as shown in \S~\ref{sec:design}.


% \begin{wrapfigure}{r}{0.45\textwidth}  % r表示图片在右侧,0.5\textwidth表示图片宽度为文本宽度的50%
%     \centering
%     \includegraphics[width=\linewidth]{figures/images/train_rewards_filter.pdf}  % 插入图片,宽度设置为wrapfigure环境的宽度
%     \caption{Impact of online prompt filtering on training rewards.} 
%     \label{fig:online_prompt_filter}
% \end{wrapfigure}

\begin{wrapfigure}{r}{0.5\textwidth}
\centering
    \vspace{-10pt}
    \includegraphics[width=\linewidth]{figures/images/train_rewards_filter.pdf}
    \caption{Impact of online prompt filtering on training rewards.} 
    \label{fig:online_prompt_filter}
    \vspace{-16pt}
\end{wrapfigure}
\textbf{Online Prompt Filtering.}
As we sample multiple trajectories for each prompt, we introduce online prompt filtering which filters prompts within a certain accuracy range.
This (1) preserves only the prompts within a certain median-level difficulty range~\citep{yang2024qwen25mathtechnicalreportmathematical} and (2) balances data distribution for the Implicit PRM online training.

We present the ablation study results in Figure \ref{fig:online_prompt_filter} using RLOO with outcome rewards only, from which we can see that the online prompt filter largely lowers the variance of RL training. 

\textbf{How PRIME addresses challenges in \S \ref{sec:challenges}.} In summary, as illustrated in Figure \ref{fig:prime-algo} and Algorithm \ref{algo:prime}, PRIME adopts implicit process rewards for efficient PRM online update (C2), then integrates token-level dense rewards with outcome rewards in MC advantage estimate (C1). The PRMs are directly initialized from SFT or base models, which foregoes explicit reward modeling (C3). 





















\newpage
% \section{Additional Results}
% \label{sec:app_results}

% \subsection{Reference Model Choice is Flexible} 

% \begin{figure}[t]
    \centering
    \begin{subfigure}{0.44\textwidth}
        \centering
        \includegraphics[width=\linewidth]{figures/images/policy_ref.png}
        \caption{Policy ref: We use the policy logprob as $ \pi_{\text{ref}}$ for PRM.} % 第一个图形的子标题
        \label{fig:policy_ref}
    \end{subfigure}
    \hfill % 在两个子图之间添加一些水平间距
    \begin{subfigure}{0.54\textwidth}
        \centering
        \includegraphics[width=\linewidth]{figures/images/sfr_ref.png}
        \caption{SFT ref: We retain the initial policy to provide  $ \pi_{\text{ref}}$ for PRM and KL.} % 第二个图形的子标题
        \label{fig:sft_ref}
    \end{subfigure}
    \caption{Comparison of different reference policy implementations. One uses the running policy's old logprobs as reference (policy ref) while the other uses the initial SFT model as the reference model (SFT ref).} % 整个图形的标题
    \label{fig:online_prm}
\end{figure}
% % \begin{wrapfigure}{r}{0.45\textwidth}  % r表示图片在右侧,0.5\textwidth表示图片宽度为文本宽度的50%
%     \centering
%     \includegraphics[width=\linewidth]{figures/images/train_rewards_ref.pdf}  % 插入图片,宽度设置为wrapfigure环境的宽度
%     \caption{Title.} 
%     \label{fig:effect_of_ref_policy}
% \end{wrapfigure}

\begin{wrapfigure}{r}{0.5\textwidth}
    \centering
    \vspace{-10pt}
    \includegraphics[width=\linewidth]{figures/images/train_rewards_ref.pdf}
    \caption{\textbf{Different reference model for PRM.} We compare two reference model selection strategies for PRIME. Using the policy model as reference and using the initial SFT model as reference. Their rewards are similar.}
    \label{fig:effect_of_ref_policy}
    \vspace{-25pt}
\end{wrapfigure}

\iffalse
\begin{minipage}{0.50\textwidth}
        \centering
        \begin{tabular}{lcc}
\toprule
\textbf{Step} & \textbf{SFT Ref} & \textbf{Policy Ref} \\ \midrule
80            & 36.8             & 36.7                \\
160           & 36.5             & 38.4                \\
240           & 41.0             & 40.5                \\
320           & 41.7             & 41.0                \\ \bottomrule
\end{tabular}
        \captionof{table}{Test Accuracy.\lifan{this is too small, remove from subfig}} 
        \label{tab:effect_of_ref_policy}
    \end{minipage}
\fi
% % We introduce online PRM, which updates with policy model rollouts and their corresponding verifier outcomes. Here we demonstrate the importance of online updates for PRMs. We compare two settings, where the online PRM is initialized by Eurus-2-7B-SFT and the offline PRM is EurusPRM-Stage1\hanbin{This seems to be the first mention of EurusPRM-Stage1, which may require a quote or additional explanation}. As shown in Figure \ref{fig:online_prm}, we can see that online PRM outperforms offline PRM by a large margin on both training and test sets. 


% %\subsection{Effect of Reference Policy}
% %\lifan{a bit confusing about reference policy and ref model for implicit prm. elaborate a bit before diving into details?}\ganqu{the same, merged. But this is not so imporatnt, consider moving to Appendix}
% We implement two variants of our algorithms to explore the effect of reference model of implicit PRM, one using the initial SFT model as the reference model (SFT ref) while the other using the running policy's old logprobs as reference (policy ref), as shown in Figure~\ref{fig:policy_ref}. The policy ref simply adopts the old logprob of the policy model as  $\pi_{\text{ref}}$, while the SFT ref remains the initial SFT model for an additional $\pi_{\text{ref}}$ calculation. We compare their performance in this section. 

% From the training rewards in Figure \ref{fig:effect_of_ref_policy}, we find the two strategies are close and have pros and cons in different aspects: The Q value calculated by implicit PRM is the expectation under the distribution of the reference model. So the updating policy could natrually serve as the reference.
% On the other hand, KL divergence calculation is only allowed when the initial SFT model is retained.



% \begin{figure*}[tbh]
    \centering
    \begin{subfigure}{0.48\textwidth}
        \centering
        \includegraphics[width=\linewidth]{figures/images/prm_acc_double_forward.pdf}
        \caption{PRM classification accuracy on training samples.} % 第一个图形的子标题
        \label{fig:prm_acc_double_forward}
    \end{subfigure}
    \hfill % 在两个子图之间添加一些水平间距
    \begin{subfigure}{0.48\textwidth}
        \centering
        \includegraphics[width=\linewidth]{figures/images/train_rewards_forward.pdf}
        \caption{Training outcome rewards.} % 第二个图形的子标题
        \label{fig:train_double_forward}
    \end{subfigure}
    \caption{\textbf{Single and double forward.} While double forward methods obtain higher accuracy after online update, the two variants achieve similar rewards during training.} % 整个图形的标题
    \label{fig:single_double_forward}
\end{figure*}


% \subsection{Single-Forward v.s. Double-Forward}
% % \lifan{review current forward flow so we know what's single forward?}\lifan{and move to appendix?}
% Since our implicit PRM is concurrently updated in training, for each rollout stage, we can update the PRM before the policy model and use the updated PRM to re-calculate the process rewards, which we call the double-forward setting. We investigate the impact of double-forward in both the training and test phases. Our default setting applies single-forward, which uses process rewards from old PRMs. We plot PRM accuracy on rollouts and training rewards in Figure \ref{fig:single_double_forward}.


% % \begin{figure*}[tbh]
    \centering
    \begin{subfigure}{0.48\textwidth}
        \centering
        \includegraphics[width=\linewidth]{figures/images/prm_acc_double_forward.pdf}
        \caption{PRM classification accuracy on training samples.} % 第一个图形的子标题
        \label{fig:prm_acc_double_forward}
    \end{subfigure}
    \hfill % 在两个子图之间添加一些水平间距
    \begin{subfigure}{0.48\textwidth}
        \centering
        \includegraphics[width=\linewidth]{figures/images/train_rewards_forward.pdf}
        \caption{Training outcome rewards.} % 第二个图形的子标题
        \label{fig:train_double_forward}
    \end{subfigure}
    \caption{\textbf{Single and double forward.} While double forward methods obtain higher accuracy after online update, the two variants achieve similar rewards during training.} % 整个图形的标题
    \label{fig:single_double_forward}
\end{figure*}


% Accordingly, we find that double-forward could increase PRM accuracy, but the training rewards remain close between the two methods. %We also compare the average test accuracy of single and double-forward\hanbin{there is a simple table in our blog} in Table \ref{tab:effect_of_single_double}. Their performances are also close. Single double-forward brings more computation overhead, we recommend the single-forward setting in practice.


% \subsection{Results of Different RL Algorithms}
% \label{sec:diff_rl_algo}

% % \begin{table}[t]
% \centering
% \caption{Results of Different RL Algorithms.}
% \label{tab:diff_rl_algo}
% \resizebox{1\textwidth}{!}{

% \begin{tabular}{llcccccc}
% \toprule
% \textbf{Step}         & \textbf{Algorithm} & \textbf{MinervaMath} & \textbf{OlympiadBench} & \textbf{HumanEval} & \textbf{LeetCode} & \textbf{LiveCodeBench} & \textbf{Avg.} \\ \midrule
% \multirow{5}{*}{256}  & PPO                & 21.7                  & 18.2                    & 62.8               & 13.3              & 17.1                    & 26.6          \\
%                       & REINFORCE          & 21.7                  & 19.0                    & 64.6               & 13.9              & 17.1                    & \textbf{27.3} \\
%                       & GRPO               & 22.8                  & 18.4                    & 59.2               & 16.1              & 17.3                    & 26.8          \\
%                       & ReMax              & 22.8                  & 19.6                    & 58.5               & 12.8              & 15.8                    & 25.9          \\
%                       & RLOO               & 18.8                  & 20.7                    & 60.4               & 16.1              & 17.8                    & 26.8          \\ \midrule
% \multirow{4}{*}{1024} & REINFORCE          & 19.5                  & 16.0                    & 57.3               & 21.1              & 16.0                    & 26.0          \\
%                       & GRPO               & 22.4                  & 20.3                    & 57.3               & 13.3              & 18.7                    & 26.4          \\
%                       & ReMax              & 24.6                  & 17.3                    & 61.0               & 21.1              & 18.6                    & 28.5          \\
%                       & RLOO               & 21.0                  & 20.6                    & 57.9               & 27.8              & 21.4                    & \textbf{29.7} \\ \bottomrule
% \end{tabular}

% }
% \end{table}

\begin{table*}[t]
\centering
\caption{Testset results of different RL algorithms.}
\label{tab:diff_rl_algo}
\resizebox{1\textwidth}{!}{

\begin{tabular}{llcccccccc}
\toprule
\textbf{Method}                  & \textbf{Step}           & \multicolumn{1}{l}{\textbf{AIME 2024}} & \multicolumn{1}{l}{\textbf{AMC}}      & \multicolumn{1}{l}{\textbf{MATH-500}} & \multicolumn{1}{l}{\textbf{MinervaMath}} & \multicolumn{1}{l}{\textbf{OlympiadBench}} & \multicolumn{1}{l}{\textbf{LeetCode}} & \multicolumn{1}{l}{\textbf{LiveCodeBench}} & \multicolumn{1}{l}{\textbf{Avg.}}     \\ \midrule
\textbf{RLOO}         & 240 & 20.0  & 47.0          & 73.2          & 36.4              & 35.4               & 28.3          & 26.7               & 36.9          \\
\rowcolor[HTML]{D7E8E8}
\textbf{RLOO w/ PRIME} & 240 & 20.0 &50.6 &78.2  &39.3 &40.3 &31.1 &27.5 &41.0 \\ 
\midrule
\textbf{REINFORCE} &240 &6.7  &47.0  &72.6 &36.0 &37.2 &27.2 &25.0 &36.0 \\
\rowcolor[HTML]{D7E8E8}
\textbf{REINFORCE w/ PRIME} & 240 &  6.7 & 50.0 & 76.4 & 36.8 &39.1 & 27.8 & 27.5 & 37.8 \\
\midrule
\textbf{GRPO} &240 &10.0  &44.6  &73.2 &37.5 &36.6 &25.0 &25.8 &36.1 \\
\rowcolor[HTML]{D7E8E8}
\textbf{GRPO w/ PRIME} &240 &16.7  &47.0  &75.0 &34.9 &38.2 &28.9 &23.9 &37.8 \\
\midrule
\textbf{PPO} & 240 & 10.0 & 41.0 & 73.6 & 36.0 &36.3 & 28.3 & 25.7 & 35.8 \\
\textbf{PRIME as Value Model} &240 &16.7  &44.6  &72.6 &34.6 &35.7 &27.8 &24.6 &36.6 \\
\rowcolor[HTML]{D7E8E8}
\textbf{PPO w/ PRIME} &240 &13.3  &50.6  &77.4 &37.1 &40.6 &30.0 &26.7 &39.4 \\
% \textbf{PRIME (response level) } &240 &16.7  &54.2  &79 &37.5 &40.4 &27.2&26.4 &40.2 \\
% \textbf{PRIME ReMax} &240 &16.7  &45.8  &75 &38.6 &37.9 &25.6 &23.7 &37.6 \\
\bottomrule
\end{tabular}

}
\end{table*}
% % The results of different RL algorithms on Llama-3.1-8B are listed in Table \ref{tab:diff_rl_algo}. Since we use a different base model and dataset for the pilot study, the benchmarks used here are slightly different from the main experiments. From the results, we find that REINFORCE-like algorithms, despite being simpler than PPO, are strong enough to produce stable results. In this paper, we choose the best performing RLOO as our RL algorithm.

% We ablate PRIME and different RL algorithms with their variants and find that the PRIME algorithm achieves the best performance for several reasons. 

% First of all, We compare different REINFORCE-like advantage estimators including REINFORCE, GRPO, and RLOO, toggling the existence of implicit process reward. To make different algorithms compatible with the compound of outcome verifier reward and process reward, we accordingly make adaptions similar to Eq. \ref{eq:adv}. For GRPO, we have
% \begin{equation}
%         A^i_t = \underbrace{\frac{r_{o}\left(\mathbf{y}^i\right)-\text{mean}( r_o\left(\mathbf{y}^j\right))}{\text{std}( r_o\left(\mathbf{y}^j\right))}}_\text{GRPO with outcome rewards} + \underbrace{\sum_{s=t}^{|\mathbf{y}^i|} \gamma^{s-t} \cdot \left[\frac{r_\phi(y^i_s)-\text{mean}\left(\frac{r_\phi \left(\mathbf{y}^j\right)}{|\mathbf{y}^j|}\right)}{\text{std}\left(\frac{r_\phi \left(\mathbf{y}^j\right)}{|\mathbf{y}^j|}\right)}\right]}_\text{GRPO with implicit process rewards}.
% \end{equation}
% For REINFORCE, we have
% \begin{equation}
% A^i_t = \underbrace{r_o\left(\mathbf{y}^i\right)}_\text{REINFORCE with outcome rewards} + \underbrace{\sum_{s=t}^{|\mathbf{y}^i|} \gamma^{s-t} \cdot r_\phi(y^i_s)}_\text{REINFORCE with implicit process rewards}.
% \end{equation}
% As shown in Table \ref{tab:diff_rl_algo}, PRIME contributes consistently regardless of the policy update method, making it a generic algorithm. 

% \iffalse
% On top of that, a process reward model prevails in providing token-level signals compared to a value estimation model. On the one hand, apparently the PRIME algorithm with RLOO advantage estimator surpasses the PPO implementation with only the verifier outcome reward model. This means while both provide online updating token-level signals, the PRIME algorithm is essentially distinct from a TD error critic model. On the other hand, we can also derive a value model from \cite{yuan2024freeprocessrewardsprocess} and the corresponding advantage estimator. 
% % \zefan{do we need to add superscript i?}
% \begin{proposition}\label{prop:implicit_value} 
% Consider an autoregressive process with an outcome reward at the last step~($T$). Define the Q value as $q_\theta^t(\mathbf{y}_{<t}, y_t):= \sum_{i=1}^{t} \beta \log \frac{\pi_\theta(y_{i}|\mathbf{y}_{<i})}{\pi_\text{ref}(y_{i}|\mathbf{y}_{<i})}$. 
% \begin{equation}
% V_t(\mathbf{y}_{<t}) = q_\theta^{t-1}(\mathbf{y}_{<t-1},\mathbf{y}_{t-1})
% \end{equation}

% \begin{equation}
% \hat A^{\text{GAE}(1,\lambda)}_t = \lambda^{T-t} \cdot
% \left[ \underbrace{r_o\left(\mathbf{y}\right)}_\text{PPO with outcome rewards} - \quad q_\theta^t(\mathbf{y}_{<t}, y_t) 
% \right] + \sum_{s=0}^{T-t} \lambda^l \cdot r_\phi(y_s)
% \end{equation}

% \end{proposition}

% \begin{proof}
% We set $r_t=0$ for $t<T$ and $r_T=r_o\left(\mathbf{y}\right)$. The above definition of $q^t_\theta$ is the exponential average of undiscounted outcome reward, so we omit the discount factor $\gamma$ for brevity. 
% \begin{equation}
% \begin{aligned}
% V_t(\mathbf{y}_{<t}) =& q_\theta^{t-1}(\mathbf{y}_{<t-1}, y_{t-1}) - r_{t-1} \\
% =&q_\theta^{t-1}(\mathbf{y}_{<t-1}, y_{t-1}) \nonumber
% \end{aligned}
% \end{equation}

% \begin{equation}
% \begin{aligned}
% \delta_t:=&r_t+V_{t+1}(\mathbf{y}_{<t-1})-V_t(\mathbf{y}_{<t})\\
% =&
% \begin{cases}
%   r_\phi(y_t),& 0\leq t < T,\\
%   r_o(\mathbf{y})-q_\theta^{t-1}(\mathbf{y}_{<t-1}), & t = T.
% \end{cases} \nonumber
% \end{aligned}
% \end{equation}

% \begin{equation}
% \hat A^{\text{GAE}(1,\lambda)}_t = \sum_{s=0}^{T-t} \lambda^l \cdot \delta_{t+s} \nonumber
% \end{equation}
% \end{proof}

% Based on the above proposition, we implement PRIME as a value model instead of a reward model, keeping its initialization and online update. As shown in Table \ref{tab:diff_rl_algo}, PRIME performs better as a reward model rather than a value model. We hypothesize that this phenomenon is related to the benefit of using advantage as process reward~\cite{setlur2024rewarding} and leave this topic for future exploration. 
% \fi

% Moreover, the PPO variant of PRIME provides no performance gain, demonstrating that the additional computation cost from the critic model is redundant. This makes it possible to compensate for the expense of the process reward model by using REINFORCE-like algorithms with simpler advantage estimators. 

% Finally, we choose the best-performing RLOO as the advantage estimator in our algorithm. 




% For PPO with value models,
% \begin{equation}
% \underbrace{r_o\left(\mathbf{y}^i\right)}_\text{PPO with outcome rewards} + \underbrace{\sum_{s=t}^{|\mathbf{y}^i|} \gamma^{s-t} \cdot r_\phi(y^i_s)}_\text{PPO with implicit process rewards} - V(\mathbf{y}_{<t})

% For GRPO,


% From the results, we find that REINFORCE-like algorithms, despite being simpler than PPO, are strong enough to produce stable results. In this paper, we choose the best-performing RLOO as our RL algorithm.


% % Please add the following required packages to your document preamble:
% \usepackage{multirow}
% \usepackage[table,xcdraw]{xcolor}
% Beamer presentation requires \usepackage{colortbl} instead of \usepackage[table,xcdraw]{xcolor}
\begin{table*}[t]
\centering
\caption{Detailed results of PRIME and RLOO w/ outcome verifier (OV). At the same 240 steps, the model trained by PRIME is generally better than the model trained by outcome rewards.}
\label{tab:dense_rewards_results}
\resizebox{1\textwidth}{!}{
\begin{tabular}{llcccccccc}
\toprule
\textbf{Method}                  & \textbf{Step}               & \multicolumn{1}{l}{\textbf{AIME 2024}} & \multicolumn{1}{l}{\textbf{AMC}}      & \multicolumn{1}{l}{\textbf{MATH-500}} & \multicolumn{1}{l}{\textbf{MinervaMath}} & \multicolumn{1}{l}{\textbf{OlympiadBench}} & \multicolumn{1}{l}{\textbf{LeetCode}} & \multicolumn{1}{l}{\textbf{LiveCodeBench}} & \multicolumn{1}{l}{\textbf{Avg.}}     \\ \midrule
\textbf{GPT-4o}          & -                           & 9.3                                    & 45.8                                  & 76.4                                  & 36.8                                      & 43.3                                       & 58.9                                  & 48.8                                       & 45.6                                  \\
\textbf{Llama-3.1-70B-Inst.}          & -                           & 20.0                                    & 37.3                                  & 65.0                                  & 37.1                                      & 30.5                                       & 35.0                                  & 34.4                                       & 37.0                                  \\
\textbf{Qwen2.5-Math-7B-Inst.}          & -                           & 13.3                                    & 50.6                                  & 79.8                                  & 34.6                                     & 40.7                                       & 11.7                                  & 11.3                                      & 34.6                                  \\
\textbf{Eurus-2-7B-SFT}          & 0                           & 3.3                                    & 30.1                                  & 66.2                                  & 32.7                                      & 29.8                                       & 21.7                                  & 17.8                                       & 28.8                                  \\ \midrule
\textbf{RLOO w/ OV Only}         & \cellcolor[HTML]{D7E8E8}240 & \cellcolor[HTML]{D7E8E8}\textbf{20.0}  & \cellcolor[HTML]{D7E8E8}47.0          & \cellcolor[HTML]{D7E8E8}73.2          & \cellcolor[HTML]{D7E8E8}36.4              & \cellcolor[HTML]{D7E8E8}35.4               & \cellcolor[HTML]{D7E8E8}28.3          & \cellcolor[HTML]{D7E8E8}26.7               & \cellcolor[HTML]{D7E8E8}36.9          \\ \midrule
                                 & 80                          & 20.0                                   & 41.0                                  & 68.2                                  & 38.2                                      & 37.0                                       & 26.7                                  & 26.6                                       & 36.8                                  \\
                                 & 160                         & 13.3                                   & 42.2                                  & 72.0                                  & 37.1                                      & 38.7                                       & 26.7                                  & 25.6                                       & 36.5                                  \\
                                 & \cellcolor[HTML]{D7E8E8}240 & \cellcolor[HTML]{D7E8E8}\textbf{20.0}  & \cellcolor[HTML]{D7E8E8}\textbf{50.6} & \cellcolor[HTML]{D7E8E8}\textbf{78.2} & \cellcolor[HTML]{D7E8E8}\textbf{39.3}     & \cellcolor[HTML]{D7E8E8}\textbf{40.3}      & \cellcolor[HTML]{D7E8E8}\textbf{31.1} & \cellcolor[HTML]{D7E8E8}\textbf{27.5}      & \cellcolor[HTML]{D7E8E8}\textbf{41.0} \\
                                 & 320                         & 16.7                                   & 51.8                                  & 77.8                                  & 39.7                                      & 41.5                                       & 36.1                                  & 28.5                                       & 41.7                                  \\
\multirow{-5}{*}{\textbf{Eurus-2-7B-PRIME}} & 592                         & 26.7                                   & 57.8                                  & 79.2                                  & 38.6                                      & 42.1                                       & 33.3                                  & 28.6                                       & 43.9                                  \\ \bottomrule
\end{tabular}
}
\end{table*}
% \subsection{Effect of dense rewards}
% \label{sec:effeoc_of_dense_rewards}
% Table ~\ref{tab:dense_rewards_results} shows the detailed results of PRIME and RLOO w/ outcome verifier (OV). We can see that at the same 240 steps, the model trained by PRIME is generally better than the model trained by outcome rewards, leading to a 4-point performance gap. PRIME could further enhance the model with more training steps. 



% \subsection{Effect of Reference Policy}
% \label{sec:effeoc_of_ref_policy}
% We delve into the comparative analysis of different reference policy implementations. As shown in Figure \ref{fig:effect_of_ref_policy}, The left one (policy ref) simply adopts the old logprob of policy model as  $\pi_{\text{ref}}$, while the right one (SFT ref) remains the initial SFT model for an additional $\pi_{\text{ref}}$ calculation.

% \subsection{``Zero'' Experiments}
% \label{sec:app_zero}
% \begin{figure*}[tbh]
    \centering
    \begin{subfigure}{0.42\textwidth}
        \centering
        \includegraphics[width=\linewidth]{figures/images/train_rewards_zero_7B.pdf}
        \caption{Outcome training rewards (10-step moving).} % 第一个图形的子标题
        %\label{fig:train_rewards}
    \end{subfigure}
    \hfill % 在两个子图之间添加一些水平间距
    \begin{subfigure}{0.57\textwidth}
        \centering
        \includegraphics[width=\linewidth]{figures/images/test_accuracy_zero_7b.pdf}
        \caption{Math test accuracy across different gradient steps.} % 第二个图形的子标题
        %\label{fig:test_accuracy}
    \end{subfigure}
    \caption{\textbf{``Zero'' RL from Qwen2.5-Math-7B.} RL from the base model converges way faster than the SFT model, surpassing the instruct version within 32 steps.} % 整个图形的标题
    \label{fig:zero_7}
\end{figure*}
% \begin{figure*}[tbh]
    \centering
    \begin{subfigure}{0.48\textwidth}
        \centering
        \includegraphics[width=\linewidth]{figures/images/train_rewards_zero_32B.pdf}
        \caption{Outcome training rewards (10-step moving).} % 第一个图形的子标题
        %\label{fig:train_rewards}
    \end{subfigure}
    \hfill % 在两个子图之间添加一些水平间距
    \begin{subfigure}{0.50\textwidth}
        \centering
        \includegraphics[width=\linewidth]{figures/images/test_accuracy_zero_32b.pdf}
        \caption{Math test accuracy across different gradient steps.} % 第二个图形的子标题
        %\label{fig:test_accuracy}
    \end{subfigure}
    \caption{\textbf{``Zero'' RL from Qwen2.5-32B-Base.} RL from a 32B base model shows more promising gain, surpassing the instruct version within 16 steps. } % 整个图形的标题
    \label{fig:zero_32}
\end{figure*}


% \begin{figure}[tbh]
%     % \vspace{-5pt}
%     \centering
%     % \begin{subfigure}{\textwidth}
%     \includegraphics[width=0.43\textwidth]{figures/images/train_rewards.pdf}
%     % \hfill
%     \includegraphics[width=0.56\textwidth]{figures/images/test_accuracy.pdf}
%     \\
%     \centering
%     % \end{subfigure}
%     % \vspace{-12pt}
%     \caption{Title}
%     \label{fig:dense_rewards}
%     \vspace{-10pt}
% \end{figure}
% \citet{deepseekai2025deepseekr1incentivizingreasoningcapability} proposed DeepSeek-R1-Zero, which is directly trained from a base model with reinforcement learning. To further investigate the ``Zero'' setting, we also perform RL from Qwen2.5-Math-7B-Base and Qwen2.5-32B-Base~\citep{qwen2.5}, skipping the SFT phase. 
% We present the experimental results in Figure~\ref{fig:zero_7} and Figure~\ref{fig:zero_32}. The observations are as follows:

% (1) \textbf{RL from base model is suprisingly efficient and effective.} Comparing PRIME from Qwen2.5-Math-7B and Eurus-2-7B-SFT, the ``Zero'' setting converges much faster. This indicates that directly performing RL from a base model might be a strong alternative to the conventional SFT-RL pipeline

% (2) \textbf{Larger models benefit more.} Comparing 7B and 32B models, we see that the 32B model gains more on both training rewards and test performance. This is aligned with the conclusion in \citet{deepseekai2025deepseekr1incentivizingreasoningcapability}.

% (3) \textbf{Saturation could be a potential issue.} Although PRIME-Zero obtains impressive performance gain, we find it quickly saturated at a very early stage (about 50 steps), which hinders further improvement like in \citet{deepseekai2025deepseekr1incentivizingreasoningcapability}. This is possibly attributed to the decrease of response diversity, and we leave this as future work.



\section{SFT Data \& Training Details}
\label{sec:sft_data_training_details}
\begin{table}[t]
\centering
\caption{Actions in action-centric chain-of-thought reasoning framework.}
\label{tab:actions}
\resizebox{0.7\textwidth}{!}{

\begin{tabular}{ll}
\toprule
\textbf{Action Name} & \textbf{Description}                                                  \\ \midrule
\textbf{ASSESS}      & Analyze current situation, identify key elements and goals            \\
\textbf{ADVANCE}     & Move forward with reasoning - calculate, conclude, or form hypothesis \\
\textbf{VERIFY}      & Check accuracy of current approach, look for errors                   \\
\textbf{SIMPLIFY}    & Break complex problems into simpler parts                             \\
\textbf{SYNTHESIZE}  & Combine multiple pieces of information into complete solution         \\
\textbf{PIVOT}       & Change strategy when current approach isn't working                   \\
\textbf{OUTPUT}      & Summarize thought process and present final answer                    \\ \bottomrule
\end{tabular}
}
\end{table}
% Please add the following required packages to your document preamble:
% \usepackage{multirow}
\begin{table}[t]
\centering
\caption{Data statistics of SFT data.}
\label{tab:sft_data_stat}
\resizebox{1\textwidth}{!}{
\begin{tabular}{llccl}
\toprule
\textbf{Task}           & \textbf{Dataset}                   & \textbf{Size} & \textbf{Avg. Response Length} & \textbf{Source}                                                       \\ \midrule
\multirow{4}{*}{Math}   & MathInstruct-MATH~\citep{yue2023mammoth}                  & 12715  & 964.01               & \href{https://huggingface.co/datasets/TIGER-Lab/MathInstruct}{https://huggingface.co/datasets/TIGER-Lab/MathInstruct}               \\
                        & OpenMathIns-2-Aug\_Math~\citep{toshniwal2024openmath2} & 15086  & 1202.25              & \href{https://huggingface.co/datasets/nvidia/OpenMathInstruct-2}{https://huggingface.co/datasets/nvidia/OpenMathInstruct-2}             \\
                        & Numina~\citep{li2024numinamath}                             & 55845  & 1331.61              & \href{https://huggingface.co/datasets/AI-MO/NuminaMath-CoT}{https://huggingface.co/datasets/AI-MO/NuminaMath-CoT}                  \\
                        & Reasoning-001~\citep{reasoning001}                      & 29831  & 1316.49              & \href{https://huggingface.co/datasets/SkunkworksAI/reasoning-0.01}{https://huggingface.co/datasets/SkunkworksAI/reasoning-0.01}           \\ \midrule
\multirow{3}{*}{Coding} & Code-Feedback~\citep{zheng2024opencodeinterpreter}                      & 27663  & 1805.16              & \href{https://huggingface.co/datasets/m-a-p/Code-Feedback}{https://huggingface.co/datasets/m-a-p/Code-Feedback}                   \\
                        & Magicoder~\citep{wei2024magicoder}                          & 24480  & 1828.72              & \href{https://huggingface.co/datasets/ise-uiuc/Magicoder-Evol-Instruct-110K}{https://huggingface.co/datasets/ise-uiuc/Magicoder-Evol-Instruct-110K} \\
                        & Magicoder-OSS~\citep{wei2024magicoder}                      & 28980  & 1850.05              & \href{https://huggingface.co/datasets/ise-uiuc/Magicoder-OSS-Instruct-75K}{https://huggingface.co/datasets/ise-uiuc/Magicoder-OSS-Instruct-75K}   \\ \midrule
Biomedicine             & UltraMedical\_mc~\citep{zhang2024ultramedical}                   & 35163  & 891.06               & \href{https://huggingface.co/datasets/TsinghuaC3I/UltraMedical}{https://huggingface.co/datasets/TsinghuaC3I/UltraMedical}              \\ \midrule
Total / Avg.            & -                                  & 229763 & 1390.75              & -                                                                     \\ \bottomrule
\end{tabular}
}
\end{table}

We first perform supervised finetuning on the base model to get a starter model for RL. 

\textbf{Action-centric chain-of-thought reasoning.} We apply imitation learning (supervised finetuning) as a warmup stage to teach models to learn certain reasoning patterns. To this end, we first design an action-centric chain-of-thought reasoning framework.
Table \ref{tab:actions} shows the actions in the action-centric chain-of-thought reasoning framework. When the model generates answers, it conducts multi-step reasoning and chooses one of the 7 actions at each step. The response begins with the ASSESS action and ends with the OUTPUT action.

\textbf{Construction of the SFT dataset.} To construct the SFT dataset, we collect reasoning instructions from several open-source datasets. It is noteworthy that we did not include many datasets with ground-truth answers in SFT, even though they are of higher quality. However, we reserve them for later RL training. The reason is that we aim to use different datasets for SFT and RL to diversify the exploration in RL, and we consider ground-truth more essential in RL than in SFT.  For completion, we employ LLaMA-3.1-70B-Instruct to answer the instructions, with a system prompt requesting the model to perform an action-centric chain-of-thought. Table \ref{tab:sft_data_stat} summarizes the key statistics of the datasets used for SFT. The datasets span mathematics, coding, and biomedicine. We finally obtain 230K SFT data and the average response length is 1390 tokens.

\textbf{SFT Training.}  During the SFT phase, we conduct full parameter fine-tuning with a learning rate of 1e-05, utilizing the AdamW optimizer alongside a cosine annealing learning rate schedule and a warmup ratio of 0.1. The batch size was set to 96, with a fixed random seed of 42. The model was trained on 230K datasets for 3 epochs.









\section{RL Data Preprocessing}
\label{sec:rl_data_process}
\subsection{RL Data Collection and Preprocessing}

We curate a high-quality RL training dataset of mathematics and coding problems with outcome verifiers (LaTeX answers for math and test cases for coding). For math, we source from NuminaMath-CoT~\citep{li2024numinamath},  which contains about 860K math problems. The problems span from Chinese high school mathematics to International Mathematical Olympiad competition questions. For coding, we source from APPS~\citep{apps}, CodeContests~\citep{li2022competition}, TACO~\citep{li2023taco}, and Codeforces\footnote{\url{https://huggingface.co/datasets/MatrixStudio/Codeforces-Python-Submissions}}. To further increase data quality, we conduct detailed cleaning and filtering. Finally, we retain 457k math problems and 27k coding problems.

\subsection{Data Filtering and Question-Type Classification}

The preprocessing pipeline employs a systematic rule-based approach to filter and classify mathematical problems to create a high-quality dataset with solvable problems, appropriate difficulty levels, and correct solutions. We exclude problems containing figures or diagrams since they require visual processing capabilities. We also remove proof questions due to difficulties in answer verification. Based on specific patterns, the remaining problems are classified into question-answering, multiple-choice, or fill-in-the-blank questions. Since fill-in-the-blank questions comprise less than 400 examples compared to the much larger set of multiple-choice questions, we focus solely on multiple-choice questions for further processing.

\subsection{Converting to Direct Question-Answer Format}


We transform multiple-choice questions into a direct question-answer format through three sequential stages: rule-based filtering, LLM-based filtering, and LLM-based formatting.

We first identify and remove questions that inherently require multiple-choice options - specifically, those where comparing specific statements or properties is essential to the problem-solving process. These questions cannot be meaningfully converted to a direct question-answer format. The initial filtering employs simple rule-based pattern matching, searching for keywords like "following" and "statement" that typically indicate option-dependent problems.

Following the rule-based filtering, we employ Llama-3.1-8B-Instruct to perform a more nuanced classification of the remaining questions. Our pilot study revealed that while the LLM occasionally misclassifies questions, it tends to err on the conservative side - marking potentially convertible questions as requiring options rather than the reverse. Given our large dataset, we accepted this conservative approach to maintain quality.

For questions classified as convertible, we implement a two-phase reformatting process: 1) Question Reformatting: Removing choice indicators and restructuring the question to elicit direct answers. 2) Solution Reformatting: Converting multiple-choice solutions into step-by-step derivations, ensuring all final answers are presented in standard LaTeX boxed format. This systematic approach maintains mathematical rigor while creating a standardized format suitable for downstream applications.

\subsection{Problem and Solution Validation}

The final stage involves merging all question-answer pairs and performing LLM-based comprehensive validation. We identify two key aspects in validation: solvability and correctness.

We leverage state-of-the-art mathematical reasoning models, including QwQ-32B-Preview~\citep{qwq-32b-preview} and Qwen2.5-Math-72B-Instruct~\citep{yang2024qwen25mathtechnicalreportmathematical}, employing a self-consistency approach to determine problem solvability, and if solvable, verify the correctness of solutions provided in the original dataset.


To enhance validation accuracy, we first analyzed sample problems to identify characteristics of solvable and unsolvable cases and created synthetic unsolvable problems featuring missing conditions or logical contradictions. Based on these samples, we developed specialized prompts to improve the models' ability to distinguish solvability. Each problem undergoes five independent validation attempts, where the LLM: 1) Provides step-by-step solutions using LaTeX formatting. 2) Identifies unsolvability due to missing conditions or logical contradictions. 3) Generates complete reasoning traces for solvable problems. 4) Presents final answers in standardized LaTeX boxed format (\texttt{\textbackslash boxed\{...\}}). 5) Document any impediments to solution completion.



We evaluate two key consistency measures across multiple validation attempts: 1) Status Consistency: agreement on problem solvability. 2) Answer Consistency: consistency of solutions across different attempts and agreement between generated solutions and ground truth. The final dataset retains only problems that demonstrate consistent solvability across validation attempts, agreement in solutions across multiple attempts, and alignment with ground truth answers. This rigorous validation process ensures the resulting dataset comprises well-defined, solvable problems with verified, accurate solutions.



\subsection{PRM Data}
\label{sec:app_prm_data}
\begin{table}[t]
\centering
\caption{Data statistics of EurusPRM training dataset.}
\label{tab:stage1_data_stat}
\resizebox{.9\textwidth}{!}{

\begin{tabular}{llccc}
\toprule
\textbf{Dataset}                  & \textbf{Generator Model} & \textbf{Num. Inst} & \textbf{Resp/Inst} & \textbf{Step-level/Response-level} \\ \midrule
\multirow{4}{*}{UltraInteract}    & Llama-3.1-8B-Inst        & 20177              & 8                  & Response-level                     \\
                                  & Llama-3.1-8B-Base        & 13570              & 8                  & Response-level                     \\
                                  & Qwen2.5-72B-Inst         & 4758               & 8                  & Response-level                     \\
                                  & Qwen2.5-Math-7B-Base     & 25713              & 8                  & Response-level                     \\
\multirow{2}{*}{Numina-SynMath}   & Llama-3.1-8B-Inst        & 4783               & 8                  & Response-level                     \\
                                  & Qwen2.5-Math-7B-Base     & 5806               & 8                  & Response-level                     \\
\multirow{2}{*}{Numina-Olympiads} & Llama-3.1-8B-Inst        & 2909               & 8                  & Response-level                     \\
                                  & Qwen2.5-Math-7B-Base     & 4739               & 8                  & Response-level                     \\ \bottomrule
\end{tabular}

}
\end{table}
% \begin{table}[t]
\centering
\caption{Data statistics of PRM stage 2 training dataset.}
\label{tab:stage2_data_stat}
\resizebox{1\textwidth}{!}{

\begin{tabular}{llccc}
\toprule
\textbf{Dataset}      & \textbf{Generator Model} & \textbf{Num. Inst} & \textbf{Resp/Inst} & \textbf{Step-level/Response-level} \\ \midrule
\multirow{2}{*}{MATH} & Llama-3.1-70B-Inst       & 4715               & 2                  & Step-level                         \\
                      & Qwen2.5-72B-Inst         & 6098               & 2                  & Step-level                         \\
UltraInteract         & Llama-3.1-70B-Inst       & 4238               & 2                  & Response-level                     \\ \bottomrule
\end{tabular}

}
\end{table}
The dataset statistics of training EurusPRM are shown in Table~\ref{tab:stage1_data_stat}.