% \begin{wrapfigure}{r}{0.45\textwidth}  % r表示图片在右侧,0.5\textwidth表示图片宽度为文本宽度的50%
%     \centering
%     \includegraphics[width=\linewidth]{figures/images/train_rewards_ref.pdf}  % 插入图片,宽度设置为wrapfigure环境的宽度
%     \caption{Title.} 
%     \label{fig:effect_of_ref_policy}
% \end{wrapfigure}

\begin{wrapfigure}{r}{0.5\textwidth}
    \centering
    \vspace{-10pt}
    \includegraphics[width=\linewidth]{figures/images/train_rewards_ref.pdf}
    \caption{\textbf{Different reference model for PRM.} We compare two reference model selection strategies for PRIME. Using the policy model as reference and using the initial SFT model as reference. Their rewards are similar.}
    \label{fig:effect_of_ref_policy}
    \vspace{-25pt}
\end{wrapfigure}

\iffalse
\begin{minipage}{0.50\textwidth}
        \centering
        \begin{tabular}{lcc}
\toprule
\textbf{Step} & \textbf{SFT Ref} & \textbf{Policy Ref} \\ \midrule
80            & 36.8             & 36.7                \\
160           & 36.5             & 38.4                \\
240           & 41.0             & 40.5                \\
320           & 41.7             & 41.0                \\ \bottomrule
\end{tabular}
        \captionof{table}{Test Accuracy.\lifan{this is too small, remove from subfig}} 
        \label{tab:effect_of_ref_policy}
    \end{minipage}
\fi