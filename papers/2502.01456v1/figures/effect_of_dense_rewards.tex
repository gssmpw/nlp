\begin{figure*}[t]
    \centering
    \begin{subfigure}{0.42\textwidth}
        \centering
        \includegraphics[width=\linewidth]{figures/images/train_rewards.pdf}
        \caption{Outcome training rewards (10-step moving).} % 第一个图形的子标题
        %\label{fig:train_rewards}
    \end{subfigure}
    \hfill % 在两个子图之间添加一些水平间距
    \begin{subfigure}{0.57\textwidth}
        \centering
        \includegraphics[width=\linewidth]{figures/images/test_accuracy.pdf}
        \caption{Test accuracy across different gradient steps.} % 第二个图形的子标题
        %\label{fig:test_accuracy}
    \end{subfigure}
    \caption{
    % %\hao{these two figs' captions are a bit off.
    % what should be on the y-axis label is in the captions.
    % the captions should be more descriptive.
    % e.g., outcome rewards (10-step moving average) throughout training,
    % test accuracy across different gradient steps.
    % the two titles can be dropped
    % }
    \textbf{The effect of dense reward.} We compare PRIME and RLOO with outcome verifier (OV). Dense rewards in PRIME lead to \textbf{$2.5\times$} sample efficiency and \textbf{$6.9\%$} performance improvement. PRIME also substantially outperforms RLOO on downstream tasks.} % 整个图形的标题
    \label{fig:dense_rewards}
    \vspace{-10pt}
\end{figure*}


% \begin{figure}[tbh]
%     % \vspace{-5pt}
%     \centering
%     % \begin{subfigure}{\textwidth}
%     \includegraphics[width=0.43\textwidth]{figures/images/train_rewards.pdf}
%     % \hfill
%     \includegraphics[width=0.56\textwidth]{figures/images/test_accuracy.pdf}
%     \\
%     \centering
%     % \end{subfigure}
%     % \vspace{-12pt}
%     \caption{Title}
%     \label{fig:dense_rewards}
%     \vspace{-10pt}
% \end{figure}