%File: anonymous-submission-latex-2025.tex
%File: anonymous-submission-latex-2025.tex
\documentclass[letterpaper]{article} % DO NOT CHANGE THIS
%\usepackage[submission]{aaai25}  % DO NOT CHANGE THIS
\usepackage{aaai25}  % DO NOT CHANGE THIS

\usepackage{times}  % DO NOT CHANGE THIS
\usepackage{helvet}  % DO NOT CHANGE THIS
\usepackage{courier}  % DO NOT CHANGE THIS
\usepackage[hyphens]{url}  % DO NOT CHANGE THIS
\usepackage{graphicx} % DO NOT CHANGE THIS
\urlstyle{rm} % DO NOT CHANGE THIS
\def\UrlFont{\rm}  % DO NOT CHANGE THIS
\usepackage{natbib}  % DO NOT CHANGE THIS AND DO NOT ADD ANY OPTIONS TO IT
\usepackage{caption} % DO NOT CHANGE THIS AND DO NOT ADD ANY OPTIONS TO IT
\frenchspacing  % DO NOT CHANGE THIS
\setlength{\pdfpagewidth}{8.5in} % DO NOT CHANGE THIS
\setlength{\pdfpageheight}{11in} % DO NOT CHANGE THIS
%
% These are recommended to typeset algorithms but not required. See the subsubsection on algorithms. Remove them if you don't have algorithms in your paper.
\usepackage{algorithm}
\usepackage{algorithmic}
\usepackage{amsthm}
\usepackage{multirow}

%
% These are are recommended to typeset listings but not required. See the subsubsection on listing. Remove this block if you don't have listings in your paper.
\usepackage{newfloat}
\usepackage{listings}
\DeclareCaptionStyle{ruled}{labelfont=normalfont,labelsep=colon,strut=off} % DO NOT CHANGE THIS
\lstset{%
	basicstyle={\footnotesize\ttfamily},% footnotesize acceptable for monospace
	numbers=left,numberstyle=\footnotesize,xleftmargin=2em,% show line numbers, remove this entire line if you don't want the numbers.
	aboveskip=0pt,belowskip=0pt,%
	showstringspaces=false,tabsize=2,breaklines=true}
\floatstyle{ruled}
\newfloat{listing}{tb}{lst}{}
\floatname{listing}{Listing}
%
% Keep the \pdfinfo as shown here. There's no need
% for you to add the /Title and /Author tags.
\pdfinfo{
/TemplateVersion (2025.1)
}

% DISALLOWED PACKAGES
% \usepackage{authblk} -- This package is specifically forbidden
% \usepackage{balance} -- This package is specifically forbidden
% \usepackage{color (if used in text)
% \usepackage{CJK} -- This package is specifically forbidden
% \usepackage{float} -- This package is specifically forbidden
% \usepackage{flushend} -- This package is specifically forbidden
% \usepackage{fontenc} -- This package is specifically forbidden
% \usepackage{fullpage} -- This package is specifically forbidden
% \usepackage{geometry} -- This package is specifically forbidden
% \usepackage{grffile} -- This package is specifically forbidden
% \usepackage{hyperref} -- This package is specifically forbidden
% \usepackage{navigator} -- This package is specifically forbidden
% (or any other package that embeds links such as navigator or hyperref)
% \indentfirst} -- This package is specifically forbidden
% \layout} -- This package is specifically forbidden
% \multicol} -- This package is specifically forbidden
% \nameref} -- This package is specifically forbidden
% \usepackage{savetrees} -- This package is specifically forbidden
% \usepackage{setspace} -- This package is specifically forbidden
% \usepackage{stfloats} -- This package is specifically forbidden
% \usepackage{tabu} -- This package is specifically forbidden
% \usepackage{titlesec} -- This package is specifically forbidden
% \usepackage{tocbibind} -- This package is specifically forbidden
% \usepackage{ulem} -- This package is specifically forbidden
% \usepackage{wrapfig} -- This package is specifically forbidden
% DISALLOWED COMMANDS
% \nocopyright -- Your paper will not be published if you use this command
% \addtolength -- This command may not be used
% \balance -- This command may not be used
% \baselinestretch -- Your paper will not be published if you use this command
% \clearpage -- No page breaks of any kind may be used for the final version of your paper
% \columnsep -- This command may not be used
% \newpage -- No page breaks of any kind may be used for the final version of your paper
% \pagebreak -- No page breaks of any kind may be used for the final version of your paperr
% \pagestyle -- This command may not be used
% \tiny -- This is not an acceptable font size.
% \vspace{- -- No negative value may be used in proximity of a caption, figure, table, section, subsection, subsubsection, or reference
% \vskip{- -- No negative value may be used to alter spacing above or below a caption, figure, table, section, subsection, subsubsection, or reference
\newcommand{\thought}[1]{{\color[rgb]{0.2,0.39,0.66}(#1)}}
\newcommand{\todo}[1]{{\color[rgb]{1.0,0.0,0.0}(#1)}}
\newcommand{\hsh}[1]{{\color{green!50!black} Henrik: #1}}
\newcommand{\st}[1]{{\color{red!50!black} Sebastian: #1}}

\newcommand{\ulm}[1]{_{\scaleto{\mathrm{#1}}{3pt}}}
\newcommand\at[2]{\left.#1\right|_{#2}}











\newtheorem{assumption}{Assumption}

\DeclareMathOperator*{\argmax}{arg\,max}
\DeclareMathOperator*{\argmin}{arg\,min}

\newcommand{\swname}[1]{\texttt{#1}}
\newcommand{\ie}{i\/.\/e\/.,\/~}
\newcommand{\eg}{e\/.\/g\/.,\/~}
\newcommand{\cf}{cf\/.\/~}

\newcommand{\fig}{Fig\/.\/~}
\newcommand{\defn}{Def\/.\/~}
\newcommand{\sect}{Sec\/.\/~}
\newcommand{\tabl}{Tab\/.\/~}
\newcommand{\algo}{Algorithm~}
\newcommand{\theo}{Theorem~}

\newcommand{\bnnl}{3 hidden layers}
\newcommand{\bnnn}{50 neurons}
\newcommand{\bnna}{tanh activations}

\newcommand{\capt}[1]{\mdseries{\emph{#1}}}

\newcommand{\videolink}{at \url{https://youtu.be/_d7AqTRjz6g}}
\newcommand{\codelink}{\url{https://github.com/wheelbot/mini-wheelbot}}

\newcommand{\fakepar}[1]{\vspace{0mm}\noindent\textbf{#1.}}

\newcommand{\needref}{\textcolor{red}{[REF]}}

\newcommand{\plotfontsize}{9pt}




\setcounter{secnumdepth}{1} %May be changed to 1 or 2 if section numbers are desired.

% The file aaai25.sty is the style file for AAAI Press
% proceedings, working notes, and technical reports.
%

% Title

% Your title must be in mixed case, not sentence case.
% That means all verbs (including short verbs like be, is, using,and go),
% nouns, adverbs, adjectives should be capitalized, including both words in hyphenated terms, while
% articles, conjunctions, and prepositions are lower case unless they
% directly follow a colon or long dash
\title{Probabilistic Foundations for Metacognition via Hybrid-AI}
\author {
    Paulo Shakarian\textsuperscript{\rm 1},
    Gerardo I. Simari\textsuperscript{\rm 2},
    Nathaniel D. Bastian\textsuperscript{\rm 3}
}
\affiliations {
    % Affiliations
    \textsuperscript{\rm 1}Arizona State University, Tempe, AZ USA\\
    \textsuperscript{\rm 2} Departament of Computer Science and Engineering, Universidad Nacional del Sur (UNS) \& \\
    Institute for Computer Science and Engineering (UNS-CONICET), Bahía Blanca, Argentina\\
    \textsuperscript{\rm 2}United States Military Academy, West Point, NY USA\\

    pshak02@asu.edu, gis@cs.uns.edu.ar, nathaniel.bastian@westpoint.edu
}



%Example, Single Author, ->> remove \iffalse,\fi and place them surrounding AAAI title to use it
\iffalse
\title{My Publication Title --- Single Author}
\author {
    Author Name
}
\affiliations{
    Affiliation\\
    Affiliation Line 2\\
    name@example.com
}
\fi

\iffalse
%Example, Multiple Authors, ->> remove \iffalse,\fi and place them surrounding AAAI title to use it
\title{My Publication Title --- Multiple Authors}
\author {
    % Authors
    First Author Name\textsuperscript{\rm 1},
    Second Author Name\textsuperscript{\rm 2},
    Third Author Name\textsuperscript{\rm 1}
}
\affiliations {
    % Affiliations
    \textsuperscript{\rm 1}Affiliation 1\\
    \textsuperscript{\rm 2}Affiliation 2\\
    firstAuthor@affiliation1.com, secondAuthor@affilation2.com, thirdAuthor@affiliation1.com
}
\fi


% REMOVE THIS: bibentry
% This is only needed to show inline citations in the guidelines document. You should not need it and can safely delete it.
\usepackage{bibentry}
% END REMOVE bibentry

\begin{document}

\maketitle

\begin{abstract}
Metacognition is the concept of reasoning about an agent's own internal processes, and it has recently received renewed attention with respect to artificial intelligence (AI) and, more specifically, machine learning systems.  This paper reviews a hybrid-AI approach known as ``error detecting and correcting rules'' (EDCR) that allows for the learning of rules to correct perceptual (e.g., neural) models.  Additionally, we introduce a probabilistic framework that adds rigor to prior empirical studies, and we use this framework to prove results on necessary and sufficient conditions for metacognitive improvement, as well as limits to the approach.  A set of future research directions is also provided.
\end{abstract}

% Uncomment the following to link to your code, datasets, an extended version or similar.
%
% \begin{links}
%     \link{Code}{https://aaai.org/example/code}
%     \link{Datasets}{https://aaai.org/example/datasets}
%     \link{Extended version}{https://aaai.org/example/extended-version}
% \end{links}

\section{Introduction}

Originally a concept from developmental psychology~\cite{flavell1979metacognition}, metacognition refers to reasoning about an agent’s own internal processes~\cite{wei24}.  This idea of metacognition, considered the human brain's ``self-monitoring process''~\cite{demetriou1993architecture}, has been studied in a diverse set of fields~\cite{li2017applications,izzo2019survey,caesar2020nuscenes}.  Over the years, its study has been proposed in the field of AI~\cite{cox2011metareasoning,cox2005metacognition}, and recent interest~\cite{wei24} has reinvigorated this discussion.\footnote{See recent workshop proceedings at~\url{https://neurosymbolic.asu.edu/metacognition/}.}
In this paper, we review a recent, hybrid-AI style of metacognition known as ``error detection and correction rules'' (EDCR) that has been explored in a variety of models and use cases (see Table~\ref{tab:summary}). 
In short, the proposal is a hybrid-AI approach in which logical rules are learned to characterize the model performance of well-trained perceptual (e.g., neural) models.  This paper serves as a review of the results of these recent articles, and improves the underlying theoretical underpinnings by framing them in terms of a probabilistic argument, providing new insights into the use of hybrid-AI for metacognition.  
In particular, our novel theoretical framework allows us to explain some empirically observed results, as well as leads us to several new open research questions.

\section{Error Detection and Correction Rules}

In this framework, we are given a well-trained model, normally denoted as $f$, that takes some continuous input (e.g., a vector) and returns a set of class labels.  We will use subscripts (e.g., $f_i$ or $f_1$) to denote multiple models when relevant.  We note that the model need not be neural (though all existing studies have focused on neural models), and in general we do not assume access to the model weights.  We now introduce a small running example.

\begin{example}
\label{ex:1}
Let $f_{car}$ be a neural model trained on data set $D_{tng}$ such that given a sample (that we denote $x$) returns a subset of labels $\{\ford,\toyota,\dodge,\us,\japan\}$.
So, for example, for sample $\omega$, $f_{car}(\omega)=\{\dodge,\us\}$.
\end{example}

\begin{table}
\begin{center}
\begin{tiny}
    \begin{tabular}{|l|l|l|l|}
    \hline
        \multirow{2}{*}{\textbf{Paper}} & \multirow{2}{*}{\textbf{Use case}} &  \textbf{Base model}   & \textbf{Source of} \\
                                     &                   & \textbf{architectures} & \textbf{conditions} \\
        \hline
        \hline
        \citet{kri24} & Vision & ViT & Hier. class  \\
                     &        &     & labels, other \\
        &&& models\\
        \hline
        \citet{xi24} & Trajectory  & CNN, LRCN, & Dom. kn, \\
         &classification & LRCNa & other models\\
        \hline
        \citet{lee24} & Time series    & CNN, RNN, & Other models\\
                     & classification & Attention     & \\
        \hline
    \end{tabular}
\end{tiny}
\caption{Summary of existing literature on EDCR.}  
\label{tab:summary}
\end{center}
\end{table}

The work on EDCR introduces the notion of a \textit{metacognitive condition}, which is some aspect of the environment, sensor, model, or metadata that may cause the model to be incorrect in a given circumstance.  In the existing work, the metacognitive condition (which we refer to as ``condition'', for short) has come from domain knowledge, model output (e.g,, labels at different levels of a hierarchy), or other models (e.g., different architectures, trained on different data) -- this is summarized in Table~\ref{tab:summary}.  
This leads us to the first type of rule, the error detecting rule, which is expressed in a first order logic syntax as follows:
\begin{eqnarray}
\mathit{error}^i_\alpha(X) \leftarrow \mathit{pred}_\alpha^i(X) \wedge \bigvee_{j\in \mathit{DC}_i}\mathit{cond}_j(X)
\end{eqnarray}
In words, this rule states that if any of the conditions in set $\textit{DC}_i$ are met, and model $i$ predicts class $\alpha$, then the model has an erroneous detection.  We note that we can replace the disjunction with a singleton and have multiple rules for a given model-class pair (i.e., using a logic program).  The second type of rule is a correction rule, which has the following syntax:
\begin{eqnarray}
\mathit{corr}_\beta^i(X) \leftarrow \bigvee_{j,\alpha\in \mathit{CC}_\beta}\left(\mathit{cond}_j(X) \wedge \mathit{pred}_\alpha^i(X) \right)
\end{eqnarray}
In words, this rule states that given a set of condition-class pairs relevant to class $\beta$ (denoted $\mathit{CC}_\beta$), for any of those pairs (e.g., model $i$ predicting $\alpha$ for the sample while condition $j$ is true), the rule dictates that the sample should be relabeled with label $\beta$.  
In both~\citet{xi24} and~\citet{lee24}, the detection step happens first (in both training and inference), allowing for samples to have some labels ``erased'' and only samples with erased labels will have new labels assigned by a correction rule.

\begin{example}
\label{ex:2}
Consider the model $f_{\mathit{car}}$ from Example~\ref{ex:1}.  Suppose we have a detection rule of the form:
\begin{eqnarray*}
\mathit{error}^\car_{\toyota}(X) \leftarrow \mathit{pred}_\toyota^\car(X) \wedge \mathit{cond}_{\us}(X)
\end{eqnarray*}
Here we have a single condition $\mathit{cond}_\us$, and we define it to be true for sample $x$ when $\us \in f_\car(x)$.
This is an example of using a class label from a different level of the hierarchy, as done in~\cite{kri24}.
Likewise, we can imagine a detection rule:
\begin{eqnarray*}
\mathit{corr}^\car_{\dodge}(X) \leftarrow \mathit{cond}_{\us}(X) \wedge \mathit{pred}_\toyota^\car(X)
\end{eqnarray*}
Here, if the condition-class pair $\mathit{cond}_\us$ and $\toyota$ are true, then the sample should be re-labeled as $\dodge$.
\end{example}

In Example~\ref{ex:2}, one could imagine having a priori knowledge that predicting $\japan$ and $\us$ together would yield an inconsistency.  
However, in such a case, it may make more sense to define the condition as the case where $\{\japan,\us\}\subseteq \fcar(x)$.  In fact, in prior work, the primary knowledge engineering task is to design the conditions, not the rules, which are learned from data, primarily using combinatorial techniques.
For example, in~\citet{xi24} detection rules are learned via submodular maximization of precision while constraining recall reduction,
while in~\citet{kri24} detection rules are learned via submodular ratio maximization to optimize toward F1.  
We also note that the detection rule of Example~\ref{ex:2} can also function as a constraint on class relationships, in this case meaning that $\us$ and $\toyota$ cannot go together.  This turns out to be useful, as the rules are already expressed in logic.  This enabled~\citet{kri24} to take learned detection rules and retrain a model with an LTN~\cite{ltn22} loss function to gain improved model performance.


\section{Probabilistic Interpretation and Results}

We now present a new probabilistic interpretation of EDCR.  First, we introduce a bit of notation: when the object in question is understood, we shall use the notation $f_i$ to refer to the set of propositions produced by model $f_i$ on the object.  So, for example, $\lefi$ means that model $f_i$ predicted label $\alpha$.  We shall also use the notation $\gt$ to denote a set of ground truth labels for the object; so, the statement $\lefi,\leng$ means that model $f_i$ incorrectly predicted class $\alpha$ for the object.  With this in mind, we can represent the precision and recall for a model as the following conditional probabilities.
\begin{eqnarray}
\textit{Precision:} & P_\alpha = \Pr(\legi \; | \; \lefi)\\
\textit{Recall:}    & R_\alpha = \Pr(\lefi \; | \; \legi)
\end{eqnarray}
We will also consider a random variable $D$ that specifies a specific distribution.  We can consider conditional probabilities with an unspecified distribution (as shown above) to be approximated from training data, while setting $D$ to be something specific would signify being out-of-distribution: e.g., the precision for model $i$ on class $\alpha$ under distribution~$d$ is written as $\Pr(\legi \; | \; \lefi,\d)$.
In what follows, we shall use $\m$ to denote the set of metacognitive conditions.  We note that this framework allows us to precisely define what it means for a condition to be error detecting, and introduce a new, formal definition to that effect.

\begin{definition}[Error Detecting]
\label{def:det}
A metacognitive condition~$c$ is \emph{error detecting} with respect to model $i$, class $\alpha$, and distribution $d$ if 
$\Pr(\legi \; | \; \lefi,\c,\d)$ is less than or equal to $\Pr(\legi \; | \; \lefi,\d)$.
\end{definition}

Intuitively, a condition is error detecting if the precision of the model drops when the condition is true.  The second property is distribution invariance.

\begin{definition}[Distribution Invariance]
A metacognitive condition $c$ is \emph{distribution invariant} with respect to model~$i$, class~$\alpha$, and set of distributions $\mathcal{D}$ if for any distributions $d \in \mathcal{D}$ it is error detecting with respect to that distribution.
\end{definition}

At first glance, distribution invariance seems to be a strong definition, and some may even think the set of conditions meeting that criteria would be small.  However, consider conditions based on constraints among classes; e.g., inconsistent assignments in a multi-class classification problem is one such case, and the configuration of the sensor is another.

With conditions in mind, we define precision and recall after applying a metacognitive condition:
\begin{eqnarray}
\textit{Precision:} & P_\alpha^c = \Pr(\legi \; | \lefi,\notc)\\
\textit{Recall:}    & R_\alpha^c = \Pr(\lefi,\notc \; | \; \legi)
\end{eqnarray}

Our first result shows how much the precision changes after applying a metacognitive condition.  The argument in this paper characterizes the change in precision using a probabilistic argument and, notably, the result is obtained without any assumptions of independence.

\begin{theorem}[Metacognitive Precision Change]
\label{thm:1}
The following identity holds:
\[
P_\alpha^{c}-P_\alpha=K\times\big (\Pr(\alpha \notin \gt \; | \; \lefi, \c)-(1-P_\alpha)\big),
\]
where $K=\frac{\Pr(\c \; | \; \lefi)}{\Pr(\notc \; | \; \lefi)}$
\end{theorem}

In~\citet{xi24}, an analogous result is shown. Both results suggest finding conditions that attempt to maximize the product of probabilities 
$\Pr(\c \; | \; \lefi)$ and $\Pr(\alpha \notin \gt \; | \; \lefi, \c)$ is desirable for precision improvement by ``erasing'' labels 
-- these correspond to support and confidence in that paper.  
It also turns out that this product has computationally desirable properties, as it is submodular (proven in~\citet{xi24}).  However, we point out that the result of~\citet{xi24} did not frame the preliminaries in terms of probability, and hence it was not clear if there were latent assumptions; this probabilistic interpretation and the corresponding proof (in the appendix) strengthen the results, as now they clearly do not rely on independence.  Further, we obtain additional insights built on this result discussed throughout this paper, and one notable insight is that we immediately obtain a necessary and  sufficient condition for obtaining improvement in precision:
\begin{eqnarray}
\Pr(\alpha \notin \gt \; | \; \lefi, \c) > 1 - P_\alpha
\end{eqnarray}
We can think of $1-P_\alpha$ as the residual of the model -- how much room it has to improve precision.  We can also think of this result as having a practical application in determining if a condition is invariant.  
For example, we would likely assume that $1-P_\alpha$ increases for out-of-distribution samples, and perhaps we could determine some $\epsilon$ where for some $\d$ that $\Pr(\alpha \notin \gt \; | \; \lefi, \c,\d)$ is within $\epsilon$ of confidence $\Pr(\alpha \notin \gt \; | \; \lefi, \c)$.  So, for example, if we identify the confidence value by examining multiple samples, and obtain an error, we can determine up to which value of $1-P_\alpha$ the condition is expected to be invariant.  Another aspect of this result is that we can also prove that our error-detecting definition (Definition~\ref{def:det}) is an equivalent condition.

\begin{theorem}[Error Detection is Necessary and Sufficient]
\label{thm:edns}
Condition $c$ is error detecting iff $P_\alpha^c  \geq P_\alpha$.
\end{theorem}
\begin{proof}
\noindent$(\rightarrow)$ By way of contradiction, assume $c$ is error detecting and $P_\alpha^c  < P_\alpha$, which gives us:
\begin{small}
\begin{eqnarray*}
&\Pr(\lefi)\Pr(\leg,\lefi,\notc) <\\
&\,\,\,\,\,\,\,\,\,\,\,\Pr(\leg,\lefi)\Pr(\lefi,\notc)
\end{eqnarray*}
\begin{eqnarray*}
&\Pr(\lefi)(\Pr(\leg,\lefi)-\\
&\,\,\,\,\,\,\,\,\,\,\,\Pr(\leg,\lefi,\c)) <\\
&\,\,\,\,\,\,\,\,\,\,\,\Pr(\leg,\lefi)(\Pr(\lefi)-\Pr(\lefi,\c))
\end{eqnarray*}
\begin{eqnarray*}
&\Pr(\lefi)(-\Pr(\leg,\lefi,\c)) <\\
&\,\,\,\,\,\,\,\,\,\,\,\Pr(\leg,\lefi)(-\Pr(\lefi,\c))
\end{eqnarray*}
% \begin{eqnarray*}
% &\Pr(\leg,\lefi)(\Pr(\lefi,\c)) < \\
% &\,\,\,\,\,\,\,\,\,\,\,\Pr(\lefi)(\Pr(\leg,\lefi,\c))
% \end{eqnarray*}
\begin{eqnarray*}
&\Pr(\leg \; | \; \lefi) < \Pr(\leg|\lefi,\c)
\end{eqnarray*}
\end{small}

However, this contradicts the definition of error detecting.  The remaining part of the proof is in the appendix.
\end{proof}

\noindent\textbf{Characterization of recall.} 
Next, we look at the change in recall. The recall prior to disregarding the classification for a prediction is
$\Pr(\lef \; | \; \leg)$.  Once we apply the condition, the recall for class $i$ becomes $\Pr(\lef,\notc \; | \; \leg)$.
Note that in doing so, we are essentially turning off model predictions, which can only reduce recall.  We can show an equivalent value to the reduction in recall with the following result.  Again, there is an analogous result in~\citet{xi24}; however our proof (in the appendix) presents a probabilistic argument and the proof clearly illustrates that no independence assumptions are made.

\begin{theorem}[Recall Reduction]
\label{thm:recall}
$\Pr(\lef \; | \; \leg) - \Pr(\lef, \notc \; | \; \leg) =$
\begin{scriptsize}
\[
\Pr(\leg \; | \; \lef,\c)\Pr(\c \; | \; \lef)\frac{\Pr(\lef \; | \; \leg)}{\Pr(\leg \; | \; \lef)}.
\]
\end{scriptsize}
\end{theorem}
\begin{proof}
\begin{small}
\begin{eqnarray*}
\Pr(\lef | \leg)- \Pr(\lef, \notc | \leg)\\
=\Pr(\lef,\c | \leg)\\
=\frac{\Pr(\lef,\c,\leg)}{\Pr(\leg)}\times\frac{\Pr(\lef,\c)}{\Pr(\lef,\c)}\\
=\frac{\Pr(\leg|\lef,\c)}{\Pr(\leg)}\times\frac{\Pr(\lef,\c)}{1}
\times\frac{\Pr(\lef)}{\Pr(\lef)}\\
%=\frac{\Pr(\leg|\lef,\c)}{\Pr(\leg)}\times\frac{\Pr(\c|\lef)}{1}
%\times\frac{\Pr(\lef)}{1}\\
=\frac{\Pr(\leg|\lef,\c)}{1}\times\frac{\Pr(\c|\lef)}{1}
\times\frac{\Pr(\lef)}{\Pr(\leg)}%\\
%=\frac{\Pr(\leg|\lef,\c)}{1}\times\frac{\Pr(\c|\lef)}{1}
%\times\frac{\Pr(\lef|\leg)}{\Pr(\leg|\lef)}
\end{eqnarray*}
\end{small}
\end{proof}



Here we see that the primary driver of recall reduction is $\Pr(\leg \; | \; \lef,\c)$, which is the probability of the model obtaining the correct answer under the metacognitive condition.  Likewise, the other term dependent on the conditions that impacts the reduction in recall is $\Pr(\c \; | \; \lef)$, which is the probability of a condition occurring with a prediction.  Note that both the decrease in recall and the increase in precision trend with this quantity. In the next section, on limitations, we shall understand how this quantity can be bounded.

\section{Limitations of Metacognitive Conditions}

We now explore some of the limits of our approach to metacognitive improvement.  Anecdotally, we noticed in prior metacognitive applications of EDCR that often detection seems easier than correction.  
In~\citet{xi24} and~\citet{lee24}, correction was conducted by changing the class label resulting from a condition-class pair that led to an error.  In the notation of this paper, such a precondition would be ``$\lefi,\c$'', meaning the condition of the model predicting class $\alpha$ while condition $c$ is also true.  This allows us to overcome the detection-induced deficit (a consequence of Theorem~\ref{thm:recall}), which was demonstrated empirically in~\citet{lee24} where such metacognitive correction could ensemble rules to directly improve recall over single-model baselines.  
However, it was less effective in improving recall in the experiments of~\citet{xi24}, where the use case consisted of five classes.  To understand why this occurs, consider a model that cannot distinguish between a set of samples, all classified under condition $c$.  A well-trained model assigns class $i$, the most probable class by training, but the probability of $i$ being correct is lower than the average precision for predictions of class $i$.  However, the next most probable class, $j$, is lower still, and picking this would lower overall loss.  As a result, without another condition or something else to distinguish these samples, the model and the metacognitive correction cannot re-assign those samples a new label that improves overall performance, while at the same time effectively identifying a case where the model had an error with a high probability.  Consider the following:

\begin{example}
\label{ex:3}
Consider the scenario from Example~\ref{ex:2}.  Suppose in cases where the error is observed (both classes $\toyota$ and $\us$ are predicted) that despite $\dodge$ being the ``best'' correction of class $\toyota$ we have the following:
\begin{eqnarray*}
\Pr(\dodge \in \fcar \; | \; \toyota \in \fcar, \us \in \fcar)\\
\leq \Pr(\dodge \in \fcar \; | \; \toyota \in \fcar)
\end{eqnarray*}
\end{example}

It turns out that the situation in Example~\ref{ex:3} leads to a reduction in precision for the class by which the label is re-assigned.  
We now formalize this argument.  Intuitively, if the precision for class $j$ conditioned on reclassifying items originally classified as $i$ under condition $c$ is lower, then the overall precision for classification of class $j$ will drop.

\begin{theorem}[Limits of Reclassification]
If we have that $\Pr(\legj \; | \; \lefi,\ci)\leq \Pr(\legj \;| \;\lefj)$, then:
\begin{small}
\[
\Pr(\legj \; | \; \lefj) \geq \Pr(\legj \; | \; \lefj \vee (\lefi,\ci).
\]
\end{small}
\end{theorem}
\begin{proof}
By way of contradiction, assume:\\[2pt]
\begin{small}
$\Pr(\legj \; | \; \lefj) < \Pr(\legj \; | \; \lefj \vee (\lefi,\ci)$
\begin{eqnarray*}
\Pr(\legj \; | \; \lefj) < \Pr(\legj \; | \; \lefj \vee (\lefi,\ci)\\[4pt]
% \frac{\Pr(\legj,\lefj)}{\Pr(\lefj)}
% <\\
% \,\,\,\,\,\frac{\Pr(\legj,\lefj)+\Pr(\legj,\lefi,\ci)}{\Pr(\lefj)+\Pr(\lefi,\ci)}\\
% \frac{\Pr(\lefj)+\Pr(\lefi,\ci)}{\Pr(\lefj)}<\\
% \,\,\,\,\,\frac{\Pr(\legj,\lefj)+\Pr(\legj,\lefi,\ci)}{\Pr(\legj,\lefj)}\\
\frac{\Pr(\lefi,\ci)}{\Pr(\lefj)}<\frac{\Pr(\legj,\lefi,\ci)}{\Pr(\legj,\lefj)}\\[4pt]
\frac{\Pr(\legj,\lefj)}{\Pr(\lefj)}<\frac{\Pr(\legj,\lefi,\ci)}{\Pr(\lefi,\ci)}\\[4pt]
\Pr(\legj \; | \;\lefj)<\Pr(\legj \; | \; \lefi,\ci),
\end{eqnarray*}
\end{small}
which contradicts the statement of the theorem.
\end{proof}
We also note there are other potential limits on reclassification due to EDCR.  Specifically, if rules are learned in a manner that allows for inconsistencies (which would be possible if rules are learned among different models independently) then proper corrective action becomes less clear.  This is an active area of inquiry.

\smallskip
\noindent\textbf{Limitations to precision improvement/recall reduction.}
In another new result from our analysis, we show we can bound the quantity $\Pr(\c \; | \; \lefi)$, which as pointed out earlier can magnify or suppress the amount of change in precision, or amplify the reduction in recall based on the prevalence of the metacognitive condition.  Specifically, this is bounded by $\Pr(\c \; | \;\leng,\lefi)$ when $c$ is error-detecting (shown in the next corollary).  
The practical application of the result, when identifying a metacognitive condition, is that we can understand the power of such a condition by only analyzing a subset of a dataset where the model was correct, which may have implications for large-scale data analytics and imbalanced classification problems.  Note that we never assume $\c$ is independent of $\leng$ or $\lefi$ -- this result stems directly from the assumption that $c$ is error detecting.


\begin{corollary}
If $c$ is error detecting, then:
\[
\Pr(\c \;| \; \lefi) \leq \Pr(\c \;| \;\leng,\lefi) 
\]
\end{corollary}
\begin{proof}
By definition of error detecting, we have:
\begin{eqnarray}
\Pr(\leng \; | \;\lefi\c)\geq \Pr(\leng \; | \; \lefi)\\[4pt]
% = \frac{\Pr(\leng,\lefi)}{\Pr(\lefi)}\times\\
% \,\,\,\,\,\,\,\,\,\,\,\,\,\,\,\,\,\frac{\Pr(\leng,\lefi,\c)\Pr(\lefi,c)}{\Pr(\leng,\lefi,\c)\Pr(\lefi,c)}\\
= \frac{\Pr(\c \; | \;\lefi)\Pr(\leng \; | \; \lefi,\c)}{\Pr(\c \; | \; \leng,\lefi)}.
\end{eqnarray}
This in turn gives us:
\begin{eqnarray}
% \Pr(\leng|\lefi\c)\geq\\
% \,\,\,\,\,\frac{\Pr(\c|\lefi)\Pr(\leng|\lefi,\c)}{\Pr(\c|\leng\,lefi)}\\
1\geq  \frac{\Pr(\c \; | \; \lefi)}{\Pr(\c \; | \; \leng,\lefi)}\\
\Pr(\c \; | \; \leng,\lefi) \geq \Pr(\c \; | \; \lefi),
\end{eqnarray}
thus completing the proof.
\end{proof}


\section{Conclusion: \\
Directions for Future Research}

In this paper, we reviewed a hybrid-AI technique for metacognition known as EDCR, and provided a probabilistic argument that supports previous findings.  This also suggests future research directions, so we end the paper with a discussion of some of these potential areas of inquiry.

\smallskip
\noindent\textbf{Role of Domain Knowledge and Inconsistency.}  
In Example~\ref{ex:2} we showed how metacognitive conditions can be used to identify inconsistencies, and the work of~\citet{kri24} demonstrates this empirically by both recovering latent constraints and using that to improve model loss via LTN.  When we view conditions from the standpoint of distribution invariance, logical inconsistency clearly meets these criteria.  This raises an interesting question: can consistency with domain knowledge be used for error correcting?  
Neurosymbolic approaches already use this concept to reduce training loss ~\cite{10637618}. However, understanding how consistency can be used for metacognitive correction remains an open question.

\smallskip
\noindent\textbf{Multi-model / Multi-modal reasoning.}  
The results of this paper suggest that EDCR can be most effective for improving model precision while sacrificing recall (Theorems~\ref{thm:1} and~\ref{thm:recall}).  The results of~\citet{lee24} essentially leverage this property to ensemble models together, as different models can be precise and their combined recall will lead to an increase if they are identifying different phenomena.  The ensembling of multi-modal models through EDCR seems to be a natural fit, as models of different modalities likely exhibit complementary capabilities.  However, work like~\citet{lee24} assumes one mode is the ``base model'' while the others are used to correct it.  The study of EDCR (or other metacognitive methods) to ensemble two or more models without a designed ``base model'' remains an open question.

\smallskip
\noindent\textbf{Online metacognition and data efficiency.}  Quantities such as $\Pr(\legi \; | \lefi, \c)$ and $\Pr(\c \; | \;  \lefi)$ (which are analogous to confidence and support) are typical byproducts of metacognitive rule learning under EDCR, and as seen in this paper are key quantities for validating invariance of conditions (Theorem~\ref{thm:edns}).  Likewise, an experiment in~\citet{xi24} shows how rules can be learned from a new distribution that are effective with a small portion of data.  Together, can these results suggest rapid adoption to a new distribution of data by identifying conditions on the fly (i.e., online learning of metacognitive conditions)?

\section*{Acknowledgments}
This research was supported by the Defense Advanced Research Projects Agency (DARPA) under Cooperative Agreement No. HR00112420370, the U.S. Army Combat Capabilities Development Command (DEVCOM) Army Research Office under Grant No. W911NF-24-1-0007, and the U.S. Army DEVCOM Army Research Lab under Support Agreement No. USMA 21050. The views expressed in this paper are those of the authors and do not reflect the official policy or position of the U.S. Military Academy, the U.S. Army, the U.S. Department of Defense, or the U.S. Government.

\pagebreak
% \section*{Appendix: Mathematical Proofs}
% An appendix with additional proofs can be found at \\ \url{https://neurosymbolic.asu.edu/metacognition/}.

\small
%\bibliography{aaai25}

\documentclass[conference, 9pt]{IEEEtran}

\usepackage[normalem]{ulem}
\usepackage{amsmath}
\usepackage{amsfonts}
\usepackage{hyperref}
\usepackage{graphicx}
\usepackage{cleveref}
\usepackage{amsthm}
\usepackage{algorithm, algpseudocode,subcaption}
\usepackage{xcolor}
\usepackage{tikz}
\usepackage{caption}
\usepackage{listings}
\usepackage{array}
\usepackage{booktabs}
\usepackage{diagbox}
\usepackage{float}
\usepackage{geometry}
\geometry{a4paper,margin=0.5in}
\newtheorem{condition}{Condition}
\newtheorem{claim}{Claim}
\newtheorem{example}{Example} 
\newtheorem{theorem}{Theorem}
\newtheorem{lemma}{Lemma} 
\newtheorem{proposition}{Proposition} 
\newtheorem{remark}{Remark}
\newtheorem{corollary}{Corollary}
\newtheorem{definition}{Definition}
\newtheorem{conjecture}{Conjecture}
\newtheorem{axiom}{Axiom}
\title{Robust Anomaly Detection via Tensor Pseudoskeleton Decomposition}
\author{Bowen Su }
\definecolor{officegreen}{rgb}{0.0, 0.5, 0.0}
\lstset{
    language=Python,
    basicstyle=\ttfamily\small,
    commentstyle=\color{blue},
    keywordstyle=\color{black},
    showstringspaces=false,
    numbers=left,
    numberstyle=\tiny,
    stepnumber=1,
    numbersep=5pt,
}



\begin{document}

\maketitle

\begin{abstract}
Anomaly detection plays a critical role in modern data-driven applications, from identifying fraudulent transactions and safeguarding network infrastructure to monitoring sensor systems for irregular patterns. Traditional approaches—such as distance-, density-, or cluster-based methods, face significant challenges when applied to high-dimensional tensor data, where complex interdependencies across dimensions amplify noise and computational complexity. To address these limitations, this paper leverages Tensor  pseudoskeleton decomposition within a tensor-robust principal component analysis framework to extract low-Tucker-rank structure while isolating sparse anomalies, ensuring robustness to anomaly detection. We establish theoretical analysis of convergence, and estimation error, demonstrating the stability and accuracy of the proposed approach. Numerical experiments on real-world spatiotemporal data from New York City taxi trip records validate the superiority of the proposed method in detecting anomalous urban events compared to existing benchmark methods. The results underscore the potential of Tensor  pseudoskeleton decomposition to enhance anomaly detection for large-scale, high-dimensional data.
\end{abstract}
\section{Introduction}
Anomaly detection is a crucial task in data analysis, with applications spanning various domains such as fraud detection~\cite{motie2024financial}, cybersecurity~\cite{wurzenberger2024analysis}, healthcare monitoring~\cite{kadir2024anomaly}, and sensor network analysis~\cite{tarish2025anomaly}. Anomalies, or outliers, represent data points or patterns that deviate significantly from the expected behavior, often signaling critical events or errors that require immediate attention. Detecting these anomalies, especially within high-dimensional and complex datasets, is challenging due to the sheer volume of data and the underlying noise that can mask unusual patterns.

Traditional anomaly detection techniques, including distance-based~\cite{Angiulli2002}, density-based~\cite{Breunig2000}, and clustering-based methods~\cite{Jiang2003,Hautamaki2004}, have shown some success in identifying anomalies in lower-dimensional datasets.
 However, these approaches often struggle when extended to high-dimensional tensor data, where intricate dependencies exist across multiple dimensions. Tensor data structures are common in fields such as video surveillance, biomedical imaging, and environmental monitoring, where data is naturally organized in multi-way arrays. The increased dimensionality not only complicates the detection of anomalies but also amplifies the computational costs, making scalability a critical concern.

In recent years, tensor decomposition methods have emerged as powerful tools for managing high-dimensional data. By transforming complex data into a lower-dimensional, interpretable form, tensor decompositions facilitate efficient storage, processing, and analysis. Among these methods, Tucker decomposition, a form of higher-order singular value decomposition, is particularly effective at capturing the core structure of tensor data. However, while Tucker decomposition enables significant dimensionality reduction, it remains sensitive to outliers, which can distort the decomposition and lead to unreliable results in anomaly detection.

To address these limitations,  Tensor  pseudoskeleton decomposition offers an alternative approach by selecting representative parts of the data, thereby preserving essential features while reducing redundancy.  Tucker pseudoskeleton decomposition provides a structured decomposition that is both computationally efficient and robust~\cite{hamm2023generalized,cai2021mode}. 

In this paper, we focus on anomaly detection within the tensor robust principal component analysis framework by leveraging a Tucker pseudoskeleton decomposition specifically tailored for high-dimensional datasets~\cite{hamm2023generalized,cai2021mode}. By incorporating sparsity and regularization constraints, our method reduces sensitivity to anomalies, enabling more accurate and resilient detection of unusual patterns. The Tucker  pseudoskeleton decomposition framework combines the strengths of Tucker decomposition’s structural insight with pseudoskeleton’s selective feature extraction while enhancing robustness against outliers~\cite{hamm2023generalized}.


\subsection{Notations and definitions}
In this section, we introduce notation and review foundational properties of Tucker-based tensor decomposition, which will be essential throughout the chapter. Tucker decomposition serves as a powerful tool for capturing the core structure of high-dimensional data, providing both a compact representation and interpretability of multi-dimensional relationships within the data.

To distinguish between different mathematical entities, we adopt the following conventions: calligraphic capital letters (e.g., $\mathcal{T}$) represent tensors, regular uppercase letters (e.g., ${X}$) denote matrices, regular lowercase letters (e.g., ${x}$) indicate vectors or scalars. For submatrices, $[X]_{I,:}$ and $[X]_{:,J}$ refer to the rows and columns of matrix ${X}$ indexed by sets $I$ and $J$, respectively. For tensors, $[\mathcal{T}]_{I_1, \dots, I_n}$ represents a subtensor of $\mathcal{T}$ with index sets $I_k$ along each mode $k$. A specific element in a tensor is accessed by the index notation $[\mathcal{T}]_{i_1, \dots, i_n}$. 

The tensor norm used in this chapter is the Frobenius norm~\cite{kolda2009tensor}, defined for a tensor \(\mathcal{T}\) as:
\begin{equation*}
    \|\mathcal{T}\|_\mathrm{F} = \sqrt{\sum_{i_1, \dots, i_n} [\mathcal{T}]_{i_1, \dots, i_n}^2}.
\end{equation*}
This norm represents the square root of the sum of the squared entries of \(\mathcal{T}\), extending the Frobenius norm from matrices to higher-order tensors.
For matrices, the Moore-Penrose Pseudoinverse is denoted by ${X}^\dagger$. The notation $[d] := \{1, \dots, d\}$ represents the set of natural numbers up to $d$.

\begin{definition}[\textbf{Tensor Matricization/Unfolding}~\cite{kolda2009tensor}]
    An $n$-mode tensor $\mathcal{T}$ can be reshaped into a matrix by unfolding it along each of its $n$ modes. The mode-$k$ unfolding of a tensor $\mathcal{T} \in \mathbb{R}^{d_1 \times \dots \times d_n}$, denoted $\mathcal{T}_{(k)}$, is a matrix of size $\mathbb{R}^{d_k \times \prod_{j \neq k} d_j}$, obtained by arranging all vectors of $\mathcal{T}$ with indices fixed in all modes except the $k$-th. This transformation, $\mathcal{T} \mapsto \mathcal{T}_{(k)}$, is referred to as the mode-$k$ unfolding operator.
\end{definition}

\begin{definition}[\textbf{Mode-$k$ Product}~\cite{kolda2009tensor}]
    Let $\mathcal{T} \in \mathbb{R}^{d_1 \times \dots \times d_n}$ and ${A} \in \mathbb{R}^{J \times d_k}$. The mode-$k$ product of $\mathcal{T}$ with ${A}$, denoted by $\mathcal{Y} = \mathcal{T} \times_k {A}$, is defined element-wise as:
    \begin{equation*}
        [\mathcal{Y}]_{i_1, \dots, i_{k-1}, j, i_{k+1}, \dots, i_n} = \sum_{s=1}^{d_k} [\mathcal{T}]_{i_1, \dots, i_{k-1}, s, i_{k+1}, \dots, i_n} [{A}]_{j, s}.
    \end{equation*}
    Alternatively, this operation can be represented in matrix form as $\mathcal{Y}_{(k)} = {A} \mathcal{T}_{(k)}$. For a sequence of tensor-matrix products across different modes, we use the notation $\mathcal{T} \times_{i=t}^{s} {A}_i$ to indicate the product $\mathcal{T} \times_{t} {A}_{t} \times_{t+1} \dots \times_{s} {A}_{s}$. This operation is referred to as the `tensor-matrix product' throughout the paper.
\end{definition}
\begin{definition}[\textbf{Tucker Rank and Tucker Decomposition}~\cite{kolda2009tensor}]
    The Tucker decomposition of a tensor $\mathcal{T}$ approximates it by expressing it as a product of a core tensor $\mathcal{C}$ and factor matrices ${A}_k$ along each mode:
    \begin{equation*}\label{eqn:Tucker_Decomposition}
        \mathcal{T} \approx \mathcal{C} \times_{i=1}^n {A}_i.
    \end{equation*}
    If the approximation in \eqref{eqn:Tucker_Decomposition} becomes an equality and the core tensor $\mathcal{C} \in \mathbb{R}^{r_1 \times \dots \times r_n}$, this is termed an exact Tucker decomposition of $\mathcal{T}$. The ranks $(r_1, \dots, r_n)$ are known as the Tucker ranks of the tensor $\mathcal{T}$.
\end{definition}

%\begin{remark}
%    Tucker decomposition can be viewed as a generalization of matrix singular value decomposition (SVD) to higher dimensions, preserving essential structure while reducing dimensionality. The HOSVD~\cite{de2000best} is a specific orthogonal form of Tucker decomposition commonly used in applications.
%\end{remark}
In the realm of matrix algebra, the pseudoskeleton decomposition technique is a good alternative to SVD~\cite{Goreinov}. Specifically, this method entails selecting specific columns ${C}$ and rows ${R}$ from a matrix ${X} \in \mathbb{R}^{d_1 \times d_2}$, and constructing a core matrix ${U} = {X}(I, J)$. The matrix ${X}$ is then reconstructed through the product ${C} {U}^\dagger {R}$, under the condition that $\operatorname{rank}({U}) = \operatorname{rank}({X})$. Expanding from matrices to tensors, the initial adaptations of pseudoskeleton decompositions applied a single-mode unfolding to 3-mode tensors~\cite{mahoney2008tensor}. To my best knowledge, the following are recent works on tensor pseudoskeleton decompositions or Tensor CUR Decompostions~\cite{hamm2023generalized,cai2021mode,cai2023robust,ahmadi2022cross,caiafa2010generalizing}. Furthermore, H. Cai, K. Hamm, and etc have presented rigorous theoretical results on the exact tensor pseudoskeleton decomposition \cite{cai2021mode,cai2023robust,hamm2023generalized}. For completeness, we present their work below.
\begin{definition}[Tensor pseudoskeleton decompositions or Tensor CUR Decompostions~\cite{hamm2023generalized,cai2021mode,cai2023robust}]
 Consider a tensor \(\mathcal{A} \in \mathbb{R}^{d_1 \times \cdots \times d_n}\) with Tucker ranks \((r_1, \dots, r_n)\). Suppose that for each mode \(i\), there exists a subset \(I_i \subseteq [d_i]\) such that 
 \[
\mathcal{A} = \mathcal{R} \times_{i=1}^{n} \left( C_i U_i^\dagger \right),
\]
where \(\mathcal{R} = [\mathcal{A}]_{I_1, \dots, I_n}, \quad C_i = [\mathcal{A}_{(i)}]_{:, J_i}, \quad \text{and} \quad U_i = [\mathcal{C}_{(i)}]_{I_i, :}, \quad J_i = \bigotimes_{j \neq i} I_j.
\)
\end{definition}
\begin{theorem}[{~\cite{hamm2023generalized,cai2021mode,cai2023robust}}]\label{thm:  Tucker Decomposition}
    For a tensor $\mathcal{A} \in \mathbb{R}^{d_1 \times \cdots \times d_n}$ with Tucker ranks $(r_1, \dots, r_n)$, consider subsets $I_i \subseteq [d_i]$ and let $J_i = \bigotimes_{j \neq i} I_j$ for each mode $i$. Define $\mathcal{R} = [\mathcal{A}]_{I_1, \dots, I_n}$, ${C}_i = [\mathcal{A}_{(i)}]_{:, J_i}$, and ${U}_i = [\mathcal{C}_{(i)}]_{I_i, :}$. The following conditions are equivalent:
    \begin{enumerate}
        \item $\mathcal{A} = \mathcal{R} \times_{i=1}^{n} ({C}_{i} {U}_i^\dagger)$,
        \item $\operatorname{rank}({U}_i) = r_i$ for all $i$,
        \item $\operatorname{rank}({C}_i) = r_i$ for all $i$, and $\mathcal{R}$ has Tucker rank $(r_1, \dots, r_n)$.
    \end{enumerate}
\end{theorem}
 For those interested in further details, It is recommended to read works~\cite{hamm2023generalized,cai2021mode,ahmadi2022cross,caiafa2010generalizing,cai2023robust,che2022perturbations,saibaba2016hoid}.



\section{Methodology}
We employ Tensor Robust Principal Component Analysis, an extension of classical Robust PCA that can operate directly on multi-dimensional (tensor) data. Unlike conventional low-rank models that assume the entire dataset is low-rank, TRPCA decomposes a given tensor into two distinct components: a low-rank component representing regular patterns and a sparse component isolating anomalies. This decomposition effectively isolates outliers in spatial-temporal data while retaining core structural patterns, providing a more flexible and robust approach to anomaly detection. By handling high-dimensional tensor data, TRPCA is particularly well-suited for scenarios where data is naturally structured as a multi-way array, allowing for the detection of unusual patterns that vary across both space and time.

In this framework, we represent the spatial-temporal data as a tensor $\mathcal{T} \in \mathbb{R}^{d_1 \times d_2 \times \cdots \times d_n}$, where each dimension $d_i$ corresponds to a specific mode of the data. For example, $d_1$ might represent spatial coordinates, $d_2$ temporal intervals, and additional dimensions might capture contextual features or sensor types. The objective is to decompose $\mathcal{T}$ into two components: a low-rank tensor $\mathcal{L}^\star$ that captures the dominant spatial-temporal structure, and a sparse tensor $\mathcal{S}^\star$ representing anomalies or outliers. The decomposition is expressed as:
\begin{equation*}
    \mathcal{T} = \mathcal{L}^\star + \mathcal{S}^\star,
\end{equation*}
where $\mathcal{L}^\star \in \mathbb{R}^{d_1 \times \cdots \times d_n}$ encapsulates the smooth, regular patterns in the data, while $\mathcal{S}^\star \in \mathbb{R}^{d_1 \times \cdots \times d_n}$ captures deviations from these patterns, isolating events that significantly differ from expected behavior. This separation allows for robust anomaly detection, as $\mathcal{S}^\star$ can pinpoint localized irregularities without interference from the regular structure. Mathematically, we formulate the anomaly detection problem as an optimization problem that seeks to minimize the reconstruction error between $\mathcal{T}$ and the sum of $\mathcal{L}$ and $\mathcal{S}$. This is achieved through the following objective:
\begin{equation*}\label{eq:trpca_formulation}
    \begin{split}
        \min_{\mathcal{R}, {C}_{i}, {U}_i, \mathcal{S}} & \quad \|\mathcal{T} - \mathcal{L} - \mathcal{S}\|_\mathrm{F} \\
        \text{subject to} & \quad \mathcal{L}= { \mathcal{R} \times_{i=1}^{n} ({C}_{i} {U}_i^\dagger)}\\
        &\quad\|\mathcal{S}\|_{\infty} \leq \alpha.
    \end{split}
\end{equation*}






\subsection{TRPCA via Tensor 
Pseudoskeleton Decomposition}

\begin{algorithm}[H]
    \caption{TRPCA via Tensor 
Pseudoskeleton Decomposition}
    \label{alg:TCPD}
    \begin{algorithmic}[1]
        \State \textbf{Input: } $\mathcal{T}  \in \mathbb{R}^{d_1 \times \cdots \times d_n}$: observed tensor; 
        $(r_1, \cdots, r_n)$: estimated Tucker rank; 
        $\varepsilon$: targeted precision; 
        $\zeta^{(0)}, \gamma$: thresholding parameters; $\{|I_i|\}_{i=1}^n,\{|J_i|\}_{i=1}^n$: cardinalities for sample indices.
        \State  Uniformly sample the indices $\{I_i\}_{i=1}^n, \{J_i\}_{i=1}^n$ 
        \State \textbf{Initialization:} $\mathcal{L}^{(0)} = 0, \mathcal{S}^{(0)} = 0, k = 0$
        \While {$e^{(k)} > \varepsilon$}
            \State \textcolor{officegreen}{// Step (I): Updating $\mathcal{S}$}
            \State $\zeta^{(k+1)} = \gamma \cdot \zeta^{(k)}$ 
            \State $\mathcal{S}^{(k+1)} = \mathrm{HT}_{\zeta^{(k+1)}}(\mathcal{T} - \mathcal{L}^{(k)})$  
            \State \textcolor{officegreen}{// Step (II): Updating $\mathcal{L}$}
            \State $\mathcal{L}^{(k+1)} = [\mathcal{T} - \mathcal{S}^{(k+1)}]_{I_1, \cdots, I_n}$
            \For{$i = 1, \cdots, n$}
                \State $C_i^{(k+1)} = [(\mathcal{T} - \mathcal{S}^{(k+1)})_{(i)}]_{:, J_i}$ 
                %\State $U_i^{(k+1)} = \mathcal{H}_{r_i}([C_i^{(k+1)}]_{I_i, :})$ 
                \State $[Q,R] = \operatorname{qr}\left([C_i^{(k+1)}]_{I_i, :}^{\top}\right)$

             \State$\mathcal{L}^{(k+1)} = \mathcal{L}^{(k+1)} \times C_i^{(k+1)}[Q]_{:,:r}[R]_{:r,:}^{\dagger}$    
            \EndFor
            
            \State $k = k + 1$ 
        \EndWhile
        \State \textbf{Output: } $\mathcal{L}^{(k+1)}, \mathcal{S}^{(k+1)}$.
    \end{algorithmic}
\end{algorithm}

\subsubsection{Step~(I): Update Sparse Component $\mathcal{S}$} 
\label{sec:updateS}
In this step, we update the sparse component \(\mathcal{S}\) — which captures data outliers — using the technique described in \cite{cai2023robust,netrapalli2014non,cai2019accelerated}. Specifically, we apply an iterative decaying threshold within the hard thresholding operator \(\mathrm{HT}_\zeta\) paired with \(\gamma\), as described in \cite{cai2023robust,CaiR2024}.
The hard thresholding operator $\mathrm{HT}_\zeta$ is defined as follows:

\begin{equation*}
    [\mathrm{HT}_{\zeta}(\mathcal{T})]_{i_1,\cdots,i_n} =
    \begin{cases}
        [\mathcal{T}]_{i_1,\cdots,i_n}, & \quad |[\mathcal{T}]_{i_1,\cdots,i_n}| > \zeta; \\
        0,  & \quad \text{otherwise.}
    \end{cases}
\end{equation*}

This operator $\mathrm{HT}_\zeta$ effectively filters out entries with magnitudes less than or equal to $\zeta$, treating them as negligible. By applying this to the tensor $\mathcal{T}$, only values deemed significant (i.e., values exceeding the threshold) remain in the updated sparse component $\mathcal{S}$, thereby enhancing the sparsity of $\mathcal{S}$.
\subsubsection{Step~(II): Update Low-Tucker-rank Component $\mathcal{L}$}
\label{sec:updateL}
In this step, we aim to update the low-Tucker-rank component \(\mathcal{L}\), which models the structured, low-rank part of the data tensor via tensor pseudoskeleton decomposition. The update process is divided into two key stages: subspace identification and projective reconstruction.
To approximate the low-rank structure along each mode, we begin by extracting the mode-\(i\) fibers from the residual tensor \(\mathcal{T} - \mathcal{S}^{(k)}\), which represents the current estimate of the sparse component subtracted from the observed data tensor. The fibers are assembled into the matrix representation:
\[
C_i^{(k)} \in \mathbb{R}^{d_i \times |J_i|},
\]
where each column of \(C_i^{(k)}\) corresponds to a mode-\(i\) fiber indexed by a subset of indices \(J_i\). We select a subset of mode-\(i\) fibers indexed by \(I_i \subseteq \{1, \dots, d_i\}\) and perform an economy-size QR decomposition on the transposed submatrix formed by these selected fibers:
\begin{equation*}
    \left[C_i^{(k)}\right]_{I_i,:}^\top = Q R,
\end{equation*}
where \(Q \in \mathbb{R}^{|J_i| \times r_i}\) is a matrix with orthonormal columns representing the estimated basis, and \(R \in \mathbb{R}^{r_i \times |I_i|}\) is an upper triangular matrix. The dimension \(r_i\) is the estimated Tucker rank along mode-\(i\). This step yields a low-dimensional orthonormal basis that approximates the column space of the matricized low-rank component along mode-\(i\), i.e., the dominant subspace of \(\mathcal{L}^\star_{(i)}\).
Once the subspace is identified, we project the full set of mode-\(i\) fibers onto this estimated low-rank subspace. This is achieved by updating the mode-\(i\) factor matrix of the Tucker decomposition as follows:
\begin{equation*}
    \mathcal{L}^{(k+1)} \leftarrow \mathcal{L}^{(k+1)} \times_i \left( C_i^{(k)} \left[Q\right]_{:,:r_i} \left[R\right]_{:r_i,:}^\dagger \right).
\end{equation*}
This projection aligns the updated factor matrices along mode-\(i\) with the estimated low-dimensional subspace. Using QR decomposition and projecting onto the selected subspace, the computational complexity for each mode is reduced from the cubic cost \(\mathcal{O}(d_i^3)\) to the more efficient:
\(
\mathcal{O}(d_i r_i^2 + r_i^3),
\)
where \(d_i\) is the dimension along mode-\(i\), and \(r_i\) is the target Tucker rank. This reduction is particularly beneficial when the Tucker rank \(r_i\) is significantly smaller than the mode dimension \(d_i.\)



\section{Theoretical Foundations}\label{sec:theory}

\begin{theorem}\label{thm:subspace}
Let $\mathcal{L}^\star \in \mathbb{R}^{d_1 \times \cdots \times d_n}$ be a rank-$(r_1,\ldots,r_n)$ Tucker tensor with factor matrices $\mathbf{U}_i \in \mathbb{R}^{d_i \times r_i}$ satisfying the $\mu$-incoherence condition:
\begin{equation*}
\max_{1 \leq j \leq d_i} \|\mathbf{U}_i(j,:)\|_2 \leq \sqrt{\frac{\mu r_i}{d_i}}, \quad \forall i \in [n].
\end{equation*}
For any mode $i$ and failure probability $\delta \in (0,1)$, if we sample row indices $I_i \subseteq [d_i]$ with cardinality 
\begin{equation*}
|I_i| \geq c_0 \mu r_i \log^3\left(\frac{\mu r_i}{\delta}\right),
\end{equation*}
then with probability at least $1-\delta$, the sampled factor matrix satisfies
\begin{equation*}
\frac{1}{2}\sqrt{\frac{|I_i|}{d_i}} \leq \sigma_{\min}\left(\mathbf{U}_i(I_i,:)\right) \leq \sigma_{\max}\left(\mathbf{U}_i(I_i,:)\right) \leq \frac{3}{2}\sqrt{\frac{|I_i|}{d_i}},
\end{equation*}
where $c_0 > 0$ is an absolute constant and $\sigma_{\min}(\cdot)$, $\sigma_{\max}(\cdot)$ denote extremal singular values.
\end{theorem}
\begin{proof}
Define the normalized sampling matrix $\mathbf{\Phi}_i = \sqrt{\frac{d_i}{|I_i|}}\mathbf{S}_i$ where $\mathbf{S}_i \in \{0,1\}^{|I_i|\times d_i}$ has exactly one 1 per row. The subsampled matrix becomes:
\[
\widetilde{\mathbf{U}}_i = \mathbf{\Phi}_i\mathbf{U}_i \in \mathbb{R}^{|I_i|\times r_i}.
\]
Applying the matrix Bernstein inequality \cite{tropp2015introduction} to $\mathbf{U}_i\mathbf{U}_i^\top$:
\[
\mathbb{P}\left(\left\|\widetilde{\mathbf{U}}_i\widetilde{\mathbf{U}}_i^\top - \mathbf{I}\right\|_2 \geq t\right) \leq 2r_i \exp\left(-\frac{t^2|I_i|}{C\mu r_i \log d_i}\right).
\]

Setting $t = 1/2$ and solving for $|I_i|$:
\[
|I_i| \geq C\mu r_i \log^3\left(\frac{\mu r_i}{\delta}\right) \implies \frac{1}{2}\mathbf{I} \preceq \widetilde{\mathbf{U}}_i\widetilde{\mathbf{U}}_i^\top \preceq \frac{3}{2}\mathbf{I}.
\]

Notice that
\[
\sigma_{\min}^2(\mathbf{U}_i(I_i,:)) = \frac{d_i}{|I_i|}\sigma_{\min}^2(\widetilde{\mathbf{U}}_i) \geq \frac{d_i}{2|I_i|}.
\]
Similarly for $\sigma_{\max}$. Rearrangement completes the proof.
\end{proof}
\begin{theorem}\label{thm:convergence}
Under the conditions of Theorem \ref{thm:subspace} and assuming $\|\mathcal{S}^\star\|_\infty \leq \frac{\zeta^{(0)}}{2\sqrt{\log d_{\max}}}$, the iterates satisfy:
\begin{equation*}
\|\mathcal{L}^{(k+1)} - \mathcal{L}^\star\|_F \leq \rho\|\mathcal{L}^{(k)} - \mathcal{L}^\star\|_F + C\sqrt{\frac{\log d_{\max}}{|I|}}\|\mathcal{S}^\star\|_\infty,
\end{equation*}
where the contraction factor \[\rho = \max_{1 \leq i \leq n} \left(1 - \frac{\sigma_{\min}^2(\mathbf{U}_i(I_i,:))}{2}\right) < 1\] and $|I| = \min\limits_i |I_i|$.
\end{theorem}

\begin{proof}
Define the errors:
\[
\Delta^{(k)} := \mathcal{L}^{(k)} - \mathcal{L}^\star, \quad \mathcal{E}^{(k)} := \mathcal{S}^{(k)} - \mathcal{S}^\star
\]
The update rule induces coupled dynamics:
\[
\Delta^{(k+1)} = \underbrace{\sum_{i=1}^n (\mathcal{P}_{\mathbf{Q}_i^{(k)}} - \mathcal{P}_{\mathbf{U}_i})\Delta^{(k)}}_{\text{Projection error}} + \underbrace{\mathcal{B}^{(k)}\mathcal{E}^{(k)}}_{\text{Sparsity propagation}}
\]
where $\mathcal{B}^{(k)}$ represents the multi-modal projection of residual errors.
From the hard thresholding operation and incoherence condition:
\begin{align}
\|\mathcal{E}^{(k)}\|_1 &\leq \gamma\|\mathcal{E}^{(k-1)}\|_1 + C_1\|\Delta^{(k)}\|_F \\
&\leq \gamma^k\|\mathcal{E}^{(0)}\|_1 + C_1\sum_{m=0}^{k-1}\gamma^{k-m-1}\|\Delta^{(m)}\|_F
\end{align}
Under the sparsity condition $\|\mathcal{S}^\star\|_\infty \leq \frac{\zeta^{(0)}}{2\sqrt{\log d_{\max}}}$:
\[
\|\mathcal{B}^{(k)}\mathcal{E}^{(k)}\|_F \leq C_2\sqrt{\log d_{\max}}\|\mathcal{S}^\star\|_\infty
\]
Using Wedin's theorem \cite{wedin1972perturbation} and Theorem \ref{thm:subspace}:
\[
\|\mathcal{P}_{\mathbf{Q}_i^{(k)}} - \mathcal{P}_{\mathbf{U}_i}\|_2 \leq C_3\sqrt{\frac{\mu r_i d_i \log d_i}{|I_i|^2}}
\]
Summing over all modes:
\[
\left\|\sum_{i=1}^n (\mathcal{P}_{\mathbf{Q}_i^{(k)}} - \mathcal{P}_{\mathbf{U}_i})\Delta^{(k)}\right\|_F \leq \left(1 - \frac{c}{|I|}\right)\|\Delta^{(k)}\|_F
\]
Combining both components:
\begin{align}
\|\Delta^{(k+1)}\|_F &\leq \left(1 - \frac{c}{|I|}\right)\|\Delta^{(k)}\|_F + C_2\sqrt{\log d_{\max}}\|\mathcal{S}^\star\|_\infty \\
&\leq \rho\|\Delta^{(k)}\|_F + C\sqrt{\frac{\log d_{\max}}{|I|}}\|\mathcal{S}^\star\|_\infty
\end{align}
where $\rho = 1 - \frac{c}{2|I|}$. Solving the recursion completes the proof.
\end{proof}

\begin{lemma}\label{lem:Sparsity}
The projected sparsity term satisfies:
\[
\|\mathcal{B}^{(k)}\mathcal{E}^{(k)}\|_F \leq C\sqrt{\frac{\log d_{\max}}{|I|}}\left(\|\mathcal{E}^{(k)}\|_1 + \|\Delta^{(k)}\|_F\right)
\]
\end{lemma}

\begin{proof}
Decompose the sparsity propagation using the following inequality:
\[
\|\mathcal{B}^{(k)}\mathcal{E}^{(k)}\|_F \leq \|\mathcal{B}^{(k)}\|_F\|\mathcal{E}^{(k)}\|_1
\]
From Theorem \ref{thm:subspace}, the projection operator norm is bounded by:
\[
\|\mathcal{B}^{(k)}\|_F \leq C\sqrt{\frac{\log d_{\max}}{|I|}}
\]
Combining with the threshold error bound completes the proof.
\end{proof}
\begin{theorem}\label{thm:error}
After $K = \mathcal{O}\left(\frac{\log(1/\epsilon)}{\log(1/\rho)}\right)$ iterations, the estimation error decomposes as:
\begin{equation*}
\|\mathcal{L}^{(K)} - \mathcal{L}^\star\|_F \leq \underbrace{C_1\sqrt{\frac{r_{\max}d_{\max}\log d_{\max}}{|I|}}}_{\text{Approximation Error}} + \underbrace{C_2\frac{\|\mathcal{S}^\star\|_\infty}{\sqrt{\log d_{\max}}}}_{\text{Optimization Error}},
\end{equation*}
where $r_{\max} = \max_i r_i$, $d_{\max} = \max_i d_i$, and $C_1, C_2 > 0$ are constants.
\end{theorem}

\begin{proof}
From Theorem \ref{thm:subspace}:
\[
\|\mathcal{L}^{(0)} - \mathcal{L}^\star\|_F \leq C\sqrt{\frac{r_{\max}d_{\max}}{|I|}}.
\]
Applying Theorem \ref{thm:convergence} recursively:
\[
\|\mathcal{L}^{(K)} - \mathcal{L}^\star\|_F \leq \rho^K C\sqrt{\frac{r_{\max}d_{\max}}{|I|}} + \frac{C'\sqrt{\log d_{\max}}}{1-\rho}\|\mathcal{S}^\star\|_\infty.
\]
Setting $\rho^K \leq \sqrt{\frac{\log d_{\max}}{r_{\max}d_{\max}}}$ yields the optimal error decomposition.
\end{proof}

\begin{corollary}[Sample Complexity]\label{cor:sample}
To achieve $\epsilon$-accuracy with $\epsilon < \|\mathcal{S}^\star\|_\infty/\sqrt{\log d_{\max}}$, the required sampling complexity per mode is:
\begin{equation*}
|I_i| \geq C\mu r_i d_i \log^3 d_i \left(\frac{r_{\max}d_{\max}}{\epsilon^2} + \frac{\|\mathcal{S}^\star\|_\infty^2}{\epsilon^2\log d_{\max}}\right).
\end{equation*}
\end{corollary}




\section{Numerical Experiments}
We utilize the NYC yellow taxi trip records from 2018 as a real-world spatiotemporal dataset~\cite{indibi2024spatiotemporal,sofuoglu2022gloss}. This dataset provides a detailed log of each taxi trip, including departure and arrival information (zones and times), the number of passengers, and tip amounts. 

In our experiments, we aggregate the data same as that in ~\cite{indibi2024spatiotemporal} by counting the number of arrivals per zone over hourly intervals. To ensure statistical significance, we restrict our analysis to 81 central zones, which represent high-traffic areas and exclude zones with minimal activity. This selection reduces noise from sparsely populated zones and provides a more robust representation of NYC’s high-demand regions.
With these parameters, we constructed a four-dimensional tensor \( \mathbf{Y} \) with dimensions \( 24 \times 7 \times 53 \times 81 \). The modes of this tensor are defined as follows: the first mode corresponds to the 24 hours of a day; the second mode represents the 7 days of the week; the third mode encompasses the 53 weeks of the year; the fourth mode covers the 81 selected central zones in New York City. Thus, each entry  in the tensor represents the count of taxi arrivals for hour \( h \), day \( d \), week \( w \), and zone \( z \), aggregate over the year. 

We evaluate our anomaly detection approach by identifying the top \(K\%\) of entries with the highest anomaly scores from the extracted sparse tensors, with \(K\) varying across multiple thresholds (0.014, 0.07, 0.14, 0.3, 0.7, 1, 2, and 3). Each top-\(K\%\) subset is then compared to compiled event list to determine how many events are correctly detected. The compiled event list is chosen same as~\cite{sofuoglu2022gloss,indibi2024spatiotemporal}.\Cref{tab:events} compares the number of events detected by our method against five benchmark methods—LR-STSS~\cite{indibi2024spatiotemporal}, LR-TS~\cite{indibi2024spatiotemporal}, LR-SS~\cite{indibi2024spatiotemporal}, and HoRPCA~\cite{li2015low, geng2014high}—across different \(K\%\) thresholds.
 The parameters for our method are set as follows: a maximum of 150 iterations, a tolerance level of \(10^{-7}\), and a Tucker rank of \((26, 6, 4, 10)\). The parameters for the other four methods are adopted from \cite{indibi2024spatiotemporal}. %The RCURC algorithm was developed specifically for a specialized random sampling framework known as the Cross Concentrated Sampling model~\cite{cai2024robust}. Thus, to apply the RCURC algorithm, we first flatten the tensor along its spatial dimension, transforming it into a matrix of size \(81 \times 8904\). Next, we perform robust CCS sampling, where rows and columns are selected based on a uniform observation rate across submatrices and a specified percentage of sampled indices. Finally, the RCURC algorithm is applied to extract the low-rank structure of the flatten matrix. The parameter settings for RCURC are as follows: the maximum number of iterations is fixed at 200, sampling densities (\(\delta\)) range from 0.1 to 0.8, tolerance levels (\(\text{tol}\)) are set to \(1 \times 10^{-5}\), outlier amplification factors (\(c\)) vary from 0.0 to 0.4, and the rank parameter \(r\) is incremented sequentially from 1 to 40. Parameters of all other four methods are same as in~\cite{indibi2024spatiotemporal}.


\begin{table}[H]
\centering
\resizebox{0.5\textwidth}{!}{%
\begin{tabular}{|c|c|c|c|c|c|c|c|c|}
\hline
\%       & 0.014 & 0.07 & 0.14 & 0.3  & 0.7  & 1    & 2    & 3    \\ \hline
Ours    & \textbf{3}     & \textbf{6} & \textbf{10} & \textbf{14} & \textbf{16} & \textbf{18} & \textbf{20} & \textbf{20} \\ \hline
LR-STSS  & \textbf{3} & {4} & {7} & {12} & {15} & {17} & {19} & {19} \\ \hline
LR-TS    & 3     & 4     & 5     & 6     & 13    & 13    & 18    & 19    \\ \hline
LR-SS    & 1     & 1     & 2     & 3     & 5     & 6     & 13    & 16    \\ \hline
%RCURC    & 0     & 0     & 1     & 1     & 5     & 8     & {11}  & {12}  \\ \hline
HoRPCA   & 0     & 0     & 2     & 2     & 2     & 3     & 7     & 10    \\ \hline
\end{tabular}%
}
\caption{Number of detected events among 20 compiled events in NYC for varying top-\(K\%\) of the anomaly scores}
\label{tab:events}
\end{table}
\begin{figure}
    \centering
    \includegraphics[width=1\linewidth]{methods_runtime.png}
    \caption{Running Time}
    \label{fig:time}
\end{figure}
As shown in Table~\ref{tab:events} and \Cref{fig:time}, \Cref{alg:TCPD} not only achieves higher event detection accuracy across various thresholds but also significantly reduces running time compared to LR-STSS, LR-TS, LR-SS, and HoRPCA. This balance of efficiency and effectiveness underscores \Cref{alg:TCPD}’s practical advantages for large-scale or real-time anomaly detection scenarios. This performance affirms the efficacy of our model parameters, including a Tucker rank configuration suited for complex, multi-dimensional datasets.
\section{Conclusion}

In this short paper, we investigate the application of tensor  pseudoskeleton decomposition for anomaly detection in high-traffic areas of New York City. Specifically, we aim to capture temporal and spatial patterns in taxi arrival data. By focusing on central zones with significant activity, the result demonstrates its possibility of  tensor  pseudoskeleton decomposition to remove sparsity and highlight urban regions with high demand. 

\bibliographystyle{IEEEtran}
\bibliography{reference}
\end{document}

\section{Proofs for Deterministic Safety Algorithms}\label{sec:proofs-det}

In this section we prove the correctness of our algorithm in \Cref{sec:warmup}.
We first prove that the \textsc{Round} procedure in \Cref{alg:aac-byz} satisfies the properties below, and then prove that \Cref{alg:skeleton} solves consensus under Byzantine faults.
\begin{description}
    \item[Strong Validity] If all correct processes propose the same value $v$ and a correct process returns a pair $\langle \textsc{Grade}, v' \rangle$, then $\textsc{Grade} = \textsc{Commit}$ and $v' = v$.
    \item[Consistency] If any correct process returns  $\langle \textsc{Commit}, v \rangle$, then no correct process returns $(\cdot, v' \ne v)$.
    \item[Termination] If all correct processes propose, then every correct process eventually returns.
\end{description}

In our proofs we rely on the following properties of Byzantine Reliable Broadcast (BRB)~\cite{book}:
\begin{description}
    \item[BRB-Validity] If a correct process $p$ broadcasts a message $m$, then every correct process eventually delivers $m$.
    \item[BRB-No-duplication] Every correct process delivers at most one message.
    \item[BRB-Integrity] If some correct process delivers a message $m$ with sender $p$ and process $p$ is correct, then $m$ was previously broadcast by $p$.
    \item[BRB-Consistency] If some correct process delivers a message $m$ and another correct process delivers a message $m$, then $m = m$.
    \item[BRB-Totality] If some message is delivered by any correct process, every correct process eventually delivers a message.
\end{description}

\begin{lemma}\label{lem:byz-round-validity}
    With Byzantine faults and $n=3f+1$, \Cref{alg:aac-byz} satisfies strong validity.
\end{lemma}
\begin{proof}
    If all correct processes propose the same value $v$, then at least $2f+1$ processes BRB-broadcast an \textsc{Init} message for $v$, and therefore at most $f$ processes BRB-broadcast an \textsc{Init} message for $1-v$. Thus $v$ will be the majority value among all \textsc{Init} messages delivered in phase 1, at all correct processes. Thus all correct processes will BRB-broadcast an \textsc{Echo} message for $v$. Furthermore, no Byzantine process can produce a valid \textsc{Echo} message for $1-v$, since to do so would require a set of $2f+1$ \textsc{Init} message with a majority value of $1-v$. This is impossible due to the properties of BRB and the fact that at most $f$ processes have BRB-broadcast an \textsc{Init} message for $1-v$.
    So, all valid \textsc{Echo} messages received by correct processes will be for $v$, so all correct processes will commit $v$ at line~\ref{line:fac-byz-commit}.
\end{proof}

\begin{lemma}
    With Byzantine faults and $n=3f+1$, \Cref{alg:aac-byz} satisfies consistency.
\end{lemma}
\begin{proof}
    If a correct process $p_1$ commits $v$ at line~\ref{line:fac-byz-commit}, then it must have delivered a set $S_1$ of $2f+1$ \textsc{Echo} messages for $v$ at line~\ref{line:fac-byz-wait-echo}. Take now another process $p_2$ and consider the set $S_2$ of $2f+1$ \textsc{Echo} messages it delivers at line~\ref{line:fac-byz-wait-echo}. By quorum intersection, $S_1$ and $S_2$ must intersect in at least $f+1$ messages. By the BRB-Consistency property, these $f+1$ messages must be identical at $p_1$ and $p_2$. Thus $p_2$ delivers at least $f+1$ \textsc{Echo} messages for $v$, which constitutes a majority of the $2f+1$ \textsc{Echo} messages it delivers overall. So if $p_2$ commits a value at line~\ref{line:fac-byz-commit}, then it must commit $v$, and if $p_2$ adopts a value at line~\ref{line:fac-byz-adopt}, then it must adopt $v$.
\end{proof}

\begin{lemma}
    With Byzantine faults and $n=3f+1$, \Cref{alg:aac-byz} satisfies termination.
\end{lemma}
\begin{proof}
    Follows immediately from the algorithm and from the properties of Byzantine Reliable Broadcast. Processes perform two phases; the only blocking step of each phase is waiting for $n-f$ messages (lines~\ref{line:fac-byz-wait-init} and~\ref{line:fac-byz-wait-echo}). This waiting eventually terminates, by the BRB-Validity property and the fact that there are at least $n-f$ correct processes.
\end{proof}

\begin{theorem}\label{thm:validity-byz}
    With Byzantine faults and $n=3f+1$, \Cref{alg:skeleton} satisfies strong validity.
\end{theorem}
\begin{proof}
    This follows from the strong validity property of the \textsc{Round} procedure (\Cref{lem:byz-round-validity}): if all correct processes propose $v$ to consensus, then all correct processes propose $v$ to \textsc{Round} in the first round, where by \Cref{lem:byz-round-validity}, all correct processes commit $v$, and thus all correct processes decide $v$ at line~\ref{line:skeleton-decide}.
\end{proof}

\begin{theorem}\label{thm:agreement-byz}
    With Byzantine faults and $n=3f+1$, \Cref{alg:skeleton} satisfies agreement.
\end{theorem}
\begin{proof}
    Let $r$ be the earliest round at which some process decides and let $p$ be a process that decides $v$ at round $r$. We will show that any other process $p'$ that decides, must decide $v$. 
    
    For $p$ to decide $v$ at round $r$, \textsc{Round} must output $(\textsc{Commit}, v)$ in that round. Thus, by the consistency property of \textsc{Round}, $\textsc{Round}(r,\cdot)$ must output $(\cdot, v)$ at all correct processes. If $\textsc{Round}(r,\cdot)$ outputs $(\textsc{Commit}, v)$ for $p'$, then $p'$ decides $v$ at round $r$ (line~\ref{line:skeleton-decide}). Otherwise, all correct processes input $v$ to $\textsc{Round}(r+1,\cdot)$, and by the strong validity property, all processes (including $p'$) will output $(\textsc{Commit}, v)$ and decide $v$ at round $r+1$.
\end{proof}

\begin{theorem}\label{thm:termination-byz}
    With Byzantine faults and $n=3f+1$, \Cref{alg:skeleton} satisfies termination.
\end{theorem}
\begin{proof}
     We can describe the execution of the protocol as a Markov chain with states $0,\ldots,n-f=2f+1$; the system is at state $i$ if $i$ correct processes have estimate ($est_i$ variable) equal to $0$ before invoking $\textsc{Round}$. Due to the strong validity property of the $\textsc{Round}$ procedure, states $0$ and $2f+1$ are absorbing states. There is a non-zero transition probability from each state (including $0$ and $2f+1$), to state $0$ or $2f+1$, or both (we show this below). Therefore, with probability $1$, the system will eventually reach one of the two absorbing states and remain there. Once this happens (i.e., once all processes have the same $est_i$ variable), the strong validity property of $\textsc{Round}$ ensures that all processes (who have not decided yet) will decide within a round.
    
    It only remains to show that there is a non-zero transition probability from each state to at least one of the absorbing states $0$ and $2f+1$. Consider a state $i \notin \{0,2f+1\}$; there is a schedule $S$ with non-zero probability which leads the system from $i$ to $0$ or $2f+1$ in one invocation of \textsc{Round}. We consider two cases:
    \begin{itemize}
        \item $i < f+1$: in this case $0$ is the minority value among correct processes. In schedule $S$, the $n-f$ \textsc{Init} messages delivered by correct process at line~\ref{line:fac-byz-wait-init} are all from correct processes. Thus, every correct process sees $i$ $0$s and $2f+1-i$ $1$; $1$ is the majority value, so all correct processes adopt it for phase 2. In phase 2, $S$ again ensures that the $n-f$ \textsc{Echo} messages delivered by correct process at line~\ref{line:fac-byz-wait-echo} are all from correct processes. Thus, all correct processes see $2f+1$ \textsc{Echo} messages for $1$ and commit $1$, bringing the system to state $0$.
        \item $i \geq f+1$: in this case $0$ is the majority value among correct processes. This case is symmetrical with respect to the previous one: the only difference is that all correct processes adopt $0$ (the majority value) at the end of phase 1, and all correct processes deliver $2f+1$ \textsc{Echo} messages for $0$, thus committing $0$ and bringing the system to state $2f+1$.
    \end{itemize}
\end{proof}
\end{document}

