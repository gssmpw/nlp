\section{Full proof of~\texorpdfstring{\Cref{thm:pc-ECC-FPT}}{}}
\label{app:pc}

% \pceccapproximation*


\pceccFPT*
\begin{proof}
    We give a branching algorithm which is essentially identical to that of~\Cref{thm:cfminecc-FPT}.
    For completeness, we repeat the details.
    Given an instance $(H = (V, E), t, b)$ of \pcECC{}, a \emph{conflict} is a triple $(v, e_1, e_2)$ consisting of a single vertex $v$ and a pair of distinctly colored hyperedges $e_1, e_2$ which both contain $v$.
    If $H$ contains no conflicts, then it is possible to satisfy every edge.
    Otherwise, we identify a conflict in $O(r|E|)$ time by scanning the set of hyperedges incident on each node.
    Once a conflict $(v, e_1, e_2)$ has been found, we branch on the two possible ways to resolve this conflict: deleting $e_1$ or deleting $e_2$.
    Here, deleting a hyperedge has the same effect as ``marking'' it as unsatisfied and no longer considering it for the duration of the algorithm.
    We note that it is simple to check in constant time whether a possible branch violates the constraints given by $t$ or $b$; these branches can be pruned.
    Because each branch increases the number of unsatisfied hyperedges by 1 and WLOG $t > b$, the search tree has depth at most $t$.
    Thus, the algorithm runs in time $O(2^{t}r|E|)$.
\end{proof}

