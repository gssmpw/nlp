\subsection{Color-fair ECC variants}\label{sec:color-fair}

Now we focus on the mini-max case of~\pmeanECC{} given by $p \rightarrow \infty$, which we refer to as \cfminECC{}.
In the following, $M_c^\lambda$ indicates the number of edges of color $c$ which are unsatisfied by a coloring $\lambda$.

\problembox{\cfminECC{}}{A $k$-edge-colored hypergraph $H = (V, E, \ell)$ and an integer $\tau$.}{Does there exist a vertex coloring $\lambda$ of $H$ with $\max_{c \in [k]} M_c^\lambda \leq \tau$?}

The optimization problem asks for a coloring $\lambda$ which minimizes $\max_{c \in [k]} M_c^\lambda$.
Also of interest is the corresponding maxi-min variant~\cfmaxECC{}, for which the question is whether there exists a coloring which satisfies at least $\tau$ edges of \emph{every} color.
These problems are not equivalent at optimality in general, though they are when restricted to hypergraphs with an equal number of edges of every color.
We begin by establishing hardness for both problems.
%
% color-fair minECC NP-hard (two reductions)
\begin{restatable}{theorem}{cfmineccNPhard}\label{thm:cfminecc-NPhard}
    \cfminECC{} is \cclass{NP}-hard even when restricted to subcubic trees with cutwidth~$2$, exactly $3$ edges of each color, and $\tau = 2$. \cfmaxECC{} is \cclass{NP}-hard even when restricted to paths with $\tau = 1$.
\end{restatable}
%
\begin{proof}[Proof sketch.]
    Here, we give a short reduction which proves only the following weaker claim:
    
    \begin{claim}
        \cfminECC{} is \cclass{NP}-hard even when restricted to hypergraphs with exactly three edges of each color, and $\tau = 2$. \cfmaxECC{} is \cclass{NP}-hard on the same class of hypergraphs with $\tau = 1$.
    \end{claim}
    
    The full proof, which shows all of the structural restrictions in the theorem statement, uses essentially the same reduction, but with additional gadgeteering. We refer the interested reader to~\Cref{appendix:color-fair}.

    \bigskip
    We reduce from \threeSAT{}, for which the input is a formula written in conjunctive normal form, consisting of $m$ clauses $C_1, C_2, \ldots, C_m$ over $n$ boolean variables $x_1, x_2, \ldots x_n$.
    For each variable $x_i$, we write $x_i$ and $\neg x_i$ for the corresponding positive and negative literals. We assume that every clause contains three literals corresponding to three distinct variables, and that for each variable $x_i$, both literals of $x_i$ appear in the formula.
    We construct an instance $(H = (V, E), \tau = 2)$ of \cfminECC{} as follows.

    For each clause $C_j$, we create a unique color $c_j$.
    Next, for each variable $x_i$, we create a vertex $v_{j_1, j_2}^i$ for each (ordered) pair of clauses $C_{j_1}, C_{j_2}$ with $C_{j_1}$ containing the positive literal $x_i$ and $C_{j_2}$ containing the negative literal $\neg x_i$.
    We say that $v_{j_1, j_2}^i$ is associated with the variable $x_i$ and the variable-clause pairs $(x_i, C_{j_1})$ and $(x_i, C_{j_2})$. Observe that if variable $x_i$ is contained in clause $C_j$, then $(x_i, C_j)$ has at least one associated vertex.

    Now we create edges. For each clause $C_j$ we create one edge $e_j^i$ of color $c_j$ for each variable $x_i$ contained in $C_j$. This edge contains every vertex associated with the variable-clause pair $(x_i, C_j)$.
    We note that some hyperedges may have size one. To exclude this possibility, we need only introduce an auxiliary vertex for each such hyperedge, which is contained in no other edges.
    We say that $e_j^i$ is \emph{positive} for $x_i$ if the positive literal $x_i$ appears in $C_j$, and \emph{negative} for $x_i$ otherwise.
    We set $\tau = 2$.
    This completes the construction.
    Observe that we have created exactly three hyperedges of each color, and that each hyperedge contains vertices associated with exactly one variable.
    It follows that hyperedges $e$ and $f$ overlap if and only if they are distinctly colored, contain vertices associated with the same variable $x_i$, and (without loss of generality) $e$ is positive for $x_i$ while $f$ is negative for $x_i$.   

    It remains to show that the reduction is correct.
    For the first direction, assume that there exists an assignment $\phi$ of boolean values to the variables $x_1, x_2, \ldots, x_n$ which satisfies every clause. We say that the assignment $\phi$ \emph{agrees} with a variable-clause pair $(x_i, C_j)$ if $C_j$ contains the positive literal $x_i$ and $\phi(x_i) = \textsf{True}$ or if $C_j$ contains the negative literal $\neg x_i$ and $\phi(x_i) = \textsf{False}$.
    We color the vertices of our constructed graph as follows.
    For each vertex $v_{j_1, j_2}^i$, if $\phi$ agrees with $(x_i, C_{j_1})$ we assign color $c_{j_1}$ to $v_{j_1, j_2}^i$, and otherwise we assign color $c_{j_2}$.
    Observe that this coloring satisfies an edge $e_j^i$ if and only if $\phi$ agrees with $(x_i, C_j)$.
    Moreover, because $\phi$ is satisfying,
    every clause $C_j$ contains at least one variable $x_i$ such that $\phi$ agrees with $(x_i, C_j)$.
    Hence, at least one edge of every color is satisfied. Because there are exactly three edges of every color, there are at most two unsatisfied edges of any color.

    For the other direction, assume that we have a coloring which leaves at most two edges of any color unsatisfied. We will create a satisfying assignment $\phi$.
    For each variable $x_i$, we set $\phi(x_i) = \textsf{True}$ if any edge which is positive for $x_i$ is satisfied, and $\phi(x_i) = \textsf{False}$ otherwise.
    We now show that $\phi$ is satisfying. Consider any clause $C_j$. There are exactly three edges with color $c_j$, and at least one of them is satisfied.
    Let $x_i$ be the corresponding variable, so the satisfied edge is $e_j^i$.
    If $C_j$ contains the positive literal $x_i$, then $e_j^i$ is positive for $x_i$ and so $\phi(x_i) = \textsf{True}$. Hence, $C_j$ is satisfied by $\phi$.
    Otherwise $C_j$ contains the negative literal $\neg x_i$, and $e_j^i$ is negative for $x_i$.
    In this case, we observe that every edge which is positive for $x_i$ intersects with $e_j^i$, and none of these edges has color $c_j$ since
    $C_j$ does not contain both literals. Hence, the satisfaction of $e_j^i$ implies that every edge which is positive for $x_i$ is unsatisfied.
    It follows that $\phi(x_i) = \textsf{False}$, meaning $C_j$ is satisfied by $\phi$.

    The result for~\cfmaxECC{} follows from the same construction, setting $\tau = 1$ instead of~$2$. Because there are exactly three edges of every color, the analysis is identical.
\end{proof}
\Cref{thm:cfminecc-NPhard} implies that it is \cclass{NP}-hard to provide any multiplicative approximation for \cfmaxECC{}, since it is \cclass{NP}-hard even to decide whether the optimum objective value is non-zero.
Our reduction also rules out efficient algorithms in graphs of bounded feedback edge number, cutwidth, treewidth + max. degree, or slim tree-cut width, in stark contrast to fixed-parameter tractability (FPT) results obtained for the standard \ECC{} problem by~\citet{kellerhals2023parameterized}.
On the positive side, we can give an algorithm parameterized by the total number $t$ of unsatisfied edges.
%
% color-fair minECC FPT
\begin{restatable}{theorem}{cfminECCFPT}\label{thm:cfminecc-FPT}
    \cfminECC{} is FPT with respect to the total number $t$ of unsatisfied edges.
\end{restatable}
\begin{proof}
    We give a branching algorithm.
    Given an instance $(H = (V, E), \tau)$ of \cfminECC{}, a \emph{conflict} is a triple $(v, e_1, e_2)$ consisting of a single vertex $v$ and a pair of distinctly colored hyperedges $e_1, e_2$ which both contain $v$.
    If $H$ contains no conflicts, then it is possible to satisfy every edge.
    Otherwise, we identify a conflict in $O(r|E|)$ time by scanning the set of hyperedges incident on each node.
    Once a conflict $(v, e_1, e_2)$ has been found, we branch on the two possible ways to resolve this conflict: deleting $e_1$ or deleting $e_2$.
    Here, deleting a hyperedge has the same effect as ``marking'' it as unsatisfied and no longer considering it for the duration of the algorithm.
    We note that it is simple to check in constant time whether a possible branch violates the constraint given by $\tau$; these branches can be pruned.
    Because each branch increases the number of unsatisfied hyperedges by 1, the search tree has depth at most $\edgedeletions$.
    Thus, by computing the search tree in level-order, the algorithm runs in time $O(2^{\edgedeletions}r|E|)$.
\end{proof}
%
One question left open by the reduction of~\Cref{thm:cfminecc-NPhard} is whether \cfminECC{} is hard when the number~$k$ of colors is bounded.
The standard~\ECC{} objective is in \cclass{P} when $k \leq 2$ and \cclass{NP}-hard whenever $k \geq 3$~\cite{amburg2020clustering}.
We resolve this question via an intermediate hardness result for a \textsc{Vertex Cover} variant which may be of independent interest.
Specifically, we show hardness for the following:

% Specifically, in~\Cref{appendix:color-fair} we prove that given a bipartite graph $G = (A \uplus B, E)$ and an integer $\alpha$, it is \cclass{NP}-hard to determine whether there exists a vertex cover
% $C$ of $G$ with $\max\{|C \cap A|, |C \cap B|\} \leq \alpha$.
% This result facilitates the resolution of our question:
\problembox{\fairbipartiteVC{}}{A bipartite graph $G = (V = A \uplus B, E)$ and an integer $\alpha$.}{Does there exist a vertex cover $C$ of $G$ with $\max\{|C \cap A|, |C \cap B|\} \leq \alpha$?}
%
\begin{restatable}{theorem}{fairbipartitevertexcoverNPhard}\label{thm:fairbipartiteVC-NPhard}
    \fairbipartiteVC{} is \cclass{NP}-hard.
\end{restatable}
\begin{proof}
    We reduce from \constrainedbipartiteVC{}, for which the input is a bipartite graph $G = (A \uplus B, E)$ along with two integers $\alpha_a$, $\alpha_b$, and the question is whether there exists a vertex cover $C$ with $|C \cap A| \leq \alpha_a$ and $|C \cap B| \leq \alpha_b$.
    This problem was shown to be \cclass{NP}-complete by~\citet{kuo1987efficient}.

    Given an instance $(G = (A \uplus B, E), \alpha_a, \alpha_b)$ of \constrainedbipartiteVC{}, we construct an instance $(G' = (A' \uplus B', E'), \alpha)$ of \fairbipartiteVC{} as follows.
    First, we observe that if $\alpha_a = \alpha_b$ then $(G, \alpha_a, \alpha_b)$ is already an instance of \fairbipartiteVC{}, in which case there is nothing to do.
    We therefore set $\beta = \alpha_b - \alpha_a$ and assume without loss of generality that $\beta > 0$.
    We begin by copying $G'$. That is, for each vertex $a \in A$ we create a vertex $a' \in A'$, for each vertex $b \in B$ we create a vertex $b' \in B'$, and for each edge $ab \in E$ we create an edge $a'b' \in E'$. We call
    the vertices (and edges) that we have created thus far \emph{original} vertices (and edges).
    Next, we create $\beta$ auxiliary vertices $\{a_i \ | \ 1 \leq i \leq \beta \} \subset A'$, and $\beta(\alpha_b + 1)$ auxiliary vertices $\{ b_{ij} \ | \ 1 \leq i \leq \beta, 1 \leq j \leq \alpha_b + 1 \} \subset B'$.
    We also add auxiliary edges $\{a_ib_{ij} \ | \ 1 \leq i \leq \beta, 1 \leq j \leq \alpha_b + 1 \} \subset E'$. Finally, we set $\alpha = \alpha_b$. This concludes the construction.

    It remains to show that the reduction is correct. For the first direction, assume that $C \subseteq V$ is a vertex cover of $G$ with $|C \cap A| \leq \alpha_a$ and $|C \cap B| \leq \alpha_b$.
    We construct a vertex cover $C'$ of $G'$ as follows. For each vertex $a \in C \cap A$, we add $a'$ to $C$, and for each vertex $b \in C \cap B$, we add $b'$ to $C$. We also add the auxiliary vertex $a_i$ to $C$, for each $1 \leq i \leq \beta$.
    Because $C$ is a vertex cover of $E$, $C'$ covers every original edge in $E'$. Because every auxiliary edge in $E'$ is incident on some auxiliary vertex $a_i$, $C'$ also covers all auxiliary edges in $E'$.
    Hence, $C'$ is a vertex cover of $G'$. Moreover, by construction $|C' \cap B'| \leq \alpha_b = \alpha$, and $|C' \cap A'| \leq \alpha_a + \beta = \alpha$.

    For the other direction, assume that there exists a vertex cover $C'$ of $G'$ with $|C' \cap A'|, |C' \cap B'| \leq \alpha$.
    We will construct a vertex cover $C$ of $G$.
    We begin by observing that $C'$ contains every auxiliary vertex $a_i$, for $1 \leq i \leq \beta$.
    Suppose otherwise, i.e., that for some fixed $i$ we have $a_i \notin C'$. Then each of the $\alpha_b + 1$ auxiliary vertices $b_{ij}$, for $1 \leq j \leq \alpha_b + 1$, is contained in $C'$.
    Since $\alpha = \alpha_b$, this contradicts that $|C' \cap B'| \leq \alpha$. So, we conclude that $a_i \in C'$ for each $1 \leq i \leq \beta$.
    We now also assume that $C' \cap B'$ consists entirely of original vertices, since the vertex set resulting from the removal of an auxiliary vertex $b_{ij}$ is still a cover, as it contains $a_i$.
    Now, for each $b' \in C'$, we add $b$ to $C$, and for each original $a' \in C'$, we add $a$ to $C$. Because the auxiliary vertices of $G'$ are not incident to any original edges, the original vertices of $C'$ must cover
    all original edges. Hence, $C$ is a vertex cover of $G$. Since $|C' \cap A'| \leq \alpha = \alpha_a + \beta$ and $C' \cap A'$ contains $\beta$ auxiliary vertices, $C' \cap A'$ must contain no more than $\alpha_a$ original vertices.
    Similarly, $C' \cap B'$ contains at most $\alpha = \alpha_b$ original vertices.
    These properties allow us to conclude that $|C \cap A| \leq \alpha_a$ and $|C \cap B| \leq \alpha_b$, as desired.
\end{proof}
%
\Cref{thm:fairbipartiteVC-NPhard} is useful to us because we can now reduce from \fairbipartiteVC{} to show hardness for \cfminECC{} and \cfmaxECC{} even when the number $k$ of colors is bounded.

\begin{restatable}{theorem}{cfmineccNPhardboundedk}\label{thm:cfminecc-NPhard-k2}
    \cfminECC{} and \cfmaxECC{} are \cclass{NP}-hard even in 2-regular hypergraphs with $k = 2$.
\end{restatable}
\begin{proof}
    Given an instance $(G = (A \uplus B, E), \alpha)$ of \fairbipartiteVC{}, we construct an instance
    $(H = (V, E_a \uplus E_b), \tau = \alpha)$ of \cfminECC{} with two colors ($c_A$ and $c_B$) as follows.
    For each edge $ab \in E$, we create a vertex $v_{ab} \in V$.
    We call this vertex a \emph{conflict vertex}, and we say that it is associated with $a \in A$ and with $b \in B$.
    Next, for each vertex $a \in A$, we create a hyperedge $e_a$ which contains every conflict vertex associated with $a$. We color this edge $c_A$.
    Similarly, for each vertex $b \in B$ we create a hyperedge $e_b$ of color $c_B$ which contains every conflict vertex associated with $b$.
    Observe that every hyperedge is nonempty, as we may assume that $G$ has no isolated vertices.
    Observe also that every vertex has degree~$2$.
    We set $\tau = \alpha$.

    To see that the reduction is correct, consider first a vertex cover $C$ of $G$ with $|C \cap A|, |C \cap B| \leq \alpha$.
    For each vertex $a \in A \setminus C$, we assign color $c_A$ to every conflict vertex associated with $a$.
    Similarly, for every $b \in B \setminus C$, we assign $c_B$ to every conflict vertex associated with $b$.
    We color remaining vertices arbitrarily. We claim that this is a valid coloring, i.e., every vertex has received exactly one color.
    To see this, observe that every vertex $v_{ab}$ in $V$ is a conflict vertex associated with exactly two vertices $a, b$ in $A \uplus B$.
    Since $C$ is a vertex cover, at least one of $a, b$ is in $C$. Hence, $v_{ab}$ is assigned exactly one color. We next claim that
    our assignment of colors leaves at most $\tau = \alpha$ hyperedges of any single color unsatisfied. Due to our coloring scheme,
    if a hyperedge $e_a$ (resp. $e_b$) associated with a vertex $a \in A$ (resp. $b \in B$) is unsatisfied, then
    $a$ (resp. $b$) is contained in $C$. Since $|C \cap A|, |C \cap B| \leq \alpha$, we conclude that at most $\tau = \alpha$ hyperedges
    of color $c_A$ (resp. $c_B$) are unsatisfied.

    For the other direction, suppose that we have a coloring $\lambda\colon V \rightarrow \{c_A, c_B\}$ which leaves at most $\tau = \alpha$ hyperedges
    of any single color unsatisfied. We construct a vertex cover $C$ of $G$ which contains $a \in A$ (resp. $b \in B$) if and only if the hyperedge associated with $a$ (resp. $b$) is unsatisfied by $\lambda$.
    It is immediate that $|C \cap A|, |C \cap B| \leq \alpha$.
    Now we claim that $C$ is a vertex cover of $G$.
    Consider any edge $ab \in E$. The conflict vertex $v_{ab}$ is contained in both $e_a$, which has color $c_A$, and $e_b$, which has color $c_B$.
    Thus, at least one of $e_a, e_b$ is unsatisfied by $\lambda$, and so $C$ contains at least one of $a,b$.
    Then $C$ is a vertex cover, as desired.

    A simple adjustment to our construction yields the claimed hardness result for \cfmaxECC{}.
    Let $\beta = |A| - |B|$, and assume without loss of generality that $\beta \geq 0$.
    If $\beta = 0$, then there are already an equal number of edges of colors $c_A$ and $c_B$, so we make no adjustment to the construction.
    If $\beta \geq 3$, we add an additional $\beta$ vertices $v_1, v_2, \ldots v_\beta$,
    and an additional $\beta$ hyperedges $e_1 = \{v_1, v_2\}, e_2 = \{v_2, v_3\}, \ldots, e_\beta = \{v_\beta, v_1\}$, constructing a cycle with $\beta$ vertices and edges (a $C_\beta$).
    We assign color $c_B$ to each of these hyperedges. We now observe that in the constructed hypergraph, there are an equal number of
    hyperedges of colors $c_A$ and $c_B$, and every vertex has degree~$2$.
    Finally, if $\beta \in \{1, 2\}$, we add $\beta + 6$ vertices $v_1, v_2 \ldots v_{\beta + 6}$. With three of these vertices we form a triangle
    with each edge having color $c_A$, and with the remaining $\beta + 3$ vertices we form a $C_{\beta+3}$ with each edge having color $c_B$.
    Once again, we observe that there are now an equal number of edges of colors $c_A$ and $c_B$ in the constructed hypergraph, and every vertex has degree~$2$.
    We set $\tau = |A| - \alpha$ if $\beta \geq 3$, or $\tau = |A| + 3 - \alpha$ otherwise.
    Correctness follows from a substantively identical analysis.
\end{proof}
%
We will now further examine connections between \cfminECC{} and variants of \textsc{Vertex Cover}.
The main idea of all of our reductions is similar to that of~\citet{veldt2023optimal}; we create a vertex for each hyperedge, and add edges for all pairs of distinctly colored overlapping hyperedges.
Recalling~\Cref{sec:generalized-mean}, we call this construction the \emph{conflict graph} associated with an edge-colored hypergraph.

We observe that in addition to the algorithm of~\Cref{thm:pmean-approximation}, a $2$-approximation for \cfminECC{} can also be obtained
by applying a result of~\citet{blum2022sparse} to the constructed conflict graph. Formally, we reduce to the following:

\optproblembox{\SVC{}}{A graph $G = (V = V_1 \cup V_2 \cup \ldots \cup V_k, E)$.}{Find a vertex cover $C \subseteq V$ of $G$ which minimizes $\max_{i \in [k]} \{|C \cap V_i|\}$.}
\begin{restatable}{proposition}{cfmineccreductiontosparsevertexcover}\label{thm:cfminecc-reduce-sparse-vc}
    \cfminECC{} admits a $2$-approximation via reduction to \SVC{}.
\end{restatable}
\begin{proof}
    Let $H = (V, E = E_1 \uplus E_2 \uplus \ldots \uplus E_k)$ be an edge-colored hypergraph with $k$ colors.
    We construct an instance $G = (V' = V_1' \cup V_2' \ldots V_k', E')$ of \SVC{}. For each hyperedge $e$ of color $i$, we create a vertex $v_e$ in $V_i'$.
    Next, for every pair of distinctly colored hyperedges $e_1, e_2$ with $e_1 \cap e_2 \neq \emptyset$ (a \emph{bad hyperedge pair}), we create an edge $v_{e_1}v_{e_2}$.
    This completes the construction of $G$.

    Now, let $\opt_G$ and $\opt_H$ be the optimal objective values for $G$ and $H$, respectively. We claim that
    $\opt_G \leq \opt_H$. Suppose that $S$ is the set of hyperedges unsatisfied by some optimal coloring $\lambda$, so $\max_{i \in [k]} |S \cap E_i| = \opt_H$.
    Because $\lambda$ satisfies all edges in $E \setminus S$, $S$ must contain at least one member of every bad hyperedge pair.
    Hence, the set $C = \{v_e \in V' \ \colon \ e \in S \}$ of vertices in $G$ corresponding to hyperedges in $S$ is a vertex cover in $G$, and has the property that $\max_{i \in [k]} |C \cap V_i'| = \opt_H$.
    It follows that $\opt_G \leq \opt_H$.

    Finally, we show how to lift a solution. Let $C \subseteq V'$ be a vertex cover in $G$ which is 2-approximate; such a cover can be computed in polynomial time~\cite{blum2022sparse}.
    Let $S = \{e \in E : v_e \in C\}$ be the hyperedges of $H$ corresponding to the vertices in $C$.
    Because $C$ is a vertex cover and the edges of $G$ correspond exactly to the bad hyperedge pairs of $H$, it is trivial to compute a
    coloring $\lambda$ which satisfies every edge in $E \setminus S$. The objective value of $\lambda$
    is at worst
    \[
        \max_{i \in [k]} |S \cap E_i| = \max_{i \in [k]} |C \cap V_i'| \leq 2\cdot \opt_G \leq 2\cdot \opt_H.
    \]

\end{proof}

In practice, the algorithm of~\Cref{thm:pmean-approximation} is preferable, as
the conflict graphs associated with edge-colored hypergraphs can be very dense and inefficient to even form explicitly.
%
% It is natural to wonder whether the reduction of~\Cref{thm:cfminecc-reduce-sparse-vc} may provide a route toward
% approximating \cfminECC{} below factor-$2$. This may seem unlikely; \SVC{} contains \problem{Vertex Cover} as a special case, so a $(2 - \Omega(1))$-approximation
% would refute the unique games conjecture~\cite{williamson2011design}. 
% However, the \SVC{} instances produced by the reduction of~\Cref{thm:cfminecc-reduce-sparse-vc} have some special structure.
% In particular, they have the simultaneous properties that the vertex classes $V_1, V_2, \ldots, V_k$ are non-overlapping, and that every edge has endpoints in distinct vertex classes.
% It remains an interesting open direction to study \SVC{} under these restrictions.
%

A limitation of both approaches is the reliance on convex optimization as a subroutine.
Hence, an appealing open direction is to develop purely combinatorial algorithms.
Toward this end, we now give a combinatorial $k$-approximation, i.e., one which is constant-factor for every fixed number of edge colors. The algorithm begins by computing a maximal matching $M$ in the conflict graph $G$.
The vertices in $G$ which are unmatched by $M$ are an independent set, so it is possible to simultaneously satisfy all of the corresponding hyperedges.
This algorithm has already been shown to be a $2$-approximation to the standard \minecc{} objective by~\citet{veldt2023optimal}, who also showed that by avoiding explicit construction of $G$, the running time can be made \emph{linear} in the size of the hypergraph, i.e., $O(\sum_{e \in E} |e|)$.
Our contribution is to show that the algorithm also gives a $k$-approximate solution to the color-fair objective; the lower bound is derived from an analysis of the $k$-partite structure of $G$.

\begin{restatable}{theorem}{cfminecccombinatorial}\label{thm-combinatorial-k-approx}
    \cfminECC{} admits a linear-time combinatorial $k$-approximation.
\end{restatable}
\begin{proof}
    We begin by proving that \SVC{} admits a polynomial-time combinatorial $k$-approximation when the vertex classes $V_1, V_2, \ldots, V_k$ are disjoint and every edge has endpoints in distinct vertex classes.
    Let $G = (V = V_1 \uplus V_2 \uplus \ldots \uplus V_k, E)$ be an instance of \SVC{} which satisfies these conditions.
    Let $\opt_G$ be the optimal objective value for \SVC{} on $G$, $M \subseteq E$ be a maximal matching in $G$, and $C \subseteq V$ be $\bigcup_{e \in M} e$.
    We claim that $C$ is a $k$-approximate solution.

    Since $M$ is maximal, $C$ is a vertex cover of $G$. Because every edge has endpoints in distinct vertex classes,
    for each $i \in [k]$ we have that
    \[
        |C \cap V_i| \leq \frac{|C|}{2} \leq |M|.
    \]

    To obtain a lower bound, we observe that because $M$ is a matching, every vertex cover of $G$ has cardinality at least $|M|$.
    It follows from the pigeon-hole principle that $\opt_G \geq |M|/k$. Thus,
    for each $i \in [k]$ we have that $|C \cap V_i| \leq k\cdot\opt_G$, as desired.

    The result for \cfminECC{} now follows from the reduction of~\cref{thm:cfminecc-reduce-sparse-vc}.
\end{proof}