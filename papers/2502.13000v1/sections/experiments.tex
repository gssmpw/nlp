\section{Experiments}

\newcommand{\algecc}{\textsf{ST}}
\newcommand{\algcf}{\textsf{CF}}
\newcommand{\algpc}{\textsf{PC}}

\begin{figure*}[t!]
    \centering
    % \subfiguretopcaptrue
    \subfigure[Constraint Violation]{
        \label{fig:pcecc_experiments_constraint}
        \includegraphics[page=5, width=0.35\paperwidth]{figures/protected_color_plots_no_legends.pdf}
    }
    ~~~
    \subfigure[Runtime]{
        \label{fig:pcecc_experiments_runtime}
        \includegraphics[page=7, width=0.35\paperwidth]{figures/protected_color_plots_no_legends.pdf}
    }
    ~~~
    \subfigure[Protected Color Objective Approximation Upper Bound]{
        \label{fig:pcecc_experiments_pc_approx}
        \includegraphics[page=6, width=0.35\paperwidth]{figures/protected_color_plots_no_legends.pdf}
    }
    ~~~
    \subfigure[MinECC Objective Approximation Upper Bound]{
        \label{fig:pcecc_experiments_st_approx}
        \includegraphics[page=8, width=0.35\paperwidth]{figures/protected_color_plots_no_legends.pdf}
    }
    $\begin{alignedat}{10}
            \textcolor[RGB]{229, 116, 100}{\bullet} \text{~Brain~}                           & ~~~ &
            \textcolor[RGB]{153, 154, 35}{\blacktriangle} \text{~Cooking~}                   & ~~~ &
            \textcolor[RGB]{73, 178, 117}{\blacksquare} \text{~DAWN~}                        & ~~~ &
            \textcolor[RGB]{67, 164, 241}{\boldsymbol{+}} \text{~MAG-10~}                    & ~~~ &
            \textcolor[RGB]{110, 155, 248}{\boldsymbol{\boxtimes}} \text{~Trivago-Clickout~} & ~~~ &
            \textcolor[RGB]{211, 108, 236}{\boldsymbol{\ast}} \text{~Walmart-Trips~}
        \end{alignedat}$
    \caption{Results running \pcECC{} on multiple datasets with a varying constraint on the number of unsatisfied edges for the color with the median amount of edges.
        Figure \subref{fig:pcecc_experiments_constraint} is the percent of constraint violation of the protected color. Line $y=x$ represents the constraint imposed by the problem definition, and $y=2x$ is the theoretical limit for our bicriteria approximation algorithm. Figure~\subref{fig:pcecc_experiments_runtime} is the runtime in seconds for solving the linear program (note that the rounding step has a negligible runtime). Figure \subref{fig:pcecc_experiments_pc_approx} is the approximation upper bound for the protected color objective. Figure \subref{fig:pcecc_experiments_st_approx} is the approximation upper bound for the standard MinECC objective.
        Approximation upper bounds are given by the objective value for the algorithm divided by the lower bound for the objective determined by the LP relaxation.
    }
    \label{fig:pcecc_experiments}
\end{figure*}


\begin{table}[t]
    \centering
    \caption{We report statistics for a standard suite of ECC datasets from~\cite{amburg2020clustering,veldt2023optimal}, as well as results and runtimes comparing the standard~(\algecc{}) \minecc{} algorithm versus the \cfminECC{}~(\algcf{}). Runtimes are given in seconds taken to solve the LP. Objective numbers are an upper bound on the approximation ratio, given by the objective value for the algorithm divided by the lower bound for the objective determined by the LP relaxation.}
    \label{tab:cfecc_experiments}
    \begin{tabular}{ llllllllllllllll }
        \toprule
        % \begin{noindent}
        \multirow{2}{*}{\textbf{Dataset}} & \multirow{2}{*}{\textbf{n}} & \multirow{2}{*}{\textbf{m}} & \multirow{2}{*}{\textbf{r}} & \multirow{2}{*}{\textbf{k}} & \multicolumn{2}{c}{\textbf{Runtime (s)}} & \multicolumn{2}{c}{\textbf{ECC Obj}} & \multicolumn{2}{c}{\textbf{CFECC Obj}} \\
        \cmidrule(lr){6-7} \cmidrule(lr){8-9} \cmidrule(lr){10-11}
         & & & & & \multicolumn{1}{c}{\algecc{}} & \multicolumn{1}{c}{\algcf{}} & \multicolumn{1}{c}{\algecc{}} & \multicolumn{1}{c}{\algcf{}} & \multicolumn{1}{c}{\algecc{}} & \multicolumn{1}{c}{\algcf{}} \\
         % \end{noindent}
        \cmidrule(lr){1-11}
        Brain            & 638    & 21180  & 2  & 2  & 0.50   & 1.57   & 1.00 & 1.02 & 1.26 & 1.00 \\
        Cooking          & 6714   & 39774  & 65 & 20 & 46.76  & 52.88  & 1.00 & 1.00 & 1.66 & 1.66 \\
        DAWN             & 2109   & 87104  & 22 & 10 & 4.49   & 11.60  & 1.00 & 1.00 & 1.39 & 1.38 \\
        MAG-10           & 80198  & 51889  & 25 & 10 & 8.40   & 15.52  & 1.00 & 1.20 & 1.48 & 1.03 \\
        Trivago-Clickout & 207974 & 247362 & 85 & 55 & 116.94 & 129.55 & 1.00 & 1.37 & 1.68 & 1.01 \\
        Walmart-Trips    & 88837  & 65898  & 25 & 44 & 156.71 & 314.03 & 1.00 & 1.05 & 2.48 & 1.56 \\
        \bottomrule
    \end{tabular}
\end{table}


While we primarily focus on theoretical results, we also implemented and evaluated the performance of our alternate objective ECC algorithms on a standard suite of benchmark hypergraphs~\cite{veldt2023optimal,amburg2020clustering}.
LP-based methods for standard ECC are already known to perform far better in practice than their theoretical guarantees, often yielding optimal solutions~\cite{amburg2020clustering}. Thus, our \maxecc{} algorithms should be viewed as theoretical results that help bridge the theory-practice gap and are thus not the focus of our experiments.

Algorithms based on LP rounding were implemented for the \cfminECC{} and \pcECC{} problems. These algorithms were implemented in the Julia programming language and were run on a research server running Ubuntu 20.04.1 with two AMD EPYC 7543 32-Core Processors and 1 TB of RAM. Our implementation of these algorithms chooses a color that has the lowest LP distance variable among all colors for each node. Note that this can only improve our approximation guarantees over the theoretical approach that only assigns a color if there is an LP variable that is less than $1/2$. We used Gurobi optimization software with default settings to solve the LP for all algorithms, which was the largest bottleneck for running the algorithms. The rounding step has a runtime less than $0.3$ seconds in all cases, which is negligible in comparison with solving the linear program. Statistics about the datasets can be found in Table~\ref{tab:cfecc_experiments}. The code for our experiments can be found at~\url{https://github.com/tommy1019/AltECC}.
We refer to our LP rounding algorithms for \cfminECC{} and \pcECC{} as  \algcf{} and \algpc{} respectively, and refer the standard \minecc{} LP rounding algorithm as \algecc{}.

\textbf{\textsc{Color-fair MinECC}}.
Table \ref{tab:cfecc_experiments} compares \algcf{} against \algecc{}. We first observe that in practice, \algcf{} achieves approximation factors that are much better than the theoretical bound of 2, and in several cases are very close to 1.
Furthermore, \algcf{} even does well approximating the standard \minecc{} objective even though it is not directly designed for it. In the worst case, it achieves a 1.37-approximation, which is only $37\%$ worse than \algecc{}, which finds an optimal solution.
% ECC_ST <- c(1.0, 1.0, 1.0, 1.0, 1.0, 1.0)
% ECC_CF <- c(1.02, 1.00, 1.00, 1.20, 1.37, 1.05)
% (ECC_CF / ECC_ST - 1)
%
% CF_ST <- c(1.26, 1.66, 1.39, 1.48, 1.68, 2.48)
% CF_CF <- c(1.0, 1.66, 1.38, 1.03, 1.01, 1.56)
% (CF_ST / CF_CF - 1)
By comparison,  \algecc{} is up to $66\%$ worse than \algcf{} for the color-fair objective.
%compared to \algcf{}, \algecc{} is up to a $66\%$ worse approximation ratio on this objective. 
In fact, on average among the datasets we tested, \algcf{} is only $10.66\%$ worse than \algecc{} on the standard \minecc{} objective, while \algecc{} is $32.62\%$ worse than \algcf{} on the color-fair objective.
In other words, the cost of incorporating fairness into the ECC framework is not too high---\algcf{} maintains good performance in terms of the \minecc{} objective while showing significant improvements for the color-fair objective.

% minstakes -> unsatifyed edges
\textbf{\textsc{Protected-Color MinECC}}.
Figure \ref{fig:pcecc_experiments} shows the constraint satisfaction, runtime, \algpc{} approximation bounds, and \algecc{} approximation bounds as $b$ (the number of unsatisfied protected edges) varies. Again in we see that our algorithm tends to far exceed theoretical guarantees.
%PC satifys up to 105% of this bound for these 4.
In most cases, Figure \ref{fig:pcecc_experiments_constraint} shows that the number of protected edges left unsatisfied by \algpc{} is below the constraint $b$, even though the algorithm can in theory violate this constraint by up to a factor 2. Results are especially good for Brain, MAG-10, Walmart, and Trivago-Clickout, where the constraint tends to be satisfied and in Figure \ref{fig:pcecc_experiments_pc_approx} we see the objective value for \algcf{} is within a factor $1.007$ of the LP lower bound.
%On $4$ out of $6$ datasets (Brain, MAG-10, Walmart, Trivago-Clickout), \algpc{} satisfies or violates the constraint by up to $5\%$, even though in theory it is only guaranteed to stay below a bounded factor of the protected color constraint.
% PC Apprix ratio within 1.006995 for these four
%For these four datasets, \algpc{} achieves approximation ratios below $1.007$, which is very near optimal. 
\algpc{} also tends to satisfy the protected color constraint for DAWN and Cooking when $b$ is below $50\%$ of the total number of protected edges, achieving a 1.3-approximation or better in this case. Once $b$ is above this $50\%$ threshold, \algpc{} satisfies no protected edges on DAWN and Cooking. This still matches our theory: if we want to \emph{guarantee} (based on our theory) that we do not leave all protected edges unsatisfied, $b$ must be less than $50\%$ of these edges. We also see in Figure \ref{fig:pcecc_experiments_st_approx} that \algpc{} does a good job of approximating the standard \minecc{} objective (factor 2 or better), while incorporating this protected color constraint.

We note that the rightmost point in the plots in Figure~\ref{fig:pcecc_experiments} corresponds to standard \minecc{} as there is no bound on unsatisfied protected edges. From this, we can examine how badly \algecc{} violates protected color constraints for small $b$ values. On MAG-10 when $b$ is $10\%$ of the number of protected edges, \algecc{} violates the constraint by a factor of $3.60$ whereas \algpc{} only violates the constraint by a factor of $1.01$. This gap is even more pronounced for the DAWN dataset at $10\%$, where \algecc{} violates the constraint by a factor of $10.3$ while \algpc{} satisfies the constraint. Thus, for situations where there is a need to protect certain edge types, standard \minecc{} algorithms do not provide meaningful solutions.

\section{Conclusions}
We have established the first approximation algorithm for hypergraph \maxecc{} and improved the best known approximation factor for graph \maxecc{}. We also introduced two ECC variants that incorporate fairness, designing new approximation algorithms which in practice far exceed their theoretical guarantees. One open direction is to try to further improve the best approximation ratio for \maxecc{}. This may require deviating significantly from previous LP rounding techniques, since it is currently very challenging to improve the approximation by even a small amount using this approach. Another open direction is to improve on the $2^{1/p}$-approximation for \pmeanECC{} when $p< 1$.