\section{Preliminaries}
For a positive integer $n$, let $[n] = \{1,2, \hdots n\}$. 
We use bold lowercase letters to denote vectors, and indicate the $i$th of entry of a vector $\textbf{x} \in \mathbb{R}^n$ by $x_i$. For a set $S$ and a positive integer $t$, let ${S \choose t}$ denote all subsets of $S$ of size $t$.
An instance of edge-colored clustering is given by a hypergraph $H = (V,E, \ell)$ where $V$ is a node set, $E$ is a set of hyperedges (usually just called \emph{edges}), and $\ell \colon E \rightarrow [k]$ is a mapping from edges to a color set $[k] = \{1,2, \hdots, k\}$.  For $e \in E$, we let $\omega_e \geq 0 $ denote a nonnegative weight associated with $e$, which equals 1 for all edges in the unweighted version of the problem. For a color $c \in [k]$, let $E_c \subseteq E$ denote the edges of color $c$.

The goal of different variants of ECC is to construct a map $\lambda \colon V \rightarrow [k]$ that associates each node with a color. An edge $e \in E$ is \emph{satisfied} if $\ell(e) = \lambda(v)$ for every $v \in e$, and is otherwise \emph{unsatisfied}. The simplest variant is to color nodes to maximize the number of satisfied edges (\maxecc{}), which can be cast as the following binary linear program
\begin{align}
	\label{eq:maxecc}
\begin{aligned}
	\text{max} \quad& \textstyle \sum_{e \in E} \omega_e z_e \\
	\text{s.t.} \quad& \forall v \in V:\\
	&\forall c \in [k], e \in E_c:\\
	&x_v^c, z_e \in \{0, 1\}
\end{aligned}
\begin{aligned}
	&\\
	&\textstyle \sum_{c=1}^k x_v^c = 1 \\
	\quad &x_v^c \geq z_e \quad \forall v \in e \\
	&\forall c \in [k], v \in V, e \in E.
\end{aligned}
\end{align}
Setting $x_{u}^c$ to 1 indicates that node $u$ is given color $c$, i.e., $\lambda(u) = c$). We use $\textbf{x}_u = \begin{bmatrix} x_u^1 & x_u^2 & \cdots & x_u^k \end{bmatrix}$ to denote the vector of distances for node $u$. For edge $e \in E$, the constraints are designed in such a way that $z_e = 1$ if and only if $e$ is satisfied. A binary LP for \minecc{} can be obtained by changing the objective function above.
%but instead minimizing the objective $\sum_{e \in E} \omega_e (1 - z_e)$. 

%\textbf{Prior results on ECC.}
%Edge-colored clustering has been well-studied in the machine learning and data mining literature from the perspective of approximation algorithms~\cite{amburg2020clustering,amburg2022diverse,veldt2023optimal,crane2024overlapping,angel2016clustering} and fixed parameter tractability results~\cite{kellerhals2023parameterized,cai2018alternating,crane2024overlapping}. 
%The earliest work on edge-colored clustering focused on the \maxecc{} objective in graphs, proving a $1/e^2$ approximation algorithm by rounding a linear programming (LP) relaxation~\cite{angel2016clustering}, The rounding scheme was subsequently improved several times~\cite{ageev2015improved,alhamdan2019approximability} leading up to an approximation factor of 0.3622~\cite{ageev20200}.~\citet{amburg2020clustering} were the first to study the problem over hypergraphs, proving approximation algorithms for \minecc{}, which were later improved~\citet{veldt2023optimal}. Other variants of ECC include a regularized objective designed to create teams that are diverse and experienced with respect to past interaction types~\cite{amburg2022diverse}, and generalized ECC objectives that allow overlap in clusters~\cite{crane2024overlapping}. Several papers have also focused on fixed-parameter tractability results for different ECC objectives~\cite{cai2018alternating,kellerhals2023parameterized,crane2024overlapping}. 

%\textbf{Additional related work.}
%ECC is also closely related to chromatic correlation clustering (CCC)~\cite{bonchi2012chromatic,bonchi2015chromatic,anava2015improved,klodt2021color,xiu2022chromatic}, an edge-colored variant of correlation clustering~\cite{bansal2004correlation}. CCC can be viewed as variant of ECC in graphs in which one may have multiple clusters of the same color, and one additionally incurs penalties for clustering two nodes in the same colored cluster if they do not share an edge of that color. We note finally that our work on balanced and fair variants of ECC closely relates to other recent work on fair clustering and its applications, where the goal is to form cluster assignments that are in some sense balanced or proportional in terms of some class or feature within the data. We refer to the seminal work of \cite{chierichetti2017fair} and Section 5.3 of a recent survey~\cite{caton2024fairness} for more details. 
%As a final area of related work, our motivation to study balanced and fair variants of ECC closely relates to other recent work on fair clustering and its various applications, which has seen a , where the goal is to form cluster assignments that are in some sense balanced or proportional in terms of some class or feature within the data. We refer to the seminal work of \cite{chierichetti2017fair} and a recent survey 