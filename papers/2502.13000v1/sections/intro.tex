\section{Introduction}
\label{sec:intro}
Edge-colored clustering (ECC) is an optimization framework for clustering datasets characterized by {categorical} relationships among data points. The problem is formally encoded as an edge-colored hypergraph (Figure~\ref{fig:eccbalance}), where each edge represents an interaction between data objects (the nodes) and the color of the edge indicates the \emph{type} or \emph{category} of that interaction. The goal is to assign colors to nodes in such a way that edges of a color tend to include nodes of that color, by minimizing or maximizing some objective function relating edge colors and node colors.
ECC algorithms have been applied to various clustering tasks where cluster labels naturally match with interaction types. For example, if nodes are researchers, edges are author lists for publications, and colors indicate publication field (computer science, biology, etc.), then ECC provides a framework for inferring researchers' fields based on publications. ECC has also been used for temporal hypergraph clustering~\cite{amburg2020clustering}, where edge colors encode time windows in which interactions occur. ECC then clusters nodes into time windows in which they are especially active. Variants of ECC have also been used for team formation~\cite{amburg2022diverse}, in which case nodes are people, edges represent team tasks, and colors indicate task type. In this setting, ECC corresponds to assigning tasks based on prior team experiences.
\begin{figure}[t]
	\centering
	\includegraphics[width = .6\linewidth]{figures/ecc-horizontal.pdf}
	\vspace{-5pt}
	\caption{The node coloring on the left satisfies 4 edges (2 blue, 2 red, 0 green). The coloring on the right only satisfies 3, but each color has a satisfied edge.}
	\label{fig:eccbalance}
%	\vspace{-15pt}
\end{figure}


\textbf{Related work and research gaps.}
Edge-colored clustering has been well-studied in the machine learning and data mining literature from the perspective of approximation algorithms~\cite{amburg2020clustering,amburg2022diverse,veldt2023optimal,crane2024overlapping,angel2016clustering} and fixed-parameter tractability results~\cite{kellerhals2023parameterized,cai2018alternating,crane2024overlapping}. It is also closely related to chromatic correlation clustering~\cite{bonchi2012chromatic,bonchi2015chromatic,anava2015improved,klodt2021color,xiu2022chromatic}, which is an edge-colored variant of correlation clustering~\cite{bansal2004correlation}.
Several variants of ECC have been encoded using different combinatorial objective functions for assigning colors to nodes. The earliest and arguably the most natural is \maxecc{}~\cite{angel2016clustering}, which seeks to maximize the number of \textit{satisfied} edges---edges in which all the nodes match the color of the edge. More recent attention has been paid to \minecc{}~\cite{amburg2020clustering}, where the goal is to minimize the number of \textit{unsatisfied} edges. The latter is the same as \maxecc{} at optimality but differs in terms of approximations and fixed-parameter tractability results.
%Nearly all approximation algorithms for \minecc{} apply to hypergraphs with arbitrary hyperedge sizes. 
The approximability of \minecc{} is currently well-understood: a recent ICML paper designed improved approximation guarantees that are tight with respect to linear programming integrality gaps and nearly tight with respect to approximation hardness bounds~\cite{veldt2023optimal}.
However, existing approximations for \maxecc{} apply only to graph inputs (whereas nearly all algorithms for \minecc{} apply to hypergraphs), and there is a much larger gap between the best approximation guarantees and known lower bounds.

Another limitation of prior work is that nearly all objective functions for ECC (including \minecc{} and \maxecc{}) only consider the total number of edges that are (un)satisfied, even if it means certain colors are disproportionately (un)satisfied. Figure~\ref{fig:eccbalance} provides a small example where the optimal  \minecc{} solution satisfies four edges but only from two colors. Meanwhile, another color assignment satisfies only three edges but includes a satisfied edge of each color. The latter choice is natural for applications where one may wish to incorporate some notion of balance or fairness in edge satisfaction. For example, if coloring nodes means assigning individuals to tasks for future team interactions, then we would like to avoid situations where one type of task is assigned no workers. Balanced and fair objectives have been studied and applied for many other recent clustering frameworks, but have yet to be explored in the context of edge-colored clustering. For prior work on fair clustering, we refer to the seminal work of~\citet{chierichetti2017fair} and a recent survey by~\citet{caton2024fairness}.

Motivated by the above, we present new contributions to ECC along two frontiers: improved  \maxecc{} algorithms, and new frameworks for balanced and fair ECC variants.


\textbf{Our contributions for \maxecc{}}.
We present the first approximation algorithm for hypergraph \maxecc{}, which has an approximation factor of
%$\frac{2^r}{(r+1)e^r}$-approximation 
$(2/e)^r(r+1)^{-1}$
where $r$ is the maximum hyperedge size. The approximation factor goes to zero as $r$ increases, but this is expected since a prior hardness result rules out the possibility of constant-factor approximations that hold for arbitrarily large $r$~\cite{veldt2023optimal}. Our result shows that non-trivial constant-factor approximations can be obtained when $r$ is a constant. We use insights from our hypergraph algorithm to obtain a new best approximation factor of $154/405 \approx 0.38$ for the graph version of the problem, improving on the previous best 0.3622-approximation shown five years ago~\cite{ageev20200}. While the increase in approximation factor appears small at face value, this improves on a long sequence of papers on \maxecc{} for the graph case~\cite{angel2016clustering,ageev2015improved,alhamdan2019approximability,ageev20200}. Obtaining even a minor increase in the approximation factor is highly non-trivial and constitutes the most technical result of our paper.

\textbf{Our contributions for balanced and fair ECC variants.}
Our first contribution towards balanced and fair variants of ECC is to introduce the generalized \pmeanECC{} objective, which uses a parameter $p$ to control balance in unsatisfied edges, and captures \minecc{} as a special case when $p = 1$. For $p = \infty$, the objective corresponds to a new problem we call \cfminECC{}, where the goal is to minimize the maximum number of edges of any color that are unsatisfied. We prove \cfminECC{} is NP-hard even for $k = 2$ colors, even though standard \minecc{} is polynomial-time solvable for $k = 2$. We then show how to obtain a 2-approximation by rounding a linear program (LP), which matches an LP integrality gap lower bound. More generally, we obtain a 2-approximation for \pmeanECC{} for every $p \geq 1$ by rounding a convex relaxation, and give a $2^{1/p}$-approximation for every $p \in (0,1)$. Our results also include several parameterized complexity and hardness results for \cfminECC{} and a related maximization variant where the goal is to maximize the minimum number of edges of any one color that are satisfied.

We also consider scenarios in which the ``importance'' of the edge colors is unequal. In particular, we
introduce \pcECC{}, where we are given a special color $c_1$ which can be thought of as encoding a protected interaction type. The goal is to color nodes to minimize the number of unsatisfied edges subject to a strict upper bound on the number of unsatisfied edges of color $c_1$. We give bicriteria approximation algorithms based on rounding a linear programming relaxation, meaning that our algorithm is allowed to violate the upper bound for the protected color by a bounded amount, in order to find a solution that has a bounded number of unsatisfied edges relative to the optimal solution.
%
Finally, although the contributions of our paper are primarily theoretical, we also implement our algorithms for \cfminECC{} and \pcECC{} on a suite of benchmark ECC datasets. Our empirical results show that (1) our algorithms outperform their theoretical guarantees in practice, and (2) our balanced and fair objectives still achieve good results with respect to the standard \minecc{} objective while better incorporating notions of balance and fairness.

