\subsection{Protected-color ECC}\label{sec:protected-class}

\cfminECC{} encodes fairness by ensuring that no single color is heavily ``punished'' (as measured by unsatisfied hyperedges), but what if we are only concerned about a single color that represents a protected interaction type?
It is always trivially possible to ensure that every edge of a given color $c_1$ is satisfied by setting $\lambda(v) = c_1$ for every vertex $v$.
This strategy, however, completely disregards all other edges.
We might reasonably assume that while we are constrained by concern for a given protected color, we still wish to
find a high-quality solution as measured by the traditional clustering objective.
This tension is captured in the following problem definition.

\problembox{\pcECC{}}{An edge-colored hypergraph $H = (V, E, \ell)$, two integers $t, b$, and a protected color $c_1$.}{Is it possible to color $V$ such that at most $t$ edges are unsatisfied, of which at most $b$ have color $c_1$?}

When $t = b$ we recover the standard $\ECC{}$ problem, so \pcECC{} is \cclass{NP}-hard.
However, by rounding fractional solutions to the following linear program we recover bicriteria $(\alpha, \beta)$-approximations. If $\opt_b$ is the minimum possible number of unsatisfied edges subject to the constraint $b$, then these algorithms guarantee at most $\alpha\cdot\opt_b$ unsatisfied edges, of which at most $\beta\cdot b$ have color $c_1$.\looseness=-1
\begin{align}
	\label{lp:pc-ecc}
\begin{aligned}
	\text{min} \quad& \textstyle \sum_{e \in E} \gamma_e \\[4pt]
	\text{s.t.} \quad& \forall v \in V:\\[6pt]
	&\forall c \in [k], e \in E_c:\\[2pt]
    &\textstyle\sum_{e \in E_{c_1}} \gamma_e \leq b \\[1pt]
	&d_v^c, \gamma_e \in [0, 1]
\end{aligned}
\begin{aligned}
	&\\[2pt]
	&\textstyle \sum_{c=1}^k d_v^c \geq k - 1 \\[5pt]
	\quad &d_v^c \leq \gamma_e \quad \forall v \in e \\
    & \\[4pt]
	&\forall c \in [k], v \in V, e \in E
\end{aligned}
\end{align}
All variables have the same meaning as in Program~(\ref{eq:p-mean-ecc}), but we have added the third constraint which encodes that at most $b$ edges of our protected color $c_1$ may be unsatisfied.
Apart from this constraint, LP~(\ref{lp:pc-ecc}) is identical to the canonical \minecc{} LP used to obtain
state-of-the-art approximations~\cite{amburg2020clustering,veldt2023optimal}, and our algorithm can be seen as
an appropriate adaption of those rounding schemes. In the following, $\rho \in (0, \frac{1}{2}]$ is a parameter which allows us to tune the approximation guarantees $\alpha$ and $\beta$.

\begin{restatable}{theorem}{pceccapproximation}\label{thm:pc-ECC-approximation}
    For every $\rho \in (0, \frac{1}{2}]$, there exists a polynomial-time $(\frac{1}{\rho}, \frac{1}{1 - \rho})$-approximation for \pcECC{}.
\end{restatable}
\begin{proof}
    We begin by computing optimal fractional values $\{\gamma_e\}, \{d_v^c\}$ for LP~(\ref{lp:pc-ecc}).

    We claim that for every pair of distinctly colored overlapping hyperedges $e, f$, $\gamma_e + \gamma_f \geq 1$.
    Otherwise, let $v \in e \cap f$, $c_e = \ell(e)$, and $c_f = \ell(f)$. Since $\gamma_e + \gamma_f < 1$,
    it follows from the second constraint of LP~(\ref{lp:pc-ecc}) that $d_v^{c_e} + d_v^{c_f} < 1$. Then
    \[
        \sum_{c = 1}^{k} d_v^c < 1 + k - 2 = k - 1,
    \]
    contradicting the first constraint of LP~(\ref{lp:pc-ecc}).

    Now we show how to color the vertices of our hypergraph $H = (V, E)$ with protected color $E_1$.
    For each hyperedge $e$, we set
    \[
        \hat{\gamma}_e = \begin{cases}
            1 \quad\text{ if } e \in E_1 \text{ and } \gamma_e \geq 1 - \rho \\
            1 \quad\text{ if } e \notin E_1 \text{ and } \gamma_e \geq \rho  \\
            0 \quad\text{ otherwise,}
        \end{cases}
    \]

    and we construct the set $S = \{e \in E \colon \hat{\gamma}_e = 1\}$.

    First we observe that since $\rho \in (0, \frac{1}{2}]$, $\frac{1}{1 - \rho} \leq \frac{1}{\rho}$, and so
    \[
        |S| = \sum_{e \in E} \hat{\gamma}_e \leq \frac{1}{\rho}\sum_{e \in E \setminus E_1} \gamma_e + \frac{1}{1 - \rho}\sum_{e \in E_1} \gamma_e \leq \frac{1}{\rho}\sum_{e \in E} \gamma_e,
    \]
    and
    \[
        |S \cap E_1| \leq \frac{1}{1 - \rho}\sum_{e \in E_1} \gamma_e \leq \frac{b}{1 - \rho}.
    \]

    All that remains is to compute a coloring $\lambda$ which satisfies every hyperedge in $E \setminus S$. If $S$ contains at least one member of every pair of distinctly colored overlapping hyperedges, then computing such a $\lambda$ is trivial.
    We conclude the proof by showing that $S$ has this characteristic.
    Let $e, f \in E$ with $e \cap f \neq \emptyset$ and $\ell(e) \neq \ell(f)$.
    Assume toward a contradiction that $e, f \notin S$.
    Because $e$ and $f$ are distinctly colored, at least one is in $E \setminus E_1$.
    If both $e, f \notin E_1$, then
    \[
        \gamma_e + \gamma_f < 2\rho \leq 1,
    \]
    but we have already shown that $\gamma_e + \gamma_f \geq 1$, so exactly one of $e$ or $f$ must be in $E_1$. Assume without loss of generality that $e \in E_1$ and $f \in E \setminus E_1$.
    Then
    \[
        \gamma_e + \gamma_f < 1 - \rho + \rho = 1,
    \]
    so we have contradicted that $S \cap \{e, f\} = \emptyset$, as desired.

\end{proof}

%Also, \pcECC{} is fixed-parameter tractable in the total number $t$ of unsatisfied edges.\looseness=-1
We also prove the following fixed-parameter tractability result, via a standard branching argument which is essentially identical to that of~\Cref{thm:cfminecc-FPT}.
We defer the details to~\Cref{app:pc}.
\begin{restatable}{theorem}{pceccFPT}\label{thm:pc-ECC-FPT}
    \pcECC{} is FPT with respect to the total number $t$ of unsatisfied edges.
\end{restatable}

Given these positive results, a natural question is whether we may efficiently solve the \minecc{} objective with \emph{multiple} protected colors.
We conclude by showing that as soon as we allow more than one protected color, it becomes hard even to determine whether any solution exists.  

\begin{restatable}{theorem}{multiplepcNPhard}\label{thm:pc-multiple-classes-NP-hard}
	Given $H = (V, E, \ell)$, two integers $b_1, b_2$, and two colors $c_1, c_2$, it is \cclass{NP}-hard to determine whether
	it is possible to color $V$ such that at most $b_1$ edges of color $c_1$ and at most $b_2$ edges of color $c_2$ are unsatisfied. 
\end{restatable}
\begin{proof}
    We once again reduce from \constrainedbipartiteVC{}, for which the input is a bipartite graph $G = (A \uplus B, E)$ along with two integers $\alpha_a$, $\alpha_b$, and the question is whether there exists a vertex cover $C$ with $|C \cap A| \leq \alpha_a$ and $|C \cap B| \leq \alpha_b$.
    We will construct the edge-colored hypergraph for which $G$ is the associated conflict graph.
    Formally, we construct an edge-colored hypergraph $H = (V, E_a \uplus E_b, \ell)$ with two colors $c_a$ and $c_b$ as follows. For each edge $e \in E$, we create a vertex $v_e \in V$.
    For each vertex $a \in A$, we create a hyperedge $e_a \in E_a$ such that $e_a = \{v_e \colon a \in e\}$. Similarly,
    for each vertex $b \in B$, we create a hyperedge $e_b \in E_b$ such that $e_b = \{v_e \colon a \in e\}$.
    For each $e_a \in E_a$ we set $\ell(e_a) = c_a$, and for each $e_b \in E_b$ we set $\ell(e_b) = c_b$.
    We say that our protected colors are $E_a$ and $E_b$, and we set the associated constraints $b_1 = \alpha_a$ and $b_2 = \alpha_b$.

    Thus, if $\lambda$ is a vertex coloring of $H$, $S_\lambda$ is the set of hyperedges unsatisfied by $\lambda$, and we have that $|S \cap E_a| \leq b_1 = \alpha_a$ and $|S \cap E_b| \leq b_2 = \alpha_b$,
    then the set $C = \{a \colon e_a \in C \} \cup \{b \colon e_b \in C \}$ has the properties that $|C \cap A| \leq \alpha_a$ and $|C \cap B| \leq \alpha_b$.
    Moreover, if $ab = e$ is an edge in $e$, then $e_a$ and $e_b$ are distinctly colored, with $e_a \cap e_b = \{v_e\}$.
    Hence, $S$ contains at least one of $e_a, e_b$, meaning that $C$ contains at least one of $a, b$, so $C$ is a vertex cover.

    For the other direction, assume that $C$ is a vertex cover of $G$ with the properties that $|C \cap A| \leq \alpha_a = b_1$ and $|C \cap B| \leq \alpha_b = b_2$,
    so the set $S = \{e_a \colon a \in C\} \cup \{e_b \colon b \in C\}$ has the properties that $|S \cap E_a| \leq b_1$ and $|S \cap E_b| \leq b_2$.
    Observe that, if $e_a$ and $e_b$ are distinctly colored hyperedges which overlap, then $ab \in E$. Thus, at least one of $a, b$ is contained in $C$, and so at least one of $e_a, e_b$ is contained in $S$.
    Then there are no pairs of distinctly colored overlapping hyperedges in $(E_a \uplus E_b)\setminus S$.
    Consequently, it is trivial to compute a coloring $\lambda$ which satisfies every hyperedge not contained in $S$, meaning that it leaves at most $b_1$ hyperedges of color $c_a$ unsatisfied, and at most $b_2$ hyperedges of color $c_b$ unsatisfied.
\end{proof}
