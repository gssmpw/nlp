Our hypergraph \maxecc{} approximation algorithm (Algorithm~\ref{alg:hyper_maxecc}) generates a uniform random priority for colors $\pi$, and independent random color thresholds $\{\alpha_c\}$ to determine which nodes want which colors. A node is then simply assigned to the highest priority color it wants. The proof of our approximation guarantee uses the following lemma, presented by~\citet{angel2016clustering} when developing the first approximation algorithm for graph \maxecc{}.
\begin{lemma}
	\label{lem:angel}
	Let $\left\{E_1, E_2, \dots, E_j \right\}$ be a set of independent events satisfying $\sum_{i=1}^j \prob[E_i] \leq 1$, then the probability that {at most one} of them happens is greater than or equal to $2/e$.
\end{lemma}
%Our new method satisfied the following result.
\begin{algorithm}[t]
    \caption{\textsf{HyperMaxECC}}
    \label{alg:hyper_maxecc}
    \begin{algorithmic}
        	\STATE Obtain optimal LP variables $\{z_e; x_v^c\}$ for the relaxation
        \STATE $\pi \leftarrow \text{uniform random ordering of colors } [k]$
        \STATE For $c \in [k]$, $\alpha_c \leftarrow $ uniform random threshold in $[0,1]$
        \FOR{$v \in V$}
        \STATE $W_v' = \{c \in [k] \colon \alpha_c < x_v^c\}$
        \IF{$|W_v'| > 0$}
        \STATE $\lambda(v) \leftarrow \argmax_{c \in W_v'} \pi(c)$
        \ELSE
        \STATE $\lambda(v) \leftarrow $ arbitrary color
        \ENDIF 
        \ENDFOR
    \end{algorithmic}
\end{algorithm}

\begin{theorem}
    Algorithm \ref{alg:hyper_maxecc} is a $\frac{1}{r+1} \left(\frac{2}{e}\right)^r$ approximation algorithm for \hypermaxecc{}.
\end{theorem}
\begin{proof}
Fix an arbitrary edge $e \in E$ with $c = \ell(e)$. Let $T_e$ denote the event that $e$ is satisfied. 
By Observation~\ref{obs:prob} it suffices to show $\prob[T_e] \geq \frac{1}{r+1} \left(\frac2e\right)^r$. 

Let $C = [k]\setminus \{c\}$ denote colors other than $c$. 
% For each $i \in C$, we would like quantify the possibility that some node in $e$ wants color $i$, as this opens up the possibility that $e$ will be dissatisfied because some node is given color $i$. Towards this goal, we identify a node in $e$ that has the highest likelihood of wanting $i$. Formally, for each color $i \in C$ we identify some node $v \in e$ satisfying $x_v^i \geq x_u^i$ for every $u \in e$ (breaking ties arbitrarily if multiple nodes satisfy this), and we define $\sigma(i) = v$. The definition of $\sigma(i)$ implies the following useful observation:
% \begin{observation}
% \label{obs:cv}
%     % One or more nodes in $e$ wants color $i$ if and only if $v = \sigma(i)$ wants color $i$. Equivalently, 
%     Color $i$ is not wanted by \textbf{any} nodes in $e$ $\iff$ color $i$ is not wanted by $v = \sigma(i)$.
% \end{observation}
To partition the color set $C$, for each $v \in e$ we define
\begin{equation*}
    C_v = \{i \in C \colon \sigma_e(i) = v\}
\end{equation*}
where we recall $\sigma_e(i)$ identifies a node $v \in e$ satisfying $x_v^i = \max_{u \in e} x_u^i$.
Let $A_v$ denote the event that at most one color in $C_v$ wants one or more nodes in $e$. From Observation~\ref{obs:cv}, $A_v$ is equivalent to the event that  $v$ wants at most 1 color in $C_v$. The probability of $A_v$ is therefore the probability that at most one of the events $\{X_v^i \colon i \in C_v \}$ happens. Combining Observation~\ref{obs:xuc} with the LP constraint $\sum_{i = 1}^k x_v^i = 1$ tells us that the assumptions of Lemma~\ref{lem:angel} hold, which in turn implies that $\prob[A_v] \geq 2/e$.
Observe now that the events $\{A_v \colon v \in e \}$ are mutually independent because the sets $\{C_v \colon v \in e\}$ are disjoint sets of colors, and the color thresholds $\{\alpha_i \colon i \in C\}$ are all drawn independently from each other. 

% To bound the probability of $T_e$ below, we will condition on the event that (i) every node in $e$ wants $c$ (i.e., event $Z_e$), and (ii) at most one color from each color set in $\{C_v \colon v \in e\}$ is wanted by one or more nodes in $e$ (i.e., $A_v$ holds for every $v \in e$). 
Since color thresholds are drawn independently for each color, we know the events $\{A_v, X_v^c \colon v \in e\}$ are all mutually independent.
%\footnote{For this independence argument to hold it is important that $c \notin C = [k] \backslash \{c\}$, meaning that nodes in $e$ wanting $c$ is not affected by which colors in $\{C_v \colon v \in e\}$ are wanted.}
Therefore, 
\begin{align*}
    \prob\left[ \left(\bigcap_{v \in e} A_v \right) \cap Z_e\right ] &= \prob[Z_e]\cdot \prod_{v \in e} \prob[A_v] \geq z_e \cdot \left(\frac2e \right)^r,
\end{align*}
where we have used Observation~\ref{obs:edgewants} and the independence of events. If both $Z_e$ and $\cap_{v \in e} A_v$ hold, this means that every node in $e$ wants color $c$, and at most $r$ distinct other colors (one for each node in $v \in e$ since there is one set $C_v$ for each $v \in e$) that want one or more nodes in $e$. Conditioned on this, in order for $e$ to be satisfied, the color $c$ must have a higher priority in the random ordering $\pi$ than the other $r$ colors, which happens with probability $1/(r+1)$. Putting this all together produces our needed probability bound:
\begin{align*}
    \prob[T_e] \geq \prob\left[T_e \mid Z_e \bigcap_{v \in e} A_v\right] \prob\left[ Z_e \bigcap_{v \in e} A_v \right] \geq \frac{z_e}{r+1}\left(\frac2e\right)^r.
\end{align*}


\end{proof}