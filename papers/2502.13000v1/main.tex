\documentclass[11pt]{article}
\usepackage[bottom=1in,left=1in, right=1in, top=1in]{geometry}
\geometry{letterpaper}

\usepackage{microtype}
\usepackage{graphicx}
\usepackage{subfigure}
\usepackage{booktabs}
\usepackage{hyperref}
 
% Attempt to make hyperref and algorithmic work together better:
\newcommand{\theHalgorithm}{\arabic{algorithm}}

% For theorems and such
\usepackage{amsmath}
\usepackage{amssymb}
\usepackage{mathtools}
\usepackage{amsthm}
\usepackage{thmtools, thm-restate}
\usepackage{notation}
\usepackage{enumitem}
\newtheorem{proposition}{Proposition}
\newtheorem{claim}{Claim}

% if you use cleveref..
\usepackage[capitalize,noabbrev]{cleveref}

% Todonotes is useful during development; simply uncomment the next line
%    and comment out the line below the next line to turn off comments
%\usepackage[disable,textsize=tiny]{todonotes}
\usepackage[textsize=tiny]{todonotes}

\usepackage{multirow}

\usepackage{times}
\usepackage{fancyhdr}
\usepackage{xcolor}
\usepackage{algorithm}
\usepackage{algorithmic}
\usepackage{natbib}
\usepackage{eso-pic}
\usepackage{forloop}
\usepackage{url}
\usepackage{subfigure}

\usepackage{notation}

\newcommand{\yrcite}[1]{\citeyearpar{#1}}
\renewcommand{\cite}[1]{\citep{#1}}

\usepackage{authblk}

\title{Edge-Colored Clustering in Hypergraphs: Beyond Minimizing Unsatisfied Edges}
\date{}
\author[1]{Alex Crane}
\author[2]{Thomas Stanley}
\author[1]{Blair D. Sullivan}
\author[2]{Nate Veldt}

\affil[1]{University of Utah}
\affil[2]{Texas A\&M University}

\begin{document}

\maketitle

\begin{abstract}
    We consider a framework for clustering edge-colored hypergraphs, where the goal is to cluster (equivalently, to \emph{color}) objects based on the primary {type} of multiway interactions they participate in. One well-studied objective is to color nodes to minimize the number of \textit{unsatisfied} hyperedges -- those containing one or more nodes whose color does not match the hyperedge color. We motivate and present advances for several directions that extend beyond this minimization problem. We first provide new algorithms for maximizing \textit{satisfied} edges, which is the same at optimality but is much more challenging to approximate, with all prior work restricted to graphs. We develop the first approximation algorithm for hypergraphs, and then refine it to improve the best-known approximation factor for graphs. We then introduce new objective functions that incorporate notions of balance and fairness, and provide new hardness results, approximations, and fixed-parameter tractability results.
\end{abstract}


\section{Introduction}


\begin{figure}[t]
\centering
\includegraphics[width=0.6\columnwidth]{figures/evaluation_desiderata_V5.pdf}
\vspace{-0.5cm}
\caption{\systemName is a platform for conducting realistic evaluations of code LLMs, collecting human preferences of coding models with real users, real tasks, and in realistic environments, aimed at addressing the limitations of existing evaluations.
}
\label{fig:motivation}
\end{figure}

\begin{figure*}[t]
\centering
\includegraphics[width=\textwidth]{figures/system_design_v2.png}
\caption{We introduce \systemName, a VSCode extension to collect human preferences of code directly in a developer's IDE. \systemName enables developers to use code completions from various models. The system comprises a) the interface in the user's IDE which presents paired completions to users (left), b) a sampling strategy that picks model pairs to reduce latency (right, top), and c) a prompting scheme that allows diverse LLMs to perform code completions with high fidelity.
Users can select between the top completion (green box) using \texttt{tab} or the bottom completion (blue box) using \texttt{shift+tab}.}
\label{fig:overview}
\end{figure*}

As model capabilities improve, large language models (LLMs) are increasingly integrated into user environments and workflows.
For example, software developers code with AI in integrated developer environments (IDEs)~\citep{peng2023impact}, doctors rely on notes generated through ambient listening~\citep{oberst2024science}, and lawyers consider case evidence identified by electronic discovery systems~\citep{yang2024beyond}.
Increasing deployment of models in productivity tools demands evaluation that more closely reflects real-world circumstances~\citep{hutchinson2022evaluation, saxon2024benchmarks, kapoor2024ai}.
While newer benchmarks and live platforms incorporate human feedback to capture real-world usage, they almost exclusively focus on evaluating LLMs in chat conversations~\citep{zheng2023judging,dubois2023alpacafarm,chiang2024chatbot, kirk2024the}.
Model evaluation must move beyond chat-based interactions and into specialized user environments.



 

In this work, we focus on evaluating LLM-based coding assistants. 
Despite the popularity of these tools---millions of developers use Github Copilot~\citep{Copilot}---existing
evaluations of the coding capabilities of new models exhibit multiple limitations (Figure~\ref{fig:motivation}, bottom).
Traditional ML benchmarks evaluate LLM capabilities by measuring how well a model can complete static, interview-style coding tasks~\citep{chen2021evaluating,austin2021program,jain2024livecodebench, white2024livebench} and lack \emph{real users}. 
User studies recruit real users to evaluate the effectiveness of LLMs as coding assistants, but are often limited to simple programming tasks as opposed to \emph{real tasks}~\citep{vaithilingam2022expectation,ross2023programmer, mozannar2024realhumaneval}.
Recent efforts to collect human feedback such as Chatbot Arena~\citep{chiang2024chatbot} are still removed from a \emph{realistic environment}, resulting in users and data that deviate from typical software development processes.
We introduce \systemName to address these limitations (Figure~\ref{fig:motivation}, top), and we describe our three main contributions below.


\textbf{We deploy \systemName in-the-wild to collect human preferences on code.} 
\systemName is a Visual Studio Code extension, collecting preferences directly in a developer's IDE within their actual workflow (Figure~\ref{fig:overview}).
\systemName provides developers with code completions, akin to the type of support provided by Github Copilot~\citep{Copilot}. 
Over the past 3 months, \systemName has served over~\completions suggestions from 10 state-of-the-art LLMs, 
gathering \sampleCount~votes from \userCount~users.
To collect user preferences,
\systemName presents a novel interface that shows users paired code completions from two different LLMs, which are determined based on a sampling strategy that aims to 
mitigate latency while preserving coverage across model comparisons.
Additionally, we devise a prompting scheme that allows a diverse set of models to perform code completions with high fidelity.
See Section~\ref{sec:system} and Section~\ref{sec:deployment} for details about system design and deployment respectively.



\textbf{We construct a leaderboard of user preferences and find notable differences from existing static benchmarks and human preference leaderboards.}
In general, we observe that smaller models seem to overperform in static benchmarks compared to our leaderboard, while performance among larger models is mixed (Section~\ref{sec:leaderboard_calculation}).
We attribute these differences to the fact that \systemName is exposed to users and tasks that differ drastically from code evaluations in the past. 
Our data spans 103 programming languages and 24 natural languages as well as a variety of real-world applications and code structures, while static benchmarks tend to focus on a specific programming and natural language and task (e.g. coding competition problems).
Additionally, while all of \systemName interactions contain code contexts and the majority involve infilling tasks, a much smaller fraction of Chatbot Arena's coding tasks contain code context, with infilling tasks appearing even more rarely. 
We analyze our data in depth in Section~\ref{subsec:comparison}.



\textbf{We derive new insights into user preferences of code by analyzing \systemName's diverse and distinct data distribution.}
We compare user preferences across different stratifications of input data (e.g., common versus rare languages) and observe which affect observed preferences most (Section~\ref{sec:analysis}).
For example, while user preferences stay relatively consistent across various programming languages, they differ drastically between different task categories (e.g. frontend/backend versus algorithm design).
We also observe variations in user preference due to different features related to code structure 
(e.g., context length and completion patterns).
We open-source \systemName and release a curated subset of code contexts.
Altogether, our results highlight the necessity of model evaluation in realistic and domain-specific settings.






\section{Improved \maxecc{} Algorithms}
\textbf{Preliminaries.} For a positive integer $n$, let $[n] = \{1,2, \hdots n\}$.
We use bold lowercase letters to denote vectors, and indicate the $i$th of entry of a vector $\textbf{x} \in \mathbb{R}^n$ by $x_i$. For a set $S$ and a positive integer $t$, let $\binom{S}{t}$ denote all subsets of $S$ of size $t$.
An instance of edge-colored clustering is given by a hypergraph $H = (V,E, \ell)$ where $V$ is a node set, $E$ is a set of hyperedges (usually just called \emph{edges}), and $\ell \colon E \rightarrow [k]$ is a mapping from edges to a color set $[k] = \{1,2, \hdots, k\}$.  For $e \in E$, we let $\omega_e \geq 0 $ denote a nonnegative weight associated with $e$, which equals 1 for all edges in the unweighted version of the problem. For a color $c \in [k]$, let $E_c \subseteq E$ denote the edges of color $c$. We use $r$ to denote the \emph{rank} of $H$, i.e., the maximum hyperedge size.

The goal of ECC is to construct a map $\lambda \colon V \rightarrow [k]$ that associates each node with a color, in order to optimize some function on edge \emph{satisfaction}. Edge $e \in E$ is \emph{satisfied} if $\ell(e) = \lambda(v)$ for each $v \in e$, and is otherwise \emph{unsatisfied}. The two most common objectives are maximizing the number of satisfied edges (\maxecc{}) or minimizing the number of unsatisfied edges (\minecc{}).

\textbf{Challenges in approximating \maxecc{}.} Although \minecc{} and \maxecc{} are equivalent at optimality, the latter is far more challenging to approximate. Due to an approximation-preserving reduction from \textsc{Independent Set}, it is NP-hard
to approximate \maxecc{} in hypergraphs of unbounded rank $r$ to within a factor $|E|^{1-\varepsilon}$~\cite{veldt2023optimal,zuckerman2006linear}. There also are simple instances (e.g., a triangle with 3 colors) where a simple $2$-approximation of~\citet{amburg2020clustering} for \minecc{} (round variables of an LP relaxation to 0 if they are strictly below $1/2$, otherwise round to 1) fails to satisfy \emph{any} edges.
These challenges do not rule out the possibility of approximating \maxecc{} when $r$ is constant. Indeed, there are many approximations for graph \maxecc{} ($r=2$), but these require lengthy proofs, rely fundamentally on the assumption that the input is a graph, and do not easily extend even to the $r = 3$ case. Here we provide a generalized approach that gives the first approximation guarantees for hypergraph \maxecc{}, when $r$ is constant. Our approach also provides a simplified way to approximate graph \maxecc{}; we design and analyze a refined algorithm that achieves a new best approximation factor for graph \maxecc{}, improving on a long line of previous algorithms~\cite{angel2016clustering,alhamdan2019approximability,ageev2015improved,ageev20200}.

\subsection{Technical preliminaries for LP rounding algorithms}
\maxecc{} can be cast as a binary linear program (BLP):
\begin{align}
	\label{eq:maxecc}
	\begin{aligned}
		\text{max} \quad  & \textstyle \sum_{e \in E} \omega_e z_e \\
		\text{s.t.} \quad & \forall v \in V:                       \\
		                  & \forall c \in [k], e \in E_c:          \\
		                  & x_v^c, z_e \in \{0, 1\}
	\end{aligned}
	\begin{aligned}
		      &                                      \\
		      & \textstyle \sum_{c=1}^k x_v^c = 1    \\
		\quad & x_v^c \geq z_e \quad \forall v \in e \\
		      & \forall c \in [k], v \in V, e \in E.
	\end{aligned}
\end{align}
Setting $x_{u}^c$ to 1 indicates that node $u$ is given color $c$ (i.e., $\lambda(u) = c$). We use $\textbf{x}_u = \begin{bmatrix} x_u^1 & x_u^2 & \cdots & x_u^k \end{bmatrix}$ to denote the vector of variables for node $u$. For edge $e \in E$, the constraints are designed in such a way that $z_e = 1$ if and only if $e$ is satisfied. A binary LP for \minecc{} can be obtained by changing the objective function to $\min \sum_{e \in E} \omega_e (1-z_e)$.

The LP relaxation for \maxecc{} can be obtained by relaxing the binary constraints in Binary Linear  Program~\eqref{eq:maxecc} to linear constraints $0 \leq x_v^c \leq 1$ and $0 \leq z_e \leq 1$. Solving this LP gives a fractional node-color assignment $x_v^c \in [0,1]$. The closer $x_v^c$ is to 1, the stronger this indicates node $v$ should be given color $c$. Our task is to round variables to assign one color to each node in a way that satisfies provable approximation guarantees.

Our algorithms (Algorithms~\ref{alg:hyper_maxecc} and~\ref{alg:graph_max_ecc}) are randomized. Both use the same random process to identify colors that a node ``wants''. For each $c \in [k]$ we {independently} generate a uniform random \emph{color threshold} $\alpha_c \in [0,1]$. If $x_u^c > \alpha_c$, we say that \emph{node $u$ wants color $c$}, or equivalently that color $c$ wants node $u$. Because a node may want more than one color, we use a random process to choose one color to assign, informed by the colors the node wants. Our approximation proofs rely on (often subtle) arguments about events that are independent from each other. We begin by presenting several useful observations that will aid in proving our results.

Our first observation is that if we can bound the expected cost of every edge in terms of the LP upper bound, it provides an overall expected approximation guarantee.
\begin{observation}
	\label{obs:prob}
	Let $\mathcal{A}$ be a randomized ECC algorithm and $p \in [0,1]$ be a fixed constant. If for each $e \in E$ we have $\prob[\text{$e$ is satisfied by  $\mathcal{A}$}] \geq p z_e$, then $\mathcal{A}$ is a $p$-approximation.
\end{observation}
Let $X_u^c$ denote the event that $u$ wants $c$, and $Z_e$ be the event that every node in edge $e \in E$ wants color $c = \ell(e)$.
\begin{observation}
	\label{obs:xuc}
	For each node $v \in V$, the events $\{X_v^c\}_{c \in [k]}$ are independent, and $\prob[X_v^c] = \prob[\alpha_c < x_v^c] = x_v^c \leq 1$.
\end{observation}
\begin{observation}
	\label{obs:edgewants}
	$\prob[Z_e] = \prob[\cap_{v \in e} X_v^c] = \min_{v \in e} x_v^c = z_e$.
\end{observation}

For an edge $e$ and color $i \neq \ell(e)$, we frequently wish to quantify the possibility that some node in $e$ wants color $i$, as this opens up the possibility that $e$ will be unsatisfied because some node $v \in e$ is given color $i$. Towards this goal, for each $e \in E$ and color $i \in [k]$ we identify one node in $e$ that has the highest likelihood of wanting $i$. Formally, we identify some \emph{representative} node $v \in e$ satisfying $x_v^i \geq x_u^i$ for every $u \in e$ (breaking ties arbitrarily if multiple nodes satisfy this), and we define $\sigma_e(i) = v$.
The definition of $\sigma_e(i)$ implies the following useful observation:
\begin{observation}
	\label{obs:cv}
	% One or more nodes in $e$ wants color $i$ if and only if $v = \sigma(i)$ wants color $i$. Equivalently, 
	Color $i$ does not want \textbf{any} nodes in $e$ $\iff$ color $i$ does not want $v = \sigma_e(i)$.
\end{observation}
Variations of Observations~\ref{obs:prob}-\ref{obs:edgewants} have often been used to prove guarantees for previous randomized ECC algorithms. However, Observation~\ref{obs:cv} is new and is key to our analysis.
% and is a key factor that allows us to design an algorithm that applies to hypergraph \maxecc{} and provides a much simpler way to get an approximation for graph \maxecc{}.
%Prior algorithms for graph \maxecc{} typically bound the probability of satisfying an edge $e$ by considering 
%as it will allow us to bound the probability of satisfying an edge $e$ in terms of certain independent events (e.g., the event that a certain color $i$ wants its representative node $v = \sigma_e(i)$) rather than reasoning about. 
% Using $\sigma_e$ to partition colors (based on which node $v$ is the representative for each color) will allow us to 


\subsection{Hypergraph MaxECC Algorithm}
%Our hypergraph \maxecc{} algorithm (Algorithm~\ref{alg:hyper_maxecc}) generates a uniform random priority for colors $\pi$, and independent random color thresholds $\{\alpha_c\}$ to determine which nodes and colors ``want'' each other. A node is then assigned to the highest priority color it wants. 
%After generating a uniform random permutation $\pi$ to encode a priority for and color thresholds $\{\alpha_c\}$, 
In addition to color thresholds $\{\alpha_c\}$, our hypergraph \maxecc{} algorithm (Algorithm~\ref{alg:hyper_maxecc}) generates a uniform random permutation  $\pi$ to define priorities for colors. A node is then assigned to the highest priority color it wants.
\begin{algorithm}[t]
	\caption{Approximation alg. for hypergraph \maxecc{}}
	\label{alg:hyper_maxecc}
	\begin{algorithmic}
		\STATE Obtain optimal variables $\{z_e; x_v^c\}$ for the LP relaxation
		\STATE $\pi \leftarrow \text{uniform random ordering of colors } [k]$
		\STATE For $c \in [k]$, $\alpha_c \leftarrow $ uniform random threshold in $[0,1]$
		\FOR{$v \in V$}
		\STATE $\mathcal{W} = \{c \in [k] \colon \alpha_c < x_v^c\}$
		\IF{$|\mathcal{W}| > 0$}
		\STATE $\lambda(v) \leftarrow \argmax_{c \in \mathcal{W}} \pi(c)$
		\ELSE
		\STATE $\lambda(v) \leftarrow $ arbitrary color
		\ENDIF
		\ENDFOR
	\end{algorithmic}
\end{algorithm}

\begin{theorem}
	\label{thm:hypermaxecc}
	Algorithm \ref{alg:hyper_maxecc} is a $\left(\frac{1}{r+1} \left(\frac{2}{e}\right)^r\right)$-approximation algorithm for \maxecc{} in hypergraphs with rank $r$.
\end{theorem}
\begin{proof}
	Fix an arbitrary edge $e \in E$ and let $c = \ell(e)$. Let $T_e$ denote the event that $e$ is satisfied.
	By Observation~\ref{obs:prob}, it suffices to show $\prob[T_e] \geq \frac{z_e}{r+1} \left(\frac2e\right)^r$.

	Let $C = [k]\setminus \{c\}$.
	To partition the color set $C$, for each $v \in e$ we define
	%	\begin{equation*}
	$C_v = \{i \in C \colon \sigma_e(i) = v\}$.
	%	\end{equation*}
	Recall that $\sigma_e(i)$ identifies a node $v \in e$ satisfying $x_v^i = \max_{u \in e} x_u^i$.
	Let $A_v$ denote the event that at most one color in $C_v$ wants one or more nodes in $e$. From Observation~\ref{obs:cv}, $A_v$ is equivalent to the event that  $v$ wants at most 1 color in $C_v$. Thus, the probability of $A_v$ is the probability that at most one of the events $\{X_v^i \colon i \in C_v \}$ happens. Observation~\ref{obs:xuc} gives $\sum_{i = 1}^k \prob[X_v^i] = \sum_{i = 1}^k x_v^i = 1$, allowing us to repurpose a supporting lemma of~\citet{angel2016clustering} on graph \maxecc{} to see that $\prob[A_v] \geq 2/e$ (Lemma~\ref{lem:angel} in Appendix~\ref{app:maxecc}).

	%	Observe now that the events $\{A_v \colon v \in e \}$ are mutually independent because the sets $\{C_v \colon v \in e\}$ are disjoint sets of colors, and the color thresholds $\{\alpha_i \colon i \in C\}$ are all drawn independently from each other. Furthermore, each $A_v$ is independent from $X_v^c$, since $c \notin C_v$. 
	%	
	Because color thresholds $\{\alpha_i\}$ are drawn independently for each color, and because color sets $\{C_v \colon v \in e\}$ are disjoint from each other and from $c$, the events $\{A_v, X_v^c \colon v \in e\}$ are mutually independent.
	Thus, using Observation~\ref{obs:edgewants} gives
	\begin{align*}
		\prob\left[ \left(\bigcap_{v \in e} A_v \right) \cap Z_e\right ] & = \prob[Z_e]\cdot \prod_{v \in e} \prob[A_v] \geq z_e \cdot \left(\frac2e \right)^r.
	\end{align*}
	If the joint event $J = (\cap_{v \in e} A_v) \cap Z_e$ holds, this means every node in $e$ wants color $c$, and at most $r$ distinct other colors (one for each node in $v \in e$ since there is one set $C_v$ for each $v \in e$) want one or more nodes in $e$. Conditioned on $J$, $e$ is satisfied if the color $c$ has a higher priority (determined by $\pi$) than the other $r$ colors, which happens with probability $1/(r+1)$. Thus,
	\begin{align*}
		{	\prob[T_e] \geq \prob\left[T_e \mid J\right] \prob\left[ J \right] \geq \frac{z_e}{r+1}\left(\frac2e\right)^r.}
	\end{align*}
\end{proof}

\textbf{Comparison with prior graph \maxecc{} algorithms.}
Algorithm~\ref{alg:hyper_maxecc} achieves a ${4}/({3e^2}) \approx 0.18$ approximation factor for graph \maxecc{} ($r = 2$). This improves upon the first ever approximation factor of $1/e^2 \approx 0.135$ for graph \maxecc{}~\cite{angel2016clustering}, and comes with a significantly simplified proof. There are two interrelated factors driving this simplified and improved guarantee.
The first is our use of Observation~\ref{obs:cv}.
\citet{angel2016clustering} bound the probability of satisfying an edge $(u,v) \in E_c$ using a delicate argument about certain dependent events we can denote by $B_v$ (the event that $v$ wants at most 1 color from $C = [k]\setminus \{c\}$) and $B_u$ (defined analogously for $u$).
%Let $\hat{A}_{u \land v}$ be the event that $u$ and $v$ each simulteneously want at most one color in $C$. 
The algorithm of \citet{angel2016clustering} requires proving that $\prob[B_{u} \cap B_v] \geq \prob[B_{u}] \prob[B_v]$, even though $B_u$ and $B_v$ are dependent. The proof of this result (Proposition 1 in~\citet{angel2016clustering}) is interesting but also lengthy, and does not work for $r > 2$. Subsequent approximation algorithms for graph \maxecc{} in turn rely on complicated generalizations of this proposition~\cite{ageev2015improved,ageev20200,alhamdan2019approximability}. In contrast, Theorem~\ref{thm:hypermaxecc} deals with mutually independent events $\{A_v \colon v \in e\}$. Observation~\ref{obs:cv} provides the key insight as to why it suffices to consider these events when bounding probabilities, leading to a far simpler analysis that extends easily to hypergraphs.

%Observation~\ref{obs:cv} works in tandem with our 
Our second key factor is the use of a global color ordering $\pi$. \citet{angel2016clustering} and other results for graph \maxecc{} apply a two-stage approach where Stage 1 identifies which nodes ``want'' which colors, and Stage 2 assigns nodes to colors \emph{independently} for each node. To illustrate the difference, consider the probability of satisfying an edge $(u,v) \in E_c$ if we condition on $u$ and $v$ both wanting $c$ and each wanting at most one other color. The algorithm of~\citet{angel2016clustering} has a $1/4$ chance of satisfying the edge (each node gets color $c$ with probability $1/2$), whereas Algorithm~\ref{alg:hyper_maxecc} has a $1/3$ chance (the probability that $c$ is given higher global priority than the other two colors). This is precisely why Algorithm~\ref{alg:hyper_maxecc}'s approximation guarantee is a factor $4/3$ larger than the guarantee of~\citet{angel2016clustering}.


\begin{algorithm}[t]
	\caption{$0.38$-approximation alg. for graph \maxecc{}}
	\label{alg:graph_max_ecc}
	\begin{algorithmic}
		\STATE Obtain optimal variables $\{z_e; x_v^c\}$ for the LP relaxation
		\STATE $\pi \leftarrow \text{uniform random ordering of colors } [k]$
		% \FOR{$c \in [k]$}
		%     \STATE $\alpha_c \leftarrow \text{ a uniform random value in } [0, 1]$
		% \ENDFOR
		\STATE For $c \in [k]$, $\alpha_c \leftarrow $ uniform random threshold in $[0,1]$
		\FOR{$v \in V$}
		\STATE $S_v \leftarrow \left\{c \in [k] \mid x_v^c \geq 2/3 \right\}$; $W_v \leftarrow [k] \setminus S_v$
		% \STATE
		\STATE $W'_v = \left\{ c \in W_v \mid \alpha_c < x_v^c \right\}$
		% \STATE
		\IF{$|W'_v| > 0$}
		\STATE $\lambda(v) \leftarrow \argmax_{i \in W'_v} \pi(i)$
		\ELSIF{$\exists c \text{ s.t. } S_v = \left\{ c \right\}$}
		\STATE $\lambda(v) \leftarrow c$
		\ELSE
		\STATE $\lambda(v) \leftarrow \text{ arbitrary color}$
		\ENDIF
		\ENDFOR
	\end{algorithmic}
\end{algorithm}


\subsection{Graph MaxECC Algorithm}
Although Algorithm~\ref{alg:hyper_maxecc} does not improve on the $0.3622$-approximation of~\citet{ageev20200} for graph \maxecc{}, we can incorporate its distinguishing features (the color ordering $\pi$ and Observation~\ref{obs:cv}) into a refined algorithm with a $154/405 \approx 0.38$ approximation factor.

Our refined algorithm (Algorithm \ref{alg:graph_max_ecc}) for graphs solves the LP relaxation, generates color thresholds $\{\alpha_i \colon i \in [k]\}$, and generates a color ordering $\pi$ in the same way as Algorithm~\ref{alg:hyper_maxecc}. It differs in that it partitions colors for each $v$ into colors that are \emph{strong} or \emph{weak} for $v$, given respectively by the sets
\begin{align*}
	S_v = \{ i \in [k] \colon x_v^i \geq 2/3\} \text{ and } W_v & = [k] - S_v.
\end{align*}
Since $\sum_{i = 1}^k x_v^i = 1$, we know $|S_v| \in \{0,1\}$. We say color $i$ is \textit{strong for $v$} if $S_v = \{i\}$, otherwise $i$ is \textit{weak for $v$}. Note that a color being \emph{strong} or \emph{weak} for $v$ is based on a fixed and non-random LP variable. This is separate from the notion of a color \emph{wanting} $v$, which is a random event.

Algorithm~\ref{alg:graph_max_ecc} first checks if $v$ wants any {weak} colors. If so, it assigns $v$ the weak color of highest priority (using $\pi$) that $v$ wants. If $v$ wants no weak colors but has a strong color, then $v$ is assigned the strong color. Prioritizing weak colors in this way appears counterintuitive, since if $v$ has a strong color $c$ it suggests that $v$ should get color $c$. However, note that $x_v^c \geq 2/3$ still implies $v$ will get color $c$ with high probability, since a large value for $x_v^c$ makes it less likely $v$ will want any weak colors. Meanwhile, prioritizing weak colors enables us to lower bound the probability that an edge $e = (u,v)$ is satisfied even if $\ell(e)$ is weak for $u$ or $v$.

In order to improve on the extensively studied problem of \maxecc{} in graphs, our analysis is much more involved than the proof of Theorem~\ref{thm:hypermaxecc} and requires proving several detailed technical lemmas that may be of interest in their own right. In order to present the lemmas used, we use the following definition.

\begin{definition}
	\label{def:prob}
	Let $m$ be a nonnegative integer and $t$ be an integer. Given a vector $\vx \in \mathbb{R}^m$, if $m \geq t > 0$, we define the function $P(\vx, t)$ as follows:
	\[
		P(\vx, t) = \sum_{I \in {[m] \choose t}} \left( \prod_{i \in I} x_i \prod_{j\in [m] \setminus I} (1 - x_j) \right),
	\]
	and for other choices of $t$ and $m$ we define
	\begin{equation*}
		P(\vx, t) = \begin{cases}
			0                         & \text{ if $m < t$ or $t < 0$} \\
			%			0 & \text{ if $t < 0$} \\
			1                         & \text{ if $0 = m = t$}        \\
			\prod_{i = 1}^m (1 - x_i) & \text{ if $0 = t < m$.}
		\end{cases}
	\end{equation*}
\end{definition}
To provide intuition, $P(\vx,t)$ is defined to encode the probability that $t$ events from a set of $m$ independent events happen. In more detail, consider a set of $m$ mutually independent events $\mathcal{X} = \{X_1, X_2, \cdots, X_m\}$, where the probability that the $i$th event happens is $x_i = \prob[X_i]$, the $i$th entry of $\vx$. The function $P$ in Definition~\ref{def:prob} is the probability that exactly $t$ of the events in $\mathcal{X}$ happen. In our proofs for \maxecc{}, we will apply this with $\mathcal{X}$ representing a subset of colors in the ECC instance, while the vector $\vx$ encodes LP variables $\{x_u^i\}$ for some node $u$ and that set of colors (which by Observation~\ref{obs:xuc} are probabilities for wanting those colors). Lemmas~\ref{lem:bounding} and \ref{lem:bounding_constraints} will aid in bounding the probability that a node is assigned a certain color, conditioned on how many other colors it wants.


The technical lemmas used in the proof of Theorem~\ref{thm:graphecc} are given below. The proofs for these theorems can be found in Appendix~\ref{app:maxecc}.
\begin{restatable}{lemma}{lemdependentynx}
	\label{lem:dependent_y_n_x}
	Consider an edge $e = (u,v)$ of color $c$ and let $c \in W_u$. When running Algorithm~\ref{alg:graph_max_ecc}, let $X_u^c$ be the event that $u$ wants $c$, $Y_u^c$ be the event that $u$ is assigned $c$ by the algorithm, and let $N_v$ be the event that $v$ wants no colors in $[k]\setminus\{c\}$.
	Then the following inequality holds:
	\[
		\prob\left[Y_u^c \mid N_v \cap X_u^c\right] \geq \prob\left[Y_u^c \mid X_u^c\right].
	\]
\end{restatable}

\begin{restatable}{lemma}{lemsumtoprod}
	\label{lem:sum_to_prod}
	Let $\beta \in [0,1]$ be a constant and $\mathbf{x} \in \mathbb{R}^m_{\geq 0}$ be a length $m$ nonnegative vector satisfying $\sum_{t=1}^m x_t \leq \beta$, then we have $\prod_{t=1}^m (1 - x_t) \geq 1 - \beta$.
\end{restatable}

\begin{restatable}{lemma}{lembounding}
	\label{lem:bounding}
	Let $m \geq 2$ be an integer and
	$a_0, a_1, \hdots, a_m$ be values such that for every $t \in [0, m-2]$
	\begin{align*}
		a_{t+1}   & \leq a_t            \\
		2 a_{t+1} & \leq a_t + a_{t+2}.
	\end{align*}
	Let $\mathcal{D} = \{ \vx \in [0,2/3]^m \colon \sum_{i = 1}^m x_i \leq 1 \}$ be the domain of a function
	$f : \mathcal{D} \rightarrow \mathbb{R}$ defined as
	\[
		f(\vx) = \sum_{t=0}^m a_t P(\vx, t).
	\]
	Then $f$ is minimized (over domain $\mathcal{D}$) by any vector $\vx^*$ with one entry set to $2/3$, one entry set to $1/3$, and every other entry set to $0$. Furthermore, we have
	\begin{equation}
		\label{eq:minval}
		f(\vx^*) = \frac{2}{9}(a_0 + a_2) + \frac{5}{9}a_1.
	\end{equation}
\end{restatable}

\begin{restatable}{lemma}{lemboundingconstraints}
	\label{lem:bounding_constraints}
	The inequalities $a_t \geq a_{t+1}$ and $a_t + a_{t+2} \geq 2a_{t+1}$ hold for sequence $a_t = \frac{1}{1+g+t}$ where $g \geq 0$ is an arbitrary fixed integer. These inequalities also hold for the sequence
	\begin{align}
		\label{eq:complicatedat}
		a_t & = \frac{2}{9}\left( \frac{1}{t+1} + \frac{1}{t+3} \right) + \frac{5}{9} \left( \frac{1}{t+2} \right).
	\end{align}
\end{restatable}

Using the above technical lemmas, we prove the following result for an arbitrary edge $e$, which by Observation~\ref{obs:prob} proves our approximation guarantee.
\begin{theorem}
	\label{thm:graphecc}
	For every $e \in E$, $\prob[\text{$e$ is satisfied}] \geq p z_e$ where $p = {154}/{405} > 0.3802$ when running Algorithm~\ref{alg:graph_max_ecc}.
\end{theorem}
\vspace{-10pt}
\begin{proof}

	Fix $e = (u,v)$, let $c = \ell(e)$, and set $C = [k] \backslash \{c\}$.
	Define $C_v = \{i \in C \colon x_v^c \geq x_u^c\}$ and $C_u = C \setminus C_v$. As before, $T_e$ is the event that $e$ is satisfied and $X_v^i$ is the event that $v$ wants color $i$. Let $Y_v^i$ be the event that $v$ is \emph{assigned} color $i$ by Algorithm~\ref{alg:graph_max_ecc}, and $N_v$ be the event that $v$ wants \emph{no} colors in $C$. If $c$ is strong for $v$, event $N_v$ is equivalent to event $Y_v^c$. Let $W_v'$ denote colors in $W_v$ that {want} $v$. Define $X_u^i$, $Y_u^i$, $W_u'$, and $N_u$ analogously for $u$. The proof is separated into (increasingly difficult) cases, based on how many of $\{u,v\}$ have $c$ as a strong color.

	\bigbreak

	\noindent
	\textbf{Case 1:} $c$ is strong for both $u$ and $v$, i.e., $S_u = S_v = \{c\}$.\\
	Let $W_v'$ denote colors in $W_v$ that want $v$, and define $W_u'$ analogously for $u$.
	Since $c$ is the strong color for both $u$ and $v$, $e$ is satisfied if and only if $W'_u = W'_v = \emptyset$. Additionally, because nodes can only have a single strong color and we know $c$ is strong for both $u$ and $v$ we know that all other colors must be weak for both $u$ and $v$. Using the property that probabilities regarding separate colors are independent we have
	\begin{align*}
		\prob\left[e \text{ is satisfied}\right] & = \prob\left[W'_u = W'_v = \emptyset\right]  = \prod_{i \in C}\prob\left[ i \notin W'_u \cup W'_v \right].
	\end{align*}
	Consider a color $i \in C_v$, which by definition means $x_v^i \geq x_u^i$. There are three options for the random threshold $\alpha_i$:
	\begin{itemize}
		\item $\alpha_i < x_u^i \leq x_v^i$, which happens with probability $x_u^i$ and implies that $i \in W'_u \cap W'_v$,
		\item $x_u^i < \alpha_i \leq x_v^i$, which happens with probability $x_v^i - x_u^i$ and implies $i \in W'_v$, $i \notin W'_u$, and
		\item $x_u^i \leq x_v^i < \alpha_i$, which happens with probability $1 - x_v^i$ and implies $i \notin W_u' \cup W_v'$.
	\end{itemize}
	Thus, for $i \in C_v$ we have $\prob\left[i \notin W'_u \cup W'_v\right] = (1 - x_v^i)$ and similarly for $i \in C_u$ we have $\prob\left[i \notin W'_u \cup W'_v\right] = (1 - x_u^i)$. This gives
	\begin{align*}
		\prod_{i \in C}\prob\left[ i \notin W'_u \cup W'_v \right] & = \prod_{i \in C_v}(1 - x_v^i)\prod_{i \in C_u}(1 - x_u^i).
	\end{align*}
	The fact that $c$ is strong for both $u$ and $v$ means $x_v^i \geq 2/3$ and $x_u^i \geq 2/3$. From the equality constraint in the LP we see that for $w\in\{u,v\}$ we have $\sum_{i \in C}x_w^i \leq 1 - 2/3 = 1/3$. Applying Lemma \ref{lem:sum_to_prod} gives
	\begin{align*}
		\prod_{i \in C_v}(1 - x_v^i)\prod_{i \in C_u}(1 - x_u^i) & \geq \frac{2}{3}\frac{2}{3} = \frac{4}{9} \geq \frac{4}{9}z_e.
	\end{align*}

	%%%%%%%%%%%%%%%%%%%%%%%%%%%%%%%%%%%%%%%%%%%%%%%%%%%%%%%%%%%%%%%%%%%%%%%%%%%%%%%%%%%%%%%%%%%%%%%%%%%%%%%%%%%%%%%%%%%%%%%%%%%%%%%%%%%%%%%%

	\bigbreak

	\noindent
	\textbf{Case 2:} $c$ is strong for one of $u$ or $v$. \\
	Without loss of generality we say $S_v = \{c\}$ and $c \in W_u$.
	Edge $e$ is satisfied if and only if $Y_u^c \cap Y_v^c$ holds.
	Because $c$ is strong for $v$, event $Y_v^c$ holds if and only if $N_v$ holds. Using the fact that $Y_u^c = Y_u^c \cap X_u^c$, we can write
	\[
		\prob\left[e \text{ is satisfied}\right] = \prob\left[Y_u^c \cap X_u^c \cap N_v\right].
	\]
	Using Bayes' Theorem, the fact that $X_u^c$ and $N_v$ are independent\footnote{Independence follows from the fact that $X_u^c$ is concerned with color $c$, while $N_v$ is concerned with a disjoint color set $C = [k]\backslash \{c\}$.}, Observation~\ref{obs:xuc} ($\prob[X_u^c] = x_u^c$), and Lemma~\ref{lem:dependent_y_n_x}, we see that
	\begin{align*}
		\prob\left[Y_u^c \cap X_u^c \cap N_v\right] & = \prob\left[ Y_u^c \mid N_v \cap X_u^c \right] \prob\left[ N_v \cap X_u^c \right]                & \text{(Bayes' Theorem)}                     \\
		                                            & = \prob\left[ Y_u^c \mid N_v \cap X_u^c \right] \prob\left[ N_v \right] \prob\left[ X_u^c \right] & \text{($X_u^c$ and $N_v$ are indepdendent)} \\
		                                            & = \prob\left[ Y_u^c \mid N_v \cap X_u^c \right] \prob\left[ N_v \right] x_u^c                     & \text{(Observation~\ref{obs:prob})}         \\
		                                            & \geq \prob\left[ Y_u^c \mid N_v \cap X_u^c \right] \prob\left[ N_v \right] z_e                    & \text{(LP constraint $x_u^c \geq z_e$)}     \\
		                                            & \geq \prob\left[ Y_u^c \mid X_u^c \right] \prob\left[ N_v \right] z_e                             & \text{(Lemma~\ref{lem:dependent_y_n_x}).}
	\end{align*}
	Using $\overline{X}_v^i$ to indicate that $v$ \emph{does not} want $i$, we get
	\[
		\prob\left[ N_v \right] = \prob\left[ \bigcap_{i \in C} \overline{X}_v^i \right] = \prod_{i \in C}(1 - x_v^i).
	\]
	As in Case 1, because $c$ is strong for $v$, we know that $\sum_{i \in C} x_v^i \leq 1/3$ and applying Lemma \ref{lem:sum_to_prod} gives us that $\prob\left[ N_v \right] \geq 2/3$.


	Finally we must bound $\prob\left[ Y_u^c \mid X_u^c \right]$. Since
	$c$ is weak for $u$ (i.e., $c \in W_u$), it will be convenient to consider all weak colors for $u$ other than $c$, which we will denote by $\hat{W}_u = W_u \backslash \{c\}$. We will then use $\hat{W}_u' = \{ i \in \hat{W}_u \colon \alpha_i < x_u^i\}$ to denote the set of colors in $\hat{W}_u$ that $u$ wants. Note that $X_u^c$ is independent of the number of colors in $\hat{W}_u$ that $u$ wants.

	Because of the global ordering of colors, conditioned on $u$ wanting $c$ and wanting exactly $t$ colors in $\hat{W}_u$, there is a $1/(t+1)$ chance that $c$ is chosen first by the permutation $\pi$, and is therefore assigned to $u$. Formally:
	\begin{equation*}
		\prob\left[ Y_u^c \mid X_u^c  \cap (|\hat{W}'_u| = t)\right] = \frac{1}{t+1}.
	\end{equation*}
	Define $p_t = \prob[ |\hat{W}'_u| = t]$ to be the probability that exactly $t$ colors from $\hat{W}_u$ are wanted by $u$. Using the vector $\vx_u[\hat{W}_u]$ to encode LP variables $\{x_u^i \colon i \in \hat{W}_u\}$ for node $u$ and the colors in $\hat{W}_u$, we see that $p_t = P(\vx_v[\hat{W}_u], t)$. Therefore, the law of total probability (combined with the fact that $X_u^c$ is independent from $|\hat{W}_u'|$) gives
	\begin{equation}
		\label{eq:mainprob}
		\prob\left[ Y_u^c \mid X_u^c \right]
		%		= p_0 + \frac{1}{2}p_1 + \cdots + \frac{1}{|\hat{W}_u| + 1}p_{|\hat{W}_u|}.
		= \sum_{t = 0}^{|\hat{W}_u|} \prob\left[ Y_u^c \mid X_u^c  \cap (|\hat{W}'_u| = t)\right] \prob[|\hat{W}'_u| = t] = \sum_{t = 0}^{|\hat{W}_u|} \frac{1}{1+t} p_t.
	\end{equation}
	Eq.~\eqref{eq:mainprob} is exactly of the form in Lemma \ref{lem:bounding} with $a_t = 1/(t+1)$. From Lemma \ref{lem:bounding_constraints}, we know this sequence $\{a_t\}$ satisfies the constraints needed by Lemma \ref{lem:bounding}, and applying Lemma~\ref{lem:bounding} shows $\prob\left[ Y_u^c \mid X_u^c \right] \geq 31/54$. Combining this with the previously established bound $\prob\left[ N_v \right] \geq 2/3$ gives
	\begin{align*}
		\prob\left[e \text{ is satisfied}\right] & \geq \prob\left[ Y_u^c \mid X_u^c \right] \prob\left[ N_v \right] z_e  \geq \frac{31}{54} \cdot \frac{2}{3} z_e = \frac{31}{81}z_e > \frac{154}{405} z_e.
	\end{align*}

	%%%%%%%%%%%%%%%%%%%%%%%%%%%%%%%%%%%%%%%%%%%%%%%%%%%%%%%%%%%%%%%%%%%%%%%%%%%%%%%%%%%%%%%%%%%%%%%%%%%%%%%%%%%%%%%%%%%%%%%%%%%%%%%%%%%%%%%%

	\bigbreak

	\noindent
	\textbf{Case 3:} $c$ is weak for both $u$ and $v$, i.e., $c \in W_u \cap W_v$.\\
	First we condition the probability of satisfying $e$ on the event $X_u^c \cap X_v^c$. If one of $X_u^c$ or $X_v^c$ does not happen, then $e$ cannot be satisfied, so we have
	\begin{align*}
		\prob\left[ \text{$e$ is satisfied} \right] & = \prob\left[ \text{$e$ is satisfied} \mid X_u^c \cap X_v^c \right] \prob\left[ X_v^c \cap X_u^c \right] = \prob\left[ \text{$e$ is satisfied} \mid X_u^c \cap X_v^c \right] z_e.
	\end{align*}
	Define $D = (W_u \cup W_v) \backslash \{c\}$ to be the set of weak colors (excluding $c$) that want at least one of $\{u,v\}$. If we condition on $u$ and $v$ both wanting $c$, it is still possible that $e$ will not be satisfied. This will happen if a weak color $i \in D$ wants either $u$ or $v$, and the permutation $\pi$ prioritizes color $i$ over $c$. We bound the probability of satisfying $e$ by conditioning on the number of colors in $D$ that want one or both of $\{u,v\}$.

	Using the same logic as in Observation~\ref{obs:cv}, we can present a simpler way to characterize whether a color $i \in D$ wants at least one of $\{u,v\}$. Formally, we partition $D$ into the sets:
	\begin{align*}
		D_u & = \{i \in D \colon x_u^i \geq x_v^i \} \\
		D_v & = D - D_v.
	\end{align*}
	Note that a color $i \in D_u$ wants one or both nodes in $e$ if and only if $i$ wants $u$, and color $i \in D_v$ wants one or both nodes in $e$ if and only if $i$ wants $v$.
	Mirroring our previous notation, let $D_u'$ denote the set of colors in $D_u$ that want node $u$ (based on the random color thresholds) and define $D_v'$ analogously. The set $D_u' \cup D_v'$ is then the set of weak colors that want one or more node in $e$. Thus, conditioned on $|D_u' \cup D_v'| = t$ and conditioned on $u$ and $v$ wanting $c$, in order for $e$ to be satisfied $c$ must come before all $t$ colors in $|D_u' \cup D_v'|$ in the random permutation $\pi$. Formally, this means
	\[
		\prob\left[ \text{$e$ is satisfied} \mid X_u^c \cap X_v^c \cap (|D'_u \cup D'_v| = t) \right] = \frac{1}{1 + t}.
	\]

	Again using the law of total probability and the fact that $X_u^c \cap X_v^c$ is independent from $|D_u' \cup D_v'|$, we see
	\begin{align}
		\label{eq:dupdvp}
		\prob[ e & \text{ is satisfied} \mid X_u^c \cap X_v^c ]
		= \sum_{t = 0}^{|D|} \frac{1}{1+t} \prob\left[ |D'_u \cup D'_v| = t \right].
	\end{align}
	Observe now that $D_u$ and $D_v$ are disjoint color sets, which implies that $|D_u'|$ and $|D_v'|$ are independent random variables. This allows us to decouple $D_u'$ and $D_v'$ in the above expression since $\prob\left[ |D'_u \cup D'_v| = t \right] = \prob\left[ |D'_u| + |D'_v| = t \right]$. We define
	\[
		p_i = \prob\left[ |D'_u| = i \right] \text{ and } q_i = \prob\left[ |D'_v| = i \right],
	\]
	so that we can write
	\[
		\prob\left[ |D'_u| + |D'_v| = t \right] = \sum_{\substack{i, j \in [0, t] \\ i + j = t}} p_i q_j = \sum_{i = 0}^{|D_u|} p_i q_{t-i} .
	\]
	This allows us to re-write Eq.~\eqref{eq:dupdvp} as
	\begin{align}
		\label{eq:decoupled}
		\prob[ e \text{ is satisfied} \mid X_u^c \cap X_v^c ] & = \sum_{i = 0}^{|D_u|} \sum_{j = 0}^{|D_v|} \frac{1}{1+i+j} p_i q_j.
	\end{align}

	Without loss of generality we now assume $|D_u| \leq |D_v|$, and proceed case by case depending on the sizes of $D_u$ and $D_v$.
	%	As a shorthand we use $\rho = \prob[ e \text{ is satisfied} \mid X_u^c \cap X_v^c ]$. 
	Our goal is to show that for all cases $\prob[ e \text{ is satisfied} \mid X_u^c \cap X_v^c ] \geq 154/405$.
	In the cases below, we use the vector $\vx_u[D_u]$ to encode LP variables for node $u$ and colors in $D_u$ so that $p_i = P(\vx_u[D_u], i)$, and likewise $q_i = P(\vx_v[D_v], i)$, to match with notation in Definition~\ref{def:prob} and Lemma~\ref{lem:bounding}.

	\bigbreak

	\noindent
	\textbf{Case 3a.} $|D_u| = |D_v| = 0$. \\
	No weak colors other than $c$ want $u$ or $v$, so $\prob[ e \text{ is satisfied} \mid X_u^c \cap X_v^c ] = 1$.

	\bigbreak

	\noindent
	\textbf{Case 3b.} $|D_u| = 0$ and $|D_v| = 1$. \\
	Letting $a$ be the single color in $D_v$,
	\begin{align*}
		\prob[ e \text{ is satisfied} \mid X_u^c \cap X_v^c ] & = p_0 q_0 + \frac{1}{2} p_0 q_1 = q_0 + \frac{1}{2} q_1 = (1 - x_v^a) + \frac{1}{2} x_v^a.
	\end{align*}
	Because $a$ is a weak color for $v$, we know $x_v^a \in [0, \frac{2}{3}]$, and the minimum value the probability can obtain is $\frac{2}{3} > \frac{154}{405}$.

	\bigbreak

	\noindent
	\textbf{Case 3c.} $|D_u| = |D_v| = 1$. \\
	Letting $a$ be the color in $D_v$ and $b$ be the color in $D_u$,
	\begin{align*}
		\prob[ e \text{ is satisfied} \mid X_u^c \cap X_v^c ] & = p_0 q_0 + \frac{1}{2}(p_0 q_1 + p_1 q_0) + \frac{1}{3} p_1 q_1 \\ &= (1 - x_u^b)(1 - x_v^a) + \frac{1}{2}((1 - x_u^b)x_v^a + x_u^b(1 - x_v^a)) + \frac{1}{3} x_u^b x_v^a.
	\end{align*}
	Because $a$ is weak for $v$ and $b$ is weak for $u$, we know $x_v^a, x_u^b \in [0, \frac{2}{3}]$. This gives a minimum value for the probability of $\frac{13}{27} > \frac{154}{405}$.

	\bigbreak

	\noindent
	\textbf{Case 3d.} $|D_u| = 0$ and $|D_v| > 1$. \\
	In this case $p_0 = 1$, and Eq.~\eqref{eq:decoupled} simplifies to
	\begin{equation*}
		\prob[ e \text{ is satisfied} \mid X_u^c \cap X_v^c ] = \sum_{j = 0}^{|D_v|} \frac{1}{j+1} q_j \geq \frac{31}{54},
	\end{equation*}
	where we have applied Lemma \ref{lem:bounding} with $a_j = \frac{1}{j+1}$.

	\bigbreak

	\noindent
	\textbf{Case 3e.} $|D_u| = 1$ and $|D_v| > 1$. \\
	Eq.~\eqref{eq:decoupled} simplifies to
	\begin{align*}
		\prob[ e \text{ is satisfied} \mid X_u^c \cap X_v^c ] & = p_0 \sum_{j = 0}^{|D_v|} \frac{q_j}{j+1}  + p_1 \sum_{j = 0}^{|D_v|} \frac{q_j}{j+2} \geq p_0 \frac{31}{54} + p_1 \frac{19}{54},
	\end{align*}
	where we applied Lemma \ref{lem:bounding} twice, once with $a_j = \frac{1}{j+1}$ and once with $a_j = \frac{1}{j+2}$. The right hand side of the above bound has a minium value of $\frac{23}{54}$.

	\bigbreak

	\noindent
	\textbf{Case 3f.} $|D_u| > 1$ and $|D_v| > 1$.
	From Eq.~\eqref{eq:decoupled} we know
	\begin{align*}
		\prob[ e \text{ is satisfied} \mid X_u^c \cap X_v^c ] & = \sum_{i = 0}^{|D_u|}  p_i \left(\sum_{j = 0}^{|D_v|} \frac{1}{1+i+j}  q_j \right).
	\end{align*}
	We then apply Lemma~\ref{lem:bounding} $|D_u|+1$ times, once for each choice of $i \in \{0,1, \hdots, |D_u|\}$. The $i$th time we apply it, we use the sequence $a_j = \frac{1}{1+i+j}$ to get
	\begin{align}
		\label{eq:lemdvtimes}
		\sum_{j = 0}^{|D_v|} \frac{1}{1+i+j}  q_j \geq \frac{2}{9}\left( \frac{1}{i+1} + \frac{1}{i+3} \right) + \frac{5}{9} \left( \frac{1}{i+2} \right).
	\end{align}
	The right hand of~\eqref{eq:lemdvtimes} defines a new sequence
	\begin{equation}
		\label{eq:uglyai}
		a_i = \left( \frac{1}{i+1} + \frac{1}{i+3} \right) + \frac{5}{9} \left( \frac{1}{i+2} \right).
	\end{equation}
	We know from Lemma~\ref{lem:bounding_constraints} that this sequence $\{a_i\}$ in Eq.~\eqref{eq:uglyai} satisfies the conditions from Lemma~\ref{lem:bounding}, so applying Lemma~\ref{lem:bounding} one more time and combining all steps gives
	\begin{align*}
		\prob[ e \text{ is satisfied} \mid X_u^c \cap X_v^c ] & = \sum_{i = 0}^{|D_u|}  p_i \left(\sum_{j = 0}^{|D_v|} \frac{1}{1+i+j}  q_j \right) \\
		&\geq \sum_{i = 0}^{|D_u|}  p_i \left( \frac{2}{9}\left( \frac{1}{i+1} + \frac{1}{i+3} \right) + \frac{5}{9} \left( \frac{1}{i+2} \right) \right) \geq \frac{154}{405}.
	\end{align*}
	Therefore, we have shown in all possible subcases of Case 3 that $\prob\left[ \text{$e$ is satisfied} \mid X_u^c \cap X_v^c \right] \geq \frac{154}{405}$.

	Cases 1,2, and 3 all together show that for every $e \in E$ we have
	\[
		\prob\left[ \text{$e$ is satisfied} \right] \geq \frac{154}{405} z_e.
	\]
\end{proof}

\section{Alternative ECC Objectives}
We now turn our attention to more general~\ECC{} objectives, as well as several which incorporate notions of balance or fairness with respect to edge colors.
\section{Omitted proofs from~\texorpdfstring{\Cref{sec:generalized-mean}}{}}
\label{app:gm}

\pmeanapproximation*
\begin{proof}

    We begin with the case where $p \geq 1$.
    It is well known that the $\ell_p$-norm is convex, and we observe that all of the constraints in the
    in Program~(\ref{eq:p-mean-ecc}) are linear. Thus, we may in polynomial time obtain optimal fractional variables $\{d_v^c\}, \{\gamma_e\}, \{m_c\}$
    for the relaxation of Program~(\ref{eq:p-mean-ecc}).

    We observe that for each vertex $v$, there is at most one color $c$ with the property that $d_v^c < \frac{1}{2}$.
    Otherwise, there are two colors $c_1, c_2$ with $d_v^{c_1}, d_v^{c_2} < \frac{1}{2}$, but then
    \[
        \sum_{c=1}^k d_v^c < 2\cdot\frac{1}{2} + (k - 2) = k - 1,
    \]
    contradicting the first constraint of Program~(\ref{eq:p-mean-ecc}).

    We now propose a coloring $\lambda$ as follows. For each color $c$ and vertex $v$, we assign $\lambda(v) = c$ if $d_v^c < \frac{1}{2}$.
    By the preceding observation, this procedure assigns at most one color to each vertex.
    If for any $v$ we have that $d_v^c \geq \frac{1}{2}$ for all colors $c$, then we set $\lambda(v)$ arbitrarily.

    Now, for each hyperedge $e$ let $\hat{\gamma}_e$ be equal to $0$ if $\lambda(v) = \ell(e)$ for all $v \in e$, or $1$ otherwise.
    We claim that if $\hat{\gamma}_e = 1$, then $\gamma_e \geq \frac{1}{2}$.
    Otherwise, letting $c = \ell(e)$, the second constraint of Program~(\ref{eq:p-mean-ecc}) implies that $d_v^c < \frac{1}{2}$ for all $v \in e$.
    Then $\lambda(v) = c$ for all $v \in e$, implying that $\hat{\gamma}_e = 0$, a contradiction.
    We conclude that for every hyperedge $e$, $\hat{\gamma}_e \leq 2\gamma_e$.

    For each color $c$, we write $\hat{m}_c$ for the number of hyperedges of color $c$ which are unsatisfied by $\lambda$.
    Equivalently,
    \[
        \hat{m}_c = \sum_{e \in E_c} \hat{\gamma}_e \leq 2\sum_{e \in E_c} \gamma_e = 2m_c,
    \]
    where the last equality comes from the third constraint of Program~\ref{eq:p-mean-ecc}.

    It follows that for every $c$, $\hat{m}_c^p \leq 2^pm_c^p$, and thus
    \[
        \sum_{c=1}^k \hat{m}_c^p \leq 2^p \cdot \sum_{c=1}^k m_c^p,
    \]

    which in turn implies that

    \[
        \left(\sum_{c=1}^k \hat{m}_c^p\right)^{1/p} \leq 2 \cdot \left(\sum_{c=1}^k m_c^p\right)^{1/p},
    \]

    so $\lambda$ is $2$-approximate.

    We now consider the case where $0 < p < 1$.
    We begin by defining, for each $c \in [k]$, a non-negative, monotone, and submodular function $f_c\colon 2^E \rightarrow \mathbb{Z}$ given by $f_c(S) = |E_c \cap S|$.
    If $\lambda$ is a vertex coloring of $H$ and $S_\lambda \subseteq E$ is the set of edges unsatisfied by $\lambda$, then $f_c(S_\lambda)$ measures the number of edges of color $c$ unsatisfied by $\lambda$.
    Because the composition of a concave function with a submodular function results in a submodular function and $0 < p < 1$, $f_c^p$ is non-negative, monotone, and submodular.
    Then the function $f\colon 2^E \rightarrow \mathbb{R}$ given by the sum
    \[
        f(S) = \sum_{c=1}^{k} f_c^p(S)
    \]
    has these same properties.

    Now, let $\lambda^*$ be an optimal vertex coloring of $H$ and let $S_{\lambda^*}$ be the set of hyperedges unsatisfied by $\lambda^*$.
    Our goal will be to compute a coloring $\lambda$ with $f(S_\lambda) \leq 2f(S_{\lambda^*})$.
    This would complete the proof, since then we have
    \[
        (f(S_\lambda))^{1/p} \leq (2f(S_{\lambda^*}))^{1/p} = 2^{1/p}f^{1/p}(S_{\lambda^*}),
    \]
    and $f^{1/p}(S_{\lambda}), f^{1/p}(S_{\lambda^*})$ are precisely the objective values attained by $\lambda$ and $\lambda^*$, respectively, according to Program~(\ref{eq:p-mean-ecc}).

    We give a classic analysis based on the Lov{\'a}sz extension~\cite{lovasz1983submodular}.
    %     but we note that in practice more recent, faster methods are preferable.
    %    In particular, since each $f_c$ measures the cardinality of a set, very fast optimization techniques based on reductions to sparse graph cut problems are available~\cite{veldt2021approximate}.
    Some readers may find it helpful to observe that the following is conceptually equivalent to the textbook $2$-approximation for \textsc{Submodular Vertex Cover}\footnote{Given a graph $G$ and a submodular function $f$, find a vertex cover $C$ minimizing $f(C)$.}, where the graph in question is the \emph{conflict graph} $G$ of our edge-colored hypergraph $H$, as defined in~\Cref{sec:color-fair}.
    The connection between vertex colorings of edge-colored hypergraphs and vertex covers of their associated conflict graphs has been observed several times in the literature~\cite{angel2016clustering,cai2018alternating,kellerhals2023parameterized,veldt2023optimal}, and also appears elsewhere in the present work, e.g., in the proofs of~\Cref{thm:cfminecc-reduce-sparse-vc,thm-combinatorial-k-approx,thm:pc-multiple-classes-NP-hard}.

    Impose an arbitrary order on the edges of $E$ and for each vector $\boldsymbol{\gamma} \in [0, 1]^{|E|}$ map edges to components of $\boldsymbol{\gamma}$ according to this order.
    Henceforth, for a hyperedge $e$ we write $\gamma_e$ for the component of $\boldsymbol{\gamma}$ corresponding to $e$. Moreover, for a
    value $\rho$ selected uniformly at random from $[0, 1]$, we write $T_\rho(\boldsymbol{\gamma}) = \{e \colon \gamma_e \geq \rho\}$.
    Then the Lov{\'a}sz extension $\hat{f}\colon [0, 1]^{|E|} \rightarrow \mathbb{R}$ of $f$ is given by
    \[
        \hat{f}(\boldsymbol{\gamma}) = \mathbb{E}[f(T_\rho(\boldsymbol{\gamma}))].
    \]
    This $\hat{f}$ is convex~\cite{lovasz1983submodular}, so we may efficiently optimize the following program:
    %
    \begin{align}
        \label{eq:lovasz}
        \begin{aligned}
            \text{min} \quad  & \hat{f}(\gamma)                 \\
            \text{s.t.} \quad & \gamma_e + \gamma_f \geq 1\quad \\
                              & \gamma_e \geq 0                 \\
        \end{aligned}
        \begin{aligned}
             &                                                                                                         \\
             & \text{for all } e, f \in E \text{ such that } e \cap f \neq \emptyset \text{ and } \ell(e) \neq \ell(f) \\
             & \text{for all } e \in E.
        \end{aligned}
    \end{align}
    Consider the binary vector given by $\gamma_e = 1$ if $e \in S_{\lambda_*}$ or $0$ otherwise; note that this vector satisfies the constraints of Program~(\ref{eq:lovasz}). Hence, we can see that the minimum value for Program~(\ref{eq:lovasz}) provides a lower bound on $f(S_{\lambda_*})$.

    Let $\boldsymbol{\gamma}$ be a minimizer of Program~(\ref{eq:lovasz}), and let
    \[
        S = \left\{e \in E \colon \gamma_e \geq \frac{1}{2}\right\}.
    \]
    The first constraint of Program~\ref{eq:lovasz} ensures that $S$ contains at least one member of every pair of overlapping and distinctly colored hyperedges.
    Thus, it is trivial to compute a coloring $\lambda$ with $S_\lambda \subseteq S$.
    It follows from monotonicity that $f(S_\lambda) \leq f(S)$, and that for each $\rho \leq \frac{1}{2}$, $f(S) \leq f(T_\rho(\boldsymbol{\gamma}))$.
    Putting it all together, we have

    \[
        \frac{f(S_\lambda)}{2} \leq \frac{f(S)}{2} = \int_{0}^{\frac{1}{2}} f(S) \ \mathrm{d}\rho \leq \int_{0}^{\frac{1}{2}} f(T_\rho(\boldsymbol{\gamma})) \ \mathrm{d}\rho \leq \int_{0}^{1} f(T_\rho(\boldsymbol{\gamma})) \ \mathrm{d}\rho = \hat{f}(\boldsymbol{\gamma}) \leq f(S_{\lambda^*}),
    \]
    which completes the proof.

\end{proof}


\section{Full proof of~\texorpdfstring{\Cref{thm:cfminecc-NPhard}}{}}
\label{appendix:color-fair}
\label{app:cf}
% NP-hardness
\cfmineccNPhard*
{
\usetikzlibrary{math}

\definecolor{c1}{RGB}{0, 114, 178}
\definecolor{c2}{RGB}{213, 94, 0}
\definecolor{c3}{RGB}{0, 158, 115}
\definecolor{c3prime}{RGB}{240, 228, 66}
\definecolor{b1}{RGB}{230, 159, 0}
\definecolor{b2}{RGB}{86, 180, 233}
\definecolor{b3}{RGB}{204, 121, 167}

% \colorlet{c1}{cyan}
% \colorlet{c2}{red}
% \colorlet{c3}{green}
% \colorlet{c3prime}{violet}
% \colorlet{b1}{brown}
% \colorlet{b2}{olive}
% \colorlet{b3}{gray}

\begin{figure}[t]
    \centering
    \begin{tikzpicture}

        \tikzmath{
            \leftx = 0;
            \dy = 1;
            \confy = 0;
            \freey = \confy - \dy;
            \bridgey = \confy - 2*\dy;
            \sparey = \confy - 3*\dy;
            \dx = .9;
            \edgewidth = 1.25;
            \keyx = \leftx + 15*\dx;
            \keyy = \confy + 1.5;
            \keydx = 1.5;
            \keydy = -.5;
            \keytitlex = \keyx-.2;}

        \node[circle, label=right:$(x_1 \lor x_2 \lor x_3) \land (x_1 \lor \neg x_2 \lor \neg x_3) \land (\neg x_1 \lor \neg x_2)$] at (\leftx, \keyy) (formula) {};

        \node[circle, draw, fill=black, label=below:$v_1^1$] at (\leftx, \freey) (free11) {};
        \node[circle, draw, fill=black, label=above:$v_{1,3}^1$] at (\leftx, \confy) (conf131) {};
        \node[circle, draw, fill=black, label=above:$v_{2,3}^1$] at (\leftx + \dx, \confy) (conf231) {};
        \node[circle, draw, fill=black, label=below:$v_2^1$] at (\leftx + \dx, \freey) (free21) {};

        \draw[c1, line width=\edgewidth pt] (free11) -- (conf131);
        \draw[c3, line width=\edgewidth pt] (conf131) -- (conf231);
        \draw[c2, line width=\edgewidth pt] (conf231) -- (free21);

        \node[circle, draw, fill=black, label=below:$b_{1,2}^1$] at (\leftx + 2*\dx, \bridgey) (bridge121) {};
        \node[circle, draw, fill=black, label=below:$b_{1,2}^2$] at (\leftx + 3*\dx, \bridgey) (bridge122) {};

        \node[circle, draw, fill=black, label=below:$v_2^2$] at (\leftx + 4*\dx, \freey) (free22) {};
        \node[circle, draw, fill=black, label=above:$v_{1,2}^2$] at (\leftx + 4*\dx, \confy) (conf122) {};
        \node[circle, draw, fill=black, label=above:$v_{1,3}^2$] at (\leftx + 5*\dx, \confy) (conf132) {};
        \node[circle, draw, fill=black, label=below:$v_3^2$] at (\leftx + 5*\dx, \freey) (free32) {};

        \draw[c2, line width=\edgewidth pt] (free22) -- (conf122);
        \draw[c1, line width=\edgewidth pt] (conf122) -- (conf132);
        \draw[c3, line width=\edgewidth pt] (conf132) -- (free32);

        \node[circle, draw, fill=black, label=below:$b_{2,3}^1$] at (\leftx + 6*\dx, \bridgey) (bridge231) {};
        \node[circle, draw, fill=black, label=below:$b_{2,3}^2$] at (\leftx + 7*\dx, \bridgey) (bridge232) {};

        \node[circle, draw, fill=black, label=below:$v_1^3$] at (\leftx + 8*\dx, \freey) (free13) {};
        \node[circle, draw, fill=black, label=above:$v_{1,2}^3$] at (\leftx + 9*\dx, \confy) (conf123) {};
        \node[circle, draw, fill=black, label=below:$v_2^3$] at (\leftx + 10*\dx, \freey) (free23) {};

        \draw[c1, line width=\edgewidth pt] (free13) -- (conf123);
        \draw[c2, line width=\edgewidth pt] (conf123) -- (free23);

        \node[circle, draw, fill=black, label=below:$b_{3,4}^1$] at (\leftx + 11*\dx, \bridgey) (bridge341) {};
        \node[circle, draw, fill=black, label=below:$b_{3,4}^2$] at (\leftx + 12*\dx, \bridgey) (bridge342) {};

        \node[circle, draw, fill=black, label=below:$v_3^e$] at (\leftx + 13*\dx, \sparey) (spare3e) {};
        \node[circle, draw, fill=black, label=below:$v_3^b$] at (\leftx + 14*\dx, \sparey) (spare3b) {};
        \node[circle, draw, fill=black, label=below:$v_3^a$] at (\leftx + 15*\dx, \sparey) (spare3a) {};
        \node[circle, draw, fill=black, label=below:$v_3^c$] at (\leftx + 16*\dx, \sparey) (spare3c) {};
        \node[circle, draw, fill=black, label=below:$v_3^d$]     at (\leftx + 17*\dx, \sparey) (spare3d) {};

        \draw[c3prime, line width=\edgewidth pt] (spare3e) -- (spare3b);
        \draw[c3, line width=\edgewidth pt] (spare3b) -- (spare3a);
        \draw[c3prime, line width=\edgewidth pt] (spare3a) -- (spare3c);
        \draw[c3prime, line width=\edgewidth pt] (spare3a) to[bend left=50] (spare3d);
    
        
        \draw[b1, line width=\edgewidth pt] (free21) -- (bridge121);
        \draw[b1, line width=\edgewidth pt] (bridge121) -- (bridge122);
        \draw[b1, line width=\edgewidth pt] (bridge122) -- (free22);

        \draw[b2, line width=\edgewidth pt] (free32) -- (bridge231);
        \draw[b2, line width=\edgewidth pt] (bridge231) -- (bridge232);
        \draw[b2, line width=\edgewidth pt] (bridge232) -- (free13);

        \draw[b3, line width=\edgewidth pt] (free23) -- (bridge341);
        \draw[b3, line width=\edgewidth pt] (bridge341) -- (bridge342);
        \draw[b3, line width=\edgewidth pt] (bridge342) -- (spare3e);


        \node[circle, label=right:{Color Key:}] at (\keytitlex, \keyy) () {};

        \node[circle] at (\keyx, \keyy + \keydy) (c1left) {};
        \node[circle, label=right:$c_1$] at (\keyx + \keydx, \keyy + \keydy) (c1right) {};
        \draw[c1, line width=\edgewidth pt] (c1left) -- (c1right);

        \node[circle] at (\keyx, \keyy + 2*\keydy) (c1left) {};
        \node[circle, label=right:$c_2$] at (\keyx + \keydx, \keyy + 2*\keydy) (c1right) {};
        \draw[c2, line width=\edgewidth pt] (c1left) -- (c1right);

        \node[circle] at (\keyx, \keyy + 3*\keydy) (c1left) {};
        \node[circle, label=right:$c_3$] at (\keyx + \keydx, \keyy + 3*\keydy) (c1right) {};
        \draw[c3, line width=\edgewidth pt] (c1left) -- (c1right);

        \node[circle] at (\keyx, \keyy + 4*\keydy) (c1left) {};
        \node[circle, label=right:$c_3'$] at (\keyx + \keydx, \keyy + 4*\keydy) (c1right) {};
        \draw[c3prime, line width=\edgewidth pt] (c1left) -- (c1right);

        \node[circle] at (\keyx, \keyy + 5*\keydy) (c1left) {};
        \node[circle, label=right:$c_{1,2}^b$] at (\keyx + \keydx, \keyy + 5*\keydy) (c1right) {};
        \draw[b1, line width=\edgewidth pt] (c1left) -- (c1right);

        \node[circle] at (\keyx, \keyy + 6*\keydy) (c1left) {};
        \node[circle, label=right:$c_{2,3}^b$] at (\keyx + \keydx, \keyy + 6*\keydy) (c1right) {};
        \draw[b2, line width=\edgewidth pt] (c1left) -- (c1right);

        \node[circle] at (\keyx, \keyy + 7*\keydy) (c1left) {};
        \node[circle, label=right:$c_{3,4}^b$] at (\keyx + \keydx, \keyy + 7*\keydy) (c1right) {};
        \draw[b3, line width=\edgewidth pt] (c1left) -- (c1right);
    \end{tikzpicture}
    \caption{\label{fig:hardness} The construction given by~\Cref{thm:cfminecc-NPhard} for the CNF formula on three clauses $C_1 = (x_1 \lor x_2 \lor x_3)$, $C_2 = (x_1 \lor \neg x_2 \lor \neg x_3)$, and $C_3 = (\neg x_1 \lor \neg x_2)$. 
    Notice that the vertices are partitioned visually into four horizontal layers. The top layer contains \emph{conflict} vertices, the second from top contains \emph{free} vertices, the second from bottom \emph{bridge} vertices, and the bottom \emph{spare} vertices.
    Every edge containing a conflict vertex is a \emph{conflict} edge, every edge containing a bridge vertex is a \emph{bridge} edge, and all other edges are \emph{spare} edges.
    The colors $c_1, c_2$ and $c_3$ correspond to the clauses $C_1, C_2$, and $C_3$. The color $c_3'$ is the \emph{spare} color associated with $C_3$. The remaining colors are \emph{bridge} colors. We refer to the proof of~\Cref{thm:cfminecc-NPhard} for a formal description of the construction and the accompanying analysis.
    }

\end{figure}
}
\begin{proof}
    We reduce from \boolSAT{}, for which the input is a formula written in conjunctive normal form, consisting of $m$ clauses $C_1, C_2, \ldots C_m$ over $n$ variables $x_1, x_2, \ldots x_n$.
    For each variable $x_i$, we write $x_i$ and $\neg x_i$ for the corresponding positive and negative literals.
    We make the standard assumptions that no clause contains both literals of any variable, and that every literal appears at least once.
    We also assume that each clause has size $2$ or $3$, and that each variable appears in at most three clauses (implying that each literal appears in at most two clauses); hardness is retained under these assumptions~\cite{tovey1984simplified}.
    Given an instance of this problem, we construct an instance $(G = (V, E), \tau = 2)$ of \cfminECC{}. See~\Cref{fig:hardness} for a visual aid.

    For each clause $C_j$, we create a unique color $c_j$. Also, if $C_j$ has size two, we create a second unique color $c_j'$, and five vertices $v_j^a, v_j^b, v_j^c, v_j^d,$ and $v_j^e$.
    We call these the \emph{spare vertices} associated with $C_j$.
    The reason for creating these vertices and the second color $c_j'$ will become apparent later.
    Next, for each variable $x_i$, we create a vertex $v_{j_1, j_2}^i$ for each (ordered) pair of clauses $C_{j_1}, C_{j_2}$ with $C_{j_1}$ containing the positive literal $x_i$ and $C_{j_2}$ containing the negative literal $\neg x_i$.
    We call $v_{j_1, j_2}^i$ a \emph{conflict vertex} for the variable $x_i$ and the variable-clause pairs $(x_i, C_{j_1})$ and $(x_i, C_{j_2})$. Observe that each variable has either one or two associated conflict vertices, as does each variable-clause pair.
    For each variable-clause pair $(x_i, C_j)$ with a single associated conflict vertex, we create an additional vertex $v_j^i$ and call this the \emph{free vertex} for the variable-clause pair $(x_i, C_j)$.

    Now we create edges.
    For each clause $C_j$, we begin by creating one edge $e_j^i$ of color $c_j$ for each variable $x_i$ contained in $C_j$. This edge contains either the two $(x_i, C_j)$ conflict vertices, or the single $(x_i, C_j)$ conflict vertex and the $(x_i, C_j)$ free vertex.
    We call the edge $e_j^i$ a \emph{conflict edge} associated both with clause $C_j$, and with the literal of $x_i$ which appears in $C_j$.
    Observe that every free vertex has degree one, and every conflict vertex has degree two.
    For clauses $C_j$ of size $3$, there are exactly three edges of color $c_j$.
    For clauses $C_j$ of size $2$, we have thus far created two edges of color $c_j$.
    For each such clause, we use the associated spare vertices to create a third edge $\{v_j^a, v_j^b\}$ of color $c_j$, which we call the \emph{spare edge} associated with $C_j$, and three edges $\{v_j^a, v_j^c\}$, $\{v_j^a, v_j^d\}$, and $\{v_j^b, v_j^e\}$, each of color $c_j'$.
    Observe that our constructed graph still has maximum degree three, and that there are now exactly $3$ edges of every color.
    Finally, we set $\tau = 2$.

    Since free vertices have degree one, they do not participate in cycles.
    It is also clear from the construction that spare vertices do not participate in cycles.
    Thus, any cycle contains only conflict vertices.
    Note that if two conflict vertices share an edge, then they are associated with the same variable-clause pair.
    It follows that every vertex in a cycle is associated with some single variable-clause pair.
    However, a variable-clause pair has at most two associated conflict vertices, so we have constructed a forest.

    Now we will add some gadgets to turn our forest into a tree. Suppose that the graph we have constructed so far has $q$ connected components.
    Impose an arbitrary order on these components, and label them $G_1, G_2, \ldots, G_q$.
    Observe that every connected component contains either only spare vertices or two free vertices.
    In both cases, it is possible to add two edges each with one (distinct) endpoint in $G_i$ without raising the maximum degree above three.
    That the endpoints are distinct will be important when we analyze the cutwidth.
    For each consecutive pair $G_i, G_{i+1}$ of connected components, we add two \emph{bridge vertices} $b_{i, i+1}^1, b_{i, i+1}^2$ and a \emph{bridge color} $c_{i, i+1}^b$.
    We add \emph{bridge edges} (chosen so as not to violate our maximum degree constraint) from $G_i$ to $b_{i,i+1}^1$, from $b_{i, i+1}^1$ to $b_{i, i+1}^2$, and from $b_{i, i+1}^2$ to $G_{i+1}$.
    We color each of these three edges with $c_{i, i+1}^b$. Observe that the graph is now connected, but it is still acyclic since there is exactly
    one path between any pair of vertices which were in different connected components before the addition of our bridge gadgets.

    We now claim that the constructed graph $G$ has cutwidth~$2$.
    To this end, we will construct a ordering $\sigma\colon V \rightarrow \mathbb{N}$ of the vertices of $G$, where $\sigma$ is injective, $\sigma(v) = 1$ indicates that $v$ is the first vertex in the ordering, $\sigma(u) = |V|$ indicates that $u$ is the last vertex in the ordering, and $\sigma(v) < \sigma(u)$ if and only if $v$ precedes $u$ in the ordering.
    We will show that for every $i$, there exist at most two edges $uv \in E$ with the property that $\sigma(u) \leq i$ and $\sigma(v) > i$. We refer to this number of edges as the \emph{width} of the cut between vertices $i$ and $i+1$ in $\sigma$.
    We begin by guaranteeing that, for every $i$, if $u$ is the last vertex of $G_i$ in $\sigma$, $v$ is the first vertex of $G_{i+1}$ in $\sigma$, and $b_{i, i+1}^1, b_{i, i+1}^2$ are the associated bridge vertices, then $\sigma(u) = \sigma(b_{i, i+1}^1) - 1 = \sigma(b_{i, i+1}^2) - 2 = \sigma(v) - 3$.
    A consequence is that for every $i$, if $u$, $v$ are vertices of $G_i$ then every vertex $w$ with the property that $\sigma(u) < \sigma(w) < \sigma(v)$ is also a vertex of $G_i$.
    Observe that every cut between two bridge vertices has width $1$, as does every cut between a bridge vertex and a non-bridge vertex.
    We therefore need only consider cuts between pairs of vertices in the same $G_i$.
    % Note that the edges of such a cut have both endpoints in $G_i$. We therefore need only consider the cutwidth of each $G_i$.
    If $G_i$ contains a conflict vertex, then $G_i$ is either a $P_3$ or a $P_4$.
    The former case arises when a variable appears in exactly two clauses, so there is one associated conflict vertex adjacent to two free vertices.
    The latter case arises when a variable appears in exactly three clauses, so there are two associated conflict vertices.
    These are adjacent.
    Additionally, each conflict vertex is adjacent to a distinct free vertex, so we have a $P_4$.
    Both the $P_3$ and $P_4$ have cutwidth $1$, as evidenced by ordering the vertices as they appear along the path.
    Moreover, we may assume that the relevant bridge edges are incident on the appropriate endpoints of the path, and so any cut between vertices of $G_i$ has width~$1$.
    Otherwise, $G_i$ contains no conflict vertices.
    In this case, $G_i$ contains only the five spare vertices $v_j^a, v_j^b, v_j^c, v_j^d$, and $v_j^e$ associated with some clause $C_j$, and the four edges are $v_j^av_j^b, v_j^av_j^c, v_j^av_j^d$, and $v_j^bv_j^e$.
    It is simple to check that this construction has cutwidth $2$, as evidenced by the ordering $\sigma(v_j^e) < \sigma(v_j^b) < \sigma(v_j^a) < \sigma(v_j^c) < \sigma(v_j^d)$.
    Once again, we may assume that the relevant bridge edges are incident on $v_j^e$ and $v_j^d$.
    Moreover, we can assume that at least one such $G_i$ exists, since \boolSAT{} instances in which every clause has size three and every variable appears at most three times are polynomial-time solvable~\cite{tovey1984simplified}.
    Thus, $G$ has cutwidth $2$.

    It remains to show that the reduction is correct. For the first direction, assume that there exists an assignment $\phi$ of boolean values to the variables $x_1, x_2, \ldots, x_n$ which satisfies every clause. We say that the assignment $\phi$ \emph{agrees} with a variable-clause pair $(x_i, C_j)$ if $C_j$ contains the positive literal $x_i$ and $\phi(x_i) = \textsf{True}$ or if $C_j$ contains the negative literal $\neg x_i$ and $\phi(x_i) = \textsf{False}$. We color the vertices of our constructed graph as follows.
    To every free vertex associated with clause $C_j$ we assign color $c_j$.
    The free vertex was only in edges of color $c_j$, so this results in zero unsatisfied edges.
    Next, to each bridge vertex we assign the associated bridge color.
    The bridge vertices were only in edges of this color, so once again this results in zero unsatisfied edges.
    Moreover, we have now guaranteed that at least one bridge edge of every bridge color is satisfied, meaning that at most two can be dissatisfied.
    Henceforth, we will not consider the remaining bridge edges.
    Next, for each clause $C_j$ of size $2$, we assign color $c_j'$ to every spare vertex associated with $C_j$. This results in exactly one unsatisfied edge of color $c_j$, and ensures that every edge of color $c_j'$ is satisfied. Finally, for each conflict vertex $v_{j_1, j_2}^i$, if $\phi$ agrees with $(x_i, C_{j_1})$ we assign color $c_{j_1}$ to $v_{j_1, j_2}^i$, and otherwise we assign color $c_{j_2}$.
    Observe that this coloring satisfies an edge $e_j^i$ if and only if $\phi$ agrees with $(x_i, C_j)$.
    Moreover, because $\phi$ is satisfying,
    every clause $C_j$ contains at least one variable $x_i$ such that $\phi$ agrees with $(x_i, C_j)$.
    Hence, at least one edge of every color is satisfied. Because there are exactly three edges of every color, there are at most two unsatisfied edges of any color.

    For the other direction, assume that we have a coloring which leaves at most two edges of any color unsatisfied. We will create a satisfying assignment $\phi$.
    For each variable $x_i$, we set $\phi(x_i) = \textsf{True}$ if any conflict edge associated with the positive literal $x_i$ is satisfied, and $\phi(x_i) = \textsf{False}$ otherwise.
    We now show that $\phi$ is satisfying. Consider any clause $C_j$. There are exactly three edges with color $c_j$, and at least one of them is satisfied.
    We may assume that the spare edge associated with $C_j$ (if such an edge exists) is unsatisfied, since satisfying this edge would require three unsatisfied edges of color $c_j'$. Thus,
    at least one conflict edge associated with $C_j$ is satisfied. Let $x_i$ be the corresponding variable, so the satisfied conflict edge is $e_j^i$.
    If $C_j$ contains the positive literal $x_i$, then $\phi(x_i) = \textsf{True}$ so $C_j$ is satisfied by $\phi$.
    Otherwise $C_j$ contains the negative literal $\neg x_i$.
    In this case, we observe that every conflict edge associated with the positive literal $x_i$ intersects with $e_j^i$ at a conflict vertex, and none of these edges has color $c_j$ since
    $C_j$ does not contain both literals. Hence, the satisfaction of $e_j^i$ implies that every conflict edge associated with the positive literal $x_i$ is unsatisfied.
    It follows that $\phi(x_i) = \textsf{False}$, meaning $C_j$ is satisfied by $\phi$.

    To show the claim for \cfmaxECC{}, we repeat the same construction, except that we omit all spare colors, vertices, and edges. The effect is that the constructed graph is a path.
    The proof of correctness is conceptually unchanged. Given a satisfying assignment $\phi$ we color vertices in the same way as before, and given a vertex coloring which satisfies
    at least one edge of every color, it remains the case that at least one conflict edge associated with every clause must be satisfied.
    The remaining analysis is similar.
\end{proof}

% %cfminECC FPT
% \cfminECCFPT*
% \begin{proof}
%     We give a branching algorithm.
%     Given an instance $(H = (V, E), \tau)$ of \cfminECC{}, a \emph{conflict} is a triple $(v, e_1, e_2)$ consisting of a single vertex $v$ and a pair of distinctly colored hyperedges $e_1, e_2$ which both contain $v$.
%     If $H$ contains no conflicts, then it is possible to satisfy every edge.
%     Otherwise, we identify a conflict in $O(r|E|)$ time by scanning the set of hyperedges incident on each node.
%     Once a conflict $(v, e_1, e_2)$ has been found, we branch on the two possible ways to resolve this conflict: deleting $e_1$ or deleting $e_2$.
%     Here, deleting a hyperedge has the same effect as ``marking'' it as unsatisfied and no longer considering it for the duration of the algorithm.
%     We note that it is simple to check in constant time whether a possible branch violates the constraint given by $\tau$; these branches can be pruned.
%     Because each branch increases the number of unsatisfied hyperedges by 1, the search tree has depth at most $\edgedeletions$.
%     Thus, by computing the search tree in level-order, the algorithm runs in time $O(2^{\edgedeletions}r|E|)$.
% \end{proof}

%


% NP-hard k = 2
% \cfmineccNPhardboundedk*


% reduction to sparse vertex cover


% combinatorial algorithm
% \cfminecccombinatorial*



\section{Full proof of~\texorpdfstring{\Cref{thm:pc-ECC-FPT}}{}}
\label{app:pc}

% \pceccapproximation*


\pceccFPT*
\begin{proof}
    We give a branching algorithm which is essentially identical to that of~\Cref{thm:cfminecc-FPT}.
    For completeness, we repeat the details.
    Given an instance $(H = (V, E), t, b)$ of \pcECC{}, a \emph{conflict} is a triple $(v, e_1, e_2)$ consisting of a single vertex $v$ and a pair of distinctly colored hyperedges $e_1, e_2$ which both contain $v$.
    If $H$ contains no conflicts, then it is possible to satisfy every edge.
    Otherwise, we identify a conflict in $O(r|E|)$ time by scanning the set of hyperedges incident on each node.
    Once a conflict $(v, e_1, e_2)$ has been found, we branch on the two possible ways to resolve this conflict: deleting $e_1$ or deleting $e_2$.
    Here, deleting a hyperedge has the same effect as ``marking'' it as unsatisfied and no longer considering it for the duration of the algorithm.
    We note that it is simple to check in constant time whether a possible branch violates the constraints given by $t$ or $b$; these branches can be pruned.
    Because each branch increases the number of unsatisfied hyperedges by 1 and WLOG $t > b$, the search tree has depth at most $t$.
    Thus, the algorithm runs in time $O(2^{t}r|E|)$.
\end{proof}



\section{Experiments}
\label{sec:experiments}
The experiments are designed to address two key research questions.
First, \textbf{RQ1} evaluates whether the average $L_2$-norm of the counterfactual perturbation vectors ($\overline{||\perturb||}$) decreases as the model overfits the data, thereby providing further empirical validation for our hypothesis.
Second, \textbf{RQ2} evaluates the ability of the proposed counterfactual regularized loss, as defined in (\ref{eq:regularized_loss2}), to mitigate overfitting when compared to existing regularization techniques.

% The experiments are designed to address three key research questions. First, \textbf{RQ1} investigates whether the mean perturbation vector norm decreases as the model overfits the data, aiming to further validate our intuition. Second, \textbf{RQ2} explores whether the mean perturbation vector norm can be effectively leveraged as a regularization term during training, offering insights into its potential role in mitigating overfitting. Finally, \textbf{RQ3} examines whether our counterfactual regularizer enables the model to achieve superior performance compared to existing regularization methods, thus highlighting its practical advantage.

\subsection{Experimental Setup}
\textbf{\textit{Datasets, Models, and Tasks.}}
The experiments are conducted on three datasets: \textit{Water Potability}~\cite{kadiwal2020waterpotability}, \textit{Phomene}~\cite{phomene}, and \textit{CIFAR-10}~\cite{krizhevsky2009learning}. For \textit{Water Potability} and \textit{Phomene}, we randomly select $80\%$ of the samples for the training set, and the remaining $20\%$ for the test set, \textit{CIFAR-10} comes already split. Furthermore, we consider the following models: Logistic Regression, Multi-Layer Perceptron (MLP) with 100 and 30 neurons on each hidden layer, and PreactResNet-18~\cite{he2016cvecvv} as a Convolutional Neural Network (CNN) architecture.
We focus on binary classification tasks and leave the extension to multiclass scenarios for future work. However, for datasets that are inherently multiclass, we transform the problem into a binary classification task by selecting two classes, aligning with our assumption.

\smallskip
\noindent\textbf{\textit{Evaluation Measures.}} To characterize the degree of overfitting, we use the test loss, as it serves as a reliable indicator of the model's generalization capability to unseen data. Additionally, we evaluate the predictive performance of each model using the test accuracy.

\smallskip
\noindent\textbf{\textit{Baselines.}} We compare CF-Reg with the following regularization techniques: L1 (``Lasso''), L2 (``Ridge''), and Dropout.

\smallskip
\noindent\textbf{\textit{Configurations.}}
For each model, we adopt specific configurations as follows.
\begin{itemize}
\item \textit{Logistic Regression:} To induce overfitting in the model, we artificially increase the dimensionality of the data beyond the number of training samples by applying a polynomial feature expansion. This approach ensures that the model has enough capacity to overfit the training data, allowing us to analyze the impact of our counterfactual regularizer. The degree of the polynomial is chosen as the smallest degree that makes the number of features greater than the number of data.
\item \textit{Neural Networks (MLP and CNN):} To take advantage of the closed-form solution for computing the optimal perturbation vector as defined in (\ref{eq:opt-delta}), we use a local linear approximation of the neural network models. Hence, given an instance $\inst_i$, we consider the (optimal) counterfactual not with respect to $\model$ but with respect to:
\begin{equation}
\label{eq:taylor}
    \model^{lin}(\inst) = \model(\inst_i) + \nabla_{\inst}\model(\inst_i)(\inst - \inst_i),
\end{equation}
where $\model^{lin}$ represents the first-order Taylor approximation of $\model$ at $\inst_i$.
Note that this step is unnecessary for Logistic Regression, as it is inherently a linear model.
\end{itemize}

\smallskip
\noindent \textbf{\textit{Implementation Details.}} We run all experiments on a machine equipped with an AMD Ryzen 9 7900 12-Core Processor and an NVIDIA GeForce RTX 4090 GPU. Our implementation is based on the PyTorch Lightning framework. We use stochastic gradient descent as the optimizer with a learning rate of $\eta = 0.001$ and no weight decay. We use a batch size of $128$. The training and test steps are conducted for $6000$ epochs on the \textit{Water Potability} and \textit{Phoneme} datasets, while for the \textit{CIFAR-10} dataset, they are performed for $200$ epochs.
Finally, the contribution $w_i^{\varepsilon}$ of each training point $\inst_i$ is uniformly set as $w_i^{\varepsilon} = 1~\forall i\in \{1,\ldots,m\}$.

The source code implementation for our experiments is available at the following GitHub repository: \url{https://anonymous.4open.science/r/COCE-80B4/README.md} 

\subsection{RQ1: Counterfactual Perturbation vs. Overfitting}
To address \textbf{RQ1}, we analyze the relationship between the test loss and the average $L_2$-norm of the counterfactual perturbation vectors ($\overline{||\perturb||}$) over training epochs.

In particular, Figure~\ref{fig:delta_loss_epochs} depicts the evolution of $\overline{||\perturb||}$ alongside the test loss for an MLP trained \textit{without} regularization on the \textit{Water Potability} dataset. 
\begin{figure}[ht]
    \centering
    \includegraphics[width=0.85\linewidth]{img/delta_loss_epochs.png}
    \caption{The average counterfactual perturbation vector $\overline{||\perturb||}$ (left $y$-axis) and the cross-entropy test loss (right $y$-axis) over training epochs ($x$-axis) for an MLP trained on the \textit{Water Potability} dataset \textit{without} regularization.}
    \label{fig:delta_loss_epochs}
\end{figure}

The plot shows a clear trend as the model starts to overfit the data (evidenced by an increase in test loss). 
Notably, $\overline{||\perturb||}$ begins to decrease, which aligns with the hypothesis that the average distance to the optimal counterfactual example gets smaller as the model's decision boundary becomes increasingly adherent to the training data.

It is worth noting that this trend is heavily influenced by the choice of the counterfactual generator model. In particular, the relationship between $\overline{||\perturb||}$ and the degree of overfitting may become even more pronounced when leveraging more accurate counterfactual generators. However, these models often come at the cost of higher computational complexity, and their exploration is left to future work.

Nonetheless, we expect that $\overline{||\perturb||}$ will eventually stabilize at a plateau, as the average $L_2$-norm of the optimal counterfactual perturbations cannot vanish to zero.

% Additionally, the choice of employing the score-based counterfactual explanation framework to generate counterfactuals was driven to promote computational efficiency.

% Future enhancements to the framework may involve adopting models capable of generating more precise counterfactuals. While such approaches may yield to performance improvements, they are likely to come at the cost of increased computational complexity.


\subsection{RQ2: Counterfactual Regularization Performance}
To answer \textbf{RQ2}, we evaluate the effectiveness of the proposed counterfactual regularization (CF-Reg) by comparing its performance against existing baselines: unregularized training loss (No-Reg), L1 regularization (L1-Reg), L2 regularization (L2-Reg), and Dropout.
Specifically, for each model and dataset combination, Table~\ref{tab:regularization_comparison} presents the mean value and standard deviation of test accuracy achieved by each method across 5 random initialization. 

The table illustrates that our regularization technique consistently delivers better results than existing methods across all evaluated scenarios, except for one case -- i.e., Logistic Regression on the \textit{Phomene} dataset. 
However, this setting exhibits an unusual pattern, as the highest model accuracy is achieved without any regularization. Even in this case, CF-Reg still surpasses other regularization baselines.

From the results above, we derive the following key insights. First, CF-Reg proves to be effective across various model types, ranging from simple linear models (Logistic Regression) to deep architectures like MLPs and CNNs, and across diverse datasets, including both tabular and image data. 
Second, CF-Reg's strong performance on the \textit{Water} dataset with Logistic Regression suggests that its benefits may be more pronounced when applied to simpler models. However, the unexpected outcome on the \textit{Phoneme} dataset calls for further investigation into this phenomenon.


\begin{table*}[h!]
    \centering
    \caption{Mean value and standard deviation of test accuracy across 5 random initializations for different model, dataset, and regularization method. The best results are highlighted in \textbf{bold}.}
    \label{tab:regularization_comparison}
    \begin{tabular}{|c|c|c|c|c|c|c|}
        \hline
        \textbf{Model} & \textbf{Dataset} & \textbf{No-Reg} & \textbf{L1-Reg} & \textbf{L2-Reg} & \textbf{Dropout} & \textbf{CF-Reg (ours)} \\ \hline
        Logistic Regression   & \textit{Water}   & $0.6595 \pm 0.0038$   & $0.6729 \pm 0.0056$   & $0.6756 \pm 0.0046$  & N/A    & $\mathbf{0.6918 \pm 0.0036}$                     \\ \hline
        MLP   & \textit{Water}   & $0.6756 \pm 0.0042$   & $0.6790 \pm 0.0058$   & $0.6790 \pm 0.0023$  & $0.6750 \pm 0.0036$    & $\mathbf{0.6802 \pm 0.0046}$                    \\ \hline
%        MLP   & \textit{Adult}   & $0.8404 \pm 0.0010$   & $\mathbf{0.8495 \pm 0.0007}$   & $0.8489 \pm 0.0014$  & $\mathbf{0.8495 \pm 0.0016}$     & $0.8449 \pm 0.0019$                    \\ \hline
        Logistic Regression   & \textit{Phomene}   & $\mathbf{0.8148 \pm 0.0020}$   & $0.8041 \pm 0.0028$   & $0.7835 \pm 0.0176$  & N/A    & $0.8098 \pm 0.0055$                     \\ \hline
        MLP   & \textit{Phomene}   & $0.8677 \pm 0.0033$   & $0.8374 \pm 0.0080$   & $0.8673 \pm 0.0045$  & $0.8672 \pm 0.0042$     & $\mathbf{0.8718 \pm 0.0040}$                    \\ \hline
        CNN   & \textit{CIFAR-10} & $0.6670 \pm 0.0233$   & $0.6229 \pm 0.0850$   & $0.7348 \pm 0.0365$   & N/A    & $\mathbf{0.7427 \pm 0.0571}$                     \\ \hline
    \end{tabular}
\end{table*}

\begin{table*}[htb!]
    \centering
    \caption{Hyperparameter configurations utilized for the generation of Table \ref{tab:regularization_comparison}. For our regularization the hyperparameters are reported as $\mathbf{\alpha/\beta}$.}
    \label{tab:performance_parameters}
    \begin{tabular}{|c|c|c|c|c|c|c|}
        \hline
        \textbf{Model} & \textbf{Dataset} & \textbf{No-Reg} & \textbf{L1-Reg} & \textbf{L2-Reg} & \textbf{Dropout} & \textbf{CF-Reg (ours)} \\ \hline
        Logistic Regression   & \textit{Water}   & N/A   & $0.0093$   & $0.6927$  & N/A    & $0.3791/1.0355$                     \\ \hline
        MLP   & \textit{Water}   & N/A   & $0.0007$   & $0.0022$  & $0.0002$    & $0.2567/1.9775$                    \\ \hline
        Logistic Regression   &
        \textit{Phomene}   & N/A   & $0.0097$   & $0.7979$  & N/A    & $0.0571/1.8516$                     \\ \hline
        MLP   & \textit{Phomene}   & N/A   & $0.0007$   & $4.24\cdot10^{-5}$  & $0.0015$    & $0.0516/2.2700$                    \\ \hline
       % MLP   & \textit{Adult}   & N/A   & $0.0018$   & $0.0018$  & $0.0601$     & $0.0764/2.2068$                    \\ \hline
        CNN   & \textit{CIFAR-10} & N/A   & $0.0050$   & $0.0864$ & N/A    & $0.3018/
        2.1502$                     \\ \hline
    \end{tabular}
\end{table*}

\begin{table*}[htb!]
    \centering
    \caption{Mean value and standard deviation of training time across 5 different runs. The reported time (in seconds) corresponds to the generation of each entry in Table \ref{tab:regularization_comparison}. Times are }
    \label{tab:times}
    \begin{tabular}{|c|c|c|c|c|c|c|}
        \hline
        \textbf{Model} & \textbf{Dataset} & \textbf{No-Reg} & \textbf{L1-Reg} & \textbf{L2-Reg} & \textbf{Dropout} & \textbf{CF-Reg (ours)} \\ \hline
        Logistic Regression   & \textit{Water}   & $222.98 \pm 1.07$   & $239.94 \pm 2.59$   & $241.60 \pm 1.88$  & N/A    & $251.50 \pm 1.93$                     \\ \hline
        MLP   & \textit{Water}   & $225.71 \pm 3.85$   & $250.13 \pm 4.44$   & $255.78 \pm 2.38$  & $237.83 \pm 3.45$    & $266.48 \pm 3.46$                    \\ \hline
        Logistic Regression   & \textit{Phomene}   & $266.39 \pm 0.82$ & $367.52 \pm 6.85$   & $361.69 \pm 4.04$  & N/A   & $310.48 \pm 0.76$                    \\ \hline
        MLP   &
        \textit{Phomene} & $335.62 \pm 1.77$   & $390.86 \pm 2.11$   & $393.96 \pm 1.95$ & $363.51 \pm 5.07$    & $403.14 \pm 1.92$                     \\ \hline
       % MLP   & \textit{Adult}   & N/A   & $0.0018$   & $0.0018$  & $0.0601$     & $0.0764/2.2068$                    \\ \hline
        CNN   & \textit{CIFAR-10} & $370.09 \pm 0.18$   & $395.71 \pm 0.55$   & $401.38 \pm 0.16$ & N/A    & $1287.8 \pm 0.26$                     \\ \hline
    \end{tabular}
\end{table*}

\subsection{Feasibility of our Method}
A crucial requirement for any regularization technique is that it should impose minimal impact on the overall training process.
In this respect, CF-Reg introduces an overhead that depends on the time required to find the optimal counterfactual example for each training instance. 
As such, the more sophisticated the counterfactual generator model probed during training the higher would be the time required. However, a more advanced counterfactual generator might provide a more effective regularization. We discuss this trade-off in more details in Section~\ref{sec:discussion}.

Table~\ref{tab:times} presents the average training time ($\pm$ standard deviation) for each model and dataset combination listed in Table~\ref{tab:regularization_comparison}.
We can observe that the higher accuracy achieved by CF-Reg using the score-based counterfactual generator comes with only minimal overhead. However, when applied to deep neural networks with many hidden layers, such as \textit{PreactResNet-18}, the forward derivative computation required for the linearization of the network introduces a more noticeable computational cost, explaining the longer training times in the table.

\subsection{Hyperparameter Sensitivity Analysis}
The proposed counterfactual regularization technique relies on two key hyperparameters: $\alpha$ and $\beta$. The former is intrinsic to the loss formulation defined in (\ref{eq:cf-train}), while the latter is closely tied to the choice of the score-based counterfactual explanation method used.

Figure~\ref{fig:test_alpha_beta} illustrates how the test accuracy of an MLP trained on the \textit{Water Potability} dataset changes for different combinations of $\alpha$ and $\beta$.

\begin{figure}[ht]
    \centering
    \includegraphics[width=0.85\linewidth]{img/test_acc_alpha_beta.png}
    \caption{The test accuracy of an MLP trained on the \textit{Water Potability} dataset, evaluated while varying the weight of our counterfactual regularizer ($\alpha$) for different values of $\beta$.}
    \label{fig:test_alpha_beta}
\end{figure}

We observe that, for a fixed $\beta$, increasing the weight of our counterfactual regularizer ($\alpha$) can slightly improve test accuracy until a sudden drop is noticed for $\alpha > 0.1$.
This behavior was expected, as the impact of our penalty, like any regularization term, can be disruptive if not properly controlled.

Moreover, this finding further demonstrates that our regularization method, CF-Reg, is inherently data-driven. Therefore, it requires specific fine-tuning based on the combination of the model and dataset at hand.

\newpage
\section*{Acknowledgements}
This work was supported in part by the Gordon \& Betty Moore Foundation under award GBMF4560 to Blair D. Sullivan, by the National Science Foundation under award IIS-1956286 to Blair D. Sullivan, and by the Army Research Office under award W911NF‐24-1-0156 to Nate Veldt.

\bibliography{refs.bib}
\bibliographystyle{plainnat}


%%%%%%%%%%%%%%%%%%%%%%%%%%%%%%%%%%%%%%%%%%%%%%%%%%%%%%%%%%%%%%%%%%%%%%%%%%%%%%%
%%%%%%%%%%%%%%%%%%%%%%%%%%%%%%%%%%%%%%%%%%%%%%%%%%%%%%%%%%%%%%%%%%%%%%%%%%%%%%%
% APPENDIX
%%%%%%%%%%%%%%%%%%%%%%%%%%%%%%%%%%%%%%%%%%%%%%%%%%%%%%%%%%%%%%%%%%%%%%%%%%%%%%%
%%%%%%%%%%%%%%%%%%%%%%%%%%%%%%%%%%%%%%%%%%%%%%%%%%%%%%%%%%%%%%%%%%%%%%%%%%%%%%%
%\newpage
\appendix
\onecolumn

\section{\maxecc{} Proofs}
\label{app:maxecc}
We use the following lemma as a small step in our approximation guarantee for hypergraph \maxecc{} (Theorem~\ref{thm:hypermaxecc}). The result was originally used in designing a $1/e^2$-approximation algorithm for graph \maxecc{}~\cite{angel2016clustering}. A full proof can be found in the work of~\cite{angel2016clustering}.
\begin{lemma}[Lemma 4 of~\citet{angel2016clustering}]
	\label{lem:angel}
	Let $\left\{X_1, X_2, \dots, X_j \right\}$ be a set of independent events satisfying $\sum_{i=1}^j \prob[X_i] \leq 1$, then the probability that {at most one} of them happens is greater than or equal to $2/e$.
\end{lemma}

The remainder of our supporting lemmas are new results that we prove for our approximation algorithm for graph \maxecc{}.

\lemdependentynx*
\begin{proof}
	For a node $u$ and color $i$ we use $\overline{X}_u^c$ to denote the event that $u$ does not want $c$.

	We first observe that it is sufficient to show positive correlation between $Y_u^c$ and $N_v$. To see this,
	we use Bayes' Theorem and the independence of $X_u^c$ and $N_v$ to write
	\begin{align*}
		\pgiv{Y_u^c}{N_v \cap X_u^c} & = \frac{\proba{Y_u^c \cap N_v \cap X_u^c}}{\proba{N_v \cap X_u^c}}                   \\
		                             & = \frac{\pgiv{N_v \cap X_u^c}{Y_u^c}\proba{Y_u^c}}{\proba{N_v \cap X_u^c}}           \\
		                             & = \frac{\pgiv{N_v \cap X_u^c}{Y_u^c}\proba{Y_u^c}}{\proba{N_v} \cdot \proba{X_u^c}},
	\end{align*}
	as well as
	\[
		\pgiv{Y_u^c}{X_u^c} = \frac{\proba{X_u^c \cap Y_u^c}}{\proba{X_u^c}} = \frac{\pgiv{X_u^c}{Y_u^c}\proba{Y_u^c}}{\proba{X_u^c}} = \frac{1\cdot \proba{Y_u^c}}{\proba{X_u^c}}.
	\]
	By rearranging terms, we see that our claim is equivalent to
	\[
		\pgiv{N_v \cap X_u^c}{Y_u^c} = \pgiv{N_v}{Y_u^c} \geq \proba{N_v}.
	\]
	We complete the proof by demonstrating the equivalent inequality $\pgiv{Y_u^c}{N_v} \geq \proba{Y_u^c}$. Impose an arbitrary order $c_1, c_2 \ldots, c_k$ on the color set, and without loss of generality assume that $\ell(e) = c = c_k$. In what follows, we always write $c$ for $c_k$ and whenever considering a subscripted color $c_i$ we assume that $i \in W_u \setminus \{c\}$.

	If node $u$ has a strong color, then without loss of generality we assume that that strong color is $c_{k-1}$.
	The remainder of the proof is written for the case where $S_u = \emptyset$, in which case the set of weak colors (other than $c$) that might want node $u$ is $W_u \setminus \{c\} = \{c_1, c_2, \hdots, c_{k-1}\}$. In other words, $W_u$ is the set of colors with indices in $[k-1]$. If instead we had $S_u = \{c_{k-1}\}$ and $W_u \setminus \{c\} = \{c_1, c_2, \hdots, c_{k-2}\}$, the proof works in exactly the same way if we instead considered the set of colors associated with indices in $[k-2]$ instead of indices in $[k-1]$.


	Now, consider all possible subsets of $[k - 1]$ which may define (according to our ordering) those weak colors (besides $c$) which are wanted by $u$. For a particular subset $S \subseteq [k - 1]$,
	we write $X_u^S$ for the event that $u$ wants exactly (not considering $c$) the colors in $S$. In the following, we write $w(S) = \frac{1}{|S| + 1}$ and call this quantity the \emph{weight} of $S$.
	The events $X_u^S$ partition the sample space,
	so we may write
	\begin{align*}
		\proba{Y_u^c} & = \sum_{S \subseteq [k - 1]} \pgiv{Y_u^c}{X_u^S}\cdot\proba{X_u^S} = \proba{X_u^c}\cdot\sum_{S \subseteq [k - 1]} w(S)\cdot\proba{X_u^S}.
	\end{align*}
	%
	In the above, the second equality follows from the observation that, conditioned on $X_u^S$ for any $S \subseteq [k - 1]$, the event $Y_u^c$ occurs if and only if (a) $u$ wants $c$ and (b) $c$ precedes each color in $S$ in the global ordering of colors.
	The conditions (a) and (b) are independent, and occur with probabilities $\proba{X_u^c}$ and $w(S)$, respectively.
	We will now make use of the independence
	of the $X_u^{c_i}$ events to write $\proba{X_u^S}$ as the product of $k - 1$ probabilities.
	\begin{align}\label{eq:yuc}
		\proba{Y_u^c} & = \proba{X_u^c}\cdot\sum_{S \subseteq [k - 1]} \left[ w(S)\cdot \left( \prod_{i \in S}\proba{X_u^{c_i}}\right)\cdot\left(\prod_{j \notin S}\proba{\overline{X}_u^{c_j}}\right) \right].
	\end{align}

	We want to write a similar expression for $\proba{Y_u^c | N_v}$. We begin by using similar reasoning as above to write
	\begin{align*}
		\pgiv{Y_u^c}{N_v} & = \sum_{S \subseteq [k - 1]} \pgiv{Y_u^c}{X_u^S \cap N_v}\cdot\pgiv{X_u^S}{N_v} \\
		                  & = \pgiv{X_u^c}{N_v}\cdot \sum_{S \subseteq [k - 1]} w(S)\cdot\pgiv{X_u^S}{N_v}  \\
		                  & = \proba{X_u^c}\cdot \sum_{S \subseteq [k - 1]} w(S)\cdot\pgiv{X_u^S}{N_v}.
	\end{align*}
	%
	Next, we observe that the $X_u^{c_i}$ events remain independent even when conditioned on $N_v$. This allows us to write

	\begin{align*}
		\pgiv{Y_u^c}{N_v} = \proba{X_u^c}\cdot\sum_{S \subseteq [k - 1]} \left[ w(S)\cdot \left( \prod_{i \in S}\pgiv{X_u^{c_i}}{N_v}\right)\cdot\left(\prod_{j \notin S}\pgiv{\overline{X}_u^{c_j}}{N_v}\right) \right].
	\end{align*}
	To simplify this expression further, we claim that for every $i \in [k - 1]$,
	\[\pgiv{X_u^{c_i}}{N_v} = \pgiv{X_u^{c_i}}{\overline{X}_v^{c_i}}. \]
	This can be derived from the observation
	that $\bigcap_{j \neq i} \overline{X}_v^{c_j}$ is independent of $X_u^{c_i}$ when conditioned on $\overline{X}_v^{c_i}$, as well as being unconditionally independent of $X_v^{c_i}$.
	We then manipulate the definition of conditional probability to see that
	\begin{align*}
		\pgiv{X_u^{c_i}}{N_v} & = \frac{\proba{X_u^{c_i} \cap \bigcap_{j \neq i} \overline{X}_v^{c_j} \cap \overline{X}_v^{c_i}}}{\proba{\bigcap_{j \neq i} \overline{X}_v^{c_j} \cap \overline{X}_v^{c_i}}}        \\
		                      & = \frac{\proba{X_u^{c_i} \cap \bigcap_{j \neq i} \overline{X}_v^{c_j} \cap \overline{X}_v^{c_i}}}{\proba{\bigcap_{j \neq i} \overline{X}_v^{c_j}}\cdot\proba{\overline{X}_v^{c_i}}} \\
		                      & = \frac{\pgiv{X_u^{c_i} \cap \bigcap_{j \neq i} \overline{X}_v^{c_j}}{\overline{X}_v^{c_i}}}{\proba{\bigcap_{j \neq i} \overline{X}_v^{c_j}}}                                       \\
		                      & = \frac{\pgiv{X_u^{c_i}}{\overline{X}_v^{c_i}}\cdot\pgiv{\bigcap_{j \neq i} \overline{X}_v^{c_j}}{\overline{X}_v^{c_i}}}{\proba{\bigcap_{j \neq i} \overline{X}_v^{c_j}}}           \\
		                      & = \frac{\pgiv{X_u^{c_i}}{\overline{X}_v^{c_i}}\cdot\proba{\bigcap_{j \neq i} \overline{X}_v^{c_j}}}{\proba{\bigcap_{j \neq i} \overline{X}_v^{c_j}}}                                \\
		                      & = \pgiv{X_u^{c_i}}{\overline{X}_v^{c_i}}.
	\end{align*}

	We can now write a suitable counterpart to Equation~\eqref{eq:yuc}:
	\begin{align}\label{eq:yuc-conditional-nv}
		\pgiv{Y_u^c}{N_v} = \proba{X_u^c}\cdot\sum_{S \subseteq [k - 1]} \left[ w(S)\cdot \left( \prod_{i \in S}\pgiv{X_u^{c_i}}{\overline{X}_v^{c_i}}\right)\cdot\left(\prod_{j \notin S}\pgiv{\overline{X}_u^{c_j}}{\overline{X}_v^{c_j}}\right) \right].
	\end{align}

	We will complete the proof by showing that the RHS of Equation~\eqref{eq:yuc} is a lower bound for the RHS of Equation~\eqref{eq:yuc-conditional-nv}. We accomplish this in $k - 1$ steps, one for each color besides $c$.
	In the first step, we begin by rearranging the terms of Equation~\eqref{eq:yuc-conditional-nv} to isolate $\pgiv{X_u^{c_1}}{\overline{X}_v^{c_1}}$ and $\pgiv{\overline{X}_u^{c_1}}{\overline{X}_v^{c_1}}$:

	\begin{align*}
		\begin{split}
			\pgiv{Y_u^c}{N_v} = \proba{X_u^c} & \cdot \biggl[ \pgiv{X_u^{c_1}}{\overline{X}_v^{c_1}}\cdot\sum_{S \ni 1} w(S)\cdot \left( \prod_{1 \neq i \in S}\pgiv{X_u^{c_i}}{\overline{X}_v^{c_i}}\right)\cdot\left(\prod_{j \notin S}\pgiv{\overline{X}_u^{c_j}}{\overline{X}_v^{c_j}}\right)                \\
			                                  & + \pgiv{\overline{X}_u^{c_1}}{\overline{X}_v^{c_1}}\cdot \sum_{S \not\ni 1} w(S) \cdot  \left( \prod_{i \in S}\pgiv{X_u^{c_i}}{\overline{X}_v^{c_i}}\right)\cdot\left(\prod_{1 \neq j \notin S}\pgiv{\overline{X}_u^{c_j}}{\overline{X}_v^{c_j}}\right) \biggr].
		\end{split}
	\end{align*}

	In the above, we refer to the first and second sums as the $c_1$-\emph{wanted coefficient} and the $c_1$-\emph{not-wanted coefficient}, respectively.
	Observe that there exists a natural bijection between the summands in these two coefficients. Specifically, we map the summand in the $c_1$-wanted coefficient
	corresponding to set $S$ to the summand in the $c_1$-not-wanted coefficient corresponding to the set $S \setminus \{1\}$. These two summands are identical up to the
	difference between $w(S)$ and $w(S \setminus \{1\})$. By definition, the former weight is smaller. Hence, the $c_1$-wanted coefficient is smaller than
	the $c_1$-not-wanted coefficient.

	Next, we claim that $\pgiv{X_u^{c_1}}{\overline{X}_v^{c_1}} \leq \proba{X_u^{c_1}}$, and (equivalently) $\pgiv{\overline{X}_u^{c_1}}{\overline{X}_v^{c_1}} \geq \proba{\overline{X}_u^{c_1}}$.
	To prove this claim, we consider two cases. If $x_u^{c_1} \leq x_v^{c_1}$, then $\pgiv{X_u^{c_1}}{\overline{X}_v^{c_1}} = 0 \leq x_u^{c_1} = \proba{X_u^{c_1}}$. Otherwise, $x_u^{c_1} > x_v^{c_1}$, and we
	have $\pgiv{X_u^{c_1}}{\overline{X}_v^{c_1}} = \frac{x_u^{c_1} - x_v^{c_1}}{1 - x_v^{c_1}} = x_u^{c_1}\frac{1 - x_v^{c_1}/x_u^{c_1}}{1 - x_v^{c_1}} \leq x_u^{c_1} = \proba{X_u^{c_1}}$.

	We now rewrite our expression for $\pgiv{Y_u^c}{N_v}$ in terms of \emph{unconditional} probabilities of $u$ (not) wanting color $c_1$. In the following, let
	$0 \leq d = \proba{X_u^{c_1}} - \pgiv{X_u^{c_1}}{\overline{X}_v^{c_1}}$.

	\begin{align*}
		\begin{split}
			\pgiv{Y_u^c}{N_v} = \proba{X_u^c} & \cdot \biggl[ (\proba{X_u^{c_1}} - d)\cdot\sum_{S \ni 1} w(S)\cdot \left( \prod_{1 \neq i \in S}\pgiv{X_u^{c_i}}{\overline{X}_v^{c_i}}\right)\cdot\left(\prod_{j \notin S}\pgiv{\overline{X}_u^{c_j}}{\overline{X}_v^{c_j}}\right)                \\
			                                  & + (\proba{\overline{X}_u^{c_1}} + d)\cdot \sum_{S \not\ni 1} w(S) \cdot  \left( \prod_{i \in S}\pgiv{X_u^{c_i}}{\overline{X}_v^{c_i}}\right)\cdot\left(\prod_{1 \neq j \notin S}\pgiv{\overline{X}_u^{c_j}}{\overline{X}_v^{c_j}}\right) \biggr].
		\end{split}
	\end{align*}

	Our observation that the $c_1$-wanted coefficient is smaller than the $c_1$-not-wanted coefficient yields the bound we desire:
	\begin{align*}
		\begin{split}
			\pgiv{Y_u^c}{N_v} \geq \proba{X_u^c} & \cdot \biggl[ \proba{X_u^{c_1}}\cdot\sum_{S \ni 1} w(S)\cdot \left( \prod_{1 \neq i \in S}\pgiv{X_u^{c_i}}{\overline{X}_v^{c_i}}\right)\cdot\left(\prod_{j \notin S}\pgiv{\overline{X}_u^{c_j}}{\overline{X}_v^{c_j}}\right)                \\
			                                     & + \proba{\overline{X}_u^{c_1}}\cdot \sum_{S \not\ni 1} w(S) \cdot  \left( \prod_{i \in S}\pgiv{X_u^{c_i}}{\overline{X}_v^{c_i}}\right)\cdot\left(\prod_{1 \neq j \notin S}\pgiv{\overline{X}_u^{c_j}}{\overline{X}_v^{c_j}}\right) \biggr].
		\end{split}
	\end{align*}

	This completes the first of $k - 1$ steps. In the next step, we begin by rearranging terms in the inequality above to isolate $\pgiv{X_u^{c_2}}{\overline{X}_v^{c_2}}$ and $\pgiv{\overline{X}_u^{c_2}}{\overline{X}_v^{c_2}}$. We then apply the same analysis, yielding another lower bound on $\pgiv{Y_u^c}{N_v}$, this time written
	in terms of unconditional probabilities of $u$ (not) wanting colors $c_1$ and $c_2$, and conditional probabilities of $u$ (not) wanting colors $c_3, c_4, \ldots c_{k-1}$. After $k - 1$ steps, our lower bound becomes identical (up to rearranging terms) to the RHS of Equation~\ref{eq:yuc}, as desired.
\end{proof}

\lemsumtoprod*
\begin{proof}
	Observe that the claim is trivially true when $m = 1$. We proceed via induction on $m$. For $m > 1$, we have $\sum_{t = 1}^{m - 1} \leq \beta - x_m$, so by the inductive hypothesis
	\[ \prod_{t = 1}^m (1 - x_t) = (1 - x_m)\cdot \prod_{t=1}^{m-1}(1 - x_t) \geq (1 - x_m)(1 - \beta + x_m) = 1 - \beta + x_m(\beta - x_m) \geq 1 - \beta.\]
\end{proof}

\lembounding*
\begin{proof}
	Let $\mathcal{S} = \{\vx \in [0,2/3]^m \colon \sum_{i = 1}^m x_i = 1\} \subsetneq \mathcal{D}$ be the set of input vectors for $f$ whose entries sum exactly to 1. Let $\mathcal{I}\subsetneq \mathcal{S}$ be the set of vectors in $\mathcal{S}$ with one entry equal to $2/3$, one entry equal to $1/3$, and all other entries equal to 0. More formally, if $\vx \in \mathcal{I}$, this means there exists two distinct indices $\{i,j\}$ such that $x_i = 2/3$, $x_j = 1/3$, and $x_k = 0$ for every $k \notin \{i,j\}$. Our goal is to prove there exists some $\vx \in \mathcal{I}$ that minimizes $f$. We prove this in two steps: (1) we show that there exists a minimizer of $f$ in $\mathcal{S}$, and then (2) we show how to convert an arbitrary vector $\vx \in \mathcal{S}$ into a new vector $\vx^* \in \mathcal{I}$ satisfying $f(\vx^*) \leq f(\vx)$. Given $\vx^* \in \mathcal{I}$, the value $f(\vx^*)$ in Eq.~\eqref{eq:minval} amounts to a simple function evaluation, realizing that $P(\vx^*,t) = 0$ for $t \geq 3$.

	\paragraph{Step 1: Proving minimizers exist in $\mathcal{S}$.}

	Let $\vx \in \mathcal{D} \setminus \mathcal{S}$, and let $y \in  [m]$ be an entry such that $x_y < 2/3$. Set 
	\begin{equation*}
	\delta = \min \left\{ 1 - \sum_{i = 1}^m x_i, \frac{2}{3}- x_y\right\} > 0
	\end{equation*} and define a new vector $\vx' \in \mathcal{D}$ by
	\begin{equation}
		x_i' = \vx'(i) = \begin{cases}
			x_i          & \text{ if $i \neq y$} \\
			x_i + \delta & \text{otherwise}.
		\end{cases}
	\end{equation}
	By our choice of $\delta$ we know that $\vx' \in \mathcal{D}$, and we will prove this satisfies $f(\vx') \leq f(\vx)$. To do so, we re-write $f(\vx)$ in a way that isolates terms involving $x_y.$ Let $\noty{} = [m] \setminus \{y\}$. Recall that $f$ is a linear combination of terms $P(\vx,t)$, where $P(\vx,t)$ represents the probability that exactly $t$ events out of a set of $m$ independent events occur. Entry $x_i$ of $\vx$ is the probability that $i$th event occurs. We can therefore re-write:
	\begin{align}
		\label{eq:pxtwithy}
		P(\vx,t) & = P(\vx[\noty{}],t) (1-x_y) + P(\vx[\noty{}], t-1) x_y.
	\end{align}
	To explain this in more detail, we use a slight abuse of terminology and refer to $[m]$ as a set of events. The first term in Eq.~\eqref{eq:pxtwithy} is the probability that exactly $t$ events in $\noty{}$ occur and that event $y$ does not occur. The second term is the probability that $y$ does occur and exactly $t -1$ events in $\noty{}$ occur. We therefore re-write $f(\vx)$ as
	\begin{align}
		f(\vx) & = \sum_{t=0}^m a_t P(\vx, t)                                                                       & (\text{definition of $f$}) \notag            \\
		       & = \sum_{t=0}^m a_t \left(P(\vx[\noty{}],t) (1-x_y) + P(\vx[\noty{}], t-1) x_y \right)              & (\text{from Eq.~\eqref{eq:pxtwithy}}) \notag \\
		       & = \sum_{t=0}^m a_t P(\vx[\noty{}],t) (1-x_y) + \sum_{t=0}^m a_t P(\vx[\noty{}], t-1) x_y           & (\text{separating terms}) \notag             \\
		       & = \sum_{t=0}^m a_t P(\vx[\noty{}],t) (1-x_y) + \sum_{t=-1}^{m-1} a_{t+1} P(\vx[\noty{}], t) x_y    & (\text{change of variables}) \notag          \\
		       & = \sum_{t=0}^{m-1} a_t P(\vx[\noty{}],t) (1-x_y) + \sum_{t=0}^{m-1} a_{t+1} P(\vx[\noty{}], t) x_y & \label{eq:fvx_p_0}                           \\
		       & = \sum_{t=0}^{m-1} P(\vx[\noty{}],t) \left( a_t(1-x_y) + a_{t+1}x_y \right)                        & \label{eq:last}
	\end{align}
	In Eq.~\eqref{eq:fvx_p_0} we have used the fact that $P(\vx[\noty{}],\ell) = 0$ if $\ell < 0$ or $\ell > m-1$.
	Observe that if we replace the value of $x_y$ with $x_y + \delta$ in Eq.~\eqref{eq:last} this gives the value of $f(\vx')$. We can then see that
	\begin{align*}
		f(\vx') & = \sum_{t=0}^{m-1} P(\vx[\noty{}],t) \left( a_t(1-x_y - \delta) + a_{t+1}(x_y + \delta) \right)       \\
		        & = \sum_{t=0}^{m-1} P(\vx[\noty{}],t) \left( a_t(1-x_y) + a_{t+1}x_y + \delta (a_{t+1} - a_t)  \right) \\
		        & = f(\vx) + \delta \sum_{t=0}^{m-1} (a_{t+1} - a_t).
	\end{align*}
	Using the constraint $a_{t+1} \leq a_t$ we notice that the last term is always less that or equal to zero. Hence, we have $f(\vx') \leq f(\vx)$. Recall that $\delta = \min\{ 1 - \sum_{i =1}^m x_i, 2/3 - x_y\}$. If $\delta = 1 - \sum_{i=1}^m x_i$, then we can see that $\vx' \in \mathcal{S}$ and we are done. Otherwise, $\delta = 2/3- x_y$, and $x_y' = 2/3$. We can then apply the same exact procedure again, by increasing the value of some index $x_z$ (where $z \neq y$), which we know satisfies $x_z \leq 1/3$. The second application of this procedure is guaranteed to produce a vector in $\mathcal{S}$.


	\paragraph{Step 2: Editing a vector from $\mathcal{S}$ to $\mathcal{I}$.}
	The proof structure for Step 2 is very similar as Step 1, but is more involved as it requires working with two entries of $\vx$ rather than one. Let $\vx \in \mathcal{S} \setminus \mathcal{I}$, which means we can identify two entries $x_y$ and $x_z$ in the vector (with $y \neq z$) satisfying the inequality
	\begin{equation}
		\label{eq:yzinequality}
		0 < x_y \leq x_z < 2/3.
	\end{equation}
	Given these entries, set $\delta = \min\{x_y, 2/3 - x_z \}$ and define a new vector $\vx'$ by
	\begin{equation}
		x_i' = \vx'(i) = \begin{cases}
			x_i          & \text{ if $i \notin \{y,z\}$} \\
			x_i - \delta & \text{ if $i = y$}            \\
			x_i + \delta & \text{ if $i = z$.}
		\end{cases}
	\end{equation}
	Observe that $\vx' \in \mathcal{S}$ and that either $x_y' = 0$ or $x_z' = 2/3$. We will show that $f(\vx') \leq f(\vx)$, by re-writing $f(\vx)$ in a way that isolates terms involving $x_y$ and $x_z$. Let $\notyz{} = [m] \backslash \{y,z\}$. Similar to the proof of Step 1, we can re-write $P(\vx,t)$ as
	% $f$ by definition is a linear combination of terms $P(\vx,t)$, and that 
	% $P(\vx,t)$ represents the probability that exactly $t$ events out of a set of $m$ independent events occur, we can write
	% . Entry $x_i$ of $\vx$ is the probability that the $i$th event occurs. We can therefore re-write:
	\begin{align}
		P(\vx,t) & = P(\vx[\notyz{}],t) (1-x_y)(1-x_z) \label{eq:pxt1}                               \\
		         & \;\;+ P(\vx[\notyz{}], t-1) \left[x_y (1-x_z) + (1-x_y)x_z\right] \label{eq:pxt2} \\
		         & \;\; + P(\vx[\notyz{}], t-2) x_y x_z. \label{eq:pxt3}
	\end{align}
	Line~\eqref{eq:pxt1} captures the probability that exactly $t$ events from the set $\notyz{}$ happen, and that neither of the events $y$ or $z$ happen. Line~\eqref{eq:pxt2} is the probability that $t-1$ events from $\notyz{}$ happen, and exactly one of the events $\{y,z\}$ happens. Finally, line~\eqref{eq:pxt3} is the probability that both $y$ and $z$ happen, and $t-2$ of the events in $\notyz{}$ happen. For simplicity we now define
	\begin{align*}
		\alpha & = (1-x_y)(1-x_z)           \\
		\beta  & = x_y (1-x_z) + (1-x_y)x_z \\
		\gamma & = x_y x_z,
	\end{align*}
	so that we can re-write $f(\vx)$ as follows:
	\begin{align}
		f(\vx) & = \sum_{t=0}^m a_t P(\vx, t)                                                                                                                                & (\text{def. of $f$}) \notag      \\
		       & = \sum_{t=0}^m a_t \left(P(\vx[\notyz{}],t)\alpha + P(\vx[\notyz{}], t-1) \beta + P(\vx[\notyz{}], t-2) \gamma  \right)                                     & (\text{exp. $P(\vx,t)$}) \notag  \\
		       & = \sum_{t=0}^m a_t P(\vx[\notyz{}],t)\alpha + \sum_{t=0}^m a_t P(\vx[\notyz{}], t-1) \beta +  \sum_{t=0}^m a_t P(\vx[\notyz{}], t-2) \gamma                 & (\text{sep. sums}) \notag        \\
		       & = \sum_{i=0}^m a_i P(\vx[\notyz{}],i)\alpha + \sum_{j=-1}^{m-1} a_{j+1} P(\vx[\notyz{}], j) \beta +  \sum_{k=-2}^{m-2} a_{k+2} P(\vx[\notyz{}], k) \gamma   & (\text{redef. variables}) \notag \\
		       & = \sum_{i=0}^m a_i P(\vx[\notyz{}],i)\alpha + \sum_{j=0}^{m-1} a_{j+1} P(\vx[\notyz{}], j) \beta +  \sum_{k=0}^{m-2} a_{k+2} P(\vx[\notyz{}], k) \gamma     & \label{eq:littlel}               \\
		       & = \sum_{i=0}^{m-2} a_i P(\vx[\notyz{}],i)\alpha + \sum_{j=0}^{m-2} a_{j+1} P(\vx[\notyz{}], j) \beta +  \sum_{k=0}^{m-2} a_{k+2} P(\vx[\notyz{}], k) \gamma & \label{eq:bigl}                  \\
		       & = \sum_{t=0}^{m-2} P(\vx[\notyz{}],t) \left( a_t \alpha +  a_{t+1}  \beta +   a_{t+2}  \gamma \right).                                                       & \label{eq:rewritef}
	\end{align}
	In Eq.~\eqref{eq:littlel} and Eq.~\eqref{eq:bigl} we have used the fact that $P(\vx[\notyz{}],\ell) = 0$ if $\ell < 0$ or $\ell > m-2$. Consider now the vector $\vx'$ obtained by perturbing entries $x_y$ and $x_z$, and define
	\begin{equation*}
		\begin{array}{llll}
			\alpha' & = (1-x_y')(1-x_z')                                               \\
			        & = (1-(x_y-\delta))(1-(x_z +\delta))                              \\
			        & = \alpha + \delta(x_y-x_z) - \delta^2                            \\                                                                                                           \\
			\beta'  & = x_y' (1-x_z') + (1-x_y')x_z'                                   \\
			        & = (x_y-\delta) (1-(x_z +\delta)) + (1-(x_y-\delta))(x_z +\delta) \\
			        & = \beta + 2\delta (x_z - x_y) +2  \delta^2                       \\                                                                     \\
			\gamma' & = x_y' x_z'                                                      \\
			        & = (x_y-\delta) (x_z +\delta)                                     \\
			        & = \gamma + \delta (x_y - x_z) - \delta^2.
		\end{array}
	\end{equation*}
	In the expression for $f(\vx)$ given in Eq.~\eqref{eq:rewritef}, the entries $x_y$ and $x_z$ appear only in the terms $\alpha$, $\beta$, and $\gamma$. Therefore, if we replace $\{\alpha, \beta, \gamma\}$ with $\{\alpha', \beta', \gamma'\}$, we obtain a similar expression for $f(\vx')$. We can then see that

	\begin{align*}
		f(\vx') - f(\vx) & = \sum_{t=0}^{m-2} P(\vx[\notyz{}],t) \left( a_t (\alpha'-\alpha) +  a_{t+1}  (\beta'-\beta) +   a_{t+2}  (\gamma'-\gamma) \right).
	\end{align*}

	Isolating the expression inside the summation we can see

	\begin{align*}
		   & a_t (\alpha'-\alpha) +  a_{t+1}  (\beta'-\beta) +   a_{t+2}  (\gamma'-\gamma)                                             \\
		=~ & a_t (\delta(x_y-x_z) - \delta^2) + a_{t+1}  (2\delta (x_z - x_y) +2  \delta^2) + a_{t+2}  (\delta (x_y - x_z) - \delta^2) \\
		=~ & a_t (\delta(x_y-x_z) - \delta^2) -  2a_{t+1}  (\delta (x_y - x_z) -  \delta^2) + a_{t+2}  (\delta (x_y - x_z) - \delta^2) \\
		=~ & (a_t - 2a_{t+1} + a_{t+2}) (\delta(x_y - x_z) - \delta^2).
	\end{align*}

	Plugging this back into the equation for $f(\vx') - f(\vx)$ and using the fact that $a_t - 2a_{t+1} + a_{t+2} \geq 0$ and $(x_y - x_z) - \delta < 0$, we have

	\[
		f(\vx') - f(\vx) = \sum_{t=0}^{m-2} P(\vx[\notyz{}],t) \left( (a_t - 2a_{t+1} + a_{t+2}) (\delta(x_y - x_z) - \delta^2)\right)\leq 0.
	\]
	% At this point we have moved entries $x_y$ and $x_z$ closer to the interval bounds $\{0,2/3\}$ while decreasing (or holding constant) the output of function $f$. We now wish choose the largest value of $\delta > 0$ (i.e., the maximum perturbation for $x_y$ and $x_z$) that still guarantees $\vx' \in [0,2/3]$. This is equivalent to finding the maximum value of $\delta$ such that
	% \begin{align*}
	% 0  \leq x_y - \delta &\implies \delta \leq x_y \\
	% x_z + \delta \leq 2/3 &\implies \delta \leq 2/3 - x_z,
	% \end{align*}
	% so we choose $\delta = \min\{x_y, 2/3 - x_z \}$.

	If we start with an arbitrary vector $\vx \in \mathcal{S}$, applying one iteration of the above procedure will ensure that \emph{at least} one entry of $\vx$ will move to one of the endpoints $\{0,2/3\}$ of the interval $[0,2/3]$. Therefore, applying this procedure $m$ (or fewer) times to an arbitrary vector $\vx \in \mathcal{S}$ will produce a vector $\vx^* \in \mathcal{I}$ that satisfies $f(\vx^*) \leq f(\vx)$.
\end{proof}

\lemboundingconstraints*
\begin{proof}
	% Observe the inequalities hold for $a_t = \frac{1}{1+g+t}$. And for the second sequence the first inequality holds. 
	The first inequality (monotonicity) is easy to check for both sequences. The most involved step is checking that the second inequality holds for the sequence in~\eqref{eq:complicatedat}. To show it holds, we note that the inequality $a_t +  a_{t+2} \geq 2a_{t+1}$ effectively corresponds to a discrete version of convexity. We first extend the definition of the sequence in~\eqref{eq:complicatedat} to all positive reals by defining the function
	\begin{equation*}
		h(x) = \frac{2}{9}\left( \frac{1}{x+1} + \frac{1}{x+3} \right) + \frac{5}{9} \left( \frac{1}{x+2} \right),
	\end{equation*}
	which we can easily prove is convex over the interval $[0,\infty)$ by noting that its second derivative is
	\[
		h''(x) = \frac{1}{9}\left( \frac{10}{(x+2)^3} + \frac{4}{(x+3)^3} + \frac{4}{(x+1)^3} \right),
	\]
	which is greater than $0$ for all $x \geq 0$. From the definition of convexity for any $\beta \in [0, 1]$ and any $x, y \geq 0$ we have
	\[
		h(x\beta + (1-\beta) y) \leq \beta h(x) + (1-\beta) h(y).
	\]
	Now if we consider the specific values $x=t$, $y=t+2$, and $\beta=1/2$ we get
	\begin{align*}
		h\left(t\frac{1}{2} + \frac{1}{2} (t+2)\right) & \leq \frac{1}{2} h(t) + \frac{1}{2} h(t+2) \\
		\implies  h(t+1)                               & \leq \frac{1}{2} h(t) + \frac{1}{2} h(t+2) \\
		\implies  2 h(t+1)                             & \leq h(t) + h(t+2)                         \\
		\implies 2a_{t+1}                              & \leq a_t + a_{t+2}.
	\end{align*}
	% where $\beta = 1/2$, $x = t$, and $z = t+2$ to recover the desired inequalities. 
	One can use a similar approach to show (even more easily) that the sequence $a_t = 1/(1+g+t)$ also satisfies the second inequality.
\end{proof}

% \section{Omitted proofs from~\texorpdfstring{\Cref{sec:generalized-mean}}{}}
\label{app:gm}

\pmeanapproximation*
\begin{proof}

    We begin with the case where $p \geq 1$.
    It is well known that the $\ell_p$-norm is convex, and we observe that all of the constraints in the
    in Program~(\ref{eq:p-mean-ecc}) are linear. Thus, we may in polynomial time obtain optimal fractional variables $\{d_v^c\}, \{\gamma_e\}, \{m_c\}$
    for the relaxation of Program~(\ref{eq:p-mean-ecc}).

    We observe that for each vertex $v$, there is at most one color $c$ with the property that $d_v^c < \frac{1}{2}$.
    Otherwise, there are two colors $c_1, c_2$ with $d_v^{c_1}, d_v^{c_2} < \frac{1}{2}$, but then
    \[
        \sum_{c=1}^k d_v^c < 2\cdot\frac{1}{2} + (k - 2) = k - 1,
    \]
    contradicting the first constraint of Program~(\ref{eq:p-mean-ecc}).

    We now propose a coloring $\lambda$ as follows. For each color $c$ and vertex $v$, we assign $\lambda(v) = c$ if $d_v^c < \frac{1}{2}$.
    By the preceding observation, this procedure assigns at most one color to each vertex.
    If for any $v$ we have that $d_v^c \geq \frac{1}{2}$ for all colors $c$, then we set $\lambda(v)$ arbitrarily.

    Now, for each hyperedge $e$ let $\hat{\gamma}_e$ be equal to $0$ if $\lambda(v) = \ell(e)$ for all $v \in e$, or $1$ otherwise.
    We claim that if $\hat{\gamma}_e = 1$, then $\gamma_e \geq \frac{1}{2}$.
    Otherwise, letting $c = \ell(e)$, the second constraint of Program~(\ref{eq:p-mean-ecc}) implies that $d_v^c < \frac{1}{2}$ for all $v \in e$.
    Then $\lambda(v) = c$ for all $v \in e$, implying that $\hat{\gamma}_e = 0$, a contradiction.
    We conclude that for every hyperedge $e$, $\hat{\gamma}_e \leq 2\gamma_e$.

    For each color $c$, we write $\hat{m}_c$ for the number of hyperedges of color $c$ which are unsatisfied by $\lambda$.
    Equivalently,
    \[
        \hat{m}_c = \sum_{e \in E_c} \hat{\gamma}_e \leq 2\sum_{e \in E_c} \gamma_e = 2m_c,
    \]
    where the last equality comes from the third constraint of Program~\ref{eq:p-mean-ecc}.

    It follows that for every $c$, $\hat{m}_c^p \leq 2^pm_c^p$, and thus
    \[
        \sum_{c=1}^k \hat{m}_c^p \leq 2^p \cdot \sum_{c=1}^k m_c^p,
    \]

    which in turn implies that

    \[
        \left(\sum_{c=1}^k \hat{m}_c^p\right)^{1/p} \leq 2 \cdot \left(\sum_{c=1}^k m_c^p\right)^{1/p},
    \]

    so $\lambda$ is $2$-approximate.

    We now consider the case where $0 < p < 1$.
    We begin by defining, for each $c \in [k]$, a non-negative, monotone, and submodular function $f_c\colon 2^E \rightarrow \mathbb{Z}$ given by $f_c(S) = |E_c \cap S|$.
    If $\lambda$ is a vertex coloring of $H$ and $S_\lambda \subseteq E$ is the set of edges unsatisfied by $\lambda$, then $f_c(S_\lambda)$ measures the number of edges of color $c$ unsatisfied by $\lambda$.
    Because the composition of a concave function with a submodular function results in a submodular function and $0 < p < 1$, $f_c^p$ is non-negative, monotone, and submodular.
    Then the function $f\colon 2^E \rightarrow \mathbb{R}$ given by the sum
    \[
        f(S) = \sum_{c=1}^{k} f_c^p(S)
    \]
    has these same properties.

    Now, let $\lambda^*$ be an optimal vertex coloring of $H$ and let $S_{\lambda^*}$ be the set of hyperedges unsatisfied by $\lambda^*$.
    Our goal will be to compute a coloring $\lambda$ with $f(S_\lambda) \leq 2f(S_{\lambda^*})$.
    This would complete the proof, since then we have
    \[
        (f(S_\lambda))^{1/p} \leq (2f(S_{\lambda^*}))^{1/p} = 2^{1/p}f^{1/p}(S_{\lambda^*}),
    \]
    and $f^{1/p}(S_{\lambda}), f^{1/p}(S_{\lambda^*})$ are precisely the objective values attained by $\lambda$ and $\lambda^*$, respectively, according to Program~(\ref{eq:p-mean-ecc}).

    We give a classic analysis based on the Lov{\'a}sz extension~\cite{lovasz1983submodular}.
    %     but we note that in practice more recent, faster methods are preferable.
    %    In particular, since each $f_c$ measures the cardinality of a set, very fast optimization techniques based on reductions to sparse graph cut problems are available~\cite{veldt2021approximate}.
    Some readers may find it helpful to observe that the following is conceptually equivalent to the textbook $2$-approximation for \textsc{Submodular Vertex Cover}\footnote{Given a graph $G$ and a submodular function $f$, find a vertex cover $C$ minimizing $f(C)$.}, where the graph in question is the \emph{conflict graph} $G$ of our edge-colored hypergraph $H$, as defined in~\Cref{sec:color-fair}.
    The connection between vertex colorings of edge-colored hypergraphs and vertex covers of their associated conflict graphs has been observed several times in the literature~\cite{angel2016clustering,cai2018alternating,kellerhals2023parameterized,veldt2023optimal}, and also appears elsewhere in the present work, e.g., in the proofs of~\Cref{thm:cfminecc-reduce-sparse-vc,thm-combinatorial-k-approx,thm:pc-multiple-classes-NP-hard}.

    Impose an arbitrary order on the edges of $E$ and for each vector $\boldsymbol{\gamma} \in [0, 1]^{|E|}$ map edges to components of $\boldsymbol{\gamma}$ according to this order.
    Henceforth, for a hyperedge $e$ we write $\gamma_e$ for the component of $\boldsymbol{\gamma}$ corresponding to $e$. Moreover, for a
    value $\rho$ selected uniformly at random from $[0, 1]$, we write $T_\rho(\boldsymbol{\gamma}) = \{e \colon \gamma_e \geq \rho\}$.
    Then the Lov{\'a}sz extension $\hat{f}\colon [0, 1]^{|E|} \rightarrow \mathbb{R}$ of $f$ is given by
    \[
        \hat{f}(\boldsymbol{\gamma}) = \mathbb{E}[f(T_\rho(\boldsymbol{\gamma}))].
    \]
    This $\hat{f}$ is convex~\cite{lovasz1983submodular}, so we may efficiently optimize the following program:
    %
    \begin{align}
        \label{eq:lovasz}
        \begin{aligned}
            \text{min} \quad  & \hat{f}(\gamma)                 \\
            \text{s.t.} \quad & \gamma_e + \gamma_f \geq 1\quad \\
                              & \gamma_e \geq 0                 \\
        \end{aligned}
        \begin{aligned}
             &                                                                                                         \\
             & \text{for all } e, f \in E \text{ such that } e \cap f \neq \emptyset \text{ and } \ell(e) \neq \ell(f) \\
             & \text{for all } e \in E.
        \end{aligned}
    \end{align}
    Consider the binary vector given by $\gamma_e = 1$ if $e \in S_{\lambda_*}$ or $0$ otherwise; note that this vector satisfies the constraints of Program~(\ref{eq:lovasz}). Hence, we can see that the minimum value for Program~(\ref{eq:lovasz}) provides a lower bound on $f(S_{\lambda_*})$.

    Let $\boldsymbol{\gamma}$ be a minimizer of Program~(\ref{eq:lovasz}), and let
    \[
        S = \left\{e \in E \colon \gamma_e \geq \frac{1}{2}\right\}.
    \]
    The first constraint of Program~\ref{eq:lovasz} ensures that $S$ contains at least one member of every pair of overlapping and distinctly colored hyperedges.
    Thus, it is trivial to compute a coloring $\lambda$ with $S_\lambda \subseteq S$.
    It follows from monotonicity that $f(S_\lambda) \leq f(S)$, and that for each $\rho \leq \frac{1}{2}$, $f(S) \leq f(T_\rho(\boldsymbol{\gamma}))$.
    Putting it all together, we have

    \[
        \frac{f(S_\lambda)}{2} \leq \frac{f(S)}{2} = \int_{0}^{\frac{1}{2}} f(S) \ \mathrm{d}\rho \leq \int_{0}^{\frac{1}{2}} f(T_\rho(\boldsymbol{\gamma})) \ \mathrm{d}\rho \leq \int_{0}^{1} f(T_\rho(\boldsymbol{\gamma})) \ \mathrm{d}\rho = \hat{f}(\boldsymbol{\gamma}) \leq f(S_{\lambda^*}),
    \]
    which completes the proof.

\end{proof}

\section{Full proof of~\texorpdfstring{\Cref{thm:cfminecc-NPhard}}{}}
\label{appendix:color-fair}
\label{app:cf}
% NP-hardness
\cfmineccNPhard*
{
\usetikzlibrary{math}

\definecolor{c1}{RGB}{0, 114, 178}
\definecolor{c2}{RGB}{213, 94, 0}
\definecolor{c3}{RGB}{0, 158, 115}
\definecolor{c3prime}{RGB}{240, 228, 66}
\definecolor{b1}{RGB}{230, 159, 0}
\definecolor{b2}{RGB}{86, 180, 233}
\definecolor{b3}{RGB}{204, 121, 167}

% \colorlet{c1}{cyan}
% \colorlet{c2}{red}
% \colorlet{c3}{green}
% \colorlet{c3prime}{violet}
% \colorlet{b1}{brown}
% \colorlet{b2}{olive}
% \colorlet{b3}{gray}

\begin{figure}[t]
    \centering
    \begin{tikzpicture}

        \tikzmath{
            \leftx = 0;
            \dy = 1;
            \confy = 0;
            \freey = \confy - \dy;
            \bridgey = \confy - 2*\dy;
            \sparey = \confy - 3*\dy;
            \dx = .9;
            \edgewidth = 1.25;
            \keyx = \leftx + 15*\dx;
            \keyy = \confy + 1.5;
            \keydx = 1.5;
            \keydy = -.5;
            \keytitlex = \keyx-.2;}

        \node[circle, label=right:$(x_1 \lor x_2 \lor x_3) \land (x_1 \lor \neg x_2 \lor \neg x_3) \land (\neg x_1 \lor \neg x_2)$] at (\leftx, \keyy) (formula) {};

        \node[circle, draw, fill=black, label=below:$v_1^1$] at (\leftx, \freey) (free11) {};
        \node[circle, draw, fill=black, label=above:$v_{1,3}^1$] at (\leftx, \confy) (conf131) {};
        \node[circle, draw, fill=black, label=above:$v_{2,3}^1$] at (\leftx + \dx, \confy) (conf231) {};
        \node[circle, draw, fill=black, label=below:$v_2^1$] at (\leftx + \dx, \freey) (free21) {};

        \draw[c1, line width=\edgewidth pt] (free11) -- (conf131);
        \draw[c3, line width=\edgewidth pt] (conf131) -- (conf231);
        \draw[c2, line width=\edgewidth pt] (conf231) -- (free21);

        \node[circle, draw, fill=black, label=below:$b_{1,2}^1$] at (\leftx + 2*\dx, \bridgey) (bridge121) {};
        \node[circle, draw, fill=black, label=below:$b_{1,2}^2$] at (\leftx + 3*\dx, \bridgey) (bridge122) {};

        \node[circle, draw, fill=black, label=below:$v_2^2$] at (\leftx + 4*\dx, \freey) (free22) {};
        \node[circle, draw, fill=black, label=above:$v_{1,2}^2$] at (\leftx + 4*\dx, \confy) (conf122) {};
        \node[circle, draw, fill=black, label=above:$v_{1,3}^2$] at (\leftx + 5*\dx, \confy) (conf132) {};
        \node[circle, draw, fill=black, label=below:$v_3^2$] at (\leftx + 5*\dx, \freey) (free32) {};

        \draw[c2, line width=\edgewidth pt] (free22) -- (conf122);
        \draw[c1, line width=\edgewidth pt] (conf122) -- (conf132);
        \draw[c3, line width=\edgewidth pt] (conf132) -- (free32);

        \node[circle, draw, fill=black, label=below:$b_{2,3}^1$] at (\leftx + 6*\dx, \bridgey) (bridge231) {};
        \node[circle, draw, fill=black, label=below:$b_{2,3}^2$] at (\leftx + 7*\dx, \bridgey) (bridge232) {};

        \node[circle, draw, fill=black, label=below:$v_1^3$] at (\leftx + 8*\dx, \freey) (free13) {};
        \node[circle, draw, fill=black, label=above:$v_{1,2}^3$] at (\leftx + 9*\dx, \confy) (conf123) {};
        \node[circle, draw, fill=black, label=below:$v_2^3$] at (\leftx + 10*\dx, \freey) (free23) {};

        \draw[c1, line width=\edgewidth pt] (free13) -- (conf123);
        \draw[c2, line width=\edgewidth pt] (conf123) -- (free23);

        \node[circle, draw, fill=black, label=below:$b_{3,4}^1$] at (\leftx + 11*\dx, \bridgey) (bridge341) {};
        \node[circle, draw, fill=black, label=below:$b_{3,4}^2$] at (\leftx + 12*\dx, \bridgey) (bridge342) {};

        \node[circle, draw, fill=black, label=below:$v_3^e$] at (\leftx + 13*\dx, \sparey) (spare3e) {};
        \node[circle, draw, fill=black, label=below:$v_3^b$] at (\leftx + 14*\dx, \sparey) (spare3b) {};
        \node[circle, draw, fill=black, label=below:$v_3^a$] at (\leftx + 15*\dx, \sparey) (spare3a) {};
        \node[circle, draw, fill=black, label=below:$v_3^c$] at (\leftx + 16*\dx, \sparey) (spare3c) {};
        \node[circle, draw, fill=black, label=below:$v_3^d$]     at (\leftx + 17*\dx, \sparey) (spare3d) {};

        \draw[c3prime, line width=\edgewidth pt] (spare3e) -- (spare3b);
        \draw[c3, line width=\edgewidth pt] (spare3b) -- (spare3a);
        \draw[c3prime, line width=\edgewidth pt] (spare3a) -- (spare3c);
        \draw[c3prime, line width=\edgewidth pt] (spare3a) to[bend left=50] (spare3d);
    
        
        \draw[b1, line width=\edgewidth pt] (free21) -- (bridge121);
        \draw[b1, line width=\edgewidth pt] (bridge121) -- (bridge122);
        \draw[b1, line width=\edgewidth pt] (bridge122) -- (free22);

        \draw[b2, line width=\edgewidth pt] (free32) -- (bridge231);
        \draw[b2, line width=\edgewidth pt] (bridge231) -- (bridge232);
        \draw[b2, line width=\edgewidth pt] (bridge232) -- (free13);

        \draw[b3, line width=\edgewidth pt] (free23) -- (bridge341);
        \draw[b3, line width=\edgewidth pt] (bridge341) -- (bridge342);
        \draw[b3, line width=\edgewidth pt] (bridge342) -- (spare3e);


        \node[circle, label=right:{Color Key:}] at (\keytitlex, \keyy) () {};

        \node[circle] at (\keyx, \keyy + \keydy) (c1left) {};
        \node[circle, label=right:$c_1$] at (\keyx + \keydx, \keyy + \keydy) (c1right) {};
        \draw[c1, line width=\edgewidth pt] (c1left) -- (c1right);

        \node[circle] at (\keyx, \keyy + 2*\keydy) (c1left) {};
        \node[circle, label=right:$c_2$] at (\keyx + \keydx, \keyy + 2*\keydy) (c1right) {};
        \draw[c2, line width=\edgewidth pt] (c1left) -- (c1right);

        \node[circle] at (\keyx, \keyy + 3*\keydy) (c1left) {};
        \node[circle, label=right:$c_3$] at (\keyx + \keydx, \keyy + 3*\keydy) (c1right) {};
        \draw[c3, line width=\edgewidth pt] (c1left) -- (c1right);

        \node[circle] at (\keyx, \keyy + 4*\keydy) (c1left) {};
        \node[circle, label=right:$c_3'$] at (\keyx + \keydx, \keyy + 4*\keydy) (c1right) {};
        \draw[c3prime, line width=\edgewidth pt] (c1left) -- (c1right);

        \node[circle] at (\keyx, \keyy + 5*\keydy) (c1left) {};
        \node[circle, label=right:$c_{1,2}^b$] at (\keyx + \keydx, \keyy + 5*\keydy) (c1right) {};
        \draw[b1, line width=\edgewidth pt] (c1left) -- (c1right);

        \node[circle] at (\keyx, \keyy + 6*\keydy) (c1left) {};
        \node[circle, label=right:$c_{2,3}^b$] at (\keyx + \keydx, \keyy + 6*\keydy) (c1right) {};
        \draw[b2, line width=\edgewidth pt] (c1left) -- (c1right);

        \node[circle] at (\keyx, \keyy + 7*\keydy) (c1left) {};
        \node[circle, label=right:$c_{3,4}^b$] at (\keyx + \keydx, \keyy + 7*\keydy) (c1right) {};
        \draw[b3, line width=\edgewidth pt] (c1left) -- (c1right);
    \end{tikzpicture}
    \caption{\label{fig:hardness} The construction given by~\Cref{thm:cfminecc-NPhard} for the CNF formula on three clauses $C_1 = (x_1 \lor x_2 \lor x_3)$, $C_2 = (x_1 \lor \neg x_2 \lor \neg x_3)$, and $C_3 = (\neg x_1 \lor \neg x_2)$. 
    Notice that the vertices are partitioned visually into four horizontal layers. The top layer contains \emph{conflict} vertices, the second from top contains \emph{free} vertices, the second from bottom \emph{bridge} vertices, and the bottom \emph{spare} vertices.
    Every edge containing a conflict vertex is a \emph{conflict} edge, every edge containing a bridge vertex is a \emph{bridge} edge, and all other edges are \emph{spare} edges.
    The colors $c_1, c_2$ and $c_3$ correspond to the clauses $C_1, C_2$, and $C_3$. The color $c_3'$ is the \emph{spare} color associated with $C_3$. The remaining colors are \emph{bridge} colors. We refer to the proof of~\Cref{thm:cfminecc-NPhard} for a formal description of the construction and the accompanying analysis.
    }

\end{figure}
}
\begin{proof}
    We reduce from \boolSAT{}, for which the input is a formula written in conjunctive normal form, consisting of $m$ clauses $C_1, C_2, \ldots C_m$ over $n$ variables $x_1, x_2, \ldots x_n$.
    For each variable $x_i$, we write $x_i$ and $\neg x_i$ for the corresponding positive and negative literals.
    We make the standard assumptions that no clause contains both literals of any variable, and that every literal appears at least once.
    We also assume that each clause has size $2$ or $3$, and that each variable appears in at most three clauses (implying that each literal appears in at most two clauses); hardness is retained under these assumptions~\cite{tovey1984simplified}.
    Given an instance of this problem, we construct an instance $(G = (V, E), \tau = 2)$ of \cfminECC{}. See~\Cref{fig:hardness} for a visual aid.

    For each clause $C_j$, we create a unique color $c_j$. Also, if $C_j$ has size two, we create a second unique color $c_j'$, and five vertices $v_j^a, v_j^b, v_j^c, v_j^d,$ and $v_j^e$.
    We call these the \emph{spare vertices} associated with $C_j$.
    The reason for creating these vertices and the second color $c_j'$ will become apparent later.
    Next, for each variable $x_i$, we create a vertex $v_{j_1, j_2}^i$ for each (ordered) pair of clauses $C_{j_1}, C_{j_2}$ with $C_{j_1}$ containing the positive literal $x_i$ and $C_{j_2}$ containing the negative literal $\neg x_i$.
    We call $v_{j_1, j_2}^i$ a \emph{conflict vertex} for the variable $x_i$ and the variable-clause pairs $(x_i, C_{j_1})$ and $(x_i, C_{j_2})$. Observe that each variable has either one or two associated conflict vertices, as does each variable-clause pair.
    For each variable-clause pair $(x_i, C_j)$ with a single associated conflict vertex, we create an additional vertex $v_j^i$ and call this the \emph{free vertex} for the variable-clause pair $(x_i, C_j)$.

    Now we create edges.
    For each clause $C_j$, we begin by creating one edge $e_j^i$ of color $c_j$ for each variable $x_i$ contained in $C_j$. This edge contains either the two $(x_i, C_j)$ conflict vertices, or the single $(x_i, C_j)$ conflict vertex and the $(x_i, C_j)$ free vertex.
    We call the edge $e_j^i$ a \emph{conflict edge} associated both with clause $C_j$, and with the literal of $x_i$ which appears in $C_j$.
    Observe that every free vertex has degree one, and every conflict vertex has degree two.
    For clauses $C_j$ of size $3$, there are exactly three edges of color $c_j$.
    For clauses $C_j$ of size $2$, we have thus far created two edges of color $c_j$.
    For each such clause, we use the associated spare vertices to create a third edge $\{v_j^a, v_j^b\}$ of color $c_j$, which we call the \emph{spare edge} associated with $C_j$, and three edges $\{v_j^a, v_j^c\}$, $\{v_j^a, v_j^d\}$, and $\{v_j^b, v_j^e\}$, each of color $c_j'$.
    Observe that our constructed graph still has maximum degree three, and that there are now exactly $3$ edges of every color.
    Finally, we set $\tau = 2$.

    Since free vertices have degree one, they do not participate in cycles.
    It is also clear from the construction that spare vertices do not participate in cycles.
    Thus, any cycle contains only conflict vertices.
    Note that if two conflict vertices share an edge, then they are associated with the same variable-clause pair.
    It follows that every vertex in a cycle is associated with some single variable-clause pair.
    However, a variable-clause pair has at most two associated conflict vertices, so we have constructed a forest.

    Now we will add some gadgets to turn our forest into a tree. Suppose that the graph we have constructed so far has $q$ connected components.
    Impose an arbitrary order on these components, and label them $G_1, G_2, \ldots, G_q$.
    Observe that every connected component contains either only spare vertices or two free vertices.
    In both cases, it is possible to add two edges each with one (distinct) endpoint in $G_i$ without raising the maximum degree above three.
    That the endpoints are distinct will be important when we analyze the cutwidth.
    For each consecutive pair $G_i, G_{i+1}$ of connected components, we add two \emph{bridge vertices} $b_{i, i+1}^1, b_{i, i+1}^2$ and a \emph{bridge color} $c_{i, i+1}^b$.
    We add \emph{bridge edges} (chosen so as not to violate our maximum degree constraint) from $G_i$ to $b_{i,i+1}^1$, from $b_{i, i+1}^1$ to $b_{i, i+1}^2$, and from $b_{i, i+1}^2$ to $G_{i+1}$.
    We color each of these three edges with $c_{i, i+1}^b$. Observe that the graph is now connected, but it is still acyclic since there is exactly
    one path between any pair of vertices which were in different connected components before the addition of our bridge gadgets.

    We now claim that the constructed graph $G$ has cutwidth~$2$.
    To this end, we will construct a ordering $\sigma\colon V \rightarrow \mathbb{N}$ of the vertices of $G$, where $\sigma$ is injective, $\sigma(v) = 1$ indicates that $v$ is the first vertex in the ordering, $\sigma(u) = |V|$ indicates that $u$ is the last vertex in the ordering, and $\sigma(v) < \sigma(u)$ if and only if $v$ precedes $u$ in the ordering.
    We will show that for every $i$, there exist at most two edges $uv \in E$ with the property that $\sigma(u) \leq i$ and $\sigma(v) > i$. We refer to this number of edges as the \emph{width} of the cut between vertices $i$ and $i+1$ in $\sigma$.
    We begin by guaranteeing that, for every $i$, if $u$ is the last vertex of $G_i$ in $\sigma$, $v$ is the first vertex of $G_{i+1}$ in $\sigma$, and $b_{i, i+1}^1, b_{i, i+1}^2$ are the associated bridge vertices, then $\sigma(u) = \sigma(b_{i, i+1}^1) - 1 = \sigma(b_{i, i+1}^2) - 2 = \sigma(v) - 3$.
    A consequence is that for every $i$, if $u$, $v$ are vertices of $G_i$ then every vertex $w$ with the property that $\sigma(u) < \sigma(w) < \sigma(v)$ is also a vertex of $G_i$.
    Observe that every cut between two bridge vertices has width $1$, as does every cut between a bridge vertex and a non-bridge vertex.
    We therefore need only consider cuts between pairs of vertices in the same $G_i$.
    % Note that the edges of such a cut have both endpoints in $G_i$. We therefore need only consider the cutwidth of each $G_i$.
    If $G_i$ contains a conflict vertex, then $G_i$ is either a $P_3$ or a $P_4$.
    The former case arises when a variable appears in exactly two clauses, so there is one associated conflict vertex adjacent to two free vertices.
    The latter case arises when a variable appears in exactly three clauses, so there are two associated conflict vertices.
    These are adjacent.
    Additionally, each conflict vertex is adjacent to a distinct free vertex, so we have a $P_4$.
    Both the $P_3$ and $P_4$ have cutwidth $1$, as evidenced by ordering the vertices as they appear along the path.
    Moreover, we may assume that the relevant bridge edges are incident on the appropriate endpoints of the path, and so any cut between vertices of $G_i$ has width~$1$.
    Otherwise, $G_i$ contains no conflict vertices.
    In this case, $G_i$ contains only the five spare vertices $v_j^a, v_j^b, v_j^c, v_j^d$, and $v_j^e$ associated with some clause $C_j$, and the four edges are $v_j^av_j^b, v_j^av_j^c, v_j^av_j^d$, and $v_j^bv_j^e$.
    It is simple to check that this construction has cutwidth $2$, as evidenced by the ordering $\sigma(v_j^e) < \sigma(v_j^b) < \sigma(v_j^a) < \sigma(v_j^c) < \sigma(v_j^d)$.
    Once again, we may assume that the relevant bridge edges are incident on $v_j^e$ and $v_j^d$.
    Moreover, we can assume that at least one such $G_i$ exists, since \boolSAT{} instances in which every clause has size three and every variable appears at most three times are polynomial-time solvable~\cite{tovey1984simplified}.
    Thus, $G$ has cutwidth $2$.

    It remains to show that the reduction is correct. For the first direction, assume that there exists an assignment $\phi$ of boolean values to the variables $x_1, x_2, \ldots, x_n$ which satisfies every clause. We say that the assignment $\phi$ \emph{agrees} with a variable-clause pair $(x_i, C_j)$ if $C_j$ contains the positive literal $x_i$ and $\phi(x_i) = \textsf{True}$ or if $C_j$ contains the negative literal $\neg x_i$ and $\phi(x_i) = \textsf{False}$. We color the vertices of our constructed graph as follows.
    To every free vertex associated with clause $C_j$ we assign color $c_j$.
    The free vertex was only in edges of color $c_j$, so this results in zero unsatisfied edges.
    Next, to each bridge vertex we assign the associated bridge color.
    The bridge vertices were only in edges of this color, so once again this results in zero unsatisfied edges.
    Moreover, we have now guaranteed that at least one bridge edge of every bridge color is satisfied, meaning that at most two can be dissatisfied.
    Henceforth, we will not consider the remaining bridge edges.
    Next, for each clause $C_j$ of size $2$, we assign color $c_j'$ to every spare vertex associated with $C_j$. This results in exactly one unsatisfied edge of color $c_j$, and ensures that every edge of color $c_j'$ is satisfied. Finally, for each conflict vertex $v_{j_1, j_2}^i$, if $\phi$ agrees with $(x_i, C_{j_1})$ we assign color $c_{j_1}$ to $v_{j_1, j_2}^i$, and otherwise we assign color $c_{j_2}$.
    Observe that this coloring satisfies an edge $e_j^i$ if and only if $\phi$ agrees with $(x_i, C_j)$.
    Moreover, because $\phi$ is satisfying,
    every clause $C_j$ contains at least one variable $x_i$ such that $\phi$ agrees with $(x_i, C_j)$.
    Hence, at least one edge of every color is satisfied. Because there are exactly three edges of every color, there are at most two unsatisfied edges of any color.

    For the other direction, assume that we have a coloring which leaves at most two edges of any color unsatisfied. We will create a satisfying assignment $\phi$.
    For each variable $x_i$, we set $\phi(x_i) = \textsf{True}$ if any conflict edge associated with the positive literal $x_i$ is satisfied, and $\phi(x_i) = \textsf{False}$ otherwise.
    We now show that $\phi$ is satisfying. Consider any clause $C_j$. There are exactly three edges with color $c_j$, and at least one of them is satisfied.
    We may assume that the spare edge associated with $C_j$ (if such an edge exists) is unsatisfied, since satisfying this edge would require three unsatisfied edges of color $c_j'$. Thus,
    at least one conflict edge associated with $C_j$ is satisfied. Let $x_i$ be the corresponding variable, so the satisfied conflict edge is $e_j^i$.
    If $C_j$ contains the positive literal $x_i$, then $\phi(x_i) = \textsf{True}$ so $C_j$ is satisfied by $\phi$.
    Otherwise $C_j$ contains the negative literal $\neg x_i$.
    In this case, we observe that every conflict edge associated with the positive literal $x_i$ intersects with $e_j^i$ at a conflict vertex, and none of these edges has color $c_j$ since
    $C_j$ does not contain both literals. Hence, the satisfaction of $e_j^i$ implies that every conflict edge associated with the positive literal $x_i$ is unsatisfied.
    It follows that $\phi(x_i) = \textsf{False}$, meaning $C_j$ is satisfied by $\phi$.

    To show the claim for \cfmaxECC{}, we repeat the same construction, except that we omit all spare colors, vertices, and edges. The effect is that the constructed graph is a path.
    The proof of correctness is conceptually unchanged. Given a satisfying assignment $\phi$ we color vertices in the same way as before, and given a vertex coloring which satisfies
    at least one edge of every color, it remains the case that at least one conflict edge associated with every clause must be satisfied.
    The remaining analysis is similar.
\end{proof}

% %cfminECC FPT
% \cfminECCFPT*
% \begin{proof}
%     We give a branching algorithm.
%     Given an instance $(H = (V, E), \tau)$ of \cfminECC{}, a \emph{conflict} is a triple $(v, e_1, e_2)$ consisting of a single vertex $v$ and a pair of distinctly colored hyperedges $e_1, e_2$ which both contain $v$.
%     If $H$ contains no conflicts, then it is possible to satisfy every edge.
%     Otherwise, we identify a conflict in $O(r|E|)$ time by scanning the set of hyperedges incident on each node.
%     Once a conflict $(v, e_1, e_2)$ has been found, we branch on the two possible ways to resolve this conflict: deleting $e_1$ or deleting $e_2$.
%     Here, deleting a hyperedge has the same effect as ``marking'' it as unsatisfied and no longer considering it for the duration of the algorithm.
%     We note that it is simple to check in constant time whether a possible branch violates the constraint given by $\tau$; these branches can be pruned.
%     Because each branch increases the number of unsatisfied hyperedges by 1, the search tree has depth at most $\edgedeletions$.
%     Thus, by computing the search tree in level-order, the algorithm runs in time $O(2^{\edgedeletions}r|E|)$.
% \end{proof}

%


% NP-hard k = 2
% \cfmineccNPhardboundedk*


% reduction to sparse vertex cover


% combinatorial algorithm
% \cfminecccombinatorial*


\section{Full proof of~\texorpdfstring{\Cref{thm:pc-ECC-FPT}}{}}
\label{app:pc}

% \pceccapproximation*


\pceccFPT*
\begin{proof}
    We give a branching algorithm which is essentially identical to that of~\Cref{thm:cfminecc-FPT}.
    For completeness, we repeat the details.
    Given an instance $(H = (V, E), t, b)$ of \pcECC{}, a \emph{conflict} is a triple $(v, e_1, e_2)$ consisting of a single vertex $v$ and a pair of distinctly colored hyperedges $e_1, e_2$ which both contain $v$.
    If $H$ contains no conflicts, then it is possible to satisfy every edge.
    Otherwise, we identify a conflict in $O(r|E|)$ time by scanning the set of hyperedges incident on each node.
    Once a conflict $(v, e_1, e_2)$ has been found, we branch on the two possible ways to resolve this conflict: deleting $e_1$ or deleting $e_2$.
    Here, deleting a hyperedge has the same effect as ``marking'' it as unsatisfied and no longer considering it for the duration of the algorithm.
    We note that it is simple to check in constant time whether a possible branch violates the constraints given by $t$ or $b$; these branches can be pruned.
    Because each branch increases the number of unsatisfied hyperedges by 1 and WLOG $t > b$, the search tree has depth at most $t$.
    Thus, the algorithm runs in time $O(2^{t}r|E|)$.
\end{proof}



\end{document}

