\section{Experiments}

\begin{table*}[t]
\centering  
\caption{Data quality assessment for the long-context synthetic dataset (\textit{LongFinanceQA}) by measuring the performance of PAI on the Loong benchmark. \textit{AS} denotes \textit{Average Scores (0-100)}, and \textit{{PR}} represents the \textit{Perfect Rate} (0-1). \colorbox{mygreen}{Green} highlights the remarkable improvements over the base model (GPT-4o-mini).}\label{tab:pai_results}\vspace{-3mm}
\renewcommand{\arraystretch}{0.93}
\resizebox{\textwidth}{!}{
\begin{tabular}{llcccccccccc}
\toprule
\multirow{2}{*}{\textbf{Model}} & \multicolumn{1}{c}{\textbf{Context}} & \multicolumn{2}{c}{\textbf{Spotlight Locating}} & \multicolumn{2}{c}{\textbf{Comparison}} & \multicolumn{2}{c}{\textbf{Clustering}} & \multicolumn{2}{c}{\textbf{Chain of Reasoning}} & \multicolumn{2}{c}{\textbf{Overall}}\\ \cmidrule(r){3-4} \cmidrule(r){5-6} \cmidrule(r){7-8} \cmidrule(r){9-10} \cmidrule(r){11-12}
 & \multicolumn{1}{c}{\textbf{Length}} & \textbf{\textit{AS}} & \textbf{\textit{PR}} & \textbf{\textit{AS}} & \textbf{\textit{PR}} & \textbf{\textit{AS}} & \textbf{\textit{PR}} & \textbf{\textit{AS}} & \textbf{\textit{PR}} & \textbf{\textit{AS}} & \textbf{\textit{PR}} \\

\midrule
\multicolumn{12}{>{\columncolor[gray]{.88}}c}{\textit{Open-Source Long-Context Large Language Models}}  \\
LLaMA-3.1-8B-Instruct & 128K & 62.42 & 0.52 & 39.13 & 0.21 & 25.96 & 0.01 & 44.20 & 0.22 & 38.79 & 0.18 \\ 
DeepSeek-R1-Qwen-32B & 128K & 51.68 & 0.41 & 49.25 & 0.34 & 41.53 & 0.16 & 45.00 & 0.30 & 45.45 & 0.27 \\ 
Qwen2-72B-Instruct & 128K & 54.17 & 0.36 &42.38 & 0.20 & 36.71 & 0.04 & 47.76 & 0.18 & 43.29 & 0.15 \\ 
Qwen2.5-72B-Instruct & 128K & 65.08 & 0.55 & 51.90 & 0.30 & 46.07 & 0.08 & 64.43 & 0.40 & 54.83 & 0.28  \\ 
% LLaMA-3-8B-Instruct-262K & 262K & 48.77 & 0.30 & 29.33 & 0.11 & 19.93 & 0.01 & 21.27 & 0.07 & 27.52 & 0.10 \\ 
% GLM4-9B-Chat & 1000K & 57.35 & 0.47 & 40.38 & 0.20 & 28.52 & 0.02 & 39.94 & 0.16 & 38.31 & 0.16\\ 
Qwen2.5-14B-Instruct-1M & 1000K &  67.50 & 0.58 & 55.12 & 0.35 & 39.05 & 0.04 & 57.81 & 0.31 & 51.30 & 0.25  \\ 

\midrule
\multicolumn{12}{>{\columncolor[gray]{.88}}c}{\textit{Closed-Source Long-Context Large Language Models}}  \\

Kimi-Chat & 200K & 60.98 & 0.50 & 34.74 & 0.13 & 28.76 & 0.04 & 38.52 & 0.15 & 37.49 & 0.16 \\

% Claude3-Haiku & 200K & 68.68 & 0.59 & 42.10 & 0.21 & 35.04 & 0.02 & 47.59 & 0.17 & 44.88 & 0.19\\

Claude3.5-Sonnet & 200K & 58.45 & 0.49 & 54.21 & 0.35 & 45.77 & 0.07 & 43.92 & 0.25 & 48.85 & 0.23\\

GPT-4o & 128K & 73.95 &0.62 & 50.50 &0.28 &44.29 &0.09 &57.95 &0.28 &53.47 &0.26 \\

Gemini-1.5-pro & 1000K & 75.02 & 0.56 & 49.94 & 0.27 & 44.10 & 0.09 & 64.97 & 0.37 & 55.37 & 0.27 \\


\midrule
GPT-4o-mini (Base) & 128K & 59.46 & 0.49 & 51.90 & 0.27 & 34.55 & 0.04 & 64.28 & 0.39 & 49.25& 0.24 \\

GPT-4o-mini w/ PAI (\textit{ours}) & 128K & \cellcolor{mygreen}{\textbf{79.74}} & \cellcolor{mygreen}\textbf{0.67} & \cellcolor{mygreen}\textbf{67.60} & \cellcolor{mygreen}\textbf{0.46} & \cellcolor{mygreen}\textbf{62.80} & \cellcolor{mygreen}\textbf{0.27} & \cellcolor{mygreen}\textbf{75.46} & \cellcolor{mygreen}\textbf{0.57} & \cellcolor{mygreen}\textbf{69.58} & \cellcolor{mygreen}\textbf{0.44} \\
% \midrule
% LLaMA-3.1-8B-Instruct & 128K & 62.42 & 0.52 & 39.13 & 0.21 & 25.96 & 0.01 & 44.20 & 0.22 & 38.79 & 0.18\\ 
% \hspace{4mm} FT on LongFinance w/o Reasoning & 256K &  &  &  &  &  &  &  &  &  &  \\
% \hspace{4mm} FT on LongFinance w/ Reasoning & 256K &  &  &  &  &  &  &  &  &  &  \\
\bottomrule
\end{tabular}
}\vspace{-3mm}
\end{table*}
\begin{table*}[t]
\centering
\caption{Performance on \textit{Financial} subset of Loong benchmark. \textit{AS} represents \textit{Avg Scores (0\textasciitilde100)} and \textit{{PR}} denotes \textit{Perfect Rate} (0\textasciitilde1). \colorbox{mygreen}{Green} highlights improvements over the base model. Full results are shown in Appendix~\ref{sec:appendix}.}\vspace{-2.5mm}
\renewcommand{\arraystretch}{0.93}
\resizebox{\textwidth}{!}{
\begin{tabular}{llcccccccccc}
\toprule
\multirow{2}{*}{\textbf{Model}} & \multicolumn{1}{c}{\textbf{Context}} & \multicolumn{2}{c}{\textbf{Spotlight Locating}} & \multicolumn{2}{c}{\textbf{Comparison}} & \multicolumn{2}{c}{\textbf{Clustering}} & \multicolumn{2}{c}{\textbf{Chain of Reasoning}} & \multicolumn{2}{c}{\textbf{Overall}}\\ \cmidrule(r){3-4} \cmidrule(r){5-6} \cmidrule(r){7-8} \cmidrule(r){9-10} \cmidrule(r){11-12}
 & \multicolumn{1}{c}{\textbf{Length}} & \textbf{\textit{AS}} & \textbf{\textit{PR}} & \textbf{\textit{AS}} & \textbf{\textit{PR}} & \textbf{\textit{AS}} & \textbf{\textit{PR}} & \textbf{\textit{AS}} & \textbf{\textit{PR}} & \textbf{\textit{AS}} & \textbf{\textit{PR}} \\

\midrule
\multicolumn{12}{>{\columncolor[gray]{.88}}c}{\textbf{$\mathtt{All\ Set}$ (10K-250K)}}  \\
% DeepSeek-R1-Qwen-32B & 128K &  53.66 & 0.46 & 52.19 & 0.39 & 39.76 & 0.17 & 65.15 & 0.51 & 49.92 & 0.34 \\
Qwen2-72B-Instruct & 128K &   59.80 & 0.47 & 61.12 & 0.43 & 34.32 & 0.06 & 74.68 & 0.50 & 53.20 & 0.32  \\
Qwen2.5-72B-Instruct & 128K &  71.07 & 0.63 & 59.14 & 0.41 & 38.23 & 0.08 & 81.09 & 0.60 & 57.36 & 0.36  \\
% LLaMA-3-8B-Instruct-262K & 262K  &  58.60 & 0.41 & 33.12 & 0.16 & 20.04 & 0.01 & 35.10 & 0.09 & 34.41 & 0.15 \\
% GLM4-9B-Chat & 1000K & 72.69 & 0.60 & 49.31 & 0.32 & 23.41 & 0.02 & 60.77 & 0.28 & 46.71 & 0.27 \\
Qwen-2.5-14B-Instruct-1M & 1000K & 78.83 & 0.72 & 65.27 & 0.50 & 36.24 & 0.07 & 79.10 & 0.64 & 59.78 & 0.41 \\
GPT-4o & 128K &  88.23 & 0.84 & 62.90 & 0.48 & 45.51 & 0.17 & 69.40 & 0.43 & 63.05 & 0.44 \\
\midrule
GPT-4o-mini (Base) & 128K & 70.90 & 0.59 & 59.37 & 0.38 & 36.33 & 0.06 & 79.58 & 0.58 & 56.50 & 0.34 \\
GPT-4o-mini w/ PAI (\textit{ours}) & 128K & \cellcolor{mygreen}91.07 & \cellcolor{mygreen}0.83 & \cellcolor{mygreen}74.40 & \cellcolor{mygreen}0.58 & \cellcolor{mygreen}61.55 & \cellcolor{mygreen}0.32 & \cellcolor{mygreen}89.63 & \cellcolor{mygreen}0.78 & \cellcolor{mygreen}75.56 & \cellcolor{mygreen}0.57 \\
\midrule
LLaMA-3.1-8B-Instruct (Base) & 128K & 67.84 & 0.56 & 47.12 & 0.30 & 24.62 & 0.02 & 63.63 & 0.34 & 45.88 & 0.26 \\ 
LongPAI (\textit{ours}) & 262K & \cellcolor{mygreen}87.02 & \cellcolor{mygreen}0.78 & \cellcolor{mygreen}64.50 & \cellcolor{mygreen}0.48 & \cellcolor{mygreen}60.95 & \cellcolor{mygreen}0.32 & \cellcolor{mygreen}81.89 & \cellcolor{mygreen}0.71 & \cellcolor{mygreen}70.54 & \cellcolor{mygreen}0.52 \\ 

 

\midrule
\multicolumn{12}{>{\columncolor[gray]{.88}}c}{\textbf{$\mathtt{Set3}$ (100K-200K)}}  \\
% DeepSeek-R1-Qwen-32B & 128K &  41.73 & 0.33 & 39.56 & 0.23 & 25.67 & 0.06 & 55.14 & 0.37 & 37.35 & 0.21 \\
Qwen2-72B-Instruct & 128K &  47.00 & 0.33 & 48.07 & 0.27 & 25.79 & 0.00 & 69.37 & 0.34 & 42.98 & 0.20  \\
Qwen2.5-72B-Instruct & 128K &  60.47 & 0.48 & 49.00 & 0.28 & 30.61 & 0.01 & 76.54 & 0.46 & 48.99 & 0.26  \\
% LLaMA-3-8B-Instruct-262K & 262K  &  54.14 & 0.40 & 22.93 & 0.05 & 15.43 & 0.00 & 29.00 & 0.00 & 28.58 & 0.11 \\
% GLM4-9B-Chat & 1000K & 74.75 & 0.65 & 41.63 & 0.24 & 21.99 & 0.01 & 49.86 & 0.17 & 43.58 & 0.25 \\
Qwen-2.5-14B-Instruct-1M & 1000K &  74.33 & 0.68 & 54.64 & 0.35 & 30.72 & 0.01 & 73.71 & 0.51 & 53.47 & 0.33 \\
GPT-4o & 128K &  87.25 & 0.83 & 46.00 & 0.31 & 36.68 & 0.08 & 64.57 & 0.40 & 54.79 & 0.36 \\
\midrule
GPT-4o-mini (Base) & 128K & 63.05 & 0.53 & 53.48 & 0.24 & 29.80 & 0.01 & 72.37 & 0.46 & 50.03 & 0.26 \\
GPT-4o-mini w/ PAI (\textit{ours}) & 128K & \cellcolor{mygreen}94.08 & \cellcolor{mygreen}0.83 & \cellcolor{mygreen}74.13 & \cellcolor{mygreen}0.63 & \cellcolor{mygreen}55.78 & \cellcolor{mygreen}0.20 & \cellcolor{mygreen}87.71 & \cellcolor{mygreen}0.77 & \cellcolor{mygreen}74.21 & \cellcolor{mygreen}0.55\\
\midrule
LLaMA-3.1-8B-Instruct (Base) & 128K & 63.38 & 0.47 & 36.04 & 0.19 & 20.28 & 0.00 & 62.49 & 0.26 & 40.45 & 0.20 \\ 
LongPAI (\textit{ours}) & 262K & \cellcolor{mygreen}88.00 & \cellcolor{mygreen}0.80 & \cellcolor{mygreen}50.07 & \cellcolor{mygreen}0.28 & \cellcolor{mygreen}55.01 & \cellcolor{mygreen}0.21 & \cellcolor{mygreen}84.03 & \cellcolor{mygreen}0.69 & \cellcolor{mygreen}65.10 & \cellcolor{mygreen}0.43 \\ 


\midrule
\multicolumn{12}{>{\columncolor[gray]{.88}}c}{\textbf{$\mathtt{Set4}$ (200K-250K)}}  \\
% DeepSeek-R1-Qwen-32B & 128K &  18.33 & 0.11 & 10.25 & 0.05 & 11.27 & 0.00 & 26.00 & 0.07 & 15.52 & 0.05 \\
Qwen2-72B-Instruct & 128K &  41.85 & 0.19 & 39.75 & 0.15 & 29.17 & 0.03 & 41.67 & 0.07 & 37.23 & 0.11  \\
Qwen2.5-72B-Instruct & 128K &  57.48 & 0.44 & 49.50 & 0.30 & 27.33 & 0.00 & 55.00 & 0.13 & 45.51 & 0.22  \\
% LLaMA-3-8B-Instruct-262K & 262K  &  34.19 & 0.07 & 20.00 & 0.05 & 20.31 & 0.00 & 20.71 & 0.00 & 24.61 & 0.03 \\
% GLM4-9B-Chat & 1000K & 40.85 & 0.19 & 29.50 & 0.05 & 18.13 & 0.00 & 25.73 & 0.00 & 28.51 & 0.07 \\
Qwen-2.5-14B-Instruct-1M & 1000K &  59.44 & 0.41 & 37.00 & 0.20 & 27.33 & 0.00 & 31.67 & 0.00 & 39.57 & 0.16 \\
GPT-4o & 128K &  69.26 & 0.56 & 50.50 & 0.35 & 30.70 & 0.00 & 50.67 & 0.07 & 49.58 & 0.25 \\
\midrule
GPT-4o-mini (Base) & 128K & 48.37 & 0.26 & 50.00 & 0.30 & 28.70 & 0.00 & 48.33 & 0.07 & 42.30 & 0.15 \\
GPT-4o-mini w/ PAI (\textit{ours}) & 128K & \cellcolor{mygreen}82.78 & \cellcolor{mygreen}0.70 & \cellcolor{mygreen}63.50 & \cellcolor{mygreen}0.35 & \cellcolor{mygreen}48.00 & \cellcolor{mygreen}0.17 & \cellcolor{mygreen}76.00 & \cellcolor{mygreen}0.53 & \cellcolor{mygreen}66.14 & \cellcolor{mygreen}0.42\\
\midrule
LLaMA-3.1-8B-Instruct (Base) & 128K & 40.74 & 0.30 & 35.85 & 0.20 & 19.77 & 0.00 & 28.73 & 0.00 & 30.88 & 0.13 \\ 
LongPAI (\textit{ours})  & 262K & \cellcolor{mygreen}77.89 & \cellcolor{mygreen}0.63 & \cellcolor{mygreen}62.00 & \cellcolor{mygreen}0.45 & \cellcolor{mygreen}48.00 & \cellcolor{mygreen}0.17 & \cellcolor{mygreen}51.33 & \cellcolor{mygreen}0.47 & \cellcolor{mygreen}60.36 & \cellcolor{mygreen}0.41  \\ 
\bottomrule
\end{tabular}
}\vspace{-2mm}
\label{tab:longpai_results}
\end{table*}

\subsection{Experimental Setup}\label{sec:4.1}

\noindent \textbf{Evaluation Benchmarks}.
We evaluate long-context understanding using two practical benchmarks: Loong~\cite{wang2024leave} and $\infty$Bench~\cite{zhang2024bench}. Loong focuses on real-world multi-document question answering and comprises 1,600 test samples across four categories, including Spotlight Locating, Comparison, Clustering, and Chain of Reasoning. These tasks assess distinct capabilities in handling long-context tasks. $\infty$Bench facilitates multilingual evaluation, assessing models on English (En.QA) and Chinese (Zh.QA) question-answering tasks that require long-range dependency and reasoning beyond short-passage retrieval.

\noindent \textbf{Evaluation Metrics}. For evaluation, Loong employs GPT-4-Turbo as a judge, scoring model responses based on accuracy, hallucinations, and completeness on a scale of 0 to 100. Meanwhile, it introduces the Perfect Rate, measuring the proportion of responses achieving a perfect score. In $\infty$Bench, model performance is measured by the F1 score~\cite{zhang2024bench}. 

\noindent \textbf{Base Models}. 
We adopt GPT-4o-mini~\cite{achiam2023gpt} and LLaMA-3.1-8B-Instruct~\cite{dubey2024llama} as our base models. Specifically, GPT-4o-mini serves as the agent of the proposed Preference Agentic Inference (PAI), while LLaMA-3.1-8B is used as the base model for LongPAI, which is fine-tuned on the LongFinanceQA dataset.


\noindent \textbf{Implementation Details}.
To enable training on long sequences (> 250K), we employ several optimization techniques, including flash-attention-2~\cite{dao2023flashattention} and ring-attention~\cite{liu2024ringattention}. Furthermore, we adopt a zigzag sharding approach~\cite{zhu2024ringflash} within ring attention for more effective load distribution across multiple GPUs. This training setup allows us to fine-tune the large language models fully.
Using 8 A100 GPUs, the long-context training is completed in three days for fine-tuning.
In addition, in training the long-context LLMs, we adopt the rotary base scaling approach~\cite{liu2024scaling} and scale up the base value (\ie, rotary base of 1,247,820) to adapt RoPE to a longer context.
For optimization, we use a constant learning rate of 2e-6 for the entire training procedure.
Following the common practice~\cite{fu2024data}, we set the batch size to 16M tokens as mentioned in \cite{dubey2024llama}.
During inference, we set the temperature as zero to eliminate the randomness. We also increase the maximum output tokens to 1,024 since the CoT reasoning requires more output tokens.


\subsection{Main Results}\vspace{-2mm}
In this section, we first assess the quality of our long-context synthetic data (LongFinanceQA) by evaluating the performance of the proposed PAI on the Loong benchmark. Next, we compare the enhanced long-context language model (LongPAI) with its base LLaMA-3.1 model and other state-of-the-art LLMs on the \textit{Finance} subset of Loong.

\noindent \textbf{Data Quality Assessment}.
Reasoning-augmented answers in LongFinanceQA are automatically generated by the PAI framework. To assess the quality of these synthetic answers, we evaluate the annotator, PAI, on the Long benchmark, using GPT-4o-mini as the agent within the PAI, referred to as GPT-4o-mini w/ PAI.
Table~\ref{tab:pai_results} shows that the GPT-4o-mini-based PAI framework significantly enhances the base model’s performance, demonstrating the effectiveness of the PAI.
In particular, compared to the standard GPT-4o-mini model, the overall average score improves by 20.3\%, with substantial gains in key tasks such as spotlight locating (+20.2\%), comparison (+15.7\%), clustering (+29.2\%), and chain of reasoning (+10.2\%).
Although the basic GPT-4o-mini is not the strongest model, GPT-4o-mini w/ PAI outperforms the state-of-the-art closed-source model, Geneni-1.5-pro~\cite{reid2024gemini}, by over 15\%. 
Moreover, the superior performance in the Spotlight Locating task (\ie, single-source QA task) \textit{highlights the quality of intermediate reasoning results generated by the PAI}, as predictions in this task serve as essential reasoning steps for the other three multi-source QA tasks.
Consequently, the strong performance on the Loong demonstrates \textit{the capability of PAI as a reliable annotator, ensuring high-quality synthetic data in long-context scenarios}.


\noindent \textbf{Method Comparisons on Existing Benchmarks}.
To evaluate the effectiveness of supervised CoT reasoning 
on long-context modeling, we measure the performance of the enhanced LongPAI model on two well-known long-context understanding benchmarks, including Loong and $\infty$Bench. 
First, we evaluate the LongPAI on an in-domain benchmark, namely the \textit{Finance} subset of Loong.
Table~\ref{tab:longpai_results} presents that supervised CoT reasoning enables LongPAI to outperform its base model, LLaMA-3.1-8B-Instruct, by 24.6\% in overall results. Furthermore, LongPAI exhibits nearly a 30\% improvement on subsets with longer content (\ie, 200K-250K). Also, LongPAI achieves a competitive performance against existing state-of-the-art language models. Remarkably, LongPAI is comparable to its teacher model, GPT-4o-mini w/ PAI. In some cases, it even surpasses its teacher model. This finding highlights the significance of long-context modeling. Meanwhile, this finding strongly challenges the recent claim that \textit{the long-context problem can be adequately addressed by short language models}~\cite{qian2024long,chen2024long}. In other words, certain long-context problems require long-context modeling, as short language models struggle to analyze and reason effectively over reduced or retrieved information. At the same time, we evaluate the LongPAI on $\infty$Bench. The results in Table~\ref{tab:infbench} show that the LongPAI outperforms its base model even on the out-domain benchmark.


\begin{table}[t!]
\centering
\caption{Comparison on En.QA and Zh.QA of $\infty$Bench. $^*$ indicates results borrowed from~\cite{zhang2024bench}.}\label{tab:infbench}\vspace{-3mm}
\renewcommand{\arraystretch}{0.9}
\resizebox{0.48\textwidth}{!}{
\small
\begin{tabular}{lcc}
\toprule
\textbf{} & \textbf{En.QA} & \textbf{Zh.QA} \\
\midrule
YaRN-Mistra$^*$ & 9.55 & 16.98 \\
Kimi-Chat$^*$ & 16.52 & 18.62 \\
Claude 2$^*$ & 11.97 & 10.53 \\
GPT-4$^*$ & 22.22 & 23.06 \\
\midrule
LLaMA-3.1-8B-Instruct (Base)& 27.11 & 29.77 \\
LongPAI (ours)& \textbf{32.46} & \textbf{32.38} \\
\bottomrule
\end{tabular}\vspace{-4mm}
}
\end{table}

\begin{table}[t!]
\centering
\caption{Ablation study on  \textit{Supervised CoT Reasoning}. Symbol $^{\S}$ represents LongPAI without supervised CoT reasoning during fine-tuning. Average Scores (0-100) are evaluated by GPT-4-Turbo. \colorbox{mygreen}{Green} highlights the remarkable improvements over the base GPT-4o-mini, while \colorbox{myred}{Red} indicates a decline.
Abbreviations: \textbf{S.L.} (Spotlight Locating), \textbf{Comp.} (Comparison), \textbf{Clust.} (Clustering), and \textbf{Chain.} (Chain of Reasoning). } 
\label{tab:cot_ablation}\vspace{-3mm}
\renewcommand{\arraystretch}{1}
\resizebox{0.49\textwidth}{!}{
\begin{tabular}{lccccc}
\toprule
\textbf{Method} & \textbf{S.L.} & \textbf{Comp.} & \textbf{Clust.} & \textbf{Chain.} & \textbf{Overall} \\
\midrule

\multicolumn{6}{>{\columncolor[gray]{.88}}c}{\textbf{$\mathtt{Set1}$ (10K-50K)}}  \\
LLaMA-3.1-8B & 89.13 & 72.33 & 31.77 & 74.0 & 60.50 \\
LongPAI$^{\S}$ & \cellcolor{mydarkgreen}93.39 & \cellcolor{myred}68.67 & \cellcolor{mydarkgreen}79.38 & \cellcolor{mygreen}83.80 & \cellcolor{mygreen}79.82 \\
LongPAI & \cellcolor{myred}85.00 & \cellcolor{mydarkgreen}90.00 & \cellcolor{mygreen}69.12 & \cellcolor{mydarkgreen}98.30 & \cellcolor{mydarkgreen}81.58\\
\midrule
\multicolumn{6}{>{\columncolor[gray]{.88}}c}{\textbf{$\mathtt{Set2}$ (50-100K)}}  \\
LLaMA-3.1-8B & 80.58 & 51.11 & 27.39 & 75.12 & 51.13 \\
LongPAI$^{\S}$ & \cellcolor{mygreen}82.12 & \cellcolor{myred}40.40 & \cellcolor{myred}11.24 & \cellcolor{myred}50.25 & \cellcolor{myred}38.11 \\
LongPAI & \cellcolor{mydarkgreen}92.88 & \cellcolor{mydarkgreen}69.40 & \cellcolor{mydarkgreen}67.57 & \cellcolor{mydarkgreen}87.38 & \cellcolor{mydarkgreen}75.49 \\
\midrule
\multicolumn{6}{>{\columncolor[gray]{.88}}c}{\textbf{$\mathtt{Set3}$ (100K-200K)}}  \\
LLaMA-3.1-8B & 63.38 & 36.04 & 20.28 & 62.49 & 40.45 \\
LongPAI$^{\S}$ & \cellcolor{myred}59.58 & \cellcolor{myred}30.43 & \cellcolor{myred}5.47 & \cellcolor{myred}37.43 & \cellcolor{myred}29.46 \\
LongPAI & \cellcolor{mydarkgreen}88.00 & \cellcolor{mydarkgreen}50.07 & \cellcolor{mydarkgreen}55.01 & \cellcolor{mydarkgreen}84.03 & \cellcolor{mydarkgreen}65.10 \\
\midrule
\multicolumn{6}{>{\columncolor[gray]{.88}}c}{\textbf{$\mathtt{Set4}$ (200K-250K)}}  \\
LLaMA-3.1-8B & 40.74 & 35.85 & 19.77 & 28.73 & 30.88 \\
LongPAI$^{\S}$ & \cellcolor{mygreen}43.37 & \cellcolor{myred}22.00 & \cellcolor{myred}5.60 & \cellcolor{myred}17.20 & \cellcolor{myred}22.14 \\
LongPAI & \cellcolor{mydarkgreen}77.89 & \cellcolor{mydarkgreen}62.00 & \cellcolor{mydarkgreen}48.00 & \cellcolor{mydarkgreen}51.33 & \cellcolor{mydarkgreen}60.36 \\
\bottomrule
\end{tabular}
}\vspace{-2mm}
\end{table}



\subsection{Discussion}
\noindent \textbf{Ablation Study on Supervised CoT Reasoning}. 
To further analyze the impact of supervised CoT reasoning, we fine-tune the base LLaMA-3.1 model on LongFinanceQA while excluding CoT reasoning steps from the augmented answers, resulting in a new model, LongPAI$^{\S}$.
Unlike the original LongPAI, LongPAI$^{\S}$ directly predicts the final answer, skipping intermediate reasoning steps.
Table~\ref{tab:cot_ablation} shows that LongPAI significantly outperforms LongPAI$^{\S}$ over different input lengths. This result strongly supports the hypothesis mentioned in Section~\ref{sec:intro}, which argues that \emph{directly guiding models to generate brief answers without intermediate reasoning steps for long-context modeling will lead to suboptimal training} (\textbf{finding 1}).
Moreover, Table~\ref{tab:cot_ablation} presents several interesting finding.
First, while LongPAI$^{\S}$ performs comparably to LongPAI on short content (10K–50K tokens), its performance declines significantly on longer content, which means \emph{CoT reasoning is necessary for long-context modeling} (\textbf{finding 2}). Furthermore, LongPAI$^{\S}$ performs well in single-source questions (Spotlight Locating) but struggles with multi-source questions (Comparison, Clustering, and Chain of Reasoning) as input length increases, demonstrating that \emph{CoT reasoning benefits complex long-context problem-solving} (\textbf{finding 3}). In sum, all these findings reaffirm the importance of the reasoning capability for long-context modeling.

\noindent \textbf{Comparison of Various Inference Frameworks}. We compare PAI with two relevant inference frameworks: PAI$^-$ and RAG.
PAI$^-$ is a variant of PAI that generates sub-questions directly, rather than first extracting properties and then generating sub-questions.
Table~\ref{tab:inference_comparison} presents that PAI$^-$ generally outperforms the base GPT-4o-mini, except on $\mathtt{Set 1}$. However, there is a gap between PAI$^-$ and PAI, highlighting the superiority of the Property Extraction Agent.
Retrieval-Augmented Generation (RAG)~\cite{lewis2020retrieval} follows a two-step process: it first retrieves the top $K$ chunks relevant to a given query, and then uses retrieved chunks to generate an answer. Following Loong~\cite{wang2024leave}, we adopt the \textit{BGE}~\cite{chen-etal-2024-m3} as the embedding choice and set $K$ to 50, selecting from 5, 10, 30, and 50. Table~\ref{tab:inference_comparison} shows that RAG struggles with long-context problems, a conclusion also reached by \cite{wang2024leave}.


\begin{figure}[tb]
    \centering
    \includegraphics[width=\linewidth,trim={0 0 0 0},clip]{figs/raw/infer_v8.pdf}
        \vspace{-20pt}
        \caption{{\ours} and Best-of-N under different search budgets. The x-axis represents the number of tokens consumed by the trajectories generated during inference averaged on all the tasks in each test set.}
        \label{fig:infer_curve}
        \vspace{-22pt}
\end{figure}

\noindent \textbf{Efficiency Analysis on PAI}. The PAI framework relies on multi-step inference, whereas LongPAI achieves results with a single inference step. We adopt the number of input tokens processed on the Loong benchmark (\ie, 1,600 samples) as a metric to compare the efficiency between PAI and LongPAI. In this comparison, PAI processes 3.53B tokens in total, determined by the GPT-4o tokenizer, whereas LongPAI requires only 112M tokens—less than 3\% of PAI's total. Despite PAI delivering the strongest overall performance, LongPAI stands out as the far more efficient approach, demonstrating a dramatic reduction in computational cost.
