
\documentclass[letterpaper,twocolumn,10pt]{article}
\usepackage{usenix2019_v3}

\usepackage{tikz}
\usepackage{comment}



\usepackage[camera]{dtrt}


\hypersetup{
     colorlinks=true,
     linkcolor=blue,
     filecolor=blue,
     citecolor = blue,
     urlcolor=blue,
 }

\usepackage[T1]{fontenc}
\usepackage{amssymb, amsmath, amsthm, booktabs}
\usepackage{graphicx}
\usepackage[capitalize]{cleveref}
\usepackage{enumitem}
\usepackage{subcaption}
\usepackage{dashbox}
\usepackage{cancel}
\usepackage{float}
\usepackage{csquotes}
\usepackage{listings}
\usepackage{algorithm}
\usepackage[noend]{algpseudocode}
\usepackage[normalem]{ulem}


\usepackage{array}
\newcolumntype{H}{>{\setbox0=\hbox\bgroup}c<{\egroup}@{}}

\newcommand{\myparagraph}[1]{\smallskip\noindent\textbf{#1.}}
\newcommand{\myparagraphit}[1]{{\smallskip\textit{#1}}}

\crefname{part}{\S}{\S\S}
\crefname{chapter}{\S}{\S\S}
\crefname{section}{\S}{\S\S}
\crefname{subsection}{\S}{\S\S}

\crefname{claim}{Claim}{Claims}
\crefname{remark}{Remark}{Remarks}

\def\UrlBreaks{\do\/\do-}

\newcommand{\pname}{\texorpdfstring{\ensuremath{\mathsf{CRATE}}}{CRATE}}

\pagestyle{plain}




\begin{document}

\newcommand{\CG}{\mathcal{G}\xspace}
\newcommand{\CV}{\mathcal{V}\xspace}
\newcommand{\CE}{\mathcal{E}\xspace}
\newcommand{\CA}{\mathcal{A}\xspace}
\newcommand{\CF}{\mathcal{F}\xspace}
\newcommand{\CR}{\mathcal{R}\xspace}
\newcommand{\CB}{\mathcal{B}\xspace}
\newcommand{\CX}{\mathcal{X}\xspace}
\newcommand{\CK}{\mathcal{K}\xspace}
\newcommand{\CM}{\mathcal{M}\xspace}
\newcommand{\CC}{\mathcal{C}\xspace}
\newcommand{\CL}{\mathcal{L}\xspace}
\newcommand{\CI}{\mathcal{I}\xspace}
\newcommand{\CQ}{\mathcal{Q}\xspace}
\newcommand{\CO}{\mathcal{O}\xspace}
\newcommand{\CP}{\mathcal{P}\xspace}
\newcommand{\CS}{\mathcal{S}\xspace}
\newcommand{\CT}{\mathcal{T}\xspace}
\newcommand{\CJ}{\mathcal{J}\xspace}
\usepackage[para]{footmisc}
\usepackage{subfig}
% \usepackage{subcaption}
% \usepackage{array}
% \usepackage{colortbl}


\input{contracts/solidity-highlighting}	%

\date{}

\title{\Large \bf \pname: Cross-Rollup Atomic Transaction Execution}

\author{
{\rm Ioannis Kaklamanis}\\
Yale University
\and
{\rm Fan Zhang}\\
Yale University
} %

\maketitle

\begin{abstract}
Blockchains have revolutionized decentralized applications, with composability enabling atomic, trustless interactions across smart contracts. However, layer 2 (L2) scalability solutions like rollups introduce fragmentation and hinder composability.
Current cross-chain protocols, including atomic swaps, bridges, and shared sequencers, lack the necessary coordination mechanisms or rely on trust assumptions, and are thus not sufficient to support full cross-rollup composability.
This paper presents $\pname$, a secure protocol for cross-rollup composability that ensures all-or-nothing and serializable execution of cross-rollup transactions (CRTs).
 $\pname$ supports rollups on distinct layer 1 (L1) chains, achieves finality in 4 rounds on L1, and only relies on the underlying L1s and the liveness of L2s. We introduce two formal models for CRTs, define atomicity within them, and formally prove the security of $\pname$. We also provide an implementation of $\pname$ along with a cross-rollup flash loan application; our experiments demonstrate that $\pname$ is practical in terms of gas usage on L1.
\end{abstract}

\section{Introduction}~\label{sec:intro}
\section{Introduction}
\label{sec:intro}


\ps{Challenges of technology scaling}

The growing demand for computing performance has always been met by increasing the number of transistors per chip, which is only possible due to CMOS technology scaling.
However, as we keep pushing the boundaries of technology scaling, we encounter multiple challenges.
Firstly, whenever we transition to a more advanced technology node, the non-recurring cost due to physical design, verification, software, mask sets, and prototyping almost doubles \cite{cost-tech-node}.
As a result, designing a chip in an advanced technology node is only economically viable if the chip is manufactured in vast quantities.
Secondly, many chip components such as I/O drivers, analog circuits, or \gls{srams} have reached their scaling limit.
This means that we cannot shrink these components further, even if we use a more advanced technology with a smaller feature size.
Thirdly, advanced technology nodes suffer from high defect rates, diminishing the yield and inflating the recurring cost.
To tackle these challenges, new chip-design paradigms have been developed.

\ps{Why 2.5D integration?}

One of these new paradigms is 2.5D integration, where multiple silicon dies called chiplets are integrated into the same package.
Once designed, a single chiplet can be reused in multiple 2.5D stacked chips, which increases the ratio of production volume to non-recurring cost.
Another advantage is that multiple chiplets - fabricated in different technologies - can be integrated into the same package.
This means that only components that can take full advantage of technology scaling are built in bleeding-edge technologies.
Components that have reached their scaling limit are fabricated in more mature and hence less costly technology nodes.
Furthermore, chiplets are smaller than monolithic chips.
Therefore, manufacturing chiplets results in less silicon area loss due to fabrication defects and hence a higher yield.
Due to these economic advantages, chip vendors such as AMD \cite{amd-chiplet} and NVIDIA \cite{chiplet-book} have adopted the 2.5D integration paradigm.  

\ps{Challenges of 2.5D integration}

An important challenge when designing 2.5D stacked chips is the construction of a low-latency and high-throughput \gls{ici}. 
To build an \gls{ici}, we connect different chiplets using \gls{d2d} links.
These links are fabricated in an organic package substrate, silicon bridge, or silicon interposer, and they are connected to the chiplets using \gls{c4} bumps or microbumps.
The number of bumps per chiplet is limited, and so is the bandwidth of \gls{d2d} links.
In addition to having lower bandwidth than links in monolithic chips, \gls{d2d} links also have higher latency.
This latency is caused by wire delay and by \gls{phys} that are necessary in both the sending and the receiving chiplet.
\gls{phys} are needed to convert between protocols, voltage levels, and frequencies, which are usually different between on-chiplet links and \gls{d2d} links.
Due to these limitations, the \gls{ici} can quickly become a bottleneck.

\ps{How we solve these challenges differently than the related work does.}

Existing approaches to maximize the performance of the \gls{ici} either optimize the placement of chiplets (with potentially heterogeneous shapes) for a predetermined \gls{ici} topology 
\cite{ho,liu,seemuth,eris,osmolovskyi,tap25d,chiou}, select one topology out of a set of candidates \cite{coskun-1, coskun-2}, or they optimize the \gls{ici} topology for a 2D grid of homogeneously shaped chiplets on an active interposer \cite{butterdonut, cluscross, kite}.
To the best of our knowledge, there is no prior work on \gls{ici} topologies for chips with heterogeneously shaped chiplets or with passive silicon interposers or silicon bridges.
To fill this gap, we propose \name, a novel optimization methodology to jointly optimize the chiplet placement and \gls{ici} topology of such architectures.
\ifnb
\else
\newpage
\fi

\ps{Details on \name~and the key idea}

The key idea is as follows: 
We optimize the chiplet placement without a predetermined topology.
For each placement generated by an optimization algorithm, we infer a placement-based \gls{ici} topology by connecting chiplets that are in close proximity in that specific placement.
We then compute the latency and throughput of this combination of placement and topology for different traffic types.
These latencies and throughputs together with the total chip area are used to compute a user-defined quality-score of the placement, which is returned to the optimization algorithm.
Based on this quality score, the algorithm can further optimize the placement.
By following this iterative process, we jointly optimize the chiplet placement and the \gls{ici} topology.

\ps{Short evaluation-summary}

We provide our open-source framework implementing the proposed placement and topology co-optimization methodology, which we evaluate using both synthetic traffic and traffic traces.
A 2D grid of chiplets with a mesh topology is used as a baseline since many proposals for 2.5D stacked chips \cite{dataflow_accel_dnn, cifher, simba, hecaton, dojo} use such an architecture.
We reduce the latency of synthetic L1-to-L2 and L2-to-memory traffic, the two most important traffic types for cache coherency traffic, by up to 28\% and 62\% respectively.
For real traffic traces, we reduce the average packet latency for almost all traces and architectures considered (reduced by an 8\% or 18\% on average depending on the configuration of \gls{phys} within a chiplet).


\section{Background}~\label{sec:background}
\section{Background of Cost Estimation} \label{sec:background}
This section first gives a brief overview of classical and learned cost estimation. 
Afterwards, we describe the learning procedure of \lcms and provide a taxonomy that guides our selection of recent \lcms for this study in \Cref{sec:methodology}.

\subsection{Traditional \& Learned Cost Estimation}
\textbf{Traditional Cost Estimation.} 
Precise cost estimates for different plan candidates in a database are crucial for the query optimizer to select optimal plans from a large search space.
Thus, a lot of engineering effort has been spent since the beginning of database development to estimate the execution costs of a query plan.
Most database systems such as MySQL \cite{widenius2002}, Oracle, PostgreSQL, or System R \cite{astrahan1976} use hand-crafted cost models to reason about the execution costs of a query plan.
These models typically provide a cost function for each physical operator in a query plan that estimates its runtime costs according to CPU usage, I/O operations, memory consumption, expected tuples, and random or sequential page accesses.
However, due to the wide variety of data, queries, and data layouts, traditional cost models need to make simplifying assumptions (e.g., independence of attributes).
These often lead to incorrect predictions of the execution cost. 
Consequently, the query optimizer makes sub-optimal decisions that degrade the query performance by increasing its runtime \cite{leis_how_2015}.
%-------------------------------------------------------

\noindent\textbf{Learned Cost Estimation.}
The need to improve prediction accuracy and the rise of machine learning motivated the idea of \lcms. 
The main idea is to approximate the complex cost functions with a learned model.
Generally, a typical model learns from previous query executions to predict execution costs like runtime.
In contrast to traditional cost models, the promise of \lcms is that they can better learn arbitrarily complex functions.
Thus, improved prediction accuracy can be expected in contrast to traditional approaches based on simplifying assumptions.
Overall, the higher accuracy is expected to lead to a selection of query plans with improved query performance.

%-----------------------------------------
\subsection{Learning Procedure of \lcms}
For our study, we look at effects that also result from the learning procedure of \lcms.
As such, we briefly review the traditional procedure as depicted in \Cref{fig:learning_procedure} to provide the necessary background:
\circles{A}~At first, a workload generator is used to create a large set of randomized, synthetic SQL-Strings that involve a variety of representative query properties such as filter predicates, joins, or aggregation types.
\circles{B}~These queries are executedses (e.g., an airline or movie database) to collect the actual costs of queries.
An important aspect here is that training procedures of many \lcms leads to biases in the dataset due to timeouts and pre-optimized queries, as discussed later.
\circles{C}~Next, various information is extracted from the workload execution.
Most importantly, the physical query plans are extracted, which serve as input to cost models.
In addition to physical plans, \lcms require different information, such as data characteristics like histograms or sample bitmaps.
\circles{D}+\circles{E}~Finally, the workload (i.e., plans and runtime) is then split for training and testing the \lcms. 

\begin{figure*}
    \centering
    \includegraphics[width=\linewidth]{./figures/training_procedure.pdf}
    \caption{
    Learning procedure of \lcms. 
    \circles{\textsc{A}} Generation of synthetic training queries. \circles{\textsc{B}} Query execution on training databases. 
    \circles{\textsc{C}} Feature (query plans, data characteristics, and sample bitmaps) and label (query runtimes) extraction to generate the training and test dataset. 
    \circles{\textsc{D}} Training of the \lcm with supervised learning. 
    \circles{\textsc{E}} Evaluation of the \lcm against unseen test data.}
    \label{fig:learning_procedure}
\end{figure*}

%------------------------------------------------------
\subsection{Taxonomy of \lcms} \label{subsec:taxonomy}
\lcms developed in the last years differ in various dimensions.
This section provides a brief taxonomy of recent \lcms to structure the different methodological approaches. 
This taxonomy will guide the selection of \lcms that we use in this study and ensure that we cover the different methodologies to analyze how they affect the ability of \lcms to support query optimization.

\noindent\textbf{Input Features.}
The first crucial dimension is the input features that a \lcm learns from.
The input features are extracted from the executed workloads (cf. \Cref{fig:learning_procedure}\circles{C}).
The query plan and the underlying data distribution need to be modeled so that a \lcm is informed to make reasonable predictions about the execution costs, which in turn affects query optimization, as we will show.
However, \lcms make use of different information for cost estimation.

\begin{enumerate}[leftmargin=*, nosep]
\item \textbf{SQL-String vs. Query Plans}: 
Some of the first models rely on the SQL string to describe a query, as it gives insights about the tables, predicates, and joins. 
However, details of the execution plan, such as physical operators or the order of joins, are not described there. 
Thus, most \lcms utilize the physical query plan, which includes the operators (e.g., scans, joins) and physical operator types  (e.g., nested loop vs. hash join).
As we will see later, this is fundamental for query optimization.

\item \textbf{Cardinalities}:
Intermediate cardinalities are an important input signal for the overall cost of a plan as they denote the number of tuples an operator needs to process \cite{leis_how_2015}.
Thus, many \lcms leverage intermediate cardinalities as input features, which are either annotated by the databases' cardinality estimator or obtained through an additional learned estimator from related work \cite{hilprecht2020deepdb, kipf2019, yang2020}.
While some \lcms also ignore cardinalities as input for cost prediction, we show in our study that they, in fact, improve the usefulness of cost estimates from \lcms for various query optimization tasks.

\item\textbf{Data Distribution}:
Another helpful factor in estimating cost is understanding the data distribution in the base tables, especially if no cardinalities are used.
For instance, the fact of how many distinct values exist in a column might influence the efficiency of physical operators (e.g., hash join). 
As such, some \lcms use data distribution represented as database statistics and histograms or sample bitmaps (which we explain later) from the base tables as inputs.
However, as we will show in our study, their effect on query optimization tasks remains unclear.

\item \textbf{Cost Estimates}: 
Finally, some of the most recent \lcms even leverage the cost estimates provided by a classical cost estimator as an input feature, which serves as a strong input signal.
This idea renders these \lcms to \textit{hybrid} as they combine a traditional cost model with a learned approach.
The study shows that this provides significant benefits.
\end{enumerate}

%-----------------------------------------------------
\noindent \textbf{Query Representation.}
Many \lcms use model architectures use a graph-based representation to encode query plans as input to the models\footnote{The graph-based representation of the queries refers to the fact whether a model leverages the query graph structure and not to the model learning architecture itself.}.
These approaches thus explicitly leverage information about the order (parent-child relationships) of operators in plans.
However, other \lcms \cite{kipf2019, akdere2012} represent a query plan (or the SQL string) as a flat vector of fixed size without modeling the operator dependencies, which we refer to as \textit{flat} representation in this paper.
While intuitively, capturing the structure and not using a flat representation should be beneficial for \lcms, the results of using graph structure in this study are not that clear. 

%-------------------------------------------------------
\noindent \textbf{Database Dependency.}
Furthermore, an important aspect is whether \lcms can generalize to unseen databases (i.e., a new set of tables) or not.
\textit{Database-agnostic} \lcms were designed \cite{hilprecht2022, zibo_liang_dace_2024} to enable cost predictions for unseen databases that were not part of the training data.
This approach has the advantage of directly providing results without requiring database-specific training data. 
In contrast, \textit{database-specific} \cite{sun2019, zhao2022, marcus2019} models cannot generalize for unknown databases.
For this study, an interesting question is if one of these classes is better suited to support query optimization tasks as database-specific can better adapt to one single database while database-agnostic models can generalize better.

%----------------------------------------
\noindent \textbf{Model Architecture.}
Finally, the presented \lcms differ largely in their learning approach.
Various learning architectures were proposed, including decision trees, tree-structured neural networks, neural units, graph neural networks, and transformer architectures.
While different architectures show different results on the cost estimation tasks, it is still open to see which architecture provides the best results for query optimization.
%----------------------------------------

\section{Technical Overview}~\label{sec:overview}





In this section, we describe our problem statement, review SVS~\cite{shared-val-seq-23}, and present a high-level overview of $\pname$.

\subsection{Problem Statement}~\label{sec:problem-statement}
We investigate protocols for the problem of executing cross-rollup transactions (CRTs) atomically. 
In this paper, we use ``rollups'' to refer to ``zk-rollups,'' systems that use SNARKs to prove the correctness of transaction execution.


\myparagraph{Assumptions, Threat Model, and Goals}
 In this work, we consider the case of $2$ rollups, which reside on distinct L1 chains. For ease of exposition, we assume the participating rollups are operated by the same (shared) Executor. Later (see~\cref{sec:discussion}), we explain how our protocol can serve for the case of rollups with \textbf{different Executors}. 
  We adopt the standard assumption that participating L1 chains are live and safe (informally, liveness means the blockchain keeps processing transactions, and safety means that the finalized blocks are consistent cross replicas; see, e.g.,~\cite{shiFoundations2020} for formal definitions.)
 We consider a computationally bounded adversary which can corrupt the Executor in order to break atomicity of users' cross-rollup transactions; he cannot, however, tamper with user-signed transactions. We assume the Executor maintains the rollup's liveness, i.e., he is always online, processing transactions on L2 and submitting state roots on L1.~\footnote{This is a standard assumption for existing rollups, which employ stake-based incentive mechanisms and replace misbehaving Executors.} 

 The goal is to allow a user to execute a sequence of transactions spanning multiple rollups in an atomic (i.e., all-or-nothing) manner while preserving the ordering of the transactions within each rollup.
 We aim to design a \textbf{C}ross-\textbf{R}ollup \textbf{A}tomic \textbf{T}ransactions (CRAT) protocol that \textbf{(a)} is as \textbf{secure} as existing zk-rollups, \textbf{(b)} is \textbf{efficient} in terms of layer 1 transactions, \textbf{(c)} places \textbf{minimal or zero trust} on any third-party, and \textbf{(d)} is \textbf{practical} in terms of on-chain (L1) gas usage.


\myparagraph{Defining Cross-Rollup Transaction (CRT)}
For the rest of this paper, we use the term ``action'' to mean a transaction executed on a single chain. Looking ahead, without formal definition and proof, promising ideas proposed in~\cite{shared-val-seq-23} are in fact insecure. 
Pinning down a formal model and proving security for $\pname$ is a non-trivial technical challenge. 
For the purposes of this overview only, we use the following simpler notations. An \emph{action} can be represented using the format specified by Ethereum Virtual Machine (EVM)~\cite{eth-evm}, i.e., 
$a = (\addr, \cdata)$, where $\addr$ is the smart contract address and $\cdata$ contains the function identifier and arguments.
Then a cross-rollup transaction (CRT) can be viewed as an ordered sequence of actions $ \crt = [a_i]_{i \in [n]}$.






\subsection{Building Block: Trigger-Action Paradigm}~\label{sec:overview-svs}
The state-of-the-art proposal is Shared Validity Sequencing (SVS)~\cite{shared-val-seq-23} and variants (e.g., \cite{espresso-circ-24}).
SVS aims to enable an action $a_1$ on $\rol{1}$ to ``trigger'' a call to action $a_2$ on another rollup $\rol{2}$ (and vice versa and potentially recursively) in an atomic fashion.
SVS assumes that both rollups reside on the same layer 1 and share the same Validator smart contract (which limits its practicality).
To achieve atomicity, the consistency between $a_1$'s input to $a_2$ (which is stored on $\rol{1}$) and $a_2$'s execution (which takes place on $\rol{2}$) must be verified by layer 1. SVS introduced a special layer 2 smart contract to facilitate efficient consistency checks using Merkle Trees.
Specifically, on all rollups $\rol{i}$, a ``general system contract'' $\gsc_i$ is deployed and trusted by all users. $\gsc_i$ exposes two main functions: $\trig(\cdot)$ and $\act(\cdot)$~\footnote{The $\gsc$ function $\act(\cdot)$ is not to be confused with the term ``action'' which represents a blockchain transaction.}. $\gsc$ stores two merkle trees, $\ttree_i$ and $\atree_i$, whose roots we denote by $\troot{i}$ and $\aroot{i}$, respectively.

\begin{figure}
    \centering
    \includegraphics[scale=0.5]{figures/svs-gsc-diagram-2-actions-crop.pdf}
    \caption{SVS~\cite{shared-val-seq-23} trigger-action paradigm. Here ``$\mathsf{CALL}(f)$'' denotes that the (smart contract) function $f$ will be executed.}
    \label{fig:orig-svs-trigger-action}
\end{figure}

\label{svs-crt-lifecycle} 
In~\cref{fig:orig-svs-trigger-action}, we illustrate the SVS workflow assuming a user who wishes to execute action $a_1 = (\addrr{1}, \cdataa{1})$ on $\rol{1}$ and $a_2 = (\addrr{2}, \cdataa{2})$ on $\rol{2}$ atomically \footnote{For instance, the pair $(a_1,a_2)$ could be a burn-mint token transfer.}.
She first ``bundles'' $(a_1,a_2)$ by creating an \emph{augmented} action $a_1'$ that executes $a_1$ and calls $\gsc_1.\trig(\addrr{2}, \cdataa{2})$ (step 1); then she sends $a_1'$ to the Executor, who in turn executes $a_1'$ on $\rol{1}$ (step 2), causing the tuple $(\addrr{2}, \cdataa{2})$ to be inserted into $\ttree_1$ (step 3). The Executor executes $\gsc_2.\act(\addrr{2}, \cdataa{2})$ on $\rol{2}$ (step 4), and as a result the same tuple $(\addrr{2}, \cdataa{2})$ is inserted into $\atree_2$ (step 5) and $a_2$ is executed.
Now atomicity boils down to a simple consistency check of whether $\troot{1} \overset{?}{=} \aroot{2}$ and $\troot{2} \overset{?}{=} \aroot{1}$, which the L1 $\vsm{}$ can efficiently check.

To see why atomicity holds, note that the execution of $a_1'$ implies (1) $a_1$ is executed, and (2) $(\addrr{2}, \cdataa{2})$ is inserted into $\ttree_1$. Either both (1) and (2) take effect or none does. Similarly, the execution of $\gsc_2.\act(\addrr{2}, \cdataa{2})$ implies (1') $a_2$ is executed on $\rol{2}$, and (2') the same triple $(\addrr{2}, \cdataa{2})$ is inserted into $\atree_2$. $\gsc_2$ ensures that both take place or none does.
Due to the collision resistance of Merkle Trees, the atomicity check above ensures that if (2) and (2') took place, so did (1) and (1').

\subsection{Technical Challenges and Our Solutions}~\label{sec:svs-challenges-and-solutions}



\begin{figure*}
    \centering
    \includegraphics[scale=0.5]{figures/svs-attack-diagram-new-crop.pdf}
    \caption{Serializability attack on SVS~\cite{shared-val-seq-23}. A malicious Executor (shown on the right in red) can execute $a_3$ before $a_1'$, flipping the order of steps (2,3) with steps (6,7). Since $a_1'$ only modifies $\ttree_1$ and $\gsc_1.\act(a_3)$ only modifies $\atree_1$, the two $\gsc$ contracts result in the same state as in the honest execution (shown in the middle), thus passing the consistency check.}
    \label{fig:svs-serializability-attack}
\end{figure*}

Although simple and elegant, SVS~\cite{shared-val-seq-23} has two major problems and fixing them requires careful modeling and protocol design. 
First, serializability -- and by extension, atomicity -- breaks down for CRTs of \emph{length greater than $2$}, as illustrated in \cref{fig:svs-serializability-attack}.
Consider $\crt := [a_1, a_2, a_3]$ of length $3$, where $a_1$ and $a_3$ are to be executed on rollup $1$ and $a_2$ on rollup $2$. A natural attempt to do this with SVS is to create augmented actions $(a_1', a_2')$ as above: $a_2'$ is the same as $a_2$ but also triggers $a_3$, and now $a_1'$ is the same as $a_1$ but also triggers $a_2'$ (step 1 in \cref{fig:svs-serializability-attack}).
The intention is that the atomic execution starts with $a_1'$ (steps 2,3), then $a_2'$ (steps 4,5) and $a_3$ (steps 6,7). 
However, a \textit{malicious} Executor can easily mount a re-ordering attack to break atomicity; in our example, he can execute $a_3$ before $a_1'$ on rollup $1$. 
Doing so still passes the consistency check as above and the validity proof verifications because each action was honestly executed. In practice, this can be a serious attack; for example, if $a_1$ and $a_3$ contain arbitrage logic, changing their execution order can cause the user to miss the arbitrage opportunity or end up worse off. 








Second, SVS requires all rollups to share the same Executor and reside on the same chain, while many rollups do not satisfy these requirements. Moreover, the Executor is required to submit each new pair of state roots \emph{together} in the same L1 transaction, along with the extended validity proofs described above. Instead, CRAT protocols should ideally be \textit{flexible} and allow rollups to reside on \textit{distinct L1 chains}, as well as be operated by distinct, possibly distrusting Executors.
To overcome this challenge, the Two-Phase Commit (2PC) paradigm is a natural choice and indeed the one we adopt; however, due to the distinct and heterogeneous nature of the entities involved -- two VSM contracts and an untrusted Executor -- applying the 2PC framework tuns out to be highly nontrivial. 

\subsubsection{Our Solutions}
We present $\pname$, along with the careful programming model and formalism needed to tackle the major flaws not addressed by the SVS~\cite{shared-val-seq-23} architecture. 

\myparagraph{Restoring serializability}
We present our new design in two steps. 
In the first step, we fix serializability in the simpler \emph{chain CRT} programming model, in which each action is limited to trigger at most one other action. However, the chain CRT model lacks the necessary expressiveness for certain applications, such as flash loans. We concretely motivate the necessity of such higher expressiveness in~\cref{subsec:application-xfl}. In the second step, we propose a more flexible programming model and modify our serializability solution to support this model.

In more detail, 
we introduce the notion of {\em sessions}. We carefully modify $\gsc$ to track CRT instances by assigning a unique session nonce and maintaining an entry nonce and an active flag.  The source action $(a_1)$ starts a new CRT session by calling a new entry point function in $\gsc$, which increments the session and entry nonces and sets the flag. Subsequent $\act(\cdot)$ calls verify that the session nonce matches the current or next expected value, ensuring consistent use of the same nonce across actions. The VSM contract checks that the sum of the two entry nonces equals the session nonce. 




\myparagraph{Supporting DAG CRTs}
We introduce the more expressive Directed Acyclic Graph (DAG) CRT programming model. In this model, each action can trigger multiple subsequent actions, which can in turn trigger additional actions, forming a directed acyclic graph (hence the name ``DAG'').  To integrate this intended functionality within our serializability solution, we represent a DAG CRT as a sequence of actions, similar to chain CRTs, but with the key distinction that each action includes multiple sub-actions to be executed, along with the ability to trigger additional sub-actions.
To support the more expressive DAG CRTs, we modify  $\gsc.\act(\cdot)$ to accept an \emph{array} of triggered actions, instead of a single action. Further, we ensure that all trigger calls and triggered actions within the same transaction share the same nonce value, we repurpose the entrypoint function to be the global entrypoint of $a_1$, all while preserving the previous session-tracking logic.


\myparagraph{Supporting distinct L1 chains}
While the above changes restore serializability, we draw on 2-Phase Commit (2PC) protocols from the distributed computing literature~\cite{bernstein-concurrency-databases-1986} to support rollups with $\vsm{}$s on distinct L1 chains.
The key is to ensure that two VSM contracts accept new state roots in an all-or-nothing fashion. Suppose the Executor is ready to submit a pair of state digests -- one to each VSM -- which were produced by at least one CRT. \done%
In the pre-commit round, one VSM is selected as the leader; it verifies and pre-commits to the submitted state digest and transitions to a paired status; the non-leader VSM then follows the same process and pre-commits only after the leader VSM. In the commit round, the two VSMs commit to the new state digest in the same order after verifying relevant evidence from the other side.

For ease of exposition, we present $\pname$ in the chain-CRT model in \cref{sec:xevm,sec:2pc-trustless}, and present modifications to support the DAG-CRT model in~\cref{sec:dag-crt-protocol}.





\subsection{Overview of \pname}






\begin{figure}
    \centering
    \includegraphics[scale=0.35]{figures/crat-diagrams-crop.pdf}
    \caption{$\pname$ Overview}
    \label{fig:protocol-diagram}
\end{figure}

We describe the end-to-end workflow of $\pname$, illustrated in~\cref{fig:protocol-diagram}, starting from the user CRT and all the way until the pair of new state digests are accepted by the VSM contracts. The user starts by creating a single transaction which encodes her intended CRT by making the appropriate calls to $\gsc.\trig(\cdot)$, and sends her transaction to the Executor (step $1$). The Executor keeps executing received transactions off-chain (phase 1 below) and, at regular intervals, he submits transaction batches -- along with the state digest they produce -- to the VSM on L1 (phase 2 below).

\parhead{Phase 1.} The Executor executes the transaction batch off-chain following the \emph{\textbf{C}ross-EVM (XEVM)} sub-protocol of $\pname$. The off-chain execution yields a new pair of state digests, and the Executor also produces validity proofs, one per rollup. This phase corresponds to step $2$ in~\cref{fig:protocol-diagram} and addresses the first challenge outlined in~\cref{sec:svs-challenges-and-solutions}. 

\parhead{Phase 2.} The Executor drives the two VSMs to perform the \emph{2PC component} of $\pname$ (steps $3-6$ in~\cref{fig:protocol-diagram}). Each VSM can be in either ``free'' or ``paired'' status, and initially both are in free status.
During the \textit{pre-commit} round, one of the two VSMs is deterministically chosen as the leader; suppose it is $\vsm{1}$. The Executor first drives $\vsm{1}$ to pre-commit (step $3$) by submitting the new state digest, validity proof, and evidence that $\vsm{2}$ is in free status; 
$\vsm{1}$ in turn verifies the proof and the evidence, accepts the digest as temporary, and switches to paired status.
Next, the Executor drives $\vsm{2}$ to pre-commit (step $4$), and $\vsm{2}$ also switches to paired status after verifying that $\vsm{1}$ is in paired status.

During the \textit{commit} round, the Executor first drives $\vsm{1}$ to commit
(step $5$) by submitting evidence that (a) $\vsm{2}$ is in paired status
and (b) the roots of the $\gsc$ contracts match. $\vsm{1}$ in turn verifies the evidence, accepts its temporary state digest as final, and stores its ``commit'' decision under a unique identifier of this 2PC instance. Next, the Executor drives $\vsm{2}$ to commit (step $6$), and $\vsm{2}$ also accepts its temporary state digest as final after verifying that that $\vsm{1}$ has issued a ``commit'' decision.

\myparagraph{Abort case} During the second round of 2PC, $\vsm{1}$ (the leader) can also be driven to abort. After verifying relevant evidence that $\vsm{2}$ has not switched to paired status, $\vsm{1}$ stores an ``abort'' decision (under the 2PC identifier), discards its temporary state digest, and ``rolls back'' to the original state digest. Similarly to the commit case, $\vsm{2}$ will also abort (if ever pre-committed) upon verifying $\vsm{1}$'s abort decision.

\myparagraph{Incorrect Evidence} The workflow above describes an honest execution. If the Executor submits incorrect evidence to the $\vsm{}$ during pre-commit or commit, verification checks will fail, causing the $\vsm{}$ to revert the transaction. In the pre-commit round, the $\vsm{}$ would remain in free status, while in the commit round, it would remain locked in paired status until proper commit/abort evidence is provided. Allowing a $\vsm{}$ to unilaterally abort is insecure, as a malicious Executor could drive one $\vsm{}$ to commit and force the other to abort. This highlights the importance of a leader-based approach, where one $\vsm{}$ decides and the other follows, as well as the non-trivial nature of applying 2PC in this setting.


\begin{remark}[VSM Communication via Bridges]~\label{rem:bridge-assumption}
    When we say a VSM ``verifies evidence'' about the other VSM's state, we mean that it verifies a state membership proof against L1 block headers relayed by a trustless blockchain bridge, such as \cite{xiew-zkbridge-22}. We further assume that a VSM only considers bridge-relayed state digests that are final (2 epochs in Ethereum~\cite{buterin-gasper-2020}), eliminating the possibility of any reorg~\cite{eth-org-reorg}.
\end{remark}

\begin{remark}[Setup]~\label{rem:vsm-setup}
    It is crucial that each VSM be aware of its rollup's GSC contract and of the other VSM. After deployment, we assume a secure, one-time setup phase at the end of which, each VSM has hardcoded the address of its rollup's GSC contract as well as the address of the other VSM. Such setups are common practice and outside this paper's scope.
\end{remark}



\parhead{Application: Cross-Rollup Flash Loan}~\label{subsec:application-xfl}
A cross-rollup flash loan involves the following seven steps:
\noindent\textbf{1. Borrow} the desired amount from a flash loan pool (on $\rol{1}$).
\noindent\textbf{2. Burn} the amount (on $\rol{1}$).
\noindent\textbf{3. Mint} the amount (on $\rol{2}$).
\noindent\textbf{4. Arbitrage} (on $\rol{2}$). 
\noindent\textbf{5. Burn} the post-arbitrage amount (on $\rol{2}$).
\noindent\textbf{5. Mint} the post-arbitrage amount (on $\rol{1}$).
\noindent\textbf{7. Repay} the borrowed amount and retain the profit (on $\rol{1}$). Typically, a Token contract supports burn-and-mint, and a User contract contains the logic for the user’s actions (e.g., arbitrage).

We now discuss why cross-rollup flash loans highlight the need for the more expressive DAG CRT model. In the execution of action $a_1$, which includes steps 1 and 2 on $\rol{1}$, both step 3 (mint) and step 4 (arbitrage) must be triggered for execution on $\rol{2}$. In theory, steps 3 and 4 could be combined into a single action $a_2$, and having $a_1$ trigger $a_2$ would eliminate the need for multiple trigger calls. However, this approach presents a challenge in determining which contract will perform the bundling and trigger $a_2$. Specifically, step 3 relates to the Token contract, while step 4 is handled by the User contract. Consequently, the Token contract should (internally) trigger the mint operation upon burning, and the User contract should trigger the arbitrage, resulting in two trigger calls at the end of $a_2$. The Executor should then pick up both trigger calls, bundle them together, and pass them as arguments to single $\gsc.\act(\cdot)$ invocation. These practical desiderata underscore the necessity for a more flexible and expressive programming model, such as our DAG CRT model.












\section{Formal Foundations}~\label{sec:foundations}
\subsection{Safety}
The SVS trigger-action paradigm~\cite{shared-val-seq-23} serves as a \emph{programming model} for cross-rollup transactions, in addition to providing an elegant atomicity check mechanism. In the simple example where $a_1$ triggers $a_2$, causing the Executor to execute $\act{}(a_2)$, it is as if $a_1$ \emph{passes} control to $a_2$, which in turn is not expected to return to its caller. This programming model is akin -- in spirit -- to the \emph{continuation-passing style (cps)} model~\cite{sussman-scheme-1998, appel-continuations-2007}, as well as to the widely used asynchronous event driven architecture~\cite{jansen-ibm-event-driven-2020, rah-even-driven-microservices-2022, ghaemi-pubsub-blockchain-2021}. 
We first define the simpler chain-CRT model, which is similar to the CRT abstraction in~\cite{lu-atomic-cc-interactions-24}, and then present the more involved DAG-CRT model.


\parhead{Chain CRTs}~\label{sec:chain-crt-model}
We model an \emph{action} as a $3$-tuple $a = (\desc{}, \code{}, \descnext{})$. Here, $\desc{} = (\rol{}, \addr{}, \cdata{})$ is the action's \emph{description}; $\rol{}$ is the target rollup (L2)~\footnote{We assume the $\rol{}$ identifier also determines the underlying L1.}, $\addr$ is the smart contract address, and $\cdata$ contains the function identifier and arguments.
Next, $\code{}$ is the \emph{raw code trace} of the action, i.e., it represents the actual opcode sequence executed, \emph{without} the triggering logic. Finally, $\descnext{}$ is \emph{the description of another action} to be executed, and its format is identical to $\desc{}$. Although not explicit, we assume that $a.\desc{}$ uniquely determines both $a.\code{}$ and $a.\descnext{}$; this correspondence is enforced by EVM rules.~\footnote{Similarly, $(\addr{}, \cdata{})$ determines the EVM opcodes to be executed.} We will overload notation and use $a.\rol{}$ to mean $a.\desc{}.\rol{}$. The field $a.\descnext{}$ can be $\nul$, in which case it is ignored.

\done{This model assumes the original SVS~\cite{shared-val-seq-23} trigger-action paradigm, restricted to allow each transaction to trigger only a single other action.}

\begin{remark}
    Looking ahead, $\descnext{}$ is precisely the field inserted in $\ttree$, while $\desc{}$ is the field inserted in $\atree$.  
\end{remark}


We define a \emph{chain CRT} as a sequence $\ccrt := [a_i]_{i \in [n]}$ which satisfies $a_n.\descnext{} = \nul$ and for all $1 \leq i \leq n-1$, \; $a_i.\descnext{} = a_{i+1}.\desc{} \neq \nul$. Since each action in the sequence is modeled to trigger the next action, we notice that each $a_i$ fully determines the sub-CRT $\ccrt [i\colon \! ]$. This means that the entire CRT $\ccrt$ can be represented by the single source action $a_1$, which is exactly the action the user will submit to the Executor. Even though $a_1$ is a compact representation of $\ccrt$, we still represent CRTs by the full sequence of actions to facilitate analysis. 


A transaction \emph{batch} is an ordered list of actions that the Executor submits along with the new state digest on L1. An action can be either \emph{local}, i.e., confined within the rollup, or \emph{non-local}, i.e., part of a CRT. 
We say a \emph{state digest} is \emph{local} if it has been produced by a batch consisting of only local actions. Otherwise we say the state digest is \emph{non-local}. We write $a_1 \underset{\batch{}}{\prec} a_2$ to mean that $a_1$ appears before $a_2$ within $\batch{}$.
We define the following predicate:
\begin{itemize}[leftmargin=*]
    \item $\actexec{}(\dst{}, \dstt{}, \batch{}, a) = \true$ if $a \in \batch{}$ \textbf{and} executing $\batch{}$ causes the state digest to transition from $\dst{}$ to $\dstt{}$.
\end{itemize}

We can now define the atomic execution predicate $\atomexec(\ccrt, \{\rol{j}, \dst{j}, \dstt{j}, \batch{j}\}_j)$, which covers an entire CRT instance $\ccrt := [a_i]_{i \in [n]}$ and is defined with respect to the set of rollups $\rolset = \cup_{i \in [n]} \{ a_i.\rol{} \}$ and their state digest pairs and batches.
This predicate evaluates to $\mathsf{True}$ if and only if (1) occurs or both (2a) and (2b) occur:
\begin{enumerate}[leftmargin=*]
    \item For all $j \in [|\rolset|]$ and for all $i \in [n]$ s.t. $a_i.\rol{} = \rol{j}$, $\actexec{}(\dst{j}, \dstt{j}, \batch{j}, a_i) = \false$.
    \item (a) For all $j \in [|\rolset|]$ and for all $i \in [n]$ s.t. $a_i.\rol{} = \rol{j}$,
            $\actexec{}(\dst{j}, \dstt{j}, \batch{j}, a_i) = \true$.
            
        (b) For all $j \in [|\rolset|]$ and for all $i_1, i_2 \in [n]$ s.t. $a_{i_1}.\rol{} = a_{i_2}.\rol{} = \rol{j}$ and $i_1 < i_2 $, it holds that $a_{i_1} \underset{\batch{}}{\prec} a_{i_2}$.
\end{enumerate}

\noindent We now proceed to formally define atomicity for chain CRTs.

\begin{definition}[CRT Atomicity]~\label{def:forward-atom-chain-crt}   
    Let $\ccrt := [a_i]_{i \in [n]}$ be a CRT, 
    and let $\mathcal{R} = \cup_{i \in [n]} \{ a_i.\rol{} \}$ be the set of associated rollups, for some $n \geq 2$. For each $j \in [|\mathcal{R}|]$, let $\dst{j}$ be the \emph{current} state digest of $\vsm{j}$. Also let $(\dstt{j}, \batch{j})$ be the (state digest, transaction batch) pair which is \emph{immediate next accepted} after running CRAT protocol $\protocol$. We say $\protocol$ satisfies \emph{atomicity} if 
    \begin{align*}
        \atomexec(\ccrt, \{\rol{j}, \dst{j}, \dstt{j}, \batch{j}\}_j) = \mathsf{True}.
    \end{align*}
\end{definition}




\parhead{DAG CRTs}~\label{sec:dag-crt-model}
Similar to chain CRTs, a DAG-CRT action is a tuple $a_i := (\desc{}, \code{}, \descnext{})$. Unlike chain-CRTs, $a.\desc{} = [sa_j.\desc{}]_j$ now is an array of \textbf{s}ub-\textbf{a}ction descriptions (hence ``$sa$''), $a.\code{}$ represents the concatenation of the $[sa_j.\code{}]_j$ components, and $a'.\descnext{} = [\descnext{}[k]]_k $ is an array of descriptions of triggered sub-actions. Similar to chain CRTs, $a.\desc{}$ uniquely determines both $a.\code{}$ and $a.\descnext{}$. The array $\descnext{}$ can be empty, in which case it is ignored. We define a DAG-CRT as a sequence $\dagcrt := [a_i]_{i \in [n]}$ which satisfies $|a_n.\descnext{}| = 0$ and for all $1 \leq i \leq n-1$, $a_i.\descnext{} = a_{i+1}.\desc{}$. Note the equality is between \emph{arrays}.


    

We can now repurpose all predicates from~\cref{sec:chain-crt-model} for DAG CRTs. Under the repurposed predicate $\atomexec{}(\dagcrt, \cdot)$), we can also re-purpose~\cref{def:forward-atom-chain-crt} to work for DAG CRTs. We omit re-stating these repurposed versions since the only difference is the use of $\dagcrt$ instead of $\ccrt$.


\subsection{Liveness and Efficiency}

\myparagraph{Liveness} A CRAT protocol satisfies liveness if the participating rollups maintain liveness, as discussed in \cref{sec:problem-statement}.

\myparagraph{Efficiency}
The key efficiency metric for cross-rollup transactions (CRTs) is latency, which refers to the time delay between submitting a CRT instance and its successful commitment or abortion. Measuring time complexity in this setting is challenging due to the heterogeneous nature of participating rollups and their differing finality rules. However, L1 transactions are the bottleneck, and we define a "round" as the duration of the slowest layer 1 transaction, based on the blockchain that takes the longest to finalize a transaction. CRAT protocols should preserve the constant round finality of existing rollups, and we capture this requirement in~\cref{def:efficiency}. 





\begin{definition}[Efficiency]~\label{def:efficiency}
    Let $\protocol$ be a CRAT protocol and let $\ccrt = [a_i]_{i \in [n]}$ be a chain-CRT of length $n$. We say $\protocol$ is \emph{efficient} if it completes in $O(1)$ rounds.
\end{definition}

Similarly, we can repurpose \cref{def:efficiency} for DAG CRTs by requiring that the protocol completes in $O(1)$ rounds regardless of the CRT length (number of actions) as well as the total number of sub-actions.



\done%






\section{\pname: Off-Chain Execution -- XEVM}~\label{sec:xevm}
\newcommand{\startsession}{\textproc{startSession}}
\newcommand{\checksessionid}{\textproc{checkSessionID}}
\newcommand{\trigger}{\textproc{trigger}}
\newcommand{\triggerinternal}{\textproc{triggerInternal}}
\newcommand{\action}{\textproc{action}}

In this section, we present XEVM, our modified execution engine for rollups. First, we extend the General System Contract ($\gsc$) introduced in~\cite{shared-val-seq-23} so that it securely supports chain CRTs of arbitrary length. Next, we present the full Executor algorithm specification for the XEVM component. 


\subsection{General System Contract}


\begin{algorithm}[ht]
    \small
    \caption{General System Contract}
    \label{alg:gsc-single-trigger}
    \begin{algorithmic}[1]

    \Function{\textcolor{chain}{startSession}}{\textcolor{chain}{$\addr{}, \cdata$}}
            \State \textcolor{chain}{ \require ($\Not \; \acalled{}$) }
            \State \textcolor{chain}{ $\enonce{} \gets \enonce{} + 1$ }
            \State \textcolor{chain}{ $\snonce{} \gets \snonce{} + 1$ }
            \State \textcolor{chain}{ $\sactive{} \gets \true$ }
            \State \textcolor{chain}{ $\tcalled{} \gets \false$ }
            \State \textcolor{chain}{ \Call{triggerInternal}{$\msgsender, \addr{}, \cdata$} }
    \EndFunction
    \Function{\textcolor{chain}{checkSessionID}}{\textcolor{chain}{$\sid{}$}}
        \If{\textcolor{chain}{ $\sid{} = \snonce{}$ } } 
            \State \textcolor{chain}{ \require ($\sactive{}$) }
        \ElsIf{\textcolor{chain}{ $\sid{} = \snonce{} + 1$} }
            \State \textcolor{chain}{ $\sactive{} \gets \true$ }
            \State \textcolor{chain}{ $\snonce{} \gets \snonce{} + 1$ }
        \Else
            \; \textcolor{chain}{ \textbf{Revert} }
        \EndIf
    \EndFunction
    \Function{triggerInternal}{$\sender, \addr{}, \cdata$}
        \State  \textcolor{chain}{ \require ($\Not \; \tcalled{}$) \textcolor{comment}{\Comment{Only one trigger per tx}} }
        \State \textcolor{chain}{$\tcalled{} \gets \true$}
        \State \textcolor{chain}{$\sid{} \gets \snonce{}$}
        \State $\tnonce{} \gets \tnonce{} + 1$
        \State $ \msghash \gets \hash(\sender, \addr{}, \cdata, \tnonce{}, \textcolor{chain}{\sid{}})$
        \State \textbf{emit T}($\sender, \addr{}, \cdata, \tnonce{}$)
        \State $\ttree.\insertt(\msghash)$
    \EndFunction
    \Function{trigger}{$\addr{}, \cdata$}
        \State \textcolor{chain}{ \Call{triggerInternal}{$\msgsender, \addr{}, \cdata$} }
    \EndFunction
    \Function{action}{$\sender, \addr{}, \cdata, \textcolor{chain}{\sid{}}$}
        \State \textcolor{chain}{$\acalled{} \gets \true$}
        \State \textcolor{chain}{\Call{checkSessionId}{$\sid{}$}}
        \State \textcolor{chain}{$\tcalled{} \gets \false$}
        \State (status, \_) $\gets \addr{}.\call(\cdata)$ \textcolor{comment}{\Comment{execute triggered action} }
        \State \require (status)
        \State $\anonce{} \gets \anonce{} + 1$
        \State $ \msghash \gets \hash(\sender, \addr{}, \cdata, \anonce{}, \textcolor{chain}{\sid{}}) $
        \State $\atree.\insertt(\msghash)$
        \If{\textcolor{chain}{ ($\Not \; \tcalled{}$) } }
            \textcolor{chain}{ $\sactive{} \gets \false$ }
        \EndIf
        \State \textcolor{chain}{$\acalled{} \gets \false$}
        
    \EndFunction
    \end{algorithmic}
\end{algorithm}



 In this section, we present our modified General System Contract ($\gsc$) which handles chain CRTs. \cref{alg:gsc-single-trigger} fully specifies our $\gsc$ contract; we highlight our contributions with \textcolor{chain}{blue color} on top of SVS~\cite{shared-val-seq-23}. Besides the Merkle trees $(\ttree, \atree)$ and nonces ($\tnonce{}, \anonce{})$, the $\gsc$ contract maintains two additional nonces $(\snonce{}, \enonce{})$ and the $\sactive{}$ flag to keep track of CRT sessions. Nonces are initialized to zero and the $\sactive{}$ flag to $\false$.

The source action $(a_1)$ launches a new CRT session by calling the new entry point function $\startsession$, passing the $(\addr, \cdata)$ fields of the action it wishes to trigger. $\startsession$ increments both $\enonce{}$ and $\snonce{}$ by one, marks the session as active ($\sactive{} \gets \true$), and calls $\triggerinternal(\msgsender, \addr, \cdata)$. $\triggerinternal$ is an internal function only supposed to be called by $\startsession$. 
$\trigger$ is the function that application smart contracts are supposed to invoke.   


The $\trigger$ function increments $\tnonce{}$, hashes the tuple $(\msgsender, \addr{}, \cdata, \tnonce{}, \sid{})$, and inserts the hash into $\ttree$. Here $\msgsender$ is the address of the caller contract and $\sid{}$ is taken to be the current $\snonce{}$ value. It also emits a Trigger event intended to be listened to by the Executor.  

The $\action$ function takes as input the tuple $(\sender, \addr{}, \cdata, \sid{})$ and can only be called by the Executor. First, it checks that $\sid{}$ matches either $\snonce{}$, which requires that $\sactive{} = \true$ , or $\snonce{} + 1$, signaling a new CRT session. Then, it sets $\xsender{} \gets \sender$ and calls the smart contract method specified by $(\addr{}, \cdata)$. The callee contract can access the public variable $\xsender{}$ which specifies the cross-rollup sender. Next, it increments $\anonce{}$, hashes the tuple $(\sender, \addr{}, \cdata, \anonce{}, \sid{})$, and inserts the hash into $\atree$. If the $\tcalled{}$ flag is raised during the call to $(\addr{}, \cdata)$ (i.e., there is a nested call to $\trigger$), this means that the current CRT session is not finished. Otherwise, this is the end of the current CRT session and it marks $\sactive{} \gets \false$. 



We assume a unique copy of the $\gsc$ contract is deployed on each rollup. Our design guarantees that the $\trigger$ function can only be called either by $\startsession$ or in a nested manner during an $\action$ call, as well as that $\startsession$ can only be called by a source action ($a_1$), and never in a nested manner during an $\action$ call.


\subsection{Executor Specification}
 

In this section, we describe the XEVM component of the Executor, formally specified in~\cref{alg:shared-executor-xevm-chain-crt}. The Executor algorithm manages the execution of both local and non-local transactions across the two rollups. We assume a function $\textproc{evm}$ which takes as input a transaction and executes it on the specified rollup according to EVM rules. The Executor maintains the current state and a state checkpoint for each rollup. To execute a transaction $\tx$ received by a user, he calls $\textproc{entryPoint}(\tx)$, which creates state checkpoints for all rollups before proceeding to call \textproc{xevm}($\tx$). The $\textproc{xevm}(\tx)$ function uses the (assumed) $\textproc{evm}$ function to execute $\tx$ and listens for a trigger event. If there is a triggered transaction $\tx'$, the Executor constructs a wrapper transaction $\tx''$ which calls $\gsc.\action(\cdot)$ with the description of $\tx'$ as argument. It then recursively calls $\textproc{xevm}(\tx'')$, and so on, until all trigger calls have been handled. If any transaction or its triggered transaction fail, this failure propagates up to $\textproc{entryPoint} $, which restores the state of all rollups to their checkpoints.


























\section{\pname: 2PC Protocol}~\label{sec:2pc-trustless}
In this section, we describe the Two-Phase Commit (2PC) protocol between the two $\vsm{}$s and driven by the Executor. We omit formally specifying the Executor's role in the 2PC protocol, since it follows directly from the $\vsm{}$ specification. As such, the entire section focuses on the $\vsm{}$ specification.


The $\vsm{}$ contract consists of three main functions ($\textproc{updateDigest}$, $\textproc{commit}$, and $\textproc{abort}$), shown in \cref{alg:vsm-generic}, which serves as a generic 2PC framework between VSMs. Concrete procedures -- such as leader determination and evidence verification -- are performed by functions $\textproc{IsLocal}, \textproc{VerPreComEvd}, \textproc{VerComEvd}$, and $\textproc{VerAbEvd}$, which we describe in this section and formally specify in \cref{alg:vsm-trustless-p1} in~\cref{apdx:vsm}.

\myparagraph{Bridge abstraction}
Recall from \cref{par:background-bridges} and \cref{rem:bridge-assumption} that we assume a trustless bridge between the underlying L1 chains. In the following, the $\ovsm$ structure contains state attributes of the other $\vsm{}$, which can be verified against the bridge using attribute proofs which we denote by $\attproofs$. 

\begin{algorithm}[ht]
    \small
    \caption{Validator Smart Contract: Generic 2PC}
    \label{alg:vsm-generic}
    \begin{algorithmic}[1] %

        \Function{UpdateDigest}{$\dstt{},  \valproof, \locevd, \pcevd$}
            \State \textbf{Require} $(\pstat = \free)$
            \State \Call{VerValProof}{$\dstt{}, \valproof$}
            \If{\Call{IsLocal}{$\dstt{}, \locevd$}} \; $\dst{} \gets \dstt{}$
            \Else
                \State \Call{VerPreComEvd}{$\dstt{}, \pcevd$}
                \State $\dstemp{} \gets \dstt{}$, \; $\pstat \gets \paired$
            \EndIf
        \EndFunction
        \Function{Commit}{$\cevd$}
            \State \require $(\pstat = \paired)$
            \State \Call{VerComEvd}{$\cevd$}
            \State $\dst{} \gets \dstemp{}$, \; $\pstat \gets \free$
        \EndFunction
        \Function{Abort}{$\abevd$}
            \State \require $(\pstat = \paired)$
            \State \Call{VerAbEvd}{$\abevd$}
            \State $\pstat \gets \free$
        \EndFunction
    \end{algorithmic}
\end{algorithm}


\myparagraph{State} The $\vsm{}$'s state consists of its ID ($\vsmid{}$) and the current rollup state digest ($\dst{}$). We also assume that it knows the address and ID of the VSM that it is connected to (see \cref{rem:vsm-setup}). The $\pstat \in \{\free, \paired \}$ variable is initialized to $\free$. All remaining state variables are specified in~\cref{alg:vsm-trustless-p1}.

\myparagraph{Entrypoint} The $\textproc{updateDigest}$ function takes as input a new state digest ($\dstt{}$), a validity proof ($\valproof$), local evidence ($\locevd$), and pre-commit evidence. First, the $\vsm{}$ verifies $\valproof$ using the (assumed) existing zk-proof verification mechanism (\textproc{VerValProof} function). Next, it parses the local evidence $(\roots, \attproofs) \gets \locevd$ which is used to determine whether $\dstt{}$ is a local or non-local state digest. It parses the claimed roots of the $\gsc$ contract  $(\troot{}, \aroot{},  \troot{}', \aroot{}') \gets \roots$ and verifies $\attproofs$ to check the membership of $(\troot{}, \aroot{})$ (resp. $(\troot{}', \aroot{}')$) against $\dst{}$ (resp. $\dstt{}$). If $\troot{}' = \troot{}$ and $\aroot{}' = \aroot{}$, then $\dstt{}$ is a $\emph{local}$ digest and thus accepted immediately $(\dst{} \gets \dstt{})$. Otherwise, $\dstt{}$ is a $\emph{non-local}$ digest and the $\vsm{}$ proceeds to pre-commit. Note that $\locevd$ additionally contains the $\snonce{}'$ and $\enonce{}'$ variables of the $\gsc$ contract, which are also verified under $\dstt{}$, and stored to be used in the commit round. 

\myparagraph{Pre-commit} If the $\vsm{}$ successfully verifies the supplied pre-commit evidence ($\pcevd$), then it stores $\dstt{}$ as the temporary digest and switches to $\paired$ status.
 In the $\textproc{VerPreComEvd}$ function, the $\vsm{}$ parses the pre-commit evidence $(\ovsm, \attproofs, \dstt{O}) \gets \pcevd$, where $(\ovsm, \attproofs)$ are verified against a bridge-relayed state digest of the other L1 chain. Next, it computes the 2PC instance identifier as $\idx \gets \textproc{GetIndex}(\dst{}, \dstt{}', \ovsm.\dst{}, \dstt{O})$, where $\textproc{GetIndex}(\cdot)$ deterministically orders and hashes the four state digests, so that both VSMs compute the same $\idx$ value. The $\idx$ identifier is stored and used to determine the leader of this 2PC instance. If the $\vsm{}$ is the \textit{leader}, it checks whether $\ovsm.\pstat = \free$. If the $\vsm{}$ is \textit{not the leader}, it checks that the computed index matches the leader VSM's index and that the leader VSM is in $\paired$ status.


\myparagraph{Commit/Abort} If the $\vsm{}$ successfully verifies the supplied commit evidence ($\cevd$), then it accepts the temporary state digest as final ($\dst{} \gets \dstemp{}$) and switches to $\free$ status.
 In the $\textproc{VerComEvd}$ function, the $\vsm{}$ parses the commit evidence $(\ovsm, \attproofs) \gets \cevd$, where $\ovsm$ and $\attproofs$ are first verified similarly to the pre-commit round. Next, it fetches the current 2PC instance identifier $\idx$ and determines the leader of this 2PC instance. If the $\vsm{}$ is the \textit{leader}, it checks whether $\ovsm.\pstat = \paired$ and $\ovsm.\idx = \idx$. It also verifies that the new pair of state digests ($\dstemp{}, \dst{}$) satisfy \emph{atomicity} by checking that (1) the Merkle roots of the $\gsc$ contracts match (i.e., $\troot{}' = \ovsm.\aroot{}'$ and $\aroot{}' = \ovsm.\troot{}'$) and (2) the sum of the two entry nonces is equal to both session nonces. If so, it stores a $\commit$ decision under $\idx$. If the $\vsm{}$ is \textit{not the leader}, it follows the leader VSM's decision under $\idx$. The $\textproc{abort}$ and $\textproc{verAbEvd}$ functions are similar to the commit case; see~\cref{alg:vsm-trustless-p1} for more details.





\section{\pname: Safety and Liveness}~\label{sec:safety-liveness}
\subsection{Safety}~\label{sec:safety-chain-crt}


In this section, we prove that $\pname$ satisfies atomicity per~\cref{def:forward-atom-chain-crt}. First, we prove that if the off-chain execution of the CRT does not satisfy all-or-nothing or serializability, then certain predicates about the two $\gsc$ states will not hold. Second, we prove that if at least one of the $\gsc$ state predicates does not hold, then the two $\vsm{}$s on L1 will not be able to complete the 2PC and commit to the new pair of faulty state digests. We use three helper lemmas which collectively imply the first step of our proof sketch. Note that \cref{lem:svs-gsc-all-or-none-exec} and \cref{lem:svs-gsc-honest-or-reverse-order} assume the SVS~\cite{shared-val-seq-23} $\gsc$ contract. Our modifications to the $\gsc$ contract do not affect these lemmas, so their implications directly apply to the more robust chain-CRT $\gsc$. 






 An observation that will prove useful is that if a non-local action $a$ is part of an L1-accepted batch, then the fields $a.\desc{}$ and $a.\descnext{}$ are inserted into the trigger and action trees, respectively -- besides $a.\code{}$ being executed. We capture this intuitive obesrvation in \cref{claim:a-exec-iff-desc-inserted-chain}. 
 
\begin{claim}~\label{claim:a-exec-iff-desc-inserted-chain}
    Let $\ttree$ and $\atree$ be the trigger and action trees of $\gsc$ after $\batch{}$ is accepted by the $\vsm{}$ of a rollup, and let $a$ be a \emph{non-local} action. Then $a \in \batch{}$ if and only if $a.\desc{} \in \atree$ and $a.\descnext{} \in \ttree$.~\footnote{This also captures the edge cases when $a$ is the first or last CRT action; e.g. if $a.\desc{} = \nul$, then trivially $a.\desc{} \in \atree$.}
\end{claim}

\begin{proof}
     $(a.\desc{}, a.\code{}, a.\descnext{})$ comprise the same action $a$ and are inserted/executed atomically when $a \in \batch{}$.
\end{proof}


\cref{lem:svs-gsc-all-or-none-exec} states that the SVS $\gsc$ satisfies the all-or-nothing property of atomicity.

\begin{lemma}~\label{lem:svs-gsc-all-or-none-exec}
    Let $\ccrt = [a_i]_{i \in [n]}$ and $\{\rol{b}, \dst{b}, \dstt{b}, \batch{b}\}_{b \in \{1,2\}}$ be as in~\cref{def:forward-atom-chain-crt}, when using the \emph{original SVS} $\gsc$ contract for off-chain execution. For $b \in \{1,2\}$, let $\{\ttree_b, \atree_b \}$ be the $\gsc_b$ trees under $\dst{b}$.
    Also assume (I) $\ttree_1 = \atree_2$ and $\ttree_2 = \atree_1$.
    Then (I) is preserved under $(\dstt{1}, \dstt{2})$ if and only if there exists $d \in \{\true, \false\}$ such that, for all $i \in [n]$, $\actexec{}(\dst{b}, \dstt{b}, \batch{b}, a_i) = d$ for both $b \in \{1,2\}$ where $a_i.\rol{} = \rol{b}$.
\end{lemma}

\begin{proof}[Proof of \cref{lem:svs-gsc-all-or-none-exec}]
    The ``if'' direction corresponds to the honest execution; its proof is simple and thus omitted. We prove the contrapositive of the ``only if'' direction. Assume there exists $1 \leq i \leq n-1$ and such that (without loss of generality)  $a_{i}'.\rol{} = \rol{1}$ and $a_{i + 1}'.\rol{} = \rol{2}$, and $\actexec{}(\dst{1}, \dstt{1}, \batch{1}, a_{i}') = d_i \neq d_{i+1} = \actexec{}(\dst{2}, \dstt{2}, \batch{2}, a_{i+1}')$. Since $\ccrt$ is a chain CRT, we know $a_i'.\descnext{} = a_{i+1}'.\desc{} \neq \nul$.
    
    \noindent\textbf{Case 1:} $d_i = \true$ and $d_{i+1} = \false$. Using \cref{claim:a-exec-iff-desc-inserted-chain}, the former implies $a_{i}'.\descnext{} \in \ttree_{1}'$, and the latter implies $a_{i+1}'.\desc{} \notin \atree_{2}'$.
            
    \noindent\textbf{Case 2:} $d_i = \false$ and $d_{i+1} = \true$. The former implies $a_{i}'.\descnext{} \notin \ttree_{1}'$, and the latter implies $a_{i+1}'.\desc{} \in \atree_{2}' $.
    In either case we get $\ttree_{1}' \neq \atree_{2}'$.
\end{proof}

\cref{lem:svs-gsc-honest-or-reverse-order} states that under the SVS $\gsc$, an Executor can only \emph{fully reverse} the order of actions within a CRT -- and still pass the $\gsc$ root check. We defer its proof to \cref{apdx:chain-crt-proofs}. In the following, we say that $\batch{}$ \emph{preserves} (resp. \emph{fully reverses}) the relative order of actions in a CRT $\ccrt = [a_i]_{i \in [n]}$ if for all $0 \leq i_1 < i_2 \leq n$ such that $a_{i_1}, a_{i_2} \in \batch{}$, it holds that $a_{i_1} \underset{\batch{}}{\prec} a_{i_2}$  (resp. $a_{i_1} \underset{\batch{}}{\succ} a_{i_2}$).

\begin{lemma}~\label{lem:svs-gsc-honest-or-reverse-order}
    Consider the same setup as in \cref{lem:svs-gsc-all-or-none-exec} and assume $\ttree_1 = \atree_2$ and $\ttree_2 = \atree_1$. Also assume $\actexec{}(\dst{b}, \dstt{b}, \batch{b}, a_i) = \true$ for $b \in \{1,2\}$ where $a_i.\rol{} = \rol{b}$. Then both $\batch{1}$ and $\batch{2}$ either preserve or fully reverse the relative order of actions if and only if $\ttree_1' = \atree_2'$ and $\ttree_2' = \atree_1'$.
\end{lemma}


\cref{lem:chain-gsc-honest-order} states that our chain-CRT $\gsc$ is safe against the ``reversal'' attack introduced in \cref{lem:svs-gsc-honest-or-reverse-order}. We defer the proof of \cref{lem:chain-gsc-honest-order} to~\cref{apdx:chain-crt-proofs}. 

\begin{lemma}~\label{lem:chain-gsc-honest-order}
    Let $\ccrt = [a_i]_{i \in [n]}$ and $\{\rol{b}, \dst{b}, \dstt{b}, \batch{b}\}_{b \in \{1,2\}}$ be as in \cref{def:forward-atom-chain-crt}, when using our \emph{chain-CRT} $\gsc$ contract for off-chain execution. For $b \in \{1,2\}$, let $\{\ttree_b, \atree_b, \sactive{b}, \snonce{b},  \\ \enonce{b} \}$ be the $\gsc_b$ state under $\dst{b}$. Assume $\actexec{}(\dst{b}, \dstt{b}, \batch{b}, a_i) = \true$ for $b \in \{1,2\}$ where $a_i.\rol{} = \rol{b}$. Also assume 

    \noindent(I) $\ttree_1 = \atree_2$ and $\ttree_2 = \atree_1$,
    
    \noindent(II) $\snonce{1} = \snonce{2} = \enonce{1} + \enonce{2}$, and

    \noindent(III) $\sactive{b} = \false$ for at least one $b \in \{1,2\}$.

    \noindent Then $\batch{1}$ and $\batch{2}$ preserve the relative order of all actions if and only if (I,II,III) are preserved under $(\dstt{1}, \dstt{2})$.
\end{lemma}

We now state and prove our main theorem.


\begin{theorem}~\label{thm:chain-crate-forward-atomicity-two-rollups}
    $\pname$ satisfies atomicity for chain-CRTs and $|\mathcal{R}| = 2$ rollups.
\end{theorem}

\begin{proof}[Proof of \cref{thm:chain-crate-forward-atomicity-two-rollups}]
    Let $\ccrt = [a_i']_{i \in [n]}$ and $\{\vsm{b}, \rol{b}, \dst{b}, \dstt{b}, \batch{b}\}_{b \in \{1,2\}}$ be as in \cref{def:forward-atom-chain-crt}, where $\{\dstt{b}, \batch{b}\}_{b \in \{1,2\}}$ are the new state digests and batches after running $\pname$. For $b \in \{1,2\}$, let $\{\ttree_b, \atree_b, \sactive{b}, \snonce{b}, \\ \enonce{b} \}$ be the $\gsc_b$ state under $\dst{b}$.
    Assume properties (I,II,II) from \cref{lem:chain-gsc-honest-order} all hold under $(\dst{1}, \dst{2})$.
    Also let $\idx = \textproc{getIndex}(\dst{1}, \dstt{1}, \dst{2}, \dstt{2})$ be the 2PC instance index.
    For contradiction, assume $\pname$ does \emph{not} satisfy atomicity, i.e., assume
    \begin{align*}
        \atomexec(\ccrt, \{\rol{b}, \dst{b}, \dstt{b}, \batch{b}\}_{b \in \{1,2\}}) = \false.
    \end{align*}
    This means either (a) the $\actexec{}(\cdot)$ predicates do not all agree, or (b) all $\actexec{}(\cdot) = \true$ and at least one pair of actions are not in order in their batch. In case (a), \cref{lem:svs-gsc-all-or-none-exec} says that property (I) is not preserved. In case (b), \cref{lem:chain-gsc-honest-order} says that at least one of (I,II,III) is not preserved. 
    In either case, using an inductive argument across all CRTs present in $\batch{1}$ and $\batch{2}$, at least one of (I,II,III) is not preserved under $(\dstt{1}, \dstt{2})$.
    
    \smallskip
    We now consider the 2PC component of $\pname$ leading to the new state digests. First note that in either case (a) or (b), at least one of $(\dstt{1}, \dstt{2})$ is a non-local digest; without loss of generality, assume $\dstt{1}$ is non-local. Thus $\vsm{1}$ must have executed $\textproc{PreCommit}$ successfully, followed by $\textproc{Commit}$, in order to commit to the new state digest $\dstt{1}$. This is because $\vsm{1}$ will see that one of $\ttree_1', \atree_1'$ is updated and will never accept $\dstt{1}$ as a new local state digest. This relies on the soundness of merkle membership proofs of the $\gsc$ state against the rollup state digest. We consider two cases:
    
   \noindent\textbf{Case 1:} $\vsm{1}$ is the leader under $\idx$. Then $\vsm{2}$ must have successfully  executed $\textproc{PreCommit}$ and entered $\paired$ status under $\idx$ 
    (otherwise, $\vsm{1}$ will never see correct proof of $\vsm{2}$ being $\paired$, and will never decide to commit). Now consider the successful execution of $\textproc{Commit}$ on $\vsm{1}$, 
    leading to a $\commit$ decision. Since $\vsm{1}$ decides to commit, it must see that all three properties (I,II,III) are preserved under $(\dstt{1}, \dstt{2})$, a contradiction. 

    \noindent\textbf{Case 2:} $\vsm{1}$ is \emph{not} the leader under $\idx$, so $\vsm{2}$ is the leader. Since $\vsm{1}$ accepts new non-local state digest ($\dstt{1}$), it must accept it after successful completion of $\textproc{Commit}$. Since $\vsm{1}$ is not the leader, it must have seen correct proof that $\vsm{2}.\decisions[\idx] = \commit$. For $\vsm{2}$ to decide to commit, it must have also successfully executed $\textproc{Commit}$ and committed. Again, it must be that all properties (I,II,III) are preserved under $(\dstt{1}, \dstt{2})$, a contradiction.  

\end{proof}

Note that the proof of \cref{thm:chain-crate-forward-atomicity-two-rollups} relies on the soundness of bridge proofs for the correctness of the L1 state digest.


\subsection{Latency and Liveness}~\label{sec:liveness-chain-crt}
$\pname$ requires two L1 transactions per rollup in sequential order: $\vsm{1}.\textproc{updateDigest}$, $\vsm{2}.\textproc{updateDigest}$, $\vsm{1}.\textproc{Commit}$, and $\vsm{2}.\textproc{Commit}$, with a similar order for aborts. This ensures an end-to-end latency of $4$ rounds, satisfying efficiency with respect to \cref{def:efficiency}. $\pname$ satisfies \emph{liveness}, as the leader VSM can always exit $\paired$ status by committing or aborting, while the non-leader VSM depends on the leader’s decision. Since both rollups are assumed live, this dependency does not cause deadlocks.





































\section{\pname: Supporting DAG CRTs}~\label{sec:dag-crt-protocol}
In this section, we
augment $\pname$ to support the more expressive DAG CRT programming model. We only modify the XEVM component of CRATE and leave the 2PC component on L1 unchanged; as a result, the liveness argument is identical to the chain-CRT version.

\parhead{General System Contract}
The chain-CRT-based $\gsc$ contract does not support the more expressive DAG CRTs, since it crucially relies on marking the session as inactive upon encountering a non-triggering action.
To support DAG CRTs, we modify XEVM as follows. 

First, $\action(\cdot)$ now accepts an \emph{array} of triggered actions, instead of a single action. Correspondingly, we modify $\trigger(\cdot)$ to accept trigger calls even after the $\tcalled{}$ flag is set to $\true$. 
To force Executors to use a single $\action$ call, we insert logic so that all trigger (resp. action) $\msghash$ values created as part of the same parent transaction use the same $\tnonce{}$ (resp. $\anonce{}$) value. At this point, notice that in the chain-CRT $\gsc$, the $\textproc{startSession}$ function is designed to route a single $\trigger$ call, thus limiting $a_1$ to single trigger call. To allow $a_1$ to make multiple trigger calls (which can originate from distinct contracts, such as in a cross-rollup flash loan), we re-purpose $\textproc{startSession}$ to be the \emph{entrypoint} of the DAG CRT. We require that the user route her compiled $a_1'$ action through $\textproc{startSession}$ by passing $(a_1'.\addr{}, a_1'.\cdata{})$ as arguments to it. $\textproc{startSession}$ then executes $a_1'$ and the $\gsc$ handles all nested $\trigger$ calls as described above. We leave $\textproc{checkSessionID}$ and all session-tracking logic unchanged.
We present our full DAG-CRT $\gsc$ contract in \cref{apdx:gsc}.

\parhead{Executor specification}
The Executor now listens for multiple trigger events, and passes the array all triggered actions as input to $\gsc.\action(\cdot)$. We reflect this change in the modified $\textproc{xevm}$ function in \cref{alg:shared-executor-xevm-dag}.


\parhead{Safety}
We now state our second main theorem, which says that $\pname$ is secure with respect to \cref{def:forward-atom-chain-crt} repurposed for DAG-CRTs. We defer the theorem's proof to \cref{apdx:dag-crt-proofs}. 


\begin{theorem}~\label{thm:dag-crate-forward-atomicity-two-rollups}
     $\pname$ satisfies atomicity for DAG CRTs and $|\mathcal{R}| = 2$ rollups.
\end{theorem}



\section{Implementation and Evaluation}~\label{sec:implementation}


We implement $\pname$ on Ethereum networks using the Foundry toolkit~\cite{foundry-book}. We abstract away the (existing) SNARK validity proofs of zk-rollups and implement the additional logic introduced by $\pname$. We also implement our motivating application, the cross-rollup flash loan, on top of $\pname$. %

\subsection{Implementation details}

Our $\pname$ implementation~\footnote{\url{https://github.com/ikaklamanis/crate}} consists of three main components: the Executor which performs both the XEVM off-chain execution on L2 and drives the 2PC on L1 (implemented in 1,200+ lines of Python), the L2-based General System ($\gsc$) contract (implemented in 100+ lines of Solidity), and the L1-based Validator ($\vsm{}$) contract (implemented in 500+ lines of Solidity). We conducted our evaluation on a machine with 80 cores powered by an Intel(R) Xeon(R) Platinum 8380 CPU @ 2.30GHz, 125 GB of memory, and 208 GB of swap space.

\myparagraph{Rollup architecture}
We use a simple rollup architecture with two rollups. Each rollup consists of an L2 chain instantiated as a Foundry~\cite{foundry-book} EVM chain, whose state is checkpointed on an L1 chain, also instantiated as a Foundry EVM chain. We deploy a copy of the $\gsc$ contract on the L2 chain, and a copy of the $\vsm{}$ contract on the L1 chain. 

\myparagraph{Executor}
We implement the Executor in Python as a standalone entity consisting of two modules. First, the XEVM module implements the Shared Executor \cref{alg:shared-executor-xevm-chain-crt}. 
Second, the 2PC module implements the Executor's role in the 2PC protocol; it fetches the state roots of both rollups, as well as membership proofs for the $\gsc$ state, using the Ethereum RPC API~\cite{eth-rpc-api}. 
We simulate the generation of the user transaction by including it as part of the Executor implementation. The Executor also simulates the bridge between the two L1s, by relaying block headers to the two VSM contracts. 

\myparagraph{$\vsm{}$ contract}
The $\vsm{}$ contract receives new state digests and follows the 2PC protocol to commit or abort. Our implementation executes the full 2PC protocol, which includes verifying MPT membership proofs that are part of the pre-commit and commit evidence. To optimize gas efficiency, we batch membership proofs where possible. We provide and evaluate two implementations for MPT verification. In the first implementation (``CRATE-MPT''), we use a smart contract~\cite{solidity-mpt} which directly verifies membership proofs on-chain; in the second one (``CRATE-SNARK''), we use Groth16~\cite{groth-16} SNARKs  and the Circom~\cite{circom-site} framework. %

\newcommand{\token}{\mathsf{xToken}}
\newcommand{\flpool}{\mathsf{xFlashLoan}}
\newcommand{\userfl}{\mathsf{xUserFL}}

\parhead{Flash Loan Application}
We implement a flexible cross-rollup flash loan infrastructure in less than 300 lines of Solidity, carefully leveraging the more expressive DAG-CRT model of $\pname$. The implementation consists of two ``service'' contracts: the $\token$ contract, which facilitates token transfers and the \emph{cross-rollup} burn-then-mint service, and the $\flpool$ contract, which supports the \emph{cross-rollup} flash loan service.
Third, we implement the $\userfl$ contract, which is executes the user-desired actions during the cross-rollup flash loan. We deploy a copy of each of the three contracts on each rollup L2 chain.
The reader can refer to \cref{apdx:xfl} for the Solidity implementation of these contracts.



\subsection{Evaluation}

\newcommand{\batchsize}{\mathsf{batchSize}}
\newcommand{\numbatches}{\mathsf{numBatches}}

\parhead{L1 Gas Usage}

Compared to existing zk-rollups, $\pname$ introduces an extra L1 transaction for finalizing the state digest and additional overhead for verifying pre-commit and commit evidence. We measure this extra L1 gas usage and compare it to Zksync Era~\cite{zksync-era}, which finalizes an L2 batch using three L1 transactions~\cite{quarkslab-zksync-workflow}. Based on data from Etherscan~\cite{etherscan} and Zksync Explorer~\cite{zksync-explorer}, the L1 gas usage of Zksync is in the range $0.9$M -- $2.1$M gas during the period June-December 2024. Since zk-rollups already incur batch submission costs, $\pname$’s evaluation excludes batch size considerations.


We define a simple workflow where the Executor receives transactions, which are grouped into batches, each containing at least one CRT. For each batch, (1) the XEVM module executes all transactions on L2, producing new state digests, and (2) the 2PC module drives the 2PC protocol between the two VSMs. We run this workflow for $100$ instances of 2PC and compute the average L1 gas usage of $\pname$. Using the MPT verifier contract, the average gas usage of $\pname$ over all 2PC instances is $1.35$M gas, with approximately $840$K gas for pre-commit and $510$K gas for commit. Thus the MPT-based $\pname$ incurs a $\mathbf{0.64-1.5}\times$ increase in gas usage compared to the referenced gas usage range of Zksync Era, which we illustrate in \cref{fig:2pc-gas-usage-increase}.



\begin{figure}
    \centering
    \includegraphics[scale=0.4]{figures/l1_gas_cost_increase_plot.pdf}
    \caption{L1 gas usage increase incurred by the MPT-based (red) and SNARK-based (green) implementation of the $\pname$ 2PC protocol, compared to the observed gas usage range of a vanilla zk-rollup (Zksync). }
    \label{fig:2pc-gas-usage-increase}
\end{figure}



\parhead{Using SNARKs instead of native MPT proofs} To further reduce L1 gas usage during 2PC, we replace Merkle membership proofs with SNARK proofs. This allows us to batch the unrelated $\locevd$ and $\pcevd$ membership proofs of the pre-commit round into a single SNARK. Under this implementation, the average gas usage of $\pname$ over all 2PC instances is $670$K gas, with approximately $340$K gas for pre-commit and $330$K gas for commit. Thus the SNARK-based $\pname$ incurs a $\mathbf{0.32-0.75}\times$ increase in gas usage compared to the observed gas usage range of Zksync Era, which we also illustrate in \cref{fig:2pc-gas-usage-increase}.
We note that in practice, a zk-rollup integrating $\pname$ can further reduce gas usage by combining the existing state transition validity SNARK with $\pname$'s pre-commit SNARK into a single SNARK. 


Since our experiments above are conducted on local Foundry chains, the depth of our MPT tries is $d = 4$, which is smaller than the depth observed on Ethereum Mainnet account MPT tries ($d=11$ at the time of writing). Larger MPT depths imply larger SNARK circuits, which in turn increase proof generation time. We use the term ``single'' MPT proof to mean a proof for a single MPT trie. To prove membership of a smart contract variable against the chain's state root, we need a ``full'' MPT proof consisting of \emph{two single} MPT proofs; one for the account trie and one for the storage trie. Under this terminology, the SNARK circuit of the pre-commit round contains \emph{two full} MPT proofs (one for $\locevd$ and one for $\pcevd$); the SNARK circuit of the commit round contains \emph{one full} MPT proof (for $\cevd$).

We separately measure the SNARK generation time of a \emph{single} MPT proof for different values of $d \in [2,16]$. For $d=16$, the proof generation time is $130$ seconds (with CPU) and $60$ seconds (with GPU). We use an AWS EC2 g4dn.xlarge instance for both CPU and GPU performance benchmarks. The instance has 4 Intel Xeon Platinum 8259CL CPUs @ 3.095 GHz, 16 GB memory, and 1 NVIDIA Tesla T4 GPU core. Since $\pname$ uses $4$ single MPT proofs for the pre-commit round, we estimate that the SNARK proof generation time of the pre-commit round (for $d=16$) is $4 \times 130 = 520$ seconds (i.e., $8.7$ minutes) with CPU, and $4 \times 60 = 240$ seconds (i.e., $4$ minutes) with GPU. Similarly, the SNARK proof generation time of the commit round is estimated to be $4.4$ minutes with CPU and $2$ minutes with GPU. Note that the \emph{proof size, verification time, and L1 gas usage remain the same} regardless of the MPT depth.





\parhead{Flash Loan application}
We measure the total gas usage of a flash loan CRT across both rollups to be $550$K gas. For reference, single-chain flash loans are reported~\cite{equalizer-fl-cost} to consume $200$K gas on average. This increase is due to the overhead of the six trigger-action calls made by our cross-rollup flash loan. Regardless, our flash loan is reasonably practical, especially considering the significantly lower L2 gas costs. 



\section{Discussion}~\label{sec:discussion}
In this section, we discuss extensions to the $\pname$ protocol and implications of our design decisions.




\parhead{Full Serializability}
\emph{Full} serializability requires that no local action can interleave between two CRT actions within a rollup batch, in addition to the relative ordering already required by \emph{weak} serializability. We can easily modify the $\gsc$ contract to support full serializability as follows. If $\action$ is ready to declare a session as inactive, then, before returning, it triggers a new $\textproc{finishSession}$ on the other $\gsc$. This ensures that both GSCs mark the current session as inactive, which was not the case in \cref{alg:gsc-single-trigger}. Further, we require that \emph{all} rollup transactions (including local ones) use $\startsession$ as their entrypoint. Then $\startsession$ can revert any transaction that attempts to execute during an ongoing CRT session. 


Unlike L1 protocols with fine-grained state locks~\cite{fal-tccsci-23}, our approach locks the entire rollup state during the CRT session, simplifying implementation. Transactions in the batch can still access the full state outside the CRT session. However, this requires routing all transactions through the $\gsc$, making it not backwards-compatible with existing applications.




\parhead{Distinct Executors}
For ease of exposition, we have described $\pname$ under a shared Executor operating both rollups. However, $\pname$ can easily be adapted to support \emph{distinct}, possibly \emph{distrusting} Executors $E_1$ and $E_2$. Upon listening to a trigger call emitted by $\gsc_1$, $E_1$ sends the triggered action information to $E_2$, who proceeds to execute a matching $\action$. $E_1$ can set proper timeouts so that he does not wait an indefinite amount of time for $E_2$ (and vice-versa). Since $E_1$ and $E_2$ need to communicate back and forth to execute CRTs, the network latency becomes the bottleneck during off-chain execution. Nevertheless, this drawback is not specific to $\pname$ but inherent to the distinct Executors setup.




\parhead{Implications on rollup finality}
Recall from \cref{sec:liveness-chain-crt} that the end-to-end latency of $\pname$ is $4$ rounds. In Ethereum, one round lasts $13$ minutes, so rollups using $\pname$ achieve finality after $4 \times 13 = 52$ minutes; this might seem much longer than the single round ($13$ minutes) required by existing zk-rollups for finality. However, most zk-rollups currently submit new state roots to L1 at intervals in the order of \emph{hours}, since SNARK proof generation is the bottleneck. Thus, both the $4\times$ latency and the $\vsm{}$ state locking introduced by $\pname$ do not affect existing rollups' finality and throughput, since the $\vsm{}$s were already not utilized during that time for any new state updates. Similarly, the increased proving time incurred by our SNARK-based implementation is also dominated by the SNARK validity proof generation time required by existing zk-rollups.

Note that the discussion above is about L1 \emph{finality}, which does not place any trust on the Executor. Today, applications may also be interested in instant confirmation of rollup transactions~\cite{zksync-instant-confirmations}. Since the Executor needs to be trusted for such faster confirmations, $\pname$ preserves the underlying zk-rollups' instant confirmation guarantees.








\section{Related Work}~\label{sec:relwork}
In this section, we compare $\pname$ to existing cross-chain systems. The state-of-the-art cross-rollup composability solution is SVS~\cite{shared-val-seq-23} and variants, such as the CIRC protocol~\cite{espresso-circ-24} by Espresso. We provide a comprehensive comparison of $\pname$ with SVS in~\cref{sec:overview}.

\myparagraph{Cross-chain Interoperability (IO)} Existing cross-chain IO~\cite{chainlink-ccip} protocols primarily adopt the framework of atomic swaps and cross-chain bridges. Atomic swap protocols usually make use of smart contracts, hashlocks, and timelocks~\cite{bitcoin-wiki-htlc,wadhwaHeHTLC2023,tsabaryMADHTLC2021} to ensure the atomic execution of cross-chain transactions. Cross chain bridges~\cite{xiew-zkbridge-22, axelar, near-bridge} connect two blockchains and facilitate the transfer of assets between them. While they enable services such as asset transfers, they cannot offer full cross-chain composability, since they lack a coordination mechanism. 
Current IO protocols~\cite{alt-chains-atomic-transfers, bitcoin-wiki-htlc, herlihy-atomic-cc-swaps-18, herlihy-cc-smr-22, sheff-het-paxos-20, anoma-chimera-chains-23} cannot accommodate complex cross-chain transactions such as flash loans without relying on strong trust assumptions.

\myparagraph{Cross-chain Composability}
Several solutions~\cite{lu-atomic-cc-interactions-24, fal-tccsci-23, zakhary-ac3-20} ensure atomicity for multi-L1 transactions, such as the 2PC4BC protocol~\cite{fal-tccsci-23}, which provides atomicity and serializability but requires $O(n)$ rounds for a transaction of length $n$. These L1 protocols rely on complex locking mechanisms and multiple rounds, making certain applications, like cross-chain flash loans, impractical. In contrast, $\pname$ operates on L2 and requires $O(1)$ rounds by executing transactions off-chain and including them in a single rollup batch. 
For instance, 2PC4BC requires at least 5 L1 rounds ($\sim60$ blocks) for flash loan execution and finality, whereas $\pname$ performs execution off-chain, with L1 involvement only needed for finality.



Shared sequencer networks like Espresso~\cite{espresso-docs}, Astria~\cite{astria-docs}, and Radius~\cite{radius-docs} leverage a shared sequencer to achieve censorship resistance and faster pre-confirmation. While these solutions enable certain types of cross-rollup transactions, their guarantees are limited to atomic inclusion (not execution) and rely on trusting the shared sequencer. 


The Superchain~\cite{superchain-explainer} is a network of layer 2 chains built on the OP Stack~\cite{optimism-website}, enabling resource sharing and asynchronous messaging but relying on a shared sequencer for cross-chain atomicity. Polygon AggLayer~\cite{agglayer} connects layer 1 and layer 2 networks via a unified bridge, supporting token transfers and message passing, but its cross-chain transactions require Ethereum settlement, leading to high asynchrony.












\section{Conclusion}~\label{sec:conclusion}
We have introduced $\pname$, a secure protocol for cross-rollup composability, enabling users to atomically execute transactions spanning two rollups, guaranteeing the all-or-nothing and serializability properties. Our protocol is flexible since it supports rollups on distinct L1s, as well as rollups operated by distinct Executors. We provided two formal models for cross-rollup transactions (CRTs), and defined atomicity within them. Our novel formal treatment gave rise to the shortcomings of other works and allowed us to rigorously prove security for $\pname$. We implemented and benchmarked all components of $\pname$ and an end-to-end cross-rollup flash loan application to demonstrate the practicality of $\pname$.




\section*{Acknowledgements}
We want to thank Fangyan Shi and Wenhao Wang for their help with the SNARK-based implementation of MPT verification, as well as Sen Yang for helpful discussions.



\bibliographystyle{plain}
\bibliography{ref,fanz}


\appendix
\newpage
\appendix
\onecolumn

\section{Why IARs (seem to) leak more privacy than DMs?}
\label{app:why_iars_leak_more}

{In the following we provide insights explaining the higher leakage observed in IARs. First, we focus on differences in architectures and models' internals. Then, we switch to explore architecture-agnostic factors like model size.}

\subsection{Inherent differences between IARs and DMs}
{We note that DMs have inherently different characteristics than IARs, and we link them to the privacy risks they exhibit. We identify three key factors:
\begin{enumerate}
    \item \textbf{Access to $p(x)$ boosts MIA}~\citep{rmia}. We note that IARs inherently expose the full information about $p(x)$ at the output (per-token logits, see~\cref{eq:ar_task}). In contrast, DMs do not, as they learn to transform $\mathcal{N}(0,I)$ to the data distribution $q(x)$ by iterative denoising process. This difference is expressed with varying MIA designs for DMs and IARs--the former exploit the predicted noise, while the latter work with $p(x)$, by focusing on the logits. Our results confirm this premise--MAR is less prone to all privacy risks, and it does not output $p(x)$. It outputs continuous tokens, sampled from a diffusion module.
    \item \textbf{AutoRegressive training exposes IARs to more data per update}. For each training sample passed through the IAR, the model "sees" $N$ different sequences to predict. Conversely, DMs only "see" a single, noisy image. This influences two factors: a) training time of the model--DMs require to be trained two times longer than IARs, on average. b) privacy leakage--IARs are exposed to more information per each update step, which translates to increased vulnerability for privacy attacks like MIAs, DI, and data extraction. VAR outputs 10 sequences of tokens, and is less prone to MIA than RAR, which outputs 256 sequences, \eg VAR-\textit{d}-20 vs. RAR-L (models of similar sizes).
    \item \textbf{Multiple \textit{independent} signals amplify leakage}. Previous works~\citep{maini2024llmdatasetinferencedid,dubinski2024cdicopyrighteddataidentification} aggregate signal from many MIAs to yield a stronger attack. Notably, each token predicted by IARs leaks unique information from the model, as it is generated from a (slightly) different prefix. Thus, per-token losses/logits that IAR-specific MIAs use, when aggregated, add up to a more informative signal, which in turn yields stronger MIAs. In contrast, DMs' outputs provide a general direction for the denoising process, and are strongly correlated. In effect, predictions at different timesteps do not provide enough \textit{novel} information to the MIA to boost its strength.
\end{enumerate}
}

{We believe that these reasons are behind greater privacy leakage that we observe for IARs than for DMs.}

\subsection{Architecture-agnostic differences between the models}

{The models evaluated in our work differ in many factors. Two of them, model size and training duration, are mostly architecture-agnostic, which means they are less related to the design choices of the specific models. As the efficacy of privacy attacks is directly related to these factors~\citep{shokri2017membershipinference}, we want to assess if our results \textit{really} show that IARs leak more than DMs. To this end, we collect five variables: TPR@FPR=1\% (MIA), $P$ (DI metric), model size, training duration, and \textit{Is IAR} for every model we evaluate in the paper (11 IARs, 8 DMs). For the first two (MIA, DI) we take them directly from~\cref{tab:mia_naive_vs_ours,tab:di_naive_vs_ours,tab:tpr_dm}. We obtain the model sizes from~\cref{tab:iar_model_details,tab:dm_details}. Training duration is expressed by a number of data points passed through the model at training, e.g., for RAR-B we have 400 epochs of ImageNet-1k train set, which amounts to $400\times1.27$M $\approx0.5$B samples seen. \textit{Is IAR} factor is a 1 if the model is IAR, 0 otherwise. We take these variables and compute pairwise Pearson’s correlation between them, using values for all the models.}

{In~\cref{tab:factors_correlation} we show correlations between factors (columns) and privacy metrics (rows). We identify the following insights:
\begin{enumerate}
    \item \textbf{Training duration} is a factor that increases vulnerability for MIA and DI for DMs the most.
    \item \textbf{Model size} influences leakage more for IARs than for DMs.
    \item \textbf{\textit{Is IAR}} factor plays the most significant role for the DI performance. It also correlates with MIA performance.
\end{enumerate}
}

{Our results show that while these two factors--model size and training duration--influence the performance of our attacks against the models, the results strengthen our notion that IARs tend to leak more privacy than IARs due to their inherent characteristics.}

\begin{table}[]
    \scriptsize
    \caption{{\textbf{Correlation between different factors and privacy leakage.} Our results show that while the model-agnostic factors correlate with the performance, the fact that the model is IAR or not also correlates with the leakage.}}
    \centering
    \begin{tabular}{ccccc}
        \toprule
        & Architecture & Training Duration & Model Size & Is IAR \\ 
         \midrule
        {$P$ (DI)} & IAR & 0.24 & -0.39 &  \\ 
        {$P$ (DI)} & DM & -0.58 & -0.32 &  \\ 
        {$P$ (DI)} & All & -0.04 & -0.28 & -0.46 \\ 
        \midrule
         {$TPR@FPR=1\%$}& IAR & 0.17 & 0.93 &  \\ 
         {$TPR@FPR=1\%$}& DM & 0.31 & 0.11 &  \\ 
         {$TPR@FPR=1\%$} & All & -0.2 & 0.87 & 0.38 \\ 
        \bottomrule
    \end{tabular}
    \label{tab:factors_correlation}
\end{table}

\section{Limitations}
\label{app:limitations}

{We acknowledge our privacy analysis of the novel IARs, and comparison to DMs suffers from two limitations. We do not evaluate our attacks on the biggest available models (like Infinity~\citep{han2024infinityscalingbitwiseautoregressive}) trained on massive (over 1B samples), messy datasets. Secondly, there are many factors crucial for MIA and DI performance, which differ in values between almost all the models. The following explains these issues in more detail.}

\subsection{On the infeasibility of high-scale experiments on extremely big models}

{We do not assess how our attacks perform when applied to models trained on datasets of the scale higher than 1M samples. It may raise concerns about the scalability of the attacks and the insights they provide to the real-world applications. Unfortunately, IARs trained on bigger datasets than ImageNet-1k (Infinity~\citep{han2024infinityscalingbitwiseautoregressive}, HART~\citep{tang2024hartefficientvisualgeneration}) do not disclose fully what their training data \textit{exactly} is. Because of that, we are unable to perform a sound evaluation of the privacy attacks. We lack the ability to assess MIA's and DI's performance correctly, as these methods rely on two assumptions: (1) we know a part of the training data (members), (2) we have access to non-members that are \textit{independent and identically distributed} (IID) with members. When we fail to satisfy (2) the methods would collapse to dataset detection~\citep{das2024blind}. Moreover, without satisfying (1) we cannot run MIA and DI at all.}

{While a \textit{methodologically correct} evaluation of the cutting-edge models is out of our reach, we aim to provide more insight into text-to-images IARs, and see how much they leak. To this end, we run our attacks on VAR-CLIP \citep{zhang2024varcliptexttoimagegeneratorvisual}, a VAR-\textit{d}16 model trained on a captioned ImageNet-1k. Our results in~\cref{tab:varclip} show that this model leaks significantly more data than its class-to-image counterpart of the same size. Moreover, the leakage is on a level similar to VAR-\textit{d}20's--a model of double the size of VAR-CLIP. We argue that the increased leakage stems from the model overfitting more to the conditioning information, which is richer for textual data than for the class labels.}

\begin{table}[h]
    \centering
    \scriptsize
    \caption{{\textbf{Leakage of VAR-CLIP compared to class-conditional VARs.} We observe increased privacy leakage over class-conditioned models, expressed by a stronger performance of our attacks.}}
    \begin{tabular}{ccc}
    \toprule
        \textbf{Model} & \textbf{TPR@FPR=1\%} & \textbf{$P$ (DI)} \\ 
        \midrule
         {VAR-CLIP} & 6.30 & 60 \\ 
         {VAR-\textit{d}16} & 2.18 & 200 \\ 
         {VAR-\textit{d}20} & 5.92 & 40 \\ 
        \bottomrule
    \end{tabular}
    \label{tab:varclip}
\end{table}

\subsection{On the impossibility of a fully standardized experimental setup between the models}

{In the ideal scenario we are able to isolate only the factors inherent to the models' architecture, and consequently, are able to draw insights which design choices lead to what privacy risks. We would call such setup \textit{standardized}, meaning that the models are \textit{almost identical}, and differ only in factors we want to explore (like architecture). However, in reality we deal with too few models, each one being trained differently, which allows only for limited insights.}

{We note the models vary in the following ways:
\begin{enumerate}
    \item \textbf{Training duration}, expressed by number of data points seen during training, \eg RAR-B sees $400\times1.27$M $\approx0.5$B samples. In DMs we evaluate the training duration varies between 0.21B to 1.79B samples seen, whereas IARs are trained with between 0.26B and 0.51B samples.
    \item \textbf{Training objectives}. DMs minimize~\cref{eq:dm_loss}, while IARs--~\cref{eq:ar_loss}. Importantly, DMs minimize the expected error \textit{over timesteps and data}, which necessitates a twice as long training duration for DMs than IARs (on average) to achieve comparable FID.
    \item \textbf{Model sizes}. IARs benefit from scaling laws~\citep{kaplan2020scaling}, and that allows them to be scaled up to sizes greater than DMs, before their performance plateaus. DMs cannot be scaled that well--the performance gains diminish faster with the increase of size. In effect, the biggest IARs we evaluate--\varbig and RAR-XXL-- are on average 2-3 times bigger than DMs. Since the size of the model impacts its vulnerability to privacy attacks, our analyses do not fully accommodate for that factor.
    \item \textbf{Two stage architectures}. All models incorporate an encoder-decoder network for training and inference, \eg VQ-VAE~\citep{esser2020taming}. Importantly, these encoders differ between models. VAR's next-scale prediction paradigm requires training of a specialized encoder that understands how to process residual token maps, used during encoding an image to the sequence of discrete tokens. Moreover, VAR and RAR work with \textit{discrete} tokens, \ie the encoder-decoder network additionally contains a quantizer module, which translates the continuous latent representations of the images to a 2D integer-only maps.
\end{enumerate}}

{Unfortunately, these factors directly prohibit a \textit{standardized comparison} of the privacy risks between DMs and IARs. We are not able to fix the training duration for all models--the generation quality of DMs would be significantly subpar than IARs (as DMs require twice the training time of IARs), and thus the results would be unsound. We incorporate the size of the models in~\cref{fig:pareto,fig:pareto_di,fig:pareto_app}, however, we acknowledge that the sizes vary between the models, and this limits our ability to fully disentangle this factor from the privacy results.}

{However, we are able to fix one factor for all the models: utility. We know the models we source are trained to the maximum of the potential each architecture allows, as we utilize models from papers that aim for exactly that--the best performance. We compare models that are the \textit{upper boundary} of what is possible within the inherent limitations and trade-offs each architecture has to offer. We are deeply aware that privacy vs utility is a balancing act: better models tend to be less private. \textbf{Thus, our study fixes one of these parameters--utility--to be the highest possible for a given model}, and under that condition we evaluate how much it leaks. We believe our results provide strong empirical evidence that DMs constitute a Pareto optimum when it comes to image generation--they are comparable in FID, while being significantly more private than the novel IAR models.}

\section{Privacy leakage under a unified attack}
\label{app:unified_mia}

{We acknowledge that the field of privacy attacks against image generative models like IARs or DMs is constantly evolving. Since our work aims to provide the current empirical insights into differences in privacy leakage between these architectures, we use \textit{the strongest available} attacks to provide an upper boundary on the privacy leakage, following literature on privacy auditing~\citep{nasr2023tight,dwork2006differential}}. 

{However, IARs and DMs are two different classes of models. In consequence, the attacks we employ are \textit{tailored} to their inherent properties, and thus the attacks vary. This might raise concerns of the following nature: what if the field progresses and a new, very potent attack is designed for DMs? Will our current empirical results hold, \ie can we \textit{really} claim IARs leak more privacy than DMs, or is it just the current MIAs against DMs that are less powerful than for IARs?}

{We believe our insights in~\cref{app:why_iars_leak_more} provide reasons why IARs \textit{inherently} leak more than DMs. To strengthen our results, we perform an \textit{architecture-agnostic}, unified attack against all models--Loss Attack~\citep{yeom2018lossmia}}.

\subsection{Loss Attack}

\begin{table}[]
    \centering
    \scriptsize
    \caption{{\textbf{Unified attack results.} We employ Loss Attack~\citep{yeom2018lossmia}, discarding any model-specific modifications that might strengthen the signal, to ensure a fair comparison between different model classes and architectures. The results strongly support our notion that IARs leak more privacy than DMs.}}
\begin{tabular}{ccccccc}
\toprule
Model & Architecture & $P$ (Dataset Inference) & TPR@FPR=1\% (MIA) & AUC (MIA) & Accuracy (MIA) \\
\midrule
VAR-$\mathit{d}$16 & IAR & 3000 & 1.50{\tiny $\pm$0.18} & 52.35{\tiny $\pm$0.40} & 50.08{\tiny $\pm$0.03} \\
VAR-$\mathit{d}$20 & IAR & 1000 & 1.67{\tiny $\pm$0.20} & 54.54{\tiny $\pm$0.40} & 50.11{\tiny $\pm$0.03} \\
VAR-$\mathit{d}$24 & IAR & 300 & 2.19{\tiny $\pm$0.20} & 59.56{\tiny $\pm$0.39} & 50.15{\tiny $\pm$0.04} \\
VAR-$\mathit{d}$30 & IAR & 40 & 4.95{\tiny $\pm$0.40} & 75.46{\tiny $\pm$0.35} & 50.32{\tiny $\pm$0.05} \\
MAR-B & IAR & 6000 & 1.43{\tiny $\pm$0.17} & 51.31{\tiny $\pm$0.30} & 50.48{\tiny $\pm$0.16} \\
MAR-L & IAR & 3000 & 1.52{\tiny $\pm$0.16} & 52.35{\tiny $\pm$0.30} & 50.70{\tiny $\pm$0.18} \\
MAR-H & IAR & 2000 & 1.61{\tiny $\pm$0.17} & 53.66{\tiny $\pm$0.30} & 51.07{\tiny $\pm$0.20} \\
RAR-B & IAR & 800 & 1.77{\tiny $\pm$0.25} & 54.92{\tiny $\pm$0.41} & 50.25{\tiny $\pm$0.06} \\
RAR-L & IAR & 400 & 2.10{\tiny $\pm$0.27} & 58.03{\tiny $\pm$0.40} & 50.39{\tiny $\pm$0.07} \\
RAR-XL & IAR & 80 & 3.40{\tiny $\pm$0.40} & 65.58{\tiny $\pm$0.38} & 50.81{\tiny $\pm$0.10} \\
RAR-XXL & IAR & 40 & 5.73{\tiny $\pm$0.52} & 74.44{\tiny $\pm$0.34} & 51.64{\tiny $\pm$0.19} \\
\midrule
LDM & DM & $>20000$ & 1.08{\tiny $\pm$0.13} & 50.13{\tiny $\pm$0.05} & 50.13{\tiny $\pm$0.11} \\
U-ViT-H/2 & DM & $>20000$ & 0.85{\tiny $\pm$0.13} & 50.11{\tiny $\pm$0.09} & 50.07{\tiny $\pm$0.18} \\
DiT-XL/2 & DM & $>20000$ & 0.84{\tiny $\pm$0.14} & 50.09{\tiny $\pm$0.05} & 50.15{\tiny $\pm$0.14} \\
MDTv1-XL/2 & DM & $>20000$ & 0.85{\tiny $\pm$0.13} & 50.05{\tiny $\pm$0.05} & 50.08{\tiny $\pm$0.14} \\
MDTv2-XL/2 & DM & $>20000$ & 0.87{\tiny $\pm$0.12} & 50.14{\tiny $\pm$0.05} & 50.16{\tiny $\pm$0.14} \\
DiMR-XL/2R & DM & $>20000$ & 0.89{\tiny $\pm$0.13} & 49.55{\tiny $\pm$0.06} & 49.70{\tiny $\pm$0.14} \\
DiMR-G/2R & DM & $>20000$ & 0.85{\tiny $\pm$0.12} & 49.54{\tiny $\pm$0.06} & 49.69{\tiny $\pm$0.13} \\
SiT-XL/2 & DM & 6000 & 0.95{\tiny $\pm$0.16} & 48.22{\tiny $\pm$0.26} & 49.97{\tiny $\pm$0.09} \\
\bottomrule
\end{tabular}
\label{tab:unified_mia_result}
\end{table}

\noindent {Loss Attack is defined as follows: (1) For each sample we perform a forward pass through the model as it would be during the training (2) We compute the model loss (specific to each model) for the samples. (3) We use the losses to perform MIA (as in~\cref{app:mias_full}), and we use the losses to perform Dataset Inference (see~\cref{app:di_section})}.

{Loss Attack differs from MIAs against DMs in the following way: instead of fixing the timestep to the most optimal one ($t=100$~\citep{carlini2023extracting}), and averaging the loss over 5 different input noises~\citep{carlini2023extracting}, we sample $t\sim\mathcal{U}[0, 1000]$, and compute the per-sample loss for \textit{a single random noise}}.

{For MAR, we roll back the modifications to the diffusion module, explained in~\cref{app:mias_on_mar}. We do not fix the timestep to the most optimal one ($t=500$), we compute the loss over 5 (default for training), instead of 64 (optimal) input noises, and we sample the masking ratio for each sample following the distribution used during training, instead of fixing it to 0.95--the optimal value.}

{For VAR and RAR, this attack is identical to the one in~\cref{tab:tpr_baseline_mias} (first row).}

{Since the DI framework relies on features obtained from different MIAs, we run DI only with the single feature--Loss Attack. We unify DI to be the same for DMs and IARs by removing the scoring function $s$ for DM-specific DI--CDI~\citep{dubinski2024cdicopyrighteddataidentification}. In effect, the procedure is \textit{identical} for DMs and IARs.}

\subsection{IARs are empirically more prone to the unified attack than DMs}

{Our results in~\cref{tab:unified_mia_result} are consistent with the results achieved with DM- and IAR-specific attacks (\cref{tab:mia_naive_vs_ours,tab:di_naive_vs_ours}) Empirical data shows IARs are more vulnerable to MIAs and DI. Loss Attack does not yield \tprat greater than random guessing (1\%) for DMs, whereas all IARs perform above random guessing. Moreover, with such a weak signal, DI ceases to be successful for DMs, requiring above 20,000 samples ($P$) to reject the null hypothesis (no significant difference between members and non-members), with one exception: SiT. Conversely, IARs retain their high vulnerability to DI, with the most private IAR--MAR-B--being similarly vulnerable to the least private DM--SiT.}

{We believe results obtained under the unified attack strengthen our message that current IARs leak more privacy than DMs.}

\section{Additional Background}

{In the following we provide additional background on Diffusion Models used for comparison to IARs, details on MIAs, and precise definition of the DI procedure, as well as a description of the sampling strategies used by IARs during generation.}

\subsection{Diffusion Models}
\label{app:dms_full}

\begin{table*}[h!]
    \scriptsize
        \newcommand{\tightcolsep}{\setlength{\tabcolsep}{3.5pt}} %
    \tightcolsep %
    \centering
        \caption{\textbf{DM details.} We report the training details for the DM  models used in this work.}
    \begin{tabular}{ccccccccc}
    \toprule
\textbf{} & LDM & {U-ViT-H/2} & {DiT-XL/2} & {MDTv1-XL/2} & {MDTv2-XL/2} & {DiMR-XL/2R} & {DiMR-G/2R} & {SiT-XL/2} \\
    \midrule
\textbf{Model parameters} & 395M & 501M & 675M & 700M & 742M & 505M & 1056M & 675M \\

    \textbf{Training steps} & 178k & 500k & 400k & 2M & 6.5M  & 1M & 1M &  7M \\
    \textbf{Batch size} & 1200 & 1024 & 256 & 256 & 256 & 1024 & 1024 & 256  \\
    \textbf{FID} & 3.60 & 2.29 & 2.27 & 1.79 & 1.58 & 1.70 & 1.63  & 2.06 \\
    \bottomrule
    \end{tabular}
    \label{tab:dm_details}
\end{table*}


{We provide a brief overview of DMs used in our experiments. All models are class-conditioned latent DMs trained on the ImageNet dataset at 256×256 resolution.}
{Except for LDM, all models utilize Vision Transformers (ViT) \citep{dosovitskiy2021imageworth16x16words} as their diffusion backbones. LDM instead employs the UNet architecture \cite{unet}, being a prior work. We refer the reader to the original publications for more details about their architectures and training strategies.}

{\textit{LDM (Latent Diffusion Model)} by \citet{rombach2022high} first propose running diffusion in a learned latent space rather than in pixel space, using a U-Net as the denoising backbone.}

{\textit{DiT-XL/2 (Diffusion Transformer)} by \citet{peebles2022dit} replaces the conventional U-Net with a ViT backbone.}

{\textit{U-ViT-H/2} by \citet{bao2022uvit} adopts a ViT-based architecture with skip connections inspired by U-Nets. It treats image patches, class labels, and diffusion timesteps as input tokens in a unified transformer space.}

{\textit{MDTv1-XL} and \textit{MDTv2-XL (Masked Diffusion Transformer)} by \citet{gao2023masked} apply a masked latent modeling strategy during training to enhance contextual learning. The model predicts missing latent tokens, improving training efficiency and sample quality. MDTv2 introduces architectural refinements that lead to further gains in fidelity and performance.}

{\textit{DiMR-XL/2R} and \textit{DiMR-G/2R} by \citet{liu2024alleviating} propose a multi-resolution diffusion framework that processes features across different spatial scales. This design improves detail preservation and reduces distortions, especially when using large patch sizes. The models also incorporate time-aware normalization to enhance temporal conditioning.}

{\textit{SiT-XL/2 (Scalable Interpolant Transformer)} by \citet{ma2024sit} extends the DiT architecture with an interpolant mechanism that decouples the noise schedule from the model. This allows for greater flexibility in diffusion dynamics without architectural changes.}

{Besides these models, we additionally evaluate emerging DMs: LFM~\citep{dao2023flow}--a flow-matching model, and DiT-MoE~\citep{fei2024scalingdiffusiontransformers16}--a mixture-of-experts DM, based on DiT~\citep{peebles2022dit}. We do not include these models for the final comparison for three reasons: (1) the released models are significantly smaller (130M parameters each) than all other models, (2) the released models achieve subpar FID scores (4.46 for LFM, unknown FID for DiT-MoE), (3) unknown details of training (number of iterations for DiT-MoE). For completeness, we perform MIA and DI, and report the values in~\cref{tab:extra_dms}.}

\begin{table}[!ht]
    \centering
    \scriptsize
    \caption{{\textbf{Results for novel DM architectures.} We see the leakage is similar to the rest of DMs.}}
    \begin{tabular}{ccc}
        \toprule
        Model & TPR@FPR=1\% & $P$ (DI) \\
        \midrule
        LFM & 1.79 & 2000\\
        DiT-MoE & 1.70 & 2000 \\
        \bottomrule
    \end{tabular}
    \label{tab:extra_dms}
\end{table}

\subsection{Membership Inference Attacks}
\label{app:mias_full}


MIAs attempt to identify whether a given input $x$, drawn from distribution $\mathcal{X}$, was part of the training dataset $\mathcal{D}_{\text{train}}$ used to train a target model $f_\theta$. We explore several MIA strategies under a gray-box setting, where the adversary has access to the model’s loss but no information about its internal parameters or gradients. The goal is to construct an \textit{attack} function $A_{f_\theta}: \mathcal{X} \rightarrow \{0, 1\}$ that predicts membership. 

\textbf{Threshold-Based attack.} 
Threshold-based attack is a key method of establishing membership status of a sample. 
It relies on a metric such as Loss~\citep{yeom2018lossmia} to determine membership. An input $x$ is classified as a member if value of the metric falls below a predefined threshold:
\begin{equation}
A_{f_\theta}(x) = \mathbb{1}[\mathcal{M}(f_\theta, x) < \gamma],
\label{eq:mia_thr}
\end{equation}
where $\mathcal{M}$ is the metric function, and $\gamma$ is the threshold. 

\textbf{\mink Metric.} 
To address the limitations of predictability in threshold-based attacks, \citet{shi2024detecting} introduced the \mink metric. This approach evaluates the least probable $ K\% $ of tokens in the input $ x $, conditioned on preceding tokens{, where $K$ is a hyperparameter, selected from $\{10,20,30,40,50\}$}. By focusing on less predictable tokens, \mink avoids over-reliance on highly predictable parts of the sequence. Membership is determined by thresholding the average negative log-likelihood of these low-probability tokens:
\[
A_{\fmodelm}(x) = \mathbb{1}[\mink(x) < \gamma].
\]

{The final value is reported for the best $K$.}

\textbf{\mink++.}  
\mink++ refines the \mink method by leveraging the insight that training samples tend to be local maxima in the modeled probability distribution. Instead of simply thresholding token probabilities, \mink++ examines whether a token forms a mode or has relatively high probability compared to other tokens in the vocabulary.

Given an input sequence $x = (x_1, x_2, \dots, x_T)$ and an autoregressive language model $f_\theta$, the \mink++ score is computed as:
\begin{equation}
\mathcal{S}_{\text{Min-K\%++}}(x) = \frac{1}{|S|} \sum_{t \in S} \frac{\log p(x_t | x_{<t}) - \mu_{x<t}}{\sigma_{x<t}},
\end{equation}
where $S$ consists of the least probable $K\%$ tokens in $x$, and $\mu_{x<t}$ and $\sigma_{x<t}$ are the mean and standard deviation of log probabilities across the vocabulary. Membership is determined by thresholding:
\begin{equation}
A_{f_\theta}(x) = \mathbb{1}[\mathcal{S}_{\text{Min-K\%++}}(x) \geq \gamma].
\end{equation}

{Similarly to \mink, \mink++ sweeps over $K\in\{10,20,30,40,50\}$, and the final result is reported for the best hyperparameter $K$.}


\textbf{zlib Ratio Attacks.}
A simple baseline attack leverages the compression ratio computed using the \textit{zlib library}~\citep{zlib2004}. This method compares the model’s perplexity with the sequence’s entropy, as determined by its zlib-compressed size. The attack is formalized as:
\[
A_{\fmodelm}(x) = \mathbb{1}\left[\frac{\mathcal{P}_{\fmodelm}(x)}{zlib(x)} < \gamma \right].
\]
The intuition is that samples from the training set tend to have lower perplexity for the model, while the zlib compression, being model-agnostic, does not exhibit such biases.

\textbf{CAMIA} introduces several context-aware signals to enhance membership inference accuracy. The \textit{slope signal} captures how quickly the per-token loss decreases over time, as members typically exhibit a steeper decline. \textit{Approximate entropy} quantifies the regularity of the loss sequence by measuring the frequency of repeating patterns, while \textit{Lempel-Ziv complexity} captures the diversity of loss fluctuations by counting unique substrings in the loss trajectory—both of which tend to be higher for non-members. The loss thresholding \textit{Count Below} approach computes the fraction of tokens with losses below a predefined threshold, exploiting the tendency of members to have more low-loss tokens. \textit{Repeated-sequence amplification} measures how much the loss decreases when an input is repeated, as non-members often show stronger loss reductions due to in-context learning.

\textbf{Surprising Tokens Attack (SURP).}  
SURP detects membership by identifying \textit{surprising tokens}, which are tokens where the model is highly confident in its prediction but assigns a low probability to the actual ground truth token. Seen data tends to be less surprising, meaning the model assigns higher probabilities to these tokens in familiar contexts.

For a given input $x = (x_1, x_2, \dots, x_T)$, surprising tokens are those where the Shannon entropy is low and the probability of the ground truth token is below a threshold:
\begin{equation}
S = \{t \mid H_t < \epsilon_e, \quad p(x_t | x_{<t}) < \tau_k \},
\end{equation}
where $H_t$ is the entropy of the model’s output at position $t$, {$\tau_k$ is the probability of the bottom $k\%$-th token. $k\in\{10,20,30,40,50\}$, and $\epsilon_e\in\{2,4,8,16\}$ are hyperparameters. }The SURP score is the average probability assigned to these surprising tokens:
\begin{equation}
\mathcal{S}_{\text{SURP}}(x) = \frac{1}{|S|} \sum_{t \in S} p(x_t | x_{<t}).
\end{equation}
Membership is determined by thresholding:
\begin{equation}
A_{f_\theta}(x) = \mathbb{1}[\mathcal{S}_{\text{SURP}}(x) \geq \gamma].
\end{equation}

{The SUPR's result for the best combination of $k$ and $\epsilon_e$ is selected as the final performance.}

\subsection{Dataset Inference}
\label{app:di_section}

\begin{figure}[h]
    \centering
    \includegraphics[width=1\linewidth]{figures/di_schema.pdf}
\caption{\textbf{Dataset Inference for IARs Procedural Steps.} The process consists of four main steps: 
\encircle{1}~\textit{Data Preparation:} Prepare the data to verify whether the (suspected) member samples \P were used to train the IAR. The (confirmed) nonmember samples \U, from the same distribution as \P, serve as the validation set.
\encircle{2}~\textit{Feature Extraction:} Run each individual MIA on all inputs from $\{\P,\U\}$ to extract membership features for all data samples. We use our MIAs tailored to IAR models. 
\encircle{3}~\textit{Score Computation:} Normalize the extracted features using MinMaxScaler to scale them into the [0,1] range and compute a membership score for each sample by summing its normalized feature values.
\encircle{4}~\textit{Statistical Testing:} Apply a statistical t-test to verify whether the scores obtained for the public suspect data points \P are statistically significantly higher than those for \U. If so, \P is marked as being part of the IAR's training set. Otherwise, the test is inconclusive and the IAR's training set is considered independent of \P.}
    \label{fig:di_schema}
\end{figure}


Scaling IARs to larger datasets raises concerns about the unauthorized use of proprietary or copyrighted data for training. With the growing adoption and increasing scale of IARs, this issue is becoming more pressing. In our work, we use DI to quantify the privacy leakage in IAR models. However, DI can  be additionaly used to establish a dispute-resolution framework for resolving illicit use of data collections in model training, ie. \textit{determine if a specific dataset was used to train a IAR.}  
 
 The framework involves three key roles. First, the \textit{victim} ($\mathcal{V}$) is the content creator who suspects that their proprietary or copyrighted data was used to train a IAR without permission. The victim provides a subset of samples ($\mathcal{P}$) they believe may have been included in the model's training dataset. Second, the \textit{suspect} ($\mathcal{A}$) refers to the IAR provider accused of using the victim's dataset during training. The suspect model ($f_\theta$) is examined to determine whether it demonstrates evidence of having been trained on $\mathcal{P}$. Finally, the \textit{arbiter} acts as a trusted third party, such as a regulatory body or law enforcement agency, tasked with conducting the dataset inference procedure. 
For instance, consider an artist whose publicly accessible but copyrighted artworks have been used without consent to train a IAR. The artist, acting as the victim ($\mathcal{V}$), provides a small subset of suspected training samples ($\mathcal{P}$). The IAR provider ($\mathcal{A}$) denies any infringement. An arbiter intervenes and obtains gray-box or white-box access to the suspect model. Using DI methodology, the arbiter determines whether the IAR demonstrates statistical evidence of training on $\mathcal{P}$. 

\subsection{Sampling Strategies}

The greedy approach selects the token with the highest probability. In the top-$k$ sampling, the highest $k$ token probabilities are retained, while all others are set to zero. The remaining non-zero probabilities are then re-normalized and used to determine the next token. Notably, when $k=1$, this method reduces to greedy sampling.
\section{Model Details}
\label{app:model_details}
In our experiments, we use a range of models from VAR~\cite{var_tian2024visualautoregressivemodelingscalable}, RAR\cite{rar_yu2024randomizedautoregressivevisualgeneration}, and MAR~\cite{mar_li2024autoregressiveimagegenerationvector} architectures, each varying in model size and architecture. The details of these models, including the number of parameters, training epochs, and FID scores, are summarized in ~\cref{tab:iar_model_details}.
The models were trained on the class-conditioned image generation on the ImageNet dataset~\cite{deng2009imagenet}.

\begin{table*}[h!]
    \centering
        \newcommand{\tightcolsep}{\setlength{\tabcolsep}{2pt}} %
    \tightcolsep %
    \caption{\textbf{Model details.} We report the training details for IAR the models used in this work.}
    \label{tab:iar_model_details}
        \tiny
    \begin{tabular}{l c c c c c c c c c c c}
        \toprule
        & \multicolumn{4}{c}{\textbf{VAR Models}} & \multicolumn{4}{c}{\textbf{RAR Models}} & \multicolumn{3}{c}{\textbf{MAR Models}} \\
        \cmidrule(r){2-5} \cmidrule(r){6-9} \cmidrule(r){10-12}
        & VAR-\textit{d}16 & VAR-\textit{d}20 & VAR-\textit{d}24 & VAR-\textit{d}30 & RAR-B & RAR-L & RAR-XL & RAR-XXL & MAR-B & MAR-L & MAR-H \\
        \midrule
        \textbf{Model parameters} & 310M & 600M  &  1.0B & 2.1B & 261M & 462M & 955M & 1.5B & 208M & 478M & 942M \\
        \textbf{Training epochs} & 200  & 250  & 300  & 350 & 400 & 400 & 400 & 400 & 400 & 400 &  400 \\
        \textbf{FID} & 3.55  & 2.95  & 2.33  & 1.92 & 1.95 & 1.70 & 1.50 & 1.48 & 2.31 & 1.78 & 1.55 \\
        \bottomrule
    \end{tabular}
\end{table*}


\section{Training and Inference Cost Estimation}
\label{app:pareto_how}
\begin{figure}[h]
    \centering
    \includegraphics[width=1\linewidth]{figures/pareto_full.pdf}
    \caption{\textbf{Comprehensive comparison of the trade-offs between IARs and DMs.}}
    \label{fig:pareto_app}
\end{figure}

Here we describe the comprehensive process of training and generation cost estimation of IARs and DMs, which results in the plot~\cref{fig:pareto_app}. We use \textit{torchprofile}~\citep{torchprofile} Python library to measure GFLOPs used for generation and training. 

In order to compute the training cost, the procedure is as follows. (1) We perform a single forward pass through the model. (2) We multiply the obtained GFLOPs cost by two, to accommodate for the backward pass cost. (3) We multiply the resulting cost of a single forward and backward pass by the amount of training samples passed through the model during training. The amount of samples is based on the numbers reported in the papers for each of the evaluated models. DMs and IARs use a different reporting methodology, with the former reporting training steps and a batch size, and the latter reporting the number of epochs. For the latter, we assume that a full pass through the ImageNet-1k training set is performed, thus we multiply the number of epochs by $1,281,167$.

Time to generate a single sample (referred to as latency) is computed by generating 640 images using code from the original models' repositories. We use the maximum batch size that fits on a single NVIDIA  RTX A4000 48GB GPU, to utilize our hardware to the maximum, in order to ensure a fair comparison. For DMs and IARs we follow the settings reported by authors of the respective papers that give the lowest FID score, \ie we use classifier-free guidance for all the models. For MAR we perform 64 steps of patches sampling. For all DMs but U-ViT we perform 250 steps of denoising, while for U-ViT the reported number is 50, which explains low latency of this model in comparison to others. We acknowledge that, in case of DMs, there are ways to lower the cost of the inference, \eg by lowering the number of denoising steps. However, we use the default, yet more costly setup for these models, as there is an inherent trade-off between generation quality and cost for DMs, which we want to avoid to make our results sound.

Single generation cost in GFLOPs is computed in a similar fashion. We utilize code provided by the authors of the respective papers for the inference, wrap it using \textit{torchprofile}, and perform a generation of a single sample. Note that here we do not measure time, and we can ignore the parallelism of hardware, as the total cost would stay the same. As we observe in~\cref{fig:pareto}, there is a discrepancy between latency and cost of generation, especially in case of RAR, where we observe an order of magnitude higher generation time than the GFLOPs cost would suggest. This phenomenon originates from the KV-Cache mechanism that is used in case of VAR and RAR during sampling. While the compute cost is lower thanks to the mechanism, the reading operation of the cache mechanism is not effectively parallelized, which results in hardware-incurred latency. We, however, acknowledge that this trade-off might become more beneficial in cases of low-power edge devices, as the computational power of these devices is more limited than the speed of memory operations.



\section{MIAs for MAR}
\label{app:mias_on_mar}




\paragraph{Adjusting Binary Mask}
\label{app:adjusting_mask}

MAR extends the IAR framework by incorporating masked prediction strategies, where masked tokens are predicted based on the visible ones. This design choice is inspired by Masked Autoencoders~\citep{he2022masked}, where selectively removing and reconstructing parts of the input allows models to learn better representations. Given that MIAs rely on detecting subtle differences in how models process known and unknown data, we hypothesize that adjusting the masking ratio during inference can amplify membership signals. By increasing the masking ratio from 0.86 (the training average) to 0.95, we create conditions where fewer tokens are available to reconstruct the original image, potentially exposing membership information more prominently.  

Our experimental results, reported in \cref{tab:adjusting_mask}, confirm that this strategy enhances MIAs' effectiveness. Specifically, \tprat for MAR-H increases from 2.18 to 2.88 (+0.70), and MAR-L sees an improvement from 1.89 to 2.25 (+0.36), demonstrating that a higher masking ratio strengthens membership signals. Notably, setting the mask ratio too high (e.g., 0.99) leads to a slight drop in  MIA performance, suggesting a balance must be struck between revealing more membership signal and overly degrading the model’s ability to generate images effectively.

\begin{table}[h!]
    \centering
    \scriptsize
    \caption{\textbf{Impact of varying mask ratio on MIAs for MAR.} We report \tprat. Higher values indicate stronger membership signals. The best-performing setting is highlighted in bold.}
    \label{tab:adjusting_mask}
    \begin{tabular}{llll}
\toprule
Mask Ratio & MAR-B & MAR-L & MAR-H \\
\midrule
0.75 & 1.64 (-0.05) & 1.65 (-0.24) & 1.81 (-0.37) \\
0.80 & 1.74 (+0.05) & 1.76 (-0.13) & 1.85 (-0.33) \\
0.85 & 1.68 (-0.01) & 1.83 (-0.06) & 2.00 (-0.18) \\
0.86  (default) & 1.69 (0.00) & 1.89 (0.00) & 2.18 (0.00) \\
0.90 & 1.65 (-0.04) & 1.88 (-0.01) & 2.22 (+0.05) \\
\textbf{0.95} & \textbf{1.88 (+0.19)} & \textbf{2.25 (+0.36)} & \textbf{2.88 (+0.70)} \\
0.99 & 1.77 (+0.08) & 1.86 (-0.03) & 2.14 (-0.04) \\

\bottomrule
\end{tabular}
\end{table}



\paragraph{Fixed Timestep}
\label{app:fixed_timestep}

MIAs on DMs have been shown to be most effective when conducted at a specific denoising step $t$~\citep{carlini2023extracting}. Since MAR utilizes a small diffusion module for token generation, we hypothesize that targeting MIAs at a fixed timestep $t$ rather than a randomly chosen one can similarly enhance MIA effectiveness. Unlike full-scale diffusion models, where the most discriminative timestep is typically around $t = 100$, our experiments reveal that for MAR models, the optimal timestep is $t = 500$.  

\cref{tab:fixed_timestep} illustrates the impact of this adjustment. When MIAs are performed at $t = 500$, MAR-H achieves a \tprat{} of 3.30, improving by +0.42 over the baseline random timestep approach. Similarly, MAR-L and MAR-B also see noticeable gains at this timestep. Notably, selecting timestep $t = 100$ significantly reduces the attack's effectiveness, with a drop of -0.38 for MAR-H. 


\begin{table}[h!]
    \centering
    \scriptsize
    \caption{\textbf{Impact of  using a fixed denoising timestep on MIAs for MAR performance.} We report \tprat. The most discriminative timestep is highlighted in bold.}
    \label{tab:fixed_timestep}
\begin{tabular}{llll}
\toprule
Timestep & MAR-B & MAR-L & MAR-H \\
\midrule
random & 1.88 (0.00) & 2.25 (0.00) & 2.88 (0.00) \\
100 & 1.60 (-0.27) & 1.90 (-0.34) & 2.50 (-0.38) \\
\textbf{500} & \textbf{1.88 (+0.00)} & \textbf{2.41 (+0.17)} & \textbf{3.30 (+0.42)} \\
700 & 1.85 (-0.03) & 2.35 (+0.10) & 3.20 (+0.32) \\
900 & 1.65 (-0.22) & 2.14 (-0.10) & 2.97 (+0.09) \\
\bottomrule
\end{tabular}
\end{table}

\paragraph{Reducing Diffusion Noise Variance}
\label{app:mul}

The MAR loss function, as defined in \cref{eq:dm_loss}, exhibits certain variance due to its dependence on randomly sampled noise $\epsilon$. During training, MAR uses four different noise samples per image. We hypothesize that increasing the number of noise samples can provide a more stable loss signal, thereby improving the performance of MIAs.  

Our results, summarized in \cref{tab:mul}, confirm that increasing the number of noise samples has a positive effect on attack performance. 






\begin{table}[h!]
    \centering
    \scriptsize
    \caption{\textbf{Impact of R
    reducing diffusion noise variance} on MIAs for MAR performance. We report \tprat. Obtaining loss for random noise sampled multiple times generally improves attack effectiveness. The best-performing setting is highlighted in bold.}
    \label{tab:mul}
\begin{tabular}{llll}
\toprule
Repeats & MAR-B & MAR-L & MAR-H \\
\midrule
4 (default) & 1.88 (0.00) & 2.41 (0.00) & 3.30 (0.00) \\
8 & 1.98 (+0.10) & 2.59 (+0.18) & 3.32 (+0.03) \\
16 & 2.01 (+0.13) & 2.50 (+0.09) & 3.19 (-0.11) \\
32 & 2.00 (+0.11) & 2.56 (+0.15) & 3.35 (+0.06) \\
\textbf{64} & \textbf{2.09 (+0.21)} &\textbf{ 2.61 (+0.20)} & \textbf{3.40 (+0.10)} \\

\bottomrule
\end{tabular}
\end{table}

\section{Full MIA Results}
\label{app:full_mia}
We report \tprat and AUC for each baseline MIA (\cref{tab:tpr_baseline_mias}, \cref{tab:auc_baseline_mias}, each improved MIA for IAR (\cref{tab:tpr_our_mias}, \cref{tab:auc_our_mias}) and each MIA for DMs (\cref{tab:tpr_dm}, \cref{tab:auc_dm}). Results are randomized over 100 experiments.

\begin{table}[h!]
    \centering
    \tiny
\caption{\textbf{\tprat for baseline MIAs.}}
\setlength{\tabcolsep}{3pt}
\begin{tabular}{cccccccccccc}
\toprule
        Model & VAR-$\mathit{d}$16 & VAR-$\mathit{d}$20 & VAR-$\mathit{d}$24 & VAR-$\mathit{d}$30 & MAR-B & MAR-L & MAR-H & RAR-B & RAR-L & RAR-XL & RAR-XXL \\
\midrule

Loss~\citep{yeom2018lossmia} & 1.50{\tiny $\pm$0.16} & 1.67{\tiny $\pm$0.20} & 2.19{\tiny $\pm$0.21} & 4.95{\tiny $\pm$0.38} & 1.42{\tiny $\pm$0.21} & 1.48{\tiny $\pm$0.19} & 1.60{\tiny $\pm$0.21} & 1.76{\tiny $\pm$0.24} & 2.10{\tiny $\pm$0.27} & 3.38{\tiny $\pm$0.42} & 5.70{\tiny $\pm$0.55} \\
Zlib~\citep{carlini2021extractLLM} & 1.55{\tiny $\pm$0.20} & 1.74{\tiny $\pm$0.20} & 2.24{\tiny $\pm$0.24} & 5.77{\tiny $\pm$0.59} & 1.41{\tiny $\pm$0.22} & 1.49{\tiny $\pm$0.21} & 1.59{\tiny $\pm$0.22} & 1.91{\tiny $\pm$0.23} & 2.45{\tiny $\pm$0.26} & 4.21{\tiny $\pm$0.31} & 7.52{\tiny $\pm$0.57} \\
Hinge~\citep{bertran2024scalable} & 1.62{\tiny $\pm$0.19} & 1.72{\tiny $\pm$0.22} & 2.14{\tiny $\pm$0.23} & 4.09{\tiny $\pm$0.40} & --- & --- & --- & 1.81{\tiny $\pm$0.17} & 1.99{\tiny $\pm$0.19} & 2.94{\tiny $\pm$0.36} & 5.16{\tiny $\pm$0.63} \\
Min-K\%~\citep{shi2024detecting} & 1.58{\tiny $\pm$0.16} & 2.04{\tiny $\pm$0.25} & 3.22{\tiny $\pm$0.38} & 12.23{\tiny $\pm$1.13} & 1.69{\tiny $\pm$0.18} & 1.89{\tiny $\pm$0.16} & 2.18{\tiny $\pm$0.23} & 2.09{\tiny $\pm$0.24} & 2.86{\tiny $\pm$0.32} & 5.83{\tiny $\pm$0.52} & 13.48{\tiny $\pm$0.98} \\
SURP~\citep{zhang2024adaptive} & 1.53{\tiny $\pm$0.17} & 1.70{\tiny $\pm$0.20} & 2.23{\tiny $\pm$0.23} & 5.02{\tiny $\pm$0.43} & --- & --- & --- & 1.84{\tiny $\pm$0.18} & 2.12{\tiny $\pm$0.30} & 3.46{\tiny $\pm$0.46} & 5.82{\tiny $\pm$0.53} \\
Min-K\%++~\citep{zhang2024min} & 1.34{\tiny $\pm$0.18} & 2.21{\tiny $\pm$0.28} & 3.73{\tiny $\pm$0.34} & 14.90{\tiny $\pm$0.96} & --- & --- & --- & 2.36{\tiny $\pm$0.29} & 3.26{\tiny $\pm$0.30} & 6.27{\tiny $\pm$0.65} & 14.63{\tiny $\pm$0.87} \\
CAMIA~\citep{chang2024context} & 1.33{\tiny $\pm$0.18} & 1.76{\tiny $\pm$0.19} & 3.07{\tiny $\pm$0.35} & 16.69{\tiny $\pm$1.16} & 1.35{\tiny $\pm$0.19} & 1.38{\tiny $\pm$0.19} & 1.44{\tiny $\pm$0.23} & 1.51{\tiny $\pm$0.17} & 1.78{\tiny $\pm$0.15} & 1.99{\tiny $\pm$0.34} & 4.34{\tiny $\pm$0.51} \\


\bottomrule
\end{tabular}
\label{tab:tpr_baseline_mias}
\end{table}


\begin{table}[h!]
\centering
\tiny
\caption{\textbf{AUC for baseline MIAs.}}
\setlength{\tabcolsep}{3pt}
\begin{tabular}{cccccccccccc}
\toprule
        Model & VAR-$\mathit{d}$16 & VAR-$\mathit{d}$20 & VAR-$\mathit{d}$24 & VAR-$\mathit{d}$30 & MAR-B & MAR-L & MAR-H & RAR-B & RAR-L & RAR-XL & RAR-XXL \\
\midrule
Loss~\citep{yeom2018lossmia} & 52.35{\tiny $\pm$0.35} & 54.53{\tiny $\pm$0.34} & 59.55{\tiny $\pm$0.35} & 75.45{\tiny $\pm$0.30} & 51.92{\tiny $\pm$0.36} & 53.33{\tiny $\pm$0.36} & 55.06{\tiny $\pm$0.34} & 54.92{\tiny $\pm$0.37} & 58.04{\tiny $\pm$0.37} & 65.59{\tiny $\pm$0.34} & 74.45{\tiny $\pm$0.30} \\
Zlib~\citep{carlini2021extractLLM} & 52.38{\tiny $\pm$0.38} & 54.59{\tiny $\pm$0.38} & 59.65{\tiny $\pm$0.37} & 75.67{\tiny $\pm$0.34} & 51.91{\tiny $\pm$0.39} & 53.32{\tiny $\pm$0.39} & 55.05{\tiny $\pm$0.38} & 55.27{\tiny $\pm$0.36} & 58.68{\tiny $\pm$0.35} & 66.85{\tiny $\pm$0.34} & 76.17{\tiny $\pm$0.30} \\
Hinge~\citep{bertran2024scalable} & 53.29{\tiny $\pm$0.39} & 56.83{\tiny $\pm$0.39} & 62.89{\tiny $\pm$0.39} & 77.36{\tiny $\pm$0.33} & --- & --- & --- & 57.07{\tiny $\pm$0.44} & 61.41{\tiny $\pm$0.44} & 71.48{\tiny $\pm$0.39} & 82.14{\tiny $\pm$0.29} \\
Min-K\%~\citep{shi2024detecting} & 53.77{\tiny $\pm$0.40} & 57.84{\tiny $\pm$0.44} & 65.49{\tiny $\pm$0.40} & 83.55{\tiny $\pm$0.30} & 51.87{\tiny $\pm$0.38} & 53.29{\tiny $\pm$0.38} & 55.05{\tiny $\pm$0.38} & 56.53{\tiny $\pm$0.38} & 61.21{\tiny $\pm$0.36} & 71.35{\tiny $\pm$0.32} & 82.33{\tiny $\pm$0.28} \\
SURP~\citep{zhang2024adaptive} & 50.46{\tiny $\pm$0.25} & 54.54{\tiny $\pm$0.38} & 59.60{\tiny $\pm$0.40} & 75.46{\tiny $\pm$0.34} & --- & --- & --- & 52.21{\tiny $\pm$0.40} & 58.02{\tiny $\pm$0.42} & 65.58{\tiny $\pm$0.41} & 74.50{\tiny $\pm$0.33} \\
Min-K\%++~\citep{zhang2024min} & 54.52{\tiny $\pm$0.41} & 57.93{\tiny $\pm$0.38} & 65.76{\tiny $\pm$0.38} & 85.33{\tiny $\pm$0.27} & --- & --- & --- & 57.82{\tiny $\pm$0.41} & 62.48{\tiny $\pm$0.38} & 75.61{\tiny $\pm$0.32} & 85.16{\tiny $\pm$0.26} \\
CAMIA~\citep{chang2024context} & 52.44{\tiny $\pm$0.44} & 55.12{\tiny $\pm$0.44} & 61.37{\tiny $\pm$0.42} & 80.16{\tiny $\pm$0.34} & 51.08{\tiny $\pm$0.42} & 51.96{\tiny $\pm$0.43} & 53.20{\tiny $\pm$0.38} & 51.40{\tiny $\pm$0.36} & 51.83{\tiny $\pm$0.39} & 59.28{\tiny $\pm$0.39} & 66.07{\tiny $\pm$0.36} \\

\bottomrule
\end{tabular}
\label{tab:auc_baseline_mias}
\end{table}



\begin{table}[h!]
    \centering
    \tiny
\caption{\textbf{\tprat for our improved MIAs for IARs.}}
\setlength{\tabcolsep}{3pt}
\begin{tabular}{cccccccccccc}
\toprule
        Model & VAR-$\mathit{d}$16 & VAR-$\mathit{d}$20 & VAR-$\mathit{d}$24 & VAR-$\mathit{d}$30 & MAR-B & MAR-L & MAR-H & RAR-B & RAR-L & RAR-XL & RAR-XXL \\
\midrule
Loss~\citep{yeom2018lossmia} & 2.16{\tiny $\pm$0.26} & 5.95{\tiny $\pm$0.54} & 24.03{\tiny $\pm$1.91} & 86.38{\tiny $\pm$0.92} & 1.54{\tiny $\pm$0.22} & 1.81{\tiny $\pm$0.21} & 2.26{\tiny $\pm$0.26} & 2.86{\tiny $\pm$0.20} & 5.50{\tiny $\pm$0.39} & 16.58{\tiny $\pm$0.97} & 40.76{\tiny $\pm$1.87} \\
Zlib~\citep{carlini2021extractLLM} & 1.75{\tiny $\pm$0.17} & 4.87{\tiny $\pm$0.41} & 20.37{\tiny $\pm$1.19} & 83.99{\tiny $\pm$0.87} & 1.51{\tiny $\pm$0.21} & 1.80{\tiny $\pm$0.23} & 2.23{\tiny $\pm$0.27} & 2.52{\tiny $\pm$0.31} & 4.56{\tiny $\pm$0.39} & 13.91{\tiny $\pm$1.02} & 41.03{\tiny $\pm$1.96} \\
Hinge~\citep{bertran2024scalable} & 0.00{\tiny $\pm$0.00} & 0.00{\tiny $\pm$0.00} & 0.00{\tiny $\pm$0.00} & 0.00{\tiny $\pm$0.00} & --- & --- & --- & 2.50{\tiny $\pm$0.20} & 4.34{\tiny $\pm$0.39} & 10.59{\tiny $\pm$0.88} & 20.23{\tiny $\pm$1.85} \\
Min-K\%~\citep{shi2024detecting} & 0.05{\tiny $\pm$0.02} & 0.06{\tiny $\pm$0.02} & 0.14{\tiny $\pm$0.04} & 1.63{\tiny $\pm$0.13} & 2.09{\tiny $\pm$0.23} & 2.6{\tiny $\pm$0.28} & 3.40{\tiny $\pm$0.30} & 4.30{\tiny $\pm$0.33} & 8.66{\tiny $\pm$0.79} & 26.14{\tiny $\pm$1.22} & 49.80{\tiny $\pm$2.15} \\
Min-K\%++~\citep{zhang2024min} & 0.39{\tiny $\pm$0.06} & 1.40{\tiny $\pm$0.11} & 4.88{\tiny $\pm$0.20} & 37.90{\tiny $\pm$0.44} & --- & --- & --- & 4.19{\tiny $\pm$0.40} & 8.24{\tiny $\pm$0.66} & 23.04{\tiny $\pm$1.14} & 43.67{\tiny $\pm$2.32} \\
CAMIA~\citep{chang2024context} & 1.83{\tiny $\pm$0.25} & 5.46{\tiny $\pm$0.52} & 20.92{\tiny $\pm$1.14} & 72.77{\tiny $\pm$1.04} & 1.00{\tiny $\pm$0.17} & 0.97{\tiny $\pm$0.13} & 1.06{\tiny $\pm$0.15} & 1.63{\tiny $\pm$0.21} & 2.60{\tiny $\pm$0.27} & 6.77{\tiny $\pm$0.47} & 17.85{\tiny $\pm$1.20} \\
\bottomrule
\end{tabular}
\label{tab:tpr_our_mias}
\end{table}


\begin{table}[h!]
\centering
\tiny
\setlength{\tabcolsep}{3pt}
\caption{\textbf{AUC for our improved MIAs for IARs.}}
\begin{tabular}{cccccccccccc}
\toprule
        Model & VAR-$\mathit{d}$16 & VAR-$\mathit{d}$20 & VAR-$\mathit{d}$24 & VAR-$\mathit{d}$30 & MAR-B & MAR-L & MAR-H & RAR-B & RAR-L & RAR-XL & RAR-XXL \\
\midrule
Loss~\citep{yeom2018lossmia} & 61.73{\tiny $\pm$0.33} & 76.26{\tiny $\pm$0.30} & 92.20{\tiny $\pm$0.15} & 98.95{\tiny $\pm$0.05} & 52.25{\tiny $\pm$0.42} & 54.60{\tiny $\pm$0.41} & 57.35{\tiny $\pm$0.40} & 65.61{\tiny $\pm$0.35} & 75.83{\tiny $\pm$0.32} & 89.64{\tiny $\pm$0.21} & 96.17{\tiny $\pm$0.12} \\
Zlib~\citep{carlini2021extractLLM} & 57.91{\tiny $\pm$0.39} & 70.86{\tiny $\pm$0.33} & 88.69{\tiny $\pm$0.24} & 98.51{\tiny $\pm$0.07} & 52.23{\tiny $\pm$0.39} & 54.57{\tiny $\pm$0.39} & 57.33{\tiny $\pm$0.39} & 62.22{\tiny $\pm$0.42} & 72.19{\tiny $\pm$0.37} & 87.51{\tiny $\pm$0.22} & 95.46{\tiny $\pm$0.13} \\
Hinge~\citep{bertran2024scalable} & 52.67{\tiny $\pm$0.36} & 56.11{\tiny $\pm$0.36} & 62.48{\tiny $\pm$0.36} & 74.63{\tiny $\pm$0.30} & --- & --- & --- & 59.66{\tiny $\pm$0.39} & 68.09{\tiny $\pm$0.35} & 81.56{\tiny $\pm$0.29} & 90.62{\tiny $\pm$0.21} \\
Min-K\%~\citep{shi2024detecting} & 59.78{\tiny $\pm$0.34} & 70.43{\tiny $\pm$0.34} & 83.10{\tiny $\pm$0.25} & 90.16{\tiny $\pm$0.27} & 53.31{\tiny $\pm$0.40} & 56.34{\tiny $\pm$0.39} & 59.98{\tiny $\pm$0.38} & 66.81{\tiny $\pm$0.38} & 78.73{\tiny $\pm$0.32} & 91.36{\tiny $\pm$0.20} & 96.97{\tiny $\pm$0.10} \\
Min-K\%++~\citep{zhang2024min} & 57.10{\tiny $\pm$0.30} & 65.44{\tiny $\pm$0.29} & 78.74{\tiny $\pm$0.25} & 93.18{\tiny $\pm$0.16} & --- & --- & --- & 65.20{\tiny $\pm$0.36} & 75.37{\tiny $\pm$0.34} & 88.29{\tiny $\pm$0.23} & 95.84{\tiny $\pm$0.14} \\
CAMIA~\citep{chang2024context} & 56.37{\tiny $\pm$0.38} & 68.18{\tiny $\pm$0.31} & 84.83{\tiny $\pm$0.24} & 96.95{\tiny $\pm$0.09} & 50.86{\tiny $\pm$0.41} & 51.15{\tiny $\pm$0.41} & 51.75{\tiny $\pm$0.41} & 57.95{\tiny $\pm$0.40} & 63.17{\tiny $\pm$0.43} & 70.43{\tiny $\pm$0.39} & 83.55{\tiny $\pm$0.31} \\
\bottomrule
\end{tabular}
\label{tab:auc_our_mias}
\end{table}




\begin{table}[h!]
    \centering
    \tiny
    \setlength{\tabcolsep}{3pt}
    \caption{\textbf{\tprat of MIAs for DMs.}}
    \begin{tabular}{ccccccccc}
    \toprule
     & LDM & U-ViT-H/2 & DiT-XL/2 & MDTv1-XL/2 & MDTv2-XL/2 & DiMR-XL/2R & DiMR-G/2R & SiT-XL/2 \\
    \midrule
    Denoising Loss~\citep{carlini2023extracting} & 1.35{\tiny $\pm$0.14} & 1.30{\tiny $\pm$0.17} & 1.42{\tiny $\pm$0.17} & 1.55{\tiny $\pm$0.18} & 1.64{\tiny $\pm$0.17} & 0.91{\tiny $\pm$0.15} & 0.88{\tiny $\pm$0.15} & 1.02{\tiny $\pm$0.13} \\
    SecMI$_{stat}$~\citep{dm2_duan23bSecMI} & 1.30{\tiny $\pm$0.20} & 1.31{\tiny $\pm$0.19} & 1.49{\tiny $\pm$0.22} & 1.35{\tiny $\pm$0.17} & 1.52{\tiny $\pm$0.22} & 1.15{\tiny $\pm$0.21} & 1.05{\tiny $\pm$0.15} & 0.00{\tiny $\pm$0.00} \\
    PIA~\citep{kong2023efficient} & 1.25{\tiny $\pm$0.16} & 1.25{\tiny $\pm$0.19} & 1.59{\tiny $\pm$0.20} & 1.72{\tiny $\pm$0.20} & 2.07{\tiny $\pm$0.24} & 1.07{\tiny $\pm$0.11} & 1.09{\tiny $\pm$0.12} & 1.14{\tiny $\pm$0.14} \\
    PIAN~\citep{kong2023efficient} & 1.03{\tiny $\pm$0.14} & 1.17{\tiny $\pm$0.16} & 0.92{\tiny $\pm$0.12} & 1.22{\tiny $\pm$0.15} & 1.50{\tiny $\pm$0.20} & 1.04{\tiny $\pm$0.13} & 1.01{\tiny $\pm$0.12} & 1.09{\tiny $\pm$0.14} \\
    GM~\citep{dubinski2024cdicopyrighteddataidentification} & 1.25{\tiny $\pm$0.17} & 1.26{\tiny $\pm$0.17} & 1.34{\tiny $\pm$0.17} & 1.18{\tiny $\pm$0.16} & 1.47{\tiny $\pm$0.19} & 1.13{\tiny $\pm$0.15} & 1.16{\tiny $\pm$0.16} & 1.38{\tiny $\pm$0.18} \\
    ML~\citep{dubinski2024cdicopyrighteddataidentification} & 1.41{\tiny $\pm$0.16} & 1.36{\tiny $\pm$0.20} & 1.50{\tiny $\pm$0.18} & 1.70{\tiny $\pm$0.16} & 1.98{\tiny $\pm$0.26} & 1.01{\tiny $\pm$0.15} & 1.10{\tiny $\pm$0.14} & 1.14{\tiny $\pm$0.12} \\
    CLiD~\citep{zhai2024clid} & 1.55{\tiny $\pm$0.19} & 1.75{\tiny $\pm$0.22} & 2.08{\tiny $\pm$0.28} & 2.72{\tiny $\pm$0.39} & 4.91{\tiny $\pm$0.44} & 0.96{\tiny $\pm$0.14} & 0.90{\tiny $\pm$0.13} & 6.38{\tiny $\pm$0.64} \\
    \bottomrule
    \end{tabular}
    \label{tab:tpr_dm}
\end{table}



\begin{table}[h!]
    \centering
    \tiny
    \setlength{\tabcolsep}{3pt}
    \caption{\textbf{AUC for MIAs for DMs.}}
    \begin{tabular}{ccccccccc}
    \toprule
     & LDM & U-ViT-H/2 & DiT-XL/2 & MDTv1-XL/2 & MDTv2-XL/2 & DiMR-XL/2R & DiMR-G/2R & SiT-XL/2 \\
    \midrule
    Denoising Loss~\citep{carlini2023extracting} & 50.53{\tiny $\pm$0.41} & 50.36{\tiny $\pm$0.42} & 51.77{\tiny $\pm$0.43} & 51.25{\tiny $\pm$0.37} & 51.65{\tiny $\pm$0.37} & 46.25{\tiny $\pm$0.40} & 46.01{\tiny $\pm$0.40} & 47.25{\tiny $\pm$0.34} \\
    SecMI$_{stat}$~\citep{dm2_duan23bSecMI} & 49.84{\tiny $\pm$0.44} & 53.15{\tiny $\pm$0.43} & 55.15{\tiny $\pm$0.46} & 54.44{\tiny $\pm$0.38} & 56.80{\tiny $\pm$0.36} & 48.73{\tiny $\pm$0.45} & 48.73{\tiny $\pm$0.44} & 50.00{\tiny $\pm$0.00} \\
    PIA~\citep{kong2023efficient} & 48.97{\tiny $\pm$0.43} & 51.77{\tiny $\pm$0.44} & 53.18{\tiny $\pm$0.42} & 52.60{\tiny $\pm$0.44} & 54.68{\tiny $\pm$0.45} & 47.31{\tiny $\pm$0.42} & 47.16{\tiny $\pm$0.41} & 49.13{\tiny $\pm$0.44} \\
    PIAN~\citep{kong2023efficient} & 49.56{\tiny $\pm$0.43} & 50.99{\tiny $\pm$0.46} & 50.14{\tiny $\pm$0.43} & 49.96{\tiny $\pm$0.42} & 51.52{\tiny $\pm$0.38} & 49.85{\tiny $\pm$0.41} & 49.79{\tiny $\pm$0.43} & 50.17{\tiny $\pm$0.37} \\
    GM~\citep{dubinski2024cdicopyrighteddataidentification} & 51.51{\tiny $\pm$0.40} & 51.19{\tiny $\pm$0.42} & 50.46{\tiny $\pm$0.46} & 50.72{\tiny $\pm$0.39} & 48.85{\tiny $\pm$0.37} & 45.97{\tiny $\pm$0.45} & 45.86{\tiny $\pm$0.45} & 50.94{\tiny $\pm$0.38} \\
    ML~\citep{dubinski2024cdicopyrighteddataidentification} & 50.36{\tiny $\pm$0.41} & 51.16{\tiny $\pm$0.41} & 52.53{\tiny $\pm$0.45} & 50.42{\tiny $\pm$0.19} & 54.65{\tiny $\pm$0.38} & 46.26{\tiny $\pm$0.38} & 49.37{\tiny $\pm$0.41} & 49.83{\tiny $\pm$0.17} \\
    CLiD~\citep{zhai2024clid} & 52.50{\tiny $\pm$0.39} & 54.27{\tiny $\pm$0.41} & 56.16{\tiny $\pm$0.41} & 57.43{\tiny $\pm$0.41} & 62.54{\tiny $\pm$0.40} & 46.20{\tiny $\pm$0.38} & 45.95{\tiny $\pm$0.41} & 78.65{\tiny $\pm$0.30} \\
    \bottomrule
    \end{tabular}
    \label{tab:auc_dm}
\end{table}



\vspace{5cm}
\section{Full DI Results}
\label{app:full_di}

We report the outcome of DI for DMs in \cref{tab:di_dm}. As an additional observation, we note that contrary to DI for IARs, shifting from the classifier to an alternative feature aggregation increases the number of samples needed to reject $H_0$. This suggests, that the linear classifier remains
necessary for DMs.


\begin{table}[h!]
\caption{\textbf{DI for DMs.} We report the minimal number of samples needed to successfully reject $H_0$.}
    \centering
    \scriptsize
    \setlength{\tabcolsep}{3pt}
    \begin{tabular}{ccccccccc}
    \toprule
     & LDM & U-ViT-H/2 & DiT-XL/2 & MDTv1-XL/2 & MDTv2-XL/2 & DiMR-XL/2R & DiMR-G/2R & SiT-XL/2 \\
    \midrule
     DI for DM & 4000 & 700 & 400 & 300 & 200 & 2000 & 200 & 300 \\
    \midrule
     No Classifier & 5000 & 4000 & 3000 & 600 & 400 & 2000 & 2000 & 500 \\
    \bottomrule
    \end{tabular}
    \label{tab:di_dm}
\end{table}









\section{Mitigation Strategy}
\label{app:mitigation}


In this section we detail our privacy risk mitigation strategy.


\begin{figure}[h!]
    \centering
    \includegraphics[width=1\linewidth]{figures/defense_eval.pdf}
    \caption{\textbf{Privacy-utility trade-off of our mitigation strategy.} We show that successfully defending VAR and RAR against MIA and DI requires adding noise that severely harms the performance. Interestingly, we are able to limit the extent of memorization for VAR, and fully defend MAR against MIA and DI.}
    \label{fig:defense}
\end{figure}


\subsection{Method}
Given an input sample $x$, we perturb the output of the IAR according to a noise scale $\sigma$, which we can adjust to balance privacy-utility trade-off. During inference, we add noise sampled from $\mathcal{N}(0,\sigma)$ to the output. For VAR and RAR, we add it to the logits, and for MAR we add them to the sampled continuous tokens.

We measure privacy leakage with our methods from~\cref{sec:our_priv_eval}. Specifically, we perform MIAs, DI, and the extraction attack. To quantify utility, we generate 10,000 images from the IARs, and compute FID~\citep{heusel2017gans} between generations and the validation set. Lower FID means better quality of the generations.

\subsection{Results}

Our results in~\cref{fig:defense} show that we can effectively lower the privacy loss by applying our mitigation strategy, however, this comes at a cost of significantly decreased utility, as highlighted by substantially increasing FID score. 

We are able to lower the MIAs success by more than half (Fig.~\ref{fig:defense}, left), with the biggest relative drop observed for RAR-XL, for which the \tprat drops from 26\% to \textbf{4.4\%}. Moreover, \textit{all} MAR models become immune to MIAs after noising their tokens, as \tprat drops to 1\% (random guessing) with $\sigma=0.001$. 
When we apply our defense to DI (Fig.~\ref{fig:defense}, second from the left), we have to increase $P$, the minimum number required to perform a successful DI attack, by an order of magnitude, with the biggest relative difference for the smallest models: VAR-16, and RAR-B, with an increase from 80 to 3000, and 200 to 8000, respectively. Such an increase means that the models are harder to attack with DI, \ie their privacy protection is boosted. Similarly to MIA, DI stops working for MAR models immediately.

Our method achieves limited success in mitigating extraction (Fig.~\ref{fig:defense}, third from the left). We are lowering the success of extraction attack only when adding significant amount of noise. However, for \varbig, which exhibits the biggest memorization, with $\sigma=1.0$ we successfully protect \textbf{93} out of 698 samples from being extracted without significantly harming the utility.
Our method, similarly to all defenses, suffers from lowered performance (Fig.~\ref{fig:defense}, right), as signal-to-noise ratio during generation gets worse when $\sigma$ increases.

\subsection{Discussion}

We show that we can mitigate privacy risks by adding noise to the outputs of IARs, at a cost of utility. Notably, all MARs become \textit{fully} immune to MIAs and DI with noise scale as small as $0.001$. This result supports previous insights from~\cref{sec:our_priv_eval}, in which we show that MARs are significantly less prone to privacy risks than VARs and RARs. We argue that logits leak significantly more information than continuous tokens, and thus, adding noise to the latter yields significantly higher protection, at a lower performance cost.

We acknowledge that our privacy leakage defense is a heuristic, and more theoretically sound approaches should be explored, \eg in the domain of Differential Privacy~\citep{dwork2006differential}. To the best of our knowledge, we make the first step towards private IARs.

\section{More About Memorization}
\label{app:more_memorization}

In this section we provide an extended analysis of memorization phenomenon in IARs. We show more examples of memorized images, highlight the relation between the prefix length $i$ and the number of extracted samples, and shed more light on our efficient extraction method, described in~\cref{sec:memorization}.

\subsection{More Memorized Images}
\label{app:more_memorization_images}
In~\cref{fig:more_memorization} we show a non-cherry-picked set of images memorized by IARs. In~\cref{fig:var30_mem_zero} we show an example of an image memorized verbatim by \varbig \textbf{without any prefix}, \ie only from the class label token. In~\cref{fig:mem_uni1} we show an image that has been memorized by both \varbig and RAR-XXL.

\begin{figure}[h]
    \centering
    \includegraphics[width=0.85\linewidth]{figures/mem_0_var30.png}
    \caption{\textbf{Image extracted from VAR-\textit{d}30 \textit{without prefix}.} (Left) memorized image, (right) generated image.}
    \label{fig:var30_mem_zero}
\end{figure}

\begin{figure}[h]
    \centering
    \includegraphics[width=0.975\linewidth]{figures/mem_uni1.png}
    \caption{\textbf{Images extracted from both VAR-\textit{d}30, and RAR-XXL.}}
    \label{fig:mem_uni1}
\end{figure}

\begin{figure}[h]
    \centering
    \includegraphics[width=0.975\linewidth]{figures/mem_uni2.png}
    \caption{\textbf{An image extracted from both VAR-\textit{d}30, and MAR-H.}}
    \label{fig:mem_uni2}
\end{figure}


\subsection{Prefix Length vs. Number of Extracted Images}
\label{app:more_memorization_i}

We analyze the effect of the prefix length on the number of extracted samples. As our method leverages conditioning on a part of the input sequence, in~\cref{fig:mem_prefix} we show an increase of extraction success with the increase in the length of the prefix. Notably, we start experiencing false-positives once the prefix length surpasses $30$ for \varbig and RAR-XXL, and $5$ for MAR-H. In effect, the results in~\cref{tab:mem_how_many} provide an upper bound of the success of our extraction method.

\begin{figure}[h]
    \centering
    \includegraphics[width=0.5\linewidth]{figures/prefix_length.pdf}
    \caption{\textbf{Prefix length and the number of extracted samples.} We show that with an increase of the prefix length, the success of our extraction method increases.}
    \label{fig:mem_prefix}
\end{figure}

\begin{table}[h]
    \centering
    \scriptsize
    \caption{Prefix length $i$ for our data extraction attack. We note that appending longer sequences leads to false positives, \ie the IARs start to generate images from the validation set.}
    \begin{tabular}{cccc}
    \toprule
        Model & \varbig & MAR-H & RAR-XXL \\
    \midrule
        Prefix length $i$ & 30 & 5 & 30 \\
    \bottomrule
    \end{tabular}
    \label{tab:prefix_length_models}
\end{table}

\subsection{Approximate distance vs. SSCD Score}
\label{app:more_memorization_distance}

In this section we underscore the effectiveness of our filtering approach.~\cref{fig:distance_vs_sscd} shows that the distances we design for the candidate selection process indeed correlates with the SSCD score. By focusing only on the top-$5$ samples for each class we effectively narrow our search to just $0.5\%$ of the training set, significantly speeding up the whole process.

\begin{figure}[h]
    \centering
    \includegraphics[width=0.5\linewidth]{figures/memorization_distance_score.png}
    \caption{\textbf{Distance function $d$ and the SSCD score.} We show that $d$ correlates with the final memorization score. This result makes our candidate selection process sound, and reduces the cost of extracting memorized samples.}
    \label{fig:distance_vs_sscd}
\end{figure}

\begin{figure}
    \centering
    \includegraphics[width=0.95\linewidth]{figures/mem_appendix.png}
    \caption{\textbf{Non-cherry-picked extracted images.} Odd columns from the left correspond to the original image, even to extracted. From left, the images are for VAR-\textit{d}30, RAR-XXL, and MAR-H.}
    \label{fig:more_memorization}
\end{figure}


\end{document}
