In this section, we compare $\pname$ to existing cross-chain systems. The state-of-the-art cross-rollup composability solution is SVS~\cite{shared-val-seq-23} and variants, such as the CIRC protocol~\cite{espresso-circ-24} by Espresso. We provide a comprehensive comparison of $\pname$ with SVS in~\cref{sec:overview}.

\myparagraph{Cross-chain Interoperability (IO)} Existing cross-chain IO~\cite{chainlink-ccip} protocols primarily adopt the framework of atomic swaps and cross-chain bridges. Atomic swap protocols usually make use of smart contracts, hashlocks, and timelocks~\cite{bitcoin-wiki-htlc,wadhwaHeHTLC2023,tsabaryMADHTLC2021} to ensure the atomic execution of cross-chain transactions. Cross chain bridges~\cite{xiew-zkbridge-22, axelar, near-bridge} connect two blockchains and facilitate the transfer of assets between them. While they enable services such as asset transfers, they cannot offer full cross-chain composability, since they lack a coordination mechanism. 
Current IO protocols~\cite{alt-chains-atomic-transfers, bitcoin-wiki-htlc, herlihy-atomic-cc-swaps-18, herlihy-cc-smr-22, sheff-het-paxos-20, anoma-chimera-chains-23} cannot accommodate complex cross-chain transactions such as flash loans without relying on strong trust assumptions.

\myparagraph{Cross-chain Composability}
Several solutions~\cite{lu-atomic-cc-interactions-24, fal-tccsci-23, zakhary-ac3-20} ensure atomicity for multi-L1 transactions, such as the 2PC4BC protocol~\cite{fal-tccsci-23}, which provides atomicity and serializability but requires $O(n)$ rounds for a transaction of length $n$. These L1 protocols rely on complex locking mechanisms and multiple rounds, making certain applications, like cross-chain flash loans, impractical. In contrast, $\pname$ operates on L2 and requires $O(1)$ rounds by executing transactions off-chain and including them in a single rollup batch. 
For instance, 2PC4BC requires at least 5 L1 rounds ($\sim60$ blocks) for flash loan execution and finality, whereas $\pname$ performs execution off-chain, with L1 involvement only needed for finality.



Shared sequencer networks like Espresso~\cite{espresso-docs}, Astria~\cite{astria-docs}, and Radius~\cite{radius-docs} leverage a shared sequencer to achieve censorship resistance and faster pre-confirmation. While these solutions enable certain types of cross-rollup transactions, their guarantees are limited to atomic inclusion (not execution) and rely on trusting the shared sequencer. 


The Superchain~\cite{superchain-explainer} is a network of layer 2 chains built on the OP Stack~\cite{optimism-website}, enabling resource sharing and asynchronous messaging but relying on a shared sequencer for cross-chain atomicity. Polygon AggLayer~\cite{agglayer} connects layer 1 and layer 2 networks via a unified bridge, supporting token transfers and message passing, but its cross-chain transactions require Ethereum settlement, leading to high asynchrony.









