
For an executed action $a$, we say $a$ is \emph{action-wrapped} if $\act{}(a.\desc{})$ is in fact executed -- instead of plain $a$. 
Similarly, we say $a$ is \emph{triggering} if $a.\descnext{} \neq \nul$, i.e., it makes at least one call to $\gsc.\trig(\cdot)$. Given a CRT $\ccrt = [a_i]$, notice that $a_1$ is the only CRT action that is \emph{not} action-wrapped. This is because $a_1$ is the user-signed entrypoint action and is executed as is, thus its $\desc{}$ field is \emph{not} inserted in the action tree. Similarly, $a_n$ is the only non-triggering action of the CRT.  






\begin{proof}[Proof of \cref{lem:svs-gsc-honest-or-reverse-order}]
    The ``only if'' direction corresponds to the honest execution; its proof is simple and omitted. We focus on the ``if'' direction. Without loss of generality, assume $n = 2k$ is even and $a_1$ is executed on $\rol{1}$. This means that the last action $a_{2k}$ is executed on $\rol{2}$. 
    Let $[i_1, \dots, i_k]$ denote the relative order of actions $\{a_{2i-1}\}_{i \in [k]}$ in $\batch{1}$, and similarly let $[j_1, \dots, j_k]$ be the relative order of actions $\{a_{2j}\}_{j \in [k]}$ in $\batch{2}$. Let $f \in [k]$ (for ``first'') be the position of the first action, i.e, $i_f = 1$, and let $l \in [k]$ (for ``last'') denote the position of the last action, i.e., $j_l = 2k$.

    Since the last action is on $\rol{2}$, all $\rol{1}$ actions $[a_{i_r}]_{r \in [k]}$ trigger another action, so the (non-null) descriptions $[a_{i_r}.\descnext{}]_{r \in [k]}$ are inserted into $\ttree_1$ in this order. Similarly, since the first action is on $\rol{1}$, all $\rol{2}$ actions are action-wrapped, so the the (non-null) descriptions $[a_{j_r}.\desc{}]_{r \in [k]}$ are inserted into $\atree_2$ in this order. Thus we must have $j_r = i_r + 1$ for all $r \in [k]$, since otherwise $\ttree_1' \neq \atree_2'$. 
    
    \noindent\textbf{Case 1:} $l = 1$. Then each $\rol{2}$ action except for $a_{j_1} = a_{2k}$ triggers another action, so the (non-null) descriptions $[a_{j_r}.\descnext{}]_{2 \leq r  \leq k}$ are inserted in $\ttree_2$ in that order. Similarly, each $\rol{1}$ action except for $a_{i_f}$ action-wrapped, so the (non-null) descriptions $[a_{i_r}.\desc{}]_{r \neq f}$ are inserted in $\atree_1$ in that order. In order to have $\ttree_2' = \atree_1'$, it must be $i_r = j_{r+1} + 1$ for all $r < f$ and $i_r = j_r$ for all $r > f$. If $f < k$, then we have both $i_{k} = j_{k}$ and $j_{k} = i_{k} + 1$, which is impossible; thus it must be $f = k$. Combining we get $i_{r+1} = i_r - 2$ and $j_{r+1} = j_r - 2$ which together with $l=1$ lead to closed forms $i_r = 2(j-r) + 1$ and $j_r = 2(j-r) + 2$. This means $[i_r]_{r \in [k]}$ (resp. $[j_r]_{r \in [k]}$) is exactly the \emph{reverse} of the honest order $[2i - 1]_{i \in [k]}$ (resp. $[2i]_{i \in [k]}$).
    
    \noindent\textbf{Case 2:} $l \geq 2$. Similar to case 1, the (non-null) descriptions $[a_{j_r}.\descnext{}]_{r \neq l}$ are inserted in $\ttree_2$ in that order, and the (non-null) descriptions $[a_{i_r}.\desc{}]_{r \neq f}$ are inserted in $\atree_1$ in that order. If $f \geq 2$, and since $l \geq 2$, we have both $i_{1} = j_{1}$ and $j_{1} = i_{1} + 1$, which is impossible; thus it must be $f = 1$, i.e., $a_{i_1} := a_1$. Similarly, if $l < k$, we reach the impossibility of $i_{k} = j_{k}$ and $j_{k} = i_{k} + 1$; thus it must be $l = k$. Combining yields closed forms $i_r = 2r - 1$ and $j_r = 2r$, i.e., both $[i_r]_{r \in [k]}$ and $[j_r]_{r \in [k]}$ precisely match the honest order.
\end{proof}


\begin{proof}[Proof of \cref{lem:chain-gsc-honest-order}]
    The ``only if'' direction corresponds to the honest execution; its proof is simple and thus omitted. We prove the contrapositive of the ``if'' direction, i.e., we begin by assuming $\batch{1}$ and $\batch{2}$ \emph{do not} preserve the relative order of actions, and we wish to show that at least one of (I,II,III) is not preserved. 
    Similar to the proof of \cref{lem:svs-gsc-honest-or-reverse-order}, assume $n = 2k$ is even and $a_1$ is executed on $\rol{1}$. This means that the last action $a_{2k}$ is executed on $\rol{2}$. If the actions' order is \emph{not} fully reversed in the two batches, then \cref{lem:svs-gsc-honest-or-reverse-order} implies that (I) is not preserved, so we are done. For the rest of the proof, we assume the actions' order is fully reversed in the two batches. Given that (III) holds under $(\dst{1}, \dst{2})$, we have two cases:
    
    \noindent\textbf{Case 1:} $\sactive{1} = \false$. The first CRT action from $\batch{1}$ to be executed is $a_{2k-1}$, which is action-wrapped. Notice that it must use $\sid{1} = \snonce{1} + 1$; if it uses $\sid{1} = \snonce{1}$, then $\textproc{checkSessionID}$ will fail since $\sactive{1} = \false$. Next, $\textproc{checkSessionID}$ will set $\snonce{1}' = \snonce{1} + 1$ and $\sactive{1}' = \true$. Since $\sid{1}$ is part of the tuple inserted in $\atree_1$, all other action-wrapped actions must be using $\sid{1}$ as their session id, too; if not, then the trees will not match and (I) is not preserved. After executing all actions except $a_1$, we have that $\snonce{1}' = \snonce{1} + 1$ and $\snonce{2}' = \snonce{2} + 1$. We now consider the execution of $a_1$ on $\rol{1}$, which is the last CRT action in $\batch{1}$.
    \noindent\textbf{Subcase 1(a):} The user action $a_1$ is \emph{proper} and calls $\textproc{startSession}$ to trigger $a_2$. Then $\enonce{1}' = \enonce{1} + 1$ and \emph{further} the session nonce now is $\snonce{1}' = \snonce{1} + 2$. This new session nonce is inserted in $\ttree_1$, which causes a mismatch with $\atree_2$, where $a_2$ was inserted along with $\snonce{2}' = \snonce{1} + 1$. Thus (I) is not preserved.
    \noindent\textbf{Subcase 1(b):} The user action $a_1$ is \emph{improper} and directly calls $\textproc{trigger}$ to trigger $a_2$. Then $\enonce{1}' = \enonce{1}$ is unchanged and still $\snonce{}' = \snonce{} + 1$. Thus we have $\enonce{1}' + \enonce{2}' \neq \snonce{1}' $ and (II) is not preserved.
    
    \noindent\textbf{Case 2:} $\sactive{1} = \true$. Then $\sactive{2} = \false$ and now $a_{2k}$ must use $\sid{2} = \snonce{2} + 1$. The argument from here on is similar to case 1.

    In either case, either property (I) or (II) is not preserved.
\end{proof}
