\section{Problem Definition}
In this section, we define the key components and objectives of our study, focusing on the structure of traffic networks and the goals of traffic forecasting.

\begin{Def}[Traffic Network]
    Given the real-world traffic scenarios, we define the traffic network as a directed graph $\mathcal{G}=\{\mathcal{V},\mathcal{E},\mathbf{A}\}$, where $\mathcal{V}$ denotes a set of $N$ nodes, each corresponding to a different sensor within the road network. $\mathcal{E}$ represents a set of edges that denote the connectivity among the nodes. $\mathbf{A}$$\in$ $\mathbb{R}^{N\times N}$ is the adjacency matrix that models the connectivity between nodes.
\end{Def}

\begin{Def}[Traffic Forecasting]
    Given historical traffic time series and the road topology, the objective of traffic flow forecasting is to predict future values of traffic time series. Specially, we represent the historical time series as a signal tensor $\bm{\mathcal{X}} =$ $\left[\mathbf{X}_1,\mathbf{X}_2,\dots,\mathbf{X}_{T}\right]$ $\in$ $\mathbb{R}^{T\times N\times C}$, where $T$ is the length of historical traffic time series and $C$ is the number of dimensions of node attributes. We aim to construct a function $f(\cdot)$ that maps the historical time series over $T$ time steps to predict the subsequent $T^\prime$ time steps:
   \begin{equation}
        \left[\mathbf{X}_{1},\mathbf{X}_{2},\dots,\mathbf{X}_{T};\mathcal{G}\right] \stackrel{f(\cdot)}{\longrightarrow} \left[\mathbf{\hat{X}}_{T+1},\mathbf{\hat{X}}_{T+2},\dots,\mathbf{\hat{X}}_{T+T^\prime}\right].
    \end{equation}
\end{Def}