% Checklist macros
\newcommand{\answerYes}[1]{\textcolor{blue}{#1}} 
\newcommand{\answerNo}[1]{\textcolor{yellow}{#1}} 
\newcommand{\answerNA}[1]{\textcolor{gray}{#1}} 
\newcommand{\answerPartial}[1]{\textcolor{green}{#1}}
\newcommand{\answerTODO}[1]{\textcolor{red}{#1}} 

\section{Reproducibility checklist}

Unless specified otherwise, please answer [Yes] to each question if the relevant information is described either in the paper itself or in a technical appendix with an explicit reference from the main paper. If you wish to explain an answer further, please do so in a section titled ``Reproducibility Checklist'' at the end of the technical appendix.

\noindent This paper:

\begin{itemize}
    \item Includes a conceptual outline and/or pseudocode description of AI methods introduced. \answerYes{Yes}
    \item Clearly delineates statements that are opinions, hypothesis, and speculation from objective facts and results. \answerYes{Yes}
    \item Provides well marked pedagogical references for less-familiare readers to gain background necessary to replicate the paper.  \answerYes{Yes}
\end{itemize}

\noindent Does this paper make theoretical contributions?  \answerYes{Yes}

\noindent If yes, please complete the list below.

\begin{itemize}
    \item All assumptions and restrictions are stated clearly and formally. (yes/partial/no) \answerYes{Yes}
    \item All novel claims are stated formally (e.g., in theorem statements). (yes/partial/no) \answerYes{Yes}
    \item Proofs of all novel claims are included. (yes/partial/no) \answerYes{Yes}
    \item Proof sketches or intuitions are given for complex and/or novel results.  (yes/partial/no) \answerYes{Yes}
    \item Appropriate citations to theoretical tools used are given. (yes/partial/no) \answerYes{Yes}
    \item All theoretical claims are demonstrated empirically to hold. (yes/partial/no/NA) \answerYes{Yes}
    \item All experimental code used to eliminate or disprove claims is included. (yes/no/NA) \answerYes{Yes}
\end{itemize}

\noindent Does this paper rely on one or more datasets? (yes/no) \answerYes{Yes}

\noindent If yes, please complete the list below.

\begin{itemize}
\item A motivation is given for why the experiments are conducted on the selected datasets. (yes/partial/no/NA) \answerYes{Yes}
\item All novel datasets introduced in this paper are included in a data appendix. (yes/partial/no/NA) \answerYes{Yes}
\item All novel datasets introduced in this paper will be made publicly available upon publication of the paper with a license that allows free usage for research purposes. (yes/partial/no/NA) \answerYes{Yes}
\item All datasets drawn from the existing literature (potentially including authors’ own previously published work) are accompanied by appropriate citations. (yes/no/NA) \answerYes{Yes}
\item All datasets drawn from the existing literature (potentially including authors’ own previously published work) are publicly available. (yes/partial/no/NA) \answerYes{Yes}
\item All datasets that are not publicly available are described in detail, with explanation why publicly available alternatives are not scientifically satisficing. (yes/partial/no/NA) \answerYes{Yes}
\end{itemize}


\noindent Does this paper include computational experiments? (yes/no)
\answerYes{Yes}

\noindent If yes, please complete the list below.
\begin{itemize}
\item Any code required for pre-processing data is included in the appendix. (yes/partial/no) \answerYes{Yes}
\item All source code required for conducting and analyzing the experiments is included in a code appendix. (yes/partial/no) \answerYes{Yes}
\item All source code required for conducting and analyzing the experiments will be made publicly available upon publication of the paper with a license that allows free usage for research purposes. (yes/partial/no) \answerYes{Yes}
\item All source code implementing new methods have comments detailing the implementation, with references to the paper where each step comes from (yes/partial/no) \answerYes{Yes}
\item If an algorithm depends on randomness, then the method used for setting seeds is described in a way sufficient to allow replication of results. (yes/partial/no/NA) \answerYes{Yes}
\item This paper specifies the computing infrastructure used for running experiments (hardware and software), including GPU/CPU models; amount of memory; operating system; names and versions of relevant software libraries and frameworks. (yes/partial/no) \answerYes{Yes}
\item This paper formally describes evaluation metrics used and explains the motivation for choosing these metrics. (yes/partial/no) \answerYes{Yes}
\item This paper states the number of algorithm runs used to compute each reported result. (yes/no) \answerYes{Yes}
\item Analysis of experiments goes beyond single-dimensional summaries of performance (e.g., average; median) to include measures of variation, confidence, or other distributional information. (yes/no) \answerYes{Yes}
\item The significance of any improvement or decrease in performance is judged using appropriate statistical tests (e.g., Wilcoxon signed-rank). (yes/partial/no) \answerYes{Yes}
\item This paper lists all final (hyper-)parameters used for each model/algorithm in the paper’s experiments. (yes/partial/no/NA) \answerYes{Yes}
\item This paper states the number and range of values tried per (hyper-) parameter during development of the paper, along with the criterion used for selecting the final parameter setting. (yes/partial/no/NA) \answerYes{Yes}
\end{itemize}