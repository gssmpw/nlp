\section{Introduction}
\label{sec.intro}

With technological advancements, a diverse array of sensors has been increasingly integrated into monitoring systems to bolster modern intelligent transportation systems (ITS) \cite{cirstea2021enhancenet,ji2022stden, dai2023dynamic}. Transportation authorities deploy a variety of sensors, such as electronic police cameras and bayonet detectors, across road networks to continuously collect essential traffic data, including flow and speed. Utilizing historical traffic flow data and road network topology, traffic forecasting aims to predict future flow variations, thereby improving daily travel and traffic management \cite{dai2021temporal,li2023dynamic}.

In the realm of traffic forecasting, considerable efforts are devoted to modeling traffic dynamics. Methods based on deep learning, especially the Spatio-Temporal Graph Neural Networks (STGNNs), have proven more effective than statistical time series analysis and shallow machine learning techniques in addressing traffic forecasting challenges. To address temporal dynamics, sequential models like RNN-based variants \cite{graves2012LSTM,deng2022multi,deng2024disentangling} and non-sequential Transformers \cite{vaswani2017Transformer} have been profoundly studied. For spatial dynamics, recent progress has been made with the Graph Neural Networks (GNNs) \cite{yin2021survey}, which represent sensors as nodes within a graph, leveraging graph structure to capture traffic patterns. Despite these substantial advancements, our study suggests that current methods still exhibit considerable potential for improvement in two critical aspects.

% In the realm of traffic forecasting, considerable efforts have been devoted to modeling traffic dynamics. Deep learning have proven more effective than statistical time series analysis and shallow machine learning techniques in addressing traffic forecasting challenges. To address temporal dynamics, sequential models like RNN-based variants \cite{graves2012LSTM} and non-sequential Transformers \cite{vaswani2017Transformer} have been profoundly studied. For spatial dynamics, recent progress has been made with the Spatial Temporal Graph Neural Networks (STGNNs) \cite{yin2021survey}, which represent sensors as nodes within a graph, leveraging graph structure to capture traffic patterns. Despite these substantial advancements, our study suggests that current methods still exhibit considerable potential for improvement in two critical aspects.

%Firstly, data utilization efficiency is not enough, especially regarding the efficiency of using the unique temporal and spatial information in traffic series data. For the temporal aspects, traffic information is related to the specific week and time it occurs, as previous studies have also demonstrated \cite{guo2019ASTGCN}. However, prior works have merely coarsely inputted these two types of information into the models without fully integrating the temporal periodicity with the traffic information. For the spatial aspects, different regions have distinct spatial characteristics based on their locations. However, prior methods based on the graph constructed by distance between two points neglect deeper correlations among nodes in 

Firstly, the prominent spatio-temporal characteristics in traffic flow can be more effectively modeled with the appropriate inductive bias. Although previous studies have demonstrated \cite{zhang2017deep,yu2019citywide,guo2019ASTGCN}, traffic patterns are strongly influenced by the specific temporal periodicities, most efforts have roughly incorporated these temporal features into the models without explicitly modeling the synergy between the long and short periodicities \cite{chen2018price,deng2024parsimony}. Regarding spatial aspects, while different locations exhibit unique spatial characteristics, most GNN methods construct static graphs based solely on the distances between two nodes, fail to consider the global interactions among all nodes in the graph \cite{yin2021survey}.

Secondly, effective trend-seasonality decomposition of traffic flow can greatly enhance the representation learning of traffic nodes \cite{wu2021Autoformer, fang2023STWave}. Utilizing this methodology improves the prediction of traffic flow by distinguishing systematic patterns and noise components. However, the application of trend-seasonality decomposition predominantly to individual nodes in a traffic network overlooks the interactions among global nodes, thereby diminishing the quality of node representations learned by GNNs.
%Traffic flow in real word is complex and variable, with each observation containing rich high-order information related to its temporal and spatial context. The failure to model high-order spatio-temporal correlations hinders the node's representation learning.

%To address these two limitations, we propose a novel method named \textbf{\underline{S}}patio-\textbf{\underline{T}}emporal \textbf{\underline{A}}uto-\textbf{\underline{C}}orrelation Network (STAC) for traffic flow prediction, 

To bridge these research gaps, we introduce a \textbf{\underline{S}}patiotemporal-aware \textbf{\underline{T}}rend-Seasonality \textbf{\underline{D}}ecomposition \textbf{\underline{N}}etwork (STDN), which enhances global node representations through a novel trend-seasonality decomposition incorporating spatio-temporal embeddings. It features three key modules: \textit{(1) Module of Spatio-Temporal Embedding Learning} models the temporal periodicity by learning the temporal embedding including specific weeks and minutes, and acquires an initial spatial location embedding from the eigenvalues and eigenvectors of the graph Laplacian matrix. \textit{(2) Module of Dynamic Relationship Graph Learning} explores the global dynamic interaction among traffic nodes, enhanced by the spatio-temporal embedding, thereby capturing the high-order relationships between each traffic node. \textit{(3) Module of Trend-Seasonality
Decomposition} aims to refine the node representations by disentangling the traffic flow into the trend-cyclical and seasonal components, which are further processed through an encoder-decoder network. The contributions of our study are summarised as follows: %Extensive experiments on three traffic datasets demonstrate that our proposed STAC is capable of achieving superior performance compared with state-of-the-art methods in different settings.
%explores the higher-order spatio-temporal relationships across regions and times, we introduce a dynamic graph convolutional network, which incorporates graph structure learning to model neighborhood interactions. Then, a trend-seasonality decomposition module is proposed to 
% combines the vectors of neighborhood interaction with the spatio-temporal embedding

\begin{itemize}

    \item We propose the \model model, a novel dynamic GCN-based framework for traffic flow prediction. To our knowledge, this is the first approach to learn disentangled representations of traffic flow in view of the spatio-temporal embeddings.
    % \item We propose the \model model, a novel dynamic GCN-based framework for traffic flow prediction. To our knowledge, this is the first time of constructing dynamic graph that incorporates both spatio-temporal embeddings learning and trend-seasonality decomposition. 
    % \item We develop an innovative spatio-temporal embedding learning approach and a novel trend-seasonality decomposition mechanism. Each component of decomposition is designed to be aware of the spatio-temporal embeddings, enriching the model's ability to capture high-order interactions across time and space.
    \item We develop a novel trend-seasonality decomposition mechanism. Each component of decomposition is designed to be aware of the spatio-temporal embeddings, enriching the model's capability to capture high-order node interactions across time and space.
    \item We conduct comprehensive multi-step traffic flow prediction experiments on three real-world datasets. The experimental results demonstrate that our method consistently surpasses various competing baselines. Additionally, the effectiveness of each module is verified through the ablation study.
    \item We release a new urban dataset named JiNan, which, unlike popular highway traffic datasets (e.g., PeMS) focuses more on the spatio-temporal dynamics of inner-city traffic. We believe this dataset will further enrich the scenario comprehensiveness in traffic flow prediction evaluations.
    %We also publish a new urban dataset called JiNan, which includes real-time traffic flow information for various vehicles at different intersections. To advance developments in the transportation field.
    
\end{itemize}