\documentclass[sigconf,balance=false,prologue,table,xcdraw]{acmart}
\usepackage{popets}
\usepackage{hyperref}
\usepackage{todonotes}
\usepackage[utf8]{inputenc}
\usepackage{xspace}
\usepackage{spverbatim}
\usepackage{booktabs}
\usepackage{rotating}
\usepackage{multirow}
\usepackage{tablefootnote}
% \usepackage[table,xcdraw]{xcolor}
\usepackage{threeparttable}

% Copyright
% \setcopyright{popets}
% \copyrightyear{YYYY}

% Issue info
\acmYear{}
\acmVolume{}
\acmNumber{}
\acmDOI{}
\acmISBN{}
\acmConference{Working draft}
\settopmatter{printacmref=false,printccs=false,printfolios=true}
\setlength {\marginparwidth }{2cm}

\newcommand{\thought}[1]{{\color[rgb]{0.2,0.39,0.66}(#1)}}
\newcommand{\todo}[1]{{\color[rgb]{1.0,0.0,0.0}(#1)}}
\newcommand{\hsh}[1]{{\color{green!50!black} Henrik: #1}}
\newcommand{\st}[1]{{\color{red!50!black} Sebastian: #1}}

\newcommand{\ulm}[1]{_{\scaleto{\mathrm{#1}}{3pt}}}
\newcommand\at[2]{\left.#1\right|_{#2}}











\newtheorem{assumption}{Assumption}

\DeclareMathOperator*{\argmax}{arg\,max}
\DeclareMathOperator*{\argmin}{arg\,min}

\newcommand{\swname}[1]{\texttt{#1}}
\newcommand{\ie}{i\/.\/e\/.,\/~}
\newcommand{\eg}{e\/.\/g\/.,\/~}
\newcommand{\cf}{cf\/.\/~}

\newcommand{\fig}{Fig\/.\/~}
\newcommand{\defn}{Def\/.\/~}
\newcommand{\sect}{Sec\/.\/~}
\newcommand{\tabl}{Tab\/.\/~}
\newcommand{\algo}{Algorithm~}
\newcommand{\theo}{Theorem~}

\newcommand{\bnnl}{3 hidden layers}
\newcommand{\bnnn}{50 neurons}
\newcommand{\bnna}{tanh activations}

\newcommand{\capt}[1]{\mdseries{\emph{#1}}}

\newcommand{\videolink}{at \url{https://youtu.be/_d7AqTRjz6g}}
\newcommand{\codelink}{\url{https://github.com/wheelbot/mini-wheelbot}}

\newcommand{\fakepar}[1]{\vspace{0mm}\noindent\textbf{#1.}}

\newcommand{\needref}{\textcolor{red}{[REF]}}

\newcommand{\plotfontsize}{9pt}


\begin{document}

%%
%% The "title" command has an optional parameter,
%% allowing the author to define a "short title" to be used in page headers.
\title[SoK: A Classification for \PPAs]{SoK: A Classification for AI-driven Personalized Privacy Assistants}

%%%%%%%%%%%%%%%% Authors' Info %%%%%%%%%%%%%%%%%
%%
%% The "author" command and its associated commands are used to define
%% the authors and their affiliations.

\author{Victor Morel}
\orcid{0000-0001-9482-8906}
\affiliation{%
  \institution{Chalmers University of Technology and University of Gothenburg}
  \city{Gothenburg}
  \country{Sweden}}
\email{morelv@chalmers.se}

\author{Leonardo Iwaya}
\orcid{0000-0001-9005-0543}
\affiliation{%
  \institution{Karlstad University}
  \city{Karlstad}
  \country{Sweden}}
\email{Leonardo.Iwaya@kau.se}

\author{Simone Fischer-Hübner}
\orcid{0000-0002-6938-4466}
\affiliation{%
  \institution{Karlstad University, Chalmers University of Technology, and University of Gothenburg}
  \city{Karlstad}
  \country{Sweden}}
\email{simone.fischer-huebner@kau.se}
%%
%% By default, the full list of authors will be used in the page
%% headers. Often, this list is too long, and will overlap
%% other information printed in the page headers. This command allows
%% the author to define a more concise list
%% of authors' names for this purpose.

\renewcommand{\shortauthors}{Morel et al.}

%%
%% The abstract is a short summary of the work to be presented in the
%% article.
\begin{abstract}
To help users make privacy-related decisions, personalized privacy assistants based on AI technology have been developed in recent years.
These AI-driven Personalized Privacy Assistants (\PPAs) can reap significant benefits for users, who may otherwise struggle to make decisions regarding their personal data in environments saturated with privacy-related decision requests. 
However, no study systematically inquired about the features of these \PPAs, their underlying technologies, or the accuracy of their decisions.
To fill this gap, we present a Systematization of Knowledge (SoK) to map the existing solutions found in the scientific literature.
We screened 1697 unique research papers over the last decade (2013-2023), constructing a classification from 39 included papers.
As a result, this SoK reviews several aspects of existing research on \PPAs in terms of types of publications, contributions, methodological quality, and other quantitative insights.
Furthermore, we provide a comprehensive classification for \PPAs, delving into their architectural choices, system contexts, types of AI used, data sources, types of decisions, and control over decisions, among other facets.
Based on our SoK, we further underline the research gaps and challenges and formulate recommendations for the design and development of \PPAs as well as avenues for future research.
\end{abstract}

%%
%% Keywords. The author(s) should pick words that accurately describe
%% the work being presented. Separate the keywords with commas.
\keywords{privacy assistant, privacy, data protection, artificial intelligence, machine learning, systematic review}


\maketitle

\section{Introduction}
\label{sec:introduction}
% \todo[inline]{First pass done (still needs references)}

% The introduction should encompass:
% \begin{itemize}
%     \item an anchor
%     \item motivation
%     \item objective
%     \item scope
%     \item contributions
%     \item outline
% \end{itemize}

% \todo[inline]{Victor: write a first version in August. @anyone feel free to drop bits of text}

%anchor
As the world becomes increasingly digitalized, people are faced with a higher number of decisions related to their privacy.
We surround ourselves daily with several apps and websites, and the number of smart gadgets and Internet of Things (IoT) devices continues to grow~\cite{noauthor_state_2024}.
%Whether by following recent privacy laws, such as the General Data Protection Regulation (GDPR), or to answer public outcry due to privacy scandals, 
Furthermore, to enforce the individuals' rights to informational self-determination and comply with privacy laws such as the General Data Protection Regulation (GDPR)~\cite{european_parliament_general_2016}, software systems regularly require us to make privacy-related decisions regarding our personal data: \textit{Do you grant this permission? Do you want to accept the cookies? Should this sensor be left on when you host friends?}
Consequently, the cognitive burden increases, leaving users in disarray, tired, and unable to decide in their best interests~\cite{choi_role_2018}.

%motivation
During the last decade, researchers have been building privacy assistants to alleviate this burden and support users in their decisions (see the patent on Personalized Privacy Assistant registered in 2023 in the US by \citet{sadeh2021personalized}).
% ~\footnote{See the patent on PPAs registered in 2023 in the US by \citet{sadeh2021personalized}}
With the progress made in Artificial Intelligence (AI), it is no surprise that some of these assistants leverage this technology, notably enabling more personalized support.
However, the extent to which AI drives these Personalized Privacy Assistants (\PPAs), their efficiency, privacy-friendliness, functioning, and eventual addressing of legal requirements remains unclear.
In fact, to the best of our knowledge, there have been no surveys or systematic reviews on the topic of \PPAs.
This lack of systematization of knowledge prevents other researchers from identifying existing gaps in the field and efficiently addressing the challenges.

%objective
To address this lack of coherence, provide a common vocabulary, and better compare and categorize the different \PPA solutions, we propose a Systematization of Knowledge (SoK) of the last decade of research.
In doing so, we aim to draw insights and lessons for future assistants and to formulate better 
%better design recommendations
recommendations for research, design, and development of \PPAs.
Formulated otherwise, we tackle the following Research Questions (RQs):
\begin{itemize}
    \item \textbf{RQ1:} \textit{What is the current state of the literature on \PPAs for automated support of end-users privacy decisions in IT systems?}
    \item \textbf{RQ2:} \textit{What are the key attributes and properties of the proposed \PPAs in the literature?}
\end{itemize}
Here, we understand agents and assistants in a broad sense (any logical entity able to support users, including unimplemented theoretical models, see our selection criteria in Table~\ref{tab:selection-criteria}); AI in a generic sense as well (see Section~\ref{subsec:AI}); and privacy decisions as individual decisions regarding one's personal information management (see Section~\ref{subsec:privacy_decisions}).

%scope
To address our RQs, we performed a Systematic Literature Review (SLR) on research papers that provided technical solutions, published between 2013 and 2023 in peer-reviewed venues, and a further snowballing process until early May 2024.
We screened 1697 unique papers from IEEE, ACM, Scopus, and Web of Science, resulting in 39 selected papers after several rounds of snowballing.
We extensively read and analyzed all the included papers, and the information extracted forms the basis of our work.

%contributions
As a result of our SLR, we make the following contributions:
\begin{description}
    \item[A Classification for \PPAs] - We propose the first classification for \PPAs, providing a common vocabulary for designers of such systems.
    \item[Data Charting \& Quantification] - We charted and quantified several aspects of \PPAs based on the aforementioned classification. 
    \item[Research Gaps \& Challenges] - We underline the current gaps in the state of the art and highlight challenges for designing \PPAs based on our data.
    \item[Recommendations \& Research Avenues] - We formulate recommendations for improving \PPAs, and propose several avenues for future research.
\end{description}


%outline
%The rest of this article is organized as follows. 
%We first provide the necessary background regarding privacy decisions, regulations and AI technologies in Section~\ref{sec:background}. We then present our methodology in Section~\ref{sec:methodology}, followed by the factual results in Section~\ref{sec:results}.
%The main contribution of our work, the classification for \PPAs, is introduced in Section~\ref{sec:classification}.
%We discuss the implications of our work and observed gaps in Section~\ref{sec:discussion},  limitations in Section~\ref{sec:validity-threats}, and finally conclude in Section~\ref{sec:conclusion}.

% The rest of this article presents background, methodology, the main contributions including a classification of \PPAs, a discussion of gaps, recommendations and research directions as well as overall conclusions.

\section{Background and Related Work}
\label{sec:background}
As a background, this section first introduces legal concepts around privacy, then provides an overview of different types of privacy decisions for which individuals could receive support from \PPAs. 
Lastly, it presents AI and Machine Learning technologies that can provide the technical foundations for \PPAs.

\subsection{Legal Background} 
\label{subsec:legal}
% \todo[inline]{Subsection may need to be moved upward}
% \todo[inline]{Simone for a first pass}
%GDPR, consent requirements, whether a decision should fall under a certain category or another.\\
%How to exercise your rights, explainability, control but not just on pure transparency.\\
%AI act on explainability (important to note the transparency requirements and the exceptions for research, for a short account https://arxiv.org/pdf/2408.01449).

\subsubsection{Roles and Obligations According to the GDPR and AI Act}
\label{legal_background}
\PPAs are using AI techniques for processing data, including personal data, to assist users with making personalized privacy-related decisions. 
For the discussion of any legal requirements regarding the use of personalized privacy assistants, the question of who the data controller is for any personal data processing by the assistants will be of relevance.

If \PPAs are installed and run by users on their own devices or other servers under their control, the users will likely act as data controllers or joint controllers with other service providers. The so-called household exemption (Art. 2(2)(c) GDPR) can take effect, meaning that the GDPR~\cite{european_parliament_general_2016} will not apply if the user is using the assistant for private purposes on their private devices or servers under their control.
If the \PPA is run not only for purely private purposes on the user's devices or controlled servers, the data controller may be another entity different from the user (e.g., the user's employer). Legal obligations need to be fulfilled by the controllers regarding data protection by design and default (Art. 25) of the assistants, security of data processed (Art. 32), implementing data subject rights, including the data subject's rights to transparency (Art. 13--15), their rights to object to profiling (Art. 21), and the right not to be subject to a decision based solely on automated processing, including profiling (Art. 22).

In addition, legal obligations according to the EU AI Act~\cite{eu_ai_act_2024} may also have to be considered for the producer and also by the deployers of AI-driven privacy assistants, including requirements for risk management (Art. 9), transparency (Art. 13, 50), robustness, security, accuracy (Art. 15). These obligations however mostly apply if \PPAs could be classified as ``high-risk'' AI systems. This should, however, seldom be the case, especially as \PPAs are typically used for users' own privacy management, which should typically not interfere with the fundamental rights of others.
Exceptions could, however, be \PPAs that are, for example, used for setting permissions for safety-critical applications impacting the safety of the users or others.
%or if they are used to manage privacy decisions related to relational personal data that do not only related to the users but may also include sensitive information about other individuals.

%GDPR: Who is controller / processor? 
%AI act: obligations of producers, "users" (deployers)

%ex ante and ex post transparency, explainability (for automated decision making), and human oversight

\subsubsection{Legal Requirements for Transparency.}
%according to GDPR and the AI act (depending on the risk level).
In cases where the data controllers of the \PPAs are not the data subjects themselves, the controllers should provide the data subjects with privacy policy information \textit{ex-ante} at the time when data are obtained from them according to Art. 13 GDPR, and \textit{ex-post} through the right to access granted in Art. 15 GDPR.
This also should include information about purposes of processing, data categories concerned, but also information about the logic involved and significance, and envisioned consequences of automated decision-making and profiling performed by the \PPAs.

The AI Act also includes obligations for transparency for the producers and deployers of limited-risk and high-risk AI systems (Art. 50). While the providers of limited-risk AI systems have to mainly ensure that humans are informed that AI systems are used, high-risk AI systems require that further clear, comprehensible and adequate information is given to the deployer (Art. 13), traceability of results via logging (Art. 12) and appropriate human oversight (Art. 14).

\subsubsection{Legal Requirements for Consent.} 
Art. 4 (11) of the GDPR defines ‘Consent’ of data subjects as any freely given, specific, informed, and unambiguous indication of the data subject's wishes by which they, by a statement or by a clear affirmative action, signifies agreement to the processing of personal data relating to them.
A valid consent has thus to fulfill several conditions. Namely, it needs to be:
\begin{itemize}
    \item \textit{Freely given}, i.e., the data subject needs to have free choices -- this is usually not the case if there is an imbalance of power in the relation between the data subject and the data controller. Furthermore, there should be no negative consequences if consent is not given. Moreover, consent may not be bundled as a non-negotiable part of terms and conditions.
    \item \textit{Specific}, which means that consent must be given for one or more specific purposes and that the data subject must have a choice in relation to them -- i.e., separate opt-in is needed for each purpose.
    \item \textit{Informed}, which means that the data subject has to be informed about certain elements that are crucial to making a choice.  This includes information about the controller’s identity, the data processing purposes, the type of data, the right to withdraw consent, any use for decisions based solely on automated processing, and risks of data transfers to third countries.
    \item Moreover, a confirming statement or \textit{affirmative action} is needed for a valid consent and requires that the data subject has taken a deliberate action to consent. Therefore, silence, pre-ticked boxes, or inactivity should not constitute consent (Recital 32 GDPR). It also means that consent cannot be fully automated.
\end{itemize}


% Maybe the following shorter text is sufficent (to be decided later):

%According to its definition in Art. 4 (11) GDPR, a valid
%Consent needs to be \textit{informed}, \textit{specific}, \textit{freely given}, and \textit{unambiguous}, which entails a \textit{clear statement} or an \textit{affirmative action} (Article 4 (11) GDPR). 

%Not sure yet, if this should be mentioned as well:
%Moreover, Art. 7(3) of the GDPR explicitly states that users must be able to withdraw consent at any time, and that it shall be as easy to withdraw as to give consent.

%\subsubsection{Automated decision making and explainability}


\subsection{Privacy Decisions}
\label{subsec:privacy_decisions}
Among the most notable definitions, Westin~\cite{westin_privacy_1968} has defined privacy as the right to informational self-determination, meaning that individuals should have the \textit{right to decide} for themselves when, how, and what information about them is communicated to others.
As mentioned, in the EU, the GDPR emphasizes that individuals should have control of their personal data (Recital 7), and thus should be empowered to make decisions about their data as one prerequisite for exercising such control.
Delving deeper into this notion of \textit{privacy decisions}, we further elaborate on this concept in the following subsections.

\subsubsection{Individual Privacy Decisions Regulated by Privacy Laws}
Some privacy decisions individuals can make to exercise control over their data are regulated under the GDPR and other privacy laws. 
These decisions notably include, but are not limited to, the \textit{decisions to grant or to withdraw consent} to data collection and processing.

Indeed, the GDPR and most other privacy laws regulate \textit{decisions to exercise data subject rights} granted by the respective laws. 
For instance, according to Art. 15-22 GDPR, data subjects have the rights to access data, request rectification or deletion of data, export data, and object to direct marketing and profiling.
Data subjects can also object in cases where the legal ground for the processing is public interest or legitimate interest, or exercise their right not to be subject to automated decision-making.

%privacy decisions of users for controlling the disclosure and conditions for the processing of their personal data

\subsubsection{Further Types of Privacy Decisions}
\label{subsec:further-decisions}
Further types of privacy decisions concerning users' choices regarding the use of their data by others, which are not directly mentioned or regulated by the GDPR, include \textit{decisions of individuals to publish or share data on their own initiative}, e.g., in social networks. In these cases, data sharing has typically not been formally triggered by a consent request to allow data sharing with another party.
%.Moreover, privacy decisions for controlling the disclosure or conditions for the processing one's personal data include 

Moreover, privacy decisions encompass \textit{privacy permission} (or access control rights) settings, which grant others certain rights for using their data and are, for instance, typically used for permission systems of mobile phone operating systems, such as Android or iOS.
Setting privacy permissions on mobile operating systems often requires consent at installation or during runtime. 
However, instead of consent, other legal grounds -- such as a contract (Art. 6 (1)(b) GDPR) --, can be used, e.g., for a banking app to forward account information when transferring money~\cite{Art29WP13}.
Let us also note the peculiar case of Global Privacy Control (GPC), a unary signal that permits or prohibits third-party tracking on the browser~\cite{human_data_2022}.
Due to its enforceability under the California Consumer Privacy Act (CCPA), it is regulated by a privacy law but is technically more akin to a privacy permission.

Additionally, some privacy-enhancing technologies and protocols allow users to decide and set \textit{privacy preferences}, which are simply indications of the users' privacy wishes of how their data should be used without actually granting any rights to others, and thus without legal mandate. 
Privacy preferences have, for instance, been used earlier by the Platform for Privacy Preferences (P3P) \cite{cranor_platform_2002} or Do Not Track (DNT) as an example for signals that can be set manually in browser settings for allowing users to specify their privacy choices.




%preferences, permissions, data sharing, reject, and what could address consent based on requirements described above (overlap possible, can be hard to distinguish sometimes).





\subsection{AI for Decision-making}
\label{subsec:AI}
%AI used more and more for decision-making, different types of AI tools, and different types of AI techniques
AI is a generic term for various strategies and techniques enabling computers and machines to simulate human intelligence and problem-solving capabilities~\cite{russell_artificial_2016}.
Machine learning (ML) is a field of AI (we subsume the former under the latter in the rest of the document) that develops and studies statistical algorithms and models, draws inferences from patterns in data, and learns and adapts without following explicit instructions.
% We subsume and include ML in our definition of AI in the rest of the document.
AI-powered tools can particularly lighten the user's cognitive load and thereby improve their decision-making, e.g., by decision support, augmentation, or automation. 

While there are different ways to categorize AI systems, we refer in the present work to the survey paper on eXplainable AI (XAI) by \citet{arrieta_explainable_2019}.
They distinguish between transparent models and those requiring post-hoc explainability (non-inherently transparent).
We leverage this reference because AI-supported decision-making must be explained under specific circumstances according to the GDPR and the AI Act~\cite{panigutti_role_2023}.

In their words: ``A model is considered to be transparent if by itself it is understandable.''~\cite{arrieta_explainable_2019}
Such models include linear regression, decision trees, k-nearest neighbors, rule-based learning, general additive models, and Bayesian models.
Nonetheless, models that are not deemed intrinsically transparent can be made explainable through the use of \textit{post-hoc} techniques.
Neural networks (especially deep and convoluted) and Support Vector Machines (SVM) typically fall under this category, as well as reinforcement learning~\cite{puiutta_explainable_2020}.



% \newpage

\section{Methodology}
\label{sec:methodology}
% \todo[inline]{V: read again, improve if needed}
% \todo[inline]{Not necessarily very detailed, we can relegate some parts to the appendix for space.}

% \begin{itemize}
%     \item Mix of Systematic Literature Review (SLR) and scoping review, following Kitchenham's guidelines to the extent possible.
%     \item Inclusion/exclusion criteria, logical queries.
%     \item Description of the different phases: databases querying, calibration, screening, several phases of snowballing, charting
%     \item Big funnel schema for the number of papers.
% \end{itemize}

% You should remind the readers of your RQ here...
% \todo[inline]{SLR protocol in the appendix, the rest will follow on GitHub}
This SoK study adopts the widely known methodology for systematic literature reviews (SLRs) proposed by \citet{kitchenham2004procedures}. The SLR methodology offers us a well-defined and rigorous sequence of methodological steps consisting of three main phases: (1) planning, (2) conducting, and (3) reporting the review. A SLR Protocol that describes the entire research process has been written for this study (a summary version of which can be found in Appendix~\ref{app:protocol}). 
Furthermore, we make our research data openly available in an anonymised GitHub repository~\footnote{\url{https://anonymous.4open.science/r/SoK_AI_PPA-E29F}} for reproducibility. 
Our material comprises the citation files of each query, the Data Extraction Forms (DEFs) of the selected papers, and the charting spreadsheet used to compile all our data. 
Thus, due to page constraints, we refer readers to these documents for methodological details.

% , integrating some components of a scoping review because of the scope (slightly larger than most SLRs) and outcomes of our work (we clarify some key concepts, examine how research is done on the topic, and analyze gaps in the area~\cite{munn_systematic_2018}).

% Motivate the review: explain that we first searched for existing reviews on the topic but did not find anything

% You can mention that we first developed the SLR Protocol, even if it was a draft document, before running the SLR. The protocol was piloted to see if the search process was adequate.

\subsection{Planning the Review}\label{sec:slr-planning}
During the planning phase, our first activity was determining the need for this SLR. Several databases were searched 
% (i.e., Scopus, IEEE Xplore, ACM Digital Library, Web of Science) 
to verify if any surveys or reviews had been conducted on \PPAs. Search terms such as privacy, data protection, assistant, agent, artificial intelligence, and machine learning were used. However, we could not identify any survey or systematic reviews on the topic, reassuring the need for an SLR.

The research questions, presented in Section~\ref{sec:introduction}, guided the remaining phases of this SLR with respect to the search process, selection criteria, and data synthesis.
% The main Research Question (RQ) that guided this SLR is the following: \textbf{What are the existing AI-driven agents and assistants for automating and supporting the privacy decisions of end-users for IT systems?} Therefore, the remaining phases of this SLR reflect the RQ in the search process, selection criteria, and data synthesis, presented in the following subsections.

\begin{figure*}
    \centering
    % \includegraphics[scale=.2]{Figures/sankeymatic_20240626_180606_2400x1200.png}
    \includegraphics[scale=.2]{sankeymatic_20241119_164508_2200x1200.png}
    \caption{Sankey chart of the selection process.}
    \label{fig:sankey}
\end{figure*}

\subsection{Conducting the Review}\label{sec:slr-conducting}
\subsubsection{Search Strategy}
Based on our RQs and previous preliminary searches when designing the SLR Protocol, we identified a list of nine relevant keywords, i.e., \textit{privacy, data protection, assistant, agent, artificial intelligence, machine learning, intelligent, automatic}, and \textit{personalized}. These keywords were used to construct the following search query:
\begin{spverbatim}
(privacy OR "data protection") AND (assistant* OR agent*) AND ("artificial intelligence" OR "machine*learning" OR intelligent OR automat* OR personali*ed)
\end{spverbatim}
As such, the search query targets papers working on three joint topics: 1) privacy (or data protection), using either 2) an assistant or an agent, and leveraging 3) artificial intelligence or personalization.  

Four scientific databases were selected, i.e., Scopus, Web of Science, IEEE Xplore, and ACM Digital Library, due to their high relevance to the areas of computer science and engineering, comprising the vast majority of published research in the field.
We also specified inclusion and exclusion criteria (see Table \ref{tab:selection-criteria}) used during the screening of publications retrieved from the databases.
Before starting the search process, two authors piloted the searches on all databases and ran a \textit{calibration exercise} to verify the consistency of the inclusion criteria. For that, the authors independently screened 10\% of the results and discussed their decisions. The conflicts were all discussed and solved, sometimes with the help of a third author. This process was repeated a second time, screening another 10\% of the papers at a point that the authors fully agreed with the consistency of the selection process.

\begin{table}[!h]
    \centering
    \small
    \begin{tabular}{|p{0.95\linewidth}|}
    \hline
       \textbf{Inclusion Criteria} \\ \hline
      % - Papers with a technological focus (computer science and engineering, information system/usability, interdisciplinary techno-legal). \\
       - Provides a technical solution (implemented or theoretical) to help end-users automate personal (and personalized) privacy decisions with an assistant (or artificial agent) in IT systems. \\
       - Papers from 2013 onward to concentrate on the state-of-the-art. \\
       - The concept of AI needs to be explicitly stated in the papers. \\ \hline \hline
       \textbf{Exclusion Criteria} \\ \hline
       - Papers with solutions that are purely theoretical without substantial explanations on how they could be implemented in practice. \\
       - Papers with solutions that solely automate the analysis of privacy policies but without any type of personalization. \\
       - Papers with poor scientific quality (e.g., lack objectives or research questions, the methodology is not described, the solution is insufficiently/vaguely described, etc.). \\
       \hline
    \end{tabular}
    \caption{Criteria for the inclusion and exclusion of studies.}
    \label{tab:selection-criteria}
\end{table}

\subsubsection{Selection Process}
Figure~\ref{fig:sankey} presents an overview of the selection process.
The querying of the databases mentioned above on October 19, 2023, yielded 2386 papers and 1697 unique entries after removing duplicates.
The screening phase lasted until November 23, 2023, and resulted in the selection of 33 papers.
Two authors then read 10\% of these 33 papers (3) and adjusted the DEF based on mutual feedback. This step helped us add new important fields to the DEF and consistently extract data from the papers.

The first data extraction phase, consisting of a full reading of each of the 33 papers, was performed over weeks 4 to 7 (included) in 2024.
Fifteen papers were excluded after full reading for different reasons: they were duplicates (i.e., same work published in different venues); they did not provide any technical solution; the automated decisions were not personalized to an end user; AI was not used for automating decisions; or they are of poor scientific quality (see our criteria in Table~\ref{tab:selection-criteria}); one paper was not available for download, we could not access it even after reaching out the authors.

We then proceeded to several snowballing phases~\cite{wohlin_guidelines_2014}, during which we checked the abstracts of all seemingly relevant\footnote{We only assessed papers cited in relevant sections such as the related work.} papers cited (backward snowballing), and screened citing papers (forward snowballing).
The snowballing process lasted from week 8 to week 19 of 2024 and resulted in 21 additional papers after exclusion, for a total of 39 papers (33-15+21).

\subsubsection{Data Extraction and Analysis}
\label{subsec:analysis}
The data extracted in the DEFs was compiled and further organized in spreadsheets during weeks 20-21.
This process also included the initial aggregation of data and the creation of frequency charts across several data categories (e.g., studies per year, types of publications, authors and affiliations, etc.)

Although we attempted to extract as much data from the studies as possible using a DEF, we found during the data analysis process the need to further categorize studies across other \textit{facets}.
For instance, additional information was compiled in the spreadsheets, such as a high-level categorization of certain fields (i.e., the type of AI used) or a critical appraisal of the user studies presented in the selected papers.
This collection of facets created during the study design and data analysis processes forms the basis of the work's final classification scheme, which is presented as part of the main results.
All authors were involved in the data analysis process and the definition of facets that further classify studies on the topic.

% \todo[inline]{Come back to this part later, to strengthen it}
By definition, an \PPA leverages AI techniques. We therefore collected information about the \textit{Type of AI used}.
AI models rely on data for training and decision-making. As such, we extracted the \textit{source of data}.
During the adjustment of the DEF, we observed that \PPAs are usually designed for a specific \textit{system context}, and for one or several \textit{types of decision}.
Connected to the system context, we extracted the \textit{choice of architecture} of the implementation (if any) to analyze the trust implications.
We also collected the methods for an \textit{empirical assessment}, presence and quality of user studies, or the means used to measure the accuracy, to gain insights on eventual benchmarks of \PPAs.
Studies can be classified as evaluation or validation research, as proposed by \citet{wieringa_requirements_2006}. 
An evaluation works in real-world practice and is implementing/deploying the solution or testing in an actual project with real test users, such as real case studies and realistic user testing of prototypes/systems.
A validation is a limited illustrative or hypothetical ``case study'' or ``use case'' performed as a lab experiment.
In general practice, prototypes are often validated by cross-sectional studies.
Finally, initially guided by legal requirements for consent and the exercise of data subject rights under the GDPR (eventually, no paper considers consent), we extracted what became \textit{user control over decisions}.

% Note that we also collected data that did not end up in the classification, such as the type of contribution or the type of paper.
% The former is nonetheless part of our results.


\subsection{Reporting the Review}
Based on the data analysis, a whole coherent narrative was written by the research team, i.e., the SoK conveying all the results, our interpretation of the main findings, and identifying research gaps.
% Threats to validity are also reviewed in \ref{sec:validity-threats}, describing the main limitations of this research.
This synthesis of this SLR on \PPAs is thus reported in the following sections of this paper.



\section{Summary of Data Charting Results}
\label{sec:results}
% \todo[inline]{Better section header, introduce Table 1.}
This section provides a brief overview of the quantitative insights generated through the data charting process (e.g., publications per year, citations, types of decisions).
Due to page constraints, further descriptions of the data charting results are provided in the appendices.
% (presented in the next section).
Most classification features are reported in Table~\ref{tab:classification}, except for the \textit{Empirical assessment} which, along with the results of our critical appraisal and the type of contribution, are reported in Table~\ref{tab:appendix_table} due to space constraints.



% \begin{table*}[]
% \tiny
% \begin{threeparttable}[b]
% \begin{tabular}{@{}llllllllllllllllllllllllllll@{}}
% \toprule
%  &  & \multicolumn{3}{l}{Type of decision} & \multicolumn{5}{l}{Type of AI used} & \multicolumn{6}{l}{Type of source of data} & \multicolumn{4}{l}{System context} & \multicolumn{3}{l}{Architecture} & \multicolumn{5}{l}{User control over decision} \\ \cmidrule(l){3-28} 
%  &  &  &  &  &  &  &  &  &  &  &  &  &  &  &  &  &  &  &  &  &  &  &  &  &  &  &  \\
%  &  &  &  &  &  &  &  &  &  &  &  &  &  &  &  &  &  &  &  &  &  &  &  &  &  &  &  \\
%  &  &  &  &  &  &  &  &  &  &  &  &  &  &  &  &  &  &  &  &  &  &  &  &  &  &  &  \\
%  &  &  &  &  &  &  &  &  &  &  &  &  &  &  &  &  &  &  &  &  &  &  &  &  &  &  &  \\
%  &  &  &  &  &  &  &  &  &  &  &  &  &  &  &  &  &  &  &  &  &  &  &  &  &  &  &  \\
% \multirow{-7}{*}{Year} & \multirow{-7}{*}{Publication} & \begin{rotate}{60} Permissions \end{rotate} & \begin{rotate}{60} Preferences \end{rotate} & \begin{rotate}{60} Data sharing \end{rotate} & \begin{rotate}{60} Classification \end{rotate} & \begin{rotate}{60} Clustering \end{rotate} & \begin{rotate}{60} Rule-based \end{rotate} & \begin{rotate}{60} Logic-based \end{rotate} & \begin{rotate}{60} Reinforcement \end{rotate} & \begin{rotate}{60} Context \end{rotate} & \begin{rotate}{60} Attitudinal data \end{rotate} & \begin{rotate}{60} Metadata \end{rotate} & \begin{rotate}{60} Type of data \end{rotate} & \begin{rotate}{60} Content of data \end{rotate} & \begin{rotate}{60} Behavioral data \end{rotate} & \begin{rotate}{60} Mobile apps \end{rotate} & \begin{rotate}{60} Social media \end{rotate} & \begin{rotate}{60} IoT \end{rotate} & \begin{rotate}{60} Cloud \end{rotate} & \begin{rotate}{60} Local \end{rotate} & \begin{rotate}{60} Remote \end{rotate} & \begin{rotate}{60} Federated \end{rotate} & \begin{rotate}{60} Informed \end{rotate} & \begin{rotate}{60} Semi-automated \end{rotate} & \begin{rotate}{60} Specific \end{rotate} & \begin{rotate}{60} Refuse \end{rotate} & \begin{rotate}{60} Revoke \end{rotate} \\ \cmidrule(r){1-2}
% \rowcolor[HTML]{C0C0C0} 
% 2014 & \citet{xie_location_2014} &  & $\bullet$ &  & $\bullet$ (NIT) &  &  &  &  & $\bullet$ &  &  &  &  &  &  &  &  &  & -- & -- & -- & No & Yes & Yes~\tablefootnote{Only location} & Yes & No \\
% 2015 & \citet{apolinarski_automating_2015} & $\bullet$ &  &  & $\bullet$ (NIT) &  &  &  &  & $\bullet$ &  &  &  &  &  & $\bullet$ &  &  &  & $\bullet$ &  &  & D & Yes & Yes & Yes & No \\
% \rowcolor[HTML]{C0C0C0} 
% 2015 & \citet{hirschprung_simplifying_2015} & $\bullet$ &  &  &  & $\bullet$ &  &  &  &  & $\bullet$ &  &  &  &  &  &  &  & $\bullet$ & -- & -- & -- & D & No~\tablefootnote{Not necessarily, depends on what they call the Configuration Options} & Yes & No~\tablefootnote{Not necessarily, depends on what they call the Configuration Options} & No \\
% 2015 & \citet{squicciarini_privacy_2015} &  &  & $\bullet$ &  &  & $\bullet$ &  &  & $\bullet$ &  & $\bullet$ &  & $\bullet$ &  &  &  &  &  & -- & -- & -- & D & Yes & Yes & Yes & No \\
% \rowcolor[HTML]{C0C0C0} 
% 2016 & \citet{liu_follow_2016} & $\bullet$ &  &  & $\bullet$ (T) & $\bullet$ &  &  &  &  & $\bullet$ &  &  &  &  & $\bullet$ &  &  &  & ? & ? &  & D, P & Yes & Yes & Yes & Yes \\
% 2016 & \citet{albertini_privacy_2016} &  &  & $\bullet$ &  &  & $\bullet$ &  &  &  & $\bullet$ &  &  &  &  &  & $\bullet$ &  &  &  & $\bullet$ &  & D & Yes & No & Yes & No \\
% \rowcolor[HTML]{C0C0C0} 
% 2016 & \citet{dong_ppm_2016} &  &  & $\bullet$ & $\bullet$ (T) &  &  &  &  & $\bullet$ &  &  &  & $\bullet$ &  &  & $\bullet$ &  &  & -- & -- & -- & -- & -- & -- & -- & -- \\
% 2017 & \citet{baarslag_automated_2017} & $\bullet$ &  &  &  &  &  & $\bullet$ &  &  & $\bullet$ &  & $\bullet$ &  &  & $\bullet$ &  &  &  & $\bullet$ & $\bullet$ &  & Unclear & Yes & Yes & Yes & No \\
% \rowcolor[HTML]{C0C0C0} 
% 2017 & \citet{fogues_sosharp_2017} &  &  & $\bullet$ & $\bullet$ (PT) &  &  &  &  & $\bullet$ & $\bullet$ &  &  &  &  &  & $\bullet$ &  &  &  & $\bullet$ &  & No & No & No & Yes & No \\
% 2017 & \citet{zhong_group-based_2017} &  &  & $\bullet$ & $\bullet$ (NIT) &  &  &  &  &  & $\bullet$ &  & $\bullet$ &  &  &  & $\bullet$ &  &  & -- & -- & -- & -- & -- & -- & -- & -- \\
% \rowcolor[HTML]{C0C0C0} 
% 2017 & \citet{misra_pacman_2017} &  &  & $\bullet$ & $\bullet$ (NIT) &  &  &  &  & $\bullet$ &  &  &  & $\bullet$ &  &  & $\bullet$ &  &  &  & $\bullet$ &  & D & Yes & Yes & Yes & No \\
% 2017 & \citet{camp_easing_2017} &  & $\bullet$ &  & $\bullet$ (NIT) &  &  &  &  &  &  &  & $\bullet$ &  & $\bullet$ &  &  &  &  & -- & -- & -- & -- & -- & -- & -- & -- \\
% \rowcolor[HTML]{C0C0C0} 
% 2017 & \citet{olejnik_smarper_2017} & $\bullet$ &  &  & $\bullet$ (T) &  &  &  &  & $\bullet$ & $\bullet$ &  &  &  &  & $\bullet$ &  &  &  & $\bullet$ &  &  & No & Yes & Yes & Yes & No \\
% 2018 & \citet{das_personalized_2018} &  & $\bullet$ &  &  &  &  &  &  &  & $\bullet$ &  &  &  &  &  &  & $\bullet$ &  &  & $\bullet$ &  & Yes & It depends & Yes & Yes & No \\
% \rowcolor[HTML]{C0C0C0} 
% 2018 & \citet{tan_context-perceptual_2018} & $\bullet$ &  &  & $\bullet$ (T) &  &  &  &  &  &  & $\bullet$ &  &  &  & $\bullet$ &  &  &  &  & $\bullet$ &  & No & No~\tablefootnote{Not by default, they have a sort of ‘user settings’ for expert users} & Yes & Not explicitly & No \\
% 2018 & \citet{wijesekera_contextualizing_2018} & $\bullet$ &  &  & $\bullet$ (NIT) &  &  &  &  & $\bullet$ &  & $\bullet$ & $\bullet$ &  &  & $\bullet$ &  &  &  & $\bullet$ & $\bullet$ &  & D, C & Yes & Yes & Yes & Yes \\
% \rowcolor[HTML]{C0C0C0} 
% 2018 & \citet{yu_leveraging_2018} &  &  & $\bullet$ & $\bullet$ (NIT) &  &  &  &  & $\bullet$ &  &  &  & $\bullet$ &  &  & $\bullet$ &  &  & -- & -- & -- & -- & -- & -- & -- & -- \\
% 2018 & \citet{bahirat_data-driven_2018} &  & $\bullet$ &  & $\bullet$ (T) &  &  &  &  &  & $\bullet$ &  &  &  &  &  &  & $\bullet$ &  & -- & -- & -- & D, P~\tablefootnote{Not consistently} & It depends & It depends & It depends & No \\
% \rowcolor[HTML]{C0C0C0} 
% 2019 & \citet{klingensmith_hypervisor-based_2019} & $\bullet$ &  &  & $\bullet$ (NIT) &  &  &  &  &  &  & $\bullet$ &  &  & $\bullet$ &  &  & $\bullet$ &  & $\bullet$ & $\bullet$ &  & D & Not always & Yes & It depends & No \\
% 2019 & \citet{barbosa_what_2019} &  & $\bullet$ &  & $\bullet$ (PT) &  &  &  &  &  & $\bullet$ & $\bullet$ &  &  &  &  &  & $\bullet$ &  & -- & -- & -- & -- & -- & -- & -- & -- \\
% \rowcolor[HTML]{C0C0C0} 
% 2019 & \citet{alom_helping_2019} &  & $\bullet$ &  & $\bullet$ (PT) &  &  &  &  & $\bullet$ & $\bullet$ &  &  &  &  &  &  &  &  & -- & -- & -- & -- & -- & -- & -- & -- \\
% 2019 & \citet{alom_adapting_2019} &  & $\bullet$ &  & $\bullet$ (NIT) &  &  &  &  &  & $\bullet$ &  &  &  & $\bullet$ &  &  & $\bullet$ &  & -- & -- & -- & -- & -- & -- & -- & -- \\
% \rowcolor[HTML]{C0C0C0} 
% 2020 & \citet{barolli_selflearning_2020} &  &  & $\bullet$ & $\bullet$ (T) & $\bullet$ &  &  &  &  &  & $\bullet$ &  &  &  &  & $\bullet$ &  &  &  & $\bullet$ &  & D & Yes & Yes & Yes & No \\
% 2020 & \citet{kaur_smart_2020} &  & $\bullet$ &  & $\bullet$ (NIT) &  &  &  &  & $\bullet$ &  & $\bullet$ &  &  &  & $\bullet$ &  & $\bullet$ &  & -- & -- & -- & -- & -- & -- & -- & -- \\
% \rowcolor[HTML]{C0C0C0} 
% 2020 & \citet{herrero_automatic_2021} &  &  & $\bullet$ & $\bullet$ (PT) &  &  &  &  &  &  &  &  & $\bullet$ &  &  & $\bullet$ &  &  &  & $\bullet$ &  & D & Yes & Yes & Yes & No \\
% 2020 & \citet{kokciyan_turp_2020} & $\bullet$ &  &  &  &  & $\bullet$ &  &  & $\bullet$ &  & $\bullet$ &  &  &  &  &  & $\bullet$ &  & -- & -- & -- & -- & -- & -- & -- & -- \\
% \rowcolor[HTML]{C0C0C0} 
% 2020 & \citet{sanchez_recommendation_2020} & $\bullet$ &  &  &  & $\bullet$ &  &  &  &  & $\bullet$ &  &  &  &  &  &  & $\bullet$ &  & -- & -- & -- & Unclear & Yes & Yes & Yes & No \\
% 2021 & \citet{barolli_reinforcement_2021} & $\bullet$ &  &  &  &  &  &  & $\bullet$ & $\bullet$ &  & $\bullet$ &  &  &  & $\bullet$ &  & $\bullet$ &  & -- & -- & -- & -- & -- & -- & -- & -- \\
% \rowcolor[HTML]{C0C0C0} 
% 2021 & \citet{lobner_explainable_2021} &  & $\bullet$ &  & $\bullet$ (T) &  &  &  &  &  &  & $\bullet$ & $\bullet$ &  & $\bullet$ &  & $\bullet$ &  &  & -- & -- & -- & -- & -- & -- & -- & -- \\
% 2022 & \citet{filipczuk_automated_2022} & $\bullet$ &  &  &  &  &  & $\bullet$ &  &  & $\bullet$ &  & $\bullet$ &  &  & $\bullet$ &  &  &  & $\bullet$ & $\bullet$ &  & D & Yes & Yes & Yes & No \\
% \rowcolor[HTML]{C0C0C0} 
% 2022 & \citet{hirschprung_game_2022} &  &  & $\bullet$ &  &  &  & $\bullet$ &  & $\bullet$ &  &  &  &  &  &  & $\bullet$ &  &  & -- & -- & -- & -- & -- & -- & -- & -- \\
% 2022 & \citet{kokciyan_taking_2022} &  & $\bullet$ &  &  &  &  & $\bullet$ &  & $\bullet$ &  &  &  &  & $\bullet$ &  &  & $\bullet$ &  & -- & -- & -- & No & It depends & Yes & Yes~\footnote{Through a feedback system} & No \\
% \rowcolor[HTML]{C0C0C0} 
% 2022 & \citet{ulusoy_panola_2022} &  &  & $\bullet$ &  &  &  &  & $\bullet$ & $\bullet$ &  &  &  &  & $\bullet$ &  & $\bullet$ &  &  & -- & -- & -- & -- & -- & -- & -- & -- \\
% 2022 & \citet{zhan_model_2022} & $\bullet$ &  &  &  &  &  & $\bullet$ & $\bullet$ &  & $\bullet$ &  &  &  &  &  &  & $\bullet$ &  & -- & -- & -- & -- & -- & -- & -- & -- \\
% \rowcolor[HTML]{C0C0C0} 
% 2022 & \citet{brandao_prediction_2022} &  & $\bullet$ &  & $\bullet$ (NIT) & $\bullet$ &  &  &  & $\bullet$ &  &  &  &  &  & $\bullet$ &  &  &  &  &  & $\bullet$ & -- & -- & -- & -- & -- \\
% 2022 & \citet{mendes_enhancing_2022} & $\bullet$ &  &  & $\bullet$ (NIT) &  &  &  &  & $\bullet$ &  &  &  &  &  & $\bullet$ &  &  &  &  & $\bullet$ &  & No & No & No & Not explicitly & No \\
% \rowcolor[HTML]{C0C0C0} 
% 2022 & \citet{shanmugarasa_automated_2022} & $\bullet$ &  &  &  & $\bullet$ &  &  &  & $\bullet$ & $\bullet$ & $\bullet$ & $\bullet$ &  &  &  &  & $\bullet$ &  & $\bullet$ &  &  & No & Yes & Yes & Yes & No \\
% 2023 & \citet{ayci_uncertainty-aware_2023} &  &  & $\bullet$ & $\bullet$ (NIT) &  &  &  &  &  & $\bullet$ & $\bullet$ &  &  &  &  & $\bullet$ &  &  &  & $\bullet$ &  & No & Yes & Yes & Yes & No \\
% \rowcolor[HTML]{C0C0C0} 
% 2023 & \citet{serramia_predicting_2023} &  & $\bullet$ &  &  &  & $\bullet$ &  &  &  & $\bullet$ & $\bullet$ &  &  &  &  &  & $\bullet$ &  &  & $\bullet$ &  & No & Yes & No & Yes & No \\ \bottomrule
% \end{tabular}

% \begin{tablenotes}
% \item[2] Only location
% \item[3] Not necessarily, depends on what they call the Configuration Options
% \item[4] Not necessarily, depends on what they call the Configuration Options
% \item[5] Not by default, they have a sort of ‘user settings’ for expert users
% \item[6] Not consistently
% \item[7] Through a feedback system
% \end{tablenotes}


% \caption{Summary table of our classification.
% ~
% % We present the \textbf{type of decision} (see Section~\ref{subsec:taxo_decision}); the \textbf{type of AI used} (see Section~\ref{subsec:taxo_ai}); the \textbf{type of source of data} (see Section~\ref{subsec:taxo_source}); the \textbf{system context} (see Section~\ref{subsec:taxo_system_context}); the \textbf{architecture} (see Section~\ref{subsec:taxo_architecture}); and finally (see Section~\ref{subsec:taxo_req}).
% % ~
% For \textbf{user control over decisions}, we further specify the elements present to inform users under \textit{Informed}, with D standing for \textit{datatype}, P for \textit{purpose}, C for \textit{controller}.
% ~
% An empty field signifies that the solution surveyed does not exhibit the characteristic (e.g. does not consider Y type of decision).
% ~
% More specifically to the Architecture, we denote with ``--'' when the criterion is not applicable (no implementation/tool is presented) and when the solution is presenting an implementation, but the paper did not specify enough information to infer its architecture.
% ~
% For the type of AI used, we specified whether the classification model is transparent (T), non-intrinsically transparent (NIT), or partially transparent (PT) because several models are used}
% \label{tab:classification}
% \end{threeparttable}
% \end{table*}

% Please add the following required packages to your document preamble:
% \usepackage{multirow}
% \usepackage[table,xcdraw]{xcolor}
% Beamer presentation requires \usepackage{colortbl} instead of \usepackage[table,xcdraw]{xcolor}
\begin{table*}[]
\tiny
\begin{threeparttable}[b]
\begin{tabular}{l|l|lll|lllll|llllll|llll|lll|llll}
\hline
 &  & \multicolumn{3}{l|}{Type of decision} & \multicolumn{5}{l|}{Type of AI used} & \multicolumn{6}{l|}{Type of source of data} & \multicolumn{4}{l|}{System context} & \multicolumn{3}{l|}{Architecture} & \multicolumn{4}{l}{User control over decision} \\ \cline{3-27} 
 &  & \multicolumn{1}{l|}{} & \multicolumn{1}{l|}{} &  & \multicolumn{1}{l|}{} & \multicolumn{1}{l|}{} & \multicolumn{1}{l|}{} & \multicolumn{1}{l|}{} &  & \multicolumn{1}{l|}{} & \multicolumn{1}{l|}{} & \multicolumn{1}{l|}{} & \multicolumn{1}{l|}{} & \multicolumn{1}{l|}{} &  & \multicolumn{1}{l|}{} & \multicolumn{1}{l|}{} & \multicolumn{1}{l|}{} &  & \multicolumn{1}{l|}{} & \multicolumn{1}{l|}{} &  & \multicolumn{1}{l|}{} & \multicolumn{1}{l|}{} & \multicolumn{1}{l|}{} &  \\
 &  & \multicolumn{1}{l|}{} & \multicolumn{1}{l|}{} &  & \multicolumn{1}{l|}{} & \multicolumn{1}{l|}{} & \multicolumn{1}{l|}{} & \multicolumn{1}{l|}{} &  & \multicolumn{1}{l|}{} & \multicolumn{1}{l|}{} & \multicolumn{1}{l|}{} & \multicolumn{1}{l|}{} & \multicolumn{1}{l|}{} &  & \multicolumn{1}{l|}{} & \multicolumn{1}{l|}{} & \multicolumn{1}{l|}{} &  & \multicolumn{1}{l|}{} & \multicolumn{1}{l|}{} &  & \multicolumn{1}{l|}{} & \multicolumn{1}{l|}{} & \multicolumn{1}{l|}{} &  \\
 &  & \multicolumn{1}{l|}{} & \multicolumn{1}{l|}{} &  & \multicolumn{1}{l|}{} & \multicolumn{1}{l|}{} & \multicolumn{1}{l|}{} & \multicolumn{1}{l|}{} &  & \multicolumn{1}{l|}{} & \multicolumn{1}{l|}{} & \multicolumn{1}{l|}{} & \multicolumn{1}{l|}{} & \multicolumn{1}{l|}{} &  & \multicolumn{1}{l|}{} & \multicolumn{1}{l|}{} & \multicolumn{1}{l|}{} &  & \multicolumn{1}{l|}{} & \multicolumn{1}{l|}{} &  & \multicolumn{1}{l|}{} & \multicolumn{1}{l|}{} & \multicolumn{1}{l|}{} &  \\
 &  & \multicolumn{1}{l|}{} & \multicolumn{1}{l|}{} &  & \multicolumn{1}{l|}{} & \multicolumn{1}{l|}{} & \multicolumn{1}{l|}{} & \multicolumn{1}{l|}{} &  & \multicolumn{1}{l|}{} & \multicolumn{1}{l|}{} & \multicolumn{1}{l|}{} & \multicolumn{1}{l|}{} & \multicolumn{1}{l|}{} &  & \multicolumn{1}{l|}{} & \multicolumn{1}{l|}{} & \multicolumn{1}{l|}{} &  & \multicolumn{1}{l|}{} & \multicolumn{1}{l|}{} &  & \multicolumn{1}{l|}{} & \multicolumn{1}{l|}{} & \multicolumn{1}{l|}{} &  \\
 &  & \multicolumn{1}{l|}{} & \multicolumn{1}{l|}{} &  & \multicolumn{1}{l|}{} & \multicolumn{1}{l|}{} & \multicolumn{1}{l|}{} & \multicolumn{1}{l|}{} &  & \multicolumn{1}{l|}{} & \multicolumn{1}{l|}{} & \multicolumn{1}{l|}{} & \multicolumn{1}{l|}{} & \multicolumn{1}{l|}{} &  & \multicolumn{1}{l|}{} & \multicolumn{1}{l|}{} & \multicolumn{1}{l|}{} &  & \multicolumn{1}{l|}{} & \multicolumn{1}{l|}{} &  & \multicolumn{1}{l|}{} & \multicolumn{1}{l|}{} & \multicolumn{1}{l|}{} &  \\
\multirow{-7}{*}{Year} & \multirow{-7}{*}{Publication} & \multicolumn{1}{l|}{\begin{rotate}{90} Permissions \end{rotate}} & \multicolumn{1}{l|}{\begin{rotate}{90} Preferences \end{rotate}} & \begin{rotate}{90} Data sharing \end{rotate} & \multicolumn{1}{l|}{\begin{rotate}{90} Classification \end{rotate}} & \multicolumn{1}{l|}{\begin{rotate}{90} Clustering \end{rotate}} & \multicolumn{1}{l|}{\begin{rotate}{90} Rule-based \end{rotate}} & \multicolumn{1}{l|}{\begin{rotate}{90} Logic-based \end{rotate}} & \begin{rotate}{90} Reinforcement \end{rotate} & \multicolumn{1}{l|}{\begin{rotate}{90} Context \end{rotate}} & \multicolumn{1}{l|}{\begin{rotate}{90} Attitudinal data \end{rotate}} & \multicolumn{1}{l|}{\begin{rotate}{90} Metadata \end{rotate}} & \multicolumn{1}{l|}{\begin{rotate}{90} Type of data \end{rotate}} & \multicolumn{1}{l|}{\begin{rotate}{90} Content of data \end{rotate}} & \begin{rotate}{90} Behavioral data \end{rotate} & \multicolumn{1}{l|}{\begin{rotate}{90} Mobile apps \end{rotate}} & \multicolumn{1}{l|}{\begin{rotate}{90} Social media \end{rotate}} & \multicolumn{1}{l|}{\begin{rotate}{90} IoT \end{rotate}} & \begin{rotate}{90} Cloud \end{rotate} & \multicolumn{1}{l|}{\begin{rotate}{90} Local \end{rotate}} & \multicolumn{1}{l|}{\begin{rotate}{90} Remote \end{rotate}} & \begin{rotate}{90} Federated \end{rotate} & \multicolumn{1}{l|}{\begin{rotate}{90} Informed \end{rotate}} & \multicolumn{1}{l|}{\begin{rotate}{90} Semi-automated \end{rotate}} & \multicolumn{1}{l|}{\begin{rotate}{90} Specific \end{rotate}} & \begin{rotate}{90} Revoke \end{rotate} \\ \hline
\rowcolor[HTML]{C0C0C0} 
2014 & \citet{xie_location_2014} & \multicolumn{1}{l|}{\cellcolor[HTML]{C0C0C0}} & \multicolumn{1}{l|}{\cellcolor[HTML]{C0C0C0}$\bullet$} &  & \multicolumn{1}{l|}{\cellcolor[HTML]{C0C0C0}$\bullet$ (NIT)} & \multicolumn{1}{l|}{\cellcolor[HTML]{C0C0C0}} & \multicolumn{1}{l|}{\cellcolor[HTML]{C0C0C0}} & \multicolumn{1}{l|}{\cellcolor[HTML]{C0C0C0}} &  & \multicolumn{1}{l|}{\cellcolor[HTML]{C0C0C0}$\bullet$} & \multicolumn{1}{l|}{\cellcolor[HTML]{C0C0C0}} & \multicolumn{1}{l|}{\cellcolor[HTML]{C0C0C0}} & \multicolumn{1}{l|}{\cellcolor[HTML]{C0C0C0}} & \multicolumn{1}{l|}{\cellcolor[HTML]{C0C0C0}} &  & \multicolumn{1}{l|}{\cellcolor[HTML]{C0C0C0}} & \multicolumn{1}{l|}{\cellcolor[HTML]{C0C0C0}} & \multicolumn{1}{l|}{\cellcolor[HTML]{C0C0C0}} &  & \multicolumn{1}{l|}{\cellcolor[HTML]{C0C0C0}--} & \multicolumn{1}{l|}{\cellcolor[HTML]{C0C0C0}--} & -- & \multicolumn{1}{l|}{\cellcolor[HTML]{C0C0C0}No} & \multicolumn{1}{l|}{\cellcolor[HTML]{C0C0C0}Yes} & \multicolumn{1}{l|}{\cellcolor[HTML]{C0C0C0}Yes~\tablefootnote{Only location}} & No \\
2015 & \citet{apolinarski_automating_2015} & \multicolumn{1}{l|}{$\bullet$} & \multicolumn{1}{l|}{} &  & \multicolumn{1}{l|}{$\bullet$ (NIT)} & \multicolumn{1}{l|}{} & \multicolumn{1}{l|}{} & \multicolumn{1}{l|}{} &  & \multicolumn{1}{l|}{$\bullet$} & \multicolumn{1}{l|}{} & \multicolumn{1}{l|}{} & \multicolumn{1}{l|}{} & \multicolumn{1}{l|}{} &  & \multicolumn{1}{l|}{$\bullet$} & \multicolumn{1}{l|}{} & \multicolumn{1}{l|}{} &  & \multicolumn{1}{l|}{$\bullet$} & \multicolumn{1}{l|}{} &  & \multicolumn{1}{l|}{D} & \multicolumn{1}{l|}{Yes} & \multicolumn{1}{l|}{Yes} & No \\
\rowcolor[HTML]{C0C0C0} 
2015 & \citet{hirschprung_simplifying_2015} & \multicolumn{1}{l|}{\cellcolor[HTML]{C0C0C0}$\bullet$} & \multicolumn{1}{l|}{\cellcolor[HTML]{C0C0C0}} &  & \multicolumn{1}{l|}{\cellcolor[HTML]{C0C0C0}} & \multicolumn{1}{l|}{\cellcolor[HTML]{C0C0C0}$\bullet$} & \multicolumn{1}{l|}{\cellcolor[HTML]{C0C0C0}} & \multicolumn{1}{l|}{\cellcolor[HTML]{C0C0C0}} &  & \multicolumn{1}{l|}{\cellcolor[HTML]{C0C0C0}} & \multicolumn{1}{l|}{\cellcolor[HTML]{C0C0C0}$\bullet$} & \multicolumn{1}{l|}{\cellcolor[HTML]{C0C0C0}} & \multicolumn{1}{l|}{\cellcolor[HTML]{C0C0C0}} & \multicolumn{1}{l|}{\cellcolor[HTML]{C0C0C0}} &  & \multicolumn{1}{l|}{\cellcolor[HTML]{C0C0C0}} & \multicolumn{1}{l|}{\cellcolor[HTML]{C0C0C0}} & \multicolumn{1}{l|}{\cellcolor[HTML]{C0C0C0}} & $\bullet$ & \multicolumn{1}{l|}{\cellcolor[HTML]{C0C0C0}--} & \multicolumn{1}{l|}{\cellcolor[HTML]{C0C0C0}--} & -- & \multicolumn{1}{l|}{\cellcolor[HTML]{C0C0C0}D} & \multicolumn{1}{l|}{\cellcolor[HTML]{C0C0C0}No~\tablefootnote{Not necessarily, depends on what they call the Configuration Options}} & \multicolumn{1}{l|}{\cellcolor[HTML]{C0C0C0}Yes} & No \\
2015 & \citet{squicciarini_privacy_2015} & \multicolumn{1}{l|}{} & \multicolumn{1}{l|}{} & $\bullet$ & \multicolumn{1}{l|}{} & \multicolumn{1}{l|}{} & \multicolumn{1}{l|}{$\bullet$} & \multicolumn{1}{l|}{} &  & \multicolumn{1}{l|}{$\bullet$} & \multicolumn{1}{l|}{} & \multicolumn{1}{l|}{$\bullet$} & \multicolumn{1}{l|}{} & \multicolumn{1}{l|}{$\bullet$} &  & \multicolumn{1}{l|}{} & \multicolumn{1}{l|}{} & \multicolumn{1}{l|}{} &  & \multicolumn{1}{l|}{--} & \multicolumn{1}{l|}{--} & -- & \multicolumn{1}{l|}{D} & \multicolumn{1}{l|}{Yes} & \multicolumn{1}{l|}{Yes} & No \\
\rowcolor[HTML]{C0C0C0} 
2016 & \citet{liu_follow_2016} & \multicolumn{1}{l|}{\cellcolor[HTML]{C0C0C0}$\bullet$} & \multicolumn{1}{l|}{\cellcolor[HTML]{C0C0C0}} &  & \multicolumn{1}{l|}{\cellcolor[HTML]{C0C0C0}$\bullet$ (T)} & \multicolumn{1}{l|}{\cellcolor[HTML]{C0C0C0}$\bullet$} & \multicolumn{1}{l|}{\cellcolor[HTML]{C0C0C0}} & \multicolumn{1}{l|}{\cellcolor[HTML]{C0C0C0}} &  & \multicolumn{1}{l|}{\cellcolor[HTML]{C0C0C0}} & \multicolumn{1}{l|}{\cellcolor[HTML]{C0C0C0}$\bullet$} & \multicolumn{1}{l|}{\cellcolor[HTML]{C0C0C0}} & \multicolumn{1}{l|}{\cellcolor[HTML]{C0C0C0}} & \multicolumn{1}{l|}{\cellcolor[HTML]{C0C0C0}} &  & \multicolumn{1}{l|}{\cellcolor[HTML]{C0C0C0}$\bullet$} & \multicolumn{1}{l|}{\cellcolor[HTML]{C0C0C0}} & \multicolumn{1}{l|}{\cellcolor[HTML]{C0C0C0}} &  & \multicolumn{1}{l|}{\cellcolor[HTML]{C0C0C0}?} & \multicolumn{1}{l|}{\cellcolor[HTML]{C0C0C0}?} &  & \multicolumn{1}{l|}{\cellcolor[HTML]{C0C0C0}D, P} & \multicolumn{1}{l|}{\cellcolor[HTML]{C0C0C0}Yes} & \multicolumn{1}{l|}{\cellcolor[HTML]{C0C0C0}Yes} & Yes \\
2016 & \citet{albertini_privacy_2016} & \multicolumn{1}{l|}{} & \multicolumn{1}{l|}{} & $\bullet$ & \multicolumn{1}{l|}{} & \multicolumn{1}{l|}{} & \multicolumn{1}{l|}{$\bullet$} & \multicolumn{1}{l|}{} &  & \multicolumn{1}{l|}{} & \multicolumn{1}{l|}{$\bullet$} & \multicolumn{1}{l|}{} & \multicolumn{1}{l|}{} & \multicolumn{1}{l|}{} &  & \multicolumn{1}{l|}{} & \multicolumn{1}{l|}{$\bullet$} & \multicolumn{1}{l|}{} &  & \multicolumn{1}{l|}{} & \multicolumn{1}{l|}{$\bullet$} &  & \multicolumn{1}{l|}{D} & \multicolumn{1}{l|}{Yes} & \multicolumn{1}{l|}{No} & No \\
\rowcolor[HTML]{C0C0C0} 
2016 & \citet{dong_ppm_2016} & \multicolumn{1}{l|}{\cellcolor[HTML]{C0C0C0}} & \multicolumn{1}{l|}{\cellcolor[HTML]{C0C0C0}} & $\bullet$ & \multicolumn{1}{l|}{\cellcolor[HTML]{C0C0C0}$\bullet$ (T)} & \multicolumn{1}{l|}{\cellcolor[HTML]{C0C0C0}} & \multicolumn{1}{l|}{\cellcolor[HTML]{C0C0C0}} & \multicolumn{1}{l|}{\cellcolor[HTML]{C0C0C0}} &  & \multicolumn{1}{l|}{\cellcolor[HTML]{C0C0C0}$\bullet$} & \multicolumn{1}{l|}{\cellcolor[HTML]{C0C0C0}} & \multicolumn{1}{l|}{\cellcolor[HTML]{C0C0C0}} & \multicolumn{1}{l|}{\cellcolor[HTML]{C0C0C0}} & \multicolumn{1}{l|}{\cellcolor[HTML]{C0C0C0}$\bullet$} &  & \multicolumn{1}{l|}{\cellcolor[HTML]{C0C0C0}} & \multicolumn{1}{l|}{\cellcolor[HTML]{C0C0C0}$\bullet$} & \multicolumn{1}{l|}{\cellcolor[HTML]{C0C0C0}} &  & \multicolumn{1}{l|}{\cellcolor[HTML]{C0C0C0}--} & \multicolumn{1}{l|}{\cellcolor[HTML]{C0C0C0}--} & -- & \multicolumn{1}{l|}{\cellcolor[HTML]{C0C0C0}--} & \multicolumn{1}{l|}{\cellcolor[HTML]{C0C0C0}--} & \multicolumn{1}{l|}{\cellcolor[HTML]{C0C0C0}--} & -- \\
2017 & \citet{baarslag_automated_2017} & \multicolumn{1}{l|}{$\bullet$} & \multicolumn{1}{l|}{} &  & \multicolumn{1}{l|}{} & \multicolumn{1}{l|}{} & \multicolumn{1}{l|}{} & \multicolumn{1}{l|}{$\bullet$} &  & \multicolumn{1}{l|}{} & \multicolumn{1}{l|}{$\bullet$} & \multicolumn{1}{l|}{} & \multicolumn{1}{l|}{$\bullet$} & \multicolumn{1}{l|}{} &  & \multicolumn{1}{l|}{$\bullet$} & \multicolumn{1}{l|}{} & \multicolumn{1}{l|}{} &  & \multicolumn{1}{l|}{$\bullet$} & \multicolumn{1}{l|}{$\bullet$} &  & \multicolumn{1}{l|}{Unclear} & \multicolumn{1}{l|}{Yes} & \multicolumn{1}{l|}{Yes} & No \\
\rowcolor[HTML]{C0C0C0} 
2017 & \citet{fogues_sosharp_2017} & \multicolumn{1}{l|}{\cellcolor[HTML]{C0C0C0}} & \multicolumn{1}{l|}{\cellcolor[HTML]{C0C0C0}} & $\bullet$ & \multicolumn{1}{l|}{\cellcolor[HTML]{C0C0C0}$\bullet$ (PT)} & \multicolumn{1}{l|}{\cellcolor[HTML]{C0C0C0}} & \multicolumn{1}{l|}{\cellcolor[HTML]{C0C0C0}} & \multicolumn{1}{l|}{\cellcolor[HTML]{C0C0C0}} &  & \multicolumn{1}{l|}{\cellcolor[HTML]{C0C0C0}$\bullet$} & \multicolumn{1}{l|}{\cellcolor[HTML]{C0C0C0}$\bullet$} & \multicolumn{1}{l|}{\cellcolor[HTML]{C0C0C0}} & \multicolumn{1}{l|}{\cellcolor[HTML]{C0C0C0}} & \multicolumn{1}{l|}{\cellcolor[HTML]{C0C0C0}} &  & \multicolumn{1}{l|}{\cellcolor[HTML]{C0C0C0}} & \multicolumn{1}{l|}{\cellcolor[HTML]{C0C0C0}$\bullet$} & \multicolumn{1}{l|}{\cellcolor[HTML]{C0C0C0}} &  & \multicolumn{1}{l|}{\cellcolor[HTML]{C0C0C0}} & \multicolumn{1}{l|}{\cellcolor[HTML]{C0C0C0}$\bullet$} &  & \multicolumn{1}{l|}{\cellcolor[HTML]{C0C0C0}No} & \multicolumn{1}{l|}{\cellcolor[HTML]{C0C0C0}No} & \multicolumn{1}{l|}{\cellcolor[HTML]{C0C0C0}No} & No \\
2017 & \citet{zhong_group-based_2017} & \multicolumn{1}{l|}{} & \multicolumn{1}{l|}{} & $\bullet$ & \multicolumn{1}{l|}{$\bullet$ (NIT)} & \multicolumn{1}{l|}{} & \multicolumn{1}{l|}{} & \multicolumn{1}{l|}{} &  & \multicolumn{1}{l|}{} & \multicolumn{1}{l|}{$\bullet$} & \multicolumn{1}{l|}{} & \multicolumn{1}{l|}{$\bullet$} & \multicolumn{1}{l|}{} &  & \multicolumn{1}{l|}{} & \multicolumn{1}{l|}{$\bullet$} & \multicolumn{1}{l|}{} &  & \multicolumn{1}{l|}{--} & \multicolumn{1}{l|}{--} & -- & \multicolumn{1}{l|}{--} & \multicolumn{1}{l|}{--} & \multicolumn{1}{l|}{--} & -- \\
\rowcolor[HTML]{C0C0C0} 
2017 & \citet{misra_pacman_2017} & \multicolumn{1}{l|}{\cellcolor[HTML]{C0C0C0}} & \multicolumn{1}{l|}{\cellcolor[HTML]{C0C0C0}} & $\bullet$ & \multicolumn{1}{l|}{\cellcolor[HTML]{C0C0C0}$\bullet$ (NIT)} & \multicolumn{1}{l|}{\cellcolor[HTML]{C0C0C0}} & \multicolumn{1}{l|}{\cellcolor[HTML]{C0C0C0}} & \multicolumn{1}{l|}{\cellcolor[HTML]{C0C0C0}} &  & \multicolumn{1}{l|}{\cellcolor[HTML]{C0C0C0}$\bullet$} & \multicolumn{1}{l|}{\cellcolor[HTML]{C0C0C0}} & \multicolumn{1}{l|}{\cellcolor[HTML]{C0C0C0}} & \multicolumn{1}{l|}{\cellcolor[HTML]{C0C0C0}} & \multicolumn{1}{l|}{\cellcolor[HTML]{C0C0C0}$\bullet$} &  & \multicolumn{1}{l|}{\cellcolor[HTML]{C0C0C0}} & \multicolumn{1}{l|}{\cellcolor[HTML]{C0C0C0}$\bullet$} & \multicolumn{1}{l|}{\cellcolor[HTML]{C0C0C0}} &  & \multicolumn{1}{l|}{\cellcolor[HTML]{C0C0C0}} & \multicolumn{1}{l|}{\cellcolor[HTML]{C0C0C0}$\bullet$} &  & \multicolumn{1}{l|}{\cellcolor[HTML]{C0C0C0}D} & \multicolumn{1}{l|}{\cellcolor[HTML]{C0C0C0}Yes} & \multicolumn{1}{l|}{\cellcolor[HTML]{C0C0C0}Yes} & No \\
2017 & \citet{camp_easing_2017} & \multicolumn{1}{l|}{} & \multicolumn{1}{l|}{$\bullet$} &  & \multicolumn{1}{l|}{$\bullet$ (NIT)} & \multicolumn{1}{l|}{} & \multicolumn{1}{l|}{} & \multicolumn{1}{l|}{} &  & \multicolumn{1}{l|}{} & \multicolumn{1}{l|}{} & \multicolumn{1}{l|}{} & \multicolumn{1}{l|}{$\bullet$} & \multicolumn{1}{l|}{} & $\bullet$ & \multicolumn{1}{l|}{} & \multicolumn{1}{l|}{} & \multicolumn{1}{l|}{} &  & \multicolumn{1}{l|}{--} & \multicolumn{1}{l|}{--} & -- & \multicolumn{1}{l|}{--} & \multicolumn{1}{l|}{--} & \multicolumn{1}{l|}{--} & -- \\
\rowcolor[HTML]{C0C0C0} 
2017 & \citet{olejnik_smarper_2017} & \multicolumn{1}{l|}{\cellcolor[HTML]{C0C0C0}$\bullet$} & \multicolumn{1}{l|}{\cellcolor[HTML]{C0C0C0}} &  & \multicolumn{1}{l|}{\cellcolor[HTML]{C0C0C0}$\bullet$ (T)} & \multicolumn{1}{l|}{\cellcolor[HTML]{C0C0C0}} & \multicolumn{1}{l|}{\cellcolor[HTML]{C0C0C0}} & \multicolumn{1}{l|}{\cellcolor[HTML]{C0C0C0}} &  & \multicolumn{1}{l|}{\cellcolor[HTML]{C0C0C0}$\bullet$} & \multicolumn{1}{l|}{\cellcolor[HTML]{C0C0C0}$\bullet$} & \multicolumn{1}{l|}{\cellcolor[HTML]{C0C0C0}} & \multicolumn{1}{l|}{\cellcolor[HTML]{C0C0C0}} & \multicolumn{1}{l|}{\cellcolor[HTML]{C0C0C0}} &  & \multicolumn{1}{l|}{\cellcolor[HTML]{C0C0C0}$\bullet$} & \multicolumn{1}{l|}{\cellcolor[HTML]{C0C0C0}} & \multicolumn{1}{l|}{\cellcolor[HTML]{C0C0C0}} &  & \multicolumn{1}{l|}{\cellcolor[HTML]{C0C0C0}$\bullet$} & \multicolumn{1}{l|}{\cellcolor[HTML]{C0C0C0}} &  & \multicolumn{1}{l|}{\cellcolor[HTML]{C0C0C0}No} & \multicolumn{1}{l|}{\cellcolor[HTML]{C0C0C0}Yes} & \multicolumn{1}{l|}{\cellcolor[HTML]{C0C0C0}Yes} & No \\
2018 & \citet{das_personalized_2018} & \multicolumn{1}{l|}{} & \multicolumn{1}{l|}{$\bullet$} &  & \multicolumn{1}{l|}{} & \multicolumn{1}{l|}{} & \multicolumn{1}{l|}{} & \multicolumn{1}{l|}{} &  & \multicolumn{1}{l|}{} & \multicolumn{1}{l|}{$\bullet$} & \multicolumn{1}{l|}{} & \multicolumn{1}{l|}{} & \multicolumn{1}{l|}{} &  & \multicolumn{1}{l|}{} & \multicolumn{1}{l|}{} & \multicolumn{1}{l|}{$\bullet$} &  & \multicolumn{1}{l|}{} & \multicolumn{1}{l|}{$\bullet$} &  & \multicolumn{1}{l|}{Yes} & \multicolumn{1}{l|}{It depends} & \multicolumn{1}{l|}{Yes} & No \\
\rowcolor[HTML]{C0C0C0} 
2018 & \citet{tan_context-perceptual_2018} & \multicolumn{1}{l|}{\cellcolor[HTML]{C0C0C0}$\bullet$} & \multicolumn{1}{l|}{\cellcolor[HTML]{C0C0C0}} &  & \multicolumn{1}{l|}{\cellcolor[HTML]{C0C0C0}$\bullet$ (T)} & \multicolumn{1}{l|}{\cellcolor[HTML]{C0C0C0}} & \multicolumn{1}{l|}{\cellcolor[HTML]{C0C0C0}} & \multicolumn{1}{l|}{\cellcolor[HTML]{C0C0C0}} &  & \multicolumn{1}{l|}{\cellcolor[HTML]{C0C0C0}} & \multicolumn{1}{l|}{\cellcolor[HTML]{C0C0C0}} & \multicolumn{1}{l|}{\cellcolor[HTML]{C0C0C0}$\bullet$} & \multicolumn{1}{l|}{\cellcolor[HTML]{C0C0C0}} & \multicolumn{1}{l|}{\cellcolor[HTML]{C0C0C0}} &  & \multicolumn{1}{l|}{\cellcolor[HTML]{C0C0C0}$\bullet$} & \multicolumn{1}{l|}{\cellcolor[HTML]{C0C0C0}} & \multicolumn{1}{l|}{\cellcolor[HTML]{C0C0C0}} &  & \multicolumn{1}{l|}{\cellcolor[HTML]{C0C0C0}} & \multicolumn{1}{l|}{\cellcolor[HTML]{C0C0C0}$\bullet$} &  & \multicolumn{1}{l|}{\cellcolor[HTML]{C0C0C0}No} & \multicolumn{1}{l|}{\cellcolor[HTML]{C0C0C0}No~\tablefootnote{Not by default, they have a sort of ‘user settings’ for expert users}} & \multicolumn{1}{l|}{\cellcolor[HTML]{C0C0C0}Yes} & No \\
2018 & \citet{wijesekera_contextualizing_2018} & \multicolumn{1}{l|}{$\bullet$} & \multicolumn{1}{l|}{} &  & \multicolumn{1}{l|}{$\bullet$ (NIT)} & \multicolumn{1}{l|}{} & \multicolumn{1}{l|}{} & \multicolumn{1}{l|}{} &  & \multicolumn{1}{l|}{$\bullet$} & \multicolumn{1}{l|}{} & \multicolumn{1}{l|}{$\bullet$} & \multicolumn{1}{l|}{$\bullet$} & \multicolumn{1}{l|}{} &  & \multicolumn{1}{l|}{$\bullet$} & \multicolumn{1}{l|}{} & \multicolumn{1}{l|}{} &  & \multicolumn{1}{l|}{$\bullet$} & \multicolumn{1}{l|}{$\bullet$} &  & \multicolumn{1}{l|}{D, C} & \multicolumn{1}{l|}{Yes} & \multicolumn{1}{l|}{Yes} & Yes \\
\rowcolor[HTML]{C0C0C0} 
2018 & \citet{yu_leveraging_2018} & \multicolumn{1}{l|}{\cellcolor[HTML]{C0C0C0}} & \multicolumn{1}{l|}{\cellcolor[HTML]{C0C0C0}} & $\bullet$ & \multicolumn{1}{l|}{\cellcolor[HTML]{C0C0C0}$\bullet$ (NIT)} & \multicolumn{1}{l|}{\cellcolor[HTML]{C0C0C0}} & \multicolumn{1}{l|}{\cellcolor[HTML]{C0C0C0}} & \multicolumn{1}{l|}{\cellcolor[HTML]{C0C0C0}} &  & \multicolumn{1}{l|}{\cellcolor[HTML]{C0C0C0}$\bullet$} & \multicolumn{1}{l|}{\cellcolor[HTML]{C0C0C0}} & \multicolumn{1}{l|}{\cellcolor[HTML]{C0C0C0}} & \multicolumn{1}{l|}{\cellcolor[HTML]{C0C0C0}} & \multicolumn{1}{l|}{\cellcolor[HTML]{C0C0C0}$\bullet$} &  & \multicolumn{1}{l|}{\cellcolor[HTML]{C0C0C0}} & \multicolumn{1}{l|}{\cellcolor[HTML]{C0C0C0}$\bullet$} & \multicolumn{1}{l|}{\cellcolor[HTML]{C0C0C0}} &  & \multicolumn{1}{l|}{\cellcolor[HTML]{C0C0C0}--} & \multicolumn{1}{l|}{\cellcolor[HTML]{C0C0C0}--} & -- & \multicolumn{1}{l|}{\cellcolor[HTML]{C0C0C0}--} & \multicolumn{1}{l|}{\cellcolor[HTML]{C0C0C0}--} & \multicolumn{1}{l|}{\cellcolor[HTML]{C0C0C0}--} & -- \\
2018 & \citet{bahirat_data-driven_2018} & \multicolumn{1}{l|}{} & \multicolumn{1}{l|}{$\bullet$} &  & \multicolumn{1}{l|}{$\bullet$ (T)} & \multicolumn{1}{l|}{} & \multicolumn{1}{l|}{} & \multicolumn{1}{l|}{} &  & \multicolumn{1}{l|}{} & \multicolumn{1}{l|}{$\bullet$} & \multicolumn{1}{l|}{} & \multicolumn{1}{l|}{} & \multicolumn{1}{l|}{} &  & \multicolumn{1}{l|}{} & \multicolumn{1}{l|}{} & \multicolumn{1}{l|}{$\bullet$} &  & \multicolumn{1}{l|}{--} & \multicolumn{1}{l|}{--} & -- & \multicolumn{1}{l|}{D, P~\tablefootnote{Not consistently}} & \multicolumn{1}{l|}{It depends} & \multicolumn{1}{l|}{It depends} & No \\
\rowcolor[HTML]{C0C0C0} 
2019 & \citet{klingensmith_hypervisor-based_2019} & \multicolumn{1}{l|}{\cellcolor[HTML]{C0C0C0}$\bullet$} & \multicolumn{1}{l|}{\cellcolor[HTML]{C0C0C0}} &  & \multicolumn{1}{l|}{\cellcolor[HTML]{C0C0C0}$\bullet$ (NIT)} & \multicolumn{1}{l|}{\cellcolor[HTML]{C0C0C0}} & \multicolumn{1}{l|}{\cellcolor[HTML]{C0C0C0}} & \multicolumn{1}{l|}{\cellcolor[HTML]{C0C0C0}} &  & \multicolumn{1}{l|}{\cellcolor[HTML]{C0C0C0}} & \multicolumn{1}{l|}{\cellcolor[HTML]{C0C0C0}} & \multicolumn{1}{l|}{\cellcolor[HTML]{C0C0C0}$\bullet$} & \multicolumn{1}{l|}{\cellcolor[HTML]{C0C0C0}} & \multicolumn{1}{l|}{\cellcolor[HTML]{C0C0C0}} & $\bullet$ & \multicolumn{1}{l|}{\cellcolor[HTML]{C0C0C0}} & \multicolumn{1}{l|}{\cellcolor[HTML]{C0C0C0}} & \multicolumn{1}{l|}{\cellcolor[HTML]{C0C0C0}$\bullet$} &  & \multicolumn{1}{l|}{\cellcolor[HTML]{C0C0C0}$\bullet$} & \multicolumn{1}{l|}{\cellcolor[HTML]{C0C0C0}$\bullet$} &  & \multicolumn{1}{l|}{\cellcolor[HTML]{C0C0C0}D} & \multicolumn{1}{l|}{\cellcolor[HTML]{C0C0C0}Not always} & \multicolumn{1}{l|}{\cellcolor[HTML]{C0C0C0}Yes} & No \\
2019 & \citet{barbosa_what_2019} & \multicolumn{1}{l|}{} & \multicolumn{1}{l|}{$\bullet$} &  & \multicolumn{1}{l|}{$\bullet$ (PT)} & \multicolumn{1}{l|}{} & \multicolumn{1}{l|}{} & \multicolumn{1}{l|}{} &  & \multicolumn{1}{l|}{} & \multicolumn{1}{l|}{$\bullet$} & \multicolumn{1}{l|}{$\bullet$} & \multicolumn{1}{l|}{} & \multicolumn{1}{l|}{} &  & \multicolumn{1}{l|}{} & \multicolumn{1}{l|}{} & \multicolumn{1}{l|}{$\bullet$} &  & \multicolumn{1}{l|}{--} & \multicolumn{1}{l|}{--} & -- & \multicolumn{1}{l|}{--} & \multicolumn{1}{l|}{--} & \multicolumn{1}{l|}{--} & -- \\
\rowcolor[HTML]{C0C0C0} 
2019 & \citet{alom_helping_2019} & \multicolumn{1}{l|}{\cellcolor[HTML]{C0C0C0}} & \multicolumn{1}{l|}{\cellcolor[HTML]{C0C0C0}$\bullet$} &  & \multicolumn{1}{l|}{\cellcolor[HTML]{C0C0C0}$\bullet$ (PT)} & \multicolumn{1}{l|}{\cellcolor[HTML]{C0C0C0}} & \multicolumn{1}{l|}{\cellcolor[HTML]{C0C0C0}} & \multicolumn{1}{l|}{\cellcolor[HTML]{C0C0C0}} &  & \multicolumn{1}{l|}{\cellcolor[HTML]{C0C0C0}$\bullet$} & \multicolumn{1}{l|}{\cellcolor[HTML]{C0C0C0}$\bullet$} & \multicolumn{1}{l|}{\cellcolor[HTML]{C0C0C0}} & \multicolumn{1}{l|}{\cellcolor[HTML]{C0C0C0}} & \multicolumn{1}{l|}{\cellcolor[HTML]{C0C0C0}} &  & \multicolumn{1}{l|}{\cellcolor[HTML]{C0C0C0}} & \multicolumn{1}{l|}{\cellcolor[HTML]{C0C0C0}} & \multicolumn{1}{l|}{\cellcolor[HTML]{C0C0C0}} &  & \multicolumn{1}{l|}{\cellcolor[HTML]{C0C0C0}--} & \multicolumn{1}{l|}{\cellcolor[HTML]{C0C0C0}--} & -- & \multicolumn{1}{l|}{\cellcolor[HTML]{C0C0C0}--} & \multicolumn{1}{l|}{\cellcolor[HTML]{C0C0C0}--} & \multicolumn{1}{l|}{\cellcolor[HTML]{C0C0C0}--} & -- \\
2019 & \citet{alom_adapting_2019} & \multicolumn{1}{l|}{} & \multicolumn{1}{l|}{$\bullet$} &  & \multicolumn{1}{l|}{$\bullet$ (NIT)} & \multicolumn{1}{l|}{} & \multicolumn{1}{l|}{} & \multicolumn{1}{l|}{} &  & \multicolumn{1}{l|}{} & \multicolumn{1}{l|}{$\bullet$} & \multicolumn{1}{l|}{} & \multicolumn{1}{l|}{} & \multicolumn{1}{l|}{} & $\bullet$ & \multicolumn{1}{l|}{} & \multicolumn{1}{l|}{} & \multicolumn{1}{l|}{$\bullet$} &  & \multicolumn{1}{l|}{--} & \multicolumn{1}{l|}{--} & -- & \multicolumn{1}{l|}{--} & \multicolumn{1}{l|}{--} & \multicolumn{1}{l|}{--} & -- \\
\rowcolor[HTML]{C0C0C0} 
2020 & \citet{barolli_selflearning_2020} & \multicolumn{1}{l|}{\cellcolor[HTML]{C0C0C0}} & \multicolumn{1}{l|}{\cellcolor[HTML]{C0C0C0}} & $\bullet$ & \multicolumn{1}{l|}{\cellcolor[HTML]{C0C0C0}$\bullet$ (T)} & \multicolumn{1}{l|}{\cellcolor[HTML]{C0C0C0}$\bullet$} & \multicolumn{1}{l|}{\cellcolor[HTML]{C0C0C0}} & \multicolumn{1}{l|}{\cellcolor[HTML]{C0C0C0}} &  & \multicolumn{1}{l|}{\cellcolor[HTML]{C0C0C0}} & \multicolumn{1}{l|}{\cellcolor[HTML]{C0C0C0}} & \multicolumn{1}{l|}{\cellcolor[HTML]{C0C0C0}$\bullet$} & \multicolumn{1}{l|}{\cellcolor[HTML]{C0C0C0}} & \multicolumn{1}{l|}{\cellcolor[HTML]{C0C0C0}} &  & \multicolumn{1}{l|}{\cellcolor[HTML]{C0C0C0}} & \multicolumn{1}{l|}{\cellcolor[HTML]{C0C0C0}$\bullet$} & \multicolumn{1}{l|}{\cellcolor[HTML]{C0C0C0}} &  & \multicolumn{1}{l|}{\cellcolor[HTML]{C0C0C0}} & \multicolumn{1}{l|}{\cellcolor[HTML]{C0C0C0}$\bullet$} &  & \multicolumn{1}{l|}{\cellcolor[HTML]{C0C0C0}D} & \multicolumn{1}{l|}{\cellcolor[HTML]{C0C0C0}Yes} & \multicolumn{1}{l|}{\cellcolor[HTML]{C0C0C0}Yes} & No \\
2020 & \citet{kaur_smart_2020} & \multicolumn{1}{l|}{} & \multicolumn{1}{l|}{$\bullet$} &  & \multicolumn{1}{l|}{$\bullet$ (NIT)} & \multicolumn{1}{l|}{} & \multicolumn{1}{l|}{} & \multicolumn{1}{l|}{} &  & \multicolumn{1}{l|}{$\bullet$} & \multicolumn{1}{l|}{} & \multicolumn{1}{l|}{$\bullet$} & \multicolumn{1}{l|}{} & \multicolumn{1}{l|}{} &  & \multicolumn{1}{l|}{$\bullet$} & \multicolumn{1}{l|}{} & \multicolumn{1}{l|}{$\bullet$} &  & \multicolumn{1}{l|}{--} & \multicolumn{1}{l|}{--} & -- & \multicolumn{1}{l|}{--} & \multicolumn{1}{l|}{--} & \multicolumn{1}{l|}{--} & -- \\
\rowcolor[HTML]{C0C0C0} 
2020 & \citet{herrero_automatic_2021} & \multicolumn{1}{l|}{\cellcolor[HTML]{C0C0C0}} & \multicolumn{1}{l|}{\cellcolor[HTML]{C0C0C0}} & $\bullet$ & \multicolumn{1}{l|}{\cellcolor[HTML]{C0C0C0}$\bullet$ (PT)} & \multicolumn{1}{l|}{\cellcolor[HTML]{C0C0C0}} & \multicolumn{1}{l|}{\cellcolor[HTML]{C0C0C0}} & \multicolumn{1}{l|}{\cellcolor[HTML]{C0C0C0}} &  & \multicolumn{1}{l|}{\cellcolor[HTML]{C0C0C0}} & \multicolumn{1}{l|}{\cellcolor[HTML]{C0C0C0}} & \multicolumn{1}{l|}{\cellcolor[HTML]{C0C0C0}} & \multicolumn{1}{l|}{\cellcolor[HTML]{C0C0C0}} & \multicolumn{1}{l|}{\cellcolor[HTML]{C0C0C0}$\bullet$} &  & \multicolumn{1}{l|}{\cellcolor[HTML]{C0C0C0}} & \multicolumn{1}{l|}{\cellcolor[HTML]{C0C0C0}$\bullet$} & \multicolumn{1}{l|}{\cellcolor[HTML]{C0C0C0}} &  & \multicolumn{1}{l|}{\cellcolor[HTML]{C0C0C0}} & \multicolumn{1}{l|}{\cellcolor[HTML]{C0C0C0}$\bullet$} &  & \multicolumn{1}{l|}{\cellcolor[HTML]{C0C0C0}D} & \multicolumn{1}{l|}{\cellcolor[HTML]{C0C0C0}Yes} & \multicolumn{1}{l|}{\cellcolor[HTML]{C0C0C0}Yes} & No \\
2020 & \citet{kokciyan_turp_2020} & \multicolumn{1}{l|}{$\bullet$} & \multicolumn{1}{l|}{} &  & \multicolumn{1}{l|}{} & \multicolumn{1}{l|}{} & \multicolumn{1}{l|}{$\bullet$} & \multicolumn{1}{l|}{} &  & \multicolumn{1}{l|}{$\bullet$} & \multicolumn{1}{l|}{} & \multicolumn{1}{l|}{$\bullet$} & \multicolumn{1}{l|}{} & \multicolumn{1}{l|}{} &  & \multicolumn{1}{l|}{} & \multicolumn{1}{l|}{} & \multicolumn{1}{l|}{$\bullet$} &  & \multicolumn{1}{l|}{--} & \multicolumn{1}{l|}{--} & -- & \multicolumn{1}{l|}{--} & \multicolumn{1}{l|}{--} & \multicolumn{1}{l|}{--} & -- \\
\rowcolor[HTML]{C0C0C0} 
2020 & \citet{sanchez_recommendation_2020} & \multicolumn{1}{l|}{\cellcolor[HTML]{C0C0C0}$\bullet$} & \multicolumn{1}{l|}{\cellcolor[HTML]{C0C0C0}} &  & \multicolumn{1}{l|}{\cellcolor[HTML]{C0C0C0}} & \multicolumn{1}{l|}{\cellcolor[HTML]{C0C0C0}$\bullet$} & \multicolumn{1}{l|}{\cellcolor[HTML]{C0C0C0}} & \multicolumn{1}{l|}{\cellcolor[HTML]{C0C0C0}} &  & \multicolumn{1}{l|}{\cellcolor[HTML]{C0C0C0}} & \multicolumn{1}{l|}{\cellcolor[HTML]{C0C0C0}$\bullet$} & \multicolumn{1}{l|}{\cellcolor[HTML]{C0C0C0}} & \multicolumn{1}{l|}{\cellcolor[HTML]{C0C0C0}} & \multicolumn{1}{l|}{\cellcolor[HTML]{C0C0C0}} &  & \multicolumn{1}{l|}{\cellcolor[HTML]{C0C0C0}} & \multicolumn{1}{l|}{\cellcolor[HTML]{C0C0C0}} & \multicolumn{1}{l|}{\cellcolor[HTML]{C0C0C0}$\bullet$} &  & \multicolumn{1}{l|}{\cellcolor[HTML]{C0C0C0}--} & \multicolumn{1}{l|}{\cellcolor[HTML]{C0C0C0}--} & -- & \multicolumn{1}{l|}{\cellcolor[HTML]{C0C0C0}Unclear} & \multicolumn{1}{l|}{\cellcolor[HTML]{C0C0C0}Yes} & \multicolumn{1}{l|}{\cellcolor[HTML]{C0C0C0}Yes} & No \\
2021 & \citet{barolli_reinforcement_2021} & \multicolumn{1}{l|}{$\bullet$} & \multicolumn{1}{l|}{} &  & \multicolumn{1}{l|}{} & \multicolumn{1}{l|}{} & \multicolumn{1}{l|}{} & \multicolumn{1}{l|}{} & $\bullet$ & \multicolumn{1}{l|}{$\bullet$} & \multicolumn{1}{l|}{} & \multicolumn{1}{l|}{$\bullet$} & \multicolumn{1}{l|}{} & \multicolumn{1}{l|}{} &  & \multicolumn{1}{l|}{$\bullet$} & \multicolumn{1}{l|}{} & \multicolumn{1}{l|}{$\bullet$} &  & \multicolumn{1}{l|}{--} & \multicolumn{1}{l|}{--} & -- & \multicolumn{1}{l|}{--} & \multicolumn{1}{l|}{--} & \multicolumn{1}{l|}{--} & -- \\
\rowcolor[HTML]{C0C0C0} 
2021 & \citet{lobner_explainable_2021} & \multicolumn{1}{l|}{\cellcolor[HTML]{C0C0C0}} & \multicolumn{1}{l|}{\cellcolor[HTML]{C0C0C0}$\bullet$} &  & \multicolumn{1}{l|}{\cellcolor[HTML]{C0C0C0}$\bullet$ (T)} & \multicolumn{1}{l|}{\cellcolor[HTML]{C0C0C0}} & \multicolumn{1}{l|}{\cellcolor[HTML]{C0C0C0}} & \multicolumn{1}{l|}{\cellcolor[HTML]{C0C0C0}} &  & \multicolumn{1}{l|}{\cellcolor[HTML]{C0C0C0}} & \multicolumn{1}{l|}{\cellcolor[HTML]{C0C0C0}} & \multicolumn{1}{l|}{\cellcolor[HTML]{C0C0C0}$\bullet$} & \multicolumn{1}{l|}{\cellcolor[HTML]{C0C0C0}$\bullet$} & \multicolumn{1}{l|}{\cellcolor[HTML]{C0C0C0}} & $\bullet$ & \multicolumn{1}{l|}{\cellcolor[HTML]{C0C0C0}} & \multicolumn{1}{l|}{\cellcolor[HTML]{C0C0C0}$\bullet$} & \multicolumn{1}{l|}{\cellcolor[HTML]{C0C0C0}} &  & \multicolumn{1}{l|}{\cellcolor[HTML]{C0C0C0}--} & \multicolumn{1}{l|}{\cellcolor[HTML]{C0C0C0}--} & -- & \multicolumn{1}{l|}{\cellcolor[HTML]{C0C0C0}--} & \multicolumn{1}{l|}{\cellcolor[HTML]{C0C0C0}--} & \multicolumn{1}{l|}{\cellcolor[HTML]{C0C0C0}--} & -- \\
2022 & \citet{filipczuk_automated_2022} & \multicolumn{1}{l|}{$\bullet$} & \multicolumn{1}{l|}{} &  & \multicolumn{1}{l|}{} & \multicolumn{1}{l|}{} & \multicolumn{1}{l|}{} & \multicolumn{1}{l|}{$\bullet$} &  & \multicolumn{1}{l|}{} & \multicolumn{1}{l|}{$\bullet$} & \multicolumn{1}{l|}{} & \multicolumn{1}{l|}{$\bullet$} & \multicolumn{1}{l|}{} &  & \multicolumn{1}{l|}{$\bullet$} & \multicolumn{1}{l|}{} & \multicolumn{1}{l|}{} &  & \multicolumn{1}{l|}{$\bullet$} & \multicolumn{1}{l|}{$\bullet$} &  & \multicolumn{1}{l|}{D} & \multicolumn{1}{l|}{Yes} & \multicolumn{1}{l|}{Yes} & No \\
\rowcolor[HTML]{C0C0C0} 
2022 & \citet{hirschprung_game_2022} & \multicolumn{1}{l|}{\cellcolor[HTML]{C0C0C0}} & \multicolumn{1}{l|}{\cellcolor[HTML]{C0C0C0}} & $\bullet$ & \multicolumn{1}{l|}{\cellcolor[HTML]{C0C0C0}} & \multicolumn{1}{l|}{\cellcolor[HTML]{C0C0C0}} & \multicolumn{1}{l|}{\cellcolor[HTML]{C0C0C0}} & \multicolumn{1}{l|}{\cellcolor[HTML]{C0C0C0}$\bullet$} &  & \multicolumn{1}{l|}{\cellcolor[HTML]{C0C0C0}$\bullet$} & \multicolumn{1}{l|}{\cellcolor[HTML]{C0C0C0}} & \multicolumn{1}{l|}{\cellcolor[HTML]{C0C0C0}} & \multicolumn{1}{l|}{\cellcolor[HTML]{C0C0C0}} & \multicolumn{1}{l|}{\cellcolor[HTML]{C0C0C0}} &  & \multicolumn{1}{l|}{\cellcolor[HTML]{C0C0C0}} & \multicolumn{1}{l|}{\cellcolor[HTML]{C0C0C0}$\bullet$} & \multicolumn{1}{l|}{\cellcolor[HTML]{C0C0C0}} &  & \multicolumn{1}{l|}{\cellcolor[HTML]{C0C0C0}--} & \multicolumn{1}{l|}{\cellcolor[HTML]{C0C0C0}--} & -- & \multicolumn{1}{l|}{\cellcolor[HTML]{C0C0C0}--} & \multicolumn{1}{l|}{\cellcolor[HTML]{C0C0C0}--} & \multicolumn{1}{l|}{\cellcolor[HTML]{C0C0C0}--} & -- \\
2022 & \citet{kokciyan_taking_2022} & \multicolumn{1}{l|}{} & \multicolumn{1}{l|}{$\bullet$} &  & \multicolumn{1}{l|}{} & \multicolumn{1}{l|}{} & \multicolumn{1}{l|}{} & \multicolumn{1}{l|}{$\bullet$} &  & \multicolumn{1}{l|}{$\bullet$} & \multicolumn{1}{l|}{} & \multicolumn{1}{l|}{} & \multicolumn{1}{l|}{} & \multicolumn{1}{l|}{} & $\bullet$ & \multicolumn{1}{l|}{} & \multicolumn{1}{l|}{} & \multicolumn{1}{l|}{$\bullet$} &  & \multicolumn{1}{l|}{--} & \multicolumn{1}{l|}{--} & -- & \multicolumn{1}{l|}{No} & \multicolumn{1}{l|}{It depends} & \multicolumn{1}{l|}{Yes} & No \\
\rowcolor[HTML]{C0C0C0} 
2022 & \citet{ulusoy_panola_2022} & \multicolumn{1}{l|}{\cellcolor[HTML]{C0C0C0}} & \multicolumn{1}{l|}{\cellcolor[HTML]{C0C0C0}} & $\bullet$ & \multicolumn{1}{l|}{\cellcolor[HTML]{C0C0C0}} & \multicolumn{1}{l|}{\cellcolor[HTML]{C0C0C0}} & \multicolumn{1}{l|}{\cellcolor[HTML]{C0C0C0}} & \multicolumn{1}{l|}{\cellcolor[HTML]{C0C0C0}} & $\bullet$ & \multicolumn{1}{l|}{\cellcolor[HTML]{C0C0C0}$\bullet$} & \multicolumn{1}{l|}{\cellcolor[HTML]{C0C0C0}} & \multicolumn{1}{l|}{\cellcolor[HTML]{C0C0C0}} & \multicolumn{1}{l|}{\cellcolor[HTML]{C0C0C0}} & \multicolumn{1}{l|}{\cellcolor[HTML]{C0C0C0}} & $\bullet$ & \multicolumn{1}{l|}{\cellcolor[HTML]{C0C0C0}} & \multicolumn{1}{l|}{\cellcolor[HTML]{C0C0C0}$\bullet$} & \multicolumn{1}{l|}{\cellcolor[HTML]{C0C0C0}} &  & \multicolumn{1}{l|}{\cellcolor[HTML]{C0C0C0}--} & \multicolumn{1}{l|}{\cellcolor[HTML]{C0C0C0}--} & -- & \multicolumn{1}{l|}{\cellcolor[HTML]{C0C0C0}--} & \multicolumn{1}{l|}{\cellcolor[HTML]{C0C0C0}--} & \multicolumn{1}{l|}{\cellcolor[HTML]{C0C0C0}--} & -- \\
2022 & \citet{zhan_model_2022} & \multicolumn{1}{l|}{$\bullet$} & \multicolumn{1}{l|}{} &  & \multicolumn{1}{l|}{} & \multicolumn{1}{l|}{} & \multicolumn{1}{l|}{} & \multicolumn{1}{l|}{$\bullet$} & $\bullet$ & \multicolumn{1}{l|}{} & \multicolumn{1}{l|}{$\bullet$} & \multicolumn{1}{l|}{} & \multicolumn{1}{l|}{} & \multicolumn{1}{l|}{} &  & \multicolumn{1}{l|}{} & \multicolumn{1}{l|}{} & \multicolumn{1}{l|}{$\bullet$} &  & \multicolumn{1}{l|}{--} & \multicolumn{1}{l|}{--} & -- & \multicolumn{1}{l|}{--} & \multicolumn{1}{l|}{--} & \multicolumn{1}{l|}{--} & -- \\
\rowcolor[HTML]{C0C0C0} 
2022 & \citet{brandao_prediction_2022} & \multicolumn{1}{l|}{\cellcolor[HTML]{C0C0C0}} & \multicolumn{1}{l|}{\cellcolor[HTML]{C0C0C0}$\bullet$} &  & \multicolumn{1}{l|}{\cellcolor[HTML]{C0C0C0}$\bullet$ (NIT)} & \multicolumn{1}{l|}{\cellcolor[HTML]{C0C0C0}$\bullet$} & \multicolumn{1}{l|}{\cellcolor[HTML]{C0C0C0}} & \multicolumn{1}{l|}{\cellcolor[HTML]{C0C0C0}} &  & \multicolumn{1}{l|}{\cellcolor[HTML]{C0C0C0}$\bullet$} & \multicolumn{1}{l|}{\cellcolor[HTML]{C0C0C0}} & \multicolumn{1}{l|}{\cellcolor[HTML]{C0C0C0}} & \multicolumn{1}{l|}{\cellcolor[HTML]{C0C0C0}} & \multicolumn{1}{l|}{\cellcolor[HTML]{C0C0C0}} &  & \multicolumn{1}{l|}{\cellcolor[HTML]{C0C0C0}$\bullet$} & \multicolumn{1}{l|}{\cellcolor[HTML]{C0C0C0}} & \multicolumn{1}{l|}{\cellcolor[HTML]{C0C0C0}} &  & \multicolumn{1}{l|}{\cellcolor[HTML]{C0C0C0}} & \multicolumn{1}{l|}{\cellcolor[HTML]{C0C0C0}} & $\bullet$ & \multicolumn{1}{l|}{\cellcolor[HTML]{C0C0C0}--} & \multicolumn{1}{l|}{\cellcolor[HTML]{C0C0C0}--} & \multicolumn{1}{l|}{\cellcolor[HTML]{C0C0C0}--} & -- \\
2022 & \citet{mendes_enhancing_2022} & \multicolumn{1}{l|}{$\bullet$} & \multicolumn{1}{l|}{} &  & \multicolumn{1}{l|}{$\bullet$ (NIT)} & \multicolumn{1}{l|}{} & \multicolumn{1}{l|}{} & \multicolumn{1}{l|}{} &  & \multicolumn{1}{l|}{$\bullet$} & \multicolumn{1}{l|}{} & \multicolumn{1}{l|}{} & \multicolumn{1}{l|}{} & \multicolumn{1}{l|}{} &  & \multicolumn{1}{l|}{$\bullet$} & \multicolumn{1}{l|}{} & \multicolumn{1}{l|}{} &  & \multicolumn{1}{l|}{} & \multicolumn{1}{l|}{$\bullet$} &  & \multicolumn{1}{l|}{No} & \multicolumn{1}{l|}{No} & \multicolumn{1}{l|}{No} & No \\
\rowcolor[HTML]{C0C0C0} 
2022 & \citet{shanmugarasa_automated_2022} & \multicolumn{1}{l|}{\cellcolor[HTML]{C0C0C0}$\bullet$} & \multicolumn{1}{l|}{\cellcolor[HTML]{C0C0C0}} &  & \multicolumn{1}{l|}{\cellcolor[HTML]{C0C0C0}} & \multicolumn{1}{l|}{\cellcolor[HTML]{C0C0C0}$\bullet$} & \multicolumn{1}{l|}{\cellcolor[HTML]{C0C0C0}} & \multicolumn{1}{l|}{\cellcolor[HTML]{C0C0C0}} &  & \multicolumn{1}{l|}{\cellcolor[HTML]{C0C0C0}$\bullet$} & \multicolumn{1}{l|}{\cellcolor[HTML]{C0C0C0}$\bullet$} & \multicolumn{1}{l|}{\cellcolor[HTML]{C0C0C0}$\bullet$} & \multicolumn{1}{l|}{\cellcolor[HTML]{C0C0C0}$\bullet$} & \multicolumn{1}{l|}{\cellcolor[HTML]{C0C0C0}} &  & \multicolumn{1}{l|}{\cellcolor[HTML]{C0C0C0}} & \multicolumn{1}{l|}{\cellcolor[HTML]{C0C0C0}} & \multicolumn{1}{l|}{\cellcolor[HTML]{C0C0C0}$\bullet$} &  & \multicolumn{1}{l|}{\cellcolor[HTML]{C0C0C0}$\bullet$} & \multicolumn{1}{l|}{\cellcolor[HTML]{C0C0C0}} &  & \multicolumn{1}{l|}{\cellcolor[HTML]{C0C0C0}No} & \multicolumn{1}{l|}{\cellcolor[HTML]{C0C0C0}Yes} & \multicolumn{1}{l|}{\cellcolor[HTML]{C0C0C0}Yes} & No \\
2023 & \citet{ayci_uncertainty-aware_2023} & \multicolumn{1}{l|}{} & \multicolumn{1}{l|}{} & $\bullet$ & \multicolumn{1}{l|}{$\bullet$ (NIT)} & \multicolumn{1}{l|}{} & \multicolumn{1}{l|}{} & \multicolumn{1}{l|}{} &  & \multicolumn{1}{l|}{} & \multicolumn{1}{l|}{$\bullet$} & \multicolumn{1}{l|}{$\bullet$} & \multicolumn{1}{l|}{} & \multicolumn{1}{l|}{} &  & \multicolumn{1}{l|}{} & \multicolumn{1}{l|}{$\bullet$} & \multicolumn{1}{l|}{} &  & \multicolumn{1}{l|}{} & \multicolumn{1}{l|}{$\bullet$} &  & \multicolumn{1}{l|}{No} & \multicolumn{1}{l|}{Yes} & \multicolumn{1}{l|}{Yes} & No \\
\rowcolor[HTML]{C0C0C0} 
2023 & \citet{serramia_predicting_2023} & \multicolumn{1}{l|}{\cellcolor[HTML]{C0C0C0}} & \multicolumn{1}{l|}{\cellcolor[HTML]{C0C0C0}$\bullet$} &  & \multicolumn{1}{l|}{\cellcolor[HTML]{C0C0C0}} & \multicolumn{1}{l|}{\cellcolor[HTML]{C0C0C0}} & \multicolumn{1}{l|}{\cellcolor[HTML]{C0C0C0}$\bullet$} & \multicolumn{1}{l|}{\cellcolor[HTML]{C0C0C0}} &  & \multicolumn{1}{l|}{\cellcolor[HTML]{C0C0C0}} & \multicolumn{1}{l|}{\cellcolor[HTML]{C0C0C0}$\bullet$} & \multicolumn{1}{l|}{\cellcolor[HTML]{C0C0C0}$\bullet$} & \multicolumn{1}{l|}{\cellcolor[HTML]{C0C0C0}} & \multicolumn{1}{l|}{\cellcolor[HTML]{C0C0C0}} &  & \multicolumn{1}{l|}{\cellcolor[HTML]{C0C0C0}} & \multicolumn{1}{l|}{\cellcolor[HTML]{C0C0C0}} & \multicolumn{1}{l|}{\cellcolor[HTML]{C0C0C0}$\bullet$} &  & \multicolumn{1}{l|}{\cellcolor[HTML]{C0C0C0}} & \multicolumn{1}{l|}{\cellcolor[HTML]{C0C0C0}$\bullet$} &  & \multicolumn{1}{l|}{\cellcolor[HTML]{C0C0C0}No} & \multicolumn{1}{l|}{\cellcolor[HTML]{C0C0C0}Yes} & \multicolumn{1}{l|}{\cellcolor[HTML]{C0C0C0}No} & No \\ \hline
\end{tabular}
\begin{tablenotes}
\item[3] Only location
\item[4] Not necessarily, depends on what they call the Configuration Options
\item[5] Not by default, they have a sort of ‘user settings’ for expert users
\item[6] Not consistently
\end{tablenotes}
\caption{Summary table of our classification.
~
For \textbf{user control over decisions}, we specify the elements present to inform users under \textit{Informed}.
~
An empty field signifies that the solution does not exhibit the characteristic (e.g., does not consider Y type of decision).
~
Under Architecture, we denote with ``--'' when the criterion is not applicable (no implementation/tool is presented) and when the solution presents an implementation, but the paper did not specify enough information to infer its architecture.
~
For the type of AI used, we specified whether the classification model is Transparent (T), Not-Inherently Transparent (NIT), or Partially Transparent (PT) because several models are used.}
\label{tab:classification}
\end{threeparttable}
\end{table*}

% \todo[inline]{Feel free to present some results, I ran out of ideas}

% \subsection{Demographics}
% \todo[inline]{Countries of affiliation, years of publication, types of venues, citations count}
Among the 39 papers surveyed, we tallied 15 different countries for the authors' affiliations (see Table~\ref{tab:countries} in Appendix \ref{app:tables}), with the USA and UK leading in numbers.
About 54\% ($n=21$) of the selected publications were published from 2019 to 2023, with the year 2021 being the most productive with 8 publications (see Table~\ref{tab:years} in the Appendix \ref{app:tables}).
% Up to 8 publications per year (in 2021), likely due to the first year of the coronavirus' aftermath during which many conferences were cancelled, leaving less possibilities to publish in computer science, which left only 2 papers in 2020) (see Table~\ref{tab:years} in the Appendix).
% \todo[inline]{Growing topic, no paper snowballed in 2024}
At the time of writing, papers were cited between 0 and 275 times with an average of 38.56 citations, a median of 14, and a standard deviation of 59, indicating a power law distribution of the citation count.
The most cited papers are \citet{liu_follow_2016} ($n=275$), \citet{yu_leveraging_2018} ($n=199$), and \citet{squicciarini_privacy_2015} ($n=130$).

Regarding the sources of data used by the \PPAs, context data, attitudinal data, and metadata were the most prevalent.
We observed a relatively balanced distribution when it comes to the types of decisions (between 12 and 15 for each type) and the system contexts (between 11 and 13, with one outlier for \textit{Cloud}).
% Note that IoT appears later (in 2018) than other system contexts in our results\footnote{Some papers may have been published on the topic earlier than 2013, date at which we started to include papers in our survey.}, and so does reinforcement learning as a type of AI used (in 2021).
For the types of AI systems that we were able to classify, most models were deemed non-intrinsically transparent (NIT, $n=13$), followed by transparent (T, $n=6$) and partially transparent (PT, $n=4$) models.
Note also that~\citet{das_personalized_2018} did not specify the type of AI used in their paper, we were therefore unable to categorize their solution in that respect (under \textit{Type of AI used}).

The publications were also classified by their types of contributions, according to the categories proposed by~\citet{kuhrmann_software_2016} and~\citet{shaw_writing_2003} (see Appendix \ref{app:protocol}).
We observed a prevalence of models ($n=23$) and tools ($n=19$), followed by frameworks ($n=9$), as shown in Table~\ref{tab:appendix_table}.

% \subsection{User studies critical appraisal}
\label{subsec:appraisal}
% \todo[inline]{Critical appraisal, types of user studies (evaluation does not mean quality of user study)}
Lastly, we also performed a critical appraisal on the user studies presented, although only when those user studies were used to evaluate the \PPA, and not when they were used for data collection for dataset building.
We used the CAT (Critically Appraised Topic) Manager App of CEBMA (the Center for Evidence-Based Management)~\cite{cebma_cat_2025}, which provides a practical yet rigorous approach to evaluate studies based on objective criteria.
The result of our critical appraisal can be found in Table~\ref{tab:appendix_table} in the Appendix \ref{app:tables}.
Out of the 39 publications, only 16 presented a user study, and in terms of quality, they mostly scored ``low'' or ``very low'' ($n=12$) according to the CEBMA checklist.
Exceptionally, only the studies of \citet{liu_follow_2016} and \citet{baarslag_automated_2017} were appraised as of high quality.

% \subsection{Meeting consent requirements}
% \todo[inline]{Based on our assessment, only two solutions could claim at managing consent in a GDPR-compliant way}
% \todo[inline]{Combine Elicitation of decision with Reject}



% \newpage

\section{Classification for \PPAs}
\label{sec:classification}
% \todo[inline]{Victor, work in progress}
We provide in this section a classification for \PPAs as the main contribution of this SoK.
The classification comprises several dimensions, i.e., features typically considered in the design of such an assistant.
These dimensions are the \textit{type of decision} (Section~\ref{subsec:taxo_decision}), the \textit{type of AI} (Section~\ref{subsec:taxo_ai}) and the \textit{source of data} (Section~\ref{subsec:taxo_source}) used in the decision, the \textit{system context} (Section~\ref{subsec:taxo_system_context}), the \textit{choice architecture} of its eventual implementation (Section~\ref{subsec:taxo_architecture}), the \textit{empirical assessment} (Section~\ref{subsec:taxo_validation}), and the extent to which \textit{users have control over the decisions} (Section~\ref{subsec:taxo_req}).

The classification and its dimensions are \textbf{data-driven} in the sense that they were derived based on what is described in the papers, reflecting the current state of the literature.
For example, considering the category of system contexts, more dimensions could be envisioned, but we limited it to the four dimensions (i.e., mobile apps, social media, IoT, and cloud) that were found in the papers.
Each feature will be explored in more detail in this section, and substantiated with non-exhaustive examples for each possible option, while an overview is provided in Figure~\ref{fig:classification}.

Note that not all dimensions are necessary for composing an \PPA.
The dimensions for the type of AI, source of data, type of decision, and system context are \textit{``mandatory,''} consisting of essential requisites that an \PPA needs to consider (solid boxes in Figure~\ref{fig:classification}).
Other dimensions such as the empirical assessment, choice architecture, and user control over decisions are \textit{``optional''} since not all the identified \PPAs were evaluated, some do not have an implementation (and therefore an architecture), and some (regrettably) do not empower users with much control for various reasons (dashed boxes in Figure~\ref{fig:classification}).

Furthermore, the papers address each of these dimensions to different extents, and their options are often non-exclusive.
For instance, all %solutions 
articles surveyed 
%comprise a 
discuss the type of decision, but this is not the case for the choice architecture; and while most solutions are composed of different sources of data and combine different AI models, the system context is often exclusive in the sense that solutions are often designed for a specific system context.

% This classification is, therefore, not an ontology or a taxonomy -- both being much stricter types of categorizations --, it is inherently and purposely loose.





\begin{figure*}
    \centering
    \includegraphics[scale=.29]{Figures/AI-driven_PPA.drawio.png}
    \caption{A schematic representation of the classification presented in Section~\ref{sec:classification}. 
    Each facet is represented as a rounded box, solid for the mandatory features and dashed for the optional ones.
    For User control over decisions (see Section~\ref{subsec:taxo_req}), we distinguish between qualities of control (solid arrows) and instruments of control (dashed arrows).}
    \label{fig:classification}
\end{figure*}






\subsection{Type of Decision}
\label{subsec:taxo_decision}
% \todo[inline]{mention that consent and other rights are not considered but can be envisioned}
Decisions taken by an \PPA can be of different types, and it is essential to distinguish them to assess the possibilities they offer.
Indeed, some decisions -- such as permissions -- have a binding character, i.e., constraining the system to act according to the user's choice, while others do not, such as preferences.
%This decision is important as some privacy decision-making systems (e.g. P3P) were not adopted notably because of the confusion between the (binding) expectations and what was delivered (non-binding).
Note that it may not always be possible to distinguish between each type of decision clearly (as discussed in Section~\ref{subsec:further-decisions}).
Other types of decisions with different implications regarding their enforcement can be envisioned by an \PPA (such as consent or deletion requests, see Section~\ref{subsec:privacy_decisions}).

\subsubsection{Permission Settings}
\label{subsub:permissions}
The first type of decisions that many \PPAs assist the users with is \textit{permission settings}, 
%in a sense akin to access control.
which, as discussed in Section~\ref{subsec:further-decisions}, correspond to access control settings.
Permissions are system-specific and binding, as the underlying operating system should enforce them.
%have a binding character and require the system to actually enforce the decision made.

%As an example, 
We typically find mobile app permissions (e.g., in \citet{baarslag_automated_2017}, mobile apps are addressed in 11 papers), but they are not restricted to the mobile environment.
\PPAs can deal with permissions in IoT environments (see, e.g., ~\cite{klingensmith_hypervisor-based_2019}, IoT is covered by 13 papers) or in the cloud~\cite{hirschprung_simplifying_2015}.

%A particular feature of permissions is that they are specific for the underlying operation system.
%require the cooperation of the underlying technical system (e.g., the mobile OS).
%Permissions only make sense insofar as they can be enforced; hence, the focus is on mobile permissions with their well-defined technical setting.

\subsubsection{Preference Settings}
The second type of decision covered by the literature is \textit{preference settings}, which, unlike %their binding counterparts
permission settings, should be understood as expressions of will.
%They are, therefore, not binding, and controllers can choose to take them into account or not.
Several works refer to preferences while they actually deal with permissions~\cite{liu_follow_2016,filipczuk_automated_2022,wijesekera_contextualizing_2018,hirschprung_simplifying_2015,shanmugarasa_automated_2022}.
It is indeed common to talk about preferences imprecisely, but they should not be confused with permissions that have a binding property.

\subsubsection{Data Sharing}
\textit{Data sharing} is the third type of privacy decision of \PPAs encountered in the reviewed literature, for which the binding character is uncertain for users (for instance, assessing whether a limitation in the audience is enforced is not always possible from a user point of view, because the underlying technical system is inaccessible to them, see for instance \citet{ulusoy_panola_2022}).
Typically, it can be difficult or even impossible to assess whether most social media platforms strictly account for the user's privacy decisions, or merely welcome them as recommendations to be applied only if possible.
Papers classified under this type of decision usually do not mention the binding character of their solution (or the lack thereof).
% Note that data sharing has its own ``bindingness'' in the sense that sharing data on a social media platform is definitive: 

% \todo[inline]{emphasize that papers don't mention the binding character most of the time, and that data sharing is orthogonal to the binding character in the sense that it is definitive}




\subsection{AI Technology Used}
\label{subsec:taxo_ai}
Another significant characteristic of \PPAs is the type of AI used.
Many solutions are based on machine learning models, such as supervised ML (classification), non-supervised ML (clustering), and reinforcement learning, sometimes combined.
It is, however, also possible to find older AI techniques grouped under the umbrella of expert or rule-based systems.

%Amongst the variety of models and techniques described in the literature , we 
%analyzed 
We also classified the different AI technologies used by the \PPAs reviewed regarding their explainability, or their inherent transparency.
%of each method.
However, XAI is only explicitly addressed by one work~\cite{lobner_explainable_2021}; the other models are therefore categorized based on~\citet{arrieta_explainable_2019}'s taxonomy.
We annotated T for Transparent in Table~\ref{tab:classification}, NIT for Not-Inherently Transparent, and PT for Partially Transparent when the solution relies on models with different levels of transparency.

\subsubsection{Transparent}
\paragraph{Classification}
Supervised machine learning, also called classification models, is a common set of techniques deployed in \PPAs.
In this context, a model is trained to classify an object of decision into a choice tailored to the users' desires.

Transparent classification models~\cite{arrieta_explainable_2019} (used in 7 papers) are composed of decision trees (used for instance in~\citet{bahirat_data-driven_2018}), k-nearest neighbors (leveraged in ~\citet{herrero_automatic_2021}), and Bayesian models (see~\citet{olejnik_smarper_2017}).

\paragraph{Clustering}
\label{subsub:clustering}
Several works use clustering techniques for their \PPA.
In this context, clustering is classically used to create a set of \textit{privacy profiles}, i.e., an archetypal ensemble of default parameters (for preferences or permissions) to which a user is then assigned.
Clustering algorithms (leveraged in 6 papers) used are hierarchical clustering (\citet{liu_follow_2016}), k-means (\citet{brandao_prediction_2022}), k-modes (\citet{shanmugarasa_automated_2022}), although several papers did not disclose the exact method used (\citet{hirschprung_simplifying_2015} for instance).

\paragraph{Rule-based}
\PPAs can be powered by non-machine-learning algorithms, based instead on rules (\citet{albertini_privacy_2016} implement association rules).
This comprises theoretical as well as practical works, with two of them out of four providing a tool (\citet{albertini_privacy_2016} and \citet{serramia_predicting_2023}).

\subsubsection{Not-inherently Transparent}
\paragraph{Classification}
Non-transparent classification models (found in 13 papers) typically encompass classic neural networks (as in~\citet{klingensmith_hypervisor-based_2019}) and deep neural networks  (see for instance~\citet{yu_leveraging_2018}); random forests (\cite{misra_pacman_2017}), Ada Boost~\cite{mendes_enhancing_2022} and Support Vector Machines (used in~\citet{wijesekera_contextualizing_2018}) complete the picture.
Post-hoc explanations must complement these models, as they are not easily understandable by themselves.

\paragraph{Reinforcement}
Reinforcement learning is the least used family of machine-learning techniques in \PPAs.
It is implemented in~\citet{barolli_reinforcement_2021} and~\citet{ulusoy_panola_2022}, both used to adapt users' feedback to their preferences, and in \citet{zhan_model_2022}.
The first paper uses it to disclose information (using permissions), while the second uses it to learn bidding preferences in a negotiation context.

\paragraph{Logic-based}
\PPAs can be based on logic (5 papers), for instance, expert systems (\citet{kokciyan_taking_2022} uses an agent-based model) or game theory (such as \citet{hirschprung_game_2022}).
These works, albeit few, span various system contexts and types of decisions.





\subsection{Source of Data}
\label{subsec:taxo_source}
% \todo[inline]{For all subsections, the relevance / potential role of the different data scources for PPAs/different types of PPAs should be discussed - why/how can be these data sources be useful?}
An \PPA can rely on various \textit{sources of data} when using AI to help with a privacy decision.
These data sources are very often combined, and a careful choice is necessary to fully exploit the potential of the models described in the previous section.


\subsubsection{Context}
\label{subsubsec:context}
\textit{Context} is an often-used data source, yet not always well-defined.
However, when it is defined, it is composed of the location~\cite{xie_location_2014}, the time, relationships with other individuals~\cite{fogues_sosharp_2017}, or the activity performed~\cite{alom_helping_2019}.

External data provided by third parties or other unrelated entities is sometimes used to predict privacy decisions, and this external data can arguably be considered context.
For instance, under this term, we find risk factors~\cite{ayci_uncertainty-aware_2023} or information related to other applications in the background~\cite{brandao_prediction_2022}.

Context is 
%arguably 
usually a crucial component for an effective \PPA because, as has been argued under theory of privacy as contextual integrity~\cite{nissenbaum_privacy_2004}, context is paramount to design appropriate information flows and to respect privacy norms.


\subsubsection{Attitudinal Data}
A few \PPAs ask users questions to elicit so-called \textit{attitudinal data} about stated practices or preferences regarding privacy recommendations to avoid the so-called cold-start problem, which arises when no past data is available to provide a recommendation.
%Take 
For example, ~\citet{camp_easing_2017} focuses on asking a minimal set of questions while keeping accuracy as high as possible, or \citet{alom_adapting_2019} asks ``a reasonable number of questions (50) to the users.''


\subsubsection{Behavioral Data}
% \todo[inline]{Weak subsection}
Another common source of data is \textit{behavioral data}.
% , which encompasses users' current settings or past decisions.
Behavioral data has the advantage of reflecting the \textit{actual} 
%settings 
privacy decisions of users to predict the next ones, as it does not simply rely on stated practices (unlike attitudinal data).
While it can be a powerful tool, it can also create a feedback loop, reinforcing the same decisions. 

% \paragraph{Current settings}
Behavioral data can encompass past decisions, such as in \citet{zhan_model_2022}, which leverage past choices to fill a knowledge base, then used them to predict privacy decisions.
It can also comprise current settings or preferences
% An \PPA can, for instance, learn from the current preferences 
on a specific type of data to infer a decision for another type~\cite{hirschprung_simplifying_2015}.
The system can also use these preferences to match users to a particular privacy profile, such as using clustering techniques (see Section~\ref{subsub:clustering}).

% \paragraph{Past decisions}


\subsubsection{Metadata}
\textit{Metadata} is data that provides information about other data, for example, the name of an application used~\cite{wijesekera_contextualizing_2018}, network requests~\cite{tan_context-perceptual_2018}, the purpose associated with processing~\cite{barbosa_what_2019}, the usage frequency of certain permissions (such as location) by an app~\cite{kaur_smart_2020}, or tags associated with images~\cite{squicciarini_privacy_2015}.
To some extent, metadata can overlap with context, for instance, when considering time or location. 
However, the articles surveyed more often refer to the time and location of collection \textit{of a certain data point} for metadata, and to the \textit{current time and location} when a decision has to be made for context.
Metadata can provide peripheral information to make decisions, although it is rarely used as a sole source of data (only 3 papers out of 13~\cite{tan_context-perceptual_2018,klingensmith_hypervisor-based_2019,barolli_selflearning_2020} rely only on metadata).

\subsubsection{Data Type}
The \textit{data type} refers to the category of data concerned by the decision, such as whether it is an image to share on social media~\cite{zhong_group-based_2017}, the location requested by an app~\cite{filipczuk_automated_2022}, or various sensor data by an IoT device~\cite{shanmugarasa_automated_2022}.
The type of data can provide accurate information about the sensitiveness of a decision (location data can, for instance, provide sensitive information regarding the users' context, e.g., from location data that reveals that a user visits a clinic or church,  medical, or religious information could be inferred),
%be seen as more sensitive than professional emails, and so regardless of its content)
yet only a relatively low number of solutions rely on the data type to build an \PPA~\cite{baarslag_automated_2017,zhong_group-based_2017,camp_easing_2017,wijesekera_contextualizing_2018,lobner_explainable_2021,filipczuk_automated_2022,shanmugarasa_automated_2022}.

\subsubsection{Content of Data}
The \textit{content of data} refers to the specific content of a data point, as the name indicates. However, we also include data that can be directly inferred from the content of data under this category.
%For instance,
For example, \citet{herrero_automatic_2021} and \citet{dong_ppm_2016} estimate the sensitivity of the content of the information to be shared to help make a decision.
Indeed, content can be leveraged to tailor decisions: a picture deemed private should not receive the same treatment as one deemed public, and a geolocation trace that may providentially allow inferring religious practice should be dealt with cautiously. 


% \subsubsection{Others}
% \todo[inline]{Discard this section, redo the classification}
% Finally, we regroup under \textit{others} all sources of data that could not be assigned under the categories above.
% We can find~\citet{barolli_selflearning_2020}, which uses the availability of information (whether it is public or not) to make a decision,~\citet{hirschprung_game_2022} leverage privacy loss (i.e. actions or events detrimental to one's privacy),~\citet{yu_leveraging_2018} the trustworthiness of the contact concerned by the privacy decision, or~\citet{squicciarini_privacy_2015} recent major changes among the user’s community about their privacy practices.







\subsection{System Context}
\label{subsec:taxo_system_context}
Most \PPAs target a specific \textit{system context}, that is, a set of technologies with distinct characteristics.
Indeed, each system context has specific requirements that one must consider when designing an \PPA.
System contexts differ by the availability of an \textbf{interface}, \textbf{computational power}, and control over the \textbf{architecture}.
% , as we will discuss in this section.

\subsubsection{Mobile Apps}
Several works focus on mobile applications, and often on Android~\cite{apolinarski_automating_2015,wijesekera_contextualizing_2018}.
Mobile ecosystems have the advantage of being well-defined ecosystems, enabling the possibility to strictly enforce privacy decisions (i.e., it is often addressed with permissions, see Section~\ref{subsub:permissions}).

Mobile phones also possess reasonable computational power (in the sense that they can run an \PPA) and a screen enabling direct user interactions.
Hence, an \PPA can be implemented directly on a smartphone (see \citet{baarslag_automated_2017}), and it can interact with and even regulate mobile apps, all of which make mobile ecosystems suitable candidates for \PPAs under the users' control.

\subsubsection{IoT}
Another widely used system context for \PPAs is the Internet of Things (IoT).
We understand IoT as a network of devices, including sensors, mechanical and digital machines, as well as consumer devices, all connected to the Internet. 
% IoT devices are typically embedded with technology such as sensors and software, including mechanical and digital machines and consumer objects.
% \footnote{\url{https://www.techtarget.com/iotagenda/definition/Internet-of-Things-IoT}}
In practice, \PPAs have been developed for smart homes~\cite{shanmugarasa_automated_2022, barbosa_what_2019}, on campuses~\cite{das_personalized_2018}, or for wearables such as fitness devices~\cite{sanchez_recommendation_2020} for instance.

Most IoT devices are usually not equipped with proper interfaces and lack computational power.
These characteristics make it challenging to build \PPAs %enforcing 
assisting with permission settings, yet not impossible (see~\citet{klingensmith_hypervisor-based_2019} for instance, who manage to do so with an \PPA located on end devices).

\subsubsection{Social Media}
According to our classification of the literature, the third major system context is social media, for which several \PPAs have been designed to help make privacy decisions.
In this case, neither the interface nor the computational power are usually limiting factors.
However, the design and implementation of social media platforms (that are usually not published openly) make it difficult to assess the binding character of privacy decisions supported by \PPAs running on social media platforms.
\PPA solutions are rather designed to support data sharing, i.e., whether a specific post should be shared on social media and with whom, than focusing on assisting users with privacy decision-making.

\subsubsection{Cloud}
% \todo[inline]{Update and elaborate}
%Finally, let us conclude this section by mentioning that other 
Another system context is cloud environments, even though only one of the reviewed articles proposes an \PPA for the cloud~\cite{hirschprung_simplifying_2015}.
Their solution offers a method to simplify information disclosure in cloud environments such as Google Drive.
However, this work is thus a lone example and contrasts with the otherwise balanced distribution of works among other system contexts.


\subsection{Architecture}
\label{subsec:taxo_architecture}
By architecture, we refer here to where the computation happens, i.e., the decision-making, and not necessarily the pre-processing steps such as building privacy profiles.
Directly connected to the architecture is the trust model of the \PPA.
While this term is usually reserved for security-oriented research, describing whether one has to trust the different entities or not provides relevant information for understanding the privacy boundaries.

Note that the location of the computation is only relevant for implemented \PPAs, and not for theoretical models. 
Similarly, most solutions surveyed do not explicitly describe a threat or trust model in their paper. 
Nonetheless, it is possible to infer that trusted parties are required in some solutions.
For instance, \citet{tan_context-perceptual_2018} describes an architecture comprising a remote classifier, in which one has to place trust, yet no trust model is described.

% \todo[inline]{add reference}

\subsubsection{Local Computation}
% \paragraph{User device}
The processing can happen locally on the user device, such as on a smartphone (see, e.g., ~\cite{olejnik_smarper_2017}), but this device can also be a home pod in an IoT context (see, e.g., ~\cite{shanmugarasa_automated_2022}). 

Creating and processing user profiles, using local AI models, and locally deriving privacy decisions have the advantage that the user can keep control over the locally processed data, including their profiles and AI models, which usually can include sensitive information about the user's preferences or behavior. 
However, local data processing also puts more responsibilities on the user to secure the devices properly against malware or other attacks. 


\subsubsection{Remote Computation}
%If the computation is not performed locally, it is then done remotely, 
The \PPA could also be based on remote (according to the user's point of view) data processing, involving a central server that 
%receives 
processes personal privacy decisions,  contextual data including, e.g., location data or another type of data. 
Remote computation raises the question of the trust placed in the 
%third 
party performing this computation to protect the data properly, to enforce the data subject's rights (e.g., to access or to delete their data and computed profiles or models), and not to use the data for any unintended purposes~\cite{tan_context-perceptual_2018}.

% \paragraph{Trusted 3rd party}
Several solutions rely on a remote third party that has to be trusted, e.g.,~\citet{baarslag_automated_2017} or~\citet{tan_context-perceptual_2018}'s solution that places trust on their own remote classifier. 
In contrast, others only require trusting the operating system (OS) on which the \PPA is implemented~\cite{olejnik_smarper_2017}, or require trusting both the OS and mobile applications~\cite{apolinarski_automating_2015}.

\subsubsection{Federated Learning}
% \paragraph{Federated learning}
Only one article, by \citet{brandao_prediction_2022}, presented an \PPA based on federated learning.
In this work, the processing of user data for the computation of locally trained neural network models happens on the user devices that share only the neural network weights with a central server, which will, in turn, average all the local weights and send back the results to the clients, which can use these new weights to continue the training process.
%while locally trained models of different users are then further used as input for further training a global model remotely with improved quality. 
Federated learning is a privacy-enhancing approach for processing the users' raw data only locally, which can achieve a performance comparable to the centralized approach (remote computation). 
Nonetheless, federated learning could still be attacked, e.g., with membership inference attacks for leaking personal data from locally trained models~\cite{shokri2017membership}.
%also on other devices (remotely) but with mathematical guarantees.

% \paragraph{Untrusted 3rd party}
% \paragraph{3rd party with unknown trust}





\subsection{Empirical Assessment}
\label{subsec:taxo_validation}
% \todo[inline]{Evaluation and validation can span over any of the means of assessment}
\PPAs' performance can be measured in terms of accuracy, but because several solutions are meant to be usable tools, assessing an \PPA encompasses more than a mere measurement of how well a privacy decision is predicted.

As mentioned in Section~\ref{subsec:analysis}, an empirical assessment can be an evaluation (see e.g.~\cite{liu_follow_2016}) or a validation (e.g.,~\cite{hirschprung_game_2022}). 

\subsubsection{User Study}
A classical way to validate a tool or a method is to conduct a user study, and we found 16 papers reporting a user study to validate usability.
A user study can have various interpretations, ranging from a simple questionnaire to rate satisfaction (such as~\citet{alom_helping_2019}) to a large-scale randomized controlled study (see for instance~\citet{liu_follow_2016}) -- the former being more akin to a mere validation, the latter a full-fledged evaluation.

Note that several works elicited data to build a dataset through a user study, which was therefore not meant as a means of assessment (annotated as $\alpha$ ~in Table~\ref{tab:appendix_table}).

\subsubsection{Purely Statistical (Dataset)}
Several works provide a validation without a user study, that is, only based on a purely statistical analysis based on a dataset (such as \citet{barolli_selflearning_2020}, \citet{herrero_automatic_2021}, \citet{bahirat_data-driven_2018}, \citet{kokciyan_turp_2020}, \citet{ayci_uncertainty-aware_2023}, \citet{zhan_model_2022}, \citet{kokciyan_taking_2022}, \citet{herrero_automatic_2021}).
Such a measure, although potentially subject to a higher degree of statistical rigor, cannot necessarily capture users' expectations and may even fall into the pitfall of Goodhart's law.\footnote{According to which ``When a measure becomes a target, it ceases to be a good measure.''}


\subsubsection{Accuracy Measurement}
%At first sight, a
Accuracy 
%seems to 
can measure the capacity of an \PPA to predict a privacy decision, but not all papers measure the same type of accuracy.
% or accuracy parameters.
\citet{tan_context-perceptual_2018} measure a privacy leakage detection, \citet{herrero_automatic_2021} whether the correct category of data is predicted or not, \citet{serramia_predicting_2023} the acceptability rate, \citet{barbosa_what_2019} the Area Under the Curve (AUC) of a binary allow/deny for a given scenario, etc.

Other works, while they do measure the accuracy of their solution to predict a privacy decision, present their work with %questionable 
limited rigor or precision.
For example, \citet{fogues_sosharp_2017} only present their results in plots.
In contrast, others, such as~\citet{olejnik_smarper_2017}, dedicate an entire subsection to explaining accuracy measurements.





\subsection{User Control Over Decisions}
\label{subsec:taxo_req}
% \todo[inline]{Semi-automated is problematic as is, let's discuss it.}
% \todo[inline]{For User control over decisions, we distinguish between qualities of control (solid arrows) and instruments of control (dashed arrows).}

Finally, \PPAs should not only assist users with making privacy decisions, but should at the same time also empower users with various options to improve \textit{control over their decisions}.
These options span over \textbf{qualities} of control (solid arrows in Figure~\ref{fig:classification}) and \textbf{instruments} of control (dashed arrows).
The former denotes adjectives that can be appended to control over a decision (akin to non-functional requirements in software engineering~\cite{pallas_privacy_2024}), and the latter denotes concrete possibilities or actions for users (similar to functional requirements).

These options are 
%somehow connected 
partly related to 
%European 
GDPR requirements for consent (introduced in Section~\ref{subsec:legal}), which are thus relevant for privacy decisions that constitute consent.
Note, however, that only a handful of papers specifically refer to legal considerations.
\citet{filipczuk_automated_2022} refer to the GDPR, \citet{mendes_enhancing_2022} acknowledge that an automated response to a permission request might not constitute legal consent, \citet{lobner_explainable_2021} base the rationale of explainability on legal requirements, and \citet{sanchez_recommendation_2020} even claim GDPR compliance.
Nonetheless, decisions for setting permissions for mobile operating systems, for instance, still require consent at installation or run time. Thus, legal requirements for consent remain relevant for these types of decisions.  

% ----------------------------------------------------
% We described in Section~\ref{subsec:legal} the legal considerations for consent under the GDPR.
% Although having a \PPA managing consent on behalf of its user has only been mentioned prospectively in the literature~\cite{colnago_informing_2020}, we argue that these considerations provide privacy-preserving guidelines for the design of \PPAs.
% We therefore examine in this section these considerations and how they should be technically interpreted for a \PPA.
% These considerations do not exhaustively encompass requirements specified in Articles 4(11), 7, and 9 of the GDPR and their interpretation in the EDPB's revised Guidelines on Consent. Instead, they reflect the current state of affairs identified in the literature.


\subsubsection{Ex-ante Transparency}
%{Informed}
Under Art 13 GDPR, data subjects should receive information if data are collected from them, 
%In an EU context, 
%personal data collection is only lawful if data subjects have been informed beforehand; 
and informing users is also an integral part of the dominant transparency paradigm in the US (the \textit{notice} of the notice and choice approach).
Informing data subjects with intelligible notices arguably improves their control over decisions.
Several \PPAs only inform about the type of data concerned by the privacy decision~\cite{albertini_privacy_2016}, some inform in addition about the controller~\cite{wieringa_requirements_2006} or of the purpose of processing~\cite{bahirat_data-driven_2018}.
%
Meeting this criterion should not be difficult in theory, although providing intelligible notices requires significant expertise in practice (as illustrated in~\cite{schaub_design_2015}).

\subsubsection{Semi-automated}
% \todo[inline]{I am not sure this subsection fits anymore: the data I collected was whether the solution surveyed provided an unambiguous decision, it is not the same to have a semi-automated solution.}
% While it may not be legal nor desirable for certain decisions to be fully automated, a certain degree of automation can be beneficial to users to reduce cognitive burden~\cite{morel_automating_2023}.
% However, certain solutions do not provide the necessary safeguards and fully automate

% Decisions can be

The semi-automated character of a decision refers to the inclusion of an affirmative action of the user to 
%conclude 
confirm the decision, 
%it is defined against a 
which is therefore not fully automated~\cite{morel_automating_2023}. 
%(such as develop by~\citet{mendes_enhancing_2022}) in which users would have to opt out ex-ante.
Most solutions provide a semi-automated decision process, although not systematically (e.g., \citet{das_personalized_2018} mention that only opt-out is possible for facial recognition), or not always (e.g., \citet{klingensmith_hypervisor-based_2019} offers different types of ``privacy profiles'', one of which -- \textit{Laissez-Faire} -- enables full automation).
\citet{tan_context-perceptual_2018} do not leave users in the loop by default, but the system allows the possibility to change the settings for ``experienced users,'' while it depends on the Configuration Option for~\citet{hirschprung_simplifying_2015}.



\subsubsection{Specific}
The specificity of a decision refers to the presentation and the possibility for users to decide on the granularity of 
%for each set of 
each data type, purpose and controller separately.
For an \PPA, it means having a fine-grained selection process, during which users should not be presented with bundled decisions.
For instance, \citet{shanmugarasa_automated_2022}'s solution works per ``situational context:'' who (is requesting data), data type, purpose, and resharability (to third parties); while~\citet{xie_location_2014} only works for one type of data (location), therefore only meeting this option in a restricted sense. 


% \subsubsection{Freely given}
% The freely given character of consent entails different considerations.
% First, the user should not be misled into consenting.
% Second, they should always have another option possible (consent must not be the only way).
% Third, they should be able to refuse and withdraw their consent as easily as it was given.

% This combination has been seldom observed in the literature, with only~\citet{liu_follow_2016} and~\citet{wijesekera_contextualizing_2018} letting users withdraw their decisions through their \PPAs.

% \subsubsection{No prior collection}
% We also analysed whether

% Consent can only be valid if data collection is conditional upon it, that is, whether no personal data is collected on this ground before its elicitation.
% Some \PPAs entirely automate the decision-making, leaving users only the possibility to reflect on the choices made ex-post.
% Although a majority of solutions to which this criterion can apply (i.e., with an implementation) integrate this feature, such as \citet{herrero_automatic_2021}, who suggest recommendations while no data is collected before a decision has been made.

% \subsubsection{Explicit}
% % \todo[inline]{we can notably mention that explicit consent is only "triggered" with special categories of data, not because of transfer to non-adequate country}
% Some decisions entail sensitive data~\footnote{Note that sensitive data is also a legal notion defined in the GDPR under the term ``special categories of data. If special categories of data are collected based with consent as a legal basis, then the consent has to be "explicit".
% %require eliciting explicit consent.''
% }.
% %, or high-risk transactions, generally speaking.
% Similarly, there can be decisions that could be classified as "high-risk" beyond privacy risks, as they could for instance impact the users' safety.
% These decisions can be dealt with an \textit{explicit} action free of ambiguity from the user to further protect the sharing of such sensitive data.
% For instance,~\citet{wijesekera_contextualizing_2018} deal separately with ``sensitive resources,'' and \citet{filipczuk_automated_2022} discuss the sensitive character of certain types of data.

% In practice, a \PPA should account differently for these special categories of data.

% \subsection{Reject}
% \todo[inline]{V: after more consideration, I suggest to make it a brand new facet (it doesn't belong under enforcement of decision, and neither Refuse and Revoke under Elicitation of decision. Let me know what you think of it.}
% Finally, we distinguish between the aforementioned positive privacy decisions taken by PPAs and \textit{reject} decisions, imbued with a negative power to opt out of or deny data collection.
% We followed~\cite {morel_automating_2023}'s classification, using their insights on the lawfulness of automating privacy decisions: while developers cannot fully automate positive privacy decisions under the GDPR, negative ones, such as reject decisions, can be set automatically as the default.

% \subsubsection{Refuse}
% % \todo[inline]{meta decision here}
% \PPAs can offer users the possibility to refuse their recommendations to keep them in the loop.\footnote{This option only applies to solutions offering recommendations -- \PPAs not considering their users in the loop, or without implementation, are not considered here.}
% As a matter of fact, not all \PPAs explicitly offer this option; certain solutions seemingly only inform users of the decision made without the possibility of objecting to it (to the extent of our understanding).
% For instance, however, the \PPA designed by \citet{wijesekera_contextualizing_2018} offers the possibility to deny permissions to applications.
% Note that not offering the possibility to refuse a recommendation generates a significant loss of agency on the user's side.


% Although connected to the freely given requirement, this consideration ought to have its dedicated section as it implies something specific for \PPAs.
% Indeed, not all \PPAs enable a negative definition of choices, where a user could define privacy decisions to refuse, object, or opt-out of processing, and so with some degree of automation (manual objection being often possible).

\subsubsection{Revoke}
% \todo[inline]{meta decision here}
Finally, we observed that some \PPAs enable users to withdraw decisions.
Here, rather than denying a decision or a recommendation, revoking operates after a given decision to withdraw it.
This feature has rarely been observed in practice -- at least explicitly -- although \citet{liu_follow_2016} allows revoking previously granted decisions.
Revoking previously made decisions, such as sharing data on social media, can be challenging to enforce.
Also, note that certain operating systems -- such as mobile OSes -- will still allow users to revoke their decisions manually, although we stress that this action is performed outside the \PPA.

% \newpage


\section{Discussion}
\label{sec:discussion}
This systematic survey provides unique insights into how state-of-the-art research has designed \PPAs during the last decade.
For instance, IoT became a system context of interest only in 2018, and we observe a similar late adoption trend for reinforcement learning after 2021.\footnote{Some papers may have been published on the topic earlier than in 2013, the year from which we started to include papers in our survey.}
However, AI techniques have been used in every system context for all types of decisions throughout the years without any apparent pattern.
While this lack of a clear pattern is not the most informative in itself, we ought to look instead at the \textbf{gaps} this survey highlights, the \textbf{challenges} \PPAs raise, then to inform better \textbf{design and development recommendations} based on these analyses.

% the areas where data is lacking.
% Indeed, we can, for instance, identify several \textbf{gaps} from the missing system contexts and posit \textbf{challenges} based on the messiness of some data points, inform better \textbf{design recommendations} based on these analyses, to finally point out avenues for \textbf{future research}.
This section discusses the issue of \textit{Evaluating \PPAs} in Section~\ref{subsec:disc_eval}, \PPAs not sufficiently addressing \textit{Privacy-by-Design} in Section~\ref{subsec:disc_pbd}, the (lack of) \textit{explanations and explainability} in Section~\ref{subsec:disc_xai}, the relationship with \textit{legal considerations} in Section~\ref{subsec:disc_leg}, the concerns surrounding \textit{system contexts} in Section~\ref{subsec:disc_contexts}, the challenges in leveraging different \textit{sources of data} in Section~\ref{subsec:disc_sources}, to finally pave \textit{research avenues} in Section~\ref{subsec:future}.

% (Section~\ref{subsec:future}).


% \todo[inline]{Opening on some general remarks, IoT came later, reinforcement learning as well.}

% \subsection{Evolution of the Field}
% \todo[inline]{My conclusion is that every techniques have been used in every system contexts, for all kinds of decision, throughout the years, without any clear pattern. Maybe it tells something as well?}

% \todo[inline]{TBW later}
% \subsubsection{trends on data sharing}
% \todo[inline]{After having looked at the data, I don't see any interesting conclusion to draw: there was papers published almost every year with each type of decision ...}

% \subsubsection{AI trends}
% \todo[inline]{analyse AI techniques in combination with context, years, type of decision etc}
% \todo[inline]{Nothing is significant, every seems either well distributed or without clear patterns (I looked at the specific type of AI used, not just the high-level classication)}
% We observe that in 2017-2018-2019, all papers (15) except two were exclusively relying on classification.
% But there doesn't seem to be any pattern in the type of classification algorithm used.

% The type of AI appears to be well distributed amongst the different system contexts, without clear patterns.

% The type of AI appears to be well distributed amongst the different types of decision, without clear patterns.



% \subsection{Gaps \& Challenges}
% \label{subsec:gaps_challenges}
% The results of the survey helped us identify several gaps and challenges within the field.
% We provide in this section the main common shortcomings and propose improvements for the design of future \PPAs.

\subsection{Evaluating \PPAs}
\label{subsec:disc_eval}
The problem of the evaluation of \PPAs is two-fold.
First, we observe that \textbf{the evaluation of \PPAs are not based on the same or on comparable accuracy metrics or measurements}.
%does not always consist of the same object when they talk of accuracy}.
Indeed, as presented in Section~\ref{subsec:taxo_validation}, accuracy is measured regarding the privacy decision, but also regarding privacy leakage, acceptability rate, etc.
Second, \textbf{our data shows a lack of user study evaluation}, and our critical appraisal shows a trend toward ``low'' or ``very low'' scores for these user studies.
Only 16 out of 39 papers mentioned that they performed a user study to evaluate their solution\footnote{Some papers include a user study for collecting data, which is not the focus of the present argument.}, only four of which are above (or equal to) 70\% based on the CEBMA critical appraisal we performed.
We acknowledge that such user studies might not be in the scope of theoretical papers (e.g., models or frameworks without prototype implementation). 
Yet, we contend that the validation offered by these theoretical papers, often cross-validation on a dataset, is far from being able to reflect reality.\\

%In a nutshell,
% We, therefore, propose that \textbf{the usability of an \PPA should be validated with a user study following the highest standards of practice, and such evaluation should notably encompass the accuracy of the privacy decision taken}.

\noindent\fbox{
    \begin{minipage}{.955\linewidth}
        \setlength\parskip{1em}
        \textit{\textbf{Recommendation:} Based on the current lack of empirical studies, we propose that the usability of an \PPA should be validated with a user study following a high standard of practice, and such evaluation should notably encompass the accuracy of the privacy decision taken.}
    \end{minipage}
}

\subsection{Lack of Privacy by Design}
\label{subsec:disc_pbd}
%{Privacy in option}
We identified a gap regarding following a privacy-by-design approach for \PPAs since hardly any of the papers we surveyed focus on or mention how the \PPAs themselves can be designed in a privacy-preserving manner.
%in privacy by design amongst the papers surveyed.
More specifically, among papers describing a technical architecture\footnote{Recall that theoretical papers are excluded from this analysis.}, \textbf{only one uses federated learning as a privacy-enhancing approach,\footnote{However, without discussing that federated learning is still vulnerable to privacy attacks.} and many require trust in a 
%3rd party} 
central server} where the \PPAs' data processing is performed.
%computation for the privacy decision is outsourced.
However, \citet{wijesekera_contextualizing_2018} provides an insightful analysis of the trade-off of having either a local (offline) or a remote computation, concluding that offline learning still performs well (almost 95\% accuracy).
Also note that the privacy threat models are rarely described, making it difficult to evaluate security and privacy assumptions critically.\\

% Our analysis yielded an obvious lack of privacy by design in \PPAs, with only one technical architecture leveraging Federated Learning but a majority (14 papers) externalizing computation remotely.
% Remote computation on a central server is not the prime choice for a privacy-friendly architecture, 
% although  privacy can to some degree be protected if the server on which the computation is performed can be trusted.
% However, the notion of trust (negatively as positively) has seldom been mentioned in our pool of papers.

\noindent\fbox{
    \begin{minipage}{.955\linewidth}
        \setlength\parskip{1em}
        \textit{\textbf{Recommendation:} We contend that \PPAs must embrace stronger privacy-by-design principles, including better design strategies but also better integration of Privacy Enhancing Technologies for achieving data minimization, such as federated learning
        %~\cite{european_network_and_information_security_agency_privacy_2014} 
        combined with differential privacy, or the use of privacy-preserving data analytics based on multi-party computation
        %~\cite{goldreich_secure_1998}, 
        homomorphic or functionally encrypted data (see also ~\cite{PAPAYA2019}).}
    %~\cite{acar_survey_2018}
    % (see~\cite{PAPAYA2019}).}
    \end{minipage}
}

% Therefore, we contend that \textbf{\PPAs must embrace stronger privacy-by-design principles}, including better design strategies but also better integration of Privacy Enhancing Technologies for achieving data minimization, such as federated learning~\cite{european_network_and_information_security_agency_privacy_2014} combined with differential privacy, or the use of privacy-preserving data analytics based on multi-party computation~\cite{goldreich_secure_1998}, homomorphic or functionally encrypted data~\cite{acar_survey_2018}.
%and the application of common data protection principles such as data minimization.

\subsection{Unexplainable AI}
\label{subsec:disc_xai}
Another pitfall identified is the lack of explanations provided by most \PPAs, combined with the lack of explainability offered by the AI models used.
\textbf{Only one of the surveyed papers explicitly addresses explainability of decisions}~\cite{lobner_explainable_2021}, and only 16 use a transparent model (see Section~\ref{subsec:taxo_ai}) to make a prediction.

The growing trend to use deep learning architectures may not facilitate the explainability of decisions, but this challenge is not insurmountable.
It is indeed possible to devise \textit{post hoc} explanations and take inspiration from other existing work on usable explanations for AI-made decisions.
Note, however, that inherent transparency can come at the expense of the quality of decisions (deep neural networks tend to perform better than their simpler counterparts, although this statement does not seem to generalize to all kinds of decisions, such as decisions made in highly unpredictable settings like social predictions~\cite{narayanan_ai_2024}).

As we discuss above in Section~\ref{legal_background}, transparency of AI can, in some specific use cases related to \PPAs, be a legal requirement for the data controller according to the GDPR, or for the provider or deployer according to the AI Act, even though this will not apply for the majority of \PPAs and use cases.\\
%According to AI act, means for transparency is especially also required as a technical feature from the system provider and deployer of high-risk AI systems (see also section ~\ref{}). While the majority of \PPAs will most likely not classify as high-risk according to the AI act's definition, still \PPAs that assist with privacy decisions that may for instance also has safety implications for the users or others may classify as high-risk.

\noindent\fbox{
    \begin{minipage}{.955\linewidth}
        \setlength\parskip{1em}
        \textit{\textbf{Recommendation:} Based on this analysis, we recommend 1) the use of inherently explainable AI models such as decision trees for the classification, whenever possible in terms of potential acceptable quality loss implications, or 2) the integration of ad-hoc explanations otherwise, e.g., for neural networks and SVMs.\\
        % Indeed, based on the newly introduced AI Act in the EU, we foresee a generalization of explainability requirements for AI systems, which include \PPAs.
        Indeed, explainability can also foster trust in technology, improving the uptake of such systems.}
    \end{minipage}
}

% \todo[inline]{Mention tradeoff explainability/quality of decisions}
% Based on this analysis, \textbf{we recommend both 1) the use of intrinsically explainable AI models such as decision trees for the classification whenever possible, and 2) the integration of ad-hoc explanations otherwise} (e.g., for neural networks and SVMs).
% Indeed, based on the newly introduced AI Act in the EU, we foresee a generalization of explainability requirements for AI systems, which include \PPAs.
% Aside from legal requirements, explainability can also foster trust in technology, improving the uptake of such systems.

\subsection{Few Legal Considerations}
\label{subsec:disc_leg}
As some privacy decisions made by \PPAs have
%directly concerned with legal issues 
legal privacy implications or issues, 
%(such as the (semi-)automation of consent, e.g. when setting permissions, or
%, although not explicitly considered in the papers surveyed), 
%and decisions made by AI systems are becoming more and more subject to regulations (see the AI Act in the EU), 
%the positioning of \PPAs towards legal concerns is becoming critical.
legal requirements, e.g., under the GDPR or the AI Act, should be discussed and considered for the design and use of \PPAs. 
Although partially explainable by 1) the geographic distribution of the solutions surveyed (only 12 papers have EU affiliations, in contrast with both the GDPR and the AI Act being EU regulations), and by 2) the timing of publications (17 papers were published before the GDPR was enacted, and none before the AI Act came into force), it can still be surprising to find \textbf{only 4 papers mentioning (but not even addressing) legal considerations}.\\

%This lack of focus can also be exemplified in data extracted from \textit{user control over decisions}, initially guided by an analysis of EU consent requirements (see Section~\ref{subsec:analysis}), from which we can derive that very few solutions can claim to address legal compliance.

\noindent\fbox{
    \begin{minipage}{.955\linewidth}
        \setlength\parskip{1em}
        \textit{\textbf{Recommendation:} 
        % Generally speaking, and in line with our previous arguments, w
        We recommend a deeper consideration of legal requirements 
        %across the board and thorough addressing when required.
        for the design of \PPAs.
        %In the latter case, it amounts 
        Such efforts should particularly amount to 1) meeting consent requirements when assisting on decisions related to consent, such as permission settings;
        % , which are related to giving consent; 
        %for solutions considering this legal basis} for data collection and processing, which may not always be the case and not even always clearly delineated (see Section~\ref{subsec:privacy_decisions}); 
        2) the introduction of \PPAs assisting and enabling users to exercise their data subject rights;
        %, for instance, for objecting to direct marketing or facilitating the right of access to data;
        %~\cite{leschke_how_2024}; 
        and 3) the incorporation of usable explanations for the logic behind the AI-based proposed decisions, and an assessment of risks and consequences addressing related requirements by the GDPR for automated decision-making or the AI Act. 
        %based on the AI Act requirements (as mentioned before).
        }
    \end{minipage}
}\\

% Generally speaking, and in line with our previous arguments, \textbf{we call for deeper consideration of legal requirements across the board} and thorough addressing when required.
% In the latter case, it amounts to 1) \textbf{meeting consent requirements for solutions considering this legal basis} for data collection and processing, which may not always be the case and not even always clearly delineated (see Section~\ref{subsec:privacy_decisions}); 2) \textbf{the introduction of \PPAs enabling the exercise of data subject rights}, for instance objecting to direct marketing, or facilitating access to data~\cite{leschke_how_2024}; and 3) \textbf{the incorporation of usable explanations and an assessment of risks}, based on the AI Act requirements (as mentioned before).

In the more general case, we contend that even when 
%legal requirements are not obligatory, 
legal privacy principles in a certain use case or context do not apply, \textbf{they can still provide valuable guidelines for the design of \PPAs}.
For instance, assisting users with making informed, unambiguous, and explicit privacy decisions (as required for consent) may foster trust in \PPAs even when the privacy decision to be made does not formally constitute consent.
%consent is not their legal basis.
Also, usable explanations of the risks and implications 
%associated with using an AI system 
when using an \PPA can in general help raise awareness among users.


\subsection{Missing System Context}
\label{subsec:disc_contexts}
% \todo[inline]{V: expand}
Our SLR covered four system contexts: mobile applications, social media, IoT, and the cloud.
We are, however, surprised \textbf{not to find other contexts, such as web browsers or Trigger-Action Platforms (TAPs)}.
The former because cookie notices are notoriously a ``hassle'' for users in modern web experience; we therefore expected to encounter solutions tackling this issue.
The latter refers to platforms offering applications for connecting otherwise unconnected devices and services using simple recipes, such as ``Every morning at 7 am, send a Slack message with the first meeting of the day from Google Calendar.'' 
Trigger-action programming has gained a lot of traction in the last years (IFTTT, the most important TAP, boasts over 27 million users~\cite{ifttt_ifttt_nodate}), yet no \PPAs specifically addressed this environment.

Both these system contexts possess their specific features: many controllers with non-standard interfaces for cookie notices, and numerous actors mediated through a single centralizing entity for TAPs.
They therefore require targeted efforts from designers to offer adequate technological solutions to manage privacy decisions.\\

\noindent\fbox{
    \begin{minipage}{.955\linewidth}
        \setlength\parskip{1em}
        \textit{\textbf{Recommendation:} We encourage researchers and developers of \PPAs to expand their efforts into a broader range of system contexts, encompassing also but not limited to web browsers and TAPs.}
    \end{minipage}
}


\subsection{Use of Varied Sources of Data, Accounting for Both Context and Personal Preferences}
\label{subsec:disc_sources}
Last, our study yielded that \PPAs leverage various sources of data (context, attitudinal data, behavioral data, type of data, content of data, and metadata), but not necessarily within the same solution.
However, utilizing several of these data sources can be a challenge in itself, as it requires careful curation of the datasets and adequate use of the AI models.
The benefits harnessed can be high, resulting in higher prediction accuracy.

We also acknowledge the difficulty of determining certain sources of data -- such as context --, or the sensitivity of data.
As mentioned in Section~\ref{subsubsec:context}, context is rarely defined.
It is, however, possible to draw inspiration for a rigorous definition from the seminal paper by \citet{barth_privacy_2006} on the formalization in a logical framework of the concept of contextual integrity coined by \citet{nissenbaum_privacy_2004}.
As for the sensitivity of data, it is notably incumbent on context when, for instance, the same type of data (e.g., location) can be deemed non-sensitive in one context (e.g., at a workplace in the middle of the week), but sensitive in another (e.g., Sunday morning near a church, thereby disclosing potential religious beliefs).\\

\noindent\fbox{
    \begin{minipage}{.955\linewidth}
        \setlength\parskip{1em}
        \textit{\textbf{Recommendation:} Based on the relative singularity of data sources, we advocate for a plurality of data sources, encompassing context as much as personal preferences.}
    \end{minipage}
}

% \subsection{Design recommendations}
% \label{subsec:design}
% Based on the aforementioned gaps and challenges, we echo in this section recommendations for the design of \PPAs.

% \subsubsection{Implementing privacy by design}
% Our analysis yielded an obvious lack of privacy by design in \PPAs, with only one technical architecture leveraging Federated Learning but a majority (14 papers) externalizing computation remotely.
% Remote computation on a central server is not the prime choice for a privacy-friendly architecture, 
% although  privacy can to some degree be protected if the server on which the computation is performed can be trusted.
% However, the notion of trust (negatively as positively) has seldom been mentioned in our pool of papers.

% Therefore, we contend that \textbf{\PPAs must embrace stronger privacy by design requirements}, including better design strategies but also better integration of Privacy Enhancing Technologies for achieving data minimisation, such as federated learning~\cite{european_network_and_information_security_agency_privacy_2014} combined with differential privacy, or the use of privacy-preserving data analytics based on multi-party computation, homomorphic or functionally encrypted data.
% %and the application of common data protection principles such as data minimization.

% \subsubsection{Clearer explanations}
% \todo[inline]{Mention tradeoff explainability/quality of decisions}
% Stemming from the lack of explainability of the decisions made by \PPAs, \textbf{we recommend both 1) the use of intrinsically explainable AI models such as decisions trees for the classification whenever possible, and 2) the integration of ad-hoc explanations otherwise} (e.g. for neural networks and SVMs).
% Indeed, based on the newly introduced AI Act in the EU, we foresee a generalization of explainability requirements for AI systems, which include \PPAs.
% Aside from legal requirements, explainability can also foster trust in technology, improving the uptake of such systems.

% \subsubsection{Consideration of legal requirements if needed}
% Generally speaking, and in line with our previous arguments, \textbf{we call for a consideration of legal requirements across the board}, and a thorough addressing when required.
% In the latter case, it amounts to 1) \textbf{meeting consent requirements for solutions considering this legal basis} for data collection and processing, which may not always be the case and not even always clearly delineated (see Section~\ref{subsec:privacy_decisions}); 2) \textbf{the introduction of \PPAs enabling the exercise of data subject rights}, for instance objecting to direct marketing, or facilitating access to data~\cite{leschke_how_2024}; and 3) \textbf{the incorporation of usable explanations and of an assessment of risks}, based on the AI Act requirements (as mentioned before).

% In the more general case, we contend that even in cases when legal requirements are not obligatory, \textbf{they can still provide useful guidelines to inform the design of \PPAs}.
% For instance, providing informed, unambiguous, and explicit decisions can foster trust in \PPAs even when consent is not their legal basis.
% Or, for example, carefully crafted presentations of the risks associated with the use of an AI system can raise awareness amongst users.

% \subsubsection{Use of varied sources of data, accounting for both context and personal preferences}
% Last, based on the relative singularity of data sources, \textbf{we advocate for a plurality of data sources encompassing context as much as personal preferences}.
% Our study yielded that \PPAs leverage various sources of data (context, attitudinal data, behavioral data, type of data, content of data, and metadata), but not necessarily at the same time.
% The utilization of several of these data sources can however be a challenge in itself, as it requires a careful curation of the datasets, and an adequate use in the AI models used.
% The benefits harnessed can be high, resulting in higher accuracies of the prediction made.

\subsection{Research Avenues}
\label{subsec:future}
In this final section of our Discussion, we investigate prospective paths for research on \PPAs, informed by the results of our study and the current social, technical, and legal landscape.

\subsubsection{The Future of \PPAs and Generative AI}
While the uptake of generative AI, such as Large Language Models (LLMs), is undeniable, their application to privacy decisions  
%yet to be observed 
is not addressed yet in the literature.
It is however already possible to find privacy assistants powered by this technology, for instance~\citet{hamid_genaipabench_2023} who provide a benchmark for evaluating Generative AI-based Privacy Assistants,~\footnote{Note that their assistants do not support users in making privacy decisions.} although we did not find papers captured by our selection criteria.
\textbf{We are only one step away from having \PPAs powered by LLMs}, which is not without raising interrogations.

Indeed, these models are intrinsically challenging to explain. 
Also, because of their inherent tendency to glitch (or ``hallucinate''), \textbf{they come with privacy challenges regarding compliance with the data accuracy principle of the GDPR,\footnote{Art 5 (i) (e) GDPR indeed stipulates that data needs to be ``accurate and, where necessary, kept up to date; every reasonable step must be taken to ensure that personal data that are inaccurate, having regard to the purposes for which they are processed, are erased or rectified without delay (‘accuracy’)''.} and should therefore be incorporated in \PPAs with caution}.
%Their use in legally binding decisions (in the EU) is out of the scope anyway because of the definition of consent requirements (i.e., consent cannot be fully automated under the GDPR); they can, however, provide valuable 
Still, future research should address opportunities and challenges of designing and using LLM-based Personalized Privacy Assistants, as well as technical and legal requirements for involving LLMs in assisting users with privacy decisions.
%recommendations, taken that their suggestions come with adequate notices and warnings.

\subsubsection{Designing Genuinely Privacy-friendly \PPAs}
A promising yet critical avenue for future research remains \textbf{to design a genuinely privacy-friendly \PPA, with the right amount of notice} to empower users and avoid the so-called ``consent fatigue.''
This right amount of notice can be a difficult balance to achieve -- some users favor more notice than others --but it is a crucial step for the uptake of such assistants.

The design should naturally be informed by the latest results in the academic literature~\cite{feng_design_2021}; it should carefully consider the number of notices, their content, their timing, etc.
However, it should also be complemented by usability studies conducted in the early stages of the prototype, as iterations over the design of the assistant are likely to be required in order to fine-tune it.


%\subsubsection{Articulation of the AI act with our question, what is the risk level in this case?}
%\todo[inline]{Simone for a first pass?}


\subsubsection{Trust in the AI assistants and automation bias}
Individuals tend to overly trust AI systems and favor AI-based decision-making while ignoring contradictory information made without automation, a phenomenon known as automation bias~\cite{cummings2017automation}, which is a problem also related to the Elisa effect first described by~\citet{weizenbaum1976computer}. If the user's decisions are biased towards following a privacy recommendation proposed by an \PPA, the users' autonomy may be negatively impacted in practice. Hence, future research should examine if users may too easily trust and rely on proposed or nudged decisions by \PPAs without critically judging or adapting proposed decisions and how such a problem could be addressed by Human-Computer Interaction research. 
%, should you trust these recommendations? do people overtrust AI systems? user studies to assess trust and how much people look into explanations.
%\todo[inline]{Simone for a first pass?}



\section{Threats to Validity Summary}
\label{sec:validity-threats}
Here, we briefly address the SLR's threats to validity and explain the strategies used to mitigate them (see details in Appendix~\ref{app:threats}). The three main threats are the following. \textbf{(1) Potential Planning Limitations:} biases may be introduced in the initial planning phase if research questions are not clearly defined and key topics are omitted, which could lead to flawed review results. To mitigate this threat, we clearly outlined two research questions and objectives and defined an SLR Protocol~\footnote{\url{
https://anonymous.4open.science/r/SoK_AI_PPA-E29F}}, piloted and refined before the review.
\textbf{(2) Validity of the Search Process:} There is also a risk that relevant studies were missed during the search process, potentially impacting the accuracy of the SLR's findings. To mitigate such threats, we followed a stepwise search process involving literature screening, complete papers reading, forward and backward snowballing, and selection by two independent reviewers. We also base the SLR on four scientific databases (i.e., Scopus, Web of Science, IEEE Xplore, and ACM Digital Library) with high relevance to computer science, privacy, and data protection.
\textbf{ (3) Potential Bias in Synthesis Process:} The synthesis of data from reviewed studies could introduce biases, such as deriving a flawed taxonomy or missed potential research gaps. To avoid bias, we created a DEF with well-known classification schemes and involved all researchers in the synthesis process for consistency checking.

\section{Conclusion}
\label{sec:conclusion}
In this SoK, we provide a classification and common vocabulary to compare and discuss \PPAs.
Although many papers (39 in our selection) have already been published on \PPAs in the last decade, they do not yet form a coherent body of literature.
\PPAs can be improved by performing standard evaluations (including their usability), integrating privacy by design in their design process, providing additional explanations for their decisions, and considering more system contexts and data sources.
We hope this survey and its classification allows users and developers of \PPAs to compare different solutions and understand their pros and cons. Moreover, the recommendations 
should help improve \PPAs in different ways, addressing the challenges raised by AI's latest developments (including LLMs), data collection, and modern regulations.
% We hope this classification will foster a new wave of \PPAs able to address the challenges raised by AI's latest developments (including LLMs), data collection, and regulations.


\begin{acks}
This work was
partially supported by the Wallenberg AI, Autonomous
Systems and Software Program (WASP) funded by the
Knut and Alice Wallenberg Foundation.
\end{acks}

% \newpage

%%
%% The next two lines define the bibliography style to be used, and
%% the bibliography file.
\bibliographystyle{plainnat}
\bibliography{APD_survey,APD_suppl}

\newpage

\appendix
\section{Supplementary material}


\subsection{Summary of the SLR Protocol}
\label{app:protocol}
See also \url{https://anonymous.4open.science/r/SoK_AI_PPA-E29F/Review_protocol.md}.

\subsubsection{Methods}
\paragraph{Design}
This SoK study adopts the widely known methodology for systematic literature reviews (SLRs) proposed by \citet{kitchenham2004procedures}. 
The SLR methodology offers us a well-defined and rigorous sequence of methodological steps consisting of three main phases: (1) planning, (2) conducting, and (3) reporting the review.

\paragraph{Eligibility Criteria}
Based on the research questions, and to reduce the likelihood of bias, the following inclusion and exclusion criteria are followed.
\begin{table}[!h]
    \centering
    \small
    \begin{tabular}{|p{0.95\linewidth}|}
    \hline
       \textbf{Inclusion Criteria} \\ \hline
      % - Papers with a technological focus (computer science and engineering, information system/usability, interdisciplinary techno-legal). \\
       - Provides a technical solution (implemented or theoretical) to help end-users automate personal (and personalized) privacy decisions with an assistant (or artificial agent) in IT systems. \\
       - Papers from 2013 onward to concentrate on the state-of-the-art. \\
       - The concept of AI needs to be explicitly stated in the papers. \\ \hline \hline
       \textbf{Exclusion Criteria} \\ \hline
       - Papers with solutions that are purely theoretical without substantial explanations on how they could be implemented in practice. \\
       - Papers with solutions that solely automate the analysis of privacy policies but without any type of personalization. \\
       - Papers with poor scientific quality (e.g., lack objectives or research questions, the methodology is not described, the solution is insufficiently/vaguely described, etc.). \\
       \hline
    \end{tabular}
    \caption{Criteria for the inclusion and exclusion of studies.}
    \label{app:selection-criteria}
\end{table}

\subsubsection{Information Sources and Search Strategy}
Four scientific databases were selected, i.e., Scopus, Web of Science, IEEE Xplore, and ACM Digital Library, due to their high relevance to the areas of computer science and engineering, comprising the vast majority of published research in the field.
We also specified inclusion and exclusion criteria (see Table \ref{app:selection-criteria}) used during the screening of publications retrieved from the databases.
Before starting the search process, two authors piloted the searches on all databases and ran a \textit{calibration exercise} to verify the consistency of the inclusion criteria. For that, the authors independently screened 10\% of the results and discussed their decisions. The conflicts were all discussed and solved, sometimes with the help of a third author. This process was repeated a second time, screening another 10\% of the papers, at a point that the authors agreed with the consistency of the selection process.

Based on our RQs and previous preliminary searches when designing the SLR Protocol, we identified a list of nine relevant keywords, i.e., privacy, data protection, assistant, agent, artificial intelligence, machine learning, intelligent, automatic, and personalized. These keywords were used to construct the following search query:
\begin{spverbatim}
(privacy OR "data protection") AND (assistant* OR agent*) AND ("artificial intelligence" OR "machine*learning" OR intelligent OR automat* OR personali*ed)
\end{spverbatim}
As such, the search query targets papers working on three joint topics: 1) privacy (or data protection), using either 2) an assistant or an agent, and leveraging 3) artificial intelligence or personalization.

\subsubsection{Study Records}
\paragraph{Data management}
To manage the screening process, we exported search results from each database and imported them to the RAYYAN software (\url{https://rayyan.ai/}), allowing two reviewers to select papers independently (i.e., double-blinded) and manage conflicts by a third reviewer. 
Duplicated publications were also removed using RAYYAN during the selection process. 
Bibliographies of final results were exported to Zotero (for citing and sharing research).

\paragraph{Data extraction}
Preliminary components to be extracted:
\begin{itemize}
    \item Bibliographic information, such as title, abstract, authors and affiliations, venues, year of publication, etc.
    \item Key information of the \PPA, such as its source(s) of data, its eventual architecture, its system context, the type of privacy decision considered, the accuracy of the decisions, the type of AI used, etc.
    \item The presence of a user study
    \item Extent of evaluation – scale of validation activity that is measured
    \item Quality assessment and critical appraisal of the studies that have validated or evaluated the \PPA
    \item Features of users control over decisions (initially guided by EU consent requirements)
\end{itemize}

\paragraph{Types of contributions}
Inspired by~\citet{kuhrmann_software_2016} and~\citet{shaw_writing_2003}, we classified publications by their types of contributions (i.e., multiple choice) according to the following:
\begin{description}
    \item[Model] Representation of observed reality by concepts after conceptualization.
    \item[Theory] Construct of cause-effect relationships.
    \item[Framework] Frameworks/methods (related to automated privacy decisions).
    \item[Guidelines] List of advice.
    \item[Lessons learned] Set of outcomes from obtained results.
    \item[Advice] Recommendation (from opinion).
    \item[Tool] A tool to automate privacy decisions.
\end{description}

\paragraph{Data synthesis}
We collated and summarized results into classification tables.
We composed a narrative synthesis for papers meeting our inclusion criteria.

Preliminary components of the data synthesis:
\begin{itemize}
    \item Overall identification and classification of \PPAs in published research
    \item Classification tables presenting features of \PPAs used to coherently organize the solutions surveyed
    \item Comparison analysis -- based on the features of \PPAs --, and narrative synthesis
\end{itemize}

\newpage

\subsection{Further Results from Data Charting and Critical Appraisal}
\label{app:tables}
% Please add the following required packages to your document preamble:
% \usepackage{booktabs}
\begin{table}[h]
\begin{tabular}{@{}lc||lc@{}}
\toprule
\textbf{Countries} & \textbf{Total} & \textbf{Countries} & \textbf{Total} \\ \midrule
United States & 13 & China & 2 \\
United Kingdom & 6 & Israel & 2 \\
Japan & 4 & Portugal & 2 \\
Netherlands & 4 & Switzerland & 1 \\
Italy & 4 & Turkey & 1 \\
Germany & 3 & Canada & 1 \\
Spain & 3 & Australia & 1 \\
India & 2 & & \\ \bottomrule
\end{tabular}
\caption{Countries of affiliation of authors of selected papers.}
\label{tab:countries}
\end{table}

% Please add the following required packages to your document preamble:
% \usepackage{booktabs}
\begin{table}[h]
\begin{tabular}{@{}lc||lc@{}}
\toprule
\textbf{Year} & \textbf{N. of Publications} & \textbf{Year} & \textbf{N. of Publications} \\ \midrule
2013 & 0 & 2019 & 4 \\
2014 & 1 & 2020 & 5 \\
2015 & 3 & 2021 & 2 \\
2016 & 3 & 2022 & 8 \\
2017 & 6 & 2023 & 2 \\
2018 & 5 \\ \bottomrule
\end{tabular}
\caption{Number of publications per year.}
\label{tab:years}
\end{table}


% \begin{table*}[]
% \tiny
% \begin{threeparttable}[b]
% \begin{tabular}{@{}llllllllll@{}}
% \toprule
%  &  &  &  & \multicolumn{6}{l}{Type of contribution} \\ \cmidrule(l){5-10} 
%  &  &  &  &  &  &  &  &  &  \\
%  &  &  &  &  &  &  &  &  &  \\
%  &  &  &  &  &  &  &  &  &  \\
%  &  &  &  &  &  &  &  &  &  \\
%  &  &  &  &  &  &  &  &  &  \\
% \multirow{-7}{*}{Publication} & \multirow{-7}{*}{Accuracy} & \multirow{-7}{*}{User study} & \multirow{-7}{*}{Critical appraisal} & \begin{rotate}{60} Framework \end{rotate} & \begin{rotate}{60} Tool \end{rotate} & \begin{rotate}{60} Model \end{rotate} & \begin{rotate}{60} Theory \end{rotate} & \begin{rotate}{60} Lessons learned \end{rotate} & \begin{rotate}{60} Advice \end{rotate} \\ \cmidrule(r){1-4}
% \rowcolor[HTML]{C0C0C0} 
% \citet{xie_location_2014} & 68\% & Online user experiment~\alpha & -- & $\bullet$ &  &  &  & $\bullet$ &  \\
% \citet{apolinarski_automating_2015} & -- & No & -- & $\bullet$ & $\bullet$ &  &  &  &  \\
% \rowcolor[HTML]{C0C0C0} 
% \citet{hirschprung_simplifying_2015} & -- & Online qualitative survey & D-, very low (55\%) & $\bullet$ &  & $\bullet$ &  &  &  \\
% \citet{squicciarini_privacy_2015} & 92.53\% & Cross sectional study~\tablefootnote{A survey-based study and a direct user evaluation} & D, very low (55\%) & $\bullet$ &  & $\bullet$ &  &  &  \\
% \rowcolor[HTML]{C0C0C0} 
% \citet{liu_follow_2016} & 78.7\% & Randomized controlled studies~\tablefootnote{Two surveys} & A, high (90\%) &  & $\bullet$ &  &  & $\bullet$ &  \\
% \citet{albertini_privacy_2016} & -- & Cross-sectional study & D, very low (55\%) &  & $\bullet$ &  &  &  &  \\
% \rowcolor[HTML]{C0C0C0} 
% \citet{dong_ppm_2016} & 89,8\% F1 & Case studies~\alpha & -- &  &  & $\bullet$ &  & $\bullet$ &  \\
% \citet{baarslag_automated_2017} & -- & Randomized controlled study~\tablefootnote{Between-participants design} & A, high (90\%) &  & $\bullet$ & $\bullet$ &  & $\bullet$ &  \\
% \rowcolor[HTML]{C0C0C0} 
% \citet{fogues_sosharp_2017} & Around 50\% & Online survey~\alpha & -- &  & $\bullet$ &  &  &  &  \\
% \citet{zhong_group-based_2017} & 79\% & Survey~\alpha & -- &  &  & $\bullet$ &  &  &  \\
% \rowcolor[HTML]{C0C0C0} 
% \citet{misra_pacman_2017} & 91.8\% & Non-controlled before-after study~\tablefootnote{Online survey} & C, limited (70\%) &  & $\bullet$ &  &  &  &  \\
% \citet{camp_easing_2017} & 85\% & Cross-sectional study~\tablefootnote{Online questionnaire} & D, very low (55\%) &  &  & $\bullet$ &  &  &  \\
% \rowcolor[HTML]{C0C0C0} 
% \citet{olejnik_smarper_2017} & More than 80\% & Yes, for data collection~\alpha & -- & $\bullet$ & $\bullet$ &  &  &  &  \\
% \citet{das_personalized_2018} & -- & No & -- &  & $\bullet$ &  &  &  &  \\
% \rowcolor[HTML]{C0C0C0} 
% \citet{tan_context-perceptual_2018} & 95\%~\tablefootnote{For privacy leakage detection (not to be confused with preferences detection)} & No & -- &  & $\bullet$ &  &  &  &  \\
% \citet{wijesekera_contextualizing_2018} & 95\% & Experience Sampling Method & D, low (60\%) &  & $\bullet$ &  &  & $\bullet$ &  \\
% \rowcolor[HTML]{C0C0C0} 
% \citet{yu_leveraging_2018} & -- & Cross-sectional study~\tablefootnote{To measure the interpretability of the approaches} & D, very low (55\%) &  &  & $\bullet$ & $\bullet$ &  &  \\
% \citet{bahirat_data-driven_2018} & 81.54\% & No & -- &  &  & $\bullet$ &  &  &  \\
% \rowcolor[HTML]{C0C0C0} 
% \citet{klingensmith_hypervisor-based_2019} & -- & No & -- &  & $\bullet$ &  &  &  &  \\
% \citet{barbosa_what_2019} & 86.8\%~\tablefootnote{AUC of binary allow/deny for a given scenario} & Survey~\alpha & -- &  &  & $\bullet$ &  &  & $\bullet$ \\
% \rowcolor[HTML]{C0C0C0} 
% \citet{alom_helping_2019} & Up to 72.2\% (satisfaction) & Cross-sectional study~\tablefootnote{User satisfaction} & D, very low (55\%) & $\bullet$ &  &  &  &  &  \\
% \citet{alom_adapting_2019} & 96.4\% and 94.5\%~\tablefootnote{Accuracy based on appreciation of evaluators} & Yes, for labeling and evaluation~\alpha & -- & $\bullet$ &  &  &  &  &  \\
% \rowcolor[HTML]{C0C0C0} 
% \citet{barolli_selflearning_2020} & -- & No & -- &  & $\bullet$ & $\bullet$ &  &  &  \\
% \citet{kaur_smart_2020} & -- & No & -- &  &  & $\bullet$ &  &  &  \\
% \rowcolor[HTML]{C0C0C0} 
% \citet{herrero_automatic_2021} & --~\tablefootnote{Accuracy based on appreciation of evaluators} & No & -- &  & $\bullet$ &  &  &  &  \\
% \citet{kokciyan_turp_2020} & -- & No & -- &  &  & $\bullet$ &  &  &  \\
% \rowcolor[HTML]{C0C0C0} 
% \citet{sanchez_recommendation_2020} & 84.74\% & Online survey to build their dataset~\alpha & -- &  &  & $\bullet$ &  &  &  \\
% \citet{barolli_reinforcement_2021} & -- & No & -- &  &  & $\bullet$ &  &  &  \\
% \rowcolor[HTML]{C0C0C0} 
% \citet{lobner_explainable_2021} & 83.33\%~\tablefootnote{Accuracy based on appreciation of evaluators} & Survey~\alpha & -- &  &  & $\bullet$ &  &  &  \\
% \citet{filipczuk_automated_2022} & 65\%~\tablefootnote{On average, but seems higher in specific case} & Between-subject experimental design & C, limited (70\%) & $\bullet$ & $\bullet$ &  &  &  &  \\
% \rowcolor[HTML]{C0C0C0} 
% \citet{hirschprung_game_2022} & -- & Cross-sectional study & D, low (60\%) & $\bullet$ &  & $\bullet$ &  &  &  \\
% \citet{kokciyan_taking_2022} & Between 41 and 92\%~\tablefootnote{Depends on several parameters} & No & -- &  &  & $\bullet$ &  &  &  \\
% \rowcolor[HTML]{C0C0C0} 
% \citet{ulusoy_panola_2022} & Around 75\%~\tablefootnote{Difficult to assess because they measure utility of decisions in a simulated setting} & No & -- &  &  & $\bullet$ &  &  &  \\
% \citet{zhan_model_2022} & 74\% & No & -- &  &  & $\bullet$ &  &  &  \\
% \rowcolor[HTML]{C0C0C0} 
% \citet{brandao_prediction_2022} & Between 82 and 88\% & Field study (cross-sectional study) & D, low (60\%) &  & $\bullet$ &  &  &  &  \\
% \citet{mendes_enhancing_2022} & 92\% & Cross-sectional study~\tablefootnote{Field study} & D, low (60\%) &  & $\bullet$ & $\bullet$ &  & $\bullet$ &  \\
% \rowcolor[HTML]{C0C0C0} 
% \citet{shanmugarasa_automated_2022} & 92.62\% & Cross-sectional study~\tablefootnote{Field study} & D, low (60\%) &  & $\bullet$ & $\bullet$ &  &  &  \\
% \citet{ayci_uncertainty-aware_2023} & 89\% & No & -- &  & $\bullet$ & $\bullet$ &  &  &  \\
% \rowcolor[HTML]{C0C0C0} 
% \citet{serramia_predicting_2023} & 3.78/5~\tablefootnote{Acceptability rate, not accuracy} & Cross-sectional study~\tablefootnote{To measure the level of comfort of the norms inferred} & D, very low (55\%) &  & $\bullet$ & $\bullet$ &  &  &  \\ \bottomrule
% \end{tabular}
% \begin{tablenotes}
% \item[\alpha] Alpha means that the user study is not meant to assess the solution, but only meant to collect data
% \item[17] A survey-based study and a direct user evaluation
% \item[18] Two surveys
% \item[19] Between-participants design
% \item[20] Online survey
% \item[21] Online questionnaire
% \item[22] For privacy leakage detection (not to be confused with preferences detection)
% \item[23] To measure the interpretability of the approaches
% \item[24] AUC of binary allow/deny for a given scenario 
% \item[25] User satisfaction
% \item[26] Accuracy based on appreciation of evaluators
% \item[27] Accuracy based on appreciation of evaluators
% \item[28] Accuracy based on appreciation of evaluators
% \item[29] On average, but seems higher in specific case
% \item[30] Depends on several parameters
% \item[31] Difficult to assess because they measure utility of decisions in a simulated setting
% \item[32] Field study
% \item[33] Field study
% \item[34] Acceptability rate, not accuracy
% \item[35] To measure the level of comfort of the norms inferred
% \end{tablenotes}
% \caption{Supplementary table of our results.
% ~
% We present the \textbf{accuracy} of the predictions (see Section~\ref{subsec:taxo_validation}); the presence or not of a \textbf{user study}, and the type of user study if applicable; the results of our \textbf{critical appraisal} (see Section~\ref{subsec:appraisal}); and finally the \textbf{type of contribution}, informing whether the solution surveyed presents a \textit{framework}, a \textit{tool} (i.e. with an implementation), a \textit{model}, \textit{lessons learned}, or \textit{advice}.
% ~
% We denote with -- when the criterion is not applicable (no implementation/tool is presented)}
% \label{tab:appendix_table}
% \end{threeparttable}
% \end{table*}

% Please add the following required packages to your document preamble:
% \usepackage{multirow}
% \usepackage[table,xcdraw]{xcolor}
% Beamer presentation requires \usepackage{colortbl} instead of \usepackage[table,xcdraw]{xcolor}
\begin{table*}[]
\tiny
\begin{threeparttable}[b]
\begin{tabular}{l|l|l|l|llllll}
\hline
 &  &  &  & \multicolumn{6}{l}{Type of contribution} \\ \cline{5-10} 
 &  &  &  & \multicolumn{1}{l|}{} & \multicolumn{1}{l|}{} & \multicolumn{1}{l|}{} & \multicolumn{1}{l|}{} & \multicolumn{1}{l|}{} &  \\
 &  &  &  & \multicolumn{1}{l|}{} & \multicolumn{1}{l|}{} & \multicolumn{1}{l|}{} & \multicolumn{1}{l|}{} & \multicolumn{1}{l|}{} &  \\
 &  &  &  & \multicolumn{1}{l|}{} & \multicolumn{1}{l|}{} & \multicolumn{1}{l|}{} & \multicolumn{1}{l|}{} & \multicolumn{1}{l|}{} &  \\
 &  &  &  & \multicolumn{1}{l|}{} & \multicolumn{1}{l|}{} & \multicolumn{1}{l|}{} & \multicolumn{1}{l|}{} & \multicolumn{1}{l|}{} &  \\
 &  &  &  & \multicolumn{1}{l|}{} & \multicolumn{1}{l|}{} & \multicolumn{1}{l|}{} & \multicolumn{1}{l|}{} & \multicolumn{1}{l|}{} &  \\
\multirow{-7}{*}{Publication} & \multirow{-7}{*}{Accuracy} & \multirow{-7}{*}{User study} & \multirow{-7}{*}{Critical appraisal} & \multicolumn{1}{l|}{\begin{rotate}{90} Framework \end{rotate}} & \multicolumn{1}{l|}{\begin{rotate}{90} Tool \end{rotate}} & \multicolumn{1}{l|}{\begin{rotate}{90} Model \end{rotate}} & \multicolumn{1}{l|}{\begin{rotate}{90} Theory \end{rotate}} & \multicolumn{1}{l|}{\begin{rotate}{90} Lessons learned \end{rotate}} & \begin{rotate}{90} Advice \end{rotate} \\ \hline
\rowcolor[HTML]{C0C0C0} 
\citet{xie_location_2014} & 68\% & Online user experiment~$\alpha$ & -- & \multicolumn{1}{l|}{\cellcolor[HTML]{C0C0C0}$\bullet$} & \multicolumn{1}{l|}{\cellcolor[HTML]{C0C0C0}} & \multicolumn{1}{l|}{\cellcolor[HTML]{C0C0C0}} & \multicolumn{1}{l|}{\cellcolor[HTML]{C0C0C0}} & \multicolumn{1}{l|}{\cellcolor[HTML]{C0C0C0}$\bullet$} &  \\
\citet{apolinarski_automating_2015} & -- & No & -- & \multicolumn{1}{l|}{$\bullet$} & \multicolumn{1}{l|}{$\bullet$} & \multicolumn{1}{l|}{} & \multicolumn{1}{l|}{} & \multicolumn{1}{l|}{} &  \\
\rowcolor[HTML]{C0C0C0} 
\citet{hirschprung_simplifying_2015} & -- & Online qualitative survey & D-, very low (55\%) & \multicolumn{1}{l|}{\cellcolor[HTML]{C0C0C0}$\bullet$} & \multicolumn{1}{l|}{\cellcolor[HTML]{C0C0C0}} & \multicolumn{1}{l|}{\cellcolor[HTML]{C0C0C0}$\bullet$} & \multicolumn{1}{l|}{\cellcolor[HTML]{C0C0C0}} & \multicolumn{1}{l|}{\cellcolor[HTML]{C0C0C0}} &  \\
\citet{squicciarini_privacy_2015} & 92.53\% & Cross sectional study~\tablefootnote{A survey-based study and a direct user evaluation} & D, very low (55\%) & \multicolumn{1}{l|}{$\bullet$} & \multicolumn{1}{l|}{} & \multicolumn{1}{l|}{$\bullet$} & \multicolumn{1}{l|}{} & \multicolumn{1}{l|}{} &  \\
\rowcolor[HTML]{C0C0C0} 
\citet{liu_follow_2016} & 78.7\% & Randomized controlled studies~\tablefootnote{Two surveys} & A, high (90\%) & \multicolumn{1}{l|}{\cellcolor[HTML]{C0C0C0}} & \multicolumn{1}{l|}{\cellcolor[HTML]{C0C0C0}$\bullet$} & \multicolumn{1}{l|}{\cellcolor[HTML]{C0C0C0}} & \multicolumn{1}{l|}{\cellcolor[HTML]{C0C0C0}} & \multicolumn{1}{l|}{\cellcolor[HTML]{C0C0C0}$\bullet$} &  \\
\citet{albertini_privacy_2016} & -- & Cross-sectional study & D, very low (55\%) & \multicolumn{1}{l|}{} & \multicolumn{1}{l|}{$\bullet$} & \multicolumn{1}{l|}{} & \multicolumn{1}{l|}{} & \multicolumn{1}{l|}{} &  \\
\rowcolor[HTML]{C0C0C0} 
\citet{dong_ppm_2016} & 89,8\% F1 & Case studies~$\alpha$ & -- & \multicolumn{1}{l|}{\cellcolor[HTML]{C0C0C0}} & \multicolumn{1}{l|}{\cellcolor[HTML]{C0C0C0}} & \multicolumn{1}{l|}{\cellcolor[HTML]{C0C0C0}$\bullet$} & \multicolumn{1}{l|}{\cellcolor[HTML]{C0C0C0}} & \multicolumn{1}{l|}{\cellcolor[HTML]{C0C0C0}$\bullet$} &  \\
\citet{baarslag_automated_2017} & -- & Randomized controlled study~\tablefootnote{Between-participants design} & A, high (90\%) & \multicolumn{1}{l|}{} & \multicolumn{1}{l|}{$\bullet$} & \multicolumn{1}{l|}{$\bullet$} & \multicolumn{1}{l|}{} & \multicolumn{1}{l|}{$\bullet$} &  \\
\rowcolor[HTML]{C0C0C0} 
\citet{fogues_sosharp_2017} & Around 50\% & Online survey~$\alpha$ & -- & \multicolumn{1}{l|}{\cellcolor[HTML]{C0C0C0}} & \multicolumn{1}{l|}{\cellcolor[HTML]{C0C0C0}$\bullet$} & \multicolumn{1}{l|}{\cellcolor[HTML]{C0C0C0}} & \multicolumn{1}{l|}{\cellcolor[HTML]{C0C0C0}} & \multicolumn{1}{l|}{\cellcolor[HTML]{C0C0C0}} &  \\
\citet{zhong_group-based_2017} & 79\% & Survey~$\alpha$ & -- & \multicolumn{1}{l|}{} & \multicolumn{1}{l|}{} & \multicolumn{1}{l|}{$\bullet$} & \multicolumn{1}{l|}{} & \multicolumn{1}{l|}{} &  \\
\rowcolor[HTML]{C0C0C0} 
\citet{misra_pacman_2017} & 91.8\% & Non-controlled before-after study~\tablefootnote{Online survey} & C, limited (70\%) & \multicolumn{1}{l|}{\cellcolor[HTML]{C0C0C0}} & \multicolumn{1}{l|}{\cellcolor[HTML]{C0C0C0}$\bullet$} & \multicolumn{1}{l|}{\cellcolor[HTML]{C0C0C0}} & \multicolumn{1}{l|}{\cellcolor[HTML]{C0C0C0}} & \multicolumn{1}{l|}{\cellcolor[HTML]{C0C0C0}} &  \\
\citet{camp_easing_2017} & 85\% & Cross-sectional study~\tablefootnote{Online questionnaire} & D, very low (55\%) & \multicolumn{1}{l|}{} & \multicolumn{1}{l|}{} & \multicolumn{1}{l|}{$\bullet$} & \multicolumn{1}{l|}{} & \multicolumn{1}{l|}{} &  \\
\rowcolor[HTML]{C0C0C0} 
\citet{olejnik_smarper_2017} & More than 80\% & Yes, for data collection~$\alpha$ & -- & \multicolumn{1}{l|}{\cellcolor[HTML]{C0C0C0}$\bullet$} & \multicolumn{1}{l|}{\cellcolor[HTML]{C0C0C0}$\bullet$} & \multicolumn{1}{l|}{\cellcolor[HTML]{C0C0C0}} & \multicolumn{1}{l|}{\cellcolor[HTML]{C0C0C0}} & \multicolumn{1}{l|}{\cellcolor[HTML]{C0C0C0}} &  \\
\citet{das_personalized_2018} & -- & No & -- & \multicolumn{1}{l|}{} & \multicolumn{1}{l|}{$\bullet$} & \multicolumn{1}{l|}{} & \multicolumn{1}{l|}{} & \multicolumn{1}{l|}{} &  \\
\rowcolor[HTML]{C0C0C0} 
\citet{tan_context-perceptual_2018} & 95\%~\tablefootnote{For privacy leakage detection (not to be confused with preferences detection)} & No & -- & \multicolumn{1}{l|}{\cellcolor[HTML]{C0C0C0}} & \multicolumn{1}{l|}{\cellcolor[HTML]{C0C0C0}$\bullet$} & \multicolumn{1}{l|}{\cellcolor[HTML]{C0C0C0}} & \multicolumn{1}{l|}{\cellcolor[HTML]{C0C0C0}} & \multicolumn{1}{l|}{\cellcolor[HTML]{C0C0C0}} &  \\
\citet{wijesekera_contextualizing_2018} & 95\% & Experience Sampling Method & D, low (60\%) & \multicolumn{1}{l|}{} & \multicolumn{1}{l|}{$\bullet$} & \multicolumn{1}{l|}{} & \multicolumn{1}{l|}{} & \multicolumn{1}{l|}{$\bullet$} &  \\
\rowcolor[HTML]{C0C0C0} 
\citet{yu_leveraging_2018} & -- & Cross-sectional study~\tablefootnote{To measure the interpretability of the approaches} & D, very low (55\%) & \multicolumn{1}{l|}{\cellcolor[HTML]{C0C0C0}} & \multicolumn{1}{l|}{\cellcolor[HTML]{C0C0C0}} & \multicolumn{1}{l|}{\cellcolor[HTML]{C0C0C0}$\bullet$} & \multicolumn{1}{l|}{\cellcolor[HTML]{C0C0C0}$\bullet$} & \multicolumn{1}{l|}{\cellcolor[HTML]{C0C0C0}} &  \\
\citet{bahirat_data-driven_2018} & 81.54\% & No & -- & \multicolumn{1}{l|}{} & \multicolumn{1}{l|}{} & \multicolumn{1}{l|}{$\bullet$} & \multicolumn{1}{l|}{} & \multicolumn{1}{l|}{} &  \\
\rowcolor[HTML]{C0C0C0} 
\citet{klingensmith_hypervisor-based_2019} & -- & No & -- & \multicolumn{1}{l|}{\cellcolor[HTML]{C0C0C0}} & \multicolumn{1}{l|}{\cellcolor[HTML]{C0C0C0}$\bullet$} & \multicolumn{1}{l|}{\cellcolor[HTML]{C0C0C0}} & \multicolumn{1}{l|}{\cellcolor[HTML]{C0C0C0}} & \multicolumn{1}{l|}{\cellcolor[HTML]{C0C0C0}} &  \\
\citet{barbosa_what_2019} & 86.8\%~\tablefootnote{AUC of binary allow/deny for a given scenario} & Survey~$\alpha$ & -- & \multicolumn{1}{l|}{} & \multicolumn{1}{l|}{} & \multicolumn{1}{l|}{$\bullet$} & \multicolumn{1}{l|}{} & \multicolumn{1}{l|}{} & $\bullet$ \\
\rowcolor[HTML]{C0C0C0} 
\citet{alom_helping_2019} & Up to 72.2\% (satisfaction) & Cross-sectional study~\tablefootnote{User satisfaction} & D, very low (55\%) & \multicolumn{1}{l|}{\cellcolor[HTML]{C0C0C0}$\bullet$} & \multicolumn{1}{l|}{\cellcolor[HTML]{C0C0C0}} & \multicolumn{1}{l|}{\cellcolor[HTML]{C0C0C0}} & \multicolumn{1}{l|}{\cellcolor[HTML]{C0C0C0}} & \multicolumn{1}{l|}{\cellcolor[HTML]{C0C0C0}} &  \\
\citet{alom_adapting_2019} & 96.4\% and 94.5\%~\tablefootnote{Accuracy based on appreciation of evaluators} & Yes, for labeling and evaluation~$\alpha$ & -- & \multicolumn{1}{l|}{$\bullet$} & \multicolumn{1}{l|}{} & \multicolumn{1}{l|}{} & \multicolumn{1}{l|}{} & \multicolumn{1}{l|}{} &  \\
\rowcolor[HTML]{C0C0C0} 
\citet{barolli_selflearning_2020} & -- & No & -- & \multicolumn{1}{l|}{\cellcolor[HTML]{C0C0C0}} & \multicolumn{1}{l|}{\cellcolor[HTML]{C0C0C0}$\bullet$} & \multicolumn{1}{l|}{\cellcolor[HTML]{C0C0C0}$\bullet$} & \multicolumn{1}{l|}{\cellcolor[HTML]{C0C0C0}} & \multicolumn{1}{l|}{\cellcolor[HTML]{C0C0C0}} &  \\
\citet{kaur_smart_2020} & -- & No & -- & \multicolumn{1}{l|}{} & \multicolumn{1}{l|}{} & \multicolumn{1}{l|}{$\bullet$} & \multicolumn{1}{l|}{} & \multicolumn{1}{l|}{} &  \\
\rowcolor[HTML]{C0C0C0} 
\citet{herrero_automatic_2021} & --~\tablefootnote{The accuracy presented is for the right category of data} & No & -- & \multicolumn{1}{l|}{\cellcolor[HTML]{C0C0C0}} & \multicolumn{1}{l|}{\cellcolor[HTML]{C0C0C0}$\bullet$} & \multicolumn{1}{l|}{\cellcolor[HTML]{C0C0C0}} & \multicolumn{1}{l|}{\cellcolor[HTML]{C0C0C0}} & \multicolumn{1}{l|}{\cellcolor[HTML]{C0C0C0}} &  \\
\citet{kokciyan_turp_2020} & -- & No & -- & \multicolumn{1}{l|}{} & \multicolumn{1}{l|}{} & \multicolumn{1}{l|}{$\bullet$} & \multicolumn{1}{l|}{} & \multicolumn{1}{l|}{} &  \\
\rowcolor[HTML]{C0C0C0} 
\citet{sanchez_recommendation_2020} & 84.74\% & Online survey to build their dataset~$\alpha$ & -- & \multicolumn{1}{l|}{\cellcolor[HTML]{C0C0C0}} & \multicolumn{1}{l|}{\cellcolor[HTML]{C0C0C0}} & \multicolumn{1}{l|}{\cellcolor[HTML]{C0C0C0}$\bullet$} & \multicolumn{1}{l|}{\cellcolor[HTML]{C0C0C0}} & \multicolumn{1}{l|}{\cellcolor[HTML]{C0C0C0}} &  \\
\citet{barolli_reinforcement_2021} & -- & No & -- & \multicolumn{1}{l|}{} & \multicolumn{1}{l|}{} & \multicolumn{1}{l|}{$\bullet$} & \multicolumn{1}{l|}{} & \multicolumn{1}{l|}{} &  \\
\rowcolor[HTML]{C0C0C0} 
\citet{lobner_explainable_2021} & 83.33\%~\tablefootnote{With interpretability of the results} & Survey~$\alpha$ & -- & \multicolumn{1}{l|}{\cellcolor[HTML]{C0C0C0}} & \multicolumn{1}{l|}{\cellcolor[HTML]{C0C0C0}} & \multicolumn{1}{l|}{\cellcolor[HTML]{C0C0C0}$\bullet$} & \multicolumn{1}{l|}{\cellcolor[HTML]{C0C0C0}} & \multicolumn{1}{l|}{\cellcolor[HTML]{C0C0C0}} &  \\
\citet{filipczuk_automated_2022} & 65\%~\tablefootnote{On average, but seems higher in specific case} & Between-subject experimental design & C, limited (70\%) & \multicolumn{1}{l|}{$\bullet$} & \multicolumn{1}{l|}{$\bullet$} & \multicolumn{1}{l|}{} & \multicolumn{1}{l|}{} & \multicolumn{1}{l|}{} &  \\
\rowcolor[HTML]{C0C0C0} 
\citet{hirschprung_game_2022} & -- & Cross-sectional study & D, low (60\%) & \multicolumn{1}{l|}{\cellcolor[HTML]{C0C0C0}$\bullet$} & \multicolumn{1}{l|}{\cellcolor[HTML]{C0C0C0}} & \multicolumn{1}{l|}{\cellcolor[HTML]{C0C0C0}$\bullet$} & \multicolumn{1}{l|}{\cellcolor[HTML]{C0C0C0}} & \multicolumn{1}{l|}{\cellcolor[HTML]{C0C0C0}} &  \\
\citet{kokciyan_taking_2022} & Between 41 and 92\%~\tablefootnote{Depends on several parameters} & No & -- & \multicolumn{1}{l|}{} & \multicolumn{1}{l|}{} & \multicolumn{1}{l|}{$\bullet$} & \multicolumn{1}{l|}{} & \multicolumn{1}{l|}{} &  \\
\rowcolor[HTML]{C0C0C0} 
\citet{ulusoy_panola_2022} & Around 75\%~\tablefootnote{Difficult to assess because they measure utility of decisions in a simulated setting} & No & -- & \multicolumn{1}{l|}{\cellcolor[HTML]{C0C0C0}} & \multicolumn{1}{l|}{\cellcolor[HTML]{C0C0C0}} & \multicolumn{1}{l|}{\cellcolor[HTML]{C0C0C0}$\bullet$} & \multicolumn{1}{l|}{\cellcolor[HTML]{C0C0C0}} & \multicolumn{1}{l|}{\cellcolor[HTML]{C0C0C0}} &  \\
\citet{zhan_model_2022} & 74\% & No & -- & \multicolumn{1}{l|}{} & \multicolumn{1}{l|}{} & \multicolumn{1}{l|}{$\bullet$} & \multicolumn{1}{l|}{} & \multicolumn{1}{l|}{} &  \\
\rowcolor[HTML]{C0C0C0} 
\citet{brandao_prediction_2022} & Between 82 and 88\% & Field study (cross-sectional study) & D, low (60\%) & \multicolumn{1}{l|}{\cellcolor[HTML]{C0C0C0}} & \multicolumn{1}{l|}{\cellcolor[HTML]{C0C0C0}$\bullet$} & \multicolumn{1}{l|}{\cellcolor[HTML]{C0C0C0}} & \multicolumn{1}{l|}{\cellcolor[HTML]{C0C0C0}} & \multicolumn{1}{l|}{\cellcolor[HTML]{C0C0C0}} &  \\
\citet{mendes_enhancing_2022} & 92\% & Cross-sectional study~\tablefootnote{Field study} & D, low (60\%) & \multicolumn{1}{l|}{} & \multicolumn{1}{l|}{$\bullet$} & \multicolumn{1}{l|}{$\bullet$} & \multicolumn{1}{l|}{} & \multicolumn{1}{l|}{$\bullet$} &  \\
\rowcolor[HTML]{C0C0C0} 
\citet{shanmugarasa_automated_2022} & 92.62\% & Cross-sectional study~\tablefootnote{Field study} & D, low (60\%) & \multicolumn{1}{l|}{\cellcolor[HTML]{C0C0C0}} & \multicolumn{1}{l|}{\cellcolor[HTML]{C0C0C0}$\bullet$} & \multicolumn{1}{l|}{\cellcolor[HTML]{C0C0C0}$\bullet$} & \multicolumn{1}{l|}{\cellcolor[HTML]{C0C0C0}} & \multicolumn{1}{l|}{\cellcolor[HTML]{C0C0C0}} &  \\
\citet{ayci_uncertainty-aware_2023} & 89\% & No & -- & \multicolumn{1}{l|}{} & \multicolumn{1}{l|}{$\bullet$} & \multicolumn{1}{l|}{$\bullet$} & \multicolumn{1}{l|}{} & \multicolumn{1}{l|}{} &  \\
\rowcolor[HTML]{C0C0C0} 
\citet{serramia_predicting_2023} & 3.78/5~\tablefootnote{Acceptability rate, not accuracy} & Cross-sectional study~\tablefootnote{To measure the level of comfort of the norms inferred} & D, very low (55\%) & \multicolumn{1}{l|}{\cellcolor[HTML]{C0C0C0}} & \multicolumn{1}{l|}{\cellcolor[HTML]{C0C0C0}$\bullet$} & \multicolumn{1}{l|}{\cellcolor[HTML]{C0C0C0}$\bullet$} & \multicolumn{1}{l|}{\cellcolor[HTML]{C0C0C0}} & \multicolumn{1}{l|}{\cellcolor[HTML]{C0C0C0}} &  \\ \hline
\end{tabular}
\begin{tablenotes}
\item[$\alpha$] Alpha means that the user study is not meant to assess the solution, but only meant to collect data
\item[16] A survey-based study and a direct user evaluation
\item[17] Two surveys
\item[18] Between-participants design
\item[19] Online survey
\item[20] Online questionnaire
\item[21] For privacy leakage detection (not to be confused with preferences detection)
\item[22] To measure the interpretability of the approaches
\item[23] AUC of binary allow/deny for a given scenario 
\item[24] User satisfaction
\item[25] Accuracy based on appreciation of evaluators
\item[26] The accuracy presented is for the right category of data
\item[27] With interpretability of the results
\item[28] On average, but seems higher in specific case
\item[29] Depends on several parameters
\item[30] Difficult to assess because they measure utility of decisions in a simulated setting
\item[31] Field study
\item[32] Field study
\item[33] Acceptability rate, not accuracy
\item[34] To measure the level of comfort of the norms inferred
\end{tablenotes}
\caption{Supplementary table of our results.
~
We present the \textbf{accuracy} of the predictions (see Section~\ref{subsec:taxo_validation}); the presence or not of a \textbf{user study}, and the type of user study if applicable; the results of our \textbf{critical appraisal} (see Section~\ref{subsec:appraisal}); and finally the \textbf{type of contribution}, informing whether the surveyed solution presents a \textit{framework}, a \textit{tool} (i.e. with an implementation), a \textit{model}, \textit{lessons learned}, or \textit{advice}.
~
We denote with -- when the criterion is not applicable (no implementation/tool is presented)}
\label{tab:appendix_table}
\end{threeparttable}
\end{table*}

% \newpage~\newpage~\newpage
% \newpage
\section{Threats to Validity}
\label{app:threats}
\subsection{Threat I -- Planning Limitations of the SLR}
The first threat relates to the planning of the SLR in terms of identifying the need and justification for this study. Here, we were concerned with identifying existing reviews (systematic and non-systematic) on the topic of \PPAs. The initial searches did not reveal any review studies on the topic, as described in Section~\ref{sec:slr-planning}, pointing to a significant gap in secondary research AI PPAs. The planning phase of the SLR is also critical to outline the research questions and provide the basis for an objective investigation of the studies that are being reviewed. If the RQs are not explicitly stated or omit the key topics, the literature review results can be flawed, overlooking the key information. To mitigate this threat, we outlined two RQs and objectives (Section~\ref{sec:slr-planning}). In summary, we seek to minimize any bias or limitations during the planning phase when defining the scope and objectives of this SLR. As a last step in the planning phase, the team finalized and cross-checked the study protocol to minimize the limitations of the SLR plan before proceeding to the subsequent phases.

\subsection{Threat II -- Validity of the Search Process}
Identifying and selecting the studies reviewed in the SLR are also significant processes to be observed. Selecting studies is a critical step; if any relevant papers are missed, the results of the SLR may be flawed. Therefore, we followed a stepwise process (Section~\ref{sec:slr-conducting}), starting with a literature screening and followed by a complete reading of papers. This selection process was carried out independently by two reviewers. We also performed forward and backward snowballing, looking for references to other potentially relevant studies. Also, this SLR restricts the selection of publications to four scientific databases: Scopus, Web of Science, IEEE Xplore, and ACM Digital Library. These databases were used due to their high relevancy to computer science, privacy, and data protection, as well as to maintain a feasible search space. This search process gives us confidence that we minimized limitations related to (i) excluding or overlooking relevant studies or (ii) including irrelevant studies that could impact the results and their reporting in the SLR.

\subsection{Threat III -- Potential Bias in the Synthesis Process}
Some threats should also be considered regarding the potential bias in synthesizing the data from the review and documenting the results. This means that if there are some limitations in the data synthesis, they directly impact the results of this SLR. Typical examples of such limitations could be a flawed research taxonomy and a mismatch of potential research gaps. To minimize the bias in synthesizing and reporting the results, we have created a data extraction form that uses well-known classification schemes, such as the ones proposed by \citet{wieringa_requirements_2006} and \citet{creswell_research_2017}, or \citet{arrieta_explainable_2019} for the classification of AI. Three researchers independently reviewed this data extraction form while revising the research protocol. While one of the researchers led the data extraction step, two other authors helped by cross-checking the work throughout the process for consistency. Three authors were involved in the creation of the classification scheme derived from the literature (i.e., shown in Section~\ref{fig:classification}), actively working on reviewing the list of categories for consistency through a series of meetings. Furthermore, this SLR also offers a complete replication package, enabling researchers to reproduce or extend this review (\url{https://anonymous.4open.science/r/SoK_AI_PPA-E29F}).

\end{document}
\endinput
%%
%% End of file `sample-sigconf.tex'.
