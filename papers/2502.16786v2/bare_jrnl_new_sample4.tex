\documentclass[lettersize,journal]{IEEEtran}
\usepackage{amsmath,amsfonts}
\usepackage{algorithmic}
\usepackage{algorithm}
\usepackage{array}
\usepackage[caption=false,font=normalsize,labelfont=sf,textfont=sf]{subfig}
\usepackage{textcomp}
\usepackage{stfloats}
\usepackage{url}
\usepackage{verbatim}
\usepackage{graphicx}
\usepackage{cite}
\usepackage{hyperref}
\usepackage{url}
\usepackage{booktabs}
\usepackage{bm}


%%begin zheng
\usepackage{color}
\usepackage{MnSymbol,wasysym}
\usepackage{multirow}
\usepackage{booktabs}
\usepackage{colortbl}
\usepackage[table]{xcolor}

%% end zheng
\hyphenation{op-tical net-works semi-conduc-tor IEEE-Xplore}
% updated with editorial comments 8/9/2021

\begin{document}

\title{SwimVG: Step-wise Multimodal Fusion and Adaption for Visual Grounding}

% \author{Liangtao Shi\IEEEauthorrefmark{1}, Ting Liu\IEEEauthorrefmark{1}, 
% Xiantao Hu, Quanjun Yin, Yue Hu, 
% Richang Hong\IEEEauthorrefmark{2}, Senior Member, IEEE

\author{Liangtao Shi, Ting Liu, 
Xiantao Hu, Yue Hu, Quanjun Yin, 
Richang Hong, Senior Member, IEEE
%,~\IEEEmembership{Staff,~IEEE,}
        % <-this % stops a space
\thanks{\IEEEauthorrefmark{1} Liangtao Shi and Ting Liu contributed equally to this paper.}
\thanks{\IEEEauthorrefmark{2} Richang Hong is the corresponding author of this paper.}

\thanks{Liangtao Shi and Richang Hong with the Key Laboratory of Knowledge Engineering with Big Data, Hefei University of Technology, Hefei 230009, China, and also with the Ministry of Education and School of Computer Science and Information Engineering, Hefei University of Technology, Hefei 230009, China (e-mail: shilt@mail.hfut.edu.cn;
hongrc.hfut@gmail.com).}


\thanks{Ting Liu, Yue Hu and Quanjun Yin are with School of systems engineering, National University of Defense Technology, Changsha, Hunan Province, 410073, China. (e-mail: liuting20@nudt.edu.cn; yquanjun@126.com; huyue11@nudt.edu.cn). Xiantao Hu with the Department of Computer Science and Engineering, Nanjing University of Science and Technology, Nanjing 210014, China (e-mail: huxiantao481@gmail.com). This research was partially supported by the National Natural Science Fund of China (Grant Nos. 62306329 and 62103425), Natural Science Fund of Hunan Province (Grant Nos. 2023JJ40676 and 2022JJ40559).}  

\thanks{}
\thanks{}
\thanks{}
}

% The paper headers
\markboth{Journal of \LaTeX\ Class Files,~Vol.~14, No.~8, August~2021}%
{Shell \MakeLowercase{\textit{et al.}}: A Sample Article Using IEEEtran.cls for IEEE Journals}

%\IEEEpubid{0000--0000/00\$00.00~\copyright~2021 IEEE}
% Remember, if you use this you must call \IEEEpubidadjcol in the second
% column for its text to clear the IEEEpubid mark.

\maketitle

\IEEEtitleabstractindextext{
\begin{abstract}
The rapid advancement of large models, driven by their exceptional abilities in learning and generalization through large-scale pre-training, has reshaped the landscape of Artificial Intelligence (AI). 
These models are now foundational to a wide range of applications, including conversational AI, recommendation systems, autonomous driving, content generation, medical diagnostics, and scientific discovery. However, their widespread deployment also exposes them to significant safety risks, raising concerns about robustness, reliability, and ethical implications.
This survey provides a systematic review of current safety research on large models, covering Vision Foundation Models (VFMs), Large Language Models (LLMs), Vision-Language Pre-training (VLP) models, Vision-Language Models (VLMs), Diffusion Models (DMs), and large-model-based Agents. 
Our contributions are summarized as follows: (1) We present a comprehensive taxonomy of safety threats to these models, including adversarial attacks, data poisoning, backdoor attacks, jailbreak and prompt injection attacks, energy-latency attacks, data and model extraction attacks, and emerging agent-specific threats. 
(2) We review defense strategies proposed for each type of attacks if available and summarize the commonly used datasets and benchmarks for safety research.
(3) Building on this, we identify and discuss the open challenges in large model safety, emphasizing the need for comprehensive safety evaluations, scalable and effective defense mechanisms, and sustainable data practices. More importantly, we highlight the necessity of collective efforts from the research community and international collaboration.
Our work can serve as a useful reference for researchers and practitioners, fostering the ongoing development of comprehensive defense systems and platforms to safeguard AI models. GitHub: \url{https://github.com/xingjunm/Awesome-Large-Model-Safety}.

\end{abstract}
\begin{IEEEkeywords}
Large Model Safety, AI Safety, Attacks and Defenses
\end{IEEEkeywords}}
\begin{IEEEkeywords}
vision and language, multimodal representation, visual grounding.
\end{IEEEkeywords}
\section{Introduction}
\label{sec::intro}

Embodied Question Answering (EQA) \cite{das2018embodied} represents a challenging task at the intersection of natural language processing, computer vision, and robotics, where an embodied agent (e.g., a UAV) must actively explore its environment to answer questions posed in natural language. While most existing research has concentrated on indoor EQA tasks \cite{gao2023room, pena2023visual}, such as exploring and answering questions within confined spaces like homes or offices \cite{liu2024aligning}, relatively little attention has been dedicated to EQA tasks in  open-ended city space. Nevertheless, extending EQA to city space is crucial for numerous real-world applications, including autonomous systems \cite{kalinowska2023embodied}, urban region profiling \cite{yan2024urbanclip}, and city planning \cite{gao2024embodiedcity}. 
% 1. 环境复杂性   
%    - 地标重复性问题(如区分相似建筑)  
%    - 动态干扰因素(交通流、行人)  
% 2. 行动复杂性  
%    - 长程导航路径规划  
%    - 移动控制、角度等  
% 3. 感知复杂性  
%    - 复合空间关系推理("A楼东侧商铺西边的车辆")  
%    - 时序依赖的观察结果整合

EQA tasks in city space (referred to as CityEQA) introduce a unique set of challenges that fundamentally differ from those encountered in indoor environments. Compared to indoor EQA, CityEQA faces three main challenges: 

1) \textbf{Environmental complexity with ambiguous objects}: 
Urban environments are inherently more complex,  featuring a diverse range of objects and structures, many of which are visually similar and difficult to distinguish without detailed semantic information (e.g., buildings, roads, and vehicles). This complexity makes it challenging to construct task instructions and specify the desired information accurately, as shown in Figure \ref{fig:example}. 

2) \textbf{Action complexity in cross-scale space}: 
The vast geographical scale of city space compels agents to adopt larger movement amplitudes to enhance exploration efficiency. However, it might risk overlooking detailed information within the scene. Therefore, agents require cross-scale action adjustment capabilities to effectively balance long-distance path planning with fine-grained movement and angular control.

3) \textbf{Perception complexity with observation dynamics}: 
% Rich semantic information in urban settings leads to varying observations depending on distance and orientation, which can impact the accuracy of answer generation. 
Observations can vary greatly depending on distance, orientation, and perspective. For example, an object may look completely different up close than it does from afar or from different angles. These differences pose challenges for consistency and can affect the accuracy of answer generation, as embodied agents must adapt to the dynamic and complex nature of urban environments.


\begin{table}
\centering
\caption{CityEQA-EC vs existing benchmarks.}
\label{table:dataset}
\renewcommand\arraystretch{1.2}
\resizebox{\linewidth}{!}{
\begin{tabular}{cccccc}
             & Place  & Open Vocab & Active & Platform  & Reference \\ \hline
EQA-v1      & Indoor & \textcolor{red}{\ding{55}}          & \textcolor{green}{\ding{51}}      & House3D      & \cite{das2018embodied}  \\
IQUAD        & Indoor & \textcolor{red}{\ding{55}}          & \textcolor{green}{\ding{51}}      & AI2-THOR     & \cite{gordon2018iqa} \\
MP3D-EQA     & Indoor & \textcolor{red}{\ding{55}}          & \textcolor{green}{\ding{51}}      & Matterport3D & \cite{wijmans2019embodied} \\
MT-EQA       & Indoor & \textcolor{red}{\ding{55}}          & \textcolor{green}{\ding{51}}      & House3D      & \cite{yu2019multi} \\
ScanQA       & Indoor & \textcolor{red}{\ding{55}}          & \textcolor{red}{\ding{55}}      & -            & \cite{azuma2022scanqa} \\
SQA3D        & Indoor & \textcolor{red}{\ding{55}}          & \textcolor{red}{\ding{55}}      & -            & \cite{masqa3d} \\
K-EQA        & Indoor & \textcolor{green}{\ding{51}}          & \textcolor{green}{\ding{51}}      & AI2-THOR     & \cite{tan2023knowledge} \\
OpenEQA      & Indoor & \textcolor{green}{\ding{51}}          & \textcolor{green}{\ding{51}}      & ScanNet/HM3D & \cite{majumdar2024openeqa} \\
 \hline
CityEQA-EC   & City (Outdoor)  & \textcolor{green}{\ding{51}}          & \textcolor{green}{\ding{51}}      & EmbodiedCity & - \\ \hline
\end{tabular}}
\end{table}

\begin{figure*}[!htb]
\centering
    \includegraphics[width=0.78\linewidth]{figures/example.pdf}
% \vspace{-0.2cm}
\caption{The typical workflow of the PMA to address City EQA tasks. There are two cars in this area, thus a valid question must contain landmarks and spatial relationships to specify a car. Given the task, PMA will sequentially complete multiple sub-tasks to find the answer.}
% \vspace{-0.2cm}
\label{fig:example}
\end{figure*}

As an initial step toward CityEQA, we developed \textbf{CityEQA-EC}, a benchmark dataset to evaluate embodied agents' performance on CityEQA tasks. The distinctions between this dataset and other EQA benchmarks are summarized in Table \ref{table:dataset}. CityEQA-EC comprises six task types characterized by open-vocabulary questions. These tasks utilize urban landmarks and spatial relationships to delineate the expected answer, adhering to human conventions while addressing object ambiguity. This design introduces significant complexity, turning CityEQA into long-horizon tasks that require embodied agents to identify and use landmarks, explore urban environments effectively, and refine observation to generate high-quality answers.

To address CityEQA tasks, we introduce the \textbf{Planner-Manager-Actor (PMA)}, a novel baseline agent powered by large models, designed to emulate human-like rationale for solving long-horizon tasks in urban environments, as illustrated in Figure \ref{fig:example}. PMA employs a hierarchical framework to generate actions and derive answers. The Planner module parses tasks and creates plans consisting of three sub-task types: navigation, exploration, and collection. The Manager oversees the execution of these plans while maintaining a global object-centric cognitive map \cite{deng2024opengraph}. This 2D grid-based representation enables precise object identification (retrieval) and efficient management of long-term landmark information. The Actor generates specific actions based on the Manager's instructions through its components: Navigator, Explorer, and Collector. Notably, the Collector integrates a Multi-Modal Large Language Model (MM-LLM) as its Vision-Language-Action (VLA) module to refine observations and generate high-quality answers.
PMA's performance is assessed against four baselines, including humans. 
Results show that humans perform best in CityEQA, while PMA achieves 60.73\% of human accuracy in answering questions, highlighting both the challenge and validity of the proposed benchmarks. 

% The Frontier-Based Exploration (FBE) Agent, widely used in indoor EQA tasks, performs worse than even a blind LLM. This underscores the importance of PMA's hierarchical framework and its use of landmarks and spatial relationships for tackling CityEQA tasks.

In summary, this paper makes the following significant contributions:
\vspace{-8pt}
\begin{itemize}[leftmargin=*]
    \item To the best of our knowledge, we present the first open-ended embodied question answering benchmark for city space, namely CityEQA-EC.
    \vspace{-7pt}
    \item We propose a novel baseline model, PMA, which is capable of solving long-horizon tasks for CityEQA tasks with a human-like rationale.
     \vspace{-7pt}
    \item Experimental results demonstrate that our approach outperforms existing baselines in tackling the CityEQA task. However, the gap with human performance highlights opportunities for future research to improve visual thinking and reasoning in embodied intelligence for city spaces.
\end{itemize}




\section{Related Works}


\noindent\textbf{3D Point Cloud Domain Adaptation and Generalization.}
Early endeavors within 3D domain adaptation (3DDA) focused on extending 2D adversarial methodologies~\cite{qin2019pointdan} to align point cloud features. Alternative methods have delved into geometry-aware self-supervised pre-tasks. Achituve \etal~\cite{achituve2021self} introduced DefRec, a technique employing self-complement tasks by reconstructing point clouds from a non-rigid distorted version, while Zou \etal~\cite{zou2021geometry} incorporating norm curves prediction as an auxiliary task. Liang \etal~\cite{liang2022point} put forth MLSP, focusing on point estimation tasks like cardinality, position, and normal. SDDA~\cite{cardace2023self} employs self-distillation to learn the point-based features. Additionally, post-hoc self-paced training~\cite{zou2021geometry,fan2022self,park2023pcadapter} has been embraced to refine adaptation to target distributions by accessing target data and conducting further finetuning based on prior knowledge from the source domain.
In contrast, the landscape of 3D domain generalization (3DDG) research remains nascent. Metasets~\cite{huang2021metasets} leverage meta-learning to address geometric variations, while PDG~\cite{wei2022learning} decomposes 3D shapes into part-based features to enhance generalization capabilities.
Despite the remarkable progress, existing studies assume that objects in both the source and target domains share the same orientation, limiting their practical application. This limitation propels our exploration into orientation-aware 3D domain generalization through intricate orientation learning.


\noindent\textbf{Rotation-generalizable Point Cloud Analysis.}
Previous works in point cloud analysis~\cite{qi2017pointnet, wang2019dynamic} enhance rotation robustness by introducing random rotations to augment point clouds. {However, generating a comprehensive set of rotated data is impractical, resulting in variable model performance across different scenes. To robustify the networks \wrt randomly rotated point clouds,} rotation-equivariance methods explore equivalent model architectures by incorporating equivalent operations~\cite{su2022svnet, Deng_2021_ICCV, luo2022equivariant} or steerable convolutions~\cite{chen2021equivariant, poulenard2021functional}.
Alternatively, rotation-invariance approaches aim to identify geometric descriptors invariant to rotations, such as distances and angles between local points~\cite{chen2019clusternet, zhang2020global} or point norms~\cite{zhao2019rotation, li2021rotation}. Besides, {Li \etal~\cite{li2021closer} have explored disambiguating the number of PCA-based canonical poses, while Kim \etal~\cite{kim2020rotation} and Chen \etal~\cite{chen2022devil} have transformed local point coordinates according to local reference frames to maintain rotation invariance. However, these methods focus on improving in-domain rotation robustness, neglecting domain shift and consequently exhibiting limited performance when applied to diverse domains. This study addresses the challenge of cross-domain generalizability together with rotation robustness and proposes novel solutions.} 

\noindent\textbf{Intricate Sample Mining}, aimed at identifying or synthesizing challenging samples that are difficult to classify correctly, seeks to rectify the imbalance between positive and negative samples for enhancing a model's discriminability. While traditional works have explored this concept in SVM optimization~\cite{felzenszwalb2009object}, shallow neural networks~\cite{dollar2009integral}, and boosted decision trees~\cite{yu2019unsupervised}, recent advances in deep learning have catalyzed a proliferation of researches in this area across various computer vision tasks. For instance, 
Lin \etal~\cite{lin2017focal} proposed a focal loss to concentrate training efforts on a selected group of hard examples in object detection, while Yu \etal~\cite{yu2019unsupervised} devised a soft multilabel-guided hard negative mining method to learn discriminative embeddings for person Re-ID. Schroff \etal~\cite{schroff2015facenet} introduced an online negative exemplar mining process to encourage spherical clusters in face embeddings for individual recognition, and Wang \etal~\cite{wang2021instance} designed an adversarially trained negative generator to yield instance-wise negative samples, bolstering the learning of unpaired image-to-image translation. In contrast to existing studies, our work presents the first attempt to mitigate the orientational shift in 3D point cloud domain generalization, by developing an effective intricate orientation mining strategy to achieve orientation-aware learning.


\section{Methodology}

\smallskip
\noindent \textbf{Participant }\rev{\textbf{Recruitment and Demographics.}} Once receiving approval from our institutions' ethics boards, we posted an open call for participants in several AI-oriented online communities \rev{on Slack and LinkedIn}. The call invited practitioners involved in some capacity with the research, development, design, or implementation of ML to participate in in-depth qualitative interviews on how they conceptualize, identify, and handle assumptions within their work. 52 individuals responded to our call, out of which we recruited 22 respondents for remote semi-structured interviews through purposive sampling \cite{sharma2017pros}. While this may not yield a statistically representative sample, it still allowed us to explore rich and unique insights into the experiences of the participants we felt most capable of answering the research questions in our study \cite{sharma2017pros,roy2015sampling}. Those who demonstrated significant experience working on ML projects, either as developers, data scientists, or product managers, as well as individuals closely involved with responsible ML artifacts, were ultimately chosen to participate in our interviews.
\rev{Most of our participants were from Global North locations and identified as males. Table \ref{tab:demographics} provides more details about our participants.}

\begin{table*}[]
\centering
\begin{tabular}{|l|l|}
\hline
\textbf{Dimension} & \textbf{Distribution} \\ \hline
Gender & Male: 16, Female: 6 \\ \hline
Region & Global North: 18, Global South: 4 \\ \hline
Role & ML/Data Engineer: 7, ML/Data Scientist: 6,  Management: 5, \\ 
 & Others (Designer/Ethicist/Academic): 4, \\ \hline
Organization Type & Tech company: 10, NGO/Civil Society: 5, Consulting: 4, \\
& Academia: 2, Government: 1 \\ \hline
\end{tabular}
\caption{\rev{Participant Demographics}}
\label{tab:demographics}
\end{table*}

\smallskip
\noindent \textbf{Interview }\rev{\textbf{Design.}} \citet{brookfield1992uncovering} emphasizes that the key to uncovering assumptions lies in analyzing the lived experience of the assumer in order to embed a specific practice within a realistic context. This motivated how we framed our questions to be reflective, allowing participants to answer in a way that stepped outside their typical frames of reference and assess their assumptions by explicitly thinking about them. The questions were also designed to explore participants' experiences without presuming outcomes while allowing participants to refute our own underlying assumptions \cite{kvale2009interviews}. The downside of this direct approach is that unconscious assumptions---the ones that inform a participant's intuition without them being privy to their existence and persistence---may fall through. 
To remedy this, we offered a second part of the interview in which we extracted specific phrases from model documentation of three popular large language models--- PaLM 2 \cite{anil2023palm}, BLOOM \cite{le2023bloom}, and Llama 2 \cite{touvron2023llama}---and asked participants to vocally analyze them. We chose these models as they varied across different dimensions of openness \cite{liesenfeld2024rethinking}. The sample texts were selected because they most directly offered an argument that follows a typical premise-conclusion structure with deliberate non-technical language that may prompt confusion at first glance. The samples are provided in appendix \ref{appendix}.

Our approach in framing the lifecycle of an assumption by inquiring about how it is conceptualized\footnote{\rev{Some readers may wonder why we focus only on conceptualization and not consider the \textit{operationalization} of an assumption. However, in our experience and based on our interviews with practitioners, ML stakeholders do not operationalize the construct ``assumption'' in practice but operationalize only the content of a specific assumption (e.g., the usage of ``representative'' in the assumption ``this data is representative''). In this view, assumptions function at a meta-level as discussed in prior works in Critical Thinking and Informal Logic (section \ref{rel:core}), and so we focus only on the conceptualization of assumptions in this work to uncover the confusion associated with the practical use of this term in ML. We leave alternative explorations to future work.}}, then identified, then handled aligns with \citet{berman2001opening}'s breakdown of an assumption as a single entity composed of assuming, feeling, thinking, and behaving. By organizing questions through assessing \textit{functions} of assumptions rather than conveying them holistically, we are able to easily distinguish what specific elements contribute to confusion around assumptions, and how participants react to that confusion. Furthermore, following the logic that initial assumptions are likely to predicate how future assumptions are handled, we attempted to frame questions in a way that allowed us to form a narrative of a participant's assumptions.

Questions are also informed by our personal experience in ML ecosystems, aligning with established practices in \textit{reflexive} qualitative research \cite{berger2015now}. The idea of assumptions being present in technical ecosystems and the motivation for the study in assessing their influence is driven by our own observations working within the space and examining it from a critical lens derived from our past and current positions as responsible ML researchers. We make this position explicit to enhance the rigor, credibility, and trustworthiness of the study and allow readers to understand the lens through which we interpreted responses.

\smallskip
\noindent \rev{\textbf{Interview Procedure.}}
\rev{The interview guide was developed by the first and second authors and was thoroughly discussed and approved by all authors. We share our complete interview guide in Appendix \ref{interview}.
We sent our consent letters ahead of the interviews and gave our participants the option of either returning the signed letter or providing verbal consent during the interview.
All our interviews were conducted in English via Zoom. While the first and second authors conducted 9 interviews together, the first author conducted 12 interviews independently, and the second author conducted 1 independently. 
We recorded our calls upon consent and manually took notes of participants who were uncomfortable with recording. 
Our participants were given the option to exit the interviews whenever they needed. 
Our interviews lasted for 60 minutes on average. We compensated participants with 30\$ for their time and contribution.}    

\smallskip
\noindent \rev{\textbf{Data }}\textbf{Analysis.} Our virtual interviews yielded approximately 25 hours of recorded audio, paired with auto-generated transcripts from Zoom. \rev{The interview data and notes were stored in the first author's institutional cloud storage.} 
\rev{As described in our interview design, our questions were broadly framed to extract how assumptions are perceived, identified, handled, and used in practice.}
Given the nature of our more open questions, we employed interpretative and descriptive qualitative analysis \cite{merriam2019qualitative} to decipher \rev{insights} within the transcribed responses. 
\rev{The first and second authors conducted the bulk of the data analysis, and the final themes were discussed and finalized among all authors. The analysis began with multiple readings of the transcripts followed by open-coding on the transcribed data, independently and manually, by the first two authors. They then iteratively went through each other's codes manually, extracted and recorded commonalities, cross-checked with one another for reliability, and finalized the codes after resolving critical disagreements by open discussion.}

\rev{In the next phase, the codes were interpreted through the assumption argument lens (section \ref{rel:core}), mapped, categorized, and structured into themes and sub-themes over multiple iterations.} 
\rev{For instance, several sub-themes such as ``forgotton assumptions'' and ``recording style'' were grouped into one of the main themes, ``informal documentation.'' These sub-themes were created by grouping several codes that revolved around how practitioners noted down their and others' assumptions. 
Further, while some sub-themes, such as ``chained assumptions'' and ``granularity'' had overlapping codes, we categorized these sub-themes into distinct themes (elaborated in sections \ref{subsec:integrate} and \ref{subsec:doc} respectively) as it offers a better frame to understand the confusions around assumptions.
Overall, as key takeaways were found around how participants personally and professionally interacted with assumptions, we were able to form ontological distinctions, procedural inconsistencies, and other confusing elements that helped us craft clear constructions of an assumption, how workflows perpetuate unchecked assumptions, and what practitioners (can) do about it.  
Our findings in section \ref{sec:findings} reflects how we inferred and organized the main themes in our data.
}
% - Answers were categorized in themes - how participants defined and identified assumptions, what givens they had going into the ML process, how they handled assumptions, and how they responded to the case study.


% \smallskip
\noindent \textbf{Limitations.} A study about assumptions will naturally possess a few assumptions itself. First, the premise of the study requires a consensus between the authors and the participants that assumptions in an ML workflow have a significance that needs to be addressed, potentially influencing answers toward having a more proactive stance toward them. Second, the samples chosen in the case study portion of the interviews were pointers extracted from lengthier and more contextualized model documentation; our selection was informed by our own assumptions about what may elicit rich responses. The samples shown were also the same for all participants. While this provided an equal frame of reference, future works could reinforce our findings through comparing similar perceptions in more diverse samples.
\section{Experiments}\label{sec:experiments}
We conduct experiments to validate the theoretical findings on both synthetic and real tasks. In this section, we focus on two illustrative settings: synthetic regression (\Cref{sec:exp_synthetic}) and real-world image regression (\Cref{sec:exp_img_reg}). For brevity, we defer more experiments on image and sentiment classification tasks to \Cref{apx:exp_img_cls,apx:exp_nlp_cls}, respectively.

\subsection{Synthetic regression}\label{sec:exp_synthetic}
We start by grounding the theoretical framework introduced in \Cref{sec:ridgeless_regression} with synthetic regression tasks. 

\paragraph{Setup.}
We concretize the downstream task $\Dcal(f_*)$ as a regression problem over Gaussian features. 
Let $f_*: \R^d \to \R$ be a linear function in a high-dimensional feature space $d=20,000$ of form $f_*(\xb) = \xb^\top \Lambda_*^{1/2} \thetab_*$ where $\Lambda_* = \diag(\lambda^*_1, \cdots, \lambda^*_d)$ is a diagonal matrix with a low rank $d_* = 300$ such that $\lambda^*_i = i^{-1}$ for $i \le d_*$ and $\lambda^*_i = 0$ otherwise; and $\thetab_* \in \R^{d}$ is a random unit vector.
Every sample $(\xb, y) \sim \Dcal(f_*)$ is generated by $\xb \sim \Ncal(\b0_d, \Ib_d)$ and $y = f_*(\xb) + z$ with $z \sim \Ncal(0, \sigma^2)$.
Given $\xb$, the associated strong and weak features in \Cref{asm:features} are generated by $\phi_s(\xb) = \Sigmab_s^{1/2} \xb$ and $\phi_w(\xb) = \Sigmab_w^{1/2} \xb$, with intrinsic dimensions $d_s = 100$ and $d_w = 200$ such that $\Sigmab_s = \sum_{i=1}^{d_s} \lambda^*_i \eb_i \eb_i^\top$ and $\Sigmab_w = \sum_{i=d_s - d_{s \wedge w} + 1}^{d_w + d_s - d_{s \wedge w}} \lambda^*_i \eb_i \eb_i^\top$. 
For all synthetic experiments, we have $\rho_s + \rho_w < 0.0004$.

In the experiments, we vary $d_{s \wedge w}$ to control the student-teacher correlation and $\sigma^2$ to control the dominance of variance over bias (characterized by $\rho_s, \rho_w$). Each error bar reflects the standard deviation over $40$ runs. 

\begin{figure}[!ht]
    \centering
    \includegraphics[width=\columnwidth]{fig/exrisk_dsw10.pdf}%\vspace{-2em}
    \caption{Scaling for excess risks on the synthetic regression task in a \emph{variance-dominated regime} with a \emph{low correlation dimension}.}\label{fig:exrisk_dsw10}
\end{figure}

\begin{figure}[!ht]
    \centering
    \includegraphics[width=\columnwidth]{fig/exrisk_dsw90.pdf}%\vspace{-2em}
    \caption{Scaling for excess risks on the synthetic regression task in a \emph{variance-dominated regime} with a \emph{high correlation dimension}.}\label{fig:exrisk_dsw90}
\end{figure}

\begin{figure}[!ht]
    \centering
    \includegraphics[width=\columnwidth]{fig/exrisk_dsw90_biased.pdf}%\vspace{-2em}
    \caption{Scaling for excess risks on the synthetic regression task when \emph{the variance is not dominant}, $\sigma^2 \approx \rho_s + \rho_w$.}\label{fig:exrisk_dsw90_biased}
\end{figure}

\paragraph{Scaling for generalization errors.}
\Cref{fig:exrisk_dsw10,fig:exrisk_dsw90,fig:exrisk_dsw90_biased} show scaling for $\exrisk(f_\wts)$ (W2S), $\exrisk(f_w)$ (Weak), $\exrisk(f_s)$ (S-Baseline), and $\exrisk(f_c)$ (S-Ceiling) with respect to the sample sizes $n, N$. The dashes show theoretical predictions in \Cref{thm:w2s_ft,pro:sft_weak,cor:sft_strong}, consistent with the empirical measurements shown in the solid lines.
In particular, we consider three cases:
\begin{itemize}%[label=(\alph*)]
    \item \Cref{fig:exrisk_dsw10}: When variance dominates ($\sigma^2 = 0.01 \gg \rho_w + \rho_s$), with a low correlation dimension $d_{s \wedge w} = 10$, $f_\wts$ outperforms both $f_w$ and $f_s$ for a moderate $n$ and a large enough $N$. However, larger sample sizes do not necessarily lead to better W2S generalization in a relative sense. For example, when $n$ keeps increasing, the strong baseline $f_s$ eventually outperforms $f_\wts$.
    \item \Cref{fig:exrisk_dsw90}: When variance dominates, with a high correlation dimension $d_{s \wedge w} = 90$, $f_\wts$ still generalizes better than $f_w$ but fails to outperform the strong baseline $f_s$. 
    \item \Cref{fig:exrisk_dsw90_biased}: When the variance is low (not dominant, \eg $\sigma^2 = 0.0004 \approx \rho_s + \rho_w$), $f_\wts$ can fail to outperform $f_w$. This suggests that variance reduction is a key advantage of W2S over supervised FT.
\end{itemize}



\begin{figure}[!ht]
    \centering
    \includegraphics[width=\columnwidth]{fig/pgr_opr_vardom.pdf}%\vspace{-2em}
    \caption{Scaling for $\pgr$ and $\opr$ under different $d_{s \wedge w}$ on the synthetic regression task in a \emph{variance-dominated regime}.}\label{fig:pgr_opr_vardom}
\end{figure}

\paragraph{Scaling for $\pgr$ and $\opr$.}
\Cref{fig:pgr_opr_vardom} show the scaling for $\pgr$ and $\opr$ with respect to sample sizes $n, N$ in the variance-dominated regime (with small non-negligible FT approximation errors), at three different correlation dimensions $d_{s \wedge w} = 90, 50, 10$. The solid and dashed lines represent the empirical measurements and lower bounds in \eqref{eq:pgr_lower_tight}, \eqref{eq:opr_lower_tight}, respectively.
\begin{itemize}
    \item Coinciding with the theoretical predictions in \Cref{cor:non_monotonic_scaling} and the performance gaps between W2S and the references in \Cref{fig:exrisk_dsw10}, we observe that the relative W2S performance in terms of $\pgr$ and $\opr$ can degenerate as $n$ increases, while the larger $N$ generally leads to better W2S generalization in the relative sense. 
    \item The lower correlation dimension $d_{s \wedge w}$ leads to higher $\pgr$ and $\opr$, \ie larger discrepancy between the strong and weak features improves W2S generalization.
\end{itemize}



\subsection{UTKFace regression}\label{sec:exp_img_reg}
Beyond the synthetic regression, we investigate W2S on a real-world image regression task -- age estimation on the UTKFace dataset~\citep{zhang2017age}. Each error bar in this section reflects standard deviation of $10$ runs. 

\paragraph{Dataset.} 
UTKFace (Aligned \& Cropped)~\citep{zhang2017age} consists of $23,708$ face images with age labels ranging from $0$ to $116$. We preprocess the images to $224 \times 224$ pixels and split the dataset into training and testing sets of sizes $20,000$ and $3,708$.
Generalization errors are estimated with the mean squared error (MSE) over the test set. 

\paragraph{Linear probing over pretrained features.}
We fix the strong student as CLIP ViT-B/32~\citep{radford2021learning} (\texttt{CLIP-B32}) and vary the weak teacher among the ResNet series~\citep{he2015deepresiduallearningimage} (\texttt{ResNet18}, \texttt{ResNet34}, \texttt{ResNet50}, \texttt{ResNet101}, \texttt{ResNet152}). We treat the backbones of these models (excluding the classification layers) as $\phi_s,\phi_w$ and finetune them via linear probing. We use ridge regression with a small fixed regularization hyperparameter $\alpha_w, \alpha_\wts, \alpha_s, \alpha_c = 10^{-6}$, close to the machine epsilon of single precision floating point numbers.

\paragraph{Intrinsic dimension.}
The intrinsic dimensions $d_w, d_s$ are measured based on the empirical covariance matrices $\Sigmab_w, \Sigmab_s$ of the weak and strong features over the entire dataset (including training and testing).
As mentioned in \Cref{fn:ridge_regression}, these covariances generally have fast decaying eigenvalues (but not exactly low-rank) in practice, effectively leading to low intrinsic dimensions under ridge regression. We estimate such low intrinsic dimensions as the minimum rank for the best low-rank approximation of $\Sigmab_w, \Sigmab_s$ with a relative error in trace less than $\tau=0.01$.

\paragraph{Correlation dimension.}
The pretrained feature dimensions (or the finetunable parameter counts) of the weak and strong models can be different in practice (see \Cref{apx:exp_img_reg}, \Cref{tab:img_reg_dim}). 
We introduce an estimation for $d_{s \wedge w}$ in this case.
Consider the (truncated) spectral decompositions $\tsvd{\Sigmab_s}{d_s} = \Vb_s \Lambdab_s \Vb_s^\top$ and $\tsvd{\Sigmab_w}{d_w} = \Vb_w \Lambdab_w \Vb_w^\top$ of two empirical covariances with different feature dimensions $D_s, D_w$ such that $\Vb_s \in \R^{D_s \times d_s}$ and $\Vb_w \in \R^{D_w \times d_w}$ consists of the top $d_s, d_w$ orthonormal eigenvectors, respectively. We estimate the correlation dimension $d_{s \wedge w}$ under different feature dimensions $D_s \ne D_w$ by matching the dimensions through a random unitary matrix~\citep{vershynin2018high} $\Gammab \in \R^{D_s \times D_w}$: $d_{s \wedge w} = \|\Vb_s^\top \Gammab \Vb_w\|_F^2$. This provides a good estimation for $d_{s \wedge w}$ because with low intrinsic dimensions $\max\{d_s, d_w\} \ll D_s, D_w$ in practice, mild dimension reduction through $\Gammab$ well preserves the essential information in $\Vb_s, \Vb_w$.

\begin{figure}[!ht]
    \centering
    \includegraphics[width=\columnwidth]{fig/pgr_opr_utkface_resnet-clip.pdf}%\vspace{-2em}
    \caption{Scaling for $\pgr$ and $\opr$ of different weak teachers with a fixed strong student on UTKFace. The legends show the comparison between $d_{s \wedge w}$ and $d_w$.}\label{fig:pgr_opr_utkface_resnet-clip}
\end{figure}

\paragraph{Discrepancies lead to better W2S.}
\Cref{fig:pgr_opr_utkface_resnet-clip} shows the scaling of $\pgr$ and $\opr$ with respect to the sample sizes $n, N$ for different weak teachers in the ResNet series with respect to a fixed student, \texttt{CLIP-B32}. 
We first observe that the relative W2S performance in terms of $\pgr$ and $\opr$ is closely related to the correlation dimension $d_{s \wedge w}$ and the intrinsic dimensions $d_s, d_w$. 
\begin{itemize}
    \item When the strong student has a lower intrinsic dimension than the weak teacher (as widely observed in practice~\citep{aghajanyan2020intrinsic}), \ie $d_s < d_w$, the relative W2S performance tends to be better than when $d_s > d_w$.
    \item The relative W2S performance tends to be better when $d_{s \wedge w}/d_w$ is lower, \ie the larger discrepancy between weak and strong features leads to better W2S generalization.
\end{itemize}
Meanwhile, both $\pgr$ and $\opr$ scale inversely with the labeled sample size $n$ and exhibit diminishing return with respect to the increasing pseudolabel size $N$, consistent with the theoretical predictions in \Cref{cor:non_monotonic_scaling} and the synthetic experiments in \Cref{fig:pgr_opr_vardom}.

\begin{figure}[!ht]
    \centering
    \includegraphics[width=\columnwidth]{fig/pgr_opr_utkface_vardom_resnet-clip.pdf}%\vspace{-2em}
    \caption{Scaling for $\pgr$ and $\opr$ on UTKFace with injected label noise: $y_i \gets y_i + \zeta_i$ where $\zeta_i \sim \Ncal(0, \varsigma^2)~\iid$.}\label{fig:pgr_opr_utkface_vardom_resnet-clip}
\end{figure}

\paragraph{Variance reduction is a key advantage of W2S.}
To investigate the impact of variance on W2S generalization, we inject noise to the training label by $y_i \gets y_i + \zeta_i$ where $\zeta_i \sim \Ncal(0, \varsigma^2)~\iid$, and $\varsigma$ controls the injected labels noise level.
In \Cref{fig:pgr_opr_utkface_vardom_resnet-clip}, we show the scaling for $\pgr$ and $\opr$ with respect to the sample sizes $n, N$ under different noise levels $\varsigma$. We observe that the relative W2S performance in terms of $\pgr$ and $\opr$ improves as the noise level $\varsigma$ increases. This provides empirical evidence that variance reduction is a key advantage of W2S over supervised FT, highlighting the importance of understanding the mechanisms of W2S in the variance-dominated regime.


\section{Conclusion}\label{sec:conclusion}

In this study, we propose M2-omni, a highly competitive omni-MLLM model to GPT-4o, characterized by its comprehensive modality and task support, as well as its exceptional performance. M2-omni demonstrates competitive performance across a diverse range of tasks, including image understanding, video understanding, interleaved image-text understanding, audio understanding and generation, as well as free-form image generation. We employ a multi-stage training approach to train M2-omni, which enables progressive modality alignment. To address the challenge of maintaining consistent performance across all modalities, we propose a step-wise balance strategy for pretraining and a dynamically adaptive balance strategy for instruction tuning, which can effectively mitigate the impact of significant variations in data volume and convergence rates across heterogeneous multimodal tasks. We publicly release M2-omni, along with its comprehensive training details, including data configurations and training procedures, to facilitate future research in this domain.



% {\small
% \bibliographystyle{plainnat}
% \bibliography{ref}
% }


{
%\fontsize{8.2pt}{9.84pt}\selectfont
\small
% \bibliographystyle{plain}
\bibliographystyle{plainnat}
\bibliography{ref}
}





\iffalse

\begin{abstract}
This document describes the most common article elements and how to use the IEEEtran class with \LaTeX \ to produce files that are suitable for submission to the IEEE.  IEEEtran can produce conference, journal, and technical note (correspondence) papers with a suitable choice of class options. 
\end{abstract}

\begin{IEEEkeywords}
Article submission, IEEE, IEEEtran, journal, \LaTeX, paper, template, typesetting.
\end{IEEEkeywords}

\section{Introduction}
\IEEEPARstart{T}{his} file is intended to serve as a ``sample article file''
for IEEE journal papers produced under \LaTeX\ using
IEEEtran.cls version 1.8b and later. The most common elements are covered in the simplified and updated instructions in ``New\_IEEEtran\_how-to.pdf''. For less common elements you can refer back to the original ``IEEEtran\_HOWTO.pdf''. It is assumed that the reader has a basic working knowledge of \LaTeX. Those who are new to \LaTeX \ are encouraged to read Tobias Oetiker's ``The Not So Short Introduction to \LaTeX ,'' available at: \url{http://tug.ctan.org/info/lshort/english/lshort.pdf} which provides an overview of working with \LaTeX.

\section{The Design, Intent, and \\ Limitations of the Templates}
The templates are intended to {\bf{approximate the final look and page length of the articles/papers}}. {\bf{They are NOT intended to be the final produced work that is displayed in print or on IEEEXplore\textsuperscript{\textregistered}}}. They will help to give the authors an approximation of the number of pages that will be in the final version. The structure of the \LaTeX\ files, as designed, enable easy conversion to XML for the composition systems used by the IEEE. The XML files are used to produce the final print/IEEEXplore pdf and then converted to HTML for IEEEXplore.

\section{Where to Get \LaTeX \ Help --- User Groups}
The following online groups are helpful to beginning and experienced \LaTeX\ users. A search through their archives can provide many answers to common questions.
\begin{list}{}{}
\item{\url{http://www.latex-community.org/}} 
\item{\url{https://tex.stackexchange.com/} }
\end{list}

\section{Other Resources}
See \cite{ref1,ref2,ref3,ref4,ref5} for resources on formatting math into text and additional help in working with \LaTeX .

\section{Text}
For some of the remainer of this sample we will use dummy text to fill out paragraphs rather than use live text that may violate a copyright.

Itam, que ipiti sum dem velit la sum et dionet quatibus apitet voloritet audam, qui aliciant voloreicid quaspe volorem ut maximusandit faccum conemporerum aut ellatur, nobis arcimus.
Fugit odi ut pliquia incitium latum que cusapere perit molupta eaquaeria quod ut optatem poreiur? Quiaerr ovitior suntiant litio bearciur?

Onseque sequaes rectur autate minullore nusae nestiberum, sum voluptatio. Et ratem sequiam quaspername nos rem repudandae volum consequis nos eium aut as molupta tectum ulparumquam ut maximillesti consequas quas inctia cum volectinusa porrum unt eius cusaest exeritatur? Nias es enist fugit pa vollum reium essusam nist et pa aceaqui quo elibusdandis deligendus que nullaci lloreri bla que sa coreriam explacc atiumquos simolorpore, non prehendunt lam que occum\cite{ref6} si aut aut maximus eliaeruntia dia sequiamenime natem sendae ipidemp orehend uciisi omnienetus most verum, ommolendi omnimus, est, veni aut ipsa volendelist mo conserum volores estisciis recessi nveles ut poressitatur sitiis ex endi diti volum dolupta aut aut odi as eatquo cullabo remquis toreptum et des accus dolende pores sequas dolores tinust quas expel moditae ne sum quiatis nis endipie nihilis etum fugiae audi dia quiasit quibus.
\IEEEpubidadjcol
Ibus el et quatemo luptatque doluptaest et pe volent rem ipidusa eribus utem venimolorae dera qui acea quam etur aceruptat.
Gias anis doluptaspic tem et aliquis alique inctiuntiur?

Sedigent, si aligend elibuscid ut et ium volo tem eictore pellore ritatus ut ut ullatus in con con pere nos ab ium di tem aliqui od magnit repta volectur suntio. Nam isquiante doluptis essit, ut eos suntionsecto debitiur sum ea ipitiis adipit, oditiore, a dolorerempos aut harum ius, atquat.

Rum rem ditinti sciendunti volupiciendi sequiae nonsect oreniatur, volores sition ressimil inus solut ea volum harumqui to see\eqref{deqn_ex1a} mint aut quat eos explis ad quodi debis deliqui aspel earcius.

\begin{equation}
\label{deqn_ex1a}
x = \sum_{i=0}^{n} 2{i} Q.
\end{equation}

Alis nime volorempera perferi sitio denim repudae pre ducilit atatet volecte ssimillorae dolore, ut pel ipsa nonsequiam in re nus maiost et que dolor sunt eturita tibusanis eatent a aut et dio blaudit reptibu scipitem liquia consequodi od unto ipsae. Et enitia vel et experferum quiat harum sa net faccae dolut voloria nem. Bus ut labo. Ita eum repraer rovitia samendit aut et volupta tecupti busant omni quiae porro que nossimodic temquis anto blacita conse nis am, que ereperum eumquam quaescil imenisci quae magnimos recus ilibeaque cum etum iliate prae parumquatemo blaceaquiam quundia dit apienditem rerit re eici quaes eos sinvers pelecabo. Namendignis as exerupit aut magnim ium illabor roratecte plic tem res apiscipsam et vernat untur a deliquaest que non cus eat ea dolupiducim fugiam volum hil ius dolo eaquis sitis aut landesto quo corerest et auditaquas ditae voloribus, qui optaspis exero cusa am, ut plibus.


\section{Some Common Elements}
\subsection{Sections and Subsections}
Enumeration of section headings is desirable, but not required. When numbered, please be consistent throughout the article, that is, all headings and all levels of section headings in the article should be enumerated. Primary headings are designated with Roman numerals, secondary with capital letters, tertiary with Arabic numbers; and quaternary with lowercase letters. Reference and Acknowledgment headings are unlike all other section headings in text. They are never enumerated. They are simply primary headings without labels, regardless of whether the other headings in the article are enumerated. 

\subsection{Citations to the Bibliography}
The coding for the citations is made with the \LaTeX\ $\backslash${\tt{cite}} command. 
This will display as: see \cite{ref1}.

For multiple citations code as follows: {\tt{$\backslash$cite\{ref1,ref2,ref3\}}}
 which will produce \cite{ref1,ref2,ref3}. For reference ranges that are not consecutive code as {\tt{$\backslash$cite\{ref1,ref2,ref3,ref9\}}} which will produce  \cite{ref1,ref2,ref3,ref9}

\subsection{Lists}
In this section, we will consider three types of lists: simple unnumbered, numbered, and bulleted. There have been many options added to IEEEtran to enhance the creation of lists. If your lists are more complex than those shown below, please refer to the original ``IEEEtran\_HOWTO.pdf'' for additional options.\\

\subsubsection*{\bf A plain  unnumbered list}
\begin{list}{}{}
\item{bare\_jrnl.tex}
\item{bare\_conf.tex}
\item{bare\_jrnl\_compsoc.tex}
\item{bare\_conf\_compsoc.tex}
\item{bare\_jrnl\_comsoc.tex}
\end{list}

\subsubsection*{\bf A simple numbered list}
\begin{enumerate}
\item{bare\_jrnl.tex}
\item{bare\_conf.tex}
\item{bare\_jrnl\_compsoc.tex}
\item{bare\_conf\_compsoc.tex}
\item{bare\_jrnl\_comsoc.tex}
\end{enumerate}

\subsubsection*{\bf A simple bulleted list}
\begin{itemize}
\item{bare\_jrnl.tex}
\item{bare\_conf.tex}
\item{bare\_jrnl\_compsoc.tex}
\item{bare\_conf\_compsoc.tex}
\item{bare\_jrnl\_comsoc.tex}
\end{itemize}





\subsection{Figures}
Fig. 1 is an example of a floating figure using the graphicx package.
 Note that $\backslash${\tt{label}} must occur AFTER (or within) $\backslash${\tt{caption}}.
 For figures, $\backslash${\tt{caption}} should occur after the $\backslash${\tt{includegraphics}}.

\begin{figure}[!t]
\centering
\includegraphics[width=2.5in]{fig1}
\caption{Simulation results for the network.}
\label{fig_1}
\end{figure}

Fig. 2(a) and 2(b) is an example of a double column floating figure using two subfigures.
 (The subfig.sty package must be loaded for this to work.)
 The subfigure $\backslash${\tt{label}} commands are set within each subfloat command,
 and the $\backslash${\tt{label}} for the overall figure must come after $\backslash${\tt{caption}}.
 $\backslash${\tt{hfil}} is used as a separator to get equal spacing.
 The combined width of all the parts of the figure should do not exceed the text width or a line break will occur.
%
\begin{figure*}[!t]
\centering
\subfloat[]{\includegraphics[width=2.5in]{fig1}%
\label{fig_first_case}}
\hfil
\subfloat[]{\includegraphics[width=2.5in]{fig1}%
\label{fig_second_case}}
\caption{Dae. Ad quatur autat ut porepel itemoles dolor autem fuga. Bus quia con nessunti as remo di quatus non perum que nimus. (a) Case I. (b) Case II.}
\label{fig_sim}
\end{figure*}

Note that often IEEE papers with multi-part figures do not place the labels within the image itself (using the optional argument to $\backslash${\tt{subfloat}}[]), but instead will
 reference/describe all of them (a), (b), etc., within the main caption.
 Be aware that for subfig.sty to generate the (a), (b), etc., subfigure
 labels, the optional argument to $\backslash${\tt{subfloat}} must be present. If a
 subcaption is not desired, leave its contents blank,
 e.g.,$\backslash${\tt{subfloat}}[].


 

\section{Tables}
Note that, for IEEE-style tables, the
 $\backslash${\tt{caption}} command should come BEFORE the table. Table captions use title case. Articles (a, an, the), coordinating conjunctions (and, but, for, or, nor), and most short prepositions are lowercase unless they are the first or last word. Table text will default to $\backslash${\tt{footnotesize}} as
 the IEEE normally uses this smaller font for tables.
 The $\backslash${\tt{label}} must come after $\backslash${\tt{caption}} as always.
 
\begin{table}[!t]
\caption{An Example of a Table\label{tab:table1}}
\centering
\begin{tabular}{|c||c|}
\hline
One & Two\\
\hline
Three & Four\\
\hline
\end{tabular}
\end{table}

\section{Algorithms}
Algorithms should be numbered and include a short title. They are set off from the text with rules above and below the title and after the last line.

\begin{algorithm}[H]
\caption{Weighted Tanimoto ELM.}\label{alg:alg1}
\begin{algorithmic}
\STATE 
\STATE {\textsc{TRAIN}}$(\mathbf{X} \mathbf{T})$
\STATE \hspace{0.5cm}$ \textbf{select randomly } W \subset \mathbf{X}  $
\STATE \hspace{0.5cm}$ N_\mathbf{t} \gets | \{ i : \mathbf{t}_i = \mathbf{t} \} | $ \textbf{ for } $ \mathbf{t}= -1,+1 $
\STATE \hspace{0.5cm}$ B_i \gets \sqrt{ \textsc{max}(N_{-1},N_{+1}) / N_{\mathbf{t}_i} } $ \textbf{ for } $ i = 1,...,N $
\STATE \hspace{0.5cm}$ \hat{\mathbf{H}} \gets  B \cdot (\mathbf{X}^T\textbf{W})/( \mathbb{1}\mathbf{X} + \mathbb{1}\textbf{W} - \mathbf{X}^T\textbf{W} ) $
\STATE \hspace{0.5cm}$ \beta \gets \left ( I/C + \hat{\mathbf{H}}^T\hat{\mathbf{H}} \right )^{-1}(\hat{\mathbf{H}}^T B\cdot \mathbf{T})  $
\STATE \hspace{0.5cm}\textbf{return}  $\textbf{W},  \beta $
\STATE 
\STATE {\textsc{PREDICT}}$(\mathbf{X} )$
\STATE \hspace{0.5cm}$ \mathbf{H} \gets  (\mathbf{X}^T\textbf{W} )/( \mathbb{1}\mathbf{X}  + \mathbb{1}\textbf{W}- \mathbf{X}^T\textbf{W}  ) $
\STATE \hspace{0.5cm}\textbf{return}  $\textsc{sign}( \mathbf{H} \beta )$
\end{algorithmic}
\label{alg1}
\end{algorithm}

Que sunt eum lam eos si dic to estist, culluptium quid qui nestrum nobis reiumquiatur minimus minctem. Ro moluptat fuga. Itatquiam ut laborpo rersped exceres vollandi repudaerem. Ulparci sunt, qui doluptaquis sumquia ndestiu sapient iorepella sunti veribus. Ro moluptat fuga. Itatquiam ut laborpo rersped exceres vollandi repudaerem. 
\section{Mathematical Typography \\ and Why It Matters}

Typographical conventions for mathematical formulas have been developed to {\bf provide uniformity and clarity of presentation across mathematical texts}. This enables the readers of those texts to both understand the author's ideas and to grasp new concepts quickly. While software such as \LaTeX \ and MathType\textsuperscript{\textregistered} can produce aesthetically pleasing math when used properly, it is also very easy to misuse the software, potentially resulting in incorrect math display.

IEEE aims to provide authors with the proper guidance on mathematical typesetting style and assist them in writing the best possible article. As such, IEEE has assembled a set of examples of good and bad mathematical typesetting \cite{ref1,ref2,ref3,ref4,ref5}. 

Further examples can be found at \url{http://journals.ieeeauthorcenter.ieee.org/wp-content/uploads/sites/7/IEEE-Math-Typesetting-Guide-for-LaTeX-Users.pdf}

\subsection{Display Equations}
The simple display equation example shown below uses the ``equation'' environment. To number the equations, use the $\backslash${\tt{label}} macro to create an identifier for the equation. LaTeX will automatically number the equation for you.
\begin{equation}
\label{deqn_ex1}
x = \sum_{i=0}^{n} 2{i} Q.
\end{equation}

\noindent is coded as follows:
\begin{verbatim}
\begin{equation}
\label{deqn_ex1}
x = \sum_{i=0}^{n} 2{i} Q.
\end{equation}
\end{verbatim}

To reference this equation in the text use the $\backslash${\tt{ref}} macro. 
Please see (\ref{deqn_ex1})\\
\noindent is coded as follows:
\begin{verbatim}
Please see (\ref{deqn_ex1})\end{verbatim}

\subsection{Equation Numbering}
{\bf{Consecutive Numbering:}} Equations within an article are numbered consecutively from the beginning of the
article to the end, i.e., (1), (2), (3), (4), (5), etc. Do not use roman numerals or section numbers for equation numbering.

\noindent {\bf{Appendix Equations:}} The continuation of consecutively numbered equations is best in the Appendix, but numbering
 as (A1), (A2), etc., is permissible.\\

\noindent {\bf{Hyphens and Periods}}: Hyphens and periods should not be used in equation numbers, i.e., use (1a) rather than
(1-a) and (2a) rather than (2.a) for subequations. This should be consistent throughout the article.

\subsection{Multi-Line Equations and Alignment}
Here we show several examples of multi-line equations and proper alignments.

\noindent {\bf{A single equation that must break over multiple lines due to length with no specific alignment.}}
\begin{multline}
\text{The first line of this example}\\
\text{The second line of this example}\\
\text{The third line of this example}
\end{multline}

\noindent is coded as:
\begin{verbatim}
\begin{multline}
\text{The first line of this example}\\
\text{The second line of this example}\\
\text{The third line of this example}
\end{multline}
\end{verbatim}

\noindent {\bf{A single equation with multiple lines aligned at the = signs}}
\begin{align}
a &= c+d \\
b &= e+f
\end{align}
\noindent is coded as:
\begin{verbatim}
\begin{align}
a &= c+d \\
b &= e+f
\end{align}
\end{verbatim}

The {\tt{align}} environment can align on multiple  points as shown in the following example:
\begin{align}
x &= y & X & =Y & a &=bc\\
x' &= y' & X' &=Y' &a' &=bz
\end{align}
\noindent is coded as:
\begin{verbatim}
\begin{align}
x &= y & X & =Y & a &=bc\\
x' &= y' & X' &=Y' &a' &=bz
\end{align}
\end{verbatim}





\subsection{Subnumbering}
The amsmath package provides a {\tt{subequations}} environment to facilitate subnumbering. An example:

\begin{subequations}\label{eq:2}
\begin{align}
f&=g \label{eq:2A}\\
f' &=g' \label{eq:2B}\\
\mathcal{L}f &= \mathcal{L}g \label{eq:2c}
\end{align}
\end{subequations}

\noindent is coded as:
\begin{verbatim}
\begin{subequations}\label{eq:2}
\begin{align}
f&=g \label{eq:2A}\\
f' &=g' \label{eq:2B}\\
\mathcal{L}f &= \mathcal{L}g \label{eq:2c}
\end{align}
\end{subequations}

\end{verbatim}

\subsection{Matrices}
There are several useful matrix environments that can save you some keystrokes. See the example coding below and the output.

\noindent {\bf{A simple matrix:}}
\begin{equation}
\begin{matrix}  0 &  1 \\ 
1 &  0 \end{matrix}
\end{equation}
is coded as:
\begin{verbatim}
\begin{equation}
\begin{matrix}  0 &  1 \\ 
1 &  0 \end{matrix}
\end{equation}
\end{verbatim}

\noindent {\bf{A matrix with parenthesis}}
\begin{equation}
\begin{pmatrix} 0 & -i \\
 i &  0 \end{pmatrix}
\end{equation}
is coded as:
\begin{verbatim}
\begin{equation}
\begin{pmatrix} 0 & -i \\
 i &  0 \end{pmatrix}
\end{equation}
\end{verbatim}

\noindent {\bf{A matrix with square brackets}}
\begin{equation}
\begin{bmatrix} 0 & -1 \\ 
1 &  0 \end{bmatrix}
\end{equation}
is coded as:
\begin{verbatim}
\begin{equation}
\begin{bmatrix} 0 & -1 \\ 
1 &  0 \end{bmatrix}
\end{equation}
\end{verbatim}

\noindent {\bf{A matrix with curly braces}}
\begin{equation}
\begin{Bmatrix} 1 &  0 \\ 
0 & -1 \end{Bmatrix}
\end{equation}
is coded as:
\begin{verbatim}
\begin{equation}
\begin{Bmatrix} 1 &  0 \\ 
0 & -1 \end{Bmatrix}
\end{equation}\end{verbatim}

\noindent {\bf{A matrix with single verticals}}
\begin{equation}
\begin{vmatrix} a &  b \\ 
c &  d \end{vmatrix}
\end{equation}
is coded as:
\begin{verbatim}
\begin{equation}
\begin{vmatrix} a &  b \\ 
c &  d \end{vmatrix}
\end{equation}\end{verbatim}

\noindent {\bf{A matrix with double verticals}}
\begin{equation}
\begin{Vmatrix} i &  0 \\ 
0 & -i \end{Vmatrix}
\end{equation}
is coded as:
\begin{verbatim}
\begin{equation}
\begin{Vmatrix} i &  0 \\ 
0 & -i \end{Vmatrix}
\end{equation}\end{verbatim}

\subsection{Arrays}
The {\tt{array}} environment allows you some options for matrix-like equations. You will have to manually key the fences, but there are other options for alignment of the columns and for setting horizontal and vertical rules. The argument to {\tt{array}} controls alignment and placement of vertical rules.

A simple array
\begin{equation}
\left(
\begin{array}{cccc}
a+b+c & uv & x-y & 27\\
a+b & u+v & z & 134
\end{array}\right)
\end{equation}
is coded as:
\begin{verbatim}
\begin{equation}
\left(
\begin{array}{cccc}
a+b+c & uv & x-y & 27\\
a+b & u+v & z & 134
\end{array} \right)
\end{equation}
\end{verbatim}

A slight variation on this to better align the numbers in the last column
\begin{equation}
\left(
\begin{array}{cccr}
a+b+c & uv & x-y & 27\\
a+b & u+v & z & 134
\end{array}\right)
\end{equation}
is coded as:
\begin{verbatim}
\begin{equation}
\left(
\begin{array}{cccr}
a+b+c & uv & x-y & 27\\
a+b & u+v & z & 134
\end{array} \right)
\end{equation}
\end{verbatim}

An array with vertical and horizontal rules
\begin{equation}
\left( \begin{array}{c|c|c|r}
a+b+c & uv & x-y & 27\\ \hline
a+b & u+v & z & 134
\end{array}\right)
\end{equation}
is coded as:
\begin{verbatim}
\begin{equation}
\left(
\begin{array}{c|c|c|r}
a+b+c & uv & x-y & 27\\
a+b & u+v & z & 134
\end{array} \right)
\end{equation}
\end{verbatim}
Note the argument now has the pipe "$\vert$" included to indicate the placement of the vertical rules.


\subsection{Cases Structures}
Many times cases can be miscoded using the wrong environment, i.e., {\tt{array}}. Using the {\tt{cases}} environment will save keystrokes (from not having to type the $\backslash${\tt{left}}$\backslash${\tt{lbrace}}) and automatically provide the correct column alignment.
\begin{equation*}
{z_m(t)} = \begin{cases}
1,&{\text{if}}\ {\beta }_m(t) \\ 
{0,}&{\text{otherwise.}} 
\end{cases}
\end{equation*}
\noindent is coded as follows:
\begin{verbatim}
\begin{equation*}
{z_m(t)} = 
\begin{cases}
1,&{\text{if}}\ {\beta }_m(t),\\ 
{0,}&{\text{otherwise.}} 
\end{cases}
\end{equation*}
\end{verbatim}
\noindent Note that the ``\&'' is used to mark the tabular alignment. This is important to get  proper column alignment. Do not use $\backslash${\tt{quad}} or other fixed spaces to try and align the columns. Also, note the use of the $\backslash${\tt{text}} macro for text elements such as ``if'' and ``otherwise.''

\subsection{Function Formatting in Equations}
Often, there is an easy way to properly format most common functions. Use of the $\backslash$ in front of the function name will in most cases, provide the correct formatting. When this does not work, the following example provides a solution using the $\backslash${\tt{text}} macro:

\begin{equation*} 
  d_{R}^{KM} = \underset {d_{l}^{KM}} {\text{arg min}} \{ d_{1}^{KM},\ldots,d_{6}^{KM}\}.
\end{equation*}

\noindent is coded as follows:
\begin{verbatim}
\begin{equation*} 
 d_{R}^{KM} = \underset {d_{l}^{KM}} 
 {\text{arg min}} \{ d_{1}^{KM},
 \ldots,d_{6}^{KM}\}.
\end{equation*}
\end{verbatim}

\subsection{ Text Acronyms Inside Equations}
This example shows where the acronym ``MSE" is coded using $\backslash${\tt{text\{\}}} to match how it appears in the text.

\begin{equation*}
 \text{MSE} = \frac {1}{n}\sum _{i=1}^{n}(Y_{i} - \hat {Y_{i}})^{2}
\end{equation*}

\begin{verbatim}
\begin{equation*}
 \text{MSE} = \frac {1}{n}\sum _{i=1}^{n}
(Y_{i} - \hat {Y_{i}})^{2}
\end{equation*}
\end{verbatim}

\section{Conclusion}
The conclusion goes here.


\section*{Acknowledgments}
This should be a simple paragraph before the References to thank those individuals and institutions who have supported your work on this article.



{\appendix[Proof of the Zonklar Equations]
Use $\backslash${\tt{appendix}} if you have a single appendix:
Do not use $\backslash${\tt{section}} anymore after $\backslash${\tt{appendix}}, only $\backslash${\tt{section*}}.
If you have multiple appendixes use $\backslash${\tt{appendices}} then use $\backslash${\tt{section}} to start each appendix.
You must declare a $\backslash${\tt{section}} before using any $\backslash${\tt{subsection}} or using $\backslash${\tt{label}} ($\backslash${\tt{appendices}} by itself
 starts a section numbered zero.)}



%{\appendices
%\section*{Proof of the First Zonklar Equation}
%Appendix one text goes here.
% You can choose not to have a title for an appendix if you want by leaving the argument blank
%\section*{Proof of the Second Zonklar Equation}
%Appendix two text goes here.}



\section{References Section}
You can use a bibliography generated by BibTeX as a .bbl file.
 BibTeX documentation can be easily obtained at:
 http://mirror.ctan.org/biblio/bibtex/contrib/doc/
 The IEEEtran BibTeX style support page is:
 http://www.michaelshell.org/tex/ieeetran/bibtex/
 
 % argument is your BibTeX string definitions and bibliography database(s)
%\bibliography{IEEEabrv,../bib/paper}
%
\section{Simple References}
You can manually copy in the resultant .bbl file and set second argument of $\backslash${\tt{begin}} to the number of references
 (used to reserve space for the reference number labels box).

\begin{thebibliography}{1}
\bibliographystyle{IEEEtran}

\bibitem{ref1}
{\it{Mathematics Into Type}}. American Mathematical Society. [Online]. Available: https://www.ams.org/arc/styleguide/mit-2.pdf

\bibitem{ref2}
T. W. Chaundy, P. R. Barrett and C. Batey, {\it{The Printing of Mathematics}}. London, U.K., Oxford Univ. Press, 1954.

\bibitem{ref3}
F. Mittelbach and M. Goossens, {\it{The \LaTeX Companion}}, 2nd ed. Boston, MA, USA: Pearson, 2004.

\bibitem{ref4}
G. Gr\"atzer, {\it{More Math Into LaTeX}}, New York, NY, USA: Springer, 2007.

\bibitem{ref5}M. Letourneau and J. W. Sharp, {\it{AMS-StyleGuide-online.pdf,}} American Mathematical Society, Providence, RI, USA, [Online]. Available: http://www.ams.org/arc/styleguide/index.html

\bibitem{ref6}
H. Sira-Ramirez, ``On the sliding mode control of nonlinear systems,'' \textit{Syst. Control Lett.}, vol. 19, pp. 303--312, 1992.

\bibitem{ref7}
A. Levant, ``Exact differentiation of signals with unbounded higher derivatives,''  in \textit{Proc. 45th IEEE Conf. Decis.
Control}, San Diego, CA, USA, 2006, pp. 5585--5590. DOI: 10.1109/CDC.2006.377165.

\bibitem{ref8}
M. Fliess, C. Join, and H. Sira-Ramirez, ``Non-linear estimation is easy,'' \textit{Int. J. Model., Ident. Control}, vol. 4, no. 1, pp. 12--27, 2008.

\bibitem{ref9}
R. Ortega, A. Astolfi, G. Bastin, and H. Rodriguez, ``Stabilization of food-chain systems using a port-controlled Hamiltonian description,'' in \textit{Proc. Amer. Control Conf.}, Chicago, IL, USA,
2000, pp. 2245--2249.

\end{thebibliography}


% \newpage

\section{Biography Section}
If you have an EPS/PDF photo (graphicx package needed), extra braces are
 needed around the contents of the optional argument to biography to prevent
 the LaTeX parser from getting confused when it sees the complicated
 $\backslash${\tt{includegraphics}} command within an optional argument. (You can create
 your own custom macro containing the $\backslash${\tt{includegraphics}} command to make things
 simpler here.)
 
\vspace{11pt}

\bf{If you include a photo:}\vspace{-33pt}
\begin{IEEEbiography}[{\includegraphics[width=1in,height=1.25in,clip,keepaspectratio]{fig1}}]{Michael Shell}
Use $\backslash${\tt{begin\{IEEEbiography\}}} and then for the 1st argument use $\backslash${\tt{includegraphics}} to declare and link the author photo.
Use the author name as the 3rd argument followed by the biography text.
\end{IEEEbiography}

\vspace{11pt}

\bf{If you will not include a photo:}\vspace{-33pt}
\begin{IEEEbiographynophoto}{John Doe}
Use $\backslash${\tt{begin\{IEEEbiographynophoto\}}} and the author name as the argument followed by the biography text.
\end{IEEEbiographynophoto}
\vfill

\fi

\end{document}


