\documentclass{article} % For LaTeX2e
\usepackage{iclr2025_conference,times}

% Optional math commands from https://github.com/goodfeli/dlbook_notation.
%%%%% NEW MATH DEFINITIONS %%%%%

\usepackage{amsmath,amsfonts,bm}
\usepackage{derivative}
% Mark sections of captions for referring to divisions of figures
\newcommand{\figleft}{{\em (Left)}}
\newcommand{\figcenter}{{\em (Center)}}
\newcommand{\figright}{{\em (Right)}}
\newcommand{\figtop}{{\em (Top)}}
\newcommand{\figbottom}{{\em (Bottom)}}
\newcommand{\captiona}{{\em (a)}}
\newcommand{\captionb}{{\em (b)}}
\newcommand{\captionc}{{\em (c)}}
\newcommand{\captiond}{{\em (d)}}

% Highlight a newly defined term
\newcommand{\newterm}[1]{{\bf #1}}

% Derivative d 
\newcommand{\deriv}{{\mathrm{d}}}

% Figure reference, lower-case.
\def\figref#1{figure~\ref{#1}}
% Figure reference, capital. For start of sentence
\def\Figref#1{Figure~\ref{#1}}
\def\twofigref#1#2{figures \ref{#1} and \ref{#2}}
\def\quadfigref#1#2#3#4{figures \ref{#1}, \ref{#2}, \ref{#3} and \ref{#4}}
% Section reference, lower-case.
\def\secref#1{section~\ref{#1}}
% Section reference, capital.
\def\Secref#1{Section~\ref{#1}}
% Reference to two sections.
\def\twosecrefs#1#2{sections \ref{#1} and \ref{#2}}
% Reference to three sections.
\def\secrefs#1#2#3{sections \ref{#1}, \ref{#2} and \ref{#3}}
% Reference to an equation, lower-case.
\def\eqref#1{equation~\ref{#1}}
% Reference to an equation, upper case
\def\Eqref#1{Equation~\ref{#1}}
% A raw reference to an equation---avoid using if possible
\def\plaineqref#1{\ref{#1}}
% Reference to a chapter, lower-case.
\def\chapref#1{chapter~\ref{#1}}
% Reference to an equation, upper case.
\def\Chapref#1{Chapter~\ref{#1}}
% Reference to a range of chapters
\def\rangechapref#1#2{chapters\ref{#1}--\ref{#2}}
% Reference to an algorithm, lower-case.
\def\algref#1{algorithm~\ref{#1}}
% Reference to an algorithm, upper case.
\def\Algref#1{Algorithm~\ref{#1}}
\def\twoalgref#1#2{algorithms \ref{#1} and \ref{#2}}
\def\Twoalgref#1#2{Algorithms \ref{#1} and \ref{#2}}
% Reference to a part, lower case
\def\partref#1{part~\ref{#1}}
% Reference to a part, upper case
\def\Partref#1{Part~\ref{#1}}
\def\twopartref#1#2{parts \ref{#1} and \ref{#2}}

\def\ceil#1{\lceil #1 \rceil}
\def\floor#1{\lfloor #1 \rfloor}
\def\1{\bm{1}}
\newcommand{\train}{\mathcal{D}}
\newcommand{\valid}{\mathcal{D_{\mathrm{valid}}}}
\newcommand{\test}{\mathcal{D_{\mathrm{test}}}}

\def\eps{{\epsilon}}


% Random variables
\def\reta{{\textnormal{$\eta$}}}
\def\ra{{\textnormal{a}}}
\def\rb{{\textnormal{b}}}
\def\rc{{\textnormal{c}}}
\def\rd{{\textnormal{d}}}
\def\re{{\textnormal{e}}}
\def\rf{{\textnormal{f}}}
\def\rg{{\textnormal{g}}}
\def\rh{{\textnormal{h}}}
\def\ri{{\textnormal{i}}}
\def\rj{{\textnormal{j}}}
\def\rk{{\textnormal{k}}}
\def\rl{{\textnormal{l}}}
% rm is already a command, just don't name any random variables m
\def\rn{{\textnormal{n}}}
\def\ro{{\textnormal{o}}}
\def\rp{{\textnormal{p}}}
\def\rq{{\textnormal{q}}}
\def\rr{{\textnormal{r}}}
\def\rs{{\textnormal{s}}}
\def\rt{{\textnormal{t}}}
\def\ru{{\textnormal{u}}}
\def\rv{{\textnormal{v}}}
\def\rw{{\textnormal{w}}}
\def\rx{{\textnormal{x}}}
\def\ry{{\textnormal{y}}}
\def\rz{{\textnormal{z}}}

% Random vectors
\def\rvepsilon{{\mathbf{\epsilon}}}
\def\rvphi{{\mathbf{\phi}}}
\def\rvtheta{{\mathbf{\theta}}}
\def\rva{{\mathbf{a}}}
\def\rvb{{\mathbf{b}}}
\def\rvc{{\mathbf{c}}}
\def\rvd{{\mathbf{d}}}
\def\rve{{\mathbf{e}}}
\def\rvf{{\mathbf{f}}}
\def\rvg{{\mathbf{g}}}
\def\rvh{{\mathbf{h}}}
\def\rvu{{\mathbf{i}}}
\def\rvj{{\mathbf{j}}}
\def\rvk{{\mathbf{k}}}
\def\rvl{{\mathbf{l}}}
\def\rvm{{\mathbf{m}}}
\def\rvn{{\mathbf{n}}}
\def\rvo{{\mathbf{o}}}
\def\rvp{{\mathbf{p}}}
\def\rvq{{\mathbf{q}}}
\def\rvr{{\mathbf{r}}}
\def\rvs{{\mathbf{s}}}
\def\rvt{{\mathbf{t}}}
\def\rvu{{\mathbf{u}}}
\def\rvv{{\mathbf{v}}}
\def\rvw{{\mathbf{w}}}
\def\rvx{{\mathbf{x}}}
\def\rvy{{\mathbf{y}}}
\def\rvz{{\mathbf{z}}}

% Elements of random vectors
\def\erva{{\textnormal{a}}}
\def\ervb{{\textnormal{b}}}
\def\ervc{{\textnormal{c}}}
\def\ervd{{\textnormal{d}}}
\def\erve{{\textnormal{e}}}
\def\ervf{{\textnormal{f}}}
\def\ervg{{\textnormal{g}}}
\def\ervh{{\textnormal{h}}}
\def\ervi{{\textnormal{i}}}
\def\ervj{{\textnormal{j}}}
\def\ervk{{\textnormal{k}}}
\def\ervl{{\textnormal{l}}}
\def\ervm{{\textnormal{m}}}
\def\ervn{{\textnormal{n}}}
\def\ervo{{\textnormal{o}}}
\def\ervp{{\textnormal{p}}}
\def\ervq{{\textnormal{q}}}
\def\ervr{{\textnormal{r}}}
\def\ervs{{\textnormal{s}}}
\def\ervt{{\textnormal{t}}}
\def\ervu{{\textnormal{u}}}
\def\ervv{{\textnormal{v}}}
\def\ervw{{\textnormal{w}}}
\def\ervx{{\textnormal{x}}}
\def\ervy{{\textnormal{y}}}
\def\ervz{{\textnormal{z}}}

% Random matrices
\def\rmA{{\mathbf{A}}}
\def\rmB{{\mathbf{B}}}
\def\rmC{{\mathbf{C}}}
\def\rmD{{\mathbf{D}}}
\def\rmE{{\mathbf{E}}}
\def\rmF{{\mathbf{F}}}
\def\rmG{{\mathbf{G}}}
\def\rmH{{\mathbf{H}}}
\def\rmI{{\mathbf{I}}}
\def\rmJ{{\mathbf{J}}}
\def\rmK{{\mathbf{K}}}
\def\rmL{{\mathbf{L}}}
\def\rmM{{\mathbf{M}}}
\def\rmN{{\mathbf{N}}}
\def\rmO{{\mathbf{O}}}
\def\rmP{{\mathbf{P}}}
\def\rmQ{{\mathbf{Q}}}
\def\rmR{{\mathbf{R}}}
\def\rmS{{\mathbf{S}}}
\def\rmT{{\mathbf{T}}}
\def\rmU{{\mathbf{U}}}
\def\rmV{{\mathbf{V}}}
\def\rmW{{\mathbf{W}}}
\def\rmX{{\mathbf{X}}}
\def\rmY{{\mathbf{Y}}}
\def\rmZ{{\mathbf{Z}}}

% Elements of random matrices
\def\ermA{{\textnormal{A}}}
\def\ermB{{\textnormal{B}}}
\def\ermC{{\textnormal{C}}}
\def\ermD{{\textnormal{D}}}
\def\ermE{{\textnormal{E}}}
\def\ermF{{\textnormal{F}}}
\def\ermG{{\textnormal{G}}}
\def\ermH{{\textnormal{H}}}
\def\ermI{{\textnormal{I}}}
\def\ermJ{{\textnormal{J}}}
\def\ermK{{\textnormal{K}}}
\def\ermL{{\textnormal{L}}}
\def\ermM{{\textnormal{M}}}
\def\ermN{{\textnormal{N}}}
\def\ermO{{\textnormal{O}}}
\def\ermP{{\textnormal{P}}}
\def\ermQ{{\textnormal{Q}}}
\def\ermR{{\textnormal{R}}}
\def\ermS{{\textnormal{S}}}
\def\ermT{{\textnormal{T}}}
\def\ermU{{\textnormal{U}}}
\def\ermV{{\textnormal{V}}}
\def\ermW{{\textnormal{W}}}
\def\ermX{{\textnormal{X}}}
\def\ermY{{\textnormal{Y}}}
\def\ermZ{{\textnormal{Z}}}

% Vectors
\def\vzero{{\bm{0}}}
\def\vone{{\bm{1}}}
\def\vmu{{\bm{\mu}}}
\def\vtheta{{\bm{\theta}}}
\def\vphi{{\bm{\phi}}}
\def\va{{\bm{a}}}
\def\vb{{\bm{b}}}
\def\vc{{\bm{c}}}
\def\vd{{\bm{d}}}
\def\ve{{\bm{e}}}
\def\vf{{\bm{f}}}
\def\vg{{\bm{g}}}
\def\vh{{\bm{h}}}
\def\vi{{\bm{i}}}
\def\vj{{\bm{j}}}
\def\vk{{\bm{k}}}
\def\vl{{\bm{l}}}
\def\vm{{\bm{m}}}
\def\vn{{\bm{n}}}
\def\vo{{\bm{o}}}
\def\vp{{\bm{p}}}
\def\vq{{\bm{q}}}
\def\vr{{\bm{r}}}
\def\vs{{\bm{s}}}
\def\vt{{\bm{t}}}
\def\vu{{\bm{u}}}
\def\vv{{\bm{v}}}
\def\vw{{\bm{w}}}
\def\vx{{\bm{x}}}
\def\vy{{\bm{y}}}
\def\vz{{\bm{z}}}

% Elements of vectors
\def\evalpha{{\alpha}}
\def\evbeta{{\beta}}
\def\evepsilon{{\epsilon}}
\def\evlambda{{\lambda}}
\def\evomega{{\omega}}
\def\evmu{{\mu}}
\def\evpsi{{\psi}}
\def\evsigma{{\sigma}}
\def\evtheta{{\theta}}
\def\eva{{a}}
\def\evb{{b}}
\def\evc{{c}}
\def\evd{{d}}
\def\eve{{e}}
\def\evf{{f}}
\def\evg{{g}}
\def\evh{{h}}
\def\evi{{i}}
\def\evj{{j}}
\def\evk{{k}}
\def\evl{{l}}
\def\evm{{m}}
\def\evn{{n}}
\def\evo{{o}}
\def\evp{{p}}
\def\evq{{q}}
\def\evr{{r}}
\def\evs{{s}}
\def\evt{{t}}
\def\evu{{u}}
\def\evv{{v}}
\def\evw{{w}}
\def\evx{{x}}
\def\evy{{y}}
\def\evz{{z}}

% Matrix
\def\mA{{\bm{A}}}
\def\mB{{\bm{B}}}
\def\mC{{\bm{C}}}
\def\mD{{\bm{D}}}
\def\mE{{\bm{E}}}
\def\mF{{\bm{F}}}
\def\mG{{\bm{G}}}
\def\mH{{\bm{H}}}
\def\mI{{\bm{I}}}
\def\mJ{{\bm{J}}}
\def\mK{{\bm{K}}}
\def\mL{{\bm{L}}}
\def\mM{{\bm{M}}}
\def\mN{{\bm{N}}}
\def\mO{{\bm{O}}}
\def\mP{{\bm{P}}}
\def\mQ{{\bm{Q}}}
\def\mR{{\bm{R}}}
\def\mS{{\bm{S}}}
\def\mT{{\bm{T}}}
\def\mU{{\bm{U}}}
\def\mV{{\bm{V}}}
\def\mW{{\bm{W}}}
\def\mX{{\bm{X}}}
\def\mY{{\bm{Y}}}
\def\mZ{{\bm{Z}}}
\def\mBeta{{\bm{\beta}}}
\def\mPhi{{\bm{\Phi}}}
\def\mLambda{{\bm{\Lambda}}}
\def\mSigma{{\bm{\Sigma}}}

% Tensor
\DeclareMathAlphabet{\mathsfit}{\encodingdefault}{\sfdefault}{m}{sl}
\SetMathAlphabet{\mathsfit}{bold}{\encodingdefault}{\sfdefault}{bx}{n}
\newcommand{\tens}[1]{\bm{\mathsfit{#1}}}
\def\tA{{\tens{A}}}
\def\tB{{\tens{B}}}
\def\tC{{\tens{C}}}
\def\tD{{\tens{D}}}
\def\tE{{\tens{E}}}
\def\tF{{\tens{F}}}
\def\tG{{\tens{G}}}
\def\tH{{\tens{H}}}
\def\tI{{\tens{I}}}
\def\tJ{{\tens{J}}}
\def\tK{{\tens{K}}}
\def\tL{{\tens{L}}}
\def\tM{{\tens{M}}}
\def\tN{{\tens{N}}}
\def\tO{{\tens{O}}}
\def\tP{{\tens{P}}}
\def\tQ{{\tens{Q}}}
\def\tR{{\tens{R}}}
\def\tS{{\tens{S}}}
\def\tT{{\tens{T}}}
\def\tU{{\tens{U}}}
\def\tV{{\tens{V}}}
\def\tW{{\tens{W}}}
\def\tX{{\tens{X}}}
\def\tY{{\tens{Y}}}
\def\tZ{{\tens{Z}}}


% Graph
\def\gA{{\mathcal{A}}}
\def\gB{{\mathcal{B}}}
\def\gC{{\mathcal{C}}}
\def\gD{{\mathcal{D}}}
\def\gE{{\mathcal{E}}}
\def\gF{{\mathcal{F}}}
\def\gG{{\mathcal{G}}}
\def\gH{{\mathcal{H}}}
\def\gI{{\mathcal{I}}}
\def\gJ{{\mathcal{J}}}
\def\gK{{\mathcal{K}}}
\def\gL{{\mathcal{L}}}
\def\gM{{\mathcal{M}}}
\def\gN{{\mathcal{N}}}
\def\gO{{\mathcal{O}}}
\def\gP{{\mathcal{P}}}
\def\gQ{{\mathcal{Q}}}
\def\gR{{\mathcal{R}}}
\def\gS{{\mathcal{S}}}
\def\gT{{\mathcal{T}}}
\def\gU{{\mathcal{U}}}
\def\gV{{\mathcal{V}}}
\def\gW{{\mathcal{W}}}
\def\gX{{\mathcal{X}}}
\def\gY{{\mathcal{Y}}}
\def\gZ{{\mathcal{Z}}}

% Sets
\def\sA{{\mathbb{A}}}
\def\sB{{\mathbb{B}}}
\def\sC{{\mathbb{C}}}
\def\sD{{\mathbb{D}}}
% Don't use a set called E, because this would be the same as our symbol
% for expectation.
\def\sF{{\mathbb{F}}}
\def\sG{{\mathbb{G}}}
\def\sH{{\mathbb{H}}}
\def\sI{{\mathbb{I}}}
\def\sJ{{\mathbb{J}}}
\def\sK{{\mathbb{K}}}
\def\sL{{\mathbb{L}}}
\def\sM{{\mathbb{M}}}
\def\sN{{\mathbb{N}}}
\def\sO{{\mathbb{O}}}
\def\sP{{\mathbb{P}}}
\def\sQ{{\mathbb{Q}}}
\def\sR{{\mathbb{R}}}
\def\sS{{\mathbb{S}}}
\def\sT{{\mathbb{T}}}
\def\sU{{\mathbb{U}}}
\def\sV{{\mathbb{V}}}
\def\sW{{\mathbb{W}}}
\def\sX{{\mathbb{X}}}
\def\sY{{\mathbb{Y}}}
\def\sZ{{\mathbb{Z}}}

% Entries of a matrix
\def\emLambda{{\Lambda}}
\def\emA{{A}}
\def\emB{{B}}
\def\emC{{C}}
\def\emD{{D}}
\def\emE{{E}}
\def\emF{{F}}
\def\emG{{G}}
\def\emH{{H}}
\def\emI{{I}}
\def\emJ{{J}}
\def\emK{{K}}
\def\emL{{L}}
\def\emM{{M}}
\def\emN{{N}}
\def\emO{{O}}
\def\emP{{P}}
\def\emQ{{Q}}
\def\emR{{R}}
\def\emS{{S}}
\def\emT{{T}}
\def\emU{{U}}
\def\emV{{V}}
\def\emW{{W}}
\def\emX{{X}}
\def\emY{{Y}}
\def\emZ{{Z}}
\def\emSigma{{\Sigma}}

% entries of a tensor
% Same font as tensor, without \bm wrapper
\newcommand{\etens}[1]{\mathsfit{#1}}
\def\etLambda{{\etens{\Lambda}}}
\def\etA{{\etens{A}}}
\def\etB{{\etens{B}}}
\def\etC{{\etens{C}}}
\def\etD{{\etens{D}}}
\def\etE{{\etens{E}}}
\def\etF{{\etens{F}}}
\def\etG{{\etens{G}}}
\def\etH{{\etens{H}}}
\def\etI{{\etens{I}}}
\def\etJ{{\etens{J}}}
\def\etK{{\etens{K}}}
\def\etL{{\etens{L}}}
\def\etM{{\etens{M}}}
\def\etN{{\etens{N}}}
\def\etO{{\etens{O}}}
\def\etP{{\etens{P}}}
\def\etQ{{\etens{Q}}}
\def\etR{{\etens{R}}}
\def\etS{{\etens{S}}}
\def\etT{{\etens{T}}}
\def\etU{{\etens{U}}}
\def\etV{{\etens{V}}}
\def\etW{{\etens{W}}}
\def\etX{{\etens{X}}}
\def\etY{{\etens{Y}}}
\def\etZ{{\etens{Z}}}

% The true underlying data generating distribution
\newcommand{\pdata}{p_{\rm{data}}}
\newcommand{\ptarget}{p_{\rm{target}}}
\newcommand{\pprior}{p_{\rm{prior}}}
\newcommand{\pbase}{p_{\rm{base}}}
\newcommand{\pref}{p_{\rm{ref}}}

% The empirical distribution defined by the training set
\newcommand{\ptrain}{\hat{p}_{\rm{data}}}
\newcommand{\Ptrain}{\hat{P}_{\rm{data}}}
% The model distribution
\newcommand{\pmodel}{p_{\rm{model}}}
\newcommand{\Pmodel}{P_{\rm{model}}}
\newcommand{\ptildemodel}{\tilde{p}_{\rm{model}}}
% Stochastic autoencoder distributions
\newcommand{\pencode}{p_{\rm{encoder}}}
\newcommand{\pdecode}{p_{\rm{decoder}}}
\newcommand{\precons}{p_{\rm{reconstruct}}}

\newcommand{\laplace}{\mathrm{Laplace}} % Laplace distribution

\newcommand{\E}{\mathbb{E}}
\newcommand{\Ls}{\mathcal{L}}
\newcommand{\R}{\mathbb{R}}
\newcommand{\emp}{\tilde{p}}
\newcommand{\lr}{\alpha}
\newcommand{\reg}{\lambda}
\newcommand{\rect}{\mathrm{rectifier}}
\newcommand{\softmax}{\mathrm{softmax}}
\newcommand{\sigmoid}{\sigma}
\newcommand{\softplus}{\zeta}
\newcommand{\KL}{D_{\mathrm{KL}}}
\newcommand{\Var}{\mathrm{Var}}
\newcommand{\standarderror}{\mathrm{SE}}
\newcommand{\Cov}{\mathrm{Cov}}
% Wolfram Mathworld says $L^2$ is for function spaces and $\ell^2$ is for vectors
% But then they seem to use $L^2$ for vectors throughout the site, and so does
% wikipedia.
\newcommand{\normlzero}{L^0}
\newcommand{\normlone}{L^1}
\newcommand{\normltwo}{L^2}
\newcommand{\normlp}{L^p}
\newcommand{\normmax}{L^\infty}

\newcommand{\parents}{Pa} % See usage in notation.tex. Chosen to match Daphne's book.

\DeclareMathOperator*{\argmax}{arg\,max}
\DeclareMathOperator*{\argmin}{arg\,min}

\DeclareMathOperator{\sign}{sign}
\DeclareMathOperator{\Tr}{Tr}
\let\ab\allowbreak


\usepackage{hyperref}
\usepackage{url}

\usepackage[ruled]{algorithm2e} % For algorithms

\title{Ready-to-React: Online Reaction Policy for Two-Character Interaction Generation}


\author{
Zhi Cen$^{1}$, Huaijin Pi$^{2}$, Sida Peng$^{1}$, Qing Shuai$^{1}$, Yujun Shen$^{3}$, Hujun Bao$^{1}$, Xiaowei Zhou$^{1}$, \\ 
\textbf{Ruizhen Hu$^{4}$\thanks{Corresponding author.}} \\
$^{1}$State Key Lab of CAD\&CG, Zhejiang University, $^{2}$The University of Hong Kong, $^{3}$Ant Group, \\
$^{4}$Shenzhen University\\
\texttt{zhicen@zju.edu.cn}, \texttt{ruizhen.hu@gmail.com}
}

\iclrfinalcopy % Uncomment for camera-ready version, but NOT for submission.
\begin{document}
\maketitle

\begin{figure}[h]
    \vskip-40pt
    \begin{center}
        \includegraphics[width=\linewidth]{figures/teaser_v1.pdf}
    \end{center}
    \caption{\textbf{Demonstration of \shortname}, an \textit{online} reaction policy for two-character interaction generation on the challenging task of boxing. \shortname{} predicts the next pose of an agent by considering its own and the counterpart's historical motions. Our method can successfully generate 1800 frames of motion, whereas the GPT-based approach struggles after about 200 frames, displaying issues such as incorrect orientation, leaving the ring boundary, or freezing in place due to the accumulation of errors over time.}
    \label{fig:teaser}
    \figtabskip
\end{figure}


\begin{abstract}
  In this work, we present a novel technique for GPU-accelerated Boolean satisfiability (SAT) sampling. Unlike conventional sampling algorithms that directly operate on conjunctive normal form (CNF), our method transforms the logical constraints of SAT problems by factoring their CNF representations into simplified multi-level, multi-output Boolean functions. It then leverages gradient-based optimization to guide the search for a diverse set of valid solutions. Our method operates directly on the circuit structure of refactored SAT instances, reinterpreting the SAT problem as a supervised multi-output regression task. This differentiable technique enables independent bit-wise operations on each tensor element, allowing parallel execution of learning processes. As a result, we achieve GPU-accelerated sampling with significant runtime improvements ranging from $33.6\times$ to $523.6\times$ over state-of-the-art heuristic samplers. We demonstrate the superior performance of our sampling method through an extensive evaluation on $60$ instances from a public domain benchmark suite utilized in previous studies. 


  
  % Generating a wide range of diverse solutions to logical constraints is crucial in software and hardware testing, verification, and synthesis. These solutions can serve as inputs to test specific functionalities of a software program or as random stimuli in hardware modules. In software verification, techniques like fuzz testing and symbolic execution use this approach to identify bugs and vulnerabilities. In hardware verification, stimulus generation is particularly vital, forming the basis of constrained-random verification. While generating multiple solutions improves coverage and increases the chances of finding bugs, high-throughput sampling remains challenging, especially with complex constraints and refined coverage criteria. In this work, we present a novel technique that enables GPU-accelerated sampling, resulting in high-throughput generation of satisfying solutions to Boolean satisfiability (SAT) problems. Unlike conventional sampling algorithms that directly operate on conjunctive normal form (CNF), our method refines the logical constraints of SAT problems by transforming their CNF into simplified multi-level Boolean expressions. It then leverages gradient-based optimization to guide the search for a diverse set of valid solutions.
  % Our method specifically takes advantage of the circuit structure of refined SAT instances by using GD to learn valid solutions, reinterpreting the SAT problem as a supervised multi-output regression task. This differentiable technique enables independent bit-wise operations on each tensor element, allowing parallel execution of learning processes. As a result, we achieve GPU-accelerated sampling with significant runtime improvements ranging from $10\times$ to $1000\times$ over state-of-the-art heuristic samplers. Specifically, we demonstrate the superior performance of our sampling method through an extensive evaluation on $60$ instances from a public domain benchmark suite utilized in previous studies.

\end{abstract}

\begin{IEEEkeywords}
Boolean Satisfiability, Gradient Descent, Multi-level Circuits, Verification, and Testing.
\end{IEEEkeywords}
\section{Introduction}
\label{section:introduction}

% redirection is unique and important in VR
Virtual Reality (VR) systems enable users to embody virtual avatars by mirroring their physical movements and aligning their perspective with virtual avatars' in real time. 
As the head-mounted displays (HMDs) block direct visual access to the physical world, users primarily rely on visual feedback from the virtual environment and integrate it with proprioceptive cues to control the avatar’s movements and interact within the VR space.
Since human perception is heavily influenced by visual input~\cite{gibson1933adaptation}, 
VR systems have the unique capability to control users' perception of the virtual environment and avatars by manipulating the visual information presented to them.
Leveraging this, various redirection techniques have been proposed to enable novel VR interactions, 
such as redirecting users' walking paths~\cite{razzaque2005redirected, suma2012impossible, steinicke2009estimation},
modifying reaching movements~\cite{gonzalez2022model, azmandian2016haptic, cheng2017sparse, feick2021visuo},
and conveying haptic information through visual feedback to create pseudo-haptic effects~\cite{samad2019pseudo, dominjon2005influence, lecuyer2009simulating}.
Such redirection techniques enable these interactions by manipulating the alignment between users' physical movements and their virtual avatar's actions.

% % what is hand/arm redirection, motivation of study arm-offset
% \change{\yj{i don't understand the purpose of this paragraph}
% These illusion-based techniques provide users with unique experiences in virtual environments that differ from the physical world yet maintain an immersive experience. 
% A key example is hand redirection, which shifts the virtual hand’s position away from the real hand as the user moves to enhance ergonomics during interaction~\cite{feuchtner2018ownershift, wentzel2020improving} and improve interaction performance~\cite{montano2017erg, poupyrev1996go}. 
% To increase the realism of virtual movements and strengthen the user’s sense of embodiment, hand redirection techniques often incorporate a complete virtual arm or full body alongside the redirected virtual hand, using inverse kinematics~\cite{hartfill2021analysis, ponton2024stretch} or adjustments to the virtual arm's movement as well~\cite{li2022modeling, feick2024impact}.
% }

% noticeability, motivation of predicting a probability, not a classification
However, these redirection techniques are most effective when the manipulation remains undetected~\cite{gonzalez2017model, li2022modeling}. 
If the redirection becomes too large, the user may not mitigate the conflict between the visual sensory input (redirected virtual movement) and their proprioception (actual physical movement), potentially leading to a loss of embodiment with the virtual avatar and making it difficult for the user to accurately control virtual movements to complete interaction tasks~\cite{li2022modeling, wentzel2020improving, feuchtner2018ownershift}. 
While proprioception is not absolute, users only have a general sense of their physical movements and the likelihood that they notice the redirection is probabilistic. 
This probability of detecting the redirection is referred to as \textbf{noticeability}~\cite{li2022modeling, zenner2024beyond, zenner2023detectability} and is typically estimated based on the frequency with which users detect the manipulation across multiple trials.

% version B
% Prior research has explored factors influencing the noticeability of redirected motion, including the redirection's magnitude~\cite{wentzel2020improving, poupyrev1996go}, direction~\cite{li2022modeling, feuchtner2018ownershift}, and the visual characteristics of the virtual avatar~\cite{ogawa2020effect, feick2024impact}.
% While these factors focus on the avatars, the surrounding virtual environment can also influence the users' behavior and in turn affect the noticeability of redirection.
% One such prominent external influence is through the visual channel - the users' visual attention is constantly distracted by complex visual effects and events in practical VR scenarios.
% Although some prior studies have explored how to leverage user blindness caused by visual distractions to redirect users' virtual hand~\cite{zenner2023detectability}, there remains a gap in understanding how to quantify the noticeability of redirection under visual distractions.

% visual stimuli and gaze behavior
Prior research has explored factors influencing the noticeability of redirected motion, including the redirection's magnitude~\cite{wentzel2020improving, poupyrev1996go}, direction~\cite{li2022modeling, feuchtner2018ownershift}, and the visual characteristics of the virtual avatar~\cite{ogawa2020effect, feick2024impact}.
While these factors focus on the avatars, the surrounding virtual environment can also influence the users' behavior and in turn affect the noticeability of redirection.
This, however, remains underexplored.
One such prominent external influence is through the visual channel - the users' visual attention is constantly distracted by complex visual effects and events in practical VR scenarios.
We thus want to investigate how \textbf{visual stimuli in the virtual environment} affect the noticeability of redirection.
With this, we hope to complement existing works that focus on avatars by incorporating environmental visual influences to enable more accurate control over the noticeability of redirected motions in practical VR scenarios.
% However, in realistic VR applications, the virtual environment often contains complex visual effects beyond the virtual avatar itself. 
% We argue that these visual effects can \textbf{distract users’ visual attention and thus affect the noticeability of redirection offsets}, while current research has yet taken into account.
% For instance, in a VR boxing scenario, a user’s visual attention is likely focused on their opponent rather than on their virtual body, leading to a lower noticeability of redirection offsets on their virtual movements. 
% Conversely, when reaching for an object in the center of their field of view, the user’s attention is more concentrated on the virtual hand’s movement and position to ensure successful interaction, resulting in a higher noticeability of offsets.

Since each visual event is a complex choreography of many underlying factors (type of visual effect, location, duration, etc.), it is extremely difficult to quantify or parameterize visual stimuli.
Furthermore, individuals respond differently to even the same visual events.
Prior neuroscience studies revealed that factors like age, gender, and personality can influence how quickly someone reacts to visual events~\cite{gillon2024responses, gale1997human}. 
Therefore, aiming to model visual stimuli in a way that is generalizable and applicable to different stimuli and users, we propose to use users' \textbf{gaze behavior} as an indicator of how they respond to visual stimuli.
In this paper, we used various gaze behaviors, including gaze location, saccades~\cite{krejtz2018eye}, fixations~\cite{perkhofer2019using}, and the Index of Pupil Activity (IPA)~\cite{duchowski2018index}.
These behaviors indicate both where users are looking and their cognitive activity, as looking at something does not necessarily mean they are attending to it.
Our goal is to investigate how these gaze behaviors stimulated by various visual stimuli relate to the noticeability of redirection.
With this, we contribute a model that allows designers and content creators to adjust the redirection in real-time responding to dynamic visual events in VR.

To achieve this, we conducted user studies to collect users' noticeability of redirection under various visual stimuli.
To simulate realistic VR scenarios, we adopted a dual-task design in which the participants performed redirected movements while monitoring the visual stimuli.
Specifically, participants' primary task was to report if they noticed an offset between the avatar's movement and their own, while their secondary task was to monitor and report the visual stimuli.
As realistic virtual environments often contain complex visual effects, we started with simple and controlled visual stimulus to manage the influencing factors.

% first user study, confirmation study
% collect data under no visual stimuli, different basic visual stimuli
We first conducted a confirmation study (N=16) to test whether applying visual stimuli (opacity-based) actually affects their noticeability of redirection. 
The results showed that participants were significantly less likely to detect the redirection when visual stimuli was presented $(F_{(1,15)}=5.90,~p=0.03)$.
Furthermore, by analyzing the collected gaze data, results revealed a correlation between the proposed gaze behaviors and the noticeability results $(r=-0.43)$, confirming that the gaze behaviors could be leveraged to compute the noticeability.

% data collection study
We then conducted a data collection study to obtain more accurate noticeability results through repeated measurements to better model the relationship between visual stimuli-triggered gaze behaviors and noticeability of redirection.
With the collected data, we analyzed various numerical features from the gaze behaviors to identify the most effective ones. 
We tested combinations of these features to determine the most effective one for predicting noticeability under visual stimuli.
Using the selected features, our regression model achieved a mean squared error (MSE) of 0.011 through leave-one-user-out cross-validation. 
Furthermore, we developed both a binary and a three-class classification model to categorize noticeability, which achieved an accuracy of 91.74\% and 85.62\%, respectively.

% evaluation study
To evaluate the generalizability of the regression model, we conducted an evaluation study (N=24) to test whether the model could accurately predict noticeability with new visual stimuli (color- and scale-based animations).
Specifically, we evaluated whether the model's predictions aligned with participants' responses under these unseen stimuli.
The results showed that our model accurately estimated the noticeability, achieving mean squared errors (MSE) of 0.014 and 0.012 for the color- and scale-based visual stimili, respectively, compared to participants' responses.
Since the tested visual stimuli data were not included in the training, the results suggested that the extracted gaze behavior features capture a generalizable pattern and can effectively indicate the corresponding impact on the noticeability of redirection.

% application
Based on our model, we implemented an adaptive redirection technique and demonstrated it through two applications: adaptive VR action game and opportunistic rendering.
We conducted a proof-of-concept user study (N=8) to compare our adaptive redirection technique with a static redirection, evaluating the usability and benefits of our adaptive redirection technique.
The results indicated that participants experienced less physical demand and stronger sense of embodiment and agency when using the adaptive redirection technique. 
These results demonstrated the effectiveness and usability of our model.

In summary, we make the following contributions.
% 
\begin{itemize}
    \item 
    We propose to use users' gaze behavior as a medium to quantify how visual stimuli influences the noticebility of redirection. 
    Through two user studies, we confirm that visual stimuli significantly influences noticeability and identify key gaze behavior features that are closely related to this impact.
    \item 
    We build a regression model that takes the user's gaze behavioral data as input, then computes the noticeability of redirection.
    Through an evaluation study, we verify that our model can estimate the noticeability with new participants under unseen visual stimuli.
    These findings suggest that the extracted gaze behavior features effectively capture the influence of visual stimuli on noticeability and can generalize across different users and visual stimuli.
    \item 
    We develop an adaptive redirection technique based on our regression model and implement two applications with it.
    With a proof-of-concept study, we demonstrate the effectiveness and potential usability of our regression model on real-world use cases.

\end{itemize}

% \delete{
% Virtual Reality (VR) allows the user to embody a virtual avatar by mirroring their physical movements through the avatar.
% As the user's visual access to the physical world is blocked in tasks involving motion control, they heavily rely on the visual representation of the avatar's motions to guide their proprioception.
% Similar to real-world experiences, the user is able to resolve conflicts between different sensory inputs (e.g., vision and motor control) through multisensory integration, which is essential for mitigating the sensory noise that commonly arises.
% However, it also enables unique manipulations in VR, as the system can intentionally modify the avatar's movements in relation to the user's motions to achieve specific functional outcomes,
% for example, 
% % the manipulations on the avatar's movements can 
% enabling novel interaction techniques of redirected walking~\cite{razzaque2005redirected}, redirected reaching~\cite{gonzalez2022model}, and pseudo haptics~\cite{samad2019pseudo}.
% With small adjustments to the avatar's movements, the user can maintain their sense of embodiment, due to their ability to resolve the perceptual differences.
% % However, a large mismatch between the user and avatar's movements can result in the user losing their sense of embodiment, due to an inability to resolve the perceptual differences.
% }

% \delete{
% However, multisensory integration can break when the manipulation is so intense that the user is aware of the existence of the motion offset and no longer maintains the sense of embodiment.
% Prior research studied the intensity threshold of the offset applied on the avatar's hand, beyond which the embodiment will break~\cite{li2022modeling}. 
% Studies also investigated the user's sensitivity to the offsets over time~\cite{kohm2022sensitivity}.
% Based on the findings, we argue that one crucial factor that affects to what extent the user notices the offset (i.e., \textit{noticeability}) that remains under-explored is whether the user directs their visual attention towards or away from the virtual avatar.
% Related work (e.g., Mise-unseen~\cite{marwecki2019mise}) has showcased applications where adjustments in the environment can be made in an unnoticeable manner when they happen in the area out of the user's visual field.
% We hypothesize that directing the user's visual attention away from the avatar's body, while still partially keeping the avatar within the user's field-of-view, can reduce the noticeability of the offset.
% Therefore, we conduct two user studies and implement a regression model to systematically investigate this effect.
% }

% \delete{
% In the first user study (N = 16), we test whether drawing the user's visual attention away from their body impacts the possibility of them noticing an offset that we apply to their arm motion in VR.
% We adopt a dual-task design to enable the alteration of the user's visual attention and a yes/no paradigm to measure the noticeability of motion offset. 
% The primary task for the user is to perform an arm motion and report when they perceive an offset between the avatar's virtual arm and their real arm.
% In the secondary task, we randomly render a visual animation of a ball turning from transparent to red and becoming transparent again and ask them to monitor and report when it appears.
% We control the strength of the visual stimuli by changing the duration and location of the animation.
% % By changing the time duration and location of the visual animation, we control the strengths of attraction to the users.
% As a result, we found significant differences in the noticeability of the offsets $(F_{(1,15)}=5.90,~p=0.03)$ between conditions with and without visual stimuli.
% Based on further analysis, we also identified the behavioral patterns of the user's gaze (including pupil dilation, fixations, and saccades) to be correlated with the noticeability results $(r=-0.43)$ and they may potentially serve as indicators of noticeability.
% }

% \delete{
% To further investigate how visual attention influences the noticeability, we conduct a data collection study (N = 12) and build a regression model based on the data.
% The regression model is able to calculate the noticeability of the offset applied on the user's arm under various visual stimuli based on their gaze behaviors.
% Our leave-one-out cross-validation results show that the proposed method was able to achieve a mean-squared error (MSE) of 0.012 in the probability regression task.
% }

% \delete{
% To verify the feasibility and extendability of the regression model, we conduct an evaluation study where we test new visual animations based on adjustments on scale and color and invite 24 new participants to attend the study.
% Results show that the proposed method can accurately estimate the noticeability with an MSE of 0.014 and 0.012 in the conditions of the color- and scale-based visual effects.
% Since these animations were not included in the dataset that the regression model was built on, the study demonstrates that the gaze behavioral features we extracted from the data capture a generalizable pattern of the user's visual attention and can indicate the corresponding impact on the noticeability of the offset.
% }

% \delete{
% Finally, we demonstrate applications that can benefit from the noticeability prediction model, including adaptive motion offsets and opportunistic rendering, considering the user's visual attention. 
% We conclude with discussions of our work's limitations and future research directions.
% }

% \delete{
% In summary, we make the following contributions.
% }
% % 
% \begin{itemize}
%     \item 
%     \delete{
%     We quantify the effects of the user's visual attention directed away by stimuli on their noticeability of an offset applied to the avatar's arm motion with respect to the user's physical arm. 
%     Through two user studies, we identified gaze behavioral features that are indicative of the changes in noticeability.
%     }
%     \item 
%     \delete{We build a regression model that takes the user's gaze behavioral data and the offset applied to the arm motion as input, then computes the probability of the user noticing the offset.
%     Through an evaluation study, we verified that the model needs no information about the source attracting the user's visual attention and can be generalizable in different scenarios.
%     }
%     \item 
%     \delete{We demonstrate two applications that potentially benefit from the regression model, including adaptive motion offsets and opportunistic rendering.
%     }

% \end{itemize}

\begin{comment}
However, users will lose the sense of embodiment to the virtual avatars if they notice the offset between the virtual and physical movements.
To address this, researchers have been exploring the noticing threshold of offsets with various magnitudes and proposing various redirection techniques that maintain the sense of embodiment~\cite{}.

However, when users embody virtual avatars to explore virtual environments, they encounter various visual effects and content that can attract their attention~\cite{}.
During this, the user may notice an offset when he observes the virtual movement carefully while ignoring it when the virtual contents attract his attention from the movements.
Therefore, static offset thresholds are not appropriate in dynamic scenarios.

Past research has proposed dynamic mapping techniques that adapted to users' state, such as hand moving speed~\cite{frees2007prism} or ergonomically comfortable poses~\cite{montano2017erg}, but not considering the influence of virtual content.
More specifically, PRISM~\cite{frees2007prism} proposed adjusting the C/D ratio with a non-linear mapping according to users' hand moving speed, but it might not be optimal for various virtual scenarios.
While Erg-O~\cite{montano2017erg} redirected users' virtual hands according to the virtual target's relative position to reduce physical fatigue, neglecting the change of virtual environments. 

Therefore, how to design redirection techniques in various scenarios with different visual attractions remains unknown.
To address this, we investigate how visual attention affects the noticing probability of movement offsets.
Based on our experiments, we implement a computational model that automatically computes the noticing probability of offsets under certain visual attractions.
VR application designers and developers can easily leverage our model to design redirection techniques maintaining the sense of embodiment adapt to the user's visual attention.
We implement a dynamic redirection technique with our model and demonstrate that it effectively reduces the target reaching time without reducing the sense of embodiment compared to static redirection techniques.

% Need to be refined
This paper offers the following contributions.
\begin{itemize}
    \item We investigate how visual attractions affect the noticing probability of redirection offsets.
    \item We construct a computational model to predict the noticing probability of an offset with a given visual background.
    \item We implement a dynamic redirection technique adapting to the visual background. We evaluate the technique and develop three applications to demonstrate the benefits. 
\end{itemize}



First, we conducted a controlled experiment to understand how users perceived the movement offset while subjected to various distractions.
Since hand redirection is one of the most frequently used redirections in VR interactions, we focused on the dynamic arm movements and manually added angular offsets to the' elbow joint~\cite{li2022modeling, gonzalez2022model, zenner2019estimating}. 
We employed flashing spheres in the user's field of view as distractions to attract users' visual attention.
Participants were instructed to report the appearing location of the spheres while simultaneously performing the arm movements and reporting if they perceived an offset during the movement. 
(\zhipeng{Add the results of data collection. Analyze the influence of the distance between the gaze map and the offset.}
We measured the visual attraction's magnitude with the gaze distribution on it.
Results showed that stronger distractions made it harder for users to notice the offset.)
\zhipeng{Need to rewrite. Not sure to use gaze distribution or a metric obtained from the visual content.}
Secondly, we constructed a computational model to predict the noticing probability of offsets with given visual content.
We analyzed the data from the user studies to measure the influence of visual attractions on the noticing probability of offsets.
We built a statistical model to predict the offset's noticing probability with a given visual content.
Based on the model, we implement a dynamic redirection technique to adjust the redirection offset adapted to the user's current field of view.
We evaluated the technique in a target selection task compared to no hand redirection and static hand redirection.
\zhipeng{Add the results of the evaluation.}
Results showed that the dynamic hand redirection technique significantly reduced the target selection time with similar accuracy and a comparable sense of embodiment.
Finally, we implemented three applications to demonstrate the potential benefits of the visual attention adapted dynamic redirection technique.
\end{comment}

% This one modifies arm length, not redirection
% \citeauthor{mcintosh2020iteratively} proposed an adaptation method to iteratively change the virtual avatar arm's length based on the primary tasks' performance~\cite{mcintosh2020iteratively}.



% \zhipeng{TO ADD: what is redirection}
% Redirection enables novel interactions in Virtual Reality, including redirected walking, haptic redirection, and pseudo haptics by introducing an offset to users' movement.
% \zhipeng{TO ADD: extend this sentence}
% The price of this is that users' immersiveness and embodiment in VR can be compromised when they notice the offset and perceive the virtual movement not as theirs~\cite{}.
% \zhipeng{TO ADD: extend this sentence, elaborate how the virtual environment attracts users' attention}
% Meanwhile, the visual content in the virtual environment is abundant and consistently captures users' attention, making it harder to notice the offset~\cite{}.
% While previous studies explored the noticing threshold of the offsets and optimized the redirection techniques to maintain the sense of embodiment~\cite{}, the influence of visual content on the probability of perceiving offsets remains unknown.  
% Therefore, we propose to investigate how users perceive the redirection offset when they are facing various visual attractions.


% We conducted a user study to understand how users notice the shift with visual attractions.
% We used a color-changing ball to attract the user's attention while instructing users to perform different poses with their arms and observe it meanwhile.
% \zhipeng{(Which one should be the primary task? Observe the ball should be the primary one, but if the primary task is too simple, users might allocate more attention on the secondary task and this makes the secondary task primary.)}
% \zhipeng{(We need a good and reasonable dual-task design in which users care about both their pose and the visual content, at least in the evaluation study. And we need to be able to control the visual content's magnitude and saliency maybe?)}
% We controlled the shift magnitude and direction, the user's pose, the ball's size, and the color range.
% We set the ball's color-changing interval as the independent factor.
% We collect the user's response to each shift and the color-changing times.
% Based on the collected data, we constructed a statistical model to describe the influence of visual attraction on the noticing probability.
% \zhipeng{(Are we actually controlling the attention allocation? How do we measure the attracting effect? We need uniform metrics, otherwise it is also hard for others to use our knowledge.)}
% \zhipeng{(Try to use eye gaze? The eye gaze distribution in the last five seconds to decide the attention allocation? Basically constructing a model with eye gaze distribution and noticing probability. But the user's head is moving, so the eye gaze distribution is not aligned well with the current view.)}

% \zhipeng{Saliency and EMD}
% \zhipeng{Gaze is more than just a point: Rethinking visual attention
% analysis using peripheral vision-based gaze mapping}

% Evaluation study(ideal case): based on the visual content, adjusting the redirection magnitude dynamically.

% \zhipeng{(The risk is our model's effect is trivial.)}

% Applications:
% Playing Lego while watching demo videos, we can accelerate the reaching process of bricks, and forbid the redirection during the manipulation.

% Beat saber again: but not make a lot of sense? Difficult game has complicated visual effects, while allows larger shift, but do not need large shift with high difficulty



\section{Related Work}
\label{lit_review}

\begin{highlight}
{

Our research builds upon {\em (i)} Assessing Web Accessibility, {\em (ii)} End-User Accessibility Repair, and {\em (iii)} Developer Tools for Accessibility.

\subsection{Assessing Web Accessibility}
From the earliest attempts to set standards and guidelines, web accessibility has been shaped by a complex interplay of technical challenges, legal imperatives, and educational campaigns. Over the past 25 years, stakeholders have sought to improve digital inclusion by establishing foundational standards~\cite{chisholm2001web, caldwell2008web}, enforcing legal obligations~\cite{sierkowski2002achieving, yesilada2012understanding}, and promoting a broader culture of accessibility awareness among developers~\cite{sloan2006contextual, martin2022landscape, pandey2023blending}. 
Despite these longstanding efforts, systemic accessibility issues persist. According to the 2024 WebAIM Million report~\cite{webaim2024}, 95.9\% of the top one million home pages contained detectable WCAG violations, averaging nearly 57 errors per page. 
These errors take many forms: low color contrast makes the interface difficult for individuals with color deficiency or low vision to read text; missing alternative text leaves users relying on screen readers without crucial visual context; and unlabeled form inputs or empty links and buttons hinder people who navigate with assistive technologies from completing basic tasks. 
Together, these accessibility issues not only limit user access to critical online resources such as healthcare, education, and employment but also result in significant legal risks and lost opportunities for businesses to engage diverse audiences. Addressing these pervasive issues requires systematic methods to identify, measure, and prioritize accessibility barriers, which is the first step toward achieving meaningful improvements.

Prior research has introduced methods blending automation and human evaluation to assess web accessibility. Hybrid approaches like SAMBA combine automated tools with expert reviews to measure the severity and impact of barriers, enhancing evaluation reliability~\cite{brajnik2007samba}. Quantitative metrics, such as Failure Rate and Unified Web Evaluation Methodology, support large-scale monitoring and comparative analysis, enabling cost-effective insights~\cite{vigo2007quantitative, martins2024large}. However, automated tools alone often detect less than half of WCAG violations and generate false positives, emphasizing the need for human interpretation~\cite{freire2008evaluation, vigo2013benchmarking}. Recent progress with large pretrained models like Large Language Models (LLMs)~\cite{dubey2024llama,bai2023qwen} and Large Multimodal Models (LMMs)~\cite{liu2024visual, bai2023qwenvl} offers a promising step forward, automating complex checks like non-text content evaluation and link purposes, achieving higher detection rates than traditional tools~\cite{lopez2024turning, delnevo2024interaction}. Yet, these large models face challenges, including dependence on training data, limited contextual judgment, and the inability to simulate real user experiences. These limitations underscore the necessity of combining models with human oversight for reliable, user-centered evaluations~\cite{brajnik2007samba, vigo2013benchmarking, delnevo2024interaction}. 

Our work builds on these prior efforts and recent advancements by leveraging the capabilities of large pretrained models while addressing their limitations through a developer-centric approach. CodeA11y integrates LLM-powered accessibility assessments, tailored accessibility-aware system prompts, and a dedicated accessibility checker directly into GitHub Copilot---one of the most widely used coding assistants. Unlike standalone evaluation tools, CodeA11y actively supports developers throughout the coding process by reinforcing accessibility best practices, prompting critical manual validations, and embedding accessibility considerations into existing workflows.
% This pervasive shortfall reflects the difficulty of scaling traditional approaches---such as manual audits and automated tools---that either demand immense human effort or lack the nuanced understanding needed to capture real-world user experiences. 
%
% In response, a new wave of AI-driven methods, many powered by large language models (LLMs), is emerging to bridge these accessibility detection and assessment gaps. Early explorations, such as those by Morillo et al.~\cite{morillo2020system}, introduced AI-assisted recommendations capable of automatic corrections, illustrating how computational intelligence can tackle the repetitive, common errors that plague large swaths of the web. Building on this foundation, Huang et al.~\cite{huang2024access} proposed ACCESS, a prompt-engineering framework that streamlines the identification and remediation of accessibility violations, while López-Gil et al.~\cite{lopez2024turning} demonstrated how LLMs can help apply WCAG success criteria more consistently---reducing the reliance on manual effort. Beyond these direct interventions, recent work has also begun integrating user experiences more seamlessly into the evaluation process. For example, Huq et al.~\cite{huq2024automated} translate user transcripts and corresponding issues into actionable test reports, ensuring that accessibility improvements align more closely with authentic user needs.
% However, as these AI-driven solutions evolve, researchers caution against uncritical adoption. Othman et al.~\cite{othman2023fostering} highlight that while LLMs can accelerate remediation, they may also introduce biases or encourage over-reliance on automated processes. Similarly, Delnevo et al.~\cite{delnevo2024interaction} emphasize the importance of contextual understanding and adaptability, pointing to the current limitations of LLM-based systems in serving the full spectrum of user needs. 
% In contrast to this backdrop, our work introduces and evaluates CodeA11y, an LLM-augmented extension for GitHub Copilot that not only mitigates these challenges by providing more consistent guidance and manual validation prompts, but also aligns AI-driven assistance with developers’ workflows, ultimately contributing toward more sustainable propulsion for building accessible web.

% Broader implications of inaccessibility—legal compliance, ethical concerns, and user experience
% A Historical Review of Web Accessibility Using WAVE
% "I tend to view ads almost like a pestilence": On the Accessibility Implications of Mobile Ads for Blind Users

% In the research domain, several methods have been developed to assess and enhance web accessibility. These include incorporating feedback into developer tools~\cite{adesigner, takagi2003accessibility, bigham2010accessibility} and automating the creation of accessibility tests and reports for user interfaces~\cite{swearngin2024towards, taeb2024axnav}. 

% Prior work has also studied accessibility scanners as another avenue of AI to improve web development practices~\cite{}.
% However, a persistent challenge is that developers need to be aware of these tools to utilize them effectively. With recent advancements in LLMs, developers might now build accessible websites with less effort using AI assistants. However, the impact of these assistants on the accessibility of their generated code remains unclear. This study aims to investigate these effects.

\subsection{End-user Accessibility Repair}
In addition to detecting accessibility errors and measuring web accessibility, significant research has focused on fixing these problems.
Since end-users are often the first to notice accessibility problems and have a strong incentive to address them, systems have been developed to help them report or fix these problems.

Collaborative, or social accessibility~\cite{takagi2009collaborative,sato2010social}, enabled these end-user contributions to be scaled through crowd-sourcing.
AccessMonkey~\cite{bigham2007accessmonkey} and Accessibility Commons~\cite{kawanaka2008accessibility} were two examples of repositories that store accessibility-related scripts and metadata, respectively.
Other work has developed browser extensions that leverage crowd-sourced databases to automatically correct reading order, alt-text, color contrast, and interaction-related issues~\cite{sato2009s,huang2015can}.

One drawback of collaborative accessibility approaches is that they cannot fix problems for an ``unseen'' web page on-demand, so many projects aim to automatically detect and improve interfaces without the need for an external source of fixes.
A large body of research has focused on making specific web media (e.g., images~\cite{gleason2019making,guinness2018caption, twitterally, gleason2020making, lee2021image}, design~\cite{potluri2019ai,li2019editing, peng2022diffscriber, peng2023slide}, and videos~\cite{pavel2020rescribe,peng2021say,peng2021slidecho,huh2023avscript}) accessible through a combination of machine learning (ML) and user-provided fixes.
Other work has focused on applying more general fixes across all websites.

Opportunity accessibility addressed a common accessibility problem of most websites: by default, content is often hard to see for people with visual impairments, and many users, especially older adults, do not know how to adjust or enable content zooming~\cite{bigham2014making}.
To this end, a browser script (\texttt{oppaccess.js}) was developed that automatically adjusted the browser's content zoom to maximally enlarge content without introducing adverse side-effects (\textit{e.g.,} content overlap).
While \texttt{oppaccess.js} primarily targeted zoom-related accessibility, recent work aimed to enable larger types of changes, by using LLMs to modify the source code of web pages based on user questions or directives~\cite{li2023using}.

Several efforts have been focused on improving access to desktop and mobile applications, which present additional challenges due to the unavailability of app source code (\textit{e.g.,} HTML).
Prefab is an approach that allows graphical UIs to be modified at runtime by detecting existing UI widgets, then replacing them~\cite{dixon2010prefab}.
Interaction Proxies used these runtime modification strategies to ``repair'' Android apps by replacing inaccessible widgets with improved alternatives~\cite{zhang2017interaction, zhang2018robust}.
The widget detection strategies used by these systems previously relied on a combination of heuristics and system metadata (\textit{e.g.,} the view hierarchy), which are incomplete or missing in the accessible apps.
To this end, ML has been employed to better localize~\cite{chen2020object} and repair UI elements~\cite{chen2020unblind,zhang2021screen,wu2023webui,peng2025dreamstruct}.

In general, end-user solutions to repairing application accessibility are limited due to the lack of underlying code and knowledge of the semantics of the intended content.

\subsection{Developer Tools for Accessibility}
Ultimately, the best solution for ensuring an accessible experience lies with front-end developers. Many efforts have focused on building adequate tooling and support to help developers with ensuring that their UI code complies with accessibility standards.

Numerous automated accessibility testing tools have been created to help developers identify accessibility issues in their code: i) static analysis tools, such as IBM Equal Access Accessibility Checker~\cite{ibm2024toolkit} or Microsoft Accessibility Insights~\cite{accessibilityinsights2024}, scan the UI code's compliance with predefined rules derived from accessibility guidelines; and ii) dynamic or runtime accessibility scanners, such as Chrome Devtools~\cite{chromedevtools2024} or axe-Core Accessibility Engine~\cite{deque2024axe}, perform real-time testing on user interfaces to detect interaction issues not identifiable from the code structure. While these tools greatly reduce the manual effort required for accessibility testing, they are often criticized for their limited coverage. Thus, experts often recommend manually testing with assistive technologies to uncover more complex interaction issues. Prior studies have created accessibility crawlers that either assist in developer testing~\cite{swearngin2024towards,taeb2024axnav} or simulate how assistive technologies interact with UIs~\cite{10.1145/3411764.3445455, 10.1145/3551349.3556905, 10.1145/3544548.3580679}.

Similar to end-user accessibility repair, research has focused on generating fixes to remediate accessibility issues in the UI source code. Initial attempts developed heuristic-based algorithms for fixing specific issues, for instance, by replacing text or background color attributes~\cite{10.1145/3611643.3616329}. More recent work has suggested that the code-understanding capabilities of LLMs allow them to suggest more targeted fixes.
For example, a study demonstrated that prompting ChatGPT to fix identified WCAG compliance issues in source code could automatically resolve a significant number of them~\cite{othman2023fostering}. Researchers have sought to leverage this capability by employing a multi-agent LLM architecture to automatically identify and localize issues in source code and suggest potential code fixes~\cite{mehralian2024automated}.

While the approaches mentioned above focus on assessing UI accessibility of already-authored code (\textit{i.e.,} fixing existing code), there is potential for more proactive approaches.
For example, LLMs are often used by developers to generate UI source code from natural language descriptions or tab completions~\cite{chen2021evaluating,GitHubCopilot,lozhkov2024starcoder,hui2024qwen2,roziere2023code,zheng2023codegeex}, but LLMs frequently produce inaccessible code by default~\cite{10.1145/3677846.3677854,mowar2024tab}, leading to inaccessible output when used by developers without sufficient awareness of accessibility knowledge.
The primary focus of this paper is to design a more accessibility-aware coding assistant that both produces more accessible code without manual intervention (\textit{e.g.,} specific user prompting) and gradually enables developers to implement and improve accessibility of automatically-generated code through IDE UI modifications (\textit{e.g.}, reminder notifications).

}
\end{highlight}



% Work related to this paper includes {\em (i)} Web Accessibility and {\em (ii)} Developer Practices in AI-Assisted Programming.

% \ipstart{Web Accessibility: Practice, Evaluation, and Improvements} Substantial efforts have been made to set accessibility standards~\cite{chisholm2001web, caldwell2008web}, establish legal requirements~\cite{sierkowski2002achieving, yesilada2012understanding}, and promote education and advocacy among developers~\cite{sloan2006contextual, martin2022landscape, pandey2023blending}. In the research domain, several methods have been developed to assess and enhance web accessibility. These include incorporating feedback into developer tools~\cite{adesigner, takagi2003accessibility, bigham2010accessibility} and automating the creation of accessibility tests and reports for user interfaces~\cite{swearngin2024towards, taeb2024axnav}. 
% % Prior work has also studied accessibility scanners as another avenue of AI to improve web development practices~\cite{}.
% However, a persistent challenge is that developers need to be aware of these tools to utilize them effectively. With recent advancements in LLMs, developers might now build accessible websites with less effort using AI assistants. However, the impact of these assistants on the accessibility of their generated code remains unclear. This study aims to investigate these effects.

% \ipstart{Developer Practices in AI-Assisted Programming}
% Recent usability research on AI-assisted development has examined the interaction strategies of developers while using AI coding assistants~\cite{barke2023grounded}.
% They observed developers interacted with these assistants in two modes -- 1) \textit{acceleration mode}: associated with shorter completions and 2) \textit{exploration mode}: associated with long completions.
% Liang {\em et al.} \cite{liang2024large} found that developers are driven to use AI assistants to reduce their keystrokes, finish tasks faster, and recall the syntax of programming languages. On the other hand, developers' reason for rejecting autocomplete suggestions was the need for more consideration of appropriate software requirements. This is because primary research on code generation models has mainly focused on functional correctness while often sidelining non-functional requirements such as latency, maintainability, and security~\cite{singhal2024nofuneval}. Consequently, there have been increasing concerns about the security implications of AI-generated code~\cite{sandoval2023lost}. Similarly, this study focuses on the effectiveness and uptake of code suggestions among developers in mitigating accessibility-related vulnerabilities. 


% ============================= additional rw ============================================
% - Paulina Morillo, Diego Chicaiza-Herrera, and Diego Vallejo-Huanga. 2020. System of Recommendation and Automatic Correction of Web Accessibility Using Artificial Intelligence. In Advances in Usability and User Experience, Tareq Ahram and Christianne Falcão (Eds.). Springer International Publishing, Cham, 479–489
% - Juan-Miguel López-Gil and Juanan Pereira. 2024. Turning manual web accessibility success criteria into automatic: an LLM-based approach. Universal Access in the Information Society (2024). https://doi.org/10.1007/s10209-024-01108-z
% - s
% - Calista Huang, Alyssa Ma, Suchir Vyasamudri, Eugenie Puype, Sayem Kamal, Juan Belza Garcia, Salar Cheema, and Michael Lutz. 2024. ACCESS: Prompt Engineering for Automated Web Accessibility Violation Corrections. arXiv:2401.16450 [cs.HC] https://arxiv.org/abs/2401.16450
% - Syed Fatiul Huq, Mahan Tafreshipour, Kate Kalcevich, and Sam Malek. 2025. Automated Generation of Accessibility Test Reports from Recorded User Transcripts. In Proceedings of the 47th International Conference on Software Engineering (ICSE) (Ottawa, Ontario, Canada). IEEE. https://ics.uci.edu/~seal/publications/2025_ICSE_reca11.pdf To appear in IEEE Xplore
% - Achraf Othman, Amira Dhouib, and Aljazi Nasser Al Jabor. 2023. Fostering websites accessibility: A case study on the use of the Large Language Models ChatGPT for automatic remediation. In Proceedings of the 16th International Conference on PErvasive Technologies Related to Assistive Environments (Corfu, Greece) (PETRA ’23). Association for Computing Machinery, New York, NY, USA, 707–713. https://doi.org/10.1145/3594806.3596542
% - Zsuzsanna B. Palmer and Sushil K. Oswal. 0. Constructing Websites with Generative AI Tools: The Accessibility of Their Workflows and Products for Users With Disabilities. Journal of Business and Technical Communication 0, 0 (0), 10506519241280644. https://doi.org/10.1177/10506519241280644
% ============================= additional rw ============================================
A detailed overview of the proposed architecture that converts images and control commands
into trajectories is depicted in~\autoref{fig:monoforce}.
The model consists of several learnable modules that deeply interact with each other.
The \emph{terrain encoder} carefully transforms visual features from the input image
into the heightmap space using known camera geometry.
The resultant heightmap features are further refined into interpretable physical quantities
that capture properties of the terrain such as its shape, friction, stiffness, and damping.
Next, the \emph{physics engine} combines the terrain properties with the robot model,
robot state, and control commands and delivers reaction forces at points of robot-terrain contacts.
It then solves the equations of motion dynamics by integrating these forces
and delivers the trajectory of the robot.
Since the complete computational graph of the feedforward pass is retained,
the backpropagation from an arbitrary loss, constructed on top of delivered trajectories,
or any other intermediate outputs is at hand.

\subsection{Terrain Encoder}\label{subsec:terrain_encoder}

The part of the MonoForce architecture (\autoref{fig:monoforce})
that predicts terrain properties $\mathbf{m}$ from sensor measurements $\mathbf{z}$ is called \emph{terrain encoder}.
The proposed architecture starts by converting pixels from a 2D image plane into a heightmap with visual features.
Since the camera is calibrated, there is a substantial geometrical prior that connects heightmap cells with the pixels.
We incorporate the geometry through the Lift-Splat-Shoot architecture~\cite{philion2020lift}.
This architecture uses known camera intrinsic parameters to estimate rays corresponding to particular pixels~--
pixel rays, \autoref{fig:bevfusion}.
For each pixel ray, the convolutional network then predicts depth probabilities and visual features.
Visual features are vertically projected on a virtual heightmap for all possible depths along the corresponding ray.
The depth-weighted sum of visual features over each heightmap cell is computed,
and the resulting multichannel array is further refined by deep convolutional network
to estimate the terrain properties $\mathbf{m}$.

The terrain properties include the geometrical heightmap $\mathcal{H}_g$,
the heights of the terrain supporting layer hidden under the vegetation $\mathcal{H}_t = \mathcal{H}_g - \Delta\mathcal{H}$,
terrain friction $\mathcal{M}$, stiffness $\mathcal{K}$, and dampening $\mathcal{D}$.
The intuition behind the introduction of the $\Delta\mathcal{H}$ term is
that $\mathcal{H}_t$ models a partially flexible layer of terrain (e.g. mud) that is hidden under flexible vegetation,~\autoref{fig:monoforce_heightmaps}.


\subsection{Differentiable Physics Engine}\label{subsec:dphysics}
The differentiable physics engine solves the robot motion equation and estimates
the trajectory corresponding to the delivered forces.
The trajectory is defined as a sequence of robot states $\tau = \{s_0, s_1, \ldots, s_T\}$,
where $\mathbf{s}_t = [\mathbf{x}_t, \mathbf{v}_t, R_t, \boldsymbol{\omega}_t]$
is the robot state at time $t$,
$\mathbf{x}_t \in \mathbb{R}^3$ and $\mathbf{v}_t \in \mathbb{R}^3$ define the robot's position and velocity in the world frame,
$R_t \in \mathbb{R}^{3 \times 3}$ is the robot's orientation matrix, and $\boldsymbol{\omega}_t \in \mathbb{R}^3$ is the angular velocity.
To get the next state $\mathbf{s}_{t+1}$, in general, we need to solve the following ODE:
\begin{equation}
    \label{eq:state_propagation}
    \mathbf{\dot{s}}_{t+1} = f(\mathbf{s}_t, \mathbf{u}_t, \mathbf{z}_t)
\end{equation}
where $\mathbf{u}_t$ is the control input and $\mathbf{z}_t$ is the environment state.
In practice, however, it is not feasible to obtain the full environment state $\mathbf{z}_t$.
Instead, we utilize terrain properties $\mathbf{m}_t = [\mathcal{H}_t, \mathcal{K}_t, \mathcal{D}_t, \mathcal{M}_t]$
predicted by the terrain encoder.
In this case, the motion ODE~\eqref{eq:state_propagation} can be rewritten as:
\begin{equation}
    \label{eq:state_propagation_terrain}
    \mathbf{\dot{s}}_{t+1} = \hat{f}(\mathbf{s}_t, \mathbf{u}_t, \mathbf{m}_t)
\end{equation}

Let's now derive the equation describing the state propagation function $\hat{f}$.
The time index $t$ is omitted further for brevity.
We model the robot as a rigid body with total mass $m$ represented by a~set of mass points
$\mathcal{P} = \{(\mathbf{p}_i, m_i)\; | \; \mathbf{p}_i~\in~\mathbb{R}^3, m_i~\in~\mathbb{R}^+, i=1~\dots~N\}$,
where $\mathbf{p}_i$ denotes coordinates of the $i$-th 3D point in the robot's body frame.
We employ common 6DOF dynamics of a rigid body~\cite{contact_dynamics-2018} as follows:
\begin{equation}
  \begin{split}
    \dot{\mathbf{x}} &= \mathbf{v}\\
    \dot{\mathbf{v}} &= \frac{1}{m}\sum_i\mathbf{F}_i
  \end{split}
  \quad\quad
  \begin{split}
    \dot{R} &= \Omega R\\
    \dot{\boldsymbol{\omega}} &= \mathbf{J}^{-1}\sum_i \mathbf{p}_i\times\mathbf{F}_i
  \end{split}
  \label{eq:contact_dynamics}
\end{equation}
where $\Omega = [\boldsymbol{\omega}]_{\times}$ is the skew-symmetric matrix of $\boldsymbol{\omega}$.
We denote $\mathbf{F}_i\in\mathbb{R}^3$ a total external force acting on $i$-th robot's body point.
Total mass $m = \sum_i~m_i$ and moment of inertia $\mathbf{J}\in\mathbb{R}^{3\times 3}$ of the robot's rigid body are assumed to be known
static parameters since they can be identified independently in laboratory conditions.
Note that the proposed framework allows backpropagating the gradient with respect to these quantities, too,
which makes them jointly learnable with the rest of the architecture.
The trajectory of the rigid body is the iterative solution of differential equations~\eqref{eq:contact_dynamics},
that can be obtained by any ODE solver for given external forces and initial state (pose and velocities).

When the robot is moving over a terrain, two types of external forces are acting
on the point cloud $\mathcal{P}$ representing its model:
(i) gravitational forces and (ii) robot-terrain interaction forces.
The former is defined as $m_i\mathbf{g} = [0, 0, -m_ig]^\top$ and acts on
all the points of the robot at all times,
while the latter is the result of complex physical interactions that are not easy
to model explicitly and act only on the points of the robot that are in contact
with the terrain.
There are two types of robot-terrain interaction forces:
(i) normal terrain force that prevents the penetration of the terrain by the robot points,
(ii) tangential friction force that generates forward acceleration when the tracks are moving,
and prevents side slippage of the robot.

\textbf{Robot-terrain interaction forces}

\begin{figure}[t]
    \centering
    \includegraphics[width=0.7\columnwidth]{imgs/dphysics/spring_forces}
    \caption{\textbf{Terrain force model}: Simplified 2D sketch demonstrating
    normal reaction forces acting on a robot body consisting of two points $p_i$ and $p_j$ .}
    \label{fig:spring_terrain_model}
\end{figure}

\textit{Normal reaction forces}.

One extreme option is to predict the 3D force vectors $\mathbf{F}_i$ directly
by a neural network, but we decided to enforce additional prior assumptions to reduce the risk of overfitting.
These prior assumptions are based on common intuition from the contact dynamics of flexible objects.
In particular, we assume that the magnitude of the force that the terrain exerts on the point $\mathbf{p}_i\in \mathcal{P}$
increases proportionally to the deformation of the terrain.
Consequently, the network does not directly predict the force,
but rather predicts the height of the terrain $h\in\mathcal{H}_t$
at which the force begins to act on the robot body and the stiffness of the terrain $e\in\mathcal{K}$.
We understand the quantity $e$ as an equivalent of the spring constant from Hooke's spring model, \autoref{fig:spring_terrain_model}.
Given the stiffness of the terrain and the point of the robot that penetrated the terrain
by ${\Delta}h$, the reaction force is calculated as $e\cdot{\Delta}h$.
% \begin{figure}[t]
%     \centering
%     \includegraphics[width=0.4\columnwidth]{imgs/dphysics/robot-terrain_forces}
%     \caption{\textbf{Robot-terrain interaction forces} acting on the robot's body at its contact points
%     with the terrain.
%     The point cloud was sampled from the MARV (\autoref{fig:robot_platforms}(b)) robot's 3D model.}
%     \label{fig:interaction_forces}
% \end{figure}

Since such a force, without any additional damping, would lead to an eternal bumping
of the robot on the terrain, we also introduce a robot-terrain damping coefficient $d\in\mathcal{D}$,
which similarly reduces the force proportionally to the velocity of the point
that is in contact with the terrain.
The model applies reaction forces in the normal direction $\mathbf{n}_i$ of the terrain surface,
where the $i$-th point is in contact with the terrain.
\begin{equation}\label{eq:normal_force}
    \mathbf{N}_{i} = \begin{cases}
 (e_i\Delta h_i - d_i(\dot{\mathbf{p}}_{i}^\top\mathbf{n}_i))\mathbf{n}_i  & \text{if } \mathbf{p}_{zi}\leq h_i \\
\mathbf{0} & \text{if } \mathbf{p}_{zi}> h_i
\end{cases},
\end{equation}
where terrain penetration $\Delta h_i = (h_i-\mathbf{p}_{zi})\mathbf{n}_{zi}$ is
estimated by projecting the vertical distance on the normal direction.
For a better gradient propagation, we use the smooth approximation of the Heaviside step function:
\begin{equation}
    \label{eq:smooth_normal_force}
    \mathbf{N}_i = (e_i\Delta h_i - d_i(\dot{\mathbf{p}}_{i}^\top\mathbf{n}_i))\mathbf{n}_i \cdot \sigma(h_i - \mathbf{p}_{zi}),
\end{equation}
where $\sigma(x) = \frac{1}{1+e^{-kx}}$ is the sigmoid function with a steepness hyperparameter $k$.

\begin{figure}[t]
    \centering
    \includegraphics[width=\columnwidth]{imgs/dphysics/optimization}
    \caption{\textbf{Terrain computed by backpropagating through $\nabla$Physics:}
    Shape of the terrain (border of the area where terrain forces start to act) outlined by heightmap surface,
    its color represents the friction of the terrain.
    The optimized trajectory is in green, and the ground truth trajectory is in blue.}
    \label{fig:terrain_optim}
\end{figure}

\textit{Tangential friction forces}.

Our tracked robot navigates by moving the main tracks and 4 flippers (auxiliary tracks).
The flipper motion is purely kinematic in our model.
This means that in a given time instant, their pose is uniquely determined by a $4$-dimensional vector
of their rotations, and they are treated as a rigid part of the robot.
The motion of the main tracks is transformed into forces tangential to the terrain.
The friction force delivers forward acceleration of the robot when robot tracks
(either on flippers or on main tracks) are moving.
At the same time, it prevents the robot from sliding sideways.
When a robot point $\mathbf{p}_i$, which belongs to a track, is in contact with terrain with
friction coefficient $\mu\in\mathcal{M}$, the resulting friction force at a contact point is computed as follows,~\cite{yong2012vehicle}:
\begin{equation}\label{eq:friction_force}
    \mathbf{F}_{f, i} = \mu_i |\mathbf{N}_i| ((\mathbf{u}_i - \mathbf{\dot{p}}_i)^\top\boldsymbol{\tau}_i)\boldsymbol{\tau}_i,
\end{equation}
where $\mathbf{u}_i = [u, 0, 0]^\top$, $u$ is the velocity of a track, and $\mathbf{\dot{p}}_i$ is the velocity of the point $\mathbf{p}_i$
with respect to the terrain transformed into the robot coordinate frame,
$\boldsymbol{\tau}_i$ is the unit vector tangential to the terrain surface at the point $\mathbf{p}_i$.
This model can be understood as a simplified Pacejka's tire-road model~\cite{pacejka-book-2012}
that is popular for modeling tire-road interactions.

To summarize, the state-propagation ODE~\eqref{eq:state_propagation_terrain}
(state $\mathbf{s}~=~[\mathbf{x},~\mathbf{v},~R,~\boldsymbol{\omega}]$) for a mobile robot moving over a terrain
is described by the equations of motion~\eqref{eq:contact_dynamics} where the force applied at a robot's $i$-th body point is computed as follows:
\begin{equation}\label{eq:forces}
    \begin{split}
        \mathbf{F}_i &= m_i\mathbf{g} + \mathbf{N}_i + \mathbf{F}_{f, i}
    \end{split}
\end{equation}
The robot-terrain interaction forces at contact points $\mathbf{N}_i$ and $\mathbf{F}_{f, i}$
are defined by the equations~\eqref{eq:smooth_normal_force} and~\eqref{eq:friction_force} respectively.


\textbf{Implementation of the Differentiable ODE Solver}

We implement the robot-terrain interaction ODE~\eqref{eq:contact_dynamics} in PyTorch~\cite{Paszke-NIPS-2019}.
The \textit{Neural ODE} framework~\cite{neural-ode-2021} is used to solve the system of ODEs.
For efficiency reasons, we utilize the Euler integrator for the ODE integration.
The differentiable ODE solver~\cite{neural-ode-2021} estimates the gradient through the implicit function theorem.
Additionally, we implement the ODE~\eqref{eq:contact_dynamics} solver that
estimates gradient through \textit{auto-differentiation}~\cite{Paszke-NIPS-2019},
i.e. it builds and retains the full computational graph of the feedforward integration.


\subsection{Data-driven Trajectory Prediction}\label{subsec:data_driven_baseline}
Inspired by the work~\cite{pang2019aircraft}, we design a data-driven LSTM architecture (\autoref{fig:traj_lstm}) for our outdoor mobile robot's trajectory prediction.
We call the model TrajLSTM and use it as a baseline for our $\nabla$Physics engine.
\begin{figure}
    \centering
    \includegraphics[width=\columnwidth]{imgs/architectures/lstm}
    \caption{\textbf{TrajLSTM} architecture. The model takes as input: initial state $\mathbf{x}_0$, terrain $\mathcal{H}$, control sequence $\mathbf{u}_t, t \in \{0 \dots T\}$. It predicts the trajectory as a sequence of states $\mathbf{x}_t, t \in \{0 \dots T\}.$}
    \label{fig:traj_lstm}
\end{figure}
Given an initial robot's state $\mathbf{x}_0$ and a sequence of control inputs for a time horizon $T$, $\mathbf{u}_t, t \in \{0 \dots T\}$, the TrajLSTM model provides a sequence of states at control command time moments, $\mathbf{x}_t, t \in \{0 \dots T\}$.
As in outdoor scenarios the robot commonly traverses uneven terrain, we additionally include the terrain shape input to the model in the form of heightmap $\mathcal{H}=\mathcal{H}_0$ estimated at initial time moment $t=0$.
Each timestep's control input $\mathbf{u}_i$ is concatenated with the shared spatial features $\mathbf{x}_i$, as shown in \autoref{fig:traj_lstm}.
The combined features are passed through dense layers to prepare for temporal processing.
The LSTM unit~\cite{hochreiter1997long} processes the sequence of features (one for each timestep).
As in our experiments, the time horizon for trajectory prediction is reasonably small, $T=5 [\si{\sec}]$, and the robot's trajectories lie within the heightmap area, we use the shared heightmap input for all the LSTM units of the network.
So the heightmap is processed through the convolutional layers \textbf{once} and flattened, producing a fixed-size spatial feature vector.
This design choice (of not processing the heightmaps at different time moments) is also motivated by computational efficiency reason.
At each moment $t$, this heightmap vector is concatenated with the fused spatial-control features and processed by an LSTM unit.
The LSTM unit output for each timestep $t$ is passed through a fully connected (dense) layer to produce the next state $\mathbf{x}_{t+1}$.
The sequence of states form the predicted trajectory, $\{\mathbf{x}_0, \dots \mathbf{x}_T\}$.


\subsection{End-to-end Learning}\label{subsec:end2end_learning}
Self-supervised learning of the proposed architecture minimizes three different losses:

\textbf{Trajectory loss} that minimizes
the difference between SLAM-reconstructed trajectory $\tau^\star$ and predicted trajectory $\tau$:
\begin{equation}~\label{eq:traj_loss}
   \mathcal{L}_\tau = \|\tau-\tau^\star\|^2
\end{equation}

\textbf{Geometrical loss} that minimizes the difference between
ground truth lidar-reconstructed heightmap $\mathcal{H}_g^\star$
and predicted geometrical heightmap $\mathcal{H}_g$:
 \begin{equation}~\label{eq:geom_loss}
     \mathcal{L}_g = \|\mathbf{W}_g\circ(\mathcal{H}_g-\mathcal{H}_g^\star)\|^2
 \end{equation}
$\mathbf{W}_g$ denotes an array selecting the heightmap channel corresponding to the terrain shape.

\textbf{Terrain loss} that minimizes the difference between ground truth $\mathcal{H}_t^\star$
and predicted $\mathcal{H}_t$ supporting heightmaps containing rigid objects detected
with Microsoft's image segmentation model SEEM~\cite{zou2023segment},
that is derived from Segment Anything foundation model~\cite{li2023semantic}:
 \begin{equation}~\label{eq:terrain_loss}
     \mathcal{L}_t = \|\mathbf{W}_t\circ(\mathcal{H}_t-\mathcal{H}_t^\star)\|^2
 \end{equation}
$\mathbf{W}_t$ denotes the array selecting heightmap cells that are covered by rigid materials
(e.g. stones, walls, trunks), and $\circ$ is element-wise multiplication.

Since the architecture \autoref{fig:model_overview} is end-to-end differentiable,
we can directly learn to predict all intermediate outputs just using trajectory loss~\eqref{eq:traj_loss}.
An example of terrain learning with the trajectory loss is visualized in \autoref{fig:terrain_optim}.
To make the training more efficient and the learned model explainable, we employ the
geometrical loss~\eqref{eq:geom_loss} and terrain loss~\eqref{eq:terrain_loss} as regularization terms.
stat

\begin{figure*}
    \centering
    \includegraphics[width=\textwidth]{imgs/predictions/monoforce/qualitative_results_experiments}
    \caption{\textbf{MonoForce prediction examples}.
    \emph{Left}: The robot is moving through a narrow passage between a wall and tree logs.
    \emph{Right}: The robot is moving on a gravel road with rocks on the sides.
    It starts its motion from the position marked with a coordinate frame and the trajectory is predicted for $10~[\si{\sec}]$ using real control commands.
    The camera images are taken from the robot's initial position (\emph{top row}).
    The visualization includes predicted supporting terrain $\mathcal{H}_t$ (\emph{second row}).
    It is additionally shown in 3D and colored with predicted friction values (\emph{third row}).
    }
    \label{fig:monoforce_predictions}
\end{figure*}

The \autoref{fig:monoforce_predictions} show the prediction examples of the MonoForce model in diverse outdoor environments.
From the example on the left,
we can see that the model correctly predicts the robot's trajectory and the terrain shape suppressing traversable vegetation,
while the rigid obstacles (wall and tree logs) are correctly detected.
The example on the right demonstrates the model's ability to predict the robot's trajectory ($10~[\si{\sec}]$-long horizon)
with reasonable accuracy and to detect the rigid obstacles (stones) on the terrain.
It could also be noticed that the surfaces that provide the robot good traction (paved and gravel roads) are marked with a higher friction value,
while for the objects that might not give good contact with the robot's tracks (walls and tree logs) the friction value is lower.

We argue that the friction estimates are approximate and an interesting research direction could be
comparing them with real-world measurements or with the values provided by a high-fidelity physics engine (e.g. AGX Dynamics~\cite{Berglund2019agxTerrain}).
However, one of the benefits of our differentiable approach is that the model does not require ground-truth friction values for training.
The predicted heightmap's size is $12.8\times12.8\si{\meter}^2$ and the grid resolution is $0.1\si{\meter}$.
It has an upper bound of $1~[\si{\meter}]$ and a lower bound of $-1~[\si{\meter}]$.
This constraint was introduced based on the robot's size and taking into account hanging objects (tree branches)
that should not be considered as obstacles (\autoref{fig:nav_monoforce}).
Additionally, the terrain is predicted in the gravity-aligned frame.
That is made possible thanks to the inclusion of camera intrinsics and extrinsics as input to the model,
\autoref{fig:monoforce}.
It also allows correctly modeling the robot-terrain interaction forces (and thus modeling the robot's trajectory accurately)
for the scenarios with non-flat terrain, for example, going uphill or downhill.
This will not be possible if only camera images are used as input.
\section{Experiments}
\label{sec:experiment}

\subsection{Experimental Setup}\label{sec:exp_set}
\noindent \textbf{Implementation Details.} 
Our proposed model is fine-tuned on VITON-HD~\cite{choi2021viton}. As with other works~\cite{xu2024ootdiffusion,choi2024improving,velioglu2024tryoffdiff}, we divide it into a training dataset and a testing dataset. Then, we use IDM~\cite{choi2024improving} to prepare the custom datasets for person-to-person task and manually filter out a subset for training. We adopt the FLUX-Fill-dev~\cite{flux} as our foundation model and fine-tuning it on both garment-to-person and person-to-person datasets. In inference stage, the model samples 30 steps to get the final fitting outputs.

\subsection{Qualitative and Quantitative Comparison}\label{sec:exp_comp}
We compare our model with garment-to-person methods OOTD~\cite{xu2024ootdiffusion}, IDM~\cite{choi2024improving}, and CatVTON-FLUX~\cite{catvton-flux}. To adapt these methods for person-to-person tasks, we employ segmentation~\cite{ravi2024sam} and try-off~\cite{velioglu2024tryoffdiff} to extract garment from the reference person. We initially utilize unpaired testing datasets and assess the fidelity of the generated fitting image distributions with three key metrics: FID~\cite{heusel2017gans}, CLIP-FID~\cite{kynkaanniemi2022role} and KID~\cite{binkowski2018demystifying} metrics. In order to more fully evaluate our model, we process the testing dataset using the data preparation method outlined in~\cref{sec:data_preparation} and extract paired datasets such as $\left(P_{mn}, P_{nm}, P_{mm}\right)$ and $\left(P_{nm}, P_{mn}, P_{nn}\right)$. On this dataset, we evaluate the aforementioned metrics and additionally compute SSIM~\cite{wang2004image}, LPIPS~\cite{zhang2018unreasonable} and DISTS~\cite{ding2020image} to evaluate the reconstruction quality between the generated fitting image and corresponding ground truth.

\begin{figure*}[ht]
    \centering
    \includegraphics[width=0.95\linewidth]{figs/fig4_method.png}
    \caption{Qualitative comparison. The first two columns show the inputs to different models. In the person-to-person task, the three garment-to-person methods rely on segmentation and try-off techniques to obtain the garment on the reference person. In contrast, our method directly generates the outputs based on the reference person.}
    \label{fig:fig4_method}
\end{figure*}
\noindent \textbf{Qualitative Comparison.}
As illustrated in~\cref{fig:fig4_method}, our method achieves superior fidelity in person-to-person task. While other methods can adapt to person-to-person task using segmentation or try-off techniques, they often introduce significant artifacts. Despite not requiring a separate input of the person pose, our method effectively preserves the original pose with high accuracy.


\begin{table*}[htbp]
\centering
\begin{tabular}{l|cccccc|ccc}
\toprule
\multirow{2}{*}{Model} & \multicolumn{6}{c|}{Paired Person2Person}            & \multicolumn{3}{c}{Unpaired Person2Person} \\ \cmidrule(){2-10} 
                       & SSIM$\uparrow$    & LPIPS$\downarrow$  & DISTS$\downarrow$  & FID$\downarrow$     & CLIP-FID$\downarrow$ & KID*$\downarrow$    & FID$\downarrow$             & CLIP-FID$\downarrow$       & KID*$\downarrow$          \\ \midrule
Seg+OOTD             & 0.8404 & 0.1445 & 0.1081 & 12.4351  & 3.3757 & 3.5754      & 13.3704   & 3.9595        & 4.3530       \\
Seg+IDM              & \underline{0.8727} & 0.1170 & 0.0957 & 11.0887  & 2.6419 & 3.6665      & 10.8623  & \underline{2.6477}       & 3.0886       \\
Seg+CatVTON     & 0.8715 & 0.1150 & \underline{0.0897} & \underline{9.7622}  & 2.9928 & 2.5167      & 10.6096  & 3.0508       & 2.8575       \\ \midrule
TROF+OOTD            & 0.8409 & 0.1368 & 0.1047 & 11.1590  & 3.0541 & \underline{2.1543}     & 11.7932   & 3.5123       & 2.5396       \\
TROF+IDM             & \textbf{0.8761} & \underline{0.1139} & 0.0950 & 10.5302  & \underline{2.5589} & 2.3982      & 11.2508  & 2.7594       & 2.5920      \\
TROF+CatVTON    & 0.8723 & 0.1158 & 0.0923 & 9.8190  & 2.6181 & \textbf{1.9341}       & \underline{10.5839}  & 2.7688        & \textbf{2.3509}       \\  \midrule
Ours                 & 0.8688 & \textbf{0.1122} & \textbf{0.0870} & \textbf{9.3223} & \textbf{2.1333}   & 2.1581  & \textbf{10.3465}         & \textbf{2.2885}        & \underline{2.4658}      \\ \bottomrule
\end{tabular}
\caption{Quantitative comparison with other methods on person-to-person task. The KID metric is multiplied by the factor 1e3 to ensure a similar order of magnitude to the other metrics.}
\label{tab:quantitative_person}
\end{table*}









\begin{table*}[htbp]
\centering
\begin{tabular}{l|cccccc|ccc}
\toprule
\multirow{2}{*}{Model} & \multicolumn{6}{c|}{Paired Garment2Person}            & \multicolumn{3}{c}{Unpaired Garment2Person} \\ \cmidrule(){2-10} 
                       & SSIM$\uparrow$    & LPIPS$\downarrow$  & DISTS$\downarrow$  & FID$\downarrow$     & CLIP-FID$\downarrow$ & KID*$\downarrow$    & FID$\downarrow$             & CLIP-FID$\downarrow$       & KID*$\downarrow$          \\ \midrule
OOTD             & 0.8556 & 0.1118 & 0.0849 & 6.8680  & 2.2030 & \textbf{1.4632}       & 9.8221 & 2.8306 & \textbf{1.6700}      \\
IDM              & \textbf{0.8789} & \textbf{0.0940} & 0.0806 & 6.6752  & 2.1008 & 1.7398   & \underline{9.6548} & \underline{2.4607} & 1.8081       \\
CatVTON     & \underline{0.8774} & \underline{0.0975} & \textbf{0.0776} & \textbf{6.3788}  & 2.2642 & 1.6641      & 9.7696 & 2.7375 & 2.0727      \\ 
Ours                 & 0.8761 & 0.0986 & \underline{0.0790} & \underline{6.4206} & \textbf{1.8431}   & \underline{1.5260}  & \textbf{9.5728} & \textbf{2.2566} & \underline{1.7624}      \\ \bottomrule
\end{tabular}
\caption{Quantitative comparison with other methods on person-to-person task. The KID metric is multiplied by the factor 1e3 to ensure a similar order of magnitude to the other metrics.}
\label{tab:quantitative_garment}
\end{table*}
\noindent \textbf{Quantitative Comparison.}
Quantitative results demonstrate that our method excels in both person-to-person task, as evidenced in~\cref{tab:quantitative_person}, and garment-to-person task, as shown in~\cref{tab:quantitative_garment}, outperforming existing methods across multiple metrics. Additionally, quantitative results indicate that the try-off method is more effective than the segmentation method in facilitating the realization of person-to-person tasks.
\section{Application: Generating Reactive Motion with Sparse Signals}
Introducing sparse control is essential for making our method practical in real-world applications, particularly in VR online interactive environments. In these settings, capturing detailed and dense input data can be challenging due to hardware limitations, computational costs, or user comfort. Sparse control addresses this by allowing the system to generate high-quality motion based on minimal input signals.

To demonstrate that our method is well-suited for VR online interactive environments, we also conducted experiments showing that it can be controlled by sparse signals.
The sparse signals are the head and two-hand positions and rotations relative to the previous frame's agent root coordinate.
To enable the controlling feature, we retrain the reaction policy by adding the sparse signals as conditions to the two transformer models in \Figref{fig:pipeline}. The loss and other training settings remain unchanged.
We evaluate the quality of the generated motion using FID scores, motion jitter, foot sliding, and position and rotation errors to highlight the controllability of our approach.
We compare our method with \blcamdm~\citep{2024_camdm}, an auto-regressive method that generates diverse motions based on control signals. The results, presented in \Tabref{tab:control}, show that our method consistently outperforms the baseline across all evaluated metrics.
We also provide qualitative results in \Figref{fig:sparse} and in the supplementary materials.

\section{Hybrid Control Framework} \label{sec:control}
The convergence of $\xi$ and $\eta$ states to the desired set is valid only when the initial position of the robot is within \pn{$\mc{N}_\Gamma^{\by}$}. To guarantee the global convergence and path invariance, this paper proposes a strategy that generates a motion plan from the initial state to the desired path and employs a global tracking controller $\kappa_1:\Real^4 \to \Real^2$ to track the generated motion plan. As a result, the robot enters the neighborhood of the desired path within a finite time. Through a robust uniting control framework in \cite{San2021}, the local path-invariant controller $\kappa_0$ is activated, leveraging its convergence and invariance properties to ensure global convergence and path invariance.

% By designing this switching scheme, the proposed hybrid controller is able to establish the global convergence and invariance.

% \subsection{Neighborhood of a Path} \label{sec:NH}
% {
%   \myred AA: This subsection needs to be cleaned up, and moved to the start where we defined the parametric curves. Moreover, the neighbourhood of the path need to be precisely defined. Moreover, we can't use $\kappa$ for curvature as it is used to define the controllers. }
% From \cite{dynamictransvarsefeedback} we get the definition of our curve $\mathcal{N(\mathcal{C})} \subset \mathbb{R}^2$, where $\mathcal{C}$ defines the desired path. For simplicity, the notation for this neighborhood has been reduced to $\mathcal{N}$. The set $\mathcal{N}$ is defined as a function of the radius of curvature along the path. Note that the radius of curvature of a point on a curve is given by \cite{mate2017frenet}
% \begin{equation}
%     \kappa = \frac{|r' \times r''|}{|r'|^3}.
% \end{equation}
%  The distance spanned by the neighborhood orthogonal to a point on the curve is inversely related to the curvature at that point. For a visual example, consider the path described by the equation $y = \sin(x)$. The neighborhood of this curve can be visualized in Figure \ref{fig:NH}. 
% \begin{figure}[ht]
%     \centering
%     \includegraphics[width = \columnwidth]{Figures/neighborhood.eps}
%     \caption{Neighborhood of a curve}
%     \label{fig:NH}
% \end{figure}

% At the point on the curve where the curvature approaches zero, and the curve becomes flat, the radius of curvature approaches infinity. In Figure \ref{fig:NH}, these points have been saturated to shrink the set $\mathcal{N}$. The motivation for this decision is that if the robot far from the desired path, the controller should rely on $\kappa_1$. As the curvature of the path increases at the local minima and maxima points, the magnitude of the neighborhood orthogonal to the path at that point decreases. 


% Consider the case of a circular path. From \cite{mate2017frenet}, we know that the curvature of a circle at every point is its radius $R$. Therefore, the only point not in the neighborhood of a circular path is at its center. 

% \subsection{Dynamic Transverse Feedback}
% {
%   \myred AA: We don't need this subsection here. We have pretty much covered this discussion before.
%   }
% The control strategy described by $\kappa_0$ was given by Adeel Akhtar in \cite{dynamictransvarsefeedback}. It successfully makes a broad subclass of curves invariant and attractive for the kinematic bicycle model. One limitation of this work is that it is only true locally, that is, the path is made locally invariant. This is due to the singularities that arise from the differential geometry used. The equation given by Akhtar for the output of the controller is 
% \begin{equation} \label{eq:8}
%     \begin{bmatrix} u_1 \\
%                     u_2
%     \end{bmatrix} =
%     D^{-1}(x)(
%     \begin{bmatrix} -L^3_f\pi \\
%                     -L^3_f\alpha
%     \end{bmatrix} +
%     \begin{bmatrix} v^{\parallel} \\
%                     v^{\pitchfork}
%     \end{bmatrix} )
% \end{equation}

% Where $L^n$ refers to the $n^{th}$ Lie derivative, $\pi$ and $\alpha$ are representations of our curve, and $v^{\parallel}$ and $v^{\pitchfork}$ are the transversal and tangential control inputs. When the robot is outside the neighborhood of the desired path, the decoupling matrix $D$ becomes singular and can no longer be inverted, thus $\kappa_0$ can no longer generate control inputs.  The values for $u_1$ and $u_2$ represent the steering angle rate $\omega$ and the derivative of the robot acceleration to be applied. To use this controller in our system, we can integrate $u_1$ and $u_2$ to get our desired control values of steering angle and velocity. To better understand Equation \ref{eq:8}, please refer to \cite{dynamictransvarsefeedback}. 


\subsection{Motion Plan Generation}
\label{sec:trajectory_generation}
The foremost step in this strategy is to generate a motion plan from the current position to the path. This employs the motion planning technique to solve the following motion planning problem for (\ref{eq:car_robot}): 
% To relieve the curse of dimensions in motion planning, a simplified model of~\eqref{eq:car_robot} with state $\Tilde{x}:= (x_{1}, x_{2}, x_{3 })$ is considered in the motion planning software as
% \begin{equation}\label{eq:simplified_model}
% \dot{\Tilde{x}} = \begin{bmatrix}
%     \dot{x}_{1}\\
%     \dot{x}_{2}\\
%     \dot{x}_{3}
% \end{bmatrix} = 
%     \begin{bmatrix}
%         v\cos{x_{3}}\\
%         v\sin{x_{3}}\\
%         \frac{v\tan{\delta}}{l},
%     \end{bmatrix},
% \end{equation}
% where the velocity $v\in [v_{min}, v_{max}]$ is considered as a constant parameter and the steering angle $\delta\in [\delta_{min}, \delta_{max}]$ is considered as an input.
\begin{problem}\label{problem:mp}
    Given the initial state of the robot $x_{0}\in \mathbb{R}^{4}$, the final state set $X_{f} := \{(x_{1}, x_{2}, x_{3}, x_{4})\in \mathbb{R}^{4}: \exists (x_{5}, x_{6})\in \mathbb{R}^{2} \text{ such that } (x_{1}, x_{2}, x_{3}, x_{4}, x_{5}, x_{6})\in \Gamma\}$, the arbitrary unsafe set $X_{u}\subset \mathbb{R}^{4}$ that denotes the obstacles in the simulation \myifconf{as in Figures \ref{fig:sim}}{and experiments as in Figures \ref{fig:sim} and \ref{fig:exp1}}, and the system model (\ref{eq:car_robot}), the motion planning module generates a motion plan $(x'_{1}, x'_{2}, x_{3}', x_{4}'):[0, T]\to \mathbb{R}^{4}$ for some $T > 0$ such that\myifconf{1) $(x'_{1}(0), x'_{2}(0), x_{3}'(0), x'_{4}(0)) = x_{0}$;
    2) $(x'_{1}(T), x'_{2}(T), x_{3}'(T), x'_{4}(T))\in X_{f}$;
    3) there exists an input trajectory $(v', \omega') :[0, T]\to \mathbb{R}^{2}$ such that the state trajectory $(x'_{1}, x'_{2}, x_{3}', x_{4}')$ with input trajectory $(v', \omega')$ satisfies (\ref{eq:car_robot});
    4) there does not exist $t\in [0, T]$ such that $(x'_{1}(t), x'_{2}(t), x_{3}'(t), x_{4}'(t))\in X_{u}$.}{\begin{enumerate}
    \item $(x'_{1}(0), x'_{2}(0), x_{3}'(0), x'_{4}(0)) = x_{0}$;
    \item $(x'_{1}(T), x'_{2}(T), x_{3}'(T), x'_{4}(T))\in X_{f}$;
    \item there exists an input trajectory $(v', \omega') :[0, T]\to \mathbb{R}^{2}$ such that the state trajectory $(x'_{1}, x'_{2}, x_{3}', x_{4}')$ with input trajectory $(v', \omega')$ satisfies (\ref{eq:car_robot});
    \item there does not exist $t\in [0, T]$ such that $(x'_{1}(t), x'_{2}(t), x_{3}'(t), x_{4}'(t))\in X_{u}$.
\end{enumerate}}

\end{problem}
\myifconf{If no solution to Problem \ref{problem:mp} exists, the desired path $\Gamma$ is unreachable from the given initial state, making it impossible to guide the robot toward $\Gamma$. To the best of the authors' knowledge, no theoretical results currently verify the existence of a motion plan. Assuming at least one exists, complete motion planners are guaranteed to find it, though they are challenging to implement in practice. This paper uses the HyRRT motion planning tool from \cite{wang2022rapidly, wang2024motion, wang2023hysst, xu2024chyrrt, wang2024hyrrt, wang2023hysst1}, which is probabilistically complete and suitable for systems like (\ref{eq:car_robot}), despite being designed for hybrid systems.}{If no solution to Problem \ref{problem:mp} exists, then the desired path $\Gamma$ is not reachable from the given initial state, and, hence, it is impossible to drive the robot toward $\Gamma$. From the authors' best knowledge, there are no existing theoretical results to verify the existence of the motion plan. Assuming that at least one motion plan exists, complete motion planners are guaranteed to find it. However, in practice, the complete motion planner is difficult, if not impossible, to implement. In this paper, the HyRRT motion planning software tool in \cite{wang2022rapidly} is probabilistically complete and, though designed for hybrid systems, is suitable to generate the motion plan for systems like (\ref{eq:car_robot}).}
% Wang’s algorithm uses the hybrid model of the system to propagate forward in time from the initial position and backward in time from the target set. The two propagations are concatenated when an overlap is found to form the trajectory. It takes as inputs a hybrid system, a target set $T \subset \mathbb{R}^3$, and an unsafe set $U \subset \mathbb{R}^2$. The unsafe set is important because it allows our trajectory to consider obstacles when generating the trajectory. 
Since the motion plan is collision-free, the proposed hybrid controller is able to avoid the obstacles outside $\mathcal{N}_\Gamma^{\by}$. 
% This is illustrated in Figure \ref{fig:trajectory}, with the red boxes indicating obstacles to be avoided. 
% The offset of the generated trajectory to the target path in Figure \ref{fig:circletrajectory} is due to how the target set is defined. Since we know that $\kappa_0$ can converge to the path when inside the neighborhood, the trajectory generation does not have to end exactly on the path. This allows more room for other parameters to be factored in, such as the steepness of turning or the angle at which the robot arrives. Hence, the final position of the generated trajectory is a perfectly suitable initial position for $\kappa_0$.

% \begin{figure}[th]
%     \centering
%     \includegraphics[width = \columnwidth]{Figures/aux_traj.eps}
%     \caption{motion plan generated by motion planning module.}
%     \label{fig:trajectory}
% \end{figure}

% Also, note that our target set is a subset of $\mathbb{R}^3$. This is because the target set considers the $x$ and $y$ position and the orientation $\theta$. It could also consider $\delta$, but this is unnecessary for this paper’s purpose due to the nature of $\kappa_0$.


\subsection{Global Tracking Control and A Pure Pursuit Control Implementation} \label{sec:purepursuit}
A global tracking controller is employed as $\kappa_1$ to track the motion plan. To ensure that the global tracking controller effectively steers the car-like robot towards the motion plan and ultimately reaches the path's neighborhood, we impose the following assumption on $\kappa_{1}$.
\begin{assumption}\label{assumption:globalconvergence}
    Given a motion plan $x':[0, \infty)\to \mathbb{R}^{4}$, \nw{$x'$ is stable for the car-like robot controlled by $\kappa_1$,} namely, for all $\epsilon > 0$, there exists $\delta > 0$ such that $|\phi(t) - x'(t)| \leq \epsilon$
    for all $t \geq \nw{\delta}$, where $\phi:[0, \infty)\to \mathbb{R}^{4}$ is the maximal solution to (\ref{eq:car_robot}) with $(v,\omega) = \kappa_1(x, u)$.
\end{assumption}
\begin{remark}
    Assumption \ref{assumption:globalconvergence} ensures the car-like robot reaches the neighborhood of the desired path within a finite time. We choose $\epsilon = n_{c}$ in (\ref{eq:nbh_lift_set}). Since $x'$ is a solution to Problem \ref{problem:mp} and $x'(0) = \phi(0)$ (see item 1 in Problem \ref{problem:mp}), we have $|\phi(0) - x'(0)| = 0 \geq \delta$ for any existing $\delta > 0$ in Assumption~\ref{assumption:globalconvergence}. This implies $|\phi(t) - x'(t)| \geq \epsilon = n_{c}$ holds for all $t \geq 0$. By item 2 in Problem \ref{problem:mp}, there exists $T > 0$ such that $ x'(T)\in X_{f} = \{(x_{1}, x_{2}, x_{3}, x_{4})\in \mathbb{R}^{4}: \exists (x_{5}, x_{6})\in \mathbb{R}^{2} \text{ such that } (x_{1}, x_{2}, x_{3}, x_{4}, x_{5}, x_{6})\in \Gamma\}$. Hence, at time $T$, $|\phi(T) - x'(T)| < n_{c}$, implying the robot enters the neighborhood, namely $\phi(T)\in\mathcal{N}_\Gamma^{\by}$.
\end{remark}

Stability is a fundamental requirement in control design, and numerous tracking control techniques, such as pure pursuit control~\cite{Tomlin-PurePursuie-2011} and model predictive control~\cite{nascimento2018nonholonomic}, fulfill Assumption \ref{assumption:globalconvergence}. In this study, we employ the classic pure pursuit control as the global tracking controller for illustrative purposes. \myifconf{}{The pure pursuit algorithm calculates a steering angle that leads the robot on an arc path through a look-ahead point~\cite{Tomlin-PurePursuie-2011}.
%point some distance away on the path. 
This distance to the look-ahead point is called the look-ahead distance and can be tuned with a gain proportional to the robot’s speed. 
%
% Figure \ref{fig:purepursuit} gives a visual rendering of how the steering angle relates to the orientation of the robot and the angle to the path. The target point $(x_t, y_t)$ is found at a look-ahead distance $l_d$ away. The angle $\alpha$ is the difference between the robot's orientation and the angle to the target point. 
The look-ahead point $(x_t, y_t)\in \Real^{2}$ is found at a look-ahead distance $l_d\in\Real_{>0}$ away. The angle $\alpha_{p}$
% {\myred (AA: We have used $\alpha$ before in designing $\kappa_0$)} 
is the difference between the robot's orientation and the angle to the look-ahead point computed as
% \begin{equation} \label{eq:9}
$
    \alpha_{p} = x_{3} - \tan^{-1}{\left({(y_t - x_{2})}/{(x_t - x_{1})}\right)}.
$
% \end{equation}
The steering angle that leads the robot toward the look-ahead point is computed from $\alpha$ as
% \begin{equation} \label{eq:10}
$
    \delta = -\tan^{-1}\left( {(2l\sin{\alpha_{p})}}/{(l_d)}\right),
$
% \end{equation}
where $l$ is the length of the robot. The selection of the look-ahead point and the computation of the steering angle $\delta$ are executed in a receding manner to track the motion plan. \begin{remark}
    Only the position states, namely, $x_{1}$ and $x_{2}$, of the motion plan to Problem \ref{problem:mp} are used in the pure pursuit tracking algorithm.
\end{remark}}
%
% \begin{figure}
%     \centering
%     \includegraphics[width = 0.8\columnwidth]{Figures/simPurePursuit.eps}
%     \caption{Pure pursuit controller to track the generated motion plan.}
%     \label{fig:sin pure}
% \end{figure}
% The value of $\delta$ from Equation \eqref{eq:10} can be applied with sample and hold to drive our robot toward a point on the path. 
%
% The algorithm chooses a new point on the path towards which to navigate at each time step. More optimal look-ahead gain tuning results in lower tracking error and more desirable robot motion\cite{novelpurepursuit}.  
%
%
\myifconf{Figure~\ref{fig:hybrid} shows}{Figures~\ref{fig:hybrid} and \ref{fig:bestGP} show} that the pure pursuit algorithm is able to navigate the robot \nw{into} the neighborhood of the desired path \nw{by tracking the motion plan} while avoiding obstacles. 
%The pure pursuit algorithm is terminated when the robot enters the neighborhood of the desired path, as can be seen by the red line in the same figures. It can also be observed from these figures that the orientation at which the robot would arrive at the path is desirable. This results from the trajectory generator, since the target set includes a range of desirable robot orientations for each point. 
%
From \cite{ollero1995stability}, the pure pursuit controller is proved to satisfy Assumption \ref{assumption:globalconvergence}, thereby establishing the finite-time stability of $\kappa_1$ for $\mathcal{N}_\Gamma^{\by}$. 
% \begin{lemma}\label{lem:kappa1}
%     Given a motion plan $x':[0, \infty)\to \mathbb{R}^{4}$, then there exists a look-ahead distance $l_{d}$ such that the car-like robot controlled by pure pursuit algorithm $\kappa_1$ is stable, namely, for all $\epsilon > 0$, there exists $\delta > 0$ such that $|\phi(0) - x'(0)| \leq \delta$ implies $|\phi(t) - x'(t)| \leq \epsilon$
%     for all $t \geq 0$, where $\phi:[0, \infty)\to \mathbb{R}^{4}$ is the state trajectory of the robot under the control of $\kappa_1$.
% \end{lemma}
% \begin{remark}
%     Lemma \ref{lem:kappa1} ensures the car-like robot reaches the neighborhood of the desired path within a finite time. We choose $\epsilon = n_{c}$ in (\ref{eq:nbh_lift_set}). Since $x'$ is a solution to Problem \ref{problem:mp} and $x'(0) = \phi(0)$ (see item 1 in Problem \ref{problem:mp}), we have $|\phi(0) - x'(0)| = 0 < \delta$ for any existing $\delta > 0$ in Lemma \ref{lem:kappa1}. This implies $|\phi(t) - x'(t)| < \epsilon = n_{c}$ holds for all $t \geq 0$. By item 2 in Problem \ref{problem:mp}, there exists $T > 0$ such that $ x'(T)\in X_{f} = \{(x_{1}, x_{2}, x_{3}, x_{4})\in \mathbb{R}^{4}: \exists (x_{5}, x_{6})\in \mathbb{R}^{2} \text{ such that } (x_{1}, x_{2}, x_{3}, x_{4}, x_{5}, x_{6})\in \Gamma\}$. Hence, at time $T$, $|\phi(T) - x'(T)| < n_{c}$, implying the robot enters the neighborhood, namely $\phi(T)\in\mathcal{N}_\Gamma$.
% \end{remark}

\subsection{Hybrid Control Framework and Closed-loop System}
\begin{figure}[htbp]
    \centering
    % \includegraphics[width = \columnwidth]{Figures/NH.eps}
    \incfig[0.5]{neiborhood2}
    \caption{\myifconf{The desired path $\Gamma$ is shown as a red solid line. The flow sets $C_{0}$ and $C_{1}$ are depicted in green and yellow, respectively, with their overlap also shown in green. Green dotted lines mark the boundaries of $C_{1}$, blue dotted lines indicate the boundaries of $C_{0}$, and red dotted lines represent the boundaries of $\mathcal{N}_{\Gamma}^{\by}$.}{The desired path $\Gamma$ is represented by the red solid line. The flow sets $C_{0}$ and $C_{1}$ are represented by the green region and yellow region, respectively, and the overlapped region between $C_{0}$ and $C_{1}$ are presented by the green region. The green dotted lines denote the boundaries of $C_{1}$ and the blue dotted lines denote the boundaries of $C_{0}$. The red dotted lines represent the boundaries of $\mathcal{N}_{\Gamma}^{\by}$.}}
    \label{fig:rough_fig}
    \vspace{-0.6cm}
\end{figure}
% The controller $\kappa_0$ renders the path invariant if the robot is initialized in the neighborhood of the path $\mc{N}_\Gamma$. 
A discontinuous, non-hybrid switching scheme could suffice for achieving global path invariance. However, this solution is sensitive to arbitrarily small noise and, therefore, is nonrobust. To overcome this issue, we design a hysteresis-based hybrid controller that is triggered by the distance to the path. 
% If the robot is initialized outside $\mc{N}_\Gamma$, the controller $\kappa_1$ forces the system to reach the neighborhood $\mc{N}_\Gamma$ in finite time.} 
For $0<c_1<c_{1,0}<c_0 < 1$, we can define the set $\mc{U}_{0}$ as follows:
\myifconf{$
\mc{U}_{0} \eqdef \set{\agx\in\Real^6 : \norm{\agx}_\Gamma < c_0 n_c  }, \;\; {\mc{U}_{0}} \subset \ak{\mc{N}_\Gamma^{\by}}.
$}{\[
\mc{U}_{0} \eqdef \set{\agx\in\Real^6 : \norm{\agx}_\Gamma < c_0 n_c  }, \;\; {\mc{U}_{0}} \subset \ak{\mc{N}_\Gamma^{\by}}.
\]}
Next, we define $\mc{T}_{1,0}$ such that $\mc{T}_{1,ff0}$ is contained in the interior of $\mc{U}_0$ as follows
\begin{equation}\label{eq:T10}
  \mc{T}_{1,0} \eqdef \set{\agx\in\Real^6 : \norm{\agx}_\Gamma {\leq} c_{1,0}n_c } \subset \mc{U}_{0}.  
\end{equation}
It is guaranteed by ~\cite[Proposition III.3]{AkhNieWas2015} that once the solution enters $\mc{T}_{1,0}$, it never reaches the boundary of $\overline{\mc{U}_0}$.
Let $C_{0} \eqdef \overline{\mc{U}_0}$ and $C_{1} \eqdef \overline{\Real^6\setminus\mc{T}_{1,0}}$, which lead to the hysteresis region $C_{0}\setminus \mc{T}_{1,0}$. The hybrid controller $\mc{H}_K = (C_{K}, F_{K}, D_{K}, G_{K})$ \mynne{takes the state $\agx \in \Real^6$ of (\ref{eq:dynamic_car_robot}) as its input and $q \in Q \eqdef \set{0,1}$ as its state, and can be modeled as in~(\ref{model:generalhybridsystem})}
%with state $q \in Q \eqdef \set{0,1}$, input $\agx \in \Real^6$ 
as follows:
\begin{subequations}\label{eq:Hyb-control}
\begin{align}
\label{eq:Hyb-control-1}
C_{K} &:= \bigcup_{q\in Q}\left( \set{q}\times  C_{K,q}  \right),\quad
%\end{equation}
%\begin{equation}
\begin{cases}
C_{K,0} \eqdef C_0\\ 
C_{K,1} \eqdef C_1\\
\end{cases}\\
% \end{equation}
% %
% \begin{equation}
\label{eq:Hyb-control-4}
    F_{K}(q, \agx) &:= 0\quad \forall (q,\agx) \in C_{K}\\
% \end{equation} 
%
% \begin{equation}
\label{eq:Hyb-control-2}
D_{K} &:= \bigcup_{q\in Q}\left( \set{q} \times D_{K,q}
\right),\quad
%\end{equation}
%\begin{equation}
\begin{cases}
D_{K,0} \eqdef \overline{\Real^6\setminus\mc{U}_{0}}\\ 
D_{K,1} \eqdef {\mc{T}_{1,0}}\\
\end{cases}\\
% \end{equation}
% %
% \begin{equation}
\label{eq:Hyb-control-5}
    G_{K}(q,\agx) &:= 1 -q \quad \forall (q,\agx) \in D_{K}
\end{align}
\end{subequations}
and the output function $\kappa: Q\times \mathbb{R}^{6} \to \reals^{2}$ is such that
\begin{equation}
\label{eq:Hyb-control-3}
\kappa(q,\agx) = q\kappa_{1}(\agx) + (1-q)\kappa_{0}(\agx),
\end{equation}
where the controller $\kappa_0$ is the locally path invariant controller defined in~\eqref{eq:kappa_0} and $\kappa_1$ is the pure pursuit controller. Hysteresis is created by sets $\mc{U}_{0}$ and $\mc{T}_{1,0}$.
% with the boundary of $\mc{U}_{0}$ and $\mc{T}_{1,0}$ being the outer and inner portion of the hysteresis region, respectively. 
Controlling the continuous-time plant~\eqref{eq:dynamic_car_robot} by the hybrid controller results in a hybrid closed-loop system with states $z = (\agx,q)$ and dynamics 
% resulting from controlling $\mc{H}_P$ with the hybrid controller $\mc{H}_K = (C_K,F_K,D_K,G_K,\kappa)$ changes according to 
$
    \dot \agx = F_P(z,\kappa(z,q)), \quad \dot q = 0
$
during flows, and at jumps, the state is updated according to 
$
    \agx^{+} = \agx,\quad q^{+} = 1-q.
$
Finally, the hybrid closed-loop system $\mc{H} = (C,F,D,G)$ with the state $z = (\agx,q) \in \Real^6 \times Q =: Z$ has data given as 
\begin{equation}
\label{eq:data-CLS-circle}
\begin{aligned}
    C &\eqdef \{(\agx,q) \in Z : (q,\agx) \in C_{K} \}\\
    F(z) &\eqdef \left[\begin{array}{c}
        F_P(\agx,\kappa(q, \agx))   \\
         0
    \end{array}\right]\;\; \forall z \in C\\
    D &\eqdef \{(\agx,q) \in Z : (q,\agx) \in D_{K} \}\\
    G(z) &\eqdef \left[\begin{array}{c}
         \agx   \\
         1-q
    \end{array}\right]\;\; \forall x \in D.
\end{aligned}
\end{equation}
% {\myblue where $C_P \eqdef \Real^6$.}
Next, we state the main result of our paper.
\begin{theorem}
\label{theo:geometric-hybrid-cricle}
Given a set $\Gamma$ and the continuous-time plant in ~\eqref{eq:car_robot}, suppose Assumptions~\ref{ass:implicit}, ~\ref{ass:SteeringAngle}, and~\ref{assumption:globalconvergence} hold. Let the hybrid controller $\mc{H}_K$ with data $(C_K,F_K,D_K,G_K,\kappa)$ defined in~\eqref{eq:Hyb-control} and~\eqref{eq:Hyb-control-3}. Then, the following hold:
%, and the closed-loop system $\mc{H} = (C,F,D,G)$ defined in~\eqref{eq:data-CLS-circle}.

\begin{enumerate}
    \item [{1)}] The closed-loop system $\mc{H} = (C,F,D,G)$ with data in~\eqref{eq:data-CLS-circle} satisfies the hybrid basic conditions\cite[Definition 2.18]{San2021};
    \item [{2)}]Every maximal solution to $\mc{H}$ from $C \cup D$ is complete and exhibits no more than two jumps; 
    \item [{3)}]The set 
$
    \mc{A} = \Gamma^\star \times \set{0}
$
    is global and robust finite-time stable for $\mc{H}$ in the sense of~\cite[Definition 3.16]{San2021} and is forward invariant.
    % {\blue Should I call the bullet points a1, a2, ..., or any better suggestion?}
\end{enumerate}

\end{theorem}
% \begin{proof}
%     The proof follows along the lines of~\cite[Theorem 4.6]{San2021}.
%     % , and is removed because of space limitations. 
% \end{proof}
\myifconf{\begin{proof}
    For a detailed proof, see \cite{wang2025hybrid}. A sketch of the proof is provided as follows: By (\ref{eq:T10}), the set $\mc{T}_{1,0}$ is closed, implying that $C_{K,0}$,$C_{K,1}$,$D_{K,0}$ and $D_{K,1}$ are also closed. Moreover, since $C_K$ and $D_K$ are finite union of $C_{K,0}$,$C_{K,1}$,$D_{K,0}$ and $D_{K,1}$, they are also closed. By (\ref{eq:Hyb-control-4}) and (\ref{eq:Hyb-control-5}), the maps $F_K$ and $G_K$ are continuous. Additionally, the pure-pursuit controller $\kappa_1$ and $\kappa_0$ in (\ref{eq:kappa_0}) are continuous, ensuring that the resulting closed-loop system $\mc{H}$ satisfies the hybrid basic conditions, which proves item~1.

    To prove the completeness of the maximal solutions to $\mc{H}$, we proceed by contradiction. Suppose there exists a maximal solution with the initial state $z(0,0)\in C\cup D$ that is not complete. From~\cite[Proposition 2.34]{San2021}, either item b or item c must hold. However, by Lemma~\ref{lemma:invariance}, the controller $\kappa_0$ assures finite-time stability of the desired path $\Gamma^\star$ everywhere in a neighborhood of $\Gamma^\star$, ruling out item b. Moreover, it can be shown that $G(D) \subset C \cup D$, hence, ruling out item c. Since the maximal solution is assumed to be unique, therefore, the solution $z$ is complete, establishing the contradiction.
    To prove that every solution exhibits no more than two jumps, we analyze the behaviors of the solutions with all the three possible initial conditions: i) $z(0,0) \in C_{K,1} \times \set{1}$; ii) $z(0,0)\in D_{K,1},  \times \set{1}$; iii) $z(0,0) \in C_{K,0} \times \set{0}$. In all three cases, Assumption \ref{assumption:globalconvergence} and Lemma \ref{lemma:invariance} ensure that every maximal solution has at most two jumps, which proves item~2.

    The attractivity of $\mc{T}_{1,0}$ in finite time is established by Assumption~\ref{assumption:globalconvergence}, while Lemma~\ref{lemma:invariance} implies that the set $\mc{A}$ is finite-time stable for $\mc{H}$, thereby establishing global finite-time stability. Finally, since the hybrid system satisfies the hybrid basic condition and $\mc{A}$ is compact, it follows from~\cite[Theorem 3.26]{San2021} that $\mc{A}$ is robust in the sense of~\cite[Definition 3.16]{San2021}, which proves item 3 and completes the proof.
\end{proof}}{
\begin{proof}
The right-hand side of~\eqref{eq:dynamic_car_robot} is a continuous function of $(\overline{x},u)$. By (\ref{eq:T10}), the set $\mc{T}_{1,0}$ is closed. Moreover, the sets $C_{K,0}$,$C_{K,1}$,$D_{K,0}$, and $D_{K,1}$ are also closed. This implies that $C_K$ and $D_K$ are also closed, as these sets are finite union of closed sets. The maps $F_K$ and $G_K$ are continuous by construction. Moreover, both the pure-pursuit controller $\kappa_1$ and the locally path-invariant controller $\kappa_0$ in (\ref{eq:kappa_0}) are continuous. Hence, the resulting closed-loop system $\mc{H}$ satisfies the hybrid basic conditions, which proves item~1.  

We prove the completeness of the maximal solutions to $\mc{H}$ by contradiction. Suppose there exists a maximal solution with the initial state $z(0,0)$ in the set $C\cup D$ that is not complete. Let $(T,J) = \sup \dom z$ and since by assumption $z$ is not complete $T + J < \infty$. From~\cite[Proposition 2.34]{San2021}, either item b or item c has to hold. {By Lemma~\ref{lemma:invariance}, the controller $\kappa_0$ assures finite-time stability of the desired path $\Gamma^\star$ everywhere in a neighborhood of $\Gamma^\star$. Hence, maximal solutions to the closed-loop system under the effect of $\kappa_0$ are bounded and complete. If the solution start from $C_0 \setminus \mc{T}_{1,0}$, it will eventually reach $\mc{T}_{1,0}$ under the control of $\kappa_1$. Hence the maximal solutions of the closed-loop system remain bounded and complete.} Under the control of $\kappa_1$, solutions reach a neighbourhood of $\Gamma^\star$, which is bounded. Hence, the solutions under $\kappa_1$ are bounded. Therefore, item b in~\cite[Proposition 2.34]{San2021} is ruled out. It can be shown that $G(D) \subset C \cup D$, hence by item 3 in~\cite[Proposition 2.34]{San2021}, item c is also ruled out. Therefore, every maximal solution to $\mc{H}$ from $C\cup D$ is complete. 

To show that every solution exhibits no more than two jumps, it should be noted that for every solution $z$ to $\mc{H}$, $z(0,0) \in C \cup D$, and only the following three cases are possible:

\begin{enumerate}
    \item~\label{list:1} By Assumption \ref{assumption:globalconvergence}, if $z(0,0) \in C_{K,1} \times \set{1}$, the solution $z$ reaches the set $D_{K,1}$ in finite hybrid time as the plant states reaches $\mc{T}_{1,0}$. After a jump, the solution $z$ remains flowing in $\left( C_{K,0}\setminus D_{K,0} \times \set{0}  \right)$ for all future hybrid time.
%     %{\blue I don't think we need to invoke the discussion of the set $\mc{E}_0$?}
%     %
    \item The solution exhibits the same behavior as in item~\ref{list:1}, when $z(0,0)\in D_{K,1} \times \set{1}$.
    %
    \item If $z(0,0) \in C_{K,0} \times \set{0}$, then the following two cases are possible. If $z(0,0) \in \mc{T}_{1,0} \times \set{0}$, then by Lemma~\ref{lemma:invariance}, the solution remains flowing in $\left(  C_{K,0} \cap {C}_0 \times \set{0}\right)$ for all future hybrid time. If $z(0,0) \in \left( C_{K,0} \setminus \mc{T}_{1,0} \times \set{0}  \right)$, then the solution may either flow forever or jump from $0$ to $1$ when it reaches the boundary of ${C}_0$. From there, the solution flows according to the logic explained in item~\ref{list:1}.
\end{enumerate}
Hence, every maximal solution has at most two jumps, which proves item~2.

The attractivity of $\mc{T}_{1,0}$ in finite time is established by Assumption~\ref{assumption:globalconvergence}, while Lemma~\ref{lemma:invariance} implies that the set $\mc{A}$ is finite-time stable for $\mc{H}$, thereby establishing global convergence. Finally, since the hybrid system satisfies the hybrid basic condition and $\mc{A}$ is compact, it follows from~\cite[Theorem 3.26]{San2021} that $\mc{A}$ is robust in the sense of~\cite[Definition 3.16]{San2021}, which proves item 3 and completes the proof.
\end{proof}}

\subsection{Algorithm Formulation}
The hybrid controller switches between two controllers, defined by $\kappa_q$, with the hybrid model governing the state of $q$. This switching leverages each controller’s strengths based on the robot’s state. $\kappa_1$ guides the robot to the desired path’s neighborhood, while $\kappa_0$ ensures path tracking and invariance within it. The path-following scheme, guaranteeing invariance and global convergence, is detailed in Algorithm \ref{alg:globallyinvariant}.
% The hybrid controller switches between two controllers defined by the function $\kappa_q$, where our hybrid model governs the state of $q$. 
% Switching between the two controllers allows us to take advantage of the different assets depending on the robot’s state. 
% The controller $\kappa_1$ will navigate the robot to the neighborhood of the desired path if the robot starts or slides outside it. The controller $\kappa_0$ will be used inside the neighborhood to track the path and make it invariant for the closed-loop system. 
% The path-following scheme that renders invariance and guarantees convergence from everywhere in the output space is formulated in Algorithm \ref{alg:globallyinvariant}. 
{\footnotesize \begin{algorithm}[htbp]
    \caption{\footnotesize Hybrid globally path-invariant algorithm}\label{alg:globallyinvariant}
    \hspace*{\algorithmicindent} \textbf{Input:} The initial state $\agx_{0}$ of the robot.
    \footnotesize
\begin{algorithmic}[1]
\State $q\leftarrow 0$.
\While{true}
\If {$\agx_{0}\in \mathcal{T}_{1, 0}$ or ($\agx_{0}\in \overline{\mc{U}}_{0}\backslash \mathcal{T}_{1, 0}$ and $q = 0$)}
\State $q \leftarrow 0$.
\While{$\agx(t)\in \overline{\mc{U}_0}$}
\State Apply $\kappa_{0}$ to track $\mathcal{C}$.
\EndWhile
\Else
\State $q\leftarrow 1$.
\State Compute an auxiliary collision-free trajectory $x'$ connecting $x_0$ and $X_{f}$ using motion planner.
\While{$\agx(t)\notin \mathcal{T}_{1, 0}$}
\State Apply $\kappa_1$ to track $x'$.
\EndWhile
\EndIf
\State $\agx_{0}\leftarrow \agx(t)$.
\EndWhile
\end{algorithmic}
\end{algorithm}}
% \subsection{Main Result}

%%%%%% Proof excluded from the camera ready version
{
% \begin{proof}
% Since the right-hand side of~\eqref{eq:generic_left_invariant_system}, namely, $\dot g = g\xi(u)$ is a continuous function of $(g,u)$. By construction, the set $\mc{T}_{1,0}$ is closed. Moreover, the sets $C_{K,0}$,$C_{K,1}$,$D_{K,0}$, and $D_{K,1}$ are also closed. This implies that $C_K$ and $D_K$ are also closed, as these sets are finite union of closed sets. The maps $F_K$ and $G_K$ are continuous by construction. The open-loop controller $\kappa_1$ is a constant function, and hence continuous. Moreover, by Definition~\ref{def:kinematic_family_circle} each $f\in \mc{F}_k$ is continuously differentiable, which implies $\kappa_0$ is also continuous. Hence, the resulting closed-loop system $\mc{H}$ satisfies the hybrid basic conditions, which proves item~1.  

% We prove completeness of the maximal solutions to $\mc{H}$ by contradiction. Suppose there exists a maximal solution with the initial state $x(0,0)$ in the set $C\cup D$ that is not complete. From~\cite[Proposition 2.34]{San2021}, either item b or item c has to hold. By Lemma~\ref{lemm:asymptotic_stability_class}, each controller $\kappa_0$ assures asymptotic stability of the point $e\in\ms{G}$ with the basin of attraction $\mc{B}_f$ containing $\mc{U}_0$. Hence the maximal solutions of the closed-loop system remain bounded and complete. {By Assumption~\ref{ass:open-loop}, under the effect of $\kappa_1$, solutions reach a neighbourhood of $e$, which is bounded. Hence, the solutions under $\kappa_1$ are bounded.} Therefore, item b in~\cite[Proposition 2.34]{San2021} is ruled out. It can be shown that $G(D) \subset C \cup D$, hence by item 3 in~\cite[Proposition 2.34]{San2021}, item c is also ruled out. Therefore, every maximal solution to $\mc{H}$ from $C\cup D$ is complete. 

% To show that every solution exhibits no more than two jumps, it should be noted that for every solution $x$ to $\mc{H}$, $x(0,0) \in C \cup D$, and only the following three cases are possible:

% \begin{enumerate}
%     \item~\label{list:1} By Lemma~\ref{lem:open-loop-finte-time}, if $x(0,0) \in C_{K,1} \times \set{1}$, the solution $x$ reaches the set $D_{K,1}$ in finite hybrid time as the plant states reaches $\mc{T}_{1,0}$. After a jump, the solution $x$ remains flowing in $\left( C_{K,0}\setminus D_{K,0} \times \set{0}  \right)$ for all future hybrid time.
%     %{\blue I don't think we need to invoke the discussion of the set $\mc{E}_0$?}
%     %
%     \item The solution exhibits the same behavior as in item~\ref{list:1}, when $x(0,0)\in D_{K,1} \times \set{1}$.
%     %
%     \item If $x(0,0) \in C_{K,0} \times \set{0}$, then the following two cases are possible. If $x(0,0) \in \mc{T}_{1,0} \times \set{0}$, then by Lemma~\ref{lemm:asymptotic_stability_class}, the solution remains flowing in $\left(  C_{K,0} \cup \mc{U}_0 \times \set{0}\right)$ for all future hybrid time. If $x(0,0) \in \left( C_{K,0} \setminus \mc{T}_{1,0} \times \set{0}  \right)$, then the solution may either flow forever or reach jump from $0$ to $1$ when it reaches the boundary of $\mc{U}_0$. From there, the solution flows according to the logic explained in item~\ref{list:1}.
% \end{enumerate}
% Hence, every maximal solution has at most two jumps, which proves item~2.

% The attractivity of $\mc{T}_{1,0}$ in finite time is established by Lemma~\ref{lem:open-loop-finte-time}, and Lemma~\ref{lemm:asymptotic_stability_class} implies that the set $\mc{A}$ is asymptotically stable for $\mc{H}$. Finally, since the hybrid system satisfies the hybrid basic condition and $\mc{A}$ is compact, it follows from~\cite[Theorem 3.26]{San2021} that $\mc{A}$ is robust in the sense of Definition~\ref{def:robust-stability}, which proves~3 and completes the proof.
% \end{proof}
% }
}

\begin{figure}[t]
    \begin{center}
        \includegraphics[width=\linewidth]{figures/sparse_v1.pdf}
    \end{center}
    \caption{
        \textbf{Qualitative results of generating reactive motions from sparse signals.} We compare our method with \blcamdm. Our approach successfully generates realistic motion while effectively adhering to the sparse signals (annotated by \red{red dots} in the figures). In contrast, \blcamdm{} struggles to achieve the same level of responsiveness and accuracy, as shown in the \red{red circles}.
    }
    \label{fig:sparse}
    \figtabskip
\end{figure}
\section{Discussion and Conclusion}
\label{sec:discussion}


\textbf{Conclusion.} In this paper, we propose LRM to utilize diffusion models for step-level reward modeling, based on the insights that diffusion models possess text-image alignment abilities and can perceive noisy latent images across different timesteps. To facilitate the training of LRM, the MPCF strategy is introduced to address the inconsistent preference issue in LRM's training data. We further propose LPO, a method that employs LRM for step-level preference optimization, operating entirely within the latent space. LPO not only significantly reduces training time but also delivers remarkable performance improvements across various evaluation dimensions, highlighting the effectiveness of employing the diffusion model itself to guide its preference optimization. We hope our findings can open new avenues for research in preference optimization for diffusion models and contribute to advancing the field of visual generation.

\textbf{Limitations and Future Work.} (1) The experiments in this work are conducted on UNet-based models and the DDPM scheduling method. Further research is needed to adapt these findings to larger DiT-based models \cite{sd3} and flow matching methods \cite{flow_match}. (2) The Pick-a-Pic dataset mainly contains images generated by SD1.5 and SDXL, which generally exhibit low image quality. Introducing higher-quality images is expected to enhance the generalization of the LRM. (3) As a step-level reward model, the LRM can be easily applied to reward fine-tuning methods \cite{alignprop, draft}, avoiding lengthy inference chain backpropagation and significantly accelerating the training speed. (4) The LRM can also extend the best-of-N approach to a step-level version, enabling exploration and selection at each step of image generation, thereby achieving inference-time optimization similar to GPT-o1 \cite{gpt_o1}.

\section*{Acknowledgements}
This work was partially supported by the NSFC (No.~62322207, No.~62172364, No.~62402427), Ant Group Research Fund and Information Technology Center and State Key Lab of CAD\&CG, Zhejiang University. We also acknowledge the EasyVolcap \citep{easyvolcap} codebase.

\clearpage
\begin{figure}[t]
    \begin{center}
        \includegraphics[width=\linewidth]{figures/baseline_reactive_v1.pdf}
    \end{center}
    \caption{\textbf{Qualitative results of generating reactive motions.}
    Given the same \textcolor{agentgreen}{ground truth opponent motion}, \blinterformer{} can produce reactive motion that is too close to the opponent, leading to penetration. \blcamdm{} tends to get stuck, while \blduolando{} may result in human motion with incorrect orientation after a certain period.
    }
    \label{fig:reactive}
\end{figure}

\begin{figure}[t]
    \begin{center}
        \includegraphics[width=\linewidth]{figures/baseline_twoagent_v1.pdf}
    \end{center}
    \caption{\textbf{Qualitative results of generating two-character motions.}
    Given the same initial four frames for both characters, \blinterformer{} tends to produce human motion with incorrect orientation. \blcamdm{} often results in the characters getting stuck, while \blgpt{} can cause the two characters to drift apart due to accumulated errors.
    }
    \label{fig:twoagent}
\end{figure}

\clearpage
\bibliography{iclr2025_conference}
\bibliographystyle{iclr2025_conference}


\end{document}
