\begin{table}[t]
\caption{
    \textbf{Quantitative results of generating reactive motion from sparse signals.}
    We compare our method with \blcamdm.
    Among them, \textbf{bold} indicates the best results.
    $\downarrow$ means lower is better. $\rightarrow$ means closer to the real data is better.
    Our method outperforms the baseline in terms of all metrics.
    }
\begin{center}
\resizebox{0.8\linewidth}{!}{
    \begin{tabular}{lrrrrrrr}
        \toprule
        \multirow{2}{*}{Methods} & \multicolumn{3}{c}{FID $\downarrow$} & \multirow{2}{*}{Jitter $\rightarrow$} & \multirow{2}{*}{FS $\rightarrow$} & \multirow{2}{*}{Pos. Err. $\downarrow$} & \multirow{2}{*}{Rot. Err. $\downarrow$} \\
        \cmidrule(r){2-4}
        & Per-frame & Per-transition & Per-clip \\
        \midrule
        Real &- &- &- &21.332 & 0.97 & - & -  \\
        \midrule
        \blcamdm & 0.697 & 1.506 & 15.169 & 47.229 & 2.25 & 14.52 & 22.40 \\
        % \midrule
        Ours & \bf{0.249} & \bf{0.263} & \bf{4.086} & \bf{21.163} & \bf{1.06} & \bf{2.72} & \bf{4.39} \\
        \bottomrule
    \end{tabular}
}
\end{center}
\label{tab:control}
\figtabskip
\end{table}