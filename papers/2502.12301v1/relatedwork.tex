\section{Related work}
There are several training datasets that comprise parallel data for Low-Resource Languages (LRLs), which we operationally define as any language beyond the first 100 or so supported by most traditional crawls and MT providers. 
The \nllbseed{} dataset is a sentence-level training set consisting of 6k sentences selected from English Wikipedia and professionally translated into 39 LRLs \cite{Nllb2022}.  
The \gatitos{} dataset \cite{Jones2023} represents another approach, consisting of a 4000-entry lexicon, translated into 170 LRLs.
There are also several professionally-translated evaluation sets, namely \floresone{} and \florestwo{} \cite{Goyal2022, Nllb2022}, and \ntrex{} \citep{ntrex}.
 
While highly multilingual, professionally translated training data is rare, 
there is a growing number of bottom-up community datasources organized through research collectives like Masakhane \citep{masakhane}, Turkish Interlingua \citep{mirzakhalov2021large, mirzakhalov2021evaluating}, and GhanaNLP~\citep{ghananlp};
and conferences and workshops like AfricaNLP,
AmericasNLP~\citep{mager-etal-2021-findings} and ArabNLP.
These datasets are usually generated by researchers fluent in the languages, and are therefore especially high quality.
In addition to providing datasets, such efforts frequently also provide models and baselines, or even public interfaces, like the Khaya Translator Web App\footnote{\url{https://ghananlp.org/project/translator-webapp/}} by GhanaNLP for West African languages, and the lesan.ai\footnote{\url{https://lesan.ai/translate}} translation website for Ethiopian languages.

Participation is especially strong from the African continent, including corpora and models for pan East-African languages \citep{babirye2022building}, languages from the Horn of Africa \citep{hornmt}, Ethiopian languages \citep{teferra-abate-etal-2018-parallel,gezmu2021extended}, Ugandan languages \citep{akera2022machine}, Emakhuwa \citep{felermino2021towards},  South-African languages \citep{eiselen-puttkammer-2014-developing}, Setswana and Sepedi \citep{marivate-etal-2020-investigating}, Yorùbá \citep{adelani-etal-2021-effect,adelani2021menyo},  Oshiwambo \citep{nekoto2022participatory}, Igbo \citep{ezeani2020igbo},
Zulu~\citep{rooweither_mabuya_2021_5035171},
Twi \citep{azunre2021english}, Gbe \citep{hacheme2021english2gbe}, Bambara \citep{tapo2021domain}, and Fon \citep{emezue-dossou-2020-ffr}. Outside of Africa, corpora have been created for languages of the Americas, including for four indigenous languages of Peru in \citet{bustamante-etal-2020-data}, the numerous results on the largely South- and Central American languages from the first AmericasNLP conference \citep{mager-etal-2021-findings}, and the Inuktitut language of Canada \citep{joanis-etal-2020-nunavut}. Datasets for lower-resourced languages of India have also sprung up, including the 13-language PMIndia \citep{haddow2020pmindia}, and datasets focused on languages of the Northeast like Mizo \citep{thihlum2020mizo}, Khasi \citep{laskar-etal-2021-enkhcorp1} and Assamese \citep{laskar-etal-2020-enascorp1}. Finally, a variety of such datasets and models are available for public use on HuggingFace\footnote{\url{https://huggingface.co/datasets?multilinguality=multilinguality:translation&task_categories=task_categories:translation&sort=downloads}} or Zenodo\footnote{\url{https://zenodo.org/communities/africanlp/}}.


In addition to professionally translated data, there are also several web-crawled datasets for LRLs, including
\textsc{MadLad} \citep{Kudugunta2023},
OSCAR \citep{oscar},
Glot500-C \citep{imanigooghari-etal-2023-glot500}, 
% NLLB \citep{Nllb2022}, 
and the Bloom library \citep{leong-etal-2022-bloom}.