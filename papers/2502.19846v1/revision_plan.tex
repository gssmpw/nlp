\twocolumn
\section*{Revision plan of Submission 1379 (Shepherd Submission 1621)}
\noindent \textbf{Dear reviewers,}

\noindent 
We would like to express our gratitude to the reviewers for their valuable feedback.
In response to the reviewers' comments, we will focus on addressing four main areas of improvement: (1) improving the experimental analysis, (2) enhancing the presentation, (3) extending the discussion on related works, and (4) clarifying our limitations.
To address these issues, we have provided a detailed explanation of our responses to the main concerns that were raised by all reviewers. We have also addressed each reviewer's comments individually.




\subsection*{Summary of the Revision Plan}
We provide a brief overview of the main changes we plan to implement in the revision. \blue{Reviewer\#1, \#2, and \#4 are referred to as R1, R2, and R3 respectively.}



% \vspace{1mm}
\noindent
% {\bf{(\colorbox{pink}{Point \textrm{1}}) Additional Experiments}}. 
\subsubsection*{\bf{(\colorbox{pink}{Point \textrm{1}})
% \label{rev:point-1}
Improving the experimental analysis}} Based on the reviewers' suggestions, we plan to enhance the experimental evaluation in the following ways:


$\bullet$ \textbf{Comparison with baselines}: We will expand the comparison with the baselines (FRL and IDS) to include quantitative metrics, including the solution size, coverage, fairness measures, and execution time \blue{(addressing R2.O1)}.

$\bullet$ \textbf{Protected groups}: We will provide a more detailed explanation of the protected groups analyzed and the rationale behind selecting them \blue{(addressing R2.O2)}.
    



\subsubsection*{\bf{(\colorbox{pink}{Point \textrm{2}})
\label{rev:point-1}
Enhancing the presentation}} To enhance the paper's presentation, we plan to make the following revisions:



$\bullet$ \textbf{Usability}: We will provide a clear explanation of how the proposed framework is intended for use by policymakers \blue{(addressing R1.O1)}. Specifically, the policymaker selects the target outcome and the relevant problem variant (e.g., imposing coverage and/or fairness constraints). The system then generates a prescription ruleset for the policymaker to implement. The framework assumes that the policymaker will publish the relevant recommendations to each subpopulation. However, if not all rules are provided to the users, disparities among subpopulations may increase.  
We will highlight this limitation of our framework and include a discussion on its intended usability in the revised paper.   

$\bullet$ \textbf{Usage of LLMs}: 
We utilized ChatGPT to translate prescription rules into natural language. This translation was done to make the rules easy to understand. However, as \blue{R1.O7} pointed out, the explanations mostly follow a template, and we do not need GPT for this task. We will replace current explanations with a template version to present them without using ChatGPT.

%We will clarify our use of LLMs and revise the prompts to ensure it does not introduce errors.  We utilized ChatGPT to translate prescription rules into natural language. Each rule included the following components: grouping pattern, intervention pattern, overall coverage, coverage for protected groups, and overall, protected, and non-protected expected utility. ChatGPT was prompted to rephrase these elements into a natural language sentence.  To minimize the risk of hallucination, we will refine our prompts by using precise templates. This approach ensures the content remains unchanged, with ChatGPT solely rephrasing the rule into a sentence without altering the numbers or meaning (addressing R1.O7).


$\bullet$ \textbf{Presentation of Result Summary}: We will adjust the colored boxes to make their content easier to read (increasing the fonts and changing the colors) and present the results summary in a more accessible format, avoiding the use of colored boxes for the results summary discussion \blue{(addressing R1.O8)}.

$\bullet$ \textbf{Patterns \& WHERE clauses}: We will clarify the relation between WHERE clause and a pattern \blue{(addressing R3.O1)}. Specifically, we will clarify that \emph{patterns} is a commonly used notation in query result explanation research~\cite{roy2014formal,wu2013scorpion,el2014interpretable,lin2021detecting}. Generally, patterns are equivalent to the WHERE clause in SQL queries. However, in this work, we focus on a specific type of pattern: a conjunction of predicates, as has been done in prior query result explanation research~\cite{roy2014formal,wu2013scorpion,el2014interpretable,lin2021detecting,DBLP:journals/pacmmod/YoungmannCGR24}.

$\bullet$ \textbf{Motivating example}: We will revise the introduction to make it more concrete by rephrasing it to include more realistic and practical examples \blue{(addressing R3.O2)}. Specifically, we will focus on the two use cases examined in our experimental evaluation: increasing the average salary of high-tech employees and improving the average credit risk score.

$\bullet$ \textbf{Typos}: We will thoroughly proofread the paper and the code repository to eliminate any typos or errors \blue{(addressing multiple comments by the reviewers  R1.11, R1.12, R2.12, R3.10)}.




\subsubsection*{\bf{(\colorbox{pink}{Point \textrm{3}})
% \label{rev:point-1}
Extending the discussion on related works}}

 To the best of our knowledge, no existing framework generates causal-based, fairness-aware actionable recommendations. However, we will broaden our discussion of related work from relevant conferences to include recent literature on alternative definitions of fairness (e.g., \cite{jain2024algorithmic, somerstep2024algorithmic}) and causal-based resource allocation approaches (e.g., \cite{majumdar2024carma, ehyaei2023robustness}) \blue{(addressing R1.O6, R1.O3)}. We will emphasize that this work focuses on four definitions of fairness -- statistical parity and bounded group loss, considering both individual and group fairness -- as an initial step. Exploring additional definitions of fairness is an interesting direction for future research.  We will fix reference [97] as mentioned in \blue{R2.12}.




\subsubsection*{\bf{(\colorbox{pink}{Point \textrm{4}})
Clarifying our
limitations. }} 
Based on the reviewers' comments, we plan to extend the discussion on the limitations of the proposed framework in the following ways:

$\bullet$ \textbf{Causal DAG quality}: We will clarify that the generated rules may be influenced by various factors, including the quality of the input causal DAG \blue{(addressing R1.O2)}. We will also add an experiment evaluating the robustness of our method when the causal graph is noisy. The example dataset (\cref{tab:data}) and the example causal DAG (\cref{fig:causal_DAG}) represent only a subset of the data, consisting of 20 variables.  
In our running example from Stack Overflow, the \verb|Exercise| variable (indicating how many times per week an individual exercises) has a directed causal path to the \verb|HoursComputer| variable (indicating how many hours per day an individual works on a computer), which in turn directly affects the \verb|Salary| variable. Thus, based on this DAG, exercising has an indirect causal effect on salary.  
The causal DAG was constructed using the causal discovery algorithm proposed in \cite{youngmann2023causal}. We acknowledge that inaccuracies in the DAG may result in flawed rules. Therefore, we assume the causal DAG is provided as part of the input and will emphasize that it is the policymaker's responsibility to validate the correctness of the causal DAG.
    


$\bullet$ \textbf{Theoretical guarantees}: We acknowledge that our algorithm does not provide formal guarantees for approximating the \probName\ problem \blue{(addressing R1.O4)}, and we will clarify this further in the revised paper. As expected, the problem is NP-hard in simple settings and even developing a good heuristics considering several parameters and constraints was non-trivial. However, we emphasize that each selected rule represents an intervention that is statistically significant. Specifically, based on causal analysis, the expected utility reflects the anticipated average increase in the outcome for the specific subpopulation to which the rule applies. Nevertheless, as we mention in Section \ref{sec:conc}, the quality of the generated rules may be influenced by several factors, including data quality, the correctness of the input causal DAG, and the method used to estimate causal effects. We will add approximations and further theoretical analysis as future work.

 
 $\bullet$ \textbf{Explainability}: Another limitation we will address in the revised paper pertains to the explainability and interpretability of the generated rules \blue{(addressing R3.O3)}. While if-then rules are generally considered interpretable by humans~\cite{lakkaraju2016interpretable}, prescription rules consist of conjunctions of predicates, which may not always be explainable. For instance, in a highly detailed dataset, a possible rule might suggest that individuals aged between 18.5 and 21.2 years should learn programming in Python. Understanding the rationale behind such a rule is not straightforward. A potential direction for future work involves binning continuous variables into semantically meaningful ranges while ensuring that sound causal inference can still be performed on the binned data. We will clarify this in the revised paper.



\vspace{4mm}
This concludes the summary of the main revisions we plan to make.
Below, you will find detailed responses to the reviewers' comments.\\


\subsection*{Meta-Review}
\vspace{1mm}
\noindent
\emph{\blue{The authors consider the task of mining from a table a set of rules that can be turned into recommendations that are fair in the sense that there is not too large of a difference in expected outcome between a protected and a non-protected group.\\
%
The reviewers request a deeper consideration of the fairness literature, to better position the results, including in the evaluation (R2.01) as well as addressing the points below.}}\\
\noindent
\emph{\blue{{\bf 5. Required Changes}\\
R1 01-08\\
R2.01/02\\
R2: 01-O2\\
}}
\textbf{Ans}. We thank the meta-reviewer and the reviewers for the revision opportunity. As discussed in the summary above and in the responses below, we will address all required changes and other comments in the revised paper.\\

\subsection*{Reviewer \#1}
\vspace{1mm}
\noindent
\colorbox{pink}{\textbf{R1.O1}}
\emph{\blue{The general framework of how this work could be used is unclear.
If at least some individuals do not hear about the recommendation, they would clearly be at a disadvantage. Conversely, if the rules are known "internally", e.g., the bank has its internal rules on who gets credit, people who know the rules play the game better than the others. This is already the case today. Clearly, the authors take a step forward by considering protected subgroups, I understand that they seek to make things more fair. But how their proposed rules can be leveraged to that purpose is unclear.
Moreover, in general, most resources are finite, e.g., total salary to pay to the developers, or credit to be given out by a bank.
If all the groups followed the recommendations, would everyone's salary really increase? Would all customers be awarded credit? Or are rules/criteria in real life used to divide a finite resource, with a more or less arbitrary threshold drawn somewhere?}} 
\textbf{Ans}. We thank the reviewer for highlighting this important issue. As noted in Point 2 above, we plan to include a discussion on the intended usability of the system and the potential risks of improper use. Additionally, we will provide a detailed discussion of additional constraints related to available resources and outline these as potential extensions for future work.

The reviewer raises an interesting point about ensuring that rules cover all individuals. To address this, we include a coverage constraint that allows for the enforcement of such a requirement. However, not all applications necessitate universal coverage. For instance, a political party may implement targeted policies to maximize votes, or a company might offer incentives to individuals more likely to purchase a product. Our framework provides the flexibility to specify these coverage requirements while maintaining fairness with respect to sensitive attributes.

\vspace{1mm}
\noindent
\colorbox{pink}{\textbf{R1.O2}} \emph{\blue{The authors' notion of causality is hard to understand in some places (examples).
The authors state in many places that their approach is causal, not correlation-based. Yet it takes a leap of faith to assume that exercising X days a week causally improves one's salary, especially for young people (considering that the bad effects of lack of exercise start showing in one's 40s or 50s).
Also, exercise does not appear at all in Table 1; nor does it appear in the partial DAG (Figure 1), nor is there some substitute attribute or anything related to the exercise aspect.
In my view, these "exercise impacts salary" examples are not helping the paper to be convincing.}} 
\textbf{Ans}. Please refer to our response in point 4 above, discussing the causal DAG quality.  

\vspace{1mm}
\noindent
\colorbox{pink}{\textbf{R1.O3}} \emph{\blue{Other aspects of fairness deserve more discussion.
The example given early on states that a certain rule whcih would improve salaries by 13000\$ for protected people and 44000\$ for everyone. Whether or not this is acceptable/desirable depends on the local purchase power and who the group consists of (does a majority of users live in India? or in the US? the calculations would not be the same). Maybe this is part of the authors' future work (mentioned at the end) of the paper.}} 
\textbf{Ans}.
The reviewer has raised an interesting point about varied purchasing power and cost of living in different countries. A potential way to fix this would be to modify the objective function. However, principled techniques to address these issues is part of future work.
We will discuss this limitation in more detail in the future work section.
%In the revised paper, we will provide additional details about the distributions in the examined datasets to give more context for the generated rules. As noted in Point 3 above, we also plan to expand our discussion of other definitions of fairness in the related work section.

\vspace{1mm}
\noindent
\colorbox{pink}{\textbf{R1.O4}} \emph{\blue{As the authors acknowledge, there are no formal guarantee of the quality of their obtained rules.}} 
\textbf{Ans}. Please refer to our answer in Point 4 above. 

\vspace{1mm}
\noindent
\colorbox{pink}{\textbf{R1.O5}} \emph{\blue{How many among the "Result Summary" statements required some experiments to establish, vs. some that can be analytically shown or exhibited with small examples?}} 
\textbf{Ans}. We agree with the reviewer on this point. Some statements in the results summary, such as the observation that fairness often comes at the cost of overall expected utility, are straightforward and intuitive. We will revise the discussion in the results summary to emphasize the more interesting and novel findings uncovered through our analysis.  


\vspace{1mm}
\noindent
\colorbox{pink}{\textbf{R1.O6}} \emph{\blue{The paper is welcome at SIGMOD where fairness is one of the topics. However, it would be good if the authors did a more comprehensive comparison with the stazte of the art in FAccT and also RecSys, to ensure: has their problem never been looked at? It would be good to make sure.}} 
\textbf{Ans}. Please refer to our response in Point 3 above. 


\vspace{1mm}
\noindent
\colorbox{pink}{\textbf{R1.O7}} \emph{\blue{Why use ChatGPT to generate very simple, template-like text (out of patterns), especially considering that it may hallucinate and introduce errors?}} 
\textbf{Ans}. Please see our response in Point 2 above. 


\vspace{1mm}
\noindent
\colorbox{pink}{\textbf{R1.O8}} \emph{\blue{It is OK to show sample rules as "figures" with smaller font, but it's a pity to do so for important content such as the Result Summary.}} 
\textbf{Ans}. Thank you for this comment. We will revise the presentation to ensure the main findings of the results are clearer and more accessible. Please refer to our response for R1.O5. 


\vspace{1mm}
\noindent
\colorbox{pink}{\textbf{R1.11 and R1.12}} \emph{\blue{Availability and Minor Remarks.}} \textbf{Ans}. Thank you for highlighting these issues. We will address them, thoroughly proofread the paper, and review the repository.


\subsection*{Reviewer \#2}
\vspace{1mm}
\noindent
\colorbox{pink}{\textbf{R2.O1}} \emph{ \blue{The experimental comparison with state-of-the-art solutions is currently limited and should be improved.
The comparison with the baselines is performed only at the qualitative level (Section 7.2). This analysis should also be extended at the quantitative level, comparing the solutions in terms of size, coverage, fairness metrics, and execution time. A more complete evaluation would prove the effectiveness of the results.}} 
\textbf{Ans}. Please see our response to Point 1 above. 

\vspace{1mm}
\noindent
\colorbox{pink}{\textbf{R2.O2}} \emph{ \blue{The paper states that, in the experiments, ‘the protected groups were selected to represent subgroups where the desired outcome was relatively low and sufficiently large (approximately 30\% of the population) to ensure the discovery of statistically significant rules.’ I find this indication unclear. Which are the protected groups? I invite the authors to clearly specify it.}} 
\textbf{Ans}. Please see our response in Point 1 above. 



\vspace{1mm}
\noindent
\colorbox{pink}{\textbf{R2.12}} \emph{\blue{-How were the three rules shown in the example selected? What are the criteria?
Moreover, it would be interesting to discuss the overall rules.
- Table 5 compares the solutions also in terms of size, coverage, and expected utility as Table 4. I suggest extending the caption.
-Typos
Survey [97] has a peer-review published version}} \textbf{Ans}. 
We chose the rules by picking one randomly from each category (one which favors the protected group, one which favors the non-protected and another one that is more balanced). We will discuss all rules for the example as well.
Thank you for pointing out the typos, the caption of Table 5, and reference [97]. We will fix these and proofread the paper.

\subsection*{Reviewer \#3}
\vspace{1mm}
\noindent
\colorbox{pink}{\textbf{R3.O1}} \emph{ \blue{Why call this a pattern? This is a WHERE clause, and the following discussion of coverage is selectivity. Why create new jargon?}} 
\textbf{Ans}. Please see our response in Point 2 above. 

\vspace{1mm}
\noindent
\colorbox{pink}{\textbf{R3.O2}} \emph{  \blue{The verbiage in the abstract and introduction - presecriptions are for change - both personal health and organization... very motherhood and apple pie, but difficult to map directly to this work. Could we bring it down from such lofty goals? Given the high-level that the introduction is written at, it is unclear what a realistic prescription could be. A clear, complete and well motivated running example, tied into the introduction that can then be used to clearly explain components of the paper - such as the granularity of a prescription, or algorithmic results later - would do well.}} 
\textbf{Ans}. We thank the reviewer for this comment. As mentioned in Point 2 above, we will revise the introduction to focus
on the two use cases examined in our experimental evaluation:
increasing the average salary of high-tech employees and improving
the average credit risk score. 


\vspace{1mm}
\noindent
\colorbox{pink}{\textbf{R3.O3}} \emph{ \blue{There is a statement about understanding and explanability in the motivation. However, the explainability component is not clearly described in teh approach or evaluation. This either needs to be removed, or strengthened.}} 
\textbf{Ans}. We agree with the reviewer on this point. We will remove these statements from the paper and include a discussion on enhancing explainability in the future work section. Please refer to our response in Point 4 above.

\vspace{1mm}
\noindent
\colorbox{pink}{\textbf{R3.10}} \textbf{Ans}. \emph{\blue{Honestly, the formatting with the callout boxes bothers me a bit. I can't tell if I'm just being old, or if there is a readability problem here}}
Thank you for pointing this out. We will revise the formatting of the callout boxes. 

