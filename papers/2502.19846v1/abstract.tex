Prescriptions, or actionable recommendations, are commonly generated across various fields to influence key outcomes such as improving public health, enhancing economic policies, or increasing business efficiency. While traditional association-based methods may identify correlations, they often fail to reveal the underlying causal factors needed for informed decision-making. On the other hand, in decision making for tasks with significant societal or economic impact, it is crucial to provide recommendations that are interpretable and justifiable, and equitable in terms of the outcome for both the protected and non-protected groups. Motivated by these two goals,  this paper introduces a fairness-aware framework leveraging causal reasoning  for generating a set of interpretable and actionable prescription rules (ruleset) toward betterment of an outcome while preventing exacerbating inequalities for protected groups. By considering group and individual fairness metrics from the literature, we ensure that both protected and non-protected groups benefit from these recommendations, providing a balanced and equitable approach to decision-making. We employ efficient optimizations to explore the vast and complex search space considering both fairness and coverage of the prescription ruleset. Empirical evaluation and case study on real-world datasets 
demonstrates the utility of our framework for different use cases.