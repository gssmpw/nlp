\section{Missing Proofs}

\begin{proof}[Proof of Lemma \ref{lemma:individual_rules}]
The utility of a rule $r$ denotes the expected increase in outcome $O$ when all individuals within the subgroup $\pattern_g$ are treated with $\pattern_t$. 
\begin{eqnarray*}
    utility(r) &=& \frac{1}{|\pattern_g|} \sum_{i\in coverage(\pattern_g)} utility_i(\pattern_t)
\end{eqnarray*}
where $utility_i(\pattern_t)$ denotes the utility for tuple $i$ with respect to treatment $\pattern_t$. Since utility(r) is an average over multiple different utilities, the utility will be higher than the expected value for certain tuples in $coverage(\pattern_g)$.
Let $i^* = \arg \max utility_i(\pattern_t)$.

Consider a new prescription rule $r' (i^*,\pattern_t)$ which considers the same treatment $\pattern_t$ for the tuple $i^*$. Therefore,
    $utility(r') = utility_i(\pattern_t) > utility(r)$.
\end{proof}


\subsection{Hardness Results}
We next study the complexity of the \probName\ problem under different constraint combinations. 
%First, we demonstrate that the \probName\ problem is an NP-hard problem~\cite{khuller1999budgeted}. 
We show that \probName\ is equivalent to optimizing a non-negative and monotone submodular function. 
Furthermore, the individual fairness constraint and rule coverage constraints are matroid constraints. Therefore, a greedy algorithm is appropriate approach to solve the problem.  


\begin{proposition}
\label{prop:unconstrained}
    The optimization objective of \probName\ problem is a non-negative submodular function.
\end{proposition}

\begin{proof}[Proof of \cref{prop:unconstrained}]
 According to \cite{lakkaraju2016interpretable}, the size objective is a non-negative and submodular function. Similarly, the expected utility—assuming each individual receives a single rule and selects the best option—is also a non-negative submodular function. As a result, the linear combination of these functions remains a non-negative submodular function, and maximizing it is known to be NP-hard \cite{khuller1999budgeted}.
\end{proof}

\begin{proposition}
\label{prop:matroid}
    Individual fairness and rule coverage constraints are matroid constraint
\end{proposition}

\begin{proof}[Proof of Proposition \ref{prop:matroid}]

We will show these constraints satisfy the following properties: 

\begin{enumerate}
    \item \textbf{Hereditary Property}: If  $S$ is an independent set, then every subset of $S$ is also an independent set.
    \item \textbf{Exchange Property}: If $S$ and $T$ are independent sets and $|S| < |T|$, then there exists an element $e \in T \setminus S$ such that $S \cup \{e\}$  is also an independent set.
\end{enumerate}
These two properties ensure that the set system behaves like a matroid.

In our setting Start by specifying the ground set is all possible rules and what qualifies as an "independent set" is a subset of rules satisfying a constraint. 

\paragraph*{\bf Individual Fairness} 
If a set of rules $R$ satisfies the individual fairness constraint, this means each rule within $R$ individually satisfies the constraint. Consequently, any subset $R' \subseteq R$ also upholds individual fairness. This further implies the exchange property, as any rule that satisfies individual fairness can be added to an individually fair set of rules while preserving individual fairness.

\paragraph*{\bf Rule coverage} 
If a set of rules $R$ satisfies the rule coverage constraint, this means each rule within $R$ individually satisfies the rule coverage constraint. Consequently, any subset $R' \subseteq R$ also satisfies the rule coverage constraint. This also implies the exchange property, as any rule that satisfies rule coverage can be added to a set of rules satisfying rule coverage while preserving the rule coverage constraints.
\end{proof}


For the group coverage, we can show that merely finding a solution that satisfies the constraints, even without maximizing expected utility, is NP-hard via a reduction from the Set Cover problem~\cite{feige1998threshold}. 

\begin{proposition}
    \label{prop:group_coverage}
    \probName\ with a group-coverage constraint is NP-hard
\end{proposition}


\begin{proof}[Proof of \ref{prop:group_coverage}]
In the decision version of the Set Cover problem, we are given a universe of elements $U = \{x_1, \ldots, x_{n'}\}$, a collection of $m$ subsets $S_1, \ldots, S_{m'} \subseteq U$ and a number $k$. The question is whether there exists a cover of $U$ of at most $k'$ subsets.


In the decision version of \probName\, 
we are searching for a set of rules $R$ such that: $f(R) \geq \tau$, where:
$$f(R) = \lambda_1 {\cdot} (l - size(R)) {+} \lambda_2 {\cdot} ExpUtility(R)$$
 such that $R$ satisfies the group-coverage constraints, defined by the parameter $\theta$. In this proof, we assume no protected group is given, namely the constraint requires that the selected ruleset $R$ would cover at least a $\theta$ fraction of the population (i.e., the protected group to be the empty group).  


Given an instance of the set cover problem, we build an instance of the \probName\ problem as follows. 
We build a relation $R$ with $m'+1$ attributes, $\attrset = (A_1, \ldots, A_{m'}, O)$, and containing $n'+m'$ tuples. For each element $x_i \in U$, we create a tuple $t_i$, such that $t_i[A_j]=1$ iff $x_i \in S_j$. 
We further add $m'$ tuples $t_{S_j}$ such that $t_{S_j}[A_j] = 1$, $t_{S_j}[O] = 0$, and $t_{S_j}[A_p] = l \neq 0$ for all $p \neq j$ where $l$ is a unique number not used anywhere else in an attribute of $R$. We set the outcome variable to be $O$.

Here, $\pattern_g$ can be any predicate.  
Note that each set of tuples defined by a pattern can only have an outcome of $0$, as the outcome of all tuples is $0$. 
Therefore, the utility of all intervention patterns is $0$. 
For \probName, we further define the threshold for the group coverage constraint $\theta = \frac{n'+k'}{n'+m'}$.
The underlying causal DAG, $G$, only contains the edges of the form $A_j \to O$ for all $1\leq j \leq m'$. 
We claim that there exists a cover of $U$ with at most $k$ sets iff there exists a solution $R$ to \probName\ such that $f(R) \geq (l-k)$. 

($\Rightarrow$) Assume we have a collection $S_{j_1}, \ldots, S_{j_k}$ such that $\cup_{j = j_1}^{j_k} S_j = U$. We show that there is a solution for \probName\ as follows. For each $S_{j_1}$, we choose for the solution the pattern $\pattern_g^{j_i}: A_{j_i} = 1$. 
We show that $R = \{(\pattern_g^{j_1}, \emptyset), \ldots, (\pattern_g^{j_k}, \emptyset)\}$ is a solution to \probName.
The intervention pattern can be any pattern, as the utility of every intervention pattern is $0$.
First, we note that all tuples of the form $t_i$ are covered by at least one grouping pattern by their definition. For the $m$ remaining tuples, we have coverage of at most $k$ tuples. These are the tuples $t_{S_{j_i}}$ that have $A_{j_i} = 1$. 
Thus, the number of covered tuples is exactly $n'+k'$ out of $n'+m'$ tuples in $R$. 
If there are fewer than $k$ tuples we can augment the original cover with arbitrary sets to obtain a cover of size $k$.

($\Leftarrow$) Assume we have a solution $R$ to \probName\ with the aforementioned parameters. We show that we can find a solution to the set cover problem. 
First, note that since $f(R) \geq (l-k)$ and the expected utility is always $0$, that means we have selected no more than $k$ rules.

Suppose $R = \{(\pattern_g^{j_1}, \emptyset), \ldots, (\pattern_g^{j_k}, \emptyset)\}$. 
We first claim that no grouping pattern that includes $A_i = 0$ in a conjunction can be included in $R$ as such a pattern will not cover any tuple $t_{S_j}$ since these tuples do not have an attribute with value $0$ by definition (and any other number other than $1$ will only cover a single tuple). Thus, the number of covered tuples will be $< \frac{n'+k'}{n'+m'} = \theta$, which would contradict the assumption that this is a valid solution to \probName\ that satisfies the coverage constraint. 
Hence, all patterns are of conjunctions of $A_i = 1$. For each intervention pattern of the form $\pattern_g = \wedge_{j=i_a}^{i_b} (A_j = 1) \land (A_p = l)$, we choose an arbitrary attribute in the conjunction $A_{j}$ if $(A_j = 1) \in \pattern_g$ and choose $S_{j}$ for the cover. Finally, if there is an uncovered element $x$ in $U$ and $R$ includes a pattern in of the form $\pattern_g = (A_j = l)$ where $l \neq 1$, we choose for the cover a set $S$ that covers $x$ arbitrarily.  
We claim that the chosen collection of sets is a cover of $U$. 
To see this, 
recall that we claimed that the coverage of $R$ is at least $n+k$. 
If the coverage includes tuples of the form $t_{S_j}$, then each pattern covers a single tuple. Suppose these patterns are $\pattern_a, \ldots, \pattern_b$. 
When building the coverage, instead of these patterns, we add a set that covers elements not yet covered by existing patterns. Thus, there are at least $b-a$ covered elements from $U$ in addition to the $n+k-(b-a)$ tuples covered by the patterns. Thus, the set cover we have assembled contains $n-(b-a) + (b-a) = n$ elements and covers all elements in $U$. 
\end{proof}
