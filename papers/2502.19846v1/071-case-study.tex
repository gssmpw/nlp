\vspace{-1mm}
\section{Case Study}
\label{sec:casestudy}

The objective of this case study is to evaluate the impact of various constraints on the solution.
We analyze two datasets, (1) German Credit (German in short) and (2) Stack Overflow (SO in short), each with a corresponding protected group, and assess the rules chosen by \sysName\ under different constraints. We aim to understand how these constraints influence coverage, utility, and disparities (for fairness) between protected and non-protected groups. \revb{We present example chosen rules under different configurations. We chose the rules by randomly picking one from each category (one that favors the protected group, one that favors the non-protected, and another that is more balanced). The full lists of rules are available in \cite{fullversion}.}


\vspace{1mm}
\paratitle{Datasets \revb{\& protected groups}}
We examine two commonly used datasets:
(1) Stack Oveflow (SO)~\cite{stackoverflowreport}, as described in \Cref{example:ex1}. Here, the goal is to increase salary. (2) German Credit~\cite{asuncion2007uci}, which contains details of bank account holders, including demographic and financial information. Here, the goal is to increase the credit score (binary). 
The corresponding 
causal DAG was constructed using  
~\cite{youngmann2023causal}. The datasets' statistics are presented in \cref{tab:datasets}. The protected groups were selected to represent subgroups where the desired outcome was relatively low and sufficiently large to ensure the discovery of statistically significant rules. \revb{The protected group in Stack Overflow is defined as individuals from countries with a low GDP, which constitutes 21.5\% of the data (the GDP attribute is categorical in this dataset). In the German data, the protected group is defined as single females, which constitutes 9.2\% of the data.} 
% \brit{StackOverflow, medical datasets\url{https://aif360.readthedocs.io/en/latest/modules/datasets.html}}



\begin{table}
	\centering
\small
		\caption{Examined datasets.}
			\label{tab:datasets}
			% \vspace{-4mm}
	\begin{tabular} 
 % {|p{15mm}|p{7mm}|p{7mm}|p{18mm}|p{18mm}|}
 {p{8mm}cccp{37mm}}
		\toprule
	\textbf{Dataset} & \textbf{Tuples}& \textbf{Atts}& \textbf{Mut Atts}&\textbf{Protected Group}
	 \\
		\midrule 

SO&38K&20&10&People from countries with a low GDP (\revb{21.5\% of the data})\\
\hline
German&1000&20&15&Single Females (\revb{9.2\% of the data}) \\
				\bottomrule
	\end{tabular}
% 	\vspace{-1mm}
\end{table}


\begin{table*}[h!]
\centering
\small
\caption{Comparison of Solutions in Terms of Size, Coverage, Expected Utility and Unfairness. \revb{IDS and FRL were used to either (i) replace step 1 of \sysName\ to find grouping patterns; (ii) replace step 2 of \sysName\ to find intervention patterns}.}
\label{tab:problem_variants}
\begin{tabular}{lccccccc}
\toprule
\textbf{Stack Overflow (SP fairness)} & \textbf{\# rules} & \textbf{coverage} & \textbf{coverage pro} & \textbf{exp utility} & \textbf{exp utility non-pro}&\textbf{exp utility pro} &\textbf{unfairness} \\
% \midrule
% No constraints  & &&&& \\
% Group coverage  & &&&& \\
% Rule coverage  & &&&& \\
% Group fairness  & &&&& \\
% Individual fairness   & &&&& \\

% Group coverage, Group fairness  & &&&& \\
% Rule coverage, Group fairness  & &&&& \\

% Group coverage, Individual fairness  & &&&& \\
% Rule coverage, Individual fairness  & &&&& \\

% \midrule
\midrule 
No constraints  & 20& 99.91\%& 99.98\%& \textbf{32634.2}& \textbf{32626.98}& \textbf{18432.66}& 14194.32 \\
Group coverage  &20& 99.84\%& 99.88\%& 32597.02& 32595.1& 18340.29& 14254.81 \\
Rule coverage  & 10& \textbf{99.99}\%& \textbf{99.99}\%& 22301.77& 22292.02& 16604.92& \textbf{5687.1}\\
Group fairness  & 8& 97.52\%& 97.81\%& 27870.77& 27998.47& 17998.66& 9999.81 \\
Individual fairness   & 20& \textbf{99.99}\%& \textbf{99.99}\%& 28014.58& 28256.35& 14241.07& 14015.28 \\
Group coverage, Group fairness  & 11& 97.95\%& 98.85\%& 27934.76& 28144.58& 18145.23& 9999.35 \\
Rule coverage, Group fairness  & 12& 99.96\%& 99.89\%& 22284.1& 22279.93& 16594.77& 5685.16\\
Group coverage, Individual fairness  & 20& 99.74\%& 99.88\%& 28057.78& 28284.25& 15128.91& 13155.34\\
Rule coverage, Individual fairness  &13& \textbf{99.99}\%& \textbf{99.99\%}& 18591.41& 18606.68& 12797.15& 5809.53 \\


\hdashline[1pt/3pt] % Dashed line above
\revb{IDS (IF clause as grouping pattern)} &\revb{16}&\revb{100\%} &\revb{100\%} &\revb{29770.43}&\revb{29988.1} &\revb{16440.82}& \revb{13547.28}\\
 % \revb{IDS (only protected)} &\revb{21}&\revb{21.5\%} & \revb{100\%}&\revb{8545.33}&\revb{0} &\revb{19666.29}& \revb{-19666.29}\\


 \revb{IDS (IF clause as intervention pattern)} &\revb{16}&\revb{100\%} &\revb{100\%} &\revb{27763.89}&\revb{27714.9} &\revb{16888.1}& \revb{10826.8}\\

  % \revb{IDS (combined)} &\revb{37}&\revb{100\%} &\revb{100\%} &\revb{30641.11}&\revb{29988.1} &\revb{16440.82}&\revb{13547.28} \\

 \revb{FRL (IF clause as grouping pattern)} &\revb{9}&\revb{99.5\%} &\revb{98.85\%} &\revb{27777.43}&\revb{27782.3} &\revb{18891.22}& \revb{8891.08}\\
 % \revb{FRL (only protected)} &\revb{7}&\revb{21.5\%} &\revb{100\%} &\revb{79883.1}&\revb{0} &\revb{16445.3}&\revb{-16445.3} \\

  % \revb{FRL (combined)} &\revb{16}&\revb{99.5\%} &\revb{100\%} &\revb{27777.43}&\revb{18891.22} &\revb{18891.22}&\revb{8891.08} \\

   \revb{FRL (IF clause as intervention pattern)} &\revb{9}&\revb{100\%} &\revb{100\%} &\revb{28999.22}&\revb{28997.8} &\revb{16453.8}& \revb{12544}\\
        
\midrule

\textbf{German Credit (BGL fairness)} & \textbf{\# rules} & \textbf{coverage} & \textbf{coverage pro} & \textbf{exp utility} & \textbf{exp utility non-pro}&\textbf{exp utility pro}&\textbf{\common{unfairness}}   \\
\midrule 
No constraints  & 17& \textbf{100.0\%}& \textbf{100.0\%}& \textbf{0.39}& \textbf{0.39}& 0.27& 0.12 \\
Group coverage  &18& \textbf{100.0\%}& \textbf{100.0\%}& \textbf{0.39}& \textbf{0.39}& 0.3& 0.09 \\
Rule coverage  & 6& 96.0\%& \textbf{100.0\%}& 0.31& 0.31& 0.3& 0.01\\
Group fairness  & 18& \textbf{100.0\%}& \textbf{100.0\%}& \textbf{0.39}& \textbf{0.39}& 0.3& 0.09 \\
Individual fairness   & 20& \textbf{100.0\%}& \textbf{100.0\%}& 0.37& 0.37& 0.23& 0.14 \\
Group coverage, Group fairness  & 6& \textbf{100.0\%}& \textbf{100.0\%}& 0.36& 0.37& \textbf{0.31}& 0.06 \\
Rule coverage, Group fairness  & 3& 90.0\%& \textbf{100.0\%}& 0.29& 0.29& \textbf{0.31}& \textbf{-0.02}\\
Group coverage, Individual fairness  & 20& \textbf{100.0\%}& \textbf{100.0\%}& 0.37& 0.37& 0.23& 0.14\\
Rule coverage, Individual fairness  &8& 96.8\%& \textbf{100.0\%}& 0.29& 0.29& 0.23& 0.06 \\

 \hdashline[1pt/3pt] % Dashed line above
 \revb{IDS (IF clause as grouping pattern)} &\revb{12}& \revb{100\%}&\revb{100\%} &\revb{0.35}&\revb{0.35} &\revb{0.3}&\revb{0.05} \\
 % \revb{IDS (only protected)} &\revb{20}& \revb{9.2\%}&\revb{100\%} &\revb{0.13}&\revb{0} &\revb{0.3}&\revb{-0.3} \\


  \revb{IDS (IF clause as intervention pattern)} &\revb{12}& \revb{100\%}&\revb{100\%} &\revb{0.34}&\revb{0.34} &\revb{0.24}&\revb{0.1} \\
 % \revb{IDS (only protected)} &\revb{20}& \revb{9.2\%}&\revb{100\%} &\revb{0.13}&\revb{0} &\revb{0.3}&\revb{-0.3} \\

 % \revb{IDS (combined)} &\revb{32}& \revb{100\%}&\revb{100\%} &\revb{0.35}&\revb{0.35} &\revb{0.3}&\revb{0.05} \\


 \revb{FRL (IF clause as grouping pattern)} &\revb{13}&\revb{100\%} &\revb{100\%} &\revb{0.26}&\revb{0.26} &\revb{0.21}&\revb{0.05} \\


 \revb{FRL (IF clause as intervention pattern)} &\revb{13}&\revb{100\%} &\revb{100\%} &\revb{0.3}&\revb{0.3} &\revb{0.23}&\revb{0.07} \\
 
 % \revb{FRL (only protected)} &\revb{11}&\revb{9.2\%} & \revb{100\%}&\revb{0.09}&\revb{0} &\revb{0.11}&\revb{-0.11} \\
  % \revb{FRL (combined)} &\revb{24}&\revb{100\%} &\revb{100\%} &\revb{0.26}&\revb{0.26} &\revb{0.21}&\revb{0.05} \\

\bottomrule
\end{tabular}
\vspace{-4mm}
\end{table*}



% \begin{table}[h!]
% \centering
% \small
% \caption{\revb{Comparison of Solutions of the Baselines in Terms of Size, Coverage, and AUC}}
% \label{tab:baselines}
% \begin{tabular}{llccc}
% \toprule
%  & \textbf{Baseline}&\textbf{\# rules} & \textbf{coverage}  & \textbf{}  \\
% % \cmidrule{2-4}
% \toprule
% \multirow{4}{*}{\textbf{Stack Overflow}}& \revb{IDS (all datasets)} &16&  \\
% & \revb{IDS (only protected)} &21&  \\
% & \revb{FRL (all datasets)} &9&  \\
% & \revb{FRL (only protected)} &7&  \\
% \midrule

% \multirow{4}{*}{\textbf{German Credit}}& \revb{IDS (all datasets)} &12&  \\
% & \revb{IDS (only protected)} &20&  \\
% & \revb{FRL (all datasets)} &13&  \\
% & \revb{FRL (only protected)} &11&  \\

% \bottomrule
% \end{tabular}

% \end{table}







\vspace{1mm}
\paratitle{Default parameters} Unless otherwise specified, the threshold of the Apriori algorithm is set to 0.1. 
For the SO dataset, the coverage thresholds are set to 0.5. 
The threshold for the SP and BFL fairness constraint is set at \$10k. For the German dataset, the coverage thresholds are set at
30\% and the fairness thresholds are set at 0.1. This configuration allows for the generation of multiple rules. 

\cut{
\vspace{1mm}
\paratitle{\common{Results Summary}} \common{Our analysis indicates the following findings:
%$1.$ Achieving individual fairness is more challenging than group fairness, regardless of the fairness definition used. This is because most rules are not individually fair.\\
$1.$ Achieving rule coverage is harder than group coverage, as many useful (i.e., high utility) rules apply only to a small fraction of the population. This is not surprising, as it follows from what we proved in Lemma~\ref{lemma:individual_rules}, where we showed that the optimal strategy is to suggest a personalized rule for each individual. \\
$3.$ Without fairness constraints, we observe a significant disparity in the expected utility between the protected and non-protected.\\
$4.$ As expected, with SP fairness constraints, the difference in expected utility between protected and non-protected is bounded. \\
$5.$ As expected, with BGL fairness constraints, which consider only the minimal gain for the protected without regard for non-protected, we may still observe a disparity between the two groups. }
}


% \vspace{-2mm}
% % Adjust the page margins
% \begin{summary}{Result Summary}
% \small
% $\star$ Fairness may come at the cost of expected utility for everyone.\\
% $\star$ Achieving individual fairness is harder than group fairness, as most rules are unfair.\\
% $\star$ Achieving rule coverage is harder than group coverage, as many useful rules apply only to a small fraction of the population.\\
% $\star$ Without fairness constraints, we observe a significant disparity in the expected utility between the protected and non-protected.\\
% $\star$ With SP fairness constraints, the difference in expected utility between protected and non-protected is bounded. \\
% $\star$ With BGL fairness constraints, which consider only the minimal gain for the protected without regard for non-protected, we may still observe a disparity between the two groups. 
% \end{summary}




\vspace{1mm}
The results are shown in \cref{tab:problem_variants}, illustrating the trade-off between utility, coverage, and fairness. Without constraints, the expected utility is substantially higher, but this comes at the expense of greater disparities between protected and non-protected groups (as indicated by the unfairness score --- the difference between the expected utility of protected and non-protected). In the examined scenarios, coverage for both groups was achieved without constraints, but other protected group definitions may require them.


% \begin{figure}[h]
%     \centering
%     \noindent
%     \boxed{
%     \parbox{\columnwidth}{
%     $\bullet$ For individuals aged 24-34, pursue an undergraduate major in CS (expected utility for protected: \textcolor{magenta}{10,292}, expected utility for non-protected: \textcolor{cyan}{22,586}).\\
%         $\bullet$ For individuals with 6-8 years of coding experience,  exercise 1-2 times per week and pursue a bachelor's degree.(expected utility for protected: \textcolor{magenta}{15,864}, expected utility for non-protected: \textcolor{cyan}{18,157}).\\
% $\bullet$  For males whose parents have a secondary school education, exercise 3-4 times per week, and work with a computer 9-12 hours a day (expected utility for protected: \textcolor{magenta}{58,548}, expected utility for non-protected: \textcolor{cyan}{41,733}).
%     }}
%     \caption{Example 3 Selected Rules out of 11 for SO (with SP group fairness and group coverage constraints)}
%     \label{fig:example_rules_so_group_fairness_group_coverage}
% \end{figure}



\paratitle{Stack Overflow}
% In the Stack Overflow scenario, the coverage requirements were met even without explicitly imposing coverage constraints. This shows that coverage requirements are not always necessary, as some protected groups can naturally be stratified, ensuring the solution applies to a sufficient number of individuals without enforcement. 
Observe that while the expected utility for both protected and non-protected groups reaches its highest value in the no-constraints variant, the unfairness score is very high. This indicates that achieving SP fairness requires compromising on the expected utility for both protected and non-protected groups. Interestingly, rule coverage and individual fairness are difficult to achieve, as most rules fail to meet these criteria. This leads to lower expected utility for all groups. On the other hand, group coverage and fairness constraints are easier to satisfy, as they offer more flexibility by allowing the selection of some unfair rules alongside those specifically designed for the protected group. 


\vspace{-1mm}
\begin{ruleset}{\textbf{3 Selected Rules out of 11 for SO (SP group fairness)}}
\small
    $\triangleright (\mathbf{S1_a})$ For individuals aged 24-34, pursue an undergraduate major in CS (exp utility protected: \textcolor{red}{10,292}, exp utility non-protected: \textcolor{blue}{22,586}).\\
        $\triangleright (\mathbf{S1_b})$ \common{For individuals with 6-8 years of coding experience, work with a computer 9 - 12 hours a day.} (exp utility protected: \textcolor{red}{17,161}, expe utility non-protected: \textcolor{blue}{19,254}).\\
$\triangleright (\mathbf{S1_c})$ \common{For males whose parents have a secondary school education, work as back-end developers} (exp utility protected: \textcolor{red}{51,542}, exp utility non-protected: \textcolor{blue}{46,354}).  
\end{ruleset}
\vspace{-1mm}



We show above the three example rules selected under group fairness constraint. The first rule $S1_a$ is more advantageous for the non-protected group, the second ($S1_b$) benefits both protected and non-protected groups similarly, while the third rule ($S1_c$) is more beneficial for the protected group. Overall, all these rules together satisfy the group fairness requirement. We also present three example rules selected under individual fairness constraints. In this case, all rules ($S2_a, S2_b, S2_c$) are nearly equally beneficial for both groups, but the overall expected utility is lower. Finally, consider the three example rules selected with no constraints. Here, all rules ($S3_a, S3_b, S3_c$ in the figure below) favor the non-protected group, highlighting the importance of including fairness constraints. %Overall, the rules selected with no constraints have a higher overall utility while fairness constraints ensure that the utility is not biased in favor of only one group.

\begin{ruleset}{\textbf{3 Selected Rules out of 20 for SO (SP individual fairness)}}
\small
    $\triangleright (\mathbf{S2_a})$ \common{For males aged 25-34 with no dependents, pursue a bachelor's degree} (exp utility protected: \textcolor{red}{16,158}, exp utility non-protected: \textcolor{blue}{18,134}).\\
        $\triangleright (\mathbf{S2_b})$ \common{For individuals aged 18 -24, work as back-end developers.} (exp utility protected: \textcolor{red}{12,664}, exp utility non-protected: \textcolor{blue}{14,101}).\\
$\triangleright (\mathbf{S2_c})$ \common{For individuals with dependents, pursue an undergraduate major in
CS} (exp utility protected: \textcolor{red}{16,124}, exp utility non-protected: \textcolor{blue}{17,138}).
\end{ruleset}

\begin{ruleset}{\textbf{3 Selected Rules out of 20 for SO (no fairness constraints)}}
\small
    $\triangleright (\mathbf{S3_a})$ \common{For White aged 25-34 with dependents, work with computer 9 - 12 hours a day and work as back-end developers} (exp utility protected: \textcolor{red}{11,147}, exp utility non-protected: \textcolor{blue}{32,248}).\\
        $\triangleright (\mathbf{S3_b})$ \common{For males aged 35-44 with dependents, work as back-end developers}. (exp utility protected: \textcolor{red}{11,189}, exp utility non-protected: \textcolor{blue}{40,207}).\\
$\triangleright (\mathbf{S3_c})$ \common{For students, pursue an undergraduate major in
CS} (exp utility for protected: \textcolor{red}{12,126}, exp utility for non-protected: \textcolor{blue}{22,174}).
\end{ruleset}

%\noindent
%The example rules from the German dataset are in full version \cite{fullversion}.

\paratitle{German}
% In this scenario as well, the coverage requirements were satisfied even without explicitly imposing coverage constraints. 
While the expected utility for both protected and non-protected peaks in the no-constraints variant, the unfairness score is relatively high. This suggests that achieving BGL fairness necessitates compromising utility for both groups. Notably, to reduce the unfairness, it is feasible to impose either a rule coverage constraint or a rule coverage constraint combined with group fairness. 
We show three rules selected under BGL group fairness constraints below.
Since we are focusing on BGL fairness, which considers only the minimal gain for the protected group without regard for the gains of the non-protected group, we still observe a disparity between the two, even with a fairness constraint in place. 
%Since this is a BGL fairness constraint, there are no restrictions on the difference between the expected utility for protected and non-protected, provided that the expected utility for the protected group exceeds a predefined threshold.

\begin{ruleset}{\textbf{3 Selected Rules out of 20 for German (group BGL fairness)}}
\small
    $\triangleright (\mathbf{G1_a})$ For people aged 24-30 with 0-2 dependents, maintain a minimum balance of 200 DM in the checking account and pursue skilled employment
    (exp utility protected: \textcolor{red}{0.26}, exp utility non-protected: \textcolor{blue}{0.35}).\\
        $\triangleright (\mathbf{G1_b})$ For people seeking a loan to purchase furniture or equipment, maintain a minimum balance of 200 DM in the checking account (exp utility protected: \textcolor{red}{0.38}, exp utility non-protected: \textcolor{blue}{0.29}).\\
        $\triangleright (\mathbf{G1_c})$ For people seeking a loan for an unspecified purpose, maintain a minimum balance of 200 DM in the checking account and own a house. (exp utility protected: \textcolor{red}{0.54}, exp utility non-protected: \textcolor{blue}{0.41}).
\end{ruleset}


%Example rules selected by \sysName\ under a group fairness constraint are shown above. 





\begin{table*}[h!]
\centering
\small
\caption{Comparison of Solutions in Terms of Fairness}
\label{tab:fairness_variants}
\begin{tabular}{lccccccc}
\toprule
\textbf{Stack Overflow (SP fairness)} & \textbf{\# rules} & \textbf{coverage} & \textbf{coverage pro} & \textbf{exp utility} & \textbf{exp utility non-pro}&\textbf{exp utility pro} &\textbf{unfairness} \\
\midrule 
%Group SP (0) &  20& 99.4\%& 99.31\%& 31682.34& 31675.57& 21051.36& 10624.21 \\
Group SP (2.5K)  &4& 97.82\%& 99.0\%& 20973.55& 20772.77& 18275.44& \textbf{2497.33} \\
Group SP (5K)  & 7& 97.31\%& 98.24\%& 22805.52& 23069.98& 18071.12& 4998.86\\
Group fairness (10K)  & 8& 97.52\%& 97.81\%& 27870.77& 27998.47& 17998.66& 9999.81 \\
Group SP (20K) & 20& \textbf{99.88\%}& \textbf{99.94\%}& \textbf{32671.11}& \textbf{32664.45}& \textbf{18423.64}& 14240.81 \\
%Individual SP (0) & 0& 0\%& 0\%& 0& 0& 0& 0 \\
\hline
Individual SP (2.5K) & 20& 99.95\%& 99.98\%& 24070.94& 24433.55& 12784.62& 11648.93 \\
Individual SP (5K)  & 20& \textbf{99.99\%}& \textbf{99.99\%}& 25526.1& 25911.22& 15327.21& \textbf{10584.01}\\
Individual SP(10K)   & 20& \textbf{99.99}\%& \textbf{99.99}\%& 28014.58& 28256.35& 14241.07& 14015.28 \\
Individual SP (20K) & 20& 99.51\%& 99.63\%& \textbf{29984.0}& \textbf{29966.29}& 14929.7& 15036.59\\
\midrule

% \textbf{German Credit (BGL fairness)} & \textbf{\# rules} & \textbf{coverage} & \textbf{coverage pro} & \textbf{exp utility} & \textbf{exp utility non-pro}&\textbf{exp utility pro}&\textbf{unfariness}   \\
% \midrule 
% Group BGL (0.10) &  18& 100.0\%& 100.0\%& 0.39& 0.39& 0.3& 0.09 \\
% Group BGL (0.25)  &0& 0\%& 0\%& 0& 0& 0& 0 \\
% Group BGL (0.30)  & 0& 0\%& 0\%& 0& 0& 0& 0\\
% Group BGL (0.35) & 0& 0\%& 0\%& 0& 0& 0& 0 \\
% Individual BGL (0.10) & 20& 100.0\%& 100.0\%& 0.37& 0.37& 0.23& 0.14 \\
% Individual BGL (0.25) & 20& 100.0\%& 100.0\%& 0.37& 0.37& 0.26& 0.11 \\
% Individual BGL (0.30)  & 18& 100.0\%& 100.0\%& 0.37& 0.37& 0.27& 0.1\\
% Individual BGL (0.35) & 15& 100.0\%& 100.0\%& 0.37& 0.37& 0.25& 0.12\\
        
        

% \bottomrule
\end{tabular}
\end{table*}