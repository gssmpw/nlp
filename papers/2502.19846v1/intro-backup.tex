%!TEX root=main.tex

\section{Introduction}
\brit{we shoud discuss causumx, highkit the fitness and not the causal part}

Decision-makers, analysts, data scientists, and policymakers frequently base their conclusions on collected data, using it to extract insights. This allows us to extract actionable recommendations aimed at influencing outcomes of interest, such as increasing product satisfaction or income levels, or decreasing the likelihood of experiencing a serious disease \cite{galhotra2022hyper,lakkaraju2016interpretable}.
Unlike prediction tasks, where achieving high accuracy is the primary goal, decision-making tasks with significant societal impact require recommendations to be actionable, interpretable and justifiable~\cite{}. In these cases, it is essential to provide a clear explanation for why a specific recommendation is made for a particular subpopulation.

Understanding the \emph{causal factors} driving a recommendation aimed at influencing a target outcome for different groups enables informed, data-driven decision-making to address inequities. For example, if a policymaker understands the causal factors affecting the salary of a particular race or gender in a specific country, they can implement corrective measures to improve the situation -- something that may not be achievable using insights based solely on associations. Identifying these causal insights is crucial for addressing societal biases and creating more equitable outcomes.

%a particular race or gender in a specific country, they can implement corrective measures to improve the situation -- something that may not be achievable using insights based solely on associations. 




In this work, we study the problem of generating causal insights (referred to as \emph{prescription rules}), which serve as actionable recommendations to improve an outcome of interest. Causal reasoning is considered as one of the most important requirements,  to generate explanations that are actionable and aligned with human reasoning.
Recent works have introduced causality to the field of database research~\cite{meliou2010causality,  meliou2014causality,salimi2020causal,10.14778/3554821.3554902}. It has been incorporated into various tasks including data discovery~\cite{galhotra2023metam,youngmann2023causal}, query result explanation~\cite{salimi2018bias,youngmann2022explaining,DBLP:journals/pacmmod/YoungmannCGR24}, and large system diagnostics~\cite{markakis2024sawmill,causalsim,sage, gudmundsdottir2017demonstration}. 

\sg{I was not sure so I commented the older version}
%\emph{Causal inference} has been thoroughly studied in AI and Statistics~\cite{pearl2009causal,rubin2005causal}. Causal analysis is a vital tool in determining the effect of a \emph{treatment} on an \emph{outcome}, and has been used in decision-making in medicine \cite{robins2000marginal}, economics \cite{banerjee2011poor}, biology \cite{shipley2016cause}, and in high-stakes areas such as identifying the root causes of failures in critical infrastructure systems to prevent catastrophic outcomes. Recent works have introduced causality to the field of database research~\cite{meliou2010causality,  meliou2014causality,salimi2020causal,10.14778/3554821.3554902}. It has been incorporated into various tasks including data discovery~\cite{galhotra2023metam,youngmann2023causal}, query result explanation~\cite{salimi2018bias,youngmann2022explaining,DBLP:journals/pacmmod/YoungmannCGR24}, and large system diagnostics~\cite{markakis2024sawmill,causalsim,sage, gudmundsdottir2017demonstration}. We propose leveraging causal inference to generate interpretable and justifiable insights (referred to as \emph{prescription rules}), which serve as actionable recommendations to improve an outcome of interest. Causal reasoning is considered as one of the most important requirements,  to generate insights that are actionable and aligned with human reasoning.


Next, we introduce a running example that will be referenced throughout the paper.




\begin{table*}[]
\footnotesize
    \centering
    	\caption{\textnormal{A subset of the Stack Overflow dataset.}}
         \label{tab:data}
    	% \vspace{-4mm}
  			\begin{tabular}[b]{|l|l|l|c|l|l|c|l|c|}
  			
				%\multicolumn{9}{c}{\textbf{Users}}\\ 
				\hline

				\textbf{ID}
    
    % \textbf{Country}& \textbf{Continent} 
    
    &\textbf{Gender} &\textbf{Ethnicity}&
				\textbf{Age} &\textbf{Role} &
				\textbf{Education} &\textbf{Country}&\textbf{Undergrad Major}&\textbf{Salary}
				\\ \hline

				1 &Male&White&26&Data Scientist & PhD& US&Computer Science&180k\\
    		2 &Non-binary&White&32&QA developer & Bachelor's degree& US&Mechanical Eng.&83k\\

 3 &Male&South Asian&29&C-suite executive  & Bachelor's degree & India&Computer Science&24k\\

  4 &Female&South Asian&25&Back-end developer  & Master's degree & India&Mathematics&7.5k\\

  5 &Male&East Asian&21&Back-end developer & Bachelor's degree & China&Computer Science&19k\\
  

        % $\ldots$ &  $\ldots$&  $\ldots$&  $\ldots$&  $\ldots$&  $\ldots$&  $\ldots$&  $\ldots$&  $\ldots$&  $\ldots$&  $\ldots$\\
    \hline
			\end{tabular}
\end{table*}




\begin{example}	%
Consider the Stack Overflow (SO) annual developer survey
\cite{stackoverflowreport}, where respondents from around the world answer
questions about their jobs and demographics. A sample of the dataset with a subset of the
attributes is presented in \cref{tab:data}.

Alex, a researcher in the UN finance department, is interested in discovering ways to increase the salaries of high-tech employees worldwide. She is looking for a set of actionable recommendations (referred to as \emph{prescription rules}) to raise the overall average salary.

Using association-based approaches~\cite{chen2018optimization,lakkaraju2016interpretable}, she may find that individuals who spend fewer than 5 hours a day on their computer have higher salaries than those who work more hours. However, this insight merely highlights a correlation: C-level executives, for instance, often spend more time in meetings and, therefore, less time at their computers. Naturally, their higher salaries are due to their position, not the amount of time they spend on their computer.

\brit{Benton, please change this part according to the experiments}
In contrast, through causal analysis, she discovers that for individuals aged 20-30 in the U.S., holding a PhD can result in a \$2,000 annual salary increase, while for those over 40 worldwide, learning Python can boost their salary by \$3,000 per year.
\sg{Discuss how rules can be unfair here}
\end{example}


Our objective is to generate a small set of prescription rules aimed at increasing (or decreasing) an outcome of interest. This is framed as an optimization problem where the goal is to select the fewest prescription rules that maximize utility (i.e., the expected increase or decrease in the outcome). However, focusing solely on maximizing utility can result in a scenario where only a small fraction of the population gets significant improvement, while others experience no benefit. Additionally, even if a large portion of the population receives recommendations, a protected subpopulation might not share the benefits and, worse, their situation could deteriorate, exacerbating inequalities.

We, therefore, enable users to incorporate various \emph{coverage and fairness constraints} along with the overall objective. Following research concerning fairness in database research~\cite{stoyanovich2020responsible,galhotra2022causal}, we assume the existence of a protected subpopulation, defined by sensitive attributes such as gender, or ethnicity. Coverage constraints ensure that a substantial fraction of the population and protected subpopulation receive at least one recommendation. Fairness constraint ensure that the expected utility of the protected population is ``comparable'' compared to the utility received by the unprotected individuals. We devise novel group and individual fairness definitions tailored for prescription rules. 

\begin{example}
    \brit{TODO: add an example to show how fairness and coverage constraints affect the solution}
\end{example}


\noindent
Our main contributions are as follows. \\
%\begin{itemize}[leftmargin=*,topsep=0pt]
{\bf (1)} We {\bf develop a framework that generates a set of prescription rules to enhance an outcome of interest.} A prescription rule consists of a \emph{grouping pattern} and a \emph{treatment pattern}, representing the target subpopulation and the actionable recommendation for that group, respectively. The strength of the causal effect of this treatment on the subgroup is used to measure the expected utility of a rule. Our objective is to identify the smallest set of prescription rules that maximizes overall expected utility. We refer to this problem as the \probName\ problem, and show that it is NP-hard.
    \par
    \noindent
{\bf (2)} We {\bf develop several coverage and fairness constraints} to ensure that the solution for the \probName\ problem is applied to a sufficient number of individuals and to minimize inequalities. We establish various definitions for coverage and fairness constraints tailored for causal analysis and analyze the complexity of the \probName\ problem subject to different constraints.  
    \par
    \noindent
{\bf (3)} We {\bf propose a general three-step algorithmic framework named \sysName}. The first step involves mining frequent grouping patterns using the Apriori algorithm~\cite{agrawal1994fast}. In the second step, we employ a lattice-based algorithm to find promising treatment patterns for each grouping pattern identified in the previous step. Finally, the third step applies a greedy approach to determine a solution. \sysName\ can be easily adapted to accommodate all variants of the \probName\ problem.
\par
\noindent
{\bf (4)} We provide a thorough {\bf experimental analysis} and {\bf multiple case study} that include two datasets, four baselines, and 18 variations of our problem with different constraints. \brit{TODO} 
%\end{itemize}


\paragraph*{Paper outline} Related work is discussed in \cref{sec:related}. Background on causal inference is provided in \Cref{sec:prelim}, and our problem formulation can be found in \cref{sec:problem}. Complexity analysis of the problem studied under different constraints is given in \cref{sec:hardness}, followed by a discussion on selecting the appropriate problem variant. Our algorithmic framework is presented in \cref{sec:algo}, a case study demonstrating the impact of different constraint configurations on the solution is given in \cref{exp:problem_variants}, and our experimental evaluation is detailed in \cref{sec:experiments}. Finally, we discuss the limitations of our framework in \cref{sec:conc}.

% \noindent
% \boxed{\parbox{\columnwidth}{$\bullet$ 
% For people with a professional degree, move to the United Kingdom
%  (coverage = 435 (20), coverage-protected = 20 (13), utility = 186855, utility-protected = 0.)\\
% $\bullet$ For graphic developers, move to the	United States
%  (coverage = 116 (29), coverage-protected = 8 (2), utility = 169431, utility-protected = 0).\\
% $\bullet$ For people who have no formal education, move to the United States
%  (coverage = 123 (34), coverage-protected = 7 (2), utility = 206742, utility-protected = 0).\\
% % \textcolor{red}{size = 38, length = 76, overlap = 64029181, utility = 1659307}\\
% \textcolor{blue}{overall coverage =674, expected utility = 187485
% coverage-protected = 35, expected utility-protected = 0}
% \sr{should mention protected group, and possibly not mention coverage in the intro or just intuitively like high coverage}
% }}


% Alex notes that although these rules result in a \$187,485 increase in the overall salary for those to whom they apply, they only affect a small fraction of the population, specifically 674 individuals. Additionally, although the expected salary increase is substantial, there is no expected increase in salary for non-males, a subpopulation of particular interest to Alex. In other words, applying these rules would result in no gain for non-males.
% \end{example}

% \begin{example}[Episode 2 - coverage and fairness constraints]
% Alex introduces coverage and fairness constraints to ensure that enough people will benefit from the rules and that they will be \emph{fair} with respect to non-males. Specifically, she demands that the benefit for a randomly chosen individual to whom one of the rules applies is nearly the same as the benefit for a randomly chosen individual who identifies as non-male and to whom one of the rules applies.

% After adding these constraints, \sysName\ recommends the following set of prescription rules:



% \noindent
% \boxed{\parbox{\columnwidth}{$\bullet$ 
% For people who have no formal education, move to the United States
%  (coverage = 123 (34), coverage-protected = 7 (2), utility = 206742, utility-protected = 0)\\
% $\bullet$ 
% For females, change role to	DevOps specialist (coverage = 2256 (47), coverage-protected = 2256 (47), utility = 90023, utility-protected = 90023).\\
% $\bullet$ For people with a Master's degree, move to the	United States
%  (coverage = 9097 (2222), coverage-protected = 642 (236), utility = 85390, utility-protected = 84201).\\
% % \textcolor{red}{size = 38, length = 76, overlap = 64029181, utility = 1659307}\\
% \textcolor{blue}{overall coverage =11476	
% , expected utility = 87601,
% coverage-protected = 2905, expected utility-protected = 88519}
% }} 







% \begin{figure}[t]
%         \centering
%         \begin{minipage}[b]{1.0\linewidth}
%             \small
%             \begin{tcolorbox}[colback=white]
%             \vspace{-2mm}
% $\bullet$ For backend developers, the treatment with the highest effect on salary is “Country = US” effect size = 78646
% \begin{itemize}
%     \item For non-male the effect is only: 59429
%     \item For male the effect is 80454
% \end{itemize}

% $\bullet$ For frontend developers, the treatment with the highest effect is :Formal Education = Bachelor's degree” effect size: 17340
% \begin{itemize}
%     \item For white the effect is 33464
%     \item For non-white the effect is 15320
% \end{itemize}


% $\bullet$ For people in Europe, the treatment with the highest effect on salary is “DevType = C-suite executive” effect size = 53254
% \begin{itemize}
%     \item For white the effect is 55112
%     \item For non-white 35249
% \end{itemize}



%             \vspace{-2mm}
%             \end{tcolorbox}
%         \end{minipage}%%
%          % \vspace{-4mm}
%         \caption{Set of prescription rules.}
%         \label{fig:so-explanation}
%     \end{figure}
