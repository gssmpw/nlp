%\vspace{-3mm}
\section{Limitations and Future Work}
\label{sec:conc}


\sysName\ generates actionable, causal-based recommendations to improve a target outcome while incorporating coverage and fairness constraints. \common{To the best of our knowledge, this is one of the first works in this direction, and several directions of future work remain. In this section, we discuss some of the current limitations of \sysName\ and future directions. }
\reva{

{\bf Generation and usage of rules by \sysName.~} \sysName\ can be used to recommend actions for different subpopulations toward optimizing a target. As an example scenario, a policymaker may select the target outcome and the parameters for coverage and fairness constraint (which may be iteratively varied and inspected based on the application).
%the relevant problem variant (e.g., coverage and fairness constraints). 
\sysName\ then generates a prescription ruleset as recommended actions for different subpopulations.
%for the policymaker to implement. 
The current framework assumes that the policymaker is trustworthy, will not misuse the rules, and will publish the relevant recommendations for each subpopulation. However, it is important to note that if not all rules are provided to all subpopulations, disparities among subpopulations may increase. Further, the generated rules do not consider global constraints, e.g., if the targeted outcome is salary in an institution, there may be a total budget, and the rules may not be applicable independently. Other variants of the problem remain future work.}

{\bf Considering constraints and resources.~} We acknowledge that the generated rules may lack robustness, meaning they might not impact all individuals receiving the recommendation in the same way. An exciting direction for future research would be to focus on ensuring the robustness of prescription rules. Another direction for future work is to account for the cost of interventions. Some interventions may be impractical (e.g., pursuing a bachelor’s degree in CS for someone who already holds a PhD in CS) or vary significantly in cost (e.g., moving to the US versus learning Python). \reva{Additionally, the gain in objective may vary across different subpopulations. For example, increase in $\$10000$ revenue would have caried impact on quality of life in different countries, depending on the cost of living and purchasing power. } Future research will incorporate intervention costs to generate budget-constrained prescription rules \reva{as well as address finite resource allocation scenarios to account for cases where the population size that can achieve improved outcomes is limited}.

{\bf Extension to multi-table data.~} \sysName\ currently supports a single-relation database without dependencies among tuples to ensure compliance with the SUTVA assumption~\cite{rubin2005causal} (discussed in \cref{sec:background-causal}). However, this assumption breaks down even in single-table databases with tuple dependencies. In single-table settings, intervention and grouping patterns are straightforward to define. Extending these definitions to multi-table databases, where grouping attributes and interventions may originate from different tables, introduces a significant challenge. This complexity arises due to many-to-many relationships and cross-table patterns. 
Previous work, such as \cite{salimi2020causal,galhotra2022hyper}, has extended causal models to handle multi-table data, but they have not explicitly targeted recommendations for subgroups. Expanding our framework to support multi-relational databases with complex dependencies remains an important direction for future research. Notably, prior work leveraging causal inference \cite{ma2023xinsight, youngmann2022explaining, salimi2018bias, DBLP:journals/pacmmod/YoungmannCGR24} has also primarily focused on single-table settings.





\revc{We also note that prescription rules consist of conjunctions of predicates, which may not always be explainable. For instance, in a highly detailed dataset, a possible rule might suggest that individuals aged 18.5- 21.2 should learn Python. Understanding the rationale behind such a rule is not straightforward. Future work would focus on binning continuous variables into meaningful ranges while preserving causal inference validity. }


Finally, it is essential to recognize that the generated rules may be influenced by several factors, including the method used to evaluate causal effects, the thresholds set for the constraints, the overall quality of the data, and the quality of the causal DAG.
\reva{Specifically, we acknowledge that inaccuracies in the DAG may lead to flawed rules. Consequently, we assume that the causal DAG is provided as part of the input, with the responsibility for validating its correctness resting on the policymaker. Nonetheless, the causal DAG only needs to specify causal dependencies between variables without detailing the nature of those dependencies.}


