
\sg{should we stop here?}
%Consequently, maximizing it is an NP-hard problem. 

We define the decision version of the \probName\ problem in the following way: Given a bound $k$ on the number of selected rules and a threshold $\tau$ on the expected utility, the question is if there exists a set of at most $k$ rules whose expected utility is at least $\tau$. We can demonstrate that this problem is NP-hard.
We prove that the unconstrained \probName\ problem via a reduction from the max-cover problem. Furthermore, adding individual fairness constraints does not 
We can show that both individual-level fairness constraints (individual parity and individual loss) are matroid constraints. Therefore, adding them does not change the complexity of the problem, as maximizing a non-negative monotone submodular function under a matroid constraint remains NP-hard~\cite{calinescu2007maximizing}.







\begin{proposition}
\label{prop:unconstrained}
    The \probName\ problem with and without individual fairness constraint is NP-hard.
\end{proposition}

\begin{proof}[Proof of \cref{prop:unconstrained}]
 According to \cite{lakkaraju2016interpretable}, the size objective is a non-negative and submodular function. Similarly, the expected utility—assuming each individual receives a single rule and selects the best option—is also a non-negative submodular function. As a result, the linear combination of these functions remains a non-negative submodular function, and maximizing it is known to be NP-hard \cite{khuller1999budgeted}.
\end{proof}

For the group-level fairness constraints, we can demonstrate that this constraint does not form a matroid and provide a reduction from the Knapsack Problem~\cite{kellerer2004introduction}.

\begin{proof}[Proof of \cref{prop:group_fairness}]
    Consider an instance of the knapsack problem with $n$ items of weight $w_i$ and value $v_i$. For each item, construct a rule $r_i$ which has fairness group fairness metric $w_i$, i.e. $utility_p(r_i) - utility_{\bar{p}}(r_i) = w_i$. Each rule has utility $v_i$, length $1$, overlap $0$.

    Now, we can observe that if \probName\ can be solved in polynomial time, we can solve the knapsack problem in polynomial time. This shows that \probName is NP-hard. 
\end{proof}

\begin{proof}[Proof of \ref{prop:group_coverage}]
In the decision version of the Set Cover problem, we are given a universe of elements $U = \{x_1, \ldots, x_{n'}\}$, a collection of $m$ subsets $S_1, \ldots, S_{m'} \subseteq U$ and a number $k$. The question is whether exists a cover of $U$ of at most $k'$ subsets.


In the decision version of \probName\, 
we are searching for a set of rule $R$ such that: (i) its size is at most $k$; (ii) its length is at most $l$; (iii) its overlap is bounded by $o$, and (iv) its utility is at least $\tau$. The goal is to find a set of rules $R \subseteq \{r_i\}_{i = 1}^l$ such that $R$ satisfies the four constraints above, as well as the group-coverage constraints, defined by the parameters $\theta$ and $\theta_p$. 


Given an instance of the set cover problem, we build an instance of the \probName\ problem as follows. 
We build a relation $R$ with $m'+1$ attributes, $\attrset = (A_1, \ldots, A_{m'}, O)$, and containing $n'+m'$ tuples. For each element $x_i \in U$, we create a tuple $t_i$, such that $t_i[A_j]=1$ iff $x_i \in S_j$. 
We further add $m'$ tuples $t_{S_j}$ such that $t_{S_j}[A_j] = 1$, $t_{S_j}[O] = 0$, and $t_{S_j}[A_p] = l \neq 0$ for all $p \neq j$ where $l$ is a unique number not used anywhere else in an attribute of $R$. We set the outcome variable to be $O$, and the protected group to be the empty group.

Here, $\pattern_g$ can be any predicate.  
Note that each set of tuples defined by a pattern can only have an outcome of $0$, as the outcome of all tuples is $0$. 
Therefore, the utility of all treatment patterns is $0$. 
For \probName, we further define $\theta = \frac{n'+k'}{n'+m'}$, $k = k'$, and $\tau = 0$. 
There is no protected group, and therefore $\theta_p$ is set to $0$.
We also set the length bound to $k$ and the overlap bound to $(n'+m')\cdot k^2$.
The underlying causal DAG, $G$, only contains the edges of the form $A_j \to O$ for all $1\leq j \leq m'$. 
We claim that there exists a cover of $U$ with at most $k$ sets iff there exists a solution $R$ to \probName\ such that $|R| \leq k$, all tuples are covered, and $Utility(R) \geq 0$. 

($\Rightarrow$) Assume we have a collection $S_{j_1}, \ldots, S_{j_k}$ such that $\cup_{j = j_1}^{j_k} S_j = U$. We show that there is a solution for \probName\ as follows. For each $S_{j_1}$, we choose for the solution the pattern $\pattern_g^{j_i}: A_{j_i} = 1$. 
We show that $R = \{(\pattern_g^{j_1}, \emptyset), \ldots, (\pattern_g^{j_k}, \emptyset)\}$ is a solution to \probName.
The treatment pattern can be any pattern, as the utility of every treatment pattern is $0$.
First, we note that all tuples of the form $t_i$ are covered by at least one grouping pattern by their definition. For the $m$ remaining tuples, we have coverage of at most $k$ tuples. These are the tuples $t_{S_{j_i}}$ that have $A_{j_i} = 1$. 
Thus, the number of covered tuples is exactly $n'+k'$ out of $n'+m'$ tuples in $R$. 
If there are fewer than $k$ tuples we can augment the original cover with arbitrary sets to obtain a cover of size $k$.

($\Leftarrow$) Assume we have a solution to \probName\ with the aforementioned parameters. We show that we can find a solution to the set cover problem. 
Suppose the cover is $R = \{(\pattern_g^{j_1}, \emptyset), \ldots, (\pattern_g^{j_k}, \emptyset)\}$. 
We first claim that no grouping pattern that includes $A_i = 0$ in a conjunction can be included in $R$ as such a pattern will not cover any tuple $t_{S_j}$ since these tuples do not have an attribute with value $0$ by definition (and any other number other than $1$ will only cover a single tuple). Thus, the number of covered tuples will be $< \frac{n'+k'}{n'+m'} = \theta$, which would contradict the assumption that this is a valid solution to \probName. 
Hence, all patterns are of conjunctions of $A_i = 1$. For each treatment pattern of the form $\pattern_g = \wedge_{j=i_a}^{i_b} (A_j = 1) \land (A_p = l)$, we choose an arbitrary attribute in the conjunction $A_{j}$ if $(A_j = 1) \in \pattern_g$ and choose $S_{j}$ for the cover. Finally, if there is an uncovered element $x$ in $U$ and $R$ includes a pattern in of the form $\pattern_g = (A_j = l)$ where $l \neq 1$, we choose for the cover a set $S$ that covers $x$ arbitrarily.  
We claim that the chosen collection of $k$ sets is a cover of $U$. 
To see this, 
recall that we claimed that the coverage of $R$ is at least $n+k$. 
If the coverage includes tuples of the form $t_{S_j}$, then each pattern covers a single tuple. Suppose these patterns are $\pattern_a, \ldots, \pattern_b$. 
When building the coverage, instead of these patterns, we add a set that covers elements not yet covered by existing patterns. Thus, there are at least $b-a$ covered elements from $U$ in addition to the $n+k-(b-a)$ tuples covered by the patterns. Thus, the set cover we have assembled contains $n-(b-a) + (b-a) = n$ elements and covers all elements in $U$. 
\end{proof}
