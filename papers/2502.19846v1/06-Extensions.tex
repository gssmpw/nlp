% \section{Extensions}

% \brit{what else should go here?}

% \subsubsection{Homogeneity}

% When a recommendation is made to a subpopulation to improve an outcome, we want to ensure it impacts all individuals similarly. Specifically, our goal is to avoid significant heterogeneity in the effectiveness of the given rules. Therefore, we define the homogeneity of a prescription rule. Rules that are not deemed homogeneous will be disregarded.
% We illustrate that via the following example.


% \begin{example}
% Consider a prescription rule $r = (\pattern_g, \pattern_t)$, where the subpopulation defined by $\pattern_g$ (i.e., $\pattern_g(\db)$) can be divided into two disjoint groups, denoted as $g_1$ and $g_2$. Specifically, $g_1 \cup g_2 = \pattern_g(\db)$ and $g_1 \cap g_2 = \emptyset$. Each group consists of 100 individuals, with no confounders to account for. In $g_1$, we have:
% $$
% CATE_{\model}(\pattern_t, O \mid g_1) = \frac{100}{100} - \frac{5}{10} = 1 - 0.5 = 0.5
% $$
% This indicates that there are 100 individuals in the treatment group and only 10 in the control group, with an average treatment effect of $0.5$ within $g_1$.
% In $g_2$, we have:
% $$
% CATE_{\model}(\pattern_t, O \mid g_2) = \frac{8}{10} - \frac{100}{100} = 0.8 - 1 = -0.2
% $$
% Here, there are 10 individuals in the treatment group and 100 in the control group, leading to an average treatment effect of $-0.2$ within $g_2$.
% For the entire subpopulation defined by $\pattern_g$, we obtain:
% $$
% CATE_{\model}(\pattern_t, O \mid \pattern_g) = \frac{108}{110} - \frac{105}{110} = 0.02
% $$
% This suggests that, overall, the rule might seem effective for this subpopulation. However, it is only effective for half of the subpopulation, while for the other half, it is not effective at all. In such cases, we would prefer not to recommend this rule, as it lacks homogeneity.
% \end{example}

% A prescription rule $r = (\pattern_g, \pattern_t)$ is considered homogeneous if the CATE within every sufficiently large subgroup of the subpopulation defined by $\pattern_g$ is close to the overall CATE for the entire subpopulation. Formally,

% \begin{definition}[Homogeneity]
% A rule $r = (\pattern_g, \pattern_t)$ is homogeneous if, for every subgroup $g \subseteq \pattern_g(\db)$ that contains more than $n_0$ individuals, the difference between the CATE within the subgroup and the overall CATE remains below a certain threshold:
% \[
% \left| CATE_{\model}(\pattern_t, O \mid g) - CATE_{\model}(\pattern_t, O \mid \pattern_g) \right| < \epsilon
% \]
% For all $g \subseteq \pattern_g(\db)$ such that $|g| > n_0$, where $n_0$ is the minimum subgroup size for consideration, and $\epsilon$ is a predefined threshold that determines the acceptable deviation from the overall CATE.
% \end{definition}

% \noindent
% \textbf{Determining Homogeneity}:
% We propose a simple method for quickly assessing the homogeneity of a given rule $r = (\pattern_g, \pattern_r)$. Instead of evaluating every possible subgroup -- an approach that would be computationally impractical due to the sheer number of potential subgroups -- we randomly sample $m$ sufficiently large subgroups from $\pattern_g(\db)$. For each sampled subgroup, we compare its CATE to the overall CATE value. If the differences between the CATE values of all sampled subgroups and the overall CATE are below the predefined threshold $\epsilon$, the rule is considered homogeneous; otherwise, it is not.

