\section{Conclusion and Limitations}

In this work, we presented \our, a hierarchical learning framework designed to enable humanoid robots to engage in real-time humanoid-human-object interactions. By decoupling the interaction process into high-level planning and low-level reactive control, \our allows humanoid robots to quickly adapt to the change of human intentions and interrupt ongoing tasks without delays. This framework incorporates a wide variety of skills, from object manipulation to expressive motion generation. We implemented \our on a real humanoid robot and demonstrated its effectiveness, flexibility, and safety in various dynamic environments.

The proposed framework provides a significant step toward making humanoid robots autonomous and responsive in real-world applications, such as assisting with daily life tasks, disaster response, and industrial automation. By allowing continuous humanoid-human-object interaction, \our provide immediate and adaptive responses, making humanoid robots suited for seamless integration into human environments.

Despite promising results, several limitations remain. First, while \our is designed to be scalable, the current implementation is constrained by the availability of high-quality training data. The generalization of the system across a broader range of tasks and environments is still a challenge, as it heavily relies on human demonstrations and teleoperation data, which is time-consuming to collect. Future work will focus on utilizing existing datasets and simulation environments to improve the scalability and generalization of the framework.
Additionally, the current implementation of \our is limited to the upper body at a fixed workspace, but a humanoid assistant should have locomotion and navigation abilities in a dynamic environment, and react with whole-body behaviors. Future work should integrate a whole-body controller to extend the framework to whole-body interaction for humanoid robots, and more general tasks with varying levels of human intervention.