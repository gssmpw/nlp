\section{Related Work}
\subsection{Data Collection from Human Demonstrations}
A substantial amount of task-specific data is essential for imitation learning. In dexterous manipulation, obtaining high-quality hand motion data is critical for training effective policies.
%
Prior work includes extracting demonstration data from human videos~\cite{sivakumar2022robotic,xu2023xskill,xu2024flow,bharadhwaj2024towards} and hand trajectories~\cite{wang2023mimicplay,yang2022learning}. While these approaches are accessible and have shown promising results, the significant visual gap between recorded human demonstrations and the robot’s perception often makes real-world transfer challenging.
%
An alternative is using dedicated hardware for data collection to bridge this gap. Hand-held grippers~\cite{song2020grasping,chi2024universal,sanches2023scalable} have proven effective in capturing robot manipulation data. However, these systems are primarily designed for parallel grippers.
%
Another widely used approach is MoCap systems, which record human demonstrations and extract hand motion data. These systems include camera-based methods~\cite{qin2023anyteleop,zimmermann2019freihand}, glove-based tracking systems~\cite{wang2024dexcap,liu2017glove,liu2019high}, marker-based tracking~\cite{zhao2012combining}, and commercial MoCap solutions~\cite{taheri2020grab,fan2023arctic}. While MoCap offers high-precision tracking, bridging the embodiment gap between human and robotic hands remains a persistent challenge.

\subsection{Dexterous Hand Teleoperation} 
Collecting high-quality human demonstrations through robotic teleoperation systems~\cite{fang2024airexo,cheng2024open,ding2024bunny,iyer2024open} also plays a critical role for advancing dexterous manipulation.
Existing research has explored teleoperation from various perspectives, including leader-follower setups such as ALOHA~\cite{zhao2023learning,zhao2024aloha,fu2024mobile,aldaco2024aloha}.
%
However, teleoperating dexterous hands remains a significant challenge.
OpenTelevision~\cite{cheng2024open} leverages VR devices to capture hand poses and streams the pose information for retargeting to robotic hands. BiDex~\cite{shawbimanual}, on the other hand, implements a teleoperation system based on commercial motion capture gloves~\cite{manusmeta} and leader arms. 
Compared to these frameworks and other glove-based systems~\cite{liu2017glove,liu2019high}, \oursystem offers distinct advantages. It eliminates the need for expensive equipment while precisely capturing fingertip positions and delivering richer haptic force feedback to the operator. This system achieves accurate dexterous hand teleoperation with a low-cost setup, making it an efficient alternative.

\subsection{Teleoperation with Haptic Force Feedback}
While recent studies rely on visual information to capture environmental characteristics, vision alone inherently limits the richness of available sensory data. 
In contrast, haptic force feedback enhances the teleoperation experience by providing greater immersion and improving perception of the robot's status and movement compared to vision-based methods. 
Bunny-VisionPro~\cite{ding2024bunny} and Liu et al.~\cite{liu2019high} apply real-time haptic feedback to enable more accurate manipulation. 
Xu et al.~\cite{xu2025immersive} build a bilateral isomorphic bimanual telerobotic system using a commercial force feedback glove~\cite{dextarobotics} to enhance perception and improve performance in complex tasks.
% Patel et al. ~\cite{patel2022haptic} apply haptic feedback in surgical robots for more precise manipulation. 
NimbRo-Avatar~\cite{schwarz2021nimbro} and Mosbach et al.~\cite{mosbach2022accelerating} integrate commercial force feedback glove~\cite{senseglove} into dexterous teleoperation systems.
However, these approaches rely on specialized or expensive equipment. In contrast, \oursystem provides a highly accurate teleoperation system with integrated haptic force feedback at a significantly lower cost and can be widely used in dexterous manipulation.
