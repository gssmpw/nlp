\begin{figure*}[t]
  \centering
  \includegraphics[width=\textwidth]{figures/demo.pdf}
  \caption{
    \textbf{Teleoperation demos.}
    a) While squeezing condensed milk, the operator regulates the flow using haptic force feedback from \oursystem.
    b) The operator grasps a slipping bottle without visual feedback.
    c) The user identifies object pairs solely through haptic force feedback.
  }
  \label{fig:demo}
  \vspace{-5mm}
\end{figure*}
\section{Introduction}

Imitation learning~(IL) has shown significant promise in addressing complex manipulation tasks~\cite{chi2023diffusion,chi2024universal,zhao2023learning,Ze2024DP3}. However, it often necessitates a substantial amount of task-specific data to train a generalizable learning policy. Efficiently collecting and ensuring the high quality of such demonstrations remains a persistent and challenging problem for the robotic community.

Teleoperation is among the most commonly used methods for collecting demonstrations, often involving the development of a wide range of devices tailored to meet diverse data acquisition requirements. These devices enable the transfer of human manipulation behaviors to various robotic platforms~\cite{fu2024mobile, fang2024airexo,cheng2024open,ding2024bunny,iyer2024open}.
%
However, when it comes to dexterous hands, their high degrees of freedom~(DoFs) and inherent complexity impose even stricter demands on operational precision and the accuracy of human motion capture. Hence, it is crucial to design an intuitive, responsive, and highly precise device specifically suited for dexterous hand teleoperation applications.

Vision-based methods are primarily used for tracking the human hand in dexterous hand teleoperation. A simple approach involves using RGB cameras~\cite{qin2023anyteleop,ultraleap}, but the accuracy of hand gesture capture is often questioned and may be further limited by visual obstacles during hand-object interactions. Motion capture~(MoCap) systems~\cite{OptiTrack,Vicon,santaera2015low,wang2024dexcap} provide stable hand tracking. However, relying solely on visual feedback for teleoperation makes intuitive control challenging for the human operator.

Haptic force feedback offers additional perceptual characteristics beyond those provided by vision, such as the ability to sense an object's weight, friction, and softness. Integrating haptic feedback into teleoperation can enrich the feedback available during interaction and enable the completion of more challenging tasks. Recently, some commercialized force feedback gloves~\cite{dextarobotics, senseglove, manusmeta, haptx} have shown promise for enabling intuitive teleoperation. However, these solutions are often prohibitively expensive and require significant integration efforts to work with existing robot learning frameworks.

In this paper, we introduce \textbf{\oursystem}, a low-cost, fully open-sourced, and easy-to-manufacture haptic force feedback glove for dexterous manipulation. The glove can be assembled in hours for a total cost of 600 USD. Key features of \oursystem include:

\textbf{21-DoF motion capture:} 
\oursystem features an anthropomorphic design resembling the human hand, providing precise motion capture and a comfortable wearing experience. Moreover, we propose a customized joint structure that integrates a compact, low-cost yet accurate joint encoder, with the entire assembly measuring less than 15~mm in thickness.

\textbf{5-DoF haptic force feedback:} 
\oursystem leverages a cable-driven mechanism to deliver force feedback to each finger while maintaining a compact and cost-effective design. Additionally, each fingertip is also equipped with a linear resonant actuator~(LRA) to provide realistic haptic feedback. This integration of force and haptic feedback creates an immersive and responsive interface for dexterous manipulation.

\textbf{Action and haptic force retargeting:} We propose a general retargeting framework. For action retargeting, the rigid constraints of the glove allow fingertip positions to be mapped from the human hand to the target robotic hand. For haptic force retargeting, the combination strategy enables users to perceive contact information during teleoperation.

The resulting system, \oursystem, provides precise hand pose motion capture and the ability to sense interactions with manipulated objects. This enables human operators to intuitively and efficiently teleoperate dexterous hands. As shown in Fig.~\ref{fig:demo}, it further supports the completion of complex manipulation tasks.
%
We evaluate the necessity of haptic force feedback through a user study and further assess the teleoperation efficiency and data accuracy of \oursystem in several quantitative experiments.

Finally, we demonstrate that \oursystem seamlessly integrates with existing methods in robot learning. We use \oursystem to teleoperate the LEAP Hand mounted on a Franka robot arm, collecting data to train imitation learning policies. To foster further research, we will \textbf{open-source} the mechanical designs, circuit designs, embedded code, assembly instructions, URDF models, retargeting methods, and MuJoCo simulation environment at \textcolor{cyan}{\url{https://do-glove.github.io/}}.