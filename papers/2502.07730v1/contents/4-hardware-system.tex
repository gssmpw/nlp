\section{Hardware System}

\subsection{Kinematic Design}

The kinematic design of \oursystem refers to the use of constraints to achieve desired movements, emulating the natural motion of a human hand. 
%
To ensure precise MoCap capabilities and a comfortable wearing experience, \oursystem are designed to closely resemble the anthropomorphic structure of the human hand.

Several studies~\cite{cerulo2017teleoperation, cerveri2007finger} model the hand skeleton as a kinematic chain, represented by a hierarchical structure of rigidly connected joints. As shown in Figure~\ref{fig:glove_explosion}, the kinematic structure of the human hand primarily consists of two types of joints: hinge joints and ball joints.

In the index, middle, ring, and pinky finger, distal interphalangeal~(DIP) and proximal interphalangeal~(PIP) joints are hinge joints with 1-DoF, allowing only flexion-extension movements. In contrast, the metacarpophalangeal~(MCP) joint is a ball joint with 2-DoF, allowing both flexion-extension and adduction-abduction movements. 

The thumb differs slightly in its structure. The interphalangeal~(IP) and metaphalangeal~(MCP) joints are hinge joints with 1 DoF, while the additional trapeziometacarpal~(TM) joint is a ball joint that supports both flexion-extension and adduction-abduction movements. To further enhance dexterity, an additional DoF at the wrist allows the thumb to perform pronation-supination movements.

To implement these joints in \oursystem, hinge joints are modeled as two linkages connected by a rotary joint, with a joint encoder installed on the rotary axis to capture motion. Ball joints are designed as a combination of two orthogonal rotary joints, each equipped with a joint encoder on its respective rotary axis.

\begin{figure}[ht]
  \centering
  \vspace{2mm}
  \includegraphics[width=0.92\linewidth]{figures/improper_linkage.pdf}
  \caption{
  \textbf{Improper linkage lengths can cause collisions} between the human finger (\textbf{link $\boldsymbol{f, g}$}) and the glove (\textbf{link $\boldsymbol{m}$}), restricting finger movements and leading to discomfort and poor MoCap performance.
  }
  \label{fig:improper_linkage}
  \vspace{-3mm}
\end{figure}

The linkage lengths in \oursystem are designed to accommodate the majority of adult human sizes. To achieve this, a standard human finger length was first modeled, and the glove's linkage parameters were simulated to ensure an optimal range of motion. As shown in Figure~\ref{fig:improper_linkage}, improper linkage lengths can obstruct the natural flexion-extension of the fingers, leading to discomfort and reduced MoCap performance. Furthermore, \oursystem features a modular design where all fingers share a common structural framework. This modularity enables users to replace linkages with customized sizes as needed, enhancing both adaptability and usability.


\subsection{Finger Design}

\begin{figure*}[t]
  \centering
  \includegraphics[width=0.95\textwidth]{figures/joint_explosion.pdf}
  \caption{
  \textbf{Exploded view of the finger assembly}, with the highlighted area indicating the basic components of a rotary joint.
  }
  \label{fig:joint_explosion}
  \vspace{-3mm}
\end{figure*}

As shown in the exploded-view in Figure~\ref{fig:glove_explosion}, \oursystem is composed of the thumb, index, middle, ring, and pinky finger assemblies, along with the palm base structure. The design of each finger assembly follows a modular approach, ensuring consistent structural elements across all fingers.

The exploded view of a single finger is illustrated in Figure~\ref{fig:joint_explosion}. The highlighted area indicates the basic components of a rotary joint. Each rotary joint is constructed using an M4×15 shoulder screw to connect the finger linkages, ball bearing, and joint encoder, secured with an M3 locknut. This design ensures smooth and reliable joint rotation.
The main body of the finger, colored white and gold, is 3D printed using PETG material for ease of fabrication and durability.

Given the limited space on the back of the human hand, the finger assembly's width is constrained to less than 26 mm. Simultaneously, to provide effective force feedback, the actuator must deliver a stall torque of at least 0.5 N·m. Additionally, adjustable stiffness requires the actuator's current to be regulated. Since the actuator is directly connected to the pulley system as a rotary joint for $MCP_B$, it is essential to measure its rotary position in real time to achieve precise joint angle control.
Considering these design requirements, the \texttt{Dynamixel XC/XL330} servo motors were selected as the actuators for force feedback. It fulfills the torque, size, and real-time position measurement needs, making it a suitable choice for \oursystem.

\vspace{1mm}
\subsubsection{\textbf{Joint Encoders}}
\leavevmode

To integrate joint encoders into the finger linkages, the encoders must be compact while maintaining high precision. Additionally, as 16 encoders are required in combination with 5 servo motors to achieve 21-DoF MoCap capabilities, the cost of each encoder needs to be affordable. Considering these constraints, we selected the \texttt{Alps RDC506018A} rotary sensor as the joint encoder. 
%
This compact encoder (W11~mm $\times$ L14.9~mm $\times$ H2.2~mm) is easily integrated into the 3D-printed joint structures. The encoder operates as a variable resistor, changing its resistance as the shaft rotates. 

The resistance changes are converted into voltage signals using a simple voltage divider circuit. Due to the encoder's linear response, the voltage output is proportional to the actual joint angle. These voltage signals are read by an Analog-Digital Converter (ADC) module. For precise conversion, we use the \texttt{TI ADS1256}, a low-noise 24-bit ADC operating at 30k samples per second. The converted signals are then sent to a microcontroller unit~(MCU), the \texttt{ST Electronics STM32F042K6T6}, which operates at a clock speed of 48 MHz. To optimize system performance and reduce OS scheduling overhead, the STM32's Direct Memory Access~(DMA) feature is utilized to accelerate joint encoder readings. Finally, the processed joint data is transmitted to the host machine via a serial port on the STM32.

The voltage readings are mapped directly to joint angles under the assumption that the supply voltage of the STM32~(approximately 3.3~V) corresponds to 360°, while the ground voltage~(0~V) corresponds to 0°. Using the ADC output voltage, the joint angle is calculated as:
\begin{equation}
    \alpha_{\text{joint}} = \frac{\text{V}_{\text{ADC}}}{\text{V}_{\text{CC}}} \cdot 360
    \label{eq:joint_angle}
\end{equation}

The primary error in this conversion comes from the linearity error of the encoder which is ±2\% according to its datasheet. This results in an angular error of ±7.2° when measuring joint angles. To mitigate this, we employ a calibration process. Using an external high-precision joint encoder, we map the voltage reading to an actual joint angle, creating a correction table for each encoder. With this calibration, the error can be reduced to within ±1°.

\vspace{2mm}
\subsubsection{\textbf{Cable-Driven Force Feedback Structure}}
\leavevmode

To provide force feedback on the human fingers, the output torque of the Dynamixel servo must be transmitted to the glove's finger linkage system. As illustrated in Figure~\ref{fig:joint_explosion}, the rotary axis of the servo and the rotary axis of the $MCP_B$ joint are misaligned. Consequently, a transmission mechanism is required to transfer the torque effectively.

Although a bevel gear system could serve as a potential solution, its implementation would require significant space to accommodate the two orthogonal gears. Additionally, transmitting large torque through gears can cause deformation in the gear shaft, leading to gear slippage. In contrast, a cable-driven mechanism offers a more compact design while ensuring stable torque transmission.

Traditional cable-driven systems typically provide unidirectional force transmission on the tension side, relying on a spring to generate force in the opposite direction. However, this approach introduces unrealistic feedback sensations. While using two servos per finger could resolve this issue, it would significantly increase the glove's weight and cost. 

\begin{figure}[ht]
  \centering
  \includegraphics[width=0.85\linewidth]{figures/pulley_system.pdf}
  \caption{
  \textbf{Pulley system} of the cable-driven mechanism.
  }
  \label{fig:pulley_system}
  \vspace{-3mm}
\end{figure}

To address these challenges, \oursystem utilizes a pulley system to provide the bi-directional force feedback, as shown in Figure~\ref{fig:pulley_system}.
%
\oursystem uses a 0.6~mm stainless steel braided wire as the cable, chosen for its strength and durability. The \textit{Servo Pulley} connects the servo to the finger linkage~(\textit{Finger Middle}) via the \textit{Finger Pulley}, maintaining a 1:1 transmission ratio. To minimize friction during transmission, the \textit{Fixed Pulley} is used to redirect the cable’s path.
%
When the \textit{Servo Pulley} rotates clockwise, the tension on \textit{Cable B} increases, causing \textit{Finger Pulley} to rotate clockwise. The extra slack on the \textit{Cable A} side is taken up by the \textit{Servo Pulley A}. Since the finger linkage is fixed to the \textit{Finger Pulley}, it also rotates clockwise, resulting in the extension movement of the $MCP_B$ joint. Similarly, when the \textit{Servo Pulley} rotates counterclockwise, the tension shifts to \textit{Cable A}, producing a flexion movement of the $MCP_B$ joint.

This configuration enables bi-directional torque transmission while maintaining a simple, compact, and cost-effective design.

\vspace{2mm}
\subsubsection{\textbf{Fingertip Haptic Feedback}}
\leavevmode

To further enhance the operator’s tactile experience, each fingertip in \oursystem is equipped with a tactile actuator. 

Traditional haptic actuators include eccentric rotating mass~(ERM) motors and linear resonate actuators~(LRAs). Limited by the inertia of the rotating mass, ERM motors are slow to start and stop, making it challenging to produce complex waveforms needed for subtle tactile sensations. On the contrary, LRAs offer linear motion, resulting in a cleaner and more precise tactile output.

In \oursystem, we use LRAs with a diameter of 8~mm and a height of 2.5~mm, installed close to the fingertips. These LRAs provide vibration stimuli by resonating at approximately 240~Hz along Z axis, which is orthogonal to the fingertip surface. Operating at 1.2~$V_\mathrm{rms}$, the LRAs generate high-quality haptic waveforms. To fully leverage the potential of the LRA, we employ the \texttt{TI DRV2605L} motor driver, which includes the licensed \texttt{Immersion TouchSense\textsuperscript{\textcopyright} 2200} haptic library. This driver supports over 100 pre-programmed waveforms, allowing \oursystem to deliver realistic and refined haptic feedback.

\subsection{Wrist Localization}

In \oursystem, we design a shell with a 1/4 inch screw connector to accommodate external wrist localization devices. For our experiments, we use the \texttt{HTC Vive Tracker} for real-time wrist position tracking. However, the design is compatible with other solutions, depending on the user's requirement.
