\section{Retargeting}

\begin{figure*}[t]
  \centering
  \includegraphics[width=\textwidth]{figures/retargeting.pdf}
  \caption{
  \textbf{Action retargeting results}: Teleoperating the LEAP Hand to grasp a toy in the real world and teleoperating the Shadow Hand, Inspire Hand, and Allegro Hand in simulation.
  }
  \label{fig:retargeting}
  \vspace{-5mm}
\end{figure*}

\subsection{Action Retargeting}

To map human hand gestures to a robotic hand, it is essential to perform action retargeting, which converts motion data from the glove into robotic hand movements. This process addresses both the embodiment gap and motion discrepancies. 
%
Previous studies~\cite{wang2024dexcap, fan2023arctic, taheri2020grab} highlight the significance of fingertips, as they are the primary contact area during object interactions. Building on this insight, we apply the 5-DoF haptic force feedback to the human operators' fingertips and adopt a retargeting method focused on fingertip positions.

Our approach combines \texttt{Forward Kinematics~(FK)} to compute human fingertip positions and \texttt{Inverse Kinematics~(IK)} to calculate the corresponding robotic hand positions. 
%
When wearing \oursystem, the human operator secures their fingertips inside the finger caps. Since the glove acts as a rigid body, the relative positions of the fingertips with respect to the glove’s origin can be accurately calculated. With \oursystem's anthropomorphic kinematic design and precise MoCap capabilities, fingertip positions are effortlessly determined using the glove’s built-in FK.
%
To map these positions to a robotic hand, we utilize Mink~\cite{Zakka_Mink_Python_inverse_2024}, a differential inverse kinematics library, to generate smooth and feasible motions for the robotic hand. 

A size discrepancy often exists between the human hand and the target robotic hand. To address this, we introduce a scaling factor when calculating IK, allowing adaptation to different robotic hand sizes. This ensures an intuitive teleoperation experience where the robotic hand naturally mirrors the human hand's gestures. For instance, when the human operator opens their hand, the robotic hand open proportionally. Similarly, when the human operator brings their thumb and index finger together, the robotic hand's thumb and index finger also touch. This ability for precise fingertip alignment is critical for tasks like grasping small objects.

In our experiments, we deploy the system on the LEAP hand~\cite{shaw2023leaphand} in real-world scenarios and test it with various robotic hands in the MuJoCo simulator. Figure~\ref{fig:retargeting} presents the retargeting results of \oursystem in both simulation and real-world environments.
 
\subsection{Haptic Force Retargeting}
\label{sec::haptic_force_retargeting}
To provide the haptic force feedback, it is first necessary to sense tactile or force information at the robotic hand's fingertips. This can be achieved using a simple tactile sensor, such as the force sensing resistor~(FSR) sensor~\cite{ding2024bunny, kappassov2015tactile}, or for better performance, by utilizing an F/T sensor~\cite{kim2020six, chen2025dexforce} or a vision-based tactile sensor~\cite{yuan2017gelsight, lin20239dtact}.

In our experimental setup, we install a \texttt{1-D force sensor} on each fingertip of the LEAP Hand, with a measurement range of 3~kg and a precision of 1~g. During our quantitative experiments (Section~\ref{sec:experiments}), we identify a combination strategy for integrating haptic and force feedback that optimizes performance. This strategy along with the corresponding thresholds and feedback patterns is summarized in Table~\ref{tab:retargeting_strategy}.

\begin{table}[h]
\centering
\resizebox{0.95\linewidth}{!}{
\begin{tabular}{lcc}
\toprule
\textbf{Force Sensor Readings (g)} & \textbf{Haptic Feedback} & \textbf{Force Feedback} \\ \midrule
\textless{}10                      & \cross                  & \cross                  \\
10--50                             & \tick                   & \cross                  \\
50--100                            & \tick                   & \tick                   \\
\textgreater{}100                  & \cross                  & \tick                   \\
\bottomrule
\end{tabular}
}
\caption{\textbf{The combination strategy} for haptic force feedback in \oursystem.}
\vspace{-3mm}
\label{tab:retargeting_strategy}
\end{table}

When the robotic hand touches an object, the force sensor readings increase. To effectively distinguish these signal from noise, we set the first threshold at 10~g to initiate haptic feedback. During a user study without visual feedback (Section~\ref{sec::user_study}), we observe that human operators are highly sensitive to force feedback. To create a more realistic experience, force feedback is applied only after the force sensor readings exceed 50~g, which serves as the second threshold. Furthermore, observations from the bottle-slipping experiment (Section~\ref{sec::bottle-slipping}) reveal that continuous haptic feedback during teleoperation can create a misleading sensation, interfering with the operator's ability to perceive the subtle properties of the object's surface. To address this, a third threshold is set at 100~g, where haptic feedback stops, leaving only force feedback active.

For force feedback, the Dynamixel servos operate in \texttt{current-based position control mode}. The force readings from the LEAP Hand fingertips are clamped to the range [0g, 3000g], and mapped linearly to the $K_P$ gain of the Dynamixel servos. For haptic feedback, we use \texttt{waveform ID~56} from the haptic engine library, corresponding to \texttt{Pulsing Sharp 1-100\%}.

This combination strategy for haptic force retargeting enables human operators to distinguish object shape, size and softness without visual feedback. It also improves performance in complex, contact-rich manipulation tasks. Further details are provided in Section~\ref{sec:experiments}.