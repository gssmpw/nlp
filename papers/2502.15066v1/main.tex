\documentclass[draft]{agujournal2019}
\usepackage{url} %this package should fix any errors with URLs in refs.
%\usepackage{lineno}
\usepackage[inline]{trackchanges} %for better track changes. finalnew option will compile document with changes incorporated.
\usepackage{soul}
%\linenumbers

\usepackage{amsmath}
\usepackage{natbib}
\usepackage{lmodern}
\usepackage{gensymb}
%\usepackage{hyperref}

\draftfalse

% Definitions for the journal names
\newcommand{\adv}{{\it Adv. Space Res.}} 
\newcommand{\annG}{{\it Ann. Geophys.}} 
\newcommand{\aap}{{\it Astronomy \& Astrophysics}}
\newcommand{\aaps}{{\it Astron. Astrophys. Suppl.}}
\newcommand{\aapr}{{\it Astron. Astrophys. Rev.}}
\newcommand{\araa}{{\it Annual Review of Astronomy and Astrophysics}}
\newcommand{\ag}{{\it Ann. Geophys.}}
\newcommand{\aj}{{\it Astron. J.}} 
\newcommand{\apj}{{\it The Astrophysical Journal}}
\newcommand{\apjl}{{\it The Astrophysical Journal Letters}}
\newcommand{\apjs}{{\it The Astrophysical Journal Supplement Series}}
\newcommand{\apss}{{\it Astrophys. Space Sci.}} 
\newcommand{\cjaa}{{\it Chin. J. Astron. Astrophys.}} 
\newcommand{\gafd}{{\it Geophys. Astrophys. Fluid Dyn.}}
\newcommand{\grl}{{\it Geophys. Res. Lett.}}
\newcommand{\ijga}{{\it Int. J. Geomagn. Aeron.}}
\newcommand{\jastp}{{\it J. Atmos. Solar-Terr. Phys.}} 
\newcommand{\jgr}{{\it J. Geophys. Res.}}
\newcommand{\mnras}{{\it Monthly Notices of the Royal Astronomical Society}}
\newcommand{\nat}{{\it Nature}}
\newcommand{\pasp}{{\it Publ. Astron. Soc. Pac.}}
\newcommand{\pasj}{{\it Publ. Astron. Soc. Japan}}
\newcommand{\pre}{{\it Phys. Rev. E}}
\newcommand{\solphys}{{\it Solar Physics}}
\newcommand{\sovast}{{\it Soviet  Astron.}} 
\newcommand{\ssr}{{\it Space Sci. Rev.}} 
\newcommand{\jasge}{{\it NRIAG Journal of Astronomy and Geophysics}}
\newcommand{\ejrss}{{\it The Egyptian Journal of Remote Sensing and Space Science}}
\chardef\us=`\_

\journalname{arXiv}

\begin{document}

\title{An Interpretable Machine Learning Approach to Understanding the Relationships between Solar Flares and Source Active Regions}

\authors{Huseyin Cavus\affil{1}, 
Jason T. L. Wang\affil{2}, 
Teja P. S. Singampalli\affil{2},
Gani Caglar Coban\affil{1},  
Hongyang Zhang\affil{2},
Abd-ur Raheem\affil{1},
Haimin Wang\affil{3}
}

\affiliation{1}{Department of Physics, Canakkale Onsekiz Mart University, 17110 Canakkale, Turkey}
\affiliation{2}{College of Computing, New Jersey Institute of Technology,  
Newark, NJ 07102, USA}
\affiliation{3}{Institute for Space Weather Sciences, New Jersey Institute of Technology, 
Newark, NJ 07102, USA}

\begin{keypoints}

\item Implementing the random forest algorithm as an interpretable machine learning approach to enable the binary classification of solar flares using the physical features of ARs collected from SolarMonitor.org
\item Today's value of Mount Wilson AR type and yesterday's value of Hale class are the most and the least governing features respectively
\item The difference in the number of spots in the AR comparing with the previous day,
i.e., NoS\_Difference,
has a distinctive effect for decision-making in both global and local interpretations
\end{keypoints}

\begin{abstract}
Solar flares are defined as outbursts on the surface of the Sun. They occur when energy accumulated in magnetic fields enclosing solar active regions (ARs) is abruptly expelled. Solar flares and associated coronal mass ejections are sources of space weather that adversely impact devices at or near Earth, including the obstruction of high-frequency radio waves utilized for communication and the deterioration of power grid operations. \\
Tracking and delivering early and precise predictions of solar flares is essential for readiness and catastrophe risk mitigation. This paper employs the random forest (RF) model to address the binary classification task, analyzing the links between solar flares and their originating ARs with observational data gathered from 2011 to 2021 by SolarMonitor.org and the XRT flare database. We seek to identify the physical features of a source AR that significantly influence its potential to trigger $\geq$C-class flares.
We found that the features of AR\_Type\_Today,  Hale\_Class\_Yesterday are the most and the least prepotent features, respectively. NoS\_Difference has a remarkable effect in decision-making in both global and local interpretations.
\end{abstract}

\section*{Plain Language Summary}
Solar flares arise from the emission of energy accumulated in the magnetic fields of solar ARs. However, the underlying mechanism for these outbursts remains unidentified. This study employs an interpretable machine learning approach to establish a solar flare prediction model, enabling the binary classification of flares ($\geq$C as the positive class and $<$C as the negative class) using the physical features of ARs obtained from SolarMonitor.org and flare data supplied from the XRT flare catalog with observational data from 2011 to 2021. 
Using the Random Forest (RF) algorithm, the performance metric values are acquired as recall of 0.81, precision of 0.82, accuracy of 0.74 and F1 score of 0.82.  
Our findings indicate that, the AR\_Type\_Today ($\beta$, $\gamma$ and $\gamma$$\delta$ types are effective in more than 90\% of positive classifications) is the most influential feature and the \\
Hale\_Class\_Yesterday ($\beta$, $\beta$$\gamma$ and $\beta$$\gamma$$\delta$ classes are dominant in over 85\% cases of positive classes) is the least influential factor for an AR generating a $\geq$C-class flare. 
The \\
NoS\_Difference feature, which takes values between $-13$ and $10$ in more than 90\% of positive test samples, significantly influences the decision-making process in both global and local interpretations.

\section{Introduction}
Active regions (ARs) on the solar disk are areas where the Sun's magnetic field is altered. They are often associated with sunspots and are the origin of violent eruptions such as coronal mass ejections (CMEs) and solar flares \citep{1995A&A...304..585H}.
The presence of sunspots indicates ARs. Ultraviolet and X-ray images of the Sun reveal these regions to be luminous because of the extraordinarily energetic events associated with ARs
\citep{1963ARA&A...1...59F}. 
These regions are surrounded by dramatic structures, such as coronal loops and solar prominences.

Different solar activities occur according to the open or closed magnetic field lines formed by the sunspot groups. Although open magnetic field lines cause the solar wind, sudden changes in magnetic field lines and reconnections of closed magnetic field lines cause solar eruptions such as CMEs and flares \citep{Priest_2014}. These eruptive events can initiate other physical activities, such as interplanetary shock waves
and geomagnetic storms, which affect the Earth's atmosphere, disrupt short-wave communications and adversely affect satellite and space operations
\citep{2022Univ....8...39M,2015ApJ...814...59V}.

Flares emerge in active regions \citep{2005ApJ...629.1141A},
especially near sunspots, where powerful magnetic fields penetrate the photosphere and connect the corona to the Sun. Flares are propelled by the immediate (minutes to tens of minutes) discharge of magnetic energy from the corona. 
\citet{2003SoPh..218..261A} reported that all the X-class flares and 55\% of the M-class flares were associated with 311 LASCO-observed CMEs between 1996 and 1999. In a more extensive mathematical investigation, \citet{2005JGRA..11012S05Y}
indicated that the association rate of CMEs significantly escalated with the magnitude of X-ray flares, 
rising from 20\% for C-class flares to 100\% for very large flares. They showed that all CMEs linked to X-class flares were observed by the LASCO coronagraphs, whereas 25-67\% of CMEs linked to C-class flares remained undetected. 
\citet{2012JAsGe...1..172Y} worked with 776 CME-flare-matched activities. They found that 67\% of the CMEs occurred after flare events. These findings indicate that although flares and CMEs
often occur together, there is no one-to-one correspondence
between them. Moreover, flare durations do not guarantee the association of CMEs \citep{1995A&A...304..585H}. 

\citet{Falconer_2012} provide a dual rationale for the increased flare productivity of active regions in their investigation of past flaring as a supplementary indicator of free magnetic energy to predict solar eruptions. Firstly, these ARs typically exhibit a complex multipolar configuration of opposite-polarity magnetic flux instead of a singular bipole arrangement; secondly, they are undergoing rapid evolution through convective flux transfer, flux emergence, and/or flux cancellation. Flares emit energy in a variety of ways, including electromagnetic radiation, particles (protons and electrons), and mass streams \citep{2012annG...55...49B}. They are distinguished by their X-ray brilliance (X-ray flux). X-class flares are the largest. M-class flares have one-tenth the energy of
X-class flares. C-class flares have one-tenth the energy of
M-class flares \citep{1989ARA&A..27..421B}.
\citet{2012ApJ...760...31K} and \citet{2016ApJ...820L..11J}
worked on magnetic classes of ARs for the period between 1992 and 2015. These authors found that $\alpha$ and $\beta$ containing ARs make up 20\% and 80\%, respectively. 

In their study on the development of the determined magnetic complexity of ARs for the 23rd solar cycle, \citet{10.1111/j.1365-2966.2010.16465.x} stated that the most strong flares and rapid coronal mass ejections (CMEs) typically originate in active regions (ARs) having intricate structures. \citet{MOHAMED2018249} 
performed a statistical analysis of the frequency of X-class flares during the declining and peak phases of Solar Cycles 23 and 24, respectively. They found a consistency between 
the number of X-class flares and the number of days of $\beta$$\gamma$$\delta$ group ARs.

In this study, we adopt an
interpretable machine learning
(ML) approach to examine the relationships between solar flares
and their source ARs using observational data collected from
2011 to 2021. By adopting the interpretable ML approach, we attempt to explain how the ML model makes decisions in determining the relationships
between flares and their source ARs.
This helps us to understand which physical properties of a source AR might play an important role in producing a flare.

The remainder of this paper is organized as follows.
Section \ref{sec:data} describes the data used in this study.
Section \ref{sec:methods} formalizes the problem studied here into a binary classification task and details our approach 
to solving the task. Section \ref{sec:results} reports the experimental results. Section \ref{sec:discussion} presents a discussion of the results. Section \ref{sec:conclusions} concludes the paper.

\section{Data}
\label{sec:data} 

The database at SolarMonitor.org was utilized to analyze
ARs and their magnetic group properties \citep{2002SoPh..209..171G}. 
It aggregates solar activity data from various sources, including the Solar Dynamics Observatory (SDO) with HMI and AIA instruments, the Global H$\alpha$ Network, 
the Solar and Heliospheric Observatory
(SOHO) with EIT and MDI instruments, 
Synoptic Optical Long-term Investigations of the Sun (SOLIS)
with full-disk chromospheric magnetograms, 
SECCHI, and databases from the National Oceanic and Atmospheric Administration (NOAA). 
It also features images from the Hinode XRT team, 
STEREO's Extreme UltraViolet Imager (EUVI),
and the Solar X-ray Imager (SXI),
as well as X-ray images from the X-ray SXT full-disk database. 

The SolarMonitor system provides AR data, 
flare forecasting information, 
and comprehensive solar disk images of the Sun. 
Its database offers near-real-time data on solar activity.
Specifically, the database includes the following physical properties
for an AR with NOAA number, the latest position, and
associated flares if available:
Hale class, AR type, sunspot area (millionths of the solar disk area), and number of spots.

Hale class has 8 values:
$\alpha$,
$\beta$,
$\alpha$$\gamma$,
$\alpha$$\delta$,
$\alpha$$\gamma$$\delta$,
$\beta$$\gamma$,
$\beta$$\delta$,
$\beta$$\gamma$$\delta$.
The Hale classes are then regrouped into five groups to obtain the Mount Wilson classification (AR type) with five values:
$\alpha$,
$\beta$,
$\gamma$,
$\delta$,
$\gamma$$\delta$
where
\begin{itemize}
\item 
$\alpha$ refers to Hale class $\alpha$;
\item 
$\beta$ refers to Hale class $\beta$; 
\item 
$\gamma$ refers to Hale classes $\alpha$$\gamma$ and $\beta$$\gamma$;
\item 
$\delta$ refers to Hale classes $\alpha$$\delta$ and $\beta$$\delta$;
\item 
$\gamma$$\delta$ refers to Hale classes $\alpha$$\gamma$$\delta$ and $\beta$$\gamma$$\delta$.
\end{itemize}

In summary, our study used 10 physical features, 
listed in Table \ref{tab:10features}, obtained from the SolarMonitor database.
Each AR record contains values of the 10 features
of the corresponding AR.

\begin{table}[h]
    \centering
    \begin{tabular}{|c|c|}
        \hline
        \textbf{Feature} & \textbf{Description} \\ \hline
        Hale\_Class\_Today & Hale class of the AR in today \\ \hline
        Hale\_Class\_Yesterday & Hale class of the AR 
        in yesterday\\ \hline
        AR\_Type\_Today & Mount Wilson classification of 
        the AR in today\\ \hline
        AR\_Type\_Yesterday & Mount Wilson classification of the AR in yesterday \\ \hline
        Spot\_Area\_Today & Area of spots of the AR in today \\ \hline
        Spot\_Area\_Yesterday & Area of spots of the AR in yesterday  \\ \hline
        Spot\_Area\_Difference  &  Spot\_Area\_Yesterday $-$
        Spot\_Area\_Today\\ \hline
        NoS\_Today & Number of spots of the AR in today \\ \hline
        NoS\_Yesterday & Number of spots of the AR in yesterday \\ \hline
        NoS\_Difference &  NoS\_Yesterday
        $-$ NoS\_Today \\ \hline 
    \end{tabular}
    \vspace*{+0.5cm}
    \caption{Physical Properties or Features of an AR Considered in Our Study}
    \label{tab:10features}
\end{table}

The flare data was sourced from the XRT flare catalog. 
This catalog contains extensive flare information, 
including positions, classes, GOES fluxes of the flare source regions, and the source AR's NOAA numbers. 
In this catalog, the magnetic group assignments of the flare source ARs were determined using the SolarMonitor database. The data from SolarMonitor.org and XRT were acquired using a web scraping script written in Python. 
We collected 986 AR-produced flare events in the period
between January 2011 and October 2021. 
After removing ARs with incomplete or missing feature values, 
we obtained 837 AR records, where each AR record
contains values of the 10 physical features of the corresponding AR and
the flare event produced by the AR.
Table \ref{tab:flareevents} summarizes
the types and counts of these flare events.

\begin{table}[h]
    \centering
    \begin{tabular}{|c|c|}
        \hline
        \textbf{Flare Type} & \textbf{Counts} \\ \hline
        A-Class & 1 \\ \hline
        B-Class & 231 \\ \hline
        C-Class & 539\\ \hline
        M-Class & 61\\ \hline
        X-Class & 5\\ \hline   
    \end{tabular}
        \vspace*{+0.5cm}
    \caption{Types and Counts of AR-Produced Flares Considered in Our Study}
    \label{tab:flareevents}
\end{table}

\section{Methodology}
\label{sec:methods}

\subsection{Classification Task}
\label{sec:task}

We consider $\geq$C-class flares,
which have been widely studied in
the literature
\citep{2023NatSR..1313665A,2024FrASS..1098609P,2018ApJ...856....7H,2020ApJ...891...10L,2019ApJ...877..121L,
2018ApJ...858..113N,2022ApJ...941....1S,2021ApJS..257...50T,2023SoPh..298..137V,2023MNRAS.521.5384Z}. 
A $\geq$C-class flare refers to a C-class,
M-class, or X-class flare and
a $<$C-class flare refers to an A-class or B-class flare.

Given an AR record with 10 physical properties or features
listed in Table \ref{tab:10features}, 
we want to solve the following binary classification problem:
does the corresponding AR produce a $\geq$C-class flare? 
Through binary classification, we attempt to understand
which physical properties or features of a source AR play an important role in determining whether
the AR produces a $\geq$C-class flare. 

\subsection{Interpretable Classification Model}
\label{sec:shap}

We adopted the random forest (RF) model to solve the binary classification task
described in Section \ref{sec:task}.
RF based models have been shown to be very effective in predicting solar flares
\citep{2017ApJ...843..104L}.
The hyperparameters were tuned using
the randomized search capability available in the Python machine learning library, {\tt scikit-learn}
\citep{10.5555/1953048.2078195}.
There were 100 trees in the random forest. 
For model interpretation, we adopted
the SHAP (SHapley Additive exPlanations) framework
\citep{DBLP:conf/nips/LundbergL17}.
SHAP is used to calculate the contribution of each feature to the final output. 
Let $F$ be the set of the 10 features in Table \ref{tab:10features}
and let $S$ be a subset of $F$.
The SHAP value of each feature $i$ in $F$, $1 \leq i \leq |F|$,
denoted $\phi_{i}$,
is defined as:
\begin{equation}
\phi_{i} = \sum_{S \subseteq F-\{i\}}\frac{|S|!(|F|-|S|-1)!}{|F|!}(C(S\cup\{j\})-C(S)).
\end{equation}
The SHAP value determines the difference in the contribution that the feature $i$ brings to the prediction if included in
a specific subset $S$, and averages the differences over every possible
combination of possible subsets $S$ of the features in terms of the
contribution function:
$C(S\cup\{j\})-C(S)$.
SHAP can be used for global interpretation (that is, the interpretation is made over the entire test set)
or local interpretation (the interpretation is made for a specific test sample).
We adopted {\tt shap.TreeExplainer} in
our study.

\section{Results}
\label{sec:results}

\subsection{Evaluation Metrics}

Given an AR record $R$ with 10 physical feature values,
we define $R$ as a true positive (TP) 
if our RF model predicts that $R$ belongs to the positive class
(i.e., the corresponding AR produces a $\geq$C-class flare)
and $R$ is indeed positive.
We define $R$ as a false positive (FP)
if our RF model predicts that $R$ is positive
while $R$ actually belongs to the negative class
(i.e., the corresponding AR produces a $<$C-class flare). 
We say $R$ is a true negative (TN) if our RF model
predicts that $R$ is negative
and $R$ is indeed negative; 
$R$ is a false negative (FN) if
our RF model predicts that $R$ is
negative while $R$ is actually positive.
When the context is clear, we also use
TP (FP, TN, and FN, respectively) 
to represent the number of
true positives
(false positives, true negatives, and false negatives, respectively) 
produced by our RF model.

The evaluation metrics used in this study include the
following:
\begin{equation}
    \text{Recall} = \frac{\mbox{TP}}{\mbox{TP} + \mbox{FN}},
\end{equation}

\begin{equation}
    \text{Precision} = \frac{\mbox{TP}}{\mbox{TP} + \mbox{FP}},
\end{equation}

\begin{equation}
    \text{Accuracy} = \frac{\mbox{TP} + \mbox{TN}}
    {\mbox{TP} + \mbox{FP} + \mbox{TN} + \mbox{FN}},
\end{equation}

\begin{equation}
\text{F1} =  \frac{2 \times \text{TP}}{2 \times \text{TP} + \text{FP}+\text{FN}}.
\end{equation}

We split the set of 837 AR records at hand into a
80\% training set and a 20\% test set.
There are a total of 605 AR records
in the positive class and
232 AR records in the negative class.
After splitting, the training set has 484 positive AR records and 185 negative AR records, totaling 669 training AR records or training samples.
The test set has 121 positive AR records
and 47 negative AR records, totaling 168 test AR records or test samples.
Each AR record in the training set contains 10 AR feature values together with a label
(positive vs. negative).
Each AR record in the test set contains 10 AR feature values without a label.
The label in a test sample will be predicted by our classification model.
We compute the TP, FP, TN, FN and all evaluation metric values based on the test set.

\subsection{Performance Assessment}

\begin{figure}
\centering
\hspace*{-0.7cm}
\includegraphics[width=0.4\linewidth]{Confusion_Matrix.png}
\caption{Confusion matrix
obtained by our RF model on
the test set with 168 test samples.
}
\label{fig:confusion_matrix}
\end{figure}

Figure \ref{fig:confusion_matrix} presents the confusion matrix obtained by our RF model, which provides a breakdown
analysis of errors that occur
when the model makes predictions in the test set. 
Based on the confusion matrix, we obtain:
recall = 0.81,
precision = 0.82,
accuracy = 0.74,
F1 = 0.82.
To assess the stability and reliability of our RF model, we also performed a five-fold cross-validation.
The average accuracy for five folds is 0.75.
These results indicate that our RF model performs reasonably well in solving the binary classification problem described in
Section \ref{sec:task}.

\subsection{Model Interpretation}
\label{sec:SHAP}

As described in Section \ref{sec:shap}, 
SHAP incorporates game theory to give each feature a
SHAP value. Positive SHAP values have a positive effect,
while negative SHAP values have a negative effect. A positive
effect increases the probability of predicting that a test
sample is in the positive class, whereas a negative effect
increases the probability of predicting that a test sample is
in the negative class. 

Figure \ref{fig:beeswarm1} presents the beeswarm plot for the RF model. In the beeswarm plot, we can observe, for each feature, the
distribution of SHAP values for all test samples. Each test
sample corresponds to a dot in each feature row. The placement
of each dot on the $x$-axis is determined by the SHAP
value that the corresponding feature of the corresponding test sample receives. When dots cluster, it shows common test samples based on their SHAP values of the corresponding feature. 
The color of a dot depends on the value of a feature
in the corresponding test sample. Red indicates a high
feature value, blue indicates a low feature value, and purple
indicates an average or moderate feature value.

\begin{figure}
\centering
\hspace*{-0.7cm}
\includegraphics[width=0.75\linewidth]{Beeswarm.png}
\caption{Beeswarm plot to visualize the positive or negative
effect of a feature for each test sample, represented by a
color dot, on the RF model’s predictions.}
\label{fig:beeswarm1}
\end{figure}

Now, focus on the NoS\_Today feature in Figure \ref{fig:beeswarm1}. There are
more test samples with negative SHAP values than with positive
SHAP values. This means that the NoS\_Today feature is
more likely to push the model’s predictions towards a negative
class. Furthermore, low feature values (blue dots) tend
to lead to negative predictions, as these low feature values
have negative SHAP values. High feature values (red dots)
tend to lead to positive predictions, as these high feature values
have positive SHAP values.

Next, focus on the Hale\_Class\_Yesterday feature in 
Figure \ref{fig:beeswarm1}.
If we compare this feature with the 
NoS\_Today feature,
this feature may appear to have fewer dots. The reason
for this appearance is that most of the SHAP values for
Hale\_Class\_Yesterday are zero or near zero. 
This causes the dots to
cluster up and makes Hale\_Class\_Yesterday seem to have fewer dots.
Furthermore, since most of the SHAP values for the feature
are zero or near zero, the Hale\_Class\_Yesterday feature neither pushes
the model’s predictions towards a negative class nor pushes
the model’s predictions towards a positive class, implying
that the Hale\_Class\_Yesterday feature plays an unimportant role in the
model’s predictions.

Figure \ref{fig:barplot} presents the bar plot
for the RF model. In
the bar plot, each feature is given the mean of the absolute
SHAP values across all test samples. This mean of the absolute
SHAP values is how the importance of each feature
is measured in the bar plot. The longer the bar in the bar
plot, the more important the corresponding feature is to the
model’s predictions. We can see in Figure \ref{fig:barplot} that the AR\_Type\_Today
feature is of the highest importance to the model’s predictions,
while the Hale\_Class\_Yesterday feature is of the lowest importance to the model’s predictions

\begin{figure}
\centering
\hspace*{-0.7cm}
\includegraphics[width=0.75\linewidth]{Bar.png}
\caption{Bar plot to display the global importance of each
feature on our RF model’s predictions.}
\label{fig:barplot}
\end{figure}

In contrast to the bar plot in Figure \ref{fig:barplot}, the beeswarm plot
in Figure \ref{fig:beeswarm1} displays a separate SHAP value for each test
sample, and shows the variability of the test samples, represented
by color dots, for each feature. 
The fewer clusters
and the more spread out from the zero in the beeswarm plot
indicate higher SHAP values (positive or negative), causing
a higher mean of absolute SHAP values and ultimately a
higher importance. Now, consider again the appearance of
Hale\_Class\_Yesterday in the beeswarm plot in Figure \ref{fig:beeswarm1}. The result of
having most of its SHAP values equal or near zero reflects
how short its bar is in the bar plot in 
Figure \ref{fig:barplot}.
On the other hand, the nature of high variability of the 
AR\_Type\_Today feature
in the beeswarm plot in 
Figure \ref{fig:beeswarm1}
causes its bar to be significantly
longer than those of the other features in the bar plot
in Figure \ref{fig:barplot}.

\begin{figure}
\centering
\hspace*{-0.7cm}
\includegraphics[width=0.7\linewidth]{Decision.png}
\caption{Decision plot to understand how our RF model produces
its predictions.}
\label{fig:dec1}
\end{figure}

Figure \ref{fig:dec1} 
presents the decision plot for our RF model.
On the $y$-axis of the decision plot, the features are displayed,
from top to bottom, according to their importance, with the most
important feature displayed at the top and the least important
feature displayed at the bottom. 
Each prediction/test sample
is represented by a line in the decision plot. The predictions
for the test samples begin at the bottom with the same base
value, which is approximately 0.67.
This base value is the
mean of the model's prediction values over all training samples
in the training set. It is used as a starting point before
considering any feature contribution. 
As a prediction/line moves from bottom to top, the SHAP value for each feature
is added to the base value. This can help to understand the
contribution of each feature in the prediction. The features
can have positive or negative contributions, pushing the corresponding
line to the right or left, respectively.
The $x$-axis shows the model output values of the test samples in the test set.
The prediction value (i.e., the model output value) of a test
sample is the probability that the test sample belongs to the
positive class. 
For example, a prediction value of 0.6 indicates
that there is a 60\% chance that the test sample belongs
to the positive class.
At the top, each prediction/line reaches
its final prediction value (predicted probability). 
The prediction value determines the color of the corresponding line. A blue/purple line indicates
a lower prediction value (lower predicted probability closer to 0), 
while a red line indicates a higher prediction
value (higher predicted probability closer to 1). 

The results shown in the beeswarm, bar, and decision
plots are consistent. Based on these results, we conclude
that AR\_Type\_Today and Hale\_Class\_Today play the most 
important roles in the model's decision-making process.
The difference between the number of sunspots of an AR in today and
the number of sunspots of the AR in yesterday, i.e.,
NoS\_Difference, also plays a very important role in the 
model’s predictions.
The three plots (beeswarm, bar, and decision) are used mainly for global interpretation based on the entire test set.
In what follows, we will present waterfall plots for local interpretation,
which look at individual predictions made by
our RF model. With waterfall plots, we can better understand
each feature and why certain behaviors lead to a specific
prediction.

Figure \ref{fig:waterfallpositive} presents the waterfall plot for a test sample predicted
to be in the positive class. 
Figure \ref{fig:waterfallnegative} presents the
waterfall plot for a test sample predicted
to be in the negative class. 
These plots
display the relative contributions of the different features in
order of importance.
In Figure \ref{fig:waterfallpositive}, we see that the three most important features\\
(NOS\_Difference, 
NoS\_Yesterday,
AR\_Type\_Today)
all make positive contributions
with positive SHAP values, 
drawing the model to predict that the test sample is positive.
In contrast, in Figure \ref{fig:waterfallnegative},
the six most important features all make negative contributions with negative SHAP values, drawing the model to predict that the test sample is negative.
When comparing the waterfall plots to the three previous plots (beeswarm, bar, and decision), it
is important to recall that the three previous plots are designed for global interpretation
based on all test samples in the test set, while
the waterfall plots are designed for local interpretation inspecting specific
test samples. 
Thus, the rankings of the importance of the features between the
three previous plots
(beeswarm, bar, and decision)
and the waterfall plots are different.
However, comparing the previous three plots
(beeswarm, bar, and decision)
and
the waterfall plots can provide
valuable insight to identify possible overlaps in importance
of the features. 
NoS\_Difference, for example, is of great importance
in all four plots
(beeswarm, bar, decision, and waterfall).

\begin{figure}
\centering
\hspace*{-0.7cm}
\includegraphics[width=0.8\linewidth]{Waterfall_Positive.png}
\caption{Waterfall plot for a test sample predicted to be positive. The plot shows the relative contribution of each feature to
the model’s prediction $f(x)$ = 0.9, 
starting from the base value $E[f(x)]$ = 0.726. 
The $x$-axis represents the model output value (predicted probability)
while the $y$-axis 
shows the features and their value.
We encode the categorical features AR\_Type and Hale\_Class where AR\_Type = 2 represents $\gamma$
and
Hale\_Class = 5 represents $\beta$$\gamma$.
The arrows display the SHAP value associated with each
feature, colored red if positive and blue if negative.}
\label{fig:waterfallpositive}
\end{figure}


\begin{figure}
\centering
\hspace*{-0.7cm}
\includegraphics[width=0.8\linewidth]{Waterfall_Negative.png}
\caption{Waterfall plot for a test sample predicted to be negative. The plot shows the relative contribution of each feature to
the model’s prediction $f(x)$ = 0.33, 
starting from the base value $E[f(x)]$ = 0.726.}
\label{fig:waterfallnegative}
\end{figure}

\section{Discussion}
\label{sec:discussion}

\begin{figure}
\centering
\hspace*{-0.7cm}
\includegraphics[width=0.7\linewidth]{AR_Type_Today.png}
\caption{Number of test samples for the varying feature values (categorical values) of AR\_Type\_Today obtained based on the 121 positive and 47 negative samples in the test set. For each feature value $v$, there is a clear difference between the number of positive test samples with the feature value $v$ and the number of negative test samples with the feature value $v$. This AR\_Type\_Today is the most important feature among the 10 physical features considered in this study. Our RF model would prefer to use this feature for flare classification.}
\label{fig:artypetoday}
\end{figure}

\begin{figure}
\centering
\hspace*{-0.7cm}
\includegraphics[width=0.7\linewidth]{Hale_Class_Yesterday.png}
\caption{Number of test samples for the varying feature values (categorical values) of Hale\_Class\_Yesterday obtained based on the 121 positive and 47 negative samples in the test set. No test sample has $\alpha$$\delta$, and hence this feature value is not shown in the figure. Compared to AR\_Type\_Today in Figure \ref{fig:artypetoday}, there are 3 features ($\alpha$, $\alpha$$\gamma$, $\alpha$$\gamma$$\delta$) based on which it is hard to distinguish between positive test samples and negative test samples. 
This Hale\_Class\_Yesterday is the least important feature among the 10 physical features considered in this study. Our RF model would prefer not to use this feature for flare classification.}
\label{fig:haleclassyesterday}
\end{figure}

\begin{figure}
\centering
\hspace*{-0.7cm}
\includegraphics[width=0.7\linewidth]{NoS_Difference.png}
\caption{Number of test samples for the varying feature values of NoS\_Difference obtained based on the 121 positive and 47 negative samples in the test set. The feature values are numerical values, ranging from $-58$ to 25. Like AR\_Type\_Today in Figure \ref{fig:artypetoday}, it is relatively easy to distinguish between positive test samples and negative test samples based on the feature values. This NoS\_Difference is of high importance in both global and local interpretations. Our RF model would prefer to use this feature for flare classification.}
\label{fig:nosdifference}
\end{figure}

The results in Section \ref{sec:SHAP}, show that the AR\_Type\_Today and Hale\_Class\_Yesterday features play the most important and least important roles in determining whether an AR produces a $\geq$C-class flare based on the test set. When looking at specific test samples more precisely
(i.e., the positive and negative test samples), we see that 
NoS\_Difference is of the highest importance in the decision-making process.

In the AR\_Type\_Today feature (Figure \ref{fig:artypetoday}), $\beta$, $\gamma$ and $\gamma$$\delta$ types are more than 90\% effective for positive (i.e. $\geq$C) classifications, while they have a share of more than 74\% in negative ($<$C) classifications.  If we analyze the Hale\_Class\_Yesterday feature (Figure \ref{fig:haleclassyesterday}) as the slightest factor, the $\beta$, $\beta$$\gamma$ and $\beta$$\gamma$$\delta$ classes are effective over 85\% cases in $\geq$C predictions, while they are 72\% dominant in $<$C predictions. 

More than 90\% of the 121 positive test samples (i.e. in 109 samples) have a \\
NoS\_Difference feature value between $-13$ and $10$ as seen in Figure \ref{fig:nosdifference}. The number of sunspots in the ARs was found to be unchanged for 19 cases. In more than 45\% of the samples, the number of sunspots increased, while in 39\%, the number of sunspots decreased. In the $<$C (negative) predictions, the NoS\_Difference feature takes values between $-3$ and $4$ in nearly 81\% of the data. In 10 out of 47 samples, the number of spots of ARs remained constant, while it increased in 45\% and decreased in 34\% of the negative samples. 

\section{Conclusions}
\label{sec:conclusions}

In this paper, we employ an interpretable machine learning approach to reveal the relationship between solar flares and their source active regions through a binary classification; in other words, one class is considered positive ($\geq$C class) and the other class would be negative ($<$C class). We seek to identify the physical features of a source active region that significantly influence the likelihood of
its generation of $\geq$C-class (i.e. C, M or X classes) flares. We used the Random Forest algorithm together with the SHapley Additive exPlanations (SHAP) method. The performance metric values of recall, precision, accuracy, and the F1 score were obtained as 0.81, 0.82, 0.74, and 0.82, respectively, in our model. 

SHAP decision plots for all test data provide a comprehensive view of our model's behavior, allowing us to gain deeper insights into its decision-making process
for each test sample. Today's value of AR type is obtained as the most influential, while yesterday's value of Hale class is found to be the least effective in this binary classification (Figures \ref{fig:beeswarm1} - \ref{fig:dec1}). According to a widely accepted hypothesis in the literature, as the number of sunspots increases, the flare intensity also increases. In our study, as in \cite{2004AAS...205.1002S}, the number of sunspots is not as important as stated. However, we realize that the difference between the number of spots in the AR compared to the previous day is a crucial feature in the decision-making process for positive and negative test samples, as seen in the waterfall plots (Figures \ref{fig:waterfallpositive} and \ref{fig:waterfallnegative}). When Figure \ref{fig:nosdifference} was carefully analyzed, it was found that this range of difference is between $-13$ and $10$ in 90\% of the positive class and between $-4$ and $3$ in 81\% of the negative class. 
We note that ARs of the types $\alpha$ and $\beta$ can be grouped as simple ARs, while other types were classified as ARs of medium and high complexity. \cite{10.1093/mnras/stw2742} reported that 79\% of the flares were generated by medium and high complexity AR groups in their study without binary classification. In the test sample (Figure \ref{fig:artypetoday}) of our study, it is found around 56\%. 

Our study reveals that (Figure \ref{fig:artypetoday}) 34\% and 66\%  of positive test samples are formed by simple and complex ARs, respectively. In negative test samples, these rates are 70\% and 30\% for simple structured and more complex ARs, respectively. In their study on the relationship between magnetic disturbances and flares in active regions, \citet{OLOKETUYI2023101972} reached results similar to ours. According to their results, 45\% of the flares in the positive class are formed by ARs with simple structure, while 55\% are formed by ARs with complex structure. For flares in the negative class, these rates are found to be 71\% and 29\% for simple and complex ARs, respectively. In \citet{Yang_2017}, it was reported that 46\% of $\geq$C class flares were caused by simple ARs while 54\% were caused by more complex ARs in their study of the statistical relationships between flare and AR properties. For $<$C class flares, they found that simple structured ARs are 60\% effective, while for complex structured ARs they gave a rate of 40\%. As can be seen, it would not be erroneous to conclude that the results of our study, in which a binary classification is defined as positive versus negative, are in line with the rates presented in the studies investigating the relationship between AR complexity and flares in the literature. 

Our results indicate that the $\beta$, $\gamma$ and $\gamma$$\delta$ types contribute to $\geq$C and $<$C classifications with more than 90\% and 74\% effectiveness, respectively. This result is in good agreement with the result obtained in Figure 2 of \citet{Sammis_2000}. They found that flares having X-ray fluxes from $10^{-6}$ $Wm^{-2}$ to $10^{-4}$  $Wm^{-2}$ (i.e. from C to X classes) are mostly produced by ARs having bipolar sunspot groups or more complex structures (namely ARs of $\beta$ or having higher complexity).  Although our study does not show that the surface area of ARs is effective, in a review article by \citet{2019LRSP...16....3T}, it is stated that the probability of flare eruption is proportional to the spot area and increases with the spot complexity ($\beta$, $\gamma$ and $\gamma$$\delta$ ) as found in the current study.

However, the causative mechanism for these eruptions remains unidentified. Consequently, traditional solar flare forecast fundamentally relies on the statistical correlation between solar flares and AR features derived from observational data. In this work, an interpretable machine learning approach was employed to elucidate the association between solar flares and their originating ARs using binary classification. 
The directions of future research include the following.
\begin{itemize}
\item Employing a broader spectrum of data, particularly examining data that encompass observational tools that have advanced in recent years, can reveal new findings.
\item Integrating data from contemporary technologies, such as sophisticated machine learning analysis tools, through a more diversified data set might yield innovative insights and improve predictive accuracy. 
\item Examining data over prolonged durations can uncover enduring trends and patterns that may not be evident in shorter datasets. This can aid in comprehending the evolution of the phenomenon and its possible implications.
\end{itemize}

\section*{Data Availability Statement}
The 10 physical features used in this study are taken from the Solar Monitor \\
database
(https://www.solarmonitor.org/).
The GOES solar flare catalog with flare class information
is taken from XRT Flare Catalog (https://xrt.cfa.harvard.edu/flare\_catalog/).

%\bibliography{flare}
% This must be in the first 5 lines to tell arXiv to use pdfLaTeX, which is strongly recommended.
\pdfoutput=1
% In particular, the hyperref package requires pdfLaTeX in order to break URLs across lines.

\documentclass[11pt]{article}

% Change "review" to "final" to generate the final (sometimes called camera-ready) version.
% Change to "preprint" to generate a non-anonymous version with page numbers.
\usepackage{acl}

% Standard package includes
\usepackage{times}
\usepackage{latexsym}

% Draw tables
\usepackage{booktabs}
\usepackage{multirow}
\usepackage{xcolor}
\usepackage{colortbl}
\usepackage{array} 
\usepackage{amsmath}

\newcolumntype{C}{>{\centering\arraybackslash}p{0.07\textwidth}}
% For proper rendering and hyphenation of words containing Latin characters (including in bib files)
\usepackage[T1]{fontenc}
% For Vietnamese characters
% \usepackage[T5]{fontenc}
% See https://www.latex-project.org/help/documentation/encguide.pdf for other character sets
% This assumes your files are encoded as UTF8
\usepackage[utf8]{inputenc}

% This is not strictly necessary, and may be commented out,
% but it will improve the layout of the manuscript,
% and will typically save some space.
\usepackage{microtype}
\DeclareMathOperator*{\argmax}{arg\,max}
% This is also not strictly necessary, and may be commented out.
% However, it will improve the aesthetics of text in
% the typewriter font.
\usepackage{inconsolata}

%Including images in your LaTeX document requires adding
%additional package(s)
\usepackage{graphicx}
% If the title and author information does not fit in the area allocated, uncomment the following
%
%\setlength\titlebox{<dim>}
%
% and set <dim> to something 5cm or larger.

\title{Wi-Chat: Large Language Model Powered Wi-Fi Sensing}

% Author information can be set in various styles:
% For several authors from the same institution:
% \author{Author 1 \and ... \and Author n \\
%         Address line \\ ... \\ Address line}
% if the names do not fit well on one line use
%         Author 1 \\ {\bf Author 2} \\ ... \\ {\bf Author n} \\
% For authors from different institutions:
% \author{Author 1 \\ Address line \\  ... \\ Address line
%         \And  ... \And
%         Author n \\ Address line \\ ... \\ Address line}
% To start a separate ``row'' of authors use \AND, as in
% \author{Author 1 \\ Address line \\  ... \\ Address line
%         \AND
%         Author 2 \\ Address line \\ ... \\ Address line \And
%         Author 3 \\ Address line \\ ... \\ Address line}

% \author{First Author \\
%   Affiliation / Address line 1 \\
%   Affiliation / Address line 2 \\
%   Affiliation / Address line 3 \\
%   \texttt{email@domain} \\\And
%   Second Author \\
%   Affiliation / Address line 1 \\
%   Affiliation / Address line 2 \\
%   Affiliation / Address line 3 \\
%   \texttt{email@domain} \\}
% \author{Haohan Yuan \qquad Haopeng Zhang\thanks{corresponding author} \\ 
%   ALOHA Lab, University of Hawaii at Manoa \\
%   % Affiliation / Address line 2 \\
%   % Affiliation / Address line 3 \\
%   \texttt{\{haohany,haopengz\}@hawaii.edu}}
  
\author{
{Haopeng Zhang$\dag$\thanks{These authors contributed equally to this work.}, Yili Ren$\ddagger$\footnotemark[1], Haohan Yuan$\dag$, Jingzhe Zhang$\ddagger$, Yitong Shen$\ddagger$} \\
ALOHA Lab, University of Hawaii at Manoa$\dag$, University of South Florida$\ddagger$ \\
\{haopengz, haohany\}@hawaii.edu\\
\{yiliren, jingzhe, shen202\}@usf.edu\\}



  
%\author{
%  \textbf{First Author\textsuperscript{1}},
%  \textbf{Second Author\textsuperscript{1,2}},
%  \textbf{Third T. Author\textsuperscript{1}},
%  \textbf{Fourth Author\textsuperscript{1}},
%\\
%  \textbf{Fifth Author\textsuperscript{1,2}},
%  \textbf{Sixth Author\textsuperscript{1}},
%  \textbf{Seventh Author\textsuperscript{1}},
%  \textbf{Eighth Author \textsuperscript{1,2,3,4}},
%\\
%  \textbf{Ninth Author\textsuperscript{1}},
%  \textbf{Tenth Author\textsuperscript{1}},
%  \textbf{Eleventh E. Author\textsuperscript{1,2,3,4,5}},
%  \textbf{Twelfth Author\textsuperscript{1}},
%\\
%  \textbf{Thirteenth Author\textsuperscript{3}},
%  \textbf{Fourteenth F. Author\textsuperscript{2,4}},
%  \textbf{Fifteenth Author\textsuperscript{1}},
%  \textbf{Sixteenth Author\textsuperscript{1}},
%\\
%  \textbf{Seventeenth S. Author\textsuperscript{4,5}},
%  \textbf{Eighteenth Author\textsuperscript{3,4}},
%  \textbf{Nineteenth N. Author\textsuperscript{2,5}},
%  \textbf{Twentieth Author\textsuperscript{1}}
%\\
%\\
%  \textsuperscript{1}Affiliation 1,
%  \textsuperscript{2}Affiliation 2,
%  \textsuperscript{3}Affiliation 3,
%  \textsuperscript{4}Affiliation 4,
%  \textsuperscript{5}Affiliation 5
%\\
%  \small{
%    \textbf{Correspondence:} \href{mailto:email@domain}{email@domain}
%  }
%}

\begin{document}
\maketitle
\begin{abstract}
Recent advancements in Large Language Models (LLMs) have demonstrated remarkable capabilities across diverse tasks. However, their potential to integrate physical model knowledge for real-world signal interpretation remains largely unexplored. In this work, we introduce Wi-Chat, the first LLM-powered Wi-Fi-based human activity recognition system. We demonstrate that LLMs can process raw Wi-Fi signals and infer human activities by incorporating Wi-Fi sensing principles into prompts. Our approach leverages physical model insights to guide LLMs in interpreting Channel State Information (CSI) data without traditional signal processing techniques. Through experiments on real-world Wi-Fi datasets, we show that LLMs exhibit strong reasoning capabilities, achieving zero-shot activity recognition. These findings highlight a new paradigm for Wi-Fi sensing, expanding LLM applications beyond conventional language tasks and enhancing the accessibility of wireless sensing for real-world deployments.
\end{abstract}

\section{Introduction}

In today’s rapidly evolving digital landscape, the transformative power of web technologies has redefined not only how services are delivered but also how complex tasks are approached. Web-based systems have become increasingly prevalent in risk control across various domains. This widespread adoption is due their accessibility, scalability, and ability to remotely connect various types of users. For example, these systems are used for process safety management in industry~\cite{kannan2016web}, safety risk early warning in urban construction~\cite{ding2013development}, and safe monitoring of infrastructural systems~\cite{repetto2018web}. Within these web-based risk management systems, the source search problem presents a huge challenge. Source search refers to the task of identifying the origin of a risky event, such as a gas leak and the emission point of toxic substances. This source search capability is crucial for effective risk management and decision-making.

Traditional approaches to implementing source search capabilities into the web systems often rely on solely algorithmic solutions~\cite{ristic2016study}. These methods, while relatively straightforward to implement, often struggle to achieve acceptable performances due to algorithmic local optima and complex unknown environments~\cite{zhao2020searching}. More recently, web crowdsourcing has emerged as a promising alternative for tackling the source search problem by incorporating human efforts in these web systems on-the-fly~\cite{zhao2024user}. This approach outsources the task of addressing issues encountered during the source search process to human workers, leveraging their capabilities to enhance system performance.

These solutions often employ a human-AI collaborative way~\cite{zhao2023leveraging} where algorithms handle exploration-exploitation and report the encountered problems while human workers resolve complex decision-making bottlenecks to help the algorithms getting rid of local deadlocks~\cite{zhao2022crowd}. Although effective, this paradigm suffers from two inherent limitations: increased operational costs from continuous human intervention, and slow response times of human workers due to sequential decision-making. These challenges motivate our investigation into developing autonomous systems that preserve human-like reasoning capabilities while reducing dependency on massive crowdsourced labor.

Furthermore, recent advancements in large language models (LLMs)~\cite{chang2024survey} and multi-modal LLMs (MLLMs)~\cite{huang2023chatgpt} have unveiled promising avenues for addressing these challenges. One clear opportunity involves the seamless integration of visual understanding and linguistic reasoning for robust decision-making in search tasks. However, whether large models-assisted source search is really effective and efficient for improving the current source search algorithms~\cite{ji2022source} remains unknown. \textit{To address the research gap, we are particularly interested in answering the following two research questions in this work:}

\textbf{\textit{RQ1: }}How can source search capabilities be integrated into web-based systems to support decision-making in time-sensitive risk management scenarios? 
% \sq{I mention ``time-sensitive'' here because I feel like we shall say something about the response time -- LLM has to be faster than humans}

\textbf{\textit{RQ2: }}How can MLLMs and LLMs enhance the effectiveness and efficiency of existing source search algorithms? 

% \textit{\textbf{RQ2:}} To what extent does the performance of large models-assisted search align with or approach the effectiveness of human-AI collaborative search? 

To answer the research questions, we propose a novel framework called Auto-\
S$^2$earch (\textbf{Auto}nomous \textbf{S}ource \textbf{Search}) and implement a prototype system that leverages advanced web technologies to simulate real-world conditions for zero-shot source search. Unlike traditional methods that rely on pre-defined heuristics or extensive human intervention, AutoS$^2$earch employs a carefully designed prompt that encapsulates human rationales, thereby guiding the MLLM to generate coherent and accurate scene descriptions from visual inputs about four directional choices. Based on these language-based descriptions, the LLM is enabled to determine the optimal directional choice through chain-of-thought (CoT) reasoning. Comprehensive empirical validation demonstrates that AutoS$^2$-\ 
earch achieves a success rate of 95–98\%, closely approaching the performance of human-AI collaborative search across 20 benchmark scenarios~\cite{zhao2023leveraging}. 

Our work indicates that the role of humans in future web crowdsourcing tasks may evolve from executors to validators or supervisors. Furthermore, incorporating explanations of LLM decisions into web-based system interfaces has the potential to help humans enhance task performance in risk control.






\section{Related Work}
\label{sec:relatedworks}

% \begin{table*}[t]
% \centering 
% \renewcommand\arraystretch{0.98}
% \fontsize{8}{10}\selectfont \setlength{\tabcolsep}{0.4em}
% \begin{tabular}{@{}lc|cc|cc|cc@{}}
% \toprule
% \textbf{Methods}           & \begin{tabular}[c]{@{}c@{}}\textbf{Training}\\ \textbf{Paradigm}\end{tabular} & \begin{tabular}[c]{@{}c@{}}\textbf{$\#$ PT Data}\\ \textbf{(Tokens)}\end{tabular} & \begin{tabular}[c]{@{}c@{}}\textbf{$\#$ IFT Data}\\ \textbf{(Samples)}\end{tabular} & \textbf{Code}  & \begin{tabular}[c]{@{}c@{}}\textbf{Natural}\\ \textbf{Language}\end{tabular} & \begin{tabular}[c]{@{}c@{}}\textbf{Action}\\ \textbf{Trajectories}\end{tabular} & \begin{tabular}[c]{@{}c@{}}\textbf{API}\\ \textbf{Documentation}\end{tabular}\\ \midrule 
% NexusRaven~\citep{srinivasan2023nexusraven} & IFT & - & - & \textcolor{green}{\CheckmarkBold} & \textcolor{green}{\CheckmarkBold} &\textcolor{red}{\XSolidBrush}&\textcolor{red}{\XSolidBrush}\\
% AgentInstruct~\citep{zeng2023agenttuning} & IFT & - & 2k & \textcolor{green}{\CheckmarkBold} & \textcolor{green}{\CheckmarkBold} &\textcolor{red}{\XSolidBrush}&\textcolor{red}{\XSolidBrush} \\
% AgentEvol~\citep{xi2024agentgym} & IFT & - & 14.5k & \textcolor{green}{\CheckmarkBold} & \textcolor{green}{\CheckmarkBold} &\textcolor{green}{\CheckmarkBold}&\textcolor{red}{\XSolidBrush} \\
% Gorilla~\citep{patil2023gorilla}& IFT & - & 16k & \textcolor{green}{\CheckmarkBold} & \textcolor{green}{\CheckmarkBold} &\textcolor{red}{\XSolidBrush}&\textcolor{green}{\CheckmarkBold}\\
% OpenFunctions-v2~\citep{patil2023gorilla} & IFT & - & 65k & \textcolor{green}{\CheckmarkBold} & \textcolor{green}{\CheckmarkBold} &\textcolor{red}{\XSolidBrush}&\textcolor{green}{\CheckmarkBold}\\
% LAM~\citep{zhang2024agentohana} & IFT & - & 42.6k & \textcolor{green}{\CheckmarkBold} & \textcolor{green}{\CheckmarkBold} &\textcolor{green}{\CheckmarkBold}&\textcolor{red}{\XSolidBrush} \\
% xLAM~\citep{liu2024apigen} & IFT & - & 60k & \textcolor{green}{\CheckmarkBold} & \textcolor{green}{\CheckmarkBold} &\textcolor{green}{\CheckmarkBold}&\textcolor{red}{\XSolidBrush} \\\midrule
% LEMUR~\citep{xu2024lemur} & PT & 90B & 300k & \textcolor{green}{\CheckmarkBold} & \textcolor{green}{\CheckmarkBold} &\textcolor{green}{\CheckmarkBold}&\textcolor{red}{\XSolidBrush}\\
% \rowcolor{teal!12} \method & PT & 103B & 95k & \textcolor{green}{\CheckmarkBold} & \textcolor{green}{\CheckmarkBold} & \textcolor{green}{\CheckmarkBold} & \textcolor{green}{\CheckmarkBold} \\
% \bottomrule
% \end{tabular}
% \caption{Summary of existing tuning- and pretraining-based LLM agents with their training sample sizes. "PT" and "IFT" denote "Pre-Training" and "Instruction Fine-Tuning", respectively. }
% \label{tab:related}
% \end{table*}

\begin{table*}[ht]
\begin{threeparttable}
\centering 
\renewcommand\arraystretch{0.98}
\fontsize{7}{9}\selectfont \setlength{\tabcolsep}{0.2em}
\begin{tabular}{@{}l|c|c|ccc|cc|cc|cccc@{}}
\toprule
\textbf{Methods} & \textbf{Datasets}           & \begin{tabular}[c]{@{}c@{}}\textbf{Training}\\ \textbf{Paradigm}\end{tabular} & \begin{tabular}[c]{@{}c@{}}\textbf{\# PT Data}\\ \textbf{(Tokens)}\end{tabular} & \begin{tabular}[c]{@{}c@{}}\textbf{\# IFT Data}\\ \textbf{(Samples)}\end{tabular} & \textbf{\# APIs} & \textbf{Code}  & \begin{tabular}[c]{@{}c@{}}\textbf{Nat.}\\ \textbf{Lang.}\end{tabular} & \begin{tabular}[c]{@{}c@{}}\textbf{Action}\\ \textbf{Traj.}\end{tabular} & \begin{tabular}[c]{@{}c@{}}\textbf{API}\\ \textbf{Doc.}\end{tabular} & \begin{tabular}[c]{@{}c@{}}\textbf{Func.}\\ \textbf{Call}\end{tabular} & \begin{tabular}[c]{@{}c@{}}\textbf{Multi.}\\ \textbf{Step}\end{tabular}  & \begin{tabular}[c]{@{}c@{}}\textbf{Plan}\\ \textbf{Refine}\end{tabular}  & \begin{tabular}[c]{@{}c@{}}\textbf{Multi.}\\ \textbf{Turn}\end{tabular}\\ \midrule 
\multicolumn{13}{l}{\emph{Instruction Finetuning-based LLM Agents for Intrinsic Reasoning}}  \\ \midrule
FireAct~\cite{chen2023fireact} & FireAct & IFT & - & 2.1K & 10 & \textcolor{red}{\XSolidBrush} &\textcolor{green}{\CheckmarkBold} &\textcolor{green}{\CheckmarkBold}  & \textcolor{red}{\XSolidBrush} &\textcolor{green}{\CheckmarkBold} & \textcolor{red}{\XSolidBrush} &\textcolor{green}{\CheckmarkBold} & \textcolor{red}{\XSolidBrush} \\
ToolAlpaca~\cite{tang2023toolalpaca} & ToolAlpaca & IFT & - & 4.0K & 400 & \textcolor{red}{\XSolidBrush} &\textcolor{green}{\CheckmarkBold} &\textcolor{green}{\CheckmarkBold} & \textcolor{red}{\XSolidBrush} &\textcolor{green}{\CheckmarkBold} & \textcolor{red}{\XSolidBrush}  &\textcolor{green}{\CheckmarkBold} & \textcolor{red}{\XSolidBrush}  \\
ToolLLaMA~\cite{qin2023toolllm} & ToolBench & IFT & - & 12.7K & 16,464 & \textcolor{red}{\XSolidBrush} &\textcolor{green}{\CheckmarkBold} &\textcolor{green}{\CheckmarkBold} &\textcolor{red}{\XSolidBrush} &\textcolor{green}{\CheckmarkBold}&\textcolor{green}{\CheckmarkBold}&\textcolor{green}{\CheckmarkBold} &\textcolor{green}{\CheckmarkBold}\\
AgentEvol~\citep{xi2024agentgym} & AgentTraj-L & IFT & - & 14.5K & 24 &\textcolor{red}{\XSolidBrush} & \textcolor{green}{\CheckmarkBold} &\textcolor{green}{\CheckmarkBold}&\textcolor{red}{\XSolidBrush} &\textcolor{green}{\CheckmarkBold}&\textcolor{red}{\XSolidBrush} &\textcolor{red}{\XSolidBrush} &\textcolor{green}{\CheckmarkBold}\\
Lumos~\cite{yin2024agent} & Lumos & IFT  & - & 20.0K & 16 &\textcolor{red}{\XSolidBrush} & \textcolor{green}{\CheckmarkBold} & \textcolor{green}{\CheckmarkBold} &\textcolor{red}{\XSolidBrush} & \textcolor{green}{\CheckmarkBold} & \textcolor{green}{\CheckmarkBold} &\textcolor{red}{\XSolidBrush} & \textcolor{green}{\CheckmarkBold}\\
Agent-FLAN~\cite{chen2024agent} & Agent-FLAN & IFT & - & 24.7K & 20 &\textcolor{red}{\XSolidBrush} & \textcolor{green}{\CheckmarkBold} & \textcolor{green}{\CheckmarkBold} &\textcolor{red}{\XSolidBrush} & \textcolor{green}{\CheckmarkBold}& \textcolor{green}{\CheckmarkBold}&\textcolor{red}{\XSolidBrush} & \textcolor{green}{\CheckmarkBold}\\
AgentTuning~\citep{zeng2023agenttuning} & AgentInstruct & IFT & - & 35.0K & - &\textcolor{red}{\XSolidBrush} & \textcolor{green}{\CheckmarkBold} & \textcolor{green}{\CheckmarkBold} &\textcolor{red}{\XSolidBrush} & \textcolor{green}{\CheckmarkBold} &\textcolor{red}{\XSolidBrush} &\textcolor{red}{\XSolidBrush} & \textcolor{green}{\CheckmarkBold}\\\midrule
\multicolumn{13}{l}{\emph{Instruction Finetuning-based LLM Agents for Function Calling}} \\\midrule
NexusRaven~\citep{srinivasan2023nexusraven} & NexusRaven & IFT & - & - & 116 & \textcolor{green}{\CheckmarkBold} & \textcolor{green}{\CheckmarkBold}  & \textcolor{green}{\CheckmarkBold} &\textcolor{red}{\XSolidBrush} & \textcolor{green}{\CheckmarkBold} &\textcolor{red}{\XSolidBrush} &\textcolor{red}{\XSolidBrush}&\textcolor{red}{\XSolidBrush}\\
Gorilla~\citep{patil2023gorilla} & Gorilla & IFT & - & 16.0K & 1,645 & \textcolor{green}{\CheckmarkBold} &\textcolor{red}{\XSolidBrush} &\textcolor{red}{\XSolidBrush}&\textcolor{green}{\CheckmarkBold} &\textcolor{green}{\CheckmarkBold} &\textcolor{red}{\XSolidBrush} &\textcolor{red}{\XSolidBrush} &\textcolor{red}{\XSolidBrush}\\
OpenFunctions-v2~\citep{patil2023gorilla} & OpenFunctions-v2 & IFT & - & 65.0K & - & \textcolor{green}{\CheckmarkBold} & \textcolor{green}{\CheckmarkBold} &\textcolor{red}{\XSolidBrush} &\textcolor{green}{\CheckmarkBold} &\textcolor{green}{\CheckmarkBold} &\textcolor{red}{\XSolidBrush} &\textcolor{red}{\XSolidBrush} &\textcolor{red}{\XSolidBrush}\\
API Pack~\cite{guo2024api} & API Pack & IFT & - & 1.1M & 11,213 &\textcolor{green}{\CheckmarkBold} &\textcolor{red}{\XSolidBrush} &\textcolor{green}{\CheckmarkBold} &\textcolor{red}{\XSolidBrush} &\textcolor{green}{\CheckmarkBold} &\textcolor{red}{\XSolidBrush}&\textcolor{red}{\XSolidBrush}&\textcolor{red}{\XSolidBrush}\\ 
LAM~\citep{zhang2024agentohana} & AgentOhana & IFT & - & 42.6K & - & \textcolor{green}{\CheckmarkBold} & \textcolor{green}{\CheckmarkBold} &\textcolor{green}{\CheckmarkBold}&\textcolor{red}{\XSolidBrush} &\textcolor{green}{\CheckmarkBold}&\textcolor{red}{\XSolidBrush}&\textcolor{green}{\CheckmarkBold}&\textcolor{green}{\CheckmarkBold}\\
xLAM~\citep{liu2024apigen} & APIGen & IFT & - & 60.0K & 3,673 & \textcolor{green}{\CheckmarkBold} & \textcolor{green}{\CheckmarkBold} &\textcolor{green}{\CheckmarkBold}&\textcolor{red}{\XSolidBrush} &\textcolor{green}{\CheckmarkBold}&\textcolor{red}{\XSolidBrush}&\textcolor{green}{\CheckmarkBold}&\textcolor{green}{\CheckmarkBold}\\\midrule
\multicolumn{13}{l}{\emph{Pretraining-based LLM Agents}}  \\\midrule
% LEMUR~\citep{xu2024lemur} & PT & 90B & 300.0K & - & \textcolor{green}{\CheckmarkBold} & \textcolor{green}{\CheckmarkBold} &\textcolor{green}{\CheckmarkBold}&\textcolor{red}{\XSolidBrush} & \textcolor{red}{\XSolidBrush} &\textcolor{green}{\CheckmarkBold} &\textcolor{red}{\XSolidBrush}&\textcolor{red}{\XSolidBrush}\\
\rowcolor{teal!12} \method & \dataset & PT & 103B & 95.0K  & 76,537  & \textcolor{green}{\CheckmarkBold} & \textcolor{green}{\CheckmarkBold} & \textcolor{green}{\CheckmarkBold} & \textcolor{green}{\CheckmarkBold} & \textcolor{green}{\CheckmarkBold} & \textcolor{green}{\CheckmarkBold} & \textcolor{green}{\CheckmarkBold} & \textcolor{green}{\CheckmarkBold}\\
\bottomrule
\end{tabular}
% \begin{tablenotes}
%     \item $^*$ In addition, the StarCoder-API can offer 4.77M more APIs.
% \end{tablenotes}
\caption{Summary of existing instruction finetuning-based LLM agents for intrinsic reasoning and function calling, along with their training resources and sample sizes. "PT" and "IFT" denote "Pre-Training" and "Instruction Fine-Tuning", respectively.}
\vspace{-2ex}
\label{tab:related}
\end{threeparttable}
\end{table*}

\noindent \textbf{Prompting-based LLM Agents.} Due to the lack of agent-specific pre-training corpus, existing LLM agents rely on either prompt engineering~\cite{hsieh2023tool,lu2024chameleon,yao2022react,wang2023voyager} or instruction fine-tuning~\cite{chen2023fireact,zeng2023agenttuning} to understand human instructions, decompose high-level tasks, generate grounded plans, and execute multi-step actions. 
However, prompting-based methods mainly depend on the capabilities of backbone LLMs (usually commercial LLMs), failing to introduce new knowledge and struggling to generalize to unseen tasks~\cite{sun2024adaplanner,zhuang2023toolchain}. 

\noindent \textbf{Instruction Finetuning-based LLM Agents.} Considering the extensive diversity of APIs and the complexity of multi-tool instructions, tool learning inherently presents greater challenges than natural language tasks, such as text generation~\cite{qin2023toolllm}.
Post-training techniques focus more on instruction following and aligning output with specific formats~\cite{patil2023gorilla,hao2024toolkengpt,qin2023toolllm,schick2024toolformer}, rather than fundamentally improving model knowledge or capabilities. 
Moreover, heavy fine-tuning can hinder generalization or even degrade performance in non-agent use cases, potentially suppressing the original base model capabilities~\cite{ghosh2024a}.

\noindent \textbf{Pretraining-based LLM Agents.} While pre-training serves as an essential alternative, prior works~\cite{nijkamp2023codegen,roziere2023code,xu2024lemur,patil2023gorilla} have primarily focused on improving task-specific capabilities (\eg, code generation) instead of general-domain LLM agents, due to single-source, uni-type, small-scale, and poor-quality pre-training data. 
Existing tool documentation data for agent training either lacks diverse real-world APIs~\cite{patil2023gorilla, tang2023toolalpaca} or is constrained to single-tool or single-round tool execution. 
Furthermore, trajectory data mostly imitate expert behavior or follow function-calling rules with inferior planning and reasoning, failing to fully elicit LLMs' capabilities and handle complex instructions~\cite{qin2023toolllm}. 
Given a wide range of candidate API functions, each comprising various function names and parameters available at every planning step, identifying globally optimal solutions and generalizing across tasks remains highly challenging.



\section{Preliminaries}
\label{Preliminaries}
\begin{figure*}[t]
    \centering
    \includegraphics[width=0.95\linewidth]{fig/HealthGPT_Framework.png}
    \caption{The \ourmethod{} architecture integrates hierarchical visual perception and H-LoRA, employing a task-specific hard router to select visual features and H-LoRA plugins, ultimately generating outputs with an autoregressive manner.}
    \label{fig:architecture}
\end{figure*}
\noindent\textbf{Large Vision-Language Models.} 
The input to a LVLM typically consists of an image $x^{\text{img}}$ and a discrete text sequence $x^{\text{txt}}$. The visual encoder $\mathcal{E}^{\text{img}}$ converts the input image $x^{\text{img}}$ into a sequence of visual tokens $\mathcal{V} = [v_i]_{i=1}^{N_v}$, while the text sequence $x^{\text{txt}}$ is mapped into a sequence of text tokens $\mathcal{T} = [t_i]_{i=1}^{N_t}$ using an embedding function $\mathcal{E}^{\text{txt}}$. The LLM $\mathcal{M_\text{LLM}}(\cdot|\theta)$ models the joint probability of the token sequence $\mathcal{U} = \{\mathcal{V},\mathcal{T}\}$, which is expressed as:
\begin{equation}
    P_\theta(R | \mathcal{U}) = \prod_{i=1}^{N_r} P_\theta(r_i | \{\mathcal{U}, r_{<i}\}),
\end{equation}
where $R = [r_i]_{i=1}^{N_r}$ is the text response sequence. The LVLM iteratively generates the next token $r_i$ based on $r_{<i}$. The optimization objective is to minimize the cross-entropy loss of the response $\mathcal{R}$.
% \begin{equation}
%     \mathcal{L}_{\text{VLM}} = \mathbb{E}_{R|\mathcal{U}}\left[-\log P_\theta(R | \mathcal{U})\right]
% \end{equation}
It is worth noting that most LVLMs adopt a design paradigm based on ViT, alignment adapters, and pre-trained LLMs\cite{liu2023llava,liu2024improved}, enabling quick adaptation to downstream tasks.


\noindent\textbf{VQGAN.}
VQGAN~\cite{esser2021taming} employs latent space compression and indexing mechanisms to effectively learn a complete discrete representation of images. VQGAN first maps the input image $x^{\text{img}}$ to a latent representation $z = \mathcal{E}(x)$ through a encoder $\mathcal{E}$. Then, the latent representation is quantized using a codebook $\mathcal{Z} = \{z_k\}_{k=1}^K$, generating a discrete index sequence $\mathcal{I} = [i_m]_{m=1}^N$, where $i_m \in \mathcal{Z}$ represents the quantized code index:
\begin{equation}
    \mathcal{I} = \text{Quantize}(z|\mathcal{Z}) = \arg\min_{z_k \in \mathcal{Z}} \| z - z_k \|_2.
\end{equation}
In our approach, the discrete index sequence $\mathcal{I}$ serves as a supervisory signal for the generation task, enabling the model to predict the index sequence $\hat{\mathcal{I}}$ from input conditions such as text or other modality signals.  
Finally, the predicted index sequence $\hat{\mathcal{I}}$ is upsampled by the VQGAN decoder $G$, generating the high-quality image $\hat{x}^\text{img} = G(\hat{\mathcal{I}})$.



\noindent\textbf{Low Rank Adaptation.} 
LoRA\cite{hu2021lora} effectively captures the characteristics of downstream tasks by introducing low-rank adapters. The core idea is to decompose the bypass weight matrix $\Delta W\in\mathbb{R}^{d^{\text{in}} \times d^{\text{out}}}$ into two low-rank matrices $ \{A \in \mathbb{R}^{d^{\text{in}} \times r}, B \in \mathbb{R}^{r \times d^{\text{out}}} \}$, where $ r \ll \min\{d^{\text{in}}, d^{\text{out}}\} $, significantly reducing learnable parameters. The output with the LoRA adapter for the input $x$ is then given by:
\begin{equation}
    h = x W_0 + \alpha x \Delta W/r = x W_0 + \alpha xAB/r,
\end{equation}
where matrix $ A $ is initialized with a Gaussian distribution, while the matrix $ B $ is initialized as a zero matrix. The scaling factor $ \alpha/r $ controls the impact of $ \Delta W $ on the model.

\section{HealthGPT}
\label{Method}


\subsection{Unified Autoregressive Generation.}  
% As shown in Figure~\ref{fig:architecture}, 
\ourmethod{} (Figure~\ref{fig:architecture}) utilizes a discrete token representation that covers both text and visual outputs, unifying visual comprehension and generation as an autoregressive task. 
For comprehension, $\mathcal{M}_\text{llm}$ receives the input joint sequence $\mathcal{U}$ and outputs a series of text token $\mathcal{R} = [r_1, r_2, \dots, r_{N_r}]$, where $r_i \in \mathcal{V}_{\text{txt}}$, and $\mathcal{V}_{\text{txt}}$ represents the LLM's vocabulary:
\begin{equation}
    P_\theta(\mathcal{R} \mid \mathcal{U}) = \prod_{i=1}^{N_r} P_\theta(r_i \mid \mathcal{U}, r_{<i}).
\end{equation}
For generation, $\mathcal{M}_\text{llm}$ first receives a special start token $\langle \text{START\_IMG} \rangle$, then generates a series of tokens corresponding to the VQGAN indices $\mathcal{I} = [i_1, i_2, \dots, i_{N_i}]$, where $i_j \in \mathcal{V}_{\text{vq}}$, and $\mathcal{V}_{\text{vq}}$ represents the index range of VQGAN. Upon completion of generation, the LLM outputs an end token $\langle \text{END\_IMG} \rangle$:
\begin{equation}
    P_\theta(\mathcal{I} \mid \mathcal{U}) = \prod_{j=1}^{N_i} P_\theta(i_j \mid \mathcal{U}, i_{<j}).
\end{equation}
Finally, the generated index sequence $\mathcal{I}$ is fed into the decoder $G$, which reconstructs the target image $\hat{x}^{\text{img}} = G(\mathcal{I})$.

\subsection{Hierarchical Visual Perception}  
Given the differences in visual perception between comprehension and generation tasks—where the former focuses on abstract semantics and the latter emphasizes complete semantics—we employ ViT to compress the image into discrete visual tokens at multiple hierarchical levels.
Specifically, the image is converted into a series of features $\{f_1, f_2, \dots, f_L\}$ as it passes through $L$ ViT blocks.

To address the needs of various tasks, the hidden states are divided into two types: (i) \textit{Concrete-grained features} $\mathcal{F}^{\text{Con}} = \{f_1, f_2, \dots, f_k\}, k < L$, derived from the shallower layers of ViT, containing sufficient global features, suitable for generation tasks; 
(ii) \textit{Abstract-grained features} $\mathcal{F}^{\text{Abs}} = \{f_{k+1}, f_{k+2}, \dots, f_L\}$, derived from the deeper layers of ViT, which contain abstract semantic information closer to the text space, suitable for comprehension tasks.

The task type $T$ (comprehension or generation) determines which set of features is selected as the input for the downstream large language model:
\begin{equation}
    \mathcal{F}^{\text{img}}_T =
    \begin{cases}
        \mathcal{F}^{\text{Con}}, & \text{if } T = \text{generation task} \\
        \mathcal{F}^{\text{Abs}}, & \text{if } T = \text{comprehension task}
    \end{cases}
\end{equation}
We integrate the image features $\mathcal{F}^{\text{img}}_T$ and text features $\mathcal{T}$ into a joint sequence through simple concatenation, which is then fed into the LLM $\mathcal{M}_{\text{llm}}$ for autoregressive generation.
% :
% \begin{equation}
%     \mathcal{R} = \mathcal{M}_{\text{llm}}(\mathcal{U}|\theta), \quad \mathcal{U} = [\mathcal{F}^{\text{img}}_T; \mathcal{T}]
% \end{equation}
\subsection{Heterogeneous Knowledge Adaptation}
We devise H-LoRA, which stores heterogeneous knowledge from comprehension and generation tasks in separate modules and dynamically routes to extract task-relevant knowledge from these modules. 
At the task level, for each task type $ T $, we dynamically assign a dedicated H-LoRA submodule $ \theta^T $, which is expressed as:
\begin{equation}
    \mathcal{R} = \mathcal{M}_\text{LLM}(\mathcal{U}|\theta, \theta^T), \quad \theta^T = \{A^T, B^T, \mathcal{R}^T_\text{outer}\}.
\end{equation}
At the feature level for a single task, H-LoRA integrates the idea of Mixture of Experts (MoE)~\cite{masoudnia2014mixture} and designs an efficient matrix merging and routing weight allocation mechanism, thus avoiding the significant computational delay introduced by matrix splitting in existing MoELoRA~\cite{luo2024moelora}. Specifically, we first merge the low-rank matrices (rank = r) of $ k $ LoRA experts into a unified matrix:
\begin{equation}
    \mathbf{A}^{\text{merged}}, \mathbf{B}^{\text{merged}} = \text{Concat}(\{A_i\}_1^k), \text{Concat}(\{B_i\}_1^k),
\end{equation}
where $ \mathbf{A}^{\text{merged}} \in \mathbb{R}^{d^\text{in} \times rk} $ and $ \mathbf{B}^{\text{merged}} \in \mathbb{R}^{rk \times d^\text{out}} $. The $k$-dimension routing layer generates expert weights $ \mathcal{W} \in \mathbb{R}^{\text{token\_num} \times k} $ based on the input hidden state $ x $, and these are expanded to $ \mathbb{R}^{\text{token\_num} \times rk} $ as follows:
\begin{equation}
    \mathcal{W}^\text{expanded} = \alpha k \mathcal{W} / r \otimes \mathbf{1}_r,
\end{equation}
where $ \otimes $ denotes the replication operation.
The overall output of H-LoRA is computed as:
\begin{equation}
    \mathcal{O}^\text{H-LoRA} = (x \mathbf{A}^{\text{merged}} \odot \mathcal{W}^\text{expanded}) \mathbf{B}^{\text{merged}},
\end{equation}
where $ \odot $ represents element-wise multiplication. Finally, the output of H-LoRA is added to the frozen pre-trained weights to produce the final output:
\begin{equation}
    \mathcal{O} = x W_0 + \mathcal{O}^\text{H-LoRA}.
\end{equation}
% In summary, H-LoRA is a task-based dynamic PEFT method that achieves high efficiency in single-task fine-tuning.

\subsection{Training Pipeline}

\begin{figure}[t]
    \centering
    \hspace{-4mm}
    \includegraphics[width=0.94\linewidth]{fig/data.pdf}
    \caption{Data statistics of \texttt{VL-Health}. }
    \label{fig:data}
\end{figure}
\noindent \textbf{1st Stage: Multi-modal Alignment.} 
In the first stage, we design separate visual adapters and H-LoRA submodules for medical unified tasks. For the medical comprehension task, we train abstract-grained visual adapters using high-quality image-text pairs to align visual embeddings with textual embeddings, thereby enabling the model to accurately describe medical visual content. During this process, the pre-trained LLM and its corresponding H-LoRA submodules remain frozen. In contrast, the medical generation task requires training concrete-grained adapters and H-LoRA submodules while keeping the LLM frozen. Meanwhile, we extend the textual vocabulary to include multimodal tokens, enabling the support of additional VQGAN vector quantization indices. The model trains on image-VQ pairs, endowing the pre-trained LLM with the capability for image reconstruction. This design ensures pixel-level consistency of pre- and post-LVLM. The processes establish the initial alignment between the LLM’s outputs and the visual inputs.

\noindent \textbf{2nd Stage: Heterogeneous H-LoRA Plugin Adaptation.}  
The submodules of H-LoRA share the word embedding layer and output head but may encounter issues such as bias and scale inconsistencies during training across different tasks. To ensure that the multiple H-LoRA plugins seamlessly interface with the LLMs and form a unified base, we fine-tune the word embedding layer and output head using a small amount of mixed data to maintain consistency in the model weights. Specifically, during this stage, all H-LoRA submodules for different tasks are kept frozen, with only the word embedding layer and output head being optimized. Through this stage, the model accumulates foundational knowledge for unified tasks by adapting H-LoRA plugins.

\begin{table*}[!t]
\centering
\caption{Comparison of \ourmethod{} with other LVLMs and unified multi-modal models on medical visual comprehension tasks. \textbf{Bold} and \underline{underlined} text indicates the best performance and second-best performance, respectively.}
\resizebox{\textwidth}{!}{
\begin{tabular}{c|lcc|cccccccc|c}
\toprule
\rowcolor[HTML]{E9F3FE} &  &  &  & \multicolumn{2}{c}{\textbf{VQA-RAD \textuparrow}} & \multicolumn{2}{c}{\textbf{SLAKE \textuparrow}} & \multicolumn{2}{c}{\textbf{PathVQA \textuparrow}} &  &  &  \\ 
\cline{5-10}
\rowcolor[HTML]{E9F3FE}\multirow{-2}{*}{\textbf{Type}} & \multirow{-2}{*}{\textbf{Model}} & \multirow{-2}{*}{\textbf{\# Params}} & \multirow{-2}{*}{\makecell{\textbf{Medical} \\ \textbf{LVLM}}} & \textbf{close} & \textbf{all} & \textbf{close} & \textbf{all} & \textbf{close} & \textbf{all} & \multirow{-2}{*}{\makecell{\textbf{MMMU} \\ \textbf{-Med}}\textuparrow} & \multirow{-2}{*}{\textbf{OMVQA}\textuparrow} & \multirow{-2}{*}{\textbf{Avg. \textuparrow}} \\ 
\midrule \midrule
\multirow{9}{*}{\textbf{Comp. Only}} 
& Med-Flamingo & 8.3B & \Large \ding{51} & 58.6 & 43.0 & 47.0 & 25.5 & 61.9 & 31.3 & 28.7 & 34.9 & 41.4 \\
& LLaVA-Med & 7B & \Large \ding{51} & 60.2 & 48.1 & 58.4 & 44.8 & 62.3 & 35.7 & 30.0 & 41.3 & 47.6 \\
& HuatuoGPT-Vision & 7B & \Large \ding{51} & 66.9 & 53.0 & 59.8 & 49.1 & 52.9 & 32.0 & 42.0 & 50.0 & 50.7 \\
& BLIP-2 & 6.7B & \Large \ding{55} & 43.4 & 36.8 & 41.6 & 35.3 & 48.5 & 28.8 & 27.3 & 26.9 & 36.1 \\
& LLaVA-v1.5 & 7B & \Large \ding{55} & 51.8 & 42.8 & 37.1 & 37.7 & 53.5 & 31.4 & 32.7 & 44.7 & 41.5 \\
& InstructBLIP & 7B & \Large \ding{55} & 61.0 & 44.8 & 66.8 & 43.3 & 56.0 & 32.3 & 25.3 & 29.0 & 44.8 \\
& Yi-VL & 6B & \Large \ding{55} & 52.6 & 42.1 & 52.4 & 38.4 & 54.9 & 30.9 & 38.0 & 50.2 & 44.9 \\
& InternVL2 & 8B & \Large \ding{55} & 64.9 & 49.0 & 66.6 & 50.1 & 60.0 & 31.9 & \underline{43.3} & 54.5 & 52.5\\
& Llama-3.2 & 11B & \Large \ding{55} & 68.9 & 45.5 & 72.4 & 52.1 & 62.8 & 33.6 & 39.3 & 63.2 & 54.7 \\
\midrule
\multirow{5}{*}{\textbf{Comp. \& Gen.}} 
& Show-o & 1.3B & \Large \ding{55} & 50.6 & 33.9 & 31.5 & 17.9 & 52.9 & 28.2 & 22.7 & 45.7 & 42.6 \\
& Unified-IO 2 & 7B & \Large \ding{55} & 46.2 & 32.6 & 35.9 & 21.9 & 52.5 & 27.0 & 25.3 & 33.0 & 33.8 \\
& Janus & 1.3B & \Large \ding{55} & 70.9 & 52.8 & 34.7 & 26.9 & 51.9 & 27.9 & 30.0 & 26.8 & 33.5 \\
& \cellcolor[HTML]{DAE0FB}HealthGPT-M3 & \cellcolor[HTML]{DAE0FB}3.8B & \cellcolor[HTML]{DAE0FB}\Large \ding{51} & \cellcolor[HTML]{DAE0FB}\underline{73.7} & \cellcolor[HTML]{DAE0FB}\underline{55.9} & \cellcolor[HTML]{DAE0FB}\underline{74.6} & \cellcolor[HTML]{DAE0FB}\underline{56.4} & \cellcolor[HTML]{DAE0FB}\underline{78.7} & \cellcolor[HTML]{DAE0FB}\underline{39.7} & \cellcolor[HTML]{DAE0FB}\underline{43.3} & \cellcolor[HTML]{DAE0FB}\underline{68.5} & \cellcolor[HTML]{DAE0FB}\underline{61.3} \\
& \cellcolor[HTML]{DAE0FB}HealthGPT-L14 & \cellcolor[HTML]{DAE0FB}14B & \cellcolor[HTML]{DAE0FB}\Large \ding{51} & \cellcolor[HTML]{DAE0FB}\textbf{77.7} & \cellcolor[HTML]{DAE0FB}\textbf{58.3} & \cellcolor[HTML]{DAE0FB}\textbf{76.4} & \cellcolor[HTML]{DAE0FB}\textbf{64.5} & \cellcolor[HTML]{DAE0FB}\textbf{85.9} & \cellcolor[HTML]{DAE0FB}\textbf{44.4} & \cellcolor[HTML]{DAE0FB}\textbf{49.2} & \cellcolor[HTML]{DAE0FB}\textbf{74.4} & \cellcolor[HTML]{DAE0FB}\textbf{66.4} \\
\bottomrule
\end{tabular}
}
\label{tab:results}
\end{table*}
\begin{table*}[ht]
    \centering
    \caption{The experimental results for the four modality conversion tasks.}
    \resizebox{\textwidth}{!}{
    \begin{tabular}{l|ccc|ccc|ccc|ccc}
        \toprule
        \rowcolor[HTML]{E9F3FE} & \multicolumn{3}{c}{\textbf{CT to MRI (Brain)}} & \multicolumn{3}{c}{\textbf{CT to MRI (Pelvis)}} & \multicolumn{3}{c}{\textbf{MRI to CT (Brain)}} & \multicolumn{3}{c}{\textbf{MRI to CT (Pelvis)}} \\
        \cline{2-13}
        \rowcolor[HTML]{E9F3FE}\multirow{-2}{*}{\textbf{Model}}& \textbf{SSIM $\uparrow$} & \textbf{PSNR $\uparrow$} & \textbf{MSE $\downarrow$} & \textbf{SSIM $\uparrow$} & \textbf{PSNR $\uparrow$} & \textbf{MSE $\downarrow$} & \textbf{SSIM $\uparrow$} & \textbf{PSNR $\uparrow$} & \textbf{MSE $\downarrow$} & \textbf{SSIM $\uparrow$} & \textbf{PSNR $\uparrow$} & \textbf{MSE $\downarrow$} \\
        \midrule \midrule
        pix2pix & 71.09 & 32.65 & 36.85 & 59.17 & 31.02 & 51.91 & 78.79 & 33.85 & 28.33 & 72.31 & 32.98 & 36.19 \\
        CycleGAN & 54.76 & 32.23 & 40.56 & 54.54 & 30.77 & 55.00 & 63.75 & 31.02 & 52.78 & 50.54 & 29.89 & 67.78 \\
        BBDM & {71.69} & {32.91} & {34.44} & 57.37 & 31.37 & 48.06 & \textbf{86.40} & 34.12 & 26.61 & {79.26} & 33.15 & 33.60 \\
        Vmanba & 69.54 & 32.67 & 36.42 & {63.01} & {31.47} & {46.99} & 79.63 & 34.12 & 26.49 & 77.45 & 33.53 & 31.85 \\
        DiffMa & 71.47 & 32.74 & 35.77 & 62.56 & 31.43 & 47.38 & 79.00 & {34.13} & {26.45} & 78.53 & {33.68} & {30.51} \\
        \rowcolor[HTML]{DAE0FB}HealthGPT-M3 & \underline{79.38} & \underline{33.03} & \underline{33.48} & \underline{71.81} & \underline{31.83} & \underline{43.45} & {85.06} & \textbf{34.40} & \textbf{25.49} & \underline{84.23} & \textbf{34.29} & \textbf{27.99} \\
        \rowcolor[HTML]{DAE0FB}HealthGPT-L14 & \textbf{79.73} & \textbf{33.10} & \textbf{32.96} & \textbf{71.92} & \textbf{31.87} & \textbf{43.09} & \underline{85.31} & \underline{34.29} & \underline{26.20} & \textbf{84.96} & \underline{34.14} & \underline{28.13} \\
        \bottomrule
    \end{tabular}
    }
    \label{tab:conversion}
\end{table*}

\noindent \textbf{3rd Stage: Visual Instruction Fine-Tuning.}  
In the third stage, we introduce additional task-specific data to further optimize the model and enhance its adaptability to downstream tasks such as medical visual comprehension (e.g., medical QA, medical dialogues, and report generation) or generation tasks (e.g., super-resolution, denoising, and modality conversion). Notably, by this stage, the word embedding layer and output head have been fine-tuned, only the H-LoRA modules and adapter modules need to be trained. This strategy significantly improves the model's adaptability and flexibility across different tasks.


\section{Experiment}
\label{s:experiment}

\subsection{Data Description}
We evaluate our method on FI~\cite{you2016building}, Twitter\_LDL~\cite{yang2017learning} and Artphoto~\cite{machajdik2010affective}.
FI is a public dataset built from Flickr and Instagram, with 23,308 images and eight emotion categories, namely \textit{amusement}, \textit{anger}, \textit{awe},  \textit{contentment}, \textit{disgust}, \textit{excitement},  \textit{fear}, and \textit{sadness}. 
% Since images in FI are all copyrighted by law, some images are corrupted now, so we remove these samples and retain 21,828 images.
% T4SA contains images from Twitter, which are classified into three categories: \textit{positive}, \textit{neutral}, and \textit{negative}. In this paper, we adopt the base version of B-T4SA, which contains 470,586 images and provides text descriptions of the corresponding tweets.
Twitter\_LDL contains 10,045 images from Twitter, with the same eight categories as the FI dataset.
% 。
For these two datasets, they are randomly split into 80\%
training and 20\% testing set.
Artphoto contains 806 artistic photos from the DeviantArt website, which we use to further evaluate the zero-shot capability of our model.
% on the small-scale dataset.
% We construct and publicly release the first image sentiment analysis dataset containing metadata.
% 。

% Based on these datasets, we are the first to construct and publicly release metadata-enhanced image sentiment analysis datasets. These datasets include scenes, tags, descriptions, and corresponding confidence scores, and are available at this link for future research purposes.


% 
\begin{table}[t]
\centering
% \begin{center}
\caption{Overall performance of different models on FI and Twitter\_LDL datasets.}
\label{tab:cap1}
% \resizebox{\linewidth}{!}
{
\begin{tabular}{l|c|c|c|c}
\hline
\multirow{2}{*}{\textbf{Model}} & \multicolumn{2}{c|}{\textbf{FI}}  & \multicolumn{2}{c}{\textbf{Twitter\_LDL}} \\ \cline{2-5} 
  & \textbf{Accuracy} & \textbf{F1} & \textbf{Accuracy} & \textbf{F1}  \\ \hline
% (\rownumber)~AlexNet~\cite{krizhevsky2017imagenet}  & 58.13\% & 56.35\%  & 56.24\%& 55.02\%  \\ 
% (\rownumber)~VGG16~\cite{simonyan2014very}  & 63.75\%& 63.08\%  & 59.34\%& 59.02\%  \\ 
(\rownumber)~ResNet101~\cite{he2016deep} & 66.16\%& 65.56\%  & 62.02\% & 61.34\%  \\ 
(\rownumber)~CDA~\cite{han2023boosting} & 66.71\%& 65.37\%  & 64.14\% & 62.85\%  \\ 
(\rownumber)~CECCN~\cite{ruan2024color} & 67.96\%& 66.74\%  & 64.59\%& 64.72\% \\ 
(\rownumber)~EmoVIT~\cite{xie2024emovit} & 68.09\%& 67.45\%  & 63.12\% & 61.97\%  \\ 
(\rownumber)~ComLDL~\cite{zhang2022compound} & 68.83\%& 67.28\%  & 65.29\% & 63.12\%  \\ 
(\rownumber)~WSDEN~\cite{li2023weakly} & 69.78\%& 69.61\%  & 67.04\% & 65.49\% \\ 
(\rownumber)~ECWA~\cite{deng2021emotion} & 70.87\%& 69.08\%  & 67.81\% & 66.87\%  \\ 
(\rownumber)~EECon~\cite{yang2023exploiting} & 71.13\%& 68.34\%  & 64.27\%& 63.16\%  \\ 
(\rownumber)~MAM~\cite{zhang2024affective} & 71.44\%  & 70.83\% & 67.18\%  & 65.01\%\\ 
(\rownumber)~TGCA-PVT~\cite{chen2024tgca}   & 73.05\%  & 71.46\% & 69.87\%  & 68.32\% \\ 
(\rownumber)~OEAN~\cite{zhang2024object}   & 73.40\%  & 72.63\% & 70.52\%  & 69.47\% \\ \hline
(\rownumber)~\shortname  & \textbf{79.48\%} & \textbf{79.22\%} & \textbf{74.12\%} & \textbf{73.09\%} \\ \hline
\end{tabular}
}
\vspace{-6mm}
% \end{center}
\end{table}
% 

\subsection{Experiment Setting}
% \subsubsection{Model Setting.}
% 
\textbf{Model Setting:}
For feature representation, we set $k=10$ to select object tags, and adopt clip-vit-base-patch32 as the pre-trained model for unified feature representation.
Moreover, we empirically set $(d_e, d_h, d_k, d_s) = (512, 128, 16, 64)$, and set the classification class $L$ to 8.

% 

\textbf{Training Setting:}
To initialize the model, we set all weights such as $\boldsymbol{W}$ following the truncated normal distribution, and use AdamW optimizer with the learning rate of $1 \times 10^{-4}$.
% warmup scheduler of cosine, warmup steps of 2000.
Furthermore, we set the batch size to 32 and the epoch of the training process to 200.
During the implementation, we utilize \textit{PyTorch} to build our entire model.
% , and our project codes are publicly available at https://github.com/zzmyrep/MESN.
% Our project codes as well as data are all publicly available on GitHub\footnote{https://github.com/zzmyrep/KBCEN}.
% Code is available at \href{https://github.com/zzmyrep/KBCEN}{https://github.com/zzmyrep/KBCEN}.

\textbf{Evaluation Metrics:}
Following~\cite{zhang2024affective, chen2024tgca, zhang2024object}, we adopt \textit{accuracy} and \textit{F1} as our evaluation metrics to measure the performance of different methods for image sentiment analysis. 



\subsection{Experiment Result}
% We compare our model against the following baselines: AlexNet~\cite{krizhevsky2017imagenet}, VGG16~\cite{simonyan2014very}, ResNet101~\cite{he2016deep}, CECCN~\cite{ruan2024color}, EmoVIT~\cite{xie2024emovit}, WSCNet~\cite{yang2018weakly}, ECWA~\cite{deng2021emotion}, EECon~\cite{yang2023exploiting}, MAM~\cite{zhang2024affective} and TGCA-PVT~\cite{chen2024tgca}, and the overall results are summarized in Table~\ref{tab:cap1}.
We compare our model against several baselines, and the overall results are summarized in Table~\ref{tab:cap1}.
We observe that our model achieves the best performance in both accuracy and F1 metrics, significantly outperforming the previous models. 
This superior performance is mainly attributed to our effective utilization of metadata to enhance image sentiment analysis, as well as the exceptional capability of the unified sentiment transformer framework we developed. These results strongly demonstrate that our proposed method can bring encouraging performance for image sentiment analysis.

\setcounter{magicrownumbers}{0} 
\begin{table}[t]
\begin{center}
\caption{Ablation study of~\shortname~on FI dataset.} 
% \vspace{1mm}
\label{tab:cap2}
\resizebox{.9\linewidth}{!}
{
\begin{tabular}{lcc}
  \hline
  \textbf{Model} & \textbf{Accuracy} & \textbf{F1} \\
  \hline
  (\rownumber)~Ours (w/o vision) & 65.72\% & 64.54\% \\
  (\rownumber)~Ours (w/o text description) & 74.05\% & 72.58\% \\
  (\rownumber)~Ours (w/o object tag) & 77.45\% & 76.84\% \\
  (\rownumber)~Ours (w/o scene tag) & 78.47\% & 78.21\% \\
  \hline
  (\rownumber)~Ours (w/o unified embedding) & 76.41\% & 76.23\% \\
  (\rownumber)~Ours (w/o adaptive learning) & 76.83\% & 76.56\% \\
  (\rownumber)~Ours (w/o cross-modal fusion) & 76.85\% & 76.49\% \\
  \hline
  (\rownumber)~Ours  & \textbf{79.48\%} & \textbf{79.22\%} \\
  \hline
\end{tabular}
}
\end{center}
\vspace{-5mm}
\end{table}


\begin{figure}[t]
\centering
% \vspace{-2mm}
\includegraphics[width=0.42\textwidth]{fig/2dvisual-linux4-paper2.pdf}
\caption{Visualization of feature distribution on eight categories before (left) and after (right) model processing.}
% 
\label{fig:visualization}
\vspace{-5mm}
\end{figure}

\subsection{Ablation Performance}
In this subsection, we conduct an ablation study to examine which component is really important for performance improvement. The results are reported in Table~\ref{tab:cap2}.

For information utilization, we observe a significant decline in model performance when visual features are removed. Additionally, the performance of \shortname~decreases when different metadata are removed separately, which means that text description, object tag, and scene tag are all critical for image sentiment analysis.
Recalling the model architecture, we separately remove transformer layers of the unified representation module, the adaptive learning module, and the cross-modal fusion module, replacing them with MLPs of the same parameter scale.
In this way, we can observe varying degrees of decline in model performance, indicating that these modules are indispensable for our model to achieve better performance.

\subsection{Visualization}
% 


% % 开始使用minipage进行左右排列
% \begin{minipage}[t]{0.45\textwidth}  % 子图1宽度为45%
%     \centering
%     \includegraphics[width=\textwidth]{2dvisual.pdf}  % 插入图片
%     \captionof{figure}{Visualization of feature distribution.}  % 使用captionof添加图片标题
%     \label{fig:visualization}
% \end{minipage}


% \begin{figure}[t]
% \centering
% \vspace{-2mm}
% \includegraphics[width=0.45\textwidth]{fig/2dvisual.pdf}
% \caption{Visualization of feature distribution.}
% \label{fig:visualization}
% % \vspace{-4mm}
% \end{figure}

% \begin{figure}[t]
% \centering
% \vspace{-2mm}
% \includegraphics[width=0.45\textwidth]{fig/2dvisual-linux3-paper.pdf}
% \caption{Visualization of feature distribution.}
% \label{fig:visualization}
% % \vspace{-4mm}
% \end{figure}



\begin{figure}[tbp]   
\vspace{-4mm}
  \centering            
  \subfloat[Depth of adaptive learning layers]   
  {
    \label{fig:subfig1}\includegraphics[width=0.22\textwidth]{fig/fig_sensitivity-a5}
  }
  \subfloat[Depth of fusion layers]
  {
    % \label{fig:subfig2}\includegraphics[width=0.22\textwidth]{fig/fig_sensitivity-b2}
    \label{fig:subfig2}\includegraphics[width=0.22\textwidth]{fig/fig_sensitivity-b2-num.pdf}
  }
  \caption{Sensitivity study of \shortname~on different depth. }   
  \label{fig:fig_sensitivity}  
\vspace{-2mm}
\end{figure}

% \begin{figure}[htbp]
% \centerline{\includegraphics{2dvisual.pdf}}
% \caption{Visualization of feature distribution.}
% \label{fig:visualization}
% \end{figure}

% In Fig.~\ref{fig:visualization}, we use t-SNE~\cite{van2008visualizing} to reduce the dimension of data features for visualization, Figure in left represents the metadata features before model processing, the features are obtained by embedding through the CLIP model, and figure in right shows the features of the data after model processing, it can be observed that after the model processing, the data with different label categories fall in different regions in the space, therefore, we can conclude that the Therefore, we can conclude that the model can effectively utilize the information contained in the metadata and use it to guide the model for classification.

In Fig.~\ref{fig:visualization}, we use t-SNE~\cite{van2008visualizing} to reduce the dimension of data features for visualization.
The left figure shows metadata features before being processed by our model (\textit{i.e.}, embedded by CLIP), while the right shows the distribution of features after being processed by our model.
We can observe that after the model processing, data with the same label are closer to each other, while others are farther away.
Therefore, it shows that the model can effectively utilize the information contained in the metadata and use it to guide the classification process.

\subsection{Sensitivity Analysis}
% 
In this subsection, we conduct a sensitivity analysis to figure out the effect of different depth settings of adaptive learning layers and fusion layers. 
% In this subsection, we conduct a sensitivity analysis to figure out the effect of different depth settings on the model. 
% Fig.~\ref{fig:fig_sensitivity} presents the effect of different depth settings of adaptive learning layers and fusion layers. 
Taking Fig.~\ref{fig:fig_sensitivity} (a) as an example, the model performance improves with increasing depth, reaching the best performance at a depth of 4.
% Taking Fig.~\ref{fig:fig_sensitivity} (a) as an example, the performance of \shortname~improves with the increase of depth at first, reaching the best performance at a depth of 4.
When the depth continues to increase, the accuracy decreases to varying degrees.
Similar results can be observed in Fig.~\ref{fig:fig_sensitivity} (b).
Therefore, we set their depths to 4 and 6 respectively to achieve the best results.

% Through our experiments, we can observe that the effect of modifying these hyperparameters on the results of the experiments is very weak, and the surface model is not sensitive to the hyperparameters.


\subsection{Zero-shot Capability}
% 

% (1)~GCH~\cite{2010Analyzing} & 21.78\% & (5)~RA-DLNet~\cite{2020A} & 34.01\% \\ \hline
% (2)~WSCNet~\cite{2019WSCNet}  & 30.25\% & (6)~CECCN~\cite{ruan2024color} & 43.83\% \\ \hline
% (3)~PCNN~\cite{2015Robust} & 31.68\%  & (7)~EmoVIT~\cite{xie2024emovit} & 44.90\% \\ \hline
% (4)~AR~\cite{2018Visual} & 32.67\% & (8)~Ours (Zero-shot) & 47.83\% \\ \hline


\begin{table}[t]
\centering
\caption{Zero-shot capability of \shortname.}
\label{tab:cap3}
\resizebox{1\linewidth}{!}
{
\begin{tabular}{lc|lc}
\hline
\textbf{Model} & \textbf{Accuracy} & \textbf{Model} & \textbf{Accuracy} \\ \hline
(1)~WSCNet~\cite{2019WSCNet}  & 30.25\% & (5)~MAM~\cite{zhang2024affective} & 39.56\%  \\ \hline
(2)~AR~\cite{2018Visual} & 32.67\% & (6)~CECCN~\cite{ruan2024color} & 43.83\% \\ \hline
(3)~RA-DLNet~\cite{2020A} & 34.01\%  & (7)~EmoVIT~\cite{xie2024emovit} & 44.90\% \\ \hline
(4)~CDA~\cite{han2023boosting} & 38.64\% & (8)~Ours (Zero-shot) & 47.83\% \\ \hline
\end{tabular}
}
\vspace{-5mm}
\end{table}

% We use the model trained on the FI dataset to test on the artphoto dataset to verify the model's generalization ability as well as robustness to other distributed datasets.
% We can observe that the MESN model shows strong competitiveness in terms of accuracy when compared to other trained models, which suggests that the model has a good generalization ability in the OOD task.

To validate the model's generalization ability and robustness to other distributed datasets, we directly test the model trained on the FI dataset, without training on Artphoto. 
% As observed in Table 3, compared to other models trained on Artphoto, we achieve highly competitive zero-shot performance, indicating that the model has good generalization ability in out-of-distribution tasks.
From Table~\ref{tab:cap3}, we can observe that compared with other models trained on Artphoto, we achieve competitive zero-shot performance, which shows that the model has good generalization ability in out-of-distribution tasks.


%%%%%%%%%%%%
%  E2E     %
%%%%%%%%%%%%


\section{Conclusion}
In this paper, we introduced Wi-Chat, the first LLM-powered Wi-Fi-based human activity recognition system that integrates the reasoning capabilities of large language models with the sensing potential of wireless signals. Our experimental results on a self-collected Wi-Fi CSI dataset demonstrate the promising potential of LLMs in enabling zero-shot Wi-Fi sensing. These findings suggest a new paradigm for human activity recognition that does not rely on extensive labeled data. We hope future research will build upon this direction, further exploring the applications of LLMs in signal processing domains such as IoT, mobile sensing, and radar-based systems.

\section*{Limitations}
While our work represents the first attempt to leverage LLMs for processing Wi-Fi signals, it is a preliminary study focused on a relatively simple task: Wi-Fi-based human activity recognition. This choice allows us to explore the feasibility of LLMs in wireless sensing but also comes with certain limitations.

Our approach primarily evaluates zero-shot performance, which, while promising, may still lag behind traditional supervised learning methods in highly complex or fine-grained recognition tasks. Besides, our study is limited to a controlled environment with a self-collected dataset, and the generalizability of LLMs to diverse real-world scenarios with varying Wi-Fi conditions, environmental interference, and device heterogeneity remains an open question.

Additionally, we have yet to explore the full potential of LLMs in more advanced Wi-Fi sensing applications, such as fine-grained gesture recognition, occupancy detection, and passive health monitoring. Future work should investigate the scalability of LLM-based approaches, their robustness to domain shifts, and their integration with multimodal sensing techniques in broader IoT applications.


% Bibliography entries for the entire Anthology, followed by custom entries
%\bibliography{anthology,custom}
% Custom bibliography entries only
\bibliography{main}
\newpage
\appendix

\section{Experiment prompts}
\label{sec:prompt}
The prompts used in the LLM experiments are shown in the following Table~\ref{tab:prompts}.

\definecolor{titlecolor}{rgb}{0.9, 0.5, 0.1}
\definecolor{anscolor}{rgb}{0.2, 0.5, 0.8}
\definecolor{labelcolor}{HTML}{48a07e}
\begin{table*}[h]
	\centering
	
 % \vspace{-0.2cm}
	
	\begin{center}
		\begin{tikzpicture}[
				chatbox_inner/.style={rectangle, rounded corners, opacity=0, text opacity=1, font=\sffamily\scriptsize, text width=5in, text height=9pt, inner xsep=6pt, inner ysep=6pt},
				chatbox_prompt_inner/.style={chatbox_inner, align=flush left, xshift=0pt, text height=11pt},
				chatbox_user_inner/.style={chatbox_inner, align=flush left, xshift=0pt},
				chatbox_gpt_inner/.style={chatbox_inner, align=flush left, xshift=0pt},
				chatbox/.style={chatbox_inner, draw=black!25, fill=gray!7, opacity=1, text opacity=0},
				chatbox_prompt/.style={chatbox, align=flush left, fill=gray!1.5, draw=black!30, text height=10pt},
				chatbox_user/.style={chatbox, align=flush left},
				chatbox_gpt/.style={chatbox, align=flush left},
				chatbox2/.style={chatbox_gpt, fill=green!25},
				chatbox3/.style={chatbox_gpt, fill=red!20, draw=black!20},
				chatbox4/.style={chatbox_gpt, fill=yellow!30},
				labelbox/.style={rectangle, rounded corners, draw=black!50, font=\sffamily\scriptsize\bfseries, fill=gray!5, inner sep=3pt},
			]
											
			\node[chatbox_user] (q1) {
				\textbf{System prompt}
				\newline
				\newline
				You are a helpful and precise assistant for segmenting and labeling sentences. We would like to request your help on curating a dataset for entity-level hallucination detection.
				\newline \newline
                We will give you a machine generated biography and a list of checked facts about the biography. Each fact consists of a sentence and a label (True/False). Please do the following process. First, breaking down the biography into words. Second, by referring to the provided list of facts, merging some broken down words in the previous step to form meaningful entities. For example, ``strategic thinking'' should be one entity instead of two. Third, according to the labels in the list of facts, labeling each entity as True or False. Specifically, for facts that share a similar sentence structure (\eg, \textit{``He was born on Mach 9, 1941.''} (\texttt{True}) and \textit{``He was born in Ramos Mejia.''} (\texttt{False})), please first assign labels to entities that differ across atomic facts. For example, first labeling ``Mach 9, 1941'' (\texttt{True}) and ``Ramos Mejia'' (\texttt{False}) in the above case. For those entities that are the same across atomic facts (\eg, ``was born'') or are neutral (\eg, ``he,'' ``in,'' and ``on''), please label them as \texttt{True}. For the cases that there is no atomic fact that shares the same sentence structure, please identify the most informative entities in the sentence and label them with the same label as the atomic fact while treating the rest of the entities as \texttt{True}. In the end, output the entities and labels in the following format:
                \begin{itemize}[nosep]
                    \item Entity 1 (Label 1)
                    \item Entity 2 (Label 2)
                    \item ...
                    \item Entity N (Label N)
                \end{itemize}
                % \newline \newline
                Here are two examples:
                \newline\newline
                \textbf{[Example 1]}
                \newline
                [The start of the biography]
                \newline
                \textcolor{titlecolor}{Marianne McAndrew is an American actress and singer, born on November 21, 1942, in Cleveland, Ohio. She began her acting career in the late 1960s, appearing in various television shows and films.}
                \newline
                [The end of the biography]
                \newline \newline
                [The start of the list of checked facts]
                \newline
                \textcolor{anscolor}{[Marianne McAndrew is an American. (False); Marianne McAndrew is an actress. (True); Marianne McAndrew is a singer. (False); Marianne McAndrew was born on November 21, 1942. (False); Marianne McAndrew was born in Cleveland, Ohio. (False); She began her acting career in the late 1960s. (True); She has appeared in various television shows. (True); She has appeared in various films. (True)]}
                \newline
                [The end of the list of checked facts]
                \newline \newline
                [The start of the ideal output]
                \newline
                \textcolor{labelcolor}{[Marianne McAndrew (True); is (True); an (True); American (False); actress (True); and (True); singer (False); , (True); born (True); on (True); November 21, 1942 (False); , (True); in (True); Cleveland, Ohio (False); . (True); She (True); began (True); her (True); acting career (True); in (True); the late 1960s (True); , (True); appearing (True); in (True); various (True); television shows (True); and (True); films (True); . (True)]}
                \newline
                [The end of the ideal output]
				\newline \newline
                \textbf{[Example 2]}
                \newline
                [The start of the biography]
                \newline
                \textcolor{titlecolor}{Doug Sheehan is an American actor who was born on April 27, 1949, in Santa Monica, California. He is best known for his roles in soap operas, including his portrayal of Joe Kelly on ``General Hospital'' and Ben Gibson on ``Knots Landing.''}
                \newline
                [The end of the biography]
                \newline \newline
                [The start of the list of checked facts]
                \newline
                \textcolor{anscolor}{[Doug Sheehan is an American. (True); Doug Sheehan is an actor. (True); Doug Sheehan was born on April 27, 1949. (True); Doug Sheehan was born in Santa Monica, California. (False); He is best known for his roles in soap operas. (True); He portrayed Joe Kelly. (True); Joe Kelly was in General Hospital. (True); General Hospital is a soap opera. (True); He portrayed Ben Gibson. (True); Ben Gibson was in Knots Landing. (True); Knots Landing is a soap opera. (True)]}
                \newline
                [The end of the list of checked facts]
                \newline \newline
                [The start of the ideal output]
                \newline
                \textcolor{labelcolor}{[Doug Sheehan (True); is (True); an (True); American (True); actor (True); who (True); was born (True); on (True); April 27, 1949 (True); in (True); Santa Monica, California (False); . (True); He (True); is (True); best known (True); for (True); his roles in soap operas (True); , (True); including (True); in (True); his portrayal (True); of (True); Joe Kelly (True); on (True); ``General Hospital'' (True); and (True); Ben Gibson (True); on (True); ``Knots Landing.'' (True)]}
                \newline
                [The end of the ideal output]
				\newline \newline
				\textbf{User prompt}
				\newline
				\newline
				[The start of the biography]
				\newline
				\textcolor{magenta}{\texttt{\{BIOGRAPHY\}}}
				\newline
				[The ebd of the biography]
				\newline \newline
				[The start of the list of checked facts]
				\newline
				\textcolor{magenta}{\texttt{\{LIST OF CHECKED FACTS\}}}
				\newline
				[The end of the list of checked facts]
			};
			\node[chatbox_user_inner] (q1_text) at (q1) {
				\textbf{System prompt}
				\newline
				\newline
				You are a helpful and precise assistant for segmenting and labeling sentences. We would like to request your help on curating a dataset for entity-level hallucination detection.
				\newline \newline
                We will give you a machine generated biography and a list of checked facts about the biography. Each fact consists of a sentence and a label (True/False). Please do the following process. First, breaking down the biography into words. Second, by referring to the provided list of facts, merging some broken down words in the previous step to form meaningful entities. For example, ``strategic thinking'' should be one entity instead of two. Third, according to the labels in the list of facts, labeling each entity as True or False. Specifically, for facts that share a similar sentence structure (\eg, \textit{``He was born on Mach 9, 1941.''} (\texttt{True}) and \textit{``He was born in Ramos Mejia.''} (\texttt{False})), please first assign labels to entities that differ across atomic facts. For example, first labeling ``Mach 9, 1941'' (\texttt{True}) and ``Ramos Mejia'' (\texttt{False}) in the above case. For those entities that are the same across atomic facts (\eg, ``was born'') or are neutral (\eg, ``he,'' ``in,'' and ``on''), please label them as \texttt{True}. For the cases that there is no atomic fact that shares the same sentence structure, please identify the most informative entities in the sentence and label them with the same label as the atomic fact while treating the rest of the entities as \texttt{True}. In the end, output the entities and labels in the following format:
                \begin{itemize}[nosep]
                    \item Entity 1 (Label 1)
                    \item Entity 2 (Label 2)
                    \item ...
                    \item Entity N (Label N)
                \end{itemize}
                % \newline \newline
                Here are two examples:
                \newline\newline
                \textbf{[Example 1]}
                \newline
                [The start of the biography]
                \newline
                \textcolor{titlecolor}{Marianne McAndrew is an American actress and singer, born on November 21, 1942, in Cleveland, Ohio. She began her acting career in the late 1960s, appearing in various television shows and films.}
                \newline
                [The end of the biography]
                \newline \newline
                [The start of the list of checked facts]
                \newline
                \textcolor{anscolor}{[Marianne McAndrew is an American. (False); Marianne McAndrew is an actress. (True); Marianne McAndrew is a singer. (False); Marianne McAndrew was born on November 21, 1942. (False); Marianne McAndrew was born in Cleveland, Ohio. (False); She began her acting career in the late 1960s. (True); She has appeared in various television shows. (True); She has appeared in various films. (True)]}
                \newline
                [The end of the list of checked facts]
                \newline \newline
                [The start of the ideal output]
                \newline
                \textcolor{labelcolor}{[Marianne McAndrew (True); is (True); an (True); American (False); actress (True); and (True); singer (False); , (True); born (True); on (True); November 21, 1942 (False); , (True); in (True); Cleveland, Ohio (False); . (True); She (True); began (True); her (True); acting career (True); in (True); the late 1960s (True); , (True); appearing (True); in (True); various (True); television shows (True); and (True); films (True); . (True)]}
                \newline
                [The end of the ideal output]
				\newline \newline
                \textbf{[Example 2]}
                \newline
                [The start of the biography]
                \newline
                \textcolor{titlecolor}{Doug Sheehan is an American actor who was born on April 27, 1949, in Santa Monica, California. He is best known for his roles in soap operas, including his portrayal of Joe Kelly on ``General Hospital'' and Ben Gibson on ``Knots Landing.''}
                \newline
                [The end of the biography]
                \newline \newline
                [The start of the list of checked facts]
                \newline
                \textcolor{anscolor}{[Doug Sheehan is an American. (True); Doug Sheehan is an actor. (True); Doug Sheehan was born on April 27, 1949. (True); Doug Sheehan was born in Santa Monica, California. (False); He is best known for his roles in soap operas. (True); He portrayed Joe Kelly. (True); Joe Kelly was in General Hospital. (True); General Hospital is a soap opera. (True); He portrayed Ben Gibson. (True); Ben Gibson was in Knots Landing. (True); Knots Landing is a soap opera. (True)]}
                \newline
                [The end of the list of checked facts]
                \newline \newline
                [The start of the ideal output]
                \newline
                \textcolor{labelcolor}{[Doug Sheehan (True); is (True); an (True); American (True); actor (True); who (True); was born (True); on (True); April 27, 1949 (True); in (True); Santa Monica, California (False); . (True); He (True); is (True); best known (True); for (True); his roles in soap operas (True); , (True); including (True); in (True); his portrayal (True); of (True); Joe Kelly (True); on (True); ``General Hospital'' (True); and (True); Ben Gibson (True); on (True); ``Knots Landing.'' (True)]}
                \newline
                [The end of the ideal output]
				\newline \newline
				\textbf{User prompt}
				\newline
				\newline
				[The start of the biography]
				\newline
				\textcolor{magenta}{\texttt{\{BIOGRAPHY\}}}
				\newline
				[The ebd of the biography]
				\newline \newline
				[The start of the list of checked facts]
				\newline
				\textcolor{magenta}{\texttt{\{LIST OF CHECKED FACTS\}}}
				\newline
				[The end of the list of checked facts]
			};
		\end{tikzpicture}
        \caption{GPT-4o prompt for labeling hallucinated entities.}\label{tb:gpt-4-prompt}
	\end{center}
\vspace{-0cm}
\end{table*}
% \section{Full Experiment Results}
% \begin{table*}[th]
    \centering
    \small
    \caption{Classification Results}
    \begin{tabular}{lcccc}
        \toprule
        \textbf{Method} & \textbf{Accuracy} & \textbf{Precision} & \textbf{Recall} & \textbf{F1-score} \\
        \midrule
        \multicolumn{5}{c}{\textbf{Zero Shot}} \\
                Zero-shot E-eyes & 0.26 & 0.26 & 0.27 & 0.26 \\
        Zero-shot CARM & 0.24 & 0.24 & 0.24 & 0.24 \\
                Zero-shot SVM & 0.27 & 0.28 & 0.28 & 0.27 \\
        Zero-shot CNN & 0.23 & 0.24 & 0.23 & 0.23 \\
        Zero-shot RNN & 0.26 & 0.26 & 0.26 & 0.26 \\
DeepSeek-0shot & 0.54 & 0.61 & 0.54 & 0.52 \\
DeepSeek-0shot-COT & 0.33 & 0.24 & 0.33 & 0.23 \\
DeepSeek-0shot-Knowledge & 0.45 & 0.46 & 0.45 & 0.44 \\
Gemma2-0shot & 0.35 & 0.22 & 0.38 & 0.27 \\
Gemma2-0shot-COT & 0.36 & 0.22 & 0.36 & 0.27 \\
Gemma2-0shot-Knowledge & 0.32 & 0.18 & 0.34 & 0.20 \\
GPT-4o-mini-0shot & 0.48 & 0.53 & 0.48 & 0.41 \\
GPT-4o-mini-0shot-COT & 0.33 & 0.50 & 0.33 & 0.38 \\
GPT-4o-mini-0shot-Knowledge & 0.49 & 0.31 & 0.49 & 0.36 \\
GPT-4o-0shot & 0.62 & 0.62 & 0.47 & 0.42 \\
GPT-4o-0shot-COT & 0.29 & 0.45 & 0.29 & 0.21 \\
GPT-4o-0shot-Knowledge & 0.44 & 0.52 & 0.44 & 0.39 \\
LLaMA-0shot & 0.32 & 0.25 & 0.32 & 0.24 \\
LLaMA-0shot-COT & 0.12 & 0.25 & 0.12 & 0.09 \\
LLaMA-0shot-Knowledge & 0.32 & 0.25 & 0.32 & 0.28 \\
Mistral-0shot & 0.19 & 0.23 & 0.19 & 0.10 \\
Mistral-0shot-Knowledge & 0.21 & 0.40 & 0.21 & 0.11 \\
        \midrule
        \multicolumn{5}{c}{\textbf{4 Shot}} \\
GPT-4o-mini-4shot & 0.58 & 0.59 & 0.58 & 0.53 \\
GPT-4o-mini-4shot-COT & 0.57 & 0.53 & 0.57 & 0.50 \\
GPT-4o-mini-4shot-Knowledge & 0.56 & 0.51 & 0.56 & 0.47 \\
GPT-4o-4shot & 0.77 & 0.84 & 0.77 & 0.73 \\
GPT-4o-4shot-COT & 0.63 & 0.76 & 0.63 & 0.53 \\
GPT-4o-4shot-Knowledge & 0.72 & 0.82 & 0.71 & 0.66 \\
LLaMA-4shot & 0.29 & 0.24 & 0.29 & 0.21 \\
LLaMA-4shot-COT & 0.20 & 0.30 & 0.20 & 0.13 \\
LLaMA-4shot-Knowledge & 0.15 & 0.23 & 0.13 & 0.13 \\
Mistral-4shot & 0.02 & 0.02 & 0.02 & 0.02 \\
Mistral-4shot-Knowledge & 0.21 & 0.27 & 0.21 & 0.20 \\
        \midrule
        
        \multicolumn{5}{c}{\textbf{Suprevised}} \\
        SVM & 0.94 & 0.92 & 0.91 & 0.91 \\
        CNN & 0.98 & 0.98 & 0.97 & 0.97 \\
        RNN & 0.99 & 0.99 & 0.99 & 0.99 \\
        % \midrule
        % \multicolumn{5}{c}{\textbf{Conventional Wi-Fi-based Human Activity Recognition Systems}} \\
        E-eyes & 1.00 & 1.00 & 1.00 & 1.00 \\
        CARM & 0.98 & 0.98 & 0.98 & 0.98 \\
\midrule
 \multicolumn{5}{c}{\textbf{Vision Models}} \\
           Zero-shot SVM & 0.26 & 0.25 & 0.25 & 0.25 \\
        Zero-shot CNN & 0.26 & 0.25 & 0.26 & 0.26 \\
        Zero-shot RNN & 0.28 & 0.28 & 0.29 & 0.28 \\
        SVM & 0.99 & 0.99 & 0.99 & 0.99 \\
        CNN & 0.98 & 0.99 & 0.98 & 0.98 \\
        RNN & 0.98 & 0.99 & 0.98 & 0.98 \\
GPT-4o-mini-Vision & 0.84 & 0.85 & 0.84 & 0.84 \\
GPT-4o-mini-Vision-COT & 0.90 & 0.91 & 0.90 & 0.90 \\
GPT-4o-Vision & 0.74 & 0.82 & 0.74 & 0.73 \\
GPT-4o-Vision-COT & 0.70 & 0.83 & 0.70 & 0.68 \\
LLaMA-Vision & 0.20 & 0.23 & 0.20 & 0.09 \\
LLaMA-Vision-Knowledge & 0.22 & 0.05 & 0.22 & 0.08 \\

        \bottomrule
    \end{tabular}
    \label{full}
\end{table*}




\end{document}


\end{document}