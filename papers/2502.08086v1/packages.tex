\usepackage{smartdiagram}


\usepackage[activate={true,nocompatibility},
            final,
            tracking=true,
            kerning=true,
            spacing=true,
            factor=1100,
            stretch=10,
            shrink=10]{} %microtype

% Text and fonts
% \usepackage[utf8]{inputenc}
\usepackage[english]{babel}
\usepackage{blindtext}
\usepackage{helvet}
% \usepackage[urw-garamond]{mathdesign}
\usepackage[T1]{fontenc}
% \usepackage[expert]{mathdesign} % math font
%\usepackage{fourier}

%\usepackage{xcolor}
% Tables
\usepackage{booktabs}
\usepackage{tabu}

% Math
\usepackage{amsmath}
\usepackage{amssymb}
\usepackage{amsthm}
%\usepackage{algorithmic}
\usepackage{algorithm}
\usepackage{siunitx}
\usepackage{amsmath,amsfonts,amsthm,mathrsfs, mathtools}
\usepackage{multirow}

% Graphics Package
\usepackage{marvosym}
\usepackage{graphicx}
\usepackage{pgfpages}
\graphicspath{{images/}{../images/}}
\usepackage{pgffor}
\usepackage{xcolor}
\usepackage{colortbl}
\usepackage{pgfplots}
\usepackage{tikz}
\usepackage{tikzsymbols}
%\usepackage{tikz-dsp}
\usetikzlibrary{automata,positioning,fadings,shadows, shapes.arrows, shapes.geometric}
\usetikzlibrary{matrix, chains, patterns}
\usetikzlibrary{arrows,shapes.gates.logic.US,shapes.gates.logic.IEC,calc,automata}
\usepgfplotslibrary{groupplots}


\usepackage{pifont}
\newcommand{\cmark}{\ding{51}}%
\newcommand{\xmark}{\ding{55}}%
%\usepackage[dvipsnames]{xcolor}

\newcommand{\xCross}{$\mathbin{\tikz [x=1.4ex,y=1.4ex,line width=.2ex, red] \draw (0,0) -- (1,1) (0,1) -- (1,0);}$}%



\definecolor{Paired-2}{RGB}{166,206,227}
\definecolor{Paired-1}{RGB}{31,120,180}
\definecolor{Paired-4}{RGB}{178,223,138}
\definecolor{Paired-3}{RGB}{51,160,44}
\definecolor{Paired-6}{RGB}{251,154,153}
\definecolor{Paired-5}{RGB}{227,26,28}
\definecolor{Paired-8}{RGB}{253,191,111}
\definecolor{Paired-7}{RGB}{255,127,0}
\definecolor{Paired-10}{RGB}{202,178,214}
\definecolor{Paired-9}{RGB}{106,61,154}
\definecolor{Paired-12}{RGB}{255,255,153}
\definecolor{Paired-11}{RGB}{177,89,40}


\usepackage{tikz-timing}[2009/05/15]

% Emoji Package
\usepackage{scalerel} % needed package to scale the pdf-images perfectly
\def\emojismiley{\scalerel*{\includegraphics{1F60C.pdf}}{O}}
\def\emojifrowny{\scalerel*{\includegraphics{1F61E.pdf}}{0}}


%===============================TIKZ CONFIGURATIONS===========================%
    \pgfdeclarelayer{background}
    \pgfdeclarelayer{foreground}
    \pgfsetlayers{background,main,foreground}
    \tikzstyle{vblock} = [draw,
                          fill=dateorange!30,
                          rotate=90,
                          anchor=north west,
                          minimum width=3.5cm,
                          minimum height=.5cm,
                         rounded corners=.05cm,
                          ]

    \tikzstyle{tile} = [draw,
                        fill=dateblue!30,
                        anchor=south west,
                        minimum width=3.5cm,
                        minimum height=.5cm,
                        rounded corners=.05cm,
                        ]

    \tikzstyle{nnlayer} = [draw,
                        fill=dateblue!30,
                        anchor=south west,
                        minimum width=1cm,
                        minimum height=.5cm,
                        rounded corners=.05cm,
                        ]

    \tikzstyle{petile} = [draw,
                        fill=dateorange!30,
                        anchor=south west,
                        minimum width=1.5cm,
                        minimum height=.5cm,
                        rounded corners=.05cm,
                        ]

    \tikzstyle{arrow} = [->,
                         >=stealth,
                         thick,
                         rounded corners=.1cm,
                         color=datemagenta,
                         ]

    \tikzstyle{arrowg} = [->,
                          >=stealth,
                          thick,
                          rounded corners=.1cm,
                          color=lightgray,
                          ]
    \tikzset{%
    table/.style={%
      matrix of nodes,
      row sep=-\pgflinewidth,
      column sep=-\pgflinewidth,
      nodes={rectangle,draw=black,text width=1.25ex,align=center},
      text depth=0.25ex,
      text height=1ex,
      nodes in empty cells
      },
    texto/.style={font=\footnotesize\sffamily},
    title/.style={font=\small\sffamily}
    }


    % Tikz functions to draw gaussian
    \pgfmathdeclarefunction{gauss}{2}{%
        \pgfmathparse{1/(#2*sqrt(2*pi))*exp(-((x-#1)^2)/(2*#2^2))}% chktex 36
    }

    \pgfmathdeclarefunction{gaussprune}{2}{%
        \pgfmathparse{or(x<-1,x>1)/(#2*sqrt(2*pi))*exp(-((x-#1)^2)/(2*#2^2))}% chktex 36
    }


%=============================================================================%



% Floats
\usepackage{stfloats}

\usepackage{perpage}
\usepackage{hyperref}
\usepackage{csquotes}

%================================= Citations ==================================%
%\usepackage[style=english]{csquotes}
%\usepackage[backend=biber, style=authoryear-icomp]{biblatex}
%\addbibresource{pres.bib}

%=============================== Beamer Options ===============================%
\usepackage{pgfpages}
% \usepackage{handoutWithNotes}
% \setbeameroption{show notes on second screen}
% \usetheme{Pittsburgh}
% \usecolortheme{beaver}

\makeatother

% \setbeamertemplate{section in toc}{%
%   {\color{darkred}\inserttocsectionnumber.}~\inserttocsection}
% \setbeamercolor{subsection in toc}{bg=white,fg=structure}
% \setbeamertemplate{subsection in toc}{%
%   \hspace{2.2em}{\color{structure}\rule[0.3ex]{3pt}{3pt}}~\inserttocsubsection\par}

\definecolor{dateblue}{RGB}{36,80,117}
\definecolor{datemagenta}{RGB}{183,50,101}
\definecolor{dateorange}{RGB}{198,84,54}
\definecolor{datebrown}{RGB}{198,140,54}
\definecolor{dateyellow}{RGB}{198,178,54}








\tikzfading[name=arrowfading, top color=transparent!0, bottom color=transparent!95]
\tikzset{arrowfill/.style={#1, general shadow={fill=black, shadow yshift=-0.8ex, path fading=arrowfading}}}
\tikzset{arrowstyle/.style n args={3}{draw=#2,arrowfill={#3}, single arrow,minimum height=#1, single arrow,
single arrow head extend=.3cm,}}

\NewDocumentCommand{\tikzfancyarrow}{O{2cm} O{dateblue} O{top color=dateblue!20, bottom color=dateblue} m}{
\tikz[baseline=-0.5ex]\node [arrowstyle={#1}{#2}{#3}] {#4};
}





\usepackage{circuitikz}





%%%%%%%%%%%%%%%%%%%%%%%%%%%%%%%%%%%%%%%%%%%%%%%%%%%%%%%%%%%%%%%%%%%

\newcommand{\implicantsol}[3][0]{
    \draw[rounded corners=3pt, fill=#3, opacity=0.3] ($(#2.north west)+(135:#1)$) rectangle ($(#2.south east)+(-45:#1)$);
    }


%internal group
%#1 - Optional. Space between node and grouping line. Default=0
%#2 - top left node
%#3 - bottom right node
%#4 - filling color
\newcommand{\implicant}[4][0]{
    \draw[rounded corners=3pt, fill=#4, opacity=0.3] ($(#2.north west)+(135:#1)$) rectangle ($(#3.south east)+(-45:#1)$);
    }

%group lateral borders
%#1 - Optional. Space between node and grouping line. Default=0
%#2 - top left node
%#3 - bottom right node
%#4 - filling color
\newcommand{\implicantcostats}[4][0]{
    \draw[rounded corners=3pt, fill=#4, opacity=0.3] ($(rf.east |- #2.north)+(90:#1)$)-| ($(#2.east)+(0:#1)$) |- ($(rf.east |- #3.south)+(-90:#1)$);
    \draw[rounded corners=3pt, fill=#4, opacity=0.3] ($(cf.west |- #2.north)+(90:#1)$) -| ($(#3.west)+(180:#1)$) |- ($(cf.west |- #3.south)+(-90:#1)$);
}

%group top-bottom borders
%#1 - Optional. Space between node and grouping line. Default=0
%#2 - top left node
%#3 - bottom right node
%#4 - filling color
\newcommand{\implicantdaltbaix}[4][0]{
    \draw[rounded corners=3pt, fill=#4, opacity=0.3] ($(cf.south -| #2.west)+(180:#1)$) |- ($(#2.south)+(-90:#1)$) -| ($(cf.south -| #3.east)+(0:#1)$);
    \draw[rounded corners=3pt, fill=#4, opacity=0.3] ($(rf.north -| #2.west)+(180:#1)$) |- ($(#3.north)+(90:#1)$) -| ($(rf.north -| #3.east)+(0:#1)$);
}

%group corners
%#1 - Optional. Space between node and grouping line. Default=0
%#2 - filling color
\newcommand{\implicantcantons}[2][0]{
    \draw[rounded corners=3pt, opacity=.3] ($(rf.east |- 0.south)+(-90:#1)$) -| ($(0.east |- cf.south)+(0:#1)$);
    \draw[rounded corners=3pt, opacity=.3] ($(rf.east |- 8.north)+(90:#1)$) -| ($(8.east |- rf.north)+(0:#1)$);
    \draw[rounded corners=3pt, opacity=.3] ($(cf.west |- 2.south)+(-90:#1)$) -| ($(2.west |- cf.south)+(180:#1)$);
    \draw[rounded corners=3pt, opacity=.3] ($(cf.west |- 10.north)+(90:#1)$) -| ($(10.west |- rf.north)+(180:#1)$);
    \fill[rounded corners=3pt, fill=#2, opacity=.3] ($(rf.east |- 0.south)+(-90:#1)$) -|  ($(0.east |- cf.south)+(0:#1)$) [sharp corners] ($(rf.east |- 0.south)+(-90:#1)$) |-  ($(0.east |- cf.south)+(0:#1)$) ;
    \fill[rounded corners=3pt, fill=#2, opacity=.3] ($(rf.east |- 8.north)+(90:#1)$) -| ($(8.east |- rf.north)+(0:#1)$) [sharp corners] ($(rf.east |- 8.north)+(90:#1)$) |- ($(8.east |- rf.north)+(0:#1)$) ;
    \fill[rounded corners=3pt, fill=#2, opacity=.3] ($(cf.west |- 2.south)+(-90:#1)$) -| ($(2.west |- cf.south)+(180:#1)$) [sharp corners]($(cf.west |- 2.south)+(-90:#1)$) |- ($(2.west |- cf.south)+(180:#1)$) ;
    \fill[rounded corners=3pt, fill=#2, opacity=.3] ($(cf.west |- 10.north)+(90:#1)$) -| ($(10.west |- rf.north)+(180:#1)$) [sharp corners] ($(cf.west |- 10.north)+(90:#1)$) |- ($(10.west |- rf.north)+(180:#1)$) ;
}

%Empty Karnaugh map 4x4
\newenvironment{Karnaugh}%
{
\begin{tikzpicture}[baseline=(current bounding box.north),scale=0.8]
\draw (0,0) grid (4,4);
\draw (0,4) -- node [pos=0.7,above right,anchor=south west] {$x_3x_4$} node [pos=0.7,below left,anchor=north east] {$x_1x_2$} ++(135:1);
%
\matrix (mapa) [matrix of nodes,
        column sep={0.8cm,between origins},
        row sep={0.8cm,between origins},
        every node/.style={minimum size=0.3mm},
        anchor=8.center,
        ampersand replacement=\&] at (0.5,0.5)
{
                       \& |(c00)| 00         \& |(c01)| 01         \& |(c11)| 11         \& |(c10)| 10         \& |(cf)| \phantom{00} \\
|(r00)| 00             \& |(0)|  \phantom{0} \& |(1)|  \phantom{0} \& |(3)|  \phantom{0} \& |(2)|  \phantom{0} \&                     \\
|(r01)| 01             \& |(4)|  \phantom{0} \& |(5)|  \phantom{0} \& |(7)|  \phantom{0} \& |(6)|  \phantom{0} \&                     \\
|(r11)| 11             \& |(12)| \phantom{0} \& |(13)| \phantom{0} \& |(15)| \phantom{0} \& |(14)| \phantom{0} \&                     \\
|(r10)| 10             \& |(8)|  \phantom{0} \& |(9)|  \phantom{0} \& |(11)| \phantom{0} \& |(10)| \phantom{0} \&                     \\
|(rf) | \phantom{00}   \&                    \&                    \&                    \&                    \&                     \\
};
}%
{
\end{tikzpicture}
}

%Empty Karnaugh map 2x4
\newenvironment{Karnaughvuit}%
{
\begin{tikzpicture}[baseline=(current bounding box.north),scale=0.8]
\draw (0,0) grid (4,2);
\draw (0,2) -- node [pos=0.7,above right,anchor=south west] {$x_2x_3$} node [pos=0.7,below left,anchor=north east] {$x_1$} ++(135:1);
%
\matrix (mapa) [matrix of nodes,
        column sep={0.8cm,between origins},
        row sep={0.8cm,between origins},
        every node/.style={minimum size=0.3mm},
        anchor=4.center,
        ampersand replacement=\&] at (0.5,0.5)
{
                      \& |(c00)| 00         \& |(c01)| 01         \& |(c11)| 11         \& |(c10)| 10         \& |(cf)| \phantom{00} \\
|(r00)| 0             \& |(0)|  \phantom{0} \& |(1)|  \phantom{0} \& |(3)|  \phantom{0} \& |(2)|  \phantom{0} \&                     \\
|(r01)| 1             \& |(4)|  \phantom{0} \& |(5)|  \phantom{0} \& |(7)|  \phantom{0} \& |(6)|  \phantom{0} \&                     \\
|(rf) | \phantom{00}  \&                    \&                    \&                    \&                    \&                     \\
};
}%
{
\end{tikzpicture}
}




\newenvironment{Karnaughvuit1}%
{
\begin{tikzpicture}[baseline=(current bounding box.north),scale=0.8]
\draw (0,0) grid (4,2);
\draw (0,2) -- node [pos=0.7,above right,anchor=south west] {$x_1x_2$} node [pos=0.7,below left,anchor=north east] {$x_5$} ++(135:1);
%
\matrix (mapa) [matrix of nodes,
        column sep={0.8cm,between origins},
        row sep={0.8cm,between origins},
        every node/.style={minimum size=0.3mm},
        anchor=4.center,
        ampersand replacement=\&] at (0.5,0.5)
{
                      \& |(c00)| 00         \& |(c01)| 01         \& |(c11)| 11         \& |(c10)| 10         \& |(cf)| \phantom{00} \\
|(r00)| 0             \& |(0)|  \phantom{0} \& |(1)|  \phantom{0} \& |(3)|  \phantom{0} \& |(2)|  \phantom{0} \&                     \\
|(r01)| 1             \& |(4)|  \phantom{0} \& |(5)|  \phantom{0} \& |(7)|  \phantom{0} \& |(6)|  \phantom{0} \&                     \\
|(rf) | \phantom{00}  \&                    \&                    \&                    \&                    \&                     \\
};
}%
{
\end{tikzpicture}
}

%Empty Karnaugh map 2x4
\newenvironment{Karnaugh4}%
{
\begin{tikzpicture}[baseline=(current bounding box.north),scale=0.8]
\draw (0,0) grid (4,1);
\draw (0,1) -- node [pos=0.7,above right,anchor=south west] {$x_1x_2$} node [pos=0.7,below left,anchor=north east] {} ++(135:1);
%
\matrix (mapa) [matrix of nodes,
        column sep={0.8cm,between origins},
        row sep={0.7cm,between origins},
        every node/.style={minimum size=0.3mm},
        anchor=5.center,
        ampersand replacement=\&] at (0.5,0.8)
{
 \& |(c00)| 00         \& |(c01)| 01         \& |(c10)| 10         \& |(c11)| 11         \& |(cf)| \phantom{00} \\
 |(rf) | \phantom{00}   \& |(0)|  \phantom{0} \& |(1)|  \phantom{0} \& |(2)|  \phantom{0} \& |(3)|  \phantom{0} \&                     \\
};
}%
{
\end{tikzpicture}
}
%Empty Karnaugh map 2x4
\newenvironment{Karnaugh42}%
{
\begin{tikzpicture}[baseline=(current bounding box.north),scale=0.8]
\draw (0,0) grid (4,1);
\draw (0,1) -- node [pos=0.7,above right,anchor=south west] {$x_1x_2$} node [pos=0.7,below left,anchor=north east] {} ++(135:1);
%
\matrix (mapa) [matrix of nodes,
        column sep={0.8cm,between origins},
        row sep={0.7cm,between origins},
        every node/.style={minimum size=0.3mm},
        anchor=5.center,
        ampersand replacement=\&] at (0.5,0.8)
{
 \& |(c00)| 00         \& |(c01)| 01         \& |(c11)| 11         \& |(c10)| 10         \& |(cf)| \phantom{00} \\
 |(rf) | \phantom{00}   \& |(0)|  \phantom{0} \& |(1)|  \phantom{0} \& |(3)|  \phantom{0} \& |(2)|  \phantom{0} \&                     \\
};
}%
{
\end{tikzpicture}
}

%Empty Karnaugh map 2x2
\newenvironment{Karnaughquatre}%
{
\begin{tikzpicture}[baseline=(current bounding box.north),scale=0.8]
\draw (0,0) grid (2,2);
\draw (0,2) -- node [pos=0.7,above right,anchor=south west] {$x_2$} node [pos=0.7,below left,anchor=north east] {$x_1$} ++(135:1);
%
\matrix (mapa) [matrix of nodes,
        column sep={0.8cm,between origins},
        row sep={0.8cm,between origins},
        every node/.style={minimum size=0.3mm},
        anchor=2.center,
        ampersand replacement=\&] at (0.5,0.5)
{
          \& |(c00)| 0          \& |(c01)| 1  \\
|(r00)| 0 \& |(0)|  \phantom{0} \& |(1)|  \phantom{0} \\
|(r01)| 1 \& |(2)|  \phantom{0} \& |(3)|  \phantom{0} \\
};
}%
{
\end{tikzpicture}
}

%Defines 8 or 16 values (0,1,X)
\newcommand{\contingut}[1]{%
\foreach \x [count=\xi from 0]  in {#1}
     \path (\xi) node {\x};
}

%Places 1 in listed positions
\newcommand{\minterms}[1]{%
    \foreach \x in {#1}
        \path (\x) node {1};
}

%Places 0 in listed positions
\newcommand{\maxterms}[1]{%
    \foreach \x in {#1}
        \path (\x) node {0};
}

%Places X in listed positions
\newcommand{\indeterminats}[1]{%
    \foreach \x in {#1}
        \path (\x) node {d};
}


\usepackage{listings}
\usetikzlibrary{positioning, shapes, arrows.meta}

\lstset{language=Verilog}

\usepackage[utf8x]{inputenc}

\definecolor{shellGreen}{RGB}{19,193,106}
\definecolor{backcolor}{rgb}{0.95,0.95,0.92}
\definecolor{mateBlack}{RGB}{45,45,50}
\definecolor{comment}{rgb}{0.1,0.6,0.2}
\definecolor{codegray}{rgb}{0.5,0.5,0.5}

\lstdefinestyle{verilog}{
   language=verilog,
   frame=single,
   basicstyle=\ttfamily\footnotesize,
   breaklines=true,
   captionpos=t,
   keepspaces=true,
   backgroundcolor=\color{backcolor},
   keywordstyle=[1]\color{blue}\bf,
   keywordstyle=[2]\color{red}\bf,
   keywordstyle=[3]\color{cyan!50}\bf,
   stringstyle=\color{orange},
   commentstyle=\color{comment},
   tabsize=2,
%   number=left,
%   numberstep=5pt,
   showspaces=false,
   showstringspaces=false,
   showtabs=false,
   %moredelim=[is][\component]{component\ }{\ is},
   morekeywords=[1]{
      library, use ,all,entity,is,port,in,out,end,architecture,of, body,
      function, variable, begin,and,or,Not,downto,ALL, signal, process, if,
      else, elsif, case, when, then, range, to, component, type, with, select,
      others, constant, inout, buffer, map, true, false, array, subtype, wait,
      wait for, generic, =, <, >, <=, >=, =>, 
   },
   alsoletter={=, <, >},
   morekeywords=[2]{
          STD, textio, std_logic_textio, STD_LOGIC_VECTOR,STD_LOGIC,IEEE,STD_LOGIC_1164, work, local, real,
          math_real, time, NUMERIC_STD,STD_LOGIC_ARITH,STD_LOGIC_UNSIGNED,
          std_logic_vector, std_logic, ieee, numeric_std, std_ulogic,
          std_logic_1164, natural, bit, bit_vector, signed, unsigned,
          boolean, integer
    },
    morekeywords=[3]{rising_edge, falling_edge, resize, to_signed, to_unsigned},
    morecomment=[l]{--},
    % morecomment=[s][\color{orange}]{'}{'},
    rulecolor=\color{black},
}
\def\component#1{%
    \textbf{\textcolor{blue}{component\ }}%
    \textcolor{green}{#1}%
    \textbf{\textcolor{blue}{\ is}}%
}

\lstset{style=verilog}

\usepackage{diagbox}


\definecolor{codegreen}{rgb}{0,0.6,0}
\definecolor{codegray}{rgb}{0.5,0.5,0.5}
\definecolor{codepurple}{rgb}{0.58,0,0.82}
\definecolor{backcolour}{rgb}{0.95,0.95,0.92}

\lstdefinestyle{mystyle}{
    backgroundcolor=\color{backcolour}\bf,   
    commentstyle=\color{codegreen}\bf,
    keywordstyle=\color{magenta}\bf,
    numberstyle=\tiny\color{codegray}\bf,
    stringstyle=\color{codepurple},
    basicstyle=\ttfamily\footnotesize,
    breakatwhitespace=false,         
    breaklines=true,                 
    captionpos=t,                    
    keepspaces=true,                
    showspaces=false,                
    showstringspaces=false,
    showtabs=false,                  
    tabsize=2
}

% \lstset{style=mystyle}