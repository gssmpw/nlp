\vspace{-0.25cm}
\section{Preliminaries}
\subsection{CircuitSAT Sampling} \label{subsec:circuitsat_sampling}
CircuitSAT sampling refers to the task of sampling solutions from the solution space of a given CircuitSAT problem. In a Boolean circuit, variables can only take on binary values of either $0$ or $1$. A digital circuit is composed of various logic gates such as AND, OR, and NOT gates, which manipulate these Boolean variables. The output of such a circuit is produced based on how these logic gates operate.


In CircuitSAT, the objective is to determine whether a given circuit, representing a Boolean expression, has an assignment of binary values to its variables that results in the circuit output valuation to $1$. The sampling aspect introduces a probabilistic dimension to this problem. Instead of finding a single solution for satisfiable problems, CircuitSAT sampling aims to generate multiple solutions or samples from the set of all possible solutions. Sampling solutions from CircuitSAT instances is an integral part of the design verification process, with significant applications in CRV \cite{Kitchen2007crv}.

CRV is a verification methodology employed in the design and testing of modern digital circuits. It involves generating random input stimuli for the design under test (DUT) while adhering to a set of predefined constraints. This method helps explore a wide range of possible input scenarios, increasing the likelihood of identifying design flaws and ensuring robustness. Inputs to the DUT are generated randomly within specified constraints. This randomness helps cover a broad spectrum of test scenarios, including corner cases that might not be easily detected with directed tests. Constraints are rules or conditions that the random inputs must satisfy. These can be functional constraints based on the design specifications or operational constraints based on practical considerations. Constraints ensure that the generated random inputs are valid and meaningful for the DUT.

For instance, consider a 4-bit multiplier with two unsigned 4-bit binary inputs and an 8-bit product as its primary output. To restrict our bug search to input pairs where the product is less than $128$, we need to set the most significant bit of the product to $0$ and find random satisfying solutions. It is worth mentioning that while exhaustive search can be used for design verification with complete coverage, it is often impractical for complex designs due to scalability and efficiency limitations. CRV provides a more feasible, efficient, and effective approach by leveraging constraints and randomization to achieve high coverage and uncover critical issues within a manageable timeframe.

One common approach for CircuitSAT sampling is the use of SAT solvers with sampling capabilities. These solvers are designed to not only determine the satisfiability of a Boolean formula but also to sample solutions from the solution space. There are various technique for efficient SAT solving, including backtracking algorithms like Davis-Putnam-Logemann-Loveland (DPLL) algorithm \cite{Davis1962DPLL}, stochastic local search methods like WalkSAT \cite{selman1993local}, and CDCL algorithms \cite{Silva1996CDCL, silva2021CDCL}. Over the past years, various algorithms and techniques have been developed for CircuitSAT/SAT sampling including randomized algorithms, Markov chain Monte Carlo (MCMC) methods, and sampling-based heuristics \cite{Impagliazzo2017RandomSAT, kitchen2009markov, Soos2020unigen3, dutra2018quicksampler, Golia2021cmsgen}. These approaches typically involve iteratively exploring the solution space, selecting candidate solutions based on certain criteria, and stochastically accepting or rejecting them. 


The process of CircuitSAT sampling using SAT solvers involves translating the structure and logic of the given circuit into an equivalent Boolean formula, typically represented in CNF. The CNF consists of a conjunction of multiple clauses (i.e., AND of multiple clauses), where each clause is a disjunction of literals (i.e., OR of literals). Literals are referred to as Boolean variables or their complements. During the conversion process, the number of inputs, outputs and intermediate signals directly contributes to the number of variables in the CNF. The functionality of logic gates and their interconnections determine the number of clauses in the CNF. 

The size and complexity of the CNF formula vary significantly depending on factors such as the number of gates, the depth of the circuit, and the number of inputs, outputs and intermediate signals. In general, the CNF formula tends to contain significantly more bit-wise operations than its corresponding circuit. The introduced complexity to the SAT instance as the result of the conversion exponentially increases the time required to find a solution using SAT solvers since the SAT problem is NP-complete. This issue becomes worse when the complexity of digital circuits increases, making SAT sampling a non-trivial task, especially for large or complex circuits.

\vspace{-0.25cm}
\subsection{Multi-Output Regression Task}
A multi-output regression task is a statistical technique used to predict multiple target variables simultaneously from a set of input variables \cite{borchani2015survey}. In this task, the primary objective is to develop a model that accurately captures the relationships between input and output variables. The model can be constructed using various techniques such as linear regression and neural networks. The model is then trained using a dataset where both input-output pairs are known. During training, the model's parameters are adjusted by minimizing the distance between the predicted outputs and the desired target variables. A common metric to measure such a distance is mean squared error (MSE) or $\ell_2$-loss. 












% \section{Theoretical Analysis}
% Convexity is a crucial concept in various fields, including machine learning, economics, engineering, and operations research. The importance of convexity in optimization lies in its ability to simplify the optimization process, ensure convergence to globally optimal solutions, provide robustness to noise, and facilitate the use of efficient optimization algorithms based on gradient information and duality principles. As such, convexity can simplify the optimization problem in our formulation for the CircuitSAT problem by ensuring that local optima are also global optima, allowing {\sc Demotic} for finding the best solutions. 

% To show the convexity of our formulation for the CircuitSAT problem, we utilize the definition of convexity. A function $f(x)$ is convex if the inequality of 
% \begin{equation}\label{convexity_condition}
%     f(\lambda x_1 + (1 - \lambda) x_2) \leq \lambda f(x_1) + (1 - \lambda) f(x_2),
% \end{equation}
% holds for any two points $x_1$ and $x_2$, and for any $\lambda \in [0, 1]$. If we can show the convexity of AND, OR and NOT gates, we can conclude that the function representing a circuit constructed using these gates is also convex as a composition of convex functions remains convex.

% Let us consider the AND gate with the function of $f(x_1, x_2) = x_1~x_2$. We start by writing and simplifying the left-hand side of Eq. (\ref{convexity_condition}) for the AND gate as follows
% \begin{align}
%     & f(\lambda x_1 + (1 - \lambda) x_1', \lambda x_2 + (1 - \lambda) x_2') \nonumber \\  &= (\lambda x_1 + (1 - \lambda) x_1') (\lambda x_2 + (1 - \lambda) x_2') \nonumber \\
%     &= \lambda^2 x_1 x_2 + (1 - \lambda)^2 x_1' x_2' + \lambda (1-\lambda)(x_1' x_2 + x_1 x_2'),
% \end{align}
% where $x_1$, $x_2$, $x_1'$ and $x_2'$ are all in the interval of $[0, 1]$. For the right-hand side of the inequality, we have 
% \begin{equation}
%     \lambda f( x_1, x_2)+(1 - \lambda) f(x_1', x_2') = \lambda x_1 x_2 + (1 - \lambda) x_1' x_2'.
% \end{equation}
% By putting the left-hand side and right-hand side equations in the inequality, we get
% \begin{align}\label{cond1}
%     &  \lambda^2 x_1 x_2 + (1 - \lambda)^2 x_1' x_2' + \lambda (1-\lambda)(x_1' x_2 + x_1 x_2') \nonumber \\ 
%     & \leq \lambda x_1 x_2 + (1 - \lambda) x_1' x_2'.
% \end{align}
% Now, let us rearrange the above equation and rewrite it as follows
% \begin{equation}
%     \lambda^2 x_1 x_2 + (1 - \lambda)^2 x_1' x_2' + \lambda (1-\lambda)(x_1' x_2 + x_1 x_2') \nonumber \\ 
%     & \leq \lambda (1-\lambda) x_1 x_2 + \lambda (1 - \lambda) x_1' x_2' + \lambda (\lambda - 1)
% \end{equation}


% Since all the variables in the above inequality are in $[0, 1]$, we can say
% \begin{align} \label{cond2}
%     & \lambda^2 x_1 x_1' + (1 - \lambda)^2 x_2 x_2' + \lambda (1-\lambda)(x_1x_2' + x_2 x_1') \nonumber \\ 
%     & \leq \lambda x_1 x_1' + (1 - \lambda) x_2 x_2' + \lambda (1-\lambda)(x_1x_2' + x_2 x_1').
% \end{align}
% Given Eq. (\ref{cond2}), we can rewrite Eq. (\ref{cond1}) as follows
% \begin{align}\label{cond1}
%     & \lambda x_1 x_1' + (1 - \lambda) x_2 x_2' + \lambda (1-\lambda)(x_1x_2' + x_2 x_1') \nonumber \\ 
%     & \leq \lambda x_1 x_2 + (1 - \lambda) x_1' x_2'.
% \end{align}


