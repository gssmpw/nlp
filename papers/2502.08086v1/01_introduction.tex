\section{Introduction}
Circuit satisfiability (CircuitSAT) solving is an integral part of testing and verification of digital circuits in modern front-end applications such as logic rewriting, false path analysis, property checking, logic synthesis and equivalence checking \cite{Mishchenko2005Optimization, Tsai2009TimingAnalyzer, Bradley2011ModelChecking, Mishchenko2006EquivalenceChecking, Zhang2021LogicSynthesis}. CircuitSAT samplers play a crucial role in generating diverse samples from the solution space, aiding in validation, analysis, and optimization of digital circuit designs \cite{dutra2018quicksampler}. Exhaustive exploration and the creation of diverse solutions are essential to guaranteeing that the design fulfills its functional requirements and operates correctly across different scenarios. Consequently, numerous sampling techniques have been developed to detect edge cases and outliers, ensure representativeness, bolster robustness, and promote the broad applicability of findings \cite{dutra2019EfficientSampling}.

High-throughput sampling is fundamental in the realm of CircuitSAT, playing a vital role in various critical tasks such as constrained random verification (CRV) \cite{Kitchen2007crv}. Its primary function lies in boosting efficiency and scalability by enabling swift exploration of vast solution spaces, which is especially crucial when dealing with complex digital circuits containing numerous inputs, outputs, and intermediate signals. Moreover, it expands coverage across solution spaces, assisting in the detection of uncommon solutions and intricate edge cases inherent in CircuitSAT problems. Furthermore, high-throughput sampling enhances statistical reliability by producing larger sample sizes, thereby reducing sampling variability in CircuitSAT formula analysis.

A common approach for solving the CircuitSAT problem and obtaining diverse solutions involves transforming the CircuitSAT problem into a Boolean Satisfiability (SAT) problem, then employing robust and advanced solvers to solve it effectively \cite{Hsu2014CircuitSAT}. More precisely, the CircuitSAT solving process begins with the formulation of the logical constraints of a digital logic into a Boolean formula, typically represented in conjunctive normal form (CNF). This formula encodes the functionality of the underlying circuits into a set of Boolean clauses according to the interconnections between the circuit's inputs, outputs and internal signals, ensuring that the resulting CNF accurately represents the original circuit's behavior \cite{Velev2004CNF}. 

The conversion process can sometimes be complex and computationally intensive, especially for large circuits. In some cases, the resulting CNF might not always be \textit{compact} due to different factors such as the size of the circuit and the number of intermediate variables. For instance, circuits containing many gates or components may result in a large CNF since each gate or component can potentially introduce additional variables and clauses. The intermediate variables can also introduce additional variables and clauses, contributing in the size of the resulting CNF. This increase in complexity of CNF can impact the efficiency of SAT solvers. 

SAT solvers employs a variety of techniques to search for a satisfying assignment to the variables in the CNF. Modern SAT solvers \cite{Niklas2003SAT, Moskewicz2001Chaff, Audemard2018Glucose} often utilize conflict-driven clause learning (CDCL) algorithm \cite{Silva1996CDCL, silva2021CDCL} which heavily relies on heuristics such as conflict-driven backtracking and clause learning. Theses heuristics help guide the search process of CDCL for a satisfying assignment efficiently. Due to the sequential nature of these heuristics and their reliance on branching and backtracking, current state-of-the-art (SOTA) SAT solvers are executed on CPUs. Accordingly, SOTA SAT samplers, which integrate a SAT solver as part of their algorithms, depend on a sequential process and are also optimized for execution on CPUs.


GPU acceleration has demonstrated substantial throughput performance advantages across diverse applications especially in machine learning \cite{Krizhevsky2012AlexNet}. In general, algorithms that exhibit parallelism and can be decomposed into independent tasks are generally suitable for GPU acceleration. This appears to align with the requirements for generating various satisfying solutions to the CircuitSAT problem, if there was a sampling method performing regular and data-parallel computations. In addition to parallelism, introducing a sampling method that operates directly on circuits by leveraging the spatial and temporal dimensions of digital computations—without converting to CNF—is crucial, as the CNF conversion process may not consistently produce the most optimal or compact representation. To this end, we introduce a novel differentiable sampler for multi-level digital circuits, called {\sc Demotic} \footnote{The code of {\sc Demotic} is available at \url{https://github.com/arashardakani/Demotic}.}, that utilizes gradient descent (GD) for learning diverse solutions to the CircuitSAT problem. We re-frame the CircuitSAT problem as a multi-output regression task, where each logic gate is modeled with a probabilistic representation. We subsequently formulate a loss function by incorporating specified constraints into the circuit. This approach allows us to maintain the integrity of the circuit structure while transforming the sampling process into a learning process. This process enables the generation of independent solutions to the CircuitSAT problem in a parallel fashion, enabling acceleration with GPUs. In summary, we make the following contributions in this paper.

\begin{itemize}
    \item We introduce a novel differentiable sampler for multi-level digital circuits, called {\sc Demotic}.
    \item We use a probabilistic representation to model logic gates and convert the CircuitSAT problem into a multi-output regression task.
    \item Our proposed sampling method maintains the original structure of the underlying digital circuit without the need for the conversion into any Boolean formulation.
    \item {\sc Demotic} enables performing the learning process in parallel, leading to GPU-accelerated sampling of satisfying formulas for the CircuitSAT problem.
    \item We demonstrate the performance of {\sc Demotic} across different CircuitSAT instances from the ISCAS-85 benchmark suite \cite{Hansen1999ISCASBench}. 
    
\end{itemize}














