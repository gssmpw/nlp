\documentclass[10pt, twocolumn, letterpaper]{article}

%%%%%%%%% PAPER TYPE  - PLEASE UPDATE FOR FINAL VERSION
\usepackage[review]{cvpr}      % To produce the REVIEW version
%\usepackage{cvpr}              % To produce the CAMERA-READY version
%\usepackage[pagenumbers]{cvpr} % To force page numbers, e.g. for an arXiv version

% Include other packages here, before hyperref.
\usepackage{graphicx}
\usepackage{amsmath}
\usepackage{amsthm}
\usepackage{amssymb}
\usepackage{booktabs}
\usepackage{times}
\usepackage{epsfig}
\usepackage{graphicx}
\usepackage{dpr_fec}

\newtheorem{proposition}{Proposition}[section]

% It is strongly recommended to use hyperref, especially for the review version.
% hyperref with option pagebackref eases the reviewers' job.
% Please disable hyperref *only* if you encounter grave issues, e.g. with the
% file validation for the camera-ready version.
%
% If you comment hyperref and then uncomment it, you should delete
% ReviewTempalte.aux before re-running LaTeX.
% (Or just hit 'q' on the first LaTeX run, let it finish, and you
%  should be clear).

%\usepackage[pagebackref=true,breaklinks=true,letterpaper=true,colorlinks,bookmarks=false]{hyperref}
\usepackage[pagebackref,breaklinks,colorlinks]{hyperref}

% Support for easy cross-referencing
\usepackage[capitalize]{cleveref}
\crefname{section}{Sec.}{Secs.}
\Crefname{section}{Section}{Sections}
\Crefname{table}{Table}{Tables}
\crefname{table}{Tab.}{Tabs.}

%%%%%%%%% PAPER ID  - PLEASE UPDATE
\def\cvprPaperID{*****} % *** Enter the CVPR Paper ID here
\def\confName{CVPR}
\def\confYear{2025}

\begin{document}

%%%%%%%%% TITLE
\title{Supplementary Materials}

\author{First Author\\
Institution1\\
Institution1 address\\
{\tt\small firstauthor@i1.org}
% For a paper whose authors are all at the same institution,
% omit the following lines up until the closing ``}''.
% Additional authors and addresses can be added with ``\and'',
% just like the second author.
% To save space, use either the email address or home page, not both
\and
Second Author\\
Institution2\\
First line of institution2 address\\
{\tt\small secondauthor@i2.org}
}
\onecolumn

\maketitle
%\thispagestyle{empty}
\appendix 


\section{Proof and Discussions}
\subsection{Proof of Proposition 5.1}~\label{appendix:prop5.1}
Recall the search direction $s(x_k)$ is defined as
\begin{align}~\label{eq:search.direction}
s(x_k) = 
\begin{cases}
- \alpha [\nabla f(x_k)]_g,& g \in \Gcal_I, \\
- \alpha [\nabla f(x_k)]_g - \gamma_k [x_k^Q]_g,& g \in \Gcal_R.
\end{cases}
\end{align}
\begin{proposition}
With forget rate $\gamma$ selection rule  and quantization step size $d$ selection rule, the search direction $s(x_k)$ is a descent direction with respect to function $f$ at the point $x_k$.
\end{proposition}
\begin{proof}
Due to the fact that negative gradient is guaranteed to be a descent direction, it suffices to show that the search direction $- \alpha [\nabla f(x_k)]_g - \gamma_k [x_k^Q]_g$ is descent direction. 
It follows that~\eqref{eq:search-direction-expression} can be rewritten as
\begin{align*}
s(x_k) = 
\begin{cases}
- \alpha [\nabla f(x_k)]_g, g \in \Gcal_I, \\
- \alpha [\nabla f(x_k)]_g - \gamma_k  [\text{sgn}(x_k) \cdot \text{clip}_{q_m}^t(|x_k|)]_g - \gamma_k \cdot d [\text{sgn}(x_k) \cdot R(x_k)]_g,  g \in \Gcal_R.
\end{cases}
\end{align*}
Denote the angle between $-[\nabla f(x)]_g$ and $-[\text{sgn}(x)\cdot \text{clip}_{q_m}^t(|x|)]_g$ as $\theta_{\gamma}$ and the angle between $-[\nabla f(x)]_g$ and $-[\text{sgn}(x)\cdot d \cdot R(x)]_g$ as $\theta_{d}$. We first decompose the search direction $s_{\text{clip}}(x_k)$ into two orthogonal directions, i.e.,
$$
s_{\text{clip}}(x_k) = \shat_{\text{clip}}(x_k) + \stilde_{\text{clip}}(x_k),
$$
where $\shat_{\text{clip}}(x)^T [\nabla f(x)]_g = 0$ and $\|[\shat_{\text{clip}}(x)]_g\| = \| \gamma [x]_g \cos(\theta_{\gamma} - 90^{\circ})\| = \gamma \sin\theta_{\gamma}\|[x]_g\|$. Given the fact that $\shat_{\text{clip}}(x)$ and $\stilde_{\text{clip}}(x)$ are orthogonal to each other, we have that
\begin{align*}
\|\stilde_{\text{clip}}(x)\|^2 & = \|s_{\text{clip}}(x)\|^2 - \|\shat_{\text{clip}}(x)\|^2 \\
& = \|-\alpha [\nabla f(x)]_g - \gamma [\text{sgn}(x)\cdot \text{clip}_{q_m}^t(|x|)]_g\|^2 \\
& - \gamma^2 \sin^2\theta_{\gamma} \|[\text{sgn}(x)\cdot \text{clip}_{q_m}^t(|x|)]_g\|^2 \\
& = \alpha^2 \|[\nabla f(x)]_g\|^2 + 2\alpha \gamma [\nabla f(x)]_g^T [\text{sgn}(x)\cdot \text{clip}_{q_m}^t(|x|)]_g \\
& + \gamma^2 \cos^2\theta_{\gamma} \|[\text{sgn}(x)\cdot \text{clip}_{q_m}^t(|x|)]_g\|^2 \\
& = \left[\alpha \|[\nabla f(x)]_g\| + \gamma \cos\theta_{\gamma} \|[\text{sgn}(x)\cdot \text{clip}_{q_m}^t(|x|)]_g\|\right]^2. 
\end{align*}
Given the norm and direction of the vector $[\dtilde(x)]_g$, we have $\dtilde(x)$ expressed as
\begin{equation*}
-\frac{\alpha \|[\nabla f(x)]_g\| + \gamma \cos(\theta_{\gamma})\|[\text{sgn}(x)\cdot \text{clip}_{q_m}^t(|x|)]_g\|}{\|[\nabla f(x)]_g\|} [\nabla f(x)]_g.
\end{equation*}

The selection rule~\eqref{eq:forget-rate-rule} ensures that  
\begin{align*}
-\alpha \|[\nabla f(x)]_g\|^2 - \gamma_k [\nabla f(x)]_g^T [\text{sgn}(x) \cdot [\text{clip}_{q_m}^{t}(|x|) ]_g ] < -\eta \alpha \|[\nabla f(x)]_g\|^2.
\end{align*}

The selection rule~\eqref{eq:forget.rate.rule} ensures that  
\begin{align*}
-\alpha \|[\nabla f(x)]_g\|^2 - \gamma_k [\nabla f(x)]_g^T [\text{sgn}(x) \cdot [\text{clip}_{q_m}^{t}(|x|) ]_g ] < -\eta \alpha \|[\nabla f(x)]_g\|^2.
\end{align*}
After that, we can choose quantization step size $d$ based on the info of $\gamma$. Our goal is to guarantee that 
\begin{equation*}
-\eta \alpha \|[\nabla f(x)]_g\|^2 - \gamma d [\nabla f(x_k)]_g^T [\text{sgn}(x_k) \cdot R(x_k)]_g < 0
\end{equation*}

\end{proof}

\subsection{Proof of Proposition 5.2}
\begin{proposition}
$s_k$ is a descent direction with respect to function $f$ at the point $x$.
\end{proposition}

{\small
\bibliographystyle{ieee_fullname}
\bibliography{egbib}
}

\end{document}