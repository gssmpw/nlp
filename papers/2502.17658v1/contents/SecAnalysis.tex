
\section{Security Analysis}

\subsection{Success Rate Analysis}
After extensive experiments, the configurable parameters in Algorithm \ref{alg:alg1} were selected to the following values: \(N=10000\), \(M=40\), \(L=20\), \(K=3\), \(\alpha=0.9\), and \(\gamma=1.13\), which were effective in demonstrating the outcome. Prolonging the attack duration reduces noise in the scores and enhances the success rate of accurately determining the weights. 

%\begin{figure}[!htbp]
%\centerline{\includegraphics[width=0.4\textwidth]%{images/ML_Attack_Successful_Rate_v3.pdf}}
%\caption{Impact of Attack Duration on the Success Rate of Weight Determination.  if you leak for example 100 different perceptrons successfully you can plot that as well. If you remember a reviewer was asking whether you can leak a particular set of weights or all  combination. {\color{red}maybe show this with bar chart? } }
%\label{fig:ML_SuccessRate}
%\end{figure}

\begin{figure}[h!]
\centering
\begin{tikzpicture}
    \begin{axis}[
        xlabel={Attack Duration (minute)},
        ylabel={Successful Rate},
        legend pos=outer north east,
        font=\scriptsize,
        xlabel style={yshift=5pt},
        ylabel style={yshift=-15pt},
        width=\linewidth*1,
        height=\linewidth*0.6,
        xmax=100, xmin=0,
        legend style={
                legend columns=-1,
                at={(0.5,1.16)},
                anchor=north, 
                font=\tiny,
                /tikz/column 2/.style={column sep=5pt}, 
            },
    ]
    \addplot[
        line width=2pt,
        color=darkgray,
    ] table {Diagrams/Data/accuracy.txt};

    \end{axis}
\end{tikzpicture}
\caption{Impact of Attack Duration on the Success Rate of Weight Determination.  if you leak for example 100 different perceptrons successfully you can plot that as well. If you remember a reviewer was asking whether you can leak a particular set of weights or all  combination.}
\label{fig:ML_SuccessRate}
\end{figure}

Figure \ref{fig:ML_SuccessRate} shows success rates increasing with longer attack durations, from 60\% at 5 minutes to a complete 100\% success rate at 50 minutes (0.02 bit/s) and beyond. Since cooling down the AMX is necessary for every timing measurement, over 99\% of the attack duration is consumed by cooling delays. Additionally, the results of the algorithm for two different attack durations are shown in Figure \ref{fig:LeakedWeights}. 

\begin{figure}
    \centering
    \includegraphics[width=0.6\linewidth]{thor_black_white_compare.png}
    \caption{{\color{red} Add the conclusion of this figure.  TODO improve the fonts and readability of the figure. }}
    \label{fig:thorleakage}
\end{figure}

{\color{red} , You can add the success rate plot here and compare. I pasted the copy you sent me. Please improve it.}


%\begin{figure}
%    \centering
%    \includegraphics[width=0.6\linewidth]{thor_black_white_compare.png}
%    \caption{{\color{red} Add the conclusion of this figure.  TODO improve the fonts and readability of the figure. }}
%    \label{fig:enter-label}
%\end{figure}
%\begin{figure}[!htbp]
%\centerline{\includegraphics[width=.8\linewidth]{images/Example.pdf}}
%\caption{{\color{red} Use two symbols like circles and stars or rectangular, that can be distinguishable in gray scale if you print with black and white. }Output of the side-channel attack for two different attack durations. In both figures, green markers represent weights of 0, and orange markers represent weights of 1. }
%\label{fig:LeakedWeights}
%\end{figure}


\begin{figure}[h!]
    \centering
    \begin{minipage}{0.5\linewidth}
        \centering
        \begin{tikzpicture}
            \begin{axis}[
                xlabel={Weight Index},
                ylabel={Score1/Score0},
                legend pos=outer north east,
                xmin=-1, xmax=65,
                font=\scriptsize,
                xlabel style={yshift=5pt},
                ylabel style={yshift=-15pt},
                width=1.1\linewidth,
                height=\linewidth,
            ]
                \addplot[
                    thick,
                    only marks,
                    line width=1.5pt,
                    mark=o,
                    mark size=1,
                    color=darkgray,
                    forget plot
                ] table {Diagrams/Data/30_min_0.txt};

                \addplot[
                    thick,
                    only marks,
                    mark=triangle,
                    line width=1.5pt,
                    mark size=1,
                    color=darkgray,
                    forget plot
                ] table {Diagrams/Data/30_min_1.txt};

                \addplot[
                    thick,
                    domain=0:70,
                    samples=2, 
                    color=black,
                    dashed,
                    forget plot
                ] {1.13};
            \end{axis}
            \node[black, thick] at (1.2,3.2) {\small (a) Attack Duration: 75 minutes};
        \end{tikzpicture}
    \end{minipage}\hfill
    \begin{minipage}{0.5\linewidth}
        \centering
        \begin{tikzpicture}
            \begin{axis}[
                xlabel={Weight Index},
                ylabel={Score1/Score0},
                legend pos=outer north east,
                xmin=-1, xmax=65,
                font=\scriptsize,
                xlabel style={yshift=5pt},
                ylabel style={yshift=-15pt},
                width=1.1\linewidth,
                height=\linewidth,
            ]
                \addplot[
                    thick,
                    only marks,
                    line width=1.5pt,
                    mark=o,
                    mark size=1,
                    color=darkgray,
                    forget plot
                ] table {Diagrams/Data/75_min_0.txt};

                \addplot[
                    thick,
                    only marks,
                    mark=triangle,
                    line width=1.5pt,
                    mark size=1,
                    color=darkgray,
                    forget plot
                ] table {Diagrams/Data/75_min_1.txt};

                \addplot[
                    thick,
                    domain=0:70,
                    samples=2, 
                    color=black,
                    dashed,
                    forget plot
                ] {1.13};
            \end{axis}
            \node[black, thick] at (1.4,3.2) {\small (b) Attack Duration: 30 minutes};
        \end{tikzpicture}
    \end{minipage}
    \caption{Output of the side-channel attack for two different attack durations. In both figures, green markers represent weights of 0, and orange markers represent weights of 1.}
    \label{fig:LeakedWeights}
\end{figure}


\subsection{Attack Requirements Analysis}
\subsubsection{Frequency Scaling Analysis}
{\color{red} add your frequency and clock analysis results and explanation here, explain in depth. }



% \paragrabf{Potential Countermeasures.}
% Eliminating the cooldown state could defend against Thor but at a high power cost since Intel AMX is an energy-intensive accelerator designed for AI tasks. Keeping AMX continuously active would be power-prohibitive.

% Masking is a proven countermeasure for protecting AI model parameters against power side-channel attacks and could be adapted for future AMX versions despite the performance overhead. Additionally, machine learning models should incorporate techniques to detect unusual usage patterns, which can help identify and thwart attacks attempting to infer parameter values using methods similar to our AMX-type attack. One well-known countermeasure to these timing attacks is to coarsen the timer. By reducing the timer's precision, it becomes much harder for attackers to measure the subtle differences in execution times that they rely on for their exploits.

% In summary, while various countermeasures exist for different attack vectors, protecting against Thor on Intel AMX accelerators requires novel approaches to power and frequency management alongside traditional techniques.



% A wealth of research exists looking at the effects of AI companions on humans, for example \citet{Brandtzaeg2022AIfriend, xie2022attachment}. Our paper instead focuses on evaluating the biases and stereotypes that chatbots perpetuate as it becomes increasingly important to mitigate their impacts.

Metrics play a crucial role in assessing {LLM}s, and a range of papers have produced quantitative evaluations of these models \citep{nangia-etal-2020-crows, dhamalabold2021, bellem2024are, wan2023biasasker}. Through the lens of gender, extensive work has been done on creating a metric for occupational bias \citep{kirk2024box, rudinger-etal-2018-wino}. \citet{bai2024measuring} is one of few papers that focus on more underlying gender biases in that it studies implicit (unintentional, automatic) rather than explicit (intentional, deliberate) bias. It does this by using the Implicit Association Test (IAT), commonly used for human biases, and modifies it to {LLM}s.

\subsection{Persona Bias in LLMs}

Research into {AI} personas find that, generally, the design and implementation of personas result in models reflecting existing human biases, as shown by \citet{cheng-etal-2023-marked}. They generated personas with different ethnicities and genders and then had the LLM describe itself in that personas voice. This output is compared to the unmarked default persona descriptions, i.e., White and Man, by finding words that statistically distinguish the two groups and comparing the generated descriptions to human-created ones. The results show that models positively stereotype and assume resilience in marked groups much more heavily than unmarked ones and much more often than humans do. \citet{wan-etal-2023-stochastic} aimed to categorise and measure ‘persona biases’ by creating a UniversalPersona dataset of generic and specific personas. These personas are measured against harmful expression (offensiveness, toxic continuation, and regard) and harmful agreement metrics (stereotype and toxic agreement). Findings show that models have fairness issues when taking on the role of a persona. This work is a continuation of that by \citet{deshpande-etal-2023-toxicity}, which shows that assigning a specific persona can increase toxicity up to six-fold. 

To uncover more implicit bias, \citet{gupta2024bias} evaluates the unintended effects of persona assignment by measuring the reasoning capability of different models on different tasks. The results are clear; although ChatGPT will unilaterally reply that there is no difference in the maths problem-solving skills between a physically-abled and disabled person, when adopting the identity of a physically-disabled person, it outputs that because of its disability, it is unable to perform calculations. The work by \citet{plaza2024angry} evaluates a more inferred bias that assumes women are more emotional than men, which {LLM}s seem to agree with; sadness is overwhelmingly linked with women, anger with men.

To date, no work has studied how assigning gendered personas to a model with an implied relationship with its user impacts model responses. Not acknowledging the user's role disregards the topic of sycophancy -- where {LLM}s may echo the opinions of the users they interact with. \citet{huang2024trustllm} and \citet{xu2024earthflatbecauseinvestigating} show that assigning the user a persona and then prompting the model with questions tends to have the model giving responses that would align with the user's persona. However, there is a research gap in how sycophancy may change when assigning a persona to the model system. The role of sycophancy is an essential question when focusing on {AI} companions, as the relationship between user and model is, at its core, intertwined \citep{sharma2023understandingsycophancylanguagemodels}.




\subsection{NN leaked Parameters Comparison}
{\color{red} , compare other NN attacks and what they leaked here vs. what can be leaked by Thor.}

