%%
%% This is file `sample-sigconf-authordraft.tex',
%% generated with the docstrip utility.
%%
%% The original source files were:
%%
%% samples.dtx  (with options: `all,proceedings,bibtex,authordraft')
%% 
%% IMPORTANT NOTICE:
%% 
%% For the copyright see the source file.
%% 
%% Any modified versions of this file must be renamed
%% with new filenames distinct from sample-sigconf-authordraft.tex.
%% 
%% For distribution of the original source see the terms
%% for copying and modification in the file samples.dtx.
%% 
%% This generated file may be distributed as long as the
%% original source files, as listed above, are part of the
%% same distribution. (The sources need not necessarily be
%% in the same archive or directory.)
%%
%%
%% Commands for TeXCount
%TC:macro \cite [option:text,text]
%TC:macro \citep [option:text,text]
%TC:macro \citet [option:text,text]
%TC:envir table 0 1
%TC:envir table* 0 1
%TC:envir tabular [ignore] word
%TC:envir displaymath 0 word
%TC:envir math 0 word
%TC:envir comment 0 0
%%
%%
%% The first command in your LaTeX source must be the \documentclass
%% command.
%%
%% For submission and review of your manuscript please change the
%% command to \documentclass[manuscript, screen, review]{acmart}.
%%
%% When submitting camera ready or to TAPS, please change the command
%% to \documentclass[sigconf]{acmart} or whichever template is required
%% for your publication.
%%
%%
%% \documentclass[sigconf,authordraft]{acmart}
%% \documentclass[manuscript, review, anonymous]{acmart}
\documentclass[sigconf]{acmart}
\usepackage{listings}
\usepackage{graphicx}
\usepackage{tabularx}
\usepackage{makecell}
\usepackage{subcaption}

%%
%% \BibTeX command to typeset BibTeX logo in the docs
\AtBeginDocument{%
  \providecommand\BibTeX{{%
    Bib\TeX}}}

%% Rights management information.  This information is sent to you
%% when you complete the rights form.  These commands have SAMPLE
%% values in them; it is your responsibility as an author to replace
%% the commands and values with those provided to you when you
%% complete the rights form.
%%
\copyrightyear{2025}
\acmYear{2025}
\setcopyright{rightsretained}
\acmConference[CHI '25]{CHI Conference on Human Factors in Computing Systems}{April 26-May 1, 2025}{Yokohama, Japan}
\acmBooktitle{CHI Conference on Human Factors in Computing Systems (CHI '25), April 26-May 1, 2025, Yokohama, Japan}\acmDOI{10.1145/3706598.3713598}
\acmISBN{979-8-4007-1394-1/25/04}
%% Submission ID.
%% Use this when submitting an article to a sponsored event. You'll
%% receive a unique submission ID from the organizers
%% of the event, and this ID should be used as the parameter to this command.
%%\acmSubmissionID{123-A56-BU3}

%%
%% For managing citations, it is recommended to use bibliography
%% files in BibTeX format.
%%
%% You can then either use BibTeX with the ACM-Reference-Format style,
%% or BibLaTeX with the acmnumeric or acmauthoryear sytles, that include
%% support for advanced citation of software artefact from the
%% biblatex-software package, also separately available on CTAN.
%%
%% Look at the sample-*-biblatex.tex files for templates showcasing
%% the biblatex styles.
%%

%%
%% The majority of ACM publications use numbered citations and
%% references.  The command \citestyle{authoryear} switches to the
%% "author year" style.
%%
%% If you are preparing content for an event
%% sponsored by ACM SIGGRAPH, you must use the "author year" style of
%% citations and references.
%% Uncommenting
%% the next command will enable that style.
%%\citestyle{acmauthoryear}


%%
%% end of the preamble, start of the body of the document source.
\begin{document}

%%
%% The "title" command has an optional parameter,
%% allowing the author to define a "short title" to be used in page headers.
%\title{The Utility Space, Challenges and Design Implications of Online Restorative Justice Tools: A Case Study with ApoloBot}

\title[The Design Space for Online Restorative Justice Tools]{The Design Space for Online Restorative Justice Tools: A Case Study with ApoloBot}

%%
%% The "author" command and its associated commands are used to define
%% the authors and their affiliations.
%% Of note is the shared affiliation of the first two authors, and the
%% "authornote" and "authornotemark" commands
%% used to denote shared contribution to the research.
\author{Bich Ngoc (Rubi) Doan}

\email{ngocdb1609@gmail.com}
\orcid{0009-0006-1767-2585}
\affiliation{%
  \institution{KAIST}
  \city{Daejeon}
  \country{Republic of Korea}
}
\authornote{now at EPFL, Switzerland.}


\author{Joseph Seering}
\email{seering@kaist.ac.kr}
\orcid{0000-0001-7606-4711}
\affiliation{%
  \institution{KAIST}
  \city{Daejeon}
  \country{Republic of Korea}
}

%%
%% By default, the full list of authors will be used in the page
%% headers. Often, this list is too long, and will overlap
%% other information printed in the page headers. This command allows
%% the author to define a more concise list
%% of authors' names for this purpose.
\renewcommand{\shortauthors}{Doan and Seering.}

%% 
%% [Added] Color text for notes
\newcommand{\todo}[1]{\textcolor{red}{TODO: #1}}
\newcommand{\note}[1]{\textcolor{blue}{Note: #1}}
\newcommand{\revision}[1]{\textcolor{black}{#1}}
\newcommand{\revisionn}[1]{\textcolor{black}{#1}}

%%
%% The abstract is a short summary of the work to be presented in the
%% article.
\begin{abstract}
   Volunteer moderators use various strategies to address online harms within their communities. Although punitive measures like content removal or account bans are common, recent research has explored the potential for restorative justice as an alternative framework to address the distinct needs of victims, offenders, and community members. In this study, we take steps toward identifying a more concrete design space for restorative justice-oriented tools by developing ApoloBot, a Discord bot designed to facilitate apologies when harm occurs in online communities. We present results from two rounds of interviews: first, with moderators giving feedback about the design of ApoloBot, and second, after a subset of these moderators have deployed ApoloBot in their communities. This study builds on prior work to yield more detailed insights regarding the potential of adopting online restorative justice tools, including opportunities, challenges, and implications for future designs.
\end{abstract}



%%
%% The code below is generated by the tool at http://dl.acm.org/ccs.cfm.
%% Please copy and paste the code instead of the example below.
%%
\begin{CCSXML}
<ccs2012>
   <concept>
       <concept_id>10003120.10003130.10011762</concept_id>
       <concept_desc>Human-centered computing~Empirical studies in collaborative and social computing</concept_desc>
       <concept_significance>500</concept_significance>
       </concept>
 </ccs2012>
\end{CCSXML}

\ccsdesc[500]{Human-centered computing~Empirical studies in collaborative and social computing}

%%
%% Keywords. The author(s) should pick words that accurately describe
%% the work being presented. Separate the keywords with commas.
\keywords{Content moderation, online harassment, alternative justice, Discord}
%% A "teaser" image appears between the author and affiliation
%% information and the body of the document, and typically spans the
%% page.
\received{12 September 2024}
\received[revised]{10 December 2024}
\received[accepted]{16 January 2025}

%%
%% This command processes the author and affiliation and title
%% information and builds the first part of the formatted document.
\maketitle

\section{Introduction}
\label{sec:introduction}
The business processes of organizations are experiencing ever-increasing complexity due to the large amount of data, high number of users, and high-tech devices involved \cite{martin2021pmopportunitieschallenges, beerepoot2023biggestbpmproblems}. This complexity may cause business processes to deviate from normal control flow due to unforeseen and disruptive anomalies \cite{adams2023proceddsriftdetection}. These control-flow anomalies manifest as unknown, skipped, and wrongly-ordered activities in the traces of event logs monitored from the execution of business processes \cite{ko2023adsystematicreview}. For the sake of clarity, let us consider an illustrative example of such anomalies. Figure \ref{FP_ANOMALIES} shows a so-called event log footprint, which captures the control flow relations of four activities of a hypothetical event log. In particular, this footprint captures the control-flow relations between activities \texttt{a}, \texttt{b}, \texttt{c} and \texttt{d}. These are the causal ($\rightarrow$) relation, concurrent ($\parallel$) relation, and other ($\#$) relations such as exclusivity or non-local dependency \cite{aalst2022pmhandbook}. In addition, on the right are six traces, of which five exhibit skipped, wrongly-ordered and unknown control-flow anomalies. For example, $\langle$\texttt{a b d}$\rangle$ has a skipped activity, which is \texttt{c}. Because of this skipped activity, the control-flow relation \texttt{b}$\,\#\,$\texttt{d} is violated, since \texttt{d} directly follows \texttt{b} in the anomalous trace.
\begin{figure}[!t]
\centering
\includegraphics[width=0.9\columnwidth]{images/FP_ANOMALIES.png}
\caption{An example event log footprint with six traces, of which five exhibit control-flow anomalies.}
\label{FP_ANOMALIES}
\end{figure}

\subsection{Control-flow anomaly detection}
Control-flow anomaly detection techniques aim to characterize the normal control flow from event logs and verify whether these deviations occur in new event logs \cite{ko2023adsystematicreview}. To develop control-flow anomaly detection techniques, \revision{process mining} has seen widespread adoption owing to process discovery and \revision{conformance checking}. On the one hand, process discovery is a set of algorithms that encode control-flow relations as a set of model elements and constraints according to a given modeling formalism \cite{aalst2022pmhandbook}; hereafter, we refer to the Petri net, a widespread modeling formalism. On the other hand, \revision{conformance checking} is an explainable set of algorithms that allows linking any deviations with the reference Petri net and providing the fitness measure, namely a measure of how much the Petri net fits the new event log \cite{aalst2022pmhandbook}. Many control-flow anomaly detection techniques based on \revision{conformance checking} (hereafter, \revision{conformance checking}-based techniques) use the fitness measure to determine whether an event log is anomalous \cite{bezerra2009pmad, bezerra2013adlogspais, myers2018icsadpm, pecchia2020applicationfailuresanalysispm}. 

The scientific literature also includes many \revision{conformance checking}-independent techniques for control-flow anomaly detection that combine specific types of trace encodings with machine/deep learning \cite{ko2023adsystematicreview, tavares2023pmtraceencoding}. Whereas these techniques are very effective, their explainability is challenging due to both the type of trace encoding employed and the machine/deep learning model used \cite{rawal2022trustworthyaiadvances,li2023explainablead}. Hence, in the following, we focus on the shortcomings of \revision{conformance checking}-based techniques to investigate whether it is possible to support the development of competitive control-flow anomaly detection techniques while maintaining the explainable nature of \revision{conformance checking}.
\begin{figure}[!t]
\centering
\includegraphics[width=\columnwidth]{images/HIGH_LEVEL_VIEW.png}
\caption{A high-level view of the proposed framework for combining \revision{process mining}-based feature extraction with dimensionality reduction for control-flow anomaly detection.}
\label{HIGH_LEVEL_VIEW}
\end{figure}

\subsection{Shortcomings of \revision{conformance checking}-based techniques}
Unfortunately, the detection effectiveness of \revision{conformance checking}-based techniques is affected by noisy data and low-quality Petri nets, which may be due to human errors in the modeling process or representational bias of process discovery algorithms \cite{bezerra2013adlogspais, pecchia2020applicationfailuresanalysispm, aalst2016pm}. Specifically, on the one hand, noisy data may introduce infrequent and deceptive control-flow relations that may result in inconsistent fitness measures, whereas, on the other hand, checking event logs against a low-quality Petri net could lead to an unreliable distribution of fitness measures. Nonetheless, such Petri nets can still be used as references to obtain insightful information for \revision{process mining}-based feature extraction, supporting the development of competitive and explainable \revision{conformance checking}-based techniques for control-flow anomaly detection despite the problems above. For example, a few works outline that token-based \revision{conformance checking} can be used for \revision{process mining}-based feature extraction to build tabular data and develop effective \revision{conformance checking}-based techniques for control-flow anomaly detection \cite{singh2022lapmsh, debenedictis2023dtadiiot}. However, to the best of our knowledge, the scientific literature lacks a structured proposal for \revision{process mining}-based feature extraction using the state-of-the-art \revision{conformance checking} variant, namely alignment-based \revision{conformance checking}.

\subsection{Contributions}
We propose a novel \revision{process mining}-based feature extraction approach with alignment-based \revision{conformance checking}. This variant aligns the deviating control flow with a reference Petri net; the resulting alignment can be inspected to extract additional statistics such as the number of times a given activity caused mismatches \cite{aalst2022pmhandbook}. We integrate this approach into a flexible and explainable framework for developing techniques for control-flow anomaly detection. The framework combines \revision{process mining}-based feature extraction and dimensionality reduction to handle high-dimensional feature sets, achieve detection effectiveness, and support explainability. Notably, in addition to our proposed \revision{process mining}-based feature extraction approach, the framework allows employing other approaches, enabling a fair comparison of multiple \revision{conformance checking}-based and \revision{conformance checking}-independent techniques for control-flow anomaly detection. Figure \ref{HIGH_LEVEL_VIEW} shows a high-level view of the framework. Business processes are monitored, and event logs obtained from the database of information systems. Subsequently, \revision{process mining}-based feature extraction is applied to these event logs and tabular data input to dimensionality reduction to identify control-flow anomalies. We apply several \revision{conformance checking}-based and \revision{conformance checking}-independent framework techniques to publicly available datasets, simulated data of a case study from railways, and real-world data of a case study from healthcare. We show that the framework techniques implementing our approach outperform the baseline \revision{conformance checking}-based techniques while maintaining the explainable nature of \revision{conformance checking}.

In summary, the contributions of this paper are as follows.
\begin{itemize}
    \item{
        A novel \revision{process mining}-based feature extraction approach to support the development of competitive and explainable \revision{conformance checking}-based techniques for control-flow anomaly detection.
    }
    \item{
        A flexible and explainable framework for developing techniques for control-flow anomaly detection using \revision{process mining}-based feature extraction and dimensionality reduction.
    }
    \item{
        Application to synthetic and real-world datasets of several \revision{conformance checking}-based and \revision{conformance checking}-independent framework techniques, evaluating their detection effectiveness and explainability.
    }
\end{itemize}

The rest of the paper is organized as follows.
\begin{itemize}
    \item Section \ref{sec:related_work} reviews the existing techniques for control-flow anomaly detection, categorizing them into \revision{conformance checking}-based and \revision{conformance checking}-independent techniques.
    \item Section \ref{sec:abccfe} provides the preliminaries of \revision{process mining} to establish the notation used throughout the paper, and delves into the details of the proposed \revision{process mining}-based feature extraction approach with alignment-based \revision{conformance checking}.
    \item Section \ref{sec:framework} describes the framework for developing \revision{conformance checking}-based and \revision{conformance checking}-independent techniques for control-flow anomaly detection that combine \revision{process mining}-based feature extraction and dimensionality reduction.
    \item Section \ref{sec:evaluation} presents the experiments conducted with multiple framework and baseline techniques using data from publicly available datasets and case studies.
    \item Section \ref{sec:conclusions} draws the conclusions and presents future work.
\end{itemize}
\section{Threats To Validity}\label{sec:ttv}
Our SLR aims to be comprehensive, but some limitations should be acknowledged. While we searched popular repositories (IEEE Xplore, ACM Digital Library, Springer, and ScienceDirect) and employed both backward and forward snowballing techniques on recent publications, the possibility of unintentional inclusion or exclusion of relevant studies remains. The authors carefully evaluated publications that fell on the borderline of inclusion/exclusion criteria to mitigate this risk.

Furthermore, our SLR focused exclusively on peer-reviewed journal articles and conference publications published in English. This decision was made to streamline the review process and ensure a high standard of research quality. However, it is important to acknowledge that relevant information may exist in other sources, such as books, theses, and non-English publications, which were not included in this review.

In addition, we deliberately excluded publications that primarily addressed network security, architecture, or systems that utilized SDN as a component. We aimed to maintain a focused review on the software security aspects of SDN itself. This means that studies analyzing SDN's role in broader contexts, such as cloud security or Internet of Things (IoT) networks, were not included. While this approach ensured a clear research focus, it may have overlooked valuable insights on the broader implications of SDN software security.

Overall, while we believe our SLR provides a comprehensive overview of the current state of research on SDN software security, readers should be aware of these limitations when interpreting our findings.
\section{System Design: ApoloBot}
\subsection{Context: Discord and Its Moderation Practices} \label{context}
In this study, we choose Discord as our research site due to its community-oriented social structure and the flexibility of its API for tool design. Unlike traditional platforms that emphasize individual profiles, Discord is centered around the concept of \textit{servers}, where communities are formed from small groups of friends to large circles with millions of people. Originally created for gamers as a third-party voice function, Discord servers now serve a variety of topics such as technology, art, and entertainment. Servers can be public~---~where a link is posted online for anyone to join~---~or private, with invitations more strictly limited. Within these spaces, members interact with each other in \textit{channels}, either through text or voice chat. Channels are areas in the server serving specific purposes, such as announcements, general chats, and topic-specific discussions. Moderators in these spaces are usually volunteers who are active community members, though those in some more formally structured servers may be paid. While their responsibilities vary from one server to another, the overarching goal is to ensure the community's safety and well-being as it grows. Moderators' duties may involve establishing community guidelines, engaging in conversations, and supporting members facing issues within the server. When incidents occur, moderators can take action using Discord's built-in functions like mute, ban, or message removal. Recently, Discord introduced AutoMod,\footnote{https://discord.com/tags/automod} an automated feature that assists moderators with tasks like content filters, action settings, infraction logging, and user verification. To further streamline and customize moderation efforts, third-party bots are extensively adopted, being used by nearly one-third of all Discord servers~\cite{Warren2021}. These bots perform various functions, including tracking server status, managing member activities, and issuing moderation actions, among many others. Popular examples are MEE6,\footnote{https://mee6.xyz/} Zeppelin,\footnote{https://zeppelin.gg/} Tatsu,\footnote{https://tatsu.gg/} and Dyno.\footnote{https://dyno.gg/} These automated solutions help alleviate the workload on human moderators, enabling them to focus on more significant issues while bots manage more routine tasks. Bots can also bring a sense of fun and engagement with features like welcome messages, role assignments, and customized activities. 

In addition to its community structure, Discord offers extensive APIs and interactive features for creating custom bots. These accessible resources add to our decision to utilize the platform and develop ApoloBot~---~a tool that embeds restorative justice principles into online community moderation. %we build a fully functional bot that can be added to any community via an invite link and is adaptable to various server setups. 

%Both of these factors–the widespread adoption of bots and the versatility of Discord development resources–motivate us to build on this technological framework and develop ApoloBot, a moderation tool that embeds restorative justice principles. The following sections detail the system implementation and workflow.

\begin{comment}
\subsection{Design Objectives}
Our research builds upon Xiao et al.’s case study examining the early opportunities and challenges for using restorative justice to address online interpersonal harm, also taking Discord as part of the research settings \cite{Xiao2023}. While the work focuses on a manual approach involving victim-offender conferences mediated by a facilitator, we take this further in a more technical approach. Utilizing ApoloBot as a tool dedicated to streamline the restoration process, we tackle the challenge of resource limitations highlighted in the study, thereby exploring additional factors for its effective implementation. We also draw from the challenges and implications of their work to shape our system’s design goals, using it as a primary reference alongside other related studies that inform our approach.

One of the significant challenges in implementing restorative justice is managing the conflicting interests among and within the involved stakeholders. As seen in Xiao et al.’s work, while several victims were open restorative approaches to address their needs and facilitate healing, concerns were raised regarding offenders’ readiness or whether restorative justice can fully resolve issues, particularly when deeper-rooted problems were involved. Offenders showed very little interest in the restorative process, as their attitude remain oriented toward avoiding punishment rather than acknowledging their wrongdoing and genuinely apologize. Our system should allow spaces for stakeholders to engage in meaningful conversations to the fullest extent possible without compelling their participation, which could lead to negative outcomes. 
\begin{quote}
\indent \textit{\textbf{Design Goal 1 (G1):} ApoloBot should be flexible, being able to accommodate the diverse needs and willingness of involved stakeholders.}
\end{quote}

Moderators are also important stakeholders, often acting as the facilitator guiding the restoration process. They meet offenders and victims to address their needs and discuss ways to resolve the harm. In online communities, they are typically responsible for determining the final outcome for offenders, be it punishment or remedy. However, adhering to restorative justice principles, resolution to harm should primarily be decided by victims and offenders \cite{Bolitho2017}. The facilitator’s main role is to give guidance, protect victims from potential harm or insincere remorse from offenders, and finally intervene only when both parties fail to reach an agreement. Ultimately, a balance is needed: while moderators should take part in the facilitation process, the input of both victims and offenders should also carry significant weight in forming the restoration outcome.

\begin{quote}
\textit{\textbf{Design Goal 2 (G2):} ApoloBot should aid moderators in facilitating the restoration process while also granting certain autonomy to users, ensuring that both victims and offenders have a voice.}
\end{quote}

Finally, the success of a tool is not solely about fulfilling functional purposes; it also depends on its adaptability to different communities’ preferences and values \cite{Jiang2023, Kiene2019}. Learning to adopt new tools presents further barriers such as unmatched tech literacy, misunderstanding with developers, unfamiliar communication norms, and increased workload \cite{Long2017, Kiene2019, Geiger2010, Jhaver2019, Seering2019}. While tools can help in addressing certain challenges, their effectiveness relies on how well it is perceived \cite{Orlikowski1992}, implemented, and maintained \cite{Jhaver2019}.

For restorative justice, these challenges are intensified by the stigma of the prevalent punitive model and extensive labor of restorative practices. To minimize these burdens, our focus is on developing tools that are easy to learn, complementing rather than replacing existing systems. In practice, a dual system incorporating punitive and restorative approaches is sometimes used \cite{Llewellyn1999}.

\begin{quote}
\textit{\textbf{Design Goal 3 (G3):} ApoloBot should be easy to learn and integrate with the existing moderation structure of the servers, serving as a valuable addition rather than a replacement.}
\end{quote}

These design goals represent a new set of guidelines for developing online restorative justice tools that fit within existing systems. We developed ApoloBot based on these guiding principles, and further implementation details will be discussed in the following sections.

\end{comment}

\subsection{System Implementation} \label{system} 
ApoloBot was developed with Javascript and operates on Node.js. It utilizes MongoDB as its database and Heroku as the hosting service. The discord.js library was employed to access necessary Discord APIs for managing user interactions and bot features. The bot’s core functionality is based on the slash command \textit{/apolomute}, for which syntax is shown in Figure \ref{fig:slashcmd}. Slash commands are familiar formats among Discord moderators, where punishments are typically executed via \textit{/mute} or \textit{/ban} commands provided by Discord's built-in system or other bots. Moderators can choose different slash commands based on the situation, allowing \textit{/apolomute} to work alongside other moderation commands. This flexibility makes ApoloBot easy to learn and integrate into any moderator's existing framework.

\begin{figure*}[h!]
  \centering
  \includegraphics[width=0.9\textwidth]{fig/slashcmd.png}
  \caption{The slash command \textit{/apolomute} that is used in the primary workflow. The first four input fields are required, where the moderator specifies the involved offender and victim, along with mute duration and reason. Optionally, proof can be attached as an image, and moderator can choose to first review the victim's apology request by setting `review-request' to True.}
  \Description{The message at the top shows an example of a harmful interaction between two users, involving some insults. Below, there is a chat box where the slash command is inputted. The command begins with a slash (/), followed by its name (apolomute), and includes several required fields: offender, victim, duration, and reason. It also suggests optional fields such as proof and review-request above the chat box.}
  \label{fig:slashcmd}
\end{figure*}

\subsection{Workflow} 
ApoloBot's procedure draws inspiration from the concept of the \textit{victim-offender conference}, a model for online restorative justice practices explored in Xiao et al.’s case study~\cite{Xiao2023}. Within this framework, victims and offenders are encouraged to meet, reflect, and resolve the harm under the guidance of a facilitator, who is the moderator in this context. The moderator's role is to ensure that the process remains safe and constructive, with the final decision ideally determined by the victims and offenders. While the original approach is manual, ApoloBot facilitates a version of this process that is tracked and guided in a style familiar to moderators who are experienced with other Discord bots. 

\revision{
Commonly, commands like \textit{/mute} or \textit{/ban} are utilized as a standardized procedure to impose punishment terms on the offender following an incident of harm. However, this approach might perpetuate a punitive mindset that deters meaningful engagement when switching to restorative justice~\cite{Xiao2023}. Gradual changes are typical in real-world justice systems, and sometimes a dual system incorporating both punitive and restorative measures is employed to cater to the diverse stakeholder needs while adapting to established structures
~\cite{Llewellyn1999}. Building on this, ApoloBot's \textit{/apolomute} extends the conventional mute by offering a potential restorative interaction point, opening an avenue for apology and constructive resolution while retaining certain familiarity with the mute action.} Figure \ref{fig:workflow} outlines an overview of ApoloBot's primary workflow, incorporating the \textit{/apolomute} slash command.

\begin{figure*}[h!]
  \centering
  \includegraphics[width=\linewidth]{fig/workflow-final.png}
  \caption{ApoloBot's Primary Workflow. The diagram shows the different pathways ApoloBot can follow based on stakeholders' decisions to approve or decline their actions. The green blocks represent the interaction points for the moderator, who keeps up with ApoloBot through their log channels. The blue and red ones depict the interaction for the victim and the offender, respectively, in their private threads. Yellow blocks indicate the case is closed and no further steps will be taken.}
  \Description{The start block is in the top left, where "Moderator identifies bad behavior (from offender)" and the workflow is initiated. From here, the process might follow different pathways depending on stakeholders' participation. If everyone approves and responds to ApoloBot prompts, they will reach the end block in the bottom left where "Offender gets released from mute before the specified duration". Otherwise, they end up with the block in the top right, which indicates "Offender gets muted for the specified duration".}
  \label{fig:workflow}
\end{figure*}

The workflow begins similarly to \revision{many current moderator response flows:} upon recognizing inappropriate behavior, the moderator mutes the offender. However, ApoloBot adds a step by involving the victim to initiate the apology process with the offender, and moderators facilitate this by specifying both the victim and the offender in the slash command syntax (Fig \ref{fig:slashcmd}). Once the command is executed, ApoloBot creates two separate threads\footnote{https://support.discord.com/hc/en-us/articles/4403205878423-Threads-FAQ}~---~one for the victim and one for the offender~---~where the interaction between ApoloBot and these stakeholders will take place. For the moderators, they interface with ApoloBot through a dedicated log channel.

In the victim's private thread, ApoloBot informs them that the offender has been muted for harmful behavior and offers the option to request an apology. If chosen, this option grants the offender a second chance to make amends and potentially have the punishment lifted. If the victim chooses to proceed, they are prompted to enter their apology request via a popup textbox (Fig \ref{fig:private-thread-victim}). 

\begin{figure*}[btp]
  \centering
     \begin{subfigure}{\textwidth}
         \centering
         \includegraphics[width=\textwidth]{fig/private-thread-victim.png}
         \caption{Victim's private thread for requesting an apology}
         \label{fig:private-thread-victim}
         \Description{The two pictures illustrate the interaction between ApoloBot and the victim in their private thread. The first picture shows the text message ApoloBot sends, which informs the victim about the case and asks if they want to request an apology. Two buttons below are provided for the victim's decision: "Yes" and "No". The second picture shows the subsequent screen after the victim selects "Yes". ApoloBot displays a popup text box, with a text field for the victim to fill in their apology request. In the bottom right, there is a "Submit" button to send this request.}
     \end{subfigure}
     \hfill
     \begin{subfigure}{\textwidth}
         \centering
         \includegraphics[width=\textwidth]{fig/private-thread-offender.png}
         \caption{Offender's private thread for requesting an apology.}
         \label{fig:private-thread-offender}
         \Description{The two pictures illustrate the interaction between ApoloBot and the offender in their private thread. Similarly, the first picture shows ApoloBot's text message. It contains the quoted apology request from the victim, and asks if the offender wants to respond with an apology. This is followed by two "Yes" and "No" decision buttons. The second picture also shows the next screen after selecting "Yes", where ApoloBot shows a popup text box. The offender fills in their apology response in a text field, and sends this via the "Submit" button.}
     \end{subfigure}
  \caption{Examples of private threads ApoloBot created from the victim's side and the offender's side.}
  \Description{}
  \label{fig:private-threads}
\end{figure*}

Following this, ApoloBot notifies the offender in their private thread that their mute can be lifted if they deliver an appropriate apology to the victim. If the offender decides to comply, they are prompted to write their apology response via a similar popup textbox (Fig \ref{fig:private-thread-offender}). 

Throughout the process, moderators receive updates from ApoloBot at every step. After receiving the apology request and response, they are responsible for reviewing the offender’s response to ensure its appropriateness (Fig \ref{fig:logs}). If approved, the apology is forwarded to the victim, who then has the final say. If the victim accepts the apology, ApoloBot notifies moderators and they can unmute the offender accordingly. 

\begin{figure*}[h!]
  \centering
  \includegraphics[width=0.6\textwidth]{fig/log.png}
  \caption{Examples of ApoloBot logs received by moderators.}
  \Description{The picture shows a series of logging messages sent by ApoloBot to the moderator, providing updates on a specific case. On the top bar, the name of the log channel is shown, followed by the thread name (update-case-5). The main interface presents messages in distinct blocks, each detailing information such as the victim and the offender's name, the current process step, and specific updates like the Victim Request or Offender Response. In the final (bottom) message block, it indicates that everyone in the process has approved, and includes a button labeled "Unmute Offender" that the moderator can click to perform the action.}
  \label{fig:logs}
\end{figure*}

This approach translates the traditional \textit{victim-offender conference} into a technical process via ApoloBot: The interactive exchange between offenders and victims allows each party to voice their perspectives when deciding the restoration outcome, and the facilitation of moderators ensures this process goes smoothly without incurring additional harm. 

At any stage, if the victim, the offender, or the moderator declines to proceed, or if the designated time expires, the process reverts to the standard punitive measures where the offender remains muted for the specified duration. This is in line with real-life restorative practices, where complete consensus may not always be feasible. 
\revision{Forcing forgiveness from victims or remorse from offenders, however, may compromise the victims' autonomy and lead to disingenuous offender responses~\cite{Llewellyn1999, Bazemore2015}.}
The system therefore supports partial participation, ensuring that engagement is voluntary and all stakeholders’ decisions are respected.
\section{Methods}
Using ApoloBot as a discussion starting point, we extend our exploration into the broader landscape of restorative justice tools through a three-phase user study with Discord moderators. Each phase involves increasing levels of commitment, starting with initial interviews, followed by tool deployment, and concluding with reflections. Given that restorative justice tools are still relatively rare in online communities, these separate phases allow us to gather valuable insights while respecting moderators' diverse willingness and interest in the new approach. All parts of this study were pre-approved by our university's Institutional Review Board (IRB).

\subsection{Phases Overview}
%To evaluate the potential for ApoloBot and restorative justice tools more broadly, we conducted a user study with three phases. 

\textbf{Phase 1. Onboarding Session (60-90 minutes):} In the first phase, we conducted individual interviews with Discord moderators to gain insights into their general moderation practices and the potential of integrating restorative justice tools. Participants were asked about their procedure to handling interpersonal harm with specific examples of past cases. We then introduced the concept of restorative justice and presented ApoloBot as a practical tool embodying a subset of these principles. This was followed by discussions on the potential application of ApoloBot and other restorative justice tools within their communities, considering critical factors such as use cases, challenges, opportunities, perceived benefits, and drawbacks. After the interview, participants were invited to a Discord sandbox server to test out ApoloBot, where they provided further feedback and decided whether to continue with the study by deploying it in the subsequent phase.

\textbf{Phase 2. In-the-wild Deployment (4 weeks):} In the second phase, a subset of interested participants deployed ApoloBot in their communities, using it whenever suitable cases arose. Throughout this period, they kept track of their bot usage and maintained weekly communication with the researchers for feedback and support.

\textbf{Phase 3. Exit Interview (60-90 minutes):} At the end of the deployment period, participants joined an exit interview to reflect on their experiences with ApoloBot, unveiling new insights into its practical aspects, including user engagement and its effects on the community. Building on these reflections and revisiting critical factors from Phase 1 interviews, we \revision{explored the underlying factors for how the deployment met or challenged initial expectations, and} broadened the discussion to assess the overall design space of ApoloBot and other online restorative justice tools.

All interviews were conducted remotely through the Discord voice chat function. Participants could withdraw from the study at any phase without penalty. Compensation was provided for fully completed phases: \$20 for Phase 1, \$50 for Phase 2, and \$30 for Phase 3, delivered via Tremendous.~\footnote{https://www.tremendous.com/}


\subsection{Recruitment and Selection of Participants}
%We utilized a combination of platforms to distribute
Our recruitment call was distributed in meta-moderation communities on Discord, Reddit, and Facebook. These are communities where Discord moderators gather to discuss various moderation topics, such as news, strategies, philosophies, and tool usage. To ensure the quality of our recruits, we used a screening survey to assess their background and moderation experience. In addition to project-specific criteria such as prior experience handling interpersonal harm and familiarity with Discord bots, we filtered out low-quality responses such as one-word answers and those containing nonsensical or irrelevant information. We contacted selected participants, and further employed snowball sampling~\cite{Biernacki1981} by asking them for referrals. A total of 16 participants were chosen for Phase 1, with six proceeding to Phases 2 and 3. Two used ApoloBot during their deployment, while the others deployed it but did not encounter any suitable use cases. A summary of the participants' demographics and their status within Phase 1 and 2 are detailed in Table \ref{table:demographics}.

\begin{table}[htbp!]
\resizebox{\columnwidth}{!}{%
\begin{tabular}{@{}l|ccc|c@{}}
 & Liberal & Moderate & Conservative & Total \\ \hline
Female & 223 & 114 & 45 & 382 \\
Male & 102 & 78 & 53 & 233 \\
Prefer not to say & 2 & 0 & 0 & 2 \\ \hline
Total & 327 & 192 & 98 & 617
\end{tabular}%
}
\caption{Annotator Demographics. All annotators are based in the United States. The table shows the number of annotators across ideology and sex categories, as self-reported to Prolific. The mean age is 38.3 (SD=12.7), and 45 annotators are immigrants (7.3\%).}
\label{tab:demographics}
\end{table}



\subsection{Qualitative Analysis}
Interview sessions in Phase 1 and Phase 3 were transcribed using CLOVA Notes. %For Phase 1 interviews, two researchers iteratively conducted thematic analysis\revision{~\cite{Braun2006}}, which involved inductive coding on the data, identifying emerging themes, and grouping into higher hierarchies. We worked together closely during this phase and followed a consensus-coding approach, having consistent meetings to merge individual codes, resolve conflict, and reach agreements on the final codebook. For this reason, calculating inter-rater reliability was not deemed necessary~\cite{McDonald2019}.
\revision{
For Phase 1 interviews, thematic analysis was conducted inductively through multiple iterations~\cite{Braun2006}. First, two researchers individually performed line-by-line open coding on eight interviews, generating initial codes that closely resembled text from the transcript such as “instant ban”, “add to modmail” and “bot seems insincere”. This was followed by focused coding~\cite{Saldaa2021}, where we identified recurring themes and sorted them into broader categories such as “escalating procedure”, “integration into existing system” and “tool perception”, which formed our initial codebook. The first author then applied this codebook across the remaining interviews, refining and adding codes as new insights emerged. After this, the two researchers met again to validate the updated codebook, consolidating higher-level themes along the dimension of moderators’ practices in addressing interpersonal harm, their stances on adopting restorative justice tools through ApoloBot’s framework, and potential impacts of implementing such a system. Finally, we aligned these diverse perspectives to outline the opportunity space and challenges associated with transitioning from traditional moderation practice to integrating restorative justice tools, laying the groundwork for our results.
}

% First-level codes included short phrases similar to text from the transcript, such as “instant ban”, “add to modmail” and “bot seems insincere”. These were then sorted into broader categories such as “escalating procedure”, “integration into existing system” and “tool perception”.

Phase 3 interviews were coded by the first author following a similar inductive process \revision{based on the codebook developed in Phase 1. While Phase 1 interviews focused on moderators’ reflections on prior experiences, Phase 3 expanded upon these by grounding the insights in practical deployment outcomes. Successful use cases from Phase 3 demonstrated how expectations from Phase 1 were met, validating the key opportunities where the tool effectively fulfilled its design intent. Equally significant were the unmet expectations, where anticipated use cases were not realized, as they revealed a new-found understanding of the practical challenges and critical areas of the opportunity space where the tool's effectiveness fell short. These observations were thus incorporated into the final codebook by combining and adding to Phase 1's codes, enhancing the framework that underpins our findings.
}

% Given that some reflections in this phase provided deeper elaborations of insights from Phase 1, we combined and consolidated several categories.

\subsection{Methodological Limitations}
As highlighted by Xiao et al., \textit{"Online communities should allow for partial success or no success without enforcing the ideal outcome, especially at the early stage of implementation when there are insufficient resources or commitments"}~\cite{Xiao2023}. Restorative justice, being relatively new and context-specific, poses significant challenges when evaluated within a brief testing period. Our study is therefore constrained by the limited empirical data available on ApoloBot usage, and the analysis presented here relies mostly on interview data from Phases 1 and 3. 
\revision{
This limitation also arises from how we shape our research focus, which is not on delivering a fully-realized restorative justice tool ready for adoption, but on developing a conceptual artifact to probe its implementation and foster critical reflections among moderators. For those who engaged with the tool, their experiences provide concrete evidence of its realized potential for effective adoption. On the other hand, investigating those who did not use the tool reveals challenges and critical gaps in its suitability within the broader online landscape, which can inform future alternatives or complements that might address the limitations.
Centering the discussion on these dual perspectives allows a deeper and more comprehensive view of how diverse online communities are currently positioned for restorative justice tools, however it might compensate the technical significance of the proposed system.}

% While these interviews offered preliminary insights into the perceived potential of ApoloBot and similar restorative justice tools, they may not offer a comprehensive assessment of their broader significance.

In addition, our study primarily gathers insights from moderators rather than victims or offenders. While this focus offers rich insights into the practical aspects of the tool adoption and execution, it lacks the perspectives of the remaining stakeholders essential to restorative justice, and thus may not fully capture the complete user experience.

Finally, even though our participants come from a wide range of international communities covering diverse topics, the fact that they are solely English speakers limits the cross-cultural generalizations that can be made based on our findings.

\section{Findings: The Opportunity Space for Online Restorative Justice Tools} \label{findings1}
Using ApoloBot as an example and starting point for discussing the broader landscape of online restorative justice tools, we present findings in relation to the opportunities and challenges highlighted by our research questions. First, this section addresses the potential for restorative justice tools \textbf{(RQ1)} by examining the opportunity space--a framework that outlines the conditions and environments where restorative justice tools can be applied effectively and bring positive impacts. Section \ref{findings2} then explores the challenges \textbf{(RQ2)} that accompany the integration of such tools.

We begin by analyzing the opportunity space, which encompasses three main scopes emerging from our interviews, with each serving distinct purposes: \textit{Community}, which influences initial adoption based on the tools' alignment with the community’s values; \textit{Moderation Practices}, determining how well they integrate into existing moderation workflows; and \textit{Case Scenarios}, focusing on the specific situations where they can be most effectively applied.

\subsection{Community: Considerations for Initial Adoption}
The decision to adopt restorative justice tools depends on specific community dynamics and characteristics. This section elaborates on the factors influencing communities' readiness for these tools, highlighting the conditions under which initial adoption is most feasible.

\subsubsection{Community Topics and Culture}
The perceived identity of a community is inseparable from the norms for behavior therein, guiding moderation practices. One recurring theme from participant interviews was the influence of a server’s primary focus or topic on the suitability of ApoloBot adoption. Per participants' responses, communities that are most likely to benefit from a restorative approach are ones with a more social focus or themed around human-centric topics. These include servers built around influential individuals or entities, such as content creators, companies, or collaborative platforms. In these spaces, tools are widely adopted for effective organization, and interaction quality among members is also highly valued as it directly impacts the reputation of the central figure or organization. P7, a content creator who moderates their own community, reflected: 
\begin{quote}

\textit{“I have my personal server that's based around the content I make. Here, me and the moderators try to make sure that we have a united fan base, since the more everyone gets along, the better it is for not only the community but also me as the creator, since I can have healthy and engaging audiences. [...] I noticed apologizing allows for really good reconciliation, so it was nice seeing this kind of system being able to make everyone happy in the end. Or at least in good standing with each other.''}

\end{quote}

Healthy resolutions facilitated by a restorative approach not only help the involved parties reconcile but also reflect positively on the influencer. P3 echoes this perspective: \textit{“[It will] align better with their brand, and it would spread, and people would talk about it, and that reinforces good brand equity.''} 

Similarly, moderators of communities that share common interests in topics valuing human interaction and inclusivity~---~such as language learning, mental health, or arts~---~see tools like ApoloBot as valuable for fostering their growth. P12 described these spaces as \textit{“where people are already intended to talk to each other and try to improve as a person''}. Restoration through apologies facilitates deep layers of communication and empathy, supporting these servers' vision of bettering members' well-being.

Conversely, communities that do not prioritize social interaction may find tools like ApoloBot less relevant. For example, servers dedicated to formal or ephemeral topics, like technical support or quick Q\&A exchange, lacked the sustained engagement and relational depth that restorative tools generally require. P11 provided the examples, stating \textit{“I don't think the idea of this bot helps since people are just in the server to solve a specific problem, or get an answer for a specific question, then just leave.”} In communities like this, members have little to no emotional investment in the server or toward each other: \textit{``People will be like `Why should I even apologize'?''} (P3). On the other hand, communities with large volumes of lower-quality social interaction~---~possibly due to negative norms associated with the genre~---~may also struggle to adopt this type of tool. P4, who moderates a server based on a major competitive game, expressed their frustration:

\begin{quote} \textit{“A lot of people get influenced by social media, and they have this mindset that it’s the norm to be toxic here. It’s very extreme, like if you imagine video games are commonly toxic then this game is ten times that. So they would come in the server with this idea in mind and be toxic, and of course they wouldn't even care to apologize even having the option to do so.”} 
\end{quote}

This highlights a fundamental challenge to the adoption of restorative tools; on platforms where inter-community mobility is high and commitment to each member is low, individuals may not be motivated to resolve conflicts constructively. P14 faced a similar issue moderating another gaming server with predominantly young members, describing that the combination of the game's nature and the immaturity of its players lead to people being \textit{``rebellious''} and \textit{``never feel[ing] remorse about what they do.''} 

In these servers, the high incidence of harm might introduce more opportunities for restorative justice tools to be utilized, yet the expected effectiveness might be limited. P14 explains the situation as having \textit{“a greater risk of people abusing the system, but a greater reward of people actually learning what they're doing wrong.”} In fact, many moderators weighed the ``risks'' more heavily than the ``rewards,'' which was a key factor in their decision not to proceed with deploying ApoloBot in Phase 2. In extreme cases, moderators pointed out that some members might not take it seriously, viewing it as a target for jokes or dismissing apologies as \textit{``weak''} or \textit{``cringe''}, further perpetrating the existing negative dynamics.

\subsubsection{Server Size}
The size of the server may also influence tool adoption: Mid-size servers with steady influx of new members are perceived to be the most ideal. They offer spaces where \textit{“there's enough room for people to disagree and potentially cause harm, and for moderators to adequately handle it''} (P3). In contrast, very large servers may experience overwhelming moderation workloads with fewer social commitments: \textit{``The chat goes 300 miles a second. And we look at something, [take] action, then move on with other stuff.'' }(P4) On the other hand, very small servers simply may not find the need for new tools, since moderators may have the capacity to facilitate conversations without needing technical support.

% In summary, the success of restorative tools adoption is closely tied to the community’s focus and its population size, which determine members' willingness to engage with the tool meaningfully. In places where inclusivity and constructive interactions are valued, ApoloBot can push forward these values, embracing a more supportive environment. This impact is especially pronounced in settings with moderate server sizes. In contrast, in communities where interactions are brief, impersonal, or rooted in unhealthy norms, these tools might not be as effective or welcomed. 

\subsection{Moderation Practice: Fitting into Current Strategies}
Moderators uphold community values through a variety of moderation strategies, which influence whether tools like ApoloBot can be compatible with their existing practices. This section explores how different approaches to moderation~---~viewed as trade-offs rather than strictly positive or negative~---~affect the practical implementation of restorative justice.
% A good fit means that ApoloBot extends moderators' existing practices, while a mismatch indicates potential irrelevance or opposition, as detailed below.

\subsubsection{Mediation Approach: Conversation vs. Action}

One critical perspective involves whether moderators' focus aligns with restorative justice values, or whether constructive conversations or punitive actions are prioritized. Successful integration of restorative justice tools requires more than preference; it involves adapting current practices to accommodate the contextual and emotional processes inherent in restorative justice principles. When discussing ApoloBot, moderators pointed out that it is \textit{``more than a binary decision of yes and no, accept or decline''} (P1), as a sense of sensibility is needed to discern first when to use the tool, and then how to evaluate apologies reasonably. This ability is more likely to be found in moderation teams that have established procedures and skills focused on fostering conversations. For instance, P3's workflow features a ``ladder of escalation'' including a defined set of steps that already align fairly well with principles in restorative justice: Interaction, Education, Action, and Moderation. ApoloBot was perceived to fit well within the first two steps where moderators engage with offenders before escalating to further punishments. By contrast, teams with reactive approaches to moderation may lack these perspectives, potentially leading to inefficient tool use and inadequate engagement from members. While these skills can be developed over time, doing so demands significant mental and physical labor, raising barriers for community moderators who are already stretched thin.

Perhaps surprisingly, our user study revealed that the tool appeals most to moderators who align with its values partially but not completely. Those who highly value interaction often prefer to handle conflicts manually, as P9 noted \textit{``Bots should only be assistive, and the key moderation "tool" should be how you portray yourself, and how you listen to other people''}. Conversely, some action-oriented moderators found potential in the tool. P6, whose moderation team previously experienced burnout from high interaction demands to a small moderation scale, sees ApoloBot as a way to ease this burden and re-engage in meaningful conversations with members.

\subsubsection{Flexibility: Fluid vs. Rule-based}

Moderation flexibility determines adaptability. Although apologies are familiar concepts and generally perceived positively, using them as structured tools to resolve online conflicts is not common. This novelty can be viewed in different ways~---~as an opportunity or as a challenge. For moderation teams with fluid dynamics, where procedures are more casual and less rigid, moderators can readily adjust their procedures to experiment with new approaches. Tools like ApoloBot provide offenders a chance to reconcile with victims, but their use is context-dependent and moderators must determine when to offer this leniency. Flexible teams provide time and space for restorative sessions to evolve, improving as moderators gain experience and trust in the process.

On the other hand, systematic and rule-based moderation structures~---~often found in very large or professional servers~---~may find the addition of tools like ApoloBot burdensome or disruptive. Rules on these servers are usually standardized, \textit{“It’s like black and white. You did this so you receive this, you follow the rules or you don't. There is less space to fit restorative justice in between”} (P3). P4 expressed how handling a large-scale server raises the bar for tool adoption and rule enforcement: 
\begin{quote} \textit{``In these places, being "flexible" might mean being messy since the moderation scale is just too large. Consistency and convenience are therefore things we value the most. ApoloBot poses a problem to both of these because firstly, different moderators might evaluate apologies differently, and secondly, the tool would require its own, independent category, which only certain dedicated moderators can handle. Given we already have so many other things going on with hundreds of channels and millions of members, I don't think this addition would be practical.''}
\end{quote}
Deviation from established norms in these servers requires significantly more commitment and resources, therefore moderators in these spaces are less likely to be enthusiastic about adopting tools like ApoloBot.

\subsubsection{Temporal Perspective: Long-term vs. Short-term Goals}
Finally, the moderators' temporal vision defines whether tools like ApoloBot are regarded as ``efficient'' for their moderation practice. ApoloBot is seen to be not favorably “efficient” in the immediate sense, as it takes time to mediate and evaluate ongoing communications among offenders and victims. Some moderators prefer prompt action, especially in highly active communities where interactions, including harmful ones, progress rapidly. \textit{“I’m just looking through the commands channel real quick and see, three days ago there were some guys spamming and being just weird like [sending] NSFW and the n-word. About three people within the span of not even one hour. Then we banned this guy for racism. A day after someone got warned for baiting, someone got banned for a DM spam. [...] People just do the craziest things, so we have to act fast.”} (P6) In these spaces, the high volume and limited time for action demands in-the-moment responses, reducing opportunities for facilitated discussions. On the other hand, some moderators in a more laid-back environment where immediate interventions are less critical, see the long-term ``efficiency'' in a comprehensive and educative approach. As P7 noted:

\begin{quote}
\textit{``Though it takes some time in the time being, making sure everyone gets along in the long run could potentially be more efficient. Because when you're punishing someone and you don't really care too much about apology, people still hold grudges and that could give a lot of drama and bad actors who can upset even more people. And that can create more moderation cases that could create more work for moderators. So it is likely, that by using this system you are removing a lot of future issues that could possibly happen.''}
\end{quote}

In this sense, moderators might spend additional time upfront, if feasible, which can potentially reduce recidivism and thus ease future moderation workload. This long-term investment allows time for tools like ApoloBot to effectively educate users, ultimately fostering more supportive and resilient communities.

\subsection{Case Scenarios: Conditions for Effective Usage}
Once adopted, restorative justice tools must be applied in the right contexts to maximize their impact. This section examines factors specific to the circumstances under which they can be used to resolve conflicts most effectively.

\subsubsection{Types of Interpersonal Harm} Moderators believe emotional and relational harms~---~such as jokes unintentionally coming off offensive, or criticism that turns into insults~---~are often seen as better intended and less serious, thus more fitting for applying restorative justice. These types of harm generally allow for some autonomy in decision-making, and tools like ApoloBot can provide a safe space for people to discuss their issues away from where the harm occurred, thereby helping to defuse emotions and prevent further rages. On the other hand, physical threats (e.g. doxxing) or financial issues (e.g. scams) are considered more severe harms, requiring more immediate and direct intervention. In these cases, initiating conversation might not be appropriate, as it's unlikely to adequately ``pay back'' the caused damage and could even exacerbate the situation. Moderators note that extreme cases may even require higher authorities, such as Discord support or law enforcement, to step in rather than relying on tools.

\subsubsection{Social Ties among Stakeholders} The effectiveness of restorative justice tools is notably enhanced when social ties are present but not overly strong, such as among new members. In these settings, \textit{``the bond is there to appreciate the restoration yet not too much to go out of their way apologizing for the action.''} (P3). Tools can help facilitate these interactions in a less confrontational manner, potentially repairing and strengthening relationships. P11, who successfully employed ApoloBot in several cases, reflected on this impact:

\begin{quote}
    \textit{``Normally, after someone offended others and their mute expired, we keep an eye on how they interact with people, especially the individual they harmed. So I'm seeing in cases I used ApoloBot, the interactions are different if you apologize versus if you don’t. With normal mutes, how these two individuals react after the incident is that they usually become non-friendly, or they hold their bad emotions since the offender wasn’t involved and encouraged to seek reconciliation. With the bot however, after some people had conflicts and they apologized, you could see them becoming normal to each other again since they got all the emotions out. It's something that made me a little bit happy, seeing people react positively afterward.''}
\end{quote}

P11's social game server, where members actively discuss gameplay though aren't closely connected, greatly benefited from ApoloBot in mending relationships after conflicts. In contrast, the tool was perceived as less beneficial at the extremes: With very weak social ties, both offenders and victims may be less concerned about their engagement due to the temporal and anonymous nature of online interactions, and with very strong ties, the involved users are either close friends who would resolve issues privately or users with ``bad blood'' who might refuse to communicate.

Overall, the opportunity space for restorative justice tools can be understood through three key perspectives: \textit{Where} (in which types of community), \textit{How} (through which moderation practices), and \textit{When} (under which scenarios) they are likely to be more or less beneficial. These factors shape the conditions for effective adoption, practice, and usage, as well as the anticipated outcomes and community reactions.

% These are the factors that define the adoption--when the bot is introduced to the community, practice--when the moderators utilize the tool within their mechanism, and usage--when it is used under specific scenarios. These utility space serves as a comprehensive guideline for the opportunities of where, by who and when ApoloBot and restorative justice tools in general can thrive in online communties. Alongside the opportunities, we would also then examine the challenges of integrating these tools as brought up by moderators through interviews and deployments. Finally, from understanding the processes and their challenges, we derive implications crucial for future work, focusing on the lessons learned from the system and areas for improvement.
\section{Findings: Challenges of Integrating Online Restorative Justice Tools} \label{findings2}
While restorative justice tools may create opportunities within the outlined space, this doesn't come without drawbacks. Our second research question focuses on the challenges of embedding restorative justice principles within a technological framework, 
%deriving insights from participants' discussion and reflection on ApoloBot's deployment.
\revisionn{
informed primarily by participants’ broader discussions during Phase 1 and also by reflections on ApoloBot’s deployment during Phase 3.
}

\subsection{Adapting to the complex and unpredictable nature of interpersonal harm} 
\subsubsection{Contextual Awareness}
One fundamental challenge for restorative justice tools is capturing the contextual nuances of interpersonal harm. Technological tools, including bots, operate based on predefined sets of actions, which can fall short when handling complex human interactions. In the case of ApoloBot, moderators are required to specify one offender and one victim for the apology process. However, participants noted that real-life dynamics are not always so clear-cut. Conflicts can involve multiple offenders and victims, their roles can overlap as the case escalates, or victims may remain completely unidentified. This challenge raises questions of when and how to balance the use of automation with human judgment. 

\subsubsection{Timing} Interpersonal harm often arises spontaneously and escalates unpredictably, making it difficult to determine an optimal time for tool intervention. There exists a niche window for appropriate use: a late response may allow issues to escalate, requiring more serious moderation actions, while early intervention by proactive moderators may negate the need for tools. Similarly, participants highlighted tensions regarding harm frequency: More frequent harm means more chances to utilize these tools, but it may also indicate the server's more permissive norms toward negative behaviors that can diminish successful restorative interventions. Yet too little harm~---~and correspondingly fewer opportunities to use tools like ApoloBot~---~may lead moderators to prefer manual interventions since \textit{``If it only happens on occasion, it's much easier to take care of them ourselves.'' }(P3). However, some participants saw value in having restorative tools as a safeguard, even if rarely used: \textit{``Having a bot is like a preventive measure. You have it in case something happens, that's why everyone has anti-raid even if they've never been raided.''} (P6). 

\subsection{Handling stakeholders' dropouts}
Interpersonal harm involves multiple stakeholders, and beyond their willingness to participate, it's important to consider their willingness to commit to the process throughout. In any tool that requires sustained participation, dropout will occur, and it is not yet known what the impact of partially-completed apologies might be.

\subsubsection{Victims' reasons for dropout} Victims may change their minds as the situation develops, reconsidering whether an apology would suffice to address the harm done, or they may request an apology but then be unsatisfied with the response. In these cases, moderators might need to do follow-up to understand the issue and determine what alternative steps should be taken. The worst case might be when victims receive no response at all to a request for an apology~---~as opposed to a direct negative response~---~since they may lose trust in the system and experience further emotional harm. As P12 noted:
\begin{quote}
    \textit{``Let's say the victim in their request, they were very heartfelt and genuinely wanted an apology, and fully expecting the offender to provide it. But if they can't receive a response, maybe either the offender just ran away with it or they responded in a way that the moderator deemed harmful. It's often disheartening to know that you've kind of bared your heart open to someone and they've taken advantage of it.''}
\end{quote}
\subsubsection{Offenders' reasons for dropout} Offenders unwilling to apologize may not learn from their mistakes, but even those with the intent to apologize may not always do so effectively. Participants compared ApoloBot to an appeal system, a similar framework where banned users are given chances to request unbans. They observed that it is common to receive inadequate appeals, sometimes with satisfactory ones coming after multiple attempts as offenders receive feedback, reflect, and revise their submissions. Therefore it is likely that apologies might not be at their best quality on the first try, requiring further guidance from moderators. Similarly, when offenders want to mend the relationship but their apology is rejected, they may experience significant emotional distress.

These dropout situations necessitate more thorough intervention, as the tool alone may not fully resolve the issue. More extensive follow-up actions beyond merely accepting or declining apologies may be required, or moderators may need to engage in less structured approaches in handling such cases.

\subsection{Overcoming negative perceptions of technological tools} \label{challenge-perception}
Embedding restorative justice in technical tools such as Discord bots can help initiate and facilitate communication among members, yet this kind of mediation might be perceived negatively under certain conditions. Despite human involvement in crafting the messages (apology request and response), the delivery through a bot might reduce its perceived authenticity. P15 highlighted this concern: \textit{``I think it's just how people interpret bot interactions. And we're very much used to chatbots on websites that are not useful and aren't controlled by a real person overseeing them.''} This inherent skepticism towards bots is rooted in their common stereotype that bots are impersonal and unhelpful. On Discord, this is exacerbated by prominent bot issues such as scams, phishing, and hacking, making users highly cautious when interacting with new tools regardless of their purpose. P3 described how initial negative perception can manifest into misconception, which fueled further negativity during their ApoloBot deployment:
\begin{quote}
    \textit{``Unfortunately we didn't quite well inform people about what it was. And that led to a group of people that were in this sort of echo chamber, where they were sharing misinformation about the bot due to their false understanding of it. [...] The primary one was about data collection. From my understanding, they thought that the bot was automatically collecting data about what everyone said, like an AI tool almost. And then that spread between some people and they were concerned, and be like, we don't want that. What they mentioned was just flat out wrong so we had to come in later and correct in greater detail of what data is collected, what the bot is about, and how it works.''}
\end{quote}
As P3 reflected, this incident quickly spread, leading to large-scale resistance among members, even persisting after clarification. This demonstrated how negative perceptions can create lasting barriers to tool adoption, if not addressed early and thoroughly.

% \subsection{Potential exploitations/misuse}
% \todo{remove or combined above}% 

% \begin{itemize}
%     \item People can learn to break the system
%     \item e.g offender use chatgpt to write apology, not (just) harass victim but harass mods; victim attacks / makes fun of offender / reject apology on purpose
%     \item e.g even security - some hackings? to get their apology approved??
% \end{itemize}% 

% Addressing these challenges is crucial to harnessing the full potential of restorative justice tools. To mitigate these obstacles and enhance their effectiveness, we turn to design implications that could mitigate these issues and align the tool more closely with the needs and expectations of its users% 

\section{Discussion of Assumptions}\label{sec:discussion}
In this paper, we have made several assumptions for the sake of clarity and simplicity. In this section, we discuss the rationale behind these assumptions, the extent to which these assumptions hold in practice, and the consequences for our protocol when these assumptions hold.

\subsection{Assumptions on the Demand}

There are two simplifying assumptions we make about the demand. First, we assume the demand at any time is relatively small compared to the channel capacities. Second, we take the demand to be constant over time. We elaborate upon both these points below.

\paragraph{Small demands} The assumption that demands are small relative to channel capacities is made precise in \eqref{eq:large_capacity_assumption}. This assumption simplifies two major aspects of our protocol. First, it largely removes congestion from consideration. In \eqref{eq:primal_problem}, there is no constraint ensuring that total flow in both directions stays below capacity--this is always met. Consequently, there is no Lagrange multiplier for congestion and no congestion pricing; only imbalance penalties apply. In contrast, protocols in \cite{sivaraman2020high, varma2021throughput, wang2024fence} include congestion fees due to explicit congestion constraints. Second, the bound \eqref{eq:large_capacity_assumption} ensures that as long as channels remain balanced, the network can always meet demand, no matter how the demand is routed. Since channels can rebalance when necessary, they never drop transactions. This allows prices and flows to adjust as per the equations in \eqref{eq:algorithm}, which makes it easier to prove the protocol's convergence guarantees. This also preserves the key property that a channel's price remains proportional to net money flow through it.

In practice, payment channel networks are used most often for micro-payments, for which on-chain transactions are prohibitively expensive; large transactions typically take place directly on the blockchain. For example, according to \cite{river2023lightning}, the average channel capacity is roughly $0.1$ BTC ($5,000$ BTC distributed over $50,000$ channels), while the average transaction amount is less than $0.0004$ BTC ($44.7k$ satoshis). Thus, the small demand assumption is not too unrealistic. Additionally, the occasional large transaction can be treated as a sequence of smaller transactions by breaking it into packets and executing each packet serially (as done by \cite{sivaraman2020high}).
Lastly, a good path discovery process that favors large capacity channels over small capacity ones can help ensure that the bound in \eqref{eq:large_capacity_assumption} holds.

\paragraph{Constant demands} 
In this work, we assume that any transacting pair of nodes have a steady transaction demand between them (see Section \ref{sec:transaction_requests}). Making this assumption is necessary to obtain the kind of guarantees that we have presented in this paper. Unless the demand is steady, it is unreasonable to expect that the flows converge to a steady value. Weaker assumptions on the demand lead to weaker guarantees. For example, with the more general setting of stochastic, but i.i.d. demand between any two nodes, \cite{varma2021throughput} shows that the channel queue lengths are bounded in expectation. If the demand can be arbitrary, then it is very hard to get any meaningful performance guarantees; \cite{wang2024fence} shows that even for a single bidirectional channel, the competitive ratio is infinite. Indeed, because a PCN is a decentralized system and decisions must be made based on local information alone, it is difficult for the network to find the optimal detailed balance flow at every time step with a time-varying demand.  With a steady demand, the network can discover the optimal flows in a reasonably short time, as our work shows.

We view the constant demand assumption as an approximation for a more general demand process that could be piece-wise constant, stochastic, or both (see simulations in Figure \ref{fig:five_nodes_variable_demand}).
We believe it should be possible to merge ideas from our work and \cite{varma2021throughput} to provide guarantees in a setting with random demands with arbitrary means. We leave this for future work. In addition, our work suggests that a reasonable method of handling stochastic demands is to queue the transaction requests \textit{at the source node} itself. This queuing action should be viewed in conjunction with flow-control. Indeed, a temporarily high unidirectional demand would raise prices for the sender, incentivizing the sender to stop sending the transactions. If the sender queues the transactions, they can send them later when prices drop. This form of queuing does not require any overhaul of the basic PCN infrastructure and is therefore simpler to implement than per-channel queues as suggested by \cite{sivaraman2020high} and \cite{varma2021throughput}.

\subsection{The Incentive of Channels}
The actions of the channels as prescribed by the DEBT control protocol can be summarized as follows. Channels adjust their prices in proportion to the net flow through them. They rebalance themselves whenever necessary and execute any transaction request that has been made of them. We discuss both these aspects below.

\paragraph{On Prices}
In this work, the exclusive role of channel prices is to ensure that the flows through each channel remains balanced. In practice, it would be important to include other components in a channel's price/fee as well: a congestion price  and an incentive price. The congestion price, as suggested by \cite{varma2021throughput}, would depend on the total flow of transactions through the channel, and would incentivize nodes to balance the load over different paths. The incentive price, which is commonly used in practice \cite{river2023lightning}, is necessary to provide channels with an incentive to serve as an intermediary for different channels. In practice, we expect both these components to be smaller than the imbalance price. Consequently, we expect the behavior of our protocol to be similar to our theoretical results even with these additional prices.

A key aspect of our protocol is that channel fees are allowed to be negative. Although the original Lightning network whitepaper \cite{poon2016bitcoin} suggests that negative channel prices may be a good solution to promote rebalancing, the idea of negative prices in not very popular in the literature. To our knowledge, the only prior work with this feature is \cite{varma2021throughput}. Indeed, in papers such as \cite{van2021merchant} and \cite{wang2024fence}, the price function is explicitly modified such that the channel price is never negative. The results of our paper show the benefits of negative prices. For one, in steady state, equal flows in both directions ensure that a channel doesn't loose any money (the other price components mentioned above ensure that the channel will only gain money). More importantly, negative prices are important to ensure that the protocol selectively stifles acyclic flows while allowing circulations to flow. Indeed, in the example of Section \ref{sec:flow_control_example}, the flows between nodes $A$ and $C$ are left on only because the large positive price over one channel is canceled by the corresponding negative price over the other channel, leading to a net zero price.

Lastly, observe that in the DEBT control protocol, the price charged by a channel does not depend on its capacity. This is a natural consequence of the price being the Lagrange multiplier for the net-zero flow constraint, which also does not depend on the channel capacity. In contrast, in many other works, the imbalance price is normalized by the channel capacity \cite{ren2018optimal, lin2020funds, wang2024fence}; this is shown to work well in practice. The rationale for such a price structure is explained well in \cite{wang2024fence}, where this fee is derived with the aim of always maintaining some balance (liquidity) at each end of every channel. This is a reasonable aim if a channel is to never rebalance itself; the experiments of the aforementioned papers are conducted in such a regime. In this work, however, we allow the channels to rebalance themselves a few times in order to settle on a detailed balance flow. This is because our focus is on the long-term steady state performance of the protocol. This difference in perspective also shows up in how the price depends on the channel imbalance. \cite{lin2020funds} and \cite{wang2024fence} advocate for strictly convex prices whereas this work and \cite{varma2021throughput} propose linear prices.

\paragraph{On Rebalancing} 
Recall that the DEBT control protocol ensures that the flows in the network converge to a detailed balance flow, which can be sustained perpetually without any rebalancing. However, during the transient phase (before convergence), channels may have to perform on-chain rebalancing a few times. Since rebalancing is an expensive operation, it is worthwhile discussing methods by which channels can reduce the extent of rebalancing. One option for the channels to reduce the extent of rebalancing is to increase their capacity; however, this comes at the cost of locking in more capital. Each channel can decide for itself the optimum amount of capital to lock in. Another option, which we discuss in Section \ref{sec:five_node}, is for channels to increase the rate $\gamma$ at which they adjust prices. 

Ultimately, whether or not it is beneficial for a channel to rebalance depends on the time-horizon under consideration. Our protocol is based on the assumption that the demand remains steady for a long period of time. If this is indeed the case, it would be worthwhile for a channel to rebalance itself as it can make up this cost through the incentive fees gained from the flow of transactions through it in steady state. If a channel chooses not to rebalance itself, however, there is a risk of being trapped in a deadlock, which is suboptimal for not only the nodes but also the channel.

\section{Conclusion}
This work presents DEBT control: a protocol for payment channel networks that uses source routing and flow control based on channel prices. The protocol is derived by posing a network utility maximization problem and analyzing its dual minimization. It is shown that under steady demands, the protocol guides the network to an optimal, sustainable point. Simulations show its robustness to demand variations. The work demonstrates that simple protocols with strong theoretical guarantees are possible for PCNs and we hope it inspires further theoretical research in this direction.
\section{Conclusion}
In this work, we propose a simple yet effective approach, called SMILE, for graph few-shot learning with fewer tasks. Specifically, we introduce a novel dual-level mixup strategy, including within-task and across-task mixup, for enriching the diversity of nodes within each task and the diversity of tasks. Also, we incorporate the degree-based prior information to learn expressive node embeddings. Theoretically, we prove that SMILE effectively enhances the model's generalization performance. Empirically, we conduct extensive experiments on multiple benchmarks and the results suggest that SMILE significantly outperforms other baselines, including both in-domain and cross-domain few-shot settings.

%%
%% The acknowledgments section is defined using the "acks" environment
%% (and NOT an unnumbered section). This ensures the proper
%% identification of the section in the article metadata, and the
%% consistent spelling of the heading.
\begin{acks}
This work was supported by the National Research Foundation of Korea (NRF) grant funded by the Korea government (MSIT) (RS-2024-00348993), as well as by a KAIST Undergraduate Research Program grant. EPFL CDH also provided financial support for the first author to attend the conference and present the work. We thank the community moderators for their active engagement and thoughtful feedback throughout the study. We also thank all members of CSTL for their constructive discussions and invaluable support.
\end{acks}

%%
%% The next two lines define the bibliography style to be used, and
%% the bibliography file.
\bibliographystyle{ACM-Reference-Format}
\bibliography{main}

%TC:ignore
\appendix
\subsection{Lloyd-Max Algorithm}
\label{subsec:Lloyd-Max}
For a given quantization bitwidth $B$ and an operand $\bm{X}$, the Lloyd-Max algorithm finds $2^B$ quantization levels $\{\hat{x}_i\}_{i=1}^{2^B}$ such that quantizing $\bm{X}$ by rounding each scalar in $\bm{X}$ to the nearest quantization level minimizes the quantization MSE. 

The algorithm starts with an initial guess of quantization levels and then iteratively computes quantization thresholds $\{\tau_i\}_{i=1}^{2^B-1}$ and updates quantization levels $\{\hat{x}_i\}_{i=1}^{2^B}$. Specifically, at iteration $n$, thresholds are set to the midpoints of the previous iteration's levels:
\begin{align*}
    \tau_i^{(n)}=\frac{\hat{x}_i^{(n-1)}+\hat{x}_{i+1}^{(n-1)}}2 \text{ for } i=1\ldots 2^B-1
\end{align*}
Subsequently, the quantization levels are re-computed as conditional means of the data regions defined by the new thresholds:
\begin{align*}
    \hat{x}_i^{(n)}=\mathbb{E}\left[ \bm{X} \big| \bm{X}\in [\tau_{i-1}^{(n)},\tau_i^{(n)}] \right] \text{ for } i=1\ldots 2^B
\end{align*}
where to satisfy boundary conditions we have $\tau_0=-\infty$ and $\tau_{2^B}=\infty$. The algorithm iterates the above steps until convergence.

Figure \ref{fig:lm_quant} compares the quantization levels of a $7$-bit floating point (E3M3) quantizer (left) to a $7$-bit Lloyd-Max quantizer (right) when quantizing a layer of weights from the GPT3-126M model at a per-tensor granularity. As shown, the Lloyd-Max quantizer achieves substantially lower quantization MSE. Further, Table \ref{tab:FP7_vs_LM7} shows the superior perplexity achieved by Lloyd-Max quantizers for bitwidths of $7$, $6$ and $5$. The difference between the quantizers is clear at 5 bits, where per-tensor FP quantization incurs a drastic and unacceptable increase in perplexity, while Lloyd-Max quantization incurs a much smaller increase. Nevertheless, we note that even the optimal Lloyd-Max quantizer incurs a notable ($\sim 1.5$) increase in perplexity due to the coarse granularity of quantization. 

\begin{figure}[h]
  \centering
  \includegraphics[width=0.7\linewidth]{sections/figures/LM7_FP7.pdf}
  \caption{\small Quantization levels and the corresponding quantization MSE of Floating Point (left) vs Lloyd-Max (right) Quantizers for a layer of weights in the GPT3-126M model.}
  \label{fig:lm_quant}
\end{figure}

\begin{table}[h]\scriptsize
\begin{center}
\caption{\label{tab:FP7_vs_LM7} \small Comparing perplexity (lower is better) achieved by floating point quantizers and Lloyd-Max quantizers on a GPT3-126M model for the Wikitext-103 dataset.}
\begin{tabular}{c|cc|c}
\hline
 \multirow{2}{*}{\textbf{Bitwidth}} & \multicolumn{2}{|c|}{\textbf{Floating-Point Quantizer}} & \textbf{Lloyd-Max Quantizer} \\
 & Best Format & Wikitext-103 Perplexity & Wikitext-103 Perplexity \\
\hline
7 & E3M3 & 18.32 & 18.27 \\
6 & E3M2 & 19.07 & 18.51 \\
5 & E4M0 & 43.89 & 19.71 \\
\hline
\end{tabular}
\end{center}
\end{table}

\subsection{Proof of Local Optimality of LO-BCQ}
\label{subsec:lobcq_opt_proof}
For a given block $\bm{b}_j$, the quantization MSE during LO-BCQ can be empirically evaluated as $\frac{1}{L_b}\lVert \bm{b}_j- \bm{\hat{b}}_j\rVert^2_2$ where $\bm{\hat{b}}_j$ is computed from equation (\ref{eq:clustered_quantization_definition}) as $C_{f(\bm{b}_j)}(\bm{b}_j)$. Further, for a given block cluster $\mathcal{B}_i$, we compute the quantization MSE as $\frac{1}{|\mathcal{B}_{i}|}\sum_{\bm{b} \in \mathcal{B}_{i}} \frac{1}{L_b}\lVert \bm{b}- C_i^{(n)}(\bm{b})\rVert^2_2$. Therefore, at the end of iteration $n$, we evaluate the overall quantization MSE $J^{(n)}$ for a given operand $\bm{X}$ composed of $N_c$ block clusters as:
\begin{align*}
    \label{eq:mse_iter_n}
    J^{(n)} = \frac{1}{N_c} \sum_{i=1}^{N_c} \frac{1}{|\mathcal{B}_{i}^{(n)}|}\sum_{\bm{v} \in \mathcal{B}_{i}^{(n)}} \frac{1}{L_b}\lVert \bm{b}- B_i^{(n)}(\bm{b})\rVert^2_2
\end{align*}

At the end of iteration $n$, the codebooks are updated from $\mathcal{C}^{(n-1)}$ to $\mathcal{C}^{(n)}$. However, the mapping of a given vector $\bm{b}_j$ to quantizers $\mathcal{C}^{(n)}$ remains as  $f^{(n)}(\bm{b}_j)$. At the next iteration, during the vector clustering step, $f^{(n+1)}(\bm{b}_j)$ finds new mapping of $\bm{b}_j$ to updated codebooks $\mathcal{C}^{(n)}$ such that the quantization MSE over the candidate codebooks is minimized. Therefore, we obtain the following result for $\bm{b}_j$:
\begin{align*}
\frac{1}{L_b}\lVert \bm{b}_j - C_{f^{(n+1)}(\bm{b}_j)}^{(n)}(\bm{b}_j)\rVert^2_2 \le \frac{1}{L_b}\lVert \bm{b}_j - C_{f^{(n)}(\bm{b}_j)}^{(n)}(\bm{b}_j)\rVert^2_2
\end{align*}

That is, quantizing $\bm{b}_j$ at the end of the block clustering step of iteration $n+1$ results in lower quantization MSE compared to quantizing at the end of iteration $n$. Since this is true for all $\bm{b} \in \bm{X}$, we assert the following:
\begin{equation}
\begin{split}
\label{eq:mse_ineq_1}
    \tilde{J}^{(n+1)} &= \frac{1}{N_c} \sum_{i=1}^{N_c} \frac{1}{|\mathcal{B}_{i}^{(n+1)}|}\sum_{\bm{b} \in \mathcal{B}_{i}^{(n+1)}} \frac{1}{L_b}\lVert \bm{b} - C_i^{(n)}(b)\rVert^2_2 \le J^{(n)}
\end{split}
\end{equation}
where $\tilde{J}^{(n+1)}$ is the the quantization MSE after the vector clustering step at iteration $n+1$.

Next, during the codebook update step (\ref{eq:quantizers_update}) at iteration $n+1$, the per-cluster codebooks $\mathcal{C}^{(n)}$ are updated to $\mathcal{C}^{(n+1)}$ by invoking the Lloyd-Max algorithm \citep{Lloyd}. We know that for any given value distribution, the Lloyd-Max algorithm minimizes the quantization MSE. Therefore, for a given vector cluster $\mathcal{B}_i$ we obtain the following result:

\begin{equation}
    \frac{1}{|\mathcal{B}_{i}^{(n+1)}|}\sum_{\bm{b} \in \mathcal{B}_{i}^{(n+1)}} \frac{1}{L_b}\lVert \bm{b}- C_i^{(n+1)}(\bm{b})\rVert^2_2 \le \frac{1}{|\mathcal{B}_{i}^{(n+1)}|}\sum_{\bm{b} \in \mathcal{B}_{i}^{(n+1)}} \frac{1}{L_b}\lVert \bm{b}- C_i^{(n)}(\bm{b})\rVert^2_2
\end{equation}

The above equation states that quantizing the given block cluster $\mathcal{B}_i$ after updating the associated codebook from $C_i^{(n)}$ to $C_i^{(n+1)}$ results in lower quantization MSE. Since this is true for all the block clusters, we derive the following result: 
\begin{equation}
\begin{split}
\label{eq:mse_ineq_2}
     J^{(n+1)} &= \frac{1}{N_c} \sum_{i=1}^{N_c} \frac{1}{|\mathcal{B}_{i}^{(n+1)}|}\sum_{\bm{b} \in \mathcal{B}_{i}^{(n+1)}} \frac{1}{L_b}\lVert \bm{b}- C_i^{(n+1)}(\bm{b})\rVert^2_2  \le \tilde{J}^{(n+1)}   
\end{split}
\end{equation}

Following (\ref{eq:mse_ineq_1}) and (\ref{eq:mse_ineq_2}), we find that the quantization MSE is non-increasing for each iteration, that is, $J^{(1)} \ge J^{(2)} \ge J^{(3)} \ge \ldots \ge J^{(M)}$ where $M$ is the maximum number of iterations. 
%Therefore, we can say that if the algorithm converges, then it must be that it has converged to a local minimum. 
\hfill $\blacksquare$


\begin{figure}
    \begin{center}
    \includegraphics[width=0.5\textwidth]{sections//figures/mse_vs_iter.pdf}
    \end{center}
    \caption{\small NMSE vs iterations during LO-BCQ compared to other block quantization proposals}
    \label{fig:nmse_vs_iter}
\end{figure}

Figure \ref{fig:nmse_vs_iter} shows the empirical convergence of LO-BCQ across several block lengths and number of codebooks. Also, the MSE achieved by LO-BCQ is compared to baselines such as MXFP and VSQ. As shown, LO-BCQ converges to a lower MSE than the baselines. Further, we achieve better convergence for larger number of codebooks ($N_c$) and for a smaller block length ($L_b$), both of which increase the bitwidth of BCQ (see Eq \ref{eq:bitwidth_bcq}).


\subsection{Additional Accuracy Results}
%Table \ref{tab:lobcq_config} lists the various LOBCQ configurations and their corresponding bitwidths.
\begin{table}
\setlength{\tabcolsep}{4.75pt}
\begin{center}
\caption{\label{tab:lobcq_config} Various LO-BCQ configurations and their bitwidths.}
\begin{tabular}{|c||c|c|c|c||c|c||c|} 
\hline
 & \multicolumn{4}{|c||}{$L_b=8$} & \multicolumn{2}{|c||}{$L_b=4$} & $L_b=2$ \\
 \hline
 \backslashbox{$L_A$\kern-1em}{\kern-1em$N_c$} & 2 & 4 & 8 & 16 & 2 & 4 & 2 \\
 \hline
 64 & 4.25 & 4.375 & 4.5 & 4.625 & 4.375 & 4.625 & 4.625\\
 \hline
 32 & 4.375 & 4.5 & 4.625& 4.75 & 4.5 & 4.75 & 4.75 \\
 \hline
 16 & 4.625 & 4.75& 4.875 & 5 & 4.75 & 5 & 5 \\
 \hline
\end{tabular}
\end{center}
\end{table}

%\subsection{Perplexity achieved by various LO-BCQ configurations on Wikitext-103 dataset}

\begin{table} \centering
\begin{tabular}{|c||c|c|c|c||c|c||c|} 
\hline
 $L_b \rightarrow$& \multicolumn{4}{c||}{8} & \multicolumn{2}{c||}{4} & 2\\
 \hline
 \backslashbox{$L_A$\kern-1em}{\kern-1em$N_c$} & 2 & 4 & 8 & 16 & 2 & 4 & 2  \\
 %$N_c \rightarrow$ & 2 & 4 & 8 & 16 & 2 & 4 & 2 \\
 \hline
 \hline
 \multicolumn{8}{c}{GPT3-1.3B (FP32 PPL = 9.98)} \\ 
 \hline
 \hline
 64 & 10.40 & 10.23 & 10.17 & 10.15 &  10.28 & 10.18 & 10.19 \\
 \hline
 32 & 10.25 & 10.20 & 10.15 & 10.12 &  10.23 & 10.17 & 10.17 \\
 \hline
 16 & 10.22 & 10.16 & 10.10 & 10.09 &  10.21 & 10.14 & 10.16 \\
 \hline
  \hline
 \multicolumn{8}{c}{GPT3-8B (FP32 PPL = 7.38)} \\ 
 \hline
 \hline
 64 & 7.61 & 7.52 & 7.48 &  7.47 &  7.55 &  7.49 & 7.50 \\
 \hline
 32 & 7.52 & 7.50 & 7.46 &  7.45 &  7.52 &  7.48 & 7.48  \\
 \hline
 16 & 7.51 & 7.48 & 7.44 &  7.44 &  7.51 &  7.49 & 7.47  \\
 \hline
\end{tabular}
\caption{\label{tab:ppl_gpt3_abalation} Wikitext-103 perplexity across GPT3-1.3B and 8B models.}
\end{table}

\begin{table} \centering
\begin{tabular}{|c||c|c|c|c||} 
\hline
 $L_b \rightarrow$& \multicolumn{4}{c||}{8}\\
 \hline
 \backslashbox{$L_A$\kern-1em}{\kern-1em$N_c$} & 2 & 4 & 8 & 16 \\
 %$N_c \rightarrow$ & 2 & 4 & 8 & 16 & 2 & 4 & 2 \\
 \hline
 \hline
 \multicolumn{5}{|c|}{Llama2-7B (FP32 PPL = 5.06)} \\ 
 \hline
 \hline
 64 & 5.31 & 5.26 & 5.19 & 5.18  \\
 \hline
 32 & 5.23 & 5.25 & 5.18 & 5.15  \\
 \hline
 16 & 5.23 & 5.19 & 5.16 & 5.14  \\
 \hline
 \multicolumn{5}{|c|}{Nemotron4-15B (FP32 PPL = 5.87)} \\ 
 \hline
 \hline
 64  & 6.3 & 6.20 & 6.13 & 6.08  \\
 \hline
 32  & 6.24 & 6.12 & 6.07 & 6.03  \\
 \hline
 16  & 6.12 & 6.14 & 6.04 & 6.02  \\
 \hline
 \multicolumn{5}{|c|}{Nemotron4-340B (FP32 PPL = 3.48)} \\ 
 \hline
 \hline
 64 & 3.67 & 3.62 & 3.60 & 3.59 \\
 \hline
 32 & 3.63 & 3.61 & 3.59 & 3.56 \\
 \hline
 16 & 3.61 & 3.58 & 3.57 & 3.55 \\
 \hline
\end{tabular}
\caption{\label{tab:ppl_llama7B_nemo15B} Wikitext-103 perplexity compared to FP32 baseline in Llama2-7B and Nemotron4-15B, 340B models}
\end{table}

%\subsection{Perplexity achieved by various LO-BCQ configurations on MMLU dataset}


\begin{table} \centering
\begin{tabular}{|c||c|c|c|c||c|c|c|c|} 
\hline
 $L_b \rightarrow$& \multicolumn{4}{c||}{8} & \multicolumn{4}{c||}{8}\\
 \hline
 \backslashbox{$L_A$\kern-1em}{\kern-1em$N_c$} & 2 & 4 & 8 & 16 & 2 & 4 & 8 & 16  \\
 %$N_c \rightarrow$ & 2 & 4 & 8 & 16 & 2 & 4 & 2 \\
 \hline
 \hline
 \multicolumn{5}{|c|}{Llama2-7B (FP32 Accuracy = 45.8\%)} & \multicolumn{4}{|c|}{Llama2-70B (FP32 Accuracy = 69.12\%)} \\ 
 \hline
 \hline
 64 & 43.9 & 43.4 & 43.9 & 44.9 & 68.07 & 68.27 & 68.17 & 68.75 \\
 \hline
 32 & 44.5 & 43.8 & 44.9 & 44.5 & 68.37 & 68.51 & 68.35 & 68.27  \\
 \hline
 16 & 43.9 & 42.7 & 44.9 & 45 & 68.12 & 68.77 & 68.31 & 68.59  \\
 \hline
 \hline
 \multicolumn{5}{|c|}{GPT3-22B (FP32 Accuracy = 38.75\%)} & \multicolumn{4}{|c|}{Nemotron4-15B (FP32 Accuracy = 64.3\%)} \\ 
 \hline
 \hline
 64 & 36.71 & 38.85 & 38.13 & 38.92 & 63.17 & 62.36 & 63.72 & 64.09 \\
 \hline
 32 & 37.95 & 38.69 & 39.45 & 38.34 & 64.05 & 62.30 & 63.8 & 64.33  \\
 \hline
 16 & 38.88 & 38.80 & 38.31 & 38.92 & 63.22 & 63.51 & 63.93 & 64.43  \\
 \hline
\end{tabular}
\caption{\label{tab:mmlu_abalation} Accuracy on MMLU dataset across GPT3-22B, Llama2-7B, 70B and Nemotron4-15B models.}
\end{table}


%\subsection{Perplexity achieved by various LO-BCQ configurations on LM evaluation harness}

\begin{table} \centering
\begin{tabular}{|c||c|c|c|c||c|c|c|c|} 
\hline
 $L_b \rightarrow$& \multicolumn{4}{c||}{8} & \multicolumn{4}{c||}{8}\\
 \hline
 \backslashbox{$L_A$\kern-1em}{\kern-1em$N_c$} & 2 & 4 & 8 & 16 & 2 & 4 & 8 & 16  \\
 %$N_c \rightarrow$ & 2 & 4 & 8 & 16 & 2 & 4 & 2 \\
 \hline
 \hline
 \multicolumn{5}{|c|}{Race (FP32 Accuracy = 37.51\%)} & \multicolumn{4}{|c|}{Boolq (FP32 Accuracy = 64.62\%)} \\ 
 \hline
 \hline
 64 & 36.94 & 37.13 & 36.27 & 37.13 & 63.73 & 62.26 & 63.49 & 63.36 \\
 \hline
 32 & 37.03 & 36.36 & 36.08 & 37.03 & 62.54 & 63.51 & 63.49 & 63.55  \\
 \hline
 16 & 37.03 & 37.03 & 36.46 & 37.03 & 61.1 & 63.79 & 63.58 & 63.33  \\
 \hline
 \hline
 \multicolumn{5}{|c|}{Winogrande (FP32 Accuracy = 58.01\%)} & \multicolumn{4}{|c|}{Piqa (FP32 Accuracy = 74.21\%)} \\ 
 \hline
 \hline
 64 & 58.17 & 57.22 & 57.85 & 58.33 & 73.01 & 73.07 & 73.07 & 72.80 \\
 \hline
 32 & 59.12 & 58.09 & 57.85 & 58.41 & 73.01 & 73.94 & 72.74 & 73.18  \\
 \hline
 16 & 57.93 & 58.88 & 57.93 & 58.56 & 73.94 & 72.80 & 73.01 & 73.94  \\
 \hline
\end{tabular}
\caption{\label{tab:mmlu_abalation} Accuracy on LM evaluation harness tasks on GPT3-1.3B model.}
\end{table}

\begin{table} \centering
\begin{tabular}{|c||c|c|c|c||c|c|c|c|} 
\hline
 $L_b \rightarrow$& \multicolumn{4}{c||}{8} & \multicolumn{4}{c||}{8}\\
 \hline
 \backslashbox{$L_A$\kern-1em}{\kern-1em$N_c$} & 2 & 4 & 8 & 16 & 2 & 4 & 8 & 16  \\
 %$N_c \rightarrow$ & 2 & 4 & 8 & 16 & 2 & 4 & 2 \\
 \hline
 \hline
 \multicolumn{5}{|c|}{Race (FP32 Accuracy = 41.34\%)} & \multicolumn{4}{|c|}{Boolq (FP32 Accuracy = 68.32\%)} \\ 
 \hline
 \hline
 64 & 40.48 & 40.10 & 39.43 & 39.90 & 69.20 & 68.41 & 69.45 & 68.56 \\
 \hline
 32 & 39.52 & 39.52 & 40.77 & 39.62 & 68.32 & 67.43 & 68.17 & 69.30  \\
 \hline
 16 & 39.81 & 39.71 & 39.90 & 40.38 & 68.10 & 66.33 & 69.51 & 69.42  \\
 \hline
 \hline
 \multicolumn{5}{|c|}{Winogrande (FP32 Accuracy = 67.88\%)} & \multicolumn{4}{|c|}{Piqa (FP32 Accuracy = 78.78\%)} \\ 
 \hline
 \hline
 64 & 66.85 & 66.61 & 67.72 & 67.88 & 77.31 & 77.42 & 77.75 & 77.64 \\
 \hline
 32 & 67.25 & 67.72 & 67.72 & 67.00 & 77.31 & 77.04 & 77.80 & 77.37  \\
 \hline
 16 & 68.11 & 68.90 & 67.88 & 67.48 & 77.37 & 78.13 & 78.13 & 77.69  \\
 \hline
\end{tabular}
\caption{\label{tab:mmlu_abalation} Accuracy on LM evaluation harness tasks on GPT3-8B model.}
\end{table}

\begin{table} \centering
\begin{tabular}{|c||c|c|c|c||c|c|c|c|} 
\hline
 $L_b \rightarrow$& \multicolumn{4}{c||}{8} & \multicolumn{4}{c||}{8}\\
 \hline
 \backslashbox{$L_A$\kern-1em}{\kern-1em$N_c$} & 2 & 4 & 8 & 16 & 2 & 4 & 8 & 16  \\
 %$N_c \rightarrow$ & 2 & 4 & 8 & 16 & 2 & 4 & 2 \\
 \hline
 \hline
 \multicolumn{5}{|c|}{Race (FP32 Accuracy = 40.67\%)} & \multicolumn{4}{|c|}{Boolq (FP32 Accuracy = 76.54\%)} \\ 
 \hline
 \hline
 64 & 40.48 & 40.10 & 39.43 & 39.90 & 75.41 & 75.11 & 77.09 & 75.66 \\
 \hline
 32 & 39.52 & 39.52 & 40.77 & 39.62 & 76.02 & 76.02 & 75.96 & 75.35  \\
 \hline
 16 & 39.81 & 39.71 & 39.90 & 40.38 & 75.05 & 73.82 & 75.72 & 76.09  \\
 \hline
 \hline
 \multicolumn{5}{|c|}{Winogrande (FP32 Accuracy = 70.64\%)} & \multicolumn{4}{|c|}{Piqa (FP32 Accuracy = 79.16\%)} \\ 
 \hline
 \hline
 64 & 69.14 & 70.17 & 70.17 & 70.56 & 78.24 & 79.00 & 78.62 & 78.73 \\
 \hline
 32 & 70.96 & 69.69 & 71.27 & 69.30 & 78.56 & 79.49 & 79.16 & 78.89  \\
 \hline
 16 & 71.03 & 69.53 & 69.69 & 70.40 & 78.13 & 79.16 & 79.00 & 79.00  \\
 \hline
\end{tabular}
\caption{\label{tab:mmlu_abalation} Accuracy on LM evaluation harness tasks on GPT3-22B model.}
\end{table}

\begin{table} \centering
\begin{tabular}{|c||c|c|c|c||c|c|c|c|} 
\hline
 $L_b \rightarrow$& \multicolumn{4}{c||}{8} & \multicolumn{4}{c||}{8}\\
 \hline
 \backslashbox{$L_A$\kern-1em}{\kern-1em$N_c$} & 2 & 4 & 8 & 16 & 2 & 4 & 8 & 16  \\
 %$N_c \rightarrow$ & 2 & 4 & 8 & 16 & 2 & 4 & 2 \\
 \hline
 \hline
 \multicolumn{5}{|c|}{Race (FP32 Accuracy = 44.4\%)} & \multicolumn{4}{|c|}{Boolq (FP32 Accuracy = 79.29\%)} \\ 
 \hline
 \hline
 64 & 42.49 & 42.51 & 42.58 & 43.45 & 77.58 & 77.37 & 77.43 & 78.1 \\
 \hline
 32 & 43.35 & 42.49 & 43.64 & 43.73 & 77.86 & 75.32 & 77.28 & 77.86  \\
 \hline
 16 & 44.21 & 44.21 & 43.64 & 42.97 & 78.65 & 77 & 76.94 & 77.98  \\
 \hline
 \hline
 \multicolumn{5}{|c|}{Winogrande (FP32 Accuracy = 69.38\%)} & \multicolumn{4}{|c|}{Piqa (FP32 Accuracy = 78.07\%)} \\ 
 \hline
 \hline
 64 & 68.9 & 68.43 & 69.77 & 68.19 & 77.09 & 76.82 & 77.09 & 77.86 \\
 \hline
 32 & 69.38 & 68.51 & 68.82 & 68.90 & 78.07 & 76.71 & 78.07 & 77.86  \\
 \hline
 16 & 69.53 & 67.09 & 69.38 & 68.90 & 77.37 & 77.8 & 77.91 & 77.69  \\
 \hline
\end{tabular}
\caption{\label{tab:mmlu_abalation} Accuracy on LM evaluation harness tasks on Llama2-7B model.}
\end{table}

\begin{table} \centering
\begin{tabular}{|c||c|c|c|c||c|c|c|c|} 
\hline
 $L_b \rightarrow$& \multicolumn{4}{c||}{8} & \multicolumn{4}{c||}{8}\\
 \hline
 \backslashbox{$L_A$\kern-1em}{\kern-1em$N_c$} & 2 & 4 & 8 & 16 & 2 & 4 & 8 & 16  \\
 %$N_c \rightarrow$ & 2 & 4 & 8 & 16 & 2 & 4 & 2 \\
 \hline
 \hline
 \multicolumn{5}{|c|}{Race (FP32 Accuracy = 48.8\%)} & \multicolumn{4}{|c|}{Boolq (FP32 Accuracy = 85.23\%)} \\ 
 \hline
 \hline
 64 & 49.00 & 49.00 & 49.28 & 48.71 & 82.82 & 84.28 & 84.03 & 84.25 \\
 \hline
 32 & 49.57 & 48.52 & 48.33 & 49.28 & 83.85 & 84.46 & 84.31 & 84.93  \\
 \hline
 16 & 49.85 & 49.09 & 49.28 & 48.99 & 85.11 & 84.46 & 84.61 & 83.94  \\
 \hline
 \hline
 \multicolumn{5}{|c|}{Winogrande (FP32 Accuracy = 79.95\%)} & \multicolumn{4}{|c|}{Piqa (FP32 Accuracy = 81.56\%)} \\ 
 \hline
 \hline
 64 & 78.77 & 78.45 & 78.37 & 79.16 & 81.45 & 80.69 & 81.45 & 81.5 \\
 \hline
 32 & 78.45 & 79.01 & 78.69 & 80.66 & 81.56 & 80.58 & 81.18 & 81.34  \\
 \hline
 16 & 79.95 & 79.56 & 79.79 & 79.72 & 81.28 & 81.66 & 81.28 & 80.96  \\
 \hline
\end{tabular}
\caption{\label{tab:mmlu_abalation} Accuracy on LM evaluation harness tasks on Llama2-70B model.}
\end{table}

%\section{MSE Studies}
%\textcolor{red}{TODO}


\subsection{Number Formats and Quantization Method}
\label{subsec:numFormats_quantMethod}
\subsubsection{Integer Format}
An $n$-bit signed integer (INT) is typically represented with a 2s-complement format \citep{yao2022zeroquant,xiao2023smoothquant,dai2021vsq}, where the most significant bit denotes the sign.

\subsubsection{Floating Point Format}
An $n$-bit signed floating point (FP) number $x$ comprises of a 1-bit sign ($x_{\mathrm{sign}}$), $B_m$-bit mantissa ($x_{\mathrm{mant}}$) and $B_e$-bit exponent ($x_{\mathrm{exp}}$) such that $B_m+B_e=n-1$. The associated constant exponent bias ($E_{\mathrm{bias}}$) is computed as $(2^{{B_e}-1}-1)$. We denote this format as $E_{B_e}M_{B_m}$.  

\subsubsection{Quantization Scheme}
\label{subsec:quant_method}
A quantization scheme dictates how a given unquantized tensor is converted to its quantized representation. We consider FP formats for the purpose of illustration. Given an unquantized tensor $\bm{X}$ and an FP format $E_{B_e}M_{B_m}$, we first, we compute the quantization scale factor $s_X$ that maps the maximum absolute value of $\bm{X}$ to the maximum quantization level of the $E_{B_e}M_{B_m}$ format as follows:
\begin{align}
\label{eq:sf}
    s_X = \frac{\mathrm{max}(|\bm{X}|)}{\mathrm{max}(E_{B_e}M_{B_m})}
\end{align}
In the above equation, $|\cdot|$ denotes the absolute value function.

Next, we scale $\bm{X}$ by $s_X$ and quantize it to $\hat{\bm{X}}$ by rounding it to the nearest quantization level of $E_{B_e}M_{B_m}$ as:

\begin{align}
\label{eq:tensor_quant}
    \hat{\bm{X}} = \text{round-to-nearest}\left(\frac{\bm{X}}{s_X}, E_{B_e}M_{B_m}\right)
\end{align}

We perform dynamic max-scaled quantization \citep{wu2020integer}, where the scale factor $s$ for activations is dynamically computed during runtime.

\subsection{Vector Scaled Quantization}
\begin{wrapfigure}{r}{0.35\linewidth}
  \centering
  \includegraphics[width=\linewidth]{sections/figures/vsquant.jpg}
  \caption{\small Vectorwise decomposition for per-vector scaled quantization (VSQ \citep{dai2021vsq}).}
  \label{fig:vsquant}
\end{wrapfigure}
During VSQ \citep{dai2021vsq}, the operand tensors are decomposed into 1D vectors in a hardware friendly manner as shown in Figure \ref{fig:vsquant}. Since the decomposed tensors are used as operands in matrix multiplications during inference, it is beneficial to perform this decomposition along the reduction dimension of the multiplication. The vectorwise quantization is performed similar to tensorwise quantization described in Equations \ref{eq:sf} and \ref{eq:tensor_quant}, where a scale factor $s_v$ is required for each vector $\bm{v}$ that maps the maximum absolute value of that vector to the maximum quantization level. While smaller vector lengths can lead to larger accuracy gains, the associated memory and computational overheads due to the per-vector scale factors increases. To alleviate these overheads, VSQ \citep{dai2021vsq} proposed a second level quantization of the per-vector scale factors to unsigned integers, while MX \citep{rouhani2023shared} quantizes them to integer powers of 2 (denoted as $2^{INT}$).

\subsubsection{MX Format}
The MX format proposed in \citep{rouhani2023microscaling} introduces the concept of sub-block shifting. For every two scalar elements of $b$-bits each, there is a shared exponent bit. The value of this exponent bit is determined through an empirical analysis that targets minimizing quantization MSE. We note that the FP format $E_{1}M_{b}$ is strictly better than MX from an accuracy perspective since it allocates a dedicated exponent bit to each scalar as opposed to sharing it across two scalars. Therefore, we conservatively bound the accuracy of a $b+2$-bit signed MX format with that of a $E_{1}M_{b}$ format in our comparisons. For instance, we use E1M2 format as a proxy for MX4.

\begin{figure}
    \centering
    \includegraphics[width=1\linewidth]{sections//figures/BlockFormats.pdf}
    \caption{\small Comparing LO-BCQ to MX format.}
    \label{fig:block_formats}
\end{figure}

Figure \ref{fig:block_formats} compares our $4$-bit LO-BCQ block format to MX \citep{rouhani2023microscaling}. As shown, both LO-BCQ and MX decompose a given operand tensor into block arrays and each block array into blocks. Similar to MX, we find that per-block quantization ($L_b < L_A$) leads to better accuracy due to increased flexibility. While MX achieves this through per-block $1$-bit micro-scales, we associate a dedicated codebook to each block through a per-block codebook selector. Further, MX quantizes the per-block array scale-factor to E8M0 format without per-tensor scaling. In contrast during LO-BCQ, we find that per-tensor scaling combined with quantization of per-block array scale-factor to E4M3 format results in superior inference accuracy across models. 

%TC:endignore
\end{document}

\endinput
%%
%% End of file `sample-sigconf-authordraft.tex'.
