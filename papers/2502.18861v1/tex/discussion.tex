\section{Discussion}
%\todo{2 problems: 1. Lack of connection with results, and 2. Engagement with previous work}

%\todo{Important to mention: The contribution of this study is not ApoloBot, but its conceptualization and how moderators perceive this? Despite being new?}

%\revision{The opportunity space and challenges highlighted in the findings provide a framework for understanding how restorative justice tools might emerge under different contexts and value systems. By evaluating the specific factors that enable or hinder their effective implementation, we establish a foundation for structuring restorative justice tools within existing moderation systems, and how it might compare to alternative approaches such as conventional punishments or manual interventions. In this section, we build upon these insights and discuss how the outcomes of restorative tools should be evaluated in light of the diverse stakeholders' needs. We then extend our focus to explore the broader potential for restorative justice tools, including their role in moderating processes beyond harm resolution, as illustrated by the case of ApoloBot. Finally, we propose key directions for future research based on these reflections.}

\revision{
By introducing ApoloBot as a tangible system that embeds restorative justice principles, we built on the more theoretical and hypothetical explorations of these concepts in the prior literature that influenced this work~\cite{Schoenebeck2021a, Schoenebeck2021b, Schoenebeck2023, Xiao2022, Xiao2023}. This enabled moderators to engage with these concepts in a practical manner and to envision how such tools might operate within their workflows. From this, our opportunity space uncovers the specific conditions where such tools can be applied, revealing nuances in community reception both in contexts where tools can thrive and where they may face resistance. Our findings offer a breadth of considerations that should be taken into account when developing future restorative justice tools. In this section, we expand our focus to discuss how we can translate these insights into the evaluation of future restorative justice tools, and we explore the broader implications for their potential designs.
}


\subsection{\revision{Assessing The Expected Outcomes of} Restorative Justice Tools}

\revision{
In contrast to punitive approaches that center around content-based penalties, restorative justice seeks to repair harm by addressing the multifaceted needs of victims, offenders, and their surrounding communities~\cite{Mccold2000}. Given these core differences in process and value orientation, evaluating restorative justice tools thus requires a more holistic approach that moves beyond the standard quantitative metrics such as utilization rates or punishment frequencies to focus on how effectively these tools meet the needs they are designed to address. These needs, however, are inherently personal and context-dependent, shaped by individual circumstances and by the goals of each specific community.
%~\cite{Mccold2000}.
%\todo{This is further complicated in the online context where... [cite prior work e.g RJ \& Chillbot? - labor of volunteer mod, online interaction dynamics... > RJ preferences? + call for evaluation]}
Therefore, a meaningful evaluation should begin with identifying \textit{what one aims to achieve}~---~the varied outcomes shaped by the diverse stakeholders needs~---~then exploring \textit{how these outcomes can be effectively measured}, by recognizing the relevant metrics accordingly~\cite{Llewellyn2013}.
%To properly evaluate this, we first need to consider \textit{what outcomes one wishes to achieve}~---~that is, the diverse expectations and needs of different stakeholders involved in the process~---~and then determine \textit{whether and how these outcomes might be measured}, using appropriate metrics.~\cite{Llewellyn2013}
}

\subsubsection{\revision{Understanding what one aims to achieve: Stakeholder's expected outcomes}}
%based on their distinct goals, values, and circumstances.
%— drive varying expectations of ApoloBot, with each group defining success in different ways depending on their unique experiences and goals.
%~---~whether a neighborhood, an online space~\cite{Xiao2023}, or a larger societal structure~\cite{Mccold2000}.
% In traditional settings, these needs are often assessed through measures like cultural fit~\cite{Ness2016}, stakeholder satisfaction~\cite{Vanfraechem2004}, victim feelings of security~\cite{Strang2003}, offender remorse~\cite{Rowe2002}, or reduced recidivism~\cite{Bonta2002}. Translating these metrics to the online context remains an open question, but our study shed some lights on how different stakeholders—moderators, victims, and offenders—shape their expectations for restorative tools like ApoloBot based on their distinct goals, values, and circumstances.
Our study shed some lights on how the different needs of the involved stakeholders~---~moderators, victims, and offenders~---~shape their diverse expectations for restorative tools like ApoloBot, with each group defining expected outcomes in different ways depending on their unique experiences and goals.

For \textit{moderators}, 
\revision{the tool’s objective lies not in how often it is used but in its ability to reinforce community cohesion, trust, and meaningful interactions. When conditions aligned, \revisionn{two} moderators (P7, P11) found tangible outcomes from ApoloBot in resolving conflicts and fostering better relationships among members. Without direct usage due to differing community dynamics and circumstances, however, some moderators saw the same value in its mere presence.}
For them, the bot serves as a preventive measure~---~a tool that is useful and enhances safety simply by being available, even if rarely utilized. As P6 noted, \textit{"Maybe the time is not there yet"}, but having ApoloBot ready for future cases is itself a form of \revision{preemptive utility, a safeguard that contributes to a more proactive environment.} Critically, some moderators point out that too frequent usage might signal a problem, indicating recurring harm and deteriorating community relationships. However, other moderators oppose this viewpoint, \revision{favoring alternatives such as manual intervention or fully-automated sanctions over ApoloBot when visible outcomes are not yet apparent. This preference was most notable in smaller, closer-knit servers that value personal interaction, or in larger, more ``commercial'' communities that prioritize scalability. ApoloBot occupies a unique middle ground on the moderation spectrum between ``automated'' and ``manual''~\cite{Jiang2023}, facilitating victim-offender communication by automating messages and embedding participation options within its features. Communities at either extreme may therefore take longer to build the trust needed to fully utilize the tool. This aligns with findings from other socially-engaged tools, such as Chillbot, where “Goldilocks zone”, mid-sized moderation teams are those that could most effectively integrate a proactive, nudge-based approach with critical results~\cite{Seering2024}.
}
% this view is not universal among moderators. Others, particularly in closer-knit communities, are skeptical about automating conflict resolution. They prefer established human-centered procedures to automated tools, and question whether the use of ApoloBot is an appropriate target for automation efforts.
%\revision{ Their goal is thus to minimize the need for such tools, viewing the ideal outcome as a community where issues are resolved without relying on automated processes.}

For \textit{victims} and \textit{offenders},  \revision{the preferred outcome for} usage of ApoloBot \revision{might be} less about completing a full restorative procedure and more about having access to an appropriate diversity of options to address harm. ApoloBot gives community members the autonomy to accept or decline participation based on their willingness to engage in the restoration process. \revision{As shown in our findings, while certain cases resulted in successful resolutions, sometimes both offender and victim may drop out for various reasons. If they do so,} it remains an open question whether this should be considered \revision{a desirable outcome}, as they are empowered to choose their preferred path, or a \revision{setback}, as a full resolution was not achieved. Sometimes reconciliation might not be the end result to aim for~---~people may seek other forms of closure: \revisionn{v}ictims might prefer to punish the offender~\cite{Xiao2022, Aliyu2024}, move past the situation~\cite{Xiao2023}, or leave the community entirely~\cite{Thomas2022}; \revisionn{l}ikewise, an offender might favor a more straightforward punitive process~\cite{Xiao2023} or choose to leave~\cite{Gao2024}. In this light, "success\revision{ful outcomes}" might mean enabling participants to express and fulfill their needs even if it doesn't result in an agreement\revision{, allowing the process itself to be as fluid as the human relationships it seeks to restore. }

%This points to a broader implication: not only the application of restorative justice tools but also their "success" is not one-size-fits-all. No single factor can imply success, and what is deemed successful to one may not hold the same meaning to another. The key may not be in striving for an ideal resolution every time, but in allowing the process itself to be as fluid as the human relationships it seeks to restore.

\subsubsection{Assessing how the outcomes can be effectively measured: Potential metrics for evaluation}
\revision{
Evaluating these diverse, nuanced outcomes is thus no simple task, as has been widely discussed in research on alternative moderation strategies~\cite{Xiao2023, Ma2023, Schoenebeck2021a}.
Recognizing the breadth of different stakeholders' goals uncovered in our findings is a crucial first step toward determining how we can effectively measure them. For example, cases where victims and offenders fail to reach consensus align with real-world restorative processes, where partial restoration still holds value~\cite{Mccold2000} and outcomes can still be assessed through personalized metrics such as stakeholder satisfaction~\cite{Vanfraechem2004}, victims' feelings of security~\cite{Strang2003}, or offenders' feelings of remorse~\cite{Rowe2002}. Online communities could adapt these measures by tracking stakeholder satisfaction and emotional responses, even for incomplete processes, to better understand the factors behind partial engagement and to identify which stakeholder needs were met or unmet. At the same time, moderators can use these insights to
assess how the tool shapes the interpersonal dynamics within their communities. Offender recidivism~\cite{Bonta2002} can also be integrated as a measure for behavioral change by using the tool's log history as part of violator profiling~\cite{Cai2021} and reflection process~\cite{Cullen2022}. These community- and stakeholder-focused metrics could help moderators fully capture how well the tool upholds their values, thereby refining its application and adjusting their usage expectations. Future work can explore these evaluation metrics to inform the deployment of restorative justice tools, ensuring that they are tailored to meet the varying needs of stakeholders and fostering more adaptive, context-sensitive interventions. 
%Furthermore, longitudinal studies can be employed to assess whether there are evolving roles of these tools that inform the refinement of existing metrics, and potentially the creation of new ones.
}

%\revision{Diverse stakeholder needs lead to varying patterns of engagement, emphasizing the need for a more refined evaluation approach that acknowledge and reflect these differences.} 

%In real-life settings, restorative outcomes can be assessed through measures such as stakeholder satisfaction~\cite{Vanfraechem2004}, victim feelings of security~\cite{Strang2003}, or offender remorse~\cite{Rowe2002}.
%, or reduced recidivism~\cite{Bonta2002}. 

%Translating these metrics to the online context remains an open question
%The key may not be in striving for an ideal resolution every time, but in allowing the process itself to be as fluid as the human relationships it seeks to restore.

% In traditional settings, these needs are often assessed through measures like cultural fit~\cite{Ness2016}, stakeholder satisfaction~\cite{Vanfraechem2004}, victim feelings of security~\cite{Strang2003}, offender remorse~\cite{Rowe2002}, or reduced recidivism~\cite{Bonta2002}. Translating these metrics to the online context remains an open question, but our study shed some lights on how different stakeholders—moderators, victims, and offenders—shape their expectations for restorative tools like ApoloBot based on their distinct goals, values, and circumstances.
%\todo{main point of this section: RJ tools require a more nuanced way of evaluation, more than just "number" or frequency of use -- since it involves communication among different stakeholders and etc (the last sentence)}

%Stakeholders have different aims and expectations, thus their definition of "success" depends on these unique values.

%Therefore, we should embrace the diverse outcomes and look at restorative justice tools from a more multidimensional perspective. No single factor can imply success, and what is deemed successful to one may not hold the same meaning to another. Ultimately, restorative justice tools may require a more nuanced approach to evaluation, one that acknowledges these differing priorities and goals. The key may not be in striving for an ideal resolution every time, but in allowing the process itself to be as fluid as the human relationships it seeks to restore.


% encourage trials and errors in deploying these tools, recognizing that their effectiveness may evolve over time as communities experiment and learn from both successes and failures? Perhaps, true success lies not in forcing outcomes, but in fostering an environment where participants feel empowered to navigate conflicts in a way that respects their individual needs and experiences.

% \subsection{The Different Approaches to implementing RJ}
% Manual (Sijia) and Technological (Keeper, ApoloBot), combined? and more?
% \begin{itemize}
%     \item Reflect back to prior work on RJ where they discuss victim-offender conference = manual
%     \item Reflect on our result - compare the tool approach? maybe when and where to use each? pros and cons?
% \end{itemize}

% \subsection{Restorative Justice Tools in the broader moderation landscape}
%\subsection{\sout{Design Implication for Future Online Restorative Justice Tools} \revision{Restorative Justice Tools within The Broader Moderation Processes}}
\subsection{Design Implication\revision{s} for Future Restorative Justice Tools}
% We now suggest design implications for future restorative justice tools, drawing on their opportunity space and challenges addressed in our empirical findings.

\revision{
While ApoloBot serves as a starting point for exploring community reactions to one implementation of a restorative approach, its focus is limited to specific aspects of restorative justice: namely, facilitating apologies. The limitations of the deployment, as discussed above, therefore point to areas where current practice may need to be refined or where complementary strategies might be considered, setting the stage for future research. Building on these observations, we propose a set of design implications to guide the development of more comprehensive and impactful future restorative justice-oriented tools.
}

% {\color{blue} % Start of blue-colored group

\subsubsection{Enabling rich interaction among stakeholders}
As discussed in Section 7.1, restorative justice tools must focus on the process of addressing stakeholder needs, and may even need to prioritize this over reaching any single outcome. This can be achieved, in part, by fostering richer and more meaningful dialogues among involved participants. ApoloBot's current design, while straightforward, only enables simple one-to-one exchange through apology requests and responses, which might not suffice in more complex cases where participants need deeper engagement. This limitation might contribute to participant dropouts, when victims are left dissatisfied or offenders struggle to express remorse effectively. To address this challenge, future tools might incorporate features for more dynamic interactions: Offenders might, for example, revise and refine their apologies based on moderator or victim feedback, reflecting genuine remorse through multiple attempts. Moderators, in turn, could offer customizable guidance, helping both parties articulate their needs and work toward meaningful resolutions.
Furthermore, tools could extend beyond dyadic exchanges to support multi-party interactions, addressing cases with overlapping roles or multiple stakeholders.
More sophisticated platforms, such as Keeper~\cite{Hughes2020}, have demonstrated the potential of restorative justice circles with socially-enhanced features like tone-setting and structured turn-taking. However, such advancements should carefully balance the benefits of improved engagement with the potential burden of increased complexity for users.

% Maybe even the process initiation - like giving user agency like victim can request ApoloBot/RJ process initiation -> address the challenge of complex stakeholder dynamics where it's sometimes hard to even identify victims + timing issue (victim can request before the case escalate too much)

\subsubsection{Enhancing trust towards socio-technical tools}
As illustrated by P3’s experience (detailed in section \ref{challenge-perception}), miscommunication combined with negative perceptions of automated moderation may erode trust and discourage engagement. This challenge is not unique to restorative justice tools but reflects broader concerns about automated systems, where distrust often stems from users’ lack of understanding about how these tools work, coupled with prior negative experiences that tarnish their perceptions~\cite{Lee2018, Hoffman2013, Kuo2023}. These issues are exacerbated by the novelty of restorative justice itself, which remains unfamiliar to many.
To address this, interface transparency has been shown to be a key factor in improving user acceptance~\cite{Kizilcec2016}, encouraging more informed interactions~\cite{Eslami2019}, and in some cases, driving long-term moderation outcomes~\cite{Jhaver2019}. Tool developers can create clear, accessible documentation or onboarding mechanisms that can be embedded or prominently displayed with the tool to clearly explain its functionalities and the human-centered restorative justice principles behind its design. Moreover, additional work can be done to explore how this explanation can be made more personalized and "humanized" to overcome the perceptions of insincere bot-driven interactions, especially when dealing with emotional matters like interpersonal harm.

%Explainable interface with thoughtful design can help users clearly understand its intent, thus build trust and engage with the tool meaningfully.

% (Our study shows that while tools can help lift the labor of manual RJ (Xiao), taking human out of the process might add additional labor in trust building)
% Trust once lost it can be hard to re-establish? - while humans can engage in confessional behaviors to rebuild trust swiftly, machines lack such emotional capacities. - what happened to P3

\subsubsection{Resource sharing and education}
The opportunity space explored in our findings highlighted groups of moderators who showed interest in ApoloBot's approach but were not yet positioned to fully utilize it. For instance, very proactive moderation teams had interest in re-engaging in restorative conversations but lacked the mediation skills necessary to do so. On the other end of the spectrum, moderators in communities with much more lax behavioral norms might perceive the tool as high-risk due to the potential for abuse, or for less sincere engagement with the tool.
%While automated tools like ApoloBot might help lift the emotional and cognitive labor of manual restorative justice facilitation~\cite{Xiao2023}, our findings showed that it might introduce a new form of labor when shifting one's technological frameworks. 
Nevertheless, these outcomes are not necessarily inevitable~---~community moderation is an evolving practice that grows through collective learning within~\cite{Cullen2022} and across~\cite{Hwang2024, Uttarapong2024} communities. Reflective practices and social learning have been shown to close knowledge gaps and reshape perceptions of tool adoption, transforming it from a burdensome transition into an engaging and even enjoyable experience~\cite{Hwang2024}. By fostering this collaborative process, a gradual shift toward deployment of restorative justice-based tools can be made more accessible when communities can share and refine their technological frame through an emergent social ecosystem. Technological platforms or frameworks enabling communities to document and share their encountered use cases can be valuable in providing inspiration and practical insights for communities to address gaps in values, functionalities and possible scenarios for restorative justice, allowing communities to adapt and align tools within their own contexts accordingly.

In addition, education plays a vital role not only for moderators but also for community members~\cite{West2018}, many of whom may lack prior exposure to restorative justice processes. Future work can further explore how educational initiatives can be informed through platform or tool design, such as via explanations, guided interactions, or prompts to provide users with actionable insights that can ease their navigation through restorative interactions. For example, in the case of ApoloBot, providing guidance on how to write a "reasonable" apology can help offenders construct meaningful responses that can better meet victims' needs. These educational elements can be further tailored to specific community values, as informed by collective learning and experimentation, to create an environment where restorative outcomes are both practical and well-aligned with the local cultural dynamics.

\subsubsection{Beyond harm resolution}
While ApoloBot's focus on harm resolution offers key insights into the core aspects of restorative justice implementation, it also carries limitations including aspects outside its current scope, such as timing before intervention or dropout follow-ups. On this account, future restorative justice-oriented tools can expand their capabilities to encompass a broader spectrum of the moderation process, addressing not only the harm by itself but also the stages \textit{pre-harm} and \textit{post-harm}.

For \textit{pre-harm interventions}, a proactive identification approach can play a crucial role in determining the right moment for restorative actions. In our study design, the harm identification process is fully manual, posing challenges related to moderator workload and timing of interventions~---~acting too early risks unnecessary interference, while acting too late can exacerbate harm. Recent advances in human-AI moderation systems have shown promise in identifying potential at-risk interactions that fall within this optimal window for restorative intervention. By analyzing relevant contextual signals such as conversation metrics~\cite{Choi2023, Schluger2022}, user histories~\cite{Im2020}, and prior moderation decisions~\cite{Chandrasekharan2019}, technologies can guide moderators’ attention to potential harm cases more efficiently, especially in large and high-traffic communities. This opens up opportunities for designing detection mechanisms specific to community dynamics that can be embedded as a more context-aware initiation of restorative workflows. 

For \textit{post-harm interventions}, a support network might provide additional avenues for engagement and healing, especially in cases where dropout occurs. Victim-support groups, as a form of restorative justice~\cite{Dignan2004}, have been previously discussed and implemented in the form of online systems designed for emotional support, advice sharing and empowerment~\cite{Dimond2013, Blackwell2017}. At the same time, it could also be valuable to explore similar consultation mechanisms for offenders, particularly those who genuinely wish to or have the potential to engage in the restorative process. Capturing the effects of these interventions and monitoring behavioral trajectories \textit{post-harm} could also provide valuable context to strengthen and tailor \textit{pre-harm} strategies. For instance, understanding patterns of accountability, remorse, or recurrence among offenders could inform predictive tools and guide proactive measures to mitigate harm.

%Overall, restorative justice-oriented tools offer a broad design space, where embedded features, bots, platforms, or integrated systems can be selectively utilized or combined to align with existing infrastructures and to meet specific community needs. This expanded vision reimagines restorative justice tools not only as solutions for conflict but also as catalysts for more proactive, inclusive, and resilient community governance.

Overall, restorative justice-oriented tools offer a broad design space, encompassing a spectrum of possibilities. They might involve \textit{embedded features} that enhance existing tools with functionalities like transparency or user feedback. At the next level, they could include dedicated \textit{bots} designed for specific restorative tasks, such as ApoloBot’s focus on facilitating apologies. For more comprehensive needs, \textit{integrated systems} could combine tools—such as pairing a bot with harm identification mechanisms—to offer greater functionality. At the highest level, \textit{platforms} like HeartMob, Keeper or knowledge-sharing spaces can provide fully realized ecosystems for restorative justice practices. By accommodating varying degrees of complexity, these tools can be customized to fit existing infrastructures and community-specific needs. This expanded vision reimagines restorative justice tools not only as solutions for conflict but also as catalysts for more proactive, inclusive, and resilient community governance.
%Researchers can \todo{future work? explore and test these designs? targetted/tailored to certain communities, or combined as a customizable, generalizable framework? see if there are evolving roles and implications of such tools?}
% } % End of blue-colored group


\begin{comment}
\subsubsection{Customization to address the diverse needs of stakeholders} While ApoloBot embedded customization into the restoration process with participation options, stakeholders' needs might exceed this scope of design. Victims and offenders may have varying perspectives on apologies, and moderators may integrate the tool differently based on existing procedures. These nuanced contexts call for a more tailored approach to restorative justice implementation.

For moderators, customization should assist their unique approaches to handling harm cases. This may include adjusting communication methods such as altering message tone rather than using a fixed template, ranging from a lighter tone for more minor cases to a more serious one for severe incidents, or adapting to a server's specific culture and language. Customizable actions with more options (warnings, bans) may also be useful, with adjustable durations that can be reduced or increased depending on the offender's response or potential misuse of the system.

For community members, especially the stakeholders involved in an incident, tools should cater to their specific needs and preferences for resolution during and after harm cases. This may involve enabling more back-and-forth conversations before a final apology, or allowing offenders to revise apologies based on feedback from victims or moderators. Similar to an appeal process, multiple apology attempts can be considered if offenders show genuine remorse and reflection. Even more broadly, tools could scale up to situations where both sides need apologies or multiple stakeholders can be involved instead of just being one-to-one.

\subsubsection{Embed social features within the technical framework}
To overcome the perceived impersonal and insincere interaction of ``talking to a bot,'' social features could be incorporated to increase the tool's appeal and trustworthiness. Currently, ApoloBot has a basic avatar and description, but participants noted how popular bots often feature distinct appearances aligned with their purpose, such as how music bots have diverse musical elements in their profiles. For restorative justice tools, which deal with personal and emotional matters, adding friendly features~---~for instance including not only the person's message but also their avatar and other personalized shared-server elements (e.g stickers, reactions, message embeds)~---~could make interactions feel more amiable. Furthermore, tools could explore ways to integrate empathetic design elements through visual cues, which prior studies have shown to enhance user engagement~\cite{Samrose2020} and emotional awareness~\cite{Rojas2022}.

\subsubsection{``Restoration'' as a broader contextual implementation} \blueout{Restoration is highly contextual: Each interpersonal harm case requires different restoration strategies, and sometimes an apology might not be the one best solution. Other alternative forms of restoration might involve:}

\begin{itemize}
    \item \protect\textbf{Simple Intervention:} Sometimes, all that's needed is to advise the involved parties to take a break and tone down their emotions, or simply explain the situation to resolve misunderstandings without the need for an apology.
    \item \protect\textbf{Non-verbal Reparation:} Restorations can take other forms beyond text. Some people might express themselves better with visuals like images \protect\cite{Makela2000}, emojis \protect\cite{Liu2018}, or even memes \protect\cite{Terzimehic2021}, which can be thoughtfully incorporated based on circumstances and cultural relevance. In addition, community-specific compensation, such as in-game items (for gaming communities) or artwork (for arts or fandom communities) may be meaningful, especially when it's a personal thing that aligns with the community values and the relationships between offenders and victims. 
    \item \protect\textbf{Educational Program:} In cases that demand more severe actions such as temporary or permanent bans, restoration might require a more structured and long-term approach. Offenders might benefit from concrete guidelines, such as goal-setting or rehabilitation programs to raise awareness of harmful conduct and guide directions for meaningful changes. Through education, offenders can work toward personal growth and demonstrate commitments to change before reinstatement.
\end{itemize}

%Future designs can explore different ways for these restorative approaches to be embedded into tools that can cater to the specific dynamics of different online communities.

\end{comment}

\begin{comment}
\subsection{Considerations for Future Research}
\todo{Hmm actually do we still need this...}
Our study mainly focused on the moderators' perceptions and reflections on ApoloBot, a tool the research team developed. 

\todo{Here we can more "aggressively" state how ApoloBot is not the main contribution i.e a tool ready for immediate deployment, rather a bridge to close the gap between conceptualizations and actual implementation(?) of RJ -> understanding potential design space -> future work!}

Future studies can include a broader range of stakeholders (victims and offenders) even during the design process to better understand their diverse perspectives and inform more targeted design objectives. Moreover, employing longitudinal studies could provide insights into the long-term impacts of restorative justice tools, beyond the short-term adoption period we examined. Researchers can assess whether there are evolving roles of these tools toward community management, further challenges regarding their maintenance, or if it is indeed beneficial to retain tools that are infrequently used. Finally, utilizing the opportunity space, further testing restorative justice tools in areas where they show promise can generalize the findings, while alternative or complementary approaches can be explored in contexts where they prove less effective.
\end{comment}

% it is valuable and necessary to hear moderated users’ voice and see how it could help reflect on and refine existing moderation design.