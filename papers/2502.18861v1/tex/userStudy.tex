\section{Methods}
Using ApoloBot as a discussion starting point, we extend our exploration into the broader landscape of restorative justice tools through a three-phase user study with Discord moderators. Each phase involves increasing levels of commitment, starting with initial interviews, followed by tool deployment, and concluding with reflections. Given that restorative justice tools are still relatively rare in online communities, these separate phases allow us to gather valuable insights while respecting moderators' diverse willingness and interest in the new approach. All parts of this study were pre-approved by our university's Institutional Review Board (IRB).

\subsection{Phases Overview}
%To evaluate the potential for ApoloBot and restorative justice tools more broadly, we conducted a user study with three phases. 

\textbf{Phase 1. Onboarding Session (60-90 minutes):} In the first phase, we conducted individual interviews with Discord moderators to gain insights into their general moderation practices and the potential of integrating restorative justice tools. Participants were asked about their procedure to handling interpersonal harm with specific examples of past cases. We then introduced the concept of restorative justice and presented ApoloBot as a practical tool embodying a subset of these principles. This was followed by discussions on the potential application of ApoloBot and other restorative justice tools within their communities, considering critical factors such as use cases, challenges, opportunities, perceived benefits, and drawbacks. After the interview, participants were invited to a Discord sandbox server to test out ApoloBot, where they provided further feedback and decided whether to continue with the study by deploying it in the subsequent phase.

\textbf{Phase 2. In-the-wild Deployment (4 weeks):} In the second phase, a subset of interested participants deployed ApoloBot in their communities, using it whenever suitable cases arose. Throughout this period, they kept track of their bot usage and maintained weekly communication with the researchers for feedback and support.

\textbf{Phase 3. Exit Interview (60-90 minutes):} At the end of the deployment period, participants joined an exit interview to reflect on their experiences with ApoloBot, unveiling new insights into its practical aspects, including user engagement and its effects on the community. Building on these reflections and revisiting critical factors from Phase 1 interviews, we \revision{explored the underlying factors for how the deployment met or challenged initial expectations, and} broadened the discussion to assess the overall design space of ApoloBot and other online restorative justice tools.

All interviews were conducted remotely through the Discord voice chat function. Participants could withdraw from the study at any phase without penalty. Compensation was provided for fully completed phases: \$20 for Phase 1, \$50 for Phase 2, and \$30 for Phase 3, delivered via Tremendous.~\footnote{https://www.tremendous.com/}


\subsection{Recruitment and Selection of Participants}
%We utilized a combination of platforms to distribute
Our recruitment call was distributed in meta-moderation communities on Discord, Reddit, and Facebook. These are communities where Discord moderators gather to discuss various moderation topics, such as news, strategies, philosophies, and tool usage. To ensure the quality of our recruits, we used a screening survey to assess their background and moderation experience. In addition to project-specific criteria such as prior experience handling interpersonal harm and familiarity with Discord bots, we filtered out low-quality responses such as one-word answers and those containing nonsensical or irrelevant information. We contacted selected participants, and further employed snowball sampling~\cite{Biernacki1981} by asking them for referrals. A total of 16 participants were chosen for Phase 1, with six proceeding to Phases 2 and 3. Two used ApoloBot during their deployment, while the others deployed it but did not encounter any suitable use cases. A summary of the participants' demographics and their status within Phase 1 and 2 are detailed in Table \ref{table:demographics}.
\begin{table*}[!ht]\scriptsize
\centering
\caption{Experiment - Participants demographics}
\label{tab:partdemograph}
\begin{tabular}{cccccccccccccccccccc}
\hline
\multicolumn{10}{c}{\textbf{Plugin Group}} & \multicolumn{10}{c}{\textbf{Control Group}} \\ \hline
\multicolumn{1}{c|}{\multirow{2}{*}{\textbf{ID}}} & \multicolumn{1}{c|}{\multirow{2}{*}{\textbf{Gender}}} & \multicolumn{1}{c|}{\multirow{2}{*}{\textbf{Persona}}} & \multicolumn{2}{c|}{\textbf{Experience}} & \multicolumn{5}{c|}{\textbf{Facets}} & \multicolumn{1}{c|}{\multirow{2}{*}{\textbf{ID}}} & \multicolumn{1}{c|}{\multirow{2}{*}{\textbf{Gender}}} & \multicolumn{1}{c|}{\multirow{2}{*}{\textbf{Persona}}} & \multicolumn{2}{c|}{\textbf{Experience}} & \multicolumn{5}{c}{\textbf{Facets}} \\ \cline{4-10} \cline{14-20} 
\multicolumn{1}{c|}{} & \multicolumn{1}{c|}{} & \multicolumn{1}{c|}{} & \multicolumn{1}{c|}{\textbf{GitHub}} & \multicolumn{1}{c|}{\textbf{OSS}} & \multicolumn{1}{c|}{\textbf{MT}} & \multicolumn{1}{c|}{\textbf{SE}} & \multicolumn{1}{c|}{\textbf{R}} & \multicolumn{1}{c|}{\textbf{IP}} & \multicolumn{1}{c|}{\textbf{L}} & \multicolumn{1}{c|}{} & \multicolumn{1}{c|}{} & \multicolumn{1}{c|}{} & \multicolumn{1}{c|}{\textbf{GitHub}} & \multicolumn{1}{c|}{\textbf{OSS}} & \multicolumn{1}{c|}{\textbf{MT}} & \multicolumn{1}{c|}{\textbf{SE}} & \multicolumn{1}{c|}{\textbf{R}} & \multicolumn{1}{c|}{\textbf{IP}} & \multicolumn{1}{c}{\textbf{L}} \\ \hline \hline

\multicolumn{1}{c|}{P1} & \multicolumn{1}{c|}{M} & \multicolumn{1}{c|}{\tikzcirclenew[fill=blue]{3pt}} & \multicolumn{1}{c|}{Never} & \multicolumn{1}{c|}{No} & \multicolumn{1}{c|}{\tikzcirclenew[fill=blue]{3pt}} & \multicolumn{1}{c|}{\tikzcirclenew[fill=blue]{3pt}} & \multicolumn{1}{c|}{\tikzcirclenew[fill=blue]{3pt}} & \multicolumn{1}{c|}{\tikzcircle[fill=orange]{3pt}} & \multicolumn{1}{c|}{\tikzcirclenew[fill=blue]{3pt}} & \multicolumn{1}{c|}{P40} & \multicolumn{1}{c|}{W} & \multicolumn{1}{c|}{\tikzcircle[fill=orange]{3pt}} & \multicolumn{1}{c|}{Once} & \multicolumn{1}{c|}{No} & \multicolumn{1}{c|}{\tikzcircle[fill=orange]{3pt}} & \multicolumn{1}{c|}{\tikzcirclenew[fill=blue]{3pt}} & \multicolumn{1}{c|}{\tikzcircle[fill=orange]{3pt}} & \multicolumn{1}{c|}{\tikzcircle[fill=orange]{3pt}} & \multicolumn{1}{c}{\tikzcirclenew[fill=blue]{3pt}} \\ \hline

\multicolumn{1}{c|}{P2} & \multicolumn{1}{c|}{W} & \multicolumn{1}{c|}{\tikzcirclenew[fill=blue]{3pt}} & \multicolumn{1}{c|}{Once} & \multicolumn{1}{c|}{No} & \multicolumn{1}{c|}{\tikzcirclenew[fill=blue]{3pt}} & \multicolumn{1}{c|}{\tikzcirclenew[fill=blue]{3pt}} & \multicolumn{1}{c|}{\tikzcirclenew[fill=blue]{3pt}} & \multicolumn{1}{c|}{\tikzcircle[fill=orange]{3pt}} & \multicolumn{1}{c|}{\tikzcirclenew[fill=blue]{3pt}} & \multicolumn{1}{c|}{P41} & \multicolumn{1}{c|}{M} & \multicolumn{1}{c|}{\tikzcirclenew[fill=blue]{3pt}} & \multicolumn{1}{c|}{Once} & \multicolumn{1}{c|}{No} & \multicolumn{1}{c|}{\tikzcirclenew[fill=blue]{3pt}} & \multicolumn{1}{c|}{\tikzcirclenew[fill=blue]{3pt}} & \multicolumn{1}{c|}{\tikzcircle[fill=orange]{3pt}} & \multicolumn{1}{c|}{\tikzcircle[fill=orange]{3pt}} & \multicolumn{1}{c}{\tikzcirclenew[fill=blue]{3pt}} \\ \hline

\multicolumn{1}{c|}{P3} & \multicolumn{1}{c|}{M} & \multicolumn{1}{c|}{\tikzcirclenew[fill=blue]{3pt}} & \multicolumn{1}{c|}{Never} & \multicolumn{1}{c|}{No} & \multicolumn{1}{c|}{\tikzcircle[fill=orange]{3pt}} & \multicolumn{1}{c|}{\tikzcirclenew[fill=blue]{3pt}} & \multicolumn{1}{c|}{\tikzcirclenew[fill=blue]{3pt}} & \multicolumn{1}{c|}{\tikzcircle[fill=orange]{3pt}} & \multicolumn{1}{c|}{\tikzcirclenew[fill=blue]{3pt}} & \multicolumn{1}{c|}{P42} & \multicolumn{1}{c|}{M} & \multicolumn{1}{c|}{\tikzcircle[fill=orange]{3pt}} & \multicolumn{1}{c|}{Never} & \multicolumn{1}{c|}{No} & \multicolumn{1}{c|}{\tikzcircle[fill=orange]{3pt}} & \multicolumn{1}{c|}{\tikzcirclenew[fill=blue]{3pt}} & \multicolumn{1}{c|}{\tikzcircle[fill=orange]{3pt}} & \multicolumn{1}{c|}{\tikzcircle[fill=orange]{3pt}} & \multicolumn{1}{c}{\tikzcirclenew[fill=blue]{3pt}} \\ \hline

\multicolumn{1}{c|}{P4} & \multicolumn{1}{c|}{M} & \multicolumn{1}{c|}{\tikzcirclenew[fill=blue]{3pt}} & \multicolumn{1}{c|}{Never} & \multicolumn{1}{c|}{No} & \multicolumn{1}{c|}{\tikzcircle[fill=orange]{3pt}} & \multicolumn{1}{c|}{\tikzcirclenew[fill=blue]{3pt}} & \multicolumn{1}{c|}{\tikzcirclenew[fill=blue]{3pt}} & \multicolumn{1}{c|}{\tikzcircle[fill=orange]{3pt}} & \multicolumn{1}{c|}{\tikzcirclenew[fill=blue]{3pt}} & \multicolumn{1}{c|}{P43} & \multicolumn{1}{c|}{M} & \multicolumn{1}{c|}{\tikzcirclenew[fill=blue]{3pt}} & \multicolumn{1}{c|}{Never} & \multicolumn{1}{c|}{No} & \multicolumn{1}{c|}{\tikzcirclenew[fill=blue]{3pt}} & \multicolumn{1}{c|}{\tikzcirclenew[fill=blue]{3pt}} & \multicolumn{1}{c|}{\tikzcirclenew[fill=blue]{3pt}} & \multicolumn{1}{c|}{\tikzcircle[fill=orange]{3pt}} & \multicolumn{1}{c}{\tikzcirclenew[fill=blue]{3pt}} \\ \hline

\multicolumn{1}{c|}{P5} & \multicolumn{1}{c|}{M} & \multicolumn{1}{c|}{\tikzcirclenew[fill=blue]{3pt}} & \multicolumn{1}{c|}{Once} & \multicolumn{1}{c|}{No} & \multicolumn{1}{c|}{\tikzcirclenew[fill=blue]{3pt}} & \multicolumn{1}{c|}{\tikzcirclenew[fill=blue]{3pt}} & \multicolumn{1}{c|}{\tikzcirclenew[fill=blue]{3pt}} & \multicolumn{1}{c|}{\tikzcircle[fill=orange]{3pt}} & \multicolumn{1}{c|}{\tikzcirclenew[fill=blue]{3pt}} & \multicolumn{1}{c|}{P44} & \multicolumn{1}{c|}{M} & \multicolumn{1}{c|}{\tikzcirclenew[fill=blue]{3pt}} & \multicolumn{1}{c|}{Never} & \multicolumn{1}{c|}{No} & \multicolumn{1}{c|}{\tikzcirclenew[fill=blue]{3pt}} & \multicolumn{1}{c|}{\tikzcirclenew[fill=blue]{3pt}} & \multicolumn{1}{c|}{\tikzcircle[fill=orange]{3pt}} & \multicolumn{1}{c|}{\tikzcircle[fill=orange]{3pt}} & \multicolumn{1}{c}{\tikzcirclenew[fill=blue]{3pt}} \\ \hline

\multicolumn{1}{c|}{P6} & \multicolumn{1}{c|}{M} & \multicolumn{1}{c|}{\tikzcirclenew[fill=blue]{3pt}} & \multicolumn{1}{c|}{Once} & \multicolumn{1}{c|}{No} & \multicolumn{1}{c|}{\tikzcirclenew[fill=blue]{3pt}} & \multicolumn{1}{c|}{\tikzcirclenew[fill=blue]{3pt}} & \multicolumn{1}{c|}{\tikzcircle[fill=orange]{3pt}} & \multicolumn{1}{c|}{\tikzcircle[fill=orange]{3pt}} & \multicolumn{1}{c|}{\tikzcirclenew[fill=blue]{3pt}} & \multicolumn{1}{c|}{P45} & \multicolumn{1}{c|}{M} & \multicolumn{1}{c|}{\tikzcirclenew[fill=blue]{3pt}} & \multicolumn{1}{c|}{Never} & \multicolumn{1}{c|}{No} & \multicolumn{1}{c|}{\tikzcirclenew[fill=blue]{3pt}} & \multicolumn{1}{c|}{\tikzcirclenew[fill=blue]{3pt}} & \multicolumn{1}{c|}{\tikzcirclenew[fill=blue]{3pt}} & \multicolumn{1}{c|}{\tikzcircle[fill=orange]{3pt}} & \multicolumn{1}{c}{\tikzcirclenew[fill=blue]{3pt}} \\ \hline

\multicolumn{1}{c|}{P7} & \multicolumn{1}{c|}{W} & \multicolumn{1}{c|}{\tikzcircle[fill=orange]{3pt}} & \multicolumn{1}{c|}{Never} & \multicolumn{1}{c|}{No} & \multicolumn{1}{c|}{\tikzcircle[fill=orange]{3pt}} & \multicolumn{1}{c|}{\tikzcirclenew[fill=blue]{3pt}} & \multicolumn{1}{c|}{\tikzcircle[fill=orange]{3pt}} & \multicolumn{1}{c|}{\tikzcircle[fill=orange]{3pt}} & \multicolumn{1}{c|}{\tikzcirclenew[fill=blue]{3pt}} & \multicolumn{1}{c|}{P46} & \multicolumn{1}{c|}{M} & \multicolumn{1}{c|}{\tikzcircle[fill=orange]{3pt}} & \multicolumn{1}{c|}{Never} & \multicolumn{1}{c|}{No} & \multicolumn{1}{c|}{\tikzcirclenew[fill=blue]{3pt}} & \multicolumn{1}{c|}{\tikzcircle[fill=orange]{3pt}} & \multicolumn{1}{c|}{\tikzcircle[fill=orange]{3pt}} & \multicolumn{1}{c|}{\tikzcircle[fill=orange]{3pt}} & \multicolumn{1}{c}{\tikzcirclenew[fill=blue]{3pt}} \\ \hline

\multicolumn{1}{c|}{P8} & \multicolumn{1}{c|}{W} & \multicolumn{1}{c|}{\tikzcircle[fill=orange]{3pt}} & \multicolumn{1}{c|}{Never} & \multicolumn{1}{c|}{No} & \multicolumn{1}{c|}{\tikzcirclenew[fill=blue]{3pt}} & \multicolumn{1}{c|}{\tikzcircle[fill=orange]{3pt}} & \multicolumn{1}{c|}{\tikzcircle[fill=orange]{3pt}} & \multicolumn{1}{c|}{\tikzcircle[fill=orange]{3pt}} & \multicolumn{1}{c|}{\tikzcirclenew[fill=blue]{3pt}} & \multicolumn{1}{c|}{P47} & \multicolumn{1}{c|}{M} & \multicolumn{1}{c|}{\tikzcirclenew[fill=blue]{3pt}} & \multicolumn{1}{c|}{Never} & \multicolumn{1}{c|}{No} & \multicolumn{1}{c|}{\tikzcirclenew[fill=blue]{3pt}} & \multicolumn{1}{c|}{\tikzcirclenew[fill=blue]{3pt}} & \multicolumn{1}{c|}{\tikzcirclenew[fill=blue]{3pt}} & \multicolumn{1}{c|}{\tikzcircle[fill=orange]{3pt}} & \multicolumn{1}{c}{\tikzcirclenew[fill=blue]{3pt}} \\ \hline

\multicolumn{1}{c|}{P9} & \multicolumn{1}{c|}{M} & \multicolumn{1}{c|}{\tikzcirclenew[fill=blue]{3pt}} & \multicolumn{1}{c|}{Once} & \multicolumn{1}{c|}{No} & \multicolumn{1}{c|}{\tikzcirclenew[fill=blue]{3pt}} & \multicolumn{1}{c|}{\tikzcirclenew[fill=blue]{3pt}} & \multicolumn{1}{c|}{\tikzcirclenew[fill=blue]{3pt}} & \multicolumn{1}{c|}{\tikzcircle[fill=orange]{3pt}} & \multicolumn{1}{c|}{\tikzcirclenew[fill=blue]{3pt}} & \multicolumn{1}{c|}{P48} & \multicolumn{1}{c|}{M} & \multicolumn{1}{c|}{\tikzcirclenew[fill=blue]{3pt}} & \multicolumn{1}{c|}{Never} & \multicolumn{1}{c|}{No} & \multicolumn{1}{c|}{\tikzcircle[fill=orange]{3pt}} & \multicolumn{1}{c|}{\tikzcirclenew[fill=blue]{3pt}} & \multicolumn{1}{c|}{\tikzcirclenew[fill=blue]{3pt}} & \multicolumn{1}{c|}{\tikzcircle[fill=orange]{3pt}} & \multicolumn{1}{c}{\tikzcirclenew[fill=blue]{3pt}} \\ \hline

\multicolumn{1}{c|}{P10} & \multicolumn{1}{c|}{W} & \multicolumn{1}{c|}{\tikzcirclenew[fill=blue]{3pt}} & \multicolumn{1}{c|}{Never} & \multicolumn{1}{c|}{No} & \multicolumn{1}{c|}{\tikzcirclenew[fill=blue]{3pt}} & \multicolumn{1}{c|}{\tikzcirclenew[fill=blue]{3pt}} & \multicolumn{1}{c|}{\tikzcircle[fill=orange]{3pt}} & \multicolumn{1}{c|}{\tikzcircle[fill=orange]{3pt}} & \multicolumn{1}{c|}{\tikzcirclenew[fill=blue]{3pt}} & \multicolumn{1}{c|}{P49} & \multicolumn{1}{c|}{M} & \multicolumn{1}{c|}{\tikzcircle[fill=orange]{3pt}} & \multicolumn{1}{c|}{Never} & \multicolumn{1}{c|}{No} & \multicolumn{1}{c|}{\tikzcircle[fill=orange]{3pt}} & \multicolumn{1}{c|}{\tikzcirclenew[fill=blue]{3pt}} & \multicolumn{1}{c|}{\tikzcircle[fill=orange]{3pt}} & \multicolumn{1}{c|}{\tikzcircle[fill=orange]{3pt}} & \multicolumn{1}{c}{\tikzcirclenew[fill=blue]{3pt}} \\ \hline

\multicolumn{1}{c|}{P11} & \multicolumn{1}{c|}{M} & \multicolumn{1}{c|}{\tikzcircle[fill=orange]{3pt}} & \multicolumn{1}{c|}{Never} & \multicolumn{1}{c|}{Some} & \multicolumn{1}{c|}{\tikzcircle[fill=orange]{3pt}} & \multicolumn{1}{c|}{\tikzcircle[fill=orange]{3pt}} & \multicolumn{1}{c|}{\tikzcircle[fill=orange]{3pt}} & \multicolumn{1}{c|}{\tikzcircle[fill=orange]{3pt}} & \multicolumn{1}{c|}{\tikzcircle[fill=orange]{3pt}} & \multicolumn{1}{c|}{P50} & \multicolumn{1}{c|}{M} & \multicolumn{1}{c|}{\tikzcirclenew[fill=blue]{3pt}} & \multicolumn{1}{c|}{Never} & \multicolumn{1}{c|}{No} & \multicolumn{1}{c|}{\tikzcirclenew[fill=blue]{3pt}} & \multicolumn{1}{c|}{\tikzcirclenew[fill=blue]{3pt}} & \multicolumn{1}{c|}{\tikzcirclenew[fill=blue]{3pt}} & \multicolumn{1}{c|}{\tikzcircle[fill=orange]{3pt}} & \multicolumn{1}{c}{\tikzcirclenew[fill=blue]{3pt}} \\ \hline

\multicolumn{1}{c|}{P12} & \multicolumn{1}{c|}{M} & \multicolumn{1}{c|}{\tikzcirclenew[fill=blue]{3pt}} & \multicolumn{1}{c|}{Never} & \multicolumn{1}{c|}{No} & \multicolumn{1}{c|}{\tikzcirclenew[fill=blue]{3pt}} & \multicolumn{1}{c|}{\tikzcirclenew[fill=blue]{3pt}} & \multicolumn{1}{c|}{\tikzcirclenew[fill=blue]{3pt}} & \multicolumn{1}{c|}{\tikzcircle[fill=orange]{3pt}} & \multicolumn{1}{c|}{\tikzcirclenew[fill=blue]{3pt}} & \multicolumn{1}{c|}{P51} & \multicolumn{1}{c|}{M} & \multicolumn{1}{c|}{\tikzcirclenew[fill=blue]{3pt}} & \multicolumn{1}{c|}{Never} & \multicolumn{1}{c|}{No} & \multicolumn{1}{c|}{\tikzcirclenew[fill=blue]{3pt}} & \multicolumn{1}{c|}{\tikzcirclenew[fill=blue]{3pt}} & \multicolumn{1}{c|}{\tikzcirclenew[fill=blue]{3pt}} & \multicolumn{1}{c|}{\tikzcircle[fill=orange]{3pt}} & \multicolumn{1}{c}{\tikzcirclenew[fill=blue]{3pt}} \\ \hline

\multicolumn{1}{c|}{P13} & \multicolumn{1}{c|}{W} & \multicolumn{1}{c|}{\tikzcircle[fill=orange]{3pt}} & \multicolumn{1}{c|}{Once} & \multicolumn{1}{c|}{No} & \multicolumn{1}{c|}{\tikzcircle[fill=orange]{3pt}} & \multicolumn{1}{c|}{\tikzcircle[fill=orange]{3pt}} & \multicolumn{1}{c|}{\tikzcircle[fill=orange]{3pt}} & \multicolumn{1}{c|}{\tikzcircle[fill=orange]{3pt}} & \multicolumn{1}{c|}{\tikzcircle[fill=orange]{3pt}} & \multicolumn{1}{c|}{P52} & \multicolumn{1}{c|}{M} & \multicolumn{1}{c|}{\tikzcircle[fill=orange]{3pt}} & \multicolumn{1}{c|}{Never} & \multicolumn{1}{c|}{Some} & \multicolumn{1}{c|}{\tikzcircle[fill=orange]{3pt}} & \multicolumn{1}{c|}{\tikzcircle[fill=orange]{3pt}} & \multicolumn{1}{c|}{\tikzcirclenew[fill=blue]{3pt}} & \multicolumn{1}{c|}{\tikzcircle[fill=orange]{3pt}} & \multicolumn{1}{c}{\tikzcircle[fill=orange]{3pt}} \\ \hline

\multicolumn{1}{c|}{P14} & \multicolumn{1}{c|}{W} & \multicolumn{1}{c|}{\tikzcircle[fill=orange]{3pt}} & \multicolumn{1}{c|}{Never} & \multicolumn{1}{c|}{Some} & \multicolumn{1}{c|}{\tikzcirclenew[fill=blue]{3pt}} & \multicolumn{1}{c|}{\tikzcirclenew[fill=blue]{3pt}} & \multicolumn{1}{c|}{\tikzcircle[fill=orange]{3pt}} & \multicolumn{1}{c|}{\tikzcircle[fill=orange]{3pt}} & \multicolumn{1}{c|}{\tikzcircle[fill=orange]{3pt}} & \multicolumn{1}{c|}{P53} & \multicolumn{1}{c|}{M} & \multicolumn{1}{c|}{\tikzcircle[fill=orange]{3pt}} & \multicolumn{1}{c|}{Once} & \multicolumn{1}{c|}{No} & \multicolumn{1}{c|}{\tikzcircle[fill=orange]{3pt}} & \multicolumn{1}{c|}{\tikzcircle[fill=orange]{3pt}} & \multicolumn{1}{c|}{\tikzcirclenew[fill=blue]{3pt}} & \multicolumn{1}{c|}{\tikzcircle[fill=orange]{3pt}} & \multicolumn{1}{c}{\tikzcircle[fill=orange]{3pt}} \\ \hline

\multicolumn{1}{c|}{P15} & \multicolumn{1}{c|}{M} & \multicolumn{1}{c|}{\tikzcircle[fill=orange]{3pt}} & \multicolumn{1}{c|}{Never} & \multicolumn{1}{c|}{No} & \multicolumn{1}{c|}{\tikzcircle[fill=orange]{3pt}} & \multicolumn{1}{c|}{\tikzcirclenew[fill=blue]{3pt}} & \multicolumn{1}{c|}{\tikzcirclenew[fill=blue]{3pt}} & \multicolumn{1}{c|}{\tikzcircle[fill=orange]{3pt}} & \multicolumn{1}{c|}{\tikzcircle[fill=orange]{3pt}} & \multicolumn{1}{c|}{P54} & \multicolumn{1}{c|}{W} & \multicolumn{1}{c|}{\tikzcircle[fill=orange]{3pt}} & \multicolumn{1}{c|}{Never} & \multicolumn{1}{c|}{No} & \multicolumn{1}{c|}{\tikzcircle[fill=orange]{3pt}} & \multicolumn{1}{c|}{\tikzcirclenew[fill=blue]{3pt}} & \multicolumn{1}{c|}{\tikzcircle[fill=orange]{3pt}} & \multicolumn{1}{c|}{\tikzcircle[fill=orange]{3pt}} & \multicolumn{1}{c}{\tikzcircle[fill=orange]{3pt}} \\ \hline

\multicolumn{1}{c|}{P16} & \multicolumn{1}{c|}{M} & \multicolumn{1}{c|}{\tikzcircle[fill=orange]{3pt}} & \multicolumn{1}{c|}{Once} & \multicolumn{1}{c|}{No} & \multicolumn{1}{c|}{\tikzcirclenew[fill=blue]{3pt}} & \multicolumn{1}{c|}{\tikzcircle[fill=orange]{3pt}} & \multicolumn{1}{c|}{\tikzcirclenew[fill=blue]{3pt}} & \multicolumn{1}{c|}{\tikzcircle[fill=orange]{3pt}} & \multicolumn{1}{c|}{\tikzcircle[fill=orange]{3pt}} & \multicolumn{1}{c|}{P55} & \multicolumn{1}{c|}{W} & \multicolumn{1}{c|}{\tikzcircle[fill=orange]{3pt}} & \multicolumn{1}{c|}{Once} & \multicolumn{1}{c|}{No} & \multicolumn{1}{c|}{\tikzcirclenew[fill=blue]{3pt}} & \multicolumn{1}{c|}{\tikzcircle[fill=orange]{3pt}} & \multicolumn{1}{c|}{\tikzcircle[fill=orange]{3pt}} & \multicolumn{1}{c|}{\tikzcircle[fill=orange]{3pt}} & \multicolumn{1}{c}{\tikzcircle[fill=orange]{3pt}} \\ \hline

\multicolumn{1}{c|}{P17} & \multicolumn{1}{c|}{W} & \multicolumn{1}{c|}{\tikzcircle[fill=orange]{3pt}} & \multicolumn{1}{c|}{Once} & \multicolumn{1}{c|}{No} & \multicolumn{1}{c|}{\tikzcircle[fill=orange]{3pt}} & \multicolumn{1}{c|}{\tikzcirclenew[fill=blue]{3pt}} & \multicolumn{1}{c|}{\tikzcircle[fill=orange]{3pt}} & \multicolumn{1}{c|}{\tikzcircle[fill=orange]{3pt}} & \multicolumn{1}{c|}{\tikzcircle[fill=orange]{3pt}} & \multicolumn{1}{c|}{P56} & \multicolumn{1}{c|}{M} & \multicolumn{1}{c|}{\tikzcircle[fill=orange]{3pt}} & \multicolumn{1}{c|}{Never} & \multicolumn{1}{c|}{No} & \multicolumn{1}{c|}{\tikzcircle[fill=orange]{3pt}} & \multicolumn{1}{c|}{\tikzcircle[fill=orange]{3pt}} & \multicolumn{1}{c|}{\tikzcirclenew[fill=blue]{3pt}} & \multicolumn{1}{c|}{\tikzcircle[fill=orange]{3pt}} & \multicolumn{1}{c}{\tikzcircle[fill=orange]{3pt}} \\ \hline

\multicolumn{1}{c|}{P18} & \multicolumn{1}{c|}{W} & \multicolumn{1}{c|}{\tikzcirclenew[fill=blue]{3pt}} & \multicolumn{1}{c|}{Once} & \multicolumn{1}{c|}{No} & \multicolumn{1}{c|}{\tikzcirclenew[fill=blue]{3pt}} & \multicolumn{1}{c|}{\tikzcirclenew[fill=blue]{3pt}} & \multicolumn{1}{c|}{\tikzcirclenew[fill=blue]{3pt}} & \multicolumn{1}{c|}{\tikzcircle[fill=orange]{3pt}} & \multicolumn{1}{c|}{\tikzcirclenew[fill=blue]{3pt}} & \multicolumn{1}{c|}{P57} & \multicolumn{1}{c|}{M} & \multicolumn{1}{c|}{\tikzcircle[fill=orange]{3pt}} & \multicolumn{1}{c|}{Few times} & \multicolumn{1}{c|}{No} & \multicolumn{1}{c|}{\tikzcircle[fill=orange]{3pt}} & \multicolumn{1}{c|}{\tikzcirclenew[fill=blue]{3pt}} & \multicolumn{1}{c|}{\tikzcircle[fill=orange]{3pt}} & \multicolumn{1}{c|}{\tikzcircle[fill=orange]{3pt}} & \multicolumn{1}{c}{\tikzcirclenew[fill=blue]{3pt}} \\ \hline

\multicolumn{1}{c|}{P19} & \multicolumn{1}{c|}{M} & \multicolumn{1}{c|}{\tikzcirclenew[fill=blue]{3pt}} & \multicolumn{1}{c|}{Never} & \multicolumn{1}{c|}{No} & \multicolumn{1}{c|}{\tikzcirclenew[fill=blue]{3pt}} & \multicolumn{1}{c|}{\tikzcirclenew[fill=blue]{3pt}} & \multicolumn{1}{c|}{\tikzcirclenew[fill=blue]{3pt}} & \multicolumn{1}{c|}{\tikzcircle[fill=orange]{3pt}} & \multicolumn{1}{c|}{\tikzcirclenew[fill=blue]{3pt}} & \multicolumn{1}{c|}{P58} & \multicolumn{1}{c|}{M} & \multicolumn{1}{c|}{\tikzcircle[fill=orange]{3pt}} & \multicolumn{1}{c|}{Once} & \multicolumn{1}{c|}{Some} & \multicolumn{1}{c|}{\tikzcircle[fill=orange]{3pt}} & \multicolumn{1}{c|}{\tikzcircle[fill=orange]{3pt}} & \multicolumn{1}{c|}{\tikzcirclenew[fill=blue]{3pt}} & \multicolumn{1}{c|}{\tikzcirclenew[fill=blue]{3pt}} & \multicolumn{1}{c}{\tikzcircle[fill=orange]{3pt}} \\ \hline

\multicolumn{1}{c|}{P20} & \multicolumn{1}{c|}{M} & \multicolumn{1}{c|}{\tikzcircle[fill=orange]{3pt}} & \multicolumn{1}{c|}{Few times} & \multicolumn{1}{c|}{No} & \multicolumn{1}{c|}{\tikzcircle[fill=orange]{3pt}} & \multicolumn{1}{c|}{\tikzcirclenew[fill=blue]{3pt}} & \multicolumn{1}{c|}{\tikzcircle[fill=orange]{3pt}} & \multicolumn{1}{c|}{\tikzcircle[fill=orange]{3pt}} & \multicolumn{1}{c|}{\tikzcircle[fill=orange]{3pt}} & \multicolumn{1}{c|}{P59} & \multicolumn{1}{c|}{M} & \multicolumn{1}{c|}{\tikzcirclenew[fill=blue]{3pt}} & \multicolumn{1}{c|}{Never} & \multicolumn{1}{c|}{No} & \multicolumn{1}{c|}{\tikzcirclenew[fill=blue]{3pt}} & \multicolumn{1}{c|}{\tikzcirclenew[fill=blue]{3pt}} & \multicolumn{1}{c|}{\tikzcirclenew[fill=blue]{3pt}} & \multicolumn{1}{c|}{\tikzcirclenew[fill=blue]{3pt}} & \multicolumn{1}{c}{\tikzcirclenew[fill=blue]{3pt}} \\ \hline

\multicolumn{1}{c|}{P21} & \multicolumn{1}{c|}{M} & \multicolumn{1}{c|}{\tikzcirclenew[fill=blue]{3pt}} & \multicolumn{1}{c|}{Often} & \multicolumn{1}{c|}{No} & \multicolumn{1}{c|}{\tikzcirclenew[fill=blue]{3pt}} & \multicolumn{1}{c|}{\tikzcirclenew[fill=blue]{3pt}} & \multicolumn{1}{c|}{\tikzcircle[fill=orange]{3pt}} & \multicolumn{1}{c|}{\tikzcircle[fill=orange]{3pt}} & \multicolumn{1}{c|}{\tikzcirclenew[fill=blue]{3pt}} & \multicolumn{1}{c|}{P60} & \multicolumn{1}{c|}{M} & \multicolumn{1}{c|}{\tikzcirclenew[fill=blue]{3pt}} & \multicolumn{1}{c|}{Once} & \multicolumn{1}{c|}{No} & \multicolumn{1}{c|}{\tikzcirclenew[fill=blue]{3pt}} & \multicolumn{1}{c|}{\tikzcircle[fill=orange]{3pt}} & \multicolumn{1}{c|}{\tikzcirclenew[fill=blue]{3pt}} & \multicolumn{1}{c|}{\tikzcirclenew[fill=blue]{3pt}} & \multicolumn{1}{c}{\tikzcirclenew[fill=blue]{3pt}} \\ \hline

\multicolumn{1}{c|}{P22} & \multicolumn{1}{c|}{W} & \multicolumn{1}{c|}{\tikzcirclenew[fill=blue]{3pt}} & \multicolumn{1}{c|}{Once} & \multicolumn{1}{c|}{No} & \multicolumn{1}{c|}{\tikzcirclenew[fill=blue]{3pt}} & \multicolumn{1}{c|}{\tikzcirclenew[fill=blue]{3pt}} & \multicolumn{1}{c|}{\tikzcirclenew[fill=blue]{3pt}} & \multicolumn{1}{c|}{\tikzcircle[fill=orange]{3pt}} & \multicolumn{1}{c|}{\tikzcircle[fill=orange]{3pt}} & \multicolumn{1}{c|}{P61} & \multicolumn{1}{c|}{M} & \multicolumn{1}{c|}{\tikzcircle[fill=orange]{3pt}} & \multicolumn{1}{c|}{Few times} & \multicolumn{1}{c|}{No} & \multicolumn{1}{c|}{\tikzcircle[fill=orange]{3pt}} & \multicolumn{1}{c|}{\tikzcircle[fill=orange]{3pt}} & \multicolumn{1}{c|}{\tikzcirclenew[fill=blue]{3pt}} & \multicolumn{1}{c|}{\tikzcirclenew[fill=blue]{3pt}} & \multicolumn{1}{c}{\tikzcircle[fill=orange]{3pt}} \\ \hline

\multicolumn{1}{c|}{P23} & \multicolumn{1}{c|}{W} & \multicolumn{1}{c|}{\tikzcirclenew[fill=blue]{3pt}} & \multicolumn{1}{c|}{Often} & \multicolumn{1}{c|}{No} & \multicolumn{1}{c|}{\tikzcirclenew[fill=blue]{3pt}} & \multicolumn{1}{c|}{\tikzcirclenew[fill=blue]{3pt}} & \multicolumn{1}{c|}{\tikzcirclenew[fill=blue]{3pt}} & \multicolumn{1}{c|}{\tikzcircle[fill=orange]{3pt}} & \multicolumn{1}{c|}{\tikzcirclenew[fill=blue]{3pt}} & \multicolumn{1}{c|}{P62} & \multicolumn{1}{c|}{M} & \multicolumn{1}{c|}{\tikzcircle[fill=orange]{3pt}} & \multicolumn{1}{c|}{Few times} & \multicolumn{1}{c|}{No} & \multicolumn{1}{c|}{\tikzcircle[fill=orange]{3pt}} & \multicolumn{1}{c|}{\tikzcirclenew[fill=blue]{3pt}} & \multicolumn{1}{c|}{\tikzcircle[fill=orange]{3pt}} & \multicolumn{1}{c|}{\tikzcircle[fill=orange]{3pt}} & \multicolumn{1}{c}{\tikzcircle[fill=orange]{3pt}} \\ \hline

\multicolumn{1}{c|}{P24} & \multicolumn{1}{c|}{M} & \multicolumn{1}{c|}{\tikzcirclenew[fill=blue]{3pt}} & \multicolumn{1}{c|}{Often} & \multicolumn{1}{c|}{No} & \multicolumn{1}{c|}{\tikzcirclenew[fill=blue]{3pt}} & \multicolumn{1}{c|}{\tikzcircle[fill=orange]{3pt}} & \multicolumn{1}{c|}{\tikzcirclenew[fill=blue]{3pt}} & \multicolumn{1}{c|}{\tikzcircle[fill=orange]{3pt}} & \multicolumn{1}{c|}{\tikzcirclenew[fill=blue]{3pt}} & \multicolumn{1}{c|}{P63} & \multicolumn{1}{c|}{W} & \multicolumn{1}{c|}{\tikzcirclenew[fill=blue]{3pt}} & \multicolumn{1}{c|}{Never} & \multicolumn{1}{c|}{No} & \multicolumn{1}{c|}{\tikzcircle[fill=orange]{3pt}} & \multicolumn{1}{c|}{\tikzcirclenew[fill=blue]{3pt}} & \multicolumn{1}{c|}{\tikzcirclenew[fill=blue]{3pt}} & \multicolumn{1}{c|}{\tikzcircle[fill=orange]{3pt}} & \multicolumn{1}{c}{\tikzcirclenew[fill=blue]{3pt}} \\ \hline

\multicolumn{1}{c|}{P25} & \multicolumn{1}{c|}{M} & \multicolumn{1}{c|}{\tikzcirclenew[fill=blue]{3pt}} & \multicolumn{1}{c|}{Often} & \multicolumn{1}{c|}{No} & \multicolumn{1}{c|}{\tikzcircle[fill=orange]{3pt}} & \multicolumn{1}{c|}{\tikzcirclenew[fill=blue]{3pt}} & \multicolumn{1}{c|}{\tikzcirclenew[fill=blue]{3pt}} & \multicolumn{1}{c|}{\tikzcirclenew[fill=blue]{3pt}} & \multicolumn{1}{c|}{\tikzcircle[fill=orange]{3pt}} & \multicolumn{1}{c|}{P64} & \multicolumn{1}{c|}{W} & \multicolumn{1}{c|}{\tikzcirclenew[fill=blue]{3pt}} & \multicolumn{1}{c|}{Never} & \multicolumn{1}{c|}{No} & \multicolumn{1}{c|}{\tikzcirclenew[fill=blue]{3pt}} & \multicolumn{1}{c|}{\tikzcirclenew[fill=blue]{3pt}} & \multicolumn{1}{c|}{\tikzcirclenew[fill=blue]{3pt}} & \multicolumn{1}{c|}{\tikzcircle[fill=orange]{3pt}} & \multicolumn{1}{c}{\tikzcirclenew[fill=blue]{3pt}} \\ \hline

\multicolumn{1}{c|}{P26} & \multicolumn{1}{c|}{W} & \multicolumn{1}{c|}{\tikzcircle[fill=orange]{3pt}} & \multicolumn{1}{c|}{Few times} & \multicolumn{1}{c|}{No} & \multicolumn{1}{c|}{\tikzcircle[fill=orange]{3pt}} & \multicolumn{1}{c|}{\tikzcircle[fill=orange]{3pt}} & \multicolumn{1}{c|}{\tikzcirclenew[fill=blue]{3pt}} & \multicolumn{1}{c|}{\tikzcircle[fill=orange]{3pt}} & \multicolumn{1}{c|}{\tikzcircle[fill=orange]{3pt}} & \multicolumn{1}{c|}{P65} & \multicolumn{1}{c|}{M} & \multicolumn{1}{c|}{\tikzcirclenew[fill=blue]{3pt}} & \multicolumn{1}{c|}{Never} & \multicolumn{1}{c|}{No} & \multicolumn{1}{c|}{\tikzcircle[fill=orange]{3pt}} & \multicolumn{1}{c|}{\tikzcirclenew[fill=blue]{3pt}} & \multicolumn{1}{c|}{\tikzcirclenew[fill=blue]{3pt}} & \multicolumn{1}{c|}{\tikzcircle[fill=orange]{3pt}} & \multicolumn{1}{c}{\tikzcirclenew[fill=blue]{3pt}} \\ \hline

\multicolumn{1}{c|}{P27} & \multicolumn{1}{c|}{M} & \multicolumn{1}{c|}{\tikzcircle[fill=orange]{3pt}} & \multicolumn{1}{c|}{Few times} & \multicolumn{1}{c|}{No} & \multicolumn{1}{c|}{\tikzcircle[fill=orange]{3pt}} & \multicolumn{1}{c|}{\tikzcirclenew[fill=blue]{3pt}} & \multicolumn{1}{c|}{\tikzcircle[fill=orange]{3pt}} & \multicolumn{1}{c|}{\tikzcircle[fill=orange]{3pt}} & \multicolumn{1}{c|}{\tikzcircle[fill=orange]{3pt}} & \multicolumn{1}{c|}{P66} & \multicolumn{1}{c|}{W} & \multicolumn{1}{c|}{\tikzcirclenew[fill=blue]{3pt}} & \multicolumn{1}{c|}{Never} & \multicolumn{1}{c|}{No} & \multicolumn{1}{c|}{\tikzcircle[fill=orange]{3pt}} & \multicolumn{1}{c|}{\tikzcirclenew[fill=blue]{3pt}} & \multicolumn{1}{c|}{\tikzcirclenew[fill=blue]{3pt}} & \multicolumn{1}{c|}{\tikzcircle[fill=orange]{3pt}} & \multicolumn{1}{c}{\tikzcirclenew[fill=blue]{3pt}} \\ \hline

\multicolumn{1}{c|}{P28} & \multicolumn{1}{c|}{M} & \multicolumn{1}{c|}{\tikzcircle[fill=orange]{3pt}} & \multicolumn{1}{c|}{Once} & \multicolumn{1}{c|}{No} & \multicolumn{1}{c|}{\tikzcircle[fill=orange]{3pt}} & \multicolumn{1}{c|}{\tikzcirclenew[fill=blue]{3pt}} & \multicolumn{1}{c|}{\tikzcirclenew[fill=blue]{3pt}} & \multicolumn{1}{c|}{\tikzcircle[fill=orange]{3pt}} & \multicolumn{1}{c|}{\tikzcircle[fill=orange]{3pt}} & \multicolumn{1}{c|}{P67} & \multicolumn{1}{c|}{M} & \multicolumn{1}{c|}{\tikzcircle[fill=orange]{3pt}} & \multicolumn{1}{c|}{Never} & \multicolumn{1}{c|}{No} & \multicolumn{1}{c|}{\tikzcircle[fill=orange]{3pt}} & \multicolumn{1}{c|}{\tikzcircle[fill=orange]{3pt}} & \multicolumn{1}{c|}{\tikzcirclenew[fill=blue]{3pt}} & \multicolumn{1}{c|}{\tikzcircle[fill=orange]{3pt}} & \multicolumn{1}{c}{\tikzcircle[fill=orange]{3pt}} \\ \hline

\multicolumn{1}{c|}{P29} & \multicolumn{1}{c|}{M} & \multicolumn{1}{c|}{\tikzcircle[fill=orange]{3pt}} & \multicolumn{1}{c|}{Never} & \multicolumn{1}{c|}{No} & \multicolumn{1}{c|}{\tikzcircle[fill=orange]{3pt}} & \multicolumn{1}{c|}{\tikzcircle[fill=orange]{3pt}} & \multicolumn{1}{c|}{\tikzcircle[fill=orange]{3pt}} & \multicolumn{1}{c|}{\tikzcircle[fill=orange]{3pt}} & \multicolumn{1}{c|}{\tikzcircle[fill=orange]{3pt}} & \multicolumn{1}{c|}{P68} & \multicolumn{1}{c|}{M} & \multicolumn{1}{c|}{\tikzcirclenew[fill=blue]{3pt}} & \multicolumn{1}{c|}{Few times} & \multicolumn{1}{c|}{No} & \multicolumn{1}{c|}{\tikzcirclenew[fill=blue]{3pt}} & \multicolumn{1}{c|}{\tikzcirclenew[fill=blue]{3pt}} & \multicolumn{1}{c|}{\tikzcirclenew[fill=blue]{3pt}} & \multicolumn{1}{c|}{\tikzcircle[fill=orange]{3pt}} & \multicolumn{1}{c}{\tikzcircle[fill=orange]{3pt}} \\ \hline

\multicolumn{1}{c|}{P30} & \multicolumn{1}{c|}{M} & \multicolumn{1}{c|}{\tikzcirclenew[fill=blue]{3pt}} & \multicolumn{1}{c|}{Few times} & \multicolumn{1}{c|}{No} & \multicolumn{1}{c|}{\tikzcirclenew[fill=blue]{3pt}} & \multicolumn{1}{c|}{\tikzcirclenew[fill=blue]{3pt}} & \multicolumn{1}{c|}{\tikzcirclenew[fill=blue]{3pt}} & \multicolumn{1}{c|}{\tikzcircle[fill=orange]{3pt}} & \multicolumn{1}{c|}{\tikzcirclenew[fill=blue]{3pt}} & \multicolumn{1}{c|}{P69} & \multicolumn{1}{c|}{M} & \multicolumn{1}{c|}{\tikzcirclenew[fill=blue]{3pt}} & \multicolumn{1}{c|}{Few times} & \multicolumn{1}{c|}{No} & \multicolumn{1}{c|}{\tikzcirclenew[fill=blue]{3pt}} & \multicolumn{1}{c|}{\tikzcirclenew[fill=blue]{3pt}} & \multicolumn{1}{c|}{\tikzcirclenew[fill=blue]{3pt}} & \multicolumn{1}{c|}{\tikzcirclenew[fill=blue]{3pt}} & \multicolumn{1}{c}{\tikzcircle[fill=orange]{3pt}} \\ \hline

\multicolumn{1}{c|}{P31} & \multicolumn{1}{c|}{M} & \multicolumn{1}{c|}{\tikzcirclenew[fill=blue]{3pt}} & \multicolumn{1}{c|}{Never} & \multicolumn{1}{c|}{No} & \multicolumn{1}{c|}{\tikzcircle[fill=orange]{3pt}} & \multicolumn{1}{c|}{\tikzcirclenew[fill=blue]{3pt}} & \multicolumn{1}{c|}{\tikzcirclenew[fill=blue]{3pt}} & \multicolumn{1}{c|}{\tikzcirclenew[fill=blue]{3pt}} & \multicolumn{1}{c|}{\tikzcirclenew[fill=blue]{3pt}} & \multicolumn{1}{c|}{P70} & \multicolumn{1}{c|}{M} & \multicolumn{1}{c|}{\tikzcirclenew[fill=blue]{3pt}} & \multicolumn{1}{c|}{Few times} & \multicolumn{1}{c|}{Some} & \multicolumn{1}{c|}{\tikzcirclenew[fill=blue]{3pt}} & \multicolumn{1}{c|}{\tikzcirclenew[fill=blue]{3pt}} & \multicolumn{1}{c|}{\tikzcirclenew[fill=blue]{3pt}} & \multicolumn{1}{c|}{\tikzcircle[fill=orange]{3pt}} & \multicolumn{1}{c}{\tikzcirclenew[fill=blue]{3pt}} \\ \hline

\multicolumn{1}{c|}{P32} & \multicolumn{1}{c|}{M} & \multicolumn{1}{c|}{\tikzcircle[fill=orange]{3pt}} & \multicolumn{1}{c|}{Never} & \multicolumn{1}{c|}{No} & \multicolumn{1}{c|}{\tikzcirclenew[fill=blue]{3pt}} & \multicolumn{1}{c|}{\tikzcirclenew[fill=blue]{3pt}} & \multicolumn{1}{c|}{\tikzcircle[fill=orange]{3pt}} & \multicolumn{1}{c|}{\tikzcircle[fill=orange]{3pt}} & \multicolumn{1}{c|}{\tikzcircle[fill=orange]{3pt}} & \multicolumn{1}{c|}{P71} & \multicolumn{1}{c|}{M} & \multicolumn{1}{c|}{\tikzcircle[fill=orange]{3pt}} & \multicolumn{1}{c|}{Few times} & \multicolumn{1}{c|}{No} & \multicolumn{1}{c|}{\tikzcircle[fill=orange]{3pt}} & \multicolumn{1}{c|}{\tikzcircle[fill=orange]{3pt}} & \multicolumn{1}{c|}{\tikzcircle[fill=orange]{3pt}} & \multicolumn{1}{c|}{\tikzcircle[fill=orange]{3pt}} & \multicolumn{1}{c}{\tikzcirclenew[fill=blue]{3pt}} \\ \hline

\multicolumn{1}{c|}{P33} & \multicolumn{1}{c|}{M} & \multicolumn{1}{c|}{\tikzcirclenew[fill=blue]{3pt}} & \multicolumn{1}{c|}{Few times} & \multicolumn{1}{c|}{No} & \multicolumn{1}{c|}{\tikzcircle[fill=orange]{3pt}} & \multicolumn{1}{c|}{\tikzcirclenew[fill=blue]{3pt}} & \multicolumn{1}{c|}{\tikzcirclenew[fill=blue]{3pt}} & \multicolumn{1}{c|}{\tikzcircle[fill=orange]{3pt}} & \multicolumn{1}{c|}{\tikzcirclenew[fill=blue]{3pt}} & \multicolumn{1}{c|}{P72} & \multicolumn{1}{c|}{M} & \multicolumn{1}{c|}{\tikzcirclenew[fill=blue]{3pt}} & \multicolumn{1}{c|}{Few times} & \multicolumn{1}{c|}{No} & \multicolumn{1}{c|}{\tikzcirclenew[fill=blue]{3pt}} & \multicolumn{1}{c|}{\tikzcircle[fill=orange]{3pt}} & \multicolumn{1}{c|}{\tikzcirclenew[fill=blue]{3pt}} & \multicolumn{1}{c|}{\tikzcircle[fill=orange]{3pt}} & \multicolumn{1}{c}{\tikzcirclenew[fill=blue]{3pt}} \\ \hline

\multicolumn{1}{c|}{P34} & \multicolumn{1}{c|}{M} & \multicolumn{1}{c|}{\tikzcirclenew[fill=blue]{3pt}} & \multicolumn{1}{c|}{Few times} & \multicolumn{1}{c|}{No} & \multicolumn{1}{c|}{\tikzcirclenew[fill=blue]{3pt}} & \multicolumn{1}{c|}{\tikzcircle[fill=orange]{3pt}} & \multicolumn{1}{c|}{\tikzcirclenew[fill=blue]{3pt}} & \multicolumn{1}{c|}{\tikzcircle[fill=orange]{3pt}} & \multicolumn{1}{c|}{\tikzcirclenew[fill=blue]{3pt}} & \multicolumn{1}{c|}{P73} & \multicolumn{1}{c|}{M} & \multicolumn{1}{c|}{\tikzcirclenew[fill=blue]{3pt}} & \multicolumn{1}{c|}{Once} & \multicolumn{1}{c|}{No} & \multicolumn{1}{c|}{\tikzcirclenew[fill=blue]{3pt}} & \multicolumn{1}{c|}{\tikzcirclenew[fill=blue]{3pt}} & \multicolumn{1}{c|}{\tikzcirclenew[fill=blue]{3pt}} & \multicolumn{1}{c|}{\tikzcircle[fill=orange]{3pt}} & \multicolumn{1}{c}{\tikzcirclenew[fill=blue]{3pt}} \\ \hline

\multicolumn{1}{c|}{P35} & \multicolumn{1}{c|}{W} & \multicolumn{1}{c|}{\tikzcirclenew[fill=blue]{3pt}} & \multicolumn{1}{c|}{Few times} & \multicolumn{1}{c|}{No} & \multicolumn{1}{c|}{\tikzcirclenew[fill=blue]{3pt}} & \multicolumn{1}{c|}{\tikzcirclenew[fill=blue]{3pt}} & \multicolumn{1}{c|}{\tikzcircle[fill=orange]{3pt}} & \multicolumn{1}{c|}{\tikzcircle[fill=orange]{3pt}} & \multicolumn{1}{c|}{\tikzcirclenew[fill=blue]{3pt}} & \multicolumn{1}{c|}{P74} & \multicolumn{1}{c|}{M} & \multicolumn{1}{c|}{\tikzcircle[fill=orange]{3pt}} & \multicolumn{1}{c|}{Never} & \multicolumn{1}{c|}{No} & \multicolumn{1}{c|}{\tikzcircle[fill=orange]{3pt}} & \multicolumn{1}{c|}{\tikzcirclenew[fill=blue]{3pt}} & \multicolumn{1}{c|}{\tikzcirclenew[fill=blue]{3pt}} & \multicolumn{1}{c|}{\tikzcircle[fill=orange]{3pt}} & \multicolumn{1}{c}{\tikzcircle[fill=orange]{3pt}} \\ \hline

\multicolumn{1}{c|}{P36} & \multicolumn{1}{c|}{M} & \multicolumn{1}{c|}{\tikzcirclenew[fill=blue]{3pt}} & \multicolumn{1}{c|}{Never} & \multicolumn{1}{c|}{No} & \multicolumn{1}{c|}{\tikzcirclenew[fill=blue]{3pt}} & \multicolumn{1}{c|}{\tikzcirclenew[fill=blue]{3pt}} & \multicolumn{1}{c|}{\tikzcirclenew[fill=blue]{3pt}} & \multicolumn{1}{c|}{\tikzcircle[fill=orange]{3pt}} & \multicolumn{1}{c|}{\tikzcirclenew[fill=blue]{3pt}} & \multicolumn{1}{c|}{P75} & \multicolumn{1}{c|}{M} & \multicolumn{1}{c|}{\tikzcirclenew[fill=blue]{3pt}} & \multicolumn{1}{c|}{Few times} & \multicolumn{1}{c|}{No} & \multicolumn{1}{c|}{\tikzcirclenew[fill=blue]{3pt}} & \multicolumn{1}{c|}{\tikzcirclenew[fill=blue]{3pt}} & \multicolumn{1}{c|}{\tikzcirclenew[fill=blue]{3pt}} & \multicolumn{1}{c|}{\tikzcircle[fill=orange]{3pt}} & \multicolumn{1}{c}{\tikzcirclenew[fill=blue]{3pt}} \\ \hline

\multicolumn{1}{c|}{P37} & \multicolumn{1}{c|}{M} & \multicolumn{1}{c|}{\tikzcirclenew[fill=blue]{3pt}} & \multicolumn{1}{c|}{Few times} & \multicolumn{1}{c|}{Some} & \multicolumn{1}{c|}{\tikzcirclenew[fill=blue]{3pt}} & \multicolumn{1}{c|}{\tikzcircle[fill=orange]{3pt}} & \multicolumn{1}{c|}{\tikzcirclenew[fill=blue]{3pt}} & \multicolumn{1}{c|}{\tikzcircle[fill=orange]{3pt}} & \multicolumn{1}{c|}{\tikzcirclenew[fill=blue]{3pt}} &

\multicolumn{10}{c}{\multirow{3}{*}{\begin{tabular}[c]{@{}c@{}}\textbf{Legend:} M: Man | W: Woman | \tikzcirclenew[fill=blue]{3pt}: Tim | \tikzcircle[fill=orange]{3pt}: Abi\\ MT: Motivation | SE: Self-efficacy | R: Risk \\ IP: Information processing | L: Learning\end{tabular}}} \\ \cline{1-10}

\multicolumn{1}{c|}{P38} & \multicolumn{1}{c|}{M} & \multicolumn{1}{c|}{\tikzcirclenew[fill=blue]{3pt}} & \multicolumn{1}{c|}{Never} & \multicolumn{1}{c|}{No} & \multicolumn{1}{c|}{\tikzcirclenew[fill=blue]{3pt}} & \multicolumn{1}{c|}{\tikzcirclenew[fill=blue]{3pt}} & \multicolumn{1}{c|}{\tikzcirclenew[fill=blue]{3pt}} & \multicolumn{1}{c|}{\tikzcircle[fill=orange]{3pt}} & \multicolumn{1}{c|}{\tikzcircle[fill=orange]{3pt}} & \multicolumn{10}{l}{} \\ \cline{1-10}

\multicolumn{1}{c|}{P39} & \multicolumn{1}{c|}{M} & \multicolumn{1}{c|}{\tikzcirclenew[fill=blue]{3pt}} & \multicolumn{1}{c|}{Few times} & \multicolumn{1}{c|}{Some} & \multicolumn{1}{c|}{\tikzcirclenew[fill=blue]{3pt}} & \multicolumn{1}{c|}{\tikzcirclenew[fill=blue]{3pt}} & \multicolumn{1}{c|}{\tikzcirclenew[fill=blue]{3pt}} & \multicolumn{1}{c|}{\tikzcircle[fill=orange]{3pt}} & \multicolumn{1}{c|}{\tikzcirclenew[fill=blue]{3pt}} & \multicolumn{10}{l}{} \\ \hline
\end{tabular}
\end{table*}



\begin{comment}


The five facets used by the GenderMag method are presented in Table~\ref{tab:gendermagfactes}. The facets are used to define personas (e.g., Abi and Tim). GenderMag highlights that differences relevant to inclusiveness lie not in a person's gender identity but in the facet values themselves~\cite{hill2017gender}. Nevertheless, Abi's facet values are more frequent in women than in other genders, and Tim's facet values are more frequent in men than in other genders. 

%(Figure~\ref{fig:abbypersona})

\begin{table}[!ht]\scriptsize
\centering
\vspace{-2.5mm}
\caption{GenderMag facets~\cite{burnett2016gendermag}}
\label{tab:gendermagfactes}
\newcommand{\pb}[1]{\parbox[t][][t]{1.0\linewidth}{#1} \vspace{-2pt}}

\begin{tabular}{p{12mm}|p{62mm}}
\hline
\multicolumn{1}{>{\centering\arraybackslash}m{12mm}|}{\textbf{GenderMag Facets}} & \multicolumn{1}{>{\centering\arraybackslash}m{62mm}}{\textbf{Definition}} \\ \hline \hline

Motivation & \pb{Women tend (statistically) to be motivated to use technology for what they can accomplish with it, whereas men are often motivated by their enjoyment of technology per se~\cite{simon2000impact, cassell2002hand, margolis2002unlocking, hou2006girls, kelleher2009barriers, burnett2010gender, burnett2011gender, hallstrom2015gender}. This difference can affect which software features users choose to use}. \\ \hline 

Information processing styles & \pb{To solve problems, people often need to process new information. Women are more likely (statistically) to process new information comprehensively—gathering fairly complete information before proceeding—but men are more likely to use selective styles—following the first promising information, then backtracking if needed~\cite{cafferata1989gender, meyers1991exploring, coursaris2008empirical, riedl2010there, meyers2015revisiting}. Each style has advantages, but either is at a disadvantage when not supported by the software.} \\ \hline

Computer self-efficacy & \pb{Self-efficacy is a person's confidence about succeeding at a specific task, which influences their use of cognitive strategies, persistence, and strategies for coping with obstacles. Empirical data have shown that women often have lower computer self-efficacy than men, which can affect their behavior with technology~\cite{margolis2002unlocking, durndell2002computer, hartzel2003self, beckwith2005effectiveness, beckwith2006tinkering, burnett2010gender, burnett2011gender, singh2013role, huffman2013using}.} \\ \hline

Risk aversion & \pb{Research shows that women statistically tend to be more risk-averse than men~\cite{weber2002domain, dohmen2011individual, charness2012strong}. These results span numerous decision-making domains, such as ethics, investment, gambling, health/safety, and career. Risk aversion with software usage can impact users' decisions as to which feature sets to use.} \\ \hline

Learning: by Process vs. by Tinkering & \pb{Research across age groups and professions reports women being statistically less likely to playfully experiment (“tinker”) with software features new to them, compared to men~\cite{beckwith2006tinkering, hou2006girls, rosner2009learning, burnett2010gender, cao2010debugging, chang2014specialization}. However, when women do tinker, they tend to be more likely to reflect during the process and thereby sometimes profit from it more than men do.} \\ \hline \hline
\end{tabular}
\end{table}

\end{comment}
&&&&

\subsection{Qualitative Analysis}
Interview sessions in Phase 1 and Phase 3 were transcribed using CLOVA Notes. %For Phase 1 interviews, two researchers iteratively conducted thematic analysis\revision{~\cite{Braun2006}}, which involved inductive coding on the data, identifying emerging themes, and grouping into higher hierarchies. We worked together closely during this phase and followed a consensus-coding approach, having consistent meetings to merge individual codes, resolve conflict, and reach agreements on the final codebook. For this reason, calculating inter-rater reliability was not deemed necessary~\cite{McDonald2019}.
\revision{
For Phase 1 interviews, thematic analysis was conducted inductively through multiple iterations~\cite{Braun2006}. First, two researchers individually performed line-by-line open coding on eight interviews, generating initial codes that closely resembled text from the transcript such as “instant ban”, “add to modmail” and “bot seems insincere”. This was followed by focused coding~\cite{Saldaa2021}, where we identified recurring themes and sorted them into broader categories such as “escalating procedure”, “integration into existing system” and “tool perception”, which formed our initial codebook. The first author then applied this codebook across the remaining interviews, refining and adding codes as new insights emerged. After this, the two researchers met again to validate the updated codebook, consolidating higher-level themes along the dimension of moderators’ practices in addressing interpersonal harm, their stances on adopting restorative justice tools through ApoloBot’s framework, and potential impacts of implementing such a system. Finally, we aligned these diverse perspectives to outline the opportunity space and challenges associated with transitioning from traditional moderation practice to integrating restorative justice tools, laying the groundwork for our results.
}

% First-level codes included short phrases similar to text from the transcript, such as “instant ban”, “add to modmail” and “bot seems insincere”. These were then sorted into broader categories such as “escalating procedure”, “integration into existing system” and “tool perception”.

Phase 3 interviews were coded by the first author following a similar inductive process \revision{based on the codebook developed in Phase 1. While Phase 1 interviews focused on moderators’ reflections on prior experiences, Phase 3 expanded upon these by grounding the insights in practical deployment outcomes. Successful use cases from Phase 3 demonstrated how expectations from Phase 1 were met, validating the key opportunities where the tool effectively fulfilled its design intent. Equally significant were the unmet expectations, where anticipated use cases were not realized, as they revealed a new-found understanding of the practical challenges and critical areas of the opportunity space where the tool's effectiveness fell short. These observations were thus incorporated into the final codebook by combining and adding to Phase 1's codes, enhancing the framework that underpins our findings.
}

% Given that some reflections in this phase provided deeper elaborations of insights from Phase 1, we combined and consolidated several categories.

\subsection{Methodological Limitations}
As highlighted by Xiao et al., \textit{"Online communities should allow for partial success or no success without enforcing the ideal outcome, especially at the early stage of implementation when there are insufficient resources or commitments"}~\cite{Xiao2023}. Restorative justice, being relatively new and context-specific, poses significant challenges when evaluated within a brief testing period. Our study is therefore constrained by the limited empirical data available on ApoloBot usage, and the analysis presented here relies mostly on interview data from Phases 1 and 3. 
\revision{
This limitation also arises from how we shape our research focus, which is not on delivering a fully-realized restorative justice tool ready for adoption, but on developing a conceptual artifact to probe its implementation and foster critical reflections among moderators. For those who engaged with the tool, their experiences provide concrete evidence of its realized potential for effective adoption. On the other hand, investigating those who did not use the tool reveals challenges and critical gaps in its suitability within the broader online landscape, which can inform future alternatives or complements that might address the limitations.
Centering the discussion on these dual perspectives allows a deeper and more comprehensive view of how diverse online communities are currently positioned for restorative justice tools, however it might compensate the technical significance of the proposed system.}

% While these interviews offered preliminary insights into the perceived potential of ApoloBot and similar restorative justice tools, they may not offer a comprehensive assessment of their broader significance.

In addition, our study primarily gathers insights from moderators rather than victims or offenders. While this focus offers rich insights into the practical aspects of the tool adoption and execution, it lacks the perspectives of the remaining stakeholders essential to restorative justice, and thus may not fully capture the complete user experience.

Finally, even though our participants come from a wide range of international communities covering diverse topics, the fact that they are solely English speakers limits the cross-cultural generalizations that can be made based on our findings.
