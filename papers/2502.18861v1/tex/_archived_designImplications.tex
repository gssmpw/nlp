\subsection{Design Implications for future Online Restorative Justice Tools}
In response to the identified challenges, participants suggested several design considerations that could enhance restorative justice tools integration viewing from the perspective of ApoloBot. [in general sth sth? > transition to the details] \note{then from here we generalize these findings for future RJ tools can improve upon our empirical findings}

\subsubsection{Customization are needed to address the diverse needs of (conflicting) stakeholders} Even though we tried to embed “customization” but it’s more in terms of participation (whether to approve/decline) - there are much more diverse needs and perceptions that we need to cater to - diff mods have diff ways to approach RJ (both ones that have already embedded and not yet but want to) and diff stakeholders have diff perceptions / receptions to it. Not just within different servers but under diff scenarios (e.g of diff severity as mentioned above) the approach might change (e.g sometimes victim might want to approve apology sometimes not?) -> need to cater to these nuanced contexts!
\begin{itemize}
    \item For mods: cater to their approach to dealing w complicated contexts like this: from decision rationale (showing additional user history, past conv, etc) to handling mechanisms (customize message to change tone - sometimes can be light for light situations and sometimes might be tougher? / server's "own language"/tone/vibe), to action (not just mute but also warn even ban; punishment duration adjustment: instead of completely lift, can set the amount of reduction or even increase in case of system abuse) -> more features to support mods in cases of complicated and nuanced contexts.
    \item For members/involved stakeholders: cater to their specific needs and resolution during and after harm cases:
    \begin{itemize}
        \item More back-and-forth conv: can send more message before final apology, offender can edit/resend apology based in victim/mod's "feedback" in case they're actually remorseful and rethink; might even give more than 2nd chance: 3rd, 4th... when they can rethink and craft a better apology - similar to appeal
        \item If both sides needed apology
        \item Multiple stakeholders instead of just 1-1
    \end{itemize}

\end{itemize}

\subsubsection{Technical Framework might be benefited with “social” embeddings}
e.g “sympathetic UI” - since it’s addressing a very personal and emotional issue (e.g some suggested adding pictures together / use webhooks / use different styles of embedded message)
→ (Somewhat) Overcome the perception of “talking to a bot” (instead of the other person)
Currently it’s a bot with simple avatar and description, so it might also not be very “trustworthy” / “appealing” for members to trust and use it

\subsubsection{"Restoration" as a broader contextual implementation} Restoration is indeed very contextual - to the case and the stakeholders involved themselves, to the moderation mechanism that mod practices as well as to the community-defined values as a whole -> Every cases might have different 'picture'(?) of restoration - sometimes apology as illustrated in ApoloBot is just not the best solution:
\begin{itemize}
    \item Sometimes it's as simple as tell them to take some rest; or just explaining w/o needing to apologize
    \item Sometimes it can take other forms than text: Some ppl said quotes or even visual things like pictures or memes or sth could also help to ease the situations - depending on, again, the server's "vibe" and their specific relationship ofc
    \item Even further: community-specific “compensation” (e.g in-game items, arts? - things that that person in that community values -> also make it personal!)
    \item Even broader: Education (set "goals" or like a "program"(?) for offenders / raise awareness; Support groups?
\end{itemize}

Overall, [summary of implications]. These are some of the generalizable findings from the empirical results of ApoloBot's case study. Much more still need to be done - need to allow further trials and errors to test and evaluate these implications and further advance the practice of RJ as it integrates and evolves with the community? And as the practice becomes more familiar even normalized esp to those who fit/need it, eventually the opportunities will be embraced while the challenges may be mitigated. \textit{"If you use the bot from the very start, you can to build up the community nicely based on that. Then if new people join, they just follow the format, what other people already do. And it will be much easier for them to adopt it, since they don't have to change any "old" habits."} (P14)