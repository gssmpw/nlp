\section{Introduction}
Online harm remains a pressing issue for content moderation. A survey by Ipsos revealed that nearly 60\% of online users reported experiencing some form of harm, while over 40\% of victims refrained from seeking external help~\cite{Dunn2023}. This highlights the shortcomings of traditional moderation approaches, which typically emphasize punishments such as account sanctions and content removals~\cite{Gillespie2018, Roberts2019}. While these methods may stop the harm as it occurs, they often fail to prevent further recurrence and meet the needs of victims and others involved. This perspective has been widely echoed within the HCI and CSCW community: several studies pointed out how punitive approaches often leave users unsupported, leading to negative emotions~\cite{Jhaver2019b, Ma2021}, difficulty making sense of their penalties~\cite{West2018, Vaccaro2020}, and struggles in reforming behaviors~\cite{Kou2021}. In response to these shortcomings, some researchers have advocated instead for alternative frameworks like restorative justice, which prioritizes mediation and engagement among stakeholders to address harm collectively.

Restorative justice focuses on repairing harm by involving offenders, victims, and sometimes community members in conversations aiming at healing and resolution. Within this process, offenders are supported to come to terms with their wrongdoing while victims can voice their needs and receive closure from the community. This approach has been effective in reducing recidivism and supporting reintegration in real-world contexts~\cite{Ness2016, Wood2016}.

Recent years have seen a growing interest in applying restorative justice methods to address harm in online environments. These methods have been proposed as a potential support mechanism for offenders such as demonetized creators~\cite{Ma2021}, banned game players~\cite{Kou2021}, and cyberbullying attackers~\cite{Aliyu2024}. Victims from various social groups also show a preference for restorative  resolution~\cite{Schoenebeck2021a, Schoenebeck2021b, Schoenebeck2023}. 
%More comprehensively, Xiao et al. conducted interviews with victims, offenders, and moderators to explore the opportunities and challenges of implementing \textit{victim-offender conferences}--a form of restorative practice--within gaming communities. 
However, much of the discussion remains conceptual. While it is evident that neither retributive nor restorative justice is a one-size-fits-all solution, and that they are effective only in specific contexts and value systems, the precise conditions behind their differential effectiveness are still not fully understood. Furthermore, practical implementations of restorative justice in online spaces are still underexplored, and it remains generally unclear how restorative practices can be structured and integrated into community moderation.

To address this gap, our study explores the implementation of restorative justice from a design perspective, specifically through tools designed for online communities. We developed ApoloBot, a Discord bot that helps moderators streamline a subset of restorative justice principles within their communities by facilitating apologies between offenders and victims, serving as a probe to explore the design space for restorative justice tools. Though restorative justice encompasses a variety of methods, apologies are one of the most basic and core elements, representing a reasonable starting point for exploration. By embedding a process for apologizing into a bot~---~one of the platform's most commonly used moderation tools~---~we provide a practical, accessible resource to \revision{examine the introduction of} restorative justice methods into the online space\revision{, along with} their broader applicability and potential impacts. 

Through interviews with 16 Discord moderators and deployment with six of them, we gather insights into how moderators perceive and engage with restorative justice tools like ApoloBot. Extending prior work, we conducted a more in-depth analysis of the opportunities and challenges associated with integrating restorative justice tools into existing moderation mechanisms. We seek to answer the following research questions:
\begin{itemize}
    \item \textbf{RQ1:} \textit{What opportunities do restorative justice tools present for online communities? Specifically, what values and contexts determine their effectiveness?}
    \item \textbf{RQ2:} \textit{What challenges accompany the implementation and usage of restorative justice tools in online communities?}
\end{itemize} 

We conclude by discussing the %criteria for tools' ``success'',
\revision{possible ways to evaluate the tools' expected outcomes, and propose design implications}
%design implications, and considerations for future research 
to drive the ongoing development of restorative justice frameworks in the online space. 