\section{Price Competition for Unweighted Matroid}
\label{app:price_comp_matroid}
In this section, we present the equilibrium analysis for the case when the buyer's constraint is a matroid constraint.
This section looks at an algorithm (Algorithm~\ref{alg:modified_greedy}) which is a natural generalization of Algorithm~\ref{alg:knapsack} in that it skips over any element that is not feasible in the bang-per-buck order given the modules selected so far.
Note that Algorithm~\ref{alg:modified_greedy} is equivalent to Algorithm~\ref{alg:greedy} when $\vecv(i) = 1$ for all $i$.
In this section, we show that for all $\vecv$, an equilibrium exists but we only prove an equilibrium quality result for the case where $\vecv(i) = 1$ for all $i$.

\begin{algorithm}
\caption{Modified Greedy Algorithm}
\label{alg:modified_greedy}
\begin{algorithmic}[1]
\State \textbf{Input}: price vector $\vec p$, values $\vec v$, budget $B$, feasible sets $\I \in 2^{[n]}$.
\State \textbf{Output}: set $G$ of modules.
\State Initialize $G \gets \emptyset$.
\State Re-arrange modules such that $\frac{\vecv(1)}{\vecp(1)} \geq \frac{\vecv(2)}{\vecp(2)} \geq \dots \geq \frac{\vecv(n)}{\vecp(n)}$ with ties broken arbitrarily.
\For{$i=1,2 \dots ,n$}
\If{$G \cup \{i\} \notin \I$}
\State \textbf{continue}
\EndIf
\If{$\vec p(G) + \vec p(i) > B$}
\State \textbf{break}
\EndIf
\State $G \gets G \cup \{i\}$
\EndFor
\end{algorithmic}
\end{algorithm}
We divide the proof of the main theorem of this section into two parts. First, in Subsection~\ref{sec:matroid_price_eqm_existance}, we show the existence of $\eps$-equilibrium prices and later in Subsection~\ref{sec:matroid_eq_analysis} we analyze the quality of all $\eps$-equilibrium prices.

\subsection{Existence of Price Equilibrium}\label{sec:matroid_price_eqm_existance}
We begin our analysis by showing the existence of $\eps$-equilibrium prices. To this end, we provide an algorithm that explicitly constructs an $\eps$-equilibrium price vector for the pricing game between the modules.

Algorithm~\ref{alg:equilibrium_dynamics} shows how to construct an equilibrium.
In addition, to simplify the notations and reduce clutter, we assume that the buyer breaks ties in favor of modules with higher $\frac{\vecv(i)}{\vecc(i)}$ and as a result, we show that Algorithm~\ref{alg:equilibrium_dynamics} converges to an exact equilibrium.
More specifically, we assume that the buyer breaks ties in the initial permutation $\pi^0$ defined in Algorithm~\ref{alg:equilibrium_dynamics}.

We note that since the buyer has no information about the module's private cost, the buyer cannot implement such a tie-breaking rule while selecting modules.
In order to simulate the desired tie-breaking rule, we can lower the price of accepted modules by an infintesimal amount which results in an $\eps$-equilibrium for any small $\eps > 0$.

The algorithm begins by initializing $\vecp$ as $\vec p = \vec c$ and an ordering $\pi^0$ which is decreasing in their bang-per-buck w.r.t.~$\vecp$ at round zero. We then define the set $T^\ind$ that denotes the set of modules that increase their price at iteration $\ind$ and $\pi^\ind$ that denotes the bang-per-buck order over modules at the price $\vec p^\ind$. Throughout the algorithm, we maintain the following invariants at iteration $\ind$: (a) the price of modules $\pi^0(\ind), \dots, \pi^0(n)$ are unchanged and equal to their cost, (b) the bang-per-buck of the modules higher than $ \frac{\vec v(\pi(\ind))}{\vec c(\pi(\ind))}$ remains higher than $ \frac{\vec v(\pi(\ind))}{\vec c(\pi(\ind))}$ at price $\vec p^\ind$,  (c) for every module in $i\in T^k$, we have $\frac{\vec v(i)}{\vec p^\ind(i)} = \frac{\vec v(\pi^0(\ind))}{\vec c(\pi^0(\ind ))}$.  

We can describe the algorithm as follows.
First, we raise the prices of all modules in $T^{k-1}$ so that their bang-per-buck meets that of module $\pi^k(k)$, whose price is still at its cost.
Next, we check whether or not $\pi^k(k) \in \spn(A^{k-1} \cup T^{k-1})$.
If not we add it to $T^{k-1}$ to obtain $T^k$.
Otherwise, we find the (unique) circuit in $A^{k-1} \cup T^{k-1} \cup \{\pi^k(k)\}$ and add it, without $\pi^k(k)$, to $A^{k-1}$ to get $A^k$.
Doing so freezes the price of these modules.
We then remove $A^{k-1}$ from $T^{k-1}$ to get $T^k$, the set of modules which can still raise their prices.
Next, we check whether or not $A^{k} \cup T^{k}$ can fit within the budget at the price $\vecp^k$.
If not then, if necessary, we lower the prices for $T^{k-1}$ (whose prices we had just raised) so that $A^{k-1} \cup T^{k-1}$ fits within the budget.
In effect, this ensures that $\greedy$ rejects $\pi^k(k)$.

If the algorithm does manage to go through every module and $T^n \neq \emptyset$ then we increase the price of every module in $T^n$ until the price of all accepted modules in $A^n \cup T^n$ is equal to the budget.

\begin{algorithm}
\caption{Equilibrium Construction}
\label{alg:equilibrium_dynamics}
\begin{algorithmic}[1]
\State Initialize price $\vecp^0 \gets \vecc$ and $\pi^0$ such that $\frac{\vecv(\pi^0(i))}{\vecc(\pi^0(i))}\geq \frac{\vecv(\pi^0(j))}{\vecc(\pi^0(j))}$ for $i < j$.
\State Initialize $A^0 \gets \emptyset, T^0 \gets \emptyset$.
%
%
\For{$\ind = 1, \ldots, n$}

\State Copy $\pi^{\ind-1}, \vec p^{\ind-1}$ into $\pi,\vec p$ for simplicity of notation.
\State Update prices:
\[
\vecp^k(\pi(j)) =
\begin{cases}
\frac{\vecp(\pi(k))}{\vecv(\pi(k))} \cdot \vecv(\pi(j)) & j \in T^k \\
\vecp(\pi(j)) & j \notin T^k
\end{cases}
\]

\If{$\pi(\ind) \notin \spn(A^{k-1} \cup T^{k-1}$)}
    \Comment{Accept $\pi(k)$.}
    \State $A^k \gets A^{k-1}$
    \State $T^k \gets T^{k-1} \cup \{\pi(k)\}$
\Else
    \Comment{Reject $\pi(k)$ and freeze modules in the circuit.}
    \State Let $C^k$ be the unique circuit in $A^{k-1} \cup T^{k-1} \cup \{\pi(k)\}$. \label{line:find_circuit} 
    \State Let $A^k \gets A^{k-1} \cup C^k \setminus \{\pi(k)\}$.
    \State Let $T^k \gets T^{k-1} \setminus A^k$.
\EndIf

\State Set $\pi^k$ as bang-per-buck ordering with respect to $\vecp^k$, breaking ties according to $\pi^0$.
\If{$\vecp^k(A^k \cup T^k) > B$} \label{line:budget_check}
\Comment{Reject last module, update prices, and terminate.}
\State $T^{k} \gets T^{k-1}$, $A^k \gets A^{k-1}$.
\State Update $\vecp^k(i) \gets \min\left\{\vecv(i) \cdot \frac{B - \vecp^{k-1}(A^{k-1})}{\vecv(T^{k-1})}, \vecp^k(i) \right\}$ for $i \in T^{k-1}$. \label{line:lower_price}
\State \textbf{return} $\vecp^k, \pi^k$
\EndIf
\EndFor

\State Update $\vecp^n(i) \gets \vecv(i) \cdot \frac{B - \vecp^n(A^{n})}{\vecv(T^n)}$ for $i \in T^n$.
\State \textbf{return} $\vecp^n, \pi^n$
\end{algorithmic}
\end{algorithm}

Note that Algorithm~\ref{alg:equilibrium_dynamics} terminates in at most $n$ iterations.
In each iteration, the algorithm needs to be able to test feasibility of a set and to be able to find circuits.

Let $\bar{\vec  p}$ be the price computed by Algorithm~\ref{alg:equilibrium_dynamics}. We first characterize $S_{\bar{\vecp}}$, the set of selected modules at the price, and obtain a structural decomposition of the set $S_{\bar {\vec p}}$ in the given underlying matroid $\mathcal M$. Throughout the analysis, we let $k^*$ be the last iteration of the dynamics.

First, we show that after the price updates at iteration $k$, the selected modules at round $k-1$ and round $k$ stays identical among the modules $\pi^{k-1}(1), \dots, \pi^{k-1}(k)$ with top $k$ bang-per-buck values. This property of our dynamics ensures that the modules that increase their price remain selected by the greedy allocation rule. Hence, the modules that update price at round $k$ increase their utility. 
  
\begin{claim}\label{claim:winners_stays_winners}
For any iteration $k \leq k^*$, we have $S_{\vecp^{k-1}} \cap \pi^{k-1}([k-1]) = S_{\vecp^{k}} \cap \pi^{k}([k-1]) = A^{k-1} \cup T^{k-1}$.
\end{claim}
\begin{proof}

Define $R^k = \pi^k([k]) \setminus (A^k \cup T^k)$.
We think of $R^k$ as the set of modules that $\greedy$ is sure to reject.
We show that the following invariants hold:
\begin{enumerate}
    \item $\pi^{k-1}([k-1]) = \pi^k([k-1])$ and $\pi^{k-1}(k) = \pi^{k}(k)$.
    \item when the prices are $\vecp^k$, the bang-per-buck of any module in $A^{k-1}$ is at least that of any module in $T^{k-1}$.
    \item when the prices are $\vecp^k$, the bang-per-buck of any module in $T^{k-1}$ is at least that of any module in $\pi^k([n]) \setminus \pi^k([k])$.
    \item $\vecp^k(A^{k-1} \cup T^{k-1}) \leq B$.
    \item if $i \leq k$ and $\pi^{k}(i) \in R^k$ then $\pi^k(i) \in \spn(A^k \cap \pi^k([i-1]))$.
\end{enumerate}
These invariants show that if $\greedy$ follows the order of $\pi^k$ then it accepts every module in $A^{k-1} \cup T^{k-1}$ when the price vector is $\vecp^k$.
In other words, $S_{\vecp^{k-1}} \cap \pi^{k-1}([k-1]) = S_{\vecp^{k}} \cap \pi^{k}([k-1])$.

The invariants are trivial if $k = 1$ since $A^{k-1} = T^{k-1} = R^{k-1} = \emptyset$.
We now assume that the invariant holds at iteration $k-1$ and prove that it remains true at iteration $k$.

For the first invariant, the only modules that increase their price is $T^{k-1} \subseteq \pi^{k-1}([k-1])$.
Since their bang-per-buck is at most that of module $\pi^{k-1}([k])$ this means that all modules in $\pi^{k-1}([k-1])$ still come before module $\pi^{k-1}([k])$.
So $\pi^{k-1}([k-1]) = \pi^k([k-1])$.
The fact that $\pi^{k-1}(k) = \pi^{k}(k)$ now easily follows because ties are broken according to $\pi^0$.

For the second invariant, we need the following claim.
\begin{claim}
  \label{claim:T_k_increasing}
  For all $i \in T^{k-1}$, we have $\vecp^k(i) \geq \vecp^{k-1}(i)$.
\end{claim}
\begin{proof}
Note that $\vecp^{k-1}(i) = \frac{\vecp(\pi^{k-1}(k-1))}{\vecv(\pi^{k-1}(k-1)))} \cdot \vecv(\pi^{k-1}(i))$.
We have that
\[
    \vecp^k(i) \geq \min\left\{
        \vecv(i) \cdot \frac{B - \vecv^{k-1}(A^{k-1})}{\vecv(T^{k-1})},
        \frac{\vecp(\pi^{k}(k))}{\vecv(\pi^{k}(k)))} \cdot \vecv(\pi^{k}(i))
    \right\}.
\]
The second term in the minimum is more than $\vecp^{k-1}(\pi^{k-1}(i))$ since $\pi^{k-1}(k-1)$ comes before $\pi^{k}(k)$ in the bang-per-buck order.
For the first term, we know that $A^{k-1} \cup T^{k-1}$ is budget-feasible (otherwise we would have terminated) so $B - \vecp^{k-1}(A^{k-1}) \geq \vecp^{k-1}(T^{k-1})$.
Thus $\vecv(i) \cdot \frac{B - \vecv^{k-1}(A^{k-1})}{\vecv(T^{k-1})} \geq \vecv(i) \cdot \frac{\vecp^{k-1}(T^{k-1})}{\vecv(T^{k-1})} = \vecp^{k-1}(i)$.
\end{proof}

Since the bang-per-buck of any module in $A^{k-1}$ is at most that of any module in $T^{k-2}$ when the price vector is $\vecp^{k-1}$ and the prices in $A^{k-1}$ do not change, we conclude, using Claim~\ref{claim:T_k_increasing} that the same is true when the price vector is $\vecp^k$.
This shows that the second invariant holds.

The third invariant is because prices of modules in $\pi^k([k-1])$ are only raised to meet the bang-per-buck of module in $\pi^k(k)$.

The fourth invariant is trivial if Line~\ref{line:budget_check} is false.
Otherwise, it is true because (i) $\vecp^{k-1}(A^{k-1}) + \vecp^{k-1}(T^{k-1})$ by the fact that the invariant holds for $k-1$, (ii) $\vecp^{k}(A^{k-1}) + \vecp^{k}(A^{k-1})$, and (iii) $\vecp^k(T^{k-1}) \leq B - \vecp^{k-1}(A^{k-1}) \leq \vecp^{k-1}(T^{k-1})$ which is just a re-arrangement of (i).

Finally, for the fifth invariant, first suppose that $i \leq k-1$.
We have that $\pi^{k-1}(i)$ forms a circuit with $A^{k-1} \cap \pi^{k-1}([i-1])$ which we call $C_i$.
When we raise prices in $T^{k-1}$, invariants $2$ and $3$ as well as the fact that $\pi^{k-1}, \pi^k$ are are consistent with $\pi^0$, imply that the relative ordering in $C_i$ does not change.
So $\pi^k(i)$ still forms a circuit with $A^{k-1} \cap \pi^{k-1}([i-1])$.
The case $i = k$ is straightforward since in this case Line~\ref{line:find_circuit} finds the circuit in $A^{k-1} \cup T^{k-1}$ and adds it to $A^k$.
In addition, observe that $A^k \cap \pi^k([k-1]) = A^k$.
\end{proof}

Let $\bar{\vecp}$ be the price computed by Algorithm~\ref{alg:equilibrium_dynamics}.
\begin{claim}
    \label{claim:matroid_greedy_output}
    $S_{\bar{\vecp}} = A^{k^*} \cup T^{k^*}$.
\end{claim}
\begin{proof}
    Claim~\ref{claim:winners_stays_winners} shows that $S_{\bar{\vecp}} \cap \pi^{k^*}([k^*-1]) = A^{k^*-1} \cup T^{k^*-1}$.
    First, suppose that $k^* < n$.
    If $A^{k^*} = A^{k^*-1}$ and $T^{k^*} = T^{k^*-1}$ before Line~\ref{line:budget_check} then Line~\ref{line:lower_price} ensures that $\vecp^{k^*}(A^{k^*-1} \cup T^{k^*-1}) = B$.
    So $S_{\bar{\vecp}} = A^{k^*-1} \cup T^{k^*-1}$ since any module not in $\pi^{k^*}([k^*-1])$ is inspected by $\greedy$ after $A^{k^*-1} \cup T^{k^*-1}$.
    
    Now suppose that $k^* = n$.
    Since Line~\ref{line:budget_check} never happens we have that $S_{\bar{\vecp}} \cap \pi^{k^*}([k^*-1]) = A^{k^*-1} \cup T^{k^*-1}$ and $A^{k^*} \cup T^{k^*}$ is feasible for $\greedy$ (and includes $A^{k^*-1} \cup T^{k^*-1}$).
    So $\greedy$ selects $A^{k^*} \cup T^{k^*}$.
\end{proof}

\begin{lemma}
We can decompose
\[
    S_{\bar{\vecp}} = T_{k^*} \cup \bigcup_{k < k^*} A_{k+1} \setminus A_k = T_{k^*} \cup \bigcup_{k < k^*} T_k \setminus (T_{k+1} \cup \pi^k(k)).
\]
\end{lemma}
\begin{proof}
The first equality follows from Claim~\ref{claim:matroid_greedy_output} and that $A_{k^*} = \bigcup_{k < k^*} A_{k+1} \setminus A_k$.
The second inequality is because $A^{k+1} \setminus A^k = T^{k} \setminus T^{k+1}$ in Algorithm~\ref{alg:equilibrium_dynamics}.
\end{proof}

\begin{theorem}
\label{thm:unweighted_eq_exist}
The vector $\bar{\vecp}$ is an equilibrium price.
\end{theorem}
\begin{proof}
If $i \notin A^{k^*} \cup T^{k^*} \cup R^{k^*} = \pi^{k^*}([k^*])$ then by Claim~\ref{claim:matroid_greedy_output}, module $i$ is not inspected by $\greedy$ before the budget constraint is met. Since module $i$ is already bidding its cost, it cannot deviate to get selected.

Suppose that $i \in R^{k^*}$.
By invariant $5$ in the proof of Claim~\ref{claim:winners_stays_winners}, we have that $\pi^{k^*}(i) \in \spn(A^{k^*} \cap \pi^{k^*}([i-1]))$.
So $i$ forms a circuit with $A^{k^*} \cap \pi^{k^*}([i-1])$ and is inspected by $\greedy$ after every other module in this circuit.
Since it is already bidding its cost, it cannot deviate to get selected by $\greedy$.

Now suppose $i \in A^{k^*}$.
This means there is some $k \in [k^*]$ such that $i \in A_{k} \setminus A_{k-1}$.
At this point, there was a circuit $C \subseteq A^{k} \cup \{\pi^k(k)\}$ that contained $i$ and after iteration $k$ all prices in $C$ were never raised.
If module $i$ raises its price then it would be inspected by $\greedy$ after every other module in $C$ and so it would be rejected. So $i$ has no incentive to deviate.

Finally, suppose $i \in T^{k^*}$.
By construction all modules in $T^{k^*}$ have the same bang-per-buck and have the lowest bang-per-buck among all modules selected by $\greedy$.
There are two cases.
If $\bar{\vecp}(A^{k^*} \cup T^{k^*}) = B$ then if $i$ increases its price, it would be rejected by $\greedy$ since there is no budget remaining when $i$ is inspected (it could come after every other module in $A^{k^*} \cup T^{k^*}$).
If $\bar{\vecp}(A^{k^*} \cup T^{k^*}) < B$ then we must have entered the if condition in Line~\ref{line:budget_check}.
Now in Line~\ref{line:lower_price}, we must set $\vecp^{k^*}$ to be the second term in the minimum.
If it was the first term, the budget constraint would be exactly met with modules $A^{k^*} \cup T^{k^*}$.
Note that $\pi^{k^*}(k^*) \in \spn(A^{k^*} \cup T^{k^*})$.
To see this, note that if not, we would have $A^{k^*} \cup T^{k^*} = A^{k^*-1} \cup T^{k^*-1}$.
Since in Line~\ref{line:lower_price}, we must set $\vecp^{k^*}$ to be the second term in the minimum, this implies that $\vecp^{k^*}(A^{k^*} \cup T^{k^*}) > B$.
We conclude then that $\bar{\vecp}(i) = \vecp^{k^*}(i)$.
In particular, this means that $i$ has the same bang-per-buck as module $\pi^{k^*}(k^*)$.
Now if $i$ raises its price then it would be inspected by $\greedy$ after $\pi^{k^*}(k^*)$.
Since $\pi^{k^*}(k^*)$ does not form a circuit with $A_{k^*} \cup T_{k^*}$, $\greedy$ would accept $\pi^{k^*}(k^*)$ but reject $i$.
So we conclude that $i$ cannot deviate.
\end{proof}

\subsection{Quality of Equilibria} \label{sec:matroid_eq_analysis}
First, we prove the following lemma that reduces our quality of equilibrium analysis to the case when the modules which are not selected at the equilibrium price sets their price equals to their respective cost.  
In this section, we will assume that $\frac{\vecv(i)}{\vecc(i)}$ are all distinct.

\begin{lemma}
Let $\vec p$ be an $\eps$-equilibrium price and $S_{\vec p}$ be the set of selected modules by $\greedy$. Then for any module $i\notin S_\vec p$, there exists an $\eps$-equilibrium prices $\vec {\bar p}$ such that (i) $\bar{\vecp} \leq \vecc(i) + \eps$, (ii)
$S_{\vec {\bar p}} \subseteq S_{\vec p}$, and (iii) $\bar{\vecp} \leq \vecp$.
\end{lemma}
\begin{proof}
Let $\pi$ be the ordering over the modules according to their bang-per-buck at price $\vec p$ with ties broken in lexicographic order. Fix a module $i \notin S_{\vecp}$ such that $\vecp(i) > \vecc(i) + \eps$.
Since $\vecp$ is an $\eps$-equilibrium price, we have that $u_i(\vecc(i) + \eps, \vecp_{-i}) = 0$. In particular, module $i$ is not accepted even if it lowered its price to $\vecc(i) + \eps$.
We now consider the deviation price vector $\vecp'$ where $\vecp'(i) = \vecc(i) + \eps$ and $\vecp'(k) = \vecp(k)$ for $k \neq i$.

Let $\pi'$ be the new ordering of modules according to their bang-per-buck at price $(\vec c(i)+\varepsilon,\vec p_{-i})$ and $j' = \pi'^{-1}(i)$ be the new position of module $i$ in the new order $\pi'$ (again, breaking ties in lexicographic order). We observe that since $i \notin S_\vec p$, either (1) $i \in \spn (\{\pi'(1),\dots,\pi'(j'-1)\})$ or (2) $i \notin \spn (\{\pi'(1),\dots,\pi'(j'-1)\})$ and $\vec c(i) + \eps + \vecp(S_{\vec p}\cap \{\pi'(1),\dots,\pi'(j'-1)\}) > B$. We analyze both cases separately. 
In both cases, by definition, $i \notin S_{\vec c(i) + \varepsilon , \vec p_{-i}}$.

\paragraph{Case 1: $i \in \spn (\{\pi'(1),\dots,\pi'(j'-1)\})$.} In this case, we claim that $S_{\vec p} = S_{\vec c(i) + \varepsilon , \vec p_{-i}}$.  We prove the claim by showing that $S_\vec p\cap \{\pi(1),\dots , \pi(n)\} = S_{\vec c(i)+\varepsilon, \vec p_{-i}} \cap \{\pi'(1),\dots , \pi'(n)\}$ where $n$ is the size of the set $M$.

We first observe that $\pi(k) = \pi'(k)$ for $k < j'$.
Therefore, we have $S_\vec p\cap \{\pi(1),\dots , \pi(k)\} = S_{\vec c(i)+\varepsilon, \vec p_{-i}} \cap \{\pi'(1),\dots , \pi'(k)\}$.
However, the modules $\pi(j'),\dots, \pi(j-1)$ have their position shifted up by one in the new ordering $\pi'$.
Then we show that for $j' \leq  k< j$, $S_\vec p \cap \{\pi(1),\dots , \pi(k)\} = S_{\vec c(j) + \varepsilon ,\vec p_{-j}} \cap \{\pi'(1),\dots , \pi'(k+1)\} $.
We prove this claim via induction on $k$.
We start with the base case where $k = j'$.
Our assumption here is that $i \in \spn (\{\pi'(1),\dots,\pi'(j'-1)\}$.
Now consider the module $\pi(j')= \pi'(j'+1)$.
We have that
\[
    S_{\vecp} \cap \{\pi(1), \ldots, \pi(j'-1)\} = S_{\vec c(i) + \varepsilon , \vec p_{-i}} \cap \{\pi'(1) \, \ldots, \pi'(j')\}.
\]
In other words, when $\greedy$ inspects module $\pi(j') = \pi'(j'+1)$, the set selected thus far is identical irrespective of whether the order $\pi$ or $\pi'$ is used.
We conclude that $\pi(j') \in S_{\vecp}$ if and only if $\pi'(j'+1) \in S_{\vec c(i) + \varepsilon ,\vec p_{-i}}$.

Next, consider any $k \in \{j'+1, \ldots, j-1\}$ and assume the induction hypothesis
\[
    S_{\vecp} \cap \{\pi(1), \ldots, \pi(k-1)\} = S_{\vec c(i) + \varepsilon , \vec p_{-i}} \cap \{\pi'(1) \, \ldots, \pi'(k)\}.
\]
The argument here is identical to that in the previous paragraph.
When $\greedy$ inspects module $\pi(k) = \pi'(k+1)$, the selected thus far is identical irrespective of whether the order $\pi$ or $\pi'$ is used.
We conclude that $\pi(k) \in S_{\vecp}$ if and only if $\pi'(k+1) \in S_{\vec c(i) + \varepsilon ,\vec p_{-i}}$.

At this point, we have established that
\[
    S_{\vecp} \cap \{\pi(1), \ldots, \pi(j-1)\} = S_{\vec c(i) + \varepsilon , \vec p_{-i}} \cap \{\pi'(1) \, \ldots, \pi'(j)\}.
\]
Our initial hypothesis of the lemma is that $j \notin S_{\vecp}$, whence we conclude that
\[
    S_{\vecp} \cap \{\pi(1), \ldots, \pi(j)\} = S_{\vec c(i) + \varepsilon , \vec p_{-i}} \cap \{\pi'(1) \, \ldots, \pi'(j)\}.
\]
For $k > j$, we have $\pi(k) = \pi'(k)$ and so $\greedy$ behaves identically irrespective of whether the ordering is $\pi$ or $\pi'$.

It remains to establish that $\vecp' = (\vec c(i)+\varepsilon,\vec p_{-i})$ is an $\eps$-equilibrium price.
First, consider a module $i' \notin S_{\vecp'}$ and $i' \neq i$ with $\vecp'(i') > \vecc(i') + \eps$.
We claim that $i'$ would remain rejected by $\greedy$ if it lowered its price to $\vecc(i') + \eps$ when the price vector of the remaining modules is $\vecp'_{-i'}$.
As we established above, $\greedy$ makes identical decisions irrespective of whether the ordering is given by $\pi$ or $\pi'$.
In particular, the set of modules selected by $\greedy$ when it inspects module $i'$ is identical when the price vector is $(\vecc(i') + \eps, \vecp_{-i'})$ or $(\vecc(i') + \eps, \vecp'_{-i'})$.
Thus, since $i'$ did not have a profitable deviation to $\vecc(i') + \eps$ in $\vecp$, we conclude that it also does not have a profitable deviation to $\vecc(i') + \eps$ in $\vecp'$.

Now consider a module $i' \in S_{\vecp'}$.
Suppose that module $i'$ has a profitable deviation to $\vecp(i') + \eps$ when the price vector is $\vecp'$ (recall that $\vecp(i') = \vecp'(i')$ for all $i' \neq i$). We claim that this is only a profitable deviation for $i'$ when the price vector is $\vecp$ which contradicts that $\vecp$ is an $\eps$-equilibrium.
To see this, note that $\vecp' \leq \vecp$.
Thus, module $i'$ with its deviation, can only come earlier in the order inspected by $\greedy$ in $\vecp$ than in $\vecp'$ and the set of accepted modules so far in $\vecp$ must be a subset of the set of accepted modules so far in $\vecp'$.
Thus, module $i'$'s deviation must also have been profitable in $\vecp$.

\paragraph{Case 2: $\vec c(i) + \vecp(S_{\vec p}\cap \{\pi'(1),\dots,\pi'(j'-1)\}) > B$.} Let $T \coloneqq S_{\vec p} \setminus S_{\vec c(i) + \varepsilon, \vec p_{-i}}$ be the set of modules which are dropped from the selected set of modules after module $i$ updated its price to $\vecc(i) + \varepsilon$. 
Here, similar to the first case,  we observe that $\pi(k') = \pi'(k')$ for all $k' < j'$. Hence, we have $S_\vec p \cap \{\pi(1),\dots, \pi(j-1)\} = S_{\vec c(i) + \varepsilon , \vec p_{-i}} \cap \{\pi(1),\dots, \pi(j-1)\}$.  Therefore, $T \subseteq M \setminus \{\pi(1),\dots, \pi(j-1)\}$.
Define the price vector $\bar{\vecp}$ as
\[
\bar{\vecp} = \begin{cases}
\max \left \{  \frac{\vec p(i) + \varepsilon}{\vec v(i)}\cdot \vec v(k) , \vec c(k)  \right\} & k \in T \\
\vecc(i) + \eps & k = i \\
\vecp(k) & k \notin T \cup \{i\}
\end{cases}.
\]
We claim that the price $\vec {\bar p} $ is an $\eps$-equilibrium price. 

We let $\bar \pi$ be the bang-per-buck ordering of modules w.r.t.~the price $\bar{\vec p}$. First, we notice that $\vec {\bar p} \leq \vec p$ point-wise. We further observe that $\pi(k) = \bar \pi(k)$ for all $k =1,\dots,  j'-1$. We consider any module $\bar i$ with $\frac{\vec v(i)}{\bar {\vec p}(i)} < \frac{\vec v(\bar i)}{\bar {\vec p}(\bar i)}$.
By construction of $\bar p$, we have that $\bar{\pi}(\bar{i}) < \bar{\pi}(i)$.
Therefore, we have that $\bar i\in S_\vec p$ if and only if $\bar i \in S_{\bar {\vec {p}}}$ as $\{\pi(1), \dots, \pi(j'-1)\} = \{\bar \pi(1), \dots, \bar \pi(j'-1)\}$. Therefore, if there exists a profitable deviation of at least $\eps$ for module $\bar i$ at price $\bar {\vec p}$ then the same price deviation is profitable of at least $\eps$ for module $\bar i$ at price $\vec p$. This implies that there is no profitable deviation of at least $\eps$ for module $\bar i$ at price $\vec {\bar p}$. 

Next, we consider module $\bar i$ such that $\frac{\vec v(i)}{\bar {\vec p}(i)}> \frac{\vec v(\bar i)}{\bar {\vec p}(\bar i)}$ and $\bar i\notin S_{\vec p}$. Since $\vec c(i) + \vecp(S_{\vec p}\cap \{\pi'(1),\dots,\pi'(j'-1)\}) > B$ and  $\pi(k) = \bar \pi(k)$ for all $k =1,\dots,  j'-1$, we have that $\bar i \notin S_{\vec {\bar p}}$. Since $\bar {\vec p} \leq  \vec p$ point-wise, we have that  if there exists a profitable deviation (of at least $\eps$) for module $\bar i$ at price $\bar {\vec p}$ then the same price deviation is profitable for module $\bar i$ at price $\vec p$ (of at least $\eps$). This implies that there is no profitable deviation for module $\bar i$ at price $\vec {\bar p}$ of at least $\eps$.

We then consider the module $\bar i$ such that $\frac{\vec v(i)}{\bar {\vec p}(i)}> \frac{\vec v(\bar i)}{\bar {\vec p}(\bar i)}$ and $\bar i\in S_{\vec p}$. We observe that $\bar i\in T$ and since $\frac{\vec v(i)}{\bar {\vec p}(i)}> \frac{\vec v(\bar i)}{\bar {\vec p}(\bar i)}$, it must be the case that $\vec {\bar {p}}(\bar i) = \vec c(\bar i)$. 

We finally consider the module $\bar i$ such that $\frac{\vec v(i)}{\bar {\vec p}(i)} = \frac{\vec v(\bar i)}{\bar {\vec p}(\bar i)}$. We first observe that $\bar i\in T$ and $\frac{\vec v(\bar i)}{\vec c(\bar i)} > \frac{\vec v(\bar i)}{\vec {\bar p}(\bar i)}  = \frac{\vec v (i)}{\bar {\vec p(i)}}$. Therefore, module $\bar i$ comes before module $i$ in the ordering $\bar \pi$. We claim that the module $\bar i \in S_{\vec {\bar p}}$. 

We observe that $\bar i$ cannot form a circuit with modules with higher bang-per-buck at price $\bar p$ than $\frac{\vec v(\bar i)}{\vec {\bar p}(\bar i)}$. For the sake of contradiction, let $\bar i$ forms a circuit $C$ with modules with higher bang-per-buck  than $\frac{\vec v(\bar i)}{\vec {\bar p}(\bar i)}$. We note that $C \subseteq \{\bar \pi_{1}, \dots, \bar \pi(j'-1)\}\cup T$. Let $i^*:= \argmin_{k\in T\cap C} \frac{\vec v(k)}{\vec p(k)}$. Since, $i^*$ forms a circuit $C$ with modules with higher bang-per-buck  than $\frac{\vec v( i^*)}{\vec { p}(i^*)}$, $i^*\notin S_{\vec {\bar p}}$ which is a contradiction. 

Let $k$ be the index of module $\bar i$ on the order $\bar \pi$. We observe that $\{\bar \pi(1),\dots , \bar \pi(k-1)\}\subseteq \{\bar \pi(1),\dots , \bar \pi(j'-1)\}\cup T$. Next, we claim that $\vec {\bar p}(S_{\vec {\bar p}} \cap \{\bar \pi(1),\dots , \bar \pi(k-1)\}) + \vec {\bar p}(k)\leq B$. This follows since, $$(S_{\vec {\bar p}} \cap \{\bar \pi(1),\dots , \bar \pi(k-1)\})\cup \bar i \subseteq S_{\vec p} \cap \{\bar \pi(1),\dots , \bar \pi(j'-1)\}\cup T \subseteq S_{\vec p},$$ we conclude that $\vec {\bar p}(S_{\vec {\bar p}} \cap \{\bar \pi(1),\dots , \bar \pi(k-1)\}) + \vec {\bar p}(k)\leq B$. This implies that $\bar i \in S_{\vec {\bar p}}$. 

Next, we want to show that there is no profitable deviation for the module $\bar i$ at price $\bar p$. If module $\bar i$ increases its price to $\vec {\bar p}(i) + \eps$ then $\frac{\vec v(\bar i)}{\vec {\bar p}(\bar i) + \eps} < \frac{\vec v( i)}{\vec {\bar p}(i) }$. Since $\vec c(i) + \vecp(S_{\vec p}\cap \{\pi'(1),\dots,\pi'(j'-1)\}) > B$, $\greedy$ will not select module $\bar i$. This concludes that there is no profitable deviation for the module $\bar i$. Combining above arguments, we conclude that $\vec {\bar p}$ is an $\eps$-equilibrium price.  

In both cases, we observe that $ S_{\bar{\vec  p}}\subseteq  S_{\vec p}$ where $\vec{ \bar{p}}(i) \leq \vec c(i) + \varepsilon$. This completes the proof of the lemma. 
\end{proof}

Due to the above lemma, we immediately obtain the following corollary. 
\begin{corollary}
Let $\vec p$ be an $\eps$-equilibrium price for the price competition game instance $\instance$.
There exists an $\eps$-equilibrium price $\vec{\bar p}$ such that $\vec {\bar {p}}(i) \leq \vec c(i)+\varepsilon$ for all $i \notin S_{\bar{\vecp}}$ and $V(S_{\bar {\vec p}}) \leq V(S_\vec p)$.
\end{corollary}
Due to the above corollary, in order to bound the approximation ratio, it is sufficient to focus on the case where all non-selected modules at equilibrium price bid their private cost.
Next, we can characterize the equilibrium prices for the matroid rank valuations in the following lemma and show that all equilibrium of the pricing game have a constant approximation ratio.

\begin{lemma}
\label{lemma:unweighted_greedy_approx}
Fix an instance $\instance$ of the price competition game where $\I$ denotes a matroid.
Let $\lambda = \max_i \vecc(i) / B$ denote the maximum normalized cost of any module.
There is a sufficiently small $\eps_0 > 0$ (depending on $\vecc, \vecp$) such that for all $\eps \in (0, \eps_0)$, the following holds.
Let $\vecp$ be an $\eps$-equilibrium price and let $S_{\vecp}$ be the set of modules selected by $\greedy$ when $\vecp(i) \leq \vecc(i) + \eps$ for all $i \notin S_{\vecp}$.
Then
\begin{enumerate}
\item $S_{\vecp} \subseteq S_{\vecc}$;
\item there exists $i \in S_{\vecc}$ such that $S_{\vecp} = S_{\vecc} \cap \pi_{\vecc}[\pi_{\vecc}^{-1}(i)]$; and 
\item $\vecv(S_{\vecc}) \leq \left( 1 + \frac{B}{B-\lambda B-\eps} \cdot \frac{\lambda B+\eps}{\lambda B} \right) \vecv(S_{\vecp}).$
\end{enumerate}
\end{lemma}
The first assertion of Lemma~\ref{lemma:unweighted_greedy_approx} shows that the set of modules selected by $\greedy$ in $\vecp$ is a subset of the set of modules selected by $\greedy$ in $\vecc$.
This is used as a stepping stone to prove the second assertion: that $\greedy$ in $\vecp$ actually selects a \emph{prefix} of the modules selected by $\greedy$ in $\vecc$.
Using this, we will prove that $\greedy$ in $\vecp$ gets, approximately, a $2$-approximation to $\greedy$ in $\vecc$.
\begin{proof}
We will choose $\eps_0$ such that for all $i \in [n-1]$, we have
\[
    \frac{\vecv(i)}{\vecc(i) + \eps_0} > \frac{\vecv(i+1)}{\vecc(i+1)}.
\]
Such an $\eps_0 > 0$ exists since we assume that $\frac{\vecv(i)}{\vecc(i)} > \frac{\vecv(i+1)}{\vecc(i+1)}$ for all $i \in [n-1]$.

Now fix $\eps \in (0, \eps_0)$. We begin with the first assertion that $S_{\vecp} \subseteq S_{\vecc}$.
For the sake of contradiction, suppose that $S_{\vecp} \setminus S_{\vecc} \neq \emptyset$ and let $i \in S_{\vecp} \setminus S_{\vecc}$.
We consider two cases and show that neither case can happen.
\paragraph{Case 1: $\vecc(S_{\vecc} \cap \pi_{\vecc}[\pi_{\vecc}^{-1}(i) - 1]) + \vecc(i) > B$.}
Observe that we have $\vecp(S_{\vecp} \cap \pi_\vecp[\pi_\vecp^{-1}(i)]) \leq B$ since we know that $i \in S_{\vecp}$.
For notation, let us write $S_{\vecc}' = S_{\vecc} \cap \pi_{\vecc}[\pi_{\vecc}^{-1}(i)-1]$ and $S_{\vecp}' = S_{\vecp} \cap \pi_{\vecp}[\pi_{\vecp}^{-1}(i) - 1]$.
We claim that $S_{\vecc}' \setminus S_{\vecp}' \neq \emptyset$.
If not then we would have $\vecc(S_{\vecc}' \cup \{i\}) \leq \vecp(S_{\vecp}' \cup \{i\}) \leq B$ which contradicts the premise of the case.
Choose an arbitrary module $i' \in S_{\vecc}' \setminus S_{\vecp}'$.
Note that module $i'$ comes before module $i$ in the bang-per-buck order according to $\vecc$ and if module $i'$ price is in $[\vecc(i'), \vecc(i') + \eps]$ then this ordering remains unchanged.
Further note that $\pi_{\vecc}[\pi_{\vecc}^{-1}(i')-1] \supseteq \pi_{\vecp}[\pi_{\vecp}^{-1}(i') - 1]$ as $\vecp \geq \vecc$ coordinate-wise.
As $i' \in S_{\vecc}'$, we know that $i' \notin \spn(\pi_{\vecc}[\pi_{\vecc}^{-1}(i')-1])$ and thus  $i' \notin \spn(\pi_{\vecp}[\pi_{\vecp}^{-1}(i')-1])$.
This means that $i'$ is matroid-feasible in $\vecp$ when inspected by $\greedy$ but as it was not selected, the budget constraint must have already been violated which must have caused $\greedy$ to terminate. As module $i$ comes after module $i'$, according to both $\pi_{\vecc}$ and $\pi_{\vecp}$, this contradicts that $i \in S_{\vecp}$.
So the premise of the present case is impossible.

\paragraph{Case 2: $\vecc(S_{\vecc} \cap \pi_{\vecc}[\pi_{\vecc}^{-1}(i) - 1]) + \vecc(i) \leq B$.}
In this case, module $i \in \spn(S_{\vecc} \cap \pi_{\vecc}[\pi_{\vecc}^{-1}(i)-1])$.
Let $C$ be the unique circuit in $S_{\vecc} \cap \pi_{\vecc}[\pi_{\vecc}^{-1}(i) - 1] \cup \{i\}$.
Note that module $i$ has the lowest bang-per-buck, according to $\vecc$, in $C$.
Since $i \in S_{\vecp}$ there must be another module $i' \in C$ such that $i' \notin S_{\vecp}$.
We know that $\vecp(i') \leq \vecc(i') + \eps$ and $\frac{\vecv(i')}{\vecp(i')} \geq \frac{\vecv(i')}{\vecc(i') + \eps} > \frac{\vecv(i)}{\vecc(i)}$.
Thus $i'$ comes before $i$ in $\vecp$.
As in the previous case, since $i'$ is independent in $\vecc$ when inspected by $\greedy$, it is also independent in $\vecp$ when inspected by $\greedy$.
Thus, $i'$ must have either been accepted by $\greedy$ or the budget constraint is violated.
In the former case, this contradicts that $i' \notin S_{\vecp}$ while in the latter case, this would contradict that $i \in S_{\vecp}$ since $\greedy$ would have terminated before inspecting module $i$.
In either case, the premise of this case is also impossible.

The two cases above prove that $S_{\vecp} \subseteq S_{\vecc}$.
We now prove the second assertion.
Suppose that the second assertion is false.
This means that there is some $i \in S_{\vecp}$ and some $i' \in S_{\vecc} \setminus S_{\vecp}$ such that $\pi_{\vecc}^{-1}(i') < \pi_{\vecc}^{-1}(i)$.
In other words, module $i'$ has better bang-per-buck than module $i$ (according to $\vecc$) but module $i'$ is not selected by $\greedy$ in $\vecp$ while module $i$ is selected by $\greedy$ in $\vecp$.
Following a similar argument in the above case analysis, it must be that module $i'$ is inspected by $\greedy$ before module $i$ when the price vector is $\vecp$ and $i'$ must be independent of the set selected by $\greedy$ thus far.
Thus if $i'$ is not selected then it must be that the budget constraint would have been violated which further implies that $i$ could not have been selected.
This would contradict that $i \in S_{\vecp}$.
This proves the second assertion.

Finally, we prove the last assertion.
If $S_{\vecp} = S_{\vecc}$ there is nothing to prove.
So let $i$ denote the module in the second assertion of the lemma and let $j = \argmax_{j' \in S_{\vecc} \setminus S_{\vecp}} \{\vecv(j') / \vecc(j')\}$; this is the next module in the bang-per-buck order according to $\vecc$.
By following a similar argument in the proof of the first assertion (particularly, for case 1),
since $j \notin \spn(S_{\vecc} \cap \pi_\vecc[\pi_\vecc^{-1}(i)])$, we also have $j \notin \spn(S_{\vecp} \cap \pi_\vecp[\pi_\vecp^{-1}(i)])$.
Thus, the reason that $j$ is not accepted by $\greedy$ in $\vecp$ is because the budget constraint is violated if module $j$ is taken.
Since $\vecp(j) \leq \vecc(j) + \eps \leq \lambda B + \eps$, we have $\vecp(S_{\vecp}) \geq B - \vecp(j) \geq B - \lambda B - \eps$.
Note also that $\frac{\vecv(j')}{\vecp(j')} \geq \frac{\vecv(j)}{\vecc(j) + \eps}$ for all $j' \in S_{\vecp}$ by definition of $\greedy$.
We conclude that
\begin{align*}
    \vecv(S_{\vecp})
    & \geq \frac{\vecv(j)}{\vecc(j) + \eps} \cdot (B - \lambda B - \eps) \\
    & = \frac{\vecv(j)}{\vecc(j)} \cdot \frac{\vecc(j)}{\vecc(j)+\eps} \cdot (B - \lambda B - \eps) \\
    & \geq \frac{\vecv(j)}{\vecc(j)} \cdot \frac{\lambda B}{\lambda B+\eps} \cdot (B - \lambda B - \eps).
\end{align*}
Next, we bound $\vecv(S_{\vecc} \setminus S_{\vecp})$.
Note that the highest bang-per-buck module (according to $\vecc$) in $S_{\vecc} \setminus S_{\vecp}$ is module $j$.
This gives the trivial bound that
\[
    \vecv(S_{\vecc} \setminus S_{\vecp})
    \leq B \cdot \frac{\vecv(j)}{\vecc(j)}
    \leq \frac{B}{B-\lambda B-\eps} \cdot \frac{\lambda B+\eps}{\lambda B} \cdot \vecv(S_{\vecp}).
\]
We conclude that
\[
    \vecv(S_{\vecc})
    = \vecv(S_{\vecp}) + \vecv(S_{\vecc} \setminus S_{\vecp})
    \leq \left( 1 + \frac{B}{B-\lambda B-\eps} \cdot \frac{\lambda B+\eps}{\lambda B} \right) \vecv(S_{\vecp}). \qedhere
\]
\end{proof}

For the remainder of this section, we assume that $\vecv(i) = 1$ for all modules $i$.
\begin{lemma}
Suppose that $\vecv(i) = 1$ for all modules $i$.
Further, suppose that $\vecc(i) / B \leq \lambda$ for all $i$.
Let $S_{\vecc}$ be the set selected by $\greedy$ when the costs are $\vecc$ and let $S_{\OPT} \in \argmax \{ |S| \,:\, \vecc(S) \leq B, S \in \I \}$. Then $|S_{\OPT}| \leq \left( 1 + \frac{\lambda}{1-\lambda} \right) |S_{\vecc}|$.
\end{lemma}
\begin{proof}
If $\greedy$ spends strictly less than $(1-\lambda) B$ then it must be that $|S_{\vecc}| = |S_{\OPT}|$ otherwise there was an independent module that could have been added, as the cost of all modules is strictly less than $\lambda B$.

So now suppose that $\vecc(S_{\vecc}) \geq (1-\lambda) B$.
Suppose that modules are sorted such that $\vecc(1) \leq \ldots \leq \vecc(n)$ and let $k = |S_{\vecc}|$.
We thus have that $k \geq \frac{(1-\lambda)B}{\vecc(k)}$.
Next, an upper bound on $|S_{\OPT}|$ is to take the first $k+1$ modules that form an independent set.
This will violate the budget constraint (otherwise $\greedy$ would have picked it).
Thus, $|S_{\OPT}| - |S_{\vecc}| \leq 1 \leq \frac{\lambda B}{\vecc(k^*)} \leq \frac{\lambda}{1-\lambda} |S_{\vecc}|$.
Rearranging gives the lemma.
\end{proof}

By combining the above two lemmas we have the following.
\begin{theorem}
\label{theorem:unweighted_approx}
Fix an instance $\instance$ where $\vecv(i) = 1$ for all $i$.
Further, suppose that $\vecc(i) / B \leq \lambda$ for all $i$.
Let $\eps > 0$ be sufficiently small, as required by Lemma~\ref{lemma:unweighted_greedy_approx}.
Let $\vecp$ be an $\eps$-equilibrium price, let $S_{\vecp}$ be the set selected by $\greedy$ when the prices are $\vecp$, and let $S_{\OPT} \in \argmax \{ |S| \,:\, \vecc(S) \leq B, S \in \I \}$. Then $|S_{\OPT}| \leq \left(1 + \frac{\lambda}{1-\lambda} \right)\left( 1 + \frac{1}{1-\lambda - \eps/B}\cdot \frac{\lambda + \eps / B}{\lambda} \right) \cdot |S_{\vecp}|$.
\end{theorem}
In particular, for any fixed instance, as $\eps \to 0$, the approximation ratio tends to $\frac{2-\lambda}{(1-\lambda)^2}$.