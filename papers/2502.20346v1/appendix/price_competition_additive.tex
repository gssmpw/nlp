\section{Technical Assumptions}
\subsection{Necessity of \texorpdfstring{$\eps$}{eps}-equilibrium}
\label{subsec:need_eps_eq}
In this brief sub-section, we argue the necessity of studying $\eps$-equilibria instead of exact equilibrium.
Consider the following example with three modules $A, B, C$.
The values are $v_A = v_B = v_C = 1$ so that the greedy algorithm simply selects modules in order of prices and a tie-breaking rule.
The costs are $c_A = 2, c_B = 3, c_C = 4$ while the budget is $10$.
Suppose that ties are broken in favor of modules $C$ then $A$ then $B$.
We consider a few cases.

\paragraph{Case 1: both $A$ and $B$ bid exactly $4$.}
We consider two smaller subcases.
Suppose first that $C$ also bids $4$.
Then the winners are $C$ and $A$. But this means $B$ can deviate, to $3.5$ for example, to increase their utility.
On the other hand, if $C$ bids strictly more than $4$ then both $A$ and $B$ are selected.
However, both of these modules can slightly increase their bid to ensure they remain selected while being paid slightly more.
So there is no equilibrium where both $A$ and $B$ bid exactly $4$.

\paragraph{Case 2: at least one of $A$ and $B$ bid strictly more than $4$.}
In this case, any equilibrium must include $C$ as a winner because if $C$ is not selected then it could always choose some bid strictly above $4$ to be selected and thus obtain positive utility.
In general (not just in this case), any equilibrium must also include $A$ and at a price of at least $4-\eps$ because if not, it could just bid $4-\eps$ for any $\eps > 0$.
Since $C$ can never bid below $4$, the budget utilization when $A$ is being considered is at most $4-\eps$.
Thus, module $B$ must be rejected but this is not an equilibrium since $B$ can deviate to $3.5$ to ensure it is selected at strictly above its price.

\paragraph{Case 3: at least one of $A$ and $B$ bids strictly less than $4$.}
If either $A$ or $B$ bids strictly less than $4$ then it is certainly accepted since $C$ cannot bid less than $4$.
But if they do bid strictly less than $4$ then they can slightly raise their bid to gain slightly more utility.

\subsection{Necessity for a lower bound on cost}
\label{subsec:need_cost_lb}
Here, we show that if we analyze $\eps$-equilibria then it is necessary to assume a lower bound on the cost of each module.
Given any fixed $\eps > 0$, consider the following example where the budget is $1$.
There are $n+1$ modules all with cost $0$.
Module $1$ has value $1$ while all other modules have value $\eps / 2$.
The optimal solution is to take all modules for a total value of $1 + n\eps / 2$.
On the other hand, an $\eps$-equilibrium is for all modules to set a price of $1$.
Module $1$ is taken because it has the highest bang-per-buck.
For all the other modules, they need to set a bid of at most $\eps / 2$ in order to be selected.
Thus, they cannot deviate to gain more than $\eps / 2$ utility.
Note that this equilibrium receives only value $1$.
As $n \to \infty$, the efficiency of this $\eps$-equilibrium becomes arbitrarily worse.

\section{Matroid Theory Preliminaries} \label{appendix:matroid_prelims}
A \emph{matroid} $\M=(E,\I)$ is a structure with elements $E$ and a family of independent sets $\I \subseteq 2^E$ satisfying the three \emph{matroid axioms}: (i) $\emptyset \in \I$, (ii) if $A \subseteq B$ and $B \in \I$ then $A \in \I$, and (iii) if $A, B \in \I$ and $|A| < |B|$ then there exists $x \in B \setminus A$ such that $A \cup \{x\} \in \I$.
A \emph{weighted matroid} incorporates a matroid $\M=(E, \I)$ with weights $w \in \R^E$ for its elements.

The rank function of a matroid $\M=(E, \I)$ is denoted by $\rank^\M$, where $\rank^\M(S) \coloneqq \max\{|T| : T\subseteq S, T \in \I\}$. The weighted version of the rank function $\rank_w^\M$ is defined for weighted matroids $(\M,w)$ as $\rank^\M_w(S) = \max\{w(T): T\subseteq S, T \in \I\}$. The span function of a matroid $\M$ is defined as $\spn^\M(S) \coloneqq \{e \in E: \rank^\M(S \cup \{e\})=\rank^\M(S)\}$.

Our proofs will make use of the following basic fact that for a matroid if $S, A$ are sets and $x$ is an element such that (i) $S \in \I$, (ii) for every $a \in A$, $S \cup \{a\} \notin \I$, and (iii) $x \not in \spn(S)$, then $x \notin \spn(S)$.

\begin{claim}
\label{claim:matroid_first_k_opt}
For any $G_k$ be the set of selected elements by Algorithm~\ref{alg:greedy} that satisfies $\vec p(G'_k)<B$ then $G_k$ is maximum weighted independent set in matroid $\mathcal I$ restricted on $\pi[k]$.  
\end{claim}
\begin{proof}
We prove this claim via induction on $k$. The claim trivially holds for $k=1$. Suppose, the claim holds until iteration $k-1$. In this case, we observe that $G^{k-1}$ is a full rank set in matroid $\mathcal I$ restricted on $\pi[k-1]$. If $\pi(k)\notin \spn(G^{k-1})$ then maximum weight independent set in $\pi[k]$ is $G^{k-1} \cup \pi[k]$ which is precisely the set $G_{k+1}$. 

In other case when $\pi(k)\in \spn(G^{k-1})$ then due to \cite[Proposition 1.1.6]{oxley2006matroid}, $\pi(k)$ forms a unique circuit $C$ with $G^{k-1}$. Now, we consider the classic greedy algorithm on matroid that orders elements in the decreasing order of weights and selects an element if it is not spanned by higher weight element. We note that minimum weight element $i$ on circuit $C$ will not be the part of optimal set as it is spanned by the set of higher weight elements. If $\pi(k)$ is the minimum weight element on $C$ then $\pi(k)$ can not be a part of the optimal solution and the current set $G^{k-1}$ remains optimal. In the other case, the optimal greedy algorithm selects the same set of elements in $G^k$ as it selected in $G^{k-1}$ whose weight is higher than weight of $\pi(k)$. Therefore, while making decision about $\pi^k$, the optimal algorithm selects $\pi(k)$ as it does not form a cycle with $G^{k-1}$ on which $\pi(k)$ has the lowest weight.
\end{proof}

\section{Missing Proofs From Section~\ref{sec:additive_price_equilibrium}}
\label{app:proofs_additive}
\subsection{Proof of Lemma~\ref{lem:additive_eps_eq_deviation}} \label{proof_eps_additive}
Since we want to bound the result of the worst $\eps$-equilibrium, we assume that $\vecp(i) \leq \vecc(i) + \eps$ for any module $i$ that is not accepted (by Lemma~\ref{lemma:worst_equil}).

Fix any $\eps$-equilibrium $\vecp$ and let $k$ denote the first module in the bang-per-buck order, according to $\vecp$, that was not selected by $\greedy$.
Since it was not selected by $\greedy$, it must be that $\greedy$ had spent $B - \vecp(k)$ up to this point.
So in $\vecp$, $\greedy$ obtained value $\vecv(S_{\vecp}) \geq \frac{\vecv(k)}{\vecp(k)} \cdot (B - \vecp(k))$.

Now, let $k'$ denote the module with the largest bang-per-buck, according to $\vecc$, that was not selected by $\greedy$ in the equilibrium with respect to $\vecp$.
The difference in value between $\greedy$ that knows the cost and $\greedy$ under the equilibrium is upper bounded by $\frac{\vecv(k')}{\vecc(k')} \cdot B$.

We have $B - \vecp(k) \geq B - M$ and
\[
    \frac{\vecv(k)}{\vecp(k)}
    \geq \frac{\vecv(k')}{\vecp(k')}
    \geq \frac{\vecv(k')}{\vecc(k')+\eps}
    \geq \frac{\vecv(k')}{(1+\eps / m) \cdot \vecc(k')}.
\]
In other words, $\vecv(S_{\vecp}) \geq \frac{\vecv(k')}{(1+\eps / m)\vecc(k')} \cdot \frac{B - M}{B} \cdot B$
Thus,
\[
    \vecv(S_{\vecc}) - \vecv(S_{\vecp})
    \leq \frac{\vecv(k')}{\vecc(k')} \cdot B
    \leq (1+\eps / m) \cdot \frac{B}{B - M} \cdot \vecv(S_{\vecp}).
\]
Rearranging, we conclude that $\vecv(S_{\vecc}) \leq \left[ 2 + \frac{\eps}{m} + \left( 1 + \frac{\eps}{m} \right) \cdot \frac{M}{B-M} \right] \cdot \vecv(S_{\vecp})$.
Observing that $\frac{M}{B-M} = \frac{\lambda}{1-\lambda}$ gives the result.



\subsection{Proof of Lemma~\ref{lemma:worst_equil}}
\label{app:proof_worst_equil}
\begin{proof}[Proof of Lemma~\ref{lemma:worst_equil}]
Let $i$ be as in the statement of the lemma and suppose that $\vec{p}(i) > \vec{c}(i) + \eps$ (otherwise the claim is trivial).
We consider a two-step process.
First, we modify the price of $i$ so that $\vec{p}'(i) = \vec{c}(i) + \eps$.
Next, define the price vector
\begin{equation}
    \label{eqn:vecp_def_additive}
    \bar{\vec{p}}(j) = \begin{cases}
        \max\left\{\vecc(j), \frac{\vecc(i) + \eps}{\vec{v}(i)} \cdot \vec{v}(j) - \frac{\eps}{2} \right\} & j \in S_{\vec p} \setminus S_{\vecp'} \\
        \vec{p}'(j) & \text{otherwise}
    \end{cases}.
\end{equation}

We first check the last three assertions in the lemma.
The first assertion is true by construction.
The third assertion follows from the following claim.
The second assertion follows from Claim~\ref{claim:Sbarvecp_subset_Svecp}.
\begin{claim}
\label{claim:barvecp_leq_vecp}
We have $\bar{\vecp} \leq \vecp' \leq \vecp$.
\end{claim}
\begin{proof}
The fact that $\vecp' \leq \vecp$ is trivial since the only difference between $\vecp'$ and $\vecp$ is to lower module $i$'s price to $\vecc(i) + \eps$.
We now need to show that for $j \in S_{\vecp} \setminus S_{\vecp'}$, we have $\bar{\vecp}(j) \leq \vecp(j)$.
Clearly, $\vecc(j) \leq \vecp(j)$ since no module sets a price below their cost.
It remains to show that $\frac{\vecc(i) + \eps}{\vecv(i)} \cdot \vecv(j) - \frac{\eps}{2} \leq \vecp(j)$.
Let $A_j$ be the set of modules that $\greedy$ considered before $j$ when the price vector is $\vecp$.
We have that $\vecp(A_j) + \vecp(j) \leq B$ because $j \in S_{\vecp}$.
Also note that $i \notin A_j$ otherwise $i \in S_{\vecp}$.
Thus, $\vecp'(A_j) + \vecp'(j) = \vecp(A_j) + \vecp(j) \leq B$.
Since $j \in S_{\vecp} \setminus S_{\vecp'}$ and the only difference is module $i$'s price, it must be that $\greedy$ considers module $i$ before module $j$ when the price vector is $\vecp'$.
This implies that $\frac{\vecp(j)}{\vecc(j)} \geq \frac{\vecc(i)+\eps}{\vecv(i)}$, as desired.
\end{proof}

\begin{claim}
\label{claim:i_notin_Sbarp}
We have that $i \notin S_{\bar{\vecp}}$.
\end{claim}
\begin{proof}
Let $A_i'$ (resp.~$\bar{A}_i$) be the set of modules that $\greedy$ considers before module $i$ when the price vector is $\vecp'$ (resp.~$\bar{\vecp}$).
Note that (i) $A_i' \subseteq \bar{A}_i$ and (ii) $\bar{\vecp}(A_i') = \vecp'(A_i')$.
The first assertion follows from Claim~\ref{claim:barvecp_leq_vecp} since module $i$ has the same price in both $\vecp'$ and $\bar{\vecp}$.
The second assertion is because $\greedy$ makes exactly the same decisions for $A_i'$ under $\vecp$ and $\vecp'$.
In particular, $A_i' \cap (S_{\vecp} \setminus S_{\vecp'}) = \emptyset$.
We thus conclude, using the definition of $\bar{\vecp}$ (Eq.~\eqref{eqn:vecp_def_additive}), that $\bar{\vecp}(j) = \vecp'(j) = \vecp(j)$ for all $j \in A_i'$.

The hypothesis of the lemma means that module $i$ cannot modify its price to $\vec{c}(i) + \eps$ and be in the accepted set.
In particular,
\[
    B < \vecp(A_i') + \vecc(i) + \eps = \vecp'(A_i') + \vecp'(i) = \bar{\vecp}(A_i') + \bar{\vecp}(i)
    \leq \bar{\vecp}(\bar{A}_i) + \bar{\vecp}(i),
\]
where the first equality is because $\vecp(j) = \vecp'(j)$ for $j \neq i$,
the second equality is by (ii) above,
and the last equality is by (i) above.
We conclude that $i \notin S_{\bar{\vecp}}$ since including it when it is considered by $\greedy$ would violate the budget constraint.
\end{proof}

\begin{claim}
\label{claim:Sbarvecp_subset_Svecp}
$S_{\bar{\vecp}} \subseteq S_{\vecp}$.
\end{claim}
\begin{proof}
We will prove the contra-positive: if $j \notin S_{\vecp}$ then $j \notin S_{\bar{\vecp}}$.
If $j = i$ then this follows from Claim~\ref{claim:i_notin_Sbarp}.
So now assume $j \neq i$.
Let $A_j'$ (resp.~$\bar{A}_j$) be the set of modules that $\greedy$ considers before module $j$ when the price vector is $\vecp'$ (resp.~$\bar{\vecp}$).
Note that the ordering, according to $\greedy$, between $j$ and $i$ is the same in both the price vectors $\vecp'$ and $\bar{\vecp}$ since both of their prices are unchanged between the two price vectors.
If $i \in A_j'$ (and hence $i \in \bar{A}_j$) then $j \notin S_{\bar{\vecp}}$) since $\greedy$ terminates after considering $i$.
Henceforth, we assume that $j \notin A_j'$.

In this case, the proof is very similar to Claim~\ref{claim:i_notin_Sbarp}.
We again have that (i) $A_j' \subseteq \bar{A}_j$ and (ii) $\bar{\vecp}(A_j') = \vecp'(A_j')$.
The first assertion follows from Claim~\ref{claim:barvecp_leq_vecp} since module $j$ has the same price in both $\vecp'$ and $\bar{\vecp}$.
The second assertion is because $\greedy$ makes exactly the same decisions for $A_j'$ under $\vecp$ and $\vecp'$.
In particular, $A_j' \cap (S_{\vecp} \setminus S_{\vecp'}) = \emptyset$.

Since $j \notin S_{\vecp}$, we have that
\[
    B < \vecp(A_j') + \vecp(j) = \vecp'(A_j') + \vecp'(j) = \bar{\vecp}(A_j') + \bar{\vecp}(j)
    \leq \bar{\vecp}(\bar{A}_j) + \bar{\vecp}(j),
\]
where the first equality is because $\vecp(j) = \vecp'(j)$ for $j \neq i$,
the second equality is by (ii) above and Eq.~\eqref{eqn:vecp_def_additive},
and the last equality is by (i) above.
We conclude that $j \notin S_{\bar{\vecp}}$ since including it when it is considered by $\greedy$ would violate the budget constraint.
\end{proof}

Finally, we check that $\bar{\vecp}$ is still an $\eps$-equilibrium which follows from the following case analysis.
\paragraph{Case 1: $j \in S_{\vec p} \cap S_{\vec p'}$.}
In this case, we have $\vec p(j) = \bar{\vecp}(j)$.
Since $\vecp$ is an $\eps$-equilibrium price, we have that module $j$ is rejected in the price vector $\vec{q} = (\vecp(j) + \eps, \vecp_{-j})$.
Hence, $j$ is also rejected in the price vector $\vec{q}' = (\vecp(j) + \eps, \vecp'_{-j})$ since all modules that came before $j$ in $\vec{q}$ also come before $j$ in $\vec{q}'$ (module $i$ is the only module with a different price between $\vec{q}$ and $\vec{q}'$).
Finally, consider $\bar{\vec{q}} = (\vecp(j) + \eps, \bar{\vecp}_{-j})$.
If $j$ comes before $i$ according to $\greedy$ in $\bar{\vec{q}}$ then $\greedy$ coincides on all modules up to and including module $j$ on $\bar{\vec{q}}$ and $\vec{q}'$.
So $j$ is rejected.
On the other hand, if $j$ comes after $i$ according to $\greedy$ then $j$ must be rejected.
This is because (i) the set of modules inspected by $\greedy$ up to and including $i$ in $\bar{\vecp}$ is a subset of the module inspected by $\greedy$ up to and including $j$ in $\bar{\vec{q}}$ and (ii) $\bar{\vecp} \leq \bar{\vec{q}}$.
We conclude that $j$ cannot raise its price by at least $\eps$ and still be included in $S_{\bar{\vecp}}$.

\paragraph{Case 2: $j \in S_{\vec p} \setminus S_{\vec p'}$.}
Since $\bar{\vec{p}}(j) \geq \frac{\vec{c}(i) + \eps}{\vec{v}(i)} - \frac{\eps}{2}$ then module $j$ cannot deviate to gain more than $\eps$ since it would have to increase its bid to at least $\frac{\vec{c}(i) + \eps}{\vec{v}(i)} + \frac{\eps}{2}$ to do so.
However, it would come after module $i$ in the bang-per-buck order and module $i$ is not accepted so neither would module $j$.

\paragraph{Case 3: $j \notin S_{\vec p}$.}
For $j = i$, we established above (Claim~\ref{claim:i_notin_Sbarp}) that $i \notin S_{\bar{\vecp}}$ and any possible deviation gives module $i$ utility strictly less than $\eps$ since $\bar{\vecp}(i) = \vecc(i) + \eps$.

Now suppose $j \neq i$.
Consider the price vector $\vec{q} = \left( \vecc(j) + \eps, \vecp \right)$.
Module $j$ must be rejected because otherwise $\vecp$ is not an $\eps$-equilibrium.
Now consider the price vector $\vec{q}' = \left( \vecc_j, \vecp' \right)$.
In this case, $j$ is still rejected because (i) if $j$ comes before $i$ according to $\greedy$ in $\vec{q}'$ then its outcome under $\greedy$ in $\vec{q}$ is the same as in $\vec{q}'$ or (ii) if $j$ comes after $i$ according to $\greedy$ then $j$ is rejected.
Finally, consider the price vector $\bar{\vec{q}} = (\vecc_j, \bar{\vecp})$.
If $j$ comes before $i$ then there may now be additional modules before $j$ in $\bar{\vec{q}}$ compared to $\vec{q}'$.
So module $j$ remains rejected.
If $j$ comes after $i$ then $j$ is rejected because $i$ is rejected.

Since module $j$ is rejected in $(\vecc(j) + \eps, \bar{\vecp}_{-j})$, it must also be rejected if it modifies its bid to any bid strictly more than $\vecc(j) + \eps$.
On the other hand, any bid lower than $\vecc(j) + \eps$ provides $j$ with strictly less than $\eps$ utility.
So any bid placed by $j$ results in utility strictly less than $\eps$.
\end{proof}

\subsection{Proof of Lemma~\ref{lemma:additive_greedy_approx_with_cost}}
\label{app:additive_greedy_approx_with_cost}
\begin{proof}[Proof of Lemma~\ref{lemma:additive_greedy_approx_with_cost}]
By rearranging, we can assume that $\frac{\vecv(1)}{\vecc(1)} > \ldots > \frac{\vecv(n)}{\vecc(n)}$.
This means that $S_{\vecc} = [k]$ for some $k \in [n]$.
An upper bound on $\vecv(S_{\OPT})$ is to take the first $k + 1$ modules.
Thus
\[
    \vecv(S_{\OPT}) - \vecv(S_{\vecc}) \leq \vecv(k+1) \leq M \cdot \frac{\vecv(k+1)}{\vecc(k+1)}.
\]
Since $k+1 \notin S_{\vecc}$, we have $\vecc([k]) \geq B - \vecc([k+1]) \geq B - M$.
Thus
\[
    \vecv(S_{\vecc}) \geq (B - M) \cdot \frac{\vecv(k+1)}{\vecc(k+1)}
\]
Combining the above two inequalities, we conclude that
\[
    \vecv(S_{\OPT}) \leq \left( 1 + \frac{M}{B - M} \right) \vecv(S_\vecc) = \left( 1 + \frac{\lambda}{1-\lambda} \right) \vecv(S_{\vecc}). \qedhere
\]
\end{proof}