\section{Price Competition for Weighted Buyer with Matroid Constraints}
\label{app:weighted_matroid}
In this section, we consider the pricing game when the platform uses Algorithm~\ref{alg:greedy} for module selection.
\subsection{Proof of Claim~\ref{claim:weighted_invarient}}\label{proof_of_weighted_invariant}
The invariants are trivial if $k = 1$ since $A^{k-1}  = \emptyset$.
We now assume that the invariant holds at iteration $k-1$ and prove that it remains true at iteration $k$.

For the first invariant, the only modules that increase their price is $T^{k-1} \subseteq \pi^{k-1}([k-1])$.
Since their bang-per-buck is at most that of module $\pi^{k-1}([k])$ this means that all modules in $\pi^{k-1}([k-1])$ still come before module $\pi^{k-1}([k])$.
So $\pi^{k-1}([k-1]) = \pi^k([k-1])$.
The fact that $\pi^{k-1}(k) = \pi^{k}(k)$ now easily follows because ties are broken according to $\pi^0$.

The second invariant is because prices of modules in $\pi^k([k-1])$ are only raised to meet the bang-per-buck of module in $\pi^k(k)$.

The third invariant is trivial if Line~\ref{line:budget_check} is false.
Otherwise, it is true because (i) $\vecp^{k-1}(A^{k-1})\leq B$ by the fact that the invariant holds for $k-1$.

At iteration $k$, if $\rank(\pi[k]) = \rank(\pi[k-1]) + 1$ then $\pi(k) \notin \spn(\pi[k-1])$. By inductive hypothesis, since $A^{k-1}$ is a full rank set in the matroid restricted to $\pi[k-1]$, module $\pi(k) \notin \spn(A^{k-1})$. Therefore if condition in Line~\ref{line:if_weighted} is satisfied and algorithm adds module $\pi(k)$ into the set $A^k$. This implies $\rank(A^k) = \rank(\pi[k])$. 
We now consider the case when $\rank(\pi[k]) = \rank(\pi[k-1])$. This implies that  $\pi(k) \in \spn(\pi[k-1])$  By inductive hypothesis, since $A^{k-1}$ is a full rank set in the matroid restricted to $\pi[k-1]$, module $\pi(k) \in \spn(A^{k-1})$. Therefore, $\pi(k)$ forms a unique circuit $C^k$ with $A^{k-1} \cup \pi[k]$. Since, $A^k \gets (A^{k-1} \cup \pi(k)) \setminus \{i\}$ removes a module from a circuit $C^k$, $\rank(A^k) = \rank(A^{k-1}) = \rank(\pi[k])$.  Hence, the forth invariant holds. 

To prove the fifth invariant, due to inductive hypothesis, we have that the module $i$ forms a circuit $C$ with the set of module $A^{k-1}$ where $i$ has the lowest value among the modules in $C$.
At round $k$, if $C\subseteq A^k$ then the invariant holds trivially. Otherwise, it must be the case that some module $j\in C$ was replaced by module $\pi^k(k)$.
This implies that the module $\pi^k(k)$ forms a circuit $C^k$ with $A^{k-1}$ that contains module $j$ and $j$ is the lowest valued module on circuit $C^k$.
We note that $i\notin C^k$.
By circuit axiom of matroids \cite[Lemma~1.1.3]{oxley2006matroid}, we have a circuit $C'\subseteq (C \cup C^k)\setminus j$, therefore, $\rank((C \cup C^k)\setminus j) \leq  |(C \cup C^k)\setminus j| -1$.
Since, fourth invariant implies that $A^k$ is a full rank set and $(C\cup C^k)\setminus \{i,j\} \subseteq A^k$, we have $\rank((C\cup C^k)\setminus \{i,j\}) = |(C\cup C^k)\setminus \{i,j\} |$.
We next claim that $i \in \spn((C\cup C^k)\setminus \{i, j\} )$.
If not then we have, $$\rank((C \cup C^k)\setminus j) = \rank((C \cup C^k)\setminus \{i, j\}) + 1 = |(C \cup C^k)\setminus \{i, j\}| + 1 = |(C \cup C^k)\setminus j|. $$ This contradicts the circuit axiom and therefore, $i \in \spn(C^k \cup C \setminus \{i,j\})$.
We note that $C^k \cup C \setminus \{i,j\} \subseteq A^k$ and all the modules in $C^k \cup C \setminus \{i,j\}$ has value higher than the value of module $i$. This concludes the proof. 

Next, we characterize the selected module at price $\bar {\vec p}$ which is an output of Algorithm~\ref{alg:equilibrium_dynamics_weighted}.
\begin{claim}
    \label{claim:matroid_greedy_weighted_output}
    $S_{\bar{\vecp}} = A^{k^*}$. In addition, $\SE$ is a maximum weight independent set in matroid $\mathcal I$ restricted to $\pi^0(1), \dots , \pi^0(k^*)$. 
\end{claim}
\begin{proof}
    Claim~\ref{claim:weighted_invarient} shows that $S_{\bar{\vecp}} \cap \pi^{k^*}([k^*-1]) = A^{k^*-1} $. The fifth invariant of Claim~\ref{claim:weighted_invarient} implies that for any $k<k^*$, the set of modules $A^k$ forms the maximum weight independent set in the matroid $\mathcal I$ restricted on modules $\pi^0[k]$ \cite[Theorem~1.8.5]{oxley2006matroid}. Now consider the greedy algorithm (Algorithm~\ref{alg:greedy}): due to invariant 2 from Claim~\ref{claim:weighted_invarient}, we observe that it iterates over $\pi^0(1), \dots, \pi^0(k^*-1)$ and selects maximum weighted independent set in the matroid $\mathcal I$ restricted on $\pi[k^*-1]$ if it does not exhaust the budget which is precisely the set $A^{k^* -1 }$. 
    
    We need to show that for any set of selected modules $S^{k'}$ at $k'$-th iteration of the of the  greedy algorithm, $\vecp(S_{k'})< B$. Consider any module $i\in S_{k'}$, we observe that the module $i$ will eventually be swapped by some module $i' \in A^{k}$. Since module $i'$ replaces the module $i$, we have $\vec v(i') > \vec v(i)$. In addition, due to bang-per-buck ordering, we have $\frac{\vec v(i)}{\vec {\bar p}(i)} \geq  \frac{\vec v(i')}{\vec {\bar p} (i')}$. This implies that $\vec {\bar p(i')} \geq \frac{\vec v(i')}{\vec v(i)} \cdot \vec {\bar p}(i) > \vec {\bar p}(i)$.  Due to our assumption $k<k^*$, we have $\vec {\bar p})(S_{k'}) \leq \vec {\bar p(A^k)}\leq B$.
    
    
    Then the greedy algorithm visits module $\pi^0(k^*)$. By definition of $k^*$, either $k^*= n$ and the budget constraint is never violated in Line~\ref{line:budget_check_weighted} or budget check condition (Line~\ref{line:budget_check_weighted}) is not satisfied after updating prices at round $k^*$. In the first case, we can observe that $A^n$ is the maximum weight independent set in $\mathcal I$ and $\vec {\bar p}^n(A^n)\leq B$. Therefore, greedy algorithm selects modules $A^n$. 
    In the second case, $A^{k^*} = A^{k^*-1}$ which is maximum weight independent set in matroid $\mathcal I $ restricted on $\pi[k^*-1]$ with $\vec {\bar p}^{k^*}(A^{k^* }) = \vec {\bar p}^{k^*}(A^{k^* -1 })\leq B$. In addition, we must have either $\pi^0(k^*) \notin \spn(A^{k^*-1})$ or $\pi^0(k^*)$ forms circuit $C^*$ with $A^{k^*-1}$ such that $i \neq \pi^0(k^*)$ is not the minimum weight element on $C^k$. In both cases, $\vec {\bar p}^k(A^*\cup \pi^0(k^*)) > B$ or $\vec {\bar p}^k(A^*\cup \pi^0(k^*) \setminus i) > B$ respectively. Therefore, greedy algorithm does not select $\pi^0(k^*)$ and terminates by outputting $A^{k^*-1} = A^{k^*}$. That concludes the proof.    
\end{proof}




\subsection{Proof of Theorem~\ref{thm:weighted_matroid_eq_exists}}\label{proof_weighted_matroid_exists}
 \begin{proof}
     Let $\barp$ be the price computed by Algorithm~\ref{alg:equilibrium_dynamics_weighted} $\optbpb$ be the bang-per-buck of modules selected modules $\SE: A^{k^*}$ by greedy algorithm (Algorithm~\ref{alg:greedy}). Since, Algorithm~\ref{alg:equilibrium_dynamics_weighted} selected maximum weighted set in matroid $\I$ restricted to $\pi^0(1), \dots , \pi^0(k^*-1)$, we claim that no module $\pi^0(k')\notin \SE$ with $k'< k^*$ can deviate their price and get selected. For the sake of contradiction, let module $\pi^0(k')$ updates its price to $p'$ and gets selected by  greedy algorithm. We note that $\pi^0(k')$ forms a circuit $C$ with $\SE$ where $i$ has the lowest value. This implies that the greedy algorithm must have ran out of budget when it reaches to the module  with lowest bang-per-buck (in this case, module $i\in \SE\setminus \{i\}$ with lowest $\frac{\vec v(i)}{\vec c(i)}$) say module $i\in C$. Since $\frac{\vec v(\pi^0(k'))}{p'}\geq  \frac{\vec v(i)}{\barp(i)}$ and $\vec v(i) > \vec v(\pi^0(k'))$. This implies that $\barp(i) > p'$. Since $\barp(\SE)\leq B$ (invariant 3 in Claim~\ref{claim:weighted_invarient}), we conclude that such deviation for module $\pi^0(k')$ can not exist. 
     
     On the other hand, for all modules $i\notin \SE$ with $\frac{\vec v(i)}{\barp(i)}< \optbpb$, we have $\barp(i) = \vec c(i)$ due to Claim~\ref{claim:weighted_invarient}. Therefore, such modules can not deviate and get selected. Finally, to conclude the proof, we need to show that no module in $\SE$ can increase their price and get selected. We first consider the case where $\barp(\SE) = B$. In this case, if any module $i\in \SE$ increases its price then it will not be selected by greedy algorithm. On the other hand, if $\barp(\SE)< B$ then $\barp(\opt(\SE \cup k^*))>B$. Then if any module in $i\in \SE$ increases its price then greedy algorithm selects $\pi^0(k^*)$ and exhausts its budget when it reaches to module $i$. Therefore, no module in $\SE$ can increase its price and improve its utility.  Therefore, we conclude that $\barp$ is an equilibrium price. 
 \end{proof}
 
 
 
 
\subsection{Proof of Lemma~\ref{lemma:unselected_lt_k_bid_cost}}\label{proof_of_lemma_poa_high_bpb}
In this section, we assume that $\eps > 0$ is sufficiently small, i.e.~$\eps < \min_i \vecc(i)$.
We first start with a couple easy claims.

\begin{claim}
\label{claim:matroid_identical_bang_per_buck}
Fix $\eps < \min_i \vecc(i)$.
Let $\vecp$ be an $\eps$-equilibrium.
For all $i, j \in S_{\vecp}$, we have $\frac{\vecv(i)}{\vecp(i)+\eps} \leq \frac{\vecv(j)}{\vecp(j)}$.
\end{claim}
\begin{proof}
    Suppose that there exists $i, j \in S_{\vecp}$ such that $\frac{\vecv(i)}{\vecp(i)+\eps} > \frac{\vecv(j)}{\vecp(j)}$.
    In this case, suppose that module $i$ increases its price to $\vecp(i) + \eps$.
    As the prices have only increased with $i$'s deviation, $\greedy$ inspects a subset (not necessarily strict) of the modules it inspected when the prices are $\vecp$.
    Since $\eps < \vecc(j)$, $j \in S_{\vecp}$, and module $i$ still comes before module $j$ in the bang-per-buck ordering after deviating, it means that module $i$ is budget-feasible.
    In particular, module $i$ is still inspected and accepted by $\greedy$.
    This contradicts that $\vecp$ is an $\eps$-equilibrium.
\end{proof}

\begin{claim}
\label{claim:eq_price_budget_or_bpb_tight}
Let $\vecp$ be an $\eps$-equilibrium and suppose $\greedy$ selects $k^*$ modules.
Then at least one of the following is true. Either $\vecp(S_\vecp) \geq B - \eps$ or for all $i \in S_{\vecp}$, we have $\frac{\vecv(i)}{\vecp(i)+\eps} \leq \frac{\vecv(k^*+1)}{\vecc(k^*+1)}$.
\end{claim}
\begin{proof}
    If neither conclusion is true then a selected module could raise its price by $\eps$ to be in front of $k^*+1$ but within the budget constraint.
    So at least one of the conclusions is true.
\end{proof}

\begin{proof}[Proof of Lemma~\ref{lemma:unselected_lt_k_bid_cost}]
    Let $i \notin [k^*] \setminus S_{\vecp}$ such that $\vecp(i) > \vecc(i) + \eps$.
    We set $\bar{\vecp}(i) = \vecc(i) + \eps$ and $\bar{\vecp}(j) = \vecp(j)$ for $j \neq i$.
    
    We now check that $\greedy$ selects the same set of modules in $\vecp$ and $\bar{\vecp}$.
    If $i$ is never selected in $\bar{\vecp}$ then it must never be selected in $\vecp$ (since its price is higher). Thus, for all other modules, $\greedy$ behaves identically in $\vecp$ and $\bar{\vecp}$.
    If $i$ is selected then it must be rejected later.
    At this point, Claim~\ref{claim:matroid_first_k_opt} shows that $\greedy$ under both $\vecp$ and $\bar{\vecp}$ must be identical.
    So thereafter, $\greedy$ is identical under both $\vecp$ and $\bar{\vecp}$.

    We need to check that $\bar{\vecp}$ is an $\eps$-equilibrium price for all $j \neq i$.
    Suppose $j$ is not selected but it can reduce its price to a point where it is budget-feasible, considered by $\greedy$, and its price exceeds its cost by $\eps$.
    Let $j' \in S_{\vecp}$ be the module that would remove $j$ if $j$'s price is $\vecc(j) + \eps$ in the price vector $\vecp$.
    If $j$ chooses a deviation to come after $j'$ then $j$ cannot be selected since $j'$ is chosen.
    Suppose that $j$ chooses a deviation to come before $j'$.
    Let $A_{j'}$ be the modules coming before $j'$, including $j'$ in $\bar{\vecp}$.
    Since the total spend by $\greedy$ is monotone increasing in each iteration and since $S_{\bar{\vecp}} \cap A_{j'}$ is budget-feasible, it must be that $\greedy$ reaches $j'$.
    Thus $j$ would be removed in this case as well.
    
    If $j \leq k^*$ and is selected then it cannot increase its price to gain more than $\eps$ utility since the solution in $\vecp$ and $\bar{\vecp}$ are identical.
    In other words, a profitable deviation in $\bar{\vecp}$ would also be a profitable deviation in $\vecp$.
\end{proof}


\subsection{Proof of Lemma~\ref{lemma:unselected_gt_k_to_cost}}\label{proof_of_key_lemma_poa}
\begin{proof}
This proof is a case analysis.
\paragraph{Case 1: $\frac{\vecv(q)}{\vecc(q) + \eps} \leq \frac{\vecv(k^*+1)}{\vecp(k^*+1)}$.}
Since $\vecp$ is an $\eps$-equilibrium, this means that module $q$ is not accepted at price $\vecc(q) + \eps$.
Let $\bar{\vecp}$ be the new price vector with $q$'s deviation to $\vecc(q) + \eps$.
Any module $q' \neq q$ coming before $k^*+1$ is unaffected and any module that comes after $k^* + 1$ cannot deviate.
To see this, if it could have deviated in $\bar{\vecp}$ then it could also have deviated in $\vecp$ since the only difference between $\bar{\vecp}$ and $\vecp$ is that in $\bar{\vecp}$, module $q$ may come in front of module $q'$.

\paragraph{Case 2: $\min_{i \in S_{\vecp}} \frac{\vecv(i)}{\vecp(i) + \eps} \geq \frac{\vecv(q)}{\vecc(q) + \eps} > \frac{\vecv(k^*+1)}{\vecp(k^*+1)}$.}
In this case, we have $\vecp(S_{\vecp}) \geq B - \eps$ by Claim~\ref{claim:eq_price_budget_or_bpb_tight}.
We claim that changing the price of module $q$ to $\vecc(q)+\eps$ gives an $\eps$-equilibrium.
No rejected module $i < k^*$ can deviate since they are already within $\eps$ of their cost.
No accepted module can increase its price by $\eps$ since it would no longer be selected as the budget constraint would be violated.
Finally, no rejected module $i > k^*$ can deviate since it would have to come in front of $k^*$ and if it was selected then that would have been a profitable deviation in $\vecp$ as well.

The last case is the most technically challenging.
\paragraph{Case 3: $\frac{\vecv(q)}{\vecc(q) + \eps} > \min_{i \in S_{\vecp}} \frac{\vecv(i)}{\vecp(i) + \eps}$.}
First, let us set $\bar{\vecp}(q) = \vecc(q) + \eps$. We consider a few subcases.
\paragraph{Case 3a: When running $\greedy$ with $q$'s deviation, $q$ could have been added but the budget was exceeded when inspecting $q$.}
In this case, we first define a temporary price $\vecp'$ by setting $\vecp'(i) = \max\left\{ \vecc(i), \frac{\vecc(q) + \eps(1-\delta)}{\vecv(q)} \cdot \vecv(i) \right\}$ for all previously accepted but now rejected modules $i$, i.e.~the new bang-per-buck of $i$ is equal to $q$ if $i$ can accept that price.
Here, $\delta \in (0, 1)$ will be a small parameter that we choose below.
Note that these modules come before $q$ in the bang-per-buck ordering.
Note that $q$ is still not accepted since now there are more modules in front of $q$.
However, it could be that some previously rejected modules are now accepted since some modules that were previously accepted cannot lower their price set by $q$.
For these modules, we raise their price so that their bang-per-buck is also $\frac{\vecv(q)}{\vecc(q) + \eps(1-\delta)}$ or their current price, whichever is greater.
We let $\bar{\vecp}$ be the price given by the three transformations discussed above (and set $\bar{\vecp}(i) = \vecp(i)$ for any module $i$ that was not discussed).

We claim that the maximum value independent set of all modules coming before $q$ is budget-feasible with the price vector $\bar{\vecp}$.
To see this, first note that the maximum value independent set has decreased in value from $\vecp$ to $\bar{\vecp}$.
Next, observe that any module that was accepted by $\greedy$ in both $\vecp$ and $\bar{\vecp}$ has its bang-per-buck either stay the same or it has increased.
On the other hand, any module that was accepted by $\greedy$ in $\vecp$ but not in $\bar{\vecp}$ may have been replaced by a module with better bang-per-buck.
We conclude that the maximum value independent set of all modules coming before $q$ must be budget-feasible with the price vector $\bar{\vecp}$.

Finally, we show that $\bar{\vecp}$ is indeed an $\eps$-equilibrium.
We require the following claim.
\begin{claim}
    \label{claim:Suq_contains_q}
    Let $S_q$ be the set selected by $\greedy$ with price vector $\bar{\vecp}$ just before $\greedy$ inspects $q$.
    If $\delta > 0$ is sufficiently small then the maximum value independent set in $S_q \cup \{q\}$ contains $q$.
\end{claim}
\begin{proof}
Let $A_q$ be the set of modules that $\greedy$ selects with only $q$'s deviation.
The premise of the case means that $A_q \cup \{q\}$ is independent but $\vecp(A_q) + \vecc(q) + \eps > B$.
Now suppose that the maximum value independent set in $S_q \cup \{q\}$ does not contain $q$.
We will show that $S_q$ must not be budget-feasible which would be a contradiction.
To do this, we will show that for sufficiently small $\delta > 0$, we have $\bar{\vecp}(S_q) - (\vecp(A_q) + \vecc(q) + \eps) > -\eta$ where $\eta = B - (\vecp(A_q) + \vecc(q) + \eps)$.
Indeed, we have
\begin{align*}
\bar{\vecp}(S_q) - (\vecp(A_q) + \vecc(q) + \eps)
& = \bar{\vecp}(S_q \cap A_q) + \bar{\vecp}(S_q \setminus A_q) - (\vecp(A_q) + \vecc(q) + \eps) \\
& \geq \vecp(S_q \cap A_q) + \bar{\vecp}(S_q \setminus A_q) - (\vecp(A_q) + \vecc(q) + \eps) \\
& = \bar{\vecp}(S_q \setminus A_q) - \vecp(A_q \setminus S_q) - \vecc(q) - \eps \\
& \geq \frac{\vecc(q) + \eps(1-\delta)}{\vecv(q)} \cdot \vecv(S_q \setminus A_q) - \frac{\vecc(q) + \eps}{\vecv(q)}{\vecv(q)} \cdot (\vecv(A_q \setminus S_q) + \vecv(q)) \\
& =  \frac{\vecc(q) + \eps}{\vecv(q)} \cdot \left( \vecv(S_q \setminus A_q) - \vecv(A_q \setminus S_q) - \vecv(q) \right) - \frac{\eps \delta \vecv(S_q \setminus A_q)}{\vecv(q)} \\
& = \frac{\vecc(q) + \eps}{\vecv(q)} \cdot \left( \vecv(S_q) - \vecv(A_q) - \vecv(q) \right) - \frac{\eps \delta \vecv(S_q \setminus A_q)}{\vecv(q)}.
\end{align*}
The first inequality is because modules in $S_q \cap A_q$ may have had their price increased from $\vecp$ to $\bar{\vecp}$.
The second inequality is because (i) any module in $S_q \setminus A_q$ is because the module had to have its price reduced after $q$'s deviation and (ii) all modules $i$ in $A_q$ (and thus in $A_q \setminus S_q$) have bang-per-buck at least $\frac{\vecv(q)}{\vecv(q) + \eps}$.
The last equality is obtained by adding and subtracting $\vecv(S_q \cap A_q)$.
Note that $\vecv(S_q) \geq \vecv(A_q) + \vecv(q)$ because $S_q$ is the maximum value independent set of all modules before and including $q$ in $\bar{\vecp}$ while $A_q \cup \{q\}$ is the maximum value independent set of all modules before and including $q$ in $\vecp$ with only $q$'s deviation (which is contained in the set of modules before and including $q$ in $\bar{\vecp}$).
We thus have
\[
\bar{\vecp}(S_q) - (\vecp(A_q) + \vecc(q) + \eps)
\geq -\frac{\eps \delta \vecv(S_q \setminus A_q)}{\vecv(q)}.
\]
A trivial upper bound on $\vecv(S_q \setminus A_q)$ is $\vecv([n])$ so taking $\delta < \frac{\eta \vecv(q)}{\eps \vecv([n])}$ is sufficient for the claim.
\end{proof}
Consider any module $i$ that is accepted.
If module $i$ deviates by more than $\eps$ then Claim~\ref{claim:Suq_contains_q} shows that either $q$ is accepted which would cause $i$ to be rejected or $\greedy$ terminates at $q$ which also means $i$ is rejected.

Now consider any module $i$ is not selected under $\bar{\vecp}$.
If $i < k^*$ was rejected in $\vecp$ and still rejected in $\bar{\vecp}$ then $i$ is bidding within $\eps$ of its cost.
If $i > k^*$ was rejected in $\vecp$ then any deviation in $\bar{\vecp}$ results in more modules in front of $i$ than in $\vecp$.
So $i$ also cannot deviate.

Finally, note that some modules $i < k^*$ which were rejected had their price increased above and then later rejected in $\bar{\vecp}$.
For these modules, we use Lemma~\ref{lemma:unselected_lt_k_bid_cost} to bring their bid back down to their cost.

\paragraph{Case 3b: When running $\greedy$ with $q$'s deviation, module $q$ is budget-feasible when $\greedy$ inspects it but is never accepted, even temporarily.}
This case is straightforward since $\greedy$ behaves identically in before and after $q$'s deviation.

\paragraph{Case 3c: When running $\greedy$ with $q$'s deviation, module $q$ is budget-feasible when $\greedy$ inspects it and is accepted but later rejected.}
In this case, we claim only $q$'s deviation is an $\eps$-equilibrium price.
First, we check that $S_{\vecp}$ remains budget-feasible.
Let $A$ be the set accepted by $\greedy$ up to but before $k^*$.
In this case, $\vecv(A) \leq \vecv(S_{\vecp})$ because, by Claim~\ref{claim:matroid_first_k_opt}, $A$ (resp.~$S_{\vecp}$) is the maximum value independent set in all the modules before $k^*$ excluding (resp.~including) $k^*$.
Moreover, we know that the price paid by $\greedy$ is always monotone increasing and $S_{\vecp}$ is budget-feasible.

Finally, we check that this is an $\eps$-equilibrium.
No accepted module can deviate because either the budget constraint is tight or its deviation would cause it to come after a module which violates the budget constraint.
Any rejected module $i < k^*$ is already bidding its cost so it cannot deviate.
Finally, let $i > k^*$ be rejected and suppose it deviates to its cost plus $\eps$.
We claim it must still be rejected.
If $\greedy$ never accepts $i$ then we are done.
Otherwise, suppose $\greedy$ does accept $i$ in $\bar{\vecp}$.
Then is must be in $\spn(S_{\vecp})$.
It also must be the case that in $\vecp$, $\greedy$ accepted $i$ but later rejected it.
Let $j$ be the last module inspected by $\greedy$ in the circuit formed by $S_{\vecp} \cup \{i\}$.
Then $j$ must be budget-feasible because $S_{\vecp}$ is and the set selected by $\greedy$ until $j$ is inspected is at most $\vecv(S_{\vecp})$.
So taking $j$ will reject $i$.
\end{proof}

\section{Proof of Theorem~\ref{thm:weighted_matroid_approx}}\label{proof_of_quality_matroid}
\begin{lemma}
\label{lemma:vecp_most_of_budget}
Let $\vecp$ be an $\eps$-equilibrium and let $k^*$ be the last module selected by $\greedy$.
Suppose that $\vecp(i) \leq \vecc(i) + \eps$ for all $i \notin S_{\vecp}$ and $i \leq k^* + 1$.
Then $\vecp(S_{\vecp}) \geq (1 - \lambda) B - \eps$ where $\lambda > \max_i \frac{\vecc(i)}{B}$.
\end{lemma}
\begin{proof}
    We assume that the maximum value set in $[k^* + 1]$ contains $k^* + 1$.
    If not and $\vecp(S_{\vecp}) < B - \eps$ then $\vecp$ is not an $\eps$-equilibrium since accepted modules can still raise their price past the bang-per-buck of $k^* + 1$.

    Let $p$ be the price of the module that $k^* + 1$ kicks out if it is added (with $p = 0$ if it does not kick out any module).
    Suppose that $\vecp(k^*+1) > \lambda \cdot B$.
    Then we claim that $\vecp(S_{\vecp}) \geq (1-\lambda)B - \eps$.
    If not, then let us take $\vecp'(k^* + 1) = B - \vecp(S_{\vecp}) + p$.
    Note that $B - \vecp(S_{\vecp}) + p \geq \lambda B + p + \eps > \vecc(k^*+1) + \eps$.
    So this means that $k^*+1$ has a profitable deviation contradicting that $\vecp(S_{\vecp}) < (1-\lambda) \cdot B - \eps$.
    
    On the other hand, if $\vecp(k^*+1) \leq \lambda \cdot B$ and $k^*+1$ was not taken then it must be that $\vecp(S_{\vecp}) \geq (1 - \lambda) B$.
\end{proof}

\begin{lemma}
Let $\vecp$ be an $\eps$-equilibrium and $\lambda = \max_i \vecc(i) / B$ and suppose that $\lambda < 1/3$.
Then
\[
    \vecv(S_{\vecc}) \leq \left( 1 + \frac{2\lambda}{1-3\lambda} + \frac{\eps / \lambda B}{1-\lambda - \eps / B} \right) \vecv(S_{\vecp}).
\]
\end{lemma}
\begin{proof}
We assume that the modules are sorted in bang-per-buck order according to $\vecp$, i.e.~$\frac{\vecv(1)}{\vecp(1)} \geq \ldots \geq \frac{\vecv(n)}{\vecp(n)}$.

By Lemma~\ref{lemma:unselected_lt_k_bid_cost} and Lemma~\ref{lemma:unselected_gt_k_to_cost}, we can assume that if $k^*$ is the last module selected by $\greedy$ then for all un-selected modules $i \neq k^* + 1$, we have $\vecp(i) \leq \vecc(i) + \eps$.

First observe that
\begin{equation}
    \label{eqn:Sp_lb}
    \vecv(S_{\vecp}) \geq \left[(1 - \lambda) B - \eps\right] \frac{\vecv(k^*)}{\vecp(k^*)},
\end{equation}
where we used Lemma~\ref{lemma:vecp_most_of_budget}.

Next we bound $\vecv(S_{\vecc} \setminus [k^*+1])$.
For any $i \geq k^* + 1$, we have that
\[
    \frac{\vecv(i)}{\vecc(i) + \eps} \leq \frac{\vecv(k^*)}{\vecp(k^*)}.
\]
Let $\delta = \frac{\eps}{\max_i c_i} = \frac{\eps}{\lambda B}$ so that $\eps \leq \delta \vecc(i)$ for all $i$.
We thus have $\frac{\vecv(i)}{\vecc(i) + \eps} \geq \frac{\vecv(i)}{(1+\delta) \vecc(i)}$ for all $i$ which implies that for any $i \geq k^* + 1$, we have
\[
    \frac{\vecv(i)}{\vecc(i)} \leq 
    (1+\delta) \cdot \frac{\vecv(k^*)}{\vecp(k^*)}.
\]
Thus,
\begin{equation}
    \label{eqn:Sc_ub}
    \vecv(S_{\vecc} \setminus [k^* + 1])
    \leq (1+\delta) \frac{\vecv(k^*)}{\vecp(k^*)} \cdot B.
\end{equation}

We now consider two cases.
\paragraph{Case 1: The maximum value independent set in $[k^*+1]$ does not contain $k^* + 1$.}
In this case, we have 
\[
    \vecv(S_{\vecp}) = \vecv(S_{\vecp} \cap [k^*+1])
    \geq \vecv(S_{\vecc} \cap [k^*+1])
\]
since Claim~\ref{claim:matroid_first_k_opt} shows that $\greedy$ on $\vecp$ finds the maximum value independent set in $[k^*]$ which, in this case, is also the maximum value independent set in $[k^* + 1]$.
We thus have
\begin{align*}
    \vecv(S_{\vecc}) - \vecv(S_{\vecp})
    & = \vecv(S_{\vecc} \cap [k^* + 1]) - \vecv(S_{\vecp}) + \vecv(S_{\vecc} \setminus [k^* + 1]) \\
    & \leq (1 + \delta) \cdot \frac{\vecv(k^*)}{\vecp(k^*)} \cdot B \\
    & \leq \frac{1+\delta}{1-\lambda - \eps / B} \vecv(S_{\vecp}),
\end{align*}
using Eq.~\eqref{eqn:Sp_lb} for the last inequality.
Rearranging, we have
\[
    \vecv(S_{\vecc}) \leq \left( 1 + \frac{1+\delta}{1-\lambda - \eps / B} \right) \cdot \vecv(S_{\vecp}).
\]
\paragraph{Case 2a: The maximum value independent set in $[k^*+1]$ does contain $k^* + 1$ and $\lambda < 1/5$.}
First, let $A = \left\{ i \in S_{\vecp} \,:\, \frac{\vecv(i)}{\vecp(i)} \leq \frac{\vecv(S_{\vecp})}{(1-3\lambda) B} \right\}$.
We claim that $\vecp(A) \geq \lambda B$.
Suppose, for the sake of contradiction, that $\vecp(A) < \lambda B$.
Then $\vecp(S_{\vecp} \setminus A) > (1-\lambda) B - 2\lambda B = (1-3\lambda)B$.
We would thus have
\[
    \vecv(S_{\vecp})
    \geq \vecv(S_{\vecp} \setminus A)
    > \frac{\vecv(S_{\vecp})}{(1-3\lambda)B} (1-3\lambda) B = \vecv(S_{\vecp}).
\]
This is absurd, so we conclude that $\vecp(A) \geq \lambda B$.

We now claim that $\vecv(k^*+1) \leq \frac{2\lambda}{1-3\lambda} \vecv(S_{\vecp})$.
Again, suppose for the sake of contradiction, that $\vecv(k^*+1) > \frac{2\lambda}{1-3\lambda} \vecv(S_{\vecp})$.
Then if it deviated to $2 \lambda B \geq 2 \vecc(k^*+1) \geq \vecc(k^*+1) + \eps$,
its bang-per-buck would be
\[
    \frac{\vecv(k^*+1)}{\vecc(k^*+1)}
    > \frac{\vecv(S_{\vecp})}{(1-3\lambda)B},
\]
which would put it in front of $A$ and there would be sufficient budget when $\greedy$ visits $k^*+1$.
Since $k^*+1$ is in the maximum value independent set in $[k^*+1]$, it must therefore be selected by $\greedy$.
This would contradict that $\vecp$ was an $\eps$-equilibrium.
We conclude that $\vecv(k^*+1) \leq \frac{2\lambda}{1-3\lambda} \vecv(S_{\vecp})$.

Now continuing as in case 1, we have
\begin{align}
    \vecv(S_{\vecc}) - \vecv(S_{\vecp})
    & = \vecv(S_{\vecc} \cap [k^* + 1]) - \vecv(S_{\vecp}) + \vecv(k^*+1) + \vecv(S_{\vecc} \setminus [k^* + 1]) \\
    & \leq \vecv(k^*+1) + (1 + \delta) \cdot \frac{\vecv(k^*)}{\vecp(k^*)} \cdot B \label{eq:recall_analysis}\\
    & \leq \left( \frac{2\lambda}{1-3\lambda} + \frac{1+\delta}{1-\lambda - \eps/B}\right) \vecv(S_{\vecp}). \qedhere
\end{align}
\end{proof}

\begin{lemma}\label{lem:greedy_approx}
Let $S_{\OPT} \in \argmax \{ S \,:\, \vecc(S) \leq B, S \in \I \}$.
Let $\lambda = \max_i \vecc(i) / B$.
Then $\vecv(S_{\OPT}) \leq \left( 1 + \frac{\lambda}{1-\lambda} \right) \cdot \vecv(S_{\vecc})$.
\end{lemma}
\begin{proof}
    Sort the modules such that $\frac{\vecv(1)}{\vecc(1)} \geq \ldots \geq \frac{\vecv(n)}{\vecc(n)}$.
    Let $k^*$ be the last module selected by $\greedy$.
    If has the same value as $S_{\OPT}$ then we are done.
    Otherwise, a standard fact is that taking the first $k^*+1$ modules that form an independent set is an upper bound on $\vecv(S_{\OPT})$.
    We have that $\vecv(S_{\vecc}) \geq \frac{\vecv(k^*)}{\vecc(k^*)} \cdot (1-\lambda) \cdot B$ and
    $\vecv(S_{\OPT}) - \vecv(S_{\vecc}) \leq \vecv(k^* + 1) \leq \frac{\vecv(k^*)}{\vecc(k^*)} \cdot \lambda B \leq \frac{\lambda}{1-\lambda} \vecv(S_{\vecc})$.
    Rearranging gives the lemma.
\end{proof}

Combining the results above, we have prove the main theorem.

