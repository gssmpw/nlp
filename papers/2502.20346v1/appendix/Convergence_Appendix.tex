\section{Missing Proofs of Section~\ref{sec:convergence_dynamics_of_learning}}
In this section, we present the missing proofs from Section~\ref{sec:convergence_dynamics_of_learning}.
In the price competition analysis, we described an algorithm to compute $\eps$-equilibrium prices for the modules for the price competition game instance $\instance$ where $\mathcal I$ is a matroid. However, the modules can not implement an $\eps$-equilibrium price without full information about the game instance $\instance$. In this section, we show that a simple price dynamics where each module implements a multiplicative weight type learning algorithm (Definition~\ref{def:multiplicative_weight_learning}) converges to an $\eps$-equilibrium.  

More formally, we consider a repeated setting of price competition game instance $\instance$ where $\mathcal I$ is a matroid and $B=1$. At each time step $t$, each module $m_i$ specifies a price $\vec p^t(i)$ from a discretized set $\mathcal B = \{\delta, 2\delta, \ldots, 1\}$. 
The platform then chooses a subset of the modules according to the greedy algorithm (Algorithm~\ref{alg:greedy}).
Importantly, the module owners are only aware of their own cost and feedback provided by the platform after each round, which includes whether they are accepted or rejected at a given price and what is the maximum price at which the module could be selected at the current round.
In particular, they do not know the buyer's valuation of their module or the existence of other modules in the market, other than what they can infer from the feedback described above. 

\paragraph{Assumptions on Discretization}Throughout the section, we assume that $\vec c(i) > \delta^{1/3}$. We note that we can assume this without loss of generality by letting the discretization be more finer. In addition, we assume that $1 \leq \vec v(i)\leq n^2$ and $\delta < \frac{1}{n^3}$. Finally, to avoid tie-breaking at the equilibrium price, when $\barp(\SE)< B$ and $\SE'$ being the optimal set of modules in matroid $\mathcal I $ restricted to $\pi[k^*]$, we have $\barp(\SE' )> B + 2\cdot \sqrt \delta$. Here, $k^*$ is the last iteration of Algorithm~\ref{alg:equilibrium_dynamics_weighted}.

\subsection{Proof of Lemma~\ref{lem:eqm_price_dominates}}
We observe that for any module $j\in H$, $\frac{\vec v(j)}{\vec c(j)} \leq  \frac{\vec v(i)}{\barp(i)}$ for $i\in \SE$. Therefore, at any price vector $\vec p_{-i}$, $\frac{\vec v(j)}{\vec p(j)} \leq  \frac{\vec v(i)}{\barp(i)}$.

Let $\pi$ be the bang-per-buck ordering over the set of modules at price $\vec p$ and $\bar S$ be the set of modules selected by the greedy algorithm at price $\vec p$. At price $\vec p$, we observe that when the greedy algorithm (Algorithm~\ref{alg:greedy}) reaches module $i$, it adds module $i$ in the working solution $G$. Since $i$ gets selected at price $\vec p$, we have that no module with a lower bang-per-buck than the module can swap module $i$ from the working solution $G$. We let $\pi(\bar i) \in H$ be the module with the highest bang-per-buck such that module $i$ does not belong to the maximum weight independent set in matroid $\mathcal I$ restricted on $\pi(1),\dots, \pi(\bar i)$. We note the greedy algorithm does not reach the module $\pi(\bar i)$ otherwise module $i$ could have been swapped out from the set $\bar S$.

Let $\pi'$ be the bang-per-buck ordering over the modules at price $\vec p'$ and $S_{\vec p'}$ be the set of selected modules at price $\vec p'$. Next, we claim that at price $\vec p'$, greedy algorithm (Algorithm~\ref{alg:greedy}) reaches the module $i$ and adds it into the working solution. Let $L'$ be the set of modules with bang-per-buck $\leq \frac{\vec v(i)}{\barp(i)}$ at price $\vec p'$. Since $H\cap L' = \emptyset$, we have that $L'\subseteq L$. Since, $i\in S_\vec p$, if the greedy algorithm (Algorithm~\ref{alg:greedy}) reaches module $i$ then it adds module $i$ in the working solution.

Let $\pi(\bar i) = \pi'(i')$. We note that since bang-per-buck of module $\frac{\vec v(\pi(\bar i))}{\vec p(\pi(\bar i))} < \frac{\vec v(i)}{\barp(i)}$, we have $\{\pi(1),\dots , \pi(\bar i)\} = \{\pi(1),\dots, \pi(i')\}$. In addition, the price of all modules that come after module $i$ on the bang-per-buck ordering $\pi'$ is unchanged from the price vector $\vec p$, we conclude that the greedy algorithm can not reach module $\pi(\bar i)$ on the ordering $\pi'$ which concludes the proof.

\subsection{Proof of Lemma~\ref{lem:module_with_worst_bpb_rejected}}
We let $\pi$ be the bang-per-buck over the modules at price $\vec p$. Let $\pi(\ell):= i^*$. We observe that $ \SE \subseteq  \pi[\ell -1]$. Let $S^*[\ell]$ be the maximum weight independent set in matroid restricted to $\pi[\ell -1]$ chosen by greedy algorithm (Algorithm~\ref{alg:greedy}). We claim that $\vec p(S^*[\ell]) >B$. 

First We note that $S^*[\ell] \cap L = \emptyset$ as $\SE \subseteq S^*[\ell]$. In addition, for any module $i\in S^*[\ell] \setminus \SE'$ must satisfy that $i\in H$. We note that $i\in \SE \setminus S^*[\ell]$ forms circuit with $S^*[\ell]$ and can be exchanged by module $j\in \SE' \setminus S^*[\ell]$ such that $j\in H$ and $\vec v(j)>\vec v(i)$. Note that for $j\in H$, $\frac{\vec v(j)}{\vec p(j)}< \frac{\vec v(i)}{\barp(i)}$ and since $\vec v(i)< \vec v(j)$, we have $\barp(i)<\vec p(j)$. This implies that $\vec p(S^*[\ell])> \vec p(\SE')> B$. Therefore, $i^*$ can not be selected by the greedy algorithm (Algorithm~\ref{alg:greedy}). This further implies that $\vec p(i^* \cup S_\vec p)>B$.

\subsection{Proof of Lemma~\ref{lem:stability_of_eqm_price}}
    We consider module $\pi^0(k^*)$ where $k^*$ denotes the last iteration of Algorithm~\ref{alg:equilibrium_dynamics_weighted}. Now, we consider module $i\in \SE$ that submits price $\geq \barp(i)(1 + \sqrt \delta)$. We first show that $\frac{\vec v(i)}{\barp(i) (1+ \sqrt \delta)} / \frac{\vec v(j)}{\barp(j) +10\delta} < 1$. This follows because $\vec c(i) > \sqrt \delta$ and $\frac{\vec v(i)}{\barp(i)} = \frac{\vec v(j)}{\barp(j)} = \optbpb$. Now,  if $\barp(\SE) = B$ then since rest of the modules comes before module $i$ in the bang-per-buck ordering, it can no-longer be affordable as $\vecp(\SE)> B - 10\delta \cdot n + \sqrt \delta >B + \sqrt \delta -\delta^{2/3}>B$ assuming $\delta < \frac{1}{n^3}$. On the other hand, if $\vec p(\SE)< B$ then let $\SE'$ be the optimal set of modules in matroid restricted to $\pi^0[k^*]$. We have that $\barp(\SE') > B$. Again, since $\frac{\vec v(i)}{\barp(i)} = \optbpb$ for modules $i\in \SE'$. Therefore, via a similar argument module $i\in \SE'$ with $\barp(i) + \delta$ can not be affordable by a greedy algorithm.
    
    Now, we consider that module $i\in \SE'\setminus \{\pi^0(k^*)\}$ sets price $\geq \barp(i) - \sqrt \delta$. In this case, module $i$ will be selected as it comes before rest of the modules in bang-per-buck ordering and since $\barp(\SE')> B + \sqrt \delta$, total price of $\vec p(\SE')> \barp(\SE') > B -10\cdot \delta \cdot n - \sqrt \delta > B - \sqrt \delta -10n\delta $ by assumption on discretization. This implies that the greedy algorithm only visits elements in $\pi^0[k^*]$ before it runs out of budget and module $i$ gets selected. Now, due to Lemma~\ref{lem:eqm_price_dominates}, we have that module $i$ also gets selected at price $\barp(i)$. 
    
\subsection*{Proof of Theorem~\ref{thm:matroid_convergence}}
We then leverage Lemma~\ref{lem:eqm_price_dominates} to define an event stating that after a sufficiently large number of iterations, all modules in $\SE$ start setting their price higher than their equilibrium prices.  More formally, we let $T_0$ be sufficiently large the round (determined later) and define $\mathcal E^0 = \{ \forall s\geq T_0:\vec p^t(i) > \Delta_i,  \forall i \in \SE  \}$.

We first observe that given any price vector $\vec p$ and conditioned on the event $\mathcal E^0$, the module with the worst bang-per-buck in the set $\SE'$ defined as $\SE' :=\SE \cup \{\pi^0(k^*)\}$ where $k^*$ is the last iteration of Algorithm~\ref{alg:equilibrium_dynamics_weighted}, will not be selected due to Lemma~\ref{lem:module_with_worst_bpb_rejected}.
Our overall proof approach is to show that under conditioned on the event $\mathcal E^0$, the modules in $\SE'$ literately stop posting the prices that lead to smaller bang-per-buck for the platform since it will not be selected at such a higher price. In order to demonstrate that, we define an order over the possible prices w.r.t. their bang-per-buck values for the buyer. Next, we define an order over the prices with respect to their bang-per-buck value. We iteratively define 
$$(  b_{(k)},   i_{(k)}): = \argmin_{\{(b,i)\in \bar{ \mathcal B}  \setminus \{(   b_{(1)},   i_{(1)}),\dots ,(   b_{(k-1)},   i_{(k-1)}) \} \}} \left \{\frac{\vec v(i)}{b} \right \}.$$
Here, $(   b_{(1)},   i_{(1)}): = \argmin_{(b,i) \in \bar{\mathcal B}} \{\frac{\vec v(i)}{b} \}$ and $\bar{\mathcal B}:=\bigcup_{i\in \SE'} \{(\Delta_i + \delta, i) , \dots , (1,i) \}$.  First, we prove the following claim that shows that the module $i_{(1)}$ will never price $b_{(1)}$ with high probability after $\poly(1/\delta)$ many rounds as it will never be selected.


Next, we extend this argument and iteratively show that the modules will stop posting prices larger than their respective $\Delta_i$s. We let $T_1 < T_2 <\dots <T_K$ and $T^*_1 < T^*_2 <\dots <T^*_K$ such that $T_k \leq T_{k}^*$ for some $K < \frac{n}{\delta}$ and the events $$\mathcal E^p:= \{ \forall s\geq T_{p} \text{ and } j\in M: b_{i_{(p)}} < b_{(p)}\},$$  satisfying the following condition: 
\begin{enumerate}
	\item We now inductively define $T_k$ as follows: suppose we are given $T_1,\dots , T_{k-1}$; $T^*_0,\dots , T^*_{k-1}$ and conditioned on the events $\mathcal E^1,\dots , \mathcal E^{k-1}$, we let $T_k^* > T^*_{k-1}$ be the smallest round (if exists) such that, 
	\begin{equation}
		\sum_{s\leq T^*_k} u_{i_{(k)},s}(\Delta_{i_{(k)}},\vec p^s_{-i_{(k)}}) - u_{i_{(k)},s}(b,\vec p^s_{-i_{(k)}}) \geq \delta^6 \cdot T_k^*
	\end{equation}
	\item  We let $T_k\leq T_k^*$ be the smallest stage such that there exists some $b_{i_{(k)}}' \notin\{b_{(1)},\dots .b_{(k-1)}\}$ for all $s \in (T_k,T_k^*]$, such that, 
	\begin{equation}
		\sum_{t\leq s} u_{i_{(k)},t} (b'_{i_{(k)},t},\vec p^s_{-i_{(k)}}) - u_{i_{(k)},t}(b_{i_{(k)}}, \vec p^s_{-i_{(k)}}) \geq \frac{C}{\gamma_s \cdot \delta}.
	\end{equation}
\end{enumerate}
Above,  $\mathcal E^p$ captures the event where the module $i_{(p)}$ sets their price with higher bang-per-buck than $p$-th lowest bang-per-buck in the set of bids. In addition, we can observe that by the definition of $T_p$,  we have that the module $i_{(p)}$ begins setting the price lower than $b_{i_{(p)}}$ with high probability due to existence of some lower bid $b$ which has significantly higher cumulative utility. Therefore, if we show an existence of bounded $T_i $s then we can essentially show that any module $i\in \SE'$ starts setting price lower than their corresponding $\Delta_i + 2\cdot \delta$ after some finite many rounds while event $\mathcal E^0$ ensures that module $i\in \SE'$ is bidding higher than $\Delta_i$. We first make the following observation: 
\begin{observation}\label{obs:dropped_price_with_hp_weighted}
	For any $p<K$ and $s\geq T_p$, we have, $$\Pr\left [\text{Price $b_{(p-1)}$ for module $i_{(p)}$ is not selected at round }s \mid \bigcap_{p'\leq p} \mathcal E^{p'} \right] = 1.$$
\end{observation}
\begin{proof}
	The proof of the observation follows from the fact that the price $b_{(p-1)}$ has the worst bang-per-buck at stage $s$ once conditioned on the event $\bigcap_{p'\leq p} \mathcal E^{p'}$. Therefore, Lemma~\ref{lem:module_with_worst_bpb_rejected} implies that the greedy algorithm does not select price $b_{(p-1)}$. 
\end{proof}
Above, we note that if Condition~\ref{condition1_weighted} satisfies for for some $k$ then for $T_k = T_k^*$, Condition~\ref{condition2_weighted} trivially satisfies. We next prove an upper bound on $T_i$ which shows an existence of desired $T_i$s which is one of the most crucial steps in proving Theorem~\ref{thm:matroid_convergence}. We recall our distorted payment rule: for initial rounds  $t\leq T_0$, each selected module $m_i\in M$ gets payment of $\vec p^t(i) + \delta^2\cdot \vec p^t(i)$  and the rest of the modules get payment of $\delta^2 \cdot \vec p^t(i)$ and later after round $t > T_0$, all selected module $m_i\in M$ gets payment of $\vec p^t(i) + \frac{\delta^4}{ \vec p^t(i)}$  and the rest of the modules get payment of $\frac{\delta^4}{ \vec p^t(i)}$.  

\begin{lemma}\label{lem:bounding_iterations}
	For any $k \geq 1$, conditioning on the event $\bigcap_{i=1}^{k} \mathcal E^i$, with probability $1$, we have,
	\begin{equation*}
		T_k \leq \frac{1}{\delta^6} \cdot \left( \sum_{i=0}^{k-1}\sqrt {T_i} + (1+\delta^2)\cdot T_0\right).
	\end{equation*}
\end{lemma}
\begin{proof}
	In order to prove the claim, we prove the following via induction on $k$: for any $k\geq 0$ and conditioned on the event $\bigcap_{p\leq k} \mathcal E^p$, with probability $1$, we have,
	\begin{equation*}
		\sum_{s\leq T_{k}} u_{i_{(k)},s} (b_{(k)}, \vec p^s_{-i_{(k)}} ) \leq \frac {(1+ 2\cdot \delta)} {b_{i_{(k)}}\delta^2} \cdot \sum_{i=1}^{(k-1)}  \sqrt {T_i}  +  (T_{k} - T_{k-1})\cdot \frac{\delta^4}{b_{i_{(k)}}} + T_0 \textbf{ and }T_k \leq \frac{1}{\delta^4} \cdot \sum_{i=0}^{k-1}\sqrt {T_i} + T_0.
	\end{equation*}
	The base case of induction for $k=0$ follows since $ \sum_{s\leq T_{1}} u_{i_{(1)},s} (b_{(1)}, \vec p^s_{-i_{(1)}} ) \leq T_0$ and $T_0 \leq T_0$. Suppose the claim holds for $k-1$. We first consider the following case:
	\paragraph{Case-1 ($i_{(k)} = i_{(k-1)}$) :} In this case, we can observe that, 
	\begin{align*}
		\sum_{s\leq T_{k}} u_{i_{(k)},s} (b_{(k)}, \vec p^s_{-i_{(k)}} ) &\leq \frac{C}{\gamma_{T_{k-1}} \cdot \delta} + \sum_{s\leq T_{k-1}} u_{i,s} (b_{(k-1)}, \vec p^s_{-i_{(k-1)}})\\
		&= \frac{\sqrt{T_{k-1}}}{ \delta} +  (T_{k} - T_{k-1})\cdot \frac{\delta^4}{b_{i_{(k)}}} + \sum_{s\leq T_{k-1}} u_{i,s} (b_{(k-1)}, \vec p^s_{-i_{(k-1)}})\\
		&\leq \frac{\sqrt{T_{k-1}}}{ \delta} +  (T_{k} - T_{k-1})\cdot \frac{\delta^4}{b_{i_{(k)}}} +\frac {(1+ 2\cdot \delta)} {b_{i_{(k)}}\delta^2} \cdot \sum_{i=1}^{(k-2)} \sqrt {T_i} + (T_{k-1} - T_{k})\cdot \frac{\delta^4}{b_{i_{(k)}}} + T_0 \\
		&\leq  \frac{\sqrt{T_{k-1}}}{ \delta} +  \frac{\sqrt{T_{k-1}}}{\delta^2 \cdot b_{i_{(k)}}} +  (T_{k} - T_{k-1})\cdot \frac{\delta^4}{b_{i_{(k)}}} \\
		& + \frac {(1+ 2\cdot \delta)} {b_{i_{(k)}}\delta^2}\cdot \sum_{i=1}^{(k-2)} \sqrt {T_i} + (T_{k-1} - T_{k})\cdot \frac{\delta^4}{b_{i_{(k)}}} + T_0 \\
		&\leq \frac {(1+ 2\cdot \delta)} {b_{i_{(k)}}\delta^2} \cdot \sum_{i=1}^{(k-1)}  \sqrt {T_i}  +  (T_{k} - T_{k-1})\cdot \frac{\delta^4}{b_{i_{(k)}}} + T_0.
	\end{align*}
	Above, the first inequality holds by the definition of $T_{k-1}$ and the fact that $i_{(k)} = i_{(k-1)}$, the equality holds because $u_{i_{(k-1)},s}(b_{(k)},\vec b_{-i_{(k-1)}}) = \frac{\delta^4}{b_{i_{(k)}}}$ for any $s\geq T_{k-1}$ due to Observation~\ref{obs:dropped_price_with_hp_weighted} once conditioned on the event $\bigcap_{i=1}^{k-1} \mathcal E^i$. The second and third inequality holds due to inductive hypothesis since $T_{k-1} - T_{k-2}\leq \frac{1}{\delta^4} \cdot \sqrt{T_{k-1}}$.
	
	\paragraph{Case-2 ($i_{(k)} \neq i_{(k-1)}$):} In this case, we let $\bar k$ be the largest index such that $i_{\bar k} = i_{(k)}$. If such $\bar k$ does not exist then we let $\bar k = 1$. We observe that either $b_{(\bar k)} = b_{(k)} +\delta$ or $\bar k = 1$ since all possible prices are discretized within the additive factor of $\delta$ and $b_{(1)} =1$. 
	We condition on the event $\bigcap_{i=1}^{k-1} \mathcal E^i$.  We define for any $k'\leq k$,  $\tilde b_{i_{(k')}} = b_{i_{(k')}} - \delta $.
	For any $k'\leq k$, by definition of $T_{k'}$, we have,
	\begin{align}\label{eq:iterative_expansion_eq_1}
		\sum_{t < T_{\bar k}} u_{i_{(\bar k)},t} (\tilde b_{i_{(\bar k)},t},\vec p^s_{-i_{(k-1)}}) &+ u_{i_{(\bar k)},T_{\bar k}} (\tilde b_{i_{(\bar k)},T_{\bar k}},\vec p^s_{-i_{(k-1)}}) \leq  \frac{C}{\gamma_{T_{\bar k}} \cdot \delta} + \sum_{t < T_{\bar k}} u_{i_{(\bar k)},t}(b_{i_{(\bar k)}}, \vec p^s_{-i_{(\bar k)}}) + 1 + \frac{\delta^4}{b_{i_{(\bar k)}}}  \notag \\
		%&= \frac{\sqrt{T_{\bar k}}}{\delta} + \sum_{t\leq T_{\bar k-1}} u_{i_{(\bar k)},t}(b_{i_{(\bar k)}}, \vec p^s_{-i_{(\bar k)}}) + (T_{\bar k} - T_{\bar k-1})\cdot \frac{\delta^4}{b_{i_{(\bar k)}}}  + 1 \\
		&\leq \frac{2\sqrt{T_{\bar k}}}{\delta} + \sum_{t\leq T_{\bar k}} u_{i_{(\bar k)},t}(b_{i_{(\bar k)}}, \vec p^s_{-i_{(\bar k)}})  \notag\\
		&\leq \frac{2\sqrt{T_{\bar k}}}{\delta} + \frac {(1+ 2\cdot \delta)} {b_{i_{(k)}}\delta^2} \cdot \sum_{i=1}^{(\bar k-1)}  \sqrt {T_i} + (T_{\bar k} - T_{\bar k-1})\cdot \frac{\delta^4}{b_{i_{(\bar k)}}} + T_0\notag\\
		&\leq \frac {(1+ 2\cdot \delta)} {b_{i_{(k)}}\delta^2} \cdot \sum_{i=1}^{\bar k}  \sqrt {T_i}  + T_0
	\end{align}
	Above, the first inequality holds because $T_{\bar k}$ be the smallest stage for which Condition~\ref{condition2_weighted} holds and the utility of module $i_{(\bar k)}$ at round $T_{\bar k}$ can be at most $1+ \frac{\delta^4}{b_{i_{(k)}}}$. The third inequality holds due to the inductive hypothesis. The second and last inequality holds because $T_{\bar k}>1$ and $\delta << \frac 1 2$. 
	
	 Next, we claim that under event $\bigcap_{p=1}^k \mathcal E^p$, from rounds $T_{\bar k}$ to $T_{k}$, price $\tilde b_{i_{(\bar k)}} =b_{i_{(k)}}$ can not be the winning price for module $i_{(k)}$. Now, notice that under event $\bigcap_{p=1}^k \mathcal E^p$, after round $s>T_{(\bar k)}$, module $i_{(\bar k-1)}$ does not get selected and for bang-per-buck ordering $\pi_s$ at round $s$ with $\pi^s[\ell] = i_{(\bar k-1)}$, due to Lemma~\ref{lem:module_with_worst_bpb_rejected}, we have $\vec p^s(S^*[\ell])>B$ where $S^*[\ell]$ be the maximum weight independent set in matroid $\mathcal I$ restricted to $\pi^s[\ell]$. Since, $b_{(k)}> \Delta_i + \sqrt \delta$, we have that if module $\bar k$ gets selected at price $\tilde b_{i_{(\bar k)}}$ then module $\vec p^s(S^*[\ell])\leq B$ which is a contradiction. Therefore, we have,
	 \begin{align}\label{eq:eq2}
	     \sum_{T_{\bar k}<t < T_{k}} u_{i_{(\bar k)},t} (\tilde b_{i_{(\bar k)},t},\vec p^s_{-i_{(k-1)}}) &\leq (T_{k}  - T_{\bar k})\cdot \frac{\delta^4}{b_{(k)}} \leq(T_{k} - T_{k-1}) \cdot \frac{\delta^4}{b_{(k)}}  + \frac {(1+ 2\cdot \delta)} {b_{i_{(k)}}\delta^2}\cdot  \sum_{i=\bar k+1}^{ k-1} \sqrt {T_{i}} 
	 \end{align}
	 Above, the second inequality holds due to the inductive hypothesis. Combining the Inequalities~\ref{eq:iterative_expansion_eq_1}, \ref{eq:eq2}, we obtain,
	 \begin{align*}
	     	\sum_{s\leq T_{k}} u_{i_{(k)},s} (b_{(k)}, \vec p^s_{-i_{(k)}} )&=  \sum_{T_{\bar k}<t < T_{k}} u_{i_{(\bar k)},t} (\tilde b_{i_{(\bar k)},t},\vec p^s_{-i_{(k-1)}}) +  \sum_{t < T_{\bar k}} u_{i_{(\bar k)},t} (\tilde b_{i_{(\bar k)},t},\vec p^s_{-i_{(k-1)}})\\
	     	&\leq \frac {(1+ 2\cdot \delta)} {b_{i_{(k)}}\delta^2} \cdot \sum_{i=1}^{(k-1)}  \sqrt {T_i}  +  (T_{k} - T_{k-1})\cdot \frac{\delta^4}{b_{i_{(k)}}} + T_0
	 \end{align*}
	 We then observe that module $i_{(k)}$ can be selected at price $b_{i_{(k)}}$ in less than $$\frac{1}{b_{(k)}} \cdot \left( \frac {(1+ 2\cdot \delta)} {b_{i_{(k)}}\delta^2} \cdot \sum_{i=1}^{(k-1)}  \sqrt {T_i}  + T_0\right), $$ many rounds.  Due to Lemma~\ref{lem:eqm_price_dominates}, we have that whenever the module gets selected at price $b_{(k)}$, it also gets selected at price $\Delta_i$. Therefore, for any $T_{k}^*\geq \frac{1}{\delta^6} \cdot \left( \sum_{i=0}^{k-1}\sqrt {T_i} + (1+\delta^2)\cdot T_0\right)$, we can bound the utility difference of $b_{(k)}$ and $\Delta$ as:
	 \begin{align*}
     &\sum_{s\leq T^*_k}( u_{i,s}(\Delta_i,\vec p^s_{-i}) - u_{i,s}(b_{i_{(k)}},\vec p^s_{-i}) ) - \delta^6\cdot T_{k^*}\\
     &\geq \left(\frac{(b_{(k)} - \Delta_i)\cdot \delta^4}{b_{(k)} \cdot  \Delta_i} -\delta^6 \right)\cdot T_k^* - \frac{(b_{(k)} - \Delta_i)}{b_{(k)} } \cdot \left( \frac {(1+ 2\cdot \delta)} {b_{i_{(k)}}\delta^2} \cdot \sum_{i=1}^{(k-1)}  \sqrt {T_i}  + T_0\right)\\
     &\geq   \left(\frac{(b_{(k)} - \Delta_i)\cdot \delta^4}{b_{(k)} \cdot  \Delta_i} -\delta^6 \right) \cdot \left( \frac{1}{\delta^6} \cdot \left( \sum_{i=0}^{k-1}\sqrt {T_i} + (1+\delta^2)\cdot T_0\right) \right) - \frac{(b_{(k)} - \Delta_i)}{b_{(k)} } \cdot \left( \frac {(1+ 2\cdot \delta)} {b_{i_{(k)}}\delta^2} \cdot \sum_{i=1}^{(k-1)}  \sqrt {T_i}  + T_0\right)\\
     &\geq \frac{(b_{(k)} - \Delta_i)}{b_{(k)} \cdot \Delta_i} \cdot\left(  \left( \frac{1}{\Delta_i \delta^2} - \frac {(1+ 2\cdot \delta)} {b_{i_{(k)}}\delta^2} - \frac{b_{(k)} \cdot \Delta_i}{(b_{(k)} - \Delta_i)} \right) \cdot \sum_{i=1}^{(k-1)}  \sqrt {T_i}  + \left( \frac{1}{\delta^2} - (1+ \delta^2)  \right) \cdot T_0 \right)\geq 0.\\
   \end{align*}
Above, the second inequality holds because of the assumption on $T_k^*$, the third inequality holds because the factor in multiplication with $T_0$ is positive and the final inequality holds because $\Delta_i > \vec c(i) >\sqrt \delta $ and $b_{(k)} > \Delta_i + 10 \cdot  \delta $. Since, $T_k \leq T_{k}^*$, this concludes the proof. 
\end{proof}
The above lemma immediately implies the following bound on $T_k$. 
\begin{lemma}
Conditioned on the event $\bigcap_{p=0}^k \mathcal E^p$, we have $T_k \leq \frac{k^2\cdot T_0}{\delta^{13}}$ with probability one.  
\end{lemma}
\begin{proof}
The lemma follows via induction on $k\geq 2$. We get $T_k \leq \frac{k\cdot (k-1)}{\delta^{2.5}} + \frac{(1+\delta^2)\cdot T_0}{\delta^6} \leq \frac{k^2\cdot T_0}{\delta^{13}}$.
\end{proof}
To complete the proof of Theorem~\ref{thm:matroid_convergence}, we need to show that $\Pr\left [\bigcap_{i=0}^K \mathcal E^i \right] \geq 1 - O \left( \poly\left ( \frac{1}{\delta} \right) \cdot \exp\left ( - \frac{1}{\delta} \right) \right). $ Since $K \leq \frac{n}{\delta}$, we have $T_K \leq  \poly \left(n, \poly \left( \frac 1  \delta \right) \right)$ which will complete the proof that all modules will be setting prices $\geq \Delta_i - \delta $ and $\leq \delta_i + 10\cdot \Delta_i$ with high probability after $\poly(1/\delta , n)$ rounds. 


\begin{lemma}\label{lem:key_bounding_probability}
	For small enough $\delta>0$, we have $\Pr\left [\bigcap_{i=1}^K \mathcal E^i \mid \mathcal E^0\right] \geq 1 - O \left( \poly\left ( \frac{1}{\delta} \right) \cdot \exp\left ( - \frac{1}{\delta} \right) \right).$
\end{lemma}
\begin{proof}
	We show via induction that for any $t<K$, we have $$\Pr \left [ \mathcal E^k \mid \bigcap_{i=0}^{k-1} \mathcal E^i \right] \geq 1 - \frac{(k^2 +1)\cdot T_1}{\delta^6}\cdot \exp \left( -\frac{1}{\delta} \right).$$ 
	For simplicity of notations, we let $ \bar{ \mathcal E}^t : = \bigcap_{i=0}^{t-1} \mathcal E^i$. We fix $T_k^* = \frac{k^2\cdot T_0}{\delta^{13}}$.
	\begin{align*}
	%\label{eq:boudning_prob}
		\Pr\left[ \forall s\geq T_k: \bigwedge_{\ell = 1}^k   \vec p^s(i_{(\ell)}) < b_{i_{(\ell)}} \mid \bar{ \mathcal E}^k  \right] &=   \Pr\left[ \forall s\geq T_k:   \vec p^s(i_{(\ell)}) <b_{i_{(k)}} \mid \bar{ \mathcal E}^k  \right] \notag \\
		&= 1 - \Pr \left[\exists s\geq T_k:\vec p^s(i_{(\ell)}) \geq b_{i_{(1)}} \mid \bar{ \mathcal E}^k  \right] \notag \\
		&\geq 1 - \sum_{T_k\leq s\leq T_k^*}\Pr \left[\vec p^s(i_{(\ell)}) \geq b_{i_{(1)}} \mid \bar{ \mathcal E}^k  \right]\\
		&+ \sum_{s\geq  T_k^*}\Pr \left[\vec p^s(i_{(\ell)}) \geq b_{i_{(1)}} \mid \bar{ \mathcal E}^k   \right] \notag\\
		&\geq  1 - T_K \cdot \exp \left(- \frac{1}{\delta}\right) - \sum_{s\geq T_k^*} \exp \left( - \gamma_s \cdot \delta^6 \cdot s \right) \notag\\
		& =  1 - \frac{(K^2 +1)\cdot T_0}{\delta^{13}} \cdot \exp \left(- \frac{1}{\delta}\right) - \sum_{s\geq T_k^*} \exp \left( - C\cdot \delta^6 \cdot \sqrt s \right)\\
		&\geq 1 - \frac{2(k^2 +1)\cdot T_0}{\delta^{13}} \cdot \exp \left( -\frac{1}{\sqrt \delta} \right).
	\end{align*} 
	Above, the first equality holds due to conditioned on the event $\bar{ \mathcal E}^k $, second inequality holds due to union bound, the third inequality holds due to the inequality $\Pr[\vec p^s(i_{(\ell)}) < b_{(k)} \mid\bar{ \mathcal E}^k ] \geq  1 - \exp \left( - \frac 1 \delta \right)$ for all $s\geq T_k$ and $T_k^* \leq  \frac{(k^2 +1)\cdot T_1}{\delta^6}$. This completes the proof of the claim. 
\end{proof}
Finally, we need to show that the probability of event $\mathcal E^0$ is exponentially high. 


\subsubsection*{Lower bounding $\Pr[\mathcal E^0]$}
To this end, we consider the following event $$\mathcal F^0:= \left\{\forall i\in \SE : \text{$\Delta_i$  gets selected for $\leq \frac{T_0}{\delta^{16}}$ from first $\frac{T_0}{\delta^{20}}\cdot (1+\delta^{18}\cdot (1-5\delta))$ rounds} \right \}.$$ 
We let $\frac{T_0}{\delta^{\ell_i}}$ be the number of times module $i$ gets selected at price $\Delta_i$. We let $T'_i= \frac{T_0\cdot \Delta_i}{\delta^{\ell_i +4}}\cdot (1+\delta^{\ell_i +2}\cdot (1-5\delta))$. We note that $\ell_i$ is a random variable.  

Note that once we conditioned on $\mathcal F^0$, we set $T_i = T_i^* = \frac{T_0\cdot \Delta_i}{\delta^{\ell_i +4}}\cdot (1+\delta^{\ell_i +2}\cdot (1-5\delta))$ for all $i\in [K]$. We observe that for any module $i\in \SE$ and price $p> \Delta_i + 10 \cdot \delta$ and $\delta_i < 1 - 10\cdot \delta$, with probability $1$, we have, 
	 \begin{align*}
	 %\label{eq:upper_bound_utility_large_bid}
     &\sum_{s\leq T^*_k}( u_{i,s}(\Delta_i,\vec p^s_{-i}) - u_{i,s}(p,\vec p^s_{-i}) )  \geq \frac{(p- \Delta_i)\cdot \delta^4}{p \cdot  \Delta_i}\cdot (T_k^* -T_0)   - (p - \Delta_i) \cdot \left( \frac{T_0}{\delta^{\ell_i}}\right) - (p - \Delta_i)\cdot  \delta^2 \cdot T_0\notag\\
     &\geq \frac{(p- \Delta_i)\cdot \delta^4}{p \cdot  \Delta_i}\cdot \left(\frac{T_0\cdot \Delta_i}{\delta^{\ell_i +4}}\cdot (1+\delta^{\ell_i+2}\cdot (1-5\delta)) -T_0 \right)  - (p - \Delta_i) \cdot \left( \frac{T_0}{\delta^{\ell_i}}\right) - (p - \Delta_i)\cdot  \delta^2 \cdot T_0\notag \\
     &\geq  \left ( \delta^2 \left (\frac{1- 5\delta }{p\cdot \Delta_i} \right)  - 1 \right)\cdot T_0\geq \delta^3 \cdot T_0
   \end{align*}
   Above, the first inequality follows because $\Delta_i$ gets selected at most $\leq \frac{T_0}{\delta^{\ell_i}}$ times and whenever the price by module $i$ being $\Delta_i$ is not selected then price  $p$ can not be selected as well (Lemma~\ref{lem:eqm_price_dominates}). The second inequality follows by the definition of $T_k^*$. The last inequality follows because $p> \Delta_i + 10 \cdot \delta$ and $\Delta_i < 1 - 10\cdot \delta$.
   
   In addition, for any bid $b< \Delta_i$, again due to Lemma~\ref{lem:eqm_price_dominates}, whenever the price by module $i$ being $\Delta_i$ is not selected then price  $p$ can not be selected as well. Therefore, we have,
	 \begin{align}
     \sum_{s\leq T^*_k}( u_{i,s}(\Delta_i,\vec p^s_{-i}) - u_{i,s}(p,\vec p^s_{-i}) ) &\geq (\Delta_i - p)\cdot  \delta^2 \cdot T_0 + \frac{(\Delta_i - p)\cdot T_0}{\delta^{\ell_i}} -\frac{( \Delta_i - p)\cdot \delta^4}{p \cdot  \Delta_i}\cdot (T_k^* -T_0) \notag \\
     &\geq (\Delta_i - p)\cdot  \delta^2 \cdot T_0 + \frac{(\Delta_i - p)\cdot T_0}{\delta^{\ell_i}} -\frac{( \Delta_i - p)\cdot \delta^4}{p \cdot  \Delta_i}\cdot \frac{T_0\cdot \Delta_i}{\delta^{\ell_i +4}} \notag  \\
     & = \frac{(p - \Delta_i)}{ \Delta_i}\cdot \left( \delta^8\cdot (1 - 5\delta ) \right) \cdot T_0 \geq \delta^9 \cdot T_0.\label{eq:upper_bound_utility_small_bid}
   \end{align}
   
   This leads to the following lemma.
   
   \begin{lemma}
   For $T_0\geq \frac{n^2}{\delta^{22}}$ and conditioned on the event $\mathcal F^0$, for all the rounds from $T': = \frac{T_0}{\delta^6}(1 + \delta^4 (1-5\delta))$ to  $\frac{T_0}{\delta^6}(1 + \delta^4 (1-5\delta)) + T_0$, all modules in $i\in \SE'$ sets prices between $[\Delta_i - 10\cdot \delta , \Delta_i + 10 \cdot \delta]$ with probability $1 - \poly(n,1/\delta) \cdot \exp(- \frac 1 {\sqrt \delta})$.  
   \end{lemma}
   \begin{proof}
       For any $s\geq T'$, we let $\mathcal H^s(i)$ be the event where module $i$ submits price from the set $[\barp(i) , \barp(i) + \sqrt \delta]$. Due to Lemma~\ref{lem:stability_of_eqm_price},for any $s>T'$, conditioning on event $\bigcap_{i}\bigcup_{T' \leq t\leq s} \mathcal H^t(i)$, we have for any $i\in \SE'$, $\barp(i)$ is accepted at any round $T' \leq t\leq s$ and $\barp(i) \sqrt \delta$ is not accepted at round $t$. This further implies that $p\notin [\Delta_i , \barp(i)+ \sqrt{\delta}]$, we have 
       \begin{equation*}
           \sum_{s\leq t}( u_{i,s}(\barp(i),\vec p^s_{-i}) - u_{i,s}(p,\vec p^s_{-i}) ) \geq (t - T')\cdot \delta + T_0\cdot \delta^9
       \end{equation*}
       This further implies that conditioned on the event $\bigcap_{i}\bigcup_{T' \leq t\leq s} \mathcal H^t(i)$ and property of multiplicative weight update algorithm, we have that
       \begin{align*}
           \Pr\left [\bigcap_i \mathcal H^t(i)\mid \bigcap_i\bigcap_{T'<s<t} \mathcal H^s(i)\right ]&\geq 1 - 2\cdot \frac{n}{\delta}\cdot \exp \left( - \frac{C\cdot (\delta^9 \cdot T_0 + \delta (t- T')}{\sqrt t} \right)\\
           & \geq  1 - 2\cdot \frac{n}{\delta}\cdot \exp \left( - \frac{C\cdot (\frac{n^2}{\delta^{13}} + \delta (t- T')}{\sqrt {t- T' + \frac{n^2}{\delta^{22}}}} \right)\\
           &\geq 1 - 2\cdot \frac{n}{\delta}\cdot \exp \left( - C\cdot \left (\frac{n^2}{\delta} + \delta \sqrt{(t- T')}\right) \right)
       \end{align*}
       Next, via conditional expectations, and assuming $\delta$ small enough, we conclude that,
       \begin{equation*}
          \Pr\left [\bigcap_i\bigcap_{s>T'} \mathcal H^s(i)\right ] \geq 1 - \poly(1/\delta , n)\cdot \exp(- 1/\sqrt \delta). 
       \end{equation*}
   \end{proof}
   On the other hand, when $\mathcal F$, does not hold, then all modules win a sufficiently large number of times that leads to for any $i\in \SE'$ and $p<\bar p - 10\cdot \delta$ we have,
   \begin{equation*}
       \sum_{s\leq T^*_K}( u_{i,s}(\Delta_i,\vec p^s_{-i}) - u_{i,s}(p,\vec p^s_{-i}) ) \geq \delta^9 \cdot T_0.
   \end{equation*}
   Since, $\mathcal E^0 \subseteq \bigcap_i\bigcap_{s>T'} \mathcal H^s(i)$, we conclude that $\Pr\left [\mathcal E^0 \right]>1 - \poly(1/\delta , n) \cdot \exp \left( - \frac{1}{\sqrt \delta} \right)$.