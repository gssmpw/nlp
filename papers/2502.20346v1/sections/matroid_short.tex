\section{Price Competition for Buyer with Matroid Constraints}
\label{sec:weighted_matroid}
In this paper, we consider two extensions of the price competition game to buyers with matroid constraints.
In the first extension, we focus on the setting where the buyer's value $\vecv(i) = 1$ for all modules $i$.
In this case, we consider a natural generalization of Algorithm~\ref{alg:knapsack} in that a module is automatically rejected if it is infeasible in the current set selected thus far (see Algorithm~\ref{alg:modified_greedy}).
Note that this is not the same as Algorithm~\ref{alg:greedy} since Algorithm~\ref{alg:greedy} allows modules that come later in the bang-per-buck order to ``replace'' modules that have already been accepted.
However, for this setting we are able to prove that an $\eps$-equilibrium always exists (Theorem~\ref{thm:unweighted_eq_exist}) and and we obtain a $(2-\lambda) / (1-\lambda)^2$-approximation where $\lambda = \max_i \vecc(i) / B$ (Theorem~\ref{theorem:unweighted_approx}).
All the details are relegated to Appendix~\ref{app:price_comp_matroid}.
In the rest of the section, we consider the pricing game when the platform uses Algorithm~\ref{alg:greedy} for module selection for any matroid and value vector. 

\subsection{Existence of Equilibria}
To prove the existence of equilibria, we consider Algorithm~\ref{alg:equilibrium_dynamics_weighted} that construct an equilibria for any given price competition instance with a matroid constraint. The algorithm begins by initializing $\vecp$ as $\vec p = \vec c$ and an ordering $\pi^0$ which is decreasing in their bang-per-buck w.r.t.~$\vecp$ at round zero. We then define the set $A^\ind$ that denotes the set of modules that increase their price at iteration $\ind$ and $\pi^\ind$ that denotes the bang-per-buck order over modules at the price $\vec p^\ind$. 

We can describe the algorithm as follows.
First, we raise the prices of all modules in $A^{k-1}$ so that their bang-per-buck meets that of module $\pi^k(k)$, whose price is still at its cost.
Next, we check whether or not $\pi^k(k) \in \spn(A^{k-1} \cup T^{k-1})$.
If not we add it to $A^{k-1}$ to obtain $A^k$.
Otherwise, we find the (unique) circuit in $A^{k-1} \cup  \{\pi^k(k)\}$. Then we remove the lowest valued module from $C^k$ and replace it with the module $\pi^0(k)$. 

Finally, if adding $\pi^0(k)$ to the set $A^k$ increases the total price to exceed the budget then we remove the module $\pi^0(k)$ from the set $A^k$ and adjust the price of all the modules in $A^{k-1}$ such that it satisfies the budget constraint and no module in $A^{k-1}$ has a profitable deviation. Throughout the algorithm, we maintain the following invariants at iteration $\ind$ whose proof is delegated to Appendix~\ref{proof_of_weighted_invariant} 
\begin{claim}\label{claim:weighted_invarient}
For any iteration $k\leq k^*$ of Algorithm ~\ref{alg:equilibrium_dynamics_weighted}, the following invariants holds:
\begin{enumerate}
    \item $\pi^{k-1}([k-1]) = \pi^k([k-1])$ and $\pi^{k-1}(k) = \pi^{k}(k)$.
    \item when the prices are $\vecp^k$, the bang-per-buck of any module in $A^{k}$ is at least that of any module in $\pi^k([n]) \setminus \pi^k([k])$.
    \item $\vecp^k(A^{k}) \leq B$.
    \item for $k<k^*$, $\rank(A^k) = \rank(\pi([k]))$. 
    \item if $i \leq k<k^*$ and $\pi^{k}(i) \in \pi[k] \setminus A^k$ then $\pi^k(i)$ forms a circuit $C$ with $A^k$ where $\pi^k(i)$ has the lowest value on $C$.
    \item $S_{\bar{\vecp}} = A^{k^*}$. In addition, $\SE$ is a maximum weight independent set in matroid $\mathcal I$ restricted to $\pi^0(1), \dots , \pi^0(k^*)$
\end{enumerate}
\end{claim}
\label{sec:weighted_matroid_equilibrium}
\begin{algorithm}
\caption{Algorithm to Compute Equilibrium for Weighted Matroid}
\label{alg:equilibrium_dynamics_weighted} 
\begin{algorithmic}[1]
\State Initialize price $\vecp^0 \gets \vecc$ and $\pi^0$ such that $\frac{\vecv(\pi^0(i))}{\vecc(\pi^0(i))}\geq \frac{\vecv(\pi^0(j))}{\vecc(\pi^0(j))}$ for $i < j$.
\State Initialize $A^0 \gets \emptyset$.
%
%
\For{$\ind = 1, \ldots, n$}

\State Copy $\pi^{\ind-1}, \vec p^{\ind-1}$ into $\pi,\vec p$ for simplicity of notation.
\State Update prices: $\vecp^k(\pi(j)) = \frac{\vecp(\pi(k))}{\vecv(\pi(k))} \cdot \vecv(\pi(j))$ if $j \in A^{k-1} $, otherwise, $ \vecp^k(\pi(j)) = \vecp(\pi(j))$.
\If{$(\pi(\ind) \notin \spn( A^{k-1})$)} \label{line:if_weighted}
    \Comment{Accept $\pi(k)$.}
    \State $A^k \gets A^{k-1} \cup \{\pi(k)\}$
\Else
    \State Let $C^k$ be the unique circuit in $A^{k-1} \cup \{\pi(k)\}$.
    \State Let module $i \in C^k$ has the lowest value.
    \State $A^k \gets (A^{k-1} \cup \pi(k)) \setminus \{i\}$
\EndIf

\State Set $\pi^k$ as bang-per-buck ordering with respect to $\vecp^k$, breaking ties according to $\pi^0$.
\If{$\vecp^k(A^k) > B$} \label{line:budget_check_weighted}
\Comment{Reject last module, update prices, and terminate.}
\State  $A^k \gets A^{k-1}$.
\State Update $\vecp^k(i) \gets \min\left\{\vecv(i) \cdot \frac{B - \vecp^{k-1}(A^{k-1})}{\vecv(A^{k-1})}, \vecp^k(i) \right\}$ for $i \in A^{k-1}$. 
\State \textbf{return} $\vecp^k, \pi^k$
\EndIf
\EndFor

\State Update $\vecp^n(i) \gets \vecv(i) \cdot \frac{B - \vecp^n(A^{n})}{\vecv(T^n)}$ for $i \in A^n$.
\State \textbf{return} $\vecp^n, \pi^n$
\end{algorithmic}
\end{algorithm}
The above invariant leads to the main theorem of this sub section whose proof is delegated to Appendix~\ref{proof_weighted_matroid_exists}.
\begin{theorem}
\label{thm:weighted_matroid_eq_exists}
 Let $\vec {\bar p}$ be the prices computed by Algorithm~\ref{alg:equilibrium_dynamics_weighted} then $\bar{\vecp}$ are equilibrium prices given that ties are broken in favor of modules with higher $\frac{\vec v(\cdot )}{\vec c(\cdot)}$. 
 \end{theorem}
 
\subsection{Quality of Equilibria}
The following is the main theorem of this section. 
\begin{theorem}
\label{thm:weighted_matroid_approx}
Let $\lambda = \max_i \vecc(i) / B$ and $S_{\OPT} \in \argmax \{ S \,:\, \vecc(S) \leq B, S \in \I \}$.
For any $\eps$-equilibrium $\vecp$, we have
\[
    \vecv(S_{\OPT})
    \leq \left( 1 + \frac{2\lambda}{1-3\lambda} + \frac{1+\eps / \lambda B}{1-\lambda - \eps / B} \right)
    \left( 1 + \frac{\lambda}{1-\lambda} \right) \cdot \vecv(S_{\vecp}).
\]
\end{theorem}
In particular, as $\eps, \lambda \to 0$, the approximation ratio approaches $2$. To prove the above theorem, we first prove the following Lemma (proof delegated to Appendix~\ref{proof_of_lemma_poa_high_bpb}) that ensures that for any $\eps$-equilibrium price $\vec p$, there exists an equilibrium price such that all non-selected module with relatively high bang-per-buck sets their price equals to their cost. 

\begin{lemma}
\label{lemma:unselected_lt_k_bid_cost}
Let $\vecp$ be an $\eps$-equilibrium price.
Let $\frac{\vecv(1)}{\vecp(1)} \geq \ldots \geq \frac{\vecv(n)}{\vecp(n)}$ be the bang-per-buck order at $\vecp$ and $S_{\vec p}$ be the set of modules selected by $\greedy$.
If module $i < k^*$ is not selected then there exists an equilibrium price $\bar{\vecp}$ such that (i) $\bar{\vecp} \leq \vecp$, (ii) $\bar{\vecp}(i) \leq \vecc(i) + \eps$, and (iii) $\vecv(S_{\bar{\vecp}}) = \vecv(S_{\vecp})$.
\end{lemma}
Next, we state the key technical lemma that ensures that given any $\eps$-equilibrium price $\vec p$, there exists a non-selected module $q$ at price $\vec p$ with relatively lower bang-per-buck and equilibrium price $\bar p$ such that $\bar p(q) \leq \vec c(q)+\eps$, the price of the rest of modules $\bar p(i)\leq \vec p(i)$, and $v(S_{\vec{\bar p}})\leq v(S_{\vec p})$. The proof of the lemma is technical and delegated to Appendix~\ref{proof_of_key_lemma_poa}.
\begin{lemma}
\label{lemma:unselected_gt_k_to_cost}
Let $\vecp$ be an $\eps$-equilibrium.
Let $\frac{\vec v(1)}{\vec p(1)}\geq \dots \geq \frac{\vec v(n)}{\vec p(n)}$ be the bang-per-buck order at price $\vec p$ and $S_{\vec p}$ be the set of selected modules by $\greedy$ such that $k^*$ be the module with the worst bang-per-buck.
Suppose that $n \geq k^* + 2$ and that if $i < k^*$ is rejected then $\vecp(i) \leq \vecc(i) + \eps$.
Let $q$ be the largest index of an un-selected module such that $\vecp(q) > \vecc(q) + \eps$.
There exists equilibrium prices such that (i) $\bar{\vecp}(i) \leq \vecp(i)$ for $i \geq k^* + 1$, (ii) $\bar{\vecp}(q) \leq \vecc(q) + \eps$, (iii) $\bar{\vecp}(i) \leq \vecc(i) + \eps$ for un-selected modules $i < k^*$, and (iv) $\vecv(S_{\bar{\vecp}}) \leq \vecv(S_{\vecp})$.
\end{lemma}
The above lemma allows us to only focus on analysing the value of equilibria where all non-selected modules are setting their price within $\eps$ additive factor of their cost. This characterizes all the bad equilibria of the pricing game and allows us to complete the proof of the main theorem in Appendix~\ref{proof_of_quality_matroid}.  