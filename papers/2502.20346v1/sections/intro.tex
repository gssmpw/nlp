\section{Introduction}
Our work is inspired by a marketplace model where sellers may participate to provide some product or service for a fee and a budget-constrained buyer whose goal is to purchase from some subset of sellers to maximize some objective function. Our primary motivation is in a computing ecosystem where the sellers may be offering specialized ``modules'' which are accessed through Application Programming Interfaces (API). The sellers would offer a price which must be paid every time a buyer issues an API call to their module. These modules may be specialized for particular tasks and may be built with proprietary data or with proprietary techniques.

We define and study the market design problem inherent in building such an ecosystem. We consider the following setup:
\begin{enumerate}[label=(\alph*),itemsep=0pt,topsep=0pt]
\item Module owners provide their module API at a price per API call.

\item Users submit requests and a budget to a central platform.

\item The platform calls a feasible subset of available APIs, respecting the budget constraint.

\item The platform aggregates the results which is provided back to the user.
\end{enumerate}

One particular example of such an ecosystem stems from the recent development of large language models (LLMs). While there may be a few flagship models that boast multiple generic capabilities, we are also witnessing a rise in specialized LLMs tailored for specific domains such as code generation (e.g., \cite{li2022competition, roziere2023code, guo2024deepseek}), medical knowledge (e.g., \cite{singhal2023large, singhal2023towards, luo2022biogpt}), legal tasks (e.g., \cite{wehnert2022applying, shui2023comprehensive}), and mathematical problem-solving (e.g., \cite{trinh2024solving}). In such an ecosystem, one can imagine that a user first enters a prompt into the generic LLM along with a budget constraint. Next, the generic LLM produces an answer by combining results from a subset of the available modules that respects the budget constraint.
Another example that fits into our marketplace model are labor markets such as Upwork \cite{upwork}. In such a market, the platform may allow workers to post hourly wages and then buyers need to assemble teams based on the workers' skills and a total budget.

The marketplace problem in this paper can also be thought of as a procurement auction. More specifically, the mechanism we consider in this paper is some type of a first-price auction. The vast majority of the literature on procurement auctions focuses on designing mechanisms that are truthful \citep{singer2010budget,chen2011approximability, balkanski2022deterministic, han2024triple, ensthaler2014dynamic, jarman2017ex}. This necessitates having a centralized platform that determines the price of every module depending on the prices of every other module. In contrast, a marketplace can be more decentralized; the price of every module can be chosen more or less independently.
One work that studies first-price procurement auctions is \cite{milgrom2020clock}.
\citet{milgrom2020clock} show that descending auctions, which are truthful, can be converted to first-price auctions.
However, one issue with such a reduction, as pointed out by \citet{milgrom2020clock}, is that these could produce additional equilibria for which there are no guarantees on their efficiency.
Thus we can also view our work as understanding what types of allocation ensure that a first-price auction contain only (approximately) efficient equilibria.
We discuss more about procurement auctions in Section~\ref{subsec:related}.

\subsection{Model and Results}
\label{subsec:model_and_results}
In this model, API prices are set by the module owners.  These prices depend on various factors.
First, module owners incur costs for module production and maintenance, including data licensing, partnerships, engineering, and compute. The module owner thus might have a private cost $c$ per API call that it wishes to recoup. The module owner can post a price $p$ per API call with the goal of optimizing its profit of $p - c$, whenever the API is called. 

Depending on a particular request, the different modules may be complementary or substitutable. Going back to the modular LLM example in the introduction, it might be that the user issued a prompt to understand treatment options for some condition. A medical module may be used to help diagnose the patient while a legal module may be useful to clarify insurance policies. However, multiple medical or legal modules may provide limited value and thus are substitutable. We assume a value function that captures the value from any subset of modules. We also assume a budget $B$ on the total that can be spent on any request. The goal is to pick a subset of modules to optimize the value subject to the budget constraint.

The value function and the budget create competition between the module owners and the module owners must strategically set their prices to maximize their profit while ensuring that they are selected. Our results concern the equilibria of this price setting game, where module owners facing repeated queries from the users do not wish to deviate from the API price they have set. 

We would like to understand the existence, potential inefficiency, and learnability of such equilibria. 
Our benchmark for inefficiency is $\opt$, which is the value obtained by the optimal allocation when each module owner sets their price equal to their cost $c$.
Our first, fairly trivial, result shows that the most natural selection rule fails miserably.
\begin{result}
If, given a set of prices, the platform always selects the optimal set of modules subject to the budget constraint then there is an equilibrium where all modules set prices equal to the budget and the user can obtain only one module. 
\end{result}
To see this, imagine that there are $n$ modules who all have cost $0$ and our budget is $1$.
Module $1$ has value $1+\eps$ while all other modules have value $1$.
The optimal value is $n+\eps$ but if all modules set a price of $1$ then we only obtain module $1$ for a value of $1+\eps$.
Moreover, no module can deviate by itself to get selected so this is, in fact, an equilibrium.

At this point, the most pressing question is whether or not there even exists an allocation rule which guarantees that all equilibria are efficient.
A second question is whether or not the allocation can be computed efficiently.
Our main result is that there is an efficient algorithm, based on a natural greedy algorithm, that guarantees that all equilibria are approximately optimal.
This result holds for a large set of value functions generalizing additive values, as described next.

\paragraph{Generalized Additive Valuations}
We assume that the value for a set of modules is additive over the set of selected modules. 
Further, we also assume that there is a matroid constraint that specifies which subset of modules is feasible to select; this captures substitutability between modules.
Matroid feasibility constraints capture many diverse settings.
For instance, a linear matroid models scenarios where a set of modules $S_1$ ``covers'' the content of another set $S_2$ 
(e.g., $S_2$ is linearly dependent on $S_1$), rendering $S_2$ redundant. A partition matroid, a special case, divides modules by expertise, allowing only one module per partition (e.g., medical or legal modules).

We consider the natural ``bang-per-buck'' algorithm, $\greedy$, which works as follows.
We first sort the modules from highest to lowest bang-per-buck (i.e.~$\frac{\vec v(i)}{\vec p(i)}$).
Then we select modules in this order while maintaining both budget and matroid feasibility.
The stopping rule of this algorithm is the \emph{first} time that the budget constraint is violated.
We give a precise definition of this algorithm can be found in Algorithm~\ref{alg:greedy}.

Our first main result, stated informally here, is that for such a selection rule, equilibria exist.
\begin{result}
\label{result:eq_exists}
When the platform implements $\greedy$, equilibria exists where no seller can profitably deviate.
\end{result}
We note that our formal results prove the existence of ``$\eps$-equilibria'' where no module can gain more than $\eps$ utility by deviating.
This is mainly to deal with tie-breaking and we can make $\eps$ arbitrarily small.
The formal results are stated in Theorem~\ref{theorem:additive_eq_exists} (for only a budget constraint), Theorem~\ref{thm:unweighted_eq_exist} (for uniform values with matroid and budget constraints), and Theorem~\ref{thm:weighted_matroid_eq_exists} (for general additive values with matroid and budget constraints).

Our second main result is that all equilibria are efficient.
\begin{result}
\label{result:approx}
Suppose that the platform implements $\greedy$ for module selection and that the cost of any module is at most $\lambda B$ for some $\lambda < 1$ (that is, it is small relative to the budget).
Let $\OPT$ be the maximum value achievable when the costs are \emph{known}.
Then any equilibrium yields a value of at least:
\begin{itemize}
\item $\frac{(1-\lambda)^2}{2 - \lambda} \cdot \OPT$ when there is only a budget constraint (Theorem~\ref{thm:additive_approx});
\item $\frac{(1-\lambda)^2}{2 - \lambda} \cdot \OPT$ when there is a matroid and a budget constraint but the values are uniform across all modules (Theorem~\ref{theorem:unweighted_approx}); and
\item $\frac{(1-3\lambda)^2}{2-3\lambda} \cdot \OPT$ when there is a matroid and a budget constraint. For this result, we also require $\lambda < 1/3$ (Theorem~\ref{thm:weighted_matroid_approx}).
\end{itemize}
Note that all bounds approach $\frac{1}{2}$ as $\lambda \to 0$.
\end{result}
Note that the last approximation result stated above is a simplified but slightly worse version of the result in Theorem~\ref{thm:weighted_matroid_approx}.
Result~\ref{result:approx} only shows that all equilibria are good.
However, it does not show how the market can achieve such any equilibrium.
Indeed, equilibrium prices are a complex function of all the parameters of the market, including the set of modules $M$, their private costs $\vec c$, the value function $\vec v$ and the underlying constraint over the modules. It is impossible for market participants to compute an equilibrium with no prior knowledge of the parameters. However, one may expect module owners to employ algorithmic techniques to adjust prices through reasonable learning algorithms, especially if they anticipate long-term participation.
Such a learning algorithm would respond to the feedback obtained in past rounds at various price levels without knowledge of the other market participants or even the value function of the platform.

Our next result proves convergence guarantees if all modules use \emph{multiplicative weight update} type no-regret learning algorithms. 

\begin{result}
For any $\eps > 0$, if the platform implements $\greedy$ for module selection and each module, unaware of market parameters, employs a multiplicative weight update-style price learning algorithm, then the price dynamics converges to an $\eps$-equilibrium price with high probability under any arbitrary matroid constraint over the modules. Consequently as $\eps \rightarrow 0$, the total expected value for a query for the platform is at least $\frac{(1-3\lambda)^2}{2-3\lambda} \cdot \OPT$, with high probability.
\end{result}
We note that our convergence results assumes that every round is identical in that the buyer's valuation function does not change over each round. If there are multiple ``types'' of buyers, one can imagine that the type of the buyer is used as context for the learning agents. In such a setting, the problem reduces to the single buyer valuation setup in this paper.

\subsection{Related Work}
\label{subsec:related}
\paragraph{Procurement Auctions.}
Our work fits into the literature of procurement auctions. There is a large body of work on designing procurement auctions subject to a budget constraint that was initiated by \citet{singer2010budget}.
These procurement auctions can either be done via sealed-bid auctions  \cite{singer2010budget,chen2011approximability} or descending clock auctions \cite{balkanski2022deterministic, han2024triple, ensthaler2014dynamic, jarman2017ex}. 

When the buyer's objective function is submodular, \citet{balkanski2022deterministic} designed deterministic clock auctions with an approximation ratio of 4.5, improving upon the approximation ratio of both deterministic and randomized mechanisms from a series of works including \cite{chen2011approximability,jalaly2021simple}. For the special case of submodular functions, \citet{leonardi2017budget} designed a $4$-approximation mechanism for unweighted matroid value functions, and \citet{gravin2020optimal} designed an optimal $(1+\sqrt 2)$-approximate mechanism for additive valuations. In addition, a series of works \cite{anari2018budget,rubinstein2022beyond} designed mechanisms with a small seller assumption $( \lambda \rightarrow 0)$ and obtained an optimal $(1-1/e)$-approximation for additive valuations and a $1/2$-approximation for monotone submodular valuations. Our result on matroid value functions matches the current best-known bound under the small seller assumption.\footnote{In addition, we ensure that all potential price equilibria are $1/2$-approximate.}

In this paper, our goal is to formally understand the marketplace model where module owners simply post a public price for their API call. This is a model that is currently being employed in the industry (e.g.~\cite{togetherpricing, vertexaipricing}) due to its simplicity. Equivalently, our model is a first-price procurement auction with a budget constraint. In such a marketplaces, since the payment is decided by the sellers, the mechanism needs to ensure that all potential equilibria prices lead to efficient outcomes which is not necessarily satisfied by existing mechanisms.

Such a marketplace setting was formally studied by \citet{milgrom2020clock} in their beautiful work which proposed a clean characterization of descending clock auctions.
\citet{milgrom2020clock} prove that the outcome of a descending clock auction can be converted to a first-price auction.
Moreover, the outcome of the descending clock auction corresponds to equilibrium bids in the first-price auction.
However, the allocation rule in the first-price auction is somewhat complex as it is defined by ``simulating'' a clock auction. A larger issue is that the conversion they describe may result in additional equilibria unless the allocation satisfies a ``non-bossy'' assumption \citep[Appendix E]{milgrom2020clock}. This assumption requires that a bidder cannot change the allocation of any other bidder without changing their own allocation. This is a non-trivial assumption to satisfy. For example, the allocation in \cite{balkanski2022deterministic}, which achieves the best approximation ratio for procurement auctions, does not satisfy the non-bossy assumption. If the allocation rule does not satisfy the non-bossy condition then the additional equilibria may have poor efficiency. In fact, it is not difficult to observe that their allocation rule leads to an equilibria that obtains $O(1/n)$ fraction of optimal value for the same example for which the optimal selection rule leads to inefficient outcome. Thus a contribution of our work can also be seen as designing allocation rules where all equilibria in a first-price auction are approximately efficient.


In another previous work, \citet{immorlica2005first} study a first-price auction in a procurement auction setting, without a budget constraint.
They study path auctions where the agents correspond to edges and the constraint is a set of edges between a source and a sink with a specified flow capacity. This makes it a covering problem whereas we consider a packing problem.
In addition, the goal of the paper is to minimize the buyer's cost while our goal is to maximize the buyer's value.

Having a marketplace model also has practical implications in that it obviates the need for complex auction infrastructure such as second-price auctions or clock auctions.
While truthful reporting is no longer optimal, our results show that if the module owners simply implement some reasonable learning algorithm, they are guaranteed to converge to an equilibrium.
This essentially allows module owners to use learning algorithms to adjust their prices based on demand feedback.
\paragraph{Pricing and Bidding Dynamics.}
One of the goals of this paper is to understand pricing dynamics in markets.
For Arrow-Debreu markets \cite{mckenzie1954equilibrium,arrow1954existence}, there is a large literature on understanding the dynamics that lead to equilibria (see for example, \cite{rabani2021invisible,cheung2013tatonnement} and references therein).
However, the models studied in these papers assume that the goods are a finite resource and are divisible. In our setting, the goods are digital and are indivisible.
Furthermore, these papers assume that agents do not directly control prices, which are set in a centralized manner based on aggregate supply and demand. In contrast, our model allows agents to directly change their prices based on the response from the market.

Our work also fits into a growing literature of understanding bidding dynamics in auctions \cite{feng2021convergence,deng2022nash,kolumbus2022auctions}.
These papers assume that bidders use some sort of no-regret or mean-based learning algorithm.
\citet{feng2021convergence, deng2022nash, kolumbus2022auctions} aim to understand the convergence properties of these dynamics. \citet{kolumbus2022auctions} also study the incentive properties of different auctions when an agent uses no-regret learning algorithms.
Our paper also assumes that agents use no-regret learning algorithms to adjust their prices.
However, a key difference is that, to the best of our knowledge, our work is the first to study no-regret dynamics in a procurement setting.

Our convergence result falls within the "learning in games" literature \cite{hannan1957approximation,giannou2021survival,hart2000simple}, which considers the setting where players have no information about game parameters and interact with the platform over multiple iterations, using an online learning algorithm to select actions from a finite set. Crucially, here the game does not change over the interactions. Our convergence result is interesting because \cite{giannou2021survival} showed that if players implement a "follow the regularized leader" type learning algorithm (which includes multiplicative weight update), then the dynamics converge to a strict-pure Nash equilibrium. However, in our setting, strict pure Nash equilibria do not exist.  Instead, the "stability" properties of our allocation rule and equilibrium prices ensure that the learning sellers converge to a pure equilibrium price.


\section{Preliminaries} \label{sec:prelim}
\paragraph{Notation.}
We will generally use bolded letters such as $\vecv$ to denote vectors.
For an index $i$, we use $\vecv(i)$ to denote the $i$th coordinate of $\vecv$ and $\vecv_{-i}$ to denote the vector $\vecv$ without the $i$th coordinate.
For a set $S$ of indices, we use $\vecv(S) = \sum_{i \in S} \vecv(i)$.
For two vectors $\vecv^1, \vecv^2$, we write $\vecv^1 \leq \vecv^2$ to indicate that $\vecv^1(i) \leq \vecv^2(i)$ for all indices $i$ (and similarly for all other comparisons). For any positive integer $n$, we let $[n] = \{1,\dots , n\}$.
We use $\pi \colon [n] \to [n]$ to denote a permutation where $\pi(i)$ denotes the element at index $i$.
For a set $S \subseteq [n]$, we write $\pi(S) = \{ \pi(i) \,:\, i \in S\}$. We also denote $\pi[k] = \pi([k])$.
In the remainder of the paper, we will generally refer to a module only via its index $i$ instead of $m_i$. We provide basic preliminaries of matroid theory in Appendix~\ref{appendix:matroid_prelims}.


\paragraph{Price Competition Game}
Throughout this paper, we use buyer interchangeably with the platform and a seller interchangeably with a module owner.
An instance of the price competition game is defined by $\langle M, \vec v, \vec c, \mathcal I, B \rangle $ where $M= \{m_1, \dots, m_k\}$ are the set of modules, $\vec v(i)\in \mathbb R_+\cup \{0\}$ is the value of module $m_i$ for the buyer, $\vec c(i) \in \mathbb R_+\cup \{0\}$ is the cost of module $m_i$, $\mathcal I\subseteq 2^M$ is the set of feasible modules for the buyer, and $B$ is the buyer's budget.  

For the given instance of a price competition game $\langle M, \vec v, \vec c, \mathcal I, B \rangle $, we assume that each module owner has full information of the instance including the values $\vec v$, costs $\vec c$ and constraints $\mathcal I$. We do note that full information about the valuation function of the buyer, constraint $\mathcal I$, and cost of other module owners is unrealistic. However, understanding the price competition game in the full-information setting already turns out to be non-trivial and acts as a building block towards understanding the modular marketplace where the module owners do not have any information about the price competition instance except their private costs. In addition, we assume that the buyer does have the information about the values $\vec v$, constraint $\mathcal I$, and budget $B$ but does not have any information about the costs of the modules $\vec c$.

Given an instance of a price competition game $\langle M, \vec v, \vec c, \mathcal I, B\rangle $, the strategy space of the module owner $m_i\in M$ is a price of their module $\vec p(i) \in [\vec c(i), B]$.\footnote{
Note that $\vecp(i) < \vecc(i)$ is always dominated by setting a price of $\vecc(i)$ since bidding less than $\vecc(i)$ always results in non-positive utility for the seller.}
We let $\mathcal A \colon \mathbb R_+^M \rightarrow 2^M$ be the selection rule for the buyer that takes the prices of the modules $\vec p$ as an input and outputs a set of modules such that for any price vector $\vec p$, we have (a) $\sum_{i\in \mathcal A(\vec p)} \vec p(i) \leq B$ and (b) $\mathcal A(\vec p) \in \mathcal I$. 

For the given price vector $\vec p$, we let $u_i^\mathcal A (\vec p) = ( \vec p(i) - \vec c(i)) \cdot \mathbbm 1[i\in \mathcal A(\vec p)]$ be the utility of module $i$ when the buyer implements selection rule $\mathcal A$.
We say that $\vec p$ is an equilibrium price for the price competition game if no module can deviate from its posted price and increase its utility, i.e.~for all $i\in [k]$ and $\vec p(i)' \neq \vec p(i)$, we have,
$u_i(\vec p(i)',\vec p_{-i}) \leq u_i (\vec p(i), \vec p_{-i})$.  Similarly, for any $\eps >0$, we say that the prices $\vec p$ is an $\eps$-equilibrium price if  for all $i\in [k]$ and $\vec p(i)' \neq \vec p(i)$, we have,
$u_i(\vec p(i)',\vec p_{-i}) \leq u_i (\vec p(i) , \vec p_{-i}) + \eps$.

To evaluate the quality of the selection rule $\mathcal A$, we compare the value obtained by the buyers at any equilibrium to an omniscient benchmark $\opt$ which is the best possible value achieved by the buyer in the case when it knows the costs of the modules $\vec c$, i.e. $\max_{S\in \mathcal I, \vec c(S)\leq B} \vec v(S).$ We say that allocation rule $\mathcal A$ has an \emph{approximation ratio} of $\alpha \in [0,1]$ if for any equilibrium price $\vec p$ for the price competition instance $\langle M, \vec v, \vec c, \mathcal I, B \rangle $ with selection rule $\mathcal A$, we have $\sum_{i\in \mathcal A(\vec p)} \vec v(i) \geq \alpha \cdot \opt$.

\paragraph{Learning Dynamics in the Price Competition Game}
In the above discussion, we defined a price competition game between the modules where we assume that all the modules possess full information about the game instance  $\langle M, \vec v, \vec c, \mathcal I, B\rangle $.  However, the cost of each module is private information, i.e. only module $m_i$ has information of their private $\vec c(i)$, and the rest of the modules are unaware of the private cost of module $m_i$. In addition, the modules have no information about the value vector $\vec v$ and underlying constraints $\mathcal I$. As a result, without full information about market parameters, the modules cannot implement equilibrium prices directly.

In this section, we formally define the learning dynamics of the price competition market, where each module owner employs a learning algorithm to set their prices. More specifically, we fix an instance of the price competition game $\langle M, \vec v, \vec c, \mathcal I, B\rangle $, and we consider a setting where a single platform repeatedly interacts with the modules over $T$ rounds.  At each round, a price competition game instance $\langle M, \vec v, \vec c, \mathcal I, B\rangle $ is played. 

We assume that module $m_i\in M$ implements an \emph{online learning algorithm} $\mathcal L(i)$ to set its price.
At round $t$, the algorithm selects a price $\vec p^t(i)$ (possibly at random) from the set $\mathcal B = \{\delta, 2\cdot \delta , \dots ,B\}$,
where $\delta \in (0, 1)$ represents the minimum price increment.
The price $\vec p^t(i)$ at round $t$ is determined based on the history of the game up to round $t-1$. At each round $t$, the platform provides feedback to module $m_i$ regarding the maximum price at which the module could have been selected in that round.
We denote the history at round $t$ for module $m_i$ as the set of prices, the set of selected modules, and the received feedback from the platform up to round $t-1$.

The goal of each module's learning algorithm $\mathcal L(i)$ is to adjust its price iteratively based on the feedback received, with the aim of converging towards an optimal pricing strategy. In particular, the platform's feedback provides each module with information that helps refine its price selection to maximize its reward over time. This repeated interaction across $T$ rounds allows modules to "learn" about the market conditions and the behavior of competitors, even with incomplete information about other modules' costs and the platform's value structure.

Given the price competition game instance $\langle M, \vec v, \vec c, \mathcal I, B\rangle $ and learning algorithms of the modules $\{\mathcal L(i) \,:\, i\in M\}$ induces a price dynamics denoted as a sequence of price vectors $\{\vec p^t \,:\, t=1,2,\dots\}$. We next define convergence of the price dynamics of the price competition game which is a generalization of the standard definition of convergence of the sequences. 
\begin{definition}\label{def:multiplicative_weight_learning}
The learning dynamics of the price competition game instance $\langle M, \vec v, \vec c, \mathcal I , B \rangle$ with learning algorithms of the modules $\{\mathcal L(i) \,:\, i\in M\}$ converges to the price vector $\vec p$ if for any $\gamma >0$, there exists $T^* \geq  g(\gamma)$ such that 
\begin{equation*}
    \Pr \left[ |\vec p^t(i) - \vec p(i)|\leq \sqrt \delta: \forall i\in M \text{ and } t\geq T^* \right]\geq 1 - \gamma.
\end{equation*}
\end{definition}
In this work, we consider the dynamics where each module implements \emph{multiplicative weight update} type learning algorithm to set the price which we define in the following definition.

\begin{definition}[Multiplicative Weight Type Learning Algorithm]
Given set of bids $\mathcal B$ and total cumulative reward of price $p\in \mathcal B$ denoted as $\sigma_t(a) = \sum_{t'\leq t} u_i^\mathcal A(p,\vec p_{-i}^t)$, we say that the algorithm $\mathcal L$ is multiplicative weight type with learning rate $\gamma_t$ if for any two prices $p,p' \in \mathcal A$ and some constant $\tau>0$, whenever $\sigma_{t}(p') \leq \sigma_t(p) - \tau \cdot t$ then the probability that the algorithm sets price $p'$ at round $t+1$ is less than $ \exp(-\gamma_t \cdot \tau \cdot t)$.
\end{definition}
