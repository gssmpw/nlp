\section{Warm-up: Price Competition for Additive Buyer}\label{sec:additive_price_equilibrium}

We first present the equilibrium quality analysis for the case when the buyer's value is an additive function over the set of modules.
We consider the greedy algorithm that is presented in Algorithm~\ref{alg:knapsack} which is just Algorithm~\ref{alg:greedy} for the unconstrained case, i.e.~$\I = 2^M$.
The algorithm first sorts all modules in terms of their ``bang-per-buck'' $\vecv(i) / \vecp(i)$ where $\vecv(i)$ is the value of module $i$ and $\vecp(i)$ is the submitted bid by module $i$.
It then selects modules in decreasing bang-per-buck order and exits as soon as it finds a module which does not fit into the budget without taking that module.

\begin{algorithm}
\caption{Greedy Algorithm for Knapsack Problem}
\label{alg:knapsack}
\begin{algorithmic}[1]
\State \textbf{Input}: price vector $\vec p$, values $\vec v$, budget $B$, \textbf{Output}: set $G$ of modules.
\State Initialize $G \gets \emptyset$.
\State Re-arrange modules such that $\frac{\vecv(1)}{\vecp(1)} \geq \frac{\vecv(2)}{\vecp(2)} \geq \dots \geq \frac{\vecv(n)}{\vecp(n)}$ with ties broken arbitrarily.
\For{$i=1,2 \dots ,n$}
\If{$\vec p(G) + \vec p(i) > B$}
\State \textbf{break}
\EndIf
\State $G \gets G \cup \{i\}$
\EndFor
\State \textbf{return} $G$
\end{algorithmic}
\end{algorithm}

In Subsection~\ref{subsec:eq_exists}, we show that $\eps$-equilibria exist when the buyer uses $\greedy$.
Note that some slack in the equilibrium notion is required.
In Subsection~\ref{subsec:need_eps_eq}, we show that exact equilibria can be very brittle and in fact, it may not exist.

In Subsection~\ref{subsec:eq_quality}, we then analyze the quality of all $\eps$-equilibria.
We show that all $\eps$-equilibria obtain at least half of the optimal solution when costs are known provided that (i) the cost of all modules relative to the budget is small and (ii) $\eps$ is small relatively to the cheapest module.
The first condition is similar to the ``small-bid'' assumption that appears in the literature (e.g.~\cite{mehta2007adwords}).
The second condition turns out to be necessary to prove an approximation result since there could be many very cheap and low value modules, which in aggregate, could contribute a significant amount of value.
Details of this necessity can be found in Subsection~\ref{subsec:need_cost_lb}.

\subsection{Existence of Equilibrium}
\label{subsec:eq_exists}
In this section, we prove the existence of an $\eps$-equilibrium.
To slightly simplify the exposition, we assume that the ratio $\frac{\vecv(i)}{\vecc(i)}$ is distinct for all modules.
If this is not the case we can perturb the value of each module by adding an independent random value drawn uniformly from the interval $[0, \eps]$ so that the aforementioned ratio is almost surely distinct for all modules.
In the argument below, this means that each module may actually be able to increase its utility by at most $\eps$ more so the argument below finds a $2\eps$-equilibrium instead of an $\eps$-equilibrium.
Henceforth, we assume $\frac{\vecv(i)}{\vecc(i)}$ are all distinct.
To simplify notation, in this section we sort modules in decreasing order of the bang-per-buck value w.r.t.~their cost, i.e.~$\frac{\vec v(1)}{\vec c(1)} > \frac{\vec v(2)}{\vec c(2)} > \dots > \frac{\vec v(n)}{\vec c(n)}$.
Let
\begin{equation}
    \label{eqn:cpv_additive}
    \cpv^* = \sup\left\{ \cpv \in \R \,:\, \sum_{i=1}^n \vecv(i) \cdot \cpv \cdot \indicator[\vecc(i) \leq \vecv(i) \cdot \cpv] \leq B \right\}.
\end{equation}
The quantity $\cpv^*$ is the smallest cost-per-value such that all modules that are taking all modules that are willing to accept the price given by this cost-per-value already exceeds our budget.
We claim that $\cpv^*$ essentially defines the equilibrium price.
\begin{theorem}
\label{theorem:additive_eq_exists}
Let $\cpv^*$ be as defined in Eq.~\eqref{eqn:cpv_additive}.
For any $\eps > 0$, there exists $\delta \geq 0$ such that the price vector $\vecp$ defined as $\vecp(i) \coloneqq \max\{ (\cpv^* - \delta) \cdot \vecv(i), \vecc(i) \}$ is an $\eps$-equilibrium.
\end{theorem}
\begin{proof}
Let $\cpv_i = \frac{\vecc(i)}{\vecv(i)}$.
Recall that we assume $\cpv_1 < \ldots < \cpv_n$.
We have two cases.
\paragraph{Case 1: $\cpv^* = \cpv_k$ for some $k \in [n]$.}
In this case, we choose $\delta$ satisfying (i) $\delta > 0$, (ii) $\delta < \cpv^* - \cpv_{k-1}$, and (iii) $\delta \vecv(i) \leq \eps / 2n$ for all $i \in [n]$.
Note that, by (ii), $\vecp(i) > \vecc(i)$ for all $i < k$ and $\vecp(i) = \vecc(i)$ for all $i \geq k$.
By definition of $\cpv^*$, $\vecp([k-1]) \leq B$ so modules $1$ through $k-1$ are accepted.

We need to show that if any module raises their bid by $\eps$ then it is rejected.
Let $i \in [n]$ and suppose module $i$ deviates to $\vecp'(i) = \vecp(i) + \eps$.
Note that
\[
    \vecp(i) + \eps \geq \cpv_k \cdot \vecv(i) - \delta \cdot \vecv(i) + \eps > \cpv_k \cdot \vecv(i)
\]
because $\delta \vecv(i) \leq \eps / 2n < \eps$.
So module $i$, with its deviation, comes after module $k$ (or is module $k$) in the cost-per-value ordering.
In that case, the budget utilization by $\greedy$ up to and including $i$ is at least
\begin{align*}
\vecp([k]) \geq \cpv_k \vec([k]) - \delta \vecv([k]) + \eps
\geq B - \eps / 2 + \eps > B.
\end{align*}
In particular, module $i$ is not accepted.
Thus any module $i < k$ cannot increase its bid by at least $\eps$ while being accepted and any module $i \geq k$ cannot choose a bid to achieve utility at least $\eps$.

\paragraph{Case 2: $\cpv^* \neq \cpv_k$ for all $k \in [n]$.}
Let $k = \max\{i \,:\, \cpv_i < \cpv^*\}$.
By definition of $\cpv^*$ and $k$, this means that for the price vector $\vecp$ given by $\vecp(i) = \max\{\cpv^* \vecv(i), \vecc(i)\}$, we have that $\greedy$ picks modules $1$ through $k$ and $\vecp([k]) = B$.
No accepted module $i \leq k$ can increase its bid by more than $\eps$ because doing so would make it go after $[k] \setminus \{i\}$ and including it would exceed the budget constraint.
On the other hand, no rejected module $i > k$ can decrease its bid because it is already bidding its cost.
We thus conclude that $\vecp$ is already an equilibrium price vector.
\end{proof}

\newcommand{\additiveiterstikzscale}{0.6}

\subsection{Equilibrium Quality}
\label{subsec:eq_quality}
Lemma~\ref{lemma:worst_equil} is the main structural result needed in this section which, at a high level, says that the worst equilibrium is achieved when rejected modules are bidding near their cost.
The high-level idea of the proof is illustrated in Figure~\ref{fig:worst_equil_sketch} which illustrates how rejected modules may lower their price to cause other modules to be rejected.
The proof is relegated to Appendix~\ref{app:proof_worst_equil}.
\begin{lemma}
\label{lemma:worst_equil}
Let $\vec p$ be a vector of $\eps$-equilibrium prices and let $S_{\vec p}$ be the set of modules selected by $\greedy$.
Fix an arbitrary $i \notin S_{\vec p}$.
There exists $\eps$-equilibrium prices $\vec{\bar{p}}$ such that (i) $\vec{\bar{p}}(i) \leq \vec{c}(i) + \eps$, (ii) $S_{\vec{\bar{p}}} \subseteq S_{\vec{p}}$, and (iii) $\bar{\vecp} \leq \vecp$ (coordinate-wise).
\end{lemma}
The following lemma shows that running $\greedy$ on any equilibrium yields roughly half of the value compared to running $\greedy$ on an instance where the costs are known. The proof is delegated to Appendix~\ref{proof_eps_additive}
\begin{lemma}\label{lem:additive_eps_eq_deviation}
Let $0 < m < M \leq B$ and $\lambda = M / B$.
Suppose that the cost of every module is in $[m, M-\eps]$.
Let $\vecp$ be any $\eps$-equilibrium and let $S_{\vecp}$ (resp.~$S_{\vecc}$) be the set returned when using $\greedy$ on $\vecp$ (resp.~$\vecc$).
Then $\frac{\vecv(S_{\vecc})}{\vecv(S_{\vecp})} \leq 2 + \frac{\eps}{m} + \left( 1 + \frac{\eps}{m} \right) \cdot \frac{\lambda}{1-\lambda}$.
\end{lemma}
In particular, if the largest cost $M$ is small compared to the budget then as $\eps \to 0$, the solution obtained in an $\eps$-equilibrium becomes a $2$-approximation to the solution obtained via $\greedy$ when the costs are known a priori.
Next, we show that $S_{\vecc}$ is itself a good approximation to $S_{\OPT}$.
The proof is standard and relegated to Appendix~\ref{app:additive_greedy_approx_with_cost}.
\begin{lemma}
\label{lemma:additive_greedy_approx_with_cost}
Let $S_{\OPT} \in \argmax \{ \vecv(S) \,:\, \vecc(S) \leq B \}$ be an optimal bundle.
Suppose that $\vecc(i) \leq M$ for every $i \in [n]$.
Let $\lambda = M / B$.
Then $\frac{\vecv(S_{\OPT})}{\vecv(S_{\vecc})} \leq 1 + \frac{\lambda}{1-\lambda}$.
\end{lemma}


\begin{theorem}
\label{thm:additive_approx}
Let $0 < m < M \leq B$ and let $\lambda = M / B$.
Suppose that the cost of every module is in $[m, M-\eps]$.
Let $\vecp$ be any $\eps$-equilibrium and let $S_{\vecp}$ be the set returned when using $\greedy$ on $\vecp$.
Let $S_{\OPT} \in \argmax \{ \vecv(S) \,:\, \vecc(S) \leq B \}$ be an optimal bundle.
Then
\[
    \frac{\vecv(S_{\OPT})}{\vecv(S_{\vecp})}
    \leq
    \left(2 + \frac{\eps}{m} + \left( 1 + \frac{\eps}{m} \right) \cdot \frac{\lambda}{1-\lambda}\right) \cdot \left(1 + \frac{\lambda}{1-\lambda} \right).
\]
\end{theorem}
As $\eps \to 0$, the above ratio becomes $\frac{2-\lambda}{(1-\lambda)^2}$.
Thus as $\lambda \to 0$ (i.e.~the cost of every module becomes small relative to the budget), we get a $2$-approximation.