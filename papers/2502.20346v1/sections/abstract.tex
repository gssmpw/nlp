We envision a marketplace where diverse entities offer specialized ``modules'' through APIs, allowing users to compose the outputs of these modules for complex tasks within a given budget. This paper studies the market design problem in such an ecosystem, where module owners strategically set prices for their APIs (to maximize their profit) and a central platform orchestrates the aggregation of module outputs at query-time. One can also think about this as a first-price procurement auction with budgets. The first observation is that if the platform’s algorithm is to find the optimal set of modules then this could result in a poor outcome, in the sense that there are price equilibria which provide arbitrarily low value for the user. We show that under a suitable version of the ``bang-per-buck'' algorithm for the knapsack problem, an $\eps$-approximate equilibrium always exists, for any arbitrary $\eps > 0$. Further, our first main result shows that with this algorithm any such equilibrium provides a constant approximation to the optimal value that the buyer could get under various constraints including (i) a budget constraint and (ii) a budget and a matroid constraint. Finally, we demonstrate that these efficient equilibria can be learned  through  decentralized  price  adjustments  by  module  owners  using  no-regret  learning  algorithms.