\documentclass[final,3p,times,onecolumn]{elsarticle}
\usepackage{amsmath}
\usepackage{amsfonts}
\usepackage{amssymb}
\usepackage{graphicx} 
\usepackage{subcaption}
\usepackage{epstopdf}
\usepackage{bm}
\usepackage{lineno}
\usepackage{listings}
\usepackage{color, soul}
\usepackage[table,xcdraw]{xcolor}
\newcommand{\argmax}{\operatornamewithlimits{argmax}}
\usepackage{multirow}
\usepackage{booktabs}
\biboptions{numbers,sort&compress}
\setlength{\parskip}{1em}
\usepackage{xr-hyper}
\usepackage{hyperref}
\externaldocument[A-]{updated_main}
\usepackage{placeins} % Add this in the preamble

\usepackage{gensymb}
\usepackage{float}
\hypersetup{colorlinks,breaklinks,
            urlcolor=Maroon,
            linkcolor=Maroon}
\hypersetup{pdfauthor={Name}}
\journal{Digital Discovery}
\usepackage{xcolor} % Optional, for color customization

% Configure the listings package
\lstset{
    language=Python,
    basicstyle=\ttfamily\small, % Use small monospace font
    keywordstyle=\color{blue}, % Keywords in blue
    commentstyle=\color{gray}, % Comments in gray
    stringstyle=\color{red}, % Strings in red
    numbers=left, % Line numbers on the left
    numberstyle=\tiny\color{gray}, % Line numbers styling
    stepnumber=1, % Line numbers step
    numbersep=5pt, % Distance of line numbers from the code
    backgroundcolor=\color{white}, % Background color for the code block
    showspaces=false, % Don't show spaces in the code
    showstringspaces=false, % Don't show spaces in strings
    showtabs=false, % Don't show tabs in the code
    frame=single, % Frame around the code block
    tabsize=4, % Tab size
    captionpos=b, % Position of the caption
    breaklines=true, % Automatic line breaking
    breakatwhitespace=true, % Break at whitespace
    escapeinside={\%*}{*)} % Escape inside code for LaTeX commands
}
% Define a new floating environment for listings
\floatstyle{ruled}
\newfloat{listing}{tbp}{lop}[section]
\floatname{listing}{Listing}

\begin{document}
\title{Supplemental Material: Physics-Informed Gaussian Process Classification for Constraint-Aware Alloy Design}

\begin{frontmatter}


\author[1]{Christofer Hardcastle}
\author[1]{Ryan O'Mullan}
\author[1,2,3]{Raymundo Arr\'oyave}
\author[1]{Brent Vela \corref{cor1}}

\address[1]{Department of Materials Science and Engineering, Texas A\&M University, College Station, TX 77843, USA}
\address[2]{J. Mike Walker '66 Department of Mechanical Engineering, Texas A\&M University, College Station, TX 77843, USA}
\address[3]{Wm Michael Barnes '64 Department of Industrial and Systems Engineering, Texas A\&M University, College Station, TX 77843, USA}



\end{frontmatter}

In Section 3.4 of the main text, we benchmarked the predictive ability of the proposed method against two control methods. Specifically, the proposed approach was a physics-informed GPC, while the control methods were: (1) a GPC with a constant prior mean function and (2) Thermo-Calc CALPHAD predictions using the TCHEA6 thermodynamic database \cite{tchea6}. For benchmarking, we employed stratified Monte Carlo cross-validation, generating 500 random train/test splits with an 80\%/20\% ratio. To evaluate overall predictive performance across all classes in the multi-class setting, we presented box-and-whisker plots of each error metric based on these cross-validation splits, considering the four-class case. In this Supplemental Material, we extend this analysis to one-vs-rest classifiers, reporting error metric distributions for distinguishing each individual class from all others. Figures \ref{fig:dist_BCC}-\ref{fig:dist_FCCsec} summarize the results for the following one-vs-rest cases:
\begin{itemize}
    \item Single Phase BCC (BCC)
    \item BCC with Secondary Phases (BCC+Sec.)
    \item Single Phase FCC (FCC)
    \item FCC with Secondary Phases (FCC+Sec.)
\end{itemize}

When predicting single-phase BCC and BCC with secondary phases (Figures \ref{fig:dist_BCC} and \ref{fig:dist_BCCsec}), the model with a physics-informed prior outperformed both the standard model and TC in precision, recall, and F1-score, with a slight improvement in precision. However, log loss scores were similar between the physics-informed and standard models, while the physics-informed model performed slightly worse in terms of Brier loss. Overall, the proposed method outperformed the control models in 4 out of 6 error metrics, with its only shortcomings in the probabilistic error metrics, where vanilla GPCs performed slightly better. When considering all error metrics simultaneously, it is evident that the physics-informed GPCs perform the best.

For FCC alloys (Figure \ref{fig:dist_FCC}, the median accuracy, recall, and F1 of the physics-informed GPCs is comparable to that of Thermo-Calc. However, Thermo-Calc exhibits a narrower accuracy, recall, and F1 distributions, indicating greater consistency in distinguishing between single-phase FCC and non-FCC phases. The percision distribution of the physics-informed GPCs outperforms that of the Thermo-Calc. The physics-uninformed GPC has the worst performance with regard to classification these deterministic classification metrics. Interestingly, Thermo-Calc excels with regards to deterministic classification metrics, it performs poorly with regards to probabalistic error metrics whereas the GPCs (both informed and uninformed) both perform well. With all metrics considered, the physics-informed GPCs perform the best.
 
For FCC plus secondary phase predictions (Figure \ref{fig:dist_FCCsec}), the median accuracy, recall, Brier loss and Log loss values of the informed GPC were comparable (if not slightly lower) than the uninformed GPC. However, the variance in these error distribution was lower in the case of the informed GPC, indicating more consistent performance. That is to say, in some cases the uninformed model will perform exceptionally well, and in some cases it will perform exceptionally poorly.


To summarize, while the proposed method does not outperform the control methods with regard to every error metric in every one-vs-rest classification senario, when the error metrics are considered holistically, it is evident that the physics-informed GPC outperform uninformed GPCs and CALPHAD predictions.


\begin{figure}[htb!]
    \centering
    \includegraphics[width=0.5\linewidth]{MainFigures/BCC.png}
    \caption{Model errors for the standard GPC (Uninf.), Thermo-calc (TC), and the GPC with the physics-informed prior (Inf.) when predicting for BCC alloys.}
    \label{fig:dist_BCC}
\end{figure}

\begin{figure}[htb!]
    \centering
    \includegraphics[width=0.5\linewidth]{MainFigures/BCC+Sec..png}
    \caption{Model errors for the standard GPC (Uninf.), Thermo-calc (TC), and the GPC with the physics-informed prior (Inf.) when predicting for BCC+Sec. alloys.}
    \label{fig:dist_BCCsec}
\end{figure}


\begin{figure}[htb!]
    \centering
    \includegraphics[width=0.5\linewidth]{MainFigures/FCC.png}
    \caption{Model errors for the standard GPC (Uninf.), Thermo-calc (TC), and the GPC with the physics-informed prior (Inf.) when predicting for FCC alloys.}
    \label{fig:dist_FCC}
\end{figure}


\begin{figure}[htb!]
    \centering
    \includegraphics[width=0.5\linewidth]{MainFigures/FCC+Sec..png}
    \caption{Model errors for the standard GPC (Uninf.), Thermo-calc (TC), and the GPC with the physics-informed prior (Inf.) when predicting for FCC+Sec. alloys.}
    \label{fig:dist_FCCsec}
\end{figure}

\FloatBarrier

%\bibliographystyle{elsarticle-num}
%\bibliography{mybibfile}

\begin{thebibliography}{1}
\expandafter\ifx\csname url\endcsname\relax
  \def\url#1{\texttt{#1}}\fi
\expandafter\ifx\csname urlprefix\endcsname\relax\def\urlprefix{URL }\fi
\expandafter\ifx\csname href\endcsname\relax
  \def\href#1#2{#2} \def\path#1{#1}\fi

\bibitem{tchea6}
\href{https://www.engineering-eye.com/THERMOCALC/details/db/pdf/thermo-calc/2022b/TCHEA6_technical_info.pdf}{Thermo-calc software tchea6 database}, accessed: May 2024.
\newline\urlprefix\url{https://www.engineering-eye.com/THERMOCALC/details/db/pdf/thermo-calc/2022b/TCHEA6_technical_info.pdf}

\end{thebibliography}


\end{document}

