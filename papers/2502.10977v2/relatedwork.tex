\section{RELATED WORK}
Before introducing the \textbf{``Bathroom Model,''} it is essential to contextualize our approach within the broader landscape of hash table optimization research. The study of hash tables has been an integral part of computer science for decades, with foundational contributions from Donald Knuth \cite{knuth1998} and Thomas H. Cormen et al. \cite{cormen2009}, whose works laid the theoretical groundwork for understanding hash table performance and collision resolution.

One of the earliest breakthroughs in hash function design came from Carter and Wegman \cite{carter1977}, who introduced the concept of \textit{universal hashing}. Their work significantly improved hash table efficiency by ensuring that hash functions distribute keys more uniformly, reducing collision probabilities. Another key development was the \textit{dynamic perfect hashing} method by Dietzfelbinger et al. \cite{dietzfelbinger1990}, which allowed hash functions to be dynamically adjusted as table occupancy changed, ensuring stable performance even as data scales.

In more recent advancements, \textit{Cuckoo hashing} was introduced by Pagh and Rodler \cite{pagh2004}, leveraging multiple hash functions and table slots to maintain high load factors while preserving constant-time lookups. This technique has been widely adopted in memory-intensive applications due to its effectiveness. Meanwhile, \textit{Bloom filters}, explored by Broder and Mitzenmacher \cite{broder2003}, have provided an alternative approach to memory-efficient membership queries, further enhancing hash table utility in network applications.

Despite these advancements, most existing hashing techniques—whether traditional probing strategies or modern hashing paradigms—depend on \textit{static or semi-static mechanisms} that do not fully utilize real-time table occupancy information. These limitations prevent optimal performance in highly dynamic environments where lookup and insertion patterns vary significantly over time.

Our \textbf{``Bathroom Model''} aims to address these shortcomings by introducing a \textit{fully adaptive probing mechanism} that dynamically refines search strategies based on observed table states. By drawing on real-world decision-making analogies, such as stall selection behavior in crowded restrooms, our approach provides a more responsive and efficient alternative to conventional hashing techniques. This research demonstrates that adaptive probing can significantly improve search efficiency without incurring substantial computational costs, offering a fresh perspective on optimizing long-established data structures.