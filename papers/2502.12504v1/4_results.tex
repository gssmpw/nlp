\section{Results}

Based on the results from our experiments, we answer our research questions.

\subsection{\underline{RQ1}: Can multi-agent LLM system simulations replicate behaviors observed in PGG lab experiments with human subjects?}

Towards answering this question, we simulate treatments from three different lab experiments with human subjects: (1) priming via game name, (2) transparency of contributions, and (3) variations in endowment. In the first two experiments, LLM-agents replicate the direction of effects with statistical significance, but overestimates the magnitude. In the third experiment, LLM-agents replicate the direction of effects but the effect is not statistically significant, although this is likely due to lack of data. \textbf{Overall, we conclude that LLM-agents are able to replicate the direction of the effect of priming and transparency treatments, \color{orange}but testing their ability to replicate the effect of a varied endowment requires further simulations.\color{black}}

\subsubsection{Does priming LLM agents via game name replicate the effect of priming humans via game name?}

In experiments with human subjects, participants presented the PGG as the "Teamwork Game" contributed roughly 60\% of their endowments while participants presented the game as the "Taxation Game" contributed roughly 40\% of their endowments \cite{Eriksson_Strimling_2014}. Across ten simulations on average, LLM agents presented the PGG as the "Teamwork Game" (Alice and Bob) contributed approximately 70\% of their \$20 endowment. In comparison, agents presented the PGG as the "Taxation Game" (Casey and David) contributed only approximately 30\% of their endowment to the public pool. These results are summarized in Figure \ref{1Agraph}. A t-test shows that the difference in contributions between the two groups is statistically significant at the p < 0.01 level  
(\(t = 7.92, p < 1\text{e}{-8} \)).

\begin{figure}[h]
    \centering
    \includegraphics[width=0.5\textwidth]{images/1A.png}
    \caption{Average contributions for "Teamwork" and "Taxation" priming conditions in simulations with LLM-agents and experiments with human subjects. Average contributions are above 60\% in the "Teamwork" priming condition for both groups. Average contributions are below 40\% in the "Taxation" priming condition for both groups.}
    \label{1Agraph}
\end{figure}

\textbf{From these results, we conclude that priming LLM agents by presenting the PGG under different names replicates the direction of the effect of doing the same on humans, but may overestimate the magnitude.}

\subsubsection{Does introducing transparency in contribution between LLM agents replicate the effect of introducing transparency in contribution between humans?}

In experiments with human subjects, participants contributed on average 6\% more of their endowment under transparency conditions as compared to no transparency. \cite{transparencytwo}. Across five simulations of each condition (ten simulations total) on average, LLM agents playing the PGG with transparency contributed approximately 60\% of their 20\$ endowment, whereas LLM agents playing the PGG without transparency contributed only approximately 35\% of their endowment. \color{orange}These results are summarized in Figure \ref{1Bgraph}\color{black}. A t-test shows that the difference in contributions between the two groups is statistically significant at the p < 0.05 level  
(\(t = 2.23, p = 0.031 \)).

\begin{figure}[h]
    \centering
    \includegraphics[width=0.5\textwidth]{images/1B.png}
    \caption{Average contributions with and without transparency for simulations. Only the difference in contribution is listed in the lab experiment from which this treatment/effect was found. \cite{transparencytwo}.}
    \label{1Bgraph}
\end{figure}

\textbf{From these results, we conclude that introducing transparency of contributions in PGGs with LLM agents replicates the direction of the effect of doing the same on humans, but may overestimate the magnitude.}

% "If controlling for the MPCR, however, as is done in specifications (6) and (8), the positive effect of F.indichoice and the negative effect of F.indipayoffs have a larger magnitude and turn significant at the 5% level. The share contributed tends to be about 6 percentage points higher than in cases where players receive feedback about the distribution of choices in their group."


\subsubsection{Does varying the endowments of LLM agents replicate the effect of varying endowments of humans?}

In experiments with human subjects, it was found "poor" and "medium" wealth individuals contributed roughly the same amount in equal and varied endowment conditions, whereas "rich" individuals contributed more in the equal endowment condition than in the varied endowment condition \cite{HARGREAVESHEAP20164}. Across five simulations each of \$20-\$20-\$20, \$50-\$50-\$50, and \$80-\$80-\$80 equal endowment conditions, LLM agents on average contributed approximately 39\%, 48\%, and 63\% of their endowments, respectively. In five simulations of a \$20-\$50-\$80 varied endowment condition, on average the LLM agent given \$20 contributed approximately 35\%; the LLM agent given \$50 contributed approximately 42\%; and the LLM agent given \$80 contribute approximately 44\%. Thus, the contribution for agents endowed with \$20 was roughly the same between the equal and varied endowment conditions, whereas agents endowed with \$50 and \$80 both contributed less in the varied condition. These results are summarized in Table \ref{1Cgraph}. T-tests shows that there is no statistically significant difference at the p < 0.1 between the contributions of LLM agents endowed with \$20 in the equal and varied endowment condition (\(t = 0.18, p = 0.86 \)), nor for the \$50 (\(t = -0.56, p = 0.58 \)) endowment - both results mirroring those of human subjects. However, a t-test shows that the difference between the contributions of LLM agents endowed with \$80 is also not statistically significant at the p < 0.1 level (\(t = 1.25, p = 0.23 \)).

\begin{figure}[h]
    \centering
    \includegraphics[width=0.75\textwidth]{images/1C.png}  % Change this value to make the figure larger
    \caption{Average contributions for equal and varied endowment conditions in simulations with LLM-agents and experiments with human subjects. Average contributions are roughly the same for humans and LLM-agents in the equal and varied endowment conditions with \$20 and \$50 endowments. Average contributions are lower in the varied condition than in the fixed condition with \$80 endowments.}
    \label{1Cgraph}
\end{figure}

% \begin{table}[htbp]
% \centering
% \begin{tabular}{|>{\columncolor{gray!20}}c|c|c|c|}
% \hline
% \rowcolor{gray!20}\multicolumn{1}{|c|}{} & \multicolumn{3}{c|}{\textbf{Initial Endowment}} \\ \hline
% \rowcolor{gray!20}\textbf{Condition} & \$20 & \$50 & \$80 \\ \hline
% Equal & 39\% & 48\% & 63\% \\ \hline
% Varied & 35\% & 42\% & 44\% \\ \hline
% p-value & 0.86 & 0.58 & 0.23 \\ \hline
% t-stat & 0.18 & -0.56 & 1.25 \\ \hline
% \end{tabular}
% \caption{\color{red}IK THIS TABLE MAKES NO SENSE\color{black}}
% \label{variedend}
% \end{table}

\color{orange}
From these results, we cannot fully conclude that LLM agents are affected by varying endowments in the same way as humans. However, the lack of difference between the average contribution in the equal and varied condition with endowments of \$20 and \$50 along with the difference between the averages in the equal and varied endowment conditions does suggest that varying endowments does have an impact similar to the effect observed in humans. The lack of statistical significance could be a result of the limited and unequal number of data points in both groups of comparison (for every endowment amount, there are 15 contributions in the equal endowment condition compared  to only 5 contributions in the varied endowment condition in our simulations).
\color{black}

\subsection{\underline{RQ2}: Can multi-agent LLM system simulations replicate human behavior extrapolated from combining the behaviors observed in multiple lab experiments with human subjects?}

Towards answering this question, we simulate two experiments designed as a combination of multiple lab experiments with human subjects: \color{red}(1) takes a priming methodology and effect from experiments with human subjects of different cooperative games than the PGG to see if the effect holds for the PGG, and (2) uses a priming methodology used on the one-shot PGG and measures if it has the same effect over multiple rounds similar to the way in which priming affects different economic games than the PGG. \color{black} \textbf{From the results of these experiments, we conclude that multi-agent LLM system simulations do replicate behavior extrapolated from combining the behaviors observed in multiple lab experiments with human subjects.}

\subsubsection{Does priming LLM agents with sentences known to increase/decrease generosity in humans have that effect in simulations of the PGG?}

In experiments of other cooperation games, priming participants via sentences alluding to "unity" resulted in participants being more generous than participants primed via sentences alluding to "proportionality" \cite{moralsmatter}. In ten simulations, LLM agents primed with the sentences alluding to "unity" (Alice and Bob) contributed on average approximately 65\% of their initial endowment. In comparison, agents primed with the sentences alluding to "proportionality" (Casey and David) contributed on average only approximately 30\% of their endowment to the public pool. The results are summarized in Figure \ref{2Agraph}. A t-test shows that the difference in contributions between the two groups is statistically significant at the p < 0.01 level (\(t = 8.76, p < 1\text{e}{-9} \)).

\begin{figure}[h]
    \centering
    \includegraphics[width=0.5\textwidth]{images/2A.png}
    \caption{Average contributions for either "unity" or "proportionality" priming conditions. Average contributions under the "unity" condition are significantly higher than that under the "proportionality" condition.}
    \label{2Agraph}
\end{figure}

\textbf{From these results, we conclude that priming LLM agents with sentences known to increase/decrease generosity in humans does result in higher/lower contributions in simulations of the PGG.}

\subsubsection{Does the effect of priming LLM agents via game name in multi-round PGGs fade over time, like priming has been observed to fade over time in other experiments with human subjects?}
\bigskip
In experiments of competition games with human subjects, the effect of primings have been found to fade over time \cite{JIMENEZJIMENEZ201594}. In ten simulations, LLM agents presented a five-round PGG as the "Teamwork Game" (Alice and Bob) on average contributed 75\% of their initial endowment in the first round, compared to 55\% of their initial endowment in the fifth round. In comparison, agents presented the PGG as the "Taxation Game" (Casey and David) contributed 30\% of their initial endowment in the first round, compared to 45\% of their initial endowment in the fifth round. Hence, for both priming conditions, the average contribution was closer to 50\% of the initial endowment in the fifth round than in the first, meaning the effect of the priming was less pronounced in the fifth round than the first. The results are summarized in Figure \ref{2Bgraph}. T-tests show that the difference in contribution amount between the rounds is statistically significant for the "Teamwork" priming condition at the p < 0.01 level (\(t = 4.32, p = 0.002 \)) and for the "Taxation" priming condition at the p < 0.05 level (\(t = -2.30, p = 0.047 \)).

\begin{figure}[h]
    \centering
    \includegraphics[width=0.5\textwidth]{images/2B.png}
    \caption{Average contributions in the first and fifth rounds for either "Teamwork" and "Taxation" priming conditions. Average contributions in the fifth round are closer to 50\% of the initial endowment, the average amount contributed without any priming \cite{labexperiments}.}
    \label{2Bgraph}
\end{figure}

\textbf{From these results, we conclude that the effect of priming LLM agents fades over time, as observed in experiments with human subjects.} The "Taxation" priming condition specifically suggests that the difference between first and fifth round contributions is not because contributions in PGGs with human subjects decrease over time \cite{labexperiments}, but rather likely the consequence of a fading priming effect.

\subsection{\underline{RQ3}: What are required mechanisms for simulations to
inform policy-makers regarding real-world situations?} 

\color{purple}

\begin{itemize}
\item[1.] Classroom with Late Policies
\begin{itemize}
\item We simulate a classroom setting, where students have 5 assignments to submit over the course of a semester.
\item We try varying late-assignment policies (lenient - all late assignments accepted, harsh - late assignments not accepted, medium - 20\% grade penalty for each late day). 
\item We add perturbations to student lives that reasonably would affect their ability to turn in assignments on time: (1) having a midterm and (2) a particular assignment being especially challenging.
\item We also give students personality types: that of a procrastinator, an overachiever, and someone who "values work-life balance."
\item We wanted to see if multi-agent LLM systems simulated reasonable interactions and dynamics within a classroom.
\item We sought to examine how students would (1) work together and (2) communicate with one another and the teacher.
\item In preliminary simulations with one room, a classroom, we found that student agents proposed working together to other student agents.
\item \color{red}Generally, we saw that procrastinators would suggest collaboration most frequently.\color{purple}
\item Student agents also would ask questions to the teacher that were both individual (ex. requesting a review) and for the group (ex. clarifying an assignment). 
\item But everybody in the class would be distracted by them.
\item Therefore, we added two rooms to the simulation environment to enable easier communication: the teacher's office, and a student work room.
\item In simulations with the lenient late policy, communication was somewhat similar with or without the added rooms.
\item Students did move into the work room to "work" as per their directives. 
\item They also used the office to ask questions to the teacher.
\item But students discussed working together in a similar way and at roughly the same frequency.
\item This was true after adding perturbations too.
\item In simulations with the medium late policy, the ways in which proposed collaboration changed once perturbations were added.
\item As opposed to broadly discussing working together, student agents sometimes asked to see other agent's assignments to copy in the work room. 
\item \color{red}Generally, we saw that procrastinators would suggest cheating most frequently.\color{purple}
\item In simulations with the harsh late policy, we saw this behavior before even adding perturbations.
\item \color{red}This makes sense, because students have been found to be more likely to cheat on exams if they have strict teachers.\color{purple}
\item We ran simulations under the lenient (LLP), harsh (HLP), and medium late policies (MLP).
\item We ran simulations under three perturbation conditions: (P0) without any, (P1) students having a midterm during the 3rd assignment, and (P2) students having the midterm AND the 2nd assignment being especially challenging.
\item We ran five simulations of each combination of late policy and perturbation (LLP-P0, LLP-P1, LLP-P2, HLP-P0, HLP-P1, HLP-P2, MLP-0, MLP-1, and MLP-2), or 45 simulations total.
\item We recorded if students (1) proposed collaboration, (2) proposed a form of cheating, and (3) communicated with the teacher.
\item \color{red}We did not record which student initiated each of the types of actions. We also did not record at which assignment it occurred.\color{purple}
\item The results are summarized in Table \ref{teachingsims} and Figure \ref{3Agraph}.
\item{\begin{center}
\begin{table}[H]
\setlength{\abovecaptionskip}{0pt}
\setlength{\belowcaptionskip}{10pt}
\begin{tabular}{|>{\columncolor{gray!20}\centering\arraybackslash}m{2cm}|>{\centering\arraybackslash}m{3cm}|>{\centering\arraybackslash}m{3cm}|>{\centering\arraybackslash}m{3cm}|} 
 \hline
 \rowcolor{gray!20} \textbf{Condition} & \textbf{Collaboration} & \textbf{Cheating} & \textbf{Communication with Teacher} \\ 
 \hline\hline
 \textbf{LLP-P0} & 40\% & 0\% & 60\% \\ 
 \hline
 \textbf{LLP-P1} & 60\% & 0\% & 40\% \\ 
 \hline
 \textbf{LLP-P2} & 60\% & 0\% & 80\% \\ 
 \hline
 \textbf{MLP-P0} & 60\% & 0\% & 80\% \\ 
 \hline
 \textbf{MLP-P1} & 80\% & 20\% & 80\% \\ 
 \hline
 \textbf{MLP-P2} & 100\% & 20\% & 100\% \\ 
 \hline
 \textbf{HLP-P0} & 100\% & 20\% & 100\% \\ 
 \hline
 \textbf{HLP-P1} & 100\% & 20\% & 100\% \\ 
 \hline
 \textbf{HLP-P2} & 100\% & 60\% & 100\% \\ 
 \hline
\end{tabular}
\caption{\label{teachingsims}Percentage of simulations in which the specified behavior was observed for all 9 simulation conditions.}
\end{table}
\end{center}
}
\item{\begin{figure}[h]
    \centering
    \includegraphics[width=0.75\textwidth]{images/3A.png}  % Change this value to make the figure larger
    \caption{Percentage of simulations in which the specified behavior was observed for all 9 simulation conditions.}
    \label{3Agraph}
\end{figure}}
\item \color{purple} We find that: (1) rooms enabled private communication which LLM-agents used to propose cheating, and (2) LLM-agents seemed to reasonably respond to changes in late policy and perturbations.
\end{itemize}
\end{itemize}

\begin{itemize}
\item[2.] Shopping Cart Return
\begin{itemize}
\item We simulate a store parking lot, where there are specific shopping cart return locations.
\item Each simulation has one agent with a plan to decide whether or not to return their shopping cart.
\item We try two perturbations that might contribute to someone not returning their cart: (1) being far from the receptacle (FFR), and (2) having a child whom should not be left unattended (HAC). \cite{scientificamerican2020shoppingcarts}
\item We have two rooms: the shopper agents car and the receptacle.
\item We find that despite the perturbations, shopper agents arill frequently return their carts if it the perturbation is merely stated ("you are far from the receptacle," and "you have a child")
\item However, alluding to what is at stake as a result of the perturbation starts causing shopper agents to stop returning their carts as often ("you are \textit{parked across the parking lot} from the receptacle," and "you have a \textit{five-month old} child").
\item We present this as \color{red}stake-prompting (SP), a way in which to guide LLM-agents towards acting more in-line with humans.\color{purple}
\item We ran five simulations of either perturbation (FFR and HAC) with and without prompts alluding to what is at stake, or 20 simulations total.
\item The results are summarized in Table ~\ref{shoppingcartsims} and Figure \ref{3Bgraph}.
\item{\begin{center}
\begin{table}[H]
\setlength{\abovecaptionskip}{0pt}
\setlength{\belowcaptionskip}{10pt}
\begin{tabular}{|>{\columncolor{gray!20}\centering\arraybackslash}m{3cm}|>{\centering\arraybackslash}m{3cm}|>{\centering\arraybackslash}m{3cm}|} 
 \hline
 \rowcolor{gray!20} \textbf{Perturbation} & \textbf{Without SP} & \textbf{With SP} \\ 
 \hline\hline
 \textbf{FRB} & 100\% & 40\% \\ 
 \hline
 \textbf{HAC} & 60\% & 20\% \\ 
 \hline
\end{tabular}
\caption{Percentage of simulations in which the shopper agent decided to \color{red} MISSING THOUGHT HERE their shopping cart for both perturbations tested and under either prompting condition. ALSO, IF THE GRAPH HAS CRYTIC ACRYNYMS like SP FRR HAC, you MUST define in the cpation. \color{black}}
\label{shoppingcartsims}
\end{table}
\end{center}
}
\item{\begin{figure}[h]
    \centering
    \includegraphics[width=0.75\textwidth]{images/3B.png}
    \caption{Percentage of simulations in which the shopper agent decided to their shopping cart for both perturbations tested and under either prompting condition.}
    \label{3Bgraph}
\end{figure}}
\item We find that alluding to what is at stake caused LLM-agents to be more \color{red} affected by perturbations more like people\color{black}.
\end{itemize}
\end{itemize}

\begin{itemize}
\item[3.] Graffiti \& Littering
\begin{itemize}
\item Finally, we simulate two scenarios: (1) a bike-parking station, where a piece of paper is left on the handlebars. people must choose whether or not to litter - there is a no littering sign but no trash can. there is either grafitti on the wall or not, people more likely to litter when there is graffitti. (2) a envelope with cash visible hanging outside of a mailbox. the mail box either has graffiti on it or not.
\item \color{red}LLM-simulations show .... WIP .... but looks like first sort of should be true, second one looks unlikely but is interesting they won't steal?\color{black}
\item IDEALLY WOULD BE: WE TRIED X AND IT DID NOT WORK... ADDING Y GOT US CLOSER TO REPLICATING
\end{itemize}
\end{itemize}

\color{black}




