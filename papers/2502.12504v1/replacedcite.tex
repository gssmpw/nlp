\section{Related Work}
\subsection{The Public Goods Game}

The public goods game (PGG) is a particularly important case of human cooperative prosocial behavior studied by psychologists and economists. In the PGG, a set of three or more players are each given an endowment of money, of which they choose a portion of to voluntarily contribute in-private to a public pool. The public pool will then be multiplied by a factor, say 2, then redistributed evenly amongst the players. In a game with four players, if every player has a \$20 endowment and contributes half of their endowment, \$10, each player will receive a payoff of \$30: \$10 from the amount of their endowment they kept and \$20 from the payoff from the public pool. If each player donates nothing, they all simply keep their initial endowment of \$20 without any additional payoff. In the PGG, working cooperatively can be mutually beneficial. However, if a player decides not to contribute to the public pool, the player can increase the payoff — in the example previously discussed, if one player decides not to contribute to the public pool, the payoff is \$35 — \$20 from the endowment the player kept, and \$15 from the payoff from the public pool. Hence, not contributing to the public pool can result in a greater payoff for an individual, despite contribution to the public pool by \textit{someone} being necessary for there to be any additional payoff for players. This game mimics situations involving prosocial behavior in the real-world, such as taxation — when paying taxes, people are forced to contribute some of their own money to a public pool to develop infrastructure and services for the broader community. People can even try to avoid paying taxes through tax havens or loopholes and freeload off public goods.

Lab experiments with human subjects show that in single-round (one-shot) PGGs, people do typically contribute a portion of their endowment — on average, 50\%. When the game is played for multiple rounds, the amount that players contribute decreases each round, until it is effectively zero ____. However, introducing variations to the game can increase or decrease prosocial behavior. Priming participants by presenting the game under different names ____ or having participants primed with words alluding to cooperation before playing the PGG ____ increases their average contribution amount. Introducing transparency of contributions also increases average contributions — having participants write their contribution on a blackboard after each round ____ or announce their contributions publicly ____ increases the average contributions made in the PGG. Finally, varying the endowments that players have to begin with in the PGG also has an effect on average contributions — in experiments with individuals given low (\$20), medium (\$50), and high (\$80) endowments, individuals with the low and medium endowments contributed the same amount in games where endowments were the same as in games where endowments were varied. However, individuals with the high endowment contributed much less in games where endowments were varied ____.
% were varied than in games where endowments were equal 

These effects are not limited to the PGG and can be seen in select real-world scenarios. In fundraisers, audiences may be primed to cooperate towards a common goal. In many settings, pledges are made publicly (that is, transparently), which increases donations ____. Similar strategies are used by policy makers seeking to levy taxes for a bridge or other local public goods ____. Furthermore, lower class individuals have been demonstrated to be more prosocial than their higher class counterparts in studies of generosity, charity, trust, and helpfulness ____. Overall, although free-loaders may be able to benefit off of public goods — whether in the PGG or in the real world — there are strategies that policy makers can use to increase prosocial behavior.

% \color{red}One can often see the above results play out in fund-raising events. Those doing fundraising will prime an audience to cooperate toward a common goal. In many settings, pledges are made publicly: this increases transparency, which in turn increases donations ____. Similar advice is given to policy makers seeking to levy taxes for a bridge or other local public goods ____. \color{black}


%