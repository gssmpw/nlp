\section{Conclusion}

This study demonstrates the potential of multi-agent LLM systems to simulate cooperative prosocial behavior, particularly in the context of public goods games. Our findings show that these systems can not only replicate the directional effects observed in human public goods game experiments but also apply priming effects from non-public goods game experiments and incorporate the fading of priming effects over time, as observed in multi-round games. The ability of LLMs to transfer effects across different contexts suggests a deeper understanding of human behavior rather than mere recapitulation of specific experimental results. Our research also highlights the need for additional mechanisms, such as private communication channels and stake-prompting, to accurately capture the nuances of human behavior in real-world scenarios.

While these results are promising, they also underscore the importance of careful implementation and interpretation when using LLM simulations to inform policy decisions. The observed exaggeration of effect magnitudes and the potential for bias inherent in LLM training data necessitate a cautious approach. Future research might focus on refining these simulation techniques, addressing limitations, and developing robust frameworks for validating LLM-based behavioral predictions against real-world outcomes. In sum, these studies suggest that LLM-based simulations, when used judiciously and with an understanding of their limitations, can become a valuable tool for policy designers, offering insights into potential human reactions to policies whose effects may be both difficult to anticipate and crucial to understand.
