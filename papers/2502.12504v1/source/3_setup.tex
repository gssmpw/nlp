% 3 sections; 1. lab experiments, 2. combinations, and 3. evaluation/results

\section{Experimental Setup}

\color{red}wrote this whole section (esp eval) using the word "treatment" either need to define it or (preferably) change it\color{black}

With the goal of better understanding the ability of multi-agent LLM systems to model prosocial human behavior, we simulate three different types of experiments: (1) simulations which aim to directly capture the effect of a treatment observed in a previous lab experiment with human subjects; (2) simulations which combine treatments/effects from multiple previous lab experiments with human subjects; and (3) select "out-of-the-lab" real-world scenarios we use as case studies.

For all experiments, we use OpenAI's GPT. Specifically, we use GPT4 (model gpt-4o-mini), which has been shown to have the ability to interpret inherently human concepts such as equity and also scores  well on a variety of standardized tests, ranging from the Bar Exam to the GRE \cite{openai2023gpt4}.

\subsection{Research Questions}

We specifically address the following three research questions:

\bigskip

\begin{itemize}[itemindent=2.5em]
  \item[\underline{\textbf{RQ1}}] Can multi-agent LLM system simulations replicate behaviors observed in PGG lab experiments with human subjects?
\begin{itemize} [leftmargin=0.5in]
\item[\underline{a.}] Does priming LLM agents via game name replicate the effect of priming humans via game name?
\item[\underline{b.}] Does introducing transparency in contribution between LLM agents replicate the effect of introducing transparency in contribution between humans?
\item[\underline{c.}] Does varying the endowments of LLM agents replicate the effect of varying endowments of humans?
\end{itemize}
\end{itemize}

\bigskip
\color{blue}

\begin{itemize}[itemindent=2.5em]
  \item[\underline{\textbf{RQ1}}] \textit{Can multi-agent LLM system simulations replicate behaviors observed in PGG lab experiments with human subjects?} We answer this research question via three experiments comparing results from simulations with LLM-agents to results from lab experiments with human subjects:
  
  \begin{itemize} [leftmargin=0.5in]
    \item[\underline{1.}] Priming via Game Name
    \item[\underline{2.}] Introducing Transparency of Contributions
    \item[\underline{3.}] Varying Endowments
  \end{itemize}
\end{itemize}

\color{black}

\bigskip

\begin{itemize}[itemindent=2.5em]

  \item[\underline{\textbf{RQ2.}}] Can multi-agent LLM system simulations replicate human behavior extrapolated from combining the behaviors observed in multiple lab experiments with human subjects?
\begin{itemize}[leftmargin=0.5in]
    \item[\underline{a.}] Does priming LLM agents with sentences known to increase/decrease generosity in humans have that effect in simulations of the PGG?
\item[\underline{b.}] Does the effect of priming LLM agents via game name in multi-round PGGs fade over time, like priming has been observed to fade over time in other experiments with human subjects?
\end{itemize}
\end{itemize}

  \bigskip
\color{blue}
\begin{itemize}[itemindent=2.5em]

  \item[\underline{\textbf{RQ2.}}] \textit{Can multi-agent LLM system simulations replicate human behavior extrapolated from combining the behaviors observed in multiple lab experiments with human subjects?} We answer this research question via two experiments combining treatments and expected effects from multiple lab experiments with human subjects:
  \begin{itemize}[leftmargin=0.5in]
    \item[\underline{1.}] Using a priming methodology from other cooperation games in the PGG
    \item[\underline{2.}] Measuring the change in effect of priming over time, using results of priming over time from a different competition game
  \end{itemize}
\end{itemize}
\color{black}

\bigskip

\begin{itemize}
  \item[\underline{\textbf{RQ3.}}] What are required mechanisms for simulations to
inform policy-makers regarding real-world situations to test policies that encourage human collaboration?
\end{itemize}

\bigskip

\color{blue}

\begin{itemize}
  \item[\underline{\textbf{RQ3.}}] \textit{What are required mechanisms for simulations to
inform policy-makers regarding real-world situations to test policies that encourage human collaboration?} We answer this research question via three real-world case studies towards identifying necessary mechanisms:
\begin{itemize}[leftmargin=0.5in]
    \item[\underline{1.}] A classroom setting with varying late-assignment policies
    \item[\underline{2.}] A store parking lot where shoppers need to return shopping carts to designated areas
    \item[\underline{3.}] Graffitti?
  \end{itemize}
\end{itemize}

\color{black}

\bigskip

We expect generally positive results for the first two research questions, given other work has shown the ability of LLMs to replicate lab experiments that are both published and unpublished \cite{hewitt2024predicting}, but still verify this in the context of the PGG towards specifically understanding the abilities of multi-agent LLM systems to simulate prosocial behavior. \color{red}The third research question is more open-ended - we use case studies of two real-world scenarios towards identifying two mechanisms required for simulations to show expected outcomes, a process which can be continued to be used by policy makers. As policy makers consider the specific scenarios which they need to simulate, they may realize and implement new mechanisms required to enable their simulation.\color{black}

\subsection{Multi-Agent-LLM System}

For our experiments, we adapt a previous introduced agent architecture for simulating social emergent behavior \cite{park2023generativeagentsinteractivesimulacra}. In this architecture, the administrator provides each agent with a name, public and private biographies, instructions (called directives), and an initial plan, which consists of a description, stop condition, and a location. The architecture also allows for the creation of rooms, between which agents are able to move. 

\subsubsection{System Setup for Public Goods Game Experiments}

In PGG experiments, there is an agent named and directed to act as the moderator. The remaining agents (varies from three to four depending upon the experiment) are given arbitrary alphabetical names (Alice, Bob, Casey, and David). Each agent's initial endowment is specified in their public biography, which all other agents can see. If there is a priming condition, it is included in the private biography as to expose the desired agent to the priming without leakage to other agents. Communication is not inherently turn-taking in the architecture; so we have to specify the order and method of contribution to player agents via their directives and initial plan. We initialize two separate rooms - the Game Room and the Moderation Room. Player agents are by default in the game room, then travel to the moderation room to tell the moderator their contribution once the player specified as being before them returns to game room from the moderation room after making their contribution. The moderator remains in the moderation room other than to announce payoffs after all players have made contributions. To realize transparency, we simply remove the moderation room, and have all agents make their contributions publicly in the game room. For the experiment with multiple rounds, we use nearly the same instruction set, other than specifying to the moderator and player agents via directives to repeat the same process for multiple rounds.

The thoughts and actions of each player (and the moderator) are displayed in separate output logs, each representing one of the agents involved in the simulation. In each of the output logs, the agent waits for the condition required (the return of the player before them), then acts on the plan of moving to the moderation room and making a contribution by speaking with the moderator. The moderator stores each contribution to memory in order to later announce payoffs. We extract the contribution amount of each agent using a script on each agent's output logs, identifying the section in which the agent speaks in the moderation room and taking the monetary amount in that sentence. We also manually verified these contribution amounts to ensure accuracy.

\subsubsection{System Setup for Real-World Case Studies}

\color{red}We use two real-world settings as case studies to identify mechanisms needed to enable informative simulations. For these case studies, we run several simulations with slightly varying conditions to generally understand the ways in which agents act, and then use that to create a mechanism which enables a behavior we identify as missing from a combination of general knowledge and informative studies.

The first is a classroom setting, where we specifically focus the impact of various late policies (no penalty, full penalty, and penalty by days late) on student communicative and collaborative behaviors. We define a professor agent (named the Professor) and three student agents with arbitrary names (Alice, Bob, and Casey). The professor agent is given instructions to announce a late policy to the class, and then begin assigning assignments. In student agents public biography, they are given one of three "personalities" - an overachiever, a hard-worker, and a balanced individual. In the private biographies of the agents, we add various perturbations to student lives that would reasonably affect their ability to turn in assignments on time, such as having a midterm in another class, having special difficulty with a particular assignment, or having a busy week in general. We do not need to implement a rigid turn-taking guideline, however we do generally add a framework within which agents act - the professor agent announces assignments on a fixed frequency in the classroom, between which student agents "work" on assignments and consider how many days late they will need to submit it based on the other factors in their lives.

Second, we simulate shoppers in a parking lot who can either return or not return their shopping cart. We define five shopper agents with arbitrary names (Alice, Bob, Casey, David, and Emery). We add various perturbations to shopper lives that affect their ability to return their shopping cart \cite{scientificamerican2020shoppingcarts} in their private biographies. Shopper agents by default have blank public biographies; but if perturbations added to the shopper's private biography would reasonably be extrapolated other people in real life (such as having a child), we also include it in the shopper's public biography. Shopper agents are not given directives, but have an initial plan to decide whether or not to return their shopping cart, and then to act on it. We also try changing the environment by adding agents who are employees of the store tasked with taking the shopping carts from the parking lot back into the store. These agents are given the name "Employee," have public biographies which indicate that they are a store employee, have an initial plan (and related directives) to move carts, and otherwise have all other fields left blank.\color{black}

\subsection{Evaluation}

\subsubsection{Replicating Observed Outcomes from Past Public Goods Game Experiments}

For experiments of the first type, we run simulations of the PGG using treatments from published lab experiments with human subjects, checking whether the effect of said treatments on multi-agents LLMs in simulation in similar to that of on humans. The treatments and effects we replicate from previous lab experiments are listed in Table ~\ref{labexperiments}.

\begin{center}
\begin{table}[H]
\setlength{\abovecaptionskip}{0pt}
\setlength{\belowcaptionskip}{10pt}
\begin{tabular}{|>{\columncolor{gray!20}\centering\arraybackslash}m{0.5cm}|p{6cm}|p{6cm}|} 
 \hline
 \rowcolor{gray!20} \textbf{Exp.} & \multicolumn{1}{c|}{\textbf{Treatment}} & \multicolumn{1}{c|}{\textbf{Expected Result}} \\ [0.5ex] 
 \hline\hline
 \textbf{1} & \textbf{Priming via Game Name:} Participants are presented the PGG as a game called either the "Taxation Game" or "Teamwork Game." & Participants presented the game as the "Taxation Game" contribute less than participants presented the "Teamwork Game." \cite{Eriksson_Strimling_2014} \\ 
 \hline
 \textbf{2} & \textbf{Transparency of Contribution:} Participants know what others contribute to the public pool (under standard conditions, contributions are made privately with respect to other players). & Participants make larger contributions when participants are aware of each other's contributions compared to when contributions are made privately without other's knowledge. \cite{transparencytwo}\\
 \hline
 \textbf{3} & \textbf{Variations in Endowment:} Participants start the PGG with different endowment amounts. & "Rich" participants contribute more when other participants are also "rich." \cite{HARGREAVESHEAP20164}  \\
 \hline
\end{tabular}
\caption{\label{labexperiments}Table of treatments and expected results (effects) from previous lab experiments of the PGG with human subjects. Actual experiments that are drawn from are cited in the second column.}
\end{table}
\end{center}




For these experiments, we simulate one-shot PGGs with three or four players, measuring whether the impact the treatments have on player contributions is similar to what has been observed in experiments with human subjects (i.e., the expected result listed). For the first experiment, we run ten simulations of the PGG with four players, two under each priming condition, and compare the average contribution amount of players from either group. For the second experiment, we also run ten simulations of the PGG with four players: five simulations with transparency of contributions and five without. We compare the average contribution amount of players in either of the two conditions. In the first and second experiments, players are all endowed with \$20. For the third experiment, we run twenty simulations with three players: fifteen simulations where all players have the same initial endowment (five where that endowment is \$20, five where it is \$50, and five where it is \$80) and five simulations where players have different initial endowments (namely, \$20, \$50, and \$80). We compare the average contributions of players with each of the three endowments in the fixed and the varied endowment conditions (i.e., we compare how much a player with \$X contributes in the condition where everyone has \$X vs. the varied condition). In all three experiments, 1.6 times the amount of the public pool is split evenly amongst the players as their payoff - players are made aware of this to inform the way in which they act, but the simulations are of one-shot PGGs.

\subsubsection{Verifying Outcomes from Combinations of Past Lab Experiments}

We also design two additional experiments based on a combination of treatments and effects from multiple lab experiments with human subjects. The first takes a priming methodology and effect from experiments with human subjects of different cooperative games than the PGG to see if the effect holds for the PGG. The second uses a priming methodology used on the one-shot PGG and measures if it has the same effect over multiple rounds similar to the way in which priming affects different economic games than the PGG. The treatments and effects in the combined experiments we design are listed in Table ~\ref{combinedexperiments}.

\begin{center}
\begin{table}[H]
\setlength{\abovecaptionskip}{0pt}
\setlength{\belowcaptionskip}{10pt}
\begin{tabular}{|>{\columncolor{gray!20}\centering\arraybackslash}m{0.5cm}|p{6cm}|p{6cm}|} 
 \hline
 \rowcolor{gray!20} \textbf{Exp.} & \multicolumn{1}{c|}{\textbf{Treatment}} & \multicolumn{1}{c|}{\textbf{Expected Result}} \\ [0.5ex] 
 \hline\hline
 \textbf{1} & \textbf{Priming via Sentences:} Participants are presented with 5 sentences alluding to either "unity" or "proportionality" concepts. & Participants presented with sentences alluding to "unity" contribute more than participants primed with "proportionality". \cite{6d947858-4a17-3462-9053-fc55b58ffee1, moralsmatter} \\ 
 \hline
 \textbf{2} & \textbf{Priming via Game Name over Multiple Rounds:} Participants are presented the PGG as a game called either the "Taxation Game" or "Teamwork Game" and play for five rounds. & The effect of priming should fade over time; although there should still be a disparity between participants in either group's initial contributions, it should lessen over time. \cite{Eriksson_Strimling_2014, JIMENEZJIMENEZ201594} \\
 \hline
\end{tabular}
\caption{\label{combinedexperiments}Table of treatments and expected results (effects) from combining previous lab experiments with human subjects. Actual experiments that are drawn from are cited in the second column.}
\end{table}
\end{center}

For both experiments, all players are endowed with \$20. For the first experiment, we run ten simulations a one-shot PGG with four players, two under each priming condition ("unity" and "proportionality"), and compare the average contribution amount of players from either group. We prime each player by exposing them to five sentences alluding to each concept, taken from the lab experiment \cite{moralsmatter}. In this experiment, 1.6 times the amount of the public pool is split evenly amongst the players as their payoff - players are made aware of this to inform the way in which they act, but the simulations are of one-shot PGGs. For the second experiment, we run ten simulations of a PGG with five rounds and four players, two under each priming condition ("Taxation" and "Teamwork"). In this experiment, payoffs are computed the same as in the previous experiment (namely, 1.6 times the amount of the public pool split evenly amongst the players). Since there are multiple rounds, the payoff and amount not contributed are summed to determine each players endowment for the next round. Within each priming condition, we compare the average contribution in the first round compared to the fifth round. We also compare the contributions between priming conditions, checking that there was an initial disparity that ultimately begins to fade.

\color{red}

\subsubsection{Real-World Case Studies}

SEE RESULTS, THERE'S A BIG BULLET POINT LIST OF EVAL / RESULTS COMBINED I'M TRYING TO FIGURE OUT HOW TO SEPARATE. 

TALK ABOUT HOW THESE CONNECT TO PROSOCIAL SITUATIONS HERE?

\color{black}