%% bare_conf.tex
%% V1.4b
%% 2015/08/26
%% by Michael Shell
%% See:
%% http://www.michaelshell.org/
%% for current contact information.
%%
%% This is a skeleton file demonstrating the use of IEEEtran.cls
%% (requires IEEEtran.cls version 1.8b or later) with an IEEE
%% conference paper.
%%
%% Support sites:
%% http://www.michaelshell.org/tex/ieeetran/
%% http://www.ctan.org/pkg/ieeetran
%% and
%% http://www.ieee.org/

%%*************************************************************************
%% Legal Notice:
%% This code is offered as-is without any warranty either expressed or
%% implied; without even the implied warranty of MERCHANTABILITY or
%% FITNESS FOR A PARTICULAR PURPOSE! 
%% User assumes all risk.
%% In no event shall the IEEE or any contributor to this code be liable for
%% any damages or losses, including, but not limited to, incidental,
%% consequential, or any other damages, resulting from the use or misuse
%% of any information contained here.
%%
%% All comments are the opinions of their respective authors and are not
%% necessarily endorsed by the IEEE.
%%
%% This work is distributed under the LaTeX Project Public License (LPPL)
%% ( http://www.latex-project.org/ ) version 1.3, and may be freely used,
%% distributed and modified. A copy of the LPPL, version 1.3, is included
%% in the base LaTeX documentation of all distributions of LaTeX released
%% 2003/12/01 or later.
%% Retain all contribution notices and credits.
%% ** Modified files should be clearly indicated as such, including  **
%% ** renaming them and changing author support contact information. **
%%*************************************************************************


% *** Authors should verify (and, if needed, correct) their LaTeX system  ***
% *** with the testflow diagnostic prior to trusting their LaTeX platform ***
% *** with production work. The IEEE's font choices and paper sizes can   ***
% *** trigger bugs that do not appear when using other class files.       ***                          ***
% The testflow support page is at:
% http://www.michaelshell.org/tex/testflow/



\documentclass[conference]{IEEEtran}
% Some Computer Society conferences also require the compsoc mode option,
% but others use the standard conference format.
%
% If IEEEtran.cls has not been installed into the LaTeX system files,
% manually specify the path to it like:
% \documentclass[conference]{../sty/IEEEtran}

% Some very useful LaTeX packages include:
% (uncomment the ones you want to load)

% *** MISC UTILITY PACKAGES ***
%
%\usepackage{ifpdf}
% Heiko Oberdiek's ifpdf.sty is very useful if you need conditional
% compilation based on whether the output is pdf or dvi.
% usage:
% \ifpdf
%   % pdf code
% \else
%   % dvi code
% \fi
% The latest version of ifpdf.sty can be obtained from:
% http://www.ctan.org/pkg/ifpdf
% Also, note that IEEEtran.cls V1.7 and later provides a builtin
% \ifCLASSINFOpdf conditional that works the same way.
% When switching from latex to pdflatex and vice-versa, the compiler may
% have to be run twice to clear warning/error messages.

% *** CITATION PACKAGES ***
%
%\usepackage{cite}
% cite.sty was written by Donald Arseneau
% V1.6 and later of IEEEtran pre-defines the format of the cite.sty package
% \cite{} output to follow that of the IEEE. Loading the cite package will
% result in citation numbers being automatically sorted and properly
% "compressed/ranged". e.g., [1], [9], [2], [7], [5], [6] without using
% cite.sty will become [1], [2], [5]--[7], [9] using cite.sty. cite.sty's
% \cite will automatically add leading space, if needed. Use cite.sty's
% noadjust option (cite.sty V3.8 and later) if you want to turn this off
% such as if a citation ever needs to be enclosed in parenthesis.
% cite.sty is already installed on most LaTeX systems. Be sure and use
% version 5.0 (2009-03-20) and later if using hyperref.sty.
% The latest version can be obtained at:
% http://www.ctan.org/pkg/cite
% The documentation is contained in the cite.sty file itself.

% *** GRAPHICS RELATED PACKAGES ***
%
\ifCLASSINFOpdf
  % \usepackage[pdftex]{graphicx}
  % declare the path(s) where your graphic files are
  % \graphicspath{{../pdf/}{../jpeg/}}
  % and their extensions so you won't have to specify these with
  % every instance of \includegraphics
  % \DeclareGraphicsExtensions{.pdf,.jpeg,.png}
\else
  % or other class option (dvipsone, dvipdf, if not using dvips). graphicx
  % will default to the driver specified in the system graphics.cfg if no
  % driver is specified.
  % \usepackage[dvips]{graphicx}
  % declare the path(s) where your graphic files are
  % \graphicspath{{../eps/}}
  % and their extensions so you won't have to specify these with
  % every instance of \includegraphics
  % \DeclareGraphicsExtensions{.eps}
\fi
% graphicx was written by David Carlisle and Sebastian Rahtz. It is
% required if you want graphics, photos, etc. graphicx.sty is already
% installed on most LaTeX systems. The latest version and documentation
% can be obtained at: 
% http://www.ctan.org/pkg/graphicx
% Another good source of documentation is "Using Imported Graphics in
% LaTeX2e" by Keith Reckdahl which can be found at:
% http://www.ctan.org/pkg/epslatex
%
% latex, and pdflatex in dvi mode, support graphics in encapsulated
% postscript (.eps) format. pdflatex in pdf mode supports graphics
% in .pdf, .jpeg, .png and .mps (metapost) formats. Users should ensure
% that all non-photo figures use a vector format (.eps, .pdf, .mps) and
% not a bitmapped formats (.jpeg, .png). The IEEE frowns on bitmapped formats
% which can result in "jaggedy"/blurry rendering of lines and letters as
% well as large increases in file sizes.
%
% You can find documentation about the pdfTeX application at:
% http://www.tug.org/applications/pdftex

% *** MATH PACKAGES ***
%
%\usepackage{amsmath}
% A popular package from the American Mathematical Society that provides
% many useful and powerful commands for dealing with mathematics.
%
% Note that the amsmath package sets \interdisplaylinepenalty to 10000
% thus preventing page breaks from occurring within multiline equations. Use:
%\interdisplaylinepenalty=2500
% after loading amsmath to restore such page breaks as IEEEtran.cls normally
% does. amsmath.sty is already installed on most LaTeX systems. The latest
% version and documentation can be obtained at:
% http://www.ctan.org/pkg/amsmath

% *** SPECIALIZED LIST PACKAGES ***
%
%\usepackage{algorithmic}
% algorithmic.sty was written by Peter Williams and Rogerio Brito.
% This package provides an algorithmic environment fo describing algorithms.
% You can use the algorithmic environment in-text or within a figure
% environment to provide for a floating algorithm. Do NOT use the algorithm
% floating environment provided by algorithm.sty (by the same authors) or
% algorithm2e.sty (by Christophe Fiorio) as the IEEE does not use dedicated
% algorithm float types and packages that provide these will not provide
% correct IEEE style captions. The latest version and documentation of
% algorithmic.sty can be obtained at:
% http://www.ctan.org/pkg/algorithms
% Also of interest may be the (relatively newer and more customizable)
% algorithmicx.sty package by Szasz Janos:
% http://www.ctan.org/pkg/algorithmicx

% *** ALIGNMENT PACKAGES ***
%
%\usepackage{array}
% Frank Mittelbach's and David Carlisle's array.sty patches and improves
% the standard LaTeX2e array and tabular environments to provide better
% appearance and additional user controls. As the default LaTeX2e table
% generation code is lacking to the point of almost being broken with
% respect to the quality of the end results, all users are strongly
% advised to use an enhanced (at the very least that provided by array.sty)
% set of table tools. array.sty is already installed on most systems. The
% latest version and documentation can be obtained at:
% http://www.ctan.org/pkg/array

% IEEEtran contains the IEEEeqnarray family of commands that can be used to
% generate multiline equations as well as matrices, tables, etc., of high
% quality.

% *** SUBFIGURE PACKAGES ***
%\ifCLASSOPTIONcompsoc
%  \usepackage[caption=false,font=normalsize,labelfont=sf,textfont=sf]{subfig}
%\else
%  \usepackage[caption=false,font=footnotesize]{subfig}
%\fi
% subfig.sty, written by Steven Douglas Cochran, is the modern replacement
% for subfigure.sty, the latter of which is no longer maintained and is
% incompatible with some LaTeX packages including fixltx2e. However,
% subfig.sty requires and automatically loads Axel Sommerfeldt's caption.sty
% which will override IEEEtran.cls' handling of captions and this will result
% in non-IEEE style figure/table captions. To prevent this problem, be sure
% and invoke subfig.sty's "caption=false" package option (available since
% subfig.sty version 1.3, 2005/06/28) as this is will preserve IEEEtran.cls
% handling of captions.
% Note that the Computer Society format requires a larger sans serif font
% than the serif footnote size font used in traditional IEEE formatting
% and thus the need to invoke different subfig.sty package options depending
% on whether compsoc mode has been enabled.
%
% The latest version and documentation of subfig.sty can be obtained at:
% http://www.ctan.org/pkg/subfig

% *** FLOAT PACKAGES ***
%
%\usepackage{fixltx2e}
% fixltx2e, the successor to the earlier fix2col.sty, was written by
% Frank Mittelbach and David Carlisle. This package corrects a few problems
% in the LaTeX2e kernel, the most notable of which is that in current
% LaTeX2e releases, the ordering of single and double column floats is not
% guaranteed to be preserved. Thus, an unpatched LaTeX2e can allow a
% single column figure to be placed prior to an earlier double column
% figure.
% Be aware that LaTeX2e kernels dated 2015 and later have fixltx2e.sty's
% corrections already built into the system in which case a warning will
% be issued if an attempt is made to load fixltx2e.sty as it is no longer
% needed.
% The latest version and documentation can be found at:
% http://www.ctan.org/pkg/fixltx2e

%\usepackage{stfloats}
% stfloats.sty was written by Sigitas Tolusis. This package gives LaTeX2e
% the ability to do double column floats at the bottom of the page as well
% as the top. (e.g., "\begin{figure*}[!b]" is not normally possible in
% LaTeX2e). It also provides a command:
%\fnbelowfloat
% to enable the placement of footnotes below bottom floats (the standard
% LaTeX2e kernel puts them above bottom floats). This is an invasive package
% which rewrites many portions of the LaTeX2e float routines. It may not work
% with other packages that modify the LaTeX2e float routines. The latest
% version and documentation can be obtained at:
% http://www.ctan.org/pkg/stfloats
% Do not use the stfloats baselinefloat ability as the IEEE does not allow
% \baselineskip to stretch. Authors submitting work to the IEEE should note
% that the IEEE rarely uses double column equations and that authors should try
% to avoid such use. Do not be tempted to use the cuted.sty or midfloat.sty
% packages (also by Sigitas Tolusis) as the IEEE does not format its papers in
% such ways.
% Do not attempt to use stfloats with fixltx2e as they are incompatible.
% Instead, use Morten Hogholm'a dblfloatfix which combines the features
% of both fixltx2e and stfloats:
%
% \usepackage{dblfloatfix}
% The latest version can be found at:
% http://www.ctan.org/pkg/dblfloatfix

% *** PDF, URL AND HYPERLINK PACKAGES ***
%
%\usepackage{url}
% url.sty was written by Donald Arseneau. It provides better support for
% handling and breaking URLs. url.sty is already installed on most LaTeX
% systems. The latest version and documentation can be obtained at:
% http://www.ctan.org/pkg/url
% Basically, \url{my_url_here}.

% *** Do not adjust lengths that control margins, column widths, etc. ***
% *** Do not use packages that alter fonts (such as pslatex).         ***
% There should be no need to do such things with IEEEtran.cls V1.6 and later.
% (Unless specifically asked to do so by the journal or conference you plan
% to submit to, of course. )
\usepackage[utf8]{inputenc}
\usepackage{cite} %because I use ieeetr as the bibliography stytle, this command is needed to sort references in text (and shorten ranges)
\usepackage{graphicx}
\usepackage{adjustbox}
\usepackage{amsmath}
\usepackage{amsfonts}
\usepackage{placeins}
\usepackage{fancyhdr}
\usepackage{balance}
\usepackage{algorithm,algpseudocode} 

\usepackage[dvipsnames]{xcolor}
 
\usepackage{amsthm}
\newtheorem{prop}{Property}
\newtheorem{observation}{Observation}
\newtheorem{thm}{Theorem}
\newtheorem{lma}{Lemma}
\newtheorem{remark}{Remark}

\usepackage{lipsum}
\usepackage{subcaption}
\usepackage{mathtools}
\usepackage{booktabs}
\usepackage{threeparttable}
\usepackage{siunitx}
% \ifCLASSOPTIONcompsoc
% \usepackage[caption=false,font=normalsize,labelfont=sf,textfont=sf]{subfig}
% \else
% \usepackage[caption=false,font=footnotesize]{subfig}
% \fi


\newcommand{\myquote}[1]{{``#1''}} 

\newcommand{\real}{\mathbb{R}}



% correct bad hyphenation here
\hyphenation{}

\begin{document}
%
% paper title
% Titles are generally capitalized except for words such as a, an, and, as,
% at, but, by, for, in, nor, of, on, or, the, to and up, which are usually
% not capitalized unless they are the first or last word of the title.
% Linebreaks \\ can be used within to get better formatting as desired.
% Do not put math or special symbols in the title.
\title{A Simulation Pipeline to Facilitate Real-World Robotic Reinforcement Learning Applications}
% Deploying Reinforcement Learning Models for Robotic Systems: A Case Study from Simulation to Physical Robot

% author names and affiliations
% use a multiple column layout for up to three different
% affiliations
%\author{\IEEEauthorblockN{Michael Shell}
%\IEEEauthorblockA{School of Electrical and\\Computer Engineering\\
%Georgia Institute of Technology\\
%Atlanta, Georgia 30332--0250\\
%Email: http://www.michaelshell.org/contact.html}
%\and
%\IEEEauthorblockN{Homer Simpson}
%\IEEEauthorblockA{Twentieth Century Fox\\
%Springfield, USA\\
%Email: homer@thesimpsons.com}
%\and
%\IEEEauthorblockN{James Kirk\\ and Montgomery Scott}
%\IEEEauthorblockA{Starfleet Academy\\
%San Francisco, California 96678--2391\\
%Telephone: (800) 555--1212\\
%Fax: (888) 555--1212}}

% conference papers do not typically use \thanks and this command
% is locked out in conference mode. If really needed, such as for
% the acknowledgment of grants, issue a \IEEEoverridecommandlockouts
% after \documentclass

% for over three affiliations, or if they all won't fit within the width
% of the page, use this alternative format:
% 
% \author{
%     \IEEEauthorblockN{
%         Jefferson Silveira\IEEEauthorrefmark{1},
%         Joshua A. Marshall\IEEEauthorrefmark{2}
%         Sidney N. Givigi, Jr.\IEEEauthorrefmark{3}
%     }
%     \IEEEauthorblockA{
%         \IEEEauthorrefmark{1}
%             Department of Electrical and Computer Engineering, Queen's University\\ Kingston, ON, Canada \\ Email: jefferson.silveira@queensu.ca\\
%         \IEEEauthorrefmark{2}
%             Department of Electrical and Computer Engineering and the Ingenuity Labs Research Institute, Queen's University\\ Kingston, ON, Canada \\
%             Email: joshua.marshall@queensu.ca
%         \IEEEauthorrefmark{3}
%             School of Computing, Queen's University\\ Kingston, ON, Canada \\ Email: sidney.givigi@queensu.ca\\
%     }
% }

\author{
    \IEEEauthorblockN{Jefferson Silveira}
    \IEEEauthorblockA{\textit{Dept. of Electrical and Computer Engineering} \\
    \textit{Queen's University}\\
    Kingston, ON, Canada \\
    jefferson.silveira@queensu.ca}
    \and
    \IEEEauthorblockN{Joshua A. Marshall}
    \IEEEauthorblockA{
    \textit{Ingenuity Labs Research Institute} \\
    \textit{Queen's University}\\
    Kingston, ON, Canada \\
   joshua.marshall@queensu.ca}
    \and
    \IEEEauthorblockN{Sidney N. Givigi, Jr}
    \IEEEauthorblockA{\textit{School of Computing} \\
    \textit{Queen's University}\\
    Kingston, ON, Canada \\
    sidney.givigi@queensu.ca}
}


% use for special paper notices
%\IEEEspecialpapernotice{(Invited Paper)}

% make the title area
\maketitle

% As a general rule, do not put math, special symbols or citations
% in the abstract
\begin{abstract}
Reinforcement learning (RL) has gained traction for its success in solving complex tasks for robotic applications. However, its deployment on physical robots remains challenging due to safety risks and the comparatively high costs of training. To avoid these problems, RL agents are often trained on simulators, which introduces a new problem related to the gap between simulation and reality. This paper presents an RL pipeline designed to help reduce the reality gap and facilitate developing and deploying RL policies for real-world robotic systems. The pipeline organizes the RL training process into an initial step for system identification and three training stages: core simulation training, high-fidelity simulation, and real-world deployment, each adding levels of realism to reduce the sim-to-real gap. Each training stage takes an input policy, improves it, and either passes the improved policy to the next stage or loops it back for further improvement. This iterative process continues until the policy achieves the desired performance. The pipeline's effectiveness is shown through a case study with the Boston Dynamics Spot mobile robot used in a surveillance application.  The case study presents the steps taken at each pipeline stage to obtain an RL agent to control the robot's position and orientation.
\end{abstract}
\IEEEpeerreviewmaketitle

\section{Introduction}
\label{sec:intro}

In recent years, Reinforcement Learning (RL) has achieved high-level performance in solving challenging robotic tasks. %robot manipulation tasks~\cite{akkaya2019solving}.
While RL itself is not new (with roots dating back to the 1960s~\cite{minsky1961steps}), its recent successes can be attributed to several factors: the availability of powerful and affordable processing units (such as CPUs and GPUs), advancements in deep learning techniques, and the development of new deep RL methods.

The RL methodology relies on three core concepts: the agent, the reward signal and the environment. The agent is the decision-maker that learns by interacting with the environment. The reward signal defines the goal of the RL problem, serving as feedback to encourage desirable behaviours and discourage undesirable ones. The environment is where the agent operates, providing observations in response to the agent's actions~\cite{sutton2018reinforcement}. The agent gathers experience to learn the appropriate actions and solve the RL problem. This is an iterative approach that can be directly applied to a physical robot. % Fig.\ \ref{fig:core_RL_diagram} illustrates the interactions between these concepts.

% \begin{figure}[t]
%     \centering
%     \includegraphics[width=0.5\linewidth, trim=0cm 0cm 0cm 0.35cm, clip]{figures/RL_diagram.pdf}
%     \caption{Core reinforcement learning diagram representing the interaction between the agent and the environment.}
%     \label{fig:core_RL_diagram}
% \end{figure}

%In robotic applications, the environment includes not only the space where the robot operates but also the robot itself, and the agent is the decision-making algorithm or policy that sends commands via the robot.  However, it is common to refer to the robot along with the learning algorithm as the agent and the environment as everything external to the robot. This paper adopts this latter convention to distinguish the robot from its surroundings.

Unfortunately, applying RL directly in real-life applications is often prohibitive, requiring the robot to collect data by iteratively acting on the environment~\cite{levine2020offline}. This iterative process can be costly and time-consuming. There may be safety concerns because the robot may take unexpected actions during training, possibly causing damage to itself or its surroundings.

\begin{figure*}
    \centering
    \includegraphics[width=0.85\linewidth]{figures/simplified_diagram_2.pdf}
    \caption{Simplified diagram describing the components of the proposed RL pipeline. Each component is optional, and the combination of the stages depends on the problem's complexity. 
    %A simple RL problem could only utilize the core simulation training stage, while more complicated approaches could take advantage of the whole pipeline.
    }
    \label{fig:simplified_diagram}
\end{figure*}

The success of RL in robotics is partly due to the development of a methodology known as off-policy RL, which enables offline training~\cite{levine2020offline} where the RL training process happens after data collection. Algorithms such as Q-learning~\cite{sutton2018reinforcement}, SAC~\cite{haarnoja2018soft}, and TD3~\cite{fujimoto2018addressing} are examples of off-policy RL. Another advantage of off-policy RL is that it allows for the use of simulated data during training. Unfortunately, this approach introduces other challenges, because discrepancies between the simulated and the real world may cause the robot to perform worse when compared to its simulated counterpart. This problem, often referred to by various terms such as the reality gap, sim-to-real gap, and sim2real gap~\cite{jakobi1995noise}, arises because off-policy RL is usually trained mostly, if not entirely, in simulation. Some studies focus on reducing this problem~\cite{chebotar2019closing, calderon2024deep, zhao2020sim}, and strategies they apply include (\textit{i}) creating high-fidelity simulations with parameters modelled directly from the robot, and (\textit{ii}) randomizing the simulation parameters to generate a policy that is robust to environmental variations. %More details on common approaches are discussed in Section \ref{sec:related}.

Considering these methodologies, this work presents a pipeline to facilitate the development of RL models on real-world robotic systems. %We introduce the pipeline through two diagrams: 
A simplified overview of each phase's components is shown in Fig.\ \ref{fig:simplified_diagram}. %, and a more detailed block diagram outlining inputs, outputs, and data flow in Section \ref{sec:pipeline}. 
The proposed pipeline combines established methodologies from the RL literature into a staged process that involves system identification, training with different levels of simulation complexity, and real-world deployment. 
% Each stage is optional and introduces increasing complexity. 
~While the individual techniques included in the pipeline are not novel, this paper's contribution lies in a systematic approach to the learning process. Unlike most studies, which often detail only the specific steps for their application, this pipeline offers a modular framework to guide practitioners in adapting RL models to real-world scenarios. To support this, a case study demonstrating the successful deployment of an RL policy on a mobile robot is also presented.

The remainder of the paper is organized as follows. Section~\ref{sec:related} presents common approaches to reduce the reality gap; Section \ref{sec:pipeline} describes the proposed pipeline and discusses its application in robotics; Section \ref{sec:case_study} describes the steps taken to train an RL policy using the proposed pipeline; Section \ref{sec:application} illustrates the results in a surveillance application. Finally, Section \ref{sec:conclusion} presents the final remarks.

\section{Related Work}
\label{sec:related}

This section presents previous studies on the simulation-to-reality gap and links these concepts to the proposed pipeline. For this purpose, it is necessary to define the concept of environment used in this paper. In robotic RL applications, the environment includes not only the space where the robot operates but also the robot itself, and the agent is the decision-making algorithm or policy that sends commands via the robot.  However, it is also common to refer to the robot along with the learning algorithm as the agent and the environment as everything external to the robot. This paper adopts this latter convention to distinguish the robot from its surroundings.

Research on RL in robotic systems has led to various methodologies for improving sample efficiency of RL algorithms and reducing the sim-to-real problem. The sample efficiency in RL refers to the number of samples (i.e., experiences) needed to achieve a certain performance level. A comprehensive survey on the topic is presented by Calderón-Cordova et al.~\cite{calderon2024deep}. They present many techniques, frameworks, tools, and a practical guide for developing RL control applications focused on robotic manipulators. Even though their work is extensive, they propose a simple pipeline containing a single simulation stage with no discussion of the various levels of simulation complexity and their effects on the reality gap. 

In contrast,~\cite{zhu2021survey} presents a survey on RL applied to bio-inspired robots that categorize RL methodologies into four main groups: 1) methods that rely on accurate simulators; 2) approaches that only use simplified kinematic or dynamic models; 3) techniques that apply RL on top of hierarchical controllers; and 4) methods that leverage human demonstrations. They also mention that while methodologies in group 1 often perform better after a sim-to-real transfer, they are less sample-efficient than the others. Our proposed pipeline supports not only a single simulation stage, as presented in~\cite{calderon2024deep} but also allows for multi-level simulation complexities. This includes options to integrate high-fidelity simulators and simplified kinematic or dynamic models as core simulators.

In another survey, Zhao et al.~\cite{zhao2020sim} discuss not only the potential benefits of high-fidelity simulation in improving the sim-to-real transfer but also present three other approaches: system identification to create a simulator tailored to a specific robot, domain randomization, and domain adaptation. Domain randomization methods involve modelling parameters from reality and randomizing their values in the simulator to cover the actual distribution of these parameters in the real world. On the other hand, domain adaptation involves the combination of two or more environments during the training process. For example, one could train a model in a high-fidelity simulator and then transfer the model to training on real-world data.

% Successful examples of domain randomization include the works of Huber et al.~\cite{huber2024domain}, Peng et al.~\cite{peng2018sim}, and Akkaya et al.~\cite{akkaya2019solving}. 
Several works successfully use domain randomization~\cite{huber2024domain,peng2018sim,akkaya2019solving}. In~\cite{huber2024domain}, a feedback approach is used to change the simulation parameters based on how the real robot performs with the transferred policy, thus allowing for automatic randomization to achieve high performance on the transferred policy. In contrast,~\cite{peng2018sim} performs a comprehensive randomization of parameters (e.g., table size, arm dynamics, controller gains, links' mass, friction, noise levels, and time steps) on a robotic arm application, %such as modifying table size, dynamics of the arm, controller gains, mass of the links,  friction coefficients, noise levels, time step variations, 
totalling 95 randomized parameters, resulting in a successful sim-to-real transfer without additional training on the physical system. Using a combination of system identification techniques, domain randomization and curriculum learning, \cite{akkaya2019solving} obtained a remarkable level of dexterity in a robotic hand when trained to solve the Rubik's cube.

% The task of controlling a robotic hand to solve the Rubik's cube is discussed in~\cite{akkaya2019solving}.The level of dexterity obtained was remarkable due to a combination of system identification techniques, domain randomization and curriculum learning .%, where they increased the task's difficulty as training progressed. 

Domain adaptation techniques involve converting input data from one domain into another, where most training is performed. For example,~\cite{james2019sim} develops a machine-learning model that converts the visual feedback data from the real world into a format that resembles simulated inputs. This translation technique allows a model trained in simulation to perform well on physical robots. Similar strategies are used in~\cite{bousmalis2018using}.

Note that these methodologies are not always necessary. For example,~\cite{hu2021sim} presents a sim-to-real pipeline to address the challenge of robot navigation in 3D cluttered environments. Their pipeline includes a simulation stage that matches the inputs and outputs to the real robot, and uses simulated sensors and state estimation techniques that closely matched the real ones. These steps achieved successful real-world transfer without requiring any adaptation or additional training. While control problems like these could be solved with classical control techniques, RL provides a data-based adaptive methodology that is independent of the robot model.

Ultimately, the requirements of the RL process depend on the complexity of the task. If the robot is passively stable and accepts simple commands, such as linear and angular velocities, it is possible to transfer the learned model without further adaptation~\cite{hu2021sim}. However, more complex problems, such as solving a Rubik's cube with 24 degrees of freedom, require more steps to achieve acceptable performance in real applications~\cite{akkaya2019solving}. %The proposed pipeline includes many components used in their applications, supporting various levels of system complexity.






\section{Proposed Reinforcement Learning Pipeline}
\label{sec:pipeline}

\begin{figure*}
    \centering
    \includegraphics[width=1\linewidth, trim=0cm 0cm 0cm 0.4cm, clip]{figures/training_pipeline_2.pdf}
    \caption{The proposed training pipeline involves three stages with increasing levels of complexity from left to right. Each stage is optional and can be revisited with different parameters until they pass a predefined performance criteria. %Each stage follows the RL data flow illustrated in Fig.\ \ref{fig:core_RL_diagram}. However,
    Each stage receives as input a policy and outputs a modified policy, allowing incremental improvement until the final policy is achieved.}
    \label{fig:training_pipeline}
\end{figure*}


The pipeline is proposed as a multi-stage process to turn RL policies into real-world applications as presented in Fig.~\ref{fig:simplified_diagram}, with four phases: system identification, core simulation training, high-fidelity training, and real-life training, which are explained further in this section.
Fig.\ \ref{fig:training_pipeline} describes the data flow and interactions between the RL agent and the environments during training. This design allows the customization of each phase to meet the specific requirements of a task. For example, tasks that involve manipulating physical objects (e.g., a Rubik’s cube~\cite{akkaya2019solving}) may benefit from utilizing all stages of training. In contrast, more straightforward tasks with no interaction with other objects, like the position control of a wheeled vehicle, might only require the system identification and core simulation stages. This flexibility is valuable, especially for researchers and industry practitioners new to RL.

The following subsections provide a detailed description of each component and their interactions in the proposed pipeline.



%The proposed pipeline, as shown in Fig.\ \ref{fig:simplified_diagram} and \ref{fig:training_pipeline}, is designed as a multi-stage process that transitions RL policies from simplified simulation environments to high-fidelity simulations and ultimately to real-world applications. Each phase can be customized to meet the specific requirements of a robotic task or even omitted, allowing practitioners to modify each component based on task complexity and available resources. This structured methodology can be valuable for researchers and industry practitioners new to RL. 

%The following subsections provide a detailed description of each component in the proposed pipeline.

\subsection{System Identification}

This stage focuses on obtaining a data-based model of the robot to be applied in the following stages. Although optional, this step can be crucial to reduce the reality gap by considering the robot's physical parameters in the simulators used in the following stages. The system identification field is vast, and techniques include identifying linear or nonlinear models that convert inputs to outputs, frequency response analysis, machine learning, and many others. %It could also be used to identify noise distributions and uncertainty levels. % in the robot. 
A good resource on system identification is provided in~\cite{brunton2022data}. Section \ref{sec:case_study} provides further details on the system identification technique used in the presented case study. 

\subsection{Core Simulation Training}

This is the first training stage in Fig.\ \ref{fig:training_pipeline}, and it allows the RL agent to learn within a simplified simulation environment. This phase could involve modelling either the kinematic or dynamic equations of motions. For example, an RL agent could be trained to control a differential-drive robot by modelling the kinematic equations while ignoring factors like friction, motor mismatch, and other sources of error, such as in \cite{hu2021sim}, where RL is applied to control a wheeled robot to navigate in rough terrain. While this simplified approach might allow the policy to transfer to the actual robot with minimal training, the transfer could also result in policy degradation. Incorporating system identification parameters into the simulation could ensure smoother transfer and more reliable performance.


The passing criteria in this and subsequent stages can be a numerical function that determines if the model passes a predefined performance score or based on practitioners' prior experience to assess model feasibility. Examples of performance metrics include the success rate for goal-based problems or the accumulated reward.
% or a decision made by the practitioner about whether to continue improving the model with further changes in the training process or to move to a future stage. 
The tools used in this stage include the gymnasium API developed for creating RL simulators and physics simulators, such as Bullet and MuJoCo, that can be used to implement rigid body dynamics.

\subsection{High-Fidelity Training}

The second training stage in Fig.\ \ref{fig:training_pipeline} involves high-fidelity simulation, incorporating realism such as gravity, friction, actuator latency and dead zones, sensor noise, and realistic renderings. This reduces the sim-to-real gap, critical for complex RL tasks. With high-fidelity simulators, domain randomization is a powerful technique to further reduce the sim-to-real gap. For example, domain randomization (e.g., cube size, friction, force ranges, inertia, and action latency) enhanced the training process in the Rubik's cube robot manipulation task~\cite{akkaya2019solving}, where it enabled successful transfer into the real robot.

Common tools used in this stage, such as Gazebo, CoppeliaSim and Isaac Sim, offer advanced features like accurate sensor modelling, real-robot input/output matching and Robot Operating System (ROS) support, enabling near-seamless transitions between simulated and real environments.

\subsection{Real-Life Training}

In the final phase, the RL model is deployed on the physical robot, and the performance degradation is evaluated. If performance is inadequate, fine-tuning the model with real-world data or addressing discrepancies through high-fidelity simulation, using either domain randomization or domain adaptation, can reduce the sim-to-real gap. This iterative process continues until the model meets the desired performance criteria.

% \subsection{Incorporating other RL methodologies in our pipeline}
% \textcolor{red}{I was planning to explain how to apply the following concepts while following the pipeline:
% Domain randomization
% Domain adaptation
% Curriculum learning
% . However, I am running out of space and I think this is not essential for the paper so I am thinking of ignoring this subsection for now.}


\subsection{Debugging RL Applications with the Pipeline}

RL requires careful integration of components, including actions, observations, reward signals, simulation, RL algorithms, and well-tuned hyperparameters. Complex robotics tasks with continuous observations and actions often require multiple neural networks, further increasing the number of hyperparameters that must be optimized.

More often than not, the initial training fails to converge to a useful policy, with issues stemming from suboptimal hyperparameters, insufficient observations, or neural networks that need more neurons or layers. In these situations, a common solution is to start with a simplified simulation and minimal observations and actions, making it easier to find workable parameters. This process can then iterate, gradually increasing task complexity or progressing through the pipeline stages until the desired performance is obtained.




\section{Case Study on a Mobile Robot}
\label{sec:case_study}

This section presents a case study on applying the discussed concepts to obtain an RL policy for the Boston Dynamics Spot robot. Spot, an agile legged robot, autonomously computes gait and foot placement on linear ($a_x$), lateral ($a_y$) and angular body velocities ($a_\theta$) commands. The objective was to train an RL model to control the robot's position and orientation to reach a desired configuration while optimizing a cost function. This controller was then used in a surveillance application.

\subsection{System Identification}
\label{sec:system_id}

In this stage, the objective was to reduce the sim-to-real gap by modelling the robot's motion constraints. The robot cannot perfectly execute commanded velocities because converting desired speeds into leg movements and actuation limits introduces errors. Training an RL agent solely on commanded velocities $\mathbf{a} = \left(a_x,a_y,a_\theta\right)$ risks poor control because these may not align with what the robot can physically achieve.

The first step was identifying the velocities the robot could execute. To do so, a grid of commanded body velocities $\mathbf{a}$  was created (Fig.\ \ref{fig:action_set_a}), and the executed body velocities $\mathbf{v} = \left(v_x,v_y,v_\theta\right)$ were measured using a motion capture system, with grid ranges shown in Table \ref{tab:system_params}. Note that $a_x$ is not symmetrical because the robot moves faster forward than backward, adding control complexity, which the RL agent can still learn to handle.

\begin{figure}%[t] 
  \begin{subfigure}[b]{0.45\linewidth}
    \centering
    \includegraphics[width=1.0\linewidth, trim=5.2cm 1.3cm 3.2cm 2.0cm, clip]{figures/action_set_input.pdf}
    \caption{} 
    \label{fig:action_set_a} 
  \end{subfigure}%% 
  \begin{subfigure}[b]{0.45\linewidth}
    \centering
    \includegraphics[width=1.0\linewidth, trim=5.0cm 1.3cm 3.4cm 1.0cm, clip]{figures/action_set_output.pdf} 
    \caption{} 
    \label{fig:action_set_b}
  \end{subfigure}%%
  \caption{Action set of the Spot robot. (a) Nominal and (b) feasible velocities with overlaid approximated velocities. }
  \label{fig:action_set} 
\end{figure}

\begin{figure}%[t]
\centering\includegraphics[width=0.9\linewidth]{figures/system_id_spot_h.pdf}
    \caption{The executed velocities of the real robot are approximated using a polynomial function approximator.}
    \label{fig:fuction_approximation}
\end{figure}

The system identification process involved finding a function that approximates the executed velocities from commanded velocities (Fig.~\ref{fig:fuction_approximation}), represented as
\begin{equation}
    \hat{\mathbf{v}} = \hat{\mathbf{f}}(\mathbf{a}),
\end{equation}
where  $\hat{\mathbf{v}}$ is the approximated body velocity and $\hat{\mathbf{f}}(\mathbf{a})$ is a multivariate function with a third-order polynomial with no bias term in each dimension of $\mathbf{a}$, resulting in:

\begin{equation}
    \hat{\mathbf{v}} = \begin{bmatrix}
           \hat{v}_x \\
           \hat{v}_y \\
           \hat{v}_{\theta}
         \end{bmatrix}
     =
     \begin{bmatrix}
         \hat{f}_x(\mathbf{a})\\
         \hat{f}_y(\mathbf{a})\\
         \hat{f}_\theta(\mathbf{a})
     \end{bmatrix},
\end{equation}
where $\hat{f}_x(\mathbf{a})$, $\hat{f}_y(\mathbf{a})$, and $  \hat{f}_\theta(\mathbf{a})$ are the polynomial functions that approximate the executed velocities in each dimension. For clarity, the approximation for $\hat{v}_x$ is given by
\begin{equation}
    \hat{v}_x = \hat{f}_x(\mathbf{a}) = \sum_{0 < i+j+l \leq 3} c_{x,ijl} (a^i_x a^j_y a^l_\theta),
\end{equation}
 where $i, j, l \in \mathbb{Z}_{\geq 0}$ and the coefficients $c_{x,ijl}$ are computed using the least squares method that minimizes the total squared error
\begin{equation}
    J_x = \sum_{n=0}^{N-1} (v_{x,n} - \hat{v}_{x,n})^2,
    % J = (\mathbf{v} - \hat{\mathbf{v}})(\mathbf{v} - \hat{\mathbf{v}})^\top.
\end{equation}
where $N$ represents the total number of samples in the grid (Fig. \ref{fig:action_set_a}), and the $n$ index represents the $n$-th element. The coefficients of $\hat{v}_y$ and $\hat{v}_{\theta}$ are computed similarly. 

Fig.\ \ref{fig:action_set_b} shows the linear regression results with the approximated velocities overlaid on the executed ones, illustrating a good fit from the polynomial regression.

\subsection{RL Formulation}

The problem was modelled as a goal-conditioned Markov Decision Process (MDP) $\langle S, G, A, T, R \rangle$, where $S$ is the state space, $G$ the goal space, $A$ the action space, $T: S\times A \rightarrow S$ the state transition function, and $R(\mathbf{a}, \mathbf{s}, \mathbf{g}),~R: S \times A\rightarrow\mathbb{R}$ the reward function that returns a scalar when taking action $\mathbf{a}$ and arriving at state $\mathbf{s}$ with goal $\mathbf{g}$. The goal is to find a policy $\pi(\mathbf{a}|\mathbf{s},\mathbf{g})$ that maximizes the agent's cumulative reward~\cite{antonyshyn2023multiple}. The policy $\pi(\mathbf{a}|\mathbf{s},\mathbf{g})$ defines a probability distribution over the action space $A$ given a combination of state $\mathbf{s}$ and goal $\mathbf{g}$. The policy can be applied probabilistically, where each action $\mathbf{a}$ is a sample of the probability distribution, or deterministically, where $\mathbf{a}$ is the distribution mean.

\subsubsection{State and Goal Spaces}

Since the goal of the RL agent is to control the position and orientation of the robot, the state was defined as
\begin{equation}
    \mathbf{s} = (x, y, \theta)\in\mathbb{R}^2\times[-\pi,\pi).
\end{equation}
Similarly, the goal is defined as
\begin{equation}
    \mathbf{g} = (x_g,y_g,\theta_g) \in[r_{\rm min},r_{\rm max}]^2\times[-\pi,\pi),
\end{equation}
where $r_{\rm min}$ and $r_{\rm max}$ represent the lower and upper limits of $x_g$ and $y_g$. Note that the state and goal combination form the observation $\mathbf{o} = (\mathbf{s},\mathbf{g})$, but separating them into state and goal vectors is more intuitive.

\subsubsection{Action Space}

The commanded action was defined as 
\begin{equation}
    \mathbf{a} = (a_x, a_y, a_\theta) \in \mathbb{R}^3.
\end{equation}
They represent the linear, lateral and angular body velocities commands sent to the robot. 

\subsubsection{Simulation and Transition Function}

%This section presents the simulator used in our pipeline's core simulation stage. 
The simulator used in the core simulation stage converts the desired action at time step $k$ into the identified executed action, as shown in Section \ref{sec:system_id}, and computes a new state resulting from the applied action. The new state is obtained through a first-order integration of the executed velocity as
\begin{align}
    \hat{\mathbf{v}}_k &= \hat{f}(\mathbf{a}_k) \\
    \mathbf{s}_k &= \mathbf{s}_{k-1} + \hat{\mathbf{v}}_k\Delta t,
\end{align}
with $\Delta t$ being the step duration.

Since this is a goal-conditioned MDP, our simulator must also inform when the robot reaches $\mathbf{g}$. To do so, the error in position and orientation are computed
\begin{align}
    e_{p,k} &= \|\mathbf{p}_g-\mathbf{p}_{s,k}\|_2, \\
    e_{\theta,k} &= |\theta_g - \theta_{k}|,
\end{align}
where $\mathbf{p}_g = (x_g, y_g)$, $\mathbf{p}_{s,k} =(x_k, y_k)$, and $||\cdot||_2$ represents the Euclidean norm. Then, the robot successfully reaches $\mathbf{g}$ when $e_{p,k} < \epsilon_p$ and $e_{\theta,k} < \epsilon_\theta$, where $\epsilon_p$ and $\epsilon_\theta$ are tolerance parameters that define the required proximity to the goal.

\subsubsection{Reward Function}

The reward function for this study was chosen based on common costs used in model predictive control applications. For interested readers on this topic, the authors of \cite{song2023reaching} present an insightful comparison between optimal control and reinforcement learning techniques and their costs. Our objective was to optimize a policy that minimizes control actions, action smoothness, and time, resulting in the following cost function:
\begin{equation}
\label{eq:cost_function}
J = \sum_{k=0}^{N-1}\left(\|\mathbf{u}_k\|_{\mathbf{R}}^2 + \|\mathbf{u}_k-\mathbf{u}_{k-1}\|_{\mathbf{S}}^2 + \lambda_k\right),
\end{equation}
where the notation $\|\mathbf{z}\|^2_\mathbf{M}= \mathbf{z}^\top \mathbf{M} \mathbf{z}$ represents the weighted Euclidean norm, with $\mathbf{z}$ as a column vector and $\mathbf{M}$ a square matrix matching the dimension of $\mathbf{z}$. The parameters $\mathbf{R}$, $\mathbf{S}$ prioritize action magnitude and variation, respectively, while $\lambda_k$ is a time-varying factor that encourages faster task completion. For example, the parameters $\lambda_k = 1$ and $\mathbf{R} = \mathbf{S} = \mathbf{0}$ define the minimum time optimal control problem \cite{rosolia2021minimum}.

Since RL maximizes a reward rather than minimizing a cost, the cost function in \eqref{eq:cost_function} is converted into a reward as follows,
\begin{equation}
\label{eq:reward}
    r_k =  -\left(\|\mathbf{u}_k\|_{\mathbf{R}}^2 + \|\mathbf{u}_k-\mathbf{u}_{k-1}\|_{\mathbf{S}}^2 + \lambda_k\right).
\end{equation}

Maximizing \eqref{eq:reward} over time corresponds to minimizing the cost in \eqref{eq:cost_function}. However, to further encourage the RL agent to reach the goal state, the time penalty is removed once the robot is within a specified proximity to the goal. Thus, we define:
\begin{equation}
\lambda_k = 
\begin{cases} 
      0, & \text{if }  e_{p,k} < \epsilon_p \text{ and } e_{\theta,k} < \epsilon_\theta \\ 
      1, & \text{otherwise} 
   \end{cases}
\end{equation}


\subsubsection{Neural Network Architecture}

In recent years, several algorithms, like DDPG~\cite{lillicrap2015continuous}, SAC~\cite{haarnoja2018soft}, and TD3~\cite{fujimoto2018addressing}, have been created for continuous state and action spaces.  Considering the pool of deep RL algorithms, we chose the Soft Actor-Critic Algorithm (SAC)~\cite{haarnoja2018soft} due to being easy to find hyperparameters that lead to good policies. The stochastic nature of SAC also helps with environment exploration, which reduces the number of parameters to tune. The RL training parameters and network architectures used are listed in Table~\ref{tab:system_params}, and a visual representation of the network architecture for the actor-critic framework is presented in Fig.\ \ref{fig:actor_critic_architecture}.

\begin{figure} 
  % \begin{subfigure}{1.0\linewidth}
    \centering
    \includegraphics[width=1.0\linewidth, trim=0cm 0cm 0cm 0cm, clip]{figures/actor_critic_architecture_2.pdf}
  %   \caption{} 
  %   \label{fig:a} 
  % \end{subfigure}%% 
  % \vspace{0.3cm} % Adjust vertical spacing as needed
  % \begin{subfigure}{0.99\linewidth}
  %   \centering
  %   \includegraphics[width=0.5\linewidth, trim=0.8cm 0.6cm 0.5cm 1.5cm, clip]{figures/RL_actor_critic_diagram.pdf} 
  %   \caption{} 
  %   \label{fig:c}
  % \end{subfigure}%%
  \caption{Actor-critic network architecture}
  \label{fig:actor_critic_architecture} 
\end{figure}

\begin{figure*}
  \begin{subfigure}[b]{0.33\linewidth}
    \centering
    \includegraphics[width=1.0\linewidth, trim=0.3cm 0.3cm 0.2cm 0.1cm, clip]{figures/Gymnasium_simulation__theta_g__0.pdf}
    \caption{Core simulation} 
    \label{fig:policy_results_a} 
  \end{subfigure}%% 
  \begin{subfigure}[b]{0.33\linewidth}
    \centering
    \includegraphics[width=0.71\linewidth, trim=0.3cm 0.3cm 0.2cm 0.1cm, clip]{figures/Gazebo_simulation__theta_g__0.pdf} 
    \caption{High-fidelity simulation} 
    \label{fig:policy_results_b}
  \end{subfigure}%%
 \begin{subfigure}[b]{0.33\linewidth}
    \centering
    \includegraphics[width=0.71\linewidth, trim=0.3cm 0.3cm 0.2cm 0.1cm, clip]{figures/Actual_robot__theta_g__0.pdf}
    \caption{Robot execution} 
    \label{fig:policy_results_c} 
  \end{subfigure}%% 
  \caption{Execution of the policy at multiple goals with $\theta_g = 0$ with (a) showing the core simulation trajectories, (b) the high-fidelity simulation trajectories, and (c) the trajectories executed by the real robot.}
  \label{fig:policy_results} 
\end{figure*}
In SAC, two networks are used: the actor and the critic. The actor network is the policy itself, taking the observation (state and the goal) as input and outputting action probabilities in terms of mean and deviation for each element. The critic network takes both the observation and a sampled action as input, and outputs estimates of the expected return of that action given the current policy (Q-value). This Q-value guides the actor’s updates, helping SAC refine the policy by assessing the value of actions to improve the long-term rewards.

\begin{table}
\caption{Simulation, RL, and neural network parameters.}
\begin{center}
\begin{threeparttable}
\begin{tabular}{l c}
    \toprule
     \textbf{Parameter} & \textbf{Value}\\ \midrule
    
    Simulation frequency & $30$ Hz \\
    Policy frequency & $10$ Hz \\
    Observation standard deviation & $0.01$ \\
    % $\epsilon_p$  &   0.1 m   \\
    % $\epsilon_\theta$  &   $\pi/8$ rad   \\
    $\mathbf{R}$ & $\textup{diag}(0.0, 0.8, 0.8)$\\
    $\mathbf{S}$ & $\textup{diag}(0.2, 0.2, 0.2)$\\
    Range of $a_x$ & $[-0.8, 1.1]$ m/s\\
    Range of $a_y$ & $[-0.7, 0.7]$ m/s\\
    Range of $a_\theta$ & $[-1.1, 1.1]$ rad/s\\
    $[r_{\rm min}, r_{\rm max}]$ & $[-2,2]$ \\
    \midrule
   
    Batch size   &   $512$\\
    $\tau$ & $0.0045$ \\
    Discount factor   &  $0.999$\\
    Learning rate   &   \SI{2e-4}{} \\
    Buffer size   &  \SI{1e6}{} \\
    Number of training steps & \SI{3e5}{} \\
    \midrule
    Actor neurons & $16$ \\
    Actor hidden layers & $2$\\
    Critic neurons & $128$ \\
    Critic hidden layers & $2$ \\
    Hidden layer activation function & ReLu\\
    Output layer function & linear \\

    \bottomrule
\end{tabular}
\end{threeparttable}
\end{center}
\label{tab:system_params}
\end{table}  

\subsection{Training and Verification Process}

The core simulation training stage was performed by using Gymnasium~\cite{towers2024gymnasium}, which is a library that simplifies the creation of simulators tailored for RL applications, and the SAC neural network was sourced from the stable-baselines 3 library~\cite{stable-baselines3}. The simulation parameters are shown in Table~\ref{tab:system_params}. To accelerate learning, another concept called Hindsight Experience Replay (HER)~\cite{andrychowicz2017hindsight} was also applied. HER is a technique that generates additional training data from failed episodes by redefining the goal to a point the robot reached and adjusting rewards accordingly. This approach enables the agent to learn not only from successes but also from failures.

Two additional training strategies were applied in this study: curriculum learning and randomization. Curriculum learning was applied to the goal tolerances $\epsilon_p$ and $\epsilon_\theta$, starting with the tolerances covering $80$~\% of the training region. When the robot reached a $95$~\% success rate, the tolerances were reduced by $20$~\% until it reached $\epsilon_p = 0.05$ m and $\epsilon_\theta = \ang{1}$. Looking at the pipeline in Fig.\ \ref{fig:training_pipeline}, this can be viewed as following the feedback loop in the core simulator until the RL agent achieves the desired tolerances. Randomization was applied to the robot's state, introducing noise in the robot's position and orientation to encourage the RL agent to learn how to control the robot under uncertain conditions, reducing the reality gap.

Once the policy achieved a $100$~\% success rate in the core simulator stage with the final tolerance levels, it was transferred to the high-fidelity simulation stage using the Gazebo simulator, which provides a physics engine, realistic sensor emulation, and a communication layer with ROS. Because the problem did not involve interacting with physical objects, there was no necessity to continue training the policy on the Gazebo simulator. However, running the policy on Gazebo was still crucial, as integration with ROS enabled the use of the same code, and localization and navigation architecture as on the real robot, allowing the algorithm to be tested under conditions that closely replicate the real-world environment. %More details on the navigation system are shown in Section~\ref{sec:application}.

After testing the RL agent in Gazebo, the policy was deemed ready for deployment on the real robot. Initial deployments yielded positive results, though some discrepancies were observed compared to the Gazebo simulation. Specifically, the RL agent struggled to stop the robot in certain scenarios, such as the goal at $x = 0$ m and $y = 2$ m in Fig.~\ref{fig:policy_results_c}., leading to overshooting and oscillations around the goal position. This was caused by the robot's latency in stopping due to inertia and a minimum action duration not accounted for in the core simulator. Thus, the tolerance to reach the goal was increased to $\epsilon_p = 0.3$ m and $\epsilon_\theta = \ang{17}$. This modification eliminated the oscillations and proved sufficient for our application, allowing us to avoid further training. The need for increased tolerances is a consequence of the reality gap. If more precise positioning is required, the pipeline allows for more training using either real-life data during training or by incorporating the robot oscillations in the high-fidelity simulator.

Fig.~\ref{fig:policy_results} shows the robot trajectories across all three stages. The  Gymnasium simulation (Fig.\ \ref{fig:policy_results_a}) shows accurate trajectories, as anticipated. Note that trajectories requiring lateral movement include some  forward/backward motion due to the RL agent being optimized for faster executions rather than shorter trajectories. In the Gazebo runs (Fig.\ \ref{fig:policy_results_b}), the tolerances were adjusted to match those used in the actual robot stage, and minor variations in the robot's starting positions were introduced to create a more realistic setup. Despite these differences, the trajectories were similar to the Gymnasium stage. Finally, in the real robot execution (Fig.\ \ref{fig:policy_results_c}), the trajectories closely matched the Gazebo results, except for a delay in the robot stopping upon reaching the goal regions.


% \vspace{-1mm}
\section{Applications of NTL}
\label{sec:applications}


NTL supports different applications, depending on which data are used as source and target domain. We introduce two applications in model intellectual property (IP) protection and then the application of harmful fine-tuning defense. 

\paragraph{Ownership verification (OV).} OV is a passive IP protection manner, which aims to verify the ownership of a deep learning model \cite{cheng2021mid,lederer2023identifying}. NTL solves ownership verification by triggering misclassification on data with pre-defined triggers \cite{wang2021non,chen2024mark,guo2024zeromark}. For example, when training, we add a shallow trigger (only known by the model owner) on the original dataset data and see them as the target domain, while the original data without the trigger is regarded as the source domain. Then, target-specified NTL is used to train a model. Therefore, the ownership can be verified via observing the performance difference of a model on the original data and the data with the pre-defined trigger. For SL model, the shallow trigger has minor influence on the model performance, and thus, the model shows similar performance on original data and data with triggers. In contrast, the NTL model specific to this pre-defined trigger has high performance on the original data but random-guess-like performance on data with the trigger. This provides evidence for verifying the model's ownership.
% 
\paragraph{Applicability authorization (AA).} AA is an active IP protection approach that ensures models can only be effective on authorized data \cite{wang2021non,xu2024idea,si2024iclguard}. NTL solves AA by degrading the model generalization outside the authorized domain. Basic solution is to add a pre-defined trigger on original data (seen as source domain), and the original data without the correct triggers is regarded as the target domain. After training by NTL, the model will only perform well on authorized data (i.e., the data with the trigger). Any unauthorized data will be randomly predicted by the NTL model. Thus, AA can be achieved.




\paragraph{Safety alignment and harmful fine-tuning defense.} 
Fine-tuning large language models (LLMs) with user's own data for downstream tasks has recently become a popular online service \cite{huang2024harmful,openai2024finetune}. However, this practice raises concerns about compromising the safety alignment of LLMs \cite{qi2023fine,yang2023shadow,zhan2023removing}, as harmful data may be present in users' datasets, whether intentionally or unintentionally. To address the risks of harmful fine-tuning, various defensive solutions \cite{huang2024booster,rosati2024representation,huang2024vaccine} have been proposed to ensure that fine-tuned LLMs can effectively refuse harmful queries. Specifically, these defense methods aim to limit the transferability of LLMs from harmless queries to harmful ones, which techniques are variants of the objectives of NTL. 
Actually, all existing NTL approaches can be applied to this task by regarding the alignment data as the source domain and the harmful data as the target domain. Then, target-specified NTL can be conducted to defend agaginst harmful fine-tuning attacks.

\section{Conclusion}
We introduce a novel approach, \algo, to reduce human feedback requirements in preference-based reinforcement learning by leveraging vision-language models. While VLMs encode rich world knowledge, their direct application as reward models is hindered by alignment issues and noisy predictions. To address this, we develop a synergistic framework where limited human feedback is used to adapt VLMs, improving their reliability in preference labeling. Further, we incorporate a selective sampling strategy to mitigate noise and prioritize informative human annotations.

Our experiments demonstrate that this method significantly improves feedback efficiency, achieving comparable or superior task performance with up to 50\% fewer human annotations. Moreover, we show that an adapted VLM can generalize across similar tasks, further reducing the need for new human feedback by 75\%. These results highlight the potential of integrating VLMs into preference-based RL, offering a scalable solution to reducing human supervision while maintaining high task success rates. 

\section*{Impact Statement}
This work advances embodied AI by significantly reducing the human feedback required for training agents. This reduction is particularly valuable in robotic applications where obtaining human demonstrations and feedback is challenging or impractical, such as assistive robotic arms for individuals with mobility impairments. By minimizing the feedback requirements, our approach enables users to more efficiently customize and teach new skills to robotic agents based on their specific needs and preferences. The broader impact of this work extends to healthcare, assistive technology, and human-robot interaction. One possible risk is that the bias from human feedback can propagate to the VLM and subsequently to the policy. This can be mitigated by personalization of agents in case of household application or standardization of feedback for industrial applications. 


\balance
\bibliographystyle{ieeetr} 
% \bibliography{mybibfile} 
\begin{thebibliography}{10}

\bibitem{minsky1961steps}
M.~Minsky, ``Steps toward artificial intelligence,'' {\em Proceedings of the IRE}, vol.~49, no.~1, pp.~8--30, 1961.

\bibitem{sutton2018reinforcement}
R.~S. Sutton and A.~G. Barto, {\em Reinforcement Learning: An Introduction}.
\newblock A Bradford Book, 2018.

\bibitem{levine2020offline}
S.~Levine, A.~Kumar, G.~Tucker, and J.~Fu, ``Offline reinforcement learning: Tutorial, review, and perspectives on open problems,'' {\em arXiv preprint arXiv:2005.01643}, 2020.

\bibitem{haarnoja2018soft}
T.~Haarnoja, A.~Zhou, P.~Abbeel, and S.~Levine, ``Soft actor-critic: Off-policy maximum entropy deep reinforcement learning with a stochastic actor,'' in {\em International Conference on Machine Learning}, pp.~1861--1870, PMLR, 2018.

\bibitem{fujimoto2018addressing}
S.~Fujimoto, H.~Hoof, and D.~Meger, ``Addressing function approximation error in actor-critic methods,'' in {\em International Conference on Machine Learning}, pp.~1587--1596, PMLR, 2018.

\bibitem{jakobi1995noise}
N.~Jakobi, P.~Husbands, and I.~Harvey, ``Noise and the reality gap: The use of simulation in evolutionary robotics,'' in {\em Advances in Artificial Life: Third European Conference on Artificial Life Granada, Spain, June 4--6, 1995 Proceedings 3}, pp.~704--720, Springer, 1995.

\bibitem{chebotar2019closing}
Y.~Chebotar, A.~Handa, V.~Makoviychuk, M.~Macklin, J.~Issac, N.~Ratliff, and D.~Fox, ``Closing the sim-to-real loop: Adapting simulation randomization with real world experience,'' in {\em 2019 International Conference on Robotics and Automation (ICRA)}, pp.~8973--8979, IEEE, 2019.

\bibitem{calderon2024deep}
C.~Calderon-Cordova, R.~Sarango, D.~Castillo, and V.~Lakshminarayanan, ``A deep reinforcement learning framework for control of robotic manipulators in simulated environments,'' {\em IEEE Access}, 2024.

\bibitem{zhao2020sim}
W.~Zhao, J.~P. Queralta, and T.~Westerlund, ``Sim-to-real transfer in deep reinforcement learning for robotics: a survey,'' in {\em 2020 IEEE symposium series on computational intelligence (SSCI)}, pp.~737--744, IEEE, 2020.

\bibitem{zhu2021survey}
W.~Zhu, X.~Guo, D.~Owaki, K.~Kutsuzawa, and M.~Hayashibe, ``A survey of sim-to-real transfer techniques applied to reinforcement learning for bioinspired robots,'' {\em IEEE Transactions on Neural Networks and Learning Systems}, vol.~34, no.~7, pp.~3444--3459, 2021.

\bibitem{huber2024domain}
J.~Huber, F.~H{\'e}l{\'e}non, H.~Watrelot, F.~B. Amar, and S.~Doncieux, ``Domain randomization for sim2real transfer of automatically generated grasping datasets,'' in {\em 2024 IEEE International Conference on Robotics and Automation (ICRA)}, pp.~4112--4118, IEEE, 2024.

\bibitem{peng2018sim}
X.~B. Peng, M.~Andrychowicz, W.~Zaremba, and P.~Abbeel, ``Sim-to-real transfer of robotic control with dynamics randomization,'' in {\em 2018 IEEE International Conference on Robotics and Automation (ICRA)}, pp.~3803--3810, IEEE, 2018.

\bibitem{akkaya2019solving}
I.~Akkaya, M.~Andrychowicz, M.~Chociej, M.~Litwin, B.~McGrew, A.~Petron, A.~Paino, M.~Plappert, G.~Powell, R.~Ribas, {\em et~al.}, ``Solving rubik's cube with a robot hand,'' {\em arXiv preprint arXiv:1910.07113}, 2019.

\bibitem{james2019sim}
S.~James, P.~Wohlhart, M.~Kalakrishnan, D.~Kalashnikov, A.~Irpan, J.~Ibarz, S.~Levine, R.~Hadsell, and K.~Bousmalis, ``Sim-to-real via sim-to-sim: Data-efficient robotic grasping via randomized-to-canonical adaptation networks,'' in {\em Proceedings of the IEEE/CVF conference on computer vision and pattern recognition}, pp.~12627--12637, 2019.

\bibitem{bousmalis2018using}
K.~Bousmalis, A.~Irpan, P.~Wohlhart, Y.~Bai, M.~Kelcey, M.~Kalakrishnan, L.~Downs, J.~Ibarz, P.~Pastor, K.~Konolige, {\em et~al.}, ``Using simulation and domain adaptation to improve efficiency of deep robotic grasping,'' in {\em 2018 IEEE International Conference on Robotics and Automation (ICRA)}, pp.~4243--4250, IEEE, 2018.

\bibitem{hu2021sim}
H.~Hu, K.~Zhang, A.~H. Tan, M.~Ruan, C.~Agia, and G.~Nejat, ``A sim-to-real pipeline for deep reinforcement learning for autonomous robot navigation in cluttered rough terrain,'' {\em IEEE Robotics and Automation Letters}, vol.~6, no.~4, pp.~6569--6576, 2021.

\bibitem{brunton2022data}
S.~L. Brunton and J.~N. Kutz, {\em Data-driven science and engineering: Machine learning, dynamical systems, and control}.
\newblock Cambridge University Press, 2022.

\bibitem{antonyshyn2023multiple}
L.~Antonyshyn, J.~Silveira, S.~Givigi, and J.~Marshall, ``Multiple mobile robot task and motion planning: A survey,'' {\em ACM Computing Surveys}, vol.~55, no.~10, pp.~1--35, 2023.

\bibitem{song2023reaching}
Y.~Song, A.~Romero, M.~M{\"u}ller, V.~Koltun, and D.~Scaramuzza, ``Reaching the limit in autonomous racing: Optimal control versus reinforcement learning,'' {\em Science Robotics}, vol.~8, no.~82, p.~eadg1462, 2023.

\bibitem{rosolia2021minimum}
U.~Rosolia and F.~Borrelli, ``Minimum time learning model predictive control,'' {\em International Journal of Robust and Nonlinear Control}, vol.~31, no.~18, pp.~8830--8854, 2021.

\bibitem{lillicrap2015continuous}
T.~P. Lillicrap, J.~J. Hunt, A.~Pritzel, N.~Heess, T.~Erez, Y.~Tassa, D.~Silver, and D.~Wierstra, ``Continuous control with deep reinforcement learning,'' {\em arXiv preprint arXiv:1509.02971}, 2015.

\bibitem{towers2024gymnasium}
M.~Towers, A.~Kwiatkowski, J.~Terry, J.~U. Balis, G.~De~Cola, T.~Deleu, M.~Goul{\~a}o, A.~Kallinteris, M.~Krimmel, A.~KG, {\em et~al.}, ``Gymnasium: A standard interface for reinforcement learning environments,'' {\em arXiv preprint arXiv:2407.17032}, 2024.

\bibitem{stable-baselines3}
A.~Raffin, A.~Hill, A.~Gleave, A.~Kanervisto, M.~Ernestus, and N.~Dormann, ``Stable-baselines3: Reliable reinforcement learning implementations,'' {\em Journal of Machine Learning Research}, vol.~22, no.~268, pp.~1--8, 2021.

\bibitem{andrychowicz2017hindsight}
M.~Andrychowicz, F.~Wolski, A.~Ray, J.~Schneider, R.~Fong, P.~Welinder, B.~McGrew, J.~Tobin, O.~Pieter~Abbeel, and W.~Zaremba, ``Hindsight experience replay,'' {\em Advances in neural information processing systems}, vol.~30, 2017.

\end{thebibliography}


\end{document}

