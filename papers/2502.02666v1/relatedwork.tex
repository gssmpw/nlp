\section{Related works}
\label{sec2}
In this section, we review the literature in two key areas: UAV-based routing for persistent surveillance and recent advancements in learning-based approaches for routing problems.

\subsection{UAV based Persistent Surveillance Problem}
In recent years, UAV routing problems have garnered significant attention in modern disaster management due to their potential to address complex logistical challenges. Specifically, models such as the Traveling Salesman Problem with Drones (TSP-D) and the Vehicle Routing Problem with Drones (VRP-D) have been pivotal in developing efficient solutions for these challenges. Scherer et al. \cite{scherer2016persistent} explored the energy and communication constraints of UAVs, developing an offline path planning algorithm for multiple UAVs and assessing the impact of base station configurations on mission performance. Concurrently, Angun and Dundar \cite{angun2020intelligent} approached the surveillance problem from an energy-capacitation perspective, framing it as a location routing problem suitable for the risk mitigation phase of disaster management. In the context of emergency operations, Calamoneri et al. \cite{calamoneri2024management} proposed the Multi-Depot Multi-Trip Vehicle Routing Problem with Total Completion Times Minimization (MDMT-VRP-TCT). Their model facilitates a fleet of UAVs in visiting disaster-affected sites multiple times from various depots, thereby enhancing rescue efforts.
%Unlike standard surveillance, persistent surveillance demands constant revisitation of target points to ensure up-to-date information collection%. 
Nigam et al. \cite{nigam2014multiple} addressed the persistent surveillance problem by segmenting a specified 2D area into a grid map tailored to the sensor footprint, assigning UAVs to repeatedly monitor these grids. Further innovations in this field include the work of Stump and Michael \cite{michael2011persistent}, who employed the Vehicle Routing Problem with Time Windows, utilizing a receding horizon strategy to maintain ongoing surveillance over discrete targets. Despite these advancements, the issue of UAV fuel limitations has been relatively underexplored. Hari et al. \cite{hari2020optimal} introduced a model that restricts the number of UAV visits to reflect fuel constraints, determining the minimal maximum revisit duration among target nodes. Traditional setups assume the existence of fixed service stations for UAV recharging. However, recent studies suggest employing mobile UGVs as dynamic recharging units to enhance operational efficiency, an approach known as the UAV-UGV cooperative routing problem. This integration of heterogeneous vehicles complicates the problem, making exact Mixed Integer Linear Programming (MILP) solutions impractical due to scalability issues. To tackle this, heuristic methods have been devised to achieve high-quality, feasible solutions within reasonable time frames. For instance, Seyedi et al. \cite{seyedi2019persistent, lin2022robust} developed a heuristic that organizes a uniform UAV-UGV team to cyclically patrol optimally partitioned surveillance areas. While effective for area coverage, this strategy may not suit scenarios with discrete mission points due to its reliance on spatial partitioning. In heterogeneous vehicle cooperative routing, a prevalent tactic is the multi-level or multi-echelon optimization approach, such as ``UGV first, UAV second," which simplifies the complexity by breaking down the problem into more manageable subproblems. Maini et al. \cite{maini2015cooperation, maini2019cooperative} implemented a bi-level strategy, initially identifying potential rendezvous points via a minimum set cover problem, followed by formulating a MILP to ascertain optimal routes for both UAV and UGV. Innovatively, Nigam et al. \cite{nigam2008persistent} proposed a value-sum strategy directing UAVs towards areas where the cumulative value is highest or a target-based tactic focusing on cells with the greatest singular value determined by latency periods. Similarly, Chour et al. \cite{chour2022reactive} tackled a UAV-UGV rendezvous problem by creating a multi-level coordination, scheduling, and planning algorithm. They solved a reward-based traveling salesman problem using a receding horizon approach to establish the UAV’s route for UGV-based recharging. However, this method bears the inherent limitation of the \textit{horizon effect} \cite{asghar2023risk}, where a route deemed optimal within a planning horizon may prove suboptimal over the complete mission duration.



\subsection{DRL for UAV-UGV cooperative routing}

In recent times, learning-based methods have emerged as a viable alternative for addressing the routing problem, its variants, and other combinatorial optimization challenges. Most deep reinforcement learning (DRL) approaches involve end-to-end encoding and decoding processes for solving routing problems. The encoder extracts node features from inputs, while the decoder utilizes these features to sequentially select node sequences, training the neural model through reinforcement learning to preferentially output high-profit nodes at each time step. Vinyals et al. \cite{vinyals2015pointer} pioneered the Pointer Network (PN), enhancing a recurrent neural network (RNN) with an attention mechanism \cite{bahdanau2014neural} and training it in a supervised manner to predict optimal tours for the Traveling Salesman Problem (TSP). However, supervised learning methods for routing are constrained by the need for extensive and costly labels generated by pre-solving the routing problems which is a time-consuming process. Addressing these limitations, Kool et al. \cite{kool2018attention} introduced a Transformer-based encoder-decoder architecture that outperformed various traditional heuristic methods across a spectrum of routing problems. Similarly, Li et al. \cite{li2021deep} utilized a DRL strategy with attention mechanisms to significantly enhance solution quality and computational efficiency in tackling the heterogeneous capacitated vehicle routing problem. Furthermore, Wu et al. \cite{wu2021reinforcement} explored the truck-and-drone-based last-mile delivery problem using reinforcement learning. They effectively divided the optimization task into customer clustering and routing stages, employing an encoder-decoder framework with RL to address the challenge. Similarly, Fan et al. \cite{fan2022deep} employed a multi-head attention mechanism alongside a DRL policy to design routes for an energy-constrained UAV, but their model relied on fixed UAV recharging stations. Bana et al. \cite{bana2024deep} also applied DRL to persistent surveillance for fuel-limited UAVs, though their approach was similarly constrained by the assumption of fixed recharging stations. Building on these insights, this paper introduces a novel DRL framework that leverages an encoder-decoder transformer architecture and a policy gradient method to optimize coordinated routing between an energy-limited UAV and a UGV acting as a mobile recharging station. Tested across diverse problem sizes and scenarios, the proposed framework demonstrates significant adaptability and generalizability, offering a promising solution for UAV-UGV cooperative routing for persistent surveillance.