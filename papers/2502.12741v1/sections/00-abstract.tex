\abstract{
%
The Worldwide LHC Computing Grid (WLCG) provides the robust computing infrastructure essential for the LHC experiments by integrating global computing resources into a cohesive entity. Simulations of different compute models present a feasible approach for evaluating future adaptations that are able to cope with future increased demands. However, running these simulations incurs a trade-off between accuracy and scalability. For example, while the simulator DCSim can provide accurate results, it falls short on scaling with the size of the simulated platform. Using Generative Machine Learning as a surrogate presents a candidate for overcoming this challenge.

In this work, we evaluate the usage of three different Machine Learning models for the simulation of distributed computing systems and assess their ability to generalize to unseen situations. We show that those models can predict central observables derived from execution traces of compute jobs with approximate accuracy but with orders of magnitude faster execution times. Furthermore, we identify potentials for improving the predictions towards better accuracy and generalizability.
}