\section{Conclusion}
\label{sec:conclusion}

We propose using surrogate modeling to evaluate the throughput of different infrastructure designs. 
%
We train three model architectures using simulator data to predict different job observables.
%
From our evaluation results, the architecture choice does not significantly influence the accuracy of the predictions at the current stage of development.
%
All three architectures decrease the execution times by orders of magnitude compared to DCSim.

At the current stage of the models inaccuracies are observed.
We suspect a lack of input information given to the models as the predominant source.
While the models are able to predict the compute times of our heterogeneous jobs scenario, where some implicit information about the infrastructure can be extracted by the model, they fail to predict the transfer time of input files due to not being aware of the data infrastructure setup that is more complex. 

Future work can build on these results by incorporating platform information into the training data to improve the predictions.
This will become essential to ensure that the model performs accurately on arbitrary workload mixes.
%
Moreover, training on real-world data instead of simulator data could enhance the capabilities and applicability of the models, also providing valuable data in regimes where the simulation, due to its scaling behavior, is not able to feasibly produce large amounts of training data.