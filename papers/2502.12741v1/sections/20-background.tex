\section{Data Generator DCSim}
\label{sec:background}

To simulate the execution of HEP workflows on parallel and distributed computing (PDC) infrastructure, we utilize the DCSim tool~\cite{Horzela2023Proceeding,Horzela2023_1000165566} as a reference simulator\footnote{Note the exact version of the simulator published here: \url{https://doi.org/10.5281/ZENODO.8300961}}. 
It is implemented using the SimGrid~\cite{SimGrid} and WRENCH~\cite{wrench} simulation frameworks.
%
SimGrid and WRENCH have been chosen since they are general-purpose, enable more accurate simulation of PDC systems than other versatile frameworks~\cite{TOMACS} while keeping the level of computational complexity at manageable levels, have been carefully validated~\cite{
% TOMACS, 
simutool_09, nstools_07, simgrid_storage,
% SMPI_TPDS,
% 7885814, 8048921,
7384330,
% stanisic,
Cornebize-cluster19}, and are utilized by a large and active community. %~\cite{simgrid-web,wrench-web}.

DCSim takes three main essential user inputs indicated as \enquote{configurations} in \autoref{fig:approach-overview}.
First, a platform description following the SimGrid standard defines the network of computing resources with all its (technical) characteristics, for instance, the network of links and interconnected computers, their routing, as well as their respective network bandwidths, latencies, numbers of CPU cores, CPU speeds, disk bandwidths, storage volumes, and other parameters.
Second, datasets and replicas present at the simulation start, their composition of files, sizes, and locations are specified in a dedicated input file.
Third, the workloads to be executed on the specified platform while possibly processing files from datasets are defined in another dedicated steering file.
The workloads are characterized by their submission time, the job collection they are composed of, their respective resource demands, and the amount of computational work that must be executed for their completion.
Additionally, parameters can be given to DCSim to steer individual components of the simulation models integrated into DCSim or to adjust their calibration. 
In this work, we use the default calibration derived for \cite{Horzela2023Proceeding}.
Using these inputs, DCSim simulates the scheduling of the jobs, their execution and data processing, and returns a list of observables for each simulated job.
Per default, observables are the start and end times of the jobs, the total times spent with I/O and compute operations, and the processed amounts of data.

It has been shown that after calibration, DCSim can achieve valid descriptions of real-world computing workflow executions and systems~\cite{Horzela2023Proceeding,Horzela2023_1000165566}.
However, the execution time of DCSim scales super-linear with the size of the simulated computing system, which limits its feasibility and necessitates workarounds in studies concerning computing systems at a global scale, like the WLCG.
We therefore utilize DCSim as a baseline for testing ML surrogates in simulation.
