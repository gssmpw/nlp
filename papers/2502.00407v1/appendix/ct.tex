% !TEX root =  ../main.tex

\section{Category theory formalization.}\label{app:CT}
This section extends the category-theoretic formalization introduced in the main paper to intervened models and abstraction.

Recall the category-theoretic definition from the main paper:
\begin{definition}[Category-theoretic SCM]\label{def:SCM_ct_app}
    An SCM is a functor $\scm^n: \Index \rightarrow \Prob$, mapping the source node of \Index to the probability space associated with the exogenous variables $(\myexogenousvals,\, \Sigma_{\myexogenousvals}, \zeta)$, the sink node of \Index to the probability space associated with the endogenous variables $(\myendogenousvals,\, \Sigma_{\myendogenousvals}, \chi)$, and the only edge of \Index to the measurable map induced by the set \myfunctional of functional assignments.
\end{definition}

\cref{fig:functor} offers a depiction of an SCM as a functor.

\begin{figure}
    \centering
    \begin{tikzpicture}[]

    \draw[dashed] (-0.5, 2) rectangle (.5, -0.5);
    \draw[dashed] (1.8, 2) rectangle (4, -0.5);

    \node at (0, 2.5) {\Index};
    \node at (3, 2.5) {\Prob};
    
    \node[circle, draw, fill,inner sep=1pt] (A) at (0, 1.5) {};
    \node[circle, draw, fill,inner sep=1pt] (B) at (0, 0) {};
    \node (C) at (3, 1.5) {$(\myexogenousvals,\, \Sigma_{\myexogenousvals}, \zeta)$};
    \node (D) at (3, 0) {$(\myendogenousvals,\, \Sigma_{\myendogenousvals}, \chi)$};

    \coordinate (A1shift) at ([yshift=-5pt]A);
    \draw[->,shorten >=2pt] (A1shift) -- (B);

    \coordinate (A2shift) at ([xshift=5pt]A);
    \draw[->, mypurple] (A2shift) -- (C);

    \coordinate (B1shift) at ([xshift=5pt]B);
    \draw[->, mypurple] (B1shift) -- (D);
    
    \draw[->] (C) -- node[right] {\mymixing} (D);
    
    \end{tikzpicture}
    \caption{An SCM is a functor (purple arrows) from \Index (right) to \Prob (left).}
    \label{fig:functor}
\end{figure}

In the same vein, we can have a functorial representation for intervened SCMs as well. However, instead of representing directly the post-interventional model $\scm^n_\iota$ as in Def. \ref{def:SCM_ct_app}, we will adopt a representation that is closer to the intervention operator itself. First, notice that, whenever the domains of the variables of an SCM are continuous, we can represent an intervention as a measurable map by relying on the truncation formula \citeSupp{pearl2009causalitySupp}:
\begin{lemma}
    Given a continuous Markovian  SCM $\scm^n = \langle (\myexogenousvals,\, \Sigma_{\myexogenousvals}, \zeta), \, (\myendogenousvals,\, \Sigma_{\myendogenousvals}, \chi)\, , \mymixing \rangle$ and an intervention $\iota$ on $\scm^n$, there exists a measurable map $\phi_\iota$ from the probability space of endogenous variables of the pre-interventional SCM $(\myendogenousvals,\, \Sigma_{\myendogenousvals}, \chi)$ to the probability space of endogenous variables of the post-interventional SCM $(\myendogenousvals_\iota,\, \Sigma_{\myendogenousvals_\iota}, \chi_\iota)$.
\end{lemma}
\emph{Proof.} Given a Markovian SCM $\scm^n$, the probability measure $\chi$ over the measure space of endogenous variables $(\myendogenousvals,\, \Sigma_{\myendogenousvals}, \chi)$ can be expressed by through the factorization over the endogenous variables $\chi=\prod_{i \in [n]} P\left(X_i |  \parents_i,Z_i\right)$.  Given intervention $\iota=\operatorname{do}(\myendogenous^{\iota} = \mathbf{x}^{\iota})$ on $\scm^n$, the new post-interventional measure $\chi^\iota$ can be computed through the truncation formula \citeSupp{pearl2009causalitySupp}:
\begin{equation}\label{eq:truncation}
\chi^{\iota}=\begin{cases}
\prod_{i\in[n],X_{i}\notin \myendogenous^{\iota}}P(X_i|\parents_i,Z_i) & \textrm{if }\myendogenous^{\iota} = \mathbf{x}^{\iota}\\
0 & \textrm{if }\myendogenous^{\iota} \neq \mathbf{x}^{\iota}
\end{cases}
\end{equation}
We can now define a measurable map $\phi^\iota$ connecting $(\myendogenousvals,\, \Sigma_{\myendogenousvals}, \chi)$ and $(\myendogenousvals,\, \Sigma_{\myendogenousvals}, \chi^\iota)$ such that $\phi^\iota_{\#}(\chi)=\chi^\iota$. Specifically, for each $X_i \in \myendogenous^{\iota}$, $\phi(X_i) = x_i^\iota$, thus guaranteeing the distribution on the second line of \cref{eq:truncation}; for each $X_i \notin \myendogenous^{\iota}$, we solve a measure transport problem \citeSupp{Marzouk_2016Supp} from $\chi(X_i)$ to $\chi^\iota(X_i)$ which, in the continuous case, guarantees a transport map over the domains that satisfies the distribution on the first line of \cref{eq:truncation}.
$\blacksquare$

We can then encode an intervened model as follows: 

\begin{definition}[Category-theoretic post-interventional SCM]\label{def:SCM_ct_intervention}
    A post-interventional SCM is a functor $\scm^n_\iota: \Index \rightarrow \Prob$, where the functor maps the source node of \Index to the probability space associated with the endogenous variables of the pre-interventional SCM $(\myendogenousvals,\, \Sigma_{\myendogenousvals}, \chi)$, the sink node of \Index to the probability space associated with the endogenous variables of the post-interventional SCM $(\myendogenousvals_\iota,\, \Sigma_{\myendogenousvals_\iota}, \chi_\iota)$, and the only edge of \Index to the function $\phi_\iota$ encoding the intervention $\iota$.
\end{definition} 

This construction gives rise to the structure in Fig. \ref{fig:functor3} and an immediate category-theory expression of abstraction equivalent to Def.\ref{def:abstraction}:
\begin{lemma}\label{lem:interventional_mixing}
    An interventionally consistent abstraction is a singular natural transformation $\abst$, that is, a morphism $\alphamap{\myendogenousvals}$ in \Prob, that, for all intervention in $\mathcal{I}$ guarantees the commutativity of the diagrams constructed from Fig. \ref{fig:functor2}. 
\end{lemma}

\emph{Proof.} 
Recall the definition of interventional consistency in \Cref{def:interv_consistency}:
\begin{equation}%\label{eq:abserr}
        \alphamap{\mathcal{Y}^h_\mathcal{I}}(P(\mathcal{Y}^\ell_\mathcal{I} \vert \doint(\mathcal{X}^\ell_\mathcal{I}))) =  P(\mathcal{Y}^h_\mathcal{I} \vert \alphamap{\mathcal{X}^h_\mathcal{I}}(\doint(\mathcal{X}^h_\mathcal{I}))).
\end{equation}
Let us relate this definition to our categorical notation. First, $\alphamap{\mathcal{Y}^h_\mathcal{I}}$ and $\alphamap{\mathcal{X}^h_\mathcal{I}}$ are components of the abstraction map $\alphamap{}$; in the categorical notation, this map correspond to $\alphamap{\myendogenousvals}$. 
The probability distribution $P(\mathcal{Y}^\ell_\mathcal{I} \vert \doint(\mathcal{X}^\ell_\mathcal{I}))$ is a distribution in the low-level model; with no loss of generality, assuming $\mathcal{Y}^\ell_\mathcal{I}$ to encompass all the non-intervened variables, this distribution correspond to the measure $\chi_\iota^\ell$; furthermore, the interventional measure $\chi_\iota^\ell$ can be obtained through the pushforward of the observational measure $\chi^\ell$ via the interventional mixing functions $\mymixing_\iota^\ell$, as by \cref{lem:interventional_mixing}.
Finally, the probability distribution $P(\mathcal{Y}^h_\mathcal{I} \vert \alphamap{\mathcal{X}^h_\mathcal{I}}(\doint(\mathcal{X}^h_\mathcal{I}))$ is a distribution in the high-level model; again, with no loss of generality, assuming $\mathcal{Y}^h_\mathcal{I}$ to encompass all the non-intervened variables, this distribution correspond to the measure $\chi_\kappa^h$, where $\kappa$ is the abstraction of the terms in $\iota$. Also, as before, the interventional measure $\chi_\kappa^h$ can be obtained through the pushforward of the observational measure $\chi^h$ via the interventional mixing functions $\mymixing_\kappa^h$, thanks to \cref{lem:interventional_mixing}.
We then obtain a rewriting of abstraction as:
\begin{equation}%\label{eq:abserr}
        \alphamap{\myendogenousvals} \circ \mymixing_\iota^\ell =  \mymixing_\kappa^h \circ \alphamap{\myendogenousvals}.
\end{equation}
corresponding to the commutativity of the right diagram in \cref{fig:functor3}, for all interventions. $\blacksquare$

\begin{figure*}
    \centering
    \begin{tikzpicture}[]

    \draw[dashed] (-0.5, 2.8) rectangle (.5, -0.5);
    \draw[dashed] (1.5, 2.8) rectangle (10.5, -0.5);

    \node at (0, 3.2) {\Index};
    \node at (6, 3.2) {\Prob};
    
    \node[circle, draw, fill,inner sep=1pt] (A) at (0, 2) {};
    \node[circle, draw, fill,inner sep=1pt] (B) at (0, 0) {};
    
    \node (C) at (3, 2) {$(\myexogenousvals^\ell,\, \Sigma_{\myexogenousvals^\ell}, \zeta^\ell)$};
    \node (D) at (6, 2) {$(\myendogenousvals^\ell,\, \Sigma_{\myendogenousvals^\ell}, \chi^\ell)$};
    \node (E) at (9, 2) {$(\myendogenousvals^\ell_\iota,\, \Sigma_{\myendogenousvals^\ell_\iota}, \chi^\ell_\iota)$};
    \draw[->] (C) -- node[above] {$\mymixing^\ell$} (D);
    \draw[->] (D) -- node[above] {$\mymixing^\ell_\iota$} (E);

    \node (F) at (3, 0) {$(\myexogenousvals^h,\, \Sigma_{\myexogenousvals^h}, \zeta^h)$};
    \node (G) at (6, 0) {$(\myendogenousvals^h,\, \Sigma_{\myendogenousvals^h}, \chi^h)$};
    \node (H) at (9, 0) {$(\myendogenousvals^h_\kappa,\, \Sigma_{\myendogenousvals^h_\kappa}, \chi^\ell_\kappa)$};
    \draw[->] (F) -- node[above] {$\mymixing^h$} (G);
    \draw[->] (G) -- node[above] {$\mymixing^h_\kappa$} (H);

    \coordinate (A1shift) at ([yshift=-5pt]A);
    \draw[->,shorten >=2pt] (A1shift) -- (B);

    \draw[blue, rounded corners=10pt]($(C.west)+(0,0.6)$) rectangle ($(D.east)+(0.1,-0.3)$);
    \draw[->, blue] (0.2,1) -- (4,1.7);

    \draw[cyan, rounded corners=10pt]($(D.west)+(0,0.6)$) rectangle ($(E.east)+(0.1,-0.3)$);
    \draw[->, cyan] (0.2,1) -- (8,1.7);

    \draw[red, rounded corners=10pt]($(F.west)+(0,0.6)$) rectangle ($(G.east)+(0.1,-0.3)$);
    \draw[->, red] (0.2,1) -- (4,0.6);

    \draw[orange, rounded corners=10pt]($(G.west)+(0,0.6)$) rectangle ($(H.east)+(0.1,-0.3)$);
    \draw[->, orange] (0.2,1) -- (8,0.6);

    \draw[->, dashed] (C) -- node[right] {$\alphamap{\myexogenousvals^{h}}$} (F);
    \draw[->, dashed] (D) -- node[right] {$\alphamap{\myendogenousvals^{h}}$} (G);
    \draw[->, dashed] (E) -- node[right] {$\alphamap{\myendogenousvals^h}$} (H);
    
    \end{tikzpicture}
    \caption{Representation of {\color{blue} $\scm^\ell$} (blue), {\color{cyan} $\scm^\ell_\iota$} (cyan), {\color{red} $\scm^h$} (red), {\color{orange} $\scm^h_\iota$} (orange) as functors. An abstraction is just a natural transformation, that is, a set of commuting arrows in \Prob (dashed black). Notice two commuting diagrams in \Prob: the first observational one rooted on the exogenous variables ($\mymixing^h \circ \alphamap{\myexogenousvals^{h}} = \alphamap{\myendogenousvals^{h}} \circ \mymixing^\ell$), the second interventional one connecting observational and interventional model ($\mymixing^h_\kappa \circ \alphamap{\myendogenousvals^{h}} = \alphamap{\myendogenousvals^{h}} \circ \mymixing^\ell_\iota$).}
    \label{fig:functor3}
\end{figure*}

