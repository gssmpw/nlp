% !TEX root =  ../main.tex
\section{Additional material for the causal abstraction of brain networks}\label{app:rw_figs}

This section provides additional material about the full and partial prior applications of CLinSEPAL to brain data, given in \Cref{sec:empirical_assessment_rw}.
Specifically, \Cref{fig:ROIsLobes} depicts the ground truth linear CA and the learned linear CA by CLinSEPAL for the full prior setting; whereas \Cref{fig:ROIsFun_ca} the results for the partial prior setting.
Regarding the partial prior setting, we also report the monitored metrics to better understand the performance of CLinSEPAL with varying degree of uncertainty (low, medium, high), as discussed in \Cref{sec:empirical_assessment_rw}.
The color coding for the partial prior setting refers to the following classification, reported unaltered from \citeSupp{gabriele2024extractingSupp}:
\begin{squishlist}
    \item Red for ROIs corresponding to cognitive functions, attention, emotion, and decision-making;
    \item Orange for those related to auditory processing, speech and language processing, and memory;
    \item Blue for those concerning memory formation and memory retrieval;
    \item Pink for those associated with sensory integration and somatosensory;
    \item Purple for the ROIs within the visual network and related to the visual memory;
    \item Green for those within the motor network;
    \item Yellow for those regarding the motor control and the posture.
\end{squishlist}

\begin{figure}
    \centering
    \includegraphics[width=1.\textwidth]{figs/30_case_study_1_day1_day1.pdf}
    \caption{The figure shows (top) the ground truth linear CA and (bottom) the learned linear CA for the simulated full prior setting in \Cref{sec:empirical_assessment_rw}.}
    \label{fig:ROIsLobes}
\end{figure}



\begin{figure}
    \centering
    \includegraphics[width=1.\textwidth]{figs/30_case_study_2_ca.pdf}
    \caption{Starting from the top, the figure shows \emph{(i)} the ground truth linear CA, and the learned linear CA for the simulated partial prior setting with \emph{(ii)} low, \emph{(iii)} medium, and \emph{(iv)} high uncertainty in \Cref{sec:empirical_assessment_rw}.}
    \label{fig:ROIsFun_ca}
\end{figure}

\begin{figure}
    \centering
    \includegraphics[width=1.\textwidth]{figs/30_case_study_2_metrics.pdf}
    \caption{Starting from the left, the figure provides the \emph{(i)} the KL divergence evaluated at the learned \Vhat, \emph{(ii)} the Frobenious absolute distance, \emph{(iii)} the true positive rate, and \emph{(iv)} the false discovery rate for the simulated partial prior setting with low, medium, and high uncertainty in \Cref{sec:empirical_assessment_rw}.}
    \label{fig:ROIsFun_metrics}
\end{figure}
