% !TEX root =  ../main.tex

\section{Category theory essentials}\label{app:CT_background}

Below are fundamental definitions and examples that are instrumental in providing the necessary background on category theory to understand our work.
For a comprehensive overview of category theory see resources such as \citeSupp{mac2013categoriesSupp,perrone2024startingSupp}.

\begin{definition}[Category]\label{def:category}
    A category $\mathsf{C}$ consists of
    \begin{squishlist}
        \item A collection of objects, viz. $X$ in $\mathsf{C}$,
        \item A collection of morphisms, viz. $f: X \rightarrow Y$ in $\mathsf{C}$;
    \end{squishlist}
    such that:
    \begin{squishlist}
        \item Each morphism $f$ has assigned two objects of the category called source and target, respectively,
        \item Each object $X$ has an identity morphism $\mathrm{id}_X: X \rightarrow X$,
        \item Given $f: X\rightarrow Y $ and $g:Y\rightarrow Z$, than the composition exists, $g \circ f = h: X \rightarrow Z$.
    \end{squishlist}
    These structures satisfy the following axioms:
    \begin{squishlist}
        \item (Unitality) $\forall f: X \rightarrow Y, \; f \circ \mathrm{id}_X=f \text{ and } \mathrm{id}_Y \circ f = f$;
        \item (Associativity) Given $f$, $g$, and $h$ such that the compositions hold, then $h \circ (g \circ f) = (h \circ g) \circ f$.
    \end{squishlist}
\end{definition}

\begin{example}
    The following are some notable examples of categories:
    \begin{squishlist}
        \item Indicate with \Poset a partial order set. \Poset can be viewed as the category whose objects are the elements $p$ and morphisms are order relations $p \leq p^\prime$. Notice that there is at most one morphism between two objects;
        \item \Vect is the category whose objects are real vector spaces and morphisms are linear maps;
        \item \Prob is the category whose objects are probability measure spaces and morphisms measurable maps.
    \end{squishlist}
\end{example}

Arrows between categories are called \emph{functors}, defined as follows:

\begin{definition}[Functor]\label{def:functor}
    Consider $\mathsf{C}$ and $\mathsf{D}$ categories. 
    A functor $F: \mathsf{C} \rightarrow \mathsf{D}$ consists of the following data:
    \begin{squishlist}
        \item For each object $X$ in \Ccat, an object $F(X)$ in $\mathsf{D}$;
        \item For each object morphism $f: X \rightarrow Y$ in \Ccat, a morphism $F(f): F(X) \rightarrow F(Y)$ in $\mathsf{D}$;
    \end{squishlist}
    such that the following axioms hold:
    \begin{squishlist}
        \item (Unitality) $\forall X $ in $ \Ccat, \; F(\mathrm{id}_X) \!=\! \mathrm{id}_{F(X)}$. In other words, the identity in \Ccat is mapped into the identity in $\mathsf{D}$.
        \item (Compositionality) $\forall f \text{ and } g $ in \Ccat such that the composition is defined, then $F(g \circ f) = F(g) \circ F(f)$. In other words, the composition in \Ccat is mapped into the composition in $\mathsf{D}$.
    \end{squishlist}
\end{definition}

To ease the notation, in the sequel, we use $F^X$ and $F^f$ to denote $F(X)$ and $F(f)$, respectively.
Finally, we can have arrows between functors as well, called \emph{natural transformations}:

\begin{definition}[Natural transformation]\label{def:nat_transf}
Consider two categories \Ccat and $\mathsf{D}$, and two functors between them, namely $F: \Ccat \rightarrow \mathsf{D}$ and $G: \Ccat \rightarrow \mathsf{D}$.
A natural transformation $\alpha: F \dotarrow G$ consists of the following data:
\begin{squishlist}
    \item For each object $X $ in \Ccat, a morphism $\alpha_{X}: F^X \rightarrow G^X$ in $\mathcal{D}$ called the component of $\alpha$ at $X$;
    \item For each morphism $f: X \rightarrow X^\prime$ in \Ccat, the following diagram commutes:
    \begin{equation}
        \begin{tikzcd}[row sep=1.5cm, column sep=1.5cm]
            F^X \arrow[r, "F^f"] \arrow[d, "\alpha_X"'] & F^{X^\prime} \arrow[d, "\alpha_{X^\prime}"] \\
            G^X \arrow[r, "G^f"'] & G^{X^\prime}
        \end{tikzcd}
    \end{equation}
\end{squishlist}
\end{definition}

A natural transformation can be thought of as a consistent system of arrows between two functors, invariant with respect to maps between the images of two functors.