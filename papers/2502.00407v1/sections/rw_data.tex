% !TEX root =  ../main.tex
\section{Causal abstraction of brain networks}\label{sec:empirical_assessment_rw}
To show the practical relevance of our approach, we apply CLinSEPAL to resting-state functional magnetic resonance imaging (rs-fMRI) data, using the dataset from \cite{gabriele2024extracting} (refer to the paper for details on the dataset). The data, publicly released as part of the \emph{Human Connectome Project} \cite{smith2013resting},
comprises recordings from $100$ healthy adults with a parcellation scheme that divides the brain into $89$ regions of interest (ROIs), $K=44$ for each hemisphere plus the shared vermis region.

We simulate a first investigating team of neuroscientists taking zero-mean stationary time series for the left hemisphere of the first adult in the dataset. They estimate the data covariance matrix using a Gaussian mixture probability model, viz. $\covlow \in \reall^{\ell \times \ell}$, with $\ell=K+1$, and interpret it as generated by an underlying, unknown, low-level SCM.

In a first fp scenario, we imagine a second investigating team that has collected data according to their causal network specified on a coarser parcellation of the same brain in $h=14$ macro ROIs. We generate the data for the second team using a ground truth linear CA $\B, \V^\star \in \stiefel{45}{14}$ based on the structural mapping in \cite{gabriele2024extracting}, and use the data for estimating the covariance matrix $\covhigh \in \reall^{h \times h}$. In this scenario it is realistic to assume knowledge of $\B$ defining how macro ROIa are mapped to ROIs. Then, to align their models, the two groups run CLinSEPAL to recover the abstraction given $\covlow,\covhigh$ and $\B$. \Cref{fig:ROIsLobes} (in \Cref{app:rw_figs}) shows that CLinSEPAL recovers $\V^\star$.

In a second pp scenario, we imagine that the second investigating team has collected data according to a causal network aggregating ROI time series into $h=8$ brain functional networks related to different activities (e.g., motor, visual, default mode). Data is generated again through a ground truth linear CA $\B, \V^\star \in \stiefel{45}{8}$ based on groupings in \cite{gabriele2024extracting} and the covariance matrix $\covhigh \in \reall^{h \times h}$ computed. In this scenario, knowledge of $\B$ is debatable as different studies in the literature suggest different relations between ROIs and functions; we then express this partial information via uncertainty over $\B$, meaning that some rows of \B have more than one entry equal to one. The two groups now run CLinSEPAL using $\covlow,\covhigh$ and an uncertain $\B$; partial knowledge compounds on an already challenging learning problem due to the high coarse-graining. \Cref{fig:ROIsFun_ca} and \Cref{fig:ROIsFun_metrics} show results with different levels of uncertainty. For low uncertainty, CLinSEPAL correctly retrieves the structure of the CA, although we observe some variation in the colors w.r.t. $\V^\star$; additionally, \KL{\Vhat} and the Frobenius absolute distance in \cref{fig:ROIsFun_metrics} show that misalignment is minimized and \Vhat very close to $\V^\star$. For medium and high uncertainty, CLinSEPAL makes some mistakes in terms of structural mapping, but \cref{fig:ROIsFun_metrics} shows that insights from the method are still valuable.