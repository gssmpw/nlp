% !TEX root =  ../main.tex
\section{Conclusion and future works}\label{sec:concl_and_fw}
In this work, we addressed the challenge of CA learning in realistic scenarios, abandoning restrictive assumptions \hyperlink{(NA1)}{(NA1)}-\hyperlink{(NA5)}{(NA5)} that limit the applicability of existing methods. We proposed an alternative category-theoretic framework for SCM and CA, and introduced the \emph{semantic embedding principle} to learn CAs that meaningfully preserve information. We formulated a general CA learning problem grounded in SEP, under a mild assumption of partial prior knowledge about the structure of CA. 
For the linear CA setting, we showed how SEP links CA to the geometry of the Stiefel manifold; as an application, we tackled the important case of Gaussian measures, with the KL divergence as a measure of alignment between the low- and high-level SCMs. We pursued two different formulations.
For the first, a nonsmooth Riemannian learning problem, we devised the LinSEPAL-ADMM and LinSEPAL-PG methods. 
For the second, a smooth Riemannian learning problem ensuring the constructiveness of the CA, we developed CLinSEPAL.
Our empirical assessment on synthetic data confirmed the effectiveness of our methods, and the application to brain data showcased the potential in real-world problems.

Our work paves the way for several exciting research directions.
First, as it emerges from our Gaussian application, \emph{linear CAs with different probability measures} deserve careful investigation.
Second, studying the \emph{nonlinear case} is a compelling avenue. 
We believe that deep and reinforcement learning paradigms, such as encoding-decoding and actor-critic architectures, hold promise for modeling nonlinear CA maps.
Lastly, we view our work as a foundational step toward \emph{observational causal abstraction learning}, bridging the gap between \emph{CA learning} and \emph{causal discovery} \cite{spirtes2016causal}. \black{Our category-theoretic framework underscores the pivotal role of exogenous variables, drawing a path to translate \emph{SCM identifiability} results into \emph{CA identifiability} results.
This suggests that, in some cases, interventional consistency may be achieved without relying on interventional data.}