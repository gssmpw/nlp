\section{Category-theory formalization}
Standard category-theoretic formalization of CA \cite{rischel2020category,otsuka2022equivalence} are based on a functorial semantics \cite{jacobs2019causal} approach mapping the graphical structure of causal models (\emph{syntax}) onto the discrete distributions of individual variables (\emph{semantics}). 
Because of our non-assumption \hyperlink{(NA2)}{(NA2)}, no knowledge of the structure of an SCM is available in our setting; thus, we propose a formalization mapping a dyadic structure (\emph{syntax}) onto the exogenous and the endogenous probability measures implied by an SCM (\emph{semantics}).

A crucial role in our modelling is that of the mixing functions \mymixing, which express the data generation process as a recursive process from the exogenous functions. This allows us to define an SCM $\scm^n$ in measure-theoretic terms as a tuple made up of the probability space of exogenous variables $(\myexogenousvals,\, \Sigma_{\myexogenousvals}, \zeta)$, the probability space of the endogenous variables $(\myendogenousvals,\, \Sigma_{\myendogenousvals}, \chi)$, and a set of measurable functions $\mymixing$ given by the mixing functions (cf. \cref{app:CA}).

We can now rely on this representation to  interpret an SCM as a category-theoretic functor from a simple index category \Index, made up only of a source and a sink object and an edge between them, to the category of probability spaces \Prob, where objects $(X,\Sigma_X, p)$ are probability spaces and morphisms $\varphi$ are measurable maps:
\begin{definition}[Category-theoretic SCM]\label{def:SCM_ct}
    An SCM is a functor $\scm^n: \Index \rightarrow \Prob$, mapping the source node of \Index to $(\myexogenousvals,\, \Sigma_{\myexogenousvals}, \zeta)$, the sink node of \Index to $(\myendogenousvals,\, \Sigma_{\myendogenousvals}, \chi)$, and the  edge of \Index to the collection \mymixing of measurable maps.
\end{definition} 



\begin{figure}
    \centering
    \begin{tikzpicture}[scale=1.]

    \draw[dashed] (-0.5, 2.8) rectangle (.5, -.8);
    \draw[dashed] (1., 2.8) rectangle (7.75, -.8);

    \node at (0, 3.2) {\Index};
    \node at (4, 3.2) {\Prob};
    
    \node[circle, draw, fill,inner sep=1pt] (A) at (0, 2) {};
    \node[circle, draw, fill,inner sep=1pt] (B) at (0, 0) {};
    
    \node (C) at (2.2, 2) {$(\myexogenousvals^\ell,\, \Sigma_{\myexogenousvals^\ell}, \zeta^\ell)$};
    \node (D) at (2.2, 0) {$(\myendogenousvals^\ell,\, \Sigma_{\myendogenousvals^\ell}, \chi^\ell)$};
    \draw[->] (C) -- node[left] {$\mymixing^\ell$} (D);

    \node (F) at (6.4, 2) {$(\myexogenousvals^h,\, \Sigma_{\myexogenousvals^h}, \zeta^h)$};
    \node (G) at (6.4, 0) {$(\myendogenousvals^h,\, \Sigma_{\myendogenousvals^h}, \chi^h)$};
    \draw[->] (F) -- node[right] {$\mymixing^h$} (G);

    \coordinate (A1shift) at ([yshift=-5pt]A);
    \draw[->,shorten >=2pt] (A1shift) -- (B);

    \draw[mypurple, rounded corners=10pt]($(C.west)+(0,0.3)$) rectangle ($(D.east)+(0.05,-0.3)$);
    % \draw[->, mypurple] (0.2,1) -- (2,1.7);
    \draw[->, mypurple] (A) -- (C);
    \draw[->, mypurple] (B) -- (D);

    \draw[cyan, rounded corners=10pt]($(F.west)+(0,0.3)$) rectangle ($(G.east)+(0.05,-0.3)$);
    % \draw[->, cyan] (0.2,1) -- (2,0.6);
    \draw[->, cyan] (A) to[bend left=20] (F);
    \draw[->, cyan] (B) to[bend right=20] (G);

    \draw[->, dashed] (C) -- node[above] {$\alphamap{\left(\myexogenousvals^{h},\, \Sigma_{\myexogenousvals^h}\right)}$} (F);
    \draw[->, dashed] (D) -- node[below] {$\alphamap{\left(\myendogenousvals^{h},\, \Sigma_{\myendogenousvals^h}\right)}$} (G);
    
    \end{tikzpicture}
    \caption{An abstraction as natural transformation, that is, a set of commuting arrows in \Prob (dashed black) from {\color{mypurple} $\scm^\ell$} (purple) to {\color{cyan} $\scm^h$} (cyan).}
    \label{fig:functor2}
\end{figure}

% \cref{app:CA} presents basic category-theoretic concepts and defines the mentioned categories.
\black{\Cref{app:CT_background} presents basic category-theoretic concepts, whereas \cref{subsec:SCM_prob} deepens \cref{def:SCM_ct}.}
CA can now be expressed as a natural transformation between two SCMs, as shown in \cref{fig:functor2}. This formulation has two important features.
First, it highlights the role of exogenous variables in a constructive abstraction showing the commutativity of the paths $\mymixing^h \circ \alphamap{\left(\myexogenousvals^{h},\, \Sigma_{\myexogenousvals^h}\right)}$ and $\alphamap{\left(\myendogenousvals^{h},\, \Sigma_{\myendogenousvals^h}\right)} \circ \mymixing^\ell$. Second, morphisms in \Prob relates measure spaces, viz. sets equipped with sigma algebras. Consequently, the natural transformation components are measurable maps with dimensionality determined by the cardinality of $\myendogenous^h$ and $\myendogenous^\ell$.
To ease the notation, we will denote $\alphamap{\left(\myexogenousvals^{h},\, \Sigma_{\myexogenousvals^h}\right)}$ by $\alphamap{\myexogenous}$ and $\alphamap{\left(\myendogenousvals^{h},\, \Sigma_{\myendogenousvals^h}\right)} $ by $\alphamap{\myendogenous}$.
Then, we can formally recast the $\alpha$-abstraction in \Prob.
\begin{definition}[$\abst$-abstraction in \Prob]\label{def:alpha_abstraction_prob}
    Given low-level $\scm^\ell$ and high-level $\scm^h$ SCMs, %as in \Cref{def:SCM_prob}, 
    an abstraction $\abst = \langle \Rset, \Qset, \amap, \alphamap{} \rangle$ is a tuple, where: \emph{(i)} \Rset is the same as in \Cref{def:abstraction}; \emph{(ii)} $\Qset \subseteq \myexogenous^\ell$ is a set of relevant exogenous variables given by the union of the set of exogenous corresponding to the endogenous in \Rset and those corresponding to their ancestors; \emph{(iii)} $\amap=\langle \amap_\myexogenous, \amap_\myendogenous \rangle$ is a pair of surjective functions mapping sets, $\amap_\myexogenous: \Qset \rightarrow \myexogenous^h$ and $\amap_\myendogenous: \Rset \rightarrow \myendogenous^h$, respectively; \emph{(iv)} $\alphamap{}=\langle \alphamap{\myexogenous}, \alphamap{\myendogenous} \rangle$ is a natural transformation made by measurable functions mapping probability spaces, $\alphamap{\myexogenous}$ for the exogenous and $\alphamap{\myendogenous}$ for the endogenous, respectively.  
\end{definition}
As \Cref{def:abstraction}, \Cref{def:alpha_abstraction_prob} makes no reference to interventional consistency.
\Cref{app:CT} explains how intervened SCMs and interventional consistency can be represented categorically.