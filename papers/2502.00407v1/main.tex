%%%%%%%% ICML 2025 EXAMPLE LATEX SUBMISSION FILE %%%%%%%%%%%%%%%%%

\documentclass{article}


%
\setlength\unitlength{1mm}
\newcommand{\twodots}{\mathinner {\ldotp \ldotp}}
% bb font symbols
\newcommand{\Rho}{\mathrm{P}}
\newcommand{\Tau}{\mathrm{T}}

\newfont{\bbb}{msbm10 scaled 700}
\newcommand{\CCC}{\mbox{\bbb C}}

\newfont{\bb}{msbm10 scaled 1100}
\newcommand{\CC}{\mbox{\bb C}}
\newcommand{\PP}{\mbox{\bb P}}
\newcommand{\RR}{\mbox{\bb R}}
\newcommand{\QQ}{\mbox{\bb Q}}
\newcommand{\ZZ}{\mbox{\bb Z}}
\newcommand{\FF}{\mbox{\bb F}}
\newcommand{\GG}{\mbox{\bb G}}
\newcommand{\EE}{\mbox{\bb E}}
\newcommand{\NN}{\mbox{\bb N}}
\newcommand{\KK}{\mbox{\bb K}}
\newcommand{\HH}{\mbox{\bb H}}
\newcommand{\SSS}{\mbox{\bb S}}
\newcommand{\UU}{\mbox{\bb U}}
\newcommand{\VV}{\mbox{\bb V}}


\newcommand{\yy}{\mathbbm{y}}
\newcommand{\xx}{\mathbbm{x}}
\newcommand{\zz}{\mathbbm{z}}
\newcommand{\sss}{\mathbbm{s}}
\newcommand{\rr}{\mathbbm{r}}
\newcommand{\pp}{\mathbbm{p}}
\newcommand{\qq}{\mathbbm{q}}
\newcommand{\ww}{\mathbbm{w}}
\newcommand{\hh}{\mathbbm{h}}
\newcommand{\vvv}{\mathbbm{v}}

% Vectors

\newcommand{\av}{{\bf a}}
\newcommand{\bv}{{\bf b}}
\newcommand{\cv}{{\bf c}}
\newcommand{\dv}{{\bf d}}
\newcommand{\ev}{{\bf e}}
\newcommand{\fv}{{\bf f}}
\newcommand{\gv}{{\bf g}}
\newcommand{\hv}{{\bf h}}
\newcommand{\iv}{{\bf i}}
\newcommand{\jv}{{\bf j}}
\newcommand{\kv}{{\bf k}}
\newcommand{\lv}{{\bf l}}
\newcommand{\mv}{{\bf m}}
\newcommand{\nv}{{\bf n}}
\newcommand{\ov}{{\bf o}}
\newcommand{\pv}{{\bf p}}
\newcommand{\qv}{{\bf q}}
\newcommand{\rv}{{\bf r}}
\newcommand{\sv}{{\bf s}}
\newcommand{\tv}{{\bf t}}
\newcommand{\uv}{{\bf u}}
\newcommand{\wv}{{\bf w}}
\newcommand{\vv}{{\bf v}}
\newcommand{\xv}{{\bf x}}
\newcommand{\yv}{{\bf y}}
\newcommand{\zv}{{\bf z}}
\newcommand{\zerov}{{\bf 0}}
\newcommand{\onev}{{\bf 1}}

% Matrices

\newcommand{\Am}{{\bf A}}
\newcommand{\Bm}{{\bf B}}
\newcommand{\Cm}{{\bf C}}
\newcommand{\Dm}{{\bf D}}
\newcommand{\Em}{{\bf E}}
\newcommand{\Fm}{{\bf F}}
\newcommand{\Gm}{{\bf G}}
\newcommand{\Hm}{{\bf H}}
\newcommand{\Id}{{\bf I}}
\newcommand{\Jm}{{\bf J}}
\newcommand{\Km}{{\bf K}}
\newcommand{\Lm}{{\bf L}}
\newcommand{\Mm}{{\bf M}}
\newcommand{\Nm}{{\bf N}}
\newcommand{\Om}{{\bf O}}
\newcommand{\Pm}{{\bf P}}
\newcommand{\Qm}{{\bf Q}}
\newcommand{\Rm}{{\bf R}}
\newcommand{\Sm}{{\bf S}}
\newcommand{\Tm}{{\bf T}}
\newcommand{\Um}{{\bf U}}
\newcommand{\Wm}{{\bf W}}
\newcommand{\Vm}{{\bf V}}
\newcommand{\Xm}{{\bf X}}
\newcommand{\Ym}{{\bf Y}}
\newcommand{\Zm}{{\bf Z}}

% Calligraphic

\newcommand{\Ac}{{\cal A}}
\newcommand{\Bc}{{\cal B}}
\newcommand{\Cc}{{\cal C}}
\newcommand{\Dc}{{\cal D}}
\newcommand{\Ec}{{\cal E}}
\newcommand{\Fc}{{\cal F}}
\newcommand{\Gc}{{\cal G}}
\newcommand{\Hc}{{\cal H}}
\newcommand{\Ic}{{\cal I}}
\newcommand{\Jc}{{\cal J}}
\newcommand{\Kc}{{\cal K}}
\newcommand{\Lc}{{\cal L}}
\newcommand{\Mc}{{\cal M}}
\newcommand{\Nc}{{\cal N}}
\newcommand{\nc}{{\cal n}}
\newcommand{\Oc}{{\cal O}}
\newcommand{\Pc}{{\cal P}}
\newcommand{\Qc}{{\cal Q}}
\newcommand{\Rc}{{\cal R}}
\newcommand{\Sc}{{\cal S}}
\newcommand{\Tc}{{\cal T}}
\newcommand{\Uc}{{\cal U}}
\newcommand{\Wc}{{\cal W}}
\newcommand{\Vc}{{\cal V}}
\newcommand{\Xc}{{\cal X}}
\newcommand{\Yc}{{\cal Y}}
\newcommand{\Zc}{{\cal Z}}

% Bold greek letters

\newcommand{\alphav}{\hbox{\boldmath$\alpha$}}
\newcommand{\betav}{\hbox{\boldmath$\beta$}}
\newcommand{\gammav}{\hbox{\boldmath$\gamma$}}
\newcommand{\deltav}{\hbox{\boldmath$\delta$}}
\newcommand{\etav}{\hbox{\boldmath$\eta$}}
\newcommand{\lambdav}{\hbox{\boldmath$\lambda$}}
\newcommand{\epsilonv}{\hbox{\boldmath$\epsilon$}}
\newcommand{\nuv}{\hbox{\boldmath$\nu$}}
\newcommand{\muv}{\hbox{\boldmath$\mu$}}
\newcommand{\zetav}{\hbox{\boldmath$\zeta$}}
\newcommand{\phiv}{\hbox{\boldmath$\phi$}}
\newcommand{\psiv}{\hbox{\boldmath$\psi$}}
\newcommand{\thetav}{\hbox{\boldmath$\theta$}}
\newcommand{\tauv}{\hbox{\boldmath$\tau$}}
\newcommand{\omegav}{\hbox{\boldmath$\omega$}}
\newcommand{\xiv}{\hbox{\boldmath$\xi$}}
\newcommand{\sigmav}{\hbox{\boldmath$\sigma$}}
\newcommand{\piv}{\hbox{\boldmath$\pi$}}
\newcommand{\rhov}{\hbox{\boldmath$\rho$}}
\newcommand{\upsilonv}{\hbox{\boldmath$\upsilon$}}

\newcommand{\Gammam}{\hbox{\boldmath$\Gamma$}}
\newcommand{\Lambdam}{\hbox{\boldmath$\Lambda$}}
\newcommand{\Deltam}{\hbox{\boldmath$\Delta$}}
\newcommand{\Sigmam}{\hbox{\boldmath$\Sigma$}}
\newcommand{\Phim}{\hbox{\boldmath$\Phi$}}
\newcommand{\Pim}{\hbox{\boldmath$\Pi$}}
\newcommand{\Psim}{\hbox{\boldmath$\Psi$}}
\newcommand{\Thetam}{\hbox{\boldmath$\Theta$}}
\newcommand{\Omegam}{\hbox{\boldmath$\Omega$}}
\newcommand{\Xim}{\hbox{\boldmath$\Xi$}}


% Sans Serif small case

\newcommand{\Gsf}{{\sf G}}

\newcommand{\asf}{{\sf a}}
\newcommand{\bsf}{{\sf b}}
\newcommand{\csf}{{\sf c}}
\newcommand{\dsf}{{\sf d}}
\newcommand{\esf}{{\sf e}}
\newcommand{\fsf}{{\sf f}}
\newcommand{\gsf}{{\sf g}}
\newcommand{\hsf}{{\sf h}}
\newcommand{\isf}{{\sf i}}
\newcommand{\jsf}{{\sf j}}
\newcommand{\ksf}{{\sf k}}
\newcommand{\lsf}{{\sf l}}
\newcommand{\msf}{{\sf m}}
\newcommand{\nsf}{{\sf n}}
\newcommand{\osf}{{\sf o}}
\newcommand{\psf}{{\sf p}}
\newcommand{\qsf}{{\sf q}}
\newcommand{\rsf}{{\sf r}}
\newcommand{\ssf}{{\sf s}}
\newcommand{\tsf}{{\sf t}}
\newcommand{\usf}{{\sf u}}
\newcommand{\wsf}{{\sf w}}
\newcommand{\vsf}{{\sf v}}
\newcommand{\xsf}{{\sf x}}
\newcommand{\ysf}{{\sf y}}
\newcommand{\zsf}{{\sf z}}


% mixed symbols

\newcommand{\sinc}{{\hbox{sinc}}}
\newcommand{\diag}{{\hbox{diag}}}
\renewcommand{\det}{{\hbox{det}}}
\newcommand{\trace}{{\hbox{tr}}}
\newcommand{\sign}{{\hbox{sign}}}
\renewcommand{\arg}{{\hbox{arg}}}
\newcommand{\var}{{\hbox{var}}}
\newcommand{\cov}{{\hbox{cov}}}
\newcommand{\Ei}{{\rm E}_{\rm i}}
\renewcommand{\Re}{{\rm Re}}
\renewcommand{\Im}{{\rm Im}}
\newcommand{\eqdef}{\stackrel{\Delta}{=}}
\newcommand{\defines}{{\,\,\stackrel{\scriptscriptstyle \bigtriangleup}{=}\,\,}}
\newcommand{\<}{\left\langle}
\renewcommand{\>}{\right\rangle}
\newcommand{\herm}{{\sf H}}
\newcommand{\trasp}{{\sf T}}
\newcommand{\transp}{{\sf T}}
\renewcommand{\vec}{{\rm vec}}
\newcommand{\Psf}{{\sf P}}
\newcommand{\SINR}{{\sf SINR}}
\newcommand{\SNR}{{\sf SNR}}
\newcommand{\MMSE}{{\sf MMSE}}
\newcommand{\REF}{{\RED [REF]}}

% Markov chain
\usepackage{stmaryrd} % for \mkv 
\newcommand{\mkv}{-\!\!\!\!\minuso\!\!\!\!-}

% Colors

\newcommand{\RED}{\color[rgb]{1.00,0.10,0.10}}
\newcommand{\BLUE}{\color[rgb]{0,0,0.90}}
\newcommand{\GREEN}{\color[rgb]{0,0.80,0.20}}

%%%%%%%%%%%%%%%%%%%%%%%%%%%%%%%%%%%%%%%%%%
\usepackage{hyperref}
\hypersetup{
    bookmarks=true,         % show bookmarks bar?
    unicode=false,          % non-Latin characters in AcrobatÕs bookmarks
    pdftoolbar=true,        % show AcrobatÕs toolbar?
    pdfmenubar=true,        % show AcrobatÕs menu?
    pdffitwindow=false,     % window fit to page when opened
    pdfstartview={FitH},    % fits the width of the page to the window
%    pdftitle={My title},    % title
%    pdfauthor={Author},     % author
%    pdfsubject={Subject},   % subject of the document
%    pdfcreator={Creator},   % creator of the document
%    pdfproducer={Producer}, % producer of the document
%    pdfkeywords={keyword1} {key2} {key3}, % list of keywords
    pdfnewwindow=true,      % links in new window
    colorlinks=true,       % false: boxed links; true: colored links
    linkcolor=red,          % color of internal links (change box color with linkbordercolor)
    citecolor=green,        % color of links to bibliography
    filecolor=blue,      % color of file links
    urlcolor=blue           % color of external links
}
%%%%%%%%%%%%%%%%%%%%%%%%%%%%%%%%%%%%%%%%%%%



% The \icmltitle you define below is probably too long as a header.
% Therefore, a short form for the running title is supplied here:
\icmltitlerunning{Causal Abstraction Learning based on the Semantic Embedding Principle}

\begin{document}

\twocolumn[
\icmltitle{Causal Abstraction Learning based on the Semantic Embedding Principle}

% It is OKAY to include author information, even for blind
% submissions: the style file will automatically remove it for you
% unless you've provided the [accepted] option to the icml2025
% package.

% List of affiliations: The first argument should be a (short)
% identifier you will use later to specify author affiliations
% Academic affiliations should list Department, University, City, Region, Country
% Industry affiliations should list Company, City, Region, Country

% You can specify symbols, otherwise they are numbered in order.
% Ideally, you should not use this facility. Affiliations will be numbered
% in order of appearance and this is the preferred way.
\icmlsetsymbol{equal}{*}

\begin{icmlauthorlist}
\icmlauthor{Gabriele D'Acunto}{sapienza}
\icmlauthor{Fabio Massimo Zennaro}{bergen}
\icmlauthor{Yorgos Felekis}{warwick}
\icmlauthor{Paolo Di Lorenzo}{sapienza}
\end{icmlauthorlist}

\icmlaffiliation{sapienza}{Department of Information Engineering, Electronics and Telecommunications, Sapienza University, Rome, Italy}
\icmlaffiliation{bergen}{Department of Informatics, University of Bergen, Bergen, Norway}
\icmlaffiliation{warwick}{Department of Computer Science, University of Warwick, Coventry, UK}

\icmlcorrespondingauthor{Gabriele D'Acunto}{gabriele.dacunto@uniroma1.it}
% \icmlcorrespondingauthor{Firstname2 Lastname2}{first2.last2@www.uk}

% You may provide any keywords that you
% find helpful for describing your paper; these are used to populate
% the "keywords" metadata in the PDF but will not be shown in the document
\icmlkeywords{structural causal models, causal abstraction, semantic embedding principle, Stiefel manifold, Riemannian optimization}

\vskip 0.3in
]

% this must go after the closing bracket ] following \twocolumn[ ...

% This command actually creates the footnote in the first column
% listing the affiliations and the copyright notice.
% The command takes one argument, which is text to display at the start of the footnote.
% The \icmlEqualContribution command is standard text for equal contribution.
% Remove it (just {}) if you do not need this facility.

\printAffiliationsAndNotice{}  % leave blank if no need to mention equal contribution
% \printAffiliationsAndNotice{\icmlEqualContribution} % otherwise use the standard text.

\begin{abstract}
Structural causal models (SCMs) allow us to investigate complex systems at multiple levels of resolution.
The causal abstraction (CA) framework formalizes the mapping between high- and low-level SCMs. 
We address CA learning in a challenging and realistic setting, where SCMs are inaccessible, interventional data is unavailable, and sample data is misaligned. 
A key principle of our framework is \emph{semantic embedding}, formalized as the high-level distribution lying on a subspace of the low-level one. 
This principle naturally links linear CA to the geometry of the \emph{Stiefel manifold}.
We present a category-theoretic approach to SCMs that enables the learning of a CA by finding a morphism between the low- and high-level probability measures, adhering to the semantic embedding principle. 
\black{Consequently, we formulate a general CA learning problem.}
\black{As an application, we solve the latter problem for linear CA; considering Gaussian measures and the Kullback-Leibler divergence as an objective.
Given the nonconvexity of the learning task, we develop three algorithms building upon existing paradigms for Riemannian optimization.}
We demonstrate that the proposed methods succeed on both synthetic and real-world brain data with different degrees of prior information about the structure of CA.  
\end{abstract}

\section{Introduction}


\begin{figure}[t]
\centering
\includegraphics[width=0.6\columnwidth]{figures/evaluation_desiderata_V5.pdf}
\vspace{-0.5cm}
\caption{\systemName is a platform for conducting realistic evaluations of code LLMs, collecting human preferences of coding models with real users, real tasks, and in realistic environments, aimed at addressing the limitations of existing evaluations.
}
\label{fig:motivation}
\end{figure}

\begin{figure*}[t]
\centering
\includegraphics[width=\textwidth]{figures/system_design_v2.png}
\caption{We introduce \systemName, a VSCode extension to collect human preferences of code directly in a developer's IDE. \systemName enables developers to use code completions from various models. The system comprises a) the interface in the user's IDE which presents paired completions to users (left), b) a sampling strategy that picks model pairs to reduce latency (right, top), and c) a prompting scheme that allows diverse LLMs to perform code completions with high fidelity.
Users can select between the top completion (green box) using \texttt{tab} or the bottom completion (blue box) using \texttt{shift+tab}.}
\label{fig:overview}
\end{figure*}

As model capabilities improve, large language models (LLMs) are increasingly integrated into user environments and workflows.
For example, software developers code with AI in integrated developer environments (IDEs)~\citep{peng2023impact}, doctors rely on notes generated through ambient listening~\citep{oberst2024science}, and lawyers consider case evidence identified by electronic discovery systems~\citep{yang2024beyond}.
Increasing deployment of models in productivity tools demands evaluation that more closely reflects real-world circumstances~\citep{hutchinson2022evaluation, saxon2024benchmarks, kapoor2024ai}.
While newer benchmarks and live platforms incorporate human feedback to capture real-world usage, they almost exclusively focus on evaluating LLMs in chat conversations~\citep{zheng2023judging,dubois2023alpacafarm,chiang2024chatbot, kirk2024the}.
Model evaluation must move beyond chat-based interactions and into specialized user environments.



 

In this work, we focus on evaluating LLM-based coding assistants. 
Despite the popularity of these tools---millions of developers use Github Copilot~\citep{Copilot}---existing
evaluations of the coding capabilities of new models exhibit multiple limitations (Figure~\ref{fig:motivation}, bottom).
Traditional ML benchmarks evaluate LLM capabilities by measuring how well a model can complete static, interview-style coding tasks~\citep{chen2021evaluating,austin2021program,jain2024livecodebench, white2024livebench} and lack \emph{real users}. 
User studies recruit real users to evaluate the effectiveness of LLMs as coding assistants, but are often limited to simple programming tasks as opposed to \emph{real tasks}~\citep{vaithilingam2022expectation,ross2023programmer, mozannar2024realhumaneval}.
Recent efforts to collect human feedback such as Chatbot Arena~\citep{chiang2024chatbot} are still removed from a \emph{realistic environment}, resulting in users and data that deviate from typical software development processes.
We introduce \systemName to address these limitations (Figure~\ref{fig:motivation}, top), and we describe our three main contributions below.


\textbf{We deploy \systemName in-the-wild to collect human preferences on code.} 
\systemName is a Visual Studio Code extension, collecting preferences directly in a developer's IDE within their actual workflow (Figure~\ref{fig:overview}).
\systemName provides developers with code completions, akin to the type of support provided by Github Copilot~\citep{Copilot}. 
Over the past 3 months, \systemName has served over~\completions suggestions from 10 state-of-the-art LLMs, 
gathering \sampleCount~votes from \userCount~users.
To collect user preferences,
\systemName presents a novel interface that shows users paired code completions from two different LLMs, which are determined based on a sampling strategy that aims to 
mitigate latency while preserving coverage across model comparisons.
Additionally, we devise a prompting scheme that allows a diverse set of models to perform code completions with high fidelity.
See Section~\ref{sec:system} and Section~\ref{sec:deployment} for details about system design and deployment respectively.



\textbf{We construct a leaderboard of user preferences and find notable differences from existing static benchmarks and human preference leaderboards.}
In general, we observe that smaller models seem to overperform in static benchmarks compared to our leaderboard, while performance among larger models is mixed (Section~\ref{sec:leaderboard_calculation}).
We attribute these differences to the fact that \systemName is exposed to users and tasks that differ drastically from code evaluations in the past. 
Our data spans 103 programming languages and 24 natural languages as well as a variety of real-world applications and code structures, while static benchmarks tend to focus on a specific programming and natural language and task (e.g. coding competition problems).
Additionally, while all of \systemName interactions contain code contexts and the majority involve infilling tasks, a much smaller fraction of Chatbot Arena's coding tasks contain code context, with infilling tasks appearing even more rarely. 
We analyze our data in depth in Section~\ref{subsec:comparison}.



\textbf{We derive new insights into user preferences of code by analyzing \systemName's diverse and distinct data distribution.}
We compare user preferences across different stratifications of input data (e.g., common versus rare languages) and observe which affect observed preferences most (Section~\ref{sec:analysis}).
For example, while user preferences stay relatively consistent across various programming languages, they differ drastically between different task categories (e.g. frontend/backend versus algorithm design).
We also observe variations in user preference due to different features related to code structure 
(e.g., context length and completion patterns).
We open-source \systemName and release a curated subset of code contexts.
Altogether, our results highlight the necessity of model evaluation in realistic and domain-specific settings.





% !TEX root =  ../main.tex
\section{Background on causality and abstraction}\label{sec:preliminaries}

This section provides the notation and key concepts related to causal modeling and abstraction theory.

\spara{Notation.} The set of integers from $1$ to $n$ is $[n]$.
The vectors of zeros and ones of size $n$ are $\zeros_n$ and $\ones_n$.
The identity matrix of size $n \times n$ is $\identity_n$. The Frobenius norm is $\frob{\mathbf{A}}$.
The set of positive definite matrices over $\reall^{n\times n}$ is $\pd^n$. The Hadamard product is $\odot$.
Function composition is $\circ$.
The domain of a function is $\dom{\cdot}$ and its kernel $\ker$.
Let $\mathcal{M}(\mathcal{X}^n)$ be the set of Borel measures over $\mathcal{X}^n \subseteq \reall^n$. Given a measure $\mu^n \in \mathcal{M}(\mathcal{X}^n)$ and a measurable map $\varphi^{\V}$, $\mathcal{X}^n \ni \mathbf{x} \overset{\varphi^{\V}}{\longmapsto} \V^\top \mathbf{x} \in \mathcal{X}^m$, we denote by $\varphi^{\V}_{\#}(\mu^n) \coloneqq \mu^n(\varphi^{\V^{-1}}(\mathbf{x}))$ the pushforward measure $\mu^m \in \mathcal{M}(\mathcal{X}^m)$. 


We now present the standard definition of SCM.

\begin{definition}[SCM, \citealp{pearl2009causality}]\label{def:SCM}
A (Markovian) structural causal model (SCM) $\scm^n$ is a tuple $\langle \myendogenous, \myexogenous, \myfunctional, \zeta^\myexogenous \rangle$, where \emph{(i)} $\myendogenous = \{X_1, \ldots, X_n\}$ is a set of $n$ endogenous random variables; \emph{(ii)} $\myexogenous =\{Z_1,\ldots,Z_n\}$ is a set of $n$ exogenous variables; \emph{(iii)} $\myfunctional$ is a set of $n$ functional assignments such that $X_i=f_i(\parents_i, Z_i)$, $\forall \; i \in [n]$, with $ \parents_i \subseteq \myendogenous \setminus \{ X_i\}$; \emph{(iv)} $\zeta^\myexogenous$ is a product probability measure over independent exogenous variables $\zeta^\myexogenous=\prod_{i \in [n]} \zeta^i$, where $\zeta^i=P(Z_i)$. 
\end{definition}
A Markovian SCM induces a directed acyclic graph (DAG) $\mathcal{G}_{\scm^n}$ where the nodes represent the variables $\myendogenous$ and the edges are determined by the structural functions $\myfunctional$; $ \parents_i$ constitutes then the parent set for $X_i$. Furthermore, we can recursively rewrite the set of structural function $\myfunctional$ as a set of mixing functions $\mymixing$ dependent only on the exogenous variables (cf. \cref{app:CA}). A key feature for studying causality is the possibility of defining interventions on the model:
\begin{definition}[Hard intervention, \citealp{pearl2009causality}]\label{def:intervention}
Given SCM $\scm^n = \langle \myendogenous, \myexogenous, \myfunctional, \zeta^\myexogenous \rangle$, a (hard) intervention $\iota = \operatorname{do}(\myendogenous^{\iota} = \mathbf{x}^{\iota})$, $\myendogenous^{\iota}\subseteq \myendogenous$,
is an operator that generates a new post-intervention SCM $\scm^n_\iota = \langle \myendogenous, \myexogenous, \myfunctional_\iota, \zeta^\myexogenous \rangle$ by replacing each function $f_i$ for $X_i\in\myendogenous^{\iota}$ with the constant $x_i^\iota\in \mathbf{x}^\iota$. 
Graphically, an intervention mutilates $\mathcal{G}_{\mathsf{M}^n}$ by removing all the incoming edges of the variables in $\myendogenous^{\iota}$.
\end{definition}

Given multiple SCMs describing the same system at different levels of granularity, CA provides the definition of an $\alpha$-abstraction map to relate these SCMs:
\begin{definition}[$\abst$-abstraction, \citealp{rischel2020category}]\label{def:abstraction}
Given low-level $\mathsf{M}^\ell$ and high-level $\mathsf{M}^h$ SCMs, an $\abst$-abstraction is a triple $\abst = \langle \Rset, \amap, \alphamap{} \rangle$, where \emph{(i)} $\Rset \subseteq \datalow$ is a subset of relevant variables in $\mathsf{M}^\ell$; \emph{(ii)} $\amap: \Rset \rightarrow \datahigh$ is a surjective function between the relevant variables of $\mathsf{M}^\ell$ and the endogenous variables of $\mathsf{M}^h$; \emph{(iii)} $\alphamap{}: \dom{\Rset} \rightarrow \dom{\datahigh}$ is a modular function $\alphamap{} = \bigotimes_{i\in[n]} \alphamap{X^h_i}$ made up by surjective functions $\alphamap{X^h_i}: \dom{\amap^{-1}(X^h_i)} \rightarrow \dom{X^h_i}$ from the outcome of low-level variables $\amap^{-1}(X^h_i) \in \datalow$ onto outcomes of the high-level variables $X^h_i \in \datahigh$.
\end{definition}
Notice that an $\abst$-abstraction simultaneously maps variables via the function $\amap$ and values through the function $\alphamap{}$. The definition itself does not place any constraint on these functions, although a common requirement in the literature is for the abstraction to satisfy \emph{interventional consistency} \cite{rubenstein2017causal,rischel2020category,beckers2019abstracting}. An important class of such well-behaved abstractions is \emph{constructive linear abstraction}, for which the following properties hold. By constructivity, \emph{(i)} $\abst$ is interventionally consistent; \emph{(ii)} all low-level variables are relevant $\Rset=\datalow$; \emph{(iii)} in addition to the map $\alphamap{}$ between endogenous variables, there exists a map ${\alphamap{}}_U$ between exogenous variables satisfying interventional consistency \cite{beckers2019abstracting,schooltink2024aligning}. By linearity, $\alphamap{} = \V^\top \in \reall^{h \times \ell}$ \cite{massidda2024learningcausalabstractionslinear}. \cref{app:CA} provides formal definitions for interventional consistency, linear and constructive abstraction.
\section{Category-theory formalization}
Standard category-theoretic formalization of CA \cite{rischel2020category,otsuka2022equivalence} are based on a functorial semantics \cite{jacobs2019causal} approach mapping the graphical structure of causal models (\emph{syntax}) onto the discrete distributions of individual variables (\emph{semantics}). 
Because of our non-assumption \hyperlink{(NA2)}{(NA2)}, no knowledge of the structure of an SCM is available in our setting; thus, we propose a formalization mapping a dyadic structure (\emph{syntax}) onto the exogenous and the endogenous probability measures implied by an SCM (\emph{semantics}).

A crucial role in our modelling is that of the mixing functions \mymixing, which express the data generation process as a recursive process from the exogenous functions. This allows us to define an SCM $\scm^n$ in measure-theoretic terms as a tuple made up of the probability space of exogenous variables $(\myexogenousvals,\, \Sigma_{\myexogenousvals}, \zeta)$, the probability space of the endogenous variables $(\myendogenousvals,\, \Sigma_{\myendogenousvals}, \chi)$, and a set of measurable functions $\mymixing$ given by the mixing functions (cf. \cref{app:CA}).

We can now rely on this representation to  interpret an SCM as a category-theoretic functor from a simple index category \Index, made up only of a source and a sink object and an edge between them, to the category of probability spaces \Prob, where objects $(X,\Sigma_X, p)$ are probability spaces and morphisms $\varphi$ are measurable maps:
\begin{definition}[Category-theoretic SCM]\label{def:SCM_ct}
    An SCM is a functor $\scm^n: \Index \rightarrow \Prob$, mapping the source node of \Index to $(\myexogenousvals,\, \Sigma_{\myexogenousvals}, \zeta)$, the sink node of \Index to $(\myendogenousvals,\, \Sigma_{\myendogenousvals}, \chi)$, and the  edge of \Index to the collection \mymixing of measurable maps.
\end{definition} 



\begin{figure}
    \centering
    \begin{tikzpicture}[scale=1.]

    \draw[dashed] (-0.5, 2.8) rectangle (.5, -.8);
    \draw[dashed] (1., 2.8) rectangle (7.75, -.8);

    \node at (0, 3.2) {\Index};
    \node at (4, 3.2) {\Prob};
    
    \node[circle, draw, fill,inner sep=1pt] (A) at (0, 2) {};
    \node[circle, draw, fill,inner sep=1pt] (B) at (0, 0) {};
    
    \node (C) at (2.2, 2) {$(\myexogenousvals^\ell,\, \Sigma_{\myexogenousvals^\ell}, \zeta^\ell)$};
    \node (D) at (2.2, 0) {$(\myendogenousvals^\ell,\, \Sigma_{\myendogenousvals^\ell}, \chi^\ell)$};
    \draw[->] (C) -- node[left] {$\mymixing^\ell$} (D);

    \node (F) at (6.4, 2) {$(\myexogenousvals^h,\, \Sigma_{\myexogenousvals^h}, \zeta^h)$};
    \node (G) at (6.4, 0) {$(\myendogenousvals^h,\, \Sigma_{\myendogenousvals^h}, \chi^h)$};
    \draw[->] (F) -- node[right] {$\mymixing^h$} (G);

    \coordinate (A1shift) at ([yshift=-5pt]A);
    \draw[->,shorten >=2pt] (A1shift) -- (B);

    \draw[mypurple, rounded corners=10pt]($(C.west)+(0,0.3)$) rectangle ($(D.east)+(0.05,-0.3)$);
    % \draw[->, mypurple] (0.2,1) -- (2,1.7);
    \draw[->, mypurple] (A) -- (C);
    \draw[->, mypurple] (B) -- (D);

    \draw[cyan, rounded corners=10pt]($(F.west)+(0,0.3)$) rectangle ($(G.east)+(0.05,-0.3)$);
    % \draw[->, cyan] (0.2,1) -- (2,0.6);
    \draw[->, cyan] (A) to[bend left=20] (F);
    \draw[->, cyan] (B) to[bend right=20] (G);

    \draw[->, dashed] (C) -- node[above] {$\alphamap{\left(\myexogenousvals^{h},\, \Sigma_{\myexogenousvals^h}\right)}$} (F);
    \draw[->, dashed] (D) -- node[below] {$\alphamap{\left(\myendogenousvals^{h},\, \Sigma_{\myendogenousvals^h}\right)}$} (G);
    
    \end{tikzpicture}
    \caption{An abstraction as natural transformation, that is, a set of commuting arrows in \Prob (dashed black) from {\color{mypurple} $\scm^\ell$} (purple) to {\color{cyan} $\scm^h$} (cyan).}
    \label{fig:functor2}
\end{figure}

% \cref{app:CA} presents basic category-theoretic concepts and defines the mentioned categories.
\black{\Cref{app:CT_background} presents basic category-theoretic concepts, whereas \cref{subsec:SCM_prob} deepens \cref{def:SCM_ct}.}
CA can now be expressed as a natural transformation between two SCMs, as shown in \cref{fig:functor2}. This formulation has two important features.
First, it highlights the role of exogenous variables in a constructive abstraction showing the commutativity of the paths $\mymixing^h \circ \alphamap{\left(\myexogenousvals^{h},\, \Sigma_{\myexogenousvals^h}\right)}$ and $\alphamap{\left(\myendogenousvals^{h},\, \Sigma_{\myendogenousvals^h}\right)} \circ \mymixing^\ell$. Second, morphisms in \Prob relates measure spaces, viz. sets equipped with sigma algebras. Consequently, the natural transformation components are measurable maps with dimensionality determined by the cardinality of $\myendogenous^h$ and $\myendogenous^\ell$.
To ease the notation, we will denote $\alphamap{\left(\myexogenousvals^{h},\, \Sigma_{\myexogenousvals^h}\right)}$ by $\alphamap{\myexogenous}$ and $\alphamap{\left(\myendogenousvals^{h},\, \Sigma_{\myendogenousvals^h}\right)} $ by $\alphamap{\myendogenous}$.
Then, we can formally recast the $\alpha$-abstraction in \Prob.
\begin{definition}[$\abst$-abstraction in \Prob]\label{def:alpha_abstraction_prob}
    Given low-level $\scm^\ell$ and high-level $\scm^h$ SCMs, %as in \Cref{def:SCM_prob}, 
    an abstraction $\abst = \langle \Rset, \Qset, \amap, \alphamap{} \rangle$ is a tuple, where: \emph{(i)} \Rset is the same as in \Cref{def:abstraction}; \emph{(ii)} $\Qset \subseteq \myexogenous^\ell$ is a set of relevant exogenous variables given by the union of the set of exogenous corresponding to the endogenous in \Rset and those corresponding to their ancestors; \emph{(iii)} $\amap=\langle \amap_\myexogenous, \amap_\myendogenous \rangle$ is a pair of surjective functions mapping sets, $\amap_\myexogenous: \Qset \rightarrow \myexogenous^h$ and $\amap_\myendogenous: \Rset \rightarrow \myendogenous^h$, respectively; \emph{(iv)} $\alphamap{}=\langle \alphamap{\myexogenous}, \alphamap{\myendogenous} \rangle$ is a natural transformation made by measurable functions mapping probability spaces, $\alphamap{\myexogenous}$ for the exogenous and $\alphamap{\myendogenous}$ for the endogenous, respectively.  
\end{definition}
As \Cref{def:abstraction}, \Cref{def:alpha_abstraction_prob} makes no reference to interventional consistency.
\Cref{app:CT} explains how intervened SCMs and interventional consistency can be represented categorically.
\section{Viewer-provider two-sided systems}

This section models the dynamics of viewer and provider populations on a recommendation platform. 
Specifically, we consider sub-group dynamics where viewers and providers are categorized into $K$ and $L$ subgroups\footnote{We can consider a ``subgroup'' of size 1. In such cases, the viewer ``population'' corresponds to the time spent by an individual viewer, while the provider ``population'' can be the amount of content produced by an individual provider.
}. Then, we model the populations, recommendation policy, payoffs, and social welfare as follows.

\begin{enumerate}[leftmargin=12pt]
    \item (Viewer/provider population)  
    Let $\lambda_{k} \in \mathbb{R}_{\geq 0}$ be the population of the viewer group $k \in [K]$ and $\lambda_{l}$ be that of the provider group $l \in [L]$. Also let $\blambda := (\lambda_{1}, \lambda_{2}, \cdots, \lambda_{K},
    \lambda_{1}, \lambda_{2}, \cdots, \lambda_{L})$ be the joint population vector of viewers and providers.
    \item (Platform's recommendation policy) 
    The platform matches each viewer group $k$ to a provider group $l$ with a recommendation policy denoted by a $K$-by-$L$ matrix $\bpi$. Specifically, its $(k,l)$-th element $\pi_{k,l}$ represents the probability of allocating the provider group $l$ to the viewer group $k$. 
    Thus $\sum_{l=1}^L \pi_{k,l} = 1, \forall k \in [K]$. For example, the uniform random policy, which assigns equal exposure probability across all provider groups is represented as given by $\bpi=\frac{1}{L}\1_{L\times K}$.
    \item (Viewer/provider payoffs) After viewer and provider groups are matched by the policy $\bpi$, their perceived payoffs can be quantified by the following metrics:
    \begin{align}\label{eq:user_satisfaction}
    \text{Viewer Satisfaction: \quad } & s_k = \textstyle \sum_{l=1}^L \pi_{k,l} q_{k,l} \,  , \\\label{eq:content_exposure}
    \text{Provider Exposure: \quad} & e_l = \textstyle\sum_{k=1}^K \pi_{k,l}\lambda_k,
    \end{align}
    where $q_{k,l}$ is the (expected) utility that viewers $k$ receive from the provider groups $l$. Eqs.~\eqref{eq:user_satisfaction} and~\eqref{eq:content_exposure} define viewer satisfaction as determined by the total utility they receive from recommendations, while providers care about the total amount of exposure they receive by recommendation. This model is prevalent is prior works including \citep{singh2018fairness, mladenov2020optimizing}.
    \item (Social welfare) Finally, we consider the following total viewer welfare as the global metric of the platform:
    \begin{align*}
        R(\bpi; \blambda) := \textstyle\sum_{k=1}^{K} \lambda_{k} s_k
    \end{align*}
    Note that here we consider the sum of viewer-side satisfaction as the social welfare, a formulation prevalent in related works~\citep{mladenov2020optimizing, huttenlocher2023matching}.
    The sum of content-side exposure simplifies to the total size of the viewer population.
\end{enumerate}

\subsection{Interaction dynamics and ``population effects''}\label{sec:dynamic_formulation}

We have so far seen a typical formulation in two-sided platforms. However, a key limitation of such formulation is to ignore potential non-stationarity in the viewer and provider populations, which is common in many real-world two-sided systems~\citep{boutilier2023modeling,  deffayet2024sardine}. 

First, consider the impact of provider population growth on the utility experience by viewers, which we call \textit{``population effects''}.
An increase in provider population naturally leads to more high-quality content. 
For example, consider a two-stage recommendation policy, where our higher-level policy $\bpi$ decides the matching between viewer and provider groups, and a second-stage policy selects individual providers among the selected group. 
Any reasonable second stage policy should be able to select a better provider from a larger provider pool~\citep{su2023value, evnine2024achieving}. 
To model such ``population effects'', we introduce the following utility decomposition:
\begin{align}
    q_{k,l} = b_{k,l} + f_{k,l}(\lambda_{l}) \label{eq:reward_decomposition}
\end{align}
where $b_{k,l}$ is the \textit{base} utility, which may indicates the matching between the preference of viewer and provider groups (e.g., this viewer group likes sports articles). In contrast, $f_{k,l}(\cdot)$ represents the quality of the provider which improves as the provider population increases. $f_{k,l}$ might be heterogeneous among different viewer and provider groups because quality might be multi-dimensional (e.g., visuals, intellects, novelty), viewers may have different preferences, and providers may have different abilities. 
We take $f_{k,l}$ to be a monotonically increasing function.

Next, consider the impact of viewer and provider payoffs on the population.
The number of viewers that a platform can maintain is related to the level of satisfaction, similarly the number of providers is related to the exposure.
We assume that viewer and provider subgroups have 
some \textit{``reference''} population $\bar{\lambda}_{k}(s_{k})$ and $\bar{\lambda}_{l}(e_{l})$ given the level of viewer satisfaction $s_k$ and provider exposure $e_l$. We assume that $\bar{\lambda}$ is a monotonically increasing function, so higher viewer satisfaction and provider exposure result in increased populations. 
Based on this, we model the viewer and provider population dynamics as:
\begin{align}
    \text{Viewer: \,}  \lambda_{t+1,k} = (1 - \eta_k) \lambda_{t,k} + \eta_k \bar{\lambda}_{k}(s_{t,k}), \label{eq:user_dynamics} \\
    \text{Content: \,}  \lambda_{t+1,l} = (1 - \eta_l) \lambda_{t,l} + \eta_l \bar{\lambda}_{l}(e_{t,l}), \label{eq:content_dynamics}
\end{align}
where $\eta \in [0, 1]$ are the \textit{reactiveness} hyperparams, determining how fast the population changes. Note that similar models are widely adopted in performative predictions~\citep{perdomo2020performative, brown2022performative}. 
We thus have that the viewer satisfaction $s_k$ depends on the provider population via ``population effects'' $f_{k,l}$, while the provider exposure directly depends on the viewer population.
The two-sided platform has complex dynamics between viewers and providers. 
Our goal will be to consider long-term objectives under such co-evolving and two-sided dynamics.

\subsection{Game-theoretic interpretation}\label{sec:game_formulation}

Next, we provide a further justification of and insight into the dynamics model by introducing a game-theoretic formulation that is equivalent to Eqs. \eqref{eq:user_dynamics} and \eqref{eq:content_dynamics}.

Consider a $(K+L)$-player game involving $K$ viewer groups and $L$ provider groups. Each viewer group selects a pure strategy $\lambda_k \in \RR_{\geq 0}$, and each provider group chooses a pure strategy $\lambda_l \in \RR_{\geq 0}$. The utility functions for the viewer and provider groups, denoted by $\{u_k\}_{k=1}^K$ and $\{v_l\}_{l=1}^L$ are defined as follows:
\begin{align}\label{eq:util_user}
    & u_k(\blambda)= \lambda_k \cdot \bar{\lambda}_k \left(\sum_{l=1}^L \pi_{k,l}\left(b_{k,l}+f_{k,l}(\lambda_l)\right)\right)-\frac{\lambda_k^2}{2}, \\ \label{eq:util_creator}
    & v_l(\blambda)= \lambda_l\cdot \bar{\lambda}_l \left(\textstyle\sum_{k=1}^K \pi_{k,l}\lambda_k\right)-\frac{\lambda_l^2}{2},
\end{align}
We denote this game as $\G(\bpi, B, f, \bar{\lambda})$, where $B$ is a $K$-by-$L$ matrix whose $(k,l)$-element is $b_{k,l}$. Proposition \ref{prop:dynamics_equivalence} establishes a connection between the game instance $\G$ and the 
formulation presented in Section \ref{sec:dynamic_formulation}.

\begin{proposition}\label{prop:dynamics_equivalence}
    If all players in $\G$ apply gradient ascent to optimize their utility functions with learning rates $\{\eta_k\}_{k=1}^K$ and $\{\eta_l\}_{l=1}^L$, the resulting joint strategy evolving dynamics are exactly given by Eqs.~\eqref{eq:user_dynamics} and \eqref{eq:content_dynamics}.
\end{proposition}

Through Proposition \ref{prop:dynamics_equivalence}, our game-theoretic formulation provides a first-principles perspective for understanding the dynamical formulation in Eqs.~\eqref{eq:user_dynamics} and \eqref{eq:content_dynamics}.\footnote{The game $\G$ resembles the Cournot Duopoly competition \cite{cournot1838recherches}. When $K = L = 1$ and $\bar{\lambda}(\mu) = a - b\mu$ and $\bar{\mu}(\lambda) = a - b\lambda$ for some positive constants $a$ and $b$, the game $\G$ corresponds exactly to the Cournot Duopoly model. The key distinction in ours is that $\bar{\mu}$ and $\bar{\lambda}$ are generic increasing functions.} 
That is, 
we can interpret $\bar{\lambda}(\cdot)$ as the marginal gain from increasing the size of a viewer or provider group by one unit. Consequently, the first terms $\lambda_k \cdot \bar{\lambda}_k(\cdot)$ and $\lambda_l \cdot \bar{\lambda}_l(\cdot)$ represent the collective payoffs for viewer and provider groups of sizes $\lambda_k$ and $\lambda_l$. 
The quadratic terms $-\frac{\lambda_k^2}{2}$ and $-\frac{\lambda_l^2}{2}$ capture the congestion costs associated with maintaining larger populations (e.g., if a provider group becomes too large, providers within the group may face intensified competition and thus reduce their productivity due to diminished marginal gains). This suggests that Eqs.~\eqref{eq:user_dynamics} and \eqref{eq:content_dynamics} are quite reasonable formulation to capture real-world interactions.
% !TEX root =  ../main.tex
\section{Problem solution}\label{sec:problem_solution}
To solve the nonsmooth and smooth Riemannian problems in \cref{sec:problem_formulation}, we leverage the following:

\begin{restatable}{proposition}{smoothnessth}\label{prop:smoothness_and_differentiability}
    Consider the function
    \begin{equation}\label{eq:general_objective}
        f(\A)\!=\!\Tr{\!\left(\A^\top \covlow \A \right)^{-1} \! \covhigh} + \log\det{\A^\top \covlow \A }\,.
    \end{equation}
    \Cref{eq:general_objective} is smooth for $\A \in \stiefel{\ell}{h}$.
    Additionally, define $\widetilde{\mathbf{A}}\coloneqq\left(\mathbf{A}^\top \covlow \mathbf{A}\right)^{-1}$.
    The gradient of $f\left(\A\right)$ is
    \begin{equation}\label{eq:gradA}
        \Egrad{\A}{f} = 2\left(\covlow\mathbf{A}\widetilde{\mathbf{A}}\right)\left(\identity_h - \covhigh\widetilde{\mathbf{A}}\right)\,,
    \end{equation}
\end{restatable}
\begin{proof}
See \cref{app:proof}.
\end{proof}

\subsection{Solution of the nonsmooth learning problem}\label{subsec:sol_nonsmoot}
Leveraging \cref{prop:smoothness_and_differentiability}, we have that \cref{eq:minKL} is constituted by a smooth yet nonconvex term, $f(\V)$, and a nonsmooth one, $h(\V)$. Hence we solve \cref{prob:nonsmooth} through two different optimization paradigms for nonsmooth Riemannian optimization: MADMM and ManPG.
We term the proposed methods \emph{LinSEPAL-ADMM} and \emph{LinSEPAL-PG}, where \textit{LinSEPAL} stands for \textbf{Lin}ear \textbf{S}emantic \textbf{E}mbedding \textbf{P}rinciple \textbf{A}bstraction \textbf{L}earner.
Next we provide a sketch of the solution and provide the full mathematical derivation in \cref{app:MADMM} and \Cref{app:ManPG}.

\spara{LinSEPAL-ADMM.} The MADMM framework appeals to our setting given the objective function separating into smooth and nonsmooth terms.
To derive the LinSEPAL-ADMM iterative algorithm, we proceed as follows.
First, the nonsmooth term $h(\V)$ is associated with a splitting variable \Y to be optimized over \rmatdim, obtaining an equivalent problem formulation (cf. \cref{eq:MADMM}). LinSEPAL-ADMM proceeds by iteratively minimizing the augmented Lagrangian with respect to the primal variables \V and \Y, while maximizing w.r.t. the scaled dual variable. 
Specifically, LinSEPAL-ADMM solves the subproblem for \V (cf. \Cref{eq:updateV}) through standard techniques for smooth optimization on the Stiefel manifold (e.g., conjugate gradient, \citealp{edelman1998geometry}).
This is the most complex update in the LinSEPAL-ADMM iterative procedure due to the nonconvex objective and the Stiefel manifold.
Next, LinSEPAL-ADMM updates \Y in closed form through the element-wise soft-thresholding operator (cf. \Cref{eq:updateY_madmm}). Finally, the scaled dual variable is updated by adding the primal residual evaluated at the current solution (cf. \cref{eq:MADMMrecursion_app}).
The stopping criteria for LinSEPAL-ADMM are established according to primal and dual feasibility optimality conditions (\citealp{boyd2011distributed}, cf. \cref{app:MADMM}).
To the best of our knowledge, the convergence guarantee for MADMM in the Riemannian space has not been proven.
Consequently, the same holds for LinSEPAL-ADMM.
\Cref{alg:linsepal_admm} summarizes the method.\\

\spara{LinSEPAL-PG.} Our LinSEPAL-PG is an iterative algorithm alternating two updates (cf. \cref{eq:ManPG_app}).
The first update is the proximal mapping providing a proximal gradient direction $\G^k$ onto the tangent space to the Stiefel manifold, using the first-order approximation of the objective around the $k$-th estimate.
The second is the update for $\V^{k+1}$, which exploits the canonical retraction (cf. \cref{eq:stiefel_retractions}) technique for projecting back $\V^k + \G^k$ from the tangent space to the manifold. The hardest step in the LinSEPAL-PG algorithm is the proximal update (cf. \cref{eq:ManPGU1}).
We solve it by declining the regularized semi-smooth Newton method \cite{xiao2018regularized} to our application (cf. \cref{app:ManPG}).
Following the rationale in \cite{si2024riemannian}, differently from the original ManPG method which uses the parameterization of the tangent space, we constrain $\G^k$ to the tangent space by exploiting the basis of the normal space to the manifold (cf. \Cref{eq:basis_nvk}).
This way, we ease the mathematical solution, with benefits from the computational perspective (cf. \citealp{si2024riemannian}).
Next, LinSEPAL-PG updates $\V^{k+1}$ in closed form (cf. \cref{eq:updateV_linsepalpg}) by applying the QR-retraction, employing an Armijo line-search procedure to determine the stepsize.
The optimization stops either when a maximum number of iterations is reached, or when $\KL{\V^{k+1}}$ is below a threshold $\tau^{\mathrm{KL}}\approx 0$.
LinSEPAL-PG inherits the global convergence of the ManPG framework, established in \cite{chen2020}.
\Cref{alg:linsepal_pg} summarizes the method.

\subsection{Solution of the smooth learning problem}\label{subsec:sol_smooth}
We provide a sketch of the solution below and the full mathematical derivation in \Cref{app:MADMMSCA_partial}.
In this case, we want to jointly optimize \Supp and \V, both being components of the linear CA, viz. $(\B \odot \Supp \odot \V)^\top$.
Hence, unlike the nonsmooth case, we constrain to the Stiefel manifold the product $(\B \odot \Supp \odot \V)$. \black{To solve \Cref{prob:nonconvex_prob_approx}, we combine the SOC, ADMM, and SCA methods.}
According to the rationale behind SOC, we add two splitting variables, namely \YO and \YT in \stiefel{\ell}{h} (cf. \cref{eq:splitting_constraints_partial}), to separate the nonconvexity of the objective function from that induced by the manifold.
The reason why we have two splitting variables is that we need to take into account the bilinear form of the first constraint in \Cref{eq:prob_madmmsca_VS}.
Additionally, to manage the second constraint in \Cref{eq:prob_madmmsca_VS}, we introduce another splitting variable $\X \in \sphere{h}{\ell}$.
Starting from the equivalent problem formulation (cf. \Cref{eq:prob_madmmsca_VS_with_splitting}), we write the (nonconvex) scaled augmented Lagrangian (cf. \Cref{eq:scaledAUL_partial}), thus arriving at the update recursion for our proposed method (cf. \Cref{eq:ADMM_partial}).
We term the latter \emph{CLinSEPAL} (Constructive LinSEPAL) to highlight that it returns constructive support for CA.
CLinSEPAL proceeds by iteratively minimizing the augmented Lagrangian w.r.t. the primal variables \V, \Supp, \YO, \YT, and \X; and maximizing it w.r.t. the scaled dual ones.
In the subproblems for \V (cf. \Cref{eq:updateV_nonconvex_partial}) and \Supp (cf. \Cref{eq:updateS_nonconvex_partial}), we adopt the SCA paradigm to manage the nonconvexity of $f(\V,\Supp^k)$ and $f(\V^{k+1},\Supp)$, respectively.
By exploiting the smoothness of $f(\V,\Supp)$ (cf. \cref{corollary:objective_partial_knowledge}), the strongly convex surrogates are derived around the current solution (cf. \cref{eq:strongly_convex_surrogate_Vpartial,eq:strongly_convex_surrogate_Spartial}).
CLinSEPAL solves the strongly convex surrogate subproblems (cf. \cref{eq:SCA_recursion_V,eq:SCA_recursion_S}) exactly.
Due to the presence of the inequality constraints, the subproblem for \Supp is a constrained quadratic programming problem.
CLinSEPAL solves it via standard techniques (e.g., \citealp{osqp}).
These two steps in CLinSEPAL can be seen as an instance of the linearized ADMM framework (Alg.1 in \citealp{lu2021linearized}) where each internal update is solved exactly. Next, CLinSEPAL solves in closed-form the updates for the three splitting variables.
Indeed, the subproblems for \YO and \YT amount to the \emph{closest orthogonal approximation problem} \cite{fan1955some,higham1986computing}, whose solution is obtained in closed form via polar decomposition.
Subsequently, the subproblem for \X is solved in closed-form according to \Cref{lemma:proximal_spDelta}.
Finally, the scaled dual variables are updated with the corresponding primal residuals.
Empirical convergence for CLinSEPAL is established when the norms of primal (cf. \cref{eq:primal_res_partial}) and dual (cf. \cref{eq:dual_res_partial}) residuals vanish, in accordance with absolute and relative tolerances (cf. \cref{eq:stopping_criteria_partial}).
\Cref{alg:clinsepal} summarizes the method.
Additionally, \Cref{subsec:CLinSEPAL_full_prior} details the solution in the special case of full prior knowledge.
% !TEX root =  ../main.tex
\begin{figure}[t]
    \centering
    \includegraphics[width=.9\columnwidth]{figs/20_full_prior_synthetic_exp.pdf}
    \caption{
    Synthetic fp results for all settings $(\ell, h)$ and methods: \emph{(i)} fraction of learned CAs that are constructive, \emph{(ii)} $\KL{\Vhat}$, \emph{(iii)} normalized  absolute Frobenius distance from $\V^\star$, and \emph{(iv)} $\mathrm{F1}$ score.
    }
    \label{fig:full_synth_data}
\end{figure}
\section{Empirical assessment on synthetic data}\label{sec:empirical_assessment}
This section provides the empirical assessment of LinSEPAL-ADMM, LinSEPAL-PG and CLinSEPAL with different degrees of prior knowledge,
from full (\emph{fp}) to partial (\emph{pp}). 
We monitor four metrics to evaluate the learned CA $\Vhat^\top$:  
\emph{(i)} \emph{constructiveness}, as required by \cref{def:semantic_embedding_principle_linear};  
\emph{(ii)} $\KL{\Vhat}$ evaluating the alignment between $\varphi^{\Vhat}_{\#}(\measurelow)$ and \measurehigh; 
\emph{(iii)} the Frobenius distance between the absolute value of \Vhat and that of the ground truth $\V^\star$, normalized by \frob{\V^\star} to make the settings comparable; 
\emph{(iv)} the $\mathrm{F1}$ score computed using the support of the learned CAs and that of $\V^\star$ to evaluate structural interventional consistency.
\cref{app:metrics} provides the definition for the above metrics and the hyper-parameters values used in the experiments.

\spara{Full prior knowledge.}
In the fp case, we investigate three different settings $(\ell,h)\in \{(12,2), (12,4), (12,6)\}$, corresponding to the cases of \emph{high}, \emph{medium-high}, and \emph{medium} coarse-graining.
We do not consider the case where $h>\ell/2$ since the abstraction for $h-\ell/2$ nodes of the low-level model would be fully specified due to the availability of full prior knowledge.
For each setting, we instantiate $S=30$ ground truth abstractions $\V^\star$, and for each simulation $s \in [S]$ we run all the methods $R=50$ times, with different initializations.
Then, for each $s$ and method, we retain the \Vhat minimizing the objective $\KL{}$.\\
\Cref{fig:full_synth_data} shows the performance of the tested methods. 
All the methods provide constructive CAs $\forall \, s \in [S]$, and reach a good level of alignment in terms of $\KL{\Vhat}$.
Recall that, while CLinSEPAL and LinSEPAL-ADMM stop the learning procedure according to primal and dual residuals convergence, LinSEPAL-PG exits when $\KL{\Vhat}$ is below a certain threshold $\tau^{\mathrm{KL}}$ (in the experiments $\tau^{\mathrm{KL}}=10^{-4}$). 
The Frobenius absolute distance shows comparable performances for the three methods, although CLinSEPAL and LinSEPAL-ADMM outperform in case $(\ell, h)=(12,4)$.
This metric tells us that, as $h$ increases, the learned \Vhat tends (in absolute terms) to the ground truth.
Interestingly, when $(\ell, h)=(12,2)$ we observe a high distance from $\V^\star$, although the learned \Vhat has the correct structure (cf. $\mathrm{F1}$ score).
This suggests that under a high coarse-graining, the size of \myker \KL{} grows, and it is more difficult for our methods to estimate $\V^\star$ under \hyperlink{(NA1)}{(NA1)}-\hyperlink{(NA5)}{(NA5)}.
Finally, the $\mathrm{F1}$ score confirms that the methods guarantee the true CA structure of \Vhat, for all the settings.
To sum up, CLinSEPAL and LinSEPAL-ADMM are slightly better choices than LinSEPAL-PG in case of full prior knowledge in our experimental setting.

\begin{figure}[t]
    \centering
    \includegraphics[width=.9\columnwidth]{figs/20_partial_prior_synthetic_exp.pdf}
    \caption{Synthetic pp results for setting $(\ell, h)=(4,2)$, all methods, and prior knowledge amounting to the correct structural mapping for $25\%$, $50\%$, or $75\%$ of the nodes. All plots as in \cref{fig:full_synth_data}.
    }
    \label{fig:partial_synth_data}
\end{figure}

\spara{Partial prior knowledge.}
In the pp case, we consider the setting $(\ell,h)\in \{(4,2)\}$ and simulate partial prior knowledge by forgetting the mapping for $25\%$, $50\%$, and $75\%$ of the variables. For each setting, we instantiate $S=30$ ground truth abstractions $\V^\star$, and for each simulation $s \in [S]$ we run all the methods $R=30$ times, with different initializations.\\
In \cref{fig:partial_synth_data}, the first plot immediately shows that only CLinSEPAL consistently returns a constructive linear CA, as guaranteed by its formulation in \cref{prob:nonconvex_prob_approx}. 
We decided to consider methods performing under a threshold of $90\%$ to be unreliable in returning constructive CAs and not to report their remaining metrics.
In the case of a limited drop of prior knowledge ($25\%$) all methods perform well, similarly to the fp case, with CLinSEPAL and LinSEPAL-ADMM slightly outperforming LinSEPAL-PG.
With a higher drop ($50\%$), LinSEPAL-PG fails to achieve our constructiveness threshold, while CLinSEPAL and LinSEPAL-ADMM still perform well, although LinSEPAL-ADMM provides a lower fraction of constructive CAs. 
Finally, with the highest drop ($75\%$) CLinSEPAL succeeds in learning a constructive CA and lowering $\KL{}$, even if the Frobenius absolute distance slightly increases. 
To sum up, for the pp setting only CLinSEPAL guarantees a constructive abstraction.
% !TEX root =  ../main.tex
\section{Causal abstraction of brain networks}\label{sec:empirical_assessment_rw}
To show the practical relevance of our approach, we apply CLinSEPAL to resting-state functional magnetic resonance imaging (rs-fMRI) data, using the dataset from \cite{gabriele2024extracting} (refer to the paper for details on the dataset). The data, publicly released as part of the \emph{Human Connectome Project} \cite{smith2013resting},
comprises recordings from $100$ healthy adults with a parcellation scheme that divides the brain into $89$ regions of interest (ROIs), $K=44$ for each hemisphere plus the shared vermis region.

We simulate a first investigating team of neuroscientists taking zero-mean stationary time series for the left hemisphere of the first adult in the dataset. They estimate the data covariance matrix using a Gaussian mixture probability model, viz. $\covlow \in \reall^{\ell \times \ell}$, with $\ell=K+1$, and interpret it as generated by an underlying, unknown, low-level SCM.

In a first fp scenario, we imagine a second investigating team that has collected data according to their causal network specified on a coarser parcellation of the same brain in $h=14$ macro ROIs. We generate the data for the second team using a ground truth linear CA $\B, \V^\star \in \stiefel{45}{14}$ based on the structural mapping in \cite{gabriele2024extracting}, and use the data for estimating the covariance matrix $\covhigh \in \reall^{h \times h}$. In this scenario it is realistic to assume knowledge of $\B$ defining how macro ROIa are mapped to ROIs. Then, to align their models, the two groups run CLinSEPAL to recover the abstraction given $\covlow,\covhigh$ and $\B$. \Cref{fig:ROIsLobes} (in \Cref{app:rw_figs}) shows that CLinSEPAL recovers $\V^\star$.

In a second pp scenario, we imagine that the second investigating team has collected data according to a causal network aggregating ROI time series into $h=8$ brain functional networks related to different activities (e.g., motor, visual, default mode). Data is generated again through a ground truth linear CA $\B, \V^\star \in \stiefel{45}{8}$ based on groupings in \cite{gabriele2024extracting} and the covariance matrix $\covhigh \in \reall^{h \times h}$ computed. In this scenario, knowledge of $\B$ is debatable as different studies in the literature suggest different relations between ROIs and functions; we then express this partial information via uncertainty over $\B$, meaning that some rows of \B have more than one entry equal to one. The two groups now run CLinSEPAL using $\covlow,\covhigh$ and an uncertain $\B$; partial knowledge compounds on an already challenging learning problem due to the high coarse-graining. \Cref{fig:ROIsFun_ca} and \Cref{fig:ROIsFun_metrics} show results with different levels of uncertainty. For low uncertainty, CLinSEPAL correctly retrieves the structure of the CA, although we observe some variation in the colors w.r.t. $\V^\star$; additionally, \KL{\Vhat} and the Frobenius absolute distance in \cref{fig:ROIsFun_metrics} show that misalignment is minimized and \Vhat very close to $\V^\star$. For medium and high uncertainty, CLinSEPAL makes some mistakes in terms of structural mapping, but \cref{fig:ROIsFun_metrics} shows that insights from the method are still valuable.
\section{Conclusion}
In this work, we propose a simple yet effective approach, called SMILE, for graph few-shot learning with fewer tasks. Specifically, we introduce a novel dual-level mixup strategy, including within-task and across-task mixup, for enriching the diversity of nodes within each task and the diversity of tasks. Also, we incorporate the degree-based prior information to learn expressive node embeddings. Theoretically, we prove that SMILE effectively enhances the model's generalization performance. Empirically, we conduct extensive experiments on multiple benchmarks and the results suggest that SMILE significantly outperforms other baselines, including both in-domain and cross-domain few-shot settings.
% \clearpage

% \section*{Accessibility}
% Authors are kindly asked to make their submissions as accessible as possible for everyone including people with disabilities and sensory or neurological differences.
% Tips of how to achieve this and what to pay attention to will be provided on the conference website \url{http://icml.cc/}.

% \section*{Software and Data}

% If a paper is accepted, we strongly encourage the publication of software and data with the
% camera-ready version of the paper whenever appropriate. This can be
% done by including a URL in the camera-ready copy. However, \textbf{do not}
% include URLs that reveal your institution or identity in your
% submission for review. Instead, provide an anonymous URL or upload
% the material as ``Supplementary Material'' into the OpenReview reviewing
% system. Note that reviewers are not required to look at this material
% when writing their review.

% Acknowledgements should only appear in the accepted version.
% \section*{Acknowledgements}

% \textbf{Do not} include acknowledgements in the initial version of
% the paper submitted for blind review.

% If a paper is accepted, the final camera-ready version can (and
% usually should) include acknowledgements.  Such acknowledgements
% should be placed at the end of the section, in an unnumbered section
% that does not count towards the paper page limit. Typically, this will 
% include thanks to reviewers who gave useful comments, to colleagues 
% who contributed to the ideas, and to funding agencies and corporate 
% sponsors that provided financial support.

\section*{Impact Statement}
\black{Our work is foundational, aiming at advancing the field of causal abstraction.
Our proposed methods can be applied to different application domains, such as neuroscience.
As demonstrated by our empirical assessment, the information resulting from their application is high-level and useful for a better understanding.
Hence, we believe that the risks associated with improper usage of our techniques are low.}

\section*{Acknowledgements}
The work of Gabriele D'Acunto and Paolo Di Lorenzo was supported by the SNS JU project 6G-GOALS \cite{strinati2024goal} under the EU's Horizon program Grant Agreement No 101139232. 
The work of Gabriele D'Acunto was also supported by the European Union under the Italian National Recovery and Resilience Plan (NRRP) of NextGenerationEU, partnership on `` Telecommunications of the Future'' (PE00000001 - program `` RESTART'').
The work of Yorgos Felekis was supported by the Onassis Foundation - Scholarship ID: F ZR 063-1/2021-2022.

\bibliography{bibliography}
\bibliographystyle{icml2025}

%%%%%%%%%%%%%%%%%%%%%%%%%%%%%%%%%%%%%%%%%%%%%%%%%%%%%%%%%%%%%%%%%%%%%%%%%%%%%%%
%%%%%%%%%%%%%%%%%%%%%%%%%%%%%%%%%%%%%%%%%%%%%%%%%%%%%%%%%%%%%%%%%%%%%%%%%%%%%%%
% APPENDIX
%%%%%%%%%%%%%%%%%%%%%%%%%%%%%%%%%%%%%%%%%%%%%%%%%%%%%%%%%%%%%%%%%%%%%%%%%%%%%%%
%%%%%%%%%%%%%%%%%%%%%%%%%%%%%%%%%%%%%%%%%%%%%%%%%%%%%%%%%%%%%%%%%%%%%%%%%%%%%%%
\newpage
\appendix
\onecolumn

% !TEX root =  ../main.tex

\section{Extended notation for the appendix}\label{app:eNot}
Below is the notation used throughout the appendices.
% Scalars are lowercase, $a$, vectors are lowercase bold, $\mathbf{a}$, matrices are uppercase bold, $\mathbf{A}$, and sets are uppercase calligraphic, $\mathcal{A}$. 
% The set of integers from $1$ to $n$ is denoted by $[n]$, and $[n]_0$ if zero is included.
The set of integers from $1$ to $n$ is $[n]$.
The vectors of zeros and ones of size $n$ are $\zeros_n$ and $\ones_n$.
The identity matrix of size $n \times n$ is $\identity_n$. 
The entry indexed by row $i$ and column $j$ is $a_{ij}=[\mathbf{A}]_{ij}$, $\mathrm{diag}(\mathbf{a})$ is the diagonal matrix having as diagonal the vector $\mathbf{a}$, while $\mathrm{diag}(\mathbf{A})$ is the diagonal of the matrix $\mathbf{A}$. 
The Frobenious norm is $\frob{\mathbf{A}}$.
The set of positive definite matrices over $\reall^{n\times n}$ is $\pd^n$.
That of symmetric ones as \sym{p}.
% We use subscripts to indicate indices, while superscripts are used for everything else.
The column-wise vectorization of a matrix is \myvec{}. 
% the Hadamard product is $\circ$.
The Hadamard product is $\odot$.
Function composition is $\circ$.

Let $\mathcal{M}(\mathcal{X}^n)$ be the set of Borel measures over $\mathcal{X}^n \subseteq \reall^n$.
Given a measure $\mu^n \in \mathcal{M}(\mathcal{X}^n)$ and a measurable map $\varphi^{\V}$, $\mathcal{X}^n \ni \mathbf{x} \overset{\varphi^{\V}}{\longmapsto} \V^\top \mathbf{x} \in \mathcal{X}^m$, we denote by $\varphi^{\V}_{\#}(\mu^n) \coloneqq \mu^n(\varphi^{\V^{-1}}(\mathbf{x}))$ the pushforward measure $\mu^m \in \mathcal{M}(\mathcal{X}^m)$. 
The proximal mapping of $h$ at $\mathbf{A}$ is $\mathrm{prox}_{\lambda \, h(\cdot)}(\mathbf{A})= \argmin_{\V} h(\V) + 1/(2\lambda) \, \frob{\V - \mathbf{A}}^2$, $\lambda \in \reall^+$. 
The Euclidean gradient of a smooth $f$ is \Egrad{ }{f}, while the Riemannian one \Rgrad{ }{h}.
The Euclidean subgradient of a nonsmooth $h$ is \Esubgrad{ }{h}, the Riemannian instead \Rsubgrad{ }{h}.
% !TEX root =  ../main.tex

\section{Category theory essentials}\label{app:CT_background}

Below are fundamental definitions and examples that are instrumental in providing the necessary background on category theory to understand our work.
For a comprehensive overview of category theory see resources such as \citeSupp{mac2013categoriesSupp,perrone2024startingSupp}.

\begin{definition}[Category]\label{def:category}
    A category $\mathsf{C}$ consists of
    \begin{squishlist}
        \item A collection of objects, viz. $X$ in $\mathsf{C}$,
        \item A collection of morphisms, viz. $f: X \rightarrow Y$ in $\mathsf{C}$;
    \end{squishlist}
    such that:
    \begin{squishlist}
        \item Each morphism $f$ has assigned two objects of the category called source and target, respectively,
        \item Each object $X$ has an identity morphism $\mathrm{id}_X: X \rightarrow X$,
        \item Given $f: X\rightarrow Y $ and $g:Y\rightarrow Z$, than the composition exists, $g \circ f = h: X \rightarrow Z$.
    \end{squishlist}
    These structures satisfy the following axioms:
    \begin{squishlist}
        \item (Unitality) $\forall f: X \rightarrow Y, \; f \circ \mathrm{id}_X=f \text{ and } \mathrm{id}_Y \circ f = f$;
        \item (Associativity) Given $f$, $g$, and $h$ such that the compositions hold, then $h \circ (g \circ f) = (h \circ g) \circ f$.
    \end{squishlist}
\end{definition}

\begin{example}
    The following are some notable examples of categories:
    \begin{squishlist}
        \item Indicate with \Poset a partial order set. \Poset can be viewed as the category whose objects are the elements $p$ and morphisms are order relations $p \leq p^\prime$. Notice that there is at most one morphism between two objects;
        \item \Vect is the category whose objects are real vector spaces and morphisms are linear maps;
        \item \Prob is the category whose objects are probability measure spaces and morphisms measurable maps.
    \end{squishlist}
\end{example}

Arrows between categories are called \emph{functors}, defined as follows:

\begin{definition}[Functor]\label{def:functor}
    Consider $\mathsf{C}$ and $\mathsf{D}$ categories. 
    A functor $F: \mathsf{C} \rightarrow \mathsf{D}$ consists of the following data:
    \begin{squishlist}
        \item For each object $X$ in \Ccat, an object $F(X)$ in $\mathsf{D}$;
        \item For each object morphism $f: X \rightarrow Y$ in \Ccat, a morphism $F(f): F(X) \rightarrow F(Y)$ in $\mathsf{D}$;
    \end{squishlist}
    such that the following axioms hold:
    \begin{squishlist}
        \item (Unitality) $\forall X $ in $ \Ccat, \; F(\mathrm{id}_X) \!=\! \mathrm{id}_{F(X)}$. In other words, the identity in \Ccat is mapped into the identity in $\mathsf{D}$.
        \item (Compositionality) $\forall f \text{ and } g $ in \Ccat such that the composition is defined, then $F(g \circ f) = F(g) \circ F(f)$. In other words, the composition in \Ccat is mapped into the composition in $\mathsf{D}$.
    \end{squishlist}
\end{definition}

To ease the notation, in the sequel, we use $F^X$ and $F^f$ to denote $F(X)$ and $F(f)$, respectively.
Finally, we can have arrows between functors as well, called \emph{natural transformations}:

\begin{definition}[Natural transformation]\label{def:nat_transf}
Consider two categories \Ccat and $\mathsf{D}$, and two functors between them, namely $F: \Ccat \rightarrow \mathsf{D}$ and $G: \Ccat \rightarrow \mathsf{D}$.
A natural transformation $\alpha: F \dotarrow G$ consists of the following data:
\begin{squishlist}
    \item For each object $X $ in \Ccat, a morphism $\alpha_{X}: F^X \rightarrow G^X$ in $\mathcal{D}$ called the component of $\alpha$ at $X$;
    \item For each morphism $f: X \rightarrow X^\prime$ in \Ccat, the following diagram commutes:
    \begin{equation}
        \begin{tikzcd}[row sep=1.5cm, column sep=1.5cm]
            F^X \arrow[r, "F^f"] \arrow[d, "\alpha_X"'] & F^{X^\prime} \arrow[d, "\alpha_{X^\prime}"] \\
            G^X \arrow[r, "G^f"'] & G^{X^\prime}
        \end{tikzcd}
    \end{equation}
\end{squishlist}
\end{definition}

A natural transformation can be thought of as a consistent system of arrows between two functors, invariant with respect to maps between the images of two functors.
% !TEX root =  ../main.tex

\section{Causality and causal abstraction.}\label{app:CA}
This section provides additional definitions and examples related to SCMs and the CA framework.

\subsection{Mixing functions}
A set of structural function in a Markovian SCM can be reduced to a set of mixing functions dependent only on the exogenous variables. 

Given an SCM $\scm^n$, recall that $\myfunctional$ is a set of $n$ functional assignments which define the values $X_i=f_i(\parents_i, Z_i)$, $\forall \; i \in [n]$, with $ \parents_i \subseteq \myendogenous \setminus \{ X_i\}$. 
Denote by $\myexogenous^{\mathcal{A}_i} \subseteq \myexogenous \setminus \{ Z_i\}$ the set of exogenous variables corresponding to the ancestors of $X_i$, where $\ancestors_i \subseteq [n] \setminus \{i\}$.
According to \myfunctional, we can identify a set of mixing functions $\mymixing=\{m_1, \ldots, m_n\}$ such that the values of the endogenous random variables are equivalently expressed as $x_i=m_i\left(\{z_j\}_{j\in \ancestors_i}, z_i\right)$, $\forall \; i \in [n]$. 

Further, we can also characterize the product probability measure implied by the SCM purely in terms of the exogenous variables, viz. $\chi^\myendogenous=\prod_{i \in [n]} P\left(X_i | \myexogenous^{\ancestors_i}, Z_i\right)$. 

As an example, consider a causal relation $x_1 \rightarrow x_2$.
%\todo[from=FMZ]{If we need space we can cut on the examples}
In the linear SCM with additive noise \citeSupp{bollen1989structuralSupp,shimizu2006linearSupp} setting we have
\begin{equation}
    \begin{cases}
        x_1=z_1\,,\\
        x_2=c_{2,1} x_1 + z_2 = c_{2,1} z_1 + z_2\,.          
    \end{cases}
\end{equation}
Again, for the post-nonlinear model \citeSupp{zhang2012identifiabilitySupp}, we get
\begin{equation}
    \begin{cases}
        x_1=f_{1,1}(z_1)=m_{1,1}(z_1)\,,\\
        \begin{aligned}
            x_2&=f_{2,2}(f_{2,1}(x_1) + z_2)\\ 
            &= (f_{2,2}\circ f_{2,1} \circ f_{1,1})(z_1) + f_{2,2}(z_2)\\
            &= m_{2,1}(z_1) + m_{2,2}(z_2)\,. 
        \end{aligned}    
    \end{cases}
\end{equation}

\subsection{Interventional consistency}
A typical requirement imposed on CA maps is that they act in a consistent way with respect to interventions \citeSupp{rischel2020categorySupp}. 

\begin{definition}[Interventional consistency]\label{def:interv_consistency}
    Given an $\abst$-abstraction between $\mathsf{M}^\ell$ and $\mathsf{M}^h$ and a set $\mathcal{I}$ of hard interventions on $\mathcal{X}^h_{\mathcal{I}}\subseteq \datahigh$, the abstraction is \emph{interventionally consistent} if, for any intervention in $\mathcal{I}$ and for every set of target variable $\mathcal{Y}^h_{\mathcal{I}}\subseteq \datahigh \setminus \mathcal{X}^h_{\mathcal{I}}$, the following diagram commutes:

    \begin{center}
    \begin{tikzpicture}[shorten >=1pt, auto, node distance=1cm, thick, scale=1.0, every node/.style={scale=1.0}]
    
    \node[] (M0_0) at (0,0) {$\dom{\mathcal{X}^\ell_\mathcal{I}}$};
    \node[] (M0_1) at (4,0) {$\dom{\mathcal{Y}^\ell_\mathcal{I}}$};
    \node[] (M1_0) at (0,-1.75) {$\dom{\mathcal{X}^h_\mathcal{I}}$};
    \node[] (M1_1) at (4,-1.75) {$\dom{\mathcal{Y}^h_\mathcal{I}}$};
    
    \draw[->, draw=mypurple]  (M0_0) to node[above,font=\small]{$P(\mathcal{Y}^\ell_\mathcal{I} \vert \doint(\mathcal{X}^\ell_\mathcal{I}))$} (M0_1);
    \draw[->, draw=cyan]  (M1_0) to node[below,font=\small]{$P(\mathcal{Y}^h_\mathcal{I}\vert \doint(\mathcal{X}^h_\mathcal{I}))$} (M1_1);
    \draw[->, draw=cyan]  (M0_0) to node[left,font=\small]{$\alphamap{\mathcal{X}^h_\mathcal{I}}$} (M1_0);
    \draw[->, draw=mypurple]  (M0_1) to node[right,font=\small]{$\alphamap{\mathcal{Y}^h_\mathcal{I}}$} (M1_1);
    
    \end{tikzpicture}
    \end{center}
    or equivalently,
    \begin{equation}%\label{eq:abserr}
        \alphamap{\mathcal{Y}^h_\mathcal{I}}(P(\mathcal{Y}^\ell_\mathcal{I} \vert \doint(\mathcal{X}^\ell_\mathcal{I}))) =  P(\mathcal{Y}^h_\mathcal{I} \vert \alphamap{\mathcal{X}^h_\mathcal{I}}(\doint(\mathcal{X}^h_\mathcal{I}))),
    \end{equation}

    where $\mathcal{X}^\ell_\mathcal{I}=\amap^{-1}(\mathcal{X}^h_\mathcal{I})$ and $\mathcal{Y}^\ell_\mathcal{I}=\amap^{-1}(\mathcal{Y}^h_\mathcal{I})$.  
\end{definition} 
Essentially, commutativity suggests that we obtain equivalent intervention outcomes in two different ways: \mypurple{\emph{(i)}} either by intervening on the low-level model and then abstracting or, \cyan{\emph{(ii)}} by abstracting to the high-level model and then intervening in an equivalent fashion.

\subsection{Linear abstraction}
The class of abstractions may be restricted by an assumption of the form of the abstraction map \citeSupp{massidda2024learningcausalabstractionslinearSupp}:

\begin{definition}[Linear abstraction]\label{def:lca}
Given an $\abst$-abstraction $\abst = \langle \Rset, \amap, \alphamap{} \rangle$ from $\mathsf{M}^\ell$ to $\mathsf{M}^h$, the abstraction is linear if $\alphamap{} = \V^\top \in \reall^{h \times \ell}$.
\end{definition}

\subsection{Constructive abstraction}
A particularly well-behaved form of abstraction is a constructive abstraction. In the context of the $\tau$-abstraction framework \citeSupp{beckers2019abstractingSupp}, a constructive abstraction is an abstraction such that: \emph{(i)} the variable mapping defines a clustering of the low-level variables (\emph{constructivity}); \emph{(ii)} consistency holds for all high-level interventions (\emph{strongness}); \emph{(iii)} the value map is surjective and it implies a map between exogenous values and between interventions ($\tau$-\emph{abstraction}). In the $\alpha$-framework a few of these properties hold by construction; thus, we define a constructive abstraction as \citeSupp{schooltink2024aligningSupp}: 

\begin{definition}[Constructive abstraction]\label{def:cca}
Given an $\abst$-abstraction $\abst = \langle \Rset, \amap, \alphamap{} \rangle$ from $\mathsf{M}^\ell$ to $\mathsf{M}^h$, the abstraction is constructive if the abstraction is interventionally consistent and implies the existence of a map ${\alphamap{}}_U: \myexogenous^\ell \rightarrow \myexogenous^h$ between exogenous variables.
\end{definition}

\subsection{Measure-theoretic definition of an SCM}\label{subsec:SCM_prob}

Any SCM can be defined in terms of the probability measure spaces underlying it:

\begin{definition}[Measure-theoretic SCM]\label{def:SCM_prob}
    A (Markovian) SCM $\scm^n$ is a triple $\langle (\myexogenousvals,\, \Sigma_{\myexogenousvals}, \zeta), \, (\myendogenousvals,\, \Sigma_{\myendogenousvals}, \chi)\, , \mymixing \rangle$  where:
    \begin{squishlist}
        \item $(\myexogenousvals,\, \Sigma_{\myexogenousvals}, \zeta)$ is a probability space associated with exogenous variables. Specifically, it consists of the product probability measure $\zeta=\zeta_1 \times \ldots \times \zeta_n$ on the product measurable space $(\myexogenousvals,\, \Sigma_{\myexogenousvals})$ where $\myexogenousvals = \myexogenousvals_1 \times \ldots \times \myexogenousvals_n$ is a product set and $ \Sigma_{\myexogenousvals} =  \Sigma_{\myexogenousvals_1} \otimes \ldots \otimes \Sigma_{\myexogenousvals_n}$ is a product $\sigma$-algebra.
        The probability measure is such that, for each $\mathcal{W}_1 \in \Sigma_{\myexogenousvals_1}, \ldots, \, \mathcal{W}_n \in \Sigma_{\myexogenousvals_n}$, we have 
        \begin{equation}
            \zeta_1 \times \ldots \times \zeta_n (\mathcal{W}_1 \times \ldots \times \mathcal{W}_n)=\zeta_1(\mathcal{W}_1) \times \ldots \times \zeta_n(\mathcal{W}_n)\,;
        \end{equation}
        \item $(\myendogenousvals,\, \Sigma_{\myendogenousvals}, \chi)$ is a probability space associated with endogenous variables consisting of a joint probability measure $\chi$ on the product measurable space $(\myendogenousvals,\, \Sigma_{\myendogenousvals})=(\myendogenousvals_1 \times \ldots \times \myendogenousvals_n,\, \Sigma_{\myendogenousvals_1} \otimes \ldots \otimes \Sigma_{\myendogenousvals_n})$;
        \item \mymixing is a set of $n$ mixing measurable maps $\varphi^{m_i}$ (cf. \Cref{def:SCM}) such that the joint probability measure $\chi$ factorizes as 
        \begin{equation}
            \chi = \bigtimes_{i=1}^n \varphi^{m_i}_{\#}\left(\mu_i \left( \myexogenousvals_i \times \myexogenousvals^{\ancestors_i} \right) \right)\,;       
        \end{equation}
        where $\myexogenousvals^{\ancestors_i}=\bigtimes_{j \in \ancestors_i} \myexogenousvals_j$, and, denoting by $\Sigma_{\myexogenousvals^{\ancestors_i}}=\bigotimes_{j \in \ancestors_i} \Sigma_{\myexogenousvals_j}$, $\mu_i$ is a probability measure on the product measurable space $\left( \myexogenousvals_i \times \myexogenousvals^{\ancestors_i},\, \Sigma_{\myexogenousvals_i} \otimes \Sigma_{\myexogenousvals^{\ancestors_i}} \right)$.
    \end{squishlist}
\end{definition}
% !TEX root =  ../main.tex

\section{Category theory formalization.}\label{app:CT}
This section extends the category-theoretic formalization introduced in the main paper to intervened models and abstraction.

Recall the category-theoretic definition from the main paper:
\begin{definition}[Category-theoretic SCM]\label{def:SCM_ct_app}
    An SCM is a functor $\scm^n: \Index \rightarrow \Prob$, mapping the source node of \Index to the probability space associated with the exogenous variables $(\myexogenousvals,\, \Sigma_{\myexogenousvals}, \zeta)$, the sink node of \Index to the probability space associated with the endogenous variables $(\myendogenousvals,\, \Sigma_{\myendogenousvals}, \chi)$, and the only edge of \Index to the measurable map induced by the set \myfunctional of functional assignments.
\end{definition}

\cref{fig:functor} offers a depiction of an SCM as a functor.

\begin{figure}
    \centering
    \begin{tikzpicture}[]

    \draw[dashed] (-0.5, 2) rectangle (.5, -0.5);
    \draw[dashed] (1.8, 2) rectangle (4, -0.5);

    \node at (0, 2.5) {\Index};
    \node at (3, 2.5) {\Prob};
    
    \node[circle, draw, fill,inner sep=1pt] (A) at (0, 1.5) {};
    \node[circle, draw, fill,inner sep=1pt] (B) at (0, 0) {};
    \node (C) at (3, 1.5) {$(\myexogenousvals,\, \Sigma_{\myexogenousvals}, \zeta)$};
    \node (D) at (3, 0) {$(\myendogenousvals,\, \Sigma_{\myendogenousvals}, \chi)$};

    \coordinate (A1shift) at ([yshift=-5pt]A);
    \draw[->,shorten >=2pt] (A1shift) -- (B);

    \coordinate (A2shift) at ([xshift=5pt]A);
    \draw[->, mypurple] (A2shift) -- (C);

    \coordinate (B1shift) at ([xshift=5pt]B);
    \draw[->, mypurple] (B1shift) -- (D);
    
    \draw[->] (C) -- node[right] {\mymixing} (D);
    
    \end{tikzpicture}
    \caption{An SCM is a functor (purple arrows) from \Index (right) to \Prob (left).}
    \label{fig:functor}
\end{figure}

In the same vein, we can have a functorial representation for intervened SCMs as well. However, instead of representing directly the post-interventional model $\scm^n_\iota$ as in Def. \ref{def:SCM_ct_app}, we will adopt a representation that is closer to the intervention operator itself. First, notice that, whenever the domains of the variables of an SCM are continuous, we can represent an intervention as a measurable map by relying on the truncation formula \citeSupp{pearl2009causalitySupp}:
\begin{lemma}
    Given a continuous Markovian  SCM $\scm^n = \langle (\myexogenousvals,\, \Sigma_{\myexogenousvals}, \zeta), \, (\myendogenousvals,\, \Sigma_{\myendogenousvals}, \chi)\, , \mymixing \rangle$ and an intervention $\iota$ on $\scm^n$, there exists a measurable map $\phi_\iota$ from the probability space of endogenous variables of the pre-interventional SCM $(\myendogenousvals,\, \Sigma_{\myendogenousvals}, \chi)$ to the probability space of endogenous variables of the post-interventional SCM $(\myendogenousvals_\iota,\, \Sigma_{\myendogenousvals_\iota}, \chi_\iota)$.
\end{lemma}
\emph{Proof.} Given a Markovian SCM $\scm^n$, the probability measure $\chi$ over the measure space of endogenous variables $(\myendogenousvals,\, \Sigma_{\myendogenousvals}, \chi)$ can be expressed by through the factorization over the endogenous variables $\chi=\prod_{i \in [n]} P\left(X_i |  \parents_i,Z_i\right)$.  Given intervention $\iota=\operatorname{do}(\myendogenous^{\iota} = \mathbf{x}^{\iota})$ on $\scm^n$, the new post-interventional measure $\chi^\iota$ can be computed through the truncation formula \citeSupp{pearl2009causalitySupp}:
\begin{equation}\label{eq:truncation}
\chi^{\iota}=\begin{cases}
\prod_{i\in[n],X_{i}\notin \myendogenous^{\iota}}P(X_i|\parents_i,Z_i) & \textrm{if }\myendogenous^{\iota} = \mathbf{x}^{\iota}\\
0 & \textrm{if }\myendogenous^{\iota} \neq \mathbf{x}^{\iota}
\end{cases}
\end{equation}
We can now define a measurable map $\phi^\iota$ connecting $(\myendogenousvals,\, \Sigma_{\myendogenousvals}, \chi)$ and $(\myendogenousvals,\, \Sigma_{\myendogenousvals}, \chi^\iota)$ such that $\phi^\iota_{\#}(\chi)=\chi^\iota$. Specifically, for each $X_i \in \myendogenous^{\iota}$, $\phi(X_i) = x_i^\iota$, thus guaranteeing the distribution on the second line of \cref{eq:truncation}; for each $X_i \notin \myendogenous^{\iota}$, we solve a measure transport problem \citeSupp{Marzouk_2016Supp} from $\chi(X_i)$ to $\chi^\iota(X_i)$ which, in the continuous case, guarantees a transport map over the domains that satisfies the distribution on the first line of \cref{eq:truncation}.
$\blacksquare$

We can then encode an intervened model as follows: 

\begin{definition}[Category-theoretic post-interventional SCM]\label{def:SCM_ct_intervention}
    A post-interventional SCM is a functor $\scm^n_\iota: \Index \rightarrow \Prob$, where the functor maps the source node of \Index to the probability space associated with the endogenous variables of the pre-interventional SCM $(\myendogenousvals,\, \Sigma_{\myendogenousvals}, \chi)$, the sink node of \Index to the probability space associated with the endogenous variables of the post-interventional SCM $(\myendogenousvals_\iota,\, \Sigma_{\myendogenousvals_\iota}, \chi_\iota)$, and the only edge of \Index to the function $\phi_\iota$ encoding the intervention $\iota$.
\end{definition} 

This construction gives rise to the structure in Fig. \ref{fig:functor3} and an immediate category-theory expression of abstraction equivalent to Def.\ref{def:abstraction}:
\begin{lemma}\label{lem:interventional_mixing}
    An interventionally consistent abstraction is a singular natural transformation $\abst$, that is, a morphism $\alphamap{\myendogenousvals}$ in \Prob, that, for all intervention in $\mathcal{I}$ guarantees the commutativity of the diagrams constructed from Fig. \ref{fig:functor2}. 
\end{lemma}

\emph{Proof.} 
Recall the definition of interventional consistency in \Cref{def:interv_consistency}:
\begin{equation}%\label{eq:abserr}
        \alphamap{\mathcal{Y}^h_\mathcal{I}}(P(\mathcal{Y}^\ell_\mathcal{I} \vert \doint(\mathcal{X}^\ell_\mathcal{I}))) =  P(\mathcal{Y}^h_\mathcal{I} \vert \alphamap{\mathcal{X}^h_\mathcal{I}}(\doint(\mathcal{X}^h_\mathcal{I}))).
\end{equation}
Let us relate this definition to our categorical notation. First, $\alphamap{\mathcal{Y}^h_\mathcal{I}}$ and $\alphamap{\mathcal{X}^h_\mathcal{I}}$ are components of the abstraction map $\alphamap{}$; in the categorical notation, this map correspond to $\alphamap{\myendogenousvals}$. 
The probability distribution $P(\mathcal{Y}^\ell_\mathcal{I} \vert \doint(\mathcal{X}^\ell_\mathcal{I}))$ is a distribution in the low-level model; with no loss of generality, assuming $\mathcal{Y}^\ell_\mathcal{I}$ to encompass all the non-intervened variables, this distribution correspond to the measure $\chi_\iota^\ell$; furthermore, the interventional measure $\chi_\iota^\ell$ can be obtained through the pushforward of the observational measure $\chi^\ell$ via the interventional mixing functions $\mymixing_\iota^\ell$, as by \cref{lem:interventional_mixing}.
Finally, the probability distribution $P(\mathcal{Y}^h_\mathcal{I} \vert \alphamap{\mathcal{X}^h_\mathcal{I}}(\doint(\mathcal{X}^h_\mathcal{I}))$ is a distribution in the high-level model; again, with no loss of generality, assuming $\mathcal{Y}^h_\mathcal{I}$ to encompass all the non-intervened variables, this distribution correspond to the measure $\chi_\kappa^h$, where $\kappa$ is the abstraction of the terms in $\iota$. Also, as before, the interventional measure $\chi_\kappa^h$ can be obtained through the pushforward of the observational measure $\chi^h$ via the interventional mixing functions $\mymixing_\kappa^h$, thanks to \cref{lem:interventional_mixing}.
We then obtain a rewriting of abstraction as:
\begin{equation}%\label{eq:abserr}
        \alphamap{\myendogenousvals} \circ \mymixing_\iota^\ell =  \mymixing_\kappa^h \circ \alphamap{\myendogenousvals}.
\end{equation}
corresponding to the commutativity of the right diagram in \cref{fig:functor3}, for all interventions. $\blacksquare$

\begin{figure*}
    \centering
    \begin{tikzpicture}[]

    \draw[dashed] (-0.5, 2.8) rectangle (.5, -0.5);
    \draw[dashed] (1.5, 2.8) rectangle (10.5, -0.5);

    \node at (0, 3.2) {\Index};
    \node at (6, 3.2) {\Prob};
    
    \node[circle, draw, fill,inner sep=1pt] (A) at (0, 2) {};
    \node[circle, draw, fill,inner sep=1pt] (B) at (0, 0) {};
    
    \node (C) at (3, 2) {$(\myexogenousvals^\ell,\, \Sigma_{\myexogenousvals^\ell}, \zeta^\ell)$};
    \node (D) at (6, 2) {$(\myendogenousvals^\ell,\, \Sigma_{\myendogenousvals^\ell}, \chi^\ell)$};
    \node (E) at (9, 2) {$(\myendogenousvals^\ell_\iota,\, \Sigma_{\myendogenousvals^\ell_\iota}, \chi^\ell_\iota)$};
    \draw[->] (C) -- node[above] {$\mymixing^\ell$} (D);
    \draw[->] (D) -- node[above] {$\mymixing^\ell_\iota$} (E);

    \node (F) at (3, 0) {$(\myexogenousvals^h,\, \Sigma_{\myexogenousvals^h}, \zeta^h)$};
    \node (G) at (6, 0) {$(\myendogenousvals^h,\, \Sigma_{\myendogenousvals^h}, \chi^h)$};
    \node (H) at (9, 0) {$(\myendogenousvals^h_\kappa,\, \Sigma_{\myendogenousvals^h_\kappa}, \chi^\ell_\kappa)$};
    \draw[->] (F) -- node[above] {$\mymixing^h$} (G);
    \draw[->] (G) -- node[above] {$\mymixing^h_\kappa$} (H);

    \coordinate (A1shift) at ([yshift=-5pt]A);
    \draw[->,shorten >=2pt] (A1shift) -- (B);

    \draw[blue, rounded corners=10pt]($(C.west)+(0,0.6)$) rectangle ($(D.east)+(0.1,-0.3)$);
    \draw[->, blue] (0.2,1) -- (4,1.7);

    \draw[cyan, rounded corners=10pt]($(D.west)+(0,0.6)$) rectangle ($(E.east)+(0.1,-0.3)$);
    \draw[->, cyan] (0.2,1) -- (8,1.7);

    \draw[red, rounded corners=10pt]($(F.west)+(0,0.6)$) rectangle ($(G.east)+(0.1,-0.3)$);
    \draw[->, red] (0.2,1) -- (4,0.6);

    \draw[orange, rounded corners=10pt]($(G.west)+(0,0.6)$) rectangle ($(H.east)+(0.1,-0.3)$);
    \draw[->, orange] (0.2,1) -- (8,0.6);

    \draw[->, dashed] (C) -- node[right] {$\alphamap{\myexogenousvals^{h}}$} (F);
    \draw[->, dashed] (D) -- node[right] {$\alphamap{\myendogenousvals^{h}}$} (G);
    \draw[->, dashed] (E) -- node[right] {$\alphamap{\myendogenousvals^h}$} (H);
    
    \end{tikzpicture}
    \caption{Representation of {\color{blue} $\scm^\ell$} (blue), {\color{cyan} $\scm^\ell_\iota$} (cyan), {\color{red} $\scm^h$} (red), {\color{orange} $\scm^h_\iota$} (orange) as functors. An abstraction is just a natural transformation, that is, a set of commuting arrows in \Prob (dashed black). Notice two commuting diagrams in \Prob: the first observational one rooted on the exogenous variables ($\mymixing^h \circ \alphamap{\myexogenousvals^{h}} = \alphamap{\myendogenousvals^{h}} \circ \mymixing^\ell$), the second interventional one connecting observational and interventional model ($\mymixing^h_\kappa \circ \alphamap{\myendogenousvals^{h}} = \alphamap{\myendogenousvals^{h}} \circ \mymixing^\ell_\iota$).}
    \label{fig:functor3}
\end{figure*}


% !TEX root =  ../main.tex

\section{Stiefel manifold}\label{app:stiefel}
We now provide a short review of the Stiefel manifold, referring the interested reader to \citeSupp{absil2008optimizationSupp,boumal2023introductionSupp} for a comprehensive discussion.

Given $\ell, \, h \in \nat, \, h < \ell$, the Stiefel manifold is the set of $\ell \times h$ matrices with orthonormal columns, mathematically
\begin{equation}\label{eq:stiefel_manifold}
    \stiefel{\ell}{h} \coloneqq \{ \V \in \rmatdim \, \mid \, \V^\top\V = \identity_h \}\,.
\end{equation}
Consider the function $g: \rmatdim \rightarrow \sym{h}$, $g(\V) \coloneqq \V^\top \V - \identity^h$.
It is well-known that $g$ is a generating function for \stiefel{\ell}{h}, thus making it an embedded submanifold of $\rmatdim$, with dimension $\mathrm{dim}\, \rmatdim - \mathrm{dim}\, \sym{h} = \ell h - h(h+1)/2$.
Given a point of the manifold $\V$, the tangent space to \stiefel{\ell}{h} can be defined implicitly as the kernel of the differential of $g$ at $\V$,
\begin{equation}\label{eq:Stiefel_t_space}
    \tangentspace{\V}{\stiefel{\ell}{h}}\coloneqq \{ \G \in \rmatdim \, | \, \V^\top \G + \G^\top \V = 0 \}\,.
\end{equation}
We consider the Riemannian metric as the restriction of the Eucliden product between two matrices in $\rmatdim$ to $\stiefel{\ell}{h}$.
Accordingly, given $\mathbf{A}, \, \mathbf{B} \in \tangentspace{\V}{\stiefel{\ell}{h}}$, we have $\Eprod{\mathbf{A}}{\mathbf{B}}{\V}= \Tr \mathbf{A}^\top \mathbf{B}$.
The tangent space linearizes the manifold around $\V$, then, we can move away from $\V$ along the directions in \tangentspace{\V}{\stiefel{\ell}{h}}. 
However, to make such a movement smooth along the manifold, we employ the \emph{retraction map} $\Retr{}{\V}: \tangentspace{\V}{\stiefel{\ell}{h}} \rightarrow \stiefel{\ell}{h}$.
The retraction has to satisfy the following conditions
\begin{equation}
        \text{\emph{(i)}} \; \Retr{}{\V}{\zeros_{\ell \times h}}=\V, \quad \text{and} \quad \text{\emph{(ii)}} \; \lim_{\G \rightarrow \zeros_{\ell \times h}} \frac{\frob{\Retr{}{\V}{\G} - (\V + \G)}}{\frob{\G}} = 0\,.
\end{equation}

Among the canonical retractions, we have 
\begin{equation}\label{eq:stiefel_retractions}
    \begin{aligned}
        \Retr{\mathrm{QR}}{\V}{\G} &= \mathrm{qf}(\V + \G)\,, \quad \text{[QR retraction]}\\
        \Retr{\mathrm{Polar}}{\V}{\G} &= (\V + \G)(\identity^h - \V^\top \V)^{\frac{1}{2}}\,, \quad \text{[Polar retraction]}\\
        \Retr{\mathrm{Caley}}{\V}{\G} &= (\identity^\ell -\frac{1}{2}\mathbf{W}(\G))^{-1}(\identity^\ell +\frac{1}{2}\mathbf{W}(\G))\V \,; \quad \text{[Caley retraction]}
    \end{aligned}
\end{equation}
where $\mathrm{qf}$ indicates the $\mathbf{Q}$ factor of the $\mathrm{QR}$ decomposition, and $\mathbf{W}(\G)=(\identity^\ell - \frac{1}{2}\V\V^\top)\G \V^\top- \V\G^\top (\identity^\ell - \frac{1}{2}\V\V^\top)$. 

Finally, the normal space to the manifold at $\V$ has the following explicit form
\begin{equation}\label{eq:explicit_normal_space}
    \normalspace{\V}{\stiefel{\ell}{h}}\coloneqq \{ \V \mathbf{S} \, \mid \, \mathbf{S} \in \sym{h} \}\,.
\end{equation}

Starting from \Cref{eq:explicit_normal_space}, the orthogonal projection to \tangentspace{\V}{\stiefel{\ell}{h}}, namely \projectiontangentspace{\V}{}, has to be such that $\G - \projectiontangentspace{\V}{\G}$ lies onto \normalspace{\V}{\stiefel{\ell}{h}}, i.e.,
\begin{equation}\label{eq:stiefel_proj_diff}
    \G - \projectiontangentspace{\V}{\G} = \V \mathbf{S}\,.
\end{equation}

Plugging \Cref{eq:stiefel_proj_diff} into \Cref{eq:Stiefel_t_space}, it can be derived that
\begin{equation}\label{eq:stiefel_projection}
    \projectiontangentspace{\V}{\G} = \left(\identity^\ell - \V \V^{\top} \right) \G + \V \frac{ \left(\V^{\top} \G - \G^\top \V \right)}{2} \,.
\end{equation}

Finally, for $\stiefel{\ell}{h}$ (and in general for Riemannian submanifolds) the Riemannian gradient of $f$ at $\V$ is the orthogonal projection of \Egrad{\V}{f} to \tangentspace{\V}{\stiefel{\ell}{h}}.
Mathematically, starting from \Cref{eq:stiefel_projection}, we have 
\begin{equation}\label{eq:stiefel_grad}
    \Rgrad{\V}{f}=\projectiontangentspace{\V}{\Egrad{\V}{f}} \,.    
\end{equation}
\documentclass[lettersize,journal]{IEEEtran}

\usepackage{amsmath,amsfonts}
\usepackage{algorithmic}
% \usepackage{algorithm}
\usepackage{array}
\usepackage[caption=false,font=normalsize,labelfont=sf,textfont=sf]{subfig}
\usepackage{textcomp}
\usepackage{stfloats}
\usepackage{url}
\usepackage{verbatim}
\usepackage{graphicx}
\usepackage{cite}
\usepackage{xspace}
\usepackage[svgnames]{xcolor}
\usepackage{multirow}
\usepackage{booktabs}
\usepackage{cleveref}
\usepackage{tcolorbox}
\usepackage{enumitem}
\usepackage[linesnumbered,ruled,vlined]{algorithm2e} % FIXME

\newcommand\mycommfont[1]{\footnotesize\ttfamily\textcolor{darkblue}{#1}}

\usepackage{listings}
\lstset{
  basicstyle=\ttfamily\scriptsize,
  numbers=left,                   % where to put the line-numbers
  numberstyle=\tiny\color{black},  % the style that is used for the line-numbers
  numbersep=3pt,                  % how far the line-numbers are from the code
  backgroundcolor=\color{white},      % choose the background color. You must add \usepackage{color}
  showstringspaces=false,         % underline spaces within strings
  frame=single,                   % adds a frame around the code
  rulecolor=\color{white},        % if not set, the frame-color may be changed on line-breaks within not-black text (e.g. commens (green here))
  breaklines=true,                % sets automatic line breaking
  commentstyle=\color{gray}\upshape
  breakatwhitespace=false,        % sets if automatic breaks should only happen at whitespace,
  xleftmargin=1.5em
}

\lstdefinelanguage{diff}{
  morecomment=[f][\color{dimgray}]{@@},
  morecomment=[f][\color{cadmiumgreen}]{+\ },
  morecomment=[f][\color{brickred}]{-\ },
}

\def\name{\textsc{Fonte}\xspace}
\def\product{SAP HANA\xspace}

\newcommand{\fixme}[1]{\textcolor{red}{#1}}
\newcommand{\todo}[1]{\textcolor{red}{#1}}
\newcommand{\refi}[1]{\textcolor{blue}{#1}}
\newcommand{\sug}[1]{{\textcolor{dkgreen}#1}}

\definecolor{dimgray}{rgb}{0.41, 0.41, 0.41}
\definecolor{brickred}{rgb}{0.8, 0.25, 0.33}
\definecolor{cadmiumgreen}{rgb}{0.0, 0.42, 0.24}
\definecolor{dkgreen}{rgb}{0,0.6,0}
\definecolor{mauve}{rgb}{0.58,0,0.82}
\definecolor{gray}{rgb}{0.4,0.4,0.4}
\definecolor{darkblue}{rgb}{0.0,0.0,0.6}
\definecolor{lightblue}{rgb}{0.0,0.0,0.9}
\definecolor{cyan}{rgb}{0.0,0.6,0.6}
\definecolor{darkred}{rgb}{0.6,0.0,0.0}

\begin{document}

\title{Identifying Bug Inducing Commits by Combining Fault Localisation and Code Change Histories}

\author{Gabin~An,
Jinsu~Choi,
Jingun~Hong,
Naryeong~Kim,
Shin~Yoo
\IEEEcompsocitemizethanks{
\IEEEcompsocthanksitem Gabin An, Jinsu Choi, Naryeong Kim, and Shin Yoo are with the School of Computing, KAIST, Daehak-ro 291, Daejeon, South Korea.\protect\\
E-mail: \{gabin.an, jinsuchoi, kimnal1234, shin.yoo\}@kaist.ac.kr.
\IEEEcompsocthanksitem Jingun Hong is with SAP Labs Korea, Seoul, South Korea.\protect\\
E-mail: jingun.hong@sap.com.}%
}
        % <-this % stops a space
% \thanks{}% <-this % stops a space
% \thanks{Manuscript received April 19, 2021; revised August 16, 2021.}}

% The paper headers
% \markboth{IEEE TRANSACTIONS ON SOFTWARE ENGINEERING, under review}%
\markboth{}%
{An \MakeLowercase{\textit{et al.}}: Identifying Bug Inducing Commits by Combining Fault Localisation and Code Change Histories}

% \IEEEpubid{0000--0000/00\$00.00~\copyright~2021 IEEE}
% Remember, if you use this you must call \IEEEpubidadjcol in the second
% column for its text to clear the IEEEpubid mark.

\maketitle

\begin{abstract}
  A Bug Inducing Commit (BIC) is a code change that introduces a bug into the codebase. Although the abnormal or unexpected behavior caused by the bug may not manifest immediately, it will eventually lead to program failures further down the line. When such a program failure is observed, identifying the relevant BIC can aid in the bug resolution process, because knowing the original intent and context behind the code change, as well as having a link to the author of that change, can facilitate bug triaging and debugging. However, existing BIC identification techniques have limitations. Bisection can be computationally expensive because it requires executing failing tests against previous versions of the codebase. Other techniques rely on the availability of specific post hoc artifacts, such as bug reports or bug fixes. In this paper, we propose a technique called \name that aims to identify the BIC with a core concept that a commit is more likely to be a BIC if it has more recently modified code elements that are highly suspicious of containing the bug. To realise this idea, \name leverages two fundamental relationships in software: the failure-to-code relationship, which can be quantified through fault localisation techniques, and the code-to-commit relationship, which can be obtained from version control systems. Our empirical evaluation using 206 real-world BICs from open-source Java projects shows that \name significantly outperforms state-of-the-art BIC identification techniques, achieving up to 45.8\% higher MRR. We also report that the ranking scores produced by \name can be used to perform weighted bisection. Finally, we apply \name to a large-scale industry project with over 10M lines of code, and show that it can rank the actual BIC within the top five commits for 87\% of the studied real batch-testing failures, and save the BIC inspection cost by 32\% on average.

\end{abstract}

\begin{IEEEkeywords}
Bug Inducing Commit, Commit Level Fault Localisation, Fault Localisation, Code Change History, Bisection, Batch Testing
\end{IEEEkeywords}


\section{Introduction}
\IEEEPARstart{I}n modern software development workflows based on Continuous Integration/Continuous Deployment (CI/CD), numerous developers simultaneously participate in the development of a single project, and multiple code changes (or commits) are continuously integrated into a shared repository. In such an environment, when an abnormal program behavior (i.e., program failure) is observed during the testing process or when a field failure occurs after release, the QA or development teams analyse and categorise the issue, assign it to the most suitable developer, who then performs debugging activities (this process is referred to as the bug resolution process). During the software development process, the commits that contain buggy source code, which eventually leads to program failures, are known as Bug Inducing Commits (BICs)~\cite{liwerski2005}. For a particular program failure, knowing which commit in the change history is more likely to be a BIC can provide significant advantages in effectively performing the bug resolution process. Firstly, it can streamline the bug assignment phase by enabling the effective assignment of a newly discovered bug to the right team or developers. This is facilitated by the connection between commits and their respective authors, coupled with the fact that 78\% of bugs are ultimately fixed by the developers who originally introduced them~\cite{Wen2016}. Secondly, it can assist in both automated and manual debugging activities. Prior work has shown that simply reverting BICs may suffice for bug fixes~\cite{Wu2017, Wen2020}, and BIC information can be utilized to improve the accuracy of Fault Localisation (FL) techniques~\cite{Wen2021}. Furthermore, the knowledge of BICs has been demonstrated to aid developers in manual debugging efforts~\cite{Wen2019, Wen2021}. Finally, it can reduce testing costs, especially in batch testing failure scenarios where tests are executed against a cumulative batch of changes to reduce overall testing costs, but some tests fail~\cite{Beheshtian2022}. In such cases, knowing the relative suspiciousness of commits in the batch can expedite the identification of the specific commit that caused the failure, potentially reducing the number of test executions required compared to the standard bisection (we show this through our experiment in the later of this paper). 

Recognising the usefulness of identifying BICs for a given failure, multiple BIC identification techniques have been proposed, which can be broadly categorised into two distinct groups. The first group employs a conventional approach known as bisection~\cite{gitbisect}, which conducts a binary search over the commit history, systematically evaluating each past program snapshot to identify whether it manifests the buggy behavior or not. The evaluation process can be performed either manually or through an automated execution of the test cases that reveal the bug. However, even with automation, the bisection process can incur significant computational overhead if building and testing a specific program version is resource-intensive~\cite{Wen2021}.
%To mitigate this issue, researchers have also focused on improving the efficiency of the search algorithm itself to reduce the overall costs. For instance, Beheshtian et al. proposed a search algorithm called BatchStop4, which is theoretically more efficient than bisection in reducing the number of iterations required to find the BIC in batch testing scenarios~\cite{Beheshtian2022}.
The second group consists of Information Retrieval (IR)-based BIC identification techniques~\cite{Wen2016, Bhagwan2018}, also known as changeset/commit-level fault localisation. These approaches reformulate BIC identification as an IR problem. While various information about the failure in a textual format, such as a bug report, is treated as a query, the commits are considered as documents. The approaches then identify the BIC by directly assessing the lexical or semantic similarity between the bug report and the commits. Although IR-based approaches do not incur the computational costs associated with their dynamic counterparts (e.g., building and testing), they have limitations. These techniques can only leverage textual information, making it challenging to capture complex failure behaviors. Additionally, their effectiveness heavily relies on the quality and completeness of the bug reports. Other than these two groups, there is a family of BIC identification techniques represented by the SZZ algorithm~\cite{liwerski2005} and its variants, but they are not applicable during the debugging time as they require a Bug Fixing Commit (BFC) as input.


\begin{figure}[t]
  \centerline{\includegraphics[width=\linewidth]{figures/Fonte-compare.pdf}}
  \caption{A comparison of BIC identification methods, highlighting the distinctions between \name, Bisection and IR-based Techniques, and their relationships with failure information, code elements, and commits.}
  \label{fig:fonte-conceptual-comparison}
\end{figure}

In this paper, we introduce \name\footnote{\name is an Italian word meaning ``source'' or ``origin''.}, a novel unsupervised approach for identifying BICs that is efficient, flexible, and available during the debugging phase. As shown in \Cref{fig:fonte-conceptual-comparison}, unlike bisection and the IR-based BIC identification methods that attempt to directly assess the relevance between the commit and the failure (or bug report), \name first quantifies the relevance of code elements to the failure, computed by a FL technique~\cite{Wong2016}, and maps these code-level scores to the commit level using the change history of the code elements. This process extends existing FL techniques beyond their traditional spatial domain (i.e., the location within the code) to a temporal one (i.e., the commit in the history of the codebase).
Compared to the bisection approach, \name requires examining only the present buggy code version where the bug manifests, making it more efficient. Additionally, \name has the ability to incorporate a wide array of failure-specific information, from test coverage to bug reports, since it is compatible with any FL technique that yields quantitative suspiciousness scores, such as Spectrum-based Fault Localisation (SBFL) or Information-Retrieval-based Fault Localisation (IRFL). Consequently, we expect that \name can naturally benefit from future advancements in FL techniques, as its design allows for seamless integration of novel FL approaches.

% We evaluate \name using a benchmark of 206 real-world Java bugs: 67 from an existing BIC dataset, and 139 that are manually curated by us. The results demonstrate that the ranking performance of \name is significantly higher than the state-of-the-art IR-based identification technqiques, achieving up to \fixme{X}\% higher MRR. Furthermore, we also propose a weighted bisection algorithm that leverages the commit scores produced by \name during the search, and show that weighted bisection can reduce the number of search iterations for \fixme{98\%} of the cases compared to the standard bisection. Since \name does not require any manual human effort, it can be easily incorporated into CI pipelines to provide developers with candidate BICs when reporting test failures. This is also demonstrated by our industry application of \name to the past CI data of \product.


We evaluate \name using a benchmark of 206 real-world Java bugs from Defects4J~\cite{Just2014} and their ground-truth BICs, comprising 67 from an existing BIC dataset~\cite{Wen2019}, and 139 that are manually curated by us. The results demonstrate that the ranking performance of \name, when combined with the Ochiai SBFL technique, significantly outperforms state-of-the-art IR-based BIC identification techniques, achieving up to 45.8\% higher Mean Reciprocal Rank (MRR). Furthermore, we propose a weighted bisection algorithm that leverages the commit scores produced by \name during the search process. Our findings show that weighted bisection can reduce the number of search iterations for 98\% of the cases compared to the standard bisection, respectively. Finally, we also apply \name to the historical CI data of \product, showcasing its practical applicability in an industry setting.

This paper presents an extended version of our previously published work~\cite{An2023}. The main extensions and contributions over the original work are summarised as follows:

\begin{itemize}
  \item \textbf{Evaluating Various Code History Tracking Tools}: While \name is compatible with any method-level code history tracking tool, our previous work only used the \texttt{git log} command to obtain the code-commit relationship. This extended study makes use of the recently proposed code change tracking tools, CodeShovel~\cite{Grund2021} and CodeTracker~\cite{Jodavi2022}, in addition to \texttt{git log} and reports their effectiveness regarding the search space reduction and recall, and also the efficiency of collecting code history.
  \item \textbf{Evaluating Diverse Failure-Code Relationships}: The previous work uses only SBFL to measure the failure-code relationships in the experiment. This extended study incorporates not only SBFL but also IRFL to establish the relationships between a failure and code elements when applying \name, and compare \name's performance based on the underlying FL technique. Furthermore, the inclusion of IRFL enables a fairer comparison between our methodology and other IR-based approaches for BIC identification.
  \item \textbf{More Realistic Assessment of FL Accuracy's Effect on \name's Performance}: Previously, to study the impact of FL accuracy on \name's effectiveness, SBFL's performance was artificially weakened by removing some tests from the test suite. In this extended work, since we incorporate both IRFL and SBFL with \name, we directly evaluate \name's performance in relation to the inherent accuracy of the underlying FL techniques. This approach provides a more authentic assessment without the need for artificial manipulations.
  \item \textbf{Richer Evaluation Dataset}: We have expanded the BIC dataset from the initial 130 bugs in our previous paper~\cite{An2023} to 206 bugs for a more comprehensive and diverse evaluation. This extended dataset contains bugs from all 16 Java projects that use Git as their version control system in the Defects4J 2.0 benchmark~\cite{Just2014}, whilst the previous dataset contained bugs from 11 projects. The reproducing package of \name with the new dataset is publicly available at our GitHub repository\footnote{\url{https://github.com/coinse/fonte/blob/extension}}.
  % \item \textbf{Stronger Comparison Baselines}: \fixme{HMCBL~\cite{Du2023}} ?
\end{itemize}

% The contributions of this paper are summarised as follows:

% \begin{itemize}
% \item \refi{We present \name, a novel and flexible technique for identifying a BIC by converting code-level suspiciousness scores into commit-level scores using code change histories. \name builds upon our previous work~\cite{An2021} by actually quantifying the suspiciousness of commits instead of simply reducing the BIC search space.}

% \item We evaluate \name with \fixme{206} real-world bugs and show that \name can
% accurately rank BIC candidates: it achieves an MRR of \fixme{0.528}, which outperforms a state-of-the-art IR-based BIC identification
% technique by \fixme{39\%}.%\fixme{340\% or 139\%}.
% % 340: Bug2Commit on C
% % 139: Bug2Commit on C_BIC

% \item We introduce weighted bisection that uses the scores assigned to
% candidate BICs by \name. Weighted bisection can reduce the required number of
% iterations for about \fixme{98\%} of studied bugs when compared to the standard
% bisection algorithm applied to the entire commit history.

% \item We apply \name to the batch testing scenario of large-scale industry software. It achieves \fixme{547\%} of MRR compared to the random baseline and can reduce the bisection iterations in 
% \fixme{78\%} of cases. % or by 32\% on average.

% \item We release a new BIC benchmark dataset for \fixme{206} Defects4J bugs. \name is publicly available at \url{https://github.com/coinse/fonte}, along with artefacts of the empirical
% evaluation in this paper.
% \end{itemize}

The remainder of the paper is structured as follows. Section~\ref{sec:background} explains the research context of this paper and defines the basic notations. Section~\ref{sec:methodology} and \ref{sec:weighted_bisection} propose \name and the novel weighted bisection method, respectively. Section~\ref{sec:eval_setup} describes the empirical settings for \name along with the research questions, and Section~\ref{sec:results} presents the results. Section~\ref{sec:industry} shows the application results of \name to the batch testing scenario in industry software. Section~\ref{sec:threats} addresses the threats to validity, and Section~\ref{sec:related_work} covers the related work of \name. Finally, Section~\ref{sec:conclusion} concludes.


\section{Background}
\label{sec:background}

This section provides the background of this paper.

\subsection{Research Context}
The debugging process is typically initiated by observing a failure that reveals the presence of a bug in the software. Prior research indicates that a single failing test case is often the most commonly available information when debugging begins~\cite{Kochhar2016}. Even when users report a field failure, the debugging activities typically commence with reproducing the failure~\cite{Artzi, Jin2012,Zimmermann2010}. This is because failure-triggering test cases are essential for confirming whether the bug has been successfully fixed or not. Once a program failure is observed and reproduced, identifying the BIC responsible for the failure can contribute to a more efficient bug triage process~\cite{Murali2021} and aid developers in better understanding the context of the buggy behaviour~\cite{Wen2016}.

While some BIC identification techniques~\cite{Bhagwan2018,Murali2021} rely on information derived from failures, such as stack traces or exception messages (in a textual format), these sources may only be indirectly linked to the contents of actual BICs. Since commits are directly coupled to specific locations in the source code, our approach focuses on leveraging the actual coverage of the failing tests as the primary source of information. Our previous work~\cite{An2021} has shown that the coverage of failing test executions, referred to as \emph{failure coverage}, can effectively reduce the search space for BICs. By simply filtering out any commit that is not related to the evolution of code elements covered by the failing tests, the search spaces for 703 bugs in the Defects4J v2.0 benchmark~\cite{Just2014} were reduced to an average of 12.4\% of their original size. This significant reduction rate suggests that failure coverage has the potential to provide a solid foundation for a BIC identification technique available during the debugging phase.

The objective of this work is to accurately identify the BIC using solely the information available at the onset of debugging, immediately following the observation and reproduction of a test failure. Building upon our previous technique for reducing the BIC search space~\cite{An2021}, we introduce an approach that can precisely quantify the relevance of each commit within the reduced search space to the observed failure. Rather than directly measuring the relevance of commits to a failure, our method leverages two fundamental types of relationships, namely the failure-code relationship and the code-commit relationship, and combines these relationships to derive the failure-commit relationships.
\subsection{Basic Notations}
\label{sec:background:notation}

Let us define the following properties of a program $P$:
\begin{itemize}

  \item A set of commits $C = \{c_1, c_2, \ldots\}$ made to $P$
  % , where $c_i$ is older than $c_{i+1}$
  \item A set of code elements $E = \{e_1, e_2, \ldots\}$ of $P$, such as
  statements or methods

  \item A set of test cases $T = \{t_1, t_2, \ldots\}$ where $T_F \subseteq T$ is a set of failing test cases
\end{itemize}

We assume that there is at least one failing test case, i.e., $|T_F| > 0$, and the bug responsible for the failure resides in the source code, i.e., some elements in $E$ cause the failure of $T_F$. We also define the following relations on sets $C$, $T$, and $E$:
% https://math.libretexts.org/Courses/Monroe_Community_College/MTH_220_Discrete_Math/6%3A_Relations/6.1%3A_Relations_on_Sets

\begin{itemize}
  \item A relation $\mathsf{Cover} \subseteq T \times E$ defines the relation between test cases and code elements in the program $P$. For every $t \in T$ and $e \in E$, $(t, e) \in \mathsf{Cover}$ if and only if the test $t$ covers $e$ during the execution.
  \item A relation $\mathsf{Evolve} \subseteq C \times E$ defines the relation between past commits and code elements in the program $P$. For every $c \in C$ and $e \in E$, $(c, e) \in \mathsf{Evolve}$ if and only if the commit $c$ is in the change history of the code element $e$.
\end{itemize}

As our ultimate goal is to find the BIC in $C$, we aim to design a scoring function $s \colon C \to \mathbb{R}$ that gives higher scores to commits that have a higher probability of being the BIC.

\section{\name: Automated BIC Identification via Dynamic, Syntactic, and Historical Analysis}
\label{sec:methodology}

This paper presents \name, a technique to automatically identify the BIC, based
on the assumption that \emph{a commit is more likely to be a bug inducing
commit if it introduced or modified a code element that is more relevant to
the observed failure}. The key idea behind \name is that the relevancy of the code elements to the observed failures can be quantified using existing FL techniques~\cite{Wong2016}, such as SBFL or IRFL, while the association of commits with code elements can be obtained by employing change history tracking tools, such as \texttt{git log}, CodeShovel~\cite{Grund2021}, or CodeTracker~\cite{Jodavi2022}. \Cref{fig:overview} illustrates the three stages of
\name, which are described below:

\begin{figure}[t]
  \centerline{\includegraphics[width=\linewidth]{figures/fonte_overview.pdf}}
  \caption{The three-stage process of \name from the reduction of the BIC search space to the final commit scoring stage.}
  \label{fig:overview}
\end{figure}

\begin{enumerate}
\item \name identifies all potentially buggy code elements using the coverage of failing test cases and discards the commits that are irrelevant to those code elements~\cite{An2021}.
\item \name additionally filters out the semantic-preserving commits that contain only style changes to the suspicious files using AST-level comparisons.
\item \name computes suspiciousness scores of the remaining commits to rank the commits in terms of potential responsibility for the failure.
\end{enumerate}

The rest of this section describes each stage in more detail.

\subsection{Stage 1: Filtering Out Failure-Irrelevant Commits}
\label{sec:methodology:stage1}

Using the notations defined in Section~\ref{sec:background:notation}, we can represent the failure-coverage-based BIC search space reduction~\cite{An2021} as follows. First, let $E_{F} \subseteq E$ denote the set of all code elements that are covered by the failing test cases:

\begin{equation}
  \label{eq:E_susp}
E_{F} = \bigcup_{t \in T_F}\{e \in E|(t, e) \in \mathsf{Cover}\}
\end{equation}

Subsequently, we obtain $C_{F} \subseteq C$, a set of commits that are involved in the evolution of at least one code element in $E_{F}$:
\begin{equation}
  \label{eq:C_susp}
C_{F} = \bigcup_{e \in E_{F}}\{c \in C|(c, e) \in \mathsf{Evolve}\}
\end{equation}

Then, all commits not contained in $C_{F}$ can be discarded from our BIC
search space because the changes introduced by those commits are not related to
any code element executed by failing executions. Consequently, the BIC search
space is reduced from $C$ to $C_{F}$.
%For more detailed explanations or examples, please refer to the original paper.

\subsection{Stage 2: Filtering Out Semantic-Preserving Commits}
\label{sec:methodology:stage2}

\begin{figure}[t]
  \lstinputlisting[language=diff]{Lang-46b-5814f50.diff}
  \caption{Changes by the commit \texttt{5814f50} in Defects4J \texttt{Lang-46}}
  \label{fig:Lang-46b-5814f50}
\end{figure}

The reduced set of candidate BICs, $C_{F}$, may still contain \emph{semantic-preserving commits}, i.e., commits that do not introduce any semantic change to the suspicious code elements. These commits can be further excluded from the BIC search space, as they cannot have altered the functional behaviour of the program so thus cannot introduce a bug~\cite{Kim2006}. An example of such a commit is shown in \Cref{fig:Lang-46b-5814f50}, which modifies the comments and encloses the single statement in the \texttt{if} block.

% To identify whether a given commit $c \in C_{F}$ is a semantic-preserving commit or not, we use the AST level comparisons~\cite{Falleri2014}.
We use the file-level AST level comparison~\cite{Falleri2014} to identify whether a given commit $c \in C_{F}$ is a semantic-preserving commit or not. First, we identify the set of failure-relevant source files, $S$, that are modified by the commit $c$ and covered by the failing test cases. Formally, any file in $S$ contains at least one code element in:
\begin{equation}
E_{F,c} = \{e \in E_{F}|(c, e) \in \mathsf{Evolve}\}
\label{eq:E_susp_c}
\end{equation}
Subsequently, for each file $s \in S$, we compare the ASTs derived from $s$ before and after the commit $c$.\footnote{The AST comparison information per file for each commit can be computed and stored promptly upon its creation, and if consistently updated in CI environments, there is no additional cost associated with computing it when applying \name.} If the ASTs are identical for all files in $S$, we consider the commit $c$ as a semantic-preserving commit.
Note that this approach does
not guarantee 100\% recall, as it is possible for two source files to yield
different ASTs while sharing the same semantic.
% Equivalent mutant
However, it can safely prune the search space due to its soundness, i.e., if it identifies a commit as semantic-preserving, it is guaranteed to be semantic-preserving. Consequently, the search space for BIC can be further reduced to $C_{BIC} = C_{F} \setminus C_{SP}$, in which $C_{SP}$ denotes all identified semantic-preserving commits in $C_{F}$.

It is worth noting that during this phase, there is also the option to consider discarding the refactoring changes~\cite{fowler2018refactoring} as well, in the same manner as RA-SZZ~\cite{Neto2018}. However, existing tools for detecting refactoring code, such as RefDiff~\cite{Silva2021}, do not guarantee 100\% precision, which introduces the risk of mistakenly excluding semantic-changing commits. Therefore, to guarantee the completeness of the narrowed-down BIC search space, \name only employs reliable AST-level comparisons.

\subsection{Stage 3: Scoring Commits using FL Scores and Code Change History}
\label{sec:methodology:stage3}

After reducing the search space, the remaining task is to evaluate how likely it is that each commit within $C_{BIC}$ is responsible for the observed failure. As mentioned earlier, our basic intuition is that if a commit had created, or modified, more suspicious code elements for the observed failures, it is more likely to be a BIC.

The suspiciousness of code elements can be quantified via an FL technique. For
example, we can apply SBFL~\cite{Wong2016} using the coverage of the test suite
$T$: note that SBFL uses only test coverage and result information, both of
which are available at the time of observing a test failure. Assuming that we
are given the suspiciousness scores, let $susp \colon E_{F} \to \mathbb{R}^{\geq 0}$ be the mapping function from each suspicious code element in $E_{F}$
to its non-negative FL score.\footnote{The constraint of FL-score being
non-negative is adopted for the sake of simplicity. Note that any FL results
can be easily transformed so that the lowest score is 0.} To convert the
code-level scores to the commit level, we propose a voting-based commit scoring
model where the FL score of a code element is distributed to its relevant
commits. The model has two main components: rank-based voting power and
depth-based decay.

\begin{table}[t]
  \centering
  \caption{Example of the voting power of code elements}
  \scalebox{1.00}{
  \begin{tabular}{l|l|rrrrr}
  \toprule
  \multicolumn{2}{l|}{\textbf{Code Element}}           & $e_1$ & $e_2$ & $e_3$ & $e_4$ & $e_5$\\\midrule
  \multicolumn{2}{l|}{\textbf{Score}}            & 1.0   & 0.6   & 0.6 & 0.6 & 0.3  \\\midrule
  \multicolumn{2}{l|}{$rank_{\mathtt{max}}$}  & 1     & 4    & 4     & 4 & 5\\
  \multicolumn{2}{l|}{$rank_{\mathtt{dense}}$}  & 1     & 2    & 2     & 2 & 3\\\midrule
  \multirow{4}{*}{$vote$}&$\alpha=0$, $\tau=\mathtt{max}$   & 1.00  & 0.25  & 0.25  & 0.25 & 0.20  \\
  &$\alpha=1$, $\tau=\mathtt{max}$   & 1.00  & 0.15  & 0.15  & 0.15  & 0.06  \\
  &$\alpha=0$, $\tau=\mathtt{dense}$ & 1.00  & 0.50  & 0.50  & 0.50  & 0.33  \\
  &$\alpha=1$, $\tau=\mathtt{dense}$ & 1.00  & 0.30  & 0.30  & 0.30  & 0.10  \\
\bottomrule
  \end{tabular}}
  \label{tab:voting}
\end{table}

\textbf{Rank-based Voting Power:} Recent work~\cite{Sohn2019,Sohn2021,
habchi2022made} showed that, when aggregating FL scores from finer granularity
elements (e.g, statements) to a coarser level (e.g., methods), it is better to
use the \emph{relative rankings} from the original level only, rather than directly
using the raw scores. The actual aggregation takes the form of voting: the higher
the ranking of a code element is in the original level, the more votes it is
assigned with for the target level. Subsequently, each code element casts its
votes to the related elements in the target level. We adopt this voting-based
method to aggregate the statement-level FL scores to commits. The
\emph{voting power} of each code element $e$ based on their FL rankings (and scores) as follows:

\begin{equation}
vote(e) = \frac{\alpha*susp(e) + (1 - \alpha)*1}{rank_{\tau}(e)}
\label{eq:vote}
\end{equation}

where $\alpha \in \{0, 1\}$ is a hyperparameter that decides whether to use the
suspiciousness value ($\alpha=1$) as a numerator or not ($\alpha=0$), and $\tau$
a hyperparameter that defines the tie-breaking scheme. We vary $\tau \in \{\mathtt{max},
\mathtt{dense}\}$: the $\mathtt{max}$ tie-breaking scheme gives the lowest (worst)
rank in the tied group to all tied elements, while $\mathtt{dense}$ gives the highest
but does not skip any ranks after ties. By design, $\tau=\mathtt{max}$ will penalise tied
elements more severely than $\tau=\mathtt{dense}$.
The example in \Cref{tab:voting} shows how the hyperparameters affect voting. Note that the relative order between FL scores is preserved in the voting power regardless of hyperparameters, i.e., $vote(e) > vote(e')$ if and only if $susp(e) > susp(e')$.

\textbf{Depth-based Decay:}
Wen et al.~\cite{Wen2016} showed that using the information about commit time
can boost the accuracy of the BIC identification. Similarly, Wu et al.~\cite{Wu2017} observed that the commit time of crash-inducing changes is closer to
the reporting time of the crashes.
%In their model to predict the crash-inducing commit, the feature related to the commit time (ITDCR) was shown to be the third most important feature.
Based on those findings, we hypothesise that \emph{older commits are less likely to
be responsible for the currently observed failure}, because if an older commit was
a BIC, it is more likely that the resulting bug has already been found and
fixed.
To capture this intuition, our approach diminishes the voting impact of a program element for older commits that have greater historical depths. The historical depth of a commit $c$, with respect to a code element $e \in E_{F,c}$ (\Cref{eq:E_susp_c}), is defined as follows:
\begin{equation}
\begin{aligned}
depth(e,c) = & |\{c' \in C_{BIC}|\\
& (c', e) \in \mathsf{Evolve} \wedge c'.time > c.time\}|
\end{aligned}
\label{eq:depth}
\end{equation}
% $depth(e,c)$ is simply the number of commits in $C_{BIC}$ that are relevant to $e$ and newer than $c$.
which is essentially a count of the commits in the set $C_{BIC}$ that are relevant to a particular code element $e$ and are more recent than the commit $c$.

\begin{figure}[t]
    \centering
    \includegraphics[width=0.95\linewidth]{figures/fonte_voting.pdf}
    \caption{Example of computing the commit scores when $\lambda = 0.1$}
    \label{fig:voting}
  \end{figure}

Bringing it all together, we use the following model to assign a score to each commit $c$ in $C_{BIC}$:
\begin{equation}
  commitScore(c) = \sum_{e \in E_{F}^c}vote(e)*(1-\lambda)^{depth(e, c)}
\label{eq:commit_score}
\end{equation}
where $\lambda \in \left[0, 1\right)$ is the decay factor: when $\lambda=0$, there is no penalty for older commits. \Cref{fig:voting} shows the example of calculating the score of commits when $\lambda=0.1$.



Finally, based on $commitScore$, the commit scoring function $s \colon C \to
\mathbb{R}^{\geq 0}$ of \name is defined as follows:
$$s(c) =
  \begin{cases}
      commitScore(c)  & \text{if } c \in C_{BIC}\\
      0              & \text{otherwise}
  \end{cases}
$$
%\end{equation}
% https://www.cuemath.com/exponential-decay-formula/

\section{Weighted Bisection}
\label{sec:weighted_bisection}

Bisection is a traditional way of finding the BIC by repeatedly narrowing down
the search space in half using binary search: it is also equipped in popular
Version Control Systems (VSCs), e.g., \texttt{git bisect} or \texttt{svn-bisect}. A standard 
bisection is performed as follows: given the last \emph{good} and earliest \emph{bad}
versions of a program, it iteratively checks whether the midpoint of
those two versions, referred to as a \emph{pivot}, contains the bug. If there
is a bug, the earliest bad point is updated to the pivot, otherwise, the last
good point is updated to the pivot. If there is a bug-revealing test case that
can automatically check the existence of a bug, this search process can be fully
automated.

However, as pointed out in previous work~\cite{Murali2021}, even though the bug
existence check can be automated, each bisect iteration may still require a
significant amount of time and computing resources, especially when the program
is large and complex, or the bug-revealing test takes a long time to execute.
Since a lengthy bisection process can block the entire debugging pipeline, we
aim to explore whether the bisection can be accelerated using the commit score
information.

\begin{figure}[t]
  \centerline{\includegraphics[width=0.95\linewidth]{figures/Math-87b.pdf}}
  \caption{Example of applying the weighted bisection to \texttt{Math-87}}
  \label{fig:search}
\end{figure}


We propose a \emph{weighted} bisection algorithm, where the search pivot is
set to a commit that will halve \emph{the amount of remaining commit scores}
instead of \emph{the number of remaining commits}, in order to to check
the point of greatest ambiguity where the chances of finding the BIC in
two directions (before and after the commit) are almost equal. For example, let us consider
the example in \Cref{fig:search} that shows the score distribution of the
commits in the reduced BIC search space of \texttt{Math-87} in Defects4J. For
\texttt{Math-87}, the score distribution is biased towards a small number of
recent commits including the real BIC (marked in red) with the third-highest score. In this
case, simply using the midpoint as a search pivot might not be a good choice
because all highly suspicious commits still remain together on one side of the
split search space: as a result, the standard bisection requires five iterations to finish.
Alternatively, if we pivot at the commit that halves the amount of
remaining scores, the bisection reaches the actual BIC more quickly, completing
the search in three iterations.

\begin{algorithm}[t]
  \small
  \SetCommentSty{mycommfont}
  \SetKwInput{KwPrecondition}{Precondition}
  \SetKwInput{KwPostcondition}{Postcondition}
  \SetKwFunction{CScore}{s}

  \KwIn{Array of commits $C$}
  \KwIn{Commit score (weight) function $s: C \rightarrow \mathbb{R}^{\geq 0}$}
  \KwOut{Bug inducing commit $c \in C$}
  \KwPrecondition{$C[i]$ is newer than $C[j]$ if and only if $i < j$}
  \KwPostcondition{$\mathtt{bad} + 1 = \mathtt{good}$}

  
  $C' \longleftarrow$ empty list\;
  \For{$c \in C$}{
      \If{$s(c) > 0$}{ \tcp{Extract Subarray with Positive Scores}
          Append $c$ to $C'$\;
      }
  }
  $\mathtt{bad}, \mathtt{good} \longleftarrow 0, |C'|$\\
  \While{$\mathtt{bad} + 1 < \mathtt{good}$}{
    $\mathtt{pivot} \leftarrow \text{argmin}_{p=\mathtt{bad}+1}^{\mathtt{good}-1}{|\sum_{i=\mathtt{bad}}^{p-1}s(C'[i]) - \sum_{i=p}^{\mathtt{good}-1}s(C'[i])|}$\\
    \If{The target bug is detected in $C'[\mathtt{pivot}]$}{
      $\mathtt{bad} \leftarrow \mathtt{pivot}$\\
    }\Else{
      $\mathtt{good} \leftarrow \mathtt{pivot}$\\
    }
  }

  \Return{$C'[\mathtt{bad}]$}
  \caption{Weighted Bisection Algorithm}
  \label{algo:search}
\end{algorithm}

Algorithm~\ref{algo:search} presents the weighted bisection algorithm.
It takes as input a chronologically sorted array of commits $C$ and a commit score function $s \in C \rightarrow \mathbb{R}^{\geq 0}$, which assigns non-negative scores to each commit, and returns the BIC.
First, it extracts a subarray $C'$ from $C$, containing only the commits with positive scores (Lines 1-4). Assuming that at least one BIC exists in the sorted sequence $C'$, the earliest bad index $\mathtt{bad}$ is initialized to 0, representing the index of the most recent commit (Line 5). Since all commits in $C'$ are BIC candidates, the last good index $\mathtt{good}$ is set to the (virtual) index immediately after the oldest commit (Line 5). The algorithm then iteratively selects a new $\mathtt{pivot}$ index from the range $[\mathtt{bad}+1, \mathtt{good}-1]$, until there are no remaining commits between $\mathtt{bad}$ and $\mathtt{good}$ (Line 6). The $\mathtt{pivot}$ selection is performed such that it minimizes the difference between the sum of scores on the left side (not including $\mathtt{pivot}$) and the sum of scores on the right side (including $\mathtt{pivot}$) (Line 7). Once a new $\mathtt{pivot}$ is selected, the commit $C'[\mathtt{pivot}]$ is inspected for the bug, either by executing the bug-revealing tests or through manual inspection (Line 8). If the bug is detected (i.e., the $\mathtt{pivot}$ is a bad commit), the $\mathtt{bad}$ index is updated to $\mathtt{pivot}$ (Line 9); otherwise (i.e., the $\mathtt{pivot}$ is a good commit), the $\mathtt{good}$ index is updated to $\mathtt{pivot}$ (Line 11). Finally, when the loop terminates, the algorithm returns the identified BIC at the $\mathtt{bad}$ index (Line 12).

It is worth noting that this algorithm is a \emph{generalised} version of the standard bisection: the standard bisection method can be considered a special case of the weighted bisection algorithm where $s$ is a non-zero constant function, assigning the same non-zero value to all commits, e.g., $\forall c \in C: s(c) = 1.0$.

\section{Evaluation Setup}
\label{sec:eval_setup}

This section describes the dataset used for evaluation (\Cref{sec:dataset}), the details of \name's implementation (\Cref{sec:implementation}), and our research questions along with their experimental protocols (\Cref{sec:rq}).

\subsection{Dataset of Bug Inducing Commits}
\label{sec:dataset}
We choose Defects4J v2.0.0~\cite{Just2014}, a collection of 835 real-world bugs
in Java open-source programs, as the source of our experimental subjects.
While Defects4J provides test suites containing the bug-revealing tests
for every bug, as well as the entire commit history for each buggy version, it
lacks the BIC information for each bug.

We, therefore, start with a
readily-available BIC dataset for 91 Defects4J bugs\footnote{https://github.com/justinwm/InduceBenchmark} constructed by Wen et al.~\cite{Wen2019}. This
dataset was created by running the bug-revealing test cases on the past
versions and finding the earliest buggy version that makes the tests fail.
However, in our experiment, we are forced to exclude 24 out of 91 data points.
Since \name is implemented using Git, it cannot trace the commit history of
nine bugs from the \texttt{JFreeChart} project which uses SVN as its version
control system. Further, we exclude 14 data points that are shown to be
inaccurate by previous work~\cite{An2021}. Lastly, \texttt{Time-23} is also
discarded, because we found that the identified commit in the dataset does not
contain any change to code, but only to the license comments. The detailed
reasons can be found in our repository. In summary, we make use of 67
ground-truth BICs from this dataset.


The original dataset of 67 ground-truth BICs identified by Wen et al.~\cite{Wen2019} encompasses bugs from only four out of the 17 projects in Defects4J. To expand the evaluation dataset, we have manually compiled an additional set of ground-truth BICs. Two authors independently pinpointed the BIC for each bug by examining bug reports, symptoms of failures, and patches provided by developers. To minimize the effort required for manual inspection, our initial focus was on all Defects4J bugs where the narrowed down BIC search space, $C_{BIC}$, comprised ten or fewer potential BICs. This process led to a consensus on 70 instances, which were then included in the dataset. Furthermore, we identified ground-truth BICs for more bugs, mainly from software projects that initially had minimal or no BIC data. An additional 76 data points, on which the authors agreed, were also incorporated. In summary, a total of 206 data points (67 from Wen et al. + 146
manually curated - 7 overlapped) are used for the evaluation of \name. \Cref{tab:subjects} shows the number of ground-truth BICs for each software project, before and after the manual curation of the additional dataset compared to the initial dataset from Wen et al.~\cite{Wen2019}. The
combined BIC dataset and the provenance of each data point are available in our
repository for further scrutiny.

\begin{table}[t]
    \centering
    \caption{Distribution of ground-truth BICs across software projects: a comparison between the initial dataset taken from Wen et al.~\cite{Wen2019} and the augmented dataset after manual data curation.}
    \scalebox{1.00}{
    \begin{tabular}{l|rr}
        \toprule
        Project & \multicolumn{2}{c}{\# BICs}\\\cmidrule{2-3}
         & Wen et al.~\cite{Wen2019} & Augmented\\\midrule
        Cli & 0  & 15\\
        Closure & 35 & 36\\
        Codec & 0 & 3\\
        Collections & 0 & 2\\
        Compress & 0 & 5\\
        Csv & 0 & 13\\
        Gson & 0 & 4\\
        JacksonCore & 0 & 5\\
        JacksonDatabind & 0 & 6 \\
        JacksonXml & 0 & 2\\
        Jsoup & 0 & 35\\
        JxPath & 0 & 4\\
        Lang & 6 & 29\\
        Math & 21 & 40\\
        Mockito & 0 & 1\\
        Time & 5 & 6\\\midrule
        Total & 67 & 206\\\bottomrule
    \end{tabular}}
    \label{tab:subjects}
\end{table}

\subsection{Implementation Details of \name}
\label{sec:implementation}

\begin{table}[t]
  \centering
  \caption{Example of Relevant Test Selection (\texttt{Time-15})}
  \scalebox{0.95}{
  \begin{tabular}{l}
  \toprule
  \textbf{Failing Test ($T_F$)}\\\midrule
  org.joda.time.field.TestFieldUtils::testSafeMultiplyLongInt\\\midrule
  \textbf{Classes Covered by the Failing Test}\\\midrule
  org.joda.time.field.\textbf{FieldUtils}\\
  org.joda.time.\textbf{IllegalFieldValueException}\\\midrule
  \textbf{Relevant Tests ($T \setminus T_F$)}\\\midrule
  org.joda.time.Test\textbf{IllegalFieldValueException}::testGJCutover\\
  org.joda.time.Test\textbf{IllegalFieldValueException}::testJulianYearZero\\
  org.joda.time.Test\textbf{IllegalFieldValueException}::testOtherConstructors\\
  org.joda.time.Test\textbf{IllegalFieldValueException}::testReadablePartialValidate\\
  org.joda.time.Test\textbf{IllegalFieldValueException}::testSetText\\
  org.joda.time.Test\textbf{IllegalFieldValueException}::testSkipDateTimeField\\
  org.joda.time.Test\textbf{IllegalFieldValueException}::testVerifyValueBounds\\
  org.joda.time.Test\textbf{IllegalFieldValueException}::testZoneTransition\\
  org.joda.time.field.Test\textbf{FieldUtils}::testSafeAddInt\\
  org.joda.time.field.Test\textbf{FieldUtils}::testSafeAddLong\\
  org.joda.time.field.Test\textbf{FieldUtils}::testSafeMultiplyLongLong\\
  org.joda.time.field.Test\textbf{FieldUtils}::testSafeSubtractLong\\
  \bottomrule
  \end{tabular}}
  \label{tab:relevant}
\end{table}

\subsubsection{Basic Properties}
We apply \name at the \emph{statement}-level granularity. Following the notations in \Cref{sec:background:notation}, $E$ is a set
of statements composing the target buggy program. 
The initial BIC search space,
$C$, is set to all commits from the very first commit up to the commit
corresponds to the buggy version\footnote{\texttt{revision.id.buggy} in
Defects4J}. From the developer-written test cases, we only use the bug-revealing (i.e.,
failing) test cases as well as their relevant test cases as $T$. A test case is
considered relevant if and only if its full name contains the name of at least one class
executed by the failing test cases. \Cref{tab:relevant} shows the example
of the relevant test selection.

\subsubsection{Construction of the $\mathsf{Cover}$ relation}
To construct the $\mathsf{Cover}$ relation between $T$ and $E$, we measure the
statement-level coverage of each test case in $T$ using \texttt{Cobertura v2.0.3} which
is included in Defects4J.

\subsubsection{Construction of the $\mathsf{Evolve}$ relation}
To establish the $\mathsf{Evolve}$ relationship between $C$ and $E$, it is necessary to track the commit history for each code element. For this purpose, we utilise the \texttt{git log} command\footnote{\texttt{git log -C -M -L<start\_line>,<end\_line>:<file>}. The flags \texttt{-C} and \texttt{-M} are used for detecting file renaming, copying, or moving across versions.} in line with our prior research~\cite{An2021}. In addition to \texttt{git log}, we explored using CodeShovel~\cite{Grund2021} and CodeTracker~\cite{Jodavi2022}, which are advanced tools for retrieving the history of method changes. 

Please note that for each statement, we retrieve the commit history of its
enclosing method and create the $\mathsf{Evolve}$ relations between the
statements and the retrieved commits to ensure high recall for commit
histories. This is also to deal with omission bugs~\cite{Zhang2007}: if a bug is caused by
omission of some statements, we cannot trace the log of the missing statements
because they literally do not exist in the current version. In that case,
tracing the log of the neighbouring statements (in the enclosing method) will
enable to find the inducing commit, as the method that encloses the omission
bug should have been covered by the failing tests~\cite{An2021}.

% However, CodeShovel~\cite{Grund2021}, a tool for retrieving method histories accurately. As CodeShovel has been shown to outperform other tools, we first used CodeShovel in our experiment.

\subsubsection{Detection of Semantic-Preserving Commits}
% For the step 3)-b) in Section~\ref{sec:methodology:stage2},
In Stage 2, before comparing the ASTs before and after a commit, we use OpenRewrite v7.21.0\footnote{https://github.com/openrewrite/rewrite} to ensure the same coding standard between the two versions of files. More
specifically, we use the \texttt{Cleanup} recipe\footnote{https://docs.openrewrite.org/reference/recipes/java/cleanup} that fixes any errors that violate CheckStyle rules.\footnote{https://checkstyle.sourceforge.io/}
This ensures that trivial differences between two versions that do not lead to
semantic differences are ignored: a good example is a commit in \texttt{Lang},
which is shown in \Cref{fig:Lang-46b-5814f50}. To compare ASTs after formatting the files, we use the
isomorphism test of GumTree v3.0.0~\cite{Falleri2014} that has time
complexity of $O(1)$.

\subsubsection{Fault Localisation\label{sec:fl}} In theory, any FL technique that produces suspiciousness scores (or rankings) can be plugged into \name. In this paper, we use two representative FL techniques, SBFL and IRFL.

\begin{itemize}
  \item SBFL: We use a widely-used SBFL formula, Ochiai~\cite{Abreu2006}, which can be expressed in our context as follows:
  $$
  Ochiai(e) = \frac{|\{t \in T_F|(t, e) \in \mathsf{Cover}\}|}{\sqrt{|T_F|*|\{t \in T|(t, e) \in \mathsf{Cover}\}|}}
  $$
  By definition, $Ochiai(e) > 0$ if and only if $e \in E_{F}$ (\Cref{eq:E_susp}).
  \item IRFL: We employ an unsupervised statement-level IRFL technique, Blues, which was proposed in a recent APR study~\cite{Motwani2023} and builds upon on BLUiR~\cite{Saha2013}, an unsupervised file-level IRFL technique. We directly utilise the pre-computed Blues suspiciousness scores for the Defects4J bugs available in their replication package.
\end{itemize}

\subsubsection{Hyperparameters}
\label{sec:hyperparameters}

In our experiments, we investigate the effects of varying hyperparameters: $\alpha \in \{0, 1\}$ and $\tau \in \{\mathtt{max},\mathtt{dense}\}$ for \Cref{eq:vote}, and $\lambda \in \{0.1, 0.2, 0.3\}$  for \Cref{eq:commit_score}.

\subsection{Research Questions}
\label{sec:rq}

We ask the following research questions in this paper:
\begin{itemize}
  \item \textbf{RQ1.} How accurately does \name find the BIC?
  \item \textbf{RQ2.} Does \name outperform other BIC identification approaches?
  \item \textbf{RQ3.} How efficient is the weighted bisection compared to the standard bisection?

% \item \textbf{RQ1. Ranking Performance of \name}: How accurately does \name rank the BIC?
% \item \textbf{RQ2. Comparison with Baselines}: How does \name's performance in identifying BICs compare against other state-of-the-art baseline techniques?
% \item \textbf{RQ3. Impact of FL Accuracy on \name}: What is the impact of FL accuracy to the performance of \name?
% \item \textbf{RQ4. Standard Bisection vs. Weighted Bisection}: How efficient is the weighted bisection compared to the standard bisection?
\end{itemize}

\section{Results}
\label{sec:results}
This section describes the evaluation methodology used to investigate each research question, along with the findings.

\subsection{\textbf{RQ1. Effectiveness of \name}}

To evaluate the efficacy of \name, we pose three subsidiary questions focusing on its search space reduction capabilities, its performance in ranking BICs, and the impact of the ranking-based voting scheme and depth-based decay.

\begin{tcolorbox}[title=RQ1-1. Search Space Reduction (Stage 1\&2)]
To what extent do the first two stages of \name reduce the search space?
\end{tcolorbox}

% \begin{figure}[t]
%   \centerline{\includegraphics[width=0.95\linewidth]{figures/RQ1_SS.pdf}}
%   \caption{Distributions of the sizes of search space}
%   \label{fig:RQ1-SS}
% \end{figure}

\begin{table}[t]
\caption{Evaluation of code history tracking tools, focusing on the average reduction in BIC search space size, the validity ratio, and the average time taken to compile commit histories. The Validity Ratio represents the proportion of bugs for which the actual BIC is located within the reduced search space $C_{BIC}$.}
\label{tab:RQ1-reduction}
\scalebox{0.95}{
\begin{tabular}{r|rrr}
  \toprule
 History & Reduction Ratio & Validity Ratio & Duration\\
 Tracking Tool & (After Stage 1, Stage 2) & & \\\midrule
\texttt{git log} & 11.3\%, 11.0\% & $\mathbf{100\%}$ $(=\frac{206}{206})$ & \textbf{27.3s}\\
CodeShovel & 11.8\%, 11.1\% & $99.0\%$ $(=\frac{204}{206})$ & 521.0s\\
CodeTracker & \textbf{11.1\%}, \textbf{10.9\%} & $98.5\%$ $(=\frac{203}{206})$ & 738.3s\\\bottomrule
\end{tabular}}
\end{table}

\noindent\textbf{Evaluation Protocol} To determine the extent to which Stages 1 and 2 can narrow down the BIC search space, we calculate the reduction ratio of the BIC search space after Stage 1 (\(|C_{susp}|/|C|\)), and Stage 2 (\(|C_{BIC}|/|C|\)). Under a consistent methodology for measuring test coverage failures, the process of reducing the search space is influenced solely by the choice of code history tracking tools, which establish the \texttt{Evolve} (or code-commit) relation. Consequently, we report the outcomes of search space reduction using \texttt{git log}, CodeShovel~\cite{Grund2021}, and CodeTracker~\cite{Jodavi2022}. Additionally, we verify the presence of the actual BIC within the narrowed search space, \(|C_{BIC}|\), as a basic validity test, and measure the time taken to collect the commit histories using each tool to access the efficiency.

\noindent\textbf{Results} \Cref{tab:RQ1-reduction} presents a comparative analysis of commit counts filtered through the initial two stages of \name, using different code history tracking tools. \texttt{git log} and CodeTracker are the most efficient, reducing the commit count to 11.3\% and 11.1\% after the first stage, respectively, and further down to 11.0\% and 10.9\% after the second stage. CodeShovel shows a slightly less efficient reduction, with 11.8\% of commits remaining after the first stage and 11.1\% after the second. 
The Validity Ratio is highest for \texttt{git log} at 100\%. This means \texttt{git log} successfully retained the actual BIC in all evaluated instances. In contrast, CodeShovel and CodeTracker have slightly lower Validity Ratios at 99.0\% and 98.5\%, respectively, indicating that these tools occasionally produce incorrect commit histories, which is detailed in our artifect, which affects their reliability in retaining the BIC throughout the reduction process.
The Duration it takes for each tool to compile the commit histories is also noted, with \texttt{git log} being the fastest at 27.3 seconds on average, followed by CodeShovel at 521.0 seconds, and CodeTracker being the slowest at 738.3 seconds. This evaluation highlights the trade-offs between accuracy, efficiency, and effectiveness in reducing the BIC search space among the different tools.

\noindent\textbf{Answer to RQ1-1} On average, the BIC search space was narrowed down to approximately 11\% of its initial size, indicating that the choice of code history tracking tool had only a minimal impact on the reduction ratio. Among the evaluated tools, \texttt{git log} stands out for its reliability and efficiency.  

\begin{tcolorbox}[title=RQ1-2. Ranking Performance (Stage 3)]
  How effectively does the third stage of \name identify the BIC within the narrowed search space?
\end{tcolorbox}

\noindent \textbf{Evaluation Protocol} If the scoring mechanism of \name is effective, BICs will receive \emph{higher} scores compared to those changes that are not responsible for bugs. Therefore, to assess the performance of \name, we employ two widely recognized ranking-based metrics.
\begin{itemize}
  \item Mean Reciprocal Rank (MRR)~\cite{Craswell2009}: The average reciprocal rank of the BIC (\emph{higher is better})
  \item Accuracy@n (Acc@n): The percentage of bugs where the ranking of the ground-truth BIC is within the top $n$ positions (\emph{higher is better})
 \end{itemize} 
In situations where multiple elements share identical scores, we apply the max-tiebreaker approach, which conservatively allocates the worst rankings to these tied elements.
Additionally, we compare the performance of \name against a \textit{random baseline} to ascertain whether the ranking algorithm performs significantly better than random chance. In a scenario where $n$ commits exist in the reduced search space, the expected rank of a BIC under random conditions would be $\frac{1 + n}{2}$. We report the MRR and Acc@n values for this baseline.

Since \texttt{git log} demonstrated both reliable and efficient performance in RQ1-1, we use it to establish the \texttt{Evolve} relation for the remaining experiments. Additionally, we evaluate \name with diverse settings to answer this question. As a base FL technique for \name, we employ either SBFL or IRFL, as described in Section \ref{sec:fl}, in combination with diverse hyperparameter configurations outlined in Section \ref{sec:hyperparameters}.

% \begin{figure}[t]
%   \centerline{\includegraphics[width=\linewidth]{figures/RQ1_MRR.pdf}}
%   \caption{MRR for each hyperparameter configuration of \name}
%   \label{fig:RQ1-MRR}
% \end{figure}


\begin{table}
  \caption{The BIC ranking performance of \name combined with IRFL and SBFL, across various hyperparameter configurations}
  \label{tab:fonte_results}
  \scalebox{0.90}{
  \begin{tabular}{c|c|c|rrr|rrr}
    \toprule
    \multicolumn{3}{c|}{Hyperparameters}& \multicolumn{3}{c|}{\name with IRFL} & \multicolumn{3}{c}{\name with SBFL} \\\midrule
    $\lambda$ & $\alpha$ & $\tau$  & MRR & Acc@1 & Acc@5 & MRR & Acc@1 & Acc@5 \\
    \midrule
      \multirow{4}{*}{0.1} & \multirow{2}{*}{0} & $\mathtt{dense}$ & \textbf{0.453} & \textbf{30.6\%} & 62.6\% & 0.474 & 31.6\% & 67.5\% \\
                                               &  & $\mathtt{max}$ & 0.445 & 28.6\% & \textbf{65.0\%} & \textbf{0.481} & \textbf{32.0\%} & 68.9\% \\\
                           & \multirow{2}{*}{1} & $\mathtt{dense}$ & 0.433 & 27.2\% & 63.6\% & 0.475 & 30.6\% & \textbf{70.4\%} \\
                                               &  & $\mathtt{max}$ & 0.432 & 27.2\% & 63.1\% & 0.473 & 31.6\% & 69.9\% \\\midrule
      \multirow{4}{*}{0.2} & \multirow{2}{*}{0} & $\mathtt{dense}$ & 0.445 & 29.1\% & 61.7\% & 0.460 & 28.6\% & 68.4\% \\
                                               &  & $\mathtt{max}$ & 0.446 & 28.6\% & 64.6\% & 0.475 & 30.6\% & 69.4\% \\
                           & \multirow{2}{*}{1} & $\mathtt{dense}$ & 0.434 & 27.7\% & 62.6\% & 0.465 & 29.1\% & \textbf{70.4\%} \\
                                               &  & $\mathtt{max}$ & 0.436 & 28.2\% & 62.6\% & 0.465 & 30.1\% & 69.4\% \\\midrule
      \multirow{4}{*}{0.3} & \multirow{2}{*}{0} & $\mathtt{dense}$ & 0.445 & 29.1\% & 61.7\% & 0.451 & 27.7\% & 68.0\% \\
                                               &  & $\mathtt{max}$ & 0.434 & 26.7\% & 62.6\% & 0.471 & 30.6\% & 69.4\% \\
                           & \multirow{2}{*}{1} & $\mathtt{dense}$ & 0.421 & 25.2\% & 62.1\% & 0.458 & 28.2\% & 69.4\% \\
                                               &  & $\mathtt{max}$ & 0.422 & 25.2\% & 62.1\% & 0.459 & 28.6\% & 68.9\% \\\midrule
    \multicolumn{3}{c|}{Random Baseline} & 0.150 & 1.5\% & 32.5\% & 0.150 & 1.5\% & 32.5\%\\
    \bottomrule
    \end{tabular}}
\end{table}


\noindent \textbf{Results} 
\Cref{tab:fonte_results} details the effectiveness of \name in ranking BICs when integrated with two distinct FL techniques: IRFL and SBFL. The evaluation spans a range of hyperparameters and is quantified using three metrics: MRR, Acc@1, and Acc@5. The top-performing results for each FL technique and metric are denoted in bold within the table.


Generally, \name, regardless of the settings used, consistently exceeds the performance of the random baseline, demonstrating its strong capability to accurately identify the actual BIC among the potential commits in the reduced search space.
With IRFL, \name achieves its best MRR at 0.453 and its highest Acc@1 at 30.6\%. The best Acc@5 is 65.0\%. In the SBFL configuration, \name attains its best MRR of 0.481, an Acc@1 of 32.0\%, and an Acc@5 of 70.4\%. The hyperparameter setting of $\lambda=0.1$ consistently delivers superior results compared to other $\lambda$ values. Examining the voting hyperparameters $alpha$ and $tau$, for IRFL, the setting of $\alpha=0$ and $\tau=\mathtt{dense}$ demonstrates strong MRR and Acc@1 scores, while $\alpha=0$ and $\tau=\mathtt{max}$ is notable for its Acc@5. Conversely, for SBFL, the combination of $\alpha=0$ and $\tau=\mathtt{max}$ is notable for higher MRR and Acc@1, whereas $\alpha=1$ and $\tau=\mathtt{dense}$ is distinguished by its Acc@5 results.

\begin{figure}[t]
  \centerline{\includegraphics[width=\linewidth]{figures/SBFL-and-IRFL.pdf}}
  \caption{Performance comparison of \name when integrated with IRFL and SBFL, categorised based on whether IRFL outperforms SBFL (IRFL wins), they have equal performance (draw), or SBFL outperforms IRFL (SBFL wins)}
  \label{fig:sbfl-irfl}
\end{figure}

We also observe that the integration of \name with SBFL yields superior results compared to combining it with IRFL. This advantage can be largely attributed to the higher FL accuracy of SBFL (using the Ochiai method) over IRFL (using the Blues method). When examining the average rankings of buggy methods across the analyzed bugs, IRFL outperforms SBFL in only 14.6\% (30 out of 206) of the cases, whereas SBFL surpasses IRFL in 60.0\% (123 out of 206) of the cases. \Cref{fig:sbfl-irfl} demonstrates that the performance of \name combined with IRFL outperforms that combined with SBFL in cases where IRFL outperforms SBFL, and vice versa. This overall trend indicates that more accurate FL results are likely to lead to better accuracy of \name in ranking BIC. Therefore, it is anticipated that \name will further benefit from advancements in more accurate and advanced FL techniques in the future.

\noindent\textbf{Answer to RQ1-2}: \name demonstrates high effectiveness in identifying the BIC within the narrowed search space. When combined with SBFL, \name achieves an MRR of 0.481, an Acc@1 of 32.0\%, and an Acc@5 of 70.4\%, indicating its ability to rank the actual BIC highly within the candidates.

\begin{tcolorbox}[title=RQ1-3. Ablation Study]
  What is the impact of Rank-based Voting Power and Depth-based Decay on the performance of \name?
\end{tcolorbox}

\noindent \textbf{Evaluation Protocol} To examine the influence of two key features of \name, Rank-based Voting Power and Depth-based Decay, on its overall performance, we conduct an ablation study. First, for evaluating the effect of Rank-based Voting Power, we evaluate a naive alternative strategy: using the FL scores directly as the voting power of each statement instead of considering the relative ranking among statements. In simpler terms, we replace \Cref{eq:vote} with the following equation:
\begin{equation}
\label{eq:vote_score}
vote(e) = susp(e)
\end{equation}
Second, to gauge the impact of Depth-based Decay, we measure how the performance of \name changes when the value of $\lambda$ is set to $0.0$, while keeping all other settings unchanged.
For the ablation study, we conducted the experiments based on the hyperparameter configuration that demonstrated the highest MRR for \name in RQ1-1 to simplify the analysis: $(\lambda, \alpha, \tau)$ is set to $(0.1, 0, \mathtt{dense})$ for \name with IRFL, $(0.1, 0, \mathtt{max})$ for \name with SBFL.
% \begin{itemize}
% \item Uniform: This scheme assigns equal weight to all lines covered by failing test cases, disregarding any FL information, i.e.:
% \begin{equation}
%   \label{eq:vote_uniform}
%   vote(e) = 1
% \end{equation}

% \item Raw Score: This scheme defines the voting power of a statement as its raw FL score, without considering its ranking relative to other statements, i.e.:
% \begin{equation}
%   \label{eq:vote_score}
%   vote(e) = susp(e)
% \end{equation}
% \end{itemize}

\begin{table}
  \caption{Performance evaluation of three variants of \name with key features ablated: without Rank-based Voting Power, without Depth-based Decay, and without both features}
  \label{tab:ablation}
  \scalebox{0.90}{
  \begin{tabular}{l|rrr}
    \toprule
 &  MRR & Acc@1 & Acc@5 \\
    \midrule
      \textbf{Fonte with IRFL ($\alpha=0, \tau=\mathtt{dense}, \lambda=0.1$)} & \textbf{0.453} & \textbf{30.6\%} & \textbf{62.6\%} \\
      - w/o Ranking-based Voting Power & 0.404 & 24.8\% & 57.3\%\\
      - w/o Depth-based Decay& 0.422 & 27.7\% & 60.7\%\\
      - w/o Both & 0.378 & 24.3\% & 52.9\%\\\midrule
      \textbf{Fonte with SBFL ($\alpha=0, \tau=\mathtt{max}, \lambda=0.1$)} & \textbf{0.481} & \textbf{32.0\%} & \textbf{68.9\%} \\
      - w/o Ranking-based Voting Power & 0.446 & 29.1\% & 61.7\%\\
      - w/o Depth-based Decay & 0.447 & 29.1\% & 64.6\%\\
      - w/o Both & 0.414 & 27.7\% & 55.3\%\\\bottomrule
    \end{tabular}}
\end{table}

\noindent \textbf{Results} \Cref{tab:ablation} presents the performance evaluation of \name with key features ablated. The absence of Ranking-based Voting Power leads to a 10.8\% decrease in performance for \name-IRFL, and a 7.3\% decrease for \name-SBFL, in terms of MRR. Similarly, removing Depth-based Decay causes a 6.8\% performance drop with \name-IRFL and a 7.3\% drop with \name-SBFL. Eliminating both features results in a reduction of MRR by 16.6\% for \name-IRFL and 13.9\% for \name-SBFL. Specifically, for \name combined with IRFL, the lack of Ranking-based Voting Power is particularly detrimental to performance, while the removal of Depth-based Decay, although less impactful, still results in a significant performance decline. This pattern is consistent when \name is combined with SBFL, especially in terms of Acc@5. Collectively, these findings highlight the crucial role that both Ranking-based Voting Power and Depth-based Decay play in the overall performance of \name, with Rank-based Voting Power playing a more critical role in enhancing performance.

\noindent\textbf{Answer to RQ1-3}: The contribution of Rank-based Voting Power and Depth-based Decay to \name's efficacy is considerable: disabling both features results in up to a 16.6\% decrease in MRR.

% \begin{table*}[t]
%   \centering
%   \caption{Comparison of the performance of \name (with $\alpha=0, \tau=\text{max}, \lambda=0.1$) to other commit ranking techniques applied to two commit search space $C_{BIC}$ (after Stages 1 and 2 of \name) and $C$, respectively. The evaluation is performed on two sets of subjects: the dataset from Wen et al. (67 subjects) and our manually curated set of 63 subjects. The performance was measured using MRR (Mean Reciprocal Rank) and accuracy@k, where k is 1, 2, 3, 5, and 10.}
%   \scalebox{0.87}{
%     \begin{tabular}{l|r|rrrrr|r|rrrrr|r|rrrrr}
%       \toprule
%       Subjects & \multicolumn{6}{c|}{\textbf{All (\# subjects = 130)}} & \multicolumn{6}{c|}{From Wen et al.~\cite{Wen2019} (\# subjects = 67)} & \multicolumn{6}{c}{Manually Curated (\# subjects = 63)}\\\midrule
%       \multirow{2}{*}{Metric} & \multirow{2}{*}{MRR} & \multicolumn{5}{c|}{Accuracy} & \multirow{2}{*}{MRR} & \multicolumn{5}{c|}{Accuracy} & \multirow{2}{*}{MRR} & \multicolumn{5}{c}{Accuracy} \\\cmidrule{3-7}\cmidrule{9-13}\cmidrule{15-19}
%           &       &     @1 &     @2 &     @3 &     @5 &     @10 &       &     @1 &     @2 &     @3 &     @5 &     @10 &       &     @1 &     @2 &     @3 &     @5 &     @10 \\\midrule
%       \multirow{2}{*}{\name} & \multirow{2}{*}{\textbf{0.528}} &     \textbf{47} &     \textbf{66} &     \textbf{85} &     \textbf{98} &     \textbf{110} & \multirow{2}{*}{0.324} &     \multirow{2}{*}{ 9} &     \multirow{2}{*}{19} &     \multirow{2}{*}{29} &     \multirow{2}{*}{38} &      \multirow{2}{*}{47} & \multirow{2}{*}{0.745} &     \multirow{2}{*}{38} &     \multirow{2}{*}{47} &     \multirow{2}{*}{56} &     \multirow{2}{*}{60} &      \multirow{2}{*}{63} \\
%                        &               & {\scriptsize (36\%)} &	{\scriptsize (51\%)} &	{\scriptsize (65\%)} &	{\scriptsize (75\%)} &	{\scriptsize (85\%)} & &  &  &  &  &  & &  &  &  &  &  \\
%                   \midrule
%     \multicolumn{19}{l}{\textbf{Other Techniques (on $C_{BIC}$)}} \\\midrule
%             Bug2Commit & 0.380 & 27 & 42 & 64 & 85 & 96& 0.235 &      7 &     13 &     19 &     26 &      33  & 0.534 &     20 &     29 &     45 &     59 &      63\\
%             FBL-BERT & 0.338 & 27 & 40 & 47 & 69 & 90& 0.158 &      5 &      9 &     11 &     14 &      27  & 0.529 &     22 &     31 &     36 &     55 &      63\\

%             Random Baseline& 0.218 &      3 &     19 &     41 &     65 &      75 & 0.065 &      0 &      2 &      4 &      6 &      12  & 0.381 &      3 &     17 &     37 &     59 &      63\\
%             Theoretical Lower Bound & 0.145 &      3 &      8 &     19 &     41 &      71 & 0.039 &      0 &      1 &      2 &      4 &       8  & 0.258 &      3 &      7 &     17 &     37 &      63\\\midrule
%         \multicolumn{19}{l}{\textbf{Other Techniques (on $C$)}} \\\midrule
%           Bug2Commit & 0.155 & 11 & 18 & 22 & 25 & 39 & 0.123 &      4 &      7 &      9 &     11 &      16 & 0.189 &      7 &     11 &     13 &     14 &      23\\
%           FBL-BERT & 0.037 & 1 & 3 & 5 & 7 & 10 & 0.037 &      1 &      2 &      2 &      3 &       4 & 0.036 &      0 &      1 &      3 &      4 &       6\\
%           Random Baseline & 0.002 &      0 &      0 &      0 &      0 &       0 & 0.002 &      0 &      0 &      0 &      0 &       0 & 0.002 &      0 &      0 &      0 &      0 &       0\\
%           Theoretical Lower Bound & 0.001 &      0 &      0 &      0 &      0 &       0 & 0.001 &      0 &      0 &      0 &      0 &       0 & 0.001 &      0 &      0 &      0 &      0 &       0\\\midrule
%           \multicolumn{19}{l}{\textbf{Ablation Study for \name}} \\\midrule
%           Skip Stage 2 & 0.490 &     39 &     64 &     82 &     97 &     110  & 0.317 &      9 &     18 &     28 &     37 &      47 & 0.675 &     30 &     46 &     54 &    60 &      63 \\
%           Use Equal Vote (No FL) & 0.436 &     39 &     56 &     67 &     79 &      88  & 0.193 &      7 &      9 &     12 &     19 &      25 & 0.694 &     32 &     47 &     55 &    60 &      63 \\
%           Max Aggr. (\Cref{eq:score_max}) & 0.317 &     17 &     36 &     50 &     73 &      97 & 0.142 &      0 &      5 &      9 &     18 &      34  & 0.503 &     17 &     31 &     41 &     55 &      63\\
%       \bottomrule
%     \end{tabular}}
%   \label{tab:RQ1-ranking}
% \end{table*}

\subsection{\textbf{RQ2. Comparison with Other Techniques}}


\noindent \textbf{Evaluation Protocol} We compare the BIC ranking performance of \name against various commit scoring baselines. To guarantee a fair comparison focused solely on the scoring mechanisms of each method and \name, we apply the baseline techniques within the same reduced BIC search space, $C_{BIC}$. The baselines include a general FL score aggregation method and two state-of-the-art IR-based techniques:

\begin{itemize}[leftmargin=1em]
% This strategy randomly shuffles the commits in the reduced search space. 
\item Max Aggregation of FL Scores: In Orca~\cite{Bhagwan2018}, the file-level FL scores are converted into the commit level using max-aggregation, that is, the highest FL score among all files changed by the commit is assigned as the commit's score. This max-aggregation method is widely utilised in FL techniques to bridge differing granularities between initial FL scores and the targeted FL granularity~\cite{Sohn2017, Lou2020}, such as from statements to methods or files to components. To illustrate and evaluate this concept, we adopt a scoring model where the score of a commit is determined by the maximum FL score among the code elements it alters:
\begin{equation}
  commitScore(c) = \max_{e \in E_{F}^c}susp(e)
\label{eq:score_max}
\end{equation}

% \footnote{https://anonymous.4open.science/r/fbl-bert-700C}
\item Bug2Commit~\cite{Murali2021}: Bug2Commit is a state-of-the-art IR-based
BIC identification method for large-scale systems, leveraging various
features of commits and bug reports. In our implementation of Bug2Commit, we opt for the Vector Space Model (VSM) because using a word-embedding model would necessitate an extra dataset comprising bug reports and commits for training. Following the approach detailed in the original study, we employ BM25~\cite{robertson1995okapi} for vectorisation. For the tokenisation process, the Ronin tokeniser is selected, recognised as the most sophisticated option available in \texttt{Spiral}~\cite{spiral2018}\footnote{https://github.com/casics/spiral}. 
We consider two features from each commit: the text of the commit message and the names of files that were altered. To represent bug reports, we evaluate two distinct configurations:
\begin{itemize}
\item Bug2Commit$_{report}$: Utilizes (1) the title and (2) the content of the bug report, both crafted by humans.
\item Bug2Commit$_{report+symptom}$: Builds on the previous configuration by additionally incorporating (3) the observed failure symptoms, such as exception messages and stack traces from failed test cases.
\end{itemize}

% (1) the failure symptoms including the exception message and stack traces from failed test cases as well as (2) the title of the bug report and (3) the content of the bug report.
% https://littlefoxdiary.tistory.com/12
\item FBL-BERT~\cite{Ciborowska2022}: FBL-BERT is a recently proposed changeset
localisation technique based on a pre-trained BERT model called
BERTOverflow~\cite{tabassum2020code}. Given a bug report, it retrieves the
relevant changesets using their scores obtained by the BERT-based model. We
fine-tune the model using the training dataset from the \texttt{JDT} project,
which is the largest training dataset provided by their repository\footnote{We
confirm that the model fine-tuned with \texttt{JDT} performs better than that
fine-tuned with \texttt{ZXing}, which has the smallest training dataset.}: this
is because no such training data is available for our target projects. We use
the ARC changeset encoding strategy, which categorises the lines in the changeset into Added, Removed, and Context groups. This method has been demonstrated to perform the best for
changeset-level retrieval in the original study~\cite{Ciborowska2022}. As
Defects4J contains the link to the original bug report for every bug, we use
the contents of the original bug report as an input query.
% \item HMCBL~\cite{Du2023}: \fixme{Add description about HMCBL} \fixme{Add description about HMCBL}\fixme{Add description about HMCBL}\fixme{Add description about HMCBL}\fixme{Add description about HMCBL}\fixme{Add description about HMCBL}\fixme{Add description about HMCBL}\fixme{Add description about HMCBL}\fixme{Add description about HMCBL}\fixme{Add description about HMCBL}\fixme{Add description about HMCBL}
\end{itemize}


\begin{table}
  \caption{Comparision of \name against commit scoring baselines. The methods marked with \dag utilise bug reports as input.}
  \label{tab:comparision}
  \centering
  \scalebox{1.00}{
  \begin{tabular}{l|rrr}
    \toprule
 &  MRR & Acc@1 & Acc@5 \\
    \midrule
      \textbf{Fonte with SBFL} & \textbf{0.481} & \textbf{32.0\%} & \textbf{68.9\%}\\
      \textbf{Fonte with IRFL} \dag & \textbf{0.453} & \textbf{30.6\%} & \textbf{62.6\%}\\\midrule
      Bug2Commit$_{report+symptom}$ & 0.330 & 17.0\% & 57.3\%\\
      Bug2Commit$_{report}$ \dag & 0.318 & 17.5\% & 54.4\% \\
      Max Aggregation of SBFL Scores & 0.288 & 12.1\% & 48.5\% \\
      Max Aggregation of IRFL Scores \dag & 0.274 & 11.7\% & 46.6\% \\
      FBL-BERT \dag & 0.246 & 14.1\% & 38.3\% \\\bottomrule
    \end{tabular}}
\end{table}


\noindent \textbf{Results} \Cref{tab:comparision} presents the ranking performance of the commit scoring baselines, sorted in descending order of MRR, along with the performance of \name. Among the baselines, Bug2Commit performs the best when utilising both the human-written bug report and failure symptoms as input to find the relevant commit. However, the results show that \name outperforms all scoring baselines across every evaluation metric. Specifically, \name with SBFL and IRFL demonstrates at least 45.8\% and 37.3\% higher MRR, respectively, compared to the baselines. This superior performance of \name holds true for any hyperparameter configuration, as shown in \Cref{tab:fonte_results}.

Furthermore, although three baselines (marked with \dag) use the same input data, i.e., the bug report, \name with IRFL achieves significantly superior performance. This highlights the effectiveness of \name's approach in converting code-level FL scores to commit-level scores. This is particularly evident when comparing \name with the max aggregation scheme, as they differ only in how the initial IRFL scores (in this case, Blues) are converted to the commit level.

% with its best hyperparameter setting ($\alpha=0$, $\tau=\text{max}$, $\lambda=0.1$) and other baseline ranking techniques, Bug2Commit and FBL-BERT, in ranking commits in $C_{BIC}$ for all subjects. We provide the breakdown of the results based on the source of the datasets, Wen et al.~\cite{Wen2019} and our manual curation (see Section~\ref{sec:dataset}), because they have different size distributions of the reduced search space $C_{BIC}$. Our manually created dataset contains only the subjects with $|C_{BIC}| \leq 10$, so that the worst rank of the BIC in $C_{BIC}$ is still within the top 10.
% Furthermore, in addition to the performance of Bug2Commit and FBL-BERT, we also provide a random baseline, which involves randomly shuffling the commits in $C_{BIC}$ and ranking them, and a theoretical lower bound, which assigns the worst possible rank to the actual BIC, to assist readers in comprehending the results for each of the datasets.

\noindent\textbf{Answer to RQ2} The scoring approach of \name outperforms both the max-aggregation method for FL scores and the state-of-the-art IR-based techniques. Notably, when \name is paired with IRFL, which relies solely on the bug report for input, it achieves superior results compared to other methods that utilise the same input data.

% \subsection{\textbf{RQ3. What is the impact of FL accuracy on the performance of \name?}}

\subsection{\textbf{RQ3. Standard Bisection vs. Weighted Bisection}}

\noindent \textbf{Evaluation Protocol} We simulate the standard and weighted bisection algorithms on all target bugs, assuming that the bug-revealing tests can perfectly reveal the existence of bugs. As commit scores, we use the best-performing configuration found in RQ1: \name with SBFL ($\alpha= 0$, $\tau=\mathtt{max}$, $\lambda=0.1$). We report how many search iterations until finding the BIC can be saved by using the weighted bisection algorithm compared to the standard bisection.

\begin{figure}[t]
  \centerline{\includegraphics[width=\linewidth]{figures/RQ2_cost_saving_by_weighted_bisection_compared_to_standard_bisection_on_C.pdf}}
  \caption{The number of saved search iterations required until finding the BIC using the weighted bisection compared to the standard bisection on the \textbf{entire} commit history, $C$}
  \label{fig:RQ2-all}
\end{figure}

\begin{figure}[t]
  \centerline{\includegraphics[width=\linewidth]{figures/RQ2_cost_saving_by_weighted_bisection_compared_to_standard_bisection_on_C_BIC.pdf}}
  \caption{The number of saved search iterations required until finding the BIC using the weighted bisection compared to the standard bisection on the \textbf{reduced} commit history, $C_{BIC}$}
  \label{fig:RQ2-reduced}
\end{figure}


\noindent \textbf{Results} \Cref{fig:RQ2-all} presents a sorted bar chart that shows the number of saved search iterations for all subjects until finding the BIC using the weighted bisection algorithm with \name-generated scores, compared to the standard bisection search on the entire commit history. The weighted bisection can reduce the search cost for approximately 98\% (202 out of 206) of the cases, saving up to 11 search iterations. On average, the number of iterations is reduced by 6.26, resulting in only 40\% of the iterations required by the standard bisection approach. Notably, there is no case where the weighted bisection degrades the performance compared to the standard bisection.

For a more conservative comparison, the weighted bisection algorithm using \name-generated scores is compared against the standard bisection search when both are applied to the reduced search space, $C_{BIC}$. \Cref{fig:RQ2-reduced} shows that the weighted bisection can reduce the number of required search iterations for 133 out of 206 subjects (64.6\%), while the number of iterations is increased in only 17 out of 206 subjects (8.3\%). In the remaining 27.2\% of cases, the number of iterations is the same as the standard bisection. The results demonstrate that the commit score information can guide the search process more efficiently. To ensure that the median of the number of saved iterations is positive, indicating a performance improvement, a one-sided Wilcoxon signed-rank test~\cite{Wilcoxon1992} is performed. The null hypothesis is that the median is negative, implying performance degradation. The obtained p-value of $1.69*10^{-19}$ allows the rejection of the null hypothesis in favor of the alternative that \emph{the median of the number of saved iterations is greater than zero, supporting the performance improvement}.

\begin{figure}[t]
  \centerline{\includegraphics[width=\linewidth]{figures/RQ3_saved_iterations.pdf}}
  \caption{Regression plot illustrating the relationship between the effectiveness of \name, measured by the normalized rank of the BIC, and the effectiveness of the weighted bisection technique, measured by the number of iterations saved compared to the standard bisection approach}
  \label{fig:RQ3-bic-rank-and-wb}
\end{figure}

We conducted an additional analysis to understand why the weighted bisection technique degrades the search efficiency for 17 subjects (8.3\%). In these cases, we found that the BIC was not ranked well by \name, either not being among the top 10 or even the top 50\% candidates. \Cref{fig:RQ3-bic-rank-and-wb} illustrates that the number of saved search iterations by using the weighted bisection exhibits a negative correlation with the normalised BIC rank, with a Pearson correlation coefficient of -0.64. This finding collectively suggests that more accurate commit scores can benefit the search process of the weighted bisection.

\noindent\textbf{Answer to RQ3} The combination of weighted bisection and \name-generated commit scores can reduce the BIC search cost for 98\% of the studied bugs, compared to the standard bisection applied to the entire commit history. On average, it saves 6.3 iterations. When the bisection is performed solely on the reduced set of candidate commits, the weighted bisection technique saves the number of search iterations in 65\% of cases, while increasing it in only 8\% of cases where the commit scores are of low quality.


% \subsection{Impact of FL Accuracy on \name \fixme{should remove or move}}

% % to protocol?

% To see how the accuracy of FL affects the performance of \name on each
% individual subject, we provide less accurate FL results to \name and observe
% how it affects the ranking performance. \refi{(note) As a measure of the FL accuracy,
% we use the highest rank of the buggy methods, where the buggy methods are
% defined as the methods fixed by the developer patch.}
% We intentionally weaken the test suite
% by removing some of the relevant passing test cases, as it is known that the
% accuracy of SBFL is highly dependent on the quality of the used test
% suite~\cite{Perez2017}. By doing so, we limit the test suite to only the test
% cases that are contained in the failing test classes. For example, in the case
% of \Cref{tab:relevant}, the relevant test cases are limited to the last
% four test cases that are in the \texttt{TestFieldUtils} class containing the
% failing test case.

% \begin{figure}[t]
%   \centerline{\includegraphics[width=0.95\linewidth]{figures/RQ3_comparison.pdf}}
%   \caption{BIC ranks of \name with the more and less accurate FL results}
%   \label{fig:RQ3}
% \end{figure}

% Among the 99 out of 130 subjects whose sets of relevant test cases are reduced,
% we observe that, in 57 subjects, the FL accuracy (in terms of the highest rank
% of buggy methods) is decreased as a result. For those 57 subjects, we see
% whether the performance of \name is affected by the accuracy of FL. In
% \Cref{fig:RQ3}, the $x$- and $y$-axis represent the BIC ranks produced by
% \name with the more (original) and less accurate FL results, respectively.
% Green markers (above the dashed line) represent the cases where the better FL
% yields the better BIC rank, while red markers indicate the opposite. The overall
% tendency is that higher FL accuracy leads to a better ranking performance of
% \name, as shown by the fact that the number of green dots above the dotted line
% is much higher than the number of red dots below the line.\footnote{\name with the worse FL results still outperforms all baselines in RQ1.}
% The one-sided Wilcoxon signed rank test for the paired ranking samples also
% results in the p-value of $2.56 * 10^{-6}$ showing that the median rank
% difference is greater than zero when the FL accuracy increases.

% \noindent\textbf{Answer to RQ3}: \name performs better when the FL results used
% as its input become better. Consequently, we expect that \name can benefit from
% more precise and sophisticated FL techniques in the future.

\section{Application to Industry Software}
\label{sec:industry}

\product is a large-scale
commercial software that consists of more than 10M lines of C++ and C. In the
CI system of \product, multiple commits that have individually passed the
pre-submit testing are merged into the delivery branch and tested together using a more extensive test suite on a daily basis. Considering
the set of multiple commits as a single batch, this is a type of \emph{Batch
Testing}~\cite{Najafi2019}. While batch testing reduces the overall test
execution cost for \product, it also has some practical drawbacks: when a test
fails, it is not immediately clear which change in the batch is responsible for
the failure~\cite{Beheshtian2022}. The current CI system of \product identifies the BIC in the batch
using automatic bisection to aid the bug assignments~\cite{Bach2022}. However, each individual
inspection during the bisection can take up to several hours due to the
compilation, installation, and test execution cost, resulting in severe
bottlenecks in the overall debugging process. The bottleneck can be
particularly problematic if integration or system-level tests fail.

\begin{figure}[h]
  \centerline{\includegraphics[width=\linewidth]{figures/batch_testing.pdf}}
  \caption{Simplified batch testing scenario}
  \label{fig:batch_testing}
\end{figure}

This motivates us to see whether \name and its weighted bisection can reduce
the number of bisection iterations. To evaluate the effectiveness of
applying \name, we collect 23 batch testing failures that occurred from July to
August 2022 and their BICs identified by the bisection from the internal CI
logs of \product. Using the data, we first check if \name can find the BIC
inside the batch accurately (\Cref{fig:batch_testing}). As the test
coverage of \product is regularly and separately updated instead of being
measured at each of the batch testings, we use the latest line-level coverage
information to calculate the Ochiai scores. Note that we do not need to compute
the Ochiai scores for all lines, but only the lines covered by the failing
tests. When applying \name, depth-based voting decay is not used
($\lambda = 0$) because all candidate commits are submitted on the same day and
have not yet been merged into the main codebase. For the other
hyperparameters, we use $\alpha=1$ and $\tau=\mathtt{max}$ which performed the best with $\lambda=0$ in our experiment with Defects4J.

\begin{table}[t]
  \centering
  \caption{Evaluation of \name on the 23 batch testing failures of \product}
  \scalebox{1.00}{
    \begin{tabular}{l|r|rrrrr}
        \toprule
     &     \multirow{2}{*}{MRR} & \multicolumn{5}{c}{Accuracy}\\\cmidrule{3-7}
     &           &     @1 &     @2 &     @3 &     @5 &     @10 \\\midrule
     \multirow{2}{*}{\name}     &   \multirow{2}{*}{0.600}   & 10  & 14 & 15 & 20 & 23\\
             &    & (43\%) & (61\%) & (65\%) & (87\%) & (100\%) \\\midrule
    Random & 0.110 & 0 & 0 & 0 & 1 & 17\\\bottomrule
    \end{tabular}}
  \label{tab:industry}
\end{table}

\Cref{tab:industry} shows the BIC ranking performance of \name in terms of
MRR and Accuracy@n. While each batch contains 18.48 commits on average, \name
can locate the actual BIC within the top 1 and 5 for 43\% and 87\% of the
failures, respectively. Compared to the random baseline, it achieves 5.5-fold
increase in MRR.
Further, we also report that the weighted bisection can reduce the
bisection iterations for 18 out of 23 cases (78\%), while it increases the cost
in only three cases (13\%). Based on this result, we plan to incorporate weighted
bisection into the CI process of \product, which is expected to save 32\% of
required iterations. Considering that each iteration can take up
to several hours, we expect a significant reduction in the average BIC
identification cost for \product in the long run.

\section{Threats to Validity}
\label{sec:threats}

% https://www.scribbr.com/methodology/internal-validity/
Threats to internal validity concern factors that can affect how confident we
are about the causal relationship between the treated factors and the effects.
\name relies on widely-adopted open-source tools to establish $\mathsf{Cover}$ and $\mathsf{Evolve}$
relations to ensure the chain of causality between the test failure and BIC identification.
We compare the validity of these different code history mining tools, that are used to build the $\mathsf{Evolve}$ relation, and use the most reliable one, \texttt{git log}, for the later experiments. We also make the performance results with the other tools publicly available in our artifact for future scrutiny, Additionally, as the baseline techniques rely on multiple sources of 
information, such as bug reports, we choose Defects4J as our benchmark as it provides 
well-established links between real-world bug reports and the buggy version, not to mention
human-written bug-revealing test cases that withheld scrutiny from the community.

Threats to external validity concern factors that may affect how well our findings
can be generalised to unseen cases. Our key findings are primarily based on experiments with
the open-source Java programs in Defects4J. Since they are not representative of
the entire population of real-world programs, only further evaluations can strengthen
our claim of generalisation. We tried to support our claim by evaluating \name with
industry-scale software written in C and C++. We do note that \name does not generalise
to bugs that are caused by non-executable files, such as configuration changes, as its
base assumption is that the test failure is caused by a bug in the source code.
We leave extension of \name to bugs caused by non-executable changes as our primary
future work.

Threats to construct validity concern how well the used metrics measure the properties
we aim to evaluate. We adopt two ranking-evaluation metrics, MRR and Accuracy@n, to
evaluate \name: both have been widely used in the IR and SE literature. Since they are based on absolute ranks, we do note that the results can be overrated
when the number of ranking candidates is small. To mitigate the threat, we also
present the expected and worst values for the measures as baselines.


\section{Related Work}
\label{sec:related_work}

Locus~\cite{Wen2016} is the first work that proposed to localise the bug at the
software change level. It takes a bug report as an input query and locates the
relevant change hunk based on the token similarities. IR-based techniques, such
as Locus, and \name can complement each other depending on circumstances. When
the failure cannot be reproduced from the bug report, IR-based techniques can
be used instead of \name. However, if the coverage of the failing and passing
tests are available, we can apply \name with SBFL to more precisely rank the
commits without relying on IR.
ChangeLocator~\cite{Wu2017} aims to find a BIC for crashes using the call stack
information. It is a learning-based approach that requires data from fixed
crashes. Unlike ChangeLocator, \name is not limited to crashes and can be
applied to general failures. Orca~\cite{Bhagwan2018} takes symptoms of bugs,
such as an exception message or customer complaints, as an input query and
outputs a ranked list of commits ordered by their relevance to the query. It
uses the TF-IQF~\cite{Yang2008} to compute the relevance scores of files, and
aggregate them to a commit level. Subsequently, it uses machine learning to
predict the risk of candidate commits for breaking ties.
Bug2Commit~\cite{Murali2021} uses multiple features extracted from bug reports
and commits, and aggregates all features by taking the average of their vector
representations. Although Bug2Commit uses an unsupervised learning approach, it
needs the historical data of project-specific bug reports and commits to train
the word embedding model. FBL-BERT~\cite{Ciborowska2022} retrieves the relevant
changeset for the input bug report using a fine-tuned BERT model that can
capture the semantics in the text. It proposes fine-grained changeset encoding methods and accelerates the retrieval by offline indexing~\cite{johnson2019billion}. The major difference between \name and the techniques
mentioned above is that \name does not require any training. Further, \name can
be combined with any code-level FL technique, without being coupled to
specific sources of information, as long as the coverage of failing executions
is available.

The weighted bisection algorithm we propose is similar to FACF (Flaky Aware Culprit Finding)~\cite{Henderson2023}, which formulates the \emph{flake-aware} bisection problem as a Bayesian inference, in that both guide the bisection process based on the probability of commits being a source of test failure. The difference between the two algorithms is that ours uses commit scores from \name to establish the initial probability distribution, while  FACF updates the probability based on the test results during the search taking into account the potential for flakiness. The original work notes that FACF can take into account any prior information about the bug inducing change in the form of an initial probability distribution. Hence, we believe that the commit scores generated by \name can be used as an effective prior distribution for the FACF framework.

There exist studies that are highly relevant to \name despite not being specifically about the BIC identification domain. FaultLocator~\cite{zhang2011localizing} is similar to \name as both use
code-level FL scores to identify suspicious changes. FaultLocator combines spectrum information with the change impact
analysis to precisely identify the failure-inducing \emph{atomic} edits out
of all edits between two versions, whereas \name aims to pinpoint BICs in the
commit history. WhoseFault~\cite{Servant2012} is a method that utilises code-level FL scores and commit history to determine the developer responsible for a bug. While it provides insights into the assignment of bugs, it does not specifically target BIC identification. As a result, it cannot be directly compared with \name in our evaluation, nor can it be integrated with our bisection algorithm. Our belief is that accurately identifying the BIC can also be used to find the developer responsible for fixing the bug, based on the authorship of the changes, in addition to helping developers understand the context in which the failure occurred.



% To use FaultLocator, we need to identify the earliest version that a test starts to fail. In that sense, FaultLocator and \name can be used together.

% An et al.~\cite{An2021} proposed to reduce the search space of BIC only using the coverage of failing executions without requiring any input that needs human effort, such as bug report or fixing commit. They only included the commits which either introduced or modified the program elements covered by failing executions in the BIC search space. The evaluation using Defects4J benchmark showed that the substantial amount of commits, 87.6\% on average, can be removed from the BIC search space. However, as they also pointed out, if the failing executions have broad coverage, there could be still a large number of commits in the reduced search space so that it is hard to accurately pinpoint the BIC.


% Locus and FBL-BERT assumes that there is a bug report written in natural language.
% While ChangeLocator uses the crash call stack trace, which can be automatically obtained, it can be applied only to the crash errors.
% Orca and Bug2Commit needs training step which requires a large amount of historical data. For example, Orca uses 92M commits to train its commit risk evaluation model. Bug2Commit need a global pool of commits and bug reports (and their features) and trains word weights for the vectorizer. For example, in their evaluation, the word embedding is trained using lots of crash reports (or regression reports) and more than XXX commits. However, it limits it applicability to where such historical is not available.


% - Change Locator, Orca: need supervised learning
% - Locus, FBL-BERT: need human-written natural language bug report
% - Bug2Commit: good baseline! Since the implementation of Bug2Commit is not publicly available, we make our own prototype of Bug2Commit. As it was developed for the use in Facebook, As a feature of bug report, we use the name of the failing tests, crash message and crash call stack (only when crash), and the error message (regression error). As a feature of commit, we use the commit message, changed hunk (with the context), and .... but you know what? Bug2commit also requires lots of training data to learn the word embedding

% Bug2Commit is the state-of-the-art IR techniques developed by Facebook.
% if it appears in a short feature as opposed to a lengthy feature.
% For this purpose, we use a BM25 [27] based vectorizer rather
% than tf-idf. BM25 is a popular scoring function used by search
% engines such as Lucene [23], an


% \begin{itemize}
%     \item Locus~\cite{Wen2016}: \textbf{Unlike our work, this can be only applied when human-written bug reports are availabe.}
%     \item ChangeLocator~\cite{Wu2017}: \textbf{while it only targets the crashing error,} \name can be applied any general faults as it use the coverage information instead of the crash call stack. \textbf{it requires training step. A supervised learning-based approach}
%     \item Orca~\cite{Bhagwan2018} \textbf{Orca also uses a learning
%     approach to break ties in ranking}

%     \refi{- Input query: A symptom of the bug (the name of the anomalous probe, an exception message, a stack-trace, the words of a customer complaint (NL))}

%     \refi{- Documents: commit (Names of the files changed, commit and review comments (NL), modified code, modified configuration parameters)}

%     \refi{- Output: a ranked list of commits}

%     \refi{- Method: Use the \textbf{machine-learning based} model of commit risk prediction trained on \textbf{two-years of history}, for each candidate commit, for each file (differnce set) and token in the symtom, calculate the td-idf score of the tuple. IDF is calculated \textbf{by analysing Orion's logs}. The commit score will be the maximum value of the TF-IDF of the token in symtoms}

%     \refi{- Limitation: \textbf{requires significant amount of historical data to train the machine-learning model and to maintain the idf value of tokens}}
%     - Differential Code Analaysis for search-space pruning: Performing a syntactic analysis on the differences that extracts the added, modified, or deleted source code by a commit.

%     - The Build Provenance Graph for search-space explosion: The graph captures dependencies between various builds.

%     - Input: A search query that describes symtoms of the problem, consistien of probe names, exception texts, log messages, etc

%     - Algorithm: find all builds related to the symptomatic build, enumerate all commits that created the builds, calculate the relevance score between the commits and symtom.
%     \item Bug2Commit~\cite{Murali2021} \refi{In the training phase it continously gathers a global pool of commits and bug reports, and trains word weights for the vectorizer.} \textbf{NEED TRAINING ALTHOUGH IT IS UNSUPERVISED LEARNING} what is there is no such corpus of bug rerports?

% \end{itemize}

% % by sm
% https://ieeexplore.ieee.org/abstract/document/9426017

\section{Conclusion}
\label{sec:conclusion}

This paper proposes \name, a BIC identification technique that is available upon the observation of a failure.
It prunes the BIC search space using failure coverage and the syntactic analysis of commits, and assigns scores
to the remaining commits using the FL scores as well as change histories of code elements. Our experiments with
206 bugs in Defects4J show that \name can effectively identify BICs with an MRR of 0.481, which significantly
outperforms the baselines including state-of-the-art BIC identification techniques. The findings indicate that \name is an effective algorithm to translate code-level suspiciousness scores into commit-level scores. Along with \name, we also
propose the weighted bisection to accelerate the BIC search utilising the commit score information and show
that it can save the search cost in 98\% of the studied cases compared to the standard bisection. Finally,
the application of \name to a large-scale industry software \product shows that \name can successfully
reduce the cost of BIC identification in a batch-testing CI scenario. 

% \section*{Acknowledgments}
% This should be a simple paragraph before the References to thank those individuals and institutions who have supported your work on this article.

% \nocite{*}
\bibliographystyle{IEEEtran}
\bibliography{references}


% \newpage

% \section{Biography Section}
% If you have an EPS/PDF photo (graphicx package needed), extra braces are
%  needed around the contents of the optional argument to biography to prevent
%  the LaTeX parser from getting confused when it sees the complicated
%  $\backslash${\tt{includegraphics}} command within an optional argument. (You can create
%  your own custom macro containing the $\backslash${\tt{includegraphics}} command to make things
%  simpler here.)
 
% \vspace{11pt}

% \bf{If you include a photo:}\vspace{-33pt}
% \begin{IEEEbiography}[{\includegraphics[width=1in,height=1.25in,clip,keepaspectratio]{fig1}}]{Michael Shell}
% Use $\backslash${\tt{begin\{IEEEbiography\}}} and then for the 1st argument use $\backslash${\tt{includegraphics}} to declare and link the author photo.
% Use the author name as the 3rd argument followed by the biography text.
% \end{IEEEbiography}

% \vspace{11pt}

% \bf{If you will not include a photo:}\vspace{-33pt}
% \begin{IEEEbiographynophoto}{John Doe}
% Use $\backslash${\tt{begin\{IEEEbiographynophoto\}}} and the author name as the argument followed by the biography text.
% \end{IEEEbiographynophoto}




\vfill

\end{document}



\subsection{Error Gap Between Aligned and Misaligned Data}\label{subsec:proof-align-misalign}







\thmalignment*

\begin{proof}

For the aligned case, we can derive the mean squared error (MSE) as follows:
\begin{equation}\label{eq:mse_aligned}
    \mathrm{MSE}_\mathrm{aligned} = \inf_{\boldsymbol{\alpha} \in R^{m^P}, \boldsymbol{\beta} \in R^{m^S}} \|\mathbf{y} - \mathbf{X}^P \boldsymbol{\alpha} - \mathbf{X}^S \boldsymbol{\beta}\|
\end{equation}
The ordinary least squares (OLS) estimator of $\boldsymbol{\alpha}$ is given by:
\begin{equation}
    \hat{\boldsymbol{\alpha}} := (\mathbf{X}^{P \top} \mathbf{X}^P)^{-1} \mathbf{X}^P (\mathbf{y} - \mathbb{E}[\mathbf{R}] \mathbf{X}^S \boldsymbol{\beta}) 
\end{equation}
For a permutation matrix $\mathbf{R}$ under uniform distribution, we have $\mathbb{E}[\mathbf{R}] = \frac{1}{n}\mathds{1}^\top \mathds{1}$. Therefore:
\begin{equation}\label{eq:alpha_hat}
    \hat{\boldsymbol{\alpha}} = (\mathbf{X}^{P \top} \mathbf{X}^P)^{-1} \mathbf{X}^P (\mathbf{y} - \frac{1}{n} \mathds{1}^\top \mathds{1} \mathbf{X}^S \boldsymbol{\beta}) 
\end{equation}
The MSE for the misaligned case can be expressed as:
\begin{align}
    \mathrm{MSE}_{\mathrm{misaligned}} 
    & = \inf_{\boldsymbol{\beta}} \inf_{\boldsymbol{\alpha}} \mathbb{E}_\mathbf{R} \|\mathbf{y} - \mathbf{X}^P \boldsymbol{\alpha} - \mathbf{R} \mathbf{X}^S \boldsymbol{\beta}\|_2^2 \\
    & = \inf_{\boldsymbol{\beta}} \mathbb{E}_\mathbf{R} \|\mathbf{y} - \mathbf{X}^P \hat{\boldsymbol{\alpha}} - \mathbf{R} \mathbf{X}^S \boldsymbol{\beta}\|_2^2 \\
\end{align}
Substituting $\hat{\boldsymbol{\alpha}}$ from equation~\ref{eq:alpha_hat}, we obtain:
\begin{align}
    \mathrm{MSE}_{\mathrm{misaligned}} 
    & = \inf_{\boldsymbol{\beta}} \mathbb{E}_\mathbf{R} \left\|\mathbf{y} - \mathbf{X}^P (\mathbf{X}^{P \top} \mathbf{X}^P)^{-1} (\mathbf{X}^P \mathbf{y} - \mathbf{X}^P \frac{1}{n} 1^\top 1 \mathbf{X}^S \boldsymbol{\beta}) - \mathbf{R} \mathbf{X}^S \boldsymbol{\beta}\right\|_2^2 \\
    & = \inf_{\boldsymbol{\beta}} \mathbb{E}_\mathbf{R} \left\| (\mathbf{I} - \mathbf{X}^P (\mathbf{X}^{P \top} \mathbf{X}^P)^{-1} \mathbf{X}^P)\mathbf{y} + (\mathbf{X}^P (\mathbf{X}^{P \top} \mathbf{X}^P)^{-1} \mathbf{X}^P \frac{1}{n} \mathds{1}^\top \mathds{1} \mathbf{X}^S \boldsymbol{\beta}) - \mathbf{R} \mathbf{X}^S \boldsymbol{\beta}\right\|_2^2 
\end{align}
Since $\mathbf{X}^P (\mathbf{X}^{P \top} \mathbf{X}^P)^{-1} \mathbf{X}^P$ is a projection matrix that projects any vector onto the column space of $\mathbf{X}^P$, and $\mathbf{X}^S \boldsymbol{\beta}$ is orthogonal to the column space of $\mathbf{X}^P$, the term $\mathbf{X}^P (\mathbf{X}^{P \top} \mathbf{X}^P)^{-1} \mathbf{X}^P \frac{1}{n} \mathds{1}^\top \mathds{1} \mathbf{X}^S \boldsymbol{\beta} = 0$. Thus:
\begin{align}
    \mathrm{MSE}_{\mathrm{misaligned}}
    & = \inf_{\boldsymbol{\beta}} \mathbb{E}_\mathbf{R} \left\| (\mathbf{I} - \mathbf{X}^P (\mathbf{X}^{P \top} \mathbf{X}^P)^{-1} \mathbf{X}^P)\mathbf{y} - \mathbf{R} \mathbf{X}^S \boldsymbol{\beta}\right\|_2^2 \\
    & = \inf_{\boldsymbol{\beta}} \mathbb{E}_\mathbf{R} \left[\left\|\mathbf{R} \mathbf{X}^S \boldsymbol{\beta}\right\|_2^2 - 2\left[(\mathbf{I} - \mathbf{X}^P (\mathbf{X}^{P \top} \mathbf{X}^P)^{-1} \mathbf{X}^P)\mathbf{y}\right]^\top \mathbf{R} \mathbf{X}^S \boldsymbol{\beta} + \left\|(\mathbf{I} - \mathbf{X}^P (\mathbf{X}^{P \top} \mathbf{X}^P)^{-1} \mathbf{X}^P)\mathbf{y}\right\|_2^2\right]
\end{align}
By properties of permutation matrices:
\begin{equation}
    \mathbb{E}_\mathbf{R}\| \mathbf{R} \mathbf{X}^S \boldsymbol{\beta}\|_2^2 = \|\mathbf{X}^S \boldsymbol{\beta}\|_2^2; \; \mathbb{E}_\mathbf{R} [\mathbf{R}]= \frac{1}{n}\mathds{1}^\top \mathds{1}
\end{equation}
Therefore:
\begin{align}
    \mathrm{MSE}_{\mathrm{misaligned}}
    & = \inf_{\boldsymbol{\beta}} \left[\left\|\mathbf{X}^S \boldsymbol{\beta}\right\|_2^2 - 2\left[(\mathbf{I} - \mathbf{X}^P (\mathbf{X}^{P \top} \mathbf{X}^P)^{-1} \mathbf{X}^P)\mathbf{y}\right]^\top \frac{1}{n}\mathds{1}^\top \mathds{1} \mathbf{X}^S \boldsymbol{\beta} + \left\|(\mathbf{I} - \mathbf{X}^P (\mathbf{X}^{P \top} \mathbf{X}^P)^{-1} \mathbf{X}^P)\mathbf{y}\right\|_2^2\right]
\end{align}
Since $\mathbf{I} - \mathbf{X}^P (\mathbf{X}^{P \top} \mathbf{X}^P)^{-1} \mathbf{X}^P$ projects any vector onto the orthogonal complement of the column space of $\mathbf{X}^P$, the term $\left[(\mathbf{I} - \mathbf{X}^P (\mathbf{X}^{P \top} \mathbf{X}^P)^{-1} \mathbf{X}^P)\mathbf{y}\right]^\top \frac{1}{n}\mathds{1}^\top \mathds{1} \mathbf{X}^S \boldsymbol{\beta} = 0$. Hence:
\begin{align}
    \mathrm{MSE}_{\mathrm{misaligned}}
    & = \inf_{\boldsymbol{\beta}} \left[\left\|\mathbf{X}^S \boldsymbol{\beta}\right\|_2^2 + \left\|(\mathbf{I} - \mathbf{X}^P (\mathbf{X}^{P \top} \mathbf{X}^P)^{-1} \mathbf{X}^P)\mathbf{y}\right\|_2^2\right] \\
    & = \inf_{\boldsymbol{\beta}} \left\|\mathbf{X}^S \boldsymbol{\beta}\right\|_2^2 + \left\|(\mathbf{I} - \mathbf{X}^P (\mathbf{X}^{P \top} \mathbf{X}^P)^{-1} \mathbf{X}^P)\mathbf{y}\right\|_2^2 \\
\end{align}
The minimum is attained at $\boldsymbol{\beta} = \mathbf{0}$, yielding:
\begin{align}
    \mathrm{MSE}_{\mathrm{misaligned}}
    & = \left\|(\mathbf{I} - \mathbf{X}^P (\mathbf{X}^{P \top} \mathbf{X}^P)^{-1} \mathbf{X}^P)\mathbf{y}\right\|_2^2 \\
    & = \inf_{\boldsymbol{\alpha} \in \mathbb{R}^{m^P}, \boldsymbol{\beta} = \mathbf{0}} \left\|\mathbf{y} - \mathbf{X}^P \boldsymbol{\alpha} - \mathbf{X}^S \boldsymbol{\beta}\right\|_2^2 \\
\end{align}
Comparing with Equation~\ref{eq:mse_aligned}, we conclude:
\begin{equation}
    \mathrm{MSE}_{\mathrm{misaligned}} \geq \inf_{\boldsymbol{\alpha} \in \mathbb{R}^{m^P}, \boldsymbol{\beta} \in \mathbb{R}^{m^S}} \left\|\mathbf{y} - \mathbf{X}^P \boldsymbol{\alpha} - \mathbf{X}^S \boldsymbol{\beta}\right\|_2^2 = \mathrm{MSE}_{\mathrm{aligned}}
\end{equation}
\end{proof}





















\subsection{Approximation Capacity of Cluster Sampler}\label{subsec:proof-cluster-sampler}

\begin{definition}[Definition of optimal cluster sampler]
    Assume the inputs are uniformly bounded by some constant $B$. 
    The optimal cluster sampler is defined by the uniform equi-continuous cluster sampler function which achieves the minimal optimization loss for the prediction task in \cref{fig:leal-framework}.
    \begin{equation}
        \textrm{Optimal cluster sampler} := \arginf_{\textrm{Uniform equi-continuous cluster sampler}} \textrm{Loss}(\textrm{cluster sampler})
    \end{equation}
    The cluster sampler is defined over bounded inputs ($|X^P|_{\infty} \leq B, |X^S|_{\infty} \leq B$) from $\mathbb{R}^{m^P} \times \mathbb{R}^{n^S \times m^S}$ and output in $\mathbb{R}^{n^S}$.
\end{definition}

\begin{remark}
    The existence of such optimal cluster sampler is guaranteed by the boundedness and uniform equi-continuity of the set of cluster sampler functions. 
\end{remark}


\thmclustersampler*

\begin{proof}
    We just need to prove the statement for small $\epsilon \leq 6$.

    The input of cluster sampler is $1 \times m^P$ and output is $n^S \times m^S$, the final prediction is to generate a sample probabilities:
    \begin{equation}
        (n^S * m^S, 1 * m^P) \to (n^S * d, 1 * C) \to (n^S * C, 1 * C) \to n^S * 1. 
    \end{equation}

    Also, since there is no weight depends on dimension $n_2$, we can reduce the approximation statement to that there exists trainable weight such that the continuous function $h$ can be approximated:
    \begin{equation}
        (1 * m^S, 1 * m^P) \to (n^S * d, 1 * C) \to (n^S * C, 1 * C) \to 1 * 1. 
    \end{equation}

    Notice that the layer operation of secondary embedding and trainable centroids weights $(C \times d)$ is continuous and the pretrained encoder as a neural network (which is a universal approximator) can approximates any continuous function $f$ composited with inverse embedding. 
    For simplicity, we will consider $m^P = m^S = 1$. 
    For any continuous function $h(p, s) \in [0, 1]$,
    we just need to show there exists trainable weight $\theta_1$, $\theta_2$ such that 
    \begin{equation}
        f(p; \theta_1) \odot g(s; \theta_2) = \sum_{i=1}^C f_i(p; \theta_1) \odot g_i(s; \theta_2). 
    \end{equation}
    Here $f(p; \theta_1) \in \mathbb{R}^C$ is a function of $p$ parameterized by $\theta_1$ and $g(s; \theta_1) \in \mathbb{R}^C$ is a function of $s$ parameterized by $\theta_2$.  
    As any continuous function $f(p, s)$ has a corresponding Taylor series expansion, it means for any $\epsilon > 0$, there exists $C$ which depends on error $\epsilon$ such that
    \begin{equation}
        \sup_p \sup_s |h(p, s) -\sum_{i=1}^C pol_{1,i}(p) pol_{2,i}(s)| \leq \frac{\epsilon}{2}. 
    \end{equation}
    Furthermore, as polynomial functions are continuous function, therefore $f_i$ can be used to approximate the polynomial function $pol_{1, i}$ and $g$ can be used to approximate the polynomial function $pol_{2, i}$.
    \begin{align}
        \sup_p |pol_{1,i}(p) - f_i(p; \theta_1)| & \leq \frac{\epsilon}{6B} \\ 
        \sup_s |pol_{2,i}(s) - g_i(s; \theta_2)| & \leq \frac{\epsilon}{6B}. 
    \end{align}
    Here $B := \max(1, \sup_p \max_{i} |pol_{1, i}(p)|, \sup_s \max_{i} |pol_{2, i}(s)|).$ 
    We show that the cluster sampler is capable to approximate any desirable continuous cluster sampler. 
    \begin{equation}
        \sup_p \sup_s |h(p, s) -\sum_{i=1}^C f_i(p; \theta_1) g_i(s; \theta_2)| \leq \frac{\epsilon}{2} + \frac{\epsilon}{6B} * B + \frac{\epsilon}{6B} (B + \frac{\epsilon}{6B}) = \frac{5}{6} \epsilon + \frac{\epsilon^2}{36B^2} < \epsilon. 
    \end{equation}
    The last inequality comes from $\epsilon < 6$. 
    The universal approximation capacity of the cluster sampler is proved. 
\end{proof}

\begin{remark}
    Since we are working with a cluster sampler with specific manually designed structure, it mainly comes from the fact the student's t-kernel introduce a suitable implicit bias to more efficiently learn the cluster sample probability $(n_2 \times 1)$. 
\end{remark}

% !TEX root =  ../main.tex
\section{LinSEPAL-ADMM}\label{app:MADMM}
Let us recall below the nonsmooth Riemannian problem we have to solve.
\nonsmoothprob*

The structure of the objective in \eqref{eq:minKL}, separating into smooth (cf. \cref{prop:smoothness_and_differentiability}) and nonsmooth terms, makes the \emph{alternating direction method of multipliers} (ADMM, \citeSupp{boyd2011distributedSupp}) an appealing optimization framework for deriving a solution.
This is the rationale behind the general framework \emph{manifold ADMM} \citeSupp{kovnatsky2016madmmSupp}, that we decline to our setting in the following, thus obtaining the LinSEPAL-ADMM algorithm.

Starting from \eqref{eq:minKL}, we add a splitting variable $\Y \in \rmatdim$ to be optimized over the Euclidean space to handle the non-smooth term $h(\V)$:

\begin{equation}\label{eq:MADMM}
    \begin{aligned}
        \min_{\V \in \stiefel{\ell}{h}, \Y \in \rmatdim} & \quad \Tr{\left( \V^\top \covlow \V\right)^{-1} \covhigh} + \log\det{\V^\top \covlow \V} + \lambda \norm{\Y}_1 \, , \\
        \textrm{subject to} & \quad \Y - \D \odot \V=\zeros_{\ell \times h}.
    \end{aligned}
    \tag{P2}
\end{equation}

At this point, following \citeSupp{boyd2011distributedSupp}, by denoting by $\scaledU \in \rmatdim$ the scaled dual variable, and by $\rho \in \reall^+$ the ADMM stepsize, the scaled augmented Lagrangian reads as

\begin{equation}\label{eq:MADMMsAUL}
    L_{\rho}\left(\V, \Y, \scaledU \right) =  \Tr{\left( \V^\top \covlow \V\right)^{-1} \covhigh} + \log\det{\V^\top \covlow \V} + \lambda \norm{\Y}_1 + \frac{\rho}{2}\frob{\D\odot \V  - \Y + \scaledU}^2.
\end{equation}

Starting from \Cref{eq:MADMMsAUL}, the ADMM updates at the $k$-th iteration are
\begin{equation}\label{eq:MADMMrecursion_app}
    \begin{aligned}
        \V^{k+1} &= \argmin_{\V \in \stiefel{\ell}{h}} L_{\rho}\left(\V, \Y^k, \scaledU^k \right),\\
        \Y^{k+1} &= \argmin_{\Y \in \rmatdim} L_{\rho}\left(\V^{k+1}, \Y, \scaledU^k \right),\\
        \scaledU^{k+1} &= \scaledU^k + \D \odot \V^{k+1} - \Y^{k+1}. 
    \end{aligned}
    \tag{R1}
\end{equation}

\spara{Solution for $\V^{k+1}$.}

The update for $\V^{k+1}$ in \eqref{eq:MADMMrecursion_app} reads as
\begin{equation}\label{eq:updateV}
    \V^{k+1} = \argmin_{\V \in \stiefel{\ell}{h}} \Tr{\left( \V^\top \covlow \V\right)^{-1} \covhigh} + \log\det{\V^\top \covlow \V} + \frac{\rho}{2}\frob{\D\odot \V  - \Y^k + \scaledU^k}^2     
\end{equation}

\Cref{eq:updateV} is a standard smooth optimization problem over the Stiefel manifold, and it can be solved by standard techniques such as those in \citeSupp{boumal2023introductionSupp}.
Newton and conjugate gradient methods for the Stiefel manifold are discussed in \citeSupp{edelman1998geometrySupp}.
In our experiments, we use the conjugate gradient implementation in \citeSupp{boumal2014manoptSupp}.

\spara{Solution for $\Y^{k+1}$.}
The update for $\Y^{k+1}$ in \eqref{eq:MADMMrecursion_app} reads as
\begin{equation}\label{eq:updateY_madmm}
    \begin{aligned}
        \Y^{k+1} &= \argmin_{\Y \in \rmatdim} \lambda\norm{\Y}_1 + \frac{\rho}{2}\frob{\D \odot \V^{k+1} - \Y + \scaledU^k}^2 = \\
        &= \mathcal{S}_{\lambda/\rho} \left( \D \odot \V^{k+1} + \scaledU^k \right);
    \end{aligned}
\end{equation}
where $\mathcal{S}_{\delta}(x)=\sign(x) \cdot \max(\abs{x}-\delta, 0)$ is the element-wise soft-thresholding operator \citeSupp{parikh2014proximalSupp}.

\spara{Stopping criteria.}
The empirical convergence of LinSEPAL-ADMM is established according to primal and dual feasibility optimality conditions \citeSupp{boyd2011distributedSupp}.
The primal residual, associated with the equality constraint in \Cref{eq:MADMM}, is
\begin{equation}\label{eq:primal_res_madmm}
    \mathbf{R}_p^{k+1}\coloneqq\Y^{k+1}-\D\odot\V^{k+1}\,.
\end{equation}
The dual residual, which can be obtained from the stationarity condition, is
\begin{equation}\label{eq:dual_res_madmm}
    \mathbf{R}_d^{k+1}\coloneqq \rho \,\D \odot\left(\Y^{k+1}-\Y^k\right)\,.
\end{equation}
As $k \rightarrow \infty$, the norm of the primal and dual residuals should vanish.
Hence, the stopping criterion can be set in terms of the norms
\begin{equation}\label{eq:norms_madmm}
    \text{\emph{(i)}}\;d_p^{k+1}=\frob{\mathbf{R}_p^{k+1}} \quad \text{and} \quad \text{\emph{(ii)}}\;d_d^{k+1}=\frob{\mathbf{R}_d^{k+1}}\,. 
\end{equation}
Specifically, given absolute and relative tolerance, namely $\tau^a$ and $\tau^r$ in $\reall_+$, respectively, convergence in practice is established following \citetSupp{boyd2011distributedSupp} when 
\begin{equation}\label{eq:convergence_linsepal}
    \text{\emph{(i)}}\; d_p \leq \tau^a\sqrt{\ell h} + \tau^r \max{\left(\frob{\Y^{k+1}}, \frob{\D \circ \V^{k+1}}\right)}\,, \quad
    \text{and} \quad
    \text{\emph{(ii)}}\; d_d \leq \tau^a\sqrt{\ell h} + \tau^r \rho  \frob{\D \circ \scaledU^{k+1}}\,.
\end{equation}

The LinSEPAL-ADMM algorithm is summarized in \cref{alg:linsepal_admm}.

\begin{algorithm}[H]
\caption{LinSEPAL-ADMM}
\label{alg:linsepal_admm}
\begin{algorithmic}[1]
\STATE \textbf{Input:} $\covlow$, $\covhigh$, $\D$, $\lambda$, $\rho$, $\tau^a$, $\tau^r$
\STATE Initialize: $\V^0 \in \stiefel{\ell}{h}$, $\Y^0 \in \rmatdim$, $\scaledU^0 \gets \D \odot \V^0 - \Y^0$
\REPEAT
    \STATE $\V^{k+1} \gets \text{Solve \cref{eq:updateV} via an off-the-shelf method for smooth Riemannian problems}$ 
    \STATE $\Y^{k+1} \gets \mathcal{S}_{\lambda/\rho}\left( \D \odot \V^{k+1} + \scaledU^k \right)$
    \STATE $\scaledU^{k+1} \gets \scaledU^k + \D \odot \V^{k+1} - \Y^{k+1}$
\UNTIL{\Cref{eq:convergence_linsepal} is satisfied}
\STATE \textbf{Output:} $\V$, $\Y$, $\scaledU$
\end{algorithmic}
\end{algorithm} 

% !TEX root =  ../main.tex
\section{LinSEPAL-PG}\label{app:ManPG}
This method is based upon the \text{manifold proximal gradient} \citeSupp{chen2020Supp} framework, which generalizes the \emph{proximal gradient} framework defined in the Euclidean space to the Stiefel manifold.
Following \citeSupp{chen2020Supp}, denoting by $\V^k$ the iterate at the step $k$, the updates recursion for solving \eqref{eq:minKL} reads as
\begin{equation}\label{eq:ManPG_app}
    \begin{aligned}
        \G^k &= \argmin_{\G \in \tangentspace{\V^k}{\stiefel{\ell}{h}}} \quad \Eprod{\Egrad{}{f\left(\V^k\right)}}{\G}{} + \frac{1}{2\rho} \frob{\G}^2 + \lambda \norm{\D \odot \left(\V^k + \G\right)}_1 \, ,\\
        \V^{k+1} &= \Retr{}{\V^k}{\G^k} \,.
    \end{aligned}
    \tag{R2}
\end{equation}
In \eqref{eq:ManPG_app}, the first update is the proximal mapping providing a proximal gradient direction $\G^k$ onto the tangent space to the Stiefel manifold, using the first-order approximation of the objective around the $k$-th estimate.
The second is the update for $\V^{k+1}$, which exploits the canonical retraction (cf. \cref{eq:stiefel_retractions}) technique for projecting back $\V^k + \G^k$ from the tangent space to the manifold.
Global convergence of the ManPG method has been established in \citeSupp{chen2020Supp}.

\spara{Solution for $\G^k$.}
\citet{chen2020} shows that the first update can be efficiently solved using the regularized semi-smooth Newton method in \citeSupp{xiao2018regularizedSupp}.
Specifically, according to \Cref{eq:Stiefel_t_space}, the feasible set $\tangentspace{\V^k}{\stiefel{\ell}{h}}$ translates into a linear constraint.
By defining $\linearop{\mathcal{A}^k}{\G} \coloneqq \G^\top \V^k + \V^{k^\top} \G$, the update is
\begin{equation}\label{eq:ManPGU1}
    \begin{aligned}
        \G^k = \argmin_{\G \in \reall^{\ell \times h}} &\quad \Eprod{\Egrad{}{f\left(\V^k\right)}}{\G}{} + \frac{1}{2\rho} \frob{\G}^2 + \lambda \norm{\D \odot \left(\V^k + \G\right)}_1 \, ,\\
        \textrm{subject to} &\quad \linearop{\mathcal{A}^k}{\G} = \zeros_{h \times h}\,.
    \end{aligned}
\end{equation}

However, following the rationale in \citeSupp{si2024riemannianSupp}, we can force $\G^k \in \tangentspace{\V^k}{\stiefel{\ell}{h}}$ by exploiting the basis $\basisN{\V^k}$ of the normal space to the manifold, namely $\normalspace{\V^k}{\stiefel{\ell}{h}}$.
To find such $\basisN{\V^k}$, recall the explicit form of \normalspace{\V}{\stiefel{\ell}{h}} in \Cref{eq:explicit_normal_space}.

The basis of \sym{h}, having dimension $s=h (h+1)/2$, is 
\begin{equation}\label{eq:basis_sym}
    \mathcal{E} \coloneqq \{ \mathbf{E}_{ij} \in \{0,1\}^{h \times h} \, \mid \, \mathbf{E}_{ij} \text{ has } e_{ij}=e_{ji}=1, \, 0 \text{ elsewhere}, \, 1\leq i \leq j \leq h\}\,.
\end{equation}
It follows from \Cref{eq:explicit_normal_space,eq:basis_sym} that
\begin{equation}\label{eq:basis_nvk}
    \basisN{\V^k} \coloneqq \{ \basisNelement{k}_{ij}=\V^k \mathbf{E}_{ij}, \, 1\leq i \leq j \leq h\}\,.
\end{equation}
At this point, the membership to $\tangentspace{\V^k}{\stiefel{\ell}{h}}$ can be expressed as 
\begin{equation}
    \Eprod{\basisNelement{k}_{ij}}{\mathbf{G}}{}=0, \quad \forall 1\leq i \leq j \leq h\,. 
\end{equation}

Hence, \eqref{eq:ManPGU1} reads as
\begin{equation}\label{eq:ManPGU1n}
    \begin{aligned}        
        \G^k = \argmin_{\G \in \reall^{\ell \times h}} &\quad \Eprod{\Egrad{}{f\left(\V^k\right)}}{\G}{} + \frac{1}{2\rho} \frob{\G}^2 + \lambda \norm{\D \odot \left(\V^k + \G\right)}_1 \, ,\\
        \textrm{subject to} &\quad \Eprod{\basisNelement{k}_{ij}}{\mathbf{G}}{}=0, \quad \forall\, 1\leq i \leq j \leq h\,.
    \end{aligned}
\end{equation}

Consider $h\left(\V^k + \G\right)=\norm{\D \odot \left(\V^k + \G\right)}_1$ and $\reall^{s} \ni \boldsymbol{\mu}=[\mu_{11}, \mu_{12}, \ldots, \mu_{ij},\dots, \mu_{hh}]$, with $1\leq i \leq j \leq h$.
The Lagrangian for \eqref{eq:ManPGU1n} is
\begin{equation}\label{eq:ManPGU1n_Lagr}
    L_\rho \left(\G, \boldsymbol{\mu} \right) = \Eprod{\Egrad{}{f\left(\V^k\right)}}{\G}{} + \frac{1}{2\rho} \frob{\G}^2 + \lambda \, h\left(\V^k + \G\right) - \sum_{1\leq i \leq j \leq h}\mu_{ij} \Eprod{\basisNelement{k}_{ij}}{\mathbf{G}}{}\,.
\end{equation}
Let us define now the matrix $\reall^{s \times \ell h} \ni \mathbf{B}^k \coloneqq [\myvec{\basisNelement{k}_{11}}, \myvec{\basisNelement{k}_{12}}, \ldots, \myvec{\basisNelement{k}_{hh}}]^\top$, where $\myvec{\basisNelement{k}_{ij}} \in \reall^{\ell h}$.
We can compactly express the $s$ equality constraints as
\begin{equation}\label{eq:compact_eq_constraint}
    \mathbf{B}^k\myvec{\G}=\zeros_{s}\,.
\end{equation}

Thus, the Karush-Kuhn-Tacker (KKT) conditions of \Cref{eq:ManPGU1} reads as
\begin{equation}\label{eq:ManPGU1_KKT}
    \text{\emph{(i)}}\; \zeros_{l \times h} \in \Esubgrad{\G}{L_{\rho}\left(\G, \boldsymbol{\mu} \right)}\,, \quad \text{and} \quad \text{\emph{(ii)}}\; \mathbf{B}^k\myvec{\G}=\zeros_{s}\,.
\end{equation}

From the stationarity condition we get
\begin{equation}\label{eq:ManPGU1_stationarity}
    \begin{aligned}
        \zeros_{l \times h} \in \G + \rho \left(\Egrad{}{f\left(\V^k\right)} - \sum_{1\leq i\leq j\leq h} \mu_{ij} \basisNelement{k}_{ij} \right) + \lambda \, \rho \,\Esubgrad{\G}{h\left(\V^k + \G\right)}\,.
    \end{aligned}
\end{equation}

At this point, recalling the inclusion property of proximal operators, viz. $\mathbf{P} = \mathrm{prox}_g(\mathbf{B}) \iff \mathbf{B}-\mathbf{P} \in \Esubgrad{}{g(\mathbf{P})}$, we have
\begin{equation}\label{eq:ManPGU1_stationarity_prox}
    \begin{aligned}
        \zeros_{l \times h} \in \underbrace{\V^k + \G}_{\mathbf{P}} - \underbrace{\left( \V^k - \rho \left(\Egrad{}{f\left(\V^k\right)} - \sum_{1\leq i\leq j\leq h} \mu_{ij} \basisNelement{k}_{ij} \right) \right)}_{\mathbf{B}(\boldsymbol{\mu})} + \lambda\, \rho \, \Esubgrad{\G}{h\underbrace{\left(\V^k + \G\right)}_{\mathbf{P}}}\,;
    \end{aligned}
\end{equation}

from which we get
\begin{equation}\label{eq:ManPGU1_stationarity_solved}
    \G(\boldsymbol{\mu}) = \mathrm{prox}_{\lambda \, \rho\, h(\cdot) } \left(\mathbf{B}\left(\boldsymbol{\mu}\right)\right) - \V^k\,.
\end{equation}

At this point, $\mathrm{prox}_{\lambda \, \rho\, h(\cdot) }$ can be computed element-wise as
\begin{equation}\label{eq:ManPGU1_stationarity_prox_solved}
    \mathrm{prox}_{\lambda \, \rho\, h(\cdot) } \left(b_{ij}\left(\boldsymbol{\mu}\right)\right) = \begin{cases}
        b_{ij}(\boldsymbol{\mu})\,, &\quad \text{if } d_{ij}=0\,, \\
        \mathcal{S}_{\lambda \, \rho}(b_{ij}(\boldsymbol{\mu}))\,, &\quad \text{otherwise}\,.
    \end{cases}
\end{equation}

Substituting \Cref{eq:ManPGU1_stationarity_solved} into \Cref{eq:ManPGU1_KKT}, we have 
\begin{equation}\label{eq:ManPGLambda_G}
    \mathbf{B}^k\myvec{\G(\boldsymbol{\mu})}=\zeros_{s}\,.
\end{equation}

Here the $r$-th entry of $\myvec{\G(\boldsymbol{\mu})}$ corresponds to the entry of $\G(\boldsymbol{\mu})$ at row $ u=(r-1) \,\mathrm{mod}\, \ell + 1$, and column $v=\lfloor{(r-1) / \ell\rfloor} + 1$ , $r \in [\ell h]$. 

At this point, we can use the regularized semi-smooth Newton method \citeSupp{xiao2018regularizedSupp} to solve \Cref{eq:ManPGLambda_G}.
Our target function is
\begin{equation}\label{eq:regNF}
    F(\boldsymbol{\mu})=\mathbf{B}^k\myvec{\G(\boldsymbol{\mu})} \, :\, \reall^s \rightarrow \reall^s. 
\end{equation}
By the chain rule of calculus, using \Cref{eq:ManPGU1_stationarity_solved}, the generalized Jacobian matrix is 
\begin{equation}\label{eq:genJ}
    \begin{aligned}
        \reall^{s \times s} \ni \mathbf{J} &= \pdv{F(\boldsymbol{\mu})}{\myvec{\G(\boldsymbol{\mu})}}\cdot \pdv{\myvec{\G(\boldsymbol{\mu})}}{\boldsymbol{\mu}} \\
        &= \mathbf{B}^k \pdv{\prox_{\lambda\, \rho\, h(\cdot)}\left(\myvec{\mathbf{B}(\boldsymbol{\mu})}\right)}{\myvec{\mathbf{B}(\boldsymbol{\mu})}}\cdot\pdv{\myvec{\mathbf{B}(\boldsymbol{\mu})}}{\boldsymbol{\mu}}\,.
    \end{aligned}
\end{equation}
The proximal-related term is a diagonal matrix $\mathbf{M} \in \reall^{\ell h \times \ell h}$, where
\begin{equation}\label{eq:Jfirst}
    m_{rr} = \begin{cases}
        1\,,\quad \text{if } \myvec{\D}_{r} = 0 \text{ or } \left(\myvec{\D}_{r} = 1 \text{ and }\abs{b_{r}}-\lambda \rho >0\right)\,. \\
        0\,, \quad \text{otherwise}.
    \end{cases}
\end{equation}
Additionally, starting from
\begin{equation}
    \begin{aligned}
        b_r(\boldsymbol{\mu}) &= \myvec{\V^k - \rho\left( \Egrad{}{f(\V^k)} - \sum_{1 \leq i \leq j \leq h} \mu_{ij} \left[\B_{ij}^k \right]_{uv} \right) } \\
        &= \myvec{\V^k - \rho\left( \Egrad{}{f(\V^k)} - \mathbf{b}_{uv}^{k^\top} \boldsymbol{\mu} \right)}\,.        
    \end{aligned}
\end{equation}

Hence, we get 
\begin{equation}\label{eq:Jsecond}
    \reall^{s} \ni \pdv{b_r(\boldsymbol{\mu})}{\boldsymbol{\mu}}=\mathbf{b}_{uv}^{k}\,.
\end{equation}

Consequently, starting from \Cref{eq:genJ}, using \Cref{eq:Jfirst,eq:Jsecond}, we finally have
\begin{equation}\label{eq:genJ_explicit}
    \mathbf{J} = \mathbf{B}^k \mathbf{M} \mathbf{C}, 
    \quad \text{ with } \reall^{\ell h \times s} \ni \mathbf{C}=\begin{pmatrix}
        \mathbf{b}_{11}^{k^\top}\\
        \mathbf{b}_{21}^{k^\top}\\
        \vdots \\
        \mathbf{b}_{\ell h}^{k^\top}  
    \end{pmatrix}\,. 
\end{equation}

Following \citeSupp{xiao2018regularizedSupp}, denoting with $\nu^k=\alpha^k\norm{F^k}_2, \, \alpha^k \in \reall^+$, we define
\begin{equation}\label{eq:regN1}
    r^k \coloneqq \left( \mathbf{J}^{k-1} + \nu^{k-1} \identity_s \right) \mathbf{d}^{k} + F^{k-1}\,.
\end{equation}

At each iteration we want to find the step $\mathbf{d}^k$ by solving \Cref{eq:regN1} inexactly, such that
\begin{equation}
    \norm{r^k}_2 \leq \tau \min{\left(1, \alpha^{k-1} \norm{F^{k-1}}_2 \norm{\mathbf{d}^k}_2\right)}\,, \quad \tau \in (0,1)\,;
\end{equation}
obtaining a trial point
\begin{equation}\label{eq:regN_trial}
    \mathbf{u}^k = \boldsymbol{\mu}^{k-1} + \mathbf{d}^k\,.
\end{equation}

Let $\beta^0=\norm{F(\boldsymbol{\mu}^0)}_2$ and $\gamma \in (0,1)$.
If $\norm{F(\mathbf{u}^k)}_2\leq \gamma \beta^{k-1}$ then we set
\begin{equation}\label{eq:Newton_step}
    \boldsymbol{\mu}^{k} = \mathbf{u}^k\,, \; \beta^{k}=\norm{F(\mathbf{u}^k)}_2\,, \; \text{ and } \alpha^{k}=\alpha^{k-1}\,. \quad \text{[Newton step]}
\end{equation}

Otherwise, let 
\begin{equation}\label{eq:safeguard_ratio}
    \xi^k = \frac{- F(\mathbf{u}^k)^\top \mathbf{d}^k}{\norm{\mathbf{d^k}}_2^2}\, . 
\end{equation}

Select $0<\phi_1\leq\phi_2<1$ and $1<\psi_1<\psi_2$.
Hence, we make a safeguard step as follows
\begin{equation}\label{eq:safeguard_step}
    \boldsymbol{\mu}^{k} = \begin{cases}
        \mathbf{v}^k\,, &\quad \text{ if } \xi^k \geq \phi_1 \text{ and } \norm{F(\mathbf{v}^k)}_2 \leq \norm{F(\boldsymbol{\mu}^{k-1})}_2, \;  \text{[projection step]} \\
        \mathbf{w}^k\,, &\quad \text{ if } \xi^k \geq \phi_1 \text{ and } \norm{F(\mathbf{v}^k)}_2 > \norm{F(\boldsymbol{\mu}^{k-1})}_2, \;  \text{[fixed-point step]} \\
        \boldsymbol{\mu}^{k-1}\,, &\quad \text{ if } \xi^k < \phi_1, \; \text{unsuccessful step}
    \end{cases}
\end{equation}
where
\begin{equation}
    \mathbf{v}^k = \boldsymbol{\mu}^{k-1} - \frac{F(\mathbf{u}^k)^\top (\boldsymbol{\mu}^{k-1} - \mathbf{u}^k)}{\norm{F(\mathbf{u}^k)}_2} F(\mathbf{u}^k), \; \mathbf{w}^k=\boldsymbol{\mu}^{k-1} - \delta F(\boldsymbol{\mu}^k), \; \delta \in \left(0, \frac{1}{\omega}\right)\,;
\end{equation}

where $\omega \in (0,1]$. 
Finally, denoting $\reall^+ \ni \Bar{\alpha} \approx 0$, the parameters $\beta^{k+1}$ and $\alpha^{k+1}$ are updated as
\begin{equation}
    \beta^{k}=\beta^{k-1}, \quad \alpha^{k} \in \begin{cases}
        (\Bar{\alpha}, \alpha^{k-1})\,, &\quad \text{if } \xi^k \geq \phi_2,\\
        [\alpha^{k-1}, \psi_1\alpha^{k-1}]\,, &\quad \text{if } \phi_1 \leq \xi^k < \phi_2,\\
        (\psi_1\alpha^{k-1}, \psi_2\alpha^{k-1}]\,, &\quad \text{otherwise.}\\
    \end{cases}
\end{equation}

At this point, we set $\G^k=\G^k(\boldsymbol{\mu}^k)$ according to \Cref{eq:ManPGU1_stationarity_solved}.

\spara{Solution for $\mathbf{V}^{k+1}$.}
Given $\V^k + \G^k \in \tangentspace{\V^k}{\stiefel{\ell}{h}}$, we have to project the point onto the manifold.
This can be accomplished via the canonical retractions in \Cref{eq:stiefel_retractions}.
However, as suggested in \citeSupp{chen2020Supp}, our LinSEPAL-PG implementation performs an Armijo line-search procedure to determine the stepsize $a$.
Hence, the update is
\begin{equation}\label{eq:updateV_linsepalpg}
    \V^{k+1}=\Retr{\mathrm{QR}}{\V^k}{a\G^k}\,.
\end{equation}

\spara{Stopping criteria.}
Empirical convergence of the LinSEPAL-PG algorithm is established either when a maximum number of iterations $K$ is reached, or when the $\KL{\V^{k+1}}$ is below a certain threshold $\tau^{\mathrm{KL}}\approx 0$.
The LinSEPAL-PG algorithm is summarized in \cref{alg:linsepal_pg}.

\begin{algorithm}[H]
\caption{LinSEPAL-PG}
\label{alg:linsepal_pg}
\begin{algorithmic}[1]
\STATE \textbf{Input:} $\covlow$, $\covhigh$, $\D$, $\lambda$, $\rho$, $\gamma \in (0,\,1)$, $\tau^{\mathrm{KL}}$, $K$
\STATE Initialize: $\V^0 \in \stiefel{\ell}{h}$, $\Y^0 \in \rmatdim$, $\scaledU^0 \in \rmatdim$
\REPEAT
    \STATE $\G^{k} \gets \text{Solve \cref{eq:ManPGU1n} via the regularized semi-smooth Newton method}$ 
    \STATE $a \gets 1$
    \REPEAT
        \STATE $a = \gamma a$
        \STATE $\bar{\V} = \;\Retr{\mathrm{QR}}{\V^k}{a\,\G^k}$
    \UNTIL{$\KL{\bar{\V}} > \KL{\V^k} - \frac{a \frob{\G^k}^2}{2\rho}$}
    \STATE $\V^{k+1} \gets \bar{\V}$
\UNTIL{$k>K$ or $\KL{\V^{k+1}}<\tau^{\mathrm{KL}}$}
\STATE \textbf{Output:} $\V$
\end{algorithmic}
\end{algorithm} 
% !TEX root =  ../main.tex
\section{CLinSEPAL}\label{app:MADMMSCA_partial}
The problem we want to solve is:

\smoothpartial*

\Cref{prob:nonconvex_prob_approx} makes it explicit that the abstraction morphism is given by three key ingredients: \emph{(i)} the given partial, structural prior information represented by \B; \emph{(ii)} the structural component \Supp to be learned, such that the resulting causal abstraction is constructive; and \emph{(iii)} the abstraction coefficients in \V determining the linear functional forms of the causal abstraction, which have to be learned as well.
Specifically, \enquote{partial} means that some rows of \B have more than one entry equal to one.

Unfortunately, \Cref{prob:nonconvex_prob_approx} is nonconvex because of the objective function and the Stiefel manifold.
Additionally, in this case, the CA results in a bilinear form $\B\odot \Supp \odot \V$, which is not jointly convex in \Supp and \V.
Consequently, the constraint $\B\odot \Supp \odot \V \in \stiefel{\ell}{h}$ has to be carefully handled.

Regarding the nonconvexity of the objective in \Cref{eq:objective_partial_knowledge}, we proceed by leveraging its smoothness.
Specifically, we have the following result.

\begin{corollary}\label{corollary:objective_partial_knowledge}
    The function $f(\V, \Supp)$ in \Cref{eq:objective_partial_knowledge} is smooth.
    Additionally, define $\mathbf{A}\coloneqq\left(\B \odot \Supp \odot \V\right)$ and $\widetilde{\mathbf{A}}\coloneqq\left(\mathbf{A}^\top \covlow \mathbf{A}\right)^{-1}$.
    The partial derivatives w.r.t. \V and \Supp are
    \begin{equation}\label{eq:gradV_partial}
        \Egrad{\V}{f} = 2\left(\B \odot \Supp\right) \odot \left(\left(\covlow\mathbf{A}\widetilde{\mathbf{A}}\right)\left(\identity_h - \covhigh\widetilde{\mathbf{A}}\right)\right)\,,
    \end{equation}
    and
    \begin{equation}\label{eq:gradS_partial}
        \Egrad{\Supp}{f} = 2\left(\B \odot \V\right) \odot \left(\left(\covlow\mathbf{A}\widetilde{\mathbf{A}}\right)\left(\identity_h - \covhigh\widetilde{\mathbf{A}}\right)\right)\,.
    \end{equation}
\end{corollary}
\begin{proof}
    Smoothness directly follows from \Cref{prop:smoothness_and_differentiability} by defining $\mathbf{A}=\left(\B \odot \Supp \odot \V\right)$, which is constrained to \stiefel{\ell}{h} as given in \Cref{eq:prob_madmmsca_VS}.
    The partial derivatives in \Cref{eq:gradV_partial,eq:gradS_partial} follow from the application of \Cref{eq:rules_matrix_calculus}, together with the chain rule for derivatives. 
\end{proof}

At this point, we leverage \Cref{corollary:objective_partial_knowledge} to provide a solution which combines ADMM \citeSupp{boyd2011distributedSupp} and SCA \citeSupp{nedic2018parallelSupp}.
Specifically, ADMM is suitable to isolate and consequently tackle the nonconvexity in different subproblems.
To manage the bilinear form within the first constraint in \Cref{eq:prob_madmmsca_VS}, we introduce two splitting variables, namely \YO and \YT in \stiefel{\ell}{h}, and the corresponding equality constraints
\begin{equation}\label{eq:splitting_constraints_partial}
    \YO - \B\odot\Supp^k\odot \V = 0_{\ell \times h} \quad \text{and} \quad \YT - \B\odot\V^{k+1}\odot\Supp = 0_{\ell \times h}\,, \; \text{respectively.}
\end{equation}
In this way, given the solution at iteration $k$ within the ADMM framework, we optimize separately over \V and \Supp while always tracking \stiefel{\ell}{h}.
Please notice that we use $\V^{k+1}$ since when optimizing over \Supp, \V has already been updated.
The rationale behind the usage of the splitting variable for handling the Stiefel manifold is the same as the \emph{splitting of orthogonality constraints} method (SOC, \citeSupp{lai2014splittingSupp}).
Additionally, to handle $\left(\B \odot \Supp\right)^\top \in \sphere{h}{\ell}$, we introduce another splitting variable $\X \in \sphere{h}{\ell}$, and the corresponding equality constraint $\X - \left(\B \odot \Supp\right)^\top=\zeros_{h\times \ell}$.
Thus, starting from \Cref{eq:prob_madmmsca_VS}, we get the following equivalent minimization problem
\begin{equation}\label{eq:prob_madmmsca_VS_with_splitting}
    \begin{aligned}
        \V^\star, \Supp^\star, \YO^\star, \YT^\star, \X^\star = \argmin_{\substack{\V \in \rmatdim\\ \Supp \in \umatdim \\ \YO \in \stiefel{\ell}{h} \\ \YT \in \stiefel{\ell}{h} \\ \X \in \sphere{h}{\ell}}} &\quad f(\V,\Supp)\,;\\
         \textrm{subject to} & \quad \YO - \B \odot \Supp^k \odot \V = \zeros_{\ell \times h}\,, \\
         & \quad \YT - \B \odot \V^{k+1} \odot \Supp = \zeros_{\ell \times h}\,, \\
         & \quad \X - \left(\B \odot \Supp\right)^\top = \zeros_{h \times \ell}\,, \\
         % & \quad \Supp \ones_h - \ones_\ell = \zeros_\ell\,, \\
         & \quad \ones_h - \left(\B \odot \Supp\right)^\top \ones_\ell \leq \zeros_h\,.
    \end{aligned}
\end{equation}

Starting from \Cref{eq:prob_madmmsca_VS_with_splitting}, considering the penalty $\rho \in \reall_+$, we introduce the scaled dual variables \scaledUO and \scaledUT in \rmatdim; and $\scaledW \in \rmatdimT$, and write the scaled augmented Lagrangian
\begin{equation}\label{eq:scaledAUL_partial}
    \begin{aligned}
    L_\rho\left(\V,\Supp,\YO,\YT,\X,\scaledUO,\scaledUT,\scaledW\right) =& f(\V,\Supp) + \frac{\rho}{2}\frob{\B\odot\Supp^k\odot\V - \YO +\scaledUO}^2 + \\
    +& \frac{\rho}{2}\frob{\B\odot\V^{k+1}\odot\Supp - \YT +\scaledUT}^2 + \frac{\rho}{2}\frob{\left(\B \odot \Supp\right)^\top - \X + \scaledW}^2\,.        
    \end{aligned}
\end{equation}

Now, we can apply ADMM iterative procedure, getting the recursion for updating the primal and scaled dual variables. 
In detail, denote by $k \in \nat$ the current iteration.
We have
\begin{equation}\label{eq:ADMM_partial}
    \begin{aligned}        
        \V^{k+1}=&\argmin_{\V \in \rmatdim} L_\rho\left(\V,\Supp^k,\YO^k,\YT^k,\X^k,\scaledUO^k,\scaledUT^k,\scaledW^k\right)\,;\\
        \Supp^{k+1}=&\argmin_{\Supp \in \umatdim} L_\rho\left(\V^{k+1},\Supp,\YO^k,\YT^k,\X^k,\scaledUO^k,\scaledUT^k,\scaledW^k\right)\,,\\
        & \textrm{subject to} \quad \ones_h - \left(\B \odot \Supp\right)^\top \ones_\ell \leq \zeros_h\,; \\
        \YO^{k+1}=&\argmin_{\YO \in \stiefel{\ell}{h}} L_\rho\left(\V^{k+1},\Supp^{k+1},\YO,\YT^k,\X^k,\scaledUO^k,\scaledUT^k,\scaledW^k\right)\,;\\
        \YT^{k+1}=&\argmin_{\YT \in \stiefel{\ell}{h}} L_\rho\left(\V^{k+1},\Supp^{k+1},\YO^{k+1},\YT,\X^k,\scaledUO^k,\scaledUT^k,\scaledW^k\right)\,;\\
        \X^{k+1}=&\argmin_{\X \in \sphere{h}{\ell}} L_\rho\left(\V^{k+1},\Supp^{k+1},\YO^{k+1},\YT^{k+1},\X,\scaledUO^k,\scaledUT^k,\scaledW^k\right)\,;\\
        \scaledUO^{k+1}=&\scaledUO^k + \left(\B\odot\Supp^{k}\odot\V^{k+1} - \YO^{k+1}\right)\,;\\
        \scaledUT^{k+1}=&\scaledUT^k + \left(\B\odot\V^{k+1}\odot\Supp^{k+1} - \YT^{k+1}\right)\,;\\
        \scaledW^{k+1}=&\scaledW^k + \left(\B \odot \Supp^{k+1}\right)^\top - \X^{k+1}\,.
    \end{aligned}
\end{equation}

Similarly to SOC, we isolate the objective nonconvexity into the first and second (nonconvex) subproblems; and the nonconvexity of the manifold into the third and fourth (nonconvex) ones.
Notably, the first and second subproblems can be managed through SCA. 
Additionally, the third and fourth nonconvex subproblems admit closed-form solutions since they boil down to the \emph{closest orthogonal
approximation problems} \citeSupp{fan1955someSupp,higham1986computingSupp}.
Thus, the latter nonconvexity is somehow resolved.
Finally, we solve the subproblem for $\X^{k+1}$ in closed form as well.

\subsection{\texorpdfstring{Update for $\V^{k+1}$}{Update for V}}\label{subsec:updateV_partial}
Starting from \Cref{eq:scaledAUL_partial,eq:ADMM_partial}, the subproblem we have to solve is
\begin{equation}\label{eq:updateV_nonconvex_partial}
    \V^{k+1}=\argmin_{\V \in \rmatdim}\quad f(\V,\Supp^k) + \frac{\rho}{2}\frob{\B\odot\Supp^k\odot\V - \YO^k +\scaledUO^k}^2\,.
\end{equation}
\Cref{eq:updateV_nonconvex_partial} is nonconvex due to the inherent nonconvexity of $f(\V,\Supp^k)$.
However, the latter function is smooth and differentiable w.r.t. \V, as given in \Cref{corollary:objective_partial_knowledge}.
Hence, we apply the SCA framework.
In detail, denote by $q$ the SCA iteration and set $\V^0=\V^k$ for $q=0$.
We derive a strongly convex surrogate $\widetilde{f}(\V;\V^q, \Supp^k)$ around the point $\V^q$ -- i.e., the solution at the iterate $q$ -- exploiting \Cref{eq:gradV_partial}:
\begin{equation}\label{eq:strongly_convex_surrogate_Vpartial}
    \widetilde{f}(\V;\V^q, \Supp^k) \coloneqq \Tr{\Egrad{\V}{f}\at{(\V^q, \Supp^k)}^\top \left(\V - \V^q\right)} + \frac{\tau}{2}\frob{\V -\V^q}^2\,.
\end{equation}
It is immediate to check that \Cref{eq:strongly_convex_surrogate_Vpartial} is a proper surrogate satisfying the stationarity condition $\Egrad{\V}{f}\at{\V^q}=\Egrad{\V}{\widetilde{f}}\at{\V^q}$.

Therefore, at each SCA iteration $q$, we solve a strongly convex problem in closed-form and then apply the usual smoothing operation by using a diminishing stepsize $\gamma^q \in \reall_+$.
Specifically,
\begin{equation}\label{eq:SCA_recursion_V}
    \begin{aligned}        
        \V^{q+1} &= \argmin_{\V \in \rmatdim}\quad \widetilde{f}(\V;\V^q, \Supp^k) +\frac{\rho}{2}\frob{\B\odot\Supp^k\odot\V - \YO^k +\scaledUO^k}^2\,, \quad\textrm{(Strongly convex problem)}\\
        \V^{q+1} &= \V^q + \gamma^k\left(\V^{q+1}-\V^q\right)\,. \quad\textrm{(Smoothing)}
    \end{aligned}
\end{equation}

The solution of the strongly-convex problem is given element-wise in \Cref{lemma:updateV_elementwise_partial}. 
\begin{lemma}\label{lemma:updateV_elementwise_partial}
    The update for $\V^{q+1}$ can be computed element-wise as
    \begin{equation}\label{updateV_elementwise_partial}
        v_{ij}^{q+1}= \frac{1}{\tau + b_{ij} s_{ij}^{k^2}}\Bigg(\rho\, b_{ij} s_{ij}^k y_{1_{ij}}^k - \rho\, b_{ij} s_{ij}^k u_{1_{ij}}^k + \tau\, v_{ij}^q - \Big[\Egrad{\V}{f\at{(\V^q, \Supp^k}}\Big]_{ij} \Bigg)\,.
    \end{equation}
\end{lemma}
\begin{proof}
    The proof follows by imposing the stationarity condition
    \begin{equation}
        \mathbf{0}_{\ell\times h} = \Egrad{\V}{f\at{(\V^q, \Supp^k}} + \tau \left(\V-\V^q\right) + \rho \, \B \odot \Supp^k \odot \left( \B \odot \Supp^k \odot \V - \YO^k +\scaledUO^k\right)\,,
    \end{equation}
    and solving for \V.
\end{proof}

Additionally, the diminishing stepsize $\gamma^k$ has to satisfy the classical stochastic approximation conditions \citeSupp{nedic2018parallelSupp},
\begin{equation}\label{eq:SCA_stepsize_conditions}
    \text{\emph{(i)}} \sum_{q=1}^\infty \gamma^q = \infty \quad  \text{and} \quad\text{\emph{(ii)}} \sum_{q=1}^\infty \left( \gamma^q \right)^2 < \infty\,.
\end{equation}
In our experiments, we use the decaying rule 
\begin{equation}\label{eq:SCA_stepsize_rule}
    \gamma^{q+1}=\gamma^q\left(1-\varepsilon \,\gamma^q\right)\,, \quad \varepsilon \in (0,1)\,.
\end{equation}
The SCA framework is guaranteed to converge to stationary points of the original nonconvex problem in \Cref{eq:updateV_nonconvex_partial} \citeSupp{nedic2018parallelSupp}.
Accordingly, we establish convergence for the update when
\begin{equation}\label{eq:convergenceV_approx}
    \frob{\V^{q+1}-\V^q}\leq \tau^c\,, \quad \tau^c \approx 0\,;
\end{equation}
and set $\V^{k+1}=\V^{q+1}$.

\subsection{\texorpdfstring{Update for $\Supp^{k+1}$}{Update for S}}\label{subsec:updateS_partial}
Starting from \Cref{eq:scaledAUL_partial,eq:ADMM_partial}, the subproblem we have to solve is
\begin{equation}\label{eq:updateS_nonconvex_partial}
    \begin{aligned}
        \Supp^{k+1}=&\argmin_{\Supp \in \umatdim} f(\V^{k},\Supp) + \frac{\rho}{2}\frob{\B \odot \V^{k+1} \odot \Supp - \YT^k +\scaledUT^k}^2 + \frac{\rho}{2}\frob{\left(\B \odot \Supp\right)^\top - \X^k + \scaledW^k}^2\,,\\
        & \textrm{subject to} \quad \ones_h - \left(\B \odot \Supp\right)^\top \ones_\ell \leq \zeros_h\,. \\
    \end{aligned}
\end{equation}
The subproblem above is nonconvex and constrained.
Similarly to \Cref{subsec:updateV_partial}, we apply the SCA framework.
Denote by $q$ the SCA iteration and set $\Supp^0=\Supp^k$ for $q=0$.
Here, the strongly convex surrogate of $f(\V^{k+1},\Supp)$ reads as
\begin{equation}\label{eq:strongly_convex_surrogate_Spartial}
    \widetilde{f}(\Supp;\V^{k+1}, \Supp^q) \coloneqq \Tr{\Egrad{\Supp}{f}\at{\left(\V^{k+1},\Supp^q\right)}^\top \left(\Supp - \Supp^q\right)} + \frac{\tau}{2}\frob{\Supp -\Supp^q}^2\,,
\end{equation}
which satisfies $\Egrad{\Supp}{f}\at{\left(\V^{k+1},\Supp^q\right)}=\Egrad{\Supp}{\widetilde{f}}\at{\left(\V^{k+1},\Supp^q\right)}$.
At each SCA iteration $q$, we solve a constrained quadratic programming (QP) problem and apply the smoothing step by using the stepsize $\gamma^q \in \reall_+$ complying with the conditions in \Cref{eq:SCA_stepsize_conditions}.
In detail, let $\myvec{\mathbf{A}}$ be the column-wise vectorization of a given matrix $\mathbf{A}$ and define
\begin{equation}\label{eq:Q_and_c_Supdate}
    \begin{aligned}
        \mathbf{Q} &= \tau\identity_{\ell h} + \rho\,\mathrm{diag}\left( \left(\myvec{\B}\odot\myvec{\V^{k+1}}\right)\odot\left(\myvec{\B}\odot\myvec{\V^{k+1}}\right)\right) + \rho\,\mathrm{diag}\left(\myvec{\B}\odot \myvec{\B}\right)\,,\\
        \mathbf{c} &= \myvec{\Egrad{\Supp}{f}\at{\left(\V^{k+1},\Supp^q\right)}} - \tau\, \myvec{\Supp^q} - \rho\, \myvec{\YT^k - \scaledUT^k} \odot \myvec{\B} \odot \myvec{\V^{k+1}} - \rho \, \myvec{\B} \odot \myvec{\left(\X^k - \scaledW^k\right)^\top}\,.
    \end{aligned}
\end{equation}
Additionally, recall that $\myvec{\mathbf{A} \mathbf{C}} = \left(\mathbf{C}^\top \otimes \identity_h \right)\myvec{\mathbf{A}}$, with $\mathbf{A} \in \reall^{h\times \ell}$ and $\mathbf{C} \in \reall^{\ell \times m}$.
Hence, denoting with $\mathbf{K}^{\ell,h}$ the commutation matrix, 
the inequality constraint can be rewritten as
\begin{equation}\label{eq:inequality_constr_S}
    \begin{aligned}
        \ones_h - \myvec{\left(\B \odot \Supp\right)^\top \ones_\ell} &= \ones_h - \left(\ones_\ell^\top \otimes \identity_h \right) \myvec{\left(\B \odot \Supp\right)^\top} \\
        &= \ones_h - \left(\ones_\ell^\top \otimes \identity_h \right) \mathbf{K}^{\ell,h} \myvec{\B \odot \Supp} \\
        &= \ones_h - \left(\ones_\ell^\top \otimes \identity_h \right) \mathbf{K}^{\ell,h} \myvec{\mathrm{diag}\left(\myvec{\B}\right) \myvec{\Supp}}\\
        &= \ones_h - \underbrace{\left(\ones_\ell^\top \otimes \identity_h \right) \mathbf{K}^{\ell,h} \mathrm{diag}\left(\myvec{\B}\right)}_{\G} \myvec{\Supp} \leq \zeros_h\,.
    \end{aligned}    
\end{equation}
At this point, starting from \Cref{eq:updateS_nonconvex_partial} and exploiting \Cref{eq:Q_and_c_Supdate,eq:inequality_constr_S}, we can pose the SCA recursion: 
\begin{equation}\label{eq:SCA_recursion_S}
    \begin{aligned}        
        \myvec{\Supp}^{q+1} =& \argmin_{\Supp \in \umatdim}\quad \frac{1}{2}\myvec{\Supp}^\top \mathbf{Q} \myvec{\Supp} + \mathbf{c}^\top \myvec{\Supp}\,, \quad\textrm{(QP problem)}\\
        &\textrm{subject to} \quad \ones_h -  \G \myvec{\Supp} \leq \zeros_h\,. \\
        \myvec{\Supp}^{q+1} &= \myvec{\Supp}^q + \gamma^k\left(\myvec{\Supp}^{q+1}-\myvec{\Supp}^q\right)\,. \quad\textrm{(Smoothing)}
    \end{aligned}
\end{equation}
The QP problem in \Cref{eq:SCA_recursion_S} can be solved through off-the-shelf quadratic programming solvers.
In our experiments, we use the OSQP \citeSupp{osqpSupp} implementation available in \texttt{cvxpy} \citeSupp{diamond2016cvxpySupp}.
Since the quadratic form involves a diagonal, positive definite matrix $\mathbf{Q}$, in case a solution exists in the feasible set determined by the inequality constraint, it is also unique.
Regarding the smoothing step, $\gamma^q$ follows \Cref{eq:SCA_stepsize_rule}.
Similarly to \Cref{subsec:updateV_partial}, we determine convergence when
\begin{equation}\label{eq:convergenceS_approx}
    \frob{\myvec{\Supp}^{q+1}-\myvec{\Supp}^q}\leq \tau^c\,, \quad \tau^c \approx 0\,;
\end{equation}
and set $\Supp^{k+1}=\Supp^{q+1}$, where $\Supp^{q+1}$ is the reshaping of $\myvec{\Supp}^{q+1}$ in matrix form.

\subsection{\texorpdfstring{Update for $\YO^{k+1}$ and $\YT^{k+1}$}{Update for Y1 and Y2}}
Starting from \Cref{eq:scaledAUL_partial,eq:ADMM_partial}, the subproblem to solve is
\begin{equation}\label{eq:updateY1_partial}
    \begin{aligned}
        \YO^{k+1}&=\argmin_{\YO \in \stiefel{\ell}{h}}\quad \frac{\rho}{2}\frob{\B \odot \Supp^{k+1} \odot \V^{k+1} - \YO +\scaledUO^k}^2\\
                &=\prox_{\stiefel{\ell}{h}}\left(\widetilde{\mathbf{Y}}_1\right)\,, \quad \text{with } \widetilde{\mathbf{Y}}_1\coloneqq\B \odot \Supp^{k+1} \odot \V^{k+1} +\scaledUO^k\,.
    \end{aligned}            
\end{equation}
The evaluation of $\prox_{\stiefel{\ell}{h}}(\widetilde{\mathbf{Y}_1})$ in \Cref{eq:updateY1_partial} is equivalent to the (unique) solution of the closest orthogonal approximation problem \citeSupp{fan1955someSupp,higham1986computingSupp}.
Specifically, it is equal to the $\mathbf{U}_{p_1}$ factor of the polar decomposition of the matrix $\widetilde{\mathbf{Y}}_1=\mathbf{U}_{p_1} \mathbf{P}_{p_1}$, namely
\begin{equation}\label{eq:updateY1_solution}
    \YO^{k+1}=\mathbf{U}_{p_1}\,.
\end{equation}

Similarly, defining $\widetilde{\mathbf{Y}}_2\coloneqq\B \odot \Supp^{k+1} \odot \V^{k+1} +\scaledUT^k$ and considering the polar decomposition $\widetilde{\mathbf{Y}}_2=\mathbf{U}_{p_2} \mathbf{P}_{p_2}$, we have
\begin{equation}\label{eq:updateY2_solution}
    \YT^{k+1}=\mathbf{U}_{p_2} \,.
\end{equation}

\subsection{\texorpdfstring{Update for $\X^{k+1}$}{Update for X}}
Starting from \Cref{eq:scaledAUL_partial,eq:ADMM_partial}, the subproblem reads as
\begin{equation}\label{eq:updateX_partial}
    \begin{aligned}
        \X^{k+1}&= \argmin_{\X \in \sphere{h}{\ell}} \frac{\rho}{2}\frob{\left(\B \odot \Supp^{k+1}\right)^\top - \X + \scaledW^k}^2\\
                &= \prox_{\sphere{h}{\ell}}\left(\left(\B \odot \Supp^{k+1}\right)^\top + \scaledW^k\right)\,.
    \end{aligned}
\end{equation}
The following result gives the solution.

\begin{lemma}\label{lemma:proximal_spDelta}
    Consider 
    \begin{equation}\label{eq:spDelta}
        \sphere{h}{\ell} \coloneqq \Big\{\mathbf{A} \in \{0,1\}^{h\times\ell} \mid  \norm{\mathbf{a}_j}_2=1 \text{ and }\\
        \sum_{i=1}^h a_{ij}=1,\,\,\forall j \in [\ell]  \Big\}\,;
\end{equation}
    and $\A \in \reall^{h \times \ell}$.
    The proximal operator 
    \begin{equation}\label{eq:proximal_spDelta}
        \prox_{\sphere{h}{\ell}}\left(\A\right)\coloneqq \argmin_{\X \in \rmatdimT} \frob{\A - \X}\,,    
    \end{equation}
     is the matrix $\X^\star$ such that
    \begin{equation}
        \forall \, j \in [\ell], \; x_{ij}^\star = \begin{cases}
            1\,, \quad &\text{if } a_{ij} = \argmin_{i} \abs{a_{ij}-1}\,,\\
            0\,, & \text{otherwise.}
        \end{cases}
    \end{equation}
\end{lemma}
\begin{proof}
    To belong to $\sphere{h}{\ell}$, $\X^\star$ must have only a single nonzero entry equal to one for each column $j \in [\ell]$.
    Consequently, the objective in \Cref{eq:proximal_spDelta} is minimized by setting, for each column $j\in [\ell]$, $x_{ij}^\star=1$ in correspondence of the element $a_{ij}$ whose absolute distance from one is minimum.
\end{proof}

\subsection{Stopping criteria}\label{subsec:stopping_criteria_partial}
The empirical convergence of the proposed method is established according to primal and dual feasibility optimality conditions.
In this case, the primal residuals associated with the equality constraints in \Cref{eq:prob_madmmsca_VS_with_splitting} are
\begin{equation}\label{eq:primal_res_partial}
    \begin{aligned}
        \mathbf{R}_{p,1}^{k+1}&\coloneqq\YO^{k+1}-\B\odot\Supp^{k}\odot\V^{k+1}\,;\\
        \mathbf{R}_{p,2}^{k+1}&\coloneqq\YT^{k+1}-\B\odot\V^{k+1}\odot\Supp^{k+1}\,;\\
        \mathbf{R}_{p,3}^{k+1}&\coloneqq \X^{k+1} - \left(\B \odot \Supp^{k+1}\right)^\top\,.
    \end{aligned}
\end{equation}

Additionally, the dual residuals obtained from the stationarity condition are
\begin{equation}\label{eq:dual_res_partial}
    \begin{aligned}        
        \mathbf{R}_{d,1}^{k+1}&\coloneqq \rho \,\B\odot \Supp^{k} \odot \left(\YO^{k+1}-\YO^k\right)\,;\\
        \mathbf{R}_{d,2}^{k+1}&\coloneqq \rho \,\B\odot \V^{k+1} \odot \left(\YT^{k+1} - \YT^k\right)\,;\\ \mathbf{R}_{d,3}^{k+1}&\coloneqq \rho \, \B \odot \left( \X^{k+1}-\X^k\right)^\top\,.
    \end{aligned}
\end{equation}

Following \citeSupp{boyd2011distributedSupp}, denoting with $\tau^a$ and $\tau^r$ in $\reall_+$ the absolute and relative tolerances, respectively, the stopping criteria to be satisfied for empirical convergence are
\begin{equation}\label{eq:stopping_criteria_partial}
    \begin{aligned}
        \frob{\mathbf{R}_{p,1}^{k+1}}&=d_{p,1}^{k+1}\leq \tau^a \sqrt{\ell h} + \tau^r \max{\left(\frob{\YO^{k+1}}, \frob{\B\odot \Supp^{k} \odot \V^{k+1}}\right)}\,,\\
        \frob{\mathbf{R}_{p,2}^{k+1}}&=d_{p,2}^{k+1}\leq \tau^a \sqrt{\ell h} + \tau^r \max{\left(\frob{\YT^{k+1}}, \frob{\B\odot \V^{k+1} \odot \Supp^{k+1}}\right)}\,,\\
        \frob{\mathbf{R}_{p,3}^{k+1}}&=d_{p,3}^{k+1}\leq \tau^a \sqrt{\ell h} + \tau^r \max{\left(\frob{\X^{k+1}}, \frob{\B \odot \Supp^{k+1}}\right)}\,,\\
        \frob{\mathbf{R}_{d,1}^{k+1}}&=d_{d,1}^{k+1}\leq \tau^a \sqrt{\ell h} + \tau^r \rho\, \frob{\B \odot \Supp^{k} \odot \scaledUO^{k+1}}\,,\\
        \frob{\mathbf{R}_{d,2}^{k+1}}&=d_{d,2}^{k+1}\leq \tau^a \sqrt{\ell h} + \tau^r \rho\, \frob{\B \odot \V^{k+1} \odot \scaledUT^{k+1}}\,,\\
        \frob{\mathbf{R}_{d,3}^{k+1}}&=d_{d,3}^{k+1}\leq \tau^a \sqrt{\ell h} + \tau^r \rho \, \frob{\B^\top \odot \scaledW^{k+1}}\,.\\
    \end{aligned}
\end{equation}

The CLinSEPAL method is summarized in \Cref{alg:clinsepal}

\begin{algorithm}[H]
\caption{CLinSEPAL}
\label{alg:clinsepal}
\begin{algorithmic}[1]
\STATE \textbf{Input:} $\covlow$, $\covhigh$, $\B$, $\rho$, $\tau$, $\varepsilon$, $\tau^c$, $\tau^a$, $\tau^r$
\STATE Initialize: $\V^0 \in \rmatdim$, $\Supp^0 = \B$, $\YO^0 \in \stiefel{\ell}{h}$, $\YT^0 \in \stiefel{\ell}{h}$, $\X^0=\B^\top$, $\scaledUO^0 \gets \B\odot\Supp^0\odot\V^0 - \YO^0$, $\scaledUT^0 \gets \B\odot\Supp^0\odot\V^0 - \YT^0$, $\scaledW^0 \gets (\B \odot \Supp^0)^\top - \X^0$
\REPEAT
    \STATE $\V^{k+1} \gets \text{Apply \cref{eq:SCA_recursion_V}}$ 
    \STATE $\Supp^{k+1} \gets \text{Apply \cref{eq:SCA_recursion_S}}$ 
    \STATE $\YO^{k+1} \gets \text{\cref{eq:updateY1_solution}}$
    \STATE $\YT^{k+1} \gets \text{\cref{eq:updateY2_solution}}$
    \STATE $\scaledUO^{k+1} \gets \scaledUO^k + \B\odot\Supp^{k}\odot\V^{k+1} - \YO^{k+1}$
    \STATE $\scaledUT^{k+1} \gets \scaledUT^k + \B\odot\V^{k+1}\odot\Supp^{k+1} - \YT^{k+1}$
    \STATE $\scaledW^{k+1} \gets \scaledW^k + \left(\B \odot \Supp^{k+1}\right)^\top - \X^{k+1}$
\UNTIL{\Cref{eq:stopping_criteria_partial} is satisfied}
\STATE \textbf{Output:} $\V$, $\Supp$, $\YO$, $\YT$, $\X$, $\scaledUO$, $\scaledUT$, $\scaledW$
\end{algorithmic}
\end{algorithm} 

\subsection{Full prior case}\label{subsec:CLinSEPAL_full_prior}
\Cref{prob:nonconvex_prob_approx} simplifies in case of full prior knowledge of \B. 
Indeed, it is not needed to learn \Supp since $\Supp \equiv \B$.
Accordingly, we get the following.

\begin{problem}\label{prob:nonconvex_prob_full}
Given $\covlow \in \pd^{\ell}$, $\covhigh \in \pd^{h}$, and $\B \in \lmatdim$, the linear constructive CA is given by the transpose of the product $\B \odot \V$, where 
    \begin{equation}\label{eq:prob_madmmsca_V}
        \begin{aligned}
            \V^\star = \argmin_{\V \in \rmatdim} &\quad f(\V)\,;\\
             \textrm{subject to} & \quad \B \odot \V \in \stiefel{\ell}{h}\,; \\
        \end{aligned}
    \end{equation}
    and
    \begin{equation}\label{eq:objective_full_knowledge}
        f(\V) \coloneqq \Tr{\left(\left(\B \odot \V\right)^\top \covlow \left(\B \odot \V\right) \right)^{-1} \covhigh} + \log\det {\left(\B \odot\V\right)^\top \covlow \left(\B \odot\V\right) }\,.
    \end{equation}
\end{problem}

The solution can be obtained in a similar manner as for the partial prior knowledge case.
Below, we report the mathematical derivation for completeness without further comments.

\begin{corollary}\label{corollary:objective_full_knowledge}
    The function $f(\V)$ in \Cref{eq:objective_full_knowledge} is smooth.
    Additionally, define $\mathbf{A}\coloneqq\left(\B \odot \V\right)$ and $\widetilde{\mathbf{A}}\coloneqq\left(\mathbf{A}^\top \covlow \mathbf{A}\right)^{-1}$.
    The gradient is
    \begin{equation}\label{eq:gradV_full}
        \Egrad{\V}{f} = 2\B \odot  \left(\left(\covlow\mathbf{A}\widetilde{\mathbf{A}}\right)\left(\identity_h - \covhigh\widetilde{\mathbf{A}}\right)\right)\,,
    \end{equation}
\end{corollary}
\begin{proof}
    Smoothness directly follows from \Cref{prop:smoothness_and_differentiability} by defining $\mathbf{A}\coloneqq\left(\B \odot \V\right)$, which is constrained to \stiefel{\ell}{h} as given in \Cref{eq:prob_madmmsca_V}.
    The gradient in \Cref{eq:gradV_full} follows from the application of \Cref{eq:rules_matrix_calculus}, together with the chain rule for derivatives. 
\end{proof}

Starting from \Cref{eq:prob_madmmsca_V}, we get the following equivalent minimization problem
\begin{equation}\label{eq:prob_madmmsca_V_with_splitting}
    \begin{aligned}
        \V^\star, \Y^\star = \argmin_{\substack{\V \in \rmatdim\\ \Y \in \stiefel{\ell}{h} \\}} &\quad f(\V)\,;\\
         \textrm{subject to} & \quad \Y - \B \odot \V = \zeros_{\ell \times h}\,.
    \end{aligned}
\end{equation}

Considering the scaled dual variable $\scaledU \in\rmatdim$, the scaled augmented Lagrangian is
\begin{equation}\label{eq:scaledAUL_full}
    L_\rho\left(\V,\Y,\scaledU\right)=f(\V) + \frac{\rho}{2}\frob{\B\odot\V - \Y +\scaledU}^2 \,.
\end{equation}

The ADMM recursion is
\begin{equation}\label{eq:ADMM_full}
    \begin{aligned}        
        \V^{k+1}=&\argmin_{\V \in \rmatdim} L_\rho\left(\V,\Y^k,\scaledU^k\right)\,;\\
        \Y^{k+1}=&\argmin_{\Y \in \stiefel{\ell}{h}} L_\rho\left(\V^{k+1},\Y,\scaledU^k\right)\,;\\
        \scaledU^{k+1}=&\scaledU^k + \left(\B\odot\V^{k+1} - \Y^{k+1}\right)\,.\\
    \end{aligned}
\end{equation}

\subsubsection{\texorpdfstring{Update for $\V^{k+1}$}{Update for V}}\label{subsec:updateV_full}
Starting from \Cref{eq:scaledAUL_full,eq:ADMM_full}, the subproblem we have to solve is
\begin{equation}\label{eq:updateV_nonconvex_full}
    \V^{k+1}=\argmin_{\V \in \rmatdim}\quad f(\V) + \frac{\rho}{2}\frob{\B\odot\V - \Y^k +\scaledU^k}^2\,.
\end{equation}
\Cref{eq:updateV_nonconvex_full} is nonconvex due to the inherent nonconvexity of $f(\V)$.
However, the latter function is smooth and differentiable w.r.t. \V, as given in \Cref{corollary:objective_full_knowledge}.
Hence, we apply the SCA framework.
In detail, denote by $q$ the SCA iteration and set $\V^0=\V^k$ for $q=0$.
We derive a strongly convex surrogate $\widetilde{f}(\V;\V^q)$ around the point $\V^q$ -- i.e., the solution at the iterate $q$ -- exploiting \Cref{eq:gradV_full}, 
\begin{equation}\label{eq:strongly_convex_surrogate_Vfull}
    \widetilde{f}(\V;\V^q) \coloneqq \Tr{\Egrad{\V}{f}\at{\V^q}^\top \left(\V - \V^q\right)} + \frac{\tau}{2}\frob{\V -\V^q}^2\,.
\end{equation}

Therefore, at each SCA iteration $q$, we solve a strongly convex problem in closed-form and then apply the usual smoothing operation by using a diminishing stepsize $\gamma^q \in \reall_+$ following \cref{eq:SCA_stepsize_rule} and satisfying \cref{eq:SCA_stepsize_conditions}.
Specifically,
\begin{equation}\label{eq:SCA_recursion_V_full}
    \begin{aligned}        
        \V^{q+1} &= \argmin_{\V \in \rmatdim}\quad \widetilde{f}(\V;\V^q) +\frac{\rho}{2}\frob{\B\odot\V - \Y^k +\scaledU^k}^2\,, \quad\textrm{(Strongly convex problem)}\\
        \V^{q+1} &= \V^q + \gamma^k\left(\V^{q+1}-\V^q\right)\,. \quad\textrm{(Smoothing)}
    \end{aligned}
\end{equation}

The solution of the strongly-convex problem is given element-wise in \Cref{lemma:updateV_elementwise_full}. 
\begin{lemma}\label{lemma:updateV_elementwise_full}
    The update for $\V^{q+1}$ can be computed element-wise as
    \begin{equation}\label{updateV_elementwise_full}
        v_{ij}^{q+1}= \frac{1}{\tau + b_{ij}}\Bigg(\rho\, b_{ij} y_{ij}^k - \rho\, b_{ij} u_{ij}^k + \tau\, v_{ij}^q - \Big[\Egrad{\V}{f\at{\V^q}}\Big]_{ij} \Bigg)\,.
    \end{equation}
\end{lemma}
\begin{proof}
    The proof follows by imposing the stationarity condition
    \begin{equation}
        \mathbf{0}_{\ell\times h} = \Egrad{\V}{f\at{\V^q}} + \tau \left(\V-\V^q\right) + \rho \, \B \odot \left( \B \odot \V - \Y^k +\scaledU^k\right)\,,
    \end{equation}
    and solving for \V.
\end{proof}

We establish convergence for the update when
\begin{equation}\label{eq:convergenceV_full}
    \frob{\V^{q+1}-\V^q}\leq \tau^c\,, \quad \tau^c \approx 0\,;
\end{equation}
and set $\V^{k+1}=\V^{q+1}$.

\subsubsection{\texorpdfstring{Update for $\Y^{k+1}$}{Update for Y}}
Starting from \Cref{eq:scaledAUL_full,eq:ADMM_full}, the subproblem to solve is
\begin{equation}\label{eq:updateY_full}
    \begin{aligned}
        \Y^{k+1}&=\argmin_{\Y \in \stiefel{\ell}{h}}\quad \frac{\rho}{2}\frob{\B \odot \V^{k+1} - \Y +\scaledU^k}^2\\
                &=\prox_{\stiefel{\ell}{h}}(\widetilde{\mathbf{Y}})\,, \quad \text{with } \widetilde{\mathbf{Y}}\coloneqq\B \odot \V^{k+1} +\scaledU^k\,.
    \end{aligned}            
\end{equation}
Denoting by $\mathbf{U}_p \mathbf{P}_p$ the polar decomposition of the matrix $\widetilde{\mathbf{Y}}$, the update is
\begin{equation}\label{eq:updateY_solution}
    \Y^{k+1}=\mathbf{U}_p\,.
\end{equation}

\subsubsection{Stopping criteria}\label{subsec:stopping_criteria_full}
The empirical convergence of the proposed method is established according to primal and dual feasibility optimality conditions \citeSupp{boyd2011distributedSupp}.
The primal residual, associated with the equality constraint in \Cref{eq:prob_madmmsca_V_with_splitting}, is
\begin{equation}\label{eq:primal_res_full}
    \mathbf{R}_p^{k+1}\coloneqq\Y^{k+1}-\B\odot\V^{k+1}\,.
\end{equation}
The dual residual, which can be obtained from the stationarity condition, is
\begin{equation}\label{eq:dual_res_full}
    \mathbf{R}_d^{k+1}\coloneqq \rho \,\B \odot\left(\Y^{k+1}-\Y^k\right)\,.
\end{equation}
As $k \rightarrow \infty$, the norm of the primal and dual residuals should vanish.
Hence, the stopping criterion can be set in terms of the norms
\begin{equation}\label{eq:norms_full}
    \text{\emph{(i)}}\;d_p^{k+1}=\frob{\mathbf{R}_p^{k+1}} \quad \text{and} \quad \text{\emph{(ii)}}\;d_d^{k+1}=\frob{\mathbf{R}_d^{k+1}}\,. 
\end{equation}
Specifically, given absolute and relative tolerance, namely $\tau^a$ and $\tau^r$ in $\reall_+$, respectively, convergence in practice is established following \citetSupp{boyd2011distributedSupp}, when 
\begin{equation}\label{eq:stopping_criteria_full}
    \text{\emph{(i)}}\; d_p^{k+1} \leq \tau^a\sqrt{\ell h} + \tau^r \max{\left(\frob{\Y^{k+1}}, \frob{\B \odot \V^{k+1}}\right)}\,, \quad
    \text{and} \quad
    \text{\emph{(ii)}}\; d_d^{k+1} \leq \tau^a\sqrt{\ell h} + \tau^r \rho  \frob{\B \odot \scaledU^{k+1}}\,.
\end{equation}

The full prior version of CLinSEPAL is summarized in \Cref{alg:clinsepal_fullprior}.

\begin{algorithm}[H]
\caption{CLinSEPAL (full prior case)}
\label{alg:clinsepal_fullprior}
\begin{algorithmic}[1]
\STATE \textbf{Input:} $\covlow$, $\covhigh$, $\B$, $\rho$, $\tau$, $\varepsilon$, $\tau^c$, $\tau^a$, $\tau^r$
\STATE Initialize: $\V^0 \in \rmatdim$, $\Y^0 \in \stiefel{\ell}{h}$, $\scaledU^0 \gets \B\odot\V^0 - \Y^0$
\REPEAT
    \STATE $\V^{k+1} \gets \text{Apply \cref{eq:SCA_recursion_V_full}}$
    \STATE $\Y^{k+1} \gets \text{\cref{eq:updateY_solution}}$
    \STATE $\scaledU^{k+1} \gets \scaledU^k + \B\odot\V^{k+1} - \Y^{k+1}$
\UNTIL{\Cref{eq:stopping_criteria_full} is satisfied}
\STATE \textbf{Output:} $\V$, $\Y$, $\scaledU$
\end{algorithmic}
\end{algorithm} 

\newcommand{\ResultEightByEightCrossbarOverheadkGE}{13.1}
\newcommand{\ResultEightByEightCrossbarOverheadPercent}{9}
\newcommand{\ResultSixteenBySixteenCrossbarOverheadkGE}{45.4}
\newcommand{\ResultSixteenBySixteenCrossbarOverheadPercent}{12}
\newcommand{\ResultAsymptoticOverheadPercent}{21.6}
\newcommand{\ResultSixteenBySixteenCrossbarFrequencyOverheadPercent}{6}
\newcommand{\ResultThirtyTwoClusterEightKiBParallelFraction}{97}
\newcommand{\ResultThirtyTwoClusterTwoKiBSpeedup}{13.5}
\newcommand{\ResultThirtyTwoClusterThirtyTwoKiBSpeedup}{16.2}
\newcommand{\ResultThirtyTwoClusterGeometricMeanSpeedup}{5.6}
\newcommand{\ResultBaselineTileNOperationalIntensity}{1.9}
\newcommand{\ResultBaselineTileNPerformanceGFLOPS}{114.4}
\newcommand{\ResultBaselineTileNPerformancePercentage}{92}
\newcommand{\ResultHybridTileNOperationalIntensityIncrease}{3.7}
\newcommand{\ResultHybridTileNPerformanceIncrease}{2.6}
\newcommand{\ResultMulticastTileNOperationalIntensityIncrease}{16.5}
\newcommand{\ResultMulticastTileNPerformanceIncrease}{3.4}
\newcommand{\ResultMulticastTileNPerformanceIncreaseOverHybridPercentage}{29}
\newcommand{\ResultMulticastTileNPerformanceGFLOPS}{391.4}

\clearpage
% !TEX root =  ../main.tex
\section{Additional material for the causal abstraction of brain networks}\label{app:rw_figs}

This section provides additional material about the full and partial prior applications of CLinSEPAL to brain data, given in \Cref{sec:empirical_assessment_rw}.
Specifically, \Cref{fig:ROIsLobes} depicts the ground truth linear CA and the learned linear CA by CLinSEPAL for the full prior setting; whereas \Cref{fig:ROIsFun_ca} the results for the partial prior setting.
Regarding the partial prior setting, we also report the monitored metrics to better understand the performance of CLinSEPAL with varying degree of uncertainty (low, medium, high), as discussed in \Cref{sec:empirical_assessment_rw}.
The color coding for the partial prior setting refers to the following classification, reported unaltered from \citeSupp{gabriele2024extractingSupp}:
\begin{squishlist}
    \item Red for ROIs corresponding to cognitive functions, attention, emotion, and decision-making;
    \item Orange for those related to auditory processing, speech and language processing, and memory;
    \item Blue for those concerning memory formation and memory retrieval;
    \item Pink for those associated with sensory integration and somatosensory;
    \item Purple for the ROIs within the visual network and related to the visual memory;
    \item Green for those within the motor network;
    \item Yellow for those regarding the motor control and the posture.
\end{squishlist}

\begin{figure}
    \centering
    \includegraphics[width=1.\textwidth]{figs/30_case_study_1_day1_day1.pdf}
    \caption{The figure shows (top) the ground truth linear CA and (bottom) the learned linear CA for the simulated full prior setting in \Cref{sec:empirical_assessment_rw}.}
    \label{fig:ROIsLobes}
\end{figure}



\begin{figure}
    \centering
    \includegraphics[width=1.\textwidth]{figs/30_case_study_2_ca.pdf}
    \caption{Starting from the top, the figure shows \emph{(i)} the ground truth linear CA, and the learned linear CA for the simulated partial prior setting with \emph{(ii)} low, \emph{(iii)} medium, and \emph{(iv)} high uncertainty in \Cref{sec:empirical_assessment_rw}.}
    \label{fig:ROIsFun_ca}
\end{figure}

\begin{figure}
    \centering
    \includegraphics[width=1.\textwidth]{figs/30_case_study_2_metrics.pdf}
    \caption{Starting from the left, the figure provides the \emph{(i)} the KL divergence evaluated at the learned \Vhat, \emph{(ii)} the Frobenious absolute distance, \emph{(iii)} the true positive rate, and \emph{(iv)} the false discovery rate for the simulated partial prior setting with low, medium, and high uncertainty in \Cref{sec:empirical_assessment_rw}.}
    \label{fig:ROIsFun_metrics}
\end{figure}

\clearpage
\bibliographystyleSupp{icml2025}
\bibliographySupp{bibliographySupp}
%%%%%%%%%%%%%%%%%%%%%%%%%%%%%%%%%%%%%%%%%%%%%%%%%%%%%%%%%%%%%%%%%%%%%%%%%%%%%%%
%%%%%%%%%%%%%%%%%%%%%%%%%%%%%%%%%%%%%%%%%%%%%%%%%%%%%%%%%%%%%%%%%%%%%%%%%%%%%%%


\end{document}


% This document was modified from the file originally made available by
% Pat Langley and Andrea Danyluk for ICML-2K. This version was created
% by Iain Murray in 2018, and modified by Alexandre Bouchard in
% 2019 and 2021 and by Csaba Szepesvari, Gang Niu and Sivan Sabato in 2022.
% Modified again in 2023 and 2024 by Sivan Sabato and Jonathan Scarlett.
% Previous contributors include Dan Roy, Lise Getoor and Tobias
% Scheffer, which was slightly modified from the 2010 version by
% Thorsten Joachims & Johannes Fuernkranz, slightly modified from the
% 2009 version by Kiri Wagstaff and Sam Roweis's 2008 version, which is
% slightly modified from Prasad Tadepalli's 2007 version which is a
% lightly changed version of the previous year's version by Andrew
% Moore, which was in turn edited from those of Kristian Kersting and
% Codrina Lauth. Alex Smola contributed to the algorithmic style files.
