\documentclass{article}

% Recommended, but optional, packages for figures and better typesetting:
\usepackage{microtype}
\usepackage{graphicx}
\usepackage{subfigure}
\usepackage{booktabs} % for professional tables

% hyperref makes hyperlinks in the resulting PDF.
% If your build breaks (sometimes temporarily if a hyperlink spans a page)
% please comment out the following usepackage line and replace
% \usepackage{icml2025} with \usepackage[nohyperref]{icml2025} above.
\usepackage{hyperref}


% Attempt to make hyperref and algorithmic work together better:
\newcommand{\theHalgorithm}{\arabic{algorithm}}

% Use the following line for the initial blind version submitted for review:
% \usepackage{icml2025}

% If accepted, instead use the following line for the camera-ready submission:
\usepackage[accepted]{include/icml2025}

% For theorems and such
\usepackage{amsmath}
\usepackage{amssymb}
\usepackage{mathtools}
\usepackage{amsthm}

% if you use cleveref..
\usepackage[capitalize,noabbrev]{cleveref}

%%%%%%%%%%%%%%%%%%%%%%%%%%%%%%%%
% THEOREMS
%%%%%%%%%%%%%%%%%%%%%%%%%%%%%%%%
\theoremstyle{plain}
\newtheorem{theorem}{Theorem}[section]
\newtheorem{proposition}[theorem]{Proposition}
\newtheorem{lemma}[theorem]{Lemma}
\newtheorem{corollary}[theorem]{Corollary}
\theoremstyle{definition}
\newtheorem{definition}[theorem]{Definition}
\newtheorem{assumption}[theorem]{Assumption}
\theoremstyle{remark}
\newtheorem{remark}[theorem]{Remark}

% Todonotes is useful during development; simply uncomment the next line
%    and comment out the line below the next line to turn off comments
\usepackage[disable,textsize=tiny]{todonotes}
% \usepackage[textsize=tiny]{todonotes}

% The \icmltitle you define below is probably too long as a header.
% Therefore, a short form for the running title is supplied here:
\icmltitlerunning{FAB-PPI}

% Our definitions
% Lists
\usepackage[inline]{enumitem}
\newlist{enuminline}{enumerate*}{1}
\setlist[enuminline]{label=(\roman*)}

% Math
\newcommand{\vol}{\mathrm{vol}}
\newcommand{\E}{\mathbb E}
\newcommand{\var}{\mathrm{var}}
\newcommand{\cov}{\mathrm{cov}}
\newcommand{\Normal}{\mathcal N}
\newcommand{\slowvar}{\mathcal L}
\newcommand{\bbR}{\mathbb R}
\newcommand{\bbE}{\mathbb E}
\newcommand{\bbP}{\mathbb P}
\newcommand{\calC}{\mathcal C}
\newcommand{\calR}{\mathcal R}
\newcommand{\calT}{\mathcal T}
\newcommand{\loA}{\underline{A}}
\newcommand{\upA}{\overline{A}} 
\newcommand{\cv}{\mathrm{cv}} 
\newcommand{\pp}{\mathrm{pp}}
\newcommand{\HS}{\mathrm{HS}}
\newcommand{\erfi}{\mathrm{erfi}}
\newcommand{\tX}{\widetilde X}
\newcommand{\OneFOne}{{}_1F_1}
\newcommand{\PP}{\mathrm{\texttt{PP}}}
\newcommand{\PPpp}{\mathrm{\texttt{PP+}}}
\newcommand{\BPP}{\mathrm{\texttt{BPP}}}
\newcommand{\BPPpp}{\mathrm{\texttt{BPP+}}}
\newcommand{\FABPPI}{\mathrm{\texttt{FABPP}}}
\newcommand{\asto}{\overset{\text{a.s.}}{\to}}

\begin{document}

\twocolumn[
\icmltitle{FAB-PPI: Frequentist, Assisted by Bayes, Prediction-Powered Inference}

% It is OKAY to include author information, even for blind
% submissions: the style file will automatically remove it for you
% unless you've provided the [accepted] option to the icml2025
% package.

% List of affiliations: The first argument should be a (short)
% identifier you will use later to specify author affiliations
% Academic affiliations should list Department, University, City, Region, Country
% Industry affiliations should list Company, City, Region, Country

% You can specify symbols, otherwise they are numbered in order.
% Ideally, you should not use this facility. Affiliations will be numbered
% in order of appearance and this is the preferred way.
\icmlsetsymbol{equal}{*}

\begin{icmlauthorlist}
    \icmlauthor{Stefano Cortinovis}{oxford}
    \icmlauthor{Fran\c cois Caron}{oxford}
\end{icmlauthorlist}

\icmlaffiliation{oxford}{Department of Statistics, University of Oxford}

\icmlcorrespondingauthor{Stefano Cortinovis}{cortinovis@stats.ox.ac.uk}

% You may provide any keywords that you
% find helpful for describing your paper; these are used to populate
% the "keywords" metadata in the PDF but will not be shown in the document
\icmlkeywords{Confidence intervals, Bayesian methods, heavy-tailed priors, horseshoe, mean estimation, statistical inference, semi-supervised inference}

\vskip 0.3in
]

% this must go after the closing bracket ] following \twocolumn[ ...

% This command actually creates the footnote in the first column
% listing the affiliations and the copyright notice.
% The command takes one argument, which is text to display at the start of the footnote.
% The \icmlEqualContribution command is standard text for equal contribution.
% Remove it (just {}) if you do not need this facility.

\printAffiliationsAndNotice{}  % leave blank if no need to mention equal contribution
% \printAffiliationsAndNotice{\icmlEqualContribution} % otherwise use the standard text.

\begin{abstract}
    Prediction-powered inference (PPI) enables valid statistical inference by combining experimental data with machine learning predictions. When a sufficient number of high-quality predictions is available, PPI results in more accurate estimates and tighter confidence intervals than traditional methods. In this paper, we propose to inform the PPI framework with prior knowledge on the quality of the predictions. The resulting method, which we call frequentist, assisted by Bayes, PPI (FAB-PPI), improves over PPI when the observed prediction quality is likely under the prior, while maintaining its frequentist guarantees. Furthermore, when using heavy-tailed priors, FAB-PPI adaptively reverts to standard PPI in low prior probability regions. We demonstrate the benefits of FAB-PPI in real and synthetic examples.
\end{abstract}

%% Introduction
\section{Introduction}
Statistical inference crucially relies on the availability of high-quality labelled data to draw actionable conclusions. As the scale of machine learning models keeps growing, their increasingly accurate predictions become a tempting alternative to labelled data in fields where the latter are traditionally scarce, such as proteomics \citep{Jumper2021}. However, blindly using potentially biased predictions as a surrogate for labelled data voids the statistical validity of the conclusions drawn. To address this, prediction-powered inference~\citep{Angelopoulos2023} provides a general framework for statistical inference in the presence of a large number of black-box predictions by combining them with a smaller number of labelled observations, which are used to \textit{correct} for the discrepancy between the predictions and the true labels. The estimators and confidence intervals (CIs) resulting from PPI are statistically valid regardless of the machine learning model used. Moreover, when the predictions are good, PPI results in more accurate estimates and shorter CIs than traditional methods that rely solely on labelled data. \looseness=-1

More formally, for an input/output pair $(X,Y)\sim \bbP=\bbP_X\times\bbP_{Y|X}$ and a convex loss function $\mathcal{L}_\theta(x,y)$, where $\theta\in\bbR^d$, we wish to estimate
\begin{align}
    \theta^\star=\underset{\theta\in\bbR^d}{\arg\min}~~\E[\mathcal{L}_\theta(X,Y)]. \label{eq:defthetastar1}
\end{align}
For instance, if $\mathcal{L}_\theta(x,y)=(\theta - y)^2/2$ is the squared loss, then $\theta^\star=\E[Y]$.
We assume that we have $n$ labelled observations $\{(X_i,Y_i)\}_{i=1}^n$ iid from $\bbP$ and $N$ unlabelled observations $\{\widetilde X_{i}\}_{i=1}^N$ iid from $\bbP_X$, which are also independent of the labelled data. The number of unlabelled observations is typically much larger than the number of labelled ones, $N\gg n$. Additionally, we are provided with a machine learning prediction rule $f$, that can be used to predict an output $f(x)$ at any input $x$. PPI aims to obtain an estimator $\widehat \theta$ and a $1-\alpha$ CI $\calC_\alpha^\pp$ for $\theta^\star$, which take advantage of $f$. Under mild assumption, $\theta^\star$ can be expressed as the solution to \looseness=-1
\begin{align}
    g_{\theta^\star}:=\E[\mathcal{L}_{\theta^\star}'(X,Y)]=0, \label{eq:defthetastar2}
\end{align}
where $\mathcal{L}_\theta'$ is a subgradient of $\mathcal{L}_\theta$ with respect to $\theta$. It is easy to see that the quantity above can be decomposed as $g_{\theta} = m_\theta + \Delta_\theta$, where 
\begin{align}
    m_\theta&:=\E[\mathcal{L}_{\theta}'(X,f(X))], \label{eq:m_theta} \\
    \Delta_\theta&:=\E[\mathcal{L}_{\theta}'(X,Y)-\mathcal{L}_{\theta}'(X,f(X))] \label{eq:Delta_theta}.
\end{align}
In this setting, $m_\theta$ represents a measure of fit of the predictor, whereas $\Delta_\theta$, called the \textit{rectifier}, accounts for the discrepancy between the predicted outputs $f(X)$ and the true outputs $Y$.
In the case of the squared loss, $\Delta_\theta=\E[f(X)-Y]$ does not depend on $\theta$, but this is not true in general.
By estimating the two quantities $m_\theta$ and $\Delta_\theta$, \citet{Angelopoulos2023a} derive an estimator and a CI for $\theta^\star$, which use both labelled and unlabelled data. The resulting CI is shorter than the classical confidence interval based solely on the labelled data when $N \gg n$ and $f$ is accurate because, in this case, $m_\theta$ can be estimated with low variance using the unlabelled data, while $\Delta_\theta$ is close to zero.

Standard PPI employs off-the-shelf estimation and CI procedures for $\Delta_\theta$, which do not take advantage of any prior knowledge on the quality of the machine learning model $f$. 
However, in many applications, we expect the latter's predictions to be
\begin{enuminline}
    \item usually very good, but
    \item sometimes prone to large errors and hallucinations.
\end{enuminline}
We propose to encode such an inductive bias with a horseshoe prior $\pi_\theta$ on $\Delta_\theta$~\citep{Carvalho2010}, which accommodates the aforementioned properties by exhibiting
\begin{enuminline}
    \item an infinitely tall spike at the origin, and
    \item Cauchy-like tails at infinity.
\end{enuminline}
In order to construct valid confidence regions for $\Delta_\theta$ using the horseshoe prior, we resort to the frequentist-assisted by Bayes (FAB) framework \citep{Pratt1961,Pratt1963,Yu2018}. 
This approach provides confidence regions such that their expected length is lower for rectifiers $\Delta_\theta$ that have high probability under $\pi_\theta$, and larger otherwise.
While the resulting confidence regions have exact coverage for any prior $\pi_\theta$, the horseshoe prior is particularly well-suited for PPI. Being concentrated around the origin, it produces shorter confidence regions when the predictions are good, i.e.~$||\Delta_\theta|| \simeq 0$.
At the same time, its heavy tails ensure robustness when the predictions are poor. Indeed, as shown by \citet{Cortinovis2024}, if $||\Delta_\theta||\gg0$, the FAB procedure with the horseshoe prior reverts to the traditional CI based on the sample mean.

In this work, we introduce FAB-PPI, a Bayes-assisted approach for PPI that encodes prior information on the quality of the machine learning predictions by specifying a prior for the rectifier $\Delta_\theta$. FAB-PPI is:
\begin{itemize}\setlength\itemsep{0em}
    \item Statistically valid, as its confidence regions have correct coverage for any choice of prior;
    \item Efficient, as its confidence regions have smaller expected length when the predictions are good;
    \item Robust, as it reverts to standard PPI when the predictions are poor, if the horseshoe prior is used;
    \item Modular, as it can be used in conjunction with power tuning \citep{Angelopoulos2023a}.
\end{itemize}

The remainder of the paper is organised as follows. \cref{sec:relatedwork} reviews related work. \cref{sec:background} provides background on control variates, PPI, and FAB confidence regions. \cref{sec:fab-ppi} describes our novel approach for PPI, called FAB-PPI. \cref{sec:experiments} demonstrates the benefits of FAB-PPI on synthetic and real data. Finally, \cref{sec:discussion} discusses limitations and further extensions of our approach.


%% Related work
\section{Related work}
\label{sec:relatedwork}
PPI \citep{Angelopoulos2023} was introduced to obtain shorter CIs for the parameters of interest by leveraging machine learning predictions in semi-supervised settings. PPI has since been extended in multiple directions. PPI\texttt{++} \citep{Angelopoulos2023a} proposes a different, loss-based formulation of PPI, leading to a more computationally efficient procedure, along with an additional power-tuning parameter to enhance PPI's performance. Stratified PPI \citep{Fisch2024} improves upon PPI by employing a data stratification strategy. Cross PPI \citep{Zrnic2024a} demonstrates how the training of $f$ can be included in the PPI pipeline. Active statistical inference \citep{Zrnic2024} applies an active learning approach to select which inputs from the unlabelled set should be labelled. Closer to our work, Bayesian PPI \citep{Hofer2024} considers an alternative PPI estimator motivated by Bayesian ideas. However, their approach provides Bayesian credible intervals, which do not offer frequentist guarantees. Additionally, their approach achieves similar experimental performance to PPI, while we demonstrate that FAB-PPI may significantly improve upon PPI. \looseness=-1

As discussed in \citet{Angelopoulos2023,Angelopoulos2023a}, PPI has close ties with control variates for variance reduction \citep[\S4.1]{Glasserman2003}. In the case of mean estimation, the form of the PPI estimator is similar to the one proposed by \citet{Zhang2019}. PPI is also related to work in semiparametric inference with missing data \citep{Robins1995}.

The concept of Bayes-optimal confidence regions originates from the work of \citet{Pratt1961,Pratt1963}. Pratt's approach, which has been given the name FAB by \citet{Yu2018}, has since been extended in multiple directions \citep{Brown1995,Farchione2008,Kabaila2013,Kabaila2022,Yu2018,Hoff2019,Hoff2023}. In particular, \citet{Cortinovis2024} show that, when combined with priors with power-law tails, FAB provides robust confidence regions that revert to classical ones in the presence of outliers. \citet{Hoff2023} applied FAB in a predictive supervised context, showing that it can lead to more accurate predictions than standard methods.


%% Background
\section{Background}
\label{sec:background}

\subsection{Control variates}
The method of control variates is a standard variance reduction technique in Monte Carlo approximation \citep[\S4.1]{Glasserman2003}. For simplicity, we present the method in the scalar case, but extensions to the multivariate setting are available.
Let $(Z,Y)$ be a pair of real-valued random variables, and assume we are interested in estimating $\E [Y]$ based on an iid sample $\{(Z_i,Y_i)\}_{i=1}^n$. Assuming $\mu=\E [Z]$ is known, one defines the control variate estimator (CVE)
\begin{align}
    \widehat Y^{\cv}_\lambda=\overline Y - \lambda (\overline Z - \mu)=\frac{1}{n}\sum_{i=1}^n \left( Y_i - \lambda(Z_i-\mu)\right), \label{eq:cvestimator}
\end{align}
where $\lambda\in\bbR$ is a tuning coefficient and $\overline Z$ and $\overline Y$ are the sample means of $(Z_i)$ and $(Y_i)$, respectively. The centered random variable $Z_i-\mu$ serves as a control variate to estimate $\E[Y]$. The CVE is a consistent and unbiased estimator of $\E[Y]$ with $\var(\widehat Y^{\cv}_\lambda)=(\var(Y)-2\lambda \cov(Z,Y)+\lambda^2 \var(Z))/n$, while $\var(\overline Y)=\var(Y)/n$. Therefore, CVE achieves smaller variance whenever $\lambda<\frac{2\cov(Z,Y)}{\var(Z)}$, and the optimal coefficient is given by $\lambda^* = \cov(Z,Y) / \var(Z)$.
In this case, $\var(\widehat Y^{\cv}_{\lambda^*}) =(1-\rho_{Z,Y}^2)\var(\overline Y)$, where $\rho_{Z,Y}$ is the correlation between $Z$ and $Y$. The more correlated $Z$ and $Y$, the larger the variance reduction.
By plugging the estimator \looseness=-1
\begin{align}
    \widehat \lambda = \frac{\sum_{i=1}^n (Z_i-\overline Z)(Y_i-\overline Y) }{\sum_{i=1}^n (Z_i-\overline Z)^2 } \label{eq:estimatorlambda}
\end{align}
for $\lambda$ in \cref{eq:cvestimator}, one has
$$
    \frac{\widehat Y^\cv_{\widehat \lambda} - \E[Y]}{s/\sqrt{n}}\to \Normal(0,1)
$$
as $n\to\infty$, where $s$ is the sample standard deviation of $\{(Y_i - \widehat\lambda Z_i)\}_{i=1,\ldots,n}$. Hence, $\widehat Y^\cv_{\widehat \lambda} \pm z_{1-\alpha/2}s/\sqrt{n}$ is an asymptotically valid $1-\alpha$ CI for $\E[Y]$, whose asymptotic width is $2z_{1-\alpha/2}\sqrt{1-\rho_{ZY}^2}\sqrt{\var(Y)}/\sqrt{n}$.

\subsection{Prediction-powered inference}
PPI~\citep{Angelopoulos2023} defines an estimator $\widehat \theta$ and a CI $\calC_\alpha^\pp$ for a parameter of interest $\theta^\star$ satisfying \cref{eq:defthetastar2}.
In particular, let $\widehat m_\theta$ and $\widehat \Delta_\theta$ be some estimators of $m_\theta$ and $\Delta_\theta$. Using \cref{eq:defthetastar2}, the estimator $\widehat \theta$ is defined as the solution, in $\theta$, to the equation
\begin{align}
    \widehat m_\theta + \widehat \Delta_\theta=0.
\end{align}
Similarly, let $\calR_\delta$ and $\calT_{\alpha-\delta}$ be $1 - \delta$ and $1 - (\alpha - \delta)$ CIs for $\Delta_\theta$ and $m_\theta$, respectively. Then, the PPI confidence interval $\calC_\alpha^\pp$ is defined as
\begin{align}
    \calC_\alpha^\pp=\left\{\theta \mid 0\in \calR_\delta + \calT_{\alpha-\delta}\right\}, \label{eq:ppconfidenceinterval}
\end{align}
where $+$ denotes the Minkowski sum.
Typical choices for $\widehat m_\theta$ and $\calT_{\alpha-\delta}$ are the sample mean of the unlabelled data,
\begin{align}
    \widehat m_\theta&=\frac{1}{N}\sum_{i=1}^{N} \mathcal{L}_{\theta}'(\widetilde X_i,f(\widetilde X_i)), \label{eq:msamplemean}
\end{align}
and classical CIs for sample means, respectively.
Different choices for $\widehat \Delta_\theta$ have been proposed in the literature, leading to different PPI estimators.

\paragraph{Standard PPI.}
\citet{Angelopoulos2023} propose to use the sample mean
\begin{align}
    \widehat \Delta^\PP_\theta&=\frac{1}{n}\sum_{i=1}^n \left(\mathcal{L}_{\theta}'(X_i,Y_i)-\mathcal{L}_{\theta}'(X_i,f(X_i))\right)\label{eq:deltasamplemean}
\end{align}
as an estimator for $\Delta_\theta$ and the associated classical CIs to construct $\calR_\delta$.
For the squared loss, the estimator $\widehat\theta^\PP$ solving $\widehat m_\theta + \widehat \Delta_\theta=0$ takes the control variate form
\begin{align}
    \widehat\theta^\PP= \overline Y - \left( \frac{1}{n}\sum_{i=1}^n f(X_i) - \frac{1}{N}\sum_{j=1}^{N} f(\tX_j)     \right) \label{eq:thetahatppi}
\end{align}
with control variate $f(X_i)-\frac{1}{N}\sum_{j=1}^{N} f(\tX_j)$ and $\lambda=1$.

\paragraph{PPI\texttt{++}.}
\citet{Angelopoulos2023a} extend standard PPI by introducing an additional control variate parameter $\lambda$, which they call power tuning parameter.
The chosen $\widehat m_\theta$ is still the sample mean \eqref{eq:msamplemean},
while $\widehat \Delta^\PPpp_\theta$ now takes the control variate form
\begingroup
    \allowdisplaybreaks
    \begin{align}
        \widehat \Delta^\PPpp_\theta&=\frac{1}{n}\sum_{i=1}^n \left(\mathcal{L}_{\theta}'(X_i,Y_i)-\mathcal{L}_{\theta}'(X_i,f(X_i))\right) \label{eq:deltahatpowertuning} \\
        & ~ - (\widehat \lambda -1 )\left( \frac{1}{n}\left [\sum_{i=1}^n \mathcal{L}_{\theta}'(X_i,f(X_i))\right ]  -\widehat m_\theta\right), \nonumber
    \end{align}
\endgroup
where $\widehat\lambda$ is estimated from the data.
In this case, the centered control variate is $\mathcal{L}_{\theta}'(X_i,f(X_i))-\widehat m_\theta$, which depends only on the machine learning predictions. For the squared loss, we obtain
\begin{align}
    \widehat\theta^\PPpp = \overline Y - \widehat\lambda\left( \frac{1}{n}\sum_{i=1}^n f(X_i) - \frac{1}{N}\sum_{j=1}^{N} f(\tX_j)     \right)
    \label{eq:thetahatppipp}
    \end{align}
with plug-in estimator
\begin{align}
    \widehat\lambda = \frac{c_n}{(1+\frac{n}{N})v_{n+N}}, \label{eq:lambdahat}
\end{align}
where $c_n$ is the sample covariance of $(Y_i,f(X_i))_{i=1}^n$ and $v_{n+N}$ is the sample variance of $((f(X_i))_{i=1}^n,(f(\tX_j))_{j=1}^N)$. The estimator \eqref{eq:thetahatppipp} is closely related (though slightly different) to the one introduced by \citet{Zhang2019} for mean estimation in semi-supervised inference.

\paragraph{CLT-based CIs.}
While the definition of the PPI confidence interval \eqref{eq:ppconfidenceinterval} allows for merging any CIs $\mathcal{R}_\delta$ and $\mathcal{T}_{\alpha-\delta}$ for $\Delta_\theta$ and $m_\theta$, in practice these are often chosen to be CLT-based CIs that, once combined into $\calC_\alpha$ give exact asymptotic coverage,
\begin{equation*}
    \liminf_{n,N \to \infty} \mathrm{Pr}(\theta^* \in \calC_\alpha) \geq 1 - \alpha.
\end{equation*}
Such CLT-based CIs rely on the following standard assumption on the estimators $\widehat m_\theta$ and $\widehat \Delta_\theta$.
\begin{assumption}[CLT assumption for PPI and PPI++]
\label{assump:clt}
    Let  $\widehat m_\theta$ be the sample mean $\eqref{eq:msamplemean}$ and consider some estimator $(\widehat \sigma^f_\theta)^2$ of $\var(\widehat m_\theta)$.
    Let $\widehat \Delta_\theta$ be either the PPI estimator \eqref{eq:deltasamplemean} or the PPI++ estimator \eqref{eq:deltahatpowertuning} and consider some estimator $\widehat \sigma_\theta$ of $\var(\widehat \Delta_\theta)$ with $\widehat\sigma_\theta/\var(\widehat \Delta_\theta) \to 1$ almost surely.  Assume that, as $\min(n,N)\to\infty$,
    \begin{align}
        (\widehat m_\theta - m_\theta) / \widehat \sigma^f_\theta & \to \Normal(0,1)\\
        (\widehat\Delta_\theta -\Delta_\theta) / \widehat\sigma_\theta &\to \Normal(0,1).
    \end{align}
\end{assumption}

\subsection{Bayes-optimal confidence regions}\label{sec:background_fab}
The FAB framework \citep{Pratt1961,Pratt1963,Yu2018} aims to construct valid confidence regions with smaller expected volume.
Let $W\mid \beta\sim\Normal(\beta,\sigma^2)$ with some prior $\pi_0(\beta)$ and denote by $\pi(w)=\int p(w\mid \beta) \pi_0(\beta)d\beta$ the corresponding marginal likelihood.
For $\alpha\in(0,1)$, let $\calC_\alpha(w)$ be an exact $1-\alpha$ confidence region for $\beta$ based on the data $w$. That is,
for any fixed $\beta_0$,
\begin{align}
    \Pr(\beta\in \calC_\alpha(W)\mid \beta=\beta_0)=1-\alpha. \label{eq:confidencelevel}
\end{align}
Let $\vol(\calC_\alpha(w))=\int_{\beta'\in\calC_\alpha(w)}d\beta'$ be the volume of $\calC_\alpha(w)$, and consider its expected value under the prior $\pi_0$,
\begin{equation}
    \E[\vol(\calC_\alpha(W))] = \int \vol(\calC_\alpha(w)) \pi(w) dw.
\end{equation}
\begin{definition}
    For $\alpha\in(0,1)$, $\sigma>0$ and a prior $\pi_0(\beta)$, the FAB confidence region $\calC_\alpha$ for the mean parameter $\beta$ of the normal model $Y\mid\beta\sim\Normal(\beta,\sigma^2)$, is the minimiser of the (Bayesian) expected volume
    \begin{align}
        \calC_\alpha = \arg\min_{\widetilde\calC_\alpha} \E[\vol(\widetilde\calC_\alpha(W))] \label{eq:bayes_optimal_C}
    \end{align}
    subject to the (frequentist) coverage constraint \eqref{eq:confidencelevel}. We write $\calC_\alpha(w)=\text{FAB-CR}(w;\pi_0,\sigma^2,\alpha)$.
\end{definition}
The solution to \cref{eq:bayes_optimal_C}, which exists and is unique if $\pi_0(\beta)$ is not degenerate \citep[Theorem~2.1]{Cortinovis2024}, may be found numerically as long as the marginal likelihood $\pi(w)$ can be evaluated pointwise.
Additional details are provided in \cref{sec:app:backgroundFAB}.
Intuitively, the FAB confidence region $\calC_\alpha(w)$ constructed through \cref{eq:bayes_optimal_C} will be smaller for values of $w$ that are likely under the marginal likelihood, and larger otherwise.
As a result of this, while FAB guarantees the right coverage for any prior, one that assigns high probability to the value of $\beta$ that generated the data is required to achieve smaller expected volume compared to the standard CI $(w\pm \sigma z_{1-\alpha/2})$, whose width does not depend on $w$.

\paragraph{Bayes-assisted estimator.} A natural estimator to use alongside the FAB confidence region $\calC_\alpha(w)$ is the posterior mean $\widehat{\beta}(W) = \mathbb{E}[\beta\mid W]$. As shown by \citep[Theorem~2.1]{Cortinovis2024}, it is always contained within the confidence region: $\widehat{\beta}(w)\in  \calC_\alpha(w)$ for any $w\in\bbR$ and any $\alpha\in(0,1)$. We refer to $\widehat{\beta}(W)$ as the Bayes-assisted estimator.


%% Method: FAB-PPI
\section{FAB-PPI}
\label{sec:fab-ppi}
Our approach, which we call FAB-PPI, combines the PPI framework with the FAB construction of confidence regions by specifying a prior on the rectifier $\Delta_\theta$. To ease the presentation, here we describe the method for $Y, \theta \in\bbR$. The general multivariate case is discussed in \cref{app:multivariate}.

As in PPI, we use the sample mean \eqref{eq:msamplemean} as the estimator of $m_\theta$.
For $\Delta_\theta$, we start by considering a consistent estimator $\widehat \Delta_\theta$, such as the sample mean \eqref{eq:deltasamplemean} used in PPI, or the control variate estimator \eqref{eq:deltahatpowertuning} used in PPI\texttt{++}.
Throughout this section, we assume that \cref{assump:clt} is satisfied. That is, a CLT holds for $\widehat{m}_\theta$ and $\widehat{\Delta}_\theta$ with respect to some estimators $(\widehat{\sigma}^f_\theta)^2$ and $\widehat{\sigma}^2_\theta$ of $\var(\widehat{m}_\theta)$ and $\var(\widehat{\Delta}_\theta)$, respectively.
In this setting, let $\pi_0(\Delta_\theta; \tau_n)$ be a prior on $\Delta_\theta$ with scale parameter $\tau_n$, which may depend on the labelled data through $\widehat \sigma_\theta$.
Denote by $\ell(w; \sigma, \tau)$ the log-marginal likelihood, evaluated at $w$, of a Gaussian likelihood model with mean $\Delta$ and variance $\sigma^2$ under the prior $\pi_0(\Delta; \tau)$,
\begin{align}
    \ell(w; \sigma, \tau) = \log \int_\bbR \Normal(w; \Delta,\sigma^2) \pi_0(\Delta; \tau) d\Delta. \nonumber
\end{align}

\subsection{Bayes-assisted PPI Estimators}
 Consider the Bayes-assisted estimator
\begin{align}
    \widehat\Delta_\theta^{\FABPPI}= \widehat\Delta_\theta + \widehat\sigma_\theta^2 \ell'\left(\widehat\Delta_\theta; \widehat\sigma_\theta, \tau_n\right)
\label{eq:FABPPIestimatorDelta}
\end{align}
for the rectifier $\Delta_\theta$.
By Tweedie's formula \citep{}, the above estimator is the posterior mean of the mean parameter of a Gaussian likelihood model under the prior $\pi_0$.
Note however that we do not assume here that $\widehat\Delta_\theta$ is normally distributed for a fixed $n$.

The FAB-PPI estimator of $\theta^\star$, denoted by $\widehat \theta^{\FABPPI}$, is then obtained as the solution, in $\theta$, to the equation
$$
    \widehat m_\theta + \widehat\Delta_\theta^{\FABPPI} = 0.
$$

\subsection{FAB-PPI Confidence Regions}\label{sec:fabppi_cr}
As in PPI, let $\calT_{\alpha-\delta}(\widehat m_\theta)$ denote a standard $1-(\alpha-\delta)$ confidence interval for $m_\theta$.
For $\Delta_\theta$, we apply the FAB framework with the prior $\pi_0$ to obtain a $1-\delta$ confidence region $\calR^{\FABPPI}_\delta(\widehat\Delta_\theta)=\text{FAB-CR}(\widehat\Delta_\theta;\pi_0(\cdot~;\tau_n),\widehat\sigma_\theta,\delta)$.
Then, the FAB-PPI confidence region $\calC_\alpha^\FABPPI$ is obtained as
\begin{align}
    \calC_\alpha^\FABPPI=\left\{\theta \mid 0\in \calR^{\FABPPI}_\delta(\widehat\Delta_\theta) + \calT_{\alpha-\delta}(\widehat m_\theta)\right\}.
    \label{eq:fabppiconfidenceinterval}
\end{align}
\cref{algo:FABPPIConvex} summarises the steps of the FAB-PPI approach in a general convex estimation problem.
\begin{algorithm}
    \caption{FAB-PPI for convex estimation}
    \label{algo:FABPPIConvex}
    {\bfseries Input:} labelled $\{(X_i,Y_i)\}_{i=1}^n$, unlabelled $\{\tX_j\}_{j=1}^N$, predictor $f$, prior $\pi_0(\cdot~; \tau_n)$, error levels $\alpha,\delta$
    \begin{algorithmic}
        \STATE Set $\widehat\lambda=1$ (FAB-PPI) or estimate $\widehat\lambda$ from data (FAB-PPI\texttt{++}) as in \citet{Angelopoulos2023a}.
        \FOR{$\theta \in \Theta_\text{grid}$}
            \STATE $\widehat m_\theta \gets \frac{1}{N}\sum_{i=1}^{N} \mathcal{L}_{\theta}'(\widetilde X_i,f(\widetilde X_i))$
            \STATE $\widehat{\xi} \gets \frac{1}{n}\sum_{i=1}^n \left(\mathcal{L}_{\theta}'(X_i,Y_i)- \widehat{\lambda} \mathcal{L}_{\theta}'(X_i,f(X_i))\right)$
            \STATE $\widehat \Delta_\theta \gets \widehat{\xi} + (\widehat \lambda -1 ) \widehat m_\theta$
            \STATE $(\widehat{\sigma}^f_\theta)^2 \gets \frac{1}{N - 1} \sum_{i=1}^{N} \left(\mathcal{L}_{\theta}'(\widetilde X_i,f(\widetilde X_i)) - \widehat m_\theta\right)^2$
            \STATE $\widehat \sigma_\xi^2 \gets \frac{1}{n-1} \sum_{i=1}^n \left(\mathcal{L}_{\theta}'(X_i,Y_i)- \widehat{\lambda} \mathcal{L}_{\theta}'(X_i,f(X_i)) - \widehat{\xi}\right)^2$
            \STATE $\widehat{\sigma}^2_\theta \gets \frac{1}{n} \widehat \sigma_\xi^2 + \frac{(\widehat \lambda -1 )^2}{N} \widehat{\sigma}^2_m$
            \STATE $\mathcal{T}_{\alpha - \delta}(\widehat m_\theta) \gets \left(\widehat{m}_\theta \pm \widehat{\sigma}^f_\theta z_{1-(\alpha - \delta)/2}\right)$
            \STATE $\calR^{\FABPPI}_\delta(\widehat\Delta_\theta)\gets \text{FAB-CR}(\widehat\Delta_\theta;\pi_0(\cdot~;\tau_n),\widehat\sigma_\theta,\delta)$
            \STATE $\widehat\Delta_\theta^{\FABPPI} \gets \widehat\Delta_\theta + \widehat\sigma_\theta^2 \ell'\left(\widehat\Delta_\theta;\widehat\sigma_\theta, \tau_n\right)$
        \ENDFOR
    \end{algorithmic}
    {\bfseries Outputs:} estimator $\widehat \theta^{\FABPPI} = \arg\min_{\Theta_\text{grid}} \left|\widehat{m}_\theta + \widehat\Delta_\theta^{\FABPPI}\right|$ and CR $\calC_\alpha^\FABPPI=\left\{\theta \mid 0\in \calR^{\FABPPI}_\delta(\widehat\Delta_\theta) + \calT_{\alpha-\delta}(\widehat m_\theta)\right\}$
\end{algorithm}

\subsection{Choosing the prior}\label{sec:fabppi_priors}
FAB-PPI is motivated by applications in which the PPI predictor $f$ is expected to be generally accurate, as measured by the rectifier $\Delta_\theta$.
Such a property may be encoded in $\pi_0(\Delta_\theta; \tau_n)$ by choosing a prior that concentrates around zero.
As mentioned in \cref{sec:background_fab}, the FAB construction of $\calR^{\FABPPI}_\delta(\widehat\Delta_\theta)$ will exhibit smaller volume compared to the classical CI, and hence result in downstream efficiency gains over standard PPI, if the true rectifier $\Delta_\theta$ is likely under $\pi_0$.
In particular, the prior scale $\tau_n$ controls the size of the potential efficiency gains and losses of FAB-PPI over PPI: the smaller $\tau_n$, the more the resulting CR will shrink (resp. grow) when $\Delta_\theta \approx 0$ (resp. $|\Delta_\theta| \gg 0$). Experimentally, we find that the choice $\tau_n = \widehat\sigma_\theta$ results in a parameter-free approach that strikes a good compromise. More general choices of $\tau_n$ are briefly mentioned in \cref{sec:discussion}.

A seemingly natural proposal for $\pi_0$ that meets the requirements above is the Gaussian prior
\begin{align}
    \pi_\text{N}(\Delta_\theta; \widehat\sigma_\theta) = \Normal(\Delta_\theta; 0, \widehat\sigma_\theta).
    \label{eq:Gaussianprior}
\end{align}
However, as we will discuss in \cref{sec:fabppi_theory}, $\pi_\text{N}$ exhibits undesirable properties for FAB-PPI. Instead, we propose to use the horseshoe prior~\citep{Carvalho2010}
\begin{align}
    \pi_\text{HS}(\Delta_\theta;\widehat\sigma_\theta)=\int_0^\infty \Normal(\Delta_\theta;0,\nu^2\widehat\sigma^2_\theta) C^+(\nu; 0,1)d\nu,
    \label{eq:horseshoeprior}
\end{align}
where $C^+(\nu; 0,1)$ denotes the pdf of the half-Cauchy distribution with location parameter 0 and scale parameter 1. In the case of $\pi_\text{HS}$, the choice of scaling $\tau_n = \widehat\sigma_\theta$ is further motivated by \citet[\S3.3]{Piironen2017}. Furthermore,  the horseshoe prior has power-law tails, making it a particularly robust choice for FAB-PPI, as discussed in \cref{sec:fabppi_theory}. Crucially, for both priors $\pi_\text{N}$ and $\pi_\text{HS}$, the marginal likelihood under a Gaussian model with standard deviation $\widehat\sigma_\theta$ can be expressed in terms of standard functions (see \cref{sec:app:backgroundhorseshoe} for the horseshoe), enabling us to compute $\widehat\Delta_\theta^{\FABPPI}$ and $\calR^{\FABPPI}_\delta(\widehat\Delta_\theta)$ in \cref{algo:FABPPIConvex}.

\subsection{Theoretical properties}\label{sec:fabppi_theory}
As shown by the following result, proved in \cref{app:thm:fabppi_coverage}, the FAB-PPI CR has exact asymptotic coverage.
\begin{theorem}[Asymptotic coverage]\label{thm:fabppi_coverage}
    For $\alpha \in (0, 1)$, let $\calC_\alpha^\FABPPI$ be the FAB-PPI confidence region \eqref{eq:fabppiconfidenceinterval} under the Gaussian prior \eqref{eq:Gaussianprior} or the horseshoe prior \eqref{eq:horseshoeprior}. Then, under \cref{assump:clt},
    \begin{equation*}
        \liminf_{\min(n,N) \to \infty} \Pr(\theta^* \in \calC_\alpha^\FABPPI) \geq 1 - \alpha.
    \end{equation*}
\end{theorem}
The proof of \cref{thm:fabppi_coverage} crucially relies on showing exact asymptotic coverage of the FAB CR $\calR^{\FABPPI}_\delta(\widehat\Delta_\theta)$. While the latter holds for both priors introduced in the previous sections, the two limits are very different. In particular, as discussed in \cref{rem:fabppi_asymptotic}, the volume of $\calR^{\FABPPI}_\delta(\widehat\Delta_\theta)$ vanishes asymptotically under $\pi_\text{HS}$, while it does not under $\pi_\text{N}$. \looseness=-1

The behaviour of $\calR^{\FABPPI}_\delta(\widehat\Delta_\theta)$ under the two priors also differs for large values of observed $\widehat\Delta_\theta$.
In case of increasing disagreement between the prior and the data, Gaussian FAB confidence regions are known to become arbitrarily large \citep{Yu2018}. 
On the other hand, thanks to its power-law tails, the horseshoe results in confidence regions that revert to the corresponding standard CI \citep{Cortinovis2024}.
Here, we state the implication of this property on FAB-PPI informally, and provide a formal proof in \cref{app:thm:fabppi_robustness}.
\begin{proposition}[Robustness under the horseshoe, informal]\label{prop:fabppi_robustness}
    For $\alpha\in(0,1)$, let $\calC_\alpha^\FABPPI$ and $\calC_\alpha^\PP$ denote, respectively, the FAB-PPI confidence region \eqref{eq:fabppiconfidenceinterval} under the horseshoe prior \eqref{eq:horseshoeprior} and the standard PPI confidence region \eqref{eq:ppconfidenceinterval}, both viewed as functions of $\widehat\Delta_\theta$. If $|\widehat\Delta_\theta| \gg 0$, then
    \begin{align*}
        \calC_\alpha^\FABPPI \simeq \calC_\alpha^\PP.
    \end{align*}
\end{proposition}
In practice, this means that, in the presence of heavily biased predictors, FAB-PPI with the horseshoe prior reverts to standard PPI. In a sense, this represents a form of robustness to prior misspecification of FAB-PPI under the horseshoe. \looseness=-1

Overall, \cref{rem:fabppi_asymptotic} and \cref{prop:fabppi_robustness} provide strong support for preferring $\pi_\text{HS}$ over $\pi_\text{N}$ within the FAB-PPI framework.

\subsection{FAB-PPI for mean estimation}\label{sec:fabppi_mean_estimation}
To provide a concrete example, a specialised version of \cref{algo:FABPPIConvex} under the squared loss is derived in \cref{app:fabppi_squared_loss}.
Here, we briefly discuss the differences between the FAB-PPI mean estimator and its standard PPI counterpart, as well as the asymptotic behaviour of the former.
Under the squared loss, the rectifier $\Delta := \Delta_\theta$ does not depend on $\theta$ and the FAB-PPI estimator $\widehat \theta^{\FABPPI}$ corresponding to the chosen estimator $\widehat\Delta$ (PPI or PPI++) is given by \looseness=-1
\begin{align}
    \widehat \theta^{\FABPPI} &=  \widehat \theta - \widehat\sigma^2 \ell'(\widehat\Delta; \widehat\sigma, \tau_n) \label{eq:mean_estimator_fabppi}\\
    &=\overline Y - \widehat\lambda\left( \frac{1}{n}\sum_{i=1}^n f(X_i) - \frac{1}{N}\sum_{j=1}^{N} f(\tX_j) \right) \nonumber \\
    &~~~~- \widehat\sigma^2 \ell'\left(\widehat\Delta; \widehat\sigma, \tau_n\right),\nonumber
\end{align}
where $\widehat\sigma^2$ is an estimator of $\var(\widehat\Delta)$, $\widehat \theta$ is the PPI estimator corresponding to $\widehat\Delta$, and $\widehat\lambda$ is set either to one (PPI) or \eqref{eq:lambdahat} (PPI++).
In both cases, the estimator $\widehat \theta^{\FABPPI}$ takes the form
$$
    \text{\normalsize Classic Estimator + PPI correction + \textbf{Bayes correction}},
$$
where the last component depends on the chosen prior. The following proposition, proved in \cref{app:prop:fabppi_consistency}, further differentiates between the priors presented in \cref{sec:fabppi_priors} in favour of the horseshoe.
\begin{proposition}[Consistency of FAB-PPI mean estimators]
    \label{prop:fabppi_consistency}
    Let $\widehat \theta^{\FABPPI}_\text{HS}$ and $\widehat \theta^{\FABPPI}_\text{N}$ be the FAB-PPI estimators \eqref{eq:mean_estimator_fabppi} under the horseshoe \eqref{eq:horseshoeprior} and Gaussian \eqref{eq:Gaussianprior} priors, respectively. If the PPI estimator $\widehat \theta$ is a consistent estimator of $\theta^\star$, then $\widehat \theta^{\FABPPI}_\text{HS}$ is a consistent estimator of $\theta^\star$, while $\widehat \theta^{\FABPPI}_\text{N}$ is not.
\end{proposition}
Intuitively, this is due to the fact that the influence of $\pi_\text{HS}$ vanishes asymptotically, while for $\pi_\text{N}$ it does not.


%% Experiments
\section{Experiments}
\label{sec:experiments}
We compare FAB-PPI and power-tuned FAB-PPI (FAB-PPI\texttt{++}) to classical inference, PPI and power-tuned PPI (PPI\texttt{++}) on both synthetic and real estimation problems.
For FAB-PPI, we use $\text{(HS)}$ and $\text{(N)}$ to indicate the use of the horseshoe and Gaussian priors defined in \cref{sec:fabppi_priors}.
As already mentioned, PPI is motivated by settings in which labelled data are scarce, while unlabelled data are abundant. Moreover, the application of FAB to PPI specifically targets the estimation of the rectifier $\Delta_\theta$.
For these reasons, we choose to focus on cases where $N \gg n$ is large enough to rule out any uncertainty on the measure of fit $m_\theta$, which we estimate using the sample mean $\widehat{m}_\theta$ \eqref{eq:msamplemean}. As a result of this, given a $1 - \delta$ confidence interval (FAB or not) $\mathcal{R}_\delta$ for $\Delta_\theta$, the corresponding $1 - \alpha$ CI for $\theta^\star$ is obtained simply by setting $\delta = \alpha$ and shifting $\mathcal{R}_\delta$ by $\widehat{m}_\theta$. This simplification allows us to evaluate the direct effect of FAB on the procedure, eliminating concerns about the loss of tightness in the CI on $\theta^*$ due to the Minkowski sum in \cref{eq:fabppiconfidenceinterval}. In all experiments, we check empirically that $N$ is large enough to make this assumption by monitoring the coverage of the resulting intervals against both the nominal level $1 - \alpha$ and the coverage of PPI intervals that also consider the uncertainty on $m_\theta$ (denoted with PPI (full) and PPI\texttt{++} (full) in \cref{app:additional_results}). \looseness=-1

\subsection{Synthetic data}\label{sec:synthetic_data}
The simulated experiments below have a common structure. We sample two datasets, $n$ labelled observations $\{(X_i,Y_i)\}_{i=1}^n$ iid from $\mathbb{P}$ and $N$ unlabelled observations $\{\widetilde{X}_i\}_{i=1}^N$ iid from $\mathbb{P}_X$. We use a prediction rule $f$ to obtain predictions $\{f(X_i)\}_{i=1}^n$ and $\{f(\widetilde{X}_i)\}_{i=1}^N$. We apply the different procedures to obtain estimates and $1 - \alpha$ confidence regions for the mean $\theta^* = \mathbb{E}[Y]$. For all experiments, we set $\alpha = 0.1$ and report the average mean squared error (MSE), interval volume, and coverage over $1000$ repetitions.

\paragraph{Biased predictions.}
We sample $X_i \overset{iid}{\sim} \mathcal{N}(0,1)$ and $Y_i = X_i + \epsilon_i$ with $\epsilon_i \overset{iid}{\sim} \mathcal{N}(0, 1)$, so that $\theta^* = \mathbb{E}[Y] = 0$.
The prediction rule is defined as $f(X_i) = X_i + \gamma$, where $\gamma \in \mathbb{R}$.
For this choice, the bias of $f$ is controlled by $\gamma$, since $\text{MSE}(f) = \gamma^2 + 1$.
For this experiment, we assume that $N$ is infinite, set $n = 200$, and vary $\gamma$ between $-1.5$ and $1.5$.
\cref{fig:simul_bias_width_main} shows the average interval volume as a function of $\gamma$ for classical inference, PPI\texttt{++}, and FAB-PPI\texttt{++} with both a horseshoe and a Gaussian prior.
\begin{figure}[ht!]
    \centering
    \includegraphics[width=0.4\textwidth]{./figures/simul_bias_width_main.pdf}
    \caption{Biased predictions study. The panel shows the average CI volume as the bias level $\gamma$ varies.}
    \label{fig:simul_bias_width_main}
\end{figure}
Results for the non power-tuned methods, as well as MSE and coverage plots, are reported in \cref{fig:simul_bias_supplementary}.
Except for the version with the Gaussian prior, all the PPI procedures outperform classical inference for every bias level $\gamma$, but the behaviour exhibited by PPI\texttt{++} deserves attention, as its CI volume is approximately constant across values of $\gamma$.
This is due to the fact that, since $N$ is taken to be infinite and $n$ is fairly large, $\widehat\lambda \approx \cov(Y, f(X)) = 1$ and the rectifier is accurately estimated with similar variance across all values of $\gamma$.
On the other hand, the CI volume for the FAB-PPI methods varies greatly with $\gamma$. When the bias is small ($\gamma \simeq 0$), the observed rectifier has a value close to $0$, leading to smaller CIs.
As the bias increases, the volume of the confidence intervals grows, until it surpasses that of the PPI intervals.
At this point, the two FAB-PPI procedures behave differently: the volume of the Gaussian intervals grows unbounded, whereas the horseshoe intervals eventually revert to the PPI ones.
This example clearly shows that FAB-PPI with a horseshoe prior allows to obtain smaller CIs when the predictions are good, while ensuring robustness as the quality of the predictions decreases (\cref{prop:fabppi_robustness}). \looseness=-1

\paragraph{Noisy predictions.}
We consider the mean estimation example of \citet[\S7.1.1]{Angelopoulos2023a}, which does not involve any covariate $X$.
We sample $Y_i \overset{iid}{\sim} \mathcal{N}(0, 1)$, so that $\theta^* = \mathbb{E}[Y] = 0$.
The prediction rule is defined as $f(X_i) = Y_i + \sigma_Y \epsilon_i$, where $\epsilon_i \overset{iid}{\sim} \mathcal{N}(0,1)$ and $\sigma_Y$ is successively set to $0.1$, $1$, and $2$.
For this experiment, we set $N = 10^6$ and vary $n$ from $100$ to $1000$.
\cref{fig:simul_noisy_width_main} shows the average interval volume as a function of $n$ for the different methods as the noise level $\sigma_Y$ varies, while similar plots for the MSE and coverage are reported in \cref{fig:simul_noisy_supplementary}.
\begin{figure}[ht!]
    \centering
    \includegraphics[width=0.45\textwidth]{./figures/simul_noisy_width_main.pdf}
    \caption{Noisy predictions study. The left, middle and right panels show the average CI volume for noise levels $\sigma_Y = 0.1, 1, 2$.}
    \label{fig:simul_noisy_width_main}
\end{figure}
\begin{figure*}[ht!]
    \centering
    \includegraphics[width=\textwidth]{./figures/real_main.pdf}
    \caption{Real data mean estimation study. The left, middle, and right panels correspond to the \textsc{alphafold}, \textsc{galaxies}, and \textsc{forest} datasets. The top, middle, and bottom rows show average MSE, CI volume, and CI coverage over $1000$ repetitions for $\alpha = 0.1$.}
    \label{fig:real_main}
\end{figure*}
In this case, the effect of power tuning matches the observations of \citet{Angelopoulos2023a}: as the noise level increases, $\hat{\lambda}$ decreases and less weight is given to the predicted labels.
When the noise is small, all PPI procedures perform similarly, and much better than classical inference.
When the noise is large, the power-tuned procedures perform similar to or better than classical inference, whereas the non-tuned alternatives lose ground.
At the intermediate noise level, the power-tuned methods clearly outperform the other baselines.
Crucially, FAB-PPI outperforms the PPI counterpart at all noise levels, with FAB-PPI\texttt{++} being the best performer overall. This is because, in this setting, while predictions exhibit increasing variance with $\sigma_Y$, they remain unbiased. As a result of this, regardless of the value of $\lambda$ used, any additional shrinkage performed on the rectifier by FAB-PPI is beneficial. This example shows that FAB-PPI\texttt{++} retains the benefits of power tuning, while also taking advantage of the adaptive shrinkage provided by the FAB procedure.

\subsection{Real data}\label{sec:real_data}
We consider several estimation experiments using the datasets presented in \citet{Angelopoulos2023} and briefly described in \cref{app:datasets}. Each dataset comes with covariate/label/prediction triples $\{X_i, Y_i, f(X_i)\}_{i=1}^N$, which we randomly split into two subsets with $n$ labelled and $N - n$ unlabelled observations, for varying values of $n$. For all experiments and methods, we report the average estimation MSE, CI volume and coverage across multiple repetitions.

We begin with three mean estimation experiments, involving the \textsc{alphafold}, \textsc{forest}, and \textsc{galaxies} datasets, and one logistic regression experiment on the \textsc{healthcare} dataset. \Cref{fig:real_main} shows the results for classical inference, PPI\texttt{++}, and FAB-PPI\texttt{++} applied to the mean estimation datasets. The mean estimation results for the non power-tuned methods are reported in \cref{fig:real_supplementary}, whereas the ones for the logistic regression experiment are reported in \cref{fig:real_logistic_supplementary}. In all cases, FAB-PPI/FAB-PPI\texttt{++} outperform classical inference and the corresponding PPI methods, both in terms of MSE and CI volume, while achieving comparable coverage.
These examples suggest that the quality of the predictions of existing machine learning models on several real datasets may fall into the regime where the adaptive shrinkage provided by the FAB framework leads to a further improvement over standard PPI. In these settings, as the predictions are good, FAB-PPI under the horseshoe and Gaussian priors exhibit similar gains, as already seen in \cref{fig:simul_bias_width_main}.

However, the same is not true in the presence of bad predictions. For instance, \cref{fig:real_quantile_supplementary} shows the results of a quantile estimation experiment on the \textsc{genes} dataset, where predictions are heavily biased.
In this case, the behaviour of the FAB-PPI methods under the horseshoe and Gaussian priors differs significantly: the former matches the performance of the PPI methods, which outperform classical inference, whereas the latter leads to much larger MSE and CIs.
As previously discussed, such desirable behaviour of FAB-PPI under the horseshoe prior is due to its robustness against large bias levels (\cref{prop:fabppi_robustness}). 
Similarly, \cref{fig:real_ols_supplementary} reports the results of a linear regression experiment on the \textsc{census} dataset.
For one of the two parameters considered (panel~(a)), FAB-PPI underperforms the alternatives under both priors for small $n$.
However, as $n$ increases, the performance under the horseshoe prior improves and eventually matches that of the PPI methods, while the Gaussian prior does not.
This example shows another facet of the horseshoe's robustness: even for moderate bias levels, as the available labelled sample size grows, disagreements between the prior and the data become apparent (i.e.~$\var(\widehat{\Delta}_\theta)$ decreases), eventually leading \cref{prop:fabppi_robustness} to kick in.


%% Discussion
\section{Discussion and extensions}
\label{sec:discussion}
We proposed FAB-PPI as a Bayes-informed method to significantly improve the performance of PPI in the presence of high quality predictions. In doing so, we showed that the horseshoe represents a sensible default prior for FAB-PPI, contrary to the seemingly natural choice of a Gaussian prior. However, several options may be worth exploring.

In particular, the horseshoe prior was chosen due to its popularity and key properties: (i) its spike at zero (ii) power-law tails and  (iii) the closed-form expression for the marginal density $\pi(y)$. However, many other scale-mixture of Gaussians models share these properties.
For example, the family of priors with a beta prime (aka inverted beta) prior over the variance~\citep{Polson2012}, which includes the horseshoe, normal-exponential-gamma~\citep{Griffin2011} and other robust priors \citep{Berger1980,Strawderman1971} as special cases, share the same three properties.
On the other hand, some other standard priors such as the Laplace prior~\citep{Park2008} or normal-gamma prior~\citep{Caron2008,Griffin2010} do not have power-law tails and therefore do not offer the same robustness guarantees. Other priors, such as the student-t, lack an analytical expression for $\pi(y)$, therefore requiring additional numerical approximation to be applied to FAB-PPI.

Furthermore, we used the scale $\sigma$ of the noise in the generative model as the scale for both the horseshoe and Gaussian priors, as this allows us to obtain a simple, parameter-free approach, which generally performs well.
Alternatively, one could consider a prior scale of $\eta\sigma$, where $\eta$ is a hyperparameter to be tuned using a validation set.
However, in the case of the horseshoe, this renders the marginal likelihood intractable. While using a rescaled horseshoe prior for FAB-PPI remains feasible through numerical integration, as shown in \cref{fig:real_forest_scaled_supplementary}, this increases the computational cost of the method.
By contrast, a rescaled Gaussian prior would not encounter this issue.
Furthermore, we conjecture that choosing a scale that does not depend on $\sigma$ may resolve the inconsistency of the estimator based on a Gaussian prior, which was discussed in \cref{sec:fabppi_mean_estimation}. \looseness=-1

As a potential drawback, FAB-PPI shares the computational limitations of the PPI approach \citep{Angelopoulos2023}, which are discussed in \citet{Angelopoulos2023a}. In particular, except for particular cases such as mean estimation and linear regression, the method requires evaluating $\widehat m_\theta + \widehat\Delta_\theta$ over a grid of values of $\theta$. This can be computationally expensive, especially in high-dimensional settings.


% Acknowledgements
\section*{Acknowledgements}
Stefano Cortinovis is supported by the EPSRC Centre for Doctoral Training in Modern Statistics and Statistical Machine Learning (EP/S023151/1).

\bibliography{references}
\bibliographystyle{include/icml2025}


%%%%%%%%%%%%%%%%%%%%%%%%%%%%%%%%%%%%%%%%%%%%%%%%%%%%%%%%%%%%%%%%%%%%%%%%%%%%%%%
%%%%%%%%%%%%%%%%%%%%%%%%%%%%%%%%%%%%%%%%%%%%%%%%%%%%%%%%%%%%%%%%%%%%%%%%%%%%%%%
% APPENDIX
%%%%%%%%%%%%%%%%%%%%%%%%%%%%%%%%%%%%%%%%%%%%%%%%%%%%%%%%%%%%%%%%%%%%%%%%%%%%%%%
%%%%%%%%%%%%%%%%%%%%%%%%%%%%%%%%%%%%%%%%%%%%%%%%%%%%%%%%%%%%%%%%%%%%%%%%%%%%%%%
\newpage
\appendix
\onecolumn

%% Add S to section, table, and figure numbers
\makeatletter
\renewcommand{\thesection}{S\@arabic\c@section}
\renewcommand{\thetable}{S\@arabic\c@table}
\renewcommand{\thefigure}{S\@arabic\c@figure}
\renewcommand{\theequation}{S\arabic{equation}}
\makeatother

%%Additional Background
\section{Additional background material}
\subsection{Horseshoe prior}
\label{sec:app:backgroundhorseshoe}
Consider the Gaussian likelihood model
$$
    Y\mid \beta\sim \Normal(\beta,\sigma^2)
$$
with standard deviation $\sigma>0$ and mean parameter $\beta\in\bbR$. The horseshoe prior~\citep{Carvalho2010} with density $\pi_\HS$ can be represented as a scale mixture of normals \citep{Andrews1974}
\begin{align}
    \beta\mid \nu^2 &\sim \Normal(0,\eta^2\sigma^2\nu^2)\\
    \nu &\sim C^+(0,1),
\end{align}
where $\eta>0$ and $C^+(0,1)$ is the half-Cauchy distribution with location parameter $0$ and scale parameter $1$. Throughout this section and the main text, we assume $\eta=1$. The rationale for this choice, along with a discussion of the general case $\eta\neq 1$, is provided at the end of this section.

The marginal likelihood is given by
\begin{align*}
    \pi(y)&=\int_{-\infty}^{\infty} \Normal(y\mid \beta,\sigma^2)\pi_\HS(\beta)d \beta\\
    &=\frac{1}{\sqrt{2\pi\sigma^2}} \int_0^\infty \frac{1}{\sqrt{1+\nu^2}}e^{-\frac{y^2}{2\sigma^2(1+\nu^2)}}p(\nu)d\nu\\
    &=\frac{2}{\pi\sqrt{2\pi\sigma^2}} \int_0^\infty e^{-\frac{y^2}{2\sigma^2(1+\nu^2)}}\frac{1}{(1+\nu^2)^{3/2}}d\nu.
\end{align*}
Using the change of variable $u=\frac{1}{1+\nu^2}$, we obtain
\begin{align*}
    \pi(y)&=\frac{1}{\pi\sqrt{2\pi\sigma^2}} \int_0^1 e^{-\frac{uy^2}{2\sigma^2}}(1-u)^{-1/2}du\\
    & = \frac{2}{\pi\sqrt{2\pi\sigma^2}} ~\OneFOne\left(1,\frac{3}{2},-\frac{y^2}{2\sigma^2}\right),
\end{align*}
where $\OneFOne$ is (Kummer's) confluent hypergeometric function of the first kind, with integral representation
\begin{align*}
    \OneFOne(a,b,z)=\frac{\Gamma(b)}{\Gamma(a)\Gamma(b-a)}\int_0^1 e^{zt}t^{a-1}(1-t)^{b-a-1}dt.
\end{align*}
Alternatively, the marginal can be expressed in function of the imaginary error function (erfi) or Dawson function (aka Dawson integral) as
\begin{align*}
    \pi(y)&=\frac{1}{\pi\sqrt{2\sigma^2}} e^{-y^2/(2\sigma^2)}\frac{\erfi(|y|/\sqrt{2\sigma^2})}{|y|/(\sqrt{2\sigma^2})}\\
    &=\frac{2}{\pi^{3/2}}\frac{1}{|y|}D\left( \frac{|y|}{\sqrt{2\sigma^2}}\right)
\end{align*}
where Dawson's function is defined as
$$
    D(z)=e^{-z^2}\int_0^z e^{t^2}dt.
$$ 
The marginal likelihood exhibits power-law tails
$$
    \pi(y)\sim C\frac{1}{|y|^2}\quad \text{as } |y|\to\infty
$$
for some constant $C>0$. Let $\ell(y)=\log\pi(y)$ denote the log-marginal likelihood. Kummer's function has the derivative
\begin{align*}
    \frac{d}{dz}\OneFOne(a,b,z)=\frac{a}{b}\OneFOne(a+1,b+1,z).
\end{align*}
It follows that
\begin{align*}
    \ell'(y) &= \frac{\pi'(y)}{\pi(y)}\\
    &=-\frac{2}{3}\frac{y}{\sigma^2}\frac{\OneFOne\left(2,\frac{5}{2},-\frac{y^2}{2\sigma^2}\right)}{\OneFOne\left(1,\frac{3}{2},-\frac{y^2}{2\sigma^2}\right)}.
\end{align*}
Applying Tweedie's formula \citep{}, we obtain the posterior mean
\begin{align}
    \E[\beta \mid y] &= y + \sigma^2 \ell'(y;\sigma)= (1-\kappa(y)) y,
\end{align}
where the shrinkage function $\kappa(y)\in(0,1)$ is given by
$$
    \kappa(y)=\frac{2}{3}\frac{\OneFOne\left(2,\frac{5}{2},-\frac{y^2}{2\sigma^2}\right)}{\OneFOne\left(1,\frac{3}{2},-\frac{y^2}{2\sigma^2}\right)}.
$$

Using the asymptotic expansion \citep[Chapter 4, Eq. (4.I.3)]{Slater1960}
$$
    \OneFOne(a,b,-z)\sim z^{-a}\frac{\Gamma(b)}{\Gamma(b-a)}
$$
as $z\to\infty$, we find
\begin{align*}
    \kappa(y)&\sim \frac{2\sigma^2}{y^2}\\
    |\bbE[\beta\mid y]-y|=\sigma^2|\ell'(y)|&\sim \frac{2\sigma^2}{|y|}
\end{align*}
as $|y|\to\infty$.

The horseshoe prior $\pi_\HS$ has two key properties: an infinite spike at zero, inducing strong shrinkage near $y=0$, and Cauchy-like tails, ensuring that strong signals remain largely unshrunk ($\kappa(y)\to 0$ and $|\bbE[\beta\mid y]-y|\to 0$  as $|y|\to\infty$). This is illustrated in \cref{fig:shrinkage_horseshoe}.
\begin{figure}[ht!]
    \centering
    \includegraphics[width=.35\textwidth]{./figures/shrinkage_horseshoe.pdf}
    \caption{Shrinkage function $\kappa(y)$ for the horseshoe prior when $\sigma^2=0.1$.}
    \label{fig:shrinkage_horseshoe}
\end{figure}

\begin{remark}[Parameterisation]
In this section and in the main text, we focused on the specific parametrisation $\eta=1$. For a general $\eta$, similar expressions can be derived for the marginal likelihood and posterior mean, replacing Kummer's $\OneFOne$ function with the more general degenerate hypergeometric function of two variables, $\Phi_1$ (see \citep[Equations (4) in the main text and (A1) in the appendix]{Carvalho2010}). While Kummer's $\OneFOne$ function is implemented in many standard scientific libraries, such as SciPy, $\Phi_1$ is not. Consequently, computing the marginal likelihood when $\eta\neq 1$ requires numerical integration. Since the evaluation of the marginal likelihood is crucial to our approach, it is therefore reasonable to set $\eta=1$ here.
\end{remark}

\subsection{FAB framework}
\label{sec:app:backgroundFAB}
In this section, we provide additional background on the FAB framework~\citep{Pratt1961,Pratt1963,Yu2018}.
Let $Y\mid \beta\sim\Normal(\beta,\sigma^2)$ with some prior $\pi_0(\beta)$. Denote by $\pi(y)=\int_\bbR p(y\mid \beta)\pi_0(\beta)d\beta$ the corresponding marginal likelihood.
For $\alpha\in(0,1)$, let $\calC_\alpha$ be the confidence procedure that solves the constrained optimisation problem
\begin{align*}
    &\calC_\alpha=\arg\min_{\widetilde \calC_\alpha} \E[\vol(\widetilde\calC_\alpha(Y))]\\
    \text{under the constraints }&\Pr(\beta\in \calC_\alpha(Y)\mid \beta=\beta')=1-\alpha \text{ for all fixed }\beta',
\end{align*}
where $\vol(\calC_\alpha(y))=\int_{\beta'\in\calC_\alpha(y)}d\beta'$ is the volume of $\calC_\alpha(y)$ and
\begin{equation}
    \E[\vol(\calC_\alpha(Y))] = \int_{\bbR} \vol(\calC_\alpha(y))\pi(y)dy
\end{equation}
is the expected volume under the marginal distribution $\pi(y)$. By the Ghosh-Pratt identity \citep{Ghosh1961,Pratt1961},
\begin{align*}
    \E[\vol(\calC_\alpha(Y))] &= \int_\bbR \vol(\calC_\alpha(y))\pi(y)dy\\
    &=\int_\bbR \int_\bbR 1_{\beta'\in \calC_\alpha(y)} d\beta'\pi(y)dy\\
    &=\int_\bbR \Pr(\beta'\in \calC_\alpha(Y))d\beta'.
\end{align*}
That is, minimising $\E[\vol(\calC_\alpha(Y))]$ is equivalent to minimising $\Pr(\beta'\in \calC_\alpha(Y))$ for each $\beta'\in\bbR$. Define the acceptance region 
$$
    A_\alpha(\beta')=\{y\mid \beta'\in \calC_\alpha(y)\}.
$$
The constrained optimisation problem above then reduces to solving, for each $\beta'$,
\begin{align*}
    &A_\alpha(\beta')=\arg\max_{\widetilde A_\alpha} \Pr(Y \notin \widetilde A_\alpha(\beta')) \\
    \text{such that }&\Pr(Y\notin A_\alpha(\beta)\mid \beta=\beta')=\alpha.
\end{align*}
The term $\Pr(Y \notin  A_\alpha(\beta'))$ may be interpreted as the power of a size-$\alpha$ test 
$$
    H_0:\beta=\beta'\text{ vs }H_1:\beta\sim\pi_0
$$
where $Y\mid \beta\sim\Normal(\beta,\sigma^2)$. By the Neyman-Pearson lemma, the most powerful test is of the form
$$
    A_\alpha(\beta')=\left \{y\mid \frac{\pi(y)}{p(y\mid \beta')}\leq k_\alpha(\beta')\right \}
$$
where $k_\alpha(\beta')$ is such that $\Pr(Y\in A_\alpha(\beta)\mid \beta=\beta')=1-\alpha$. The acceptance region is an interval $[\loA_\alpha(\beta'),\upA_\alpha(\beta')]$ \citep[Theorem 2.1]{Cortinovis2024}. Defining
$$
    w_\alpha(\beta')=\frac{1}{\alpha}\Phi\left( \frac{\loA_\alpha(\beta')-\beta'}{\sigma}\right),
$$
the confidence region is given by
$$
    \calC_\alpha(y)=\{\beta' \mid \loA_\alpha(\beta')=\beta' -\sigma z_{1-\alpha w_\alpha(\beta')}\leq y \leq \beta'+\sigma z_{1-\alpha(1-w_\alpha(\beta')) }=\upA_\alpha(\beta')\}.
$$
The function $w_\alpha(\beta')\in[0,1]$ is called the spending function or tail function \citep{Puza2006,Yu2018}, and represents the proportion of the $\alpha$ rejection budget allocated to the left tail of the acceptance interval $[\loA_\alpha(\beta'),\upA_\alpha(\beta')]$.

The spending function $w_\alpha$ satisfies several key properties, which will be useful for our asymptotic analysis. Most of these originate from \citet{Cortinovis2024}. Under mild assumptions on the prior, satisfied for the models considered in this paper, $w_\alpha(\beta)$ is continuous in $\beta$. If the prior $\pi_0$ is symmetric around zero, we have
\begin{align}
    w_\alpha(-\beta')=1-w_\alpha(\beta').
    \label{eq:propw1}
\end{align}

Additionally, if the prior $\pi_0(\beta):=\pi_0(\beta;\sigma)$ admits $\sigma$ as a scale parameter, writing $w_\alpha(\beta;\sigma)$ for the corresponding tail function, we have
\begin{align}
    w_\alpha(\beta;\sigma)=w_\alpha\left(\frac{\beta}{\sigma};1\right).
    \label{eq:propw2}
\end{align}

We now describe other properties of the spending function in the case of a Gaussian prior and of a prior with power-law tails, such as the horseshoe.
\begin{proposition}[FAB with a Gaussian prior \citep{Pratt1963,Yu2018}]
    \label{prop:fabgaussianprop}
    If the prior $\pi_0(\beta)=\mathcal N(\beta;0,\tau^2\sigma^2)$ is Gaussian, the spending function is given by $w_\alpha(\beta)=g_\alpha^{-1}\left(\frac{2\beta}{\sigma\tau^2}\right)$ where $g_\alpha:(0,1)\to\bbR$ is the one-to-one function
    \begin{equation}
        g_\alpha(\omega)=\Phi^{-1}(\alpha \omega)-\Phi^{-1}(\alpha(1-\omega)).
        \label{eq:galpha}
    \end{equation}
    $w_\alpha$ is strictly increasing and
    $$
        \lim_{\beta\to\infty} w_\alpha(\beta)=1.
    $$%
\end{proposition}
\begin{proposition}[FAB with a prior with power-law tails \citep{Cortinovis2024}]\label{prop:horseshoefab}
    Let $\pi_0(\beta;\sigma)$ be a symmetric prior such that the marginal density $\pi(y)$ has power-law tails:
    $$
        \pi(y)\sim \frac{C_\sigma}{|y|^{2\beta+1}} \text{ as }|y|\to\infty
    $$
    for some constant $C_\sigma$ and some $\beta>0$. Then,
    $$
        \lim_{\beta\to\infty} w_\alpha(\beta)=\lim_{\beta\to -\infty} w_\alpha(\beta)=\frac{1}{2}.
    $$
\end{proposition}
The difference between the spending functions of the two priors greatly affects the resulting FAB confidence regions. In particular, while both priors lead to confidence regions that are shorter than the classical one when the observed $y$ is close to zero, their behaviour differs as the disagreement between the prior and the data increases. In particular, the FAB confidence regions under the Gaussian prior become unbounded as $|y|$ grows, while the horseshoe prior leads to confidence regions that eventually revert to the classical confidence interval. This is illustrated in \cref{fig:fab_comparison_supplementary}.
\begin{figure}
    \centering
    \includegraphics[width=\textwidth]{./figures/fab_comparison_supplementary.pdf}
    \caption{Comparison of the FAB procedures under a Gaussian ($\tau^2 = 1$) and a horseshoe ($\eta = 1$) priors when $\sigma^2 = 1$ and $\alpha = 0.1$.}
    \label{fig:fab_comparison_supplementary}
\end{figure}


%% Detailed derivations
\section{Derivations}

\subsection{FAB-PPI for mean estimation}\label{app:fabppi_squared_loss}
Here, we outline the steps to derive the FAB-PPI mean estimator presented in \cref{eq:mean_estimator_fabppi}, as well as the corresponding FAB-PPI confidence region.
The convex loss function that corresponds to estimating $\theta^* = \mathbb{E}[Y]$ is the squared loss $\mathcal{L}_\theta(x, y) = \frac{1}{2}\left(\theta - y\right)^2$. In this case, the subgradient of $\mathcal{L}_\theta$ with respect to $\theta$ is given by $\mathcal{L}_\theta'(x, y) = \theta - y$. As a result of this, the measure of fit $m_\theta$ and the rectifier $\Delta_\theta$ take the form
\begin{align*}
    m_\theta &= \mathbb{E}[\mathcal{L}_\theta'(X, f(X))] = \theta - \mathbb{E}[f(X)], \\
    \Delta_\theta &= \mathbb{E}[\mathcal{L}_\theta(X, Y) - \mathcal{L}_\theta'(X, f(X))] = \mathbb{E}[f(X) - Y].
\end{align*}
In particular, under the squared loss, the rectifier $\Delta_\theta$ does not depend on $\theta$, and we indicate this by dropping the subscript $\theta$ and writing $\Delta := \Delta_\theta$.

In order to apply FAB-PPI to this setting, we follow the steps outlined in \cref{sec:fab-ppi}. In particular, we use the sample mean of the unlabelled data \eqref{eq:msamplemean} as the estimator $\widehat{m}_\theta$ of $m_\theta$,
\begin{equation*}
    \widehat{m}_\theta = \frac{1}{N} \sum_{i=1}^N \mathcal{L}_\theta'(X_i, f(X_i)) = \theta - \frac{1}{N}\sum_{i=1}^N f(\tX_i),
\end{equation*}
and either the sample mean \eqref{eq:deltasamplemean} or the control variate estimator \eqref{eq:deltahatpowertuning} as the estimator $\widehat{\Delta}$ of $\Delta$, as in PPI and PPI\texttt{++}, respectively. To avoid repetitions, in this section we write $\widehat{\Delta}$ as the following general control variate estimator with tuning parameter $\lambda \in \mathbb{R}$,
\begin{align*}
    \widehat \Delta&=\frac{1}{n}\sum_{i=1}^n \left(\mathcal{L}_{\theta}'(X_i,Y_i)-\mathcal{L}_{\theta}'(X_i,f(X_i))\right) - (\lambda - 1) \left( \frac{1}{n}\left [\sum_{i=1}^n \mathcal{L}_{\theta}'(X_i,f(X_i))\right ]  -\widehat m_\theta\right) \\
    &= \frac{1}{n}\sum_{i=1}^n \left(\mathcal{L}_{\theta}'(X_i,Y_i)- \lambda \mathcal{L}_{\theta}'(X_i,f(X_i))\right) + (\lambda - 1) \widehat m_\theta \\
    &= -\overline Y + \lambda \left(\frac{1}{n}\sum_{i=1}^n f(X_i) - \frac{1}{N}\sum_{j=1}^N f(\tX_j)\right) + \frac{1}{N}\sum_{j=1}^N f(\tX_j).
\end{align*}
The sample mean estimator \eqref{eq:deltasamplemean} and the control variate estimator \eqref{eq:deltahatpowertuning} under the squared loss are recovered by setting $\lambda$ to $1$ and $\widehat\lambda$ \eqref{eq:lambdahat}, respectively. From this, the standard PPI mean estimators \eqref{eq:thetahatppi} and \eqref{eq:thetahatppipp} are obtained by solving the equation $\widehat m_\theta + \widehat \Delta = 0$ for $\theta$.

Instead, we first define the Bayes assisted estimator \eqref{eq:FABPPIestimatorDelta} under the chosen prior $\pi_0(\Delta; \tau_n)$,
\begin{equation*}
    \widehat\Delta^{\FABPPI} = \widehat\Delta + \widehat\sigma^2 \ell'\left(\widehat\Delta; \widehat\sigma, \tau_n\right),
\end{equation*}
where $\widehat\sigma^2$ is an estimator of $\var(\widehat\Delta)$ and $\ell_\theta'(z; \sigma, \tau)$ is the derivative of the log-marginal likelihood of a Gaussian likelihood model with mean $\Delta$ and variance $\sigma^2$ under the prior $\pi_0(\Delta, \tau)$. Then, the FAB-PPI mean estimator $\widehat\theta^{\FABPPI}$ under $\pi_0$ is given by the solution to the equation
\begin{equation*}
    \widehat m_\theta + \widehat\Delta^{\FABPPI} = 0
\end{equation*}
in $\theta$, that takes the form
\begin{align*}
    \widehat\theta^{\FABPPI} &= \overline Y - \lambda \left(\frac{1}{n}\sum_{i=1}^n f(X_i) - \frac{1}{N}\sum_{j=1}^N f(\tX_j)\right) - \widehat\sigma^2 \ell'\left(\widehat\Delta; \widehat\sigma, \tau_n\right),
\end{align*}
which matches the expression in \cref{eq:mean_estimator_fabppi}. Furthermore, by recognising that the first two terms in the above expression match \eqref{eq:thetahatppipp}, we can alternatively write the FAB-PPI mean estimator as
\begin{align*}
    \widehat\theta^{\FABPPI} = \widehat{\theta} - \widehat\sigma^2 \ell'\left(\widehat\Delta; \widehat\sigma, \tau_n\right),
\end{align*}
where $\widehat\theta$ is the corresponding standard PPI mean estimator.

Given $\alpha \in (0, 1)$, we construct the FAB-PPI confidence region $\calC_\alpha^{\FABPPI}$ for the mean as described in \cref{sec:fabppi_cr}. In particular, let $\mathcal{T}_{\alpha - \delta}(\widehat m_\theta)$ denote a standard $1 - (\alpha - \delta)$ confidence interval for $m_\theta$,
\begin{align*}
    \mathcal{T}_{\alpha - \delta}(\widehat m_\theta) = (\widehat m_\theta \pm \widehat\sigma^f z_{1 - (\alpha - \delta) / 2}) = \left(\theta - \widehat{\theta}^f \pm \widehat\sigma^f z_{1 - (\alpha - \delta) / 2}\right),
\end{align*}
where $\widehat\theta^f := \frac{1}{N}\sum_{i=1}^N f(\tX_i)$ for conciseness, and $(\widehat\sigma^f)^2$ is an estimator of $\var(\widehat m_\theta)$. Then, we apply the FAB framework under the prior $\pi_0(\Delta; \tau_n)$ to obtain a $1 - \delta$ confidence region for $\Delta$,
\begin{align*}
    \calR^{\FABPPI}_\delta(\widehat\Delta) = \text{FAB-CR}(\widehat\Delta; \pi_0(\cdot~;\tau_n), \widehat\sigma, \delta),
\end{align*}
where, again, $\widehat\sigma^2$ is an estimator of $\var(\widehat\Delta)$. Finally, to avoid making assumptions on the specific form of $\calR^{\FABPPI}_\delta(\widehat\Delta)$, we use $(\inf\calR^{\FABPPI}_\delta), \sup(\calR^{\FABPPI}_\delta)) \supseteq \calR^{\FABPPI}_\delta(\widehat\Delta)$ in the definition of $\calC_\alpha^{\FABPPI}$ \eqref{eq:fabppiconfidenceinterval} to obtain the FAB-PPI interval
\begin{align*}
    \calC_\alpha^{\FABPPI} &= \left\{\theta \mid 0\in \left(\theta - \widehat\theta^f \pm \widehat\sigma^f z_{1 - (\alpha - \delta) / 2}\right) + (\inf(\calR^{\FABPPI}_\delta), \sup(\calR^{\FABPPI}_\delta))\right\} \\
    &= \left\{\theta \mid 0\in \left(\theta - \widehat\theta^f - \widehat\sigma^f z_{1 - (\alpha - \delta) / 2} + \inf(\calR^{\FABPPI}_\delta), \theta - \widehat\theta^f + \widehat\sigma^f z_{1 - (\alpha - \delta) / 2} + \sup(\calR^{\FABPPI}_\delta)\right)\right\} \\
    &= \left(\widehat\theta^f - \widehat\sigma^f z_{1 - (\alpha - \delta) / 2} - \sup(\calR^{\FABPPI}_\delta), \widehat\theta^f + \widehat\sigma^f z_{1 - (\alpha - \delta) / 2} - \inf(\calR^{\FABPPI}_\delta)\right).
\end{align*}
\cref{algo:FABPPIMean} summarises the FAB-PPI approach under the squared loss, where $\widehat\xi$ is defined for notational convenience and the corresponding sample variances are used as $(\widehat\sigma^f)^2$ and $\widehat\sigma^2$.
\begin{algorithm}
    \caption{FAB-PPI for mean estimation}
    \label{algo:FABPPIMean}
    {\bfseries Input:} labelled $\{(X_i,Y_i)\}_{i=1}^n$, unlabelled $\{\tX_j\}_{j=1}^N$, predictor $f$, prior $\pi_0(\cdot~; \tau_n)$, error levels $\alpha$, $\delta$
    \begin{algorithmic}
        \STATE Set $\widehat\lambda=1$ (FAB-PPI) or estimate $\widehat\lambda$ from data (FAB-PPI\texttt{++}) using \cref{eq:lambdahat}
        \STATE $\widehat\theta^f \gets \frac{1}{N}\sum_{j=1}^N f(\tX_j)$
        \STATE $\widehat\xi \gets \frac{1}{n}\sum_{i=1}^n (\widehat\lambda f(X_i) - Y_i)$
        \STATE $\widehat \Delta \gets \widehat\xi  - (\widehat\lambda-1)\widehat\theta^f$
        \STATE $(\widehat\sigma^f)^2 \gets \frac{1}{N-1}\sum_{j=1}^N (f(\tX_j)-\widehat\theta^f)^2$
        \STATE $\widehat\sigma_\xi^2 = \frac{1}{n-1}\sum_{i=1}^n (\widehat\lambda f(X_i) - Y_i - \widehat\xi)^2$
        \STATE $\widehat\sigma^2 \gets \frac{1}{n}\widehat\sigma_\xi^2 + \frac{(\widehat\lambda-1)^2}{N}(\widehat\sigma^f)^2$
        \STATE $\calR^{\FABPPI}_\delta \gets \text{FAB-CR}(\widehat \Delta;\pi_0(\cdot~;\tau_n), \widehat\sigma,\delta)$
    \end{algorithmic}
    {\bfseries Outputs:} estimator $\widehat \theta^{\FABPPI} =  \widehat\theta^f - \widehat \Delta - \widehat\sigma^2 \ell'(\widehat\Delta; \widehat\sigma, \tau_n)$ and CR $\calC_\alpha^\FABPPI=(\widehat\theta^f-\sigma^f z_{1-(\alpha-\delta)/2}-\sup(\calR^{\FABPPI}_\delta),\widehat\theta^f+\sigma^f z_{1-(\alpha-\delta)/2}-\inf(\calR^{\FABPPI}_\delta))$
\end{algorithm}


%% Proofs
\section{Proofs}
\label{app:proofs}
\subsection{\cref{thm:fabppi_coverage} - Asymptotic coverage of FAB-PPI under the Gaussian and horseshoe priors}
\label{app:thm:fabppi_coverage}
Under the prior $\pi_0(\cdot; \widehat{\sigma}_\theta )$, the FAB confidence region for the rectifier $\Delta_\theta$ is
\begin{align}
    \calR^{\FABPPI}_\delta(\widehat{\Delta}_\theta; \widehat{\sigma}_\theta )=\{\Delta_\theta  \mid \Delta_\theta -\widehat{\sigma}_\theta z_{1-\delta w_\delta(\Delta_\theta; \widehat{\sigma}_\theta)}<\widehat\Delta_\theta<\Delta_\theta + \widehat{\sigma}_\theta z_{1-\delta(1-w_\delta(\Delta_\theta; \widehat{\sigma}_\theta))}\},
\label{eq:FABdelta}
\end{align}
where $w_\delta(\cdot; \widehat{\sigma}_\theta)$ is the FAB spending function. The proof of asymptotic coverage is organised as follows. First, we show that $\calR^{\FABPPI}_\delta(\widehat{\Delta}_\theta; \widehat{\sigma}_\theta )$ is asymptotically a $1-\delta$ confidence interval. This is established via \cref{lemma:coverage_gaussian_delta} for the Gaussian prior, and via \cref{lemma:coverage_horseshoe_delta1,lemma:fabcr_horseshoe_convergence} for the horseshoe prior. This result is then combined with the asymptotic coverage of the standard sample mean estimator for $m_\theta$ to conclude the asymptotic coverage of the FAB-PPI estimator of $\theta^\star$.

We first prove the following lemma for the Gaussian prior, demonstrating that the rectifier has the correct asymptotic coverage.
 
\begin{lemma}
    \label{lemma:coverage_gaussian_delta}
    Let $\widehat{\Delta}_\theta$ be a consistent estimator of $\Delta_\theta$ such that a CLT holds for $\widehat{\Delta}_\theta$, i.e.
    \begin{equation*}
        \frac{\widehat{\Delta}_\theta - \Delta_\theta}{\widehat{\sigma}_\theta } \to \mathcal{N}(0, 1)
    \end{equation*}
    as $\min(n,N) \to \infty$, where $\frac{\widehat{\sigma}_\theta^2}{\var(\widehat{\Delta}_\theta)}\to 1$ almost surely. Let $\pi_{0}(\cdot; \widehat{\sigma}_\theta)$ be the Gaussian prior \eqref{eq:Gaussianprior} for $\Delta_\theta$ and consider the corresponding $1 - \delta$ FAB confidence region
    \begin{equation*}
        \calR^{\FABPPI}_\delta(\widehat{\Delta}_\theta; \widehat{\sigma}_\theta ) = \text{FAB-CR}\left(\widehat{\Delta}_\theta; \pi_{0}\left(\cdot~; \widehat{\sigma}_\theta\right), \widehat{\sigma}_\theta, \delta\right).
    \end{equation*}
    Then
    \begin{align}
        {\lim}_{\min(n,N)\to\infty}\Pr(\Delta_\theta \in \calR^{\FABPPI}_\delta(\widehat{\Delta}_\theta; \widehat{\sigma}_\theta )\mid \Delta_\theta)\geq 1-\delta.
    \end{align}
\end{lemma}
\begin{proof}
    Using \cref{prop:fabgaussianprop}, for any $x>0$, $\sigma>0$,
    \begin{align*}
        \frac{2x}{\sigma}&=g_\delta(g^{-1}_\delta(2x/\sigma))\\
        &=g_\delta(w_\delta(x;\sigma))\\
        &=z_{1-\delta(1-w_\delta(x;\sigma))}-z_{1-\delta w_\delta(x;\sigma)}
    \end{align*}
    and
    \begin{align*}
        \sigma z_{1-\delta(1-w_\delta(x; \sigma))}&=\sigma z_{1-\delta(1-w_\delta(x/\sigma; 1))}\\
        &= 2x + \sigma z_{1-\delta w_\delta(x/\sigma;1)}.
    \end{align*}
    The FAB confidence region \eqref{eq:FABdelta} can therefore be written as
    \begin{align*}
        \calR^{\FABPPI}_\delta(\widehat{\Delta}_\theta; \widehat{\sigma}_\theta )&=\{\Delta_\theta>0  \mid \Delta_\theta -\widehat{\sigma}_\theta z_{1-\delta w_\delta(\Delta_\theta; \widehat{\sigma}_\theta)}<\widehat\Delta_\theta< 3\Delta_\theta + \widehat{\sigma}_\theta z_{1-\delta w_\delta(\Delta_\theta; \widehat{\sigma}_\theta)}\}\\
        &~~~\cup
        \{\Delta_\theta<0  \mid 3\Delta_\theta -\widehat{\sigma}_\theta z_{1-\delta (1-w_\delta(\Delta_\theta; \widehat{\sigma}_\theta))}<\widehat\Delta_\theta<\Delta_\theta + \widehat{\sigma}_\theta z_{1-\delta(1-w_\delta(\Delta_\theta; \widehat{\sigma}_\theta))}\}.
    \end{align*}
    Additionally, for any $x>0$,
    \begin{align*}
        z_{1-\delta w_\delta(x; \widehat{\sigma}_\theta)}\to z_{1-\delta}\\
        z_{1-\delta (1-w_\delta(-x; \widehat{\sigma}_\theta))}\to z_{1-\delta}
    \end{align*}
    almost surely as $\min(n,N)\to\infty$.

    It follows from \cref{eq:FABdelta} that, for any $z_{1-\delta}>\epsilon>0$, there exist $N_0$ such that for all $n,N$ with $\min(n,N)\geq N_0$, the FAB confidence region $\calR^{\FABPPI}_\delta(\widehat{\Delta}_\theta; \widehat{\sigma}_\theta )$ contains the set
    \begin{align*}
        \mathcal S_\delta(\widehat{\Delta}_\theta; \widehat{\sigma}_\theta )&=\left \{\Delta_\theta>0 \mid \Delta_\theta - \widehat{\sigma}_\theta (z_{1-\delta}-\epsilon) < \widehat \Delta_\theta < 3\Delta_\theta +\widehat{\sigma}_\theta (z_{1-\delta}-\epsilon) \right\}\\
        &~~~\cup
        \left \{\Delta_\theta<0 \mid 3\Delta_\theta -\widehat{\sigma}_\theta (z_{1-\delta}-\epsilon) < \widehat \Delta_\theta <  \Delta_\theta - \widehat{\sigma}_\theta (z_{1-\delta}-\epsilon) \right\}.
    \end{align*}
    For any fixed $\Delta_\theta>0$,
    \begin{align}
        \Pr(\Delta_\theta\in \mathcal S_\delta(\widehat{\Delta}_\theta; \widehat{\sigma}_\theta ))=\Pr\left(-(z_{1-\delta}-\epsilon) <   \frac{\widehat\Delta_\theta-\Delta_\theta}{\widehat\sigma_\theta}<\frac{2\Delta_\theta}{\widehat\sigma_\theta} +z_{1-\delta}-\epsilon \right).
    \end{align}
    Noting that $\frac{2\Delta_\theta}{\widehat\sigma_\theta} +z_{1-\delta}-\epsilon\to\infty$ a.s. as $\min(n,N)\to\infty$, we obtain that
    \begin{align}
        {\lim}_{\min(n,N)\to\infty}\Pr(\Delta_\theta \in \calR^{\FABPPI}_\delta(\widehat{\Delta}_\theta; \widehat{\sigma}_\theta )\mid \Delta_\theta)
        \geq {\lim}_{\min(n,N)\to\infty}\Pr(\Delta_\theta \in \mathcal S_\delta(\widehat{\Delta}_\theta; \widehat{\sigma}_\theta )\mid \Delta_\theta)
        \geq 1-\delta.
    \end{align}
\end{proof}

\begin{lemma}
    \label{lemma:coverage_horseshoe_delta1}
    For $\delta\in(0,1)$, let $w_\delta(\cdot; \widehat{\sigma}_\theta)$ be the FAB spending function associated to the FAB confidence region $$\calR^{\FABPPI}_\delta(\bullet)=\text{FAB-CR}(\bullet;\pi_0(\cdot~;\widehat\sigma_\theta ),\widehat\sigma_\theta,\delta).$$
    Assume $\widehat{\sigma}_\theta\to 0$ almost surely as $\min(n,N)\to\infty$. If $\pi_0$ is the horseshoe prior \eqref{eq:horseshoeprior}, then, for any $z\in\bbR$,
    \begin{equation*}
        w_\delta(z; \widehat{\sigma}_\theta)\to\frac{1}{2}\text{ a.s. as }\min(n,N)\to\infty.
    \end{equation*}
\end{lemma}
\begin{proof}
    As described in \cref{sec:app:backgroundFAB}, the spending function $w_\delta$, which is continuous, satisfies,  for any $z\in\bbR$ and $\sigma>0$,
    \begin{equation*}
        w_\delta\left(z; \sigma \right) = w_\delta\left(\frac{z}{\sigma}; 1\right).
    \end{equation*}
    By \cref{prop:horseshoefab}, we have
    $$
        \lim_{z\to\infty} w_\delta(z;1)=\lim_{z\to -\infty} w_\delta(z;1)=w_\delta(0;1)=\frac{1}{2}.
    $$
    Since $\widehat{\sigma}_\theta\to 0$ almost surely and $w_\delta$ is continuous, for any $z\in\bbR$, we conclude that
    \begin{equation*}
    w_\delta\left(z; \widehat{\sigma}_\theta \right)  \to\frac{1}{2}\text{ a.s. as }\min(n,N)\to\infty.
    \end{equation*}
\end{proof}

\begin{lemma}
    \label{lemma:fabcr_horseshoe_convergence}
    Let $\widehat{\Delta}_\theta$ be a consistent estimator of $\Delta_\theta$ such that a CLT holds for $\widehat{\Delta}_\theta$, i.e.
    \begin{equation*}
        \frac{\widehat{\Delta}_\theta - \Delta_\theta}{\widehat{\sigma}_\theta } \to \mathcal{N}(0, 1)
    \end{equation*}
    as $\min(n,N) \to \infty$, where $\frac{\widehat{\sigma}_\theta^2}{\var(\widehat{\Delta}_\theta)}\to 1$ almost surely. Specify a symmetric prior $\pi_{0}(\cdot; \widehat{\sigma}_\theta)$  for $\Delta_\theta$ and consider the corresponding $1 - \delta$ FAB confidence region
    \begin{equation*}
        \calR^{\FABPPI}_\delta(\widehat{\Delta}_\theta; \widehat{\sigma}_\theta ) = \text{FAB-CR}\left(\widehat{\Delta}_\theta; \pi_{0}\left(\cdot~; \widehat{\sigma}_\theta\right), \widehat{\sigma}_\theta, \delta\right),
    \end{equation*}
    with associated weight function $w_\delta(\Delta_\theta; \widehat{\sigma}_\theta)$. If, for any $x\in\bbR$,
    \begin{align}
        w_\delta\left(x; \widehat{\sigma}_\theta\right)\to \frac{1}{2}\text{ a.s. as }\min(n,N)\to\infty, \label{eq:wtendstohalf}
    \end{align}
    then the confidence region $\calR^{\FABPPI}_\delta(\widehat{\Delta}_\theta; \widehat{\sigma}_\theta )$  reverts to the classical $1 - \delta$ $z$-interval for $\Delta_\theta$, i.e., almost surely,
    \begin{equation*}
        \lim_{\min(n,N) \to \infty} \frac{\sup(\calR^{\FABPPI}_\delta(\widehat{\Delta}_\theta; \widehat{\sigma}_\theta )) - \widehat{\Delta}_\theta}{\widehat{\sigma}_\theta} = \lim_{\min(n,N) \to \infty} \frac{\widehat{\Delta}_\theta - \inf(\calR^{\FABPPI}_\delta(\widehat{\Delta}_\theta; \widehat{\sigma}_\theta ))}{\widehat{\sigma}_\theta} = z_{1 - \delta / 2},
    \end{equation*}
    with
    \begin{align}
        {\lim}_{\min(n,N)\to\infty}\Pr(\Delta_\theta \in \calR^{\FABPPI}_\delta(\widehat{\Delta}_\theta; \widehat{\sigma}_\theta )\mid \Delta_\theta)=1-\delta.\label{eq:coverageDelta}
    \end{align}
\end{lemma}
\begin{proof}
    As $\Phi^{-1}$ is continuous, \cref{eq:wtendstohalf} implies that, for any $x\in\bbR$,
    $$
        \lim_{\min(n,N)\to\infty} z_{1-\delta w_\delta(x; \widehat{\sigma}_\theta)}=\lim_{\min(n,N)\to\infty} z_{1-\delta(1-w_\delta(x; \widehat{\sigma}_\theta))}=z_{1-\delta/2}
    $$
    almost surely. From \cref{eq:FABdelta}, it follows that, for any $\epsilon>0$ and for $\min(n,N)$ large enough,
    \begin{align*}
        z_{1 - \delta / 2}-\epsilon \leq \frac{\sup(\calR^{\FABPPI}_\delta(\widehat{\Delta}_\theta; \widehat{\sigma}_\theta )) - \widehat{\Delta}_\theta}{\widehat{\sigma}_\theta}\leq z_{1 - \delta / 2}+\epsilon\\
        z_{1 - \delta / 2}-\epsilon \leq \frac{\widehat{\Delta}_\theta - \inf(\calR^{\FABPPI}_\delta(\widehat{\Delta}_\theta; \widehat{\sigma}_\theta ))}{\widehat{\sigma}_\theta} \leq z_{1 - \delta / 2}+\epsilon.
    \end{align*}
    \eqref{eq:coverageDelta} then follows directly, which completes the proof.
\end{proof}

With the above two lemmas, we can now prove \cref{thm:fabppi_coverage}, which we restate here in extended form.
\begin{theorem}
    Consider a convex estimation problem whose solution can be expressed as in \cref{eq:defthetastar2}. For all $\theta \in \mathbb{R}$, define $\widehat{\Delta}_\theta$ and $\widehat{m}_\theta$ as in \cref{sec:fab-ppi} and let
    \begin{align*}
        \calR^{\FABPPI}_\delta(\widehat{\Delta}_\theta; \widehat{\sigma}_\theta ) &= \text{FAB-CR}\left(\widehat{\Delta}_\theta; \pi_{0}\left(\cdot~; \widehat{\sigma}_\theta\right), \widehat{\sigma}_\theta, \delta\right),\nonumber\\
        \calT_{\alpha-\delta}(\widehat m_\theta; \widehat\sigma_\theta^f) &= \left(\widehat{m}_\theta - \widehat\sigma_\theta^f z_{1-(\alpha - \delta) / 2}, \widehat{m}_\theta + \widehat\sigma_\theta^f  z_{1-(\alpha - \delta) / 2}\right),
    \end{align*}
    where $\frac{\widehat{\sigma}_\theta^2}{\var(\hat{\Delta}_\theta)}\to 1$ and $\frac{(\widehat\sigma_\theta^f)^2}{\var(\hat{m}_\theta)}\to 1$ almost surely as $\min(n,N)\to\infty$. Then, the FAB-PPI confidence region $\calC_\alpha^\FABPPI$, defined as
    \begin{equation*}
        \calC_\alpha^\FABPPI=\left\{\theta \mid 0\in \calR^{\FABPPI}_\delta(\widehat\Delta_\theta; \widehat{\sigma}_\theta) + \calT_{\alpha-\delta}(\widehat m_\theta; \widehat\sigma_\theta^f)\right\},
    \end{equation*}
    has correct asymptotic coverage, i.e.~it satifies
    \begin{equation*}
        \liminf_{\min(n,N) \to \infty} \Pr(\theta^* \in \calC_\alpha^\FABPPI) = \liminf_{\min(n,N) \to \infty} \Pr(0 \in \calR^{\FABPPI}_\delta(\widehat\Delta_{\theta^*}; \widehat{\sigma}_{\theta^\star}) + \calT_{\alpha-\delta}(\widehat m_{\theta^*}; \widehat\sigma_{\theta^\star}^f)) \geq 1 - \alpha.
    \end{equation*}
\end{theorem}
\begin{proof}
    By \cref{lemma:coverage_gaussian_delta} (Gaussian prior) and \cref{lemma:coverage_horseshoe_delta1,lemma:fabcr_horseshoe_convergence} (horseshoe prior), $\calR^{\FABPPI}_\delta(\widehat\Delta_{\theta^\star}; \widehat{\sigma}_{\theta^\star})$  is an asymptotically valid $1-\delta$ confidence region for $\Delta_{\theta^\star}$, that is
    \begin{align*}
        \liminf_{n, N \to \infty} \text{Pr}(\Delta_{\theta^\star} \in \calR^{\FABPPI}_\delta(\widehat\Delta_{\theta^\star}; \widehat{\sigma}_{\theta^\star})) &\geq 1 - \delta.
    \end{align*}
    Similarly, the CLT for $\widehat m_\theta$ implies that
    \begin{align*}
        \liminf_{n, N \to \infty} \text{Pr}(m_{\theta^*} \in \calT_{\alpha-\delta}(\widehat m_{\theta^*};\widehat\sigma_{\theta^*}^f)) &\geq 1 - (\alpha - \delta).
    \end{align*}
    Consider the event
    \begin{equation*}
        E = \{\Delta_{\theta^*} \in \calR^{\FABPPI}_\delta(\widehat\Delta_{\theta^*}; \widehat{\sigma}_{\theta^\star})\} \cap \{m_{\theta^*} \in \calT_{\alpha-\delta}(\widehat m_{\theta^*};\widehat\sigma_{\theta^*}^f)\}.
    \end{equation*}
    By Boole's inequality,
    \begin{align*}
        \liminf_{n,N\to\infty} \mathrm{Pr}(E) &\geq 1 - \limsup_{n,N\to\infty}\mathrm{Pr}(\{\Delta_{\theta^*} \notin \calR^{\FABPPI}_\delta(\widehat\Delta_{\theta^*}; \widehat{\sigma}_{\theta^\star})\} \cup \{m_{\theta^*} \notin \calT_{\alpha-\delta}(\widehat m_{\theta^*};\widehat\sigma_{\theta^*}^f)\}) \\
        &\geq 1 - \limsup_{n\to\infty}\mathrm{Pr}(\{\Delta_{\theta^*} \notin \calR^{\FABPPI}_\delta(\widehat\Delta_{\theta^*}; \widehat{\sigma}_{\theta^\star})\}) - \limsup_{N\to\infty} \mathrm{Pr}(\{m_{\theta^*} \notin \calT_{\alpha-\delta}(\widehat m_{\theta^*};\widehat\sigma_{\theta^*}^f)\}) \\
        &\geq 1 - \delta - (\alpha - \delta) \\
        &= 1 - \alpha.
    \end{align*}
    Furthermore, on the event $E$, we have that
    \begin{equation*}
        0 = \Delta_{\theta^*} + m_{\theta^*} \in \calR^{\FABPPI}_\delta(\widehat\Delta_{\theta^*}; \widehat{\sigma}_{\theta^\star}) + \calT_{\alpha-\delta}(\widehat m_{\theta^*};\widehat\sigma_{\theta^*}^f),
    \end{equation*}
    where the first equality follows from \cref{eq:defthetastar2}. As a result of this,
    \begin{equation*}
        \liminf_{n,N \to \infty} \text{Pr}(0 \in \calR^{\FABPPI}_\delta(\widehat\Delta_{\theta^*}; \widehat{\sigma}_{\theta^\star}) + \calT_{\alpha-\delta}(\widehat m_{\theta^*};\widehat\sigma_{\theta^*}^f)) \geq 1 - \alpha,
    \end{equation*}
    as desired.
\end{proof}

\begin{remark}\label{rem:fabppi_asymptotic}
    With both the Gaussian and horseshoe priors, we obtain asymptotic coverage. However, the asymptotic confidence regions differ significantly. In the Gaussian case, the volume of the confidence region does not vanish asymptotically. Instead, the confidence region converges to $(\frac{\Delta_\theta}{3},\Delta_\theta)$, with volume of $\frac{2}{3}|\Delta_\theta|$.
    In contrast, when using the horseshoe prior \eqref{eq:horseshoeprior}, we revert to the usual CLT-based confidence intervals, and the volume of the confidence region converges to zero almost surely.
\end{remark}

\subsection{\cref{prop:fabppi_robustness} - Robustness of FAB-PPI under the horseshoe prior}
\label{app:thm:fabppi_robustness}
Let $\pi_0$ be the horseshoe prior \eqref{eq:horseshoeprior}, and consider the FAB confidence region $\calR^{\FABPPI}_\delta(\widehat{\Delta}_\theta; \widehat{\sigma}_\theta )$ for $\Delta_\theta$, as defined in \cref{eq:FABdelta}.

We first state a corollary of \citet[Theorem 2.2]{Cortinovis2024}, which follows from the power-law tails of the marginal likelihood under the horseshoe prior (see \cref{sec:app:backgroundhorseshoe}). The corollary states that, if $|\widehat{\Delta}_\theta|$ is very large, then the standard CLT-based confidence interval, $(\widehat{\Delta}_\theta\pm \widehat{\sigma}_\theta z_{1-\delta/2})$, is recovered.

\begin{corollary}(\citet[Theorem 2.2]{Cortinovis2024})
    \label{prop:limitstandardCI}
    For any $\sigma>0$,
    \begin{align*}
        \lim_{|\Delta|\to\infty} \sup(\calR^{\FABPPI}_\delta(\Delta; \sigma )) - \Delta&=\lim_{|\Delta|\to\infty} \Delta-\inf(\calR^{\FABPPI}_\delta(\Delta; \sigma))=\sigma z_{1-\delta/2}.
    \end{align*}
\end{corollary}

Define
\begin{align}
    \mathcal S_{\alpha,\delta}(\widehat\Delta_\theta, \widehat{\sigma}_\theta, \widehat m_\theta, \widehat\sigma_\theta^f)
    =\calR^{\FABPPI}_\delta(\widehat\Delta_\theta; \widehat{\sigma}_\theta) + \calT_{\alpha-\delta}(\widehat m_\theta; \widehat\sigma_\theta^f),
\end{align}
where
$$
    \calT_{\alpha-\delta}(\widehat m_\theta; \widehat\sigma_\theta^f)
    = \left(\widehat{m}_\theta - \widehat\sigma_\theta^f z_{1-(\alpha - \delta) / 2}, \widehat{m}_\theta + \widehat\sigma_\theta^f  z_{1-(\alpha - \delta) / 2}\right)
$$
is the standard CLT-based confidence interval for $m_\theta$. From Corollary \ref{prop:limitstandardCI}, for any fixed $\sigma>0$, $m\in\bbR$, $\sigma^f>0$,
\begin{align*}
    \lim_{|\Delta|\to\infty} \sup(\mathcal S_{\alpha,\delta}(\Delta, \sigma, m, \sigma^f)) - (\Delta+m) &=\lim_{|\Delta|\to\infty} (\Delta+m)-\inf(\mathcal S_{\alpha,\delta}(\Delta, \sigma, m, \sigma^f))=\sigma z_{1-\delta/2} + \sigma^f z_{1-(\alpha-\delta)/2}.
\end{align*}
Therefore, if $|\widehat{\Delta}_\theta|\gg 0$, the confidence region $\mathcal S_{\alpha,\delta}(\widehat\Delta_\theta, \widehat{\sigma}_\theta, \widehat m_\theta, \widehat\sigma_\theta^f)$ reverts to the standard interval 
$$
    (\widehat\Delta_\theta+\widehat m_\theta \pm (\widehat{\sigma}_\theta z_{1-\delta/2}+ \widehat{\sigma}^f_\theta z_{1-(\alpha-\delta)/2}) ).
$$
It follows that, if $\inf_{\theta'\in\bbR} |\widehat\Delta_{\theta'}|\gg 0$  the confidence region for $\theta^\star$, defined as
\begin{equation*}
    \calC_\alpha^\FABPPI=\left\{\theta \mid 0\in \calR^{\FABPPI}_\delta(\widehat\Delta_\theta; \widehat{\sigma}_\theta) + \calT_{\alpha-\delta}(\widehat m_\theta; \widehat\sigma_\theta^f)\right\},
\end{equation*}    
reverts to the standard, CLT-based PPI confidence region
\begin{equation*}
    \calC_\alpha^\text{PP}=\left\{\theta\in\bbR \mid  -\widehat{\sigma}_\theta z_{1-\delta/2}- \widehat{\sigma}^f_\theta z_{1-(\alpha-\delta)/2}  \leq \widehat\Delta_\theta+\widehat m_\theta\leq  \widehat{\sigma}_\theta z_{1-\delta/2}+ \widehat{\sigma}^f_\theta z_{1-(\alpha-\delta)/2}           \right\}.
\end{equation*}

\subsection{\cref{prop:fabppi_consistency} - Consistency of FAB-PPI mean estimators}
\label{app:prop:fabppi_consistency}
Below, we use $\BPP$ and $\BPPpp$ to distinguish between the estimators $\widehat\theta^\FABPPI$, $\widehat\theta$, $\widehat\Delta$ and $\widehat\sigma$ in the two cases of FAB-PPI and FAB-PPI\texttt{++}. Then, the FAB-PPI and FAB-PPI\texttt{++} mean estimators are given by
\begin{align*}
    \widehat \theta^{\BPP} &=  \widehat \theta^{\PP} - (\widehat\sigma^{\PP})^2 \ell'\left(\widehat\Delta^\PP; \widehat\sigma^{\PP}, \widehat\sigma^{\PP}\right),\\
    \widehat \theta^{\BPPpp} &=  \widehat \theta^{\PPpp} - (\widehat\sigma^{\PPpp})^2 \ell'\left(\widehat\Delta^\PPpp;\widehat\sigma^{\PPpp},\widehat\sigma^{\PP}\right),
\end{align*}
where we recall that $\ell(y;\sigma,\tau)=\log \int_\bbR \Normal(y;\Delta,\sigma^2)\pi_0(\Delta;\tau)d\Delta$ where $\tau$ is a scale parameter of the prior $\pi_0$. By assumption, both PPI estimators, $\widehat \theta^{\PP}$ and $\widehat \theta^{\PPpp}$, are strongly consistent estimators of $\theta^\star$. It remains to prove that
\begin{align}
    (\widehat\sigma^{\PP})^2 \ell'\left(\widehat\Delta^\PP;\widehat\sigma^{\PP}, \widehat\sigma^{\PP} \right)\to 0\label{eq:fabppicon_conv1}\\
    (\widehat\sigma^{\PPpp})^2 \ell'\left(\widehat\Delta^\PPpp;\widehat\sigma^{\PPpp}, \widehat\sigma^{\PP} \right)\to 0\label{eq:fabppicon_conv2}
\end{align}
almost surely, as $\min(n,N)\to\infty$. For any $\sigma>0$, we have $\ell'(y;\sigma,\sigma)=\frac{1}{\sigma}\ell'(y/\sigma; 1,1)$.

Under the horseshoe prior \eqref{eq:horseshoeprior}, $\ell_{\text{HS}}'(y; 1,1)$ is bounded.  Therefore \eqref{eq:fabppicon_conv1} and \eqref{eq:fabppicon_conv2} hold almost surely by sandwiching.

Under the Gaussian prior \eqref{eq:Gaussianprior},
$$
    \ell_{\text{N}}'(y;\sigma,\sigma)=-\frac{y}{2\sigma^2}.
$$
Hence, since $\widehat\Delta^{\PP}\to \Delta$ and $\widehat\Delta^{\PPpp}\to \Delta$ almost surely, where we recall that $\Delta=\mathbb E[f(X)-Y]$, we obtain
\begin{align*}
    (\widehat\sigma^{\PP})^2 \ell_{\text{N}}'\left(\widehat\Delta^\PP; \widehat\sigma^{\PP},\widehat\sigma^{\PP}\right)\to -\frac{\Delta}{2}\\
    (\widehat\sigma^{\PPpp})^2 \ell_{\text{N}}'\left(\widehat\Delta^\PPpp; \widehat\sigma^{\PPpp},\widehat\sigma^{\PP}\right)\to -\frac{\Delta}{2}
\end{align*}
almost surely as $\min(n,N)\to\infty$, which implies that the FAB-PPI mean estimators under the Gaussian prior \eqref{eq:Gaussianprior} are not consistent.


%% Multivariate FAB-PPI
\section{Multivariate FAB-PPI}
\label{app:multivariate}
Here we extend FAB-PPI to the multivariate case, where $\theta,m_\theta,\Delta_\theta \in \bbR^d$. While most of the methodology remains the same as in the univariate case, we now need to specify a multivariate prior for $\Delta_\theta$, for which we consider independent horseshoe priors on each dimension.

\subsection{Multivariate Bayesian PPI Estimators}
As in the univariate case, we use the sample mean $\widehat{m}_\theta$ as the estimator of $m_\theta$.
Similarly, we consider some consistent estimator $\widehat \Delta_\theta$ of $\Delta_\theta$, such as the sample mean \eqref{eq:deltasamplemean}, as in PPI, or the control variate estimator \eqref{eq:deltahatpowertuning}, as in PPI\texttt{++}.
Crucially, we assume that a multivariate CLT holds for this estimator, that is
\begin{equation*}
    \widehat\Sigma_\theta^{-1/2} \left(\widehat\Delta_\theta -\Delta_\theta\right)\to \Normal(0, \mathrm{I})
\end{equation*}
as $\min(n,N)\to\infty$, where $\widehat\Sigma_\theta$ is an estimator of $\cov(\widehat\Delta_\theta)$. Again, this holds for both the PPI and PPI\texttt{++} estimators~\citep{Angelopoulos2023,Angelopoulos2023a}.
We consider $d$ independent priors, $\pi_0(\Delta_{\theta,k};\widehat\sigma_{\theta,k})$ for $k = 1,\dots,d$, on the components of $\Delta_\theta$, where $\widehat\sigma_{\theta,k}^2$ is the $k$-th diagonal element of $\widehat\Sigma_\theta$. The multivariate FAB-PPI estimator $\widehat\Delta_{\theta}^{\FABPPI}$ is formed by stacking the individual estimators
\begin{align*}
    \widehat\Delta_{\theta,k}^{\FABPPI}= \widehat\Delta_{\theta,k} + \widehat\sigma_{\theta,k}^2 \ell'\left(\widehat\Delta_{\theta, k};\widehat\sigma_{\theta,k}\right)
\end{align*}
for each dimension $k = 1,\dots,d$. Importantly, note that the $k$-th dimension of $\widehat\Delta_{\theta}^{\FABPPI}$ only depends on the $k$-th dimension of the observed $(\mathcal{L}_\theta'(X_i, Y_i) - \mathcal{L}_\theta'(X_i, f(X_i)))$ that are used to estimate $\Delta_{\theta,k}$.
The FAB-PPI estimator of $\theta^\star$ then becomes the solution, in $\theta$, to the equation
$$
\widehat m_\theta + \widehat\Delta_\theta^{\FABPPI} = \mathbf{0} \in \mathbb{R}^d.
$$

\subsection{Multivariate FAB-PPI Confidence Regions}
As in the univariate case, let $\calT_{\alpha-\delta}(\widehat m_\theta)$ denote a standard $1-(\alpha-\delta)$ confidence interval for $m_\theta$.
For $\Delta_\theta$, we apply the FAB framework with independent horseshoe priors to each dimension $\Delta_{\theta,k}$ and use a union bound to obtain a $1-\delta$ confidence region for $\Delta_\theta$. In particular, let $\calR^{\FABPPI}_{\delta/d}(\widehat\Delta_{\theta,k},\widehat\sigma_{\theta,k})=\text{FAB-CR}(\widehat\Delta_{\theta,k};\pi_0(\cdot~;\widehat\sigma_{\theta,k}),\widehat\sigma_{\theta, k}, \delta/d)$ be a $1 - \delta/d$ FAB confidence region for $\Delta_{\theta,k}$ under the horseshoe prior $\pi_{0}(\cdot~;\widehat\sigma_{\theta,k})$. Then,
\begin{equation*}
    \calR^{\FABPPI}_{\delta}(\widehat\Delta_{\theta},\widehat\sigma_{\theta}) = \left\{\Delta_\theta \mid \Delta_{\theta,k} \in \calR^{\FABPPI}_{\delta/d}(\widehat\Delta_{\theta,k},\widehat\sigma_{\theta,k}),\ k = 1,\dots,d\right\}
\end{equation*}
where $\widehat\Delta_{\theta}=(\widehat\Delta_{\theta,1},\ldots,\widehat\Delta_{\theta,d})$, $\widehat\sigma_{\theta}=(\widehat\sigma_{\theta,1},\ldots,\widehat\sigma_{\theta,d})$,
is a $1-\delta$ multivariate FAB confidence region for $\Delta_\theta$ by a union bound. With this, the multivariate FAB-PPI confidence region $\calC_\alpha^\FABPPI$ is given by
\begin{align*}
    \calC_\alpha^\FABPPI=\left\{\theta \mid \mathbf{0} \in \calR^{\FABPPI}_\delta(\widehat\Delta_\theta,\widehat\sigma_{\theta}) + \calT_{\alpha-\delta}(\widehat m_\theta, \widehat\sigma^f_{\theta})\right\},
\end{align*}
exactly as in the univariate case. Moreover, also multivariate FAB-PPI enjoys exact asymptotic coverage as $\min(n,N) \to \infty$. In particular, \cref{thm:fabppi_coverage} can be easily extended to the multivariate case by applying a union bound over the dimensions of $\Delta_\theta$.


%% Experimental details
\section{Experimental details}
\subsection{Datasets}\label{app:datasets}
Here we provide a brief description of each dataset used for the mean estimation experiments in \cref{sec:real_data}. For additional details, the reader may refer to \citet{Angelopoulos2023}. All of the datasets were downloaded from the examples provided as part of the \texttt{ppi-py} package~\citep{Angelopoulos2023a}.

\paragraph{Alphafold.}
The \textsc{alphafold} dataset contains the following features for $N = 10802$ protein residues analysed by \citet{Bludau2022}: whether the residue is phosphorylated ($Z_i \in \{0,1\})$, whether the residue is part of an intrinsically disordered region (IDR, $Y_i \in \{0,1\}$), and the prediction of the AlphaFold model~\citep{Jumper2021} for the probability of $Y_i$ being equal to one ($f(X_i) \in [0,1]$). The goal is to estimate the odds ratio of a protein being phosphorylated and being part of an IDR region, i.e.
\begin{equation*}
    \theta^\star = \frac{\mu_1 / (1 - \mu_1)}{\mu_0 / (1 - \mu_0)},
\end{equation*}
where $\mu_1 = \mathrm{Pr}(Y = 1 \mid Z = 1)$ and $\mu_0 = \mathrm{Pr}(Y = 1 \mid Z = 0)$. Following \citet{Angelopoulos2023}, given $\alpha \in (0, 1)$, we construct $1 -
\alpha / 2$ confidence intervals $\mathcal{C}_0 = [l_0, u_0]$ and $\mathcal{C}_1 = [l_1, u_1]$ for $\mu_0$ and $\mu_1$, respectively. Then, by a union bound, the interval
\begin{equation*}
    \mathcal{C} = \left\{\frac{c_1}{1 - c_1} \cdot \frac{1 - c_0}{c_0} \colon c_0 \in \mathcal{C}_0, c_1 \in \mathcal{C}_1 \right\} = \left(\frac{l_1}{1 - l_1} \cdot \frac{1 - u_0}{u_0}, \frac{u_1}{1 - u_1} \cdot \frac{1 - l_0}{l_0}\right)
\end{equation*}
has coverage at least $1 - \alpha$.

\paragraph{Forest.}
The \textsc{forest} dataset contains the following features for $N = 1596$ parcels of land in the Amazon rainforest examined during field visits~\citep{Bullock2020}: whether the parcel has been subject to deforestation ($Y_i \in \{0, 1\}$) and the prediction of a gradient-boosted tree model for the probability of $Y_i$ being equal to one ($f(X_i) \in [0, 1]$). The goal is to estimate the fraction of Amazon rainforest lost to deforestation, i.e.~$\theta^* = \mathbb{E}[Y]$.

\paragraph{Galaxies.}
The \textsc{galaxies} dataset contains the following features for $N = 16743$ images from the Galaxy Zoo 2 initiative \citep{Willett2013}: whether the galaxy has spiral arms ($Y_i \in \{0, 1\}$) and the prediction of a ResNet50 model \citep{He2016} for the probability of $Y_i$ being equal to one ($f(X_i) \in [0, 1]$). The goal is to estimate the fraction of galaxies with spiral arms, i.e.~$\theta^* = \mathbb{E}[Y]$.

\paragraph{Genes.}
The \textsc{genes} dataset contains the following features for $N = 61150$ gene promoter sequences: the expression level of the gene induced by the promoter and the prediction of a transformer model for the same quantity \citep{Vaishnav2022}. The goal is to estimate the median expression level across genes.

\paragraph{Census.}
The \textsc{census} dataset contains the following features for $N = 380091$ individuals from the 2019 California census: the individual's age, sex, and yearly income, as well as the prediction of a gradient-boosted tree model trained on the previous year's raw data for the individual's income. The goal is to estimate the ordinary least squares (OLS) regression coefficients when regressing income on age and sex.

\paragraph{Healthcare.}
The \textsc{healthcare} dataset contains the following features for $N = 318215$ individuals from the 2019 California census: the individual's yearly income and whether they have health insurance ($Y_i \in \{0, 1\}$), as well as the prediction of a gradient-boosted tree model trained on the previous year's raw data for the probability of $Y_i$ being equal to one ($f(X_i) \in [0, 1]$). The goal is to estimate the logistic regression coefficient when regressing health insurance status on income.


%% Additional results
\section{Additional results}
\label{app:additional_results}
\subsection{Experiments with synthetic data}
The complete results for the experiments discussed in \cref{sec:experiments} are presented here. The legend names for the figures are as in \cref{sec:experiments}.

\subsubsection{Biased predictions simulation study}\label{app:simul_bias_supplementary}
\cref{fig:simul_bias_supplementary} shows the average MSE, CI volume, and CI coverage as a function of the bias level $\gamma$ for the biased predictions study in \cref{sec:synthetic_data}.
\begin{figure}[ht!]
    \centering
    \includegraphics[width=\textwidth]{./figures/simul_bias_supplementary.pdf}
    \caption{Full results for the biased predictions study. The left, middle, and right panels show the average MSE, CI volume, and CI coverage as the bias level $\gamma$ varies.}
    \label{fig:simul_bias_supplementary}
\end{figure}
Compared to \cref{fig:simul_bias_width_main}, we include results for the non power-tuned methods, as well as for the ones that take into account the uncertainty on the measure of fit $m_\theta$ (i.e.~PPI (full) and PPI\texttt{++} (full)). In this example, power-tuning does not play a significant role and the same conclusions as in \cref{sec:synthetic_data} hold. In particular, standard PPI induces shorter CIs than classical inference with constant volume across bias levels. On the other hand, FAB methods induce shorter CIs when the predictions are good. As the prediction bias increases, the volume of the FAB CIs with Gaussian prior grows unbounded, while the horseshoe prior eventually reverts to the PPI intervals. Furthermore, the coverage plot shows that the methods tested achieve similar coverage to the nominal level and to PPI (full) and PPI\texttt{++} (full).

\subsubsection{Noisy predictions simulation study}\label{app:simul_noisy_supplementary}
\cref{fig:simul_noisy_supplementary} shows the average MSE, CI volume, and CI coverage as a function of $n$ for the values of $\sigma_Y$ considered in the noisy predictions study of \cref{sec:synthetic_data}.
\begin{figure}[ht!]
    \centering
    \includegraphics[width=\textwidth]{./figures/simul_noisy_supplementary.pdf}
    \caption{Full results for the noisy predictions study. The left, middle, and right panels correspond to noise levels $\sigma_Y = 0.1, 1, 2$, respectively. The top, middle, and bottom rows show average MSE, CI volume, and CI coverage, respectively.}
    \label{fig:simul_noisy_supplementary}
\end{figure}
Compared to \cref{fig:simul_noisy_width_main}, we include results for the methods that use the Gaussian prior (FAB-PPI (N) and FAB-PPI\texttt{++} (N)) and those that take into account the uncertainty on the measure of fit $m_\theta$ (i.e.~PPI (full) and PPI\texttt{++} (full)). Like the CI volume plots in the main text, the MSE plots clearly show the benefits of both power-tuning and adaptive shrinkage through the horseshoe prior: as $\sigma_Y$ increases, the power-tuned methods clearly outperform the standard alternatives, while shrinkage always helps compared to standard PPI because the predictions remain unbiased. In this case, the Gaussian prior performs similarly to the horseshoe as the prediction rule $f$ is unbiased. The coverage plots confirm that all methods achieve comparable coverage across noise levels.

\subsection{Experiments with real data}\label{app:real_supplementary}
\subsubsection{Mean estimation}\label{app:mean_estimation}
\paragraph{Full comparison.}
\cref{fig:real_supplementary} shows the average MSE, CI volume, and CI coverage as a function of $n$ for the three datasets considered in \cref{sec:real_data}.
\begin{figure}[ht!]
    \centering
    \includegraphics[width=\textwidth]{./figures/real_supplementary.pdf}
    \caption{Full results for  mean estimation experiment on real data. The left, middle, and right panels correspond to the \textsc{alphafold}, \textsc{galaxies}, and \textsc{forest} datasets, respectively. The top, middle, and bottom rows show average MSE, CI volume, and CI coverage, respectively over $1000$ repetitions for $\alpha = 0.1$.}
    \label{fig:real_supplementary}
\end{figure}
Compared to \cref{fig:real_main}, we include results for the non power-tuned methods, as well as for the ones that take into account the uncertainty on the measure of fit $m_\theta$ (i.e.~PPI (full) and PPI\texttt{++} (full)). The results are consistent with those presented in \cref{sec:real_data}. In particular, FAB methods outperform the standard PPI alternatives and classical inference, while achieving comparable coverage. For the datasets and the values of $n$ considered, power-tuned methods perform similarly to the non-tuned ones. Among the FAB methods, the horseshoe and Gaussian priors achieve similar performance.

\paragraph{Example intervals.}
\cref{fig:real_intervals_supplementary} shows 10 randomly chosen intervals for the classical, PPI\texttt{++}, and FAB-PPI\texttt{++} methods for the three datasets considered in \cref{sec:real_data} and different choices of the number of labelled observations $n$.
\begin{figure}[ht!]
    \centering
    \includegraphics[width=\textwidth]{./figures/real_intervals_supplementary.pdf}
    \caption{Each subfigure includes 10 randomly chosen intervals for the classical, PPI\texttt{++} and FAB-PPI\texttt{++} methods. The left, middle, and right panels refer to the \textsc{alphafold}, \textsc{galaxies}, and \textsc{forest} datasets, respectively. The top, middle, and bottom rows refer correspond to different values of $n$.}
    \label{fig:real_intervals_supplementary}
\end{figure}

\paragraph{Varying the prior scale.}
We repeat the mean estimation experiment on the \textsc{forest} dataset while varying the scale of the horseshoe prior used for FAB-PPI\texttt{++} in \cref{app:mean_estimation}. In addition to the scale $\widehat{\sigma}$ used in the main text, we consider the sample independent scale $1 / \sqrt{n}$ and the data independent scale $1$. As already mentioned, the computation of the FAB-PPI confidence regions under a horseshoe prior with scale other than $\widehat{\sigma}$ involves numerical integration to compute the corresponding marginal likelihood.
\Cref{fig:real_forest_scaled_supplementary} shows the average MSE, CI volume, and CI coverage for each of these choices, as well as for classical inference and PPI\texttt{++}.
\begin{figure}[ht!]
    \centering
    \includegraphics[width=\textwidth]{./figures/real_forest_scaled_supplementary.pdf}
    \caption{Mean estimation experiment on the \textsc{forest} dataset with varying horseshoe prior scale. The left, middle, and right panels show average MSE, CI volume, and CI coverage over $100$ repetitions for $\alpha = 0.1$.}
    \label{fig:real_forest_scaled_supplementary}
\end{figure}
While the scale $\widehat{\sigma}$ achieves the best performance, the other scales also provide shorter CIs than classical inference and PPI\texttt{++}. In particular, the sample independent scale $1 / \sqrt{n}$ results in good performance across all metrics without requiring the estimation of $\widehat{\sigma}$.

\subsubsection{Logistic regression}\label{app:logistic}
\cref{fig:real_logistic_supplementary} shows the average MSE, CI volume, and CI coverage as a function of $n$ for the logistic regression experiment on the \textsc{healthcare} dataset mentioned in \cref{sec:real_data}.
\begin{figure}[ht!]
    \centering
    \includegraphics[width=\textwidth]{./figures/real_logistic_supplementary.pdf}
    \caption{Logistic regression experiment on the \textsc{healthcare} dataset. The left, middle, and right panels show average MSE, CI volume, and CI coverage over $1000$ repetitions for $\alpha = 0.1$.}
    \label{fig:real_logistic_supplementary}
\end{figure}
As mentioned in the main text, FAB methods outperform the standard PPI alternatives and classical inference, while achieving comparable coverage. Among the FAB methods, the horseshoe and Gaussian priors achieve similar performance.

\subsubsection{Quantile estimation}\label{app:quantile}
\cref{fig:real_quantile_supplementary} shows the average MSE, CI volume, and CI coverage as a function of $n$ for the quantile estimation experiment on the \textsc{genes} dataset mentioned in \cref{sec:real_data}.
\begin{figure}[ht!]
    \centering
    \includegraphics[width=\textwidth]{./figures/real_quantile_supplementary.pdf}
    \caption{Quantile estimation experiment on the \textsc{genes} dataset. The left, middle, and right panels show average MSE, CI volume, and CI coverage over $100$ repetitions for $\alpha = 0.1$.}
    \label{fig:real_quantile_supplementary}
\end{figure}
The predictions contained in this dataset are highly biased and this is reflected in the performance of the FAB-PPI methods. In particular, the Gaussian prior underperforms both classical inference and standard PPI, while the horseshoe prior achieves similar performance to standard PPI thanks to its robustness against large bias levels.

\subsubsection{Linear regression}\label{app:ols}
\cref{fig:real_quantile_supplementary} shows the average MSE, CI volume, and CI coverage as a function of $n$ for the linear regression experiment on the \textsc{census} dataset mentioned in \cref{sec:real_data}.
\begin{figure}[ht!]
    \centering
    \includegraphics[width=\textwidth]{./figures/real_ols_supplementary.pdf}
    \caption{Linear regression experiment on the \textsc{census} dataset. The (a) and (b) panels correspond to the two covariates in the dataset. The left, middle, and right panels show average MSE, CI volume, and CI coverage over $1000$ repetitions for $\alpha = 0.1$.}
    \label{fig:real_ols_supplementary}
\end{figure}
More specifically, panel (a) and (b) corresponds to the OLS parameters associated with the \textit{age} and \textit{sex} covariates, respectively. On the one hand, FAB-PPI seems to perform well for the \textit{sex} covariate, with similar performance between the Gaussian and horseshoe priors, and slightly improved MSE and CI volume compared to classical inference and standard PPI. On the other hand, the performance of FAB-PPI for the \textit{age} covariate seems to be affected by bias in the dataset predictions. In particular, FAB-PPI under the Gaussian prior underperforms the alternatives for all $n$. On the other hand, while the horseshoe prior achieves worse performance than the other methods for small $n$, its performance improve as $n$ grows, and it eventually matches standard PPI. This suggests that, as $n$ increases and $\var(\widehat{\Delta}_\theta)$ decreases, the observed value of the rectifier is increasingly considered as extreme, causing the influence from the horseshoe prior to eventually vanish thanks to its robustness to extreme bias levels.


\end{document}


% This document was modified from the file originally made available by
% Pat Langley and Andrea Danyluk for ICML-2K. This version was created
% by Iain Murray in 2018, and modified by Alexandre Bouchard in
% 2019 and 2021 and by Csaba Szepesvari, Gang Niu and Sivan Sabato in 2022.
% Modified again in 2023 and 2024 by Sivan Sabato and Jonathan Scarlett.
% Previous contributors include Dan Roy, Lise Getoor and Tobias
% Scheffer, which was slightly modified from the 2010 version by
% Thorsten Joachims & Johannes Fuernkranz, slightly modified from the
% 2009 version by Kiri Wagstaff and Sam Roweis's 2008 version, which is
% slightly modified from Prasad Tadepalli's 2007 version which is a
% lightly changed version of the previous year's version by Andrew
% Moore, which was in turn edited from those of Kristian Kersting and
% Codrina Lauth. Alex Smola contributed to the algorithmic style files.
