Here we extend FAB-PPI to the multivariate case, where $\theta,m_\theta,\Delta_\theta \in \bbR^d$. While most of the methodology remains the same as in the univariate case, we now need to specify a multivariate prior for $\Delta_\theta$, for which we consider independent horseshoe priors on each dimension.

\subsection{Multivariate Bayesian PPI Estimators}
As in the univariate case, we use the sample mean $\widehat{m}_\theta$ as the estimator of $m_\theta$.
Similarly, we consider some consistent estimator $\widehat \Delta_\theta$ of $\Delta_\theta$, such as the sample mean \eqref{eq:deltasamplemean}, as in PPI, or the control variate estimator \eqref{eq:deltahatpowertuning}, as in PPI\texttt{++}.
Crucially, we assume that a multivariate CLT holds for this estimator, that is
\begin{equation*}
    \widehat\Sigma_\theta^{-1/2} \left(\widehat\Delta_\theta -\Delta_\theta\right)\to \Normal(0, \mathrm{I})
\end{equation*}
as $\min(n,N)\to\infty$, where $\widehat\Sigma_\theta$ is an estimator of $\cov(\widehat\Delta_\theta)$. Again, this holds for both the PPI and PPI\texttt{++} estimators~\citep{Angelopoulos2023,Angelopoulos2023a}.
We consider $d$ independent priors, $\pi_0(\Delta_{\theta,k};\widehat\sigma_{\theta,k})$ for $k = 1,\dots,d$, on the components of $\Delta_\theta$, where $\widehat\sigma_{\theta,k}^2$ is the $k$-th diagonal element of $\widehat\Sigma_\theta$. The multivariate FAB-PPI estimator $\widehat\Delta_{\theta}^{\FABPPI}$ is formed by stacking the individual estimators
\begin{align*}
    \widehat\Delta_{\theta,k}^{\FABPPI}= \widehat\Delta_{\theta,k} + \widehat\sigma_{\theta,k}^2 \ell'\left(\widehat\Delta_{\theta, k};\widehat\sigma_{\theta,k}\right)
\end{align*}
for each dimension $k = 1,\dots,d$. Importantly, note that the $k$-th dimension of $\widehat\Delta_{\theta}^{\FABPPI}$ only depends on the $k$-th dimension of the observed $(\mathcal{L}_\theta'(X_i, Y_i) - \mathcal{L}_\theta'(X_i, f(X_i)))$ that are used to estimate $\Delta_{\theta,k}$.
The FAB-PPI estimator of $\theta^\star$ then becomes the solution, in $\theta$, to the equation
$$
\widehat m_\theta + \widehat\Delta_\theta^{\FABPPI} = \mathbf{0} \in \mathbb{R}^d.
$$

\subsection{Multivariate FAB-PPI Confidence Regions}
As in the univariate case, let $\calT_{\alpha-\delta}(\widehat m_\theta)$ denote a standard $1-(\alpha-\delta)$ confidence interval for $m_\theta$.
For $\Delta_\theta$, we apply the FAB framework with independent horseshoe priors to each dimension $\Delta_{\theta,k}$ and use a union bound to obtain a $1-\delta$ confidence region for $\Delta_\theta$. In particular, let $\calR^{\FABPPI}_{\delta/d}(\widehat\Delta_{\theta,k},\widehat\sigma_{\theta,k})=\text{FAB-CR}(\widehat\Delta_{\theta,k};\pi_0(\cdot~;\widehat\sigma_{\theta,k}),\widehat\sigma_{\theta, k}, \delta/d)$ be a $1 - \delta/d$ FAB confidence region for $\Delta_{\theta,k}$ under the horseshoe prior $\pi_{0}(\cdot~;\widehat\sigma_{\theta,k})$. Then,
\begin{equation*}
    \calR^{\FABPPI}_{\delta}(\widehat\Delta_{\theta},\widehat\sigma_{\theta}) = \left\{\Delta_\theta \mid \Delta_{\theta,k} \in \calR^{\FABPPI}_{\delta/d}(\widehat\Delta_{\theta,k},\widehat\sigma_{\theta,k}),\ k = 1,\dots,d\right\}
\end{equation*}
where $\widehat\Delta_{\theta}=(\widehat\Delta_{\theta,1},\ldots,\widehat\Delta_{\theta,d})$, $\widehat\sigma_{\theta}=(\widehat\sigma_{\theta,1},\ldots,\widehat\sigma_{\theta,d})$,
is a $1-\delta$ multivariate FAB confidence region for $\Delta_\theta$ by a union bound. With this, the multivariate FAB-PPI confidence region $\calC_\alpha^\FABPPI$ is given by
\begin{align*}
    \calC_\alpha^\FABPPI=\left\{\theta \mid \mathbf{0} \in \calR^{\FABPPI}_\delta(\widehat\Delta_\theta,\widehat\sigma_{\theta}) + \calT_{\alpha-\delta}(\widehat m_\theta, \widehat\sigma^f_{\theta})\right\},
\end{align*}
exactly as in the univariate case. Moreover, also multivariate FAB-PPI enjoys exact asymptotic coverage as $\min(n,N) \to \infty$. In particular, \cref{thm:fabppi_coverage} can be easily extended to the multivariate case by applying a union bound over the dimensions of $\Delta_\theta$.
