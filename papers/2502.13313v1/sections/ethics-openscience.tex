\section{Ethics considerations}
\label{sec:ethics}

This study utilized publicly available datasets~\cite{shi-etal-2022-selective, pii}, some of which included identifiable information such as personal details. However, third-party organizations pre-processed and validated the datasets to ensure that no real individuals’ data were present, thus mitigating potential privacy concerns.

This project received ethical clearance from the Ethical Review Board of the affiliated institution on October 21, 2024 (Approval No. 24-09-4), with no ethical concerns raised.


\section{Open science}
\label{sec:openscience}
This work promotes transparency and reproducibility in research on privacy and utility in large language models (LLMs). To enable further investigation, we will release:
\begin{itemize}
    \item[1.] Code and Framework: The implementation of our proposed privacy measurement framework, which distinguishes between sensitive and non-sensitive tokens, along with scripts for privacy leakage analysis and utility-efficiency evaluation.
    \item[2.] Datasets and Preprocessing information: Links to publicly available datasets used in this study, along with preprocessing scripts to ensure reproducibility. Sensitive data were excluded or anonymized to comply with ethical standards.
    \item [3.] Evaluation Pipeline: An open-source pipeline for assessing privacy leakage and the trade-offs between privacy, utility, and efficiency in LLMs. 
\end{itemize}

These resources aim to support reproducibility and further research into privacy-aware, efficient LLM development.