%%%%%%%%%%%%%%%%%%%%%%%%%%%%%%%%%%%%%%%%%%%%%%%%%%%%%%%%%%%%%%%%%%%%%%%%%%%%%%%%
% Template for USENIX papers.
%
% History:
%
% - TEMPLATE for Usenix papers, specifically to meet requirements of
%   USENIX '05. originally a template for producing IEEE-format
%   articles using LaTeX. written by Matthew Ward, CS Department,
%   Worcester Polytechnic Institute. adapted by David Beazley for his
%   excellent SWIG paper in Proceedings, Tcl 96. turned into a
%   smartass generic template by De Clarke, with thanks to both the
%   above pioneers. Use at your own risk. Complaints to /dev/null.
%   Make it two column with no page numbering, default is 10 point.
%
% - Munged by Fred Douglis <douglis@research.att.com> 10/97 to
%   separate the .sty file from the LaTeX source template, so that
%   people can more easily include the .sty file into an existing
%   document. Also changed to more closely follow the style guidelines
%   as represented by the Word sample file.
%
% - Note that since 2010, USENIX does not require endnotes. If you
%   want foot of page notes, don't include the endnotes package in the
%   usepackage command, below.
% - This version uses the latex2e styles, not the very ancient 2.09
%   stuff.
%
% - Updated July 2018: Text block size changed from 6.5" to 7"
%
% - Updated Dec 2018 for ATC'19:
%
%   * Revised text to pass HotCRP's auto-formatting check, with
%     hotcrp.settings.submission_form.body_font_size=10pt, and
%     hotcrp.settings.submission_form.line_height=12pt
%
%   * Switched from \endnote-s to \footnote-s to match Usenix's policy.
%
%   * \section* => \begin{abstract} ... \end{abstract}
%
%   * Make template self-contained in terms of bibtex entires, to allow
%     this file to be compiled. (And changing refs style to 'plain'.)
%
%   * Make template self-contained in terms of figures, to
%     allow this file to be compiled. 
%
%   * Added packages for hyperref, embedding fonts, and improving
%     appearance.
%   
%   * Removed outdated text.
%
%%%%%%%%%%%%%%%%%%%%%%%%%%%%%%%%%%%%%%%%%%%%%%%%%%%%%%%%%%%%%%%%%%%%%%%%%%%%%%%%

\documentclass[letterpaper,twocolumn,10pt]{article}
%\usepackage{usenix2019_v3}
\usepackage{usenix-2020-09}

% to be able to draw some self-contained figs
\usepackage{tikz}
\usepackage{amsmath}

% inlined bib file
\usepackage{filecontents}

%-------------------------------------------------------------------------------
% \begin{filecontents}{\jobname.bib}
% %-------------------------------------------------------------------------------
% @Book{arpachiDusseau18:osbook,
%   author =       {Arpaci-Dusseau, Remzi H. and Arpaci-Dusseau Andrea C.},
%   title =        {Operating Systems: Three Easy Pieces},
%   publisher =    {Arpaci-Dusseau Books, LLC},
%   year =         2015,
%   edition =      {1.00},
%   note =         {\url{http://pages.cs.wisc.edu/~remzi/OSTEP/}}
% }
% @InProceedings{waldspurger02,
%   author =       {Waldspurger, Carl A.},
%   title =        {Memory resource management in {VMware ESX} server},
%   booktitle =    {USENIX Symposium on Operating System Design and
%                   Implementation (OSDI)},
%   year =         2002,
%   pages =        {181--194},
%   note =         {\url{https://www.usenix.org/legacy/event/osdi02/tech/waldspurger/waldspurger.pdf}}}
% \end{filecontents}

\usepackage{hyperref}
\usepackage{url}
\usepackage{graphicx}
\usepackage{subcaption}
\usepackage{caption}
\usepackage{algorithm, algpseudocode}
\usepackage{amsmath}
\usepackage{amsfonts}
\usepackage{multirow}
\usepackage{color,array, soul}
\usepackage{xcolor}
\usepackage{mdframed}

\usepackage{dsfont}
\usepackage{colortbl}

\usepackage{booktabs}
% \usepackage{fontspec} % Enables UTF-8 support with custom fonts
\usepackage{placeins}
\usepackage{enumitem}
\usepackage{listings}
\usepackage[utf8]{inputenc}
\lstset{
    breaklines=true,
    breakatwhitespace=false,
    columns=fullflexible,
    basicstyle=\ttfamily,
    literate={% Fix for special characters
        {ÿ}{{y}}1            % Replace ÿ with y (or any placeholder you prefer)
        {€}{{EUR}}1          % Replace € with EUR or any suitable placeholder
        {�}{{y}}1
        {©}{{cc}}1
        {Ö}{{o}}1
        {™}{{TM}}1
        {®}{R}1
        {“}{"}1
        {”}{"}1
        {_}{{\_}}1           % Handle underscores
        {^}{{\textasciicircum}}1 % Handle caret if needed
        {[}{{\textbf{[}}}1   % Handle opening bracket
        {]}{{\textbf{]}}}1   % Handle closing bracket
        {&}{{\&}}1           % Handle ampersand
    }
}

\newcommand{\tb}[1]{\textcolor{blue}{#1}}
\newcommand{\tr}[1]{\textcolor{red}{#1}}
\newcommand{\todo}[1]{\textcolor{orange}{TODO:#1}}


\newcommand{\model}{\gM}
\newcommand{\weights}{W}

\definecolor{lavender}{RGB}{230, 230, 250}
\definecolor{gemmagreen}{HTML}{00cc96}
%-------------------------------------------------------------------------------
\begin{document}
%-------------------------------------------------------------------------------

%don't want date printed
\date{}

% make title bold and 14 pt font (Latex default is non-bold, 16 pt)
\title{\Large \bf Revisiting Privacy, Utility, and Efficiency Trade-offs\\when Fine-Tuning Large Language Models}

%for single author (just remove % characters)
\author{
 {\rm Soumi\ Das}\\
MPI--SWS, Germany
\and
{\rm Camila Kolling}\\
MPI--SWS, Germany
% % copy the following lines to add more authors
 \and
 {\rm Mohammad Aflah Khan}\\
MPI--SWS, Germany
\and
 {\rm Mahsa Amani}\\
MPI--SWS, Germany
\and
{\rm Bishwamittra Ghosh}\\
MPI--SWS, Germany
% % copy the following lines to add more authors
 \and
 {\rm Qinyuan Wu}\\
 MPI--SWS, Germany
 \and
  {\rm Till Speicher}\\
MPI--SWS, Germany
\and
{\rm Krishna P. Gummadi}\\
MPI--SWS, Germany
% % copy the following lines to add more authors
} % end author

\maketitle

%-------------------------------------------------------------------------------
\begin{abstract}
%-------------------------------------------------------------------------------
We study the inherent trade-offs in \emph{minimizing privacy risks and maximizing utility, while maintaining high computational efficiency}, when fine-tuning large language models (LLMs).
%
A number of recent works in privacy research have attempted to mitigate privacy risks posed by memorizing fine-tuning data by using differentially private training methods (e.g., DP), albeit at a significantly higher computational cost (inefficiency). 
%
In parallel, several works in systems research have focussed on developing (parameter) efficient fine-tuning methods (e.g., LoRA), but few works, if any, investigated whether such efficient methods enhance or diminish privacy risks.
%
In this paper, we investigate this gap and arrive at a surprising conclusion: efficient fine-tuning methods like LoRA mitigate privacy-risks similar to private fine-tuning methods like DP.  
%
Our empirical finding directly contradicts prevailing wisdom that privacy and efficiency objectives are at odds during fine-tuning.
%
Our finding is established by (a) carefully defining measures of privacy and utility that distinguish between memorizing \emph{sensitive and non-sensitive} tokens in training and test datasets used in fine-tuning and (b) extensive evaluations using multiple open-source language models from Pythia, Gemma, and Llama families and different domain-specific datasets.
%

\end{abstract}

\section{Introduction}


\begin{figure}[t]
\centering
\includegraphics[width=0.6\columnwidth]{figures/evaluation_desiderata_V5.pdf}
\vspace{-0.5cm}
\caption{\systemName is a platform for conducting realistic evaluations of code LLMs, collecting human preferences of coding models with real users, real tasks, and in realistic environments, aimed at addressing the limitations of existing evaluations.
}
\label{fig:motivation}
\end{figure}

\begin{figure*}[t]
\centering
\includegraphics[width=\textwidth]{figures/system_design_v2.png}
\caption{We introduce \systemName, a VSCode extension to collect human preferences of code directly in a developer's IDE. \systemName enables developers to use code completions from various models. The system comprises a) the interface in the user's IDE which presents paired completions to users (left), b) a sampling strategy that picks model pairs to reduce latency (right, top), and c) a prompting scheme that allows diverse LLMs to perform code completions with high fidelity.
Users can select between the top completion (green box) using \texttt{tab} or the bottom completion (blue box) using \texttt{shift+tab}.}
\label{fig:overview}
\end{figure*}

As model capabilities improve, large language models (LLMs) are increasingly integrated into user environments and workflows.
For example, software developers code with AI in integrated developer environments (IDEs)~\citep{peng2023impact}, doctors rely on notes generated through ambient listening~\citep{oberst2024science}, and lawyers consider case evidence identified by electronic discovery systems~\citep{yang2024beyond}.
Increasing deployment of models in productivity tools demands evaluation that more closely reflects real-world circumstances~\citep{hutchinson2022evaluation, saxon2024benchmarks, kapoor2024ai}.
While newer benchmarks and live platforms incorporate human feedback to capture real-world usage, they almost exclusively focus on evaluating LLMs in chat conversations~\citep{zheng2023judging,dubois2023alpacafarm,chiang2024chatbot, kirk2024the}.
Model evaluation must move beyond chat-based interactions and into specialized user environments.



 

In this work, we focus on evaluating LLM-based coding assistants. 
Despite the popularity of these tools---millions of developers use Github Copilot~\citep{Copilot}---existing
evaluations of the coding capabilities of new models exhibit multiple limitations (Figure~\ref{fig:motivation}, bottom).
Traditional ML benchmarks evaluate LLM capabilities by measuring how well a model can complete static, interview-style coding tasks~\citep{chen2021evaluating,austin2021program,jain2024livecodebench, white2024livebench} and lack \emph{real users}. 
User studies recruit real users to evaluate the effectiveness of LLMs as coding assistants, but are often limited to simple programming tasks as opposed to \emph{real tasks}~\citep{vaithilingam2022expectation,ross2023programmer, mozannar2024realhumaneval}.
Recent efforts to collect human feedback such as Chatbot Arena~\citep{chiang2024chatbot} are still removed from a \emph{realistic environment}, resulting in users and data that deviate from typical software development processes.
We introduce \systemName to address these limitations (Figure~\ref{fig:motivation}, top), and we describe our three main contributions below.


\textbf{We deploy \systemName in-the-wild to collect human preferences on code.} 
\systemName is a Visual Studio Code extension, collecting preferences directly in a developer's IDE within their actual workflow (Figure~\ref{fig:overview}).
\systemName provides developers with code completions, akin to the type of support provided by Github Copilot~\citep{Copilot}. 
Over the past 3 months, \systemName has served over~\completions suggestions from 10 state-of-the-art LLMs, 
gathering \sampleCount~votes from \userCount~users.
To collect user preferences,
\systemName presents a novel interface that shows users paired code completions from two different LLMs, which are determined based on a sampling strategy that aims to 
mitigate latency while preserving coverage across model comparisons.
Additionally, we devise a prompting scheme that allows a diverse set of models to perform code completions with high fidelity.
See Section~\ref{sec:system} and Section~\ref{sec:deployment} for details about system design and deployment respectively.



\textbf{We construct a leaderboard of user preferences and find notable differences from existing static benchmarks and human preference leaderboards.}
In general, we observe that smaller models seem to overperform in static benchmarks compared to our leaderboard, while performance among larger models is mixed (Section~\ref{sec:leaderboard_calculation}).
We attribute these differences to the fact that \systemName is exposed to users and tasks that differ drastically from code evaluations in the past. 
Our data spans 103 programming languages and 24 natural languages as well as a variety of real-world applications and code structures, while static benchmarks tend to focus on a specific programming and natural language and task (e.g. coding competition problems).
Additionally, while all of \systemName interactions contain code contexts and the majority involve infilling tasks, a much smaller fraction of Chatbot Arena's coding tasks contain code context, with infilling tasks appearing even more rarely. 
We analyze our data in depth in Section~\ref{subsec:comparison}.



\textbf{We derive new insights into user preferences of code by analyzing \systemName's diverse and distinct data distribution.}
We compare user preferences across different stratifications of input data (e.g., common versus rare languages) and observe which affect observed preferences most (Section~\ref{sec:analysis}).
For example, while user preferences stay relatively consistent across various programming languages, they differ drastically between different task categories (e.g. frontend/backend versus algorithm design).
We also observe variations in user preference due to different features related to code structure 
(e.g., context length and completion patterns).
We open-source \systemName and release a curated subset of code contexts.
Altogether, our results highlight the necessity of model evaluation in realistic and domain-specific settings.





\putsec{related}{Related Work}

\noindent \textbf{Efficient Radiance Field Rendering.}
%
The introduction of Neural Radiance Fields (NeRF)~\cite{mil:sri20} has
generated significant interest in efficient 3D scene representation and
rendering for radiance fields.
%
Over the past years, there has been a large amount of research aimed at
accelerating NeRFs through algorithmic or software
optimizations~\cite{mul:eva22,fri:yu22,che:fun23,sun:sun22}, and the
development of hardware
accelerators~\cite{lee:cho23,li:li23,son:wen23,mub:kan23,fen:liu24}.
%
The state-of-the-art method, 3D Gaussian splatting~\cite{ker:kop23}, has
further fueled interest in accelerating radiance field
rendering~\cite{rad:ste24,lee:lee24,nie:stu24,lee:rho24,ham:mel24} as it
employs rasterization primitives that can be rendered much faster than NeRFs.
%
However, previous research focused on software graphics rendering on
programmable cores or building dedicated hardware accelerators. In contrast,
\name{} investigates the potential of efficient radiance field rendering while
utilizing fixed-function units in graphics hardware.
%
To our knowledge, this is the first work that assesses the performance
implications of rendering Gaussian-based radiance fields on the hardware
graphics pipeline with software and hardware optimizations.

%%%%%%%%%%%%%%%%%%%%%%%%%%%%%%%%%%%%%%%%%%%%%%%%%%%%%%%%%%%%%%%%%%%%%%%%%%
\myparagraph{Enhancing Graphics Rendering Hardware.}
%
The performance advantage of executing graphics rendering on either
programmable shader cores or fixed-function units varies depending on the
rendering methods and hardware designs.
%
Previous studies have explored the performance implication of graphics hardware
design by developing simulation infrastructures for graphics
workloads~\cite{bar:gon06,gub:aam19,tin:sax23,arn:par13}.
%
Additionally, several studies have aimed to improve the performance of
special-purpose hardware such as ray tracing units in graphics
hardware~\cite{cho:now23,liu:cha21} and proposed hardware accelerators for
graphics applications~\cite{lu:hua17,ram:gri09}.
%
In contrast to these works, which primarily evaluate traditional graphics
workloads, our work focuses on improving the performance of volume rendering
workloads, such as Gaussian splatting, which require blending a huge number of
fragments per pixel.

%%%%%%%%%%%%%%%%%%%%%%%%%%%%%%%%%%%%%%%%%%%%%%%%%%%%%%%%%%%%%%%%%%%%%%%%%%
%
In the context of multi-sample anti-aliasing, prior work proposed reducing the
amount of redundant shading by merging fragments from adjacent triangles in a
mesh at the quad granularity~\cite{fat:bou10}.
%
While both our work and quad-fragment merging (QFM)~\cite{fat:bou10} aim to
reduce operations by merging quads, our proposed technique differs from QFM in
many aspects.
%
Our method aims to blend \emph{overlapping primitives} along the depth
direction and applies to quads from any primitive. In contrast, QFM merges quad
fragments from small (e.g., pixel-sized) triangles that \emph{share} an edge
(i.e., \emph{connected}, \emph{non-overlapping} triangles).
%
As such, QFM is not applicable to the scenes consisting of a number of
unconnected transparent triangles, such as those in 3D Gaussian splatting.
%
In addition, our method computes the \emph{exact} color for each pixel by
offloading blending operations from ROPs to shader units, whereas QFM
\emph{approximates} pixel colors by using the color from one triangle when
multiple triangles are merged into a single quad.


\section{Quantifying Privacy and Utility}
\label{sec:quantifying_privacy_utility}
In this section, we introduce the distinction between sensitive and non-sensitive entities when quantifying privacy and utility of an LLM. We conduct case studies to compare our quantification with existing measures. Finally, we demonstrate how the privacy threat is unintentionally exaggerated in existing methods due to the lack of distinction between sensitive and non-sensitive entities.

\subsection{Rethinking Privacy and Utility}

Existing studies at the intersection of privacy and natural language processing \cite{zhao-etal-2022-provably,shi-etal-2022-selective,li2022large,yu2022differentially} seek to enhance privacy while maintaining model utility. Utility is generally assessed based on model performance, such as loss, accuracy, or perplexity \textit{across the entire test dataset}. Privacy is evaluated in terms of performance measures on the \textit{entire training dataset} or theoretical guarantees in differential privacy (DP).


Natural language text may contain both sensitive and non-sensitive words, referred to as entities. For example, sensitive entities include names, addresses, phone numbers, order IDs, and other personally identifiable information. In contrast, non-sensitive entities generally involve semantic and/or syntactic completions following predictable patterns in language generation tasks. Informally, sensitive entities are drawn from a large search space (e.g., \textit{a random sequence of digits}), resulting in high entropy and low predictability. In contrast, non-sensitive entities are more restricted in their occurrences (e.g., \textit{a subject is typically followed by a verb}), leading to low entropy and high predictability. 
Several studies \cite{biderman2024emergent, shi-etal-2022-selective, zhao-etal-2022-provably} distinguish between sensitive and non-sensitive entities in their proposed privacy leakage mitigation methods. However, the distinction is not leveraged in the \textit{quantification of privacy and utility}, which is essential for a granular evaluation as discussed next.

\noindent
\textbf{Quantification of privacy and utility.} In this work, we quantify privacy and utility by accounting for sensitive and non-sensitive entities. Considering a training dataset and a test dataset in a general LLM training pipeline, we quantify \textbf{privacy} as the \textit{recollection of sensitive entities in the training data} and \textbf{utility} as the \textit{prediction of non-sensitive entities in the test data}. Our motivation for the quantification is two-fold: (1) privacy of a model is generally related to training data, while utility is the model's performance on the test data. (2) when quantifying privacy, we deliberately disregard non-sensitive entities, since they are more predictable and not sensitive to a specific person or entity. Similarly, in quantifying utility, we ignore sensitive entities in the test data, since the sensitive entities are rare (and possibly unseen during training), whereas predicting non-sensitive entities indicates the general language understanding ability of LLMs. Next, we provide two pieces of evidence supporting why the distinction is important.

\subsection{Why do we distinguish between sensitive and non-sensitive entities?}
\label{sec:sens_non_sens_distinction}

In this section, we present evidence supporting the importance of distinguishing between sensitive and non-sensitive entities in natural language text while quantifying privacy and utility of an LLM. 

\begin{figure}[t!]
    \centering
        \begin{subfigure}{.48\linewidth}
       \includegraphics[scale=0.25]{figures-paper/section2/train-privacy-customersim--pythia-gpt4.pdf}
        \caption{Privacy measure in Pythia}
        \label{fig:pyt_priv}
    \end{subfigure}
    \hfil
    \begin{subfigure}{.48\linewidth}
        \includegraphics[scale=0.25]{figures-paper/section2/test-utility-customersim--pythia-gpt4.pdf}
        \caption{Utility measure in Pythia}
        \label{fig:pyt_util}
    \end{subfigure}
    
    \begin{subfigure}{.48\linewidth}
       \includegraphics[scale=0.25]{figures-paper/section2/train-privacy-customersim--gemma-gpt4.pdf}
        \caption{Privacy measure in Gemma}
        \label{fig:gemma_priv}
    \end{subfigure}
    \hfil
    \begin{subfigure}{.48\linewidth}
        \includegraphics[scale=0.25]{figures-paper/section2/test-utility-customersim--gemma-gpt4.pdf}
        \caption{Utility measure in Gemma}
        \label{fig:gemma_util}
    \end{subfigure}


    \begin{subfigure}{.48\linewidth}
       \includegraphics[scale=0.25]{figures-paper/section2/train-privacy-customersim--llama2-gpt4.pdf}
        \caption{Privacy measure in Llama2}
        \label{fig:llama_priv}
    \end{subfigure}
    \hfil
    \begin{subfigure}{.48\linewidth}
        \includegraphics[scale=0.25]{figures-paper/section2/test-utility-customersim--llama2-gpt4.pdf}
        \caption{Utility measure in Llama2}
        \label{fig:llama_util}
    \end{subfigure}
    
    \caption{
    Our measures offer a more precise assessment of privacy and utility when fine-tuning LLMs by distinguishing between sensitive and non-sensitive tokens, revealing higher privacy (higher loss) for sensitive tokens and better utility (lower loss) for non-sensitive tokens compared to traditional measures that overlook this sensitivity-based distinction.
    }
    \label{fig:FFT_All}
\end{figure}


\paragraph{\textbf{I. Analyzing privacy and utility while fine-tuning an LLM
}}
%: existing vs.\ our measures}} 
%\tr{provide the intuition}
In order to align with LLM terminology, hereafter, we use tokens to denote entities. Fine-tuning involves iterating an LLM on a specific dataset containing both sensitive and non-sensitive tokens. 
We illustrate how our measure of privacy and utility compares to existing measure in a typical fine-tuning scenario, highlighting a key difference: our approach distinguishes between sensitive and non-sensitive tokens, whereas the existing measure does not.

\textbf{Results.} In Figure~\ref{fig:FFT_All}, we demonstrate measures of privacy (left column) and utility (right column) while fine-tuning three LLM models on Customersim dataset \cite{shi-etal-2022-selective} (experimental details
are provided at the end of this section). In particular, we show training loss on the left column and test loss on the right column. Importantly, we separately compute the loss for both sensitive and non-sensitive tokens in both training and test datasets. Intuitively, a higher loss denotes more privacy and less utility.


\textbf{Privacy is overestimated in the existing measure.} In Figure~\ref{fig:pyt_priv}, we compute privacy using our measure, as well as the existing one. The existing measure of privacy considers \textit{all tokens in the training data}, where low training loss denotes less privacy, while our measure considers \textit{only the sensitive tokens in the training data}. Using our measure, a notable disparity emerges: \textit{sensitive tokens exhibit significantly higher loss than non-sensitive ones}, particularly in the initial training epochs, as sensitive tokens are less predictable. This eventually indicates that the loss over all tokens (existing measure) would be much lower initially than the loss over only sensitive tokens (our measure), thus overestimating privacy threats much earlier. 
%by unintentionally accounting for non-sensitive tokens having little to no privacy threat. 
Similar trends are observed for other models in Figures \ref{fig:gemma_priv} and \ref{fig:llama_priv}.


\textbf{Utility is underestimated in the existing measure.}
In Figure \ref{fig:pyt_util}, we compute utility using our measure and the prevailing one. The existing utility measure is related to the test loss of all tokens, where lower test loss indicates better utility. We can observe that our measure that considers the test loss on only non-sensitive tokens provides better utility than the existing measure. Similar trends are observed for other models in Figures \ref{fig:gemma_util} and \ref{fig:llama_util}. 
%In our measure, we only consider the test loss of non-sensitive tokens, which is even lower than the test loss of all tokens. 
Existing measure would lead to better utility at earlier epochs compared to our measure
%Using our measure, better utility (lower test loss) comes at later epochs compared to existing measures 
i.e. one would select checkpoints at epochs $4$, $2$ and $2$ using existing measures for Pythia, Gemma and Llama2 respectively and at epochs $5$, $6$, and $6$ using our measure. Thus, utility is underestimated in the existing measure. Figure \ref{fig:app_FFT_All_log} in Appendix \ref{appendix:sens_non_sens_distinction} shows the change more distinctly in log-scale. For the same fine-tuning scenario, our measure certifies better utility than the existing measure, which unintentionally considers the loss of sensitive tokens that are rare. The key to our findings is how we annotate sensitive and non-sensitive tokens in real-world datasets. Next, we expand our experimental details leading to our results.


\begin{table}[t!]
\scriptsize
\centering
\caption{Example of samples from datasets with sensitive tokens highlighted.}
\label{tab:snippets}
\begin{tabular}{|ll|}
\hline
\multicolumn{2}{|c|}{\textbf{CustomerSim}}                                                                                                                                                                                                                                                                                                                                                                                                                                                                                                                                                                                                                                                                                                                                                                                                                                                                                                                                                                                                                                                                                                                                                                                                                                                                                                                                                                                                                                                                                                                                                                                                                                                                                                     \\ \hline
\multicolumn{1}{|l|}{\begin{tabular}[c]{@{}l@{}}SYS: Hello, I am the customer \\ support bot. What can \\ I do for you?\\ USR: Hello robot. I ordered a pot \\ several days ago but I can't track it.\\ SYS: Could you verify your full name?\\ USR: \hl{Patrick Schug}\\ SYS: Verify your order number please.\\ USR: It's \hl{843-58572-7002}.\\ SYS: You can track your package \\ with your tracking number \\, which is \hl{AGZIM5T6KL}. \\ Are you happy about my answer?\\ USR: All good. See you.\\ SYS: Have a nice day! Bye.\end{tabular}}                                                                                                                                                                                                                                                                              & \begin{tabular}[c]{@{}l@{}}SYS: Hello, I am the customer \\ support bot. What can \\ I do for you?\\ USR: Hi. Where is my package?\\ SYS: Could you verify \\ your full name?\\ USR: I am \hl{Catherine Pena}.\\ SYS: Verify your phone number.\\ USR: You can reach me at \\ \hl{547.302.3744}.\\ SYS: The tracking number is \\ \hl{VVTPHDB6VK}. \\ Anything else?\\ USR: All good.\end{tabular}                                                                                                                                                                                                                                                                                                                                                                                                                                                                                                      \\ \hline
\multicolumn{2}{|c|}{\textbf{SynBio}}                                                                                                                                                                                                                                                                                                                                                                                                                                                                                                                                                                                                                                                                                                                                                                                                                                                                                                                                                                                                                                                                                                                                                                                                                                                                                                                                                                                                                                                                                                                                                                                                                                                                                                            \\ \hline
\multicolumn{1}{|l|}{\begin{tabular}[c]{@{}l@{}}My name is \hl{Alexander Tanaka}, and I'm a \\ saleswoman with a year of experience. I \\ recently completed a project that \\ involved developing and implementing \\ a new sales strategy  for \\ my company. I started by analyzing our \\ current sales data to identify areas where \\ we could improve...
\end{tabular}} & \begin{tabular}[c]{@{}l@{}}My name is \hl{Phillip Martinez}, and \\ I would like to share some aspects \\ of my life's journey with \\ you. I have had the pleasure of  \\ living in various places throughout \\ my life,  but I currently reside \\ at \hl{4537 Tanglewood Trail}\\
... you can reach me via email at \\ \hl{phillip-martinez@outlook.com}\\ or by phone at \hl{+86 19144 1648}.\end{tabular} \\ \hline
\end{tabular}
\end{table}


\noindent
\textbf{Experimental setup and methodology.} We perform our analysis on two datasets: CustomerSim~\cite{shi-etal-2022-selective}, a simulated dialog dataset for conversation generation and SynBio (originally called PII)~\cite{pii}, an LLM generated dataset representing student biographies containing personal identifiable information. Table \ref{tab:snippets} shows some excerpts from the datasets. We use three open-source models during evaluation: Pythia 1B \cite{biderman2023pythia}, Gemma 2B\cite{team2024gemma}, and Llama2 7B \cite{touvron2023llama}.


We leverage two tools for annotating sensitive information in a given text: Presidio~\cite{MsPresidio}, which helps in identification of private entities in text, and GPT-4 \cite{achiam2023gpt}, which is provided with a particular prompt for returning the annotated portions. An example of such a prompt for annotating samples is provided in Appendix~\ref{appendix:example-priv-annotatation}.


We run two surveys, each among 40 Prolific\footnote{\url{https://www.prolific.com}} users, to gauge the usefulness of the two tools.
We provide the details of the survey in Appendix~\ref{appendix:human-survey-priv-annotatations}.
Figure~\ref{fig:survey_results} shows the results on the CustomerSim dataset which depicts that 75\% participants found GPT4's annotations to be accurate while Presidio annotations were mostly mixed or under-annotated.
\textit{Hence, throughout the rest of the paper we show our results using GPT-4 annotations}, and those using Presidio annotations are shown in Appendix~\ref{appendix:exploring-tradeoffs}.


To summarize, the degree of difference in computing privacy and utility using our measure and the existing one depends on the ratio of sensitive to non-sensitive tokens. A higher ratio would result in a higher difference in the measure, and vice versa. Considering the distinction between sensitive and non-sensitive tokens, we show that the existing measure can both exaggerate privacy threats and underestimate the utility in LLMs. 
In this context, we re-examine a prior study~\cite{biderman2024emergent} to better support our claim that reported privacy threats are exaggerated.

\begin{figure}[!h]
    \centering
     \includegraphics[scale=0.28]{figures-paper/section3/plot1_multiplechoice_customersim.pdf}
    \caption{
    % GPT-4 demonstrates higher accuracy than Presidio for identifying privacy-sensitive information, as rated by human annotators.
    GPT-4 shows higher annotation accuracy, with 75\% of participants rating its annotations to be accurate while Presidio annotations were mostly mixed or under-annotated.
    }
    \label{fig:survey_results}
\end{figure}

\begin{table}[]
\caption{
Examples of memorized sequences from \cite{biderman2024emergent}, often containing predictable and non-sensitive patterns, like mathematical series and licensing text.
}
\label{tab:memseq}
\scriptsize
\centering
\scalebox{0.9}{
\begin{tabular}{p{0.5\columnwidth}|p{0.5\columnwidth}}
    % \multicolumn{2}{c}{\textbf{Memorized Sequence}} \\
    % \hline
    Prompt  & Generation \\
    \toprule
264. 
  265. 
  266. 
  267. 
  268. 
  269. 
  270. 
  271. & 272. 
  273. 
  274. 
  275. 
  276. 
  277. 
  278. 
  279. \\
  \midrule
active.disabled:focus,
.datepicker table tr td.active.disabled:hover:focus,
.datepicker table tr td.active:active, & .datepicker table tr td.active:hover:active,
.datepicker table tr td.active.disabled:active,
.datepicker table tr td \\
\midrule
$\langle$rel=``Chapter" href=``Char.html"$\rangle$$ \langle$link title=``Clflags" rel=``Chapter" href=``Clflags.html"$\rangle$ $\langle$ & link title=``Complex" rel=``Chapter" href=``Complex.html"$\rangle$ $ \langle$link title=``Condition" rel=``Chapter" href=``Condition.html"$\rangle$ \\

\midrule
    amp amp amp amp amp amp amp amp  amp amp amp amp amp amp amp amp & amp amp amp amp amp amp amp amp amp  amp amp amp amp amp amp amp  \\

\midrule
	.word 0
	.word 0
	.word 0
	.word 0
	.word 0
	.word 0 & .word 0
	.word 0
	.word 0
	.word 0
	.word 0
	.word 0
	.word \\
  \bottomrule
\end{tabular}}
\end{table}




 
\paragraph{\textbf{II. Examining memorized sequences from \cite{biderman2024emergent}:}}
We consider a case study to analyze the reported memorized strings by~\cite{biderman2024emergent}, %which are marked as possible privacy threats of the LLM.
Our goal is to examine whether the memorized strings contain sensitive information or mere syntactic and semantic patterns.

\textbf{Experimental setup.} The authors in \cite{biderman2024emergent} considered the task of predicting whether a model memorizes specific training data points from the Pile dataset~\cite{gao2020pile}, which is used to train base LLM models. Among published memorized strings, we randomly choose $5,000$ strings from the \textit{pythia-1b-dup} split~\cite{biderman2024emergent}. A representative list of memorized strings is in Table~\ref{tab:memseq}, where the strings often follow syntactic and semantic patterns, such as completion of mathematical series, code snippets, licensing agreements, etc. Therefore, \textit{our hypothesis is that most of the memorized strings contain a great amount of non-sensitive and highly predictable tokens.} To validate our hypothesis, we query for the source of memorized strings with respect to the training dataset, Pile, which aggregates data from multiple sources such as Pile-Cc, OpenWebText, ArXiv, etc. We leverage GPT-4 model to accomplish our task -- given a memorized string, we ask for the source of the string from the list of Pile sections. The prompt template for GPT-4 is the following.

\noindent
%\begin{quote}
\begin{mdframed}[backgroundcolor=lavender, linewidth=0pt]
\small
\textit{You are provided with the following text: \{\textcolor{blue}{memorized--sequence}\}.\newline
Which section of the Pile dataset does the text belong to? Choose from the list below. You can select 1 or 2 options separated by a comma. Please respond with only the option number.\newline
a. Pile-CC
b. PubMed Central
c. Books3 \\
d. OpenWebText2
e. ArXiv
f. GitHub
g. FreeLaw \\
h. Stack Exchange
i. USPTO Backgrounds \\
j. PubMed Abstracts
k. Gutenberg (PG-19) \\
l. OpenSubtitles
m. Wikipedia (en) \\
n. DM Mathematics
o. Ubuntu IRC \\
p. BookCorpus2
q. EuroParl 
r. HackerNews \\
s. YoutubeSubtitles 
t. PhilPapers \\
u. NIH ExPorter
v. Enron Emails}
\end{mdframed}
%\end{quote}

\begin{figure}[!h]
    \centering
        \subfloat{
       \includegraphics[scale=0.22]{figures-paper/section2/the_pile_dataset_pie_chart.pdf}
        }
        \subfloat{
       \includegraphics[scale=0.22]{figures-paper/section2/predicted_pile_sections_random_5000_gpt-4_v2_avg_pie_chart.pdf}
        }
    \caption{
    Memorized sequences are predominantly sourced from GitHub and ArXiv, despite these sections being mid-range in the original Pile dataset, suggesting that memorized content is largely non-sensitive and may pose a lower privacy risk than previously assumed.
    }
    \label{fig:pile-data}
\end{figure}
 %\vspace{-15pt}


\textbf{Results.} In Figure~\ref{fig:pile-data}, we present two pie charts illustrating the distributions across $22$ distinct sections or data sources within the Pile dataset. The left chart represents the original content distribution of sections within the Pile dataset, while the right chart depicts the distribution of sources of memorized sequences as predicted by GPT-4. 

In the right chart in Figure~\ref{fig:pile-data}, the memorized strings are predicted mostly from GitHub, followed by ArXiv, while the rest of the sources are largely under-represented. Herein, both GitHub and ArXiv are relatively in the middle range in terms of contents in the original dataset in the top chart.  However, analyzing the typical data in these sections, GitHub appears as a source of structured format code with repeated predictable patterns, which is commonly tagged as non-sensitive data. Similarly, the Pile dataset includes {\LaTeX} files uploaded to ArXiv, since {\LaTeX} is a common typesetting language for scientific research papers~\cite{gao2020pile}. As such, highly memorized strings in the Pile dataset are non-sensitive in nature.


\textbf{Validating GPT-4 predictions.} GPT-4 predictions may be erroneous. Hence, we conduct a verification test to evaluate the accuracy of GPT-4's predictions. For this assessment, we sample $200$ random strings from each of the 22 sections of the Pile dataset \cite{gao2020pile}, and prompt GPT-4 to predict the source of the strings. Unlike the previous experiment, \textit{the ground-truth of string source is known in this validation experiment}. Figure~\ref{fig:pile-data-verify} illustrates the accuracy for each section, indicating that $50\%$ of the sections exhibit an accuracy rate of at least $90\%$ with $4.5\%$ being the base accuracy of a random predictor.
Furthermore, GPT-4 predicts the correct source on an average of $78\%$ strings across all $22$ sections of the Pile dataset. In addition, misclassified strings are often assigned to sections of a similar category, e.g., \textit{NIH Explorer misclassified as PubMed Central} (more details in Appendix~\ref{appendix:validating-gpt4-preds}). \textit{Therefore, the GPT-4 predictions can be considered as reliable.}

\begin{figure}[!t]
    \centering
        \includegraphics[scale=0.35]{figures-paper/section2/the_pile_dataset_200_sample_per_section_bar_chart.pdf}
    \caption{
    GPT-4 achieves an average accuracy of 78\% in predicting the source of memorized strings across Pile dataset sections,
    % , with half of the sections reaching at least 90\% accuracy, 
    reinforcing the reliability of GPT-4 and supporting our position that privacy concerns in prior work are overestimated without distinguishing token sensitivity.
    }
    \label{fig:pile-data-verify}
\end{figure}

\begin{figure}[!t]
    \centering
    % \begin{subfigure}{.8\linewidth}
    %    \includegraphics[width=\linewidth,height=4cm]{figures/plot_q1.pdf}
    % \end{subfigure}
    % \begin{subfigure}{.9\linewidth}
       \includegraphics[scale=0.3]{figures-paper/section3/plot2_q1.pdf}
    % \end{subfigure}
    
    \caption{
    Most participants classified the memorized sequences detected by~\cite{biderman2024emergent} as non-sensitive, with fewer than 10\% marking them as privacy-sensitive, indicating that the perceived privacy risk of these strings is generally low.
    }
    \label{fig:human-survey-mem-txt}
\end{figure}


Finally, we conduct a human survey on Prolific to evaluate the extent of sensitive information present in  randomly chosen $100$ memorized strings from \cite{biderman2024emergent}. 
The survey results are summarized in Figure~\ref{fig:human-survey-mem-txt}. The majority of participants classified memorized strings as non-sensitive, while  $< 10\%$ participants disagree and mark the strings as containing privacy-sensitive information.
%Figure~\ref{fig:human-survey-mem-txt} shows the responses by question, illustrating a consistent trend across individual sequences. 
\textit{Therefore, most crowdsourced participants do not perceive the sampled stings as containing privacy-sensitive content.} 
%Overall, the data reveal a strong tendency for non-sensitivity classification across the memorized sequences tested.
Further details on the setup and results are provided in Appendix~\ref{appendix:human-survey-priv-mem-txt}. Thus, by distinguishing between sensitive and non-sensitive entities, we demonstrate a deeper understanding of actual privacy threat.
\section{Privacy-Utility-Efficiency tradeoffs in different fine-tuning methods}
\label{sec:trade_off}
%Mention about the datasets and models used in the experiments.

\begin{figure}[t]
    \centering
    \begin{subfigure}{0.49\linewidth}
    \centering
    \includegraphics[scale=0.22]{figures-paper/section3/tradeoff_pu_customersim-200-gpt4.pdf}
    \caption{CustomerSim}
    \label{fig:ffta}
    \end{subfigure}
    \begin{subfigure}{0.49\linewidth}
    \centering
    \includegraphics[scale=0.22]{figures-paper/section3/tradeoff_pu_pii-8000-gpt4.pdf}
    \caption{SynBio}
    \label{fig:fftb}    
    \end{subfigure}
    %\captionsetup{font=scriptsize}
    \caption{
    Privacy-utility trade-off shows that privacy increases with higher training loss on sensitive tokens, while utility improves with lower test loss on non-sensitive tokens, enabling desired checkpoint selection to balance both objectives.
    }
    \label{fig:fft}
\end{figure}

\begin{figure*}[h!]
    \centering
        \begin{subfigure}{.32\linewidth}
       \includegraphics[scale=0.25,height=3.5cm]{figures-paper/section3/SCIQ_FFT.pdf}
        \caption{SCIQ Benchmark}
        \label{fig:sciq_fft}
    \end{subfigure}
    \begin{subfigure}{.32\linewidth}
       \includegraphics[scale=0.25,height=3.5cm]{figures-paper/section3/MMLU_FFT.pdf}
        \caption{MMLU Benchmark}
        \label{fig:mmlu_fft}
    \end{subfigure}
     \begin{subfigure}{.32\linewidth}
       \includegraphics[scale=0.25,height=3.5cm]{figures-paper/section3/Hellaswag_FFT.pdf}
        \caption{HellaSwag Benchmark}
        \label{fig:hs_fft}
    \end{subfigure}
    \caption{
    % \textbf{Benchmark performance of fully fine-tuned Gemma model}
    Full fine-tuning of the Gemma model leads to a significant drop in accuracy on benchmark datasets, with declines of approximately 75\%, 9\%, and 30\% for SCIQ, MMLU, and HellaSwag, respectively.
    }
    \label{fig:bench-fft}
\end{figure*}

We use the distinction between sensitive and non-sensitive tokens to study the privacy and utility impact of training models with three different fine-tuning methods: full fine-tuning (FFT), Differentially Privacy (DP), and Low-Rank Adaptation (LoRA).
We also investigate the computational efficiency of each method.
Our goal is to answer the following questions: ``\textit{How prone is each method to recollecting the sensitive parts of the training data? (Privacy)'' , ``How effective is each method at predicting non-sensitive parts of test data? (Utility)'' , ``What is the computational cost associated with each method? (Efficiency})''.

To answer these questions, we use each fine-tuning method to train three models from different families, Pythia-1B~\cite{biderman2023pythia}, Gemma-2B~\cite{team2024gemma} and Llama2-7B~\cite{touvron2023llama}, on two datasets, CustomerSim and SynBio.
More information about the datasets and our methodology for distinguishing between sensitive and non-sensitive tokens can be found in Section~\ref{sec:sens_non_sens_distinction}.
We train the models for 50 epochs on each dataset.
Details on hyperparameters can be found in Appendix~\ref{appendix:hp-models}.

For each fine-tuning method, model and dataset we report three metrics: 1) privacy as the loss on sensitive tokens (annotated by GPT-4) in the training data, 2) utility as the loss on non-sensitive tokens (the remaining tokens) on a held-out test set, and 3) efficiency based on the relative amount of computation and memory usage of each method.
Additionally, to assess how fine-tuning affects the underlying abilities and knowledge of the base model, we measure the performance of the fine-tuned Gemma model (trained with CustomerSim data) on general language understanding benchmarks: SCIQ \cite{SciQ}, a dataset of over 13,000 crowdsourced questions on Physics, Chemistry, and Biology; MMLU \cite{hendryckstest2021, hendrycks2021ethics}, a large multi-task dataset covering various domains of knowledge; and Hellaswag \cite{zellers-etal-2019-hellaswag}, a dataset for commonsense inference.

\noindent
\textbf{Update rules:}
For each fine-tuning method, we describe how updated weights $W_{t + 1}$ are computed from the previous weights $W_t$ in each step, where $W_0$ are the weights of the pre-trained base model before fine-tuning.
We use $X$ to refer to a batch of $|X|$ datapoints and $x_i$ to refer to individual datapoint, $\mathcal{M}_{W}$ to refer to the model parameterized by weights $W$, and $\mathcal{L}(\mathcal{M}_{W}(X), X)$ to refer to the autoregressive cross-entropy loss of the model on data $X$.
We denote by $\nabla_{W} \mathcal{L}(...)$ as the gradient of the loss wrt. weights $W$ and $\eta$ is the learning rate.

\noindent
\textbf{Efficiency:}
The efficiency of each method is determined by the amount of computation it requires, and also other factors such as memory requirements, which can affect the usable batch-size and thus the overall training throughput.
Following~\cite{kaplan2020scaling}, we estimate the amount of training compute ($C$) in floating point operations (FLOPs) for full fine-tuning as $C_{\text{FFT}} = 6 D N$, where $D$ is the number of training tokens and $N$, the number of model parameters.
For each method, we report its compute requirements relative to the FFT-baseline based on measurements using the PyTorch profiler\footnote{\url{https://pytorch.org/docs/stable/profiler}}.
We also comment on other factors that affect training throughput.


\subsection{Method 1 : Full fine-tuning}

\textbf{Update rules:}
Full fine-tuning (FFT) updates all model parameters at each step:
\begin{equation}
    W_{t + 1} = W_t - \eta \nabla_{W_t} \mathcal{L}(\mathcal{M}_{W_{t}}(X), X)
\end{equation}



\noindent
\textbf{Privacy-Utility trade-off:}
Figures~\ref{fig:ffta} and~\ref{fig:fftb} show the \textit{privacy-utility trade-off} for the CustomerSim and SynBio datasets, respectively.
In these figures, privacy increases with the training loss on sensitive tokens (\textit{up $\Uparrow$ on the y-axis}), while utility increases when the test loss on non-sensitive tokens decreases (\textit{right $\Longrightarrow$ on the x-axis}).
Each curve starts with the baseline performance of the pre-trained model.
For CustomerSim (Figure~\ref{fig:ffta}), as training advances (\textit{denoted by an arrow $\rightarrow$} on the lines), privacy progressively decreases (\textit{lower on the y-axis}), while utility improves (\textit{rightward on the x-axis}) for approximately the first 5 epochs across all models before stabilizing and eventually declining (\textit{leftward on the x-axis}).
However, for SynBio (Figure \ref{fig:fftb}), the privacy-utility trade-off primarily worsens for Gemma and Llama models. On examining these curves, one can select a desired checkpoint that aligns with specified privacy and utility thresholds.

\noindent
\textbf{Impact on benchmark datasets:}
Figures \ref{fig:sciq_fft}, \ref{fig:mmlu_fft}, and \ref{fig:hs_fft} show the fully fine-tuned Gemma model's accuracy at each epoch for the three benchmarks: SCIQ, MMLU, and Hellaswag, respectively. Note that the accuracy corresponding to the first point represents the performance of the pre-trained model. We observe that full fine-tuning shows a substantial decline in accuracy (around $0.75$, $0.09$, and $0.3$ decrease in accuracy in SCIQ, MMLU, and HellaSwag,  respectively).

\noindent
\textbf{Efficiency:}
FFT serves as our efficiency baseline.
It has moderate compute requirements (discussed above), and relatively high memory requirements, since in addition to the input-dependent activations, we need to keep four numbers per model parameter in GPU memory: the parameter value, its gradient, and two optimizer states (first and second moments of the gradient for Adam \cite{kingma2014adam}).

\noindent
\textbf{Takeaway:}
FFT offers poor privacy-utility trade-offs, since gains in utility in most cases come at the cost of a significant loss in privacy.
During FFT, models learn to both predict the training distribution better, but also quickly learn to recollect sensitive tokens.
In addition, FFT deteriorates the base performance of the model, as can be seen by the rapid decline of the benchmark scores.
FFT is moderately efficient and has relatively high memory requirements. The degree of measures along the \textit{trade-off, knowledge retention, and efficiency} are: 

\indent Utility-privacy trade-offs: \textit{poor} \\
\indent Retention of base performance: \textit{poor} \\
\indent Efficiency: \textit{moderate}


\begin{figure*}[h!]
    \centering
    % First row of subfigures (CustomerSim dataset)
    \begin{subfigure}{0.3\textwidth}
        \centering
        \includegraphics[scale=0.28,height=3.5cm]{figures-paper/section3/tradeoff_dpsgd_pu_sp_customersim-200_pythia-gpt4.pdf}
        \caption{Pythia (CustomerSim)}
        \label{fig:dp_csima}
    \end{subfigure}
    \begin{subfigure}{0.3\textwidth}
        \centering
        \includegraphics[scale=0.28,height=3.5cm]{figures-paper/section3/tradeoff_dpsgd_pu_sp_customersim-200_gemma-gpt4.pdf}
        \caption{Gemma (CustomerSim)}
        \label{fig:dp_csimb}
    \end{subfigure}
    \begin{subfigure}{0.3\textwidth}
        \centering
        \includegraphics[scale=0.28,height=3.5cm]{figures-paper/section3/tradeoff_dpsgd_pu_sp_customersim-200_llama2-gpt4.pdf}
        \caption{Llama2 (CustomerSim)}
        \label{fig:dp_csimc}
    \end{subfigure}

    % Second row of subfigures (SynBio dataset)
    \begin{subfigure}{0.3\textwidth}
        \centering
        \includegraphics[scale=0.28,height=3.5cm]{figures-paper/section3/tradeoff_dpsgd_pu_sp_pii-8000_pythia-gpt4.pdf}
        \caption{Pythia (SynBio)}
        \label{fig:dp_piia}
    \end{subfigure}
    \begin{subfigure}{0.3\textwidth}
        \centering
        \includegraphics[scale=0.28,height=3.5cm]{figures-paper/section3/tradeoff_dpsgd_pu_sp_pii-8000_gemma-gpt4.pdf}
        \caption{Gemma (SynBio)}
        \label{fig:dp_piib}
    \end{subfigure}    
    \begin{subfigure}{0.3\textwidth}
        \centering
        \includegraphics[scale=0.28,height=3.5cm]{figures-paper/section3/tradeoff_dpsgd_pu_sp_pii-8000_llama2-gpt4.pdf}
        \caption{Llama2 (SynBio)}
        \label{fig:dp_piic}
    \end{subfigure}

    \caption{
        Privacy-utility trade-offs for DP across models and datasets. \emph{Top row:} For the \emph{CustomerSim} dataset, DP enhances privacy across all models, though higher noise scales reduce utility, especially for larger models. \emph{Bottom row:} For the \emph{SynBio} dataset, DP similarly improves privacy, with utility declines at higher noise levels.
        Gemma shows a unique utility drop after two epochs, highlighting the complexity of privacy-utility trade-off.
    }
    \label{fig:combined_dp}
\end{figure*}


\subsection{Method 2 : Differential Privacy}

Differential Privacy (DP) algorithms ~\cite{10.1145/2976749.2978318} aim to safeguard the privacy of individual training data points by limiting their influence on the gradient updates during training.

\noindent
\textbf{Update rules:}
DP clips the \textit{$l_2$} norm of each data point’s gradient at a threshold $T$, followed by adding noise with magnitude $\sigma$ to each clipped gradient.
The purpose of clipping is to reduce data sensitivity by ensuring that the impact of data points with high gradient magnitudes on the model parameters is limited.
% The added noise serves to smooth the probability distribution \bgcomment{probability distribution of what?}, preventing any data point’s outcome from being disproportionately likely.
Adding noise further obscures the contribution of individual data points, making it difficult to infer specific data points from the model.

\vspace{-10pt}
\small{
\begin{equation}
%\begin{align}
\begin{split}
    W_{t + 1} &= W_t - \eta \, \text{Noise} \left( \frac{1}{B} \sum_i \text{Clip}\left( \nabla_{W_t} \mathcal{L}(\mathcal{M}_{W_t}(x_i), x_i) \right) \right) \\
    \text{Clip}(y) &= y / \max \left( 1, \frac{\lVert y \rVert_2}{T} \right) \\
    \text{Noise}(y) &= y + \mathcal{N}(0, \sigma^2 T^2 \mathds{1})
\end{split}
%\end{align}
\end{equation}
}
\normalsize
\vspace{-10pt}

We vary the noise hyperparameter $\sigma$ for the experiments.
The clipping gradient norm $T$ is fixed at $10^{-2}$, as in \cite{shi-etal-2022-selective}.

\begin{figure*}[h!]
    \centering
        \begin{subfigure}{.3\linewidth}
       \includegraphics[scale=0.25,height=3.5cm]{figures-paper/section3/SCIQ_DPSGD.pdf}
        \caption{SCIQ Benchmark}
        \label{fig:sciq_dpsgd}
    \end{subfigure}
    \begin{subfigure}{.3\linewidth}
       \includegraphics[scale=0.25,height=3.5cm]{figures-paper/section3/MMLU_DPSGD.pdf}
        \caption{MMLU Benchmark}
        \label{fig:mmlu_dpsgd}
    \end{subfigure}
     \begin{subfigure}{.3\linewidth}
       \includegraphics[scale=0.25,height=3.5cm]{figures-paper/section3/Hellaswag_DPSGD.pdf}
        \caption{HellaSwag Benchmark}
        \label{fig:hs_dpsgd}
    \end{subfigure}
    \caption{
    % Benchmark performance of DP-SGD fine-tuned Gemma model
    Fine-tuning Gemma with DP results in a substantial accuracy declines of 70\%, 10\%, and 30\% across the benchmarks.
    }
    \label{fig:bench-dpsgd}
\end{figure*}

\noindent
\textbf{Privacy-Utility trade-off:}
Figures \ref{fig:dp_csima}, \ref{fig:dp_csimb}, and \ref{fig:dp_csimc} illustrate the privacy-utility trade-off when varying the noise $\sigma$ on the CustomerSim dataset across the Pythia, Gemma, and Llama2 models.
As in previous plots, each curve begins with the performance of the pre-trained model.
It is evident that DP maintains privacy effectively with minimal degradation.
But there is a trade-off between privacy and utility.
For all models, lower noise values such as $\sigma = 0.1$ are able to achieve better utility than higher ones such as $\sigma = 0.5$ and $\sigma = 0.9$, but they also decrease privacy more.
While the total amount of utility achievable with DP is limited, especially for the larger Llama2-7B model, overall, it provides good privacy-utility trade-offs.
Similar trends can be observed for the SynBio dataset in Figures~\ref{fig:dp_piia} and \ref{fig:dp_piic}, but for Figure~\ref{fig:dp_piib} (with Gemma) utility declines after approximately two epochs.  

\noindent
\textbf{Impact on benchmark datasets:} Figures~\ref{fig:sciq_dpsgd}, \ref{fig:mmlu_dpsgd} and \ref{fig:hs_dpsgd} show the benchmark accuracy over epochs for the DP fine-tuned Gemma model with $\sigma=0.1$ on SCIQ, MMLU, and HellaSwag datasets, respectively. Accuracy drops gradually with increasing epochs for all benchmarks, stabilizing at lower levels, indicating that fine-tuning with DP significantly reduces the knowledge retention capacity.

\noindent
\textbf{Efficiency:}
Differential privacy comes with a high computational cost, since the gradients of each datapoint need norm and clipping computations, and additional noise values are added.
Empirically, we observe a relative FLOPs requirement of $C_{\text{DP}} / C_{\text{FFT}} = 1.33$.
Additionally, the per-sample operations required for clipping mean that we need to keep copies of the gradient for each datapoint in memory, which requires substantially more GPU memory, which decreases the feasible batch size and the overall training throughput even further.

\begin{figure*}[t]
    \centering
    % First row of subfigures (LoRA with rank 16 on CustomerSim)
    \begin{subfigure}{0.3\textwidth}
        \centering
        \includegraphics[scale=0.28,height=3.5cm]{figures-paper/section3/tradeoff_lora_pu_sp_rank_16_customersim-200_pythia-gpt4.pdf}
        \caption{Pythia (Rank 16)}
        \label{fig:lora_16_csima}
    \end{subfigure}
    \begin{subfigure}{0.3\textwidth}
        \centering
        \includegraphics[scale=0.28,height=3.5cm]{figures-paper/section3/tradeoff_lora_pu_sp_rank_16_customersim-200_gemma-gpt4.pdf}
        \caption{Gemma (Rank 16)}
        \label{fig:lora_16_csimb}
    \end{subfigure}
    \begin{subfigure}{0.3\textwidth}
        \centering
        \includegraphics[scale=0.28,height=3.5cm]{figures-paper/section3/tradeoff_lora_pu_sp_rank_16_customersim-200_llama2-gpt4.pdf}
        \caption{Llama2 (Rank 16)}
        \label{fig:lora_16_csimc}
    \end{subfigure}

    % Second row of subfigures (LoRA with rank 32 on CustomerSim)
    \begin{subfigure}{0.3\textwidth}
        \centering
        \includegraphics[scale=0.28,height=3.5cm]{figures-paper/section3/tradeoff_lora_pu_sp_rank_32_customersim-200_pythia-gpt4.pdf}
        \caption{Pythia (Rank 32)}
        \label{fig:lora_32_csima}
    \end{subfigure}
    \begin{subfigure}{0.3\textwidth}
        \centering
        \includegraphics[scale=0.28,height=3.5cm]{figures-paper/section3/tradeoff_lora_pu_sp_rank_32_customersim-200_gemma-gpt4.pdf}
        \caption{Gemma (Rank 32)}
        \label{fig:lora_32_csimb}
    \end{subfigure}
    \begin{subfigure}{0.3\textwidth}
        \centering
        \includegraphics[scale=0.28,height=3.5cm]{figures-paper/section3/tradeoff_lora_pu_sp_rank_32_customersim-200_llama2-gpt4.pdf}
        \caption{Llama2 (Rank 32)}
        \label{fig:lora_32_csimc}
    \end{subfigure}

    \caption{
        Privacy-utility trade-offs for LoRA fine-tuning on the \emph{CustomerSim} dataset. \emph{Top row:} LoRA with rank 16 shows that smaller models achieve a better privacy-utility trade-off, while larger models retain utility but experience reduced privacy with more epochs. \emph{Bottom row:} LoRA with rank 32 yields similar results, with smaller models performing better in privacy-utility trade-off and larger models maintaining utility but with privacy reduction as epochs increase.
    }
    \label{fig:combined_lora}
\end{figure*}

\noindent
\textbf{Takeaway:} DP offers a reasonable privacy-utility trade-off and is less susceptible to learning sensitive data. Increasing noise rates lead to poor utility across all models.
Convergence with DP
%\bgcomment{convergence of what?} 
is not guaranteed, particularly in larger models such as Llama2.
% This issue accounts for the lower utility observed in the Llama2 7B models in Figures \ref{fig:dp_csimc} and \ref{fig:dp_piic}.
The privacy gains come at the cost of efficiency.
Additionally, fine-tuning with DP leads to quick decline in the benchmark performance. The degree of measures along the \textit{tradeoff, knowledge retention, and efficiency} are: 

\indent Utility-privacy trade-offs: \textit{moderate} \\
\indent Retention of base performance: \textit{poor} \\
\indent Efficiency: \textit{poor}

\begin{figure*}[ht!]
    \centering
    % First row of subfigures (LoRA with rank 16 on SynBio)
    \begin{subfigure}{0.3\textwidth}
        \centering
        \includegraphics[scale=0.28,height=3.5cm]{figures-paper/section3/tradeoff_lora_pu_sp_rank_16_pii-8000_pythia-gpt4.pdf}
        \caption{Pythia (Rank 16)}
        \label{fig:lora_16_piia}
    \end{subfigure}
    \begin{subfigure}{0.3\textwidth}
        \centering
        \includegraphics[scale=0.28,height=3.5cm]{figures-paper/section3/tradeoff_lora_pu_sp_rank_16_pii-8000_gemma-gpt4.pdf}
        \caption{Gemma (Rank 16)}
        \label{fig:lora_16_piib}
    \end{subfigure}
    \begin{subfigure}{0.3\textwidth}
        \centering
        \includegraphics[scale=0.28,height=3.5cm]{figures-paper/section3/tradeoff_lora_pu_sp_rank_16_pii-8000_llama2-gpt4.pdf}
        \caption{Llama2 (Rank 16)}
        \label{fig:lora_16_piic}
    \end{subfigure}

    % Second row of subfigures (LoRA with rank 32 on SynBio)
    \begin{subfigure}{0.3\textwidth}
        \centering
        \includegraphics[scale=0.28,height=3.5cm]{figures-paper/section3/tradeoff_lora_pu_sp_rank_32_pii-8000_pythia-gpt4.pdf}
        \caption{Pythia (Rank 32)}
        \label{fig:lora_32_piia}
    \end{subfigure}
    \begin{subfigure}{0.3\textwidth}
        \centering
        \includegraphics[scale=0.28,height=3.5cm]{figures-paper/section3/tradeoff_lora_pu_sp_rank_32_pii-8000_gemma-gpt4.pdf}
        \caption{Gemma (Rank 32)}
        \label{fig:lora_32_piib}
    \end{subfigure}    
    \begin{subfigure}{0.3\textwidth}
        \centering
        \includegraphics[scale=0.28,height=3.5cm]{figures-paper/section3/tradeoff_lora_pu_sp_rank_32_pii-8000_llama2-gpt4.pdf}
        \caption{Llama2 (Rank 32)}
        \label{fig:lora_32_piic}
    \end{subfigure}

    \caption{
        Privacy-utility trade-offs for LoRA fine-tuning on the \emph{SynBio} dataset. \emph{Top row:} LoRA with rank 16 shows that smaller models achieve a better privacy-utility trade-off, while larger models retain utility but experience reduced privacy with more epochs. \emph{Bottom row:} LoRA with rank 32 yields similar results, with smaller models performing better in privacy-utility trade-off and larger models maintaining utility but with privacy reduction as epochs increase.
    }
    \label{fig:combined_lora_synbio}
\end{figure*}


\begin{figure*}[ht!]
    \centering
        \begin{subfigure}{.3\linewidth}
       \includegraphics[scale=0.3,height=3.5cm]{figures-paper/section3/SCIQ_LoRA.pdf}
        \caption{SCIQ Benchmark}
        \label{fig:sciq_lora}
    \end{subfigure}
    \begin{subfigure}{.3\linewidth}
       \includegraphics[scale=0.3,height=3.5cm]{figures-paper/section3/MMLU_LoRA.pdf}
        \caption{MMLU Benchmark}
        \label{fig:mmlu_lora}
    \end{subfigure}
     \begin{subfigure}{.3\linewidth}
       \includegraphics[scale=0.3,height=3.5cm]{figures-paper/section3/Hellaswag_LoRA.pdf}
        \caption{HellaSwag Benchmark}
        \label{fig:hs_lora}
    \end{subfigure}
    \caption{
    LoRA fine-tuned model maintains accuracy levels close to the pre-trained model with declines of 5\%, 3\% and 3\% across SCIQ, MMLU, and Hellaswag benchmarks, highlighting its effectiveness in knowledge retention.
    }
    \label{fig:bench-lora}
\end{figure*}


\subsection{Method 3 : Low-Rank Adaptation (LoRA)}

LoRA~\cite{hu2022lora} is a parameter-efficient fine-tuning method developed to reduce the compute and memory requirements of fine-tuning LLMs, and to reduce the size of storing fine-tuned checkpoints.
It enjoys large popularity for LLM fine-tuning.

\textbf{Update rules:}

\begin{equation}
\begin{split}
    W_{t + 1} &= W_0 + \frac{\alpha}{r} \Delta W_{t + 1}
 \\
    \Delta W_{t + 1} &= \Delta W_t - \eta \nabla_{\Delta W_t} \mathcal{L}(\mathcal{M}_{W_t}(X), X)
\end{split}
\label{eq:compute_fft}
\end{equation}



LoRA freezes the weights of the pretrained base model $W_0$ and only fine-tunes an adapter matrix $\Delta W_t$.
During training, $\Delta W_t$ is stored separately from $W_0$ as two low-rank matrices with rank $r$: $\Delta W_t = B_t A_t$ with $B_t \in \mathbb{R}^{d \times r}, A_t \in \mathbb{R}^{r \times k}, W_0 \in \mathbb{R}^{d \times k}$.
$r$ is typically very small, often between $4$ and $32$.
$\alpha$ is a constant that controls how much the LoRA adapters affect the behavior of the base weights $W_0$.

We hypothesize that LoRA's low-rank updates restrict the model's capacity to memorize precise details, which could have an effect similar to the noisy updates in DP.
Therefore, LoRA has the potential to provide better privacy-utility trade-offs than FFT, similar to DP, while also being computationally more efficient.


\noindent
\textbf{Privacy-Utility trade-off:}
We investigate the effects of varying both the rank $r$ and scaling parameter $\alpha$ of LoRA.
We use common rank values of $16$ and $32$ and $\alpha \in \{16,32,64,128\}$.
While LoRA has been explored for privacy in conjunction with DP~\cite{yu2022differentiallyprivatefinetuninglanguage}, there has been no prior work that specifically examines the privacy benefits of LoRA alone. 

Figures~\ref{fig:lora_16_csima}-~\ref{fig:lora_16_csimc} show the trade-off with rank $16$ and varying parameters of $\alpha$ for the CustomerSim dataset across Pythia, Gemma and Llama2 models. As seen in Figures \ref{fig:lora_16_csima} and \ref{fig:lora_16_csimb}, smaller-scale models (Pythia and Gemma) exhibit a better privacy-utility trade-off when the rank and $\alpha$ values are equal, compared to the other configurations. For the larger model, the privacy declines at later epochs, while utility is mostly retained.
This trend is similar to the one observed for rank $32$ in Figures~\ref{fig:lora_32_csima}-~\ref{fig:lora_32_csimc}. 
We also analyze the trade-off for the SynBio dataset in Figures~\ref{fig:lora_16_piia}-~\ref{fig:lora_16_piic} for rank 16, and Figures~\ref{fig:lora_32_piia}-~\ref{fig:lora_32_piic} for rank 32, which make similar observations. 
However, due to the more unstructured nature of the SynBio dataset, there is a larger reduction in utility after certain epochs compared to the CustomerSim dataset.

\noindent
\textbf{Impact on benchmark datasets:} We use the LoRA model with configuration $r=16, \alpha=16$ for this experiment. Figures \ref{fig:sciq_lora}-~\ref{fig:hs_lora} illustrate the LoRA fine-tuned model's accuracy at each epoch for the three benchmarks. 
%Note that the accuracy corresponding to the first point on x-axis represents the performance of the pre-trained model.
The LoRA-based fine-tuned Gemma model retains performance levels close to those of the pre-trained model.

\noindent
\textbf{Efficiency:}
During training, LoRA has slightly larger compute requirements for the forward pass than full fine-tuning, since additional FLOPs are required for the adapter matrices $\Delta W$, though they are much smaller than the full base weight $W_0$.
However, during the backward pass, LoRA requires less compute, since no gradients need to be computed for the base weights $W_0$.
We observe a relative FLOPs requirement of $C_{\text{LoRA}} / C_{\text{FFT}} \approx 0.65$.
The original paper~\cite{hu2022lora} reports a 25\% speedup during training.
LoRA has needs of less GPU memory than FFT, since no optimizer states and gradients need to be stored for the base weights $W_0$, which makes it possible to run it with larger batch-sizes and thus an overall increased training throughput.

\begin{figure*}[t]
    \centering
    % First row of subfigures (CustomerSim dataset)
    \begin{subfigure}{0.3\textwidth}
        \centering
        \includegraphics[scale=0.25]{figures-paper/section4/tradeoff_pu_all_fix_customersim_pythia-gpt4.pdf}
        \caption{Pythia (CustomerSim)}
        \label{fig:fdl_csim1}
    \end{subfigure}
    \begin{subfigure}{0.3\textwidth}
        \centering
        \includegraphics[scale=0.25]{figures-paper/section4/tradeoff_pu_all_fix_customersim_gemma-gpt4.pdf}
        \caption{Gemma (CustomerSim)}
        \label{fig:fdl_csim2}
    \end{subfigure}
    \begin{subfigure}{0.3\textwidth}
        \centering
        \includegraphics[scale=0.25]{figures-paper/section4/tradeoff_pu_all_fix_customersim_llama2-gpt4.pdf}
        \caption{Llama2 (CustomerSim)}
        \label{fig:fdl_csim3}
    \end{subfigure}

    % Second row of subfigures (SynBio dataset)
    \begin{subfigure}{0.3\textwidth}
        \centering
        \includegraphics[scale=0.25]{figures-paper/section4/tradeoff_pu_all_fix_pii_pythia-gpt4.pdf}
        \caption{Pythia (SynBio)}
        \label{fig:fdl_pii1}
    \end{subfigure}
    \begin{subfigure}{0.3\textwidth}
        \centering
        \includegraphics[scale=0.25]{figures-paper/section4/tradeoff_pu_all_fix_pii_gemma-gpt4.pdf}
        \caption{Gemma (SynBio)}
        \label{fig:fdl_pii2}
    \end{subfigure}
    \begin{subfigure}{0.3\textwidth}
        \centering
        \includegraphics[scale=0.25]{figures-paper/section4/tradeoff_pu_all_fix_pii_llama2-gpt4.pdf}
        \caption{Llama2 (SynBio)}
        \label{fig:fdl_pii3}
    \end{subfigure}

    \caption{
        Privacy-utility tradeoffs for different fine-tuning methods on the \emph{CustomerSim} and \emph{SynBio} datasets. \emph{Top row:} On the CustomerSim dataset, DP achieves high privacy with a significant computational cost, full fine-tuning maximizes utility but offers less privacy, and LoRA provides a balanced trade-off with the lowest computational overhead. \emph{Bottom row:} Similar trends are observed for the SynBio dataset.
        % , with DP-SGD excelling in privacy at high computational cost, full fine-tuning performing best in utility, and LoRA balancing privacy, utility, and efficiency.
    }
    \label{fig:combined_fdl}
\end{figure*}

\noindent
\textbf{Takeaway:}
We are one of the first works to explore the privacy benefits of parameter-efficient fine-tuning methods, particularly LoRA.
We vary LoRA's $r$ and $\alpha$ hyperparameters and observe that all configurations are able to achieve high utility, while especially lower $r$ and $\alpha$ values also preserve a high degree of privacy.
For each model, the optimal privacy-utility trade-off value is achieved with $r = \alpha$.
% Higher $\alpha$ values lead to marginally better utilities than the lower ones. However, for each model, we observe that the optimal privacy-utility trade-off value is achievable when $r=\alpha$. It is also worth noting that LoRA is the most cost-effective approach among all the other fine-tuning methods. 
In addition, we observe that LoRA, after being fine-tuned on the CustomerSim dataset, did not lose much of its abilities on the benchmark datasets, maintaining results comparable to those of pre-trained model.
Finally, LoRA is much more computationally and memory efficient than FFT and especially DP.
Overall it provides the best trade-offs in terms of utility, privacy and efficiency and shows that privacy can be achieved without additional computational costs. The degree of measures along the \textit{trade-off, knowledge retention, and efficiency} are: 

\indent Utility-privacy trade-offs: \textit{good} \\
\indent Retention of base performance: \textit{moderate} \\
\indent Efficiency: \textit{good}
\section{Comparison of Fine-tuning Methods}
\label{sec:comparison}

In this section, we compare all fine-tuning methods across three key aspects: privacy, utility, and efficiency.
Recall that privacy is measured as training loss on sensitive tokens and utility is measured as test loss on non-sensitive tokens.
For efficiency, we measure floating point operations (FLOPs) based on the number of operations incurred (e.g., matrix multiplication, addition, etc.) during training.
% We report this metric for one instance as it scales linearly.
In this section, for DP we use a noise ratio of $\sigma = 0.1$ and for LoRA we use $r=16,\alpha=16 $. Figures~\ref{fig:fdl_csim1}-\ref{fig:fdl_csim3} present the Pareto plots for CustomerSim, comparing the three fine-tuning strategies across three dimensions: privacy, utility, and efficiency. 

\noindent
\textbf{Privacy}: Regarding \textit{privacy}, as previously noted, full fine-tuning shows poor privacy performance over extended training, while DP achieves the highest privacy levels.
LoRA provides a similar degree of privacy compared to DP throughout most training epochs for small-scale models (Pythia and Gemma), but declines with the large-scale model (Llama2).

\noindent
\textbf{Utility:} In terms of \textit{utility}, full fine-tuning maintains relatively strong performance throughout most of its training (\textit{right on the x-axis}).
DP while yielding good utility in smaller models (Pythia and Gemma), performs poorly in the larger model (Llama2) and overall needs more epochs to reach higher utility levels.
In contrast, LoRA consistently preserves higher utility across the entire training period.


\noindent
\textbf{Efficiency}: The color bar in Figure~\ref{fig:combined_fdl} highlights the FLOPs intensity associated with each model across all fine-tuning strategies.
% For CustomerSim (Figures~\ref{fig:fdl_csim1}, \ref{fig:fdl_csim2} and \ref{fig:fdl_csim3}), in terms of \textit{efficiency},
DP requires the \textit{highest number of FLOPs} (in \textcolor{red}{red}) due to the need for per-sample gradient computation, where each sample corresponds to a token within each training sequence.
FFT demands a \textit{moderate number of FLOPs} (in \textcolor{blue}{blue}), proportional to the total number of parameters.
Finally, LoRA requires the \textit{fewest FLOPs} (in \textcolor{green}{green}), as especially during backpropagation, most operations only operate on the low-rank matrices.
% Similar patterns are observed for the SynBio dataset in Figures~\ref{fig:fdl_pii1}, \ref{fig:fdl_pii2} and \ref{fig:fdl_pii3}). 

\begin{figure*}[!t]
    \centering
        \begin{subfigure}{.32\linewidth}
       \includegraphics[scale=0.25]{figures-paper/section4/SCIQ.pdf}
        \caption{Benchmark SCIQ}
        \label{fig:sciq_all}
    \end{subfigure}
    \begin{subfigure}{.32\linewidth}
       \includegraphics[scale=0.25]{figures-paper/section4/MMLU.pdf}
        \caption{Benchmark MMLU}
        \label{fig:mmlu_all}
    \end{subfigure}
     \begin{subfigure}{.32\linewidth}
       \includegraphics[scale=0.25]{figures-paper/section4/Hellaswag.pdf}
        \caption{Benchmark HellaSwag}
        \label{fig:hs_all}
    \end{subfigure}
    \caption{
    % Comparison of all methods on the benchmark datasets.   
    LoRA demonstrates superior knowledge retention on benchmark datasets throughout the training regime on the CustomerSim dataset, while FFT and DP show fragility with sharp and gradual performance declines, respectively.
    }
    \label{fig:bench-all}
\end{figure*}

\noindent
\textbf{Benchmark performance:} Figure~\ref{fig:bench-all} shows strong knowledge retention capabilities of LoRA on the three benchmarks after being fine-tuned on the CustomerSim dataset for 50 epochs.
FFT and DP, on the other hand, decline sharply and gradually, respectively from the pretrained base model performance.
% The sharp and gradual declines in the performance of FFT and DP-SGD respectively hint at fragility.
%\sdcomment{add here}

\noindent
Assessing all the above aspects, we can observe the following: 

1. \textit{Full fine-tuning} achieves \textit{high utility initially, but starts diminishing} after a few epochs, and also witnesses a significant \textit{drop in privacy}.
It has a relatively \textit{high computational cost}.
% due to the large number of FLOPs associated proportional to the number of model parameters.
Additionally, the fully fine-tuned model's performance \textit{diminishes significantly on benchmark datasets}.

2. \textit{DP} offers the \textit{strongest privacy protection} and achieves a \textit{reasonable privacy-utility tradeoff in smaller models}.
However, this tradeoff deteriorates in larger models.
% due to the extensive number of noisy gradient updates.
DP incurs the \textit{highest computational cost} as its per-sample noisy gradient updates significantly increase FLOPs and also memory requirements.
Additionally, models fine-tuned with DP exhibit a \textit{gradual decline in their benchmark performance} over the course of training.

3. \textit{LoRA}, a parameter-efficient fine-tuning method, maintains \textit{high utility} and achieves \textit{privacy levels comparable to DP} in smaller models, though this advantage reduces in larger ones. 
Figure~\ref{fig:fdl_csim3} shows that while LoRA preserves less privacy as training progresses, it is possible to select checkpoints that balance strong privacy with utility.
This finding challenges the prevailing notion that privacy must come at the cost of high efficiency, demonstrating that \textit{\textbf{LoRA can offer privacy benefits}}.
Moreover, LoRA-tuned models \textit{retain performance on benchmark datasets} close to that of pre-trained models throughout training. Figure \ref{tab:overall} in Appendix \ref{appendix:comparison} shows a comprehensive comparison across all the fine-tuning methods. 
%\input{sections/exp-generalizability}
\section{Conclusions}

We provided deterministic distributed algorithms to efficiently simulate a round of algorithms designed for the CONGEST model on the Beeping Networks. This allowed us to improve polynomially the time complexity of several (also graph) problems on Beeping  Networks. The first simulation by the Local Broadcast algorithm is shorter by a polylogarithmic factor than the other, more general one -- yet still powerful enough to implement some algorithms, including the prominent solution to Network Decomposition~\cite{ghaffari2021improved}.
The more general one could be used for solving problems such as MIS.
We also considered efficient pipelining of messages via several layers of BN.
%We also proved that our solutions could not be substantially improved if the considered problems require content-oblivious local broadcast, by proving an almost-tight lower bound.

Two important lines of research arise from our work.
First, whether some (graph) problems do not need local broadcast to be solved deterministically, and whether their time complexity could be asymptotically below $\Delta^2$.
Second, could a lower bound on any deterministic local broadcast algorithm, better than $\Omega(\Delta\log n)$, be proved?
%our lower bound be tightened and extended to any, not necessarily content-oblivious \mam{and non-adaptive}, solutions to the Local Broadcast problem?

% \todo{Propose to develop algorithms that work in time depending on the diameter of the network}

% \todo{Discussion of noisy beeping channel.}
\section{Ethics considerations}
\label{sec:ethics}

This study utilized publicly available datasets~\cite{shi-etal-2022-selective, pii}, some of which included identifiable information such as personal details. However, third-party organizations pre-processed and validated the datasets to ensure that no real individuals’ data were present, thus mitigating potential privacy concerns.

This project received ethical clearance from the Ethical Review Board of the affiliated institution on October 21, 2024 (Approval No. 24-09-4), with no ethical concerns raised.


\section{Open science}
\label{sec:openscience}
This work promotes transparency and reproducibility in research on privacy and utility in large language models (LLMs). To enable further investigation, we will release:
\begin{itemize}
    \item[1.] Code and Framework: The implementation of our proposed privacy measurement framework, which distinguishes between sensitive and non-sensitive tokens, along with scripts for privacy leakage analysis and utility-efficiency evaluation.
    \item[2.] Datasets and Preprocessing information: Links to publicly available datasets used in this study, along with preprocessing scripts to ensure reproducibility. Sensitive data were excluded or anonymized to comply with ethical standards.
    \item [3.] Evaluation Pipeline: An open-source pipeline for assessing privacy leakage and the trade-offs between privacy, utility, and efficiency in LLMs. 
\end{itemize}

These resources aim to support reproducibility and further research into privacy-aware, efficient LLM development.

% %-------------------------------------------------------------------------------
% \section{Introduction}
% %-------------------------------------------------------------------------------

% A paragraph of text goes here. Lots of text. Plenty of interesting
% text. Text text text text text text text text text text text text text
% text text text text text text text text text text text text text text
% text text text text text text text text text text text text text text
% text text text text text text text.
% More fascinating text. Features galore, plethora of promises.

% %-------------------------------------------------------------------------------
% \section{Footnotes, Verbatim, and Citations}
% %-------------------------------------------------------------------------------

% Footnotes should be places after punctuation characters, without any
% spaces between said characters and footnotes, like so.%
% \footnote{Remember that USENIX format stopped using endnotes and is
%   now using regular footnotes.} And some embedded literal code may
% look as follows.

% \begin{verbatim}
% int main(int argc, char *argv[]) 
% {
%     return 0;
% }
% \end{verbatim}

% Now we're going to cite somebody. Watch for the cite tag. Here it
% comes. Arpachi-Dusseau and Arpachi-Dusseau co-authored an excellent OS
% book, which is also really funny~\cite{arpachiDusseau18:osbook}, and
% Waldspurger got into the SIGOPS hall-of-fame due to his seminal paper
% about resource management in the ESX hypervisor~\cite{waldspurger02}.

% The tilde character (\~{}) in the tex source means a non-breaking
% space. This way, your reference will always be attached to the word
% that preceded it, instead of going to the next line.

% And the 'cite' package sorts your citations by their numerical order
% of the corresponding references at the end of the paper, ridding you
% from the need to notice that, e.g, ``Waldspurger'' appears after
% ``Arpachi-Dusseau'' when sorting references
% alphabetically~\cite{waldspurger02,arpachiDusseau18:osbook}. 

% It'd be nice and thoughtful of you to include a suitable link in each
% and every bibtex entry that you use in your submission, to allow
% reviewers (and other readers) to easily get to the cited work, as is
% done in all entries found in the References section of this document.

% Now we're going take a look at Section~\ref{sec:figs}, but not before
% observing that refs to sections and citations and such are colored and
% clickable in the PDF because of the packages we've included.

% %-------------------------------------------------------------------------------
% \section{Floating Figures and Lists}
% \label{sec:figs}
% %-------------------------------------------------------------------------------


% %---------------------------
% \begin{figure}
% \begin{center}
% \begin{tikzpicture}
%   \draw[thin,gray!40] (-2,-2) grid (2,2);
%   \draw[<->] (-2,0)--(2,0) node[right]{$x$};
%   \draw[<->] (0,-2)--(0,2) node[above]{$y$};
%   \draw[line width=2pt,blue,-stealth](0,0)--(1,1)
%         node[anchor=south west]{$\boldsymbol{u}$};
%   \draw[line width=2pt,red,-stealth](0,0)--(-1,-1)
%         node[anchor=north east]{$\boldsymbol{-u}$};
% \end{tikzpicture}
% \end{center}
% \caption{\label{fig:vectors} Text size inside figure should be as big as
%   caption's text. Text size inside figure should be as big as
%   caption's text. Text size inside figure should be as big as
%   caption's text. Text size inside figure should be as big as
%   caption's text. Text size inside figure should be as big as
%   caption's text. }
% \end{figure}
% %% %---------------------------


% Here's a typical reference to a floating figure:
% Figure~\ref{fig:vectors}. Floats should usually be placed where latex
% wants then. Figure\ref{fig:vectors} is centered, and has a caption
% that instructs you to make sure that the size of the text within the
% figures that you use is as big as (or bigger than) the size of the
% text in the caption of the figures. Please do. Really.

% In our case, we've explicitly drawn the figure inlined in latex, to
% allow this tex file to cleanly compile. But usually, your figures will
% reside in some file.pdf, and you'd include them in your document
% with, say, \textbackslash{}includegraphics.

% Lists are sometimes quite handy. If you want to itemize things, feel
% free:

% \begin{description}
  
% \item[fread] a function that reads from a \texttt{stream} into the
%   array \texttt{ptr} at most \texttt{nobj} objects of size
%   \texttt{size}, returning returns the number of objects read.

% \item[Fred] a person's name, e.g., there once was a dude named Fred
%   who separated usenix.sty from this file to allow for easy
%   inclusion.
% \end{description}

% \noindent
% The noindent at the start of this paragraph in its tex version makes
% it clear that it's a continuation of the preceding paragraph, as
% opposed to a new paragraph in its own right.


% \subsection{LaTeX-ing Your TeX File}
% %-----------------------------------

% People often use \texttt{pdflatex} these days for creating pdf-s from
% tex files via the shell. And \texttt{bibtex}, of course. Works for us.

% %-------------------------------------------------------------------------------
% \section*{Acknowledgments}
% %-------------------------------------------------------------------------------

% The USENIX latex style is old and very tired, which is why
% there's no \textbackslash{}acks command for you to use when
% acknowledging. Sorry.

% %-------------------------------------------------------------------------------
% \section*{Availability}
% %-------------------------------------------------------------------------------

% USENIX program committees give extra points to submissions that are
% backed by artifacts that are publicly available. If you made your code
% or data available, it's worth mentioning this fact in a dedicated
% section.

%-------------------------------------------------------------------------------
\bibliographystyle{plain}
\bibliography{main}

\subsection{Lloyd-Max Algorithm}
\label{subsec:Lloyd-Max}
For a given quantization bitwidth $B$ and an operand $\bm{X}$, the Lloyd-Max algorithm finds $2^B$ quantization levels $\{\hat{x}_i\}_{i=1}^{2^B}$ such that quantizing $\bm{X}$ by rounding each scalar in $\bm{X}$ to the nearest quantization level minimizes the quantization MSE. 

The algorithm starts with an initial guess of quantization levels and then iteratively computes quantization thresholds $\{\tau_i\}_{i=1}^{2^B-1}$ and updates quantization levels $\{\hat{x}_i\}_{i=1}^{2^B}$. Specifically, at iteration $n$, thresholds are set to the midpoints of the previous iteration's levels:
\begin{align*}
    \tau_i^{(n)}=\frac{\hat{x}_i^{(n-1)}+\hat{x}_{i+1}^{(n-1)}}2 \text{ for } i=1\ldots 2^B-1
\end{align*}
Subsequently, the quantization levels are re-computed as conditional means of the data regions defined by the new thresholds:
\begin{align*}
    \hat{x}_i^{(n)}=\mathbb{E}\left[ \bm{X} \big| \bm{X}\in [\tau_{i-1}^{(n)},\tau_i^{(n)}] \right] \text{ for } i=1\ldots 2^B
\end{align*}
where to satisfy boundary conditions we have $\tau_0=-\infty$ and $\tau_{2^B}=\infty$. The algorithm iterates the above steps until convergence.

Figure \ref{fig:lm_quant} compares the quantization levels of a $7$-bit floating point (E3M3) quantizer (left) to a $7$-bit Lloyd-Max quantizer (right) when quantizing a layer of weights from the GPT3-126M model at a per-tensor granularity. As shown, the Lloyd-Max quantizer achieves substantially lower quantization MSE. Further, Table \ref{tab:FP7_vs_LM7} shows the superior perplexity achieved by Lloyd-Max quantizers for bitwidths of $7$, $6$ and $5$. The difference between the quantizers is clear at 5 bits, where per-tensor FP quantization incurs a drastic and unacceptable increase in perplexity, while Lloyd-Max quantization incurs a much smaller increase. Nevertheless, we note that even the optimal Lloyd-Max quantizer incurs a notable ($\sim 1.5$) increase in perplexity due to the coarse granularity of quantization. 

\begin{figure}[h]
  \centering
  \includegraphics[width=0.7\linewidth]{sections/figures/LM7_FP7.pdf}
  \caption{\small Quantization levels and the corresponding quantization MSE of Floating Point (left) vs Lloyd-Max (right) Quantizers for a layer of weights in the GPT3-126M model.}
  \label{fig:lm_quant}
\end{figure}

\begin{table}[h]\scriptsize
\begin{center}
\caption{\label{tab:FP7_vs_LM7} \small Comparing perplexity (lower is better) achieved by floating point quantizers and Lloyd-Max quantizers on a GPT3-126M model for the Wikitext-103 dataset.}
\begin{tabular}{c|cc|c}
\hline
 \multirow{2}{*}{\textbf{Bitwidth}} & \multicolumn{2}{|c|}{\textbf{Floating-Point Quantizer}} & \textbf{Lloyd-Max Quantizer} \\
 & Best Format & Wikitext-103 Perplexity & Wikitext-103 Perplexity \\
\hline
7 & E3M3 & 18.32 & 18.27 \\
6 & E3M2 & 19.07 & 18.51 \\
5 & E4M0 & 43.89 & 19.71 \\
\hline
\end{tabular}
\end{center}
\end{table}

\subsection{Proof of Local Optimality of LO-BCQ}
\label{subsec:lobcq_opt_proof}
For a given block $\bm{b}_j$, the quantization MSE during LO-BCQ can be empirically evaluated as $\frac{1}{L_b}\lVert \bm{b}_j- \bm{\hat{b}}_j\rVert^2_2$ where $\bm{\hat{b}}_j$ is computed from equation (\ref{eq:clustered_quantization_definition}) as $C_{f(\bm{b}_j)}(\bm{b}_j)$. Further, for a given block cluster $\mathcal{B}_i$, we compute the quantization MSE as $\frac{1}{|\mathcal{B}_{i}|}\sum_{\bm{b} \in \mathcal{B}_{i}} \frac{1}{L_b}\lVert \bm{b}- C_i^{(n)}(\bm{b})\rVert^2_2$. Therefore, at the end of iteration $n$, we evaluate the overall quantization MSE $J^{(n)}$ for a given operand $\bm{X}$ composed of $N_c$ block clusters as:
\begin{align*}
    \label{eq:mse_iter_n}
    J^{(n)} = \frac{1}{N_c} \sum_{i=1}^{N_c} \frac{1}{|\mathcal{B}_{i}^{(n)}|}\sum_{\bm{v} \in \mathcal{B}_{i}^{(n)}} \frac{1}{L_b}\lVert \bm{b}- B_i^{(n)}(\bm{b})\rVert^2_2
\end{align*}

At the end of iteration $n$, the codebooks are updated from $\mathcal{C}^{(n-1)}$ to $\mathcal{C}^{(n)}$. However, the mapping of a given vector $\bm{b}_j$ to quantizers $\mathcal{C}^{(n)}$ remains as  $f^{(n)}(\bm{b}_j)$. At the next iteration, during the vector clustering step, $f^{(n+1)}(\bm{b}_j)$ finds new mapping of $\bm{b}_j$ to updated codebooks $\mathcal{C}^{(n)}$ such that the quantization MSE over the candidate codebooks is minimized. Therefore, we obtain the following result for $\bm{b}_j$:
\begin{align*}
\frac{1}{L_b}\lVert \bm{b}_j - C_{f^{(n+1)}(\bm{b}_j)}^{(n)}(\bm{b}_j)\rVert^2_2 \le \frac{1}{L_b}\lVert \bm{b}_j - C_{f^{(n)}(\bm{b}_j)}^{(n)}(\bm{b}_j)\rVert^2_2
\end{align*}

That is, quantizing $\bm{b}_j$ at the end of the block clustering step of iteration $n+1$ results in lower quantization MSE compared to quantizing at the end of iteration $n$. Since this is true for all $\bm{b} \in \bm{X}$, we assert the following:
\begin{equation}
\begin{split}
\label{eq:mse_ineq_1}
    \tilde{J}^{(n+1)} &= \frac{1}{N_c} \sum_{i=1}^{N_c} \frac{1}{|\mathcal{B}_{i}^{(n+1)}|}\sum_{\bm{b} \in \mathcal{B}_{i}^{(n+1)}} \frac{1}{L_b}\lVert \bm{b} - C_i^{(n)}(b)\rVert^2_2 \le J^{(n)}
\end{split}
\end{equation}
where $\tilde{J}^{(n+1)}$ is the the quantization MSE after the vector clustering step at iteration $n+1$.

Next, during the codebook update step (\ref{eq:quantizers_update}) at iteration $n+1$, the per-cluster codebooks $\mathcal{C}^{(n)}$ are updated to $\mathcal{C}^{(n+1)}$ by invoking the Lloyd-Max algorithm \citep{Lloyd}. We know that for any given value distribution, the Lloyd-Max algorithm minimizes the quantization MSE. Therefore, for a given vector cluster $\mathcal{B}_i$ we obtain the following result:

\begin{equation}
    \frac{1}{|\mathcal{B}_{i}^{(n+1)}|}\sum_{\bm{b} \in \mathcal{B}_{i}^{(n+1)}} \frac{1}{L_b}\lVert \bm{b}- C_i^{(n+1)}(\bm{b})\rVert^2_2 \le \frac{1}{|\mathcal{B}_{i}^{(n+1)}|}\sum_{\bm{b} \in \mathcal{B}_{i}^{(n+1)}} \frac{1}{L_b}\lVert \bm{b}- C_i^{(n)}(\bm{b})\rVert^2_2
\end{equation}

The above equation states that quantizing the given block cluster $\mathcal{B}_i$ after updating the associated codebook from $C_i^{(n)}$ to $C_i^{(n+1)}$ results in lower quantization MSE. Since this is true for all the block clusters, we derive the following result: 
\begin{equation}
\begin{split}
\label{eq:mse_ineq_2}
     J^{(n+1)} &= \frac{1}{N_c} \sum_{i=1}^{N_c} \frac{1}{|\mathcal{B}_{i}^{(n+1)}|}\sum_{\bm{b} \in \mathcal{B}_{i}^{(n+1)}} \frac{1}{L_b}\lVert \bm{b}- C_i^{(n+1)}(\bm{b})\rVert^2_2  \le \tilde{J}^{(n+1)}   
\end{split}
\end{equation}

Following (\ref{eq:mse_ineq_1}) and (\ref{eq:mse_ineq_2}), we find that the quantization MSE is non-increasing for each iteration, that is, $J^{(1)} \ge J^{(2)} \ge J^{(3)} \ge \ldots \ge J^{(M)}$ where $M$ is the maximum number of iterations. 
%Therefore, we can say that if the algorithm converges, then it must be that it has converged to a local minimum. 
\hfill $\blacksquare$


\begin{figure}
    \begin{center}
    \includegraphics[width=0.5\textwidth]{sections//figures/mse_vs_iter.pdf}
    \end{center}
    \caption{\small NMSE vs iterations during LO-BCQ compared to other block quantization proposals}
    \label{fig:nmse_vs_iter}
\end{figure}

Figure \ref{fig:nmse_vs_iter} shows the empirical convergence of LO-BCQ across several block lengths and number of codebooks. Also, the MSE achieved by LO-BCQ is compared to baselines such as MXFP and VSQ. As shown, LO-BCQ converges to a lower MSE than the baselines. Further, we achieve better convergence for larger number of codebooks ($N_c$) and for a smaller block length ($L_b$), both of which increase the bitwidth of BCQ (see Eq \ref{eq:bitwidth_bcq}).


\subsection{Additional Accuracy Results}
%Table \ref{tab:lobcq_config} lists the various LOBCQ configurations and their corresponding bitwidths.
\begin{table}
\setlength{\tabcolsep}{4.75pt}
\begin{center}
\caption{\label{tab:lobcq_config} Various LO-BCQ configurations and their bitwidths.}
\begin{tabular}{|c||c|c|c|c||c|c||c|} 
\hline
 & \multicolumn{4}{|c||}{$L_b=8$} & \multicolumn{2}{|c||}{$L_b=4$} & $L_b=2$ \\
 \hline
 \backslashbox{$L_A$\kern-1em}{\kern-1em$N_c$} & 2 & 4 & 8 & 16 & 2 & 4 & 2 \\
 \hline
 64 & 4.25 & 4.375 & 4.5 & 4.625 & 4.375 & 4.625 & 4.625\\
 \hline
 32 & 4.375 & 4.5 & 4.625& 4.75 & 4.5 & 4.75 & 4.75 \\
 \hline
 16 & 4.625 & 4.75& 4.875 & 5 & 4.75 & 5 & 5 \\
 \hline
\end{tabular}
\end{center}
\end{table}

%\subsection{Perplexity achieved by various LO-BCQ configurations on Wikitext-103 dataset}

\begin{table} \centering
\begin{tabular}{|c||c|c|c|c||c|c||c|} 
\hline
 $L_b \rightarrow$& \multicolumn{4}{c||}{8} & \multicolumn{2}{c||}{4} & 2\\
 \hline
 \backslashbox{$L_A$\kern-1em}{\kern-1em$N_c$} & 2 & 4 & 8 & 16 & 2 & 4 & 2  \\
 %$N_c \rightarrow$ & 2 & 4 & 8 & 16 & 2 & 4 & 2 \\
 \hline
 \hline
 \multicolumn{8}{c}{GPT3-1.3B (FP32 PPL = 9.98)} \\ 
 \hline
 \hline
 64 & 10.40 & 10.23 & 10.17 & 10.15 &  10.28 & 10.18 & 10.19 \\
 \hline
 32 & 10.25 & 10.20 & 10.15 & 10.12 &  10.23 & 10.17 & 10.17 \\
 \hline
 16 & 10.22 & 10.16 & 10.10 & 10.09 &  10.21 & 10.14 & 10.16 \\
 \hline
  \hline
 \multicolumn{8}{c}{GPT3-8B (FP32 PPL = 7.38)} \\ 
 \hline
 \hline
 64 & 7.61 & 7.52 & 7.48 &  7.47 &  7.55 &  7.49 & 7.50 \\
 \hline
 32 & 7.52 & 7.50 & 7.46 &  7.45 &  7.52 &  7.48 & 7.48  \\
 \hline
 16 & 7.51 & 7.48 & 7.44 &  7.44 &  7.51 &  7.49 & 7.47  \\
 \hline
\end{tabular}
\caption{\label{tab:ppl_gpt3_abalation} Wikitext-103 perplexity across GPT3-1.3B and 8B models.}
\end{table}

\begin{table} \centering
\begin{tabular}{|c||c|c|c|c||} 
\hline
 $L_b \rightarrow$& \multicolumn{4}{c||}{8}\\
 \hline
 \backslashbox{$L_A$\kern-1em}{\kern-1em$N_c$} & 2 & 4 & 8 & 16 \\
 %$N_c \rightarrow$ & 2 & 4 & 8 & 16 & 2 & 4 & 2 \\
 \hline
 \hline
 \multicolumn{5}{|c|}{Llama2-7B (FP32 PPL = 5.06)} \\ 
 \hline
 \hline
 64 & 5.31 & 5.26 & 5.19 & 5.18  \\
 \hline
 32 & 5.23 & 5.25 & 5.18 & 5.15  \\
 \hline
 16 & 5.23 & 5.19 & 5.16 & 5.14  \\
 \hline
 \multicolumn{5}{|c|}{Nemotron4-15B (FP32 PPL = 5.87)} \\ 
 \hline
 \hline
 64  & 6.3 & 6.20 & 6.13 & 6.08  \\
 \hline
 32  & 6.24 & 6.12 & 6.07 & 6.03  \\
 \hline
 16  & 6.12 & 6.14 & 6.04 & 6.02  \\
 \hline
 \multicolumn{5}{|c|}{Nemotron4-340B (FP32 PPL = 3.48)} \\ 
 \hline
 \hline
 64 & 3.67 & 3.62 & 3.60 & 3.59 \\
 \hline
 32 & 3.63 & 3.61 & 3.59 & 3.56 \\
 \hline
 16 & 3.61 & 3.58 & 3.57 & 3.55 \\
 \hline
\end{tabular}
\caption{\label{tab:ppl_llama7B_nemo15B} Wikitext-103 perplexity compared to FP32 baseline in Llama2-7B and Nemotron4-15B, 340B models}
\end{table}

%\subsection{Perplexity achieved by various LO-BCQ configurations on MMLU dataset}


\begin{table} \centering
\begin{tabular}{|c||c|c|c|c||c|c|c|c|} 
\hline
 $L_b \rightarrow$& \multicolumn{4}{c||}{8} & \multicolumn{4}{c||}{8}\\
 \hline
 \backslashbox{$L_A$\kern-1em}{\kern-1em$N_c$} & 2 & 4 & 8 & 16 & 2 & 4 & 8 & 16  \\
 %$N_c \rightarrow$ & 2 & 4 & 8 & 16 & 2 & 4 & 2 \\
 \hline
 \hline
 \multicolumn{5}{|c|}{Llama2-7B (FP32 Accuracy = 45.8\%)} & \multicolumn{4}{|c|}{Llama2-70B (FP32 Accuracy = 69.12\%)} \\ 
 \hline
 \hline
 64 & 43.9 & 43.4 & 43.9 & 44.9 & 68.07 & 68.27 & 68.17 & 68.75 \\
 \hline
 32 & 44.5 & 43.8 & 44.9 & 44.5 & 68.37 & 68.51 & 68.35 & 68.27  \\
 \hline
 16 & 43.9 & 42.7 & 44.9 & 45 & 68.12 & 68.77 & 68.31 & 68.59  \\
 \hline
 \hline
 \multicolumn{5}{|c|}{GPT3-22B (FP32 Accuracy = 38.75\%)} & \multicolumn{4}{|c|}{Nemotron4-15B (FP32 Accuracy = 64.3\%)} \\ 
 \hline
 \hline
 64 & 36.71 & 38.85 & 38.13 & 38.92 & 63.17 & 62.36 & 63.72 & 64.09 \\
 \hline
 32 & 37.95 & 38.69 & 39.45 & 38.34 & 64.05 & 62.30 & 63.8 & 64.33  \\
 \hline
 16 & 38.88 & 38.80 & 38.31 & 38.92 & 63.22 & 63.51 & 63.93 & 64.43  \\
 \hline
\end{tabular}
\caption{\label{tab:mmlu_abalation} Accuracy on MMLU dataset across GPT3-22B, Llama2-7B, 70B and Nemotron4-15B models.}
\end{table}


%\subsection{Perplexity achieved by various LO-BCQ configurations on LM evaluation harness}

\begin{table} \centering
\begin{tabular}{|c||c|c|c|c||c|c|c|c|} 
\hline
 $L_b \rightarrow$& \multicolumn{4}{c||}{8} & \multicolumn{4}{c||}{8}\\
 \hline
 \backslashbox{$L_A$\kern-1em}{\kern-1em$N_c$} & 2 & 4 & 8 & 16 & 2 & 4 & 8 & 16  \\
 %$N_c \rightarrow$ & 2 & 4 & 8 & 16 & 2 & 4 & 2 \\
 \hline
 \hline
 \multicolumn{5}{|c|}{Race (FP32 Accuracy = 37.51\%)} & \multicolumn{4}{|c|}{Boolq (FP32 Accuracy = 64.62\%)} \\ 
 \hline
 \hline
 64 & 36.94 & 37.13 & 36.27 & 37.13 & 63.73 & 62.26 & 63.49 & 63.36 \\
 \hline
 32 & 37.03 & 36.36 & 36.08 & 37.03 & 62.54 & 63.51 & 63.49 & 63.55  \\
 \hline
 16 & 37.03 & 37.03 & 36.46 & 37.03 & 61.1 & 63.79 & 63.58 & 63.33  \\
 \hline
 \hline
 \multicolumn{5}{|c|}{Winogrande (FP32 Accuracy = 58.01\%)} & \multicolumn{4}{|c|}{Piqa (FP32 Accuracy = 74.21\%)} \\ 
 \hline
 \hline
 64 & 58.17 & 57.22 & 57.85 & 58.33 & 73.01 & 73.07 & 73.07 & 72.80 \\
 \hline
 32 & 59.12 & 58.09 & 57.85 & 58.41 & 73.01 & 73.94 & 72.74 & 73.18  \\
 \hline
 16 & 57.93 & 58.88 & 57.93 & 58.56 & 73.94 & 72.80 & 73.01 & 73.94  \\
 \hline
\end{tabular}
\caption{\label{tab:mmlu_abalation} Accuracy on LM evaluation harness tasks on GPT3-1.3B model.}
\end{table}

\begin{table} \centering
\begin{tabular}{|c||c|c|c|c||c|c|c|c|} 
\hline
 $L_b \rightarrow$& \multicolumn{4}{c||}{8} & \multicolumn{4}{c||}{8}\\
 \hline
 \backslashbox{$L_A$\kern-1em}{\kern-1em$N_c$} & 2 & 4 & 8 & 16 & 2 & 4 & 8 & 16  \\
 %$N_c \rightarrow$ & 2 & 4 & 8 & 16 & 2 & 4 & 2 \\
 \hline
 \hline
 \multicolumn{5}{|c|}{Race (FP32 Accuracy = 41.34\%)} & \multicolumn{4}{|c|}{Boolq (FP32 Accuracy = 68.32\%)} \\ 
 \hline
 \hline
 64 & 40.48 & 40.10 & 39.43 & 39.90 & 69.20 & 68.41 & 69.45 & 68.56 \\
 \hline
 32 & 39.52 & 39.52 & 40.77 & 39.62 & 68.32 & 67.43 & 68.17 & 69.30  \\
 \hline
 16 & 39.81 & 39.71 & 39.90 & 40.38 & 68.10 & 66.33 & 69.51 & 69.42  \\
 \hline
 \hline
 \multicolumn{5}{|c|}{Winogrande (FP32 Accuracy = 67.88\%)} & \multicolumn{4}{|c|}{Piqa (FP32 Accuracy = 78.78\%)} \\ 
 \hline
 \hline
 64 & 66.85 & 66.61 & 67.72 & 67.88 & 77.31 & 77.42 & 77.75 & 77.64 \\
 \hline
 32 & 67.25 & 67.72 & 67.72 & 67.00 & 77.31 & 77.04 & 77.80 & 77.37  \\
 \hline
 16 & 68.11 & 68.90 & 67.88 & 67.48 & 77.37 & 78.13 & 78.13 & 77.69  \\
 \hline
\end{tabular}
\caption{\label{tab:mmlu_abalation} Accuracy on LM evaluation harness tasks on GPT3-8B model.}
\end{table}

\begin{table} \centering
\begin{tabular}{|c||c|c|c|c||c|c|c|c|} 
\hline
 $L_b \rightarrow$& \multicolumn{4}{c||}{8} & \multicolumn{4}{c||}{8}\\
 \hline
 \backslashbox{$L_A$\kern-1em}{\kern-1em$N_c$} & 2 & 4 & 8 & 16 & 2 & 4 & 8 & 16  \\
 %$N_c \rightarrow$ & 2 & 4 & 8 & 16 & 2 & 4 & 2 \\
 \hline
 \hline
 \multicolumn{5}{|c|}{Race (FP32 Accuracy = 40.67\%)} & \multicolumn{4}{|c|}{Boolq (FP32 Accuracy = 76.54\%)} \\ 
 \hline
 \hline
 64 & 40.48 & 40.10 & 39.43 & 39.90 & 75.41 & 75.11 & 77.09 & 75.66 \\
 \hline
 32 & 39.52 & 39.52 & 40.77 & 39.62 & 76.02 & 76.02 & 75.96 & 75.35  \\
 \hline
 16 & 39.81 & 39.71 & 39.90 & 40.38 & 75.05 & 73.82 & 75.72 & 76.09  \\
 \hline
 \hline
 \multicolumn{5}{|c|}{Winogrande (FP32 Accuracy = 70.64\%)} & \multicolumn{4}{|c|}{Piqa (FP32 Accuracy = 79.16\%)} \\ 
 \hline
 \hline
 64 & 69.14 & 70.17 & 70.17 & 70.56 & 78.24 & 79.00 & 78.62 & 78.73 \\
 \hline
 32 & 70.96 & 69.69 & 71.27 & 69.30 & 78.56 & 79.49 & 79.16 & 78.89  \\
 \hline
 16 & 71.03 & 69.53 & 69.69 & 70.40 & 78.13 & 79.16 & 79.00 & 79.00  \\
 \hline
\end{tabular}
\caption{\label{tab:mmlu_abalation} Accuracy on LM evaluation harness tasks on GPT3-22B model.}
\end{table}

\begin{table} \centering
\begin{tabular}{|c||c|c|c|c||c|c|c|c|} 
\hline
 $L_b \rightarrow$& \multicolumn{4}{c||}{8} & \multicolumn{4}{c||}{8}\\
 \hline
 \backslashbox{$L_A$\kern-1em}{\kern-1em$N_c$} & 2 & 4 & 8 & 16 & 2 & 4 & 8 & 16  \\
 %$N_c \rightarrow$ & 2 & 4 & 8 & 16 & 2 & 4 & 2 \\
 \hline
 \hline
 \multicolumn{5}{|c|}{Race (FP32 Accuracy = 44.4\%)} & \multicolumn{4}{|c|}{Boolq (FP32 Accuracy = 79.29\%)} \\ 
 \hline
 \hline
 64 & 42.49 & 42.51 & 42.58 & 43.45 & 77.58 & 77.37 & 77.43 & 78.1 \\
 \hline
 32 & 43.35 & 42.49 & 43.64 & 43.73 & 77.86 & 75.32 & 77.28 & 77.86  \\
 \hline
 16 & 44.21 & 44.21 & 43.64 & 42.97 & 78.65 & 77 & 76.94 & 77.98  \\
 \hline
 \hline
 \multicolumn{5}{|c|}{Winogrande (FP32 Accuracy = 69.38\%)} & \multicolumn{4}{|c|}{Piqa (FP32 Accuracy = 78.07\%)} \\ 
 \hline
 \hline
 64 & 68.9 & 68.43 & 69.77 & 68.19 & 77.09 & 76.82 & 77.09 & 77.86 \\
 \hline
 32 & 69.38 & 68.51 & 68.82 & 68.90 & 78.07 & 76.71 & 78.07 & 77.86  \\
 \hline
 16 & 69.53 & 67.09 & 69.38 & 68.90 & 77.37 & 77.8 & 77.91 & 77.69  \\
 \hline
\end{tabular}
\caption{\label{tab:mmlu_abalation} Accuracy on LM evaluation harness tasks on Llama2-7B model.}
\end{table}

\begin{table} \centering
\begin{tabular}{|c||c|c|c|c||c|c|c|c|} 
\hline
 $L_b \rightarrow$& \multicolumn{4}{c||}{8} & \multicolumn{4}{c||}{8}\\
 \hline
 \backslashbox{$L_A$\kern-1em}{\kern-1em$N_c$} & 2 & 4 & 8 & 16 & 2 & 4 & 8 & 16  \\
 %$N_c \rightarrow$ & 2 & 4 & 8 & 16 & 2 & 4 & 2 \\
 \hline
 \hline
 \multicolumn{5}{|c|}{Race (FP32 Accuracy = 48.8\%)} & \multicolumn{4}{|c|}{Boolq (FP32 Accuracy = 85.23\%)} \\ 
 \hline
 \hline
 64 & 49.00 & 49.00 & 49.28 & 48.71 & 82.82 & 84.28 & 84.03 & 84.25 \\
 \hline
 32 & 49.57 & 48.52 & 48.33 & 49.28 & 83.85 & 84.46 & 84.31 & 84.93  \\
 \hline
 16 & 49.85 & 49.09 & 49.28 & 48.99 & 85.11 & 84.46 & 84.61 & 83.94  \\
 \hline
 \hline
 \multicolumn{5}{|c|}{Winogrande (FP32 Accuracy = 79.95\%)} & \multicolumn{4}{|c|}{Piqa (FP32 Accuracy = 81.56\%)} \\ 
 \hline
 \hline
 64 & 78.77 & 78.45 & 78.37 & 79.16 & 81.45 & 80.69 & 81.45 & 81.5 \\
 \hline
 32 & 78.45 & 79.01 & 78.69 & 80.66 & 81.56 & 80.58 & 81.18 & 81.34  \\
 \hline
 16 & 79.95 & 79.56 & 79.79 & 79.72 & 81.28 & 81.66 & 81.28 & 80.96  \\
 \hline
\end{tabular}
\caption{\label{tab:mmlu_abalation} Accuracy on LM evaluation harness tasks on Llama2-70B model.}
\end{table}

%\section{MSE Studies}
%\textcolor{red}{TODO}


\subsection{Number Formats and Quantization Method}
\label{subsec:numFormats_quantMethod}
\subsubsection{Integer Format}
An $n$-bit signed integer (INT) is typically represented with a 2s-complement format \citep{yao2022zeroquant,xiao2023smoothquant,dai2021vsq}, where the most significant bit denotes the sign.

\subsubsection{Floating Point Format}
An $n$-bit signed floating point (FP) number $x$ comprises of a 1-bit sign ($x_{\mathrm{sign}}$), $B_m$-bit mantissa ($x_{\mathrm{mant}}$) and $B_e$-bit exponent ($x_{\mathrm{exp}}$) such that $B_m+B_e=n-1$. The associated constant exponent bias ($E_{\mathrm{bias}}$) is computed as $(2^{{B_e}-1}-1)$. We denote this format as $E_{B_e}M_{B_m}$.  

\subsubsection{Quantization Scheme}
\label{subsec:quant_method}
A quantization scheme dictates how a given unquantized tensor is converted to its quantized representation. We consider FP formats for the purpose of illustration. Given an unquantized tensor $\bm{X}$ and an FP format $E_{B_e}M_{B_m}$, we first, we compute the quantization scale factor $s_X$ that maps the maximum absolute value of $\bm{X}$ to the maximum quantization level of the $E_{B_e}M_{B_m}$ format as follows:
\begin{align}
\label{eq:sf}
    s_X = \frac{\mathrm{max}(|\bm{X}|)}{\mathrm{max}(E_{B_e}M_{B_m})}
\end{align}
In the above equation, $|\cdot|$ denotes the absolute value function.

Next, we scale $\bm{X}$ by $s_X$ and quantize it to $\hat{\bm{X}}$ by rounding it to the nearest quantization level of $E_{B_e}M_{B_m}$ as:

\begin{align}
\label{eq:tensor_quant}
    \hat{\bm{X}} = \text{round-to-nearest}\left(\frac{\bm{X}}{s_X}, E_{B_e}M_{B_m}\right)
\end{align}

We perform dynamic max-scaled quantization \citep{wu2020integer}, where the scale factor $s$ for activations is dynamically computed during runtime.

\subsection{Vector Scaled Quantization}
\begin{wrapfigure}{r}{0.35\linewidth}
  \centering
  \includegraphics[width=\linewidth]{sections/figures/vsquant.jpg}
  \caption{\small Vectorwise decomposition for per-vector scaled quantization (VSQ \citep{dai2021vsq}).}
  \label{fig:vsquant}
\end{wrapfigure}
During VSQ \citep{dai2021vsq}, the operand tensors are decomposed into 1D vectors in a hardware friendly manner as shown in Figure \ref{fig:vsquant}. Since the decomposed tensors are used as operands in matrix multiplications during inference, it is beneficial to perform this decomposition along the reduction dimension of the multiplication. The vectorwise quantization is performed similar to tensorwise quantization described in Equations \ref{eq:sf} and \ref{eq:tensor_quant}, where a scale factor $s_v$ is required for each vector $\bm{v}$ that maps the maximum absolute value of that vector to the maximum quantization level. While smaller vector lengths can lead to larger accuracy gains, the associated memory and computational overheads due to the per-vector scale factors increases. To alleviate these overheads, VSQ \citep{dai2021vsq} proposed a second level quantization of the per-vector scale factors to unsigned integers, while MX \citep{rouhani2023shared} quantizes them to integer powers of 2 (denoted as $2^{INT}$).

\subsubsection{MX Format}
The MX format proposed in \citep{rouhani2023microscaling} introduces the concept of sub-block shifting. For every two scalar elements of $b$-bits each, there is a shared exponent bit. The value of this exponent bit is determined through an empirical analysis that targets minimizing quantization MSE. We note that the FP format $E_{1}M_{b}$ is strictly better than MX from an accuracy perspective since it allocates a dedicated exponent bit to each scalar as opposed to sharing it across two scalars. Therefore, we conservatively bound the accuracy of a $b+2$-bit signed MX format with that of a $E_{1}M_{b}$ format in our comparisons. For instance, we use E1M2 format as a proxy for MX4.

\begin{figure}
    \centering
    \includegraphics[width=1\linewidth]{sections//figures/BlockFormats.pdf}
    \caption{\small Comparing LO-BCQ to MX format.}
    \label{fig:block_formats}
\end{figure}

Figure \ref{fig:block_formats} compares our $4$-bit LO-BCQ block format to MX \citep{rouhani2023microscaling}. As shown, both LO-BCQ and MX decompose a given operand tensor into block arrays and each block array into blocks. Similar to MX, we find that per-block quantization ($L_b < L_A$) leads to better accuracy due to increased flexibility. While MX achieves this through per-block $1$-bit micro-scales, we associate a dedicated codebook to each block through a per-block codebook selector. Further, MX quantizes the per-block array scale-factor to E8M0 format without per-tensor scaling. In contrast during LO-BCQ, we find that per-tensor scaling combined with quantization of per-block array scale-factor to E4M3 format results in superior inference accuracy across models. 


%%%%%%%%%%%%%%%%%%%%%%%%%%%%%%%%%%%%%%%%%%%%%%%%%%%%%%%%%%%%%%%%%%%%%%%%%%%%%%%%
\end{document}
%%%%%%%%%%%%%%%%%%%%%%%%%%%%%%%%%%%%%%%%%%%%%%%%%%%%%%%%%%%%%%%%%%%%%%%%%%%%%%%%

%%  LocalWords:  endnotes includegraphics fread ptr nobj noindent
%%  LocalWords:  pdflatex acks
