\begin{table}[!ht]
    \centering
    \small
    \caption{The prompt template for classifying the topic and format of a web page. The first two row shows the templates for system and user prompts, in which {\tt$\{$domain$\}$} becomes either ``topic'' and ``format'' and {\tt$\{$instructions$\}$} are substituted with the content of the bottom two rows.}
    \icmlskip{0.1in}
\begin{tabular}{lp{0.8\textwidth}}
\toprule
& \multicolumn{1}{c}{Prompt templates} \\
\midrule
System & {\tt Your task is to classify the $\{$domain$\}$ of web pages into one of the following 24 categories:} \\
    & {\tt $\{$choices$\}$ } \\
    & \\
    & {\tt $\{$instructions$\}$} \\
\midrule
User & {\tt Consider the following web page:} \\
    & \\
    & {\tt URL: `$\{$url$\}$`} \\
    & {\tt Content: ```} \\
    & {\tt $\{$text$\}$} \\
    & {\tt ```} \\
    & \\
    & {\tt Your task is to classify the $\{$domain$\}$ of web pages into one of the following 24 categories:} \\
    & {\tt $\{$choices$\}$ } \\
    & \\
    & {\tt $\{$instructions$\}$} \\
\midrule
& \multicolumn{1}{c}{Instructions} \\
\midrule
{\topics Topic} & \tt Choose which topic from the above list is the best match for describing what the web page content is about. If the content is about multiple topics, choose the one that is most prominent.
Remember to focus on the topic, and not the format, e.g., a book excerpt about a first date is related to `Social Life' and not `Literature'.
The URL might help you understand the content. Avoid shortcuts such as word overlap between the page and the topic descriptions or simple patterns in the URL.
Start your response with the single-letter ID of the correct topic followed by an explanation. \\
\midrule
{\formats Format} & \tt
Choose which format from the above list is the best match for describing the style, purpose and origin of the web page content. If the content has multiple formats, choose the one that is most prominent.
Remember to focus on the format, and not the topic, e.g., a research paper about legal issues does not count as `Legal Notices'.
The URL might help you understand the content. Avoid shortcuts such as word overlap between the page and the format descriptions or simple patterns in the URL, for example `.../blog/...' may also occur for organizational announcements, comment sections, and other formats.
Start your response with the single-letter ID of the correct format followed by an explanation. \\
\bottomrule
\end{tabular}
    \icmlskip{-0.1in}
    \label{tab:templates}
\end{table}
