\section{Conclusion}
\mname{} quantifies the importance of textual tokens corresponding to generated audio by leveraging both factual and counterfactual reasoning frameworks. This approach enables the generation of faithful explanations, providing actionable insights for users to edit audio and assisting developers in debugging. Consequently, \mname{} enhances the transparency and trustworthiness of TAG models.

\section{Acknowledgements}
This work was supported by NCSOFT, the Institute of Information \& Communications Technology Planning \& evaluation (IITP) grant, and the National Research Foundation of Korea (NRF) grant funded by the Korean government (MSIT) (RS-2019-II190421, IITP-2025-RS-2020-II201821, RS-2024-00438686, RS-2024-00436936, RS-2024-00360227, RS-2023-0022544, NRF-2021M3H4A1A02056037, RS-2024-00448809).  This research was also partially supported by the Culture, Sports, and Tourism R\&D Program through the Korea Creative Content Agency grant funded by the Ministry of Culture, Sports and Tourism in 2024 (RS-2024-00333068, RS-2024-00348469 (25\%)).
\newpage
