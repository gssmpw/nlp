\documentclass{article}



\usepackage{arxiv}

\usepackage[utf8]{inputenc}
\usepackage[T1]{fontenc}   
\usepackage{hyperref}      
\usepackage{url}           
\usepackage{booktabs}      
\usepackage{amsfonts}      
\usepackage{nicefrac}      
\usepackage{microtype}     
\usepackage{lipsum}		
\usepackage{graphicx}
\usepackage{natbib}
\usepackage{doi}
\usepackage{amsmath}
\usepackage{amssymb}


\title{Low Dimensional Koopman Generalized Eigenfunctions Representation: An Approach to Address Koopman High-Dimensionality Problem}


\date{} 					

\author{Simone Martini$^*$, Margareta Stefanovic, Kimon P. Valavanis\\ \\
*\small Corresponding author\\ \small Ritchie School of Engineering and Computer Science, University of Denver\\ \tt\small simone.martini@du.edu, margareta.stefanovic@du.edu, kimon.valavanis@du.edu}



\renewcommand{\headeright}{}
\renewcommand{\undertitle}{}
\renewcommand{\shorttitle}{Low Dimensional Koopman}


\hypersetup{
pdftitle={A template for the arxiv style},
pdfsubject={q-bio.NC, q-bio.QM},
pdfauthor={David S.~Hippocampus, Elias D.~Striatum},
pdfkeywords={First keyword, Second keyword, More},
}

\begin{document}
\maketitle

\begin{abstract}
This Paper introduces a methodology to achieve a lower dimensional Koopman quasi linear representation of nonlinear dynamics using Koopman generalized eigenfunctions.
The methodology is presented for the analytically derived Koopman formulation of rigid body dynamics but can be generalized to any data-driven or analytically derived generalized eigenfunction set. The presented approach aim at achieving a representation for which the number of Koopman observables matches the number of input leading to an exact linearization solution instead of resorting to the least square approximation method. The methodology is tested by designing a linear quadratic (LQ) flight controller of a quadrotor unmanned aerial vehicle (UAV). Hardware in the loop simulations validate the applicability of this approach to real-time implementation in presence of noise and sensor delays.
\end{abstract}


% keywords can be removed
\keywords{Koopman Operator \and Nonlinear Control \and Underactuated Control}


\section{Introduction}
This Paper introduces a methodology to achieve a lower dimensional Koopman quasi linear representation of nonlinear dynamics using Koopman generalized eigenfunctions.
The methodology is presented for the analytically derived Koopman formulation of rigid body dynamics but can be generalized to any data-driven or analytically derived generalized eigenfunction set. The presented approach aim at achieving a representation for which the number of Koopman observables matches the number of input leading to an exact linearization solution instead of resorting to the least square approximation method. The methodology is tested by designing a linear quadratic (LQ) flight controller of a quadrotor unmanned aerial vehicle (UAV). Hardware in the loop simulations validate the applicability of this approach to real-time implementation in presence of noise and sensor delays.

\section{Notation \& Background Information}
\subsection{Notation}
Denote $0_n$ and $I_n$ as the $n \times n$ zero matrix and identity matrix, respectively. Fix the unit vector in the third dimension as $\mathbf{e_3}= [0,0,1]^{\top}$. Given the vectors $a, b \in \mathbb{R}^{n}$ and matrices $A\in\mathbb{R}^{n\times n},B\in\mathbb{R}^{m\times m}$, let the operator ${\mathcal{T}}$ refer to the block transpose operation such that $[a,b]^{\mathcal{T}} \triangleq [a^{\top},b^{\top}]^{\top}$, define the operation $diag$ so that $diag(a) = a^{\top}I_n$, and denote the matrix $C=blkdiag(A,B)\in\mathbb{R}^{(n+m)\times(n+m)}$ as the block diagonal matrix with $A$ and $B$ diagonal elements. Let $S(\cdot)\in \mathbb{R}^{3\times 3}$ denote the skew symmetric matrix which can be used interchangeably to represent the cross product since, given $a, b \in \mathbb{R}^{3}$, $S(a)b \triangleq a \times b$.
\subsection{Koopman Theory}
To summarize Koopman operator theory, consider the general continuous-time autonomous nonlinear dynamics on a smooth $n$-dimensional manifold $\mathcal{M}$, where the evolution of the state vector $x \in \mathcal{M}\subset\mathbb{R}^n$ is represented by
$\frac{d}{{dt}}x(t) = f(x)$
with corresponding smooth Lipschitz continuous flow $F_t:\mathcal{M}\mapsto\mathcal{M}$, such that 
$F_t(x({t_0})) = x({t_0}) + \int_{{t_0}}^{{t_0} + t} {f(x(\tau ))} d\tau$,
with initial time $t_0\geq0$. The Koopman operator semigroup $\mathcal{K}_{t\geq0}$, referred to as the Koopman operator, acting on scalar function $b(x):\mathcal{M}\mapsto\mathbb{C}$, called observable, is defined on the state space $\mathcal{M}$ as
$\mathcal{K}_tb=b\circ F_t$.
The Koopman operator is, therefore, an infinite dimensional linear operator that maps the evolution of the state observables (i.e. nonlinear functions of the state variables) \cite{kaiser2021data,Bevanda,Mezic,mezic2022numerical}, embedding the original state nonlinear dynamics into a Koopman observables' linear dynamics. This new framework has recently found wide adoption in data-driven application for aerospace
\cite{Abraham2, mamakoukas2022robust,folkestad2022koopman, zheng2023optimal,manaa2024koopman} and robotics \cite{sinha2022koopman,goyal2022impedance,Bruder}. However, using a set of model-based analytically derived observables has been shown to provide better approximation of the original dynamics \cite{Chen,Zinage,Zinage2,martini2023koopman,martini2024koopman}.
\subsection{Analytically Derived Koopman Formulation of Rigid Body Dynamics}
The attitude and position rigid body dynamics are presented as a set of nonlinear differential equations \cite{Lee,siciliano2010robotics}:
\setlength{\abovedisplayskip}{0pt}   % No space before the equation
\setlength{\belowdisplayskip}{0pt}
\begin{align}\label{eq:eqmot}
\dot R = RS(\nu) \;, \quad \quad J\dot \nu = M - S(\nu)J\nu \;, \quad \quad \dot p = Rv \;, \quad \quad \dot v = \frac{1}{m}F - S(\nu)v - gR^{\top}\mathbf{e_3}
\end{align}
where $R\in \mathbb{R}^{3\times 3}$ is the rotation matrix based on the Euler angles configuration $\eta \in \mathbb{R}^{3}$; $\nu, v \in \mathbb{R}^{3}$ are the angular and linear velocity expressed in the body fixed frame; $J = diag(I_x,I_y,I_z)\in \mathbb{R}^{3\times 3}$ is the inertia tensor; $M, F \in \mathbb{R}^{3}$ are the external torque and force; $p \in \mathbb{R}^{3}$ is the position vector, $m$ is the total mass, and $g$ is the gravitational acceleration. 


In authors last work \cite{martini2024koopman}, a Koopman analytical formulation of rigid dynamics is proposed. 
This is achieved by analytically deriving a special set of Koopman observables known as generalized eigenfunction of the Koopman operator $\varkappa$, which represent an efficient embedding of the nonlinear dynamics \cite{9022864}. The resulting Koopman quadrotor quasi linear dynamics is $\dot \varkappa = A\varkappa + B(\varkappa)u \label{complete_ss}$ with six inputs $u = [T,\tau]^{\top}$, a state dependent control matrix $B(\varkappa)$ and a state matrix $A$ composed of six Jordan blocks, each related to a variable to control.
To design a control strategy, the system is rewritten as $\dot \varkappa = A\varkappa + B^{\star}U^{\star}(\varkappa)$ where $U^{\star}(\varkappa) = B(\varkappa)u$ and $B^{\star} = I_N$. Hence, the control law can be formulated on a linear time invariant (LTI) system. The control input is then computed by solving the least square optimization problem:
\begin{equation}
    \{\text{min:} (B(x)u-B^{\star}U^{\star}(x))^{\top}(B(x)u-B^{\star}U^{\star}(x))\}
\end{equation}
whose solution is $u = B^{\dagger}(x)B^{\star}U^{\star}(x)$. To guarantee the validity of such procedure it is crucial that the optimization error $B(x)u-B^{\star}U^{\star}(x)$ remains sufficiently small during operation. Since this is not always feasible, the methodology presented in this work aim at matching the number of input to the number of output to to obtain a nonsingular control matrix for which $B^{\dagger}(x)=B^{-1}(x)$.

\section{Koopman Lower Dimensional Model}
Given the authors previous work, the presented methodology is introduced by applying it to the analytically derived Koopman formualtion for rigid body dynamics, however, this approach could be generalized to any Koopman generalized eigenfunctions set, data-driven or analytically derived.
First, we exploit the property for which every linear combination of a linear operator generalized eigenfunctions is itself a generalized eigenfunction of said linear operator. Considering the six sets of Koopman generalized eigenfunctions $ \varkappa = \left( \varkappa_1, \varkappa_2, \varkappa_3, \varkappa_4,\varkappa_5,\varkappa_6 \right)$ where 
\begin{equation}
    \varkappa_{i} = (\{\varkappa_{i,k}\}^{N_{\nu}-1}_{k=0})=(\{\nu_k(j)\}^{N_{\nu}-1}_{k=0})
\end{equation} 
for $i = 1,2,3$ and $j=i$, and 
\begin{equation}
    \varkappa_{i} = (\{\varkappa_{i,k}\}^{N_{z}-1}_{k=0})=(\{z_k(j)\}^{N_{z}-1}_{k=0})
\end{equation}
for $i = 4,5,6$ and $j = i-3$, we obtain that the evolution of the infinite dimensional set of eigenfunctions $\varkappa_i$ is
\begin{subequations}
\begin{align}
        \dot \varkappa_{i,0} &= \varkappa_{i,1} + B_{i,0}(\varkappa)u\\
        \dot \varkappa_{i,1} &= \varkappa_{i,2} + B_{i,1}(\varkappa)u\\
        & \;\; \vdots  \nonumber
\end{align}
\end{subequations}
Using the above property, we create a new set of Koopman generalized eigenfunctions as 
\begin{subequations}
\begin{align}
        \underbrace{\dot{\left({k_{i,0}\varkappa_{i,0}+k_{i,1}\varkappa_{i,1}+ \cdots}\right)}}_{\dot y_{i,0}} &= \underbrace{\left({k_{i,0}\varkappa_{i,1}+k_{i,1}\varkappa_{i,2}+ \cdots}\right)}_{y_{i,1}} + \underbrace{\left({k_{i,0}B_{i,0}(\varkappa)+k_{i,1}B_{i,1}(\varkappa)+ \cdots}\right)}_{g_{i,0}}u \\
        \underbrace{\dot{\left({k_{i,0}\varkappa_{i,1}+k_{i,1}\varkappa_{i,2}+ \cdots}\right)}}_{\dot y_{i,1}} &= \underbrace{\left({k_{i,0}\varkappa_{i,2}+k_{i,1}\varkappa_{i,3}+ \cdots}\right)}_{y_{i,2}} + \underbrace{\left({k_{i,0}B_{i,1}(\varkappa)+k_{i,1}B_{i,2}(\varkappa)+ \cdots}\right)}_{g_{i,1}}u \\
        &\;\;\vdots \nonumber
\end{align}
\end{subequations}
where $y_k$ is a newly constructed Koopman generalized eigenfunction and $k_{i,j}$ is a constant coefficient with $i,j,k$ being integers so that $i \in [1,6]$ and $j,k \in [1,\infty)$.
\begin{align}
&\dot y_i = A_i y_i + G_i(\varkappa_i)u, \qquad
A_{i} = \left[\!
    \begin{array}{ccccc}
0 & 1 & 0 & \cdots & 0\\
0 & 0 & 1 & \cdots & 0 \\
\vdots & \vdots & \vdots & \ddots & \vdots
    \end{array}
    \!\right], \qquad 
G_{i}(\varkappa_i) = \left[\!
\begin{array}{c}
g_{i,0} \\
g_{i,1} \\
\vdots
\end{array}
\!\right]\nonumber\\ \label{eqn:KoopPosform1}
\end{align}
being $A_i$ a Jordan block with respect to the eigenvalue 0, $y_i$ is an infinite set Koopman generalized eigenfunctions and $\mathrm{span}\left\{y_{i,1},y_{i,2},\cdots\right\}$ is a Koopman invariant subspace.
Hence, we can create infinite sets of Koopman eigenfunction by repeating this procedure indefinitely . The resulting generalized eigenfunctions will be less and less interpretable, nevertheless, they will embed the original nonlinear dynamics.

\section{Proposed Control Methodology}
To avoid the use of the least square approximation, we propose a truncation of the newly generated set of Koopman eigenfunctions so that the number of outputs matches the number of inputs. Considering the the six input of the force and torque vector acting on the rigid body dynamics consider the following.
Truncating to the zero order and with the assumption $y_{i,1} \ll B_{i,0,y}$ we get
\begin{equation}
    \dot y_{i,0} \approx 0 \cdot y_{i,0} + g_{i,0} u 
\end{equation}
where $g_{i,0} = \sum^{\infty}_{k=0} k_{i,k} B_{i,k}(\varkappa)$. By applying this methodology to each set of the rigid body dynamics generalized eigenfunctions we obtain the vector of six generalized Koopman observables $\Upsilon = \left(y_{1,0},y_{2,0},y_{3,0},y_{4,0},y_{5,0},y_{6,0}\right)\in \mathbb{R}^{4\times 1}$, each related to one of the six Jordan blocks to control and mathcing the six control inputs ($F,\tau$). We highlight the fact that the obtained cannot be considered Koopman generalized eigenfunctions give the truncation to the zero order. However we adopt this approximation for control purpure delegating the compensation to the control algorithm.
The resulting dynamics are
\begin{equation}
    \dot \Upsilon(x) \approx 0_{4\times 4} \cdot \Upsilon(x) + G(\varkappa)u 
\end{equation}
where $G(\varkappa) = \left[ g_{1,0}, g_{2,0}, g_{3,0}, g_{4,0}, g_{5,0}, g_{6,0} \right]^{\top} \in \mathbb{R}^{6\times 6}$. The claim, which is supported by our current preliminary results, is that this lower dimensional quasi-linear formulation is still able to embed most of the original system nonlinear dynamics and can be used to formulate a control law. Having thus matched the number of outputs to the number of inputs, $G(\varkappa)$ can be made nonsingular, leading to an exact solution instead of least square approximation.
Therefore we can now perform a Koopman based exact linearization for which
\begin{equation}
    \dot \Upsilon(x) \approx G(x)G(x)^{-1}u^* = u^*\label{eqn:newKobslinear}
\end{equation}
Hence the resulting dynamics is representable as a single integrator with $u^*$ being the virtual control input to be designed and $u = )G(x)^{-1}u^* = u^*$.
The unique achievement is that the resulting formulation allows the design of any single-loop linear control law to steer the nonlinear dynamics.

\section{Quadrotor Case Study}
\begin{align}\label{eq:eqmot1}
\dot R = RS(\nu) \;, \quad \quad J\dot \nu = M - S(\nu)J\nu \;, \quad \quad \dot p = Rv \;, \quad \quad \dot v = \frac{1}{m}T\mathbf{e_3} - S(\nu)v - gR^{\top}\mathbf{e_3}
\end{align}


From \cite{martini2024koopman_rigid}, the resulting system achieves a quasi-linear form which does not feature underactuation.
To achieve a sufficiently rich embedding of the nonlinear quadrotor dynamics, system dimension of $N = 16$ is selected. Hence, the need to resort to the least square approximation method to design the control law. Despite previous successful results, the associated high dimensionality intensifies the controller tuning process. Moreover, the use of least square approximation increases the approximation error, already caused by the Koopman formulation truncation.

By applying the proposed methodology to each set of the quadrotor generalized eigenfunctions we obtain the vector of four generalized Koopman observables $\Upsilon = \left(y_{1,0},y_{4,0},y_{5,0},y_{6,0}\right)\in \mathbb{R}^{4\times 1}$, each related to one of the four Jordan blocks to control
($y_{2,0},y_{3,0}$ are discarded since the roll and pitch angular velocity dynamics are embedded within the other generalized eigenfunctions).
% Note that, the Jordan blocks related to the roll and pitch angular velocity can be discarded since their dynamics is coupled with the other Koopman generalized eigenfunctions. 
The resulting dynamics are
\begin{equation}
    \dot \Upsilon(x) \approx 0_{4\times 4} \cdot \Upsilon(x) + G(\varkappa)u 
\end{equation}
where $G(\varkappa) = \left[ g_{1,0}, g_{4,0}, g_{5,0}, g_{6,0} \right]^{\top} \in \mathbb{R}^{4\times 4}$. The claim, which is supported by our current preliminary results, is that this lower dimensional quasi-linear formulation is still able to embed most of the original system nonlinear dynamics and can be used to formulate a control law for the quadrotor UAV. Having thus reduced the number of outputs to the number of inputs, $G(\varkappa)$ can be made nonsingular, leading to an exact solution instead of least square approximation. The unique achievement is that the resulting formulation allows the design of any single-loop linear control law to steer the nonlinear dynamics of the quadrotor.\\
Using \eqref{eqn:newKobslinear} we proceed by designing a Finite time LQ controller in terms of the Koopman error, $e_{\Upsilon}(x) = \Upsilon_d(x_d) - \Upsilon(x)$ which dynamics are 
\begin{equation}
    \dot e_{\Upsilon}(x) = -u^* + u^*_d = U^* \label{eqn:newKerrorlinear}
\end{equation}
The LQ control is designed to minimize the following cost function
\begin{equation}
    J = \frac{1}{2}\int_0^{t_f}  {({e_{\Upsilon}^{\top}}Qe_{\Upsilon} + U^{*T}RU^*)dt} 
\end{equation}
It follows that the resulting control input is computed as 
\begin{equation}
    u = G(x)^{-1}\left( R^{-1}Ke_{\Upsilon} + u^*_d \right)
\end{equation}
where $K$ is the solution of the algebric Riccati differential equation
\begin{equation}
    0 = -Q + KR^{-1}K
\end{equation}
The resulting control law is adopted fo the trajectory tracking task of a square trajectory and tested in hardware in the loop (HIL) experiments.
The KLQ controller is deployed on pixhawk 2.1 cube black while the quadrotor dynamics is simulated using Jmavsim simulator using a similar setup to \cite{Martini2024EMRAC,sonmez2025Reinforcement}. 
As shown in figure \ref{fig:3D} the controller is able to keep up with real-time computation and achieve satisfactory trajectory tracking.
\begin{figure}[h]
    \centering
    \includegraphics[width=0.5\linewidth]{Control_sim_HIL_3D.png}
    \caption{HIL 3D Trajectory}
    \label{fig:3D}
\end{figure}
\section{Conclusions}
In this work, we presented a methodology to achieve a low dimensional Koopman representation based on Koopman generalized eigenfunctions. The novel formulation is designed to match the number of input to the number of output in order to avoid least square approximation when performing Koopman based exact linearization. The obtained formulation, when applied to the rigid body dynamics leads to a quasi-linear Koopman based model, which can be exactly linearized resulting into a single integrator dynamics. Practically, any linear controller can be designed in the presented formulation. When applied to the quadrotor case study, the resulting quasi-linear model is fully-actuated and it is used to design a single loop LQ flight controller. The presented hardware in the loop results showcase the methodology implementability to real-time applications in presence of noise and delays. The unique results achieved in this work highlight the practical advantages of Koopman based strategies posing them as viable solution to the underactuation problem.
% \bibliographystyle{unsrtnat}
% \bibliography{Bibliografia} 

\begin{thebibliography}{23}
\providecommand{\natexlab}[1]{#1}
\providecommand{\url}[1]{\texttt{#1}}
\expandafter\ifx\csname urlstyle\endcsname\relax
  \providecommand{\doi}[1]{doi: #1}\else
  \providecommand{\doi}{doi: \begingroup \urlstyle{rm}\Url}\fi

\bibitem[Kaiser et~al.(2021)Kaiser, Kutz, and Brunton]{kaiser2021data}
Eurika Kaiser, J~Nathan Kutz, and Steven~L Brunton.
\newblock Data-driven discovery of {K}oopman eigenfunctions for control.
\newblock \emph{Machine Learning: Science and Technology}, 2\penalty0 (3):\penalty0 035023, 2021.

\bibitem[Bevanda et~al.(2021)Bevanda, Sosnowski, and Hirche]{Bevanda}
Petar Bevanda, Stefan Sosnowski, and Sandra Hirche.
\newblock {K}oopman operator dynamical models: Learning, analysis and control.
\newblock \emph{Annual Reviews in Control}, 52:\penalty0 197--212, 2021.

\bibitem[Budi{\v{s}}i{\'c} et~al.(2012)Budi{\v{s}}i{\'c}, Mohr, and Mezi{\'c}]{Mezic}
Marko Budi{\v{s}}i{\'c}, Ryan Mohr, and Igor Mezi{\'c}.
\newblock Applied koopmanism.
\newblock \emph{Chaos: An Interdisciplinary Journal of Nonlinear Science}, 22\penalty0 (4):\penalty0 047510, 2012.

\bibitem[Mezi{\'c}(2022)]{mezic2022numerical}
Igor Mezi{\'c}.
\newblock On numerical approximations of the {K}oopman operator.
\newblock \emph{Mathematics}, 10\penalty0 (7):\penalty0 1180, 2022.

\bibitem[Abraham and Murphey(2019)]{Abraham2}
Ian Abraham and Todd~D Murphey.
\newblock Active learning of dynamics for data-driven control using {K}oopman operators.
\newblock \emph{IEEE Transactions on Robotics}, 35\penalty0 (5):\penalty0 1071--1083, 2019.

\bibitem[Mamakoukas et~al.(2022)Mamakoukas, Di~Cairano, and Vinod]{mamakoukas2022robust}
Giorgos Mamakoukas, Stefano Di~Cairano, and Abraham~P Vinod.
\newblock Robust model predictive control with data-driven {K}oopman operators.
\newblock In \emph{2022 American Control Conference (ACC)}, pages 3885--3892. IEEE, 2022.

\bibitem[Folkestad(2022)]{folkestad2022koopman}
Carl~AA Folkestad.
\newblock \emph{{K}oopman-based Learning and Control of Agile Robotic Systems}.
\newblock California Institute of Technology, 2022.

\bibitem[Zheng et~al.(2023)Zheng, Huang, and Fettweis]{zheng2023optimal}
Ketong Zheng, Peng Huang, and Gerhard~P Fettweis.
\newblock Optimal control of quadrotor attitude system using data-driven approximation of {K}oopman operator.
\newblock \emph{IFAC-PapersOnLine}, 56\penalty0 (2):\penalty0 834--840, 2023.

\bibitem[Manaa et~al.(2024)Manaa, Abdallah, Abido, and Ali]{manaa2024koopman}
Zeyad~M Manaa, Ayman~M Abdallah, Mohammad~A Abido, and Syed S~Azhar Ali.
\newblock {K}oopman-{LQR} controller for quadrotor {UAV}s from data.
\newblock \emph{arXiv preprint arXiv:2406.17973}, 2024.

\bibitem[Sinha and Wang(2022)]{sinha2022koopman}
Anirban Sinha and Yue Wang.
\newblock {K}oopman operator--based knowledge-guided reinforcement learning for safe human--robot interaction.
\newblock \emph{Frontiers in Robotics and AI}, 9:\penalty0 779194, 2022.

\bibitem[Goyal et~al.(2022)Goyal, Hussain, Martinez-Marroquin, Brown, and Jamwal]{goyal2022impedance}
Tanishka Goyal, Shahid Hussain, Elisa Martinez-Marroquin, Nicholas~AT Brown, and Prashant~K Jamwal.
\newblock Impedance control of a wrist rehabilitation robot based on autodidact stiffness learning.
\newblock \emph{IEEE Transactions on Medical Robotics and Bionics}, 4\penalty0 (3):\penalty0 796--806, 2022.

\bibitem[Bruder et~al.(2019)Bruder, Gillespie, Remy, and Vasudevan]{Bruder}
Daniel Bruder, Brent Gillespie, C~David Remy, and Ram Vasudevan.
\newblock Modeling and control of soft robots using the {K}oopman operator and model predictive control.
\newblock \emph{arXiv preprint arXiv:1902.02827}, 2019.

\bibitem[Chen et~al.(2022)Chen, Shan, and Wen]{Chen}
Ti~Chen, Jinjun Shan, and Hao Wen.
\newblock {K}oopman-operator-based attitude dynamics and control on so (3).
\newblock In \emph{Distributed Attitude Consensus of Multiple Flexible Spacecraft}, pages 177--210. Springer, 2022.

\bibitem[Zinage and Bakolas(2021)]{Zinage}
Vrushabh Zinage and Efstathios Bakolas.
\newblock {K}oopman operator based modeling for quadrotor control on se (3).
\newblock \emph{IEEE Control Systems Letters}, 6:\penalty0 752--757, 2021.

\bibitem[Zinage and Bakolas(2022)]{Zinage2}
Vrushabh Zinage and Efstathios Bakolas.
\newblock {K}oopman operator based modeling and control of rigid body motion represented by dual quaternions.
\newblock In \emph{2022 American Control Conference (ACC)}, pages 3997--4002. IEEE, 2022.

\bibitem[Martini et~al.(2023)Martini, Rizzo, Stefanovic, Livreri, Rutherford, and Valavanis]{martini2023koopman}
Simone Martini, Alessandro Rizzo, Margareta Stefanovic, Patrizia Livreri, Matthew~J Rutherford, and Kimon~P Valavanis.
\newblock {K}oopman operator based modeling and control of quadrotors.
\newblock In \emph{IUTAM Symposium on Optimal Guidance and Control for Autonomous Systems}, pages 253--266. Springer, 2023.

\bibitem[Martini et~al.(2024{\natexlab{a}})Martini, Valavanis, Stefanovic, Rizzo, and Rutherford]{martini2024koopman}
Simone Martini, Kimon~P Valavanis, Margareta Stefanovic, Alessandro Rizzo, and Matthew~J Rutherford.
\newblock {K}oopman-based reduced order controller design for quadrotors.
\newblock In \emph{2024 International Conference on Unmanned Aircraft Systems (ICUAS)}, pages 370--375. IEEE, 2024{\natexlab{a}}.

\bibitem[Lee et~al.(2010)Lee, Leok, and McClamroch]{Lee}
Taeyoung Lee, Melvin Leok, and N~Harris McClamroch.
\newblock Geometric tracking control of a quadrotor {UAV} on {SE} (3).
\newblock In \emph{49th IEEE conference on decision and control (CDC)}, pages 5420--5425. IEEE, 2010.

\bibitem[Siciliano et~al.(2010)Siciliano, Sciavicco, Villani, and Oriolo]{siciliano2010robotics}
Bruno Siciliano, Lorenzo Sciavicco, Luigi Villani, and Giuseppe Oriolo.
\newblock \emph{Robotics: modelling, planning and control}.
\newblock Springer Science \& Business Media, 2010.

\bibitem[Korda and Mezić(2020)]{9022864}
Milan Korda and Igor Mezić.
\newblock Optimal construction of {K}oopman eigenfunctions for prediction and control.
\newblock \emph{IEEE Transactions on Automatic Control}, 65\penalty0 (12):\penalty0 5114--5129, 2020.
\newblock \doi{10.1109/TAC.2020.2978039}.

\bibitem[Martini et~al.(2024{\natexlab{b}})Martini, Valavanis, and Stefanovic]{martini2024koopman_rigid}
Simone Martini, Kimon~P Valavanis, and Margareta Stefanovic.
\newblock Koopman analytical modeling of position and attitude dynamics: a case study for quadrotor control.
\newblock \emph{arXiv preprint arXiv:2407.16100}, 2024{\natexlab{b}}.

\bibitem[Martini et~al.(2024{\natexlab{c}})Martini, Mennea, Mihalkov, Rizzo, Valavanis, Sorniotti, and Montanaro]{Martini2024EMRAC}
Simone Martini, Salvatore~M Mennea, M~Mihalkov, Alessandro Rizzo, Kimon~P Valavanis, Aldo Sorniotti, and Umberto Montanaro.
\newblock Design and {HIL} testing of enhanced {MRAC} algorithms to improve tracking performance of {LQ}-strategies for quadrotor {UAV}s.
\newblock In \emph{2024 IEEE 20th International Conference on Automation Science and Engineering (CASE)}. IEEE, 2024{\natexlab{c}}.

\bibitem[Sönmez et~al.(2025)Sönmez, Montecchio, Martini, Rutherford, Rizzo, and Valavanis]{sonmez2025Reinforcement}
Serhat Sönmez, Luca Montecchio, Simone Martini, Matthew~J Rutherford, Margareta Rizzo, Alessandro~andStefanovic, and Kimon~P Valavanis.
\newblock Reinforcement learning based prediction of pid controller gains for quadrotor uavs.
\newblock \emph{arXiv preprint}, 2025.

\end{thebibliography}

\end{document}
