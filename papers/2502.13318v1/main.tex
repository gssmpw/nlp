%%%%%%%%%%%%%%%%%%%%%%% file template.tex %%%%%%%%%%%%%%%%%%%%%%%%%
%
% This is a general template file for the LaTeX package SVJour3
% for Springer journals.          Springer Heidelberg 2010/09/16
%
% Copy it to a new file with a new name and use it as the basis
% for your article. Delete % signs as needed.
%
% This template includes a few options for different layouts and
% content for various journals. Please consult a previous issue of
% your journal as needed.
%
%%%%%%%%%%%%%%%%%%%%%%%%%%%%%%%%%%%%%%%%%%%%%%%%%%%%%%%%%%%%%%%%%%%
%
% First comes an example EPS file -- just ignore it and
% proceed on the \documentclass line
% your LaTeX will extract the file if required
\begin{filecontents*}{example.eps}
%!PS-Adobe-3.0 EPSF-3.0
%%BoundingBox: 19 19 221 221
%%CreationDate: Mon Sep 29 1997
%%Creator: programmed by hand (JK)
%%EndComments
gsave
newpath
  20 20 moveto
  20 220 lineto
  220 220 lineto
  220 20 lineto
closepath
2 setlinewidth
gsave
  .4 setgray fill
grestore
stroke
grestore
\end{filecontents*}
%
\RequirePackage{fix-cm}
%
%\documentclass{svjour3}                     % onecolumn (standard format)
%\documentclass[smallcondensed]{svjour3}     % onecolumn (ditto)
%\documentclass[smallextended]{svjour3}       % onecolumn (second format)
\documentclass[twocolumn]{svjour3}          % twocolumn
%
\smartqed  % flush right qed marks, e.g. at end of proof
%
\usepackage{graphicx}
\usepackage{makecell}
\usepackage{amsmath,amssymb,amsfonts}
\usepackage{algorithmic}
\usepackage{graphicx}
\usepackage{textcomp}
\usepackage{xcolor}
\usepackage[title]{appendix}
\usepackage{array,multirow}
\usepackage[font=small]{caption}
\usepackage{balance}  % for  \balance command ON LAST PAGE  (only there!)

\usepackage{color, colortbl}
\definecolor{Gray}{gray}{0.8}
\definecolor{lGray}{gray}{0.9}
\usepackage{multicol}

\usepackage{enumitem}

% THIS COMMAND WILL SHRINK PAPER BY 2% BY REMOVING UNNEEDED SPACES!
%\renewcommand{\baselinestretch}{0.}

%EXTRA PACKAGE
\DeclareMathOperator*{\argmax2}{arg\,max}
\DeclareMathOperator*{\argmin2}{arg\,min}
\usepackage[linesnumbered,algoruled,boxed,lined,noend]{algorithm2e}
\newcommand\mycommfont[1]{\footnotesize\ttfamily\textcolor{blue}{#1}}
\SetCommentSty{mycommfont}


\newcommand{\commentRedXMARK}[1]{{\color[rgb]{1,0,0}#1}}


\newcommand{\commentBlue}[1]{{\color[rgb]{0,0,1}#1}}
\newcommand{\midtilde}{\raisebox{-0.25\baselineskip}{\textasciitilde}}
\usepackage{comment}
\usepackage{url}
%\newtheorem{problem}{Problem}
%\newtheorem{theorem}{Theorem}
%\newtheorem{definition}{Definition}
%\newtheorem{example}{Example}
%\newtheorem{lemma}{Lemma}
\newcommand{\anomaly}{{\textit{anomaly} }}
\newcommand{\discord}{{\textit{discord} }}
\newcommand{\outliers}{{\textit{outliers} }}
\newcommand{\Anomaly}{{\textit{Anomaly} }}
\newcommand{\Discord}{{\textit{Discord} }}
\newcommand{\Outliers}{{\textit{Outliers} }}





% themis red comment
\newcommand{\tp}[1]{{\color{red} {\bf ??? #1 ???}}\normalcolor}







\usepackage{tikz}
\newcommand{\commentRed}[1]{{\color[rgb]{0,0,0}~#1}}
\newcommand{\greencheck}{}%
\DeclareRobustCommand{\greencheck}{%
\textbf{
  \tikz\fill[scale=0.4, color=green]
  (0,.35) -- (.25,0) -- (1,.7) -- (.25,.15) -- cycle;%
}}
\usepackage{pifont}
\newcommand{\xmark}{\text{\ding{55}}}
\newcommand{\redmark}{\text{\commentRedXMARK{\xmark}}}%
%
% \usepackage{mathptmx}      % use Times fonts if available on your TeX system
%
% insert here the call for the packages your document requires
%\usepackage{latexsym}
% etc.
%
% please place your own definitions here and don't use \def but
% \newcommand{}{}
%
% Insert the name of "your journal" with
% \journalname{myjournal}
%
\begin{document}

\title{VUS: Effective and Efficient Accuracy Measures for Time-Series Anomaly Detection%\thanks{Grants or other notes
%about the article that should go on the front page should be
%placed here. General acknowledgments should be placed at the end of the article.}
}
%\subtitle{Do you have a subtitle?\\ If so, write it here}

%\titlerunning{Short form of title}        % if too long for running head

\author{Paul Boniol \and
        Ashwin K. Krishna \and
        Marine Bruel \and
        Qinghua Liu \and 
        Mingyi Huang \and 
        Themis Palpanas \and
        Ruey S. Tsay \and
        Aaron Elmore \and
        Michael J. Franklin \and
        John Paparrizos \and%etc.
}

%\authorrunning{Short form of author list} % if too long for running head

\institute{Paul Boniol \at
              45 rue d'Ulm, 75005, Paris \\
              \email{boniol.paul@gmail.com}
}

\date{Received: date / Accepted: date}
% The correct dates will be entered by the editor


\maketitle

\begin{abstract}
Anomaly detection (AD) is a fundamental task for time-series analytics with important implications for the downstream performance of many applications. In contrast to other domains where AD mainly focuses on point-based anomalies (i.e., outliers in standalone observations), AD for time series is also concerned with range-based anomalies (i.e., outliers spanning multiple observations). Nevertheless, it is common to use traditional point-based information retrieval measures, such as Precision, Recall, and F-score, to assess the quality of methods by thresholding the anomaly score to mark each point as an anomaly or not. However, mapping discrete labels into continuous data introduces unavoidable shortcomings, complicating the evaluation of range-based anomalies. Notably, the choice of evaluation measure may significantly bias the experimental outcome. Despite over six decades of attention, there has never been a large-scale systematic quantitative and qualitative analysis of time-series AD evaluation measures. This paper extensively evaluates quality measures for time-series AD to assess their robustness under noise, misalignments, and different anomaly cardinality ratios. Our results indicate that measures producing quality values independently of a threshold (i.e., AUC-ROC and AUC-PR) are more suitable for time-series AD. Motivated by this observation, we first extend the AUC-based measures to account for range-based anomalies. Then, we introduce a new family of parameter-free and threshold-independent measures, Volume Under the Surface (VUS), to evaluate methods while varying parameters. We also introduce two optimized implementations for VUS that reduce significantly the execution time of the initial implementation. Our findings demonstrate that our four measures are significantly more robust in assessing the quality of time-series AD methods.
%\keywords{Time Series \and Anomaly Detection \and Evaluation Measures}
% \PACS{PACS code1 \and PACS code2 \and more}
% \subclass{MSC code1 \and MSC code2 \and more}
\end{abstract}



% 
% 
The widespread integration of communication networks and smart devices in modern control systems has increased the vulnerability of industrial systems to online cyber-attacks, e.g., Industroyer, Blackenergy, etc \citep{osti_1505628}.
% Modern control systems have seen a large push to include communication networks and smart devices to increase performance, made possible by improvements in communication device cost and energy consumption. This trend has been coupled with the usage of open-standard communication protocols among industrial control systems, making them vulnerable to online cyber-attacks such as Industroyer, Blackenergy, etc \citep{osti_1505628}. 
To counter this, methods have been developed to improve security by achieving attack detection, mitigation, and monitoring, among others \citep{sandberg2022secure}. This paper focuses on active attack diagnosis to mitigate stealthy attacks. 
%
%\subsection{Literature review}

Active diagnosis techniques rely on the inclusion of additional moduli to control systems
% inclusion within the control system of additional moduli 
to alter the behavior of the system compared to information known by the attacker. 
For instance, the concept of additive watermarking was introduced in \cite{mo2015physical}, where noise signals of known mean and variance are added at the plant and compensated for it at the controller. 
This compensation, however, is not exact, causing some performance degradation. Thus, trade-offs between performance and detectability  are necessary \citep{zhu2023detection}.
% A later work \citep{zhu2023detection} designs the watermark signal by trading performance for detection. Thus, although additive watermarking serves as a good detection scheme, they endure performance losses even in the nominal case. 

In encrypted control \citep{darup2021encrypted}, the sensor data is encrypted, sent to the controller, and then operated on directly. Encrypted input signals are sent back to the plant for decryption. Although encryption is widespread in IT security, in control systems it presents some concerns, such as the introduction of time delays \citep{stabile2024verifiable}, while it may present inherent weaknesses \citep{alisic2023model}.
% they are not preferred as they introduce time delays \citep{stabile2024verifiable} which can cause instability, and some encryption schemes can be very weak  \citep{alisic2023model}. 

In moving target defense \citep{griffioen2020moving}, the plant is augmented with fictitious dynamics, known to the controller. The plant output is transmitted to the controller along with the fictitious states over a network under attack. 
The additional measurements then aide in the detection of attacks. 
This comes at the cost of higher communication bandwidth needs, which increases rapidly with the dimension of the augmented systems.
% Since the dynamics of the fictitious dynamics are exactly known to the controller, the attack is detected easily. However, when the scale of the system increases, the communication bandwidth used by moving the target defense approach increases rapidly. 

Other recently proposed works include two-way coding \citep{fang2019two}, a weak encryuption technique, and dynamic masking \citep{abdalmoaty2023privacy}, which enhances privacy as well as security, have been shown to be effective against zero-dynamics attacks.
% Two-way coding \citep{fang2019two} and dynamic masking \citep{abdalmoaty2023privacy} are other recently proposed approaches. Two-way coding is another form of weak encryption technique whilst dynamic masking proposes an architecture that enhances both privacy and security. These schemes are shown to be effective against zero dynamics attacks but remain to be studied for other classes of attacks. 
% Recent extensions include \citep{mukherjee2021secure,ramos2024privacy}.
% Some other works which are related are \citep{mukherjee2021secure}, an extension of \cite{fang2019two}. The work \citep{ramos2024privacy} is an extension of moving target defense for multi-agent systems. 
Furthermore, filtering techniques for attack detection are proposed by \cite{murguia2020security,hashemi2022codesign,escudero2023safety}, while not focusing on stealthy attacks.
% The works \citep{murguia2020security,hashemi2022codesign,escudero2023safety} develop filtering techniques to guarantee safety, without being focused on stealthy covert attacks.

Multiplicative watermarking (mWM) has been proposed by the authors as a diagnosis technique \citep{ferrari2020switching}. mWM consists of a pair of filters on each communication channel between the plant and its controller; the scheme is affine to weak encryption, whereby ``encoding'' and ``decoding'' are done by changing signals' dynamic characteristics through inverse pairs of filters. This enables original signals to be recovered exactly, and thus does not lead to performance degradation.
% A multiplicative watermark is an affine to a weak encryption technique, through which the signal is ``encoded'' by a filter, changing its dynamic behavior. The use of inverse pairs means that the original signal can be recovered, through ``decoding'' via an inverse filter. As such, differently to techniques based on additive watermarking, no performance is lost due to the injection of noise, and there are no bandwidth limitations.

%\subsection{Contributions}
One of the critical features of multiplicative watermarking is that to detect stealthy attacks, the mWM filter parameters must be switched over time. In this paper, an algorithm to optimally design the mWM parameters after a switching event is presented, enhancing detection performance, without changing the switching time.
% This is done without changing the switching time, which is taken as given.

\textcolor{black}{
To formalize the filter design problem, we suppose the defender is interested in optimal performance against adversaries injecting covert attacks with matched system parameters \citep{smith2015covert}, including the mWM parameters prior to the switch. This scenario represents a worst case where malicious agents can take full control of the system while remaining undetected.
Thus, the attack strategy is explicitly included within the formulation of the closed-loop system, and the mWM filters are chosen by solving an optimization problem minimizing the attack-energy-constrained output-to-output gain (AEC-OOG) \citep{anand2023risk}, a variation of the output-to-output gain proposed in  \cite{teixeira2015strategic}.
}
The main contributions of this paper are:
% We consider an adversary injecting a covert attack with matched system parameters \citep{smith2015covert}, i.e., an attacker with full knowledge of the control system parameters, including those of the mWM filters before the switch. This scenario is taken as a worst case, as it has been shown that this class of attacks can be made stealthy. To quantitatively define a cost, the output-to-output gain (OOG) \citep{teixeira2015strategic} is leveraged,
% a metric introduced to evaluate the impact of an additive attack in a control system. %Specifically, OOG evaluates the worst-case performance loss that an attacker injecting an undetectable attack can obtain. 
% Here, the maximum performance loss caused by a stealthy adversary with limited energy is taken, the attack-energy-constrained OOG (AEC-OOG) \citep{anand2023risk}. The main contributions of this paper are:
\begin{enumerate}
%[label=\alph*.]
\item The problem of optimally designing the switching mWM filters is formulated as an optimization problem, with the AEC-OOG is taken as the objective;%where the AEC-OOG is taken as the impact metric; 
\item The worst-case scenario of a covert attack with exact knowledge of plant and mWM filter parameters is embedded within the design problem;
% The optimization problem is defined to incorporate the worst-case scenario of a covert attack with exact knowledge of plant and mWM filter parameters;
\item The feasibility of the optimization problem is shown to be dependent only on stability conditions; 
\item A solution scheme is proposed to promote randomization of the mWM filter parameters such that an eavesdropping adversary cannot remain stealthy.
\end{enumerate} 

This builds on the results of \cite{ferrari2020switching}, where the focus was on the design of the switching protocols, rather than the parameters themselves.
Compared to previous work \citep{gallo2021design}, this paper introduces an optimization problem which is always feasible (thanks to the use of AEC-OOG in the objective), while also considering a more sophisticated class of covert attacks, where the presence of watermark is known to the adversary. 
Moreover, this paper poses a different objective than \citep{zhang2023hybrid}; indeed, while \citep{zhang2023hybrid} provided a design strategy to ensure certain privacy properties, in this paper we address the problem of optimal parameter design following a switching event.


%\subsection{Organization}
The rest of the paper is organized as follows. 
After formulating the problem in Section~\ref{sec:PF}, we propose our design algorithm in Section~\ref{sec:main}, and analyze its properties. It is then evaluated through a numerical example in Section~\ref{sec:NE}, and concluding remarks are given Section~\ref{sec:Con}.
% We provide the problem background in Section~\ref{sec:PF}. We formulate the design problem in Section~\ref{sec:main}, together with an analysis of its properties. The proposed algorithm is evaluated through a numerical example in Section \ref{sec:NE}. Concluding remarks are offered in Section \ref{sec:Con}.
\section{Background and Related Work}
\label{sec:background}

We first introduce formal notations useful for the rest of the paper (Section \ref{sec:notation}). Then, we review in detail previously proposed evaluation measures for time-series AD methods (Section \ref{acc_measure}). 


\subsection{Time-Series and Anomaly Score Notations}
\label{sec:notation}
We review notations for the time series and anomaly score sequence.
\newline \textbf{Time Series: } A time series $T \in \mathbb{R}^n $ is a sequence of real-valued numbers $T_i\in\mathbb{R}$ $[T_1,T_2,...,T_n]$, where $n=|T|$ is the length of $T$, and $T_i$ is the $i^{th}$ point of $T$. We are typically interested in local regions of the time series, known as subsequences. A subsequence $T_{i,\ell} \in \mathbb{R}^\ell$ of a time series $T$ is a continuous subset of the values of $T$ of length $\ell$ starting at position $i$. Formally, $T_{i,\ell} = [T_i, T_{i+1},...,T_{i+\ell-1}]$.	
\newline \textbf{Anomaly Score Sequence: } For a time series $T \in \mathbb{R}^n $, an AD method $A$ returns an anomaly score sequence $S_T$. For point-based approaches (i.e., methods that return a score for each point of $T$), we have $S_T \in \mathbb{R}^n$. For range-based approaches (i.e., methods that return a score for each subsequence of a given length $\ell$), we have $S_T \in \mathbb{R}^{n-\ell}$. Overall, for range-based (or subsequence-based) approaches, we define $S_T = [{S_T}_1,{S_T}_2,...,{S_T}_{n-\ell}]$ with ${S_T}_i \in [0,1]$.


\subsection{Accuracy Evaluation Measures for AD}
\label{acc_measure}

We present previously proposed quality measures for evaluating the accuracy of an AD method, given its anomaly score. We first discuss threshold-based and then threshold-independent measures.

\subsubsection{Threshold-based AD Evaluation Measures} \hfill\\
The anomaly score $S_T$ produced by an AD method $A$ highlights the parts of the time series $T$ considered as abnormal. The highest values in the anomaly score correspond to the most abnormal points. Threshold-based measures require setting a threshold to mark each point as an anomaly or not. Usually, this threshold is set to $\mu(S_T) + \alpha*\sigma(S_T)$, with $\alpha$ set to 3~\cite{statisticaloutliers}, where $\mu(S_T)$ is the mean and $\sigma(S_T)$ is the standard deviation $S_T$. Given a threshold $Thres$, we compute the $pred \in \{0,1\}^n$ as follows:


\begin{equation}
\begin{split}
&\forall i \in [1,|S_T|], pred_i = \left.
\begin{cases}
0,& \text{if: } {S_T}_i < Thres \\
1,& \text{if: } {S_T}_i \geq Thres 
\end{cases}
\right.
\end{split}
\end{equation}

Threshold-based measures compare $pred$ to $label \in \{0,1\}^n$, which indicates the true (human provided) labeled anomalies. Given the identity vector $I=[1,1,...,1]$, the points detected as anomalies or not fall into the following four categories:
\begin{itemize}[noitemsep,topsep=0pt,parsep=0pt,partopsep=0pt,leftmargin=0.5cm]
	\item {\bf True Positive (TP)}: Number of points that have been correctly identified as anomalies. Formally: $TP = label^\top \cdot pred$.
	\item {\bf True Negative (TN)}: Number of points that have been correctly identified as normal. Formally: $TN = (I-label)^\top \cdot (I-pred)$.
	\item {\bf False Positive (FP)}: Number of points that have been wrongly identified as anomalies. Formally: $FP = (I-label)^\top \cdot pred$.
	\item {\bf False Negative (FN)}: Number of points that have been wrongly identified as normal. Formally: $FN = label^\top \cdot (I-pred)$.
\end{itemize}
Given these categories, several quality measures have been proposed to assess the accuracy of AD methods.
\newline \textbf{Precision: } We define Precision (or positive predictive value) as the number of correctly identified anomalies over the total number of points detected as anomalies by the method:

\begin{equation}
Precision = \frac{TP}{TP+FP}
\end{equation}
\newline \textbf{Recall: } We define Recall (or True Positive Rate (TPR), $tpr$) as the number of correctly identified anomalies over all anomalies:

\begin{equation}
Recall = \frac{TP}{TP+FN}
\end{equation}

\noindent \textbf{False Positive Rate (FPR): } A supplemental measure to the Recall is the FPR, $fpr$, defined as the number of points wrongly identified as anomalies over the total number of normal points:

\begin{equation}
fpr = \frac{FP}{FP+TN}
\end{equation}
\newline \textbf{F-Score: } Precision and Recall evaluate two different aspects of the AD quality. A measure that combines these two aspects is the harmonic mean $F_{\beta}$, with non-negative real values for $\beta$:
\begin{equation}
F_{\beta} = \frac{(1+\beta^2)*Precision*Recall}{\beta^2*Precision+Recall}
\end{equation}
\noindent Usually, $\beta$ is set to 1, balancing the importance between Precision and Recall. In this paper, $F_1$ is referred to as F or F-score.
\newline \textbf{Precision@k: } All previous measures require an anomaly score threshold to be computed. An alternative approach is to measure the Precision using a subset of anomalies corresponding to the $k$ highest value in the anomaly score $S_T$. This is equivalent to setting the threshold such that only the $k$ highest values are retrieved. 

To address the shortcomings of the point-based measures, a range-based definition was proposed, extending the traditional Precision and Recall \cite{tatbul_precision_2018}. This definition considers several factors: (i) whether a subsequence is detected or not (ExistenceReward or ER); (ii) how many points in the subsequence are detected (OverlapReward or OR); (iii) which part of the subsequence is detected (position-dependent weight function); and (iv) how many fragmented regions correspond to one real subsequence outlier (CardinalityFactor or CF). Formally, we define $R=\{R_1,...R_{N_r}\}$ as the set of anomaly ranges, with $R_k=\{pos_i,pos_{i+1}, ..., pos_{i+j}\}$ and $\forall pos \in R_k, label_{pos} = 1$, and $P=\{P_1,...P_{N_p}\}$ as the set of predicted anomaly ranges, with $P_k=\{pos_i,pos_{i+1}, ..., pos_{i+j}\}$ and $\forall pos \in R_k, pred_{pos} = 1$. Then, we define ER, OR, and CF as follows:

%\vspace{-0.3cm}
\begin{equation}
\footnotesize{
\begin{split}
&ER(R_i,P) = \left.
\begin{cases}
1, &\text{if } \sum_{j=1}^{N_p} |R_i \cap P_j| \geq 1\\
0, &\text{otherwise}
\end{cases}
\right. \\
&CF(R_i,P) = \left.
\begin{cases}
1, &\text{if } \exists P_i \in P, |R_i \cap P_i| \geq 1\\
\gamma(R_i,P), &\text{otherwise}
\end{cases}
\right. \\
&OR(R_i,P) = CF(R_i,P)*\sum_{j=1}^{N_p} \omega(R_i,R_i \cap P_j, \delta)
\end{split}
}
%\vspace{-0.1cm}
\end{equation}

\noindent The $\gamma(),\omega()$, and $\delta()$ are tunable functions that capture the cardinality, size, and position of the overlap respectively.
The default parameters are set to $\gamma()=1,\delta()=1$ and $\omega()$ to the overlap ratio covered by the predicted anomaly range~\cite{tatbul_precision_2018}.
\newline \textbf{Rprecision and Rrecall~\cite{tatbul_precision_2018}: } Based on the above, we define:

%\vspace{-0.2cm}
\begin{equation}
\footnotesize{
\begin{split}
Rprecision(R,P) &= \frac{\sum_{i=1}^{N_p} Rprecision_s(R,P_i)}{N_p}\\
Rprecision_s(R,P_i) &= CF(P_i,R)*\sum_{j=1}^{N_r} \omega(P_i,P_i \cap R_j, \delta)
\end{split}
}
%\vspace{-0.2cm}
\end{equation}

%%\vspace{-0.2cm}
\begin{equation}
\footnotesize{
\begin{split}
Rrecall(R,P) &= \frac{\sum_{i=1}^{N_r} Rrecall_s(R_i,P)}{N_r} \\
Rrecall_s(R_i,P) &= \alpha*ER(R_i,P) + (1-\alpha)*OR(R_i,P)
\end{split}
}
%%\vspace{-0.1cm}
\end{equation}
\noindent The parameter $\alpha$ is user defined. The default value is $\alpha=0$.
\newline \textbf{Range F-score (RF)~\cite{tatbul_precision_2018}: } As described previously, the F-score combines Precision and Recall. Similarly, we define $RF_{\beta}$, for $\beta>0$ as follows:

%\vspace{-0.3cm}
\begin{equation}
RF_{\beta} = \frac{(1+\beta^2)*Rprecision*Rrecall}{\beta^2*Rprecision+Rrecall}
%\vspace{-0.1cm}
\end{equation}

\noindent As before, $\beta$ is set to 1. In this paper, $RF_1$ is referred to as RF-score.

\subsubsection{Threshold-independent AD Evaluation Measures} \hfill\\
Until now, we introduced accuracy measures requiring to threshold the produced anomaly score of AD methods. However, the accuracy values vary significantly when the threshold changes. To evaluate a method holistically using its corresponding anomaly score, two measures from the AUC family of measures are used.
\newline \textbf{AUC-ROC~\cite{FAWCETT2006861}: } The Area Under the Receiver Operating Characteristics curve (AUC-ROC) is defined as the area under the curve corresponding to TPR on the y-axis and FPR on the x-axis when we vary the anomaly score threshold. The area under the curve is computed using the trapezoidal rule. For that purpose, we define $Th$ as an ordered set of thresholds between 0 and 1. Formally, we have $Th=[Th_0,Th_1,...Th_N]$ with $0=Th_0<Th_1<...<Th_N=1$. Therefore, $AUC\text{-}ROC$ is defined as follows:
%\vspace{-0.2cm}
\begin{equation}
\begin{split}
&AUC\text{-}ROC = \frac{1}{2}\sum_{k=1}^{N} \Delta^{k}_{TPR}*\Delta^{k}_{FPR}\\
&\text{with: } \left.
\begin{cases}
\Delta^{k}_{FPR} &= FPR(Th_{k})-FPR(Th_{k-1})\\
\Delta^{k}_{TPR} &= TPR(Th_{k-1})+TPR(Th_{k})
\end{cases}
\right. 
\end{split}
\label{equAUCROC}
\end{equation}
\newline \textbf{AUC-PR~\cite{10.1145/1143844.1143874}: } The Area Under the Precision-Recall curve (AUC-PR) is defined as the area under the curve corresponding to the Recall on the x-axis and Precision on the y-axis when we vary the anomaly score threshold. 
As before, the area under the curve can be calculated using the trapezoidal rule, defined as follows:

{\footnotesize
\begin{equation}
\begin{split}
&AUC\text{-}PR = \frac{1}{2}\sum_{k=1}^{N} \Delta^{k}_{Precision}*\Delta^{k}_{Recall}\\
&\text{with: } \left.
\begin{cases}
\Delta^{k}_{Recall} &= Recall(Th_{k})-Recall(Th_{k-1})\\
\Delta^{k}_{Precision} &= Precision(Th_{k-1})+Precision(Th_{k})
\end{cases}
\right. 
\end{split}
%\vspace{-0.1cm}
\label{equAUCPR}
\end{equation}
}

\noindent As discussed in~\cite{10.1145/1143844.1143874}, linear interpolation in PR space may result in an overly optimistic estimate of performance. Therefore, we adopt an alternative interpolation method, Stepwise Interpolation, to approximate the area under the curve by calculating the average precision of the PR curve:

%\vspace{-0.2cm}
\begin{equation}
AUC\text{-}PR = \sum_{k=1}^{N} Precision(Th_{k})*\Delta^{k}_{Recall}
%\vspace{-0.2cm}
\end{equation}

\noindent For consistency, we use the above equation in this paper to compute AUC-PR.


\section{Problem motivation and limitations}
\label{sec:problem}

\begin{figure}
 \centering
 \includegraphics[height=13cm,width=\linewidth]{figures/param_influence_intro.pdf}
 %\vspace{-0.7cm}
 \caption{Evaluation measures when we vary the (a) threshold, (b) lag, (c) noise, and (d) normal/abnormal ratio. Example with Isolation Forest methods over a snippet of an ECG time series~\cite{goldberger_physiobank_2000}.}
 \label{fig:limitation_robust}
 \vspace{-0.1cm}
\end{figure}

Having introduced existing measures to assess the quality of range-based anomalies, we now elaborate on their critical limitations.

%\vspace{-0.1cm}
\subsection{Limitations of Threshold-based Measures}
%\vspace{-0.1cm}

The need to threshold the anomaly score severely impacts the accuracy measures. First, Figure~\ref{fig:limitation_robust}(a) depicts an electrocardiogram time series with an arrhythmia in red (Figure~\ref{fig:limitation_robust}(a.1)) and the corresponding anomaly score computed with Isolation Forest~\cite{liu_isolation_2008} (Figure~\ref{fig:limitation_robust}(a.2)) for one threshold equal to $\mu(score) + \sigma(score)$ and for another threshold $\mu(score) + 0.6*\sigma(score)$ (Figures~\ref{fig:limitation_robust}(a.3) and (a.4)). We compute the different accuracy measures for the first threshold (blue bars in Figure~\ref{fig:limitation_robust}(a.5)) and the second threshold (orange bars in Figure~\ref{fig:limitation_robust}(a.5)) and their differences (Figure~\ref{fig:limitation_robust}(a.6)). We observe that the threshold choice has a substantial impact on Precision, Rprecision, F and RF scores. On the contrary, the threshold-independent measures (i.e., measures computing all possible thresholds), namely, AUC-ROC and AUC-PR, show a clear advantage.

Overall, the threshold choice depends on the application and the type of input time series. Setting the threshold automatically is hard and almost impossible when we compare different categories of AD methods across heterogeneous benchmarks. To illustrate this point, we consider two transformations of the anomaly score that correspond to practical cases we observed (e.g., different methods introduce different lag and noise levels to the anomaly score).
\newline \textbf{Influence of Noise: } Some AD methods applied to some specific time series might result in a noisy anomaly score. In addition, due to manufacturing issues or external causes, a sensor can inject noise into the time series, which then propagates on the anomaly score. Figure~\ref{fig:limitation_robust}(c) depicts two cases: the first corresponds to an anomaly score without any noise (Figure~\ref{fig:limitation_robust}(c.2)). The second corresponds to an anomaly score with noise (Figure~\ref{fig:limitation_robust}(c.2)). We applied on both cases the same threshold $\mu(score) + \sigma(score)$. We observe in Figure~\ref{fig:limitation_robust}(c.6) that most of the threshold-based measures are strongly impacted by noise. This is caused by the fact that the score fluctuates around the threshold, making threshold-based measures less robust to noise. On the contrary, AUC-ROC and AUC-PR are much less influenced by noise, returning approximately the same value.
\newline \textbf{Influence of Normal/Abnormal Ratio: } Depending on the domain and the task, the number of anomalies and, consequently, the normal/abnormal ratio changes drastically. A variation in this ratio might cause a variation in the threshold, which leads to variations in threshold-based accuracy measure values. This is explained by the fact that if an anomaly score detects the anomalies correctly, the standard deviation of that score will be higher for a time series with more anomalies. Figure~\ref{fig:limitation_robust}(d) depicts two cases: one time series snippet with a $0.2$ ratio (Figure~\ref{fig:limitation_robust}(d.2)) and one time series snippet with a $0.05$ ratio (Figure~\ref{fig:limitation_robust}(d.4)). We observe that this change implies a larger variation for several threshold-based measures. Thus, the latter confirms the limitations and the non-robustness of threshold-based measures to the anomaly cardinality ratio.

%\vspace{-0.2cm}
\subsection{Limitations of Point-based Measures}
%\vspace{-0.1cm}

In the previous section, we illustrated the limitations of threshold-based measures. By design and because of their independence from the threshold choice, AUC-ROC and AUC-PR measures are robust to those limitations. However, those measures are designed for point-based outliers. Each point is considered independently and the detection of each point has an equivalent contribution to AUC. In contrast, we need to consider two factors, the range detection and the existence detection, for the subsequence AD problem.

The range detection has the same methodology as point detection. We prefer that the algorithm detects every point in the subsequence anomaly. The existence detection is a loose but crucial estimation for the anomaly subsequence detector: detecting a tiny segment of one subsequence outlier is still of great value. 
\newline \textbf{Mismatch between the anomaly score and labels: } Compared to point-based AD, range-based AD encourages accurate capturing of each subsequence anomaly, but the existence detection is good enough to be partially rewarded. Two other reasons support the application of this coarse estimation. 

First, there is no consistent labeling tradition among datasets. Some may label the whole period as an anomaly if this period does not repeat the typical pattern, while others may only mark a partial period. Figure~\ref{fig:influence_labeling}(3) depicts different labeling strategies. Figures~\ref{fig:influence_labeling}(ex1), (ex2), and (ex3) depict three real examples corresponding to three different labeling strategies that we observed in existing datasets (see Table~\ref{table:charac}). Even if we specify that each period should share the same label, the next question is how to define the starting and end points of a period. Given accurate starting or end points, it is also challenging to label a small segment in one period. Unlike a point outlier, which is an evident deviation from the trend line of the time series, range-based anomalies may not have outrageous values. This difficulty of labeling is inevitable when we assign the discrete labels to a continuous time series. There may be a transition region between the two statuses, but we have to decide on a discontinuous jumping point artificially.

Second, many algorithms, for instance, LOF~\cite{breunig_lof_2000} and iForest~\cite{liu_isolation_2008}, would first apply a sliding window to convert a 1-D time series to a set of high-dimensional data points. We denote the original time series as ($T_1$, $T_2$, $\dots$, $T_n$), and suppose the length of window is $\ell$, then the converted data set is $\{(T_i, \dots, T_{i+\ell-1})|i \in \{1, \dots T-\ell+1\}\}$. The label of point $T_k$ in the time series is defined as the label of high-dimensional point ($T_{k-\ell/2}$, $\dots$, $T_{k+\ell/2-1}$) in the transformed dataset. The conversion from a time series to a dataset has one consequence: every dimension in the high-dimensional point is equally important. So, an abnormal value at the middle or end of this point has the same ability to make it an outlier in the high-dimensional space. Usually, if the sliding window covers more anomaly points, a good algorithm should give a higher anomaly score to the converted data point. However, there are some exceptions, such as that one abnormal value at the beginning or the end of sliding windows is enough to make the converted point an outlier. 
To summarize, an anomaly subsequence $(T_s, \dots, T_e)$ may induce a high anomaly score for the range [$T_{s-\ell/2}, T_{e+\ell/2}$]. A perfect result is that the peak of the anomaly score is slightly broader than the whole abnormal region. The latter is illustrated in Figure~\ref{fig:influence_labeling}(2). However, the anomaly score is not perfect. A high score may be assigned at the range [$T_{s-\ell/2}, T_s$], which fails to reveal the entire range of the outlier but succeeds in indicating the starting region. AUC-based measures will give a low value since there is no overlap between the peak and the outlier.
\newline \textbf{Overall Limitations due to Lag: } A lag can be injected into the anomaly score depending on the choice of AD methods. Overall, such a lag may also exist due to the approximation made during the labeling phase. As illustrated in Figure~\ref{fig:limitation_robust}(b), such a lag (even small) has a substantial impact on \textit{all} existing evaluation measures. For example, in Figure~\ref{fig:limitation_robust}(b) AUC-PR fluctuates between $0.75$ and $0.50$ for a lag of $0.25$ of the labeled section length. Among all measures, only the AUC-ROC measure demonstrates to be less sensitive to such lag (as well as noise and normal/abnormal ratio).


\begin{figure}[tb]
 \centering
 \includegraphics[height=5.5cm,width=\linewidth]{figures/influence_eval_short.pdf}
 %\vspace{-0.7cm}
 \caption{Influence of the anomaly detection method score (2) and labeling strategy (3), illustrated with three examples.}
 \label{fig:influence_labeling}
 %\vspace{-0.2cm}
\end{figure}

%\vspace{-0.2cm}
\subsection{Problem Definition}
%\vspace{-0.1cm}

In summary, our goal is to develop a new anomaly score threshold-independent evaluation measure based on the robust principles of AUC. A promising direction is an extension of AUC for the range-based AD with the following desired properties: 


\noindent{\bf Robust to Lag}: Two similar anomaly scores with a slight lag difference should return approximately the same accuracy measures. For example, a high anomaly score near the border of the anomaly should be rewarded as close as a high anomaly score in the middle of the range-based anomaly.
	%\item 

\noindent{\bf Robust to Noise}: Two similar anomaly scores with and without noise should return similar accuracy.
	%\item 

\noindent{\bf Robust to the Anomaly Cardinality Ratio}: This ratio should not impact the accuracy measures.
	%\item 

\noindent{\bf High Separability between Accurate and Inaccurate Methods}: The accuracy measure should well separate accurate from inaccurate methods.
	%\item 

\noindent{\bf Consistent}: Finally, an appropriate accuracy measure should produce consistent scores for similar time series (i.e., rank different anomaly detection methods in a consistent manner).
%\end{itemize}

\noindent Next, we present new accuracy measures to satisfy these properties.

\begin{figure*}
  \centering
  \includegraphics[width=\linewidth]{figures/porposed_approaches.pdf}
  %\vspace{-0.7cm}
  \caption{Illustration of previous quality measures compared to our proposed measures. By varying the buffer window, VUS constructs a surface of TPR, FPR, and window. The volume under the surface is a measure of AUC for various windows. }
  \label{fig:auc_volume}
  %\vspace{-0.1cm}
\end{figure*}

\begin{figure}
  \centering
  \includegraphics[width=\linewidth]{figures/label_extension.pdf}
  %\vspace{-0.7cm}
  \caption{\commentRed{Illustration of proposed label extension strategy.}}
  \label{fig:label_extension}
  %\vspace{-0.1cm}
\end{figure}

%\vspace{-0.1cm}
\section{Our Measures: Range-AUC and VUS}
%\vspace{-0.1cm}

We first present new range-based extensions for ROC and PR curves by introducing a new continuous label to enable more flexibility in measuring detected anomaly ranges. We then present the Volume Under the Surface (VUS) for ROC and PR curves. VUS extends the mathematical model of Range-AUC measures by varying the buffer length. \commentRed{An alternative solution is to learn the necessary parameters and thresholds. However, such a solution works only under supervised settings and may impact the generalizability to new datasets. For the specific case of unsupervised learning, the threshold selection can only be achieved using statistical heuristics. The most common strategy to set the threshold unsupervisely is to set it to $\mu(S_T) + \alpha*\sigma(S_T)$, with $\alpha=3$~\cite{statisticaloutliers}. We will use this strategy when comparing our proposed measures to threshold-based measures.}

%\vspace{-0.2cm}
\subsection{Range-AUC-ROC and Range-AUC-PR}
\label{sec:range-auc}
%\vspace{-0.1cm}

To compute the ROC curve and PR curve for a subsequence, we need to extend to definitions of TPR, FPR, and Precision. 
The first step is to add a buffer region at the boundary of outliers. The idea is that there should be a transition region between the normal and abnormal subsequences to accommodate the false tolerance of labeling in the ground truth (as discussed, this is unavoidable due to the mapping of discrete data to continuous time series). An extra benefit is that this buffer will give credit to the high anomaly score in the vicinity of the outlier boundary, which is what we expected with the application of a sliding window originally. 

Figure ~\ref{fig:auc_volume}(b) shows the original binary labels (in blue), and Figure ~\ref{fig:auc_volume}(c) the new label with buffer region (in orange). By default, the width of the buffer region at each side is half of the period $w$ of the time series (the period is an intrinsic characteristic of the time series). Differently, this parameter can be set into the average length of anomaly sizes or can be set to a desired value by the user.

The traditional binary label is extended to a continuous value. Formally, for a given buffer length $\ell$, the positions $s,e \in [0,|label|]$ the beginning and end indexes of a labeled anomaly (i.e., sections of continuous $1$ in $label$), we define the continuous $label_r$ as follows:
%\vspace{-0.1cm}
\begin{equation}
\footnotesize{
\begin{split}
&\forall i \in [0,|label|], \quad label_{\ell i} \\
& = \begin{cases}
\bigg(1-\frac{|s-i|}{\ell}\bigg)^{\frac{1}{2}}, & \text{if } s-\frac{\ell}{2} \leq i < s \text{ and } {pred}_i = 1, \\
1, & \text{if } s \leq i < e, \\
\bigg(1-\frac{|e-i|}{\ell}\bigg)^{\frac{1}{2}}, & \text{if } e \leq i < e+\frac{\ell}{2} \text{ and } {pred}_i = 1, \\
0, & \text{else}.
\end{cases}
\end{split}
\label{label_equation}
}
%\vspace{-0.1cm}
\end{equation}

\commentRed{
\noindent Specifically, if no predicted anomaly exists within the extended buffer region, we set ${label_{\ell}}_i$ to $0$ to prevent unnecessary false negatives caused by excessive label extension, as illustrated in Figure~\ref{fig:label_extension}.
}
\noindent When the buffer regions of two discontinuous outliers overlap, the label will be the superposition of these two orange curves with one as the maximum value. Using this new continuous label, one can compute $TP_\ell$, $FP_\ell$, $TN_\ell$ and $FN_\ell$ similarly as follows:
%\vspace{-0.2cm}
\begin{equation}
{\small
\begin{split}
&TP_{\ell} = label_{\ell}^\top \cdot pred &FP_{\ell} = (I- label_{\ell})^\top \cdot pred \\
&TN_{\ell} = (I- label_{\ell})^\top \cdot (I-pred) &FN_{\ell} = label_{\ell}^\top \cdot (I-pred) \\
\end{split}
} % font size
%\vspace{-0.2cm}
\end{equation}
\noindent The total number of positive points P in this case naively should be $P_{{\ell}_0} = TP_{\ell}+ FN_{\ell} = label_{\ell}^\top \cdot I$. Here, we define it as:
%\vspace{-0.2cm}
\begin{equation}
%\begin{split}
P_{\ell} = (label+label_{\ell})^\top \cdot \frac{I}{2} \text{, } N_{\ell} = |label_{\ell}|-P_{\ell}
%\end{split}
%\vspace{-0.2cm}
\end{equation}
\noindent The reason is twofold. When the length of the outlier is several periods, $P_{{\ell}_0}$ and $P_{\ell}$ are similar because the ratio of the buffer region to the whole anomaly region is small. When the length of the outlier is only half-period, the size of the buffer region is nearly two times the original abnormal region. In other words, to pursue false tolerance, the relative change we make to the ground truth is too significant. We use the average of $label$ and $label_{\ell}$ to limit this change.

We finally generalize the point-based $Recall$, $Precision$, and $FPR$ to the range-based variants. Formally, following the definition of $R$ and $P$ as the set of anomalies range and detected predicted anomaly range (see Section~\ref{acc_measure}), we define $TPR_{\ell}$, $FPR_{\ell}$, and $Precision_{\ell}$:
%\vspace{-0.2cm}
\begin{equation}
{\small
\begin{split}
TPR_\ell&=Recall_{\ell}=\frac{TP_{\ell}}{P_{\ell}}*\sum_{R_i \in R} \frac{ExistenceR(R_i,P)}{|R|} \\
FPR_{\ell}&=\frac{FP_{\ell}}{N_{\ell}} \text{, } Precision_{\ell}=\frac{TP_{\ell}}{TP_{\ell}+FP_{\ell}} \\
\end{split}
} % font size
%\vspace{-0.2cm}
\label{eqution_constant}
\end{equation}
\noindent Note that $TPR_r=Recall_r$. Moreover, for the recall computation, we incorporate the idea of Existence Reward \cite{tatbul_precision_2018}, which is the ratio of the number of detected subsequence outliers to the total number of subsequence outliers. However, consistent with their work \cite{tatbul_precision_2018}, we do not include the Existence ratio in the definition of range-precision. We can then compute R-AUC-ROC and R-AUC-PR using Equation~\ref{equAUCROC} and Equation~\ref{equAUCPR}.
\newline \textbf{Relation between Range-ROC and Range-PR: } PR curve is a supplement to the ROC curve. In a highly unbalanced dataset, because the number of positive points is too small, at the same level of FPR, it is easy to have a high TPR (or $TPR_{\ell}$) at the cost of low precision.  There are deep connections between ROC and PR \cite{10.1145/1143844.1143874}. First, ROC and PR have one-to-one mapping for a given dataset because the confusion matrix is uniquely determined given TPR and FPR. This relation is broken for the range method because we include an extra Existence factor for range-TPR. Therefore, the confusion matrix cannot be decided in the range-ROC space. Secondly, for a point-based version, if one ROC curve \textit{dominates} another ROC curve, its corresponding PR curve would also dominate another one. Here, dominate means the curve is always higher or equal to another one. Because of the Existence factor, this rule is also lifted for the range definition. This is true only if both of the methods have the same existence ratio. However, this is not always guaranteed. Finally, a maximized AUC does not necessarily correspond to a maximized AP. This holds for the range definition.

\subsection{VUS: Volume Under the Surface}
\label{sec:vus}

Our range-AUC family of measures chooses the width of the buffer region to be half of a subsequence length $\ell$ of the time series. Such buffer length can be either set based on the knowledge of an expert (e.g., the usual size of arrhythmia in an electrocardiogram) or set automatically using the time series's period. \commentRed{The latter can be computed using different strategies: (I) using the Fourier transform to identify the most relevant period of the time series, or (ii) computing the cross-correlation and retrieving the lag value (i.e., subsequence length) that locally maximize the correlation. In practice, we observe that computing the cross-correlation of a time series and selecting the length corresponding to the first local maximal is accurate. In addition, the latter allows users to consider the shortest period length, significantly limiting the execution time of most of the AD methods and the range-AUC measures.} 

Since the period is an intrinsic property of the time series, we can compare various algorithms on the same basis. However, a different approach may get a slightly different period. In addition, there are multi-period time series. So other groups may get different range-AUC because of the difference in the period. As a matter of fact, the parameter $\ell$, if not well set, can strongly influence range-AUC measures. To eliminate this influence, we introduce two generalizations of range-AUC measures.

The solution is to compute ROC and PR curves for different buffer lengths from 0 to $\ell$ as shown in Figure~\ref{fig:auc_volume}(d). Therefore, ROC and PR curves become a surface in a three-dimensional space. Then, the overall accuracy measure corresponds to the Volume Under the Surface (VUS) for either the ROC surface (VUS-ROC) or PR surface (VUS-PR). As the R-AUC-ROC and R-AUC-PR are measures independent of the threshold on the anomaly score, the VUS-ROC and VUS-PR are independent of both the threshold and buffer length. Formally, given $Th=[Th_0,Th_1,...Th_N]$ with $0=Th_0<Th_1<...<Th_N=1$, and $\mathcal{L}=[\ell_0,\ell_1,...,\ell_L]$ with $0=\ell_0<\ell_1< ... < \ell_L = \ell$, we have:
%\vspace{-0.1cm}
\begin{equation}
\footnotesize{
\begin{split}
&VUS\text{-}ROC = \frac{1}{4}\sum_{w=1}^{L} \sum_{k=1}^{N} \Delta^{(k,w)} * \Delta^{w} \text{, with: }\\
&\left.
\begin{cases}
\Delta^{(k,w)} &= \Delta^{k}_{TPR_{\ell_w}}*\Delta^{k}_{FPR_{\ell_w}}+\Delta^{k}_{TPR_{\ell_{w-1}}}*\Delta^{k}_{FPR_{\ell_{w-1}}} \\
\Delta^{k}_{FPR_{\ell_w}} &= FPR_{\ell_w}(Th_{k})-FPR_{\ell_w}(Th_{k-1}) \\
\Delta^{k}_{TPR_{\ell_w}} &= TPR_{\ell_w}(Th_{k-1})+TPR_{\ell_w}(Th_{k}) \\
\Delta^{w} &= |\ell_w - \ell_{w-1}|
\end{cases}
\right. 
\end{split}
\label{equVUSROC}
}
%\vspace{-0.1cm}
\end{equation}

%\vspace{-0.1cm}
\begin{equation}
\footnotesize{
\begin{split}
&VUS\text{-}PR = \frac{1}{2}\sum_{w=1}^{L} \sum_{k=1}^{N} \Delta^{(k,w)} * \Delta^{w} \text{, with: }\\
&\left.
\begin{cases}
\Delta^{(k,w)} &= {Precision_{\ell_w}(Th_k)}*\Delta^{k}_{Re_{\ell_w}}\\  &\quad+{Precision_{\ell_{w-1}}(Th_k)}*\Delta^{k}_{Re_{\ell_{w-1}}} \\
\Delta^{k}_{Re_{\ell_w}} &= Recall_{\ell_w}(Th_{k})-Recall_{\ell_w}(Th_{k-1}) \\
% \Delta^{k}_{Pr_{\ell_w}} &= Precision_{\ell_w}(Th_{k-1})+Precision_{\ell_w}(Th_{k}) \\
\Delta^{w} &= |\ell_w - \ell_{w-1}|
\end{cases}
\right. 
\end{split}
\label{equVUSPR}
}
%\vspace{-0.1cm}
\end{equation} 

From the above equations, VUS measures are more expensive to compute than range-AUC measures.
Thus, the application of VUS versus range-AUC depends on our knowledge of which buffer length to set. If one user knows which would be the most appropriate buffer length, range-AUC-based measures are preferable compared to VUS-based measures.
However, if there exists an uncertainty on $\ell$, then setting a range and using VUS increases the flexibility of the usage and the robustness of the evaluation. Finally, more parameters than $\ell$ can be included in VUS-based measures. If, in addition to $\ell$, there is a need to define a range for another parameter (such as the normal model length $\ell_{N_M}$ of NormA), the two-dimensional surface is transformed into a three-dimensional hyper-surface. In general, for $P$ parameters, the value is the volume under a $|P|-1$ hyper-surface. 






\subsubsection{{\bf Complexity Analysis}}\hfill\\

This section analyzes the complexity of the VUS-based measures. 
We take into account both computation time, and memory usage.

\begin{algorithm}[tb]
{\small
    \caption{\textbf{VUS algorithm}}\label{alg:VUS}
    \label{alg:vus}
    \SetKwInOut{Input}{input}
    \SetKwInOut{Output}{output}
    \Input{Original Labels $label$, anomaly score $S_{T}$, maximum Buffer Length $L$, thresholds $N$}
    \Output{VUS\_ROC, VUS\_PR}
    \BlankLine
    $Th$ $\leftarrow$ $Thresholds(N)$\;
    $\mathcal{L}$ $\leftarrow$ $Buffer\_Lengths(L)$\;
    AUC $\leftarrow$ [],
    AP $\leftarrow$ []\;
    \tcp{Iterate through the buffer Lengths}
    \ForEach{$\ell \in \mathcal{L}$ }
    {
        $Create$ $label_\ell$ from $label$ and $\ell$\;
        $seq$= $Anomaly\_Index(label_\ell)$\;
        $list\_TPR_{\ell}$ $\leftarrow$ [],
        $list\_FPR_{\ell}$ $\leftarrow$ [],
        $list\_Prec_{\ell}$ $\leftarrow$ []\;
        \tcp{Iterate through the thresholds}
        \ForEach{$threshold \in Th$}
        {   
            $pred$ $\leftarrow$ $S_{T}>threshold$\;
            $Change$ $label_\ell$ to $label_\ell^{thres}$ based on $pred$\;
            $product$ $\leftarrow$ $label_\ell^{thres}*pred$\;
            $SumPred$ $\leftarrow$ $\sum_{p\in pred} p$\;
            $SumLabel$ $\leftarrow$ $\sum_{p\in label_\ell^{thres}} p$\;
            $TP_\ell$ $\leftarrow$ 0\;
            \ForEach{$seg \in seq_L$}
            {
                $TP_\ell$ $\leftarrow$ $TP_\ell$ + $\sum_{p\in product[seg[0]:(seg[1]+1)]}p$
            }
            $Compute$ $FP_\ell$, $P_\ell$, $N_\ell$\ from $TP_\ell$, $SumPred$, $SumLabel$\;% $\leftarrow$ $\sum_{p\in product}p$\;
            %$FP_\ell$ $\leftarrow$ $\sum_{p\in product}p$\;
            %$P_\ell$ $\leftarrow$ $\sum_{l_1,l_2\in label,label_\ell} \frac{(l_1+l_2)}{2}$
            
            $Existence_{seq}$ $\leftarrow$ 0\;
            \tcp{Iterate through the anomalies}
            \ForEach{$seg \in seq$}
            {
                \If{$\sum_{p\in product[seg[0]:(seg[1]+1)]}p>0$}
                {
                    $Existence_{seq}$ $\leftarrow$ $Existence_{seq}$ + 1
                }
                $Existence$ $\leftarrow$ $\frac{Existence_{seq}}{|seq|}$
                  
            }
            $Append$ $\frac{TP_\ell*Existence}{P_\ell}$ in $list\_TPR_{\ell}$\;
            $Append$ $\frac{FP_\ell}{N_\ell}$ in $list\_FPR_{\ell}$\;
            $Append$ $\frac{TP_\ell}{TP_\ell+FP_\ell}$ in $list\_Prec_{\ell}$\;
        }
        $Compute$ AUC\_r, AP\_r $from$ $list\_TPR_{\ell}$,$list\_FPR_{\ell}$ and $list\_Prec_{\ell}$\;
        $Append$ AUC\_r, AP\_r $in$ AUC, AP\;
    }
    \tcp{Avg. across thresholds and buffer lengths}
    VUS\_ROC $\leftarrow$ $\frac{\sum_{a\in AUC}a}{|\mathcal{L}|}$,
    VUS\_PR $\leftarrow$ $\frac{\sum_{a\in AP}a}{|\mathcal{L}|}$\;
 % font size
 }
\end{algorithm}

{\bf [Time Complexity]}
The time complexity of VUS (both VUS-ROC and VUS-PR) is determined by varying two parameters, namely the buffer length $\ell \in \mathcal{L}$ and the number of thresholds $N$.
This is further illustrated in Algorithm~\ref{alg:vus}, which breaks down the computation steps. 
It comprises a nested loop that demonstrates the variation of the parameters buffer length \commentRed{($\mathcal{L}$ lengths in total)} and number of thresholds \commentRed{($N$ thresholds in total)}. \commentRed{Therefore, VUS complexity is quadratic to $N$ and $L$. Then, for each threshold and length (inside the loop) the computational complexity is $O(\alpha \ell_a + T_1 + T_2)$}, where $\alpha$ is the number of anomalies, $\ell_a$ refers to the mean length of anomalies, and $T_1, T_2$ refer to computations in the order of length of the time series $T$ involved in the anomaly detection. 
There is a distinction between $T_1$ and $T_2$ because their practical implementations are optimized to different extents, producing very different execution times. 
Here, $O(T_1)$ is the complexity of the calculations involving the entire time series, such as $pred$ (i.e., a boolean sequence indicating if a point of the anomaly score $S_T$ is above a given threshold), and $label_\ell$ (i.e., the modified label sequence with buffer regions). $O(T_2)$ refers to the complexity of the computation of $product$, $TP_\ell$, $FP_\ell$, $P_\ell$, and $N_\ell$, which has a cost of $|T|$, but is less optimized than the previously described computation. 
Moreover, $\alpha \ell_a$ corresponds to the computation of $Existence$. Thus, the total complexity of the algorithm is $O(NL(\alpha \ell_a+T_1+T_2))$. 
In practice, $\alpha \ell_a$ is negligible compared to $T_1$ or $T_2$, and VUS complexity can be approximated to $O(NL(T_1+T_2))$.

{\bf [Space Complexity]}
The space complexity can be obtained from the pseudo-code in Algorithm~\ref{alg:vus}. 
The computation of VUS-ROC and VUS-PR is performed by iterating over the set of buffer lengths ($\mathcal{L}$) and the set of thresholds ($N$). 
Thus, the space complexity of VUS is $O(NL)$.

\subsection{A faster Implementation of VUS}
\label{sec:fasterimpl}

\begin{figure}[tb]
  \centering
  \includegraphics[width=\linewidth]{figures/Static_Dyn.pdf}
  %\vspace*{-0.5cm}
  \caption{Synthetic illustration of an anomaly score (a) and labels (b) of a given time series. We differentiate \textbf{static sections} that are invariant to the change of threshold and buffer length, and \textbf{dynamic sections} that have an impact on the accuracy.}
  \label{fig:static_dyn}
\end{figure}

As theoretically explained in the previous section, VUS's computation heavily depends on the time series length, as well as on the number of buffer lengths considered. In this section, we propose a novel implementation that significantly reduces the theoretical computation of the VUS measures.
 

\begin{algorithm}[tb]
{\small
    \caption{\textbf{\textbf{VUS}$_{opt}$}}\label{alg:VUS_opt}
    \SetKwInOut{Input}{input}
    \SetKwInOut{Output}{output}
    \Input{Original Labels $T$, anomaly score $S_{T}$, maximum Buffer Length $L$, thresholds $N$}
    \Output{VUS-ROC, VUS-PR}
    \BlankLine
    $Th$ $\leftarrow$ $Thresholds(N)$,
    $\mathcal{L}$ $\leftarrow$ $Buffer\_Lengths(L)$\;
    $Create$ $label_L$ from $label$ and $L$\;
    \tcp{Extract anomalies positions for buffer length L}
    $seq_L$ $\leftarrow$ $Anomaly\_Index(label_L)$\;
    $AUC$ $\leftarrow$ [], 
    $AP$ $\leftarrow$ []\;
    \tcp{Static Part}
    \tcp{Iterate through thresholds only}
    \ForEach{$threshold \in Th$}
    {\label{line_vus:static_b}
        $pred$ $\leftarrow$ $S_{T}>threshold$\;
        $SumPred$ $\leftarrow$ $\sum_{p\in pred} p$\;
    }\label{line_vus:static_e}
    \tcp{Dynamic Part}
    \tcp{Iterate through the buffer Lengths}
    \ForEach{$\ell \in \mathcal{L}$ }
    {\label{line_vus:dyn_b}
        $Create$ $label_\ell$ from $label$ and $\ell$\;
        $seq$= $Anomaly\_Index(label_\ell)$\;
        $l\_TPR_{\ell}$ $\leftarrow$ [], 
        $l\_FPR_{\ell}$ $\leftarrow$ [], 
        $l\_Prec_{\ell}$ $\leftarrow$ []\;
        \tcp{Iterate through the thresholds}
        \ForEach{$threshold \in Th$}
        {   
            $pred$ $\leftarrow$ $S_{T}>threshold$\;
            $Change$ $label_\ell$ to $label_\ell^{thres}$ based on $pred$\;
            $product$ $\leftarrow$ $label_\ell^{thres}$*$pred$\;
            $SumLabel$ $\leftarrow$ $\sum_{p\in label_\ell^{thres}} p$\;
            $TP_\ell$ $\leftarrow$ 0\;
            \ForEach{$seg \in seq_L$}
            {
                $TP_\ell$ $\leftarrow$ $TP_\ell$ + $\sum_{p\in product[seg[0]:(seg[1]+1)]}p$
            }
            $Compute$ $FP_\ell$, $P_\ell$, $N_\ell$\ from $TP_\ell$, $SumPred$, $SumLabel$\;
            
            $Existence_{seq}$ $\leftarrow$ 0\;
            \tcp{Iterate through the anomalies}
            \ForEach{$seg \in seq$}
            {
                \If{$\sum_{p\in product[seg[0]:(seg[1]+1)]}p>0$}
                {
                    $Existence_{seq}$ $\leftarrow$ $Existence_{seq}$ + 1
                }
                $Existence$ $\leftarrow$ $\frac{Existence_{seq}}{|seq|}$
                  
            }
            $Append$ $\frac{TP_\ell*Existence}{P_\ell}$ in $l\_TPR_{\ell}$\;
            $Append$ $\frac{FP_\ell}{N_\ell}$ in $l\_FPR_{\ell}$\;
            $Append$ $\frac{TP_\ell}{TP_\ell+FP_\ell}$ in $l\_Prec_{\ell}$\;
        }
        $Compute$ $AUC_r$, $AP_r$ $from$ $l\_TPR_{\ell}$,$l\_FPR_{\ell}$ and $l\_Prec_{\ell}$\;
        $Append$ $AUC_r$, $AP_r$ $in$ $AUC$, $AP$\;
    } \label{line_vus:dyn_e}
    \tcp{Avg. across thresholds and buffer lengths}
    VUS-ROC $\leftarrow$ $\frac{\sum_{a\in AUC}a}{|\mathcal{L}|}$, 
    VUS-PR $\leftarrow$ $\frac{\sum_{a\in AP}a}{|\mathcal{L}|}$\;
 % font size
 }
\end{algorithm}

\begin{algorithm}
{\small
    \caption{\textbf{VUS$_{opt}^{mem}$}}\label{alg:VUS_opt^{mem}}
    \SetKwInOut{Input}{input}
    \SetKwInOut{Output}{output}
    \Input{Original Labels $T$, anomaly score $S_{T}$, maximum Buffer Length $L$, thresholds $N$}
    \Output{VUS-ROC, VUS-PR}
    \BlankLine
    $Th$ $\leftarrow$ $Thresholds(N)$,
    $\mathcal{L}$ $\leftarrow$ $Buffer\_Lengths(L)$\;
    $Create$ $label_L$ from $label$ and $L$\;
    \tcp{Extract anomalies positions for buffer length L}
    $seq_L$ $\leftarrow$ $Anomaly\_Index(label_L)$\;
    $AUC$ $\leftarrow$ [], 
    $AP$ $\leftarrow$ []\; 
    $Pred_{Th}$ $\leftarrow$ []\;
    \tcp{Static Part}
    \tcp{Iterate only through thresholds}
    \ForEach{$threshold \in Th$}
    {
        $pred$ $\leftarrow$ $S_{T}>threshold$\;
        $Pred_{Th}$ $\leftarrow$ Append with $pred$\;
        $SumPred$ $\leftarrow$ $\sum_{p\in pred} p$\;
    }
    \tcp{Dynamic Part}
    \tcp{Iterate through the buffer Lengths}
    \ForEach{$\ell \in \mathcal{L}$ }
    {
        $Create$ $label_\ell$ from $label$ and $\ell$\;
        $seq$= $Anomaly\_Index(label_\ell)$\;
        $l\_TPR_{\ell}$ $\leftarrow$ [],
        $l\_FPR_{\ell}$ $\leftarrow$ [],
        $l\_Prec_{\ell}$ $\leftarrow$ []\;
        \tcp{Iterate through the thresholds}
        count $\leftarrow$ 0\;
        \ForEach{$threshold \in Th$}
        {  
            $Change$ $label_\ell$ to $label_\ell^{thres}$ based on $Pred_{Th}[threshold]$\;
            $product$ $\leftarrow$ $label_\ell^{thres}*Pred_{Th}[threshold]$\;
            $SumLabel$ $\leftarrow$ $\sum_{p\in label_\ell^{thres}} p$\;
            $TP_\ell$ $\leftarrow$ 0\;
            \ForEach{$seg \in seq_L$}
            {
                $TP_\ell$ $\leftarrow$ $TP_\ell$ + $\sum_{p\in product[seg[0]:(seg[1]+1)]}p$
            }
            $Compute$ $FP_\ell$, $P_\ell$, $N_\ell$\ from $TP_\ell$, $SumPred$, $SumLabel$\;
            $Existence_{seq}$ $\leftarrow$ 0\;
            \tcp{Iterate through the anomalies}
            \ForEach{$seg \in seq$}
            {
                \If{$\sum_{p\in product[seg[0]:(seg[1]+1)]}p>0$}
                {
                    $Existence_{seq}$ $\leftarrow$ $Existence_{seq}$ + 1
                }
                $Existence$ $\leftarrow$ $\frac{Existence_{seq}}{|seq|}$
                  
            }
            $Append$ $\frac{TP_\ell*Existence}{P_\ell}$ in $l\_TPR_{\ell}$\;
            $Append$ $\frac{FP_\ell}{N_\ell}$ in $l\_FPR_{\ell}$\;
            $Append$ $\frac{TP_\ell}{TP_\ell+FP_\ell}$ in $l\_Prec_{\ell}$\;   
        }
        $Compute$ AUC\_r, AP\_r $from$ $l\_TPR_{\ell}$,$l\_FPR_{\ell}$ and $l\_Prec_{\ell}$\;
        $Append$ AUC\_r, AP\_r $in$ AUC, AP\;
    }
    \tcp{Avg. across thresholds and buffer lengths}
    VUS\_ROC $\leftarrow$ $\frac{\sum_{a\in AUC}a}{|\mathcal{L}|}$,
    VUS\_PR $\leftarrow$ $\frac{\sum_{a\in AP}a}{|\mathcal{L}|}$\;
 % font size
 }
\end{algorithm}










\subsubsection{Dynamic versus Static sections}\hfill\\

The variations of thresholds and buffer length affect the modified labels (i.e., $label_\ell$) and $pred$, which cause changes in the values of True and False Positive Rates ($TPR$ and $FPR$). 
However, not all sections of the time series are affected by these variations. 
The data points, whose labels are not affected by the change in the buffer length for a given threshold, have the same $TPR$ and $FPR$ (i.e., data points that lie outside the maximum possible buffer length of an anomaly). 

As a result, we can segment the time series into two categories: $Dynamic$ and $Static$. The first category corresponds to sections of the time series containing labels affected by the variation of buffer length. The second category corresponds to sections that are not affected by these changes. Figure~\ref{fig:static_dyn} illustrates this segmentation, enabling us to compute the same measures with significantly reduced computational costs.





\begin{figure}[tb]
  \centering
  \includegraphics[width=\linewidth]{figures/dyn_stat_2.pdf}
  %\vspace*{-0.5cm}
  \caption{Synthetic illustration of the labels evolution with $L$. In contrast to dynamic sections (in green), the buffer length has no impact on VUS within the static sections (in grey).}
  \label{fig:static_dyn_2}
\end{figure}





\subsubsection{Algorithmic Implementation}\hfill\\

The optimization described above can be performed following two different strategies:

\begin{itemize}
\item {\bf VUS$_{opt}$}: In this version, we split the time series anomaly scores $S_T$ and $label_\ell$ into static and dynamic sections. We compute the constant required to calculate VUS only once for the static sections, and once for each buffer length and threshold value for the dynamic sections.
\item {\bf VUS$_{opt}^{mem}$}: This version is an extension of the previous, with a code-wise modification that leads to a further decrease in time complexity at the expense of increased space complexity.
Given the large main memory sizes of modern servers (and even desktops and laptops), VUS$_{opt}^{mem}$ represents a very attractive solution in practice.
\end{itemize}

Due to the consideration of splitting data into static and dynamic regions, the implementation has the following advantages:

\begin{itemize}
\item The static split avoids repetitive calculations that would have involved the same values repeatedly in a nested loop (i.e., computing only the accuracy values for each threshold for the static region, since buffer size does not affect static regions).
\item The calculations of $TP$ and $N$ in Equation~\ref{eqution_constant} essentially add up to zero in the above consideration of the static part, and do not need to be computed. 
\item The overall computational time is similar to that of the Range-AUC measures for a relatively small number of anomalies, but even more importantly, it does not increase when the number of anomalies gets significantly larger.
\end{itemize}

The computational steps of $VUS_{opt}$ and $VUS_{opt}^{mem}$ are shown in Algorithm~\ref{alg:VUS_opt} and Algorithm~\ref{alg:VUS_opt^{mem}}, respectively.
These two algorithms are divided into two different sections: (i) the static part in which we compute VUS for sections of the time series without anomalies, and (ii) the dynamic part in which we compute VUS only for the time series sections that contain anomalies.
In the following sections, we analyze in detail the theoretical complexity (space and time).

\noindent{\bf [VUS$_{opt}$ Time and Space Complexity]}: The VUS$_{opt}$ computation is similar to the original VUS computation (cf. Algorithm~\ref{alg:vus}) for the calculations of the dynamic part. 
However, it differs in the static part, as its calculations that involve predictions and labels are unaffected by buffer length. 
The static part computation (Lines~\ref{line_vus:static_b}-\ref{line_vus:static_e}) involves the predictions (according to all possible thresholds in $Th$) and their summation. 
Thus, the complexity for the static sections is $O(N(T_1+T_2))$. 
For the dynamic part (Lines~\ref{line_vus:dyn_b}-\ref{line_vus:dyn_e}), the computations (for each threshold and buffer length) are only performed for the sections containing anomalies (i.e., dynamic sections in Figure~\ref{fig:static_dyn}). Thus, the complexity of the dynamic part computation is $O(\alpha \ell_a)$.
We also have to compute the predictions (score values above a given threshold) for each dynamic section, which have a complexity of $O(T_2)$. 
Thus the total complexity adds up to $O(N(T_1+T_2))+O(NL(\alpha \ell_a+T_2))$.
In addition, the space complexity of the dynamic computation with the nested loop of thresholds and buffer length is $O(NL)$, and $O(N)$ for the static part. Therefore, the overall space complexity of VUS$_{opt}$ is $O(NL)$.

\noindent{\bf [VUS$_{opt}^{mem}$ Time and Space Complexity]}
As shown in Algorithm~\ref{alg:VUS_opt^{mem}}, the complexity of the static sections remains unchanged compared to VUS$_{opt}$. However, the complexity related to the dynamic sections is reduced by reusing the saved predictions calculated in the static part (as illustrated in Figure~\ref{fig:static_dyn_2}, it is not affected by buffer lengths).
This reduces the dynamic complexity to  $O(\alpha \ell_a)$, adding up to a total complexity of $O(N(T_1+T_2)+ NL\alpha \ell_a)$. 
For VUS$_{opt}^{mem}$, similarly to VUS$_{opt}$, the space complexity of the dynamic computation containing the nested loop of thresholds and buffer length is $O(NL)$. However, due to the storage and indexing of predictions, the computations related to the static sections result in a space complexity of $O(NT)$. Thus, the total space complexity of VUS$_{opt}^{mem}$ is $O(N(L+T))$. 
The time and space complexity of all three VUS implementations are listed in Table~\ref{tab:complexity_summary}.

\begin{table}[tb]
    \centering
    \caption{{Space and time complexity of VUS implementations}}
    \scalebox{0.88}{
    \begin{tabular}{|c|c|c|}
    \hline
         Version & Time & Space \\
         \hline
         $VUS$ & $O(NL(\alpha \ell_a+T_1+T_2))$ & $O(NL)$\\ 
 VUS$_{opt}$ & $O(N(T_1+T_2+L(\alpha \ell_a+T_2)))$ &  $O(NL)$\\
 VUS$_{opt}^{mem}$ & $O(N(T_1+T_2+L\alpha \ell_a))$ & $O(N(L+T))$\\ 
 \hline
    \end{tabular}
    }
    \label{tab:complexity_summary}
\end{table}



In this section, we empirically compare the proposed algorithm on both sequence windows and time windows with existing methods.
\paragraph{Datasets} For the sequence-based model, we used two synthetic datasets and two cross-language datasets. The statistics of the datasets are provided in Table \ref{table:statistics}:

\begin{table}[t]
    \centering
    \caption{The statistics of the datasets. The datasets satisfy $1 \leq \|\vx\|\|\vy\| \leq R $.}
    \label{table:statistics}
    \begin{tabular}{|c|c|c|c|c|c|}
    \hline
        Dataset & $n$ & $m_x$ & $m_y$ & $N$ & $R$ \\ \hline
        SYNTHETIC(1) & 100,000 & 1,000 & 2,000 & 50,000 & 65 \\ \hline
        SYNTHETIC(2) & 100,000 & 1,000 & 2,000 & 50,000 & 724 \\ \hline
        APR & 23,235 & 28,017 & 42,833 & 10,000 & 773 \\ \hline
        PAN11 & 88,977 & 5,121 & 9,959 & 10,000 & 5,548 \\ \hline
        EURO & 475,834 & 7,247 & 8,768 & 100,000 & 107,840 \\ \hline
    \end{tabular}
\end{table}

\begin{itemize}
    \item Synthetic: The elements of the two synthetic datasets are initially uniformly sampled from the range (0,1), then multiplied by a coefficient to adjust the maximum column squared norm $R$. The X matrix has 1,000 rows, and the Y matrix has 2,000 rows, each with 100,000 columns. The window size is set to 50,000.
    \item APR: The Amazon Product Reviews (APR) dataset is a publicly available collection containing product reviews and related information from the Amazon website. This dataset consists of millions of sentences in both English and French. We structured it into a review matrix where the X matrix has 28,017 rows, and the Y matrix has 42,833 rows, with both matrices sharing 23,235 columns. The window size is 10,000.
    \item PAN11: PANPC-11 (PAN11) is a dataset designed for text analysis, particularly for tasks such as plagiarism detection, author identification, and near-duplicate detection. The dataset includes texts in English and French. The X and Y matrices contain 5,121 and 9,959 rows, respectively, with both matrices having 88,977 columns. The window size is 10,000.
\end{itemize}
We evaluate the time-based model on another real-world dataset:
\begin{itemize}
    \item EURO: The Europarl (EURO) dataset is a widely used multilingual parallel corpus, comprising the proceedings of the European Parliament. We selected a subset of its English and French text portions. The X and Y matrices contain 7,247 and 8,768 rows, respectively, and both matrices share 475,834 columns. Timestamps are generated using the $Poisson$ $Arrival$ $Process$ with a rate parameter of $\lambda=2$. The window size is set to 100,000, with approximately 30,000 columns of data on average in each window.
\end{itemize}

\paragraph{Setup} For the sequence-based model, we compare the proposed hDS-COD and  aDS-COD with EH-COD~\cite{yao2024approximate} and DI-COD~\cite{yao2024approximate}. We do not consider the Sampling algorithm as a baseline, as its performance is inferior to that of EH-COD and DI-CID, as demonstrated in \cite{yao2024approximate}. %The hDS-COD is adjusted by the parameter $\ell$ and the maximum number of levels $L = \log{R}$, where $R$ is the prior estimate of the maximum squared column norm of the dataset. DI-COD similarly requires a prior estimate of $R$ to limit the maximum number of levels $L = \log{(R/\varepsilon})$. In contrast, aDS-COD and EH-COD do not require an estimate of $R$; their error-space balance is controlled by the parameter $\ell = \frac{1}{\varepsilon}$. 
For the time-based model, we compare the proposed hDS-COD and  aDS-COD with EH-COD and the Sampling algorithm since DI-COD cannot be applied to time-based sliding window model. To achieve the same error bound, the maximum number of levels for hDS-COD is set to $L = \log{(\varepsilon NR)}$, and the initial threshold for aDS-COD is set to $1$.

Our experiments aim to illustrate the trade-offs between space and approximation errors. The x-axis represents two metrics for space: final sketch size and total space cost. The final sketch size refers to the number of columns in the result sketches $\mA$ and $\mB$ generated by the algorithm, representing a compression ratio. The total space cost refers to the maximum space required during the algorithm's execution, measured by the number of columns.We evaluate the approximation performance of all algorithms based on correlation errors $\operatorname{corr-err}(\mathbf{X}_W \mathbf{Y}_W^\top, \mathbf{A} \mathbf{B}^\top)$, which is reflected on the y-axis. Every 1,000 iterations, all algorithms query the window and record the average and maximum errors across all sampled windows.

The experiments for all algorithms were conducted using MATLAB (R2023a), with all algorithms running on a Windows server equipped with 32GB of memory and a single processor of Intel i9-13900K.

\paragraph{Performance} Figure \ref{fig:error vs l} and Figure \ref{fig:error vs space} illustrate the space efficiency comparison of the algorithms on sequence-based datasets. Panels (a-d) show the average errors across all sampled windows, while panels (e-h) display the maximum errors.

Figure \ref{fig:error vs l} evaluates the compression effect of the final sketch. The hDS-COD, aDS-COD, and EH-COD show similar compression performances. But the DS series is more stable, particularly on the synthetic datasets, where they significantly outperform EH-COD and DI-COD. The performance of hDS-COD and aDS-COD is nearly the same, indicating that the adaptive threshold trick in aDS-COD does not have a noticeable negative impact on it, maintaining the same error as hDS-COD.

Figure \ref{fig:error vs space} measures the total space cost of the algorithms. hDS-COD and aDS-COD show a significant advantage over existing methods, as they can achieve the  $\varepsilon$-approximation error with much less space. For the same space cost, the correlation errors of hDS-COD and aDS-COD are much smaller than those of EH-COD and DI-COD. Also, aDS-COD has better space efficiency than hDS-COD because aDS only uses a single-level structure while hDS requires $\log R+1$ levels. We find that hDS-COD requires more space on  SYNTHETIC(2) dataset compared to SYNTHETIC(1) dataset. This phenomenon occurs because SYNTHETIC(2) dataset has a larger $R$, which confirms the dependence on $R$ as stated in Theorem~\ref{thm:hds}. 

Figure \ref{fig:time-based} compares the performance of algorithms on time-based windows. Panels (a) and (b) present the error against the final sketch size, which show that our aDS-COD and hDS-COD algorithms enjoy similar performance as EH-COD and significantly outperform the sampling algorithm. On the other hand, as shown in panels (c) and (d), our methods outperform baselines in terms of total space cost.

Software development is increasingly conceived as a collaboration activity between developers and AIs. Indeed, IDEs already implement features to enable interactive development, with AI suggesting implementations that are reused by developers.

Although multiple studies show this interaction can be successful, there is still limited understanding of how the models must be configured and used in the context of code generation tasks. This study addresses this gap, systematically investigating the impact of several key parameters, including the repeated submission of a prompt to accommodate for the non-deterministic nature of the models.

Our study reveals several key findings about the usage of ChatGPT. In particular, we discovered how creativity, although up to a limited extent, is useful to increase the range of methods whose code can be generated correctly. A major role is played by parameter top-p, which is commonly underrated, and instead has a major impact on the correctness of the results, with lower values producing better results. Finally, prompts should be submitted multiple times, with $5$ repetitions combined with a temperature of $1.2$ resulting in an effective configuration in our experiments.  

Future work concerns two main research directions. One is about replicating this experiment with other AI assistants, to validate our findings in multiple contexts. The second research direction concerns finding strategies to deal with the need to submit the same prompt multiple times to obtain a useful result, and thus developing approaches able to select or merge multiple responses automatically. 


\begin{acknowledgements}
We thank the anonymous reviewers whose comments have greatly improved this manuscript. We also thank Yuhao Kang for his help during the early phase of this work. This research was supported in part by NetApp, Cisco Systems, Exelon Utilities, HPC resources from GENCI–IDRIS (Grants 2020-101471, 2021-101925), and EU Horizon projects AI4Europe (101070000), TwinODIS (101160009), ARMADA
(101168951), DataGEMS (101188416) and RECITALS (101168490).
\end{acknowledgements}

% BibTeX users please use one of
%\bibliographystyle{spphys}      % basic style, author-year citations
\bibliographystyle{spmpsci}      % mathematics and physical sciences
%\bibliographystyle{spphys}       % APS-like style for physics
\bibliography{references}   % name your BibTeX data base


\end{document}
% end of file template.tex
