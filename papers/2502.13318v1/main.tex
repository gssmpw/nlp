%%%%%%%%%%%%%%%%%%%%%%% file template.tex %%%%%%%%%%%%%%%%%%%%%%%%%
%
% This is a general template file for the LaTeX package SVJour3
% for Springer journals.          Springer Heidelberg 2010/09/16
%
% Copy it to a new file with a new name and use it as the basis
% for your article. Delete % signs as needed.
%
% This template includes a few options for different layouts and
% content for various journals. Please consult a previous issue of
% your journal as needed.
%
%%%%%%%%%%%%%%%%%%%%%%%%%%%%%%%%%%%%%%%%%%%%%%%%%%%%%%%%%%%%%%%%%%%
%
% First comes an example EPS file -- just ignore it and
% proceed on the \documentclass line
% your LaTeX will extract the file if required
\begin{filecontents*}{example.eps}
%!PS-Adobe-3.0 EPSF-3.0
%%BoundingBox: 19 19 221 221
%%CreationDate: Mon Sep 29 1997
%%Creator: programmed by hand (JK)
%%EndComments
gsave
newpath
  20 20 moveto
  20 220 lineto
  220 220 lineto
  220 20 lineto
closepath
2 setlinewidth
gsave
  .4 setgray fill
grestore
stroke
grestore
\end{filecontents*}
%
\RequirePackage{fix-cm}
%
%\documentclass{svjour3}                     % onecolumn (standard format)
%\documentclass[smallcondensed]{svjour3}     % onecolumn (ditto)
%\documentclass[smallextended]{svjour3}       % onecolumn (second format)
\documentclass[twocolumn]{svjour3}          % twocolumn
%
\smartqed  % flush right qed marks, e.g. at end of proof
%
\usepackage{graphicx}
\usepackage{makecell}
\usepackage{amsmath,amssymb,amsfonts}
\usepackage{algorithmic}
\usepackage{graphicx}
\usepackage{textcomp}
\usepackage{xcolor}
\usepackage[title]{appendix}
\usepackage{array,multirow}
\usepackage[font=small]{caption}
\usepackage{balance}  % for  \balance command ON LAST PAGE  (only there!)

\usepackage{color, colortbl}
\definecolor{Gray}{gray}{0.8}
\definecolor{lGray}{gray}{0.9}
\usepackage{multicol}

\usepackage{enumitem}

% THIS COMMAND WILL SHRINK PAPER BY 2% BY REMOVING UNNEEDED SPACES!
%\renewcommand{\baselinestretch}{0.}

%EXTRA PACKAGE
\DeclareMathOperator*{\argmax2}{arg\,max}
\DeclareMathOperator*{\argmin2}{arg\,min}
\usepackage[linesnumbered,algoruled,boxed,lined,noend]{algorithm2e}
\newcommand\mycommfont[1]{\footnotesize\ttfamily\textcolor{blue}{#1}}
\SetCommentSty{mycommfont}


\newcommand{\commentRedXMARK}[1]{{\color[rgb]{1,0,0}#1}}


\newcommand{\commentBlue}[1]{{\color[rgb]{0,0,1}#1}}
\newcommand{\midtilde}{\raisebox{-0.25\baselineskip}{\textasciitilde}}
\usepackage{comment}
\usepackage{url}
%\newtheorem{problem}{Problem}
%\newtheorem{theorem}{Theorem}
%\newtheorem{definition}{Definition}
%\newtheorem{example}{Example}
%\newtheorem{lemma}{Lemma}
\newcommand{\anomaly}{{\textit{anomaly} }}
\newcommand{\discord}{{\textit{discord} }}
\newcommand{\outliers}{{\textit{outliers} }}
\newcommand{\Anomaly}{{\textit{Anomaly} }}
\newcommand{\Discord}{{\textit{Discord} }}
\newcommand{\Outliers}{{\textit{Outliers} }}





% themis red comment
\newcommand{\tp}[1]{{\color{red} {\bf ??? #1 ???}}\normalcolor}







\usepackage{tikz}
\newcommand{\commentRed}[1]{{\color[rgb]{0,0,0}~#1}}
\newcommand{\greencheck}{}%
\DeclareRobustCommand{\greencheck}{%
\textbf{
  \tikz\fill[scale=0.4, color=green]
  (0,.35) -- (.25,0) -- (1,.7) -- (.25,.15) -- cycle;%
}}
\usepackage{pifont}
\newcommand{\xmark}{\text{\ding{55}}}
\newcommand{\redmark}{\text{\commentRedXMARK{\xmark}}}%
%
% \usepackage{mathptmx}      % use Times fonts if available on your TeX system
%
% insert here the call for the packages your document requires
%\usepackage{latexsym}
% etc.
%
% please place your own definitions here and don't use \def but
% \newcommand{}{}
%
% Insert the name of "your journal" with
% \journalname{myjournal}
%
\begin{document}

\title{VUS: Effective and Efficient Accuracy Measures for Time-Series Anomaly Detection%\thanks{Grants or other notes
%about the article that should go on the front page should be
%placed here. General acknowledgments should be placed at the end of the article.}
}
%\subtitle{Do you have a subtitle?\\ If so, write it here}

%\titlerunning{Short form of title}        % if too long for running head

\author{Paul Boniol \and
        Ashwin K. Krishna \and
        Marine Bruel \and
        Qinghua Liu \and 
        Mingyi Huang \and 
        Themis Palpanas \and
        Ruey S. Tsay \and
        Aaron Elmore \and
        Michael J. Franklin \and
        John Paparrizos \and%etc.
}

%\authorrunning{Short form of author list} % if too long for running head

\institute{Paul Boniol \at
              45 rue d'Ulm, 75005, Paris \\
              \email{boniol.paul@gmail.com}
}

\date{Received: date / Accepted: date}
% The correct dates will be entered by the editor


\maketitle

\begin{abstract}
Anomaly detection (AD) is a fundamental task for time-series analytics with important implications for the downstream performance of many applications. In contrast to other domains where AD mainly focuses on point-based anomalies (i.e., outliers in standalone observations), AD for time series is also concerned with range-based anomalies (i.e., outliers spanning multiple observations). Nevertheless, it is common to use traditional point-based information retrieval measures, such as Precision, Recall, and F-score, to assess the quality of methods by thresholding the anomaly score to mark each point as an anomaly or not. However, mapping discrete labels into continuous data introduces unavoidable shortcomings, complicating the evaluation of range-based anomalies. Notably, the choice of evaluation measure may significantly bias the experimental outcome. Despite over six decades of attention, there has never been a large-scale systematic quantitative and qualitative analysis of time-series AD evaluation measures. This paper extensively evaluates quality measures for time-series AD to assess their robustness under noise, misalignments, and different anomaly cardinality ratios. Our results indicate that measures producing quality values independently of a threshold (i.e., AUC-ROC and AUC-PR) are more suitable for time-series AD. Motivated by this observation, we first extend the AUC-based measures to account for range-based anomalies. Then, we introduce a new family of parameter-free and threshold-independent measures, Volume Under the Surface (VUS), to evaluate methods while varying parameters. We also introduce two optimized implementations for VUS that reduce significantly the execution time of the initial implementation. Our findings demonstrate that our four measures are significantly more robust in assessing the quality of time-series AD methods.
%\keywords{Time Series \and Anomaly Detection \and Evaluation Measures}
% \PACS{PACS code1 \and PACS code2 \and more}
% \subclass{MSC code1 \and MSC code2 \and more}
\end{abstract}



\section{Introduction}

Large language models (LLMs) have achieved remarkable success in automated math problem solving, particularly through code-generation capabilities integrated with proof assistants~\citep{lean,isabelle,POT,autoformalization,MATH}. Although LLMs excel at generating solution steps and correct answers in algebra and calculus~\citep{math_solving}, their unimodal nature limits performance in plane geometry, where solution depends on both diagram and text~\citep{math_solving}. 

Specialized vision-language models (VLMs) have accordingly been developed for plane geometry problem solving (PGPS)~\citep{geoqa,unigeo,intergps,pgps,GOLD,LANS,geox}. Yet, it remains unclear whether these models genuinely leverage diagrams or rely almost exclusively on textual features. This ambiguity arises because existing PGPS datasets typically embed sufficient geometric details within problem statements, potentially making the vision encoder unnecessary~\citep{GOLD}. \cref{fig:pgps_examples} illustrates example questions from GeoQA and PGPS9K, where solutions can be derived without referencing the diagrams.

\begin{figure}
    \centering
    \begin{subfigure}[t]{.49\linewidth}
        \centering
        \includegraphics[width=\linewidth]{latex/figures/images/geoqa_example.pdf}
        \caption{GeoQA}
        \label{fig:geoqa_example}
    \end{subfigure}
    \begin{subfigure}[t]{.48\linewidth}
        \centering
        \includegraphics[width=\linewidth]{latex/figures/images/pgps_example.pdf}
        \caption{PGPS9K}
        \label{fig:pgps9k_example}
    \end{subfigure}
    \caption{
    Examples of diagram-caption pairs and their solution steps written in formal languages from GeoQA and PGPS9k datasets. In the problem description, the visual geometric premises and numerical variables are highlighted in green and red, respectively. A significant difference in the style of the diagram and formal language can be observable. %, along with the differences in formal languages supported by the corresponding datasets.
    \label{fig:pgps_examples}
    }
\end{figure}



We propose a new benchmark created via a synthetic data engine, which systematically evaluates the ability of VLM vision encoders to recognize geometric premises. Our empirical findings reveal that previously suggested self-supervised learning (SSL) approaches, e.g., vector quantized variataional auto-encoder (VQ-VAE)~\citep{unimath} and masked auto-encoder (MAE)~\citep{scagps,geox}, and widely adopted encoders, e.g., OpenCLIP~\citep{clip} and DinoV2~\citep{dinov2}, struggle to detect geometric features such as perpendicularity and degrees. 

To this end, we propose \geoclip{}, a model pre-trained on a large corpus of synthetic diagram–caption pairs. By varying diagram styles (e.g., color, font size, resolution, line width), \geoclip{} learns robust geometric representations and outperforms prior SSL-based methods on our benchmark. Building on \geoclip{}, we introduce a few-shot domain adaptation technique that efficiently transfers the recognition ability to real-world diagrams. We further combine this domain-adapted GeoCLIP with an LLM, forming a domain-agnostic VLM for solving PGPS tasks in MathVerse~\citep{mathverse}. 
%To accommodate diverse diagram styles and solution formats, we unify the solution program languages across multiple PGPS datasets, ensuring comprehensive evaluation. 

In our experiments on MathVerse~\citep{mathverse}, which encompasses diverse plane geometry tasks and diagram styles, our VLM with a domain-adapted \geoclip{} consistently outperforms both task-specific PGPS models and generalist VLMs. 
% In particular, it achieves higher accuracy on tasks requiring geometric-feature recognition, even when critical numerical measurements are moved from text to diagrams. 
Ablation studies confirm the effectiveness of our domain adaptation strategy, showing improvements in optical character recognition (OCR)-based tasks and robust diagram embeddings across different styles. 
% By unifying the solution program languages of existing datasets and incorporating OCR capability, we enable a single VLM, named \geovlm{}, to handle a broad class of plane geometry problems.

% Contributions
We summarize the contributions as follows:
We propose a novel benchmark for systematically assessing how well vision encoders recognize geometric premises in plane geometry diagrams~(\cref{sec:visual_feature}); We introduce \geoclip{}, a vision encoder capable of accurately detecting visual geometric premises~(\cref{sec:geoclip}), and a few-shot domain adaptation technique that efficiently transfers this capability across different diagram styles (\cref{sec:domain_adaptation});
We show that our VLM, incorporating domain-adapted GeoCLIP, surpasses existing specialized PGPS VLMs and generalist VLMs on the MathVerse benchmark~(\cref{sec:experiments}) and effectively interprets diverse diagram styles~(\cref{sec:abl}).

\iffalse
\begin{itemize}
    \item We propose a novel benchmark for systematically assessing how well vision encoders recognize geometric premises, e.g., perpendicularity and angle measures, in plane geometry diagrams.
	\item We introduce \geoclip{}, a vision encoder capable of accurately detecting visual geometric premises, and a few-shot domain adaptation technique that efficiently transfers this capability across different diagram styles.
	\item We show that our final VLM, incorporating GeoCLIP-DA, effectively interprets diverse diagram styles and achieves state-of-the-art performance on the MathVerse benchmark, surpassing existing specialized PGPS models and generalist VLM models.
\end{itemize}
\fi

\iffalse

Large language models (LLMs) have made significant strides in automated math word problem solving. In particular, their code-generation capabilities combined with proof assistants~\citep{lean,isabelle} help minimize computational errors~\citep{POT}, improve solution precision~\citep{autoformalization}, and offer rigorous feedback and evaluation~\citep{MATH}. Although LLMs excel in generating solution steps and correct answers for algebra and calculus~\citep{math_solving}, their uni-modal nature limits performance in domains like plane geometry, where both diagrams and text are vital.

Plane geometry problem solving (PGPS) tasks typically include diagrams and textual descriptions, requiring solvers to interpret premises from both sources. To facilitate automated solutions for these problems, several studies have introduced formal languages tailored for plane geometry to represent solution steps as a program with training datasets composed of diagrams, textual descriptions, and solution programs~\citep{geoqa,unigeo,intergps,pgps}. Building on these datasets, a number of PGPS specialized vision-language models (VLMs) have been developed so far~\citep{GOLD, LANS, geox}.

Most existing VLMs, however, fail to use diagrams when solving geometry problems. Well-known PGPS datasets such as GeoQA~\citep{geoqa}, UniGeo~\citep{unigeo}, and PGPS9K~\citep{pgps}, can be solved without accessing diagrams, as their problem descriptions often contain all geometric information. \cref{fig:pgps_examples} shows an example from GeoQA and PGPS9K datasets, where one can deduce the solution steps without knowing the diagrams. 
As a result, models trained on these datasets rely almost exclusively on textual information, leaving the vision encoder under-utilized~\citep{GOLD}. 
Consequently, the VLMs trained on these datasets cannot solve the plane geometry problem when necessary geometric properties or relations are excluded from the problem statement.

Some studies seek to enhance the recognition of geometric premises from a diagram by directly predicting the premises from the diagram~\citep{GOLD, intergps} or as an auxiliary task for vision encoders~\citep{geoqa,geoqa-plus}. However, these approaches remain highly domain-specific because the labels for training are difficult to obtain, thus limiting generalization across different domains. While self-supervised learning (SSL) methods that depend exclusively on geometric diagrams, e.g., vector quantized variational auto-encoder (VQ-VAE)~\citep{unimath} and masked auto-encoder (MAE)~\citep{scagps,geox}, have also been explored, the effectiveness of the SSL approaches on recognizing geometric features has not been thoroughly investigated.

We introduce a benchmark constructed with a synthetic data engine to evaluate the effectiveness of SSL approaches in recognizing geometric premises from diagrams. Our empirical results with the proposed benchmark show that the vision encoders trained with SSL methods fail to capture visual \geofeat{}s such as perpendicularity between two lines and angle measure.
Furthermore, we find that the pre-trained vision encoders often used in general-purpose VLMs, e.g., OpenCLIP~\citep{clip} and DinoV2~\citep{dinov2}, fail to recognize geometric premises from diagrams.

To improve the vision encoder for PGPS, we propose \geoclip{}, a model trained with a massive amount of diagram-caption pairs.
Since the amount of diagram-caption pairs in existing benchmarks is often limited, we develop a plane diagram generator that can randomly sample plane geometry problems with the help of existing proof assistant~\citep{alphageometry}.
To make \geoclip{} robust against different styles, we vary the visual properties of diagrams, such as color, font size, resolution, and line width.
We show that \geoclip{} performs better than the other SSL approaches and commonly used vision encoders on the newly proposed benchmark.

Another major challenge in PGPS is developing a domain-agnostic VLM capable of handling multiple PGPS benchmarks. As shown in \cref{fig:pgps_examples}, the main difficulties arise from variations in diagram styles. 
To address the issue, we propose a few-shot domain adaptation technique for \geoclip{} which transfers its visual \geofeat{} perception from the synthetic diagrams to the real-world diagrams efficiently. 

We study the efficacy of the domain adapted \geoclip{} on PGPS when equipped with the language model. To be specific, we compare the VLM with the previous PGPS models on MathVerse~\citep{mathverse}, which is designed to evaluate both the PGPS and visual \geofeat{} perception performance on various domains.
While previous PGPS models are inapplicable to certain types of MathVerse problems, we modify the prediction target and unify the solution program languages of the existing PGPS training data to make our VLM applicable to all types of MathVerse problems.
Results on MathVerse demonstrate that our VLM more effectively integrates diagrammatic information and remains robust under conditions of various diagram styles.

\begin{itemize}
    \item We propose a benchmark to measure the visual \geofeat{} recognition performance of different vision encoders.
    % \item \sh{We introduce geometric CLIP (\geoclip{} and train the VLM equipped with \geoclip{} to predict both solution steps and the numerical measurements of the problem.}
    \item We introduce \geoclip{}, a vision encoder which can accurately recognize visual \geofeat{}s and a few-shot domain adaptation technique which can transfer such ability to different domains efficiently. 
    % \item \sh{We develop our final PGPS model, \geovlm{}, by adapting \geoclip{} to different domains and training with unified languages of solution program data.}
    % We develop a domain-agnostic VLM, namely \geovlm{}, by applying a simple yet effective domain adaptation method to \geoclip{} and training on the refined training data.
    \item We demonstrate our VLM equipped with GeoCLIP-DA effectively interprets diverse diagram styles, achieving superior performance on MathVerse compared to the existing PGPS models.
\end{itemize}

\fi 


\section{Background and Related Work}
\label{sec:background}

We first introduce
notations useful for the rest of the paper (Section~\ref{sec:notation}). Then, we review
existing time-series clustering methods and discuss their limitations related to interpretability and user interaction (Section~\ref{sec:clus_methods}). We then motivate and introduce the usage of Graph embedding for time series clustering (Section~\ref{sec:graphforts}). We finally properly define the problem tackled in this paper (Section~\ref{sec:Probdef}).

\subsection{Time Series and Graph Notation}
\label{sec:notation}

\textbf{Time Series: }A \rev{univariate } time series $T \in \mathbb{R}^n $ is a sequence of real-valued numbers $T_i\in\mathbb{R}$ $[T_1,T_2,...,T_n]$, where $n=|T|$ is the length of $T$, and $T_i$ is the $i^{th}$ point of $T$. \rev{In the rest of this paper , we refer to univariate time series as {\it time series}. } We are typically interested in local regions of the time series, known as subsequences. A subsequence $T_{i,\ell} \in \mathbb{R}^\ell$ of a time series $T$ is a continuous subset of the values of $T$ of length $\ell$ starting at position $i$. Formally, $T_{i,\ell} = [T_i, T_{i+1},...,T_{i+\ell-1}]$.
A dataset $\mathcal{D}$ is a set of time series (possibly of different lengths). 
We define the size of $\mathcal{D}$ as $|\mathcal{D}|$.

\noindent \textbf{Graph: }
We introduce some basic definitions for graphs, which we will use in this paper.
We define a Node Set $\mathcal{N}$ as a set of unique integers.
Given a Node Set $\mathcal{N}$, an Edge Set $\mathcal{E}$ is then a set composed of tuples $(x_i,x_j)$, where $x_i,x_j \in \mathcal{N}$. 
Given a Node Set $\mathcal{N}$, an Edge Set $\mathcal{E}$ (pairs of nodes in $\mathcal{N}$), a Graph $\mathcal{G}$ is an ordered pair $\mathcal{G}=(\mathcal{N},\mathcal{E})$.
A directed graph $\mathcal{G}$ is an ordered pair $\mathcal{G}=(\mathcal{N},\mathcal{E})$ where $\mathcal{N}$ is a Node Set, and $\mathcal{E}$ is an ordered Edge Set.
In the rest of this paper, we will only use directed graphs, denoted as $\mathcal{G}$.

\subsection{Time Series Clustering}
\label{sec:clus_methods}
Time series clustering plays a pivotal role in uncovering meaningful patterns within temporal datasets, where the primary goal is to group similar time series for insightful analysis. The specific challenge addressed by partitional time series clustering involves partitioning a dataset of $n$ time series, denoted as $D$, into $k$ clusters $(C = \{C_{1}, C_{2}, ..., C_{k}\})$, where the number of clusters, $k$, is predetermined. Although the choice of $k$ can be determined using wrapper methods, it is assumed to be fixed in advance in many experiments. 

\subsubsection{Raw-Based Approaches}

Clustering algorithms for time series can either operate directly on raw data or use transformations to derive features before clustering. Algorithms specifically designed for time series often prefer using raw data~\cite{liao2005clustering} for several reasons. First, \textbf{Preservation of Information}: raw time series preserve intricate details of temporal patterns, ensuring information fidelity. Second, \textbf{Temporal Dependencies}: they capture dynamic patterns and temporal interdependencies that might be lost during feature extraction. Finally, \textbf{Data Exploration}: direct analysis of raw time series fosters the discovery of unexpected patterns and trends, supporting an exploratory, discovery-driven approach.

In the raw-based approach, the $k$-Means algorithm is commonly used to identify patterns and similarities within temporal data. Each time series is treated as a multidimensional vector, with data points representing observations over time. The goal is to partition the time series into $k$ clusters, optimizing the assignment of series to cluster centers.

$k$-Shape~\cite{10.1145/2949741.2949758} is a well-known algorithm for time series clustering in this approach. It starts by randomly selecting initial cluster centers and uses a specialized distance metric to measure dissimilarity based on shapes, addressing temporal dynamics. The algorithm iteratively assigns each time series to the cluster with the smallest shape-based distance and updates cluster centers based on the mean shape. This process continues until convergence, resulting in clusters that highlight shape similarities and provide insights into temporal dynamics.

\rev{A recent study~\cite{10.14778/3611540.3611622} has compared $k$-Shape with a large amount of clustering baselines and has demonstrated that none of the recent baselines outperform significantly $k$-Shape. }
\rev{Finally, several raw-basd appraoch are using the Dynamic Time Warping (DTW), a well-known algorithm for measuring the similarity between two time series by aligning them through time axis warping~\cite{10.1145/2783258.2783286}. This process minimizes the distance between sequences, accommodating shifts and stretches in the time dimension. Although DTW is highly accurate, it is computationally intensive, often necessitating optimizations for improved efficiency. SOMTimeS~\cite{10.1007/s10618-023-00979-9} is an algorithm that addresses this challenge. It is a self-organizing map (SOM) designed for clustering time series data, utilizing DTW as a distance measure to effectively align and compare time series. SOMTimeS enhances computational efficiency by implementing a pruning strategy that significantly reduces the number of DTW calculations required during the training.} 

\dt{However, it is important to note that raw-based approaches may face challenges when dealing with noise in raw time series, potentially obscuring meaningful patterns. Additionally, the direct clustering of raw time series might result in clusters lacking clear distinctions or meaningful insights, making extracting valuable information from the data challenging.}

\subsubsection{Feature-Based Approaches}
Adopting a feature-based \cite{wang2006characteristic} approach offers several advantages that address the challenges associated with raw-based methods: 
\textbf{Dimensionality Reduction}, feature extraction often reduces dimensionality. Indeed, lower-dimensional feature representations enhance computational efficiency and reduce the risk of the curse of dimensionality; \textbf{Enhanced Discrimination}, feature selection can be tailored to emphasize specific characteristics crucial for discrimination, enhancing the ability of clustering algorithms to distinguish subtle differences between time series, leading to more accurate clustering; and \textbf{Interpretability}, clusters derived from features often yield more interpretable results than those directly from raw time series. Interpretability is crucial for extracting insights, and features-based approaches provide clearer explanations for grouping.

\rev{TS3C~\cite{8960542} is an example of feature-based clustering approach. The latter segments the time series and extracts a set of statistical features to represent them. The features are then used to cluster time series. }
FeatTS~\cite{tiano2021featts} emerges as a feature-based algorithm tailored for univariate time series clustering. It leverages the TSFresh library and employs Principal Features Analysis to extract salient features from time series data. 
Building upon this foundation, the authors introduce Time2Feat (T2F)~\cite{bonifati2022time2feat}, an algorithm designed explicitly for multivariate time series clustering. T2F distinguishes itself by adopting two distinct feature extraction approaches named intra and inter-signal feature extraction. 
Furthermore, the authors enhance the feature selection process by introducing a grid search method for selecting optimal features for clustering multivariate time series.
FeatTS and Time2Feat are two approaches that accommodate unsupervised and semi-supervised clustering scenarios. Notably, Time2Feat stands out as the current state-of-the-art solution for multivariate time series clustering, showcasing the advancements in feature extraction and selection methodologies.

\dt{Nevertheless, feature-based approaches may suffer from the information loss induced by the transformation of the original time series (i.e., sequential patterns) into features. 
Handling high-dimensional feature spaces poses a challenge, especially when dealing with numerous derived features. This complexity can compromise the interpretability of the algorithm, making it difficult for users to examine all the extracted features.
Additionally, the sensitivity of feature engineering poses a challenge, where selecting inappropriate features or applying unsuitable transformations may affect clustering quality.}


\subsubsection{Deep Learning Approaches}

In recent years, deep learning strategies~\cite{alqahtani2021deep} have emerged to address the challenges of time series clustering, demonstrating strong performance. These approaches leverage deep learning's ability to directly interface with time series data, marking a significant shift from traditional methods. A key advantage is the \textbf{Hierarchy of Abstraction}, where deep learning architectures capture complex relationships within temporal data, revealing intricate patterns essential for effective clustering. Additionally, these methods exhibit \textbf{High Efficacy}, surpassing traditional techniques with superior accuracy and efficiency.

The Deep Auto-Encoder (DAE)~\cite{tian2014learning} is a popular solution, serving as an unsupervised model for representation learning. It transforms raw input data into new space representations, extracting valuable features through encoding. The DAE architecture, characterized by seven fully connected layers, effectively harnesses learned features through an internal layer. These features are subsequently input into a clustering loss function, minimizing the distance between data points and their respective assigned cluster centers.

Another approach is Deep Temporal Clustering (DTC)~\cite{DTCAlgorithm}, which uses the DAE for feature representation and clustering. The DTC's clustering layer optimizes the Kullback-Leibler (KL) divergence objective, aligning with a self-training target distribution. The encoding process influences clustering performance, with learned representations fed into the $k$-Means algorithm for final clustering.

\dt{In conclusion, despite the high efficacy and the ability to uncover complex patterns, deep learning approaches encounter fundamental challenges related to the interpretability of their decision-making processes. The inherent complexity of these machine learning models often results in a lack of transparency in understanding obtained results. Furthermore, these approaches may struggle to integrate domain-specific knowledge seamlessly, presenting obstacles to guiding the clustering process based on expert insights.}


\subsection{Graph Embedding of Time Series}
\label{sec:graphforts}


We argue that to address the challenges of maintaining information preservation while maximizing interpretability, a viable solution is to represent time series as a (suitably constructed) graph. 
Constructing such a graph involves processing subsequences of the time series dataset. 
These subsequences represent various types of patterns and their temporal succession. This approach furnishes clustering methods with substantial information, contributing to the sustenance of high accuracy. Additionally, this approach facilitates user-friendly navigation through the time series.

Various methods have been proposed to convert time series into graphs for specific analytical tasks to overcome the challenges mentioned earlier. 
Series2Graph embeds univariate time series into a directed graph~\cite{Series2GraphPaper, DistributedS2G}, primarily employed for anomaly detection. Similarly, approaches like Time2Graph~\cite{9477138} utilize graph representations of time series to address time series classification.
The advantages of such time series graph representations are threefold. 
First, these graph representations are easily interpretable by any user. Second, constructed directly from subsequences of the time series, they preserve essential information. Lastly, a unified embedding can significantly reduce execution time, as evidenced in anomaly detection scenarios~\cite{DistributedS2G}.

However, prior works on graph embedding for time series were often proposed either under supervised settings~\cite{9477138}, simplifying the graph construction task, or in an unsupervised manner but for continuous time series~\cite{Series2GraphPaper}.
For the specific task of time series clustering, a graph embedding of time series necessitates the unsupervised construction of a single, comprehensive graph for an entire dataset, encompassing multiple continuous time series. While a straightforward solution would be building one graph per time series, this approach diminishes the interpretability advantage of having a single, concise graph.


\subsection{Problem Formulation}
\label{sec:Probdef}

\begin{figure}[tb]
 \centering
\includegraphics[width=\linewidth]{figures/graphoid.pdf}

 \caption{$\lambda$-Graphoids and $\gamma$-Graphoids for different $\lambda$ and $\gamma$.}
 \label{fig:def}
\vspace{-0.3cm}
\end{figure}

We propose a novel 
approach to time series clustering by employing a graph embedding. The essence of this methodology lies in transforming a time series dataset into a sequence of abstract states corresponding to different subsequences within the dataset. These states are represented as nodes, denoted by $\mathcal{N}$, in a directed graph, $\mathcal{G}=(\mathcal{N},\mathcal{E})$. The edges, $\mathcal{E}$, encode the frequency with which one state occurs after another~\cite{Series2GraphPaper}. We define this graph as follows:

\begin{definition}[Graph Embedding]   
Let a time series dataset be defined as $\mathcal{D} = \{T_1,T_2,...,T_n\}$. Let $\mathcal{G}=(\mathcal{N},\mathcal{E})$ be a directed graph. 
$\mathcal{G}$ is the graph embedding of $\mathcal{D}$ if there exists a function $\mathcal{M}$ such that for any $T \in \mathcal{D}$, $\mathcal{M}(T) =  \langle N^{(1)},N^{(2)},...,N^{(m)} \rangle $, and  $\forall i \in [1,m], N^{(i)} \in \mathcal{N}$, and $\forall i \in [1,m-1], (N^{(i)},N^{(i+1)}) \in \mathcal{E}$. 
\label{defGraph}  
\end{definition}

The rest of the section considers $\mathcal{M}(T)$ as a subgraph.
For the sake of interpretability, it is imperative that different sections of the graph distinctly capture the similarity between time series. Nodes, representing similar patterns from various time series, and edges, denoting possible transitions between these patterns, play a pivotal role in this context. Consequently, a comparable time series is found to be placed within the same region of the graph.
As a result, in a given dataset $\mathcal{D} = \{T_0, T_1, ..., T_n\}$, a designated cluster of time series, denoted as $C_i \subset \mathcal{D}$, corresponds to a discernible subgraph $\mathcal{G}_{C_i} \subset \mathcal{G}$. This subgraph is formally called a ``\textit{Graphoid}."


\begin{definition}[$Graphoid$]   
Let $\mathcal{D}$ be a time series dataset and $C_i \subset \mathcal{D}$ a given cluster such as $C_i = \{T_1,T_2,...,T_{k'}\}$. Let $\mathcal{G}$ be the graph embedding of $\mathcal{D}$ resulting from a function $\mathcal{M}$. We define the {\it Graphoid} of $C_i$ as:
{\small
\[
\mathcal{G}_{C_i} = \bigcup_{T \in C_i} \mathcal{M}(T)
\]
}
\label{defGraphoid}  
\end{definition}

In Definition~\ref{defGraphoid}, the {\it Graphoid} of a given cluster contains all the nodes and edges that at least one time series of that cluster crossed. Therefore, a node of $\mathcal{G}$ may belong to multiple graphoids. As a consequence, no distinction is made between nodes that contain subsequences of all time series of one cluster (i.e., {\bf representativity}) or time series subsequences of one cluster only (i.e., {\bf exclusivity}) and nodes that are contained in all clusters. We thus introduce the two following definitions:

\begin{definition}[Node representativity]   
Let $\mathcal{D}$ a time series dataset and $C = \{C_1,C_2,...,C_k\}$ a clustering partition. Let $\mathcal{G}=(\mathcal{N},\mathcal{E})$ be the graph embedding of $\mathcal{D}$ resulting from a function $\mathcal{M}$. We define the {\it Representativity} of node $N \in \mathcal{N}$ as $Re(N) = (|N|_{C_1}, ..., |N|_{C_k}$ with $|N|_{C_i}$ defined as:
{\small
\[
|N|_{C_i} = \frac{1}{{|C_i|}}\sum_{T \in C_i} 1_{[N \in \mathcal{M}(T)]}
\]
}
\label{defnodeRep}  
\end{definition}

\begin{definition}[Node Exclusivity]   
Let $\mathcal{D}$ a time series dataset and $C = \{C_1,C_2,...,C_k\}$ a clustering partition. Let $\mathcal{G}=(\mathcal{N},\mathcal{E})$ be the graph embedding of $\mathcal{D}$ resulting from a function $\mathcal{M}$. We define the {\it Exclusivity} of node $N \in \mathcal{N}$ as $Ex(N) = (Pr_{C_1}(N), ..., Pr_{C_k}(N))$ with $Pr_{C_i}(N)$ defined as:
{\small
\[
Pr_{C_i}(N) = \frac{|C_i||N|_{C_i}}{\sum_{T \in \mathcal{D}} 1_{[N \in \mathcal{M}(T)]}}
\]
}
\label{defnodeEx}  
\end{definition}

In other words, the {\bf representativity} of a node is the number of time series of a given cluster that crossed the node divided by the total number of time series within that cluster. 
The {\bf exclusivity} of a node is the number of time series of a given cluster that crossed the node divided by the total number of time series that crossed that same node. The same definitions can be used for edges. 
Based on the above definitions, we can restrict the definition of a {\it Graphoid} based on exclusivity and representativity. 
We thus introduce $\lambda$-$Graphoid$ and $\gamma$-$Graphoid$ defined as follows:

\begin{definition}[$\lambda$-$Graphoid$]   
For a given dataset $\mathcal{D}$, $\mathcal{G}$ the graph embedding of $\mathcal{D}$, and a given cluster $C_i$. The $\lambda$-$Graphoid$ of $C_i$ is defined as $\mathcal{G}^{\lambda}_{C_i} = (\mathcal{N}^{\lambda}_{C_i},\mathcal{E}^{\lambda}_{C_i})$ such as $\forall N \in \mathcal{N}^{\lambda}_{C_i}, \forall E \in \mathcal{E}^{\lambda}_{C_i}, Pr_{C_i}(N) \geq \lambda$ and $Pr_{C_i}(E) \geq \lambda$.
\label{deflambdaGraph}  
\end{definition}

\begin{definition}[$\gamma$-$Graphoid$]   
For a given dataset $\mathcal{D}$, $\mathcal{G}$ the graph embedding of $\mathcal{D}$, and a given cluster $C_i$. The $\gamma$-$Graphoid$ of $C_i$ is defined as $\mathcal{G}^{\gamma}_{C_i} = (\mathcal{N}^{\gamma}_{C_i},\mathcal{E}^{\gamma}_{C_i})$ such as $\forall N \in \mathcal{N}^{\gamma}_{C_i}, \forall E \in \mathcal{E}^{\gamma}_{C_i}, |N|_{C_i} \geq \gamma$ and $|E|_{C_i} \geq \gamma$.
\label{defgammaGraph}  
\end{definition}

The concepts introduced above can be better illustrated using Figure~\ref{fig:def}. The $\lambda$-$Graphoid$ and $\gamma$-$Graphoid$ are influenced by the chosen values of $\lambda$ and $\gamma$. As demonstrated in Figure~\ref{fig:def}, higher values of $\lambda$ and $\gamma$ lead to more restrictive graphoids. In the illustration, nodes $N^{(5)},N^{(6)},N^{(7)}$ are exclusively crossed by time series from cluster $C_1$, highlighting unique patterns specific to this cluster. On the other hand, node $N^{(3)}$ is traversed by all time series of $C_1$ and is considered the most representative pattern for this cluster. However, it also represents a common pattern for $C_0$.

This scenario emphasizes the trade-off between representativity and exclusivity. While $N^{(3)}$ provides a comprehensive representation of $C_1$, it lacks exclusivity to this cluster. In contrast, $N^{(5)}, N^{(6)}, N^{(7)}$ offer exclusive patterns for $C_1$ but might not be present in all time series of this cluster, limiting the interpretability. Thus, finding an optimal balance between higher values of $\lambda$ and $\gamma$ is crucial for maximizing the interpretability of a clustering partition through graph embedding.
Based on the above definitions, we can state the following:

\begin{lemma}
    For a given clustering partition $C = \{C_1,C_2,...,C_k\}$, if $\lambda \leq k$, then $\bigcup_{C_i \in C} \mathcal{G}^{\lambda}_{C_i} = \mathcal{G}$.
    if $\lambda > 0.5$, then $\bigcap_{C_i \in C} \mathcal{G}^{\lambda}_{C_i} = \emptyset$.
    \label{Lemma1} 
\end{lemma} 

\begin{lemma}
    For a given clustering partition $C = \{C_1,C_2,...,C_k\}$, if $\forall C_i \in C, \mathcal{G}^{\lambda=1}_{C_i} = \mathcal{G}^{\gamma=\frac{1}{|C_i|}}_{C_i}$, then $\bigcap_{C_i \in C} \mathcal{G}_{C_i} = \emptyset$
    \label{Lemma2} 
\end{lemma} 

Lemma~\ref{Lemma2} corresponds to the perfect partition, in which each graph region exclusively represents one cluster. 
However, we do not need to have all the nodes of a {\it Graphoid} exclusively representing only one cluster. 
It is sufficient to have one node for each cluster with $|N|_{C_i} = Pr_{C_i}(N) = 1$.  
Therefore, the problem we want to solve is the following.

\begin{problem}[Time Series Graph Clustering]
	Given a dataset $\mathcal{D}$, automatically construct the graph $\mathcal{G}(\mathcal{N},\mathcal{E})$, and compute a clustering partition $C = \{C_1,...,C_k\}$ of $\mathcal{D}$, such that:  
{\small
 \[
 \forall C_i \in C, |\mathcal{G}^{\lambda=1}_{C_i}  \cap \mathcal{G}^{\gamma=1}_{C_i} |> 0
 \]
 }
 \label{problem}
\end{problem}

As this problem is impossible to solve in some use cases, the objective is to find the largest possible values of $\lambda$ and $\gamma$, such that the condition in Problem~\ref{problem} holds and the values of $\lambda$ and $\gamma$ indicate the quality of the clustering interpretability. Table~\ref{SymbolTable} summarizes the symbols used in this paper.

\begin{table}[tb]
\centering
\begin{tabular}{c|c}
\hline
{\bf Symbol} & {\bf Description} \\
\hline
$T$									& a time series (of length $|T|$) \\
$\ell$ 								& subsequence length\\
$\mathcal{D}$ 						& a dataset of time series\\
$C_i$							    & a cluster of a clustering partition $C$\\
$\mathcal{L}$						& labels generated by clustering\\
$k$							        & number of clusters\\
$\mathcal{N},\mathcal{E}$			& set of nodes and edges\\
$d$			& degrees of a set of nodes\\
$\mathcal{G}$			            & directed graph\\
$\mathcal{M}$ 						& function that transform $T$ into $\mathcal{G}$ \\
$\mathcal{G}_{\ell}$ 				& graph embedding built with length $\ell$ \\
$F_{\mathcal{G},\ell}$ 						& feature matrix for graph $\mathcal{G}_\ell$ \\
$M_C$ 								& consensus matrix \\
$\mathcal{G}_{C_i}$,$\mathcal{G}^{\lambda}_{C_i},\mathcal{G}^{\gamma}_{C_i}$ 		& graphoid, $\lambda$-graphoid and $\gamma$-graphoid of $C_i$ \\
$|N|_{C_i}$, $Pr_{C_i}(N)$						& representativity, exclusivity of node $N$ for $C_i$\\
\hline
\end{tabular}
\caption{Table of symbols}
\label{SymbolTable}
\end{table}
%We define the optimization problem as follows:
%\begin{equation}
%	\underset{x \in \mathbb{R}^n}{\text{min}} f(\Theta)
%\end{equation}
%
%This can be decomposed to the following:
%\begin{equation*}
%\underset{\bar{\mathbf{s}}\in \mathbb{R}^n}{\text{min}} \bar{\mathbf{m}}(\bar{\mathbf{s}}) = \sum\limits_{i=1}^{n} \underset{\bar{\mathbf{s}}_i}{\text{min}} \left( \bar{\mathbf{g}}_i\bar{\mathbf{s}}_i  + \frac{\lambda_i}{2} \bar{\mathbf{s}}_i^2 + \frac{\mu}{3} |\bar{\mathbf{s}_i}|^3\right)
%\end{equation*}
%
%We re-write the solution from Algorithm \ref{alg:LSR1ARC} as $\bar{\mathbf{s}}^{*} = - \mathbf{C}\bar{\mathbf{g}}$, where $\mathbf{C} = diag(c_1, \ldots, c_n)$ and $c_i = \frac{2}{\lambda_i + \sqrt{\lambda_i^2 + 4\mu|\bar{\mathbf{g}}_{i}|}}$. We provide the complete formulation and solution to the subproblem in sections \ref{sec:B0}, \ref{sec:limsetting} and \ref{sec:closedformsolve}
%
%\subsection{Limited-memory}\label{sec:limsetting}
%We acknowledge that a $n \times n$ Hessian approximation cannot be stored. Hence we propose a limited-memory solution. We use the compact representation equation (\ref{eqn:compactSR1}) to define our approximation update. Now, $\Psi$ is an $m \times n$ matrix where $m \ll n$. We perform a `thin' QR-decomposition on $\Psi = \mathbf{Q}\mathbf{R}$. This yields
%
%
%\begin{equation*}	
%\mathbf{B}_{k+1} \ = \ \mathbf{B}_0 + 
%	\begin{bmatrix}
%	\\
%	\mathbf{Q}_{k+1}\mathbf{R}_{k+1}  \\
%	\phantom{t}
%	\end{bmatrix}
%	\hspace{-.3cm}
%	\begin{array}{c}
%	\left  [ \  \mathbf{M}_{k+1}^{\phantom{h}}  \right ] \\
%	\\
%	\\
%	\end{array}
%	\hspace{-.3cm}
%	\begin{array}{c}
%	\left [  \ \quad \mathbf{R}_{k+1}^{\top}\mathbf{Q}_{k+1}^{\top}\quad \ \right ] \\
%	\\
%	\\
%	\end{array},
%\end{equation*}
%where $\mathbf{R} \in \mathbb{R}^{m \times m}$, $\mathbf{Q} \in \mathbb{R}^{m \times n}$. Next, we compute the spectral decomposition on the matrix $\mathbf{RMR}^\top$ to yield $\mathbf{P}\Lambda \mathbf{P}^\top$. We define $\mathbf{U}_\parallel = \mathbf{QP}$ and form the following representation:
%\begin{equation*}
%\mathbf{B}_{k+1} \ = \ \mathbf{B}_0 + 
%	\begin{bmatrix}
%	\\
%	\mathbf{U_{\parallel}} \\
%	\phantom{t}
%	\end{bmatrix}
%	\hspace{-.3cm}
%	\begin{array}{c}
%	\left  [ \  \Lambda^{\phantom{h}}  \right ] \\
%	\\
%	\\
%	\end{array}
%	\hspace{-.3cm}
%	\begin{array}{c}
%	\left [  \ \quad \mathbf{U}_{\parallel}^{\top}\quad \ \right ] \\
%	\\
%	\\
%	\end{array}.
%\end{equation*}
%
%\subsection{Dynamic initialization of $\mathbf{B}_0$}\label{sec:B0} 
%
%We define $\mathbf{B}_0 = \delta \mathbf{I}$, where $\delta = 0 < \delta < \hat{\lambda}_i $.
%$\hat{\lambda}_i$ denotes the smallest eigenvalue of the generalized eigenvalue problem 
%\begin{equation*}
%	(\mathbf{D}_k + \mathbf{L}_k + \mathbf{L}_k^{\top})\mathbf{u} = \hat{\lambda}_i \mathbf{S}^{\top}_k \mathbf{S}_k \mathbf{u}
%\end{equation*}
%For further information, refer to \cite{Erway2020TrustregionAF} (Lemma 2.4)
%
%\subsection{Closed-form solution}\label{sec:closedformsolve}
%Now we are ready to discuss the closed-form solution of the cubic-regularized model. Suppose we approximate a $n \times n$ Hessian. The closed form solution in the new space of variables is given by
%\begin{equation*}
%	\bar{\mathbf{s}}^* = -\mathbf{\mathbf{C}}\bar{\mathbf{g}}.
%\end{equation*}
%Here, $\mathbf{C} = diag(c_1,\ldots, c_n)$ and $c_i \overset{\text{def}}{=} \frac{2}{\lambda_i^2 + \sqrt{\lambda_i^2 + 4 \mu |\bar{g}_i|}}$, where $\bar{\mathbf{g}} = \mathbf{U}^\top \mathbf{g}$. To get the closed form solution in the original space, we transform the new space back to the original space as
%\begin{equation*}
%	\mathbf{s}^* = \mathbf{U} \bar{\mathbf{s}}^*.
%\end{equation*}
%However, we only operate on a limited memory approximation of the Hessian. We define $\mathbf{U} = [ \mathbf{U}_\parallel, \mathbf{U}_\perp] \in \mathbb{R}^{n \times n}$, where we only operate on $\mathbf{U}_\parallel$. Thus the formulation is defined as 
%\begin{align*}
%	\mathbf{B}u &= \delta \mathbf{u},\\
%	\mathbf{B}u &= (\delta +\lambda_i) \mathbf{u},
%\end{align*}
%where $\delta$ is computed using the formulation described in \ref{sec:B0} and $\lambda_i$ is defined by the eigen value decomposition in . Thus the solution in the new space is given by,
%\begin{align*}
%	\bar{\mathbf{s}}_\parallel^{*} = \mathbf{U}_{\parallel}^{\top}\mathbf{s}^{*},\quad \bar{\mathbf{s}}^{*}_{\perp} = \mathbf{U}_{\perp}^{\top}\mathbf{s}^{*},\\
%	\bar{\mathbf{g}}_{\parallel} = \mathbf{U}_{\parallel}^{\top}\mathbf{g}, \quad \bar{\mathbf{g}}_{\perp} = \mathbf{U}_{\perp}^{\top}\mathbf{g}.
%\end{align*}
%We make the following observations:
%\begin{enumerate}
%	\item $\delta$ is a multiple eigen value. This means $\mathbf{U}_{\perp}$ is not uniquely defined.
%	\item $\mathbf{s}^*$ depends on the choice of $\mathbf{U}_{\perp}$.
%	\item $\bar{\mathbf{g}}_{\perp}$ is used for computing $\bar{\mathbf{s^{*}}}_{\perp}$ and it requires $\mathbf{U}_{\perp}^{\top}\mathbf{g}$.
%\end{enumerate} 
%
%$\mathbf{U}_{\perp}^{\top}\mathbf{g}$ may be prohibitively expensive to compute and store, unless $\mathbf{U}_{\perp}$ is chosen in a special way. We define $\mathbf{g}_{\perp} = (\mathbf{I}_n - \mathbf{U}_{\parallel} \mathbf{U}_{\parallel}^{\top})\mathbf{g}$ which can be re-written as $\mathbf{g}_{\perp} =\mathbf{g} - \mathbf{U}_{\parallel} \mathbf{g}_{\parallel}$.
%
%\subsection{Adaptive regularized cubics}
Once we compute the step in Algorithm \ref{alg:LSR1ARC}, we compute the ratio between the reduction in the actual objective function and the reduction in the model  defined by
		\begin{equation}\label{eq:ratio}
		\rho_k = (f(\Theta_k) - f(\Theta_{k+1}))/(m(\mathbf{s}^*)).
		\end{equation}
where
\begin{equation*}
m(\mathbf{s}^*) = q(\mathbf{s}^*) + \frac{\mu}{3} (\norm{\mathbf{C}_\parallel \bar{\mathbf{g}}_{\parallel}}_3^3 + (\alpha^*)^3\norm{\mathbf{g}_{\perp}}_2^3).
\end{equation*}
and
$$q(\mathbf{s}^*) = \bar{\mathbf{g}}_\parallel^{\top}(\mathbf{C}_{\parallel}^2 \Lambda_{\parallel} + \mathbf{C}_{\parallel}\bar{\mathbf{g}}_{\parallel} + \frac{\delta_k(\alpha^{\star})^2 - 2\alpha^*}{2} \norm{\mathbf{g}_{\perp}}.
$$


\section{Experimental Analysis}
\label{sec:exp}
We now describe in detail our experimental analysis. The experimental section is organized as follows:
%\begin{enumerate}[noitemsep,topsep=0pt,parsep=0pt,partopsep=0pt,leftmargin=0.5cm]
%\item 

\noindent In {\bf 
Section~\ref{exp:setup}}, we introduce the datasets and methods to evaluate the previously defined accuracy measures.

%\item
\noindent In {\bf 
Section~\ref{exp:qual}}, we illustrate the limitations of existing measures with some selected qualitative examples.

%\item 
\noindent In {\bf 
Section~\ref{exp:quant}}, we continue by measuring quantitatively the benefits of our proposed measures in terms of {\it robustness} to lag, noise, and normal/abnormal ratio.

%\item 
\noindent In {\bf 
Section~\ref{exp:separability}}, we evaluate the {\it separability} degree of accurate and inaccurate methods, using the existing and our proposed approaches.

%\item
\noindent In {\bf 
Section~\ref{sec:entropy}}, we conduct a {\it consistency} evaluation, in which we analyze the variation of ranks that an AD method can have with an accuracy measures used.

%\item 
\noindent In {\bf 
Section~\ref{sec:exectime}}, we conduct an {\it execution time} evaluation, in which we analyze the impact of different parameters related to the accuracy measures and the time series characteristics. 
We focus especially on the comparison of the different VUS implementations.
%\end{enumerate}

\begin{table}[tb]
\caption{Summary characteristics (averaged per dataset) of the public datasets of TSB-UAD (S.: Size, Ano.: Anomalies, Ab.: Abnormal, Den.: Density)}
\label{table:charac}
%\vspace{-0.2cm}
\footnotesize
\begin{center}
\scalebox{0.82}{
\begin{tabular}{ |r|r|r|r|r|r|} 
 \hline
\textbf{\begin{tabular}[c]{@{}c@{}}Dataset \end{tabular}} & 
\textbf{\begin{tabular}[c]{@{}c@{}}S. \end{tabular}} & 
\textbf{\begin{tabular}[c]{c@{}} Len.\end{tabular}} & 
\textbf{\begin{tabular}[c]{c@{}} \# \\ Ano. \end{tabular}} &
\textbf{\begin{tabular}[c]{c@{}c@{}} \# \\ Ab. \\ Points\end{tabular}} &
\textbf{\begin{tabular}[c]{c@{}c@{}} Ab. \\ Den. \\ (\%)\end{tabular}} \\ \hline
Dodgers \cite{10.1145/1150402.1150428} & 1 & 50400   & 133.0     & 5612.0  &11.14 \\ \hline
SED \cite{doi:10.1177/1475921710395811}& 1 & 100000   & 75.0     & 3750.0  & 3.7\\ \hline
ECG \cite{goldberger_physiobank_2000}   & 52 & 230351  & 195.6     & 15634.0  &6.8 \\ \hline
IOPS \cite{IOPS}   & 58 & 102119  & 46.5     & 2312.3   &2.1 \\ \hline
KDD21 \cite{kdd} & 250 &77415   & 1      & 196.5   &0.56 \\ \hline
MGAB \cite{markus_thill_2020_3762385}   & 10 & 100000  & 10.0     & 200.0   &0.20 \\ \hline
NAB \cite{ahmad_unsupervised_2017}   & 58 & 6301   & 2.0      & 575.5   &8.8 \\ \hline
NASA-M. \cite{10.1145/3449726.3459411}   & 27 & 2730   & 1.33      & 286.3   &11.97 \\ \hline
NASA-S. \cite{10.1145/3449726.3459411}   & 54 & 8066   & 1.26      & 1032.4   &12.39 \\ \hline
SensorS. \cite{YAO20101059}   & 23 & 27038   & 11.2     & 6110.4   &22.5 \\ \hline
YAHOO \cite{yahoo}  & 367 & 1561   & 5.9      & 10.7   &0.70 \\ \hline 
\end{tabular}}
\end{center}
\end{table}











\subsection{Experimental Setup and Settings}
\label{exp:setup}
%\vspace{-0.1cm}

\begin{figure*}[tb]
  \centering
  \includegraphics[width=1\linewidth]{figures/quality.pdf}
  %\vspace{-0.7cm}
  \caption{Comparison of evaluation measures (proposed measures illustrated in subplots (b,c,d,e); all others summarized in subplots (f)) on two examples ((A)AE and OCSM applied on MBA(805) and (B) LOF and OCSVM applied on MBA(806)), illustrating the limitations of existing measures for scores with noise or containing a lag. }
  \label{fig:quality}
  %\vspace{-0.1cm}
\end{figure*}

We implemented the experimental scripts in Python 3.8 with the following main dependencies: sklearn 0.23.0, tensorflow 2.3.0, pandas 1.2.5, and networkx 2.6.3. In addition, we used implementations from our TSB-UAD benchmark suite.\footnote{\scriptsize \url{https://www.timeseries.org/TSB-UAD}} For reproducibility purposes, we make our datasets and code available.\footnote{\scriptsize \url{https://www.timeseries.org/VUS}}
\newline \textbf{Datasets: } For our evaluation purposes, we use the public datasets identified in our TSB-UAD benchmark. The latter corresponds to $10$ datasets proposed in the past decades in the literature containing $900$ time series with labeled anomalies. Specifically, each point in every time series is labeled as normal or abnormal. Table~\ref{table:charac} summarizes relevant characteristics of the datasets, including their size, length, and statistics about the anomalies. In more detail:

\begin{itemize}
    \item {\bf SED}~\cite{doi:10.1177/1475921710395811}, from the NASA Rotary Dynamics Laboratory, records disk revolutions measured over several runs (3K rpm speed).
	\item {\bf ECG}~\cite{goldberger_physiobank_2000} is a standard electrocardiogram dataset and the anomalies represent ventricular premature contractions. MBA(14046) is split to $47$ series.
	\item {\bf IOPS}~\cite{IOPS} is a dataset with performance indicators that reflect the scale, quality of web services, and health status of a machine.
	\item {\bf KDD21}~\cite{kdd} is a composite dataset released in a SIGKDD 2021 competition with 250 time series.
	\item {\bf MGAB}~\cite{markus_thill_2020_3762385} is composed of Mackey-Glass time series with non-trivial anomalies. Mackey-Glass data series exhibit chaotic behavior that is difficult for the human eye to distinguish.
	\item {\bf NAB}~\cite{ahmad_unsupervised_2017} is composed of labeled real-world and artificial time series including AWS server metrics, online advertisement clicking rates, real time traffic data, and a collection of Twitter mentions of large publicly-traded companies.
	\item {\bf NASA-SMAP} and {\bf NASA-MSL}~\cite{10.1145/3449726.3459411} are two real spacecraft telemetry data with anomalies from Soil Moisture Active Passive (SMAP) satellite and Curiosity Rover on Mars (MSL).
	\item {\bf SensorScope}~\cite{YAO20101059} is a collection of environmental data, such as temperature, humidity, and solar radiation, collected from a sensor measurement system.
	\item {\bf Yahoo}~\cite{yahoo} is a dataset consisting of real and synthetic time series based on the real production traffic to some of the Yahoo production systems.
\end{itemize}


\textbf{Anomaly Detection Methods: }  For the experimental evaluation, we consider the following baselines. 

\begin{itemize}
\item {\bf Isolation Forest (IForest)}~\cite{liu_isolation_2008} constructs binary trees based on random space splitting. The nodes (subsequences in our specific case) with shorter path lengths to the root (averaged over every random tree) are more likely to be anomalies. 
\item {\bf The Local Outlier Factor (LOF)}~\cite{breunig_lof_2000} computes the ratio of the neighbor density to the local density. 
\item {\bf Matrix Profile (MP)}~\cite{yeh_time_2018} detects as anomaly the subsequence with the most significant 1-NN distance. 
\item {\bf NormA}~\cite{boniol_unsupervised_2021} identifies the normal patterns based on clustering and calculates each point's distance to normal patterns weighted using statistical criteria. 
\item {\bf Principal Component Analysis (PCA)}~\cite{aggarwal_outlier_2017} projects data to a lower-dimensional hyperplane. Outliers are points with a large distance from this plane. 
\item {\bf Autoencoder (AE)} \cite{10.1145/2689746.2689747} projects data to a lower-dimensional space and reconstructs it. Outliers are expected to have larger reconstruction errors. 
\item {\bf LSTM-AD}~\cite{malhotra_long_2015} use an LSTM network that predicts future values from the current subsequence. The prediction error is used to identify anomalies.
\item {\bf Polynomial Approximation (POLY)} \cite{li_unifying_2007} fits a polynomial model that tries to predict the values of the data series from the previous subsequences. Outliers are detected with the prediction error. 
\item {\bf CNN} \cite{8581424} built, using a convolutional deep neural network, a correlation between current and previous subsequences, and outliers are detected by the deviation between the prediction and the actual value. 
\item {\bf One-class Support Vector Machines (OCSVM)} \cite{scholkopf_support_1999} is a support vector method that fits a training dataset and finds the normal data's boundary.
\end{itemize}

\subsection{Qualitative Analysis}
\label{exp:qual}



We first use two examples to demonstrate qualitatively the limitations of existing accuracy evaluation measures in the presence of lag and noise, and to motivate the need for a new approach. 
These two examples are depicted in Figure~\ref{fig:quality}. 
The first example, in Figure~\ref{fig:quality}(A), corresponds to OCSVM and AE on the MBA(805) dataset (named MBA\_ECG805\_data.out in the ECG dataset). 

We observe in Figure~\ref{fig:quality}(A)(a.1) and (a.2) that both scores identify most of the anomalies (highlighted in red). However, the OCSVM score points to more false positives (at the end of the time series) and only captures small sections of the anomalies. On the contrary, the AE score points to fewer false positives and captures all abnormal subsequences. Thus we can conclude that, visually, AE should obtain a better accuracy score than OCSVM. Nevertheless, we also observe that the AE score is lagged with the labels and contains more noise. The latter has a significant impact on the accuracy of evaluation measures. First, Figure~\ref{fig:quality}(A)(c) is showing that AUC-PR is better for OCSM (0.73) than for AE (0.57). This is contradictory with what is visually observed from Figure~\ref{fig:quality}(A)(a.1) and (a.2). However, when using our proposed measure R-AUC-PR, OCSVM obtains a lower score (0.83) than AE (0.89). This confirms that, in this example, a buffer region before the labels helps to capture the true value of an anomaly score. Overall, Figure~\ref{fig:quality}(A)(f) is showing in green and red the evolution of accuracy score for the 13 accuracy measures for AE and OCSVM, respectively. The latter shows that, in addition to Precision@k and Precision, our proposed approach captures the quality order between the two methods well.

We now present a second example, on a different time series, illustrated in Figure~\ref{fig:quality}(B). 
In this case, we demonstrate the anomaly score of OCSVM and LOF (depicted in Figure~\ref{fig:quality}(B)(a.1) and (a.2)) applied on the MBA(806) dataset (named MBA\_ECG806\_data.out in the ECG dataset). 
We observe that both methods produce the same level of noise. However, LOF points to fewer false positives and captures more sections of the abnormal subsequences than OCSVM. 
Nevertheless, the LOF score is slightly lagged with the labels such that the maximum values in the LOF score are slightly outside of the labeled sections. 
Thus, as illustrated in Figure~\ref{fig:quality}(B)(f), even though we can visually consider that LOF is performing better than OCSM, all usual measures (Precision, Recall, F, precision@k, and AUC-PR) are judging OCSM better than AE. On the contrary, measures that consider lag (Rprecision, Rrecall, RF) rank the methods correctly. 
However, due to threshold issues, these measures are very close for the two methods. Overall, only AUC-ROC and our proposed measures give a higher score for LOF than for OCSVM.

\subsection{Quantitative Analysis}
\label{exp:case}

\begin{figure}[t]
  \centering
  \includegraphics[width=1\linewidth]{figures/eval_case_study.pdf}
  %\vspace*{-0.7cm}
  \caption{\commentRed{
  Comparison of evaluation measures for synthetic data examples across various scenarios. S8 represents the oracle case, where predictions perfectly align with labeled anomalies. Problematic cases are highlighted in the red region.}}
  %\vspace*{-0.5cm}
  \label{fig:eval_case_study}
\end{figure}
\commentRed{
We present the evaluation results for different synthetic data scenarios, as shown in Figure~\ref{fig:eval_case_study}. These scenarios range from S1, where predictions occur before the ground truth anomaly, to S12, where predictions fall within the ground truth region. The red-shaded regions highlight problematic cases caused by a lack of adaptability to lags. For instance, in scenarios S1 and S2, a slight shift in the prediction leads to measures (e.g., AUC-PR, F score) that fail to account for lags, resulting in a zero score for S1 and a significant discrepancy between the results of S1 and S2. Thus, we observe that our proposed VUS effectively addresses these issues and provides robust evaluations results.}

%\subsection{Quantitative Analysis}
%\subsection{Sensitivity and Separability Analysis}
\subsection{Robustness Analysis}
\label{exp:quant}


\begin{figure}[tb]
  \centering
  \includegraphics[width=1\linewidth]{figures/lag_sensitivity_analysis.pdf}
  %\vspace*{-0.7cm}
  \caption{For each method, we compute the accuracy measures 10 times with random lag $\ell \in [-0.25*\ell,0.25*\ell]$ injected in the anomaly score. We center the accuracy average to 0.}
  %\vspace*{-0.5cm}
  \label{fig:lagsensitivity}
\end{figure}

We have illustrated with specific examples several of the limitations of current measures. 
We now evaluate quantitatively the robustness of the proposed measures when compared to the currently used measures. 
We first evaluate the robustness to noise, lag, and normal versus abnormal points ratio. We then measure their ability to separate accurate and inaccurate methods.
%\newline \textbf{Sensitivity Analysis: } 
We first analyze the robustness of different approaches quantitatively to different factors: (i) lag, (ii) noise, and (iii) normal/abnormal ratio. As already mentioned, these factors are realistic. For instance, lag can be either introduced by the anomaly detection methods (such as methods that produce a score per subsequences are only high at the beginning of abnormal subsequences) or by human labeling approximation. Furthermore, even though lag and noises are injected, an optimal evaluation metric should not vary significantly. Therefore, we aim to measure the variance of the evaluation measures when we vary the lag, noise, and normal/abnormal ratio. We proceed as follows:

\begin{enumerate}[noitemsep,topsep=0pt,parsep=0pt,partopsep=0pt,leftmargin=0.5cm]
\item For each anomaly detection method, we first compute the anomaly score on a given time series.
\item We then inject either lag $l$, noise $n$ or change the normal/abnormal ratio $r$. For 10 different values of $l \in [-0.25*\ell,0.25*\ell]$, $n \in [-0.05*(max(S_T)-min(S_T)),0.05*(max(S_T)-min(S_T))]$ and $r \in [0.01,0.2]$, we compute the 13 different measures.
\item For each evaluation measure, we compute the standard deviation of the ten different values. Figure~\ref{fig:lagsensitivity}(b) depicts the different lag values for six AD methods applied on a data series in the ECG dataset.
\item We compute the average standard deviation for the 13 different AD quality measures. For example, figure~\ref{fig:lagsensitivity}(a) depicts the average standard deviation for ten different lag values over the AD methods applied on the MBA(805) time series.
\item We compute the average standard deviation for the every time series in each dataset (as illustrated in Figure~\ref{fig:sensitivity_per_data}(b to j) for nine datasets of the benchmark.
\item We compute the average standard deviation for the every dataset (as illustrated in Figure~\ref{fig:sensitivity_per_data}(a.1) for lag, Figure~\ref{fig:sensitivity_per_data}(a.2) for noise and Figure~\ref{fig:sensitivity_per_data}(a.3) for normal/abnormal ratio).
\item We finally compute the Wilcoxon test~\cite{10.2307/3001968} and display the critical diagram over the average standard deviation for every time series (as illustrated in Figure~\ref{fig:sensitivity}(a.1) for lag, Figure~\ref{fig:sensitivity}(a.2) for noise and Figure~\ref{fig:sensitivity}(a.3) for normal/abnormal ratio).
\end{enumerate}

%height=8.5cm,

\begin{figure}[tb]
  \centering
  \includegraphics[width=\linewidth]{figures/sensitivity_per_data_long.pdf}
%  %\vspace*{-0.3cm}
  \caption{Robustness Analysis for nine datasets: we report, over the entire benchmark, the average standard deviation of the accuracy values of the measures, under varying (a.1) lag, (a.2) noise, and (a.3) normal/abnormal ratio. }
  \label{fig:sensitivity_per_data}
\end{figure}

\begin{figure*}[tb]
  \centering
  \includegraphics[width=\linewidth]{figures/sensitivity_analysis.pdf}
  %\vspace*{-0.7cm}
  \caption{Critical difference diagram computed using the signed-rank Wilkoxon test (with $\alpha=0.1$) for the robustness to (a.1) lag, (a.2) noise and (a.3) normal/abnormal ratio.}
  \label{fig:sensitivity}
\end{figure*}

The methods with the smallest standard deviation can be considered more robust to lag, noise, or normal/abnormal ratio from the above framework. 
First, as stated in the introduction, we observe that non-threshold-based measures (such as AUC-ROC and AUC-PR) are indeed robust to noise (see Figure~\ref{fig:sensitivity_per_data}(a.2)), but not to lag. Figure~\ref{fig:sensitivity}(a.1) demonstrates that our proposed measures VUS-ROC, VUS-PR, R-AUC-ROC, and R-AUC-PR are significantly more robust to lag. Similarly, Figure~\ref{fig:sensitivity}(a.2) confirms that our proposed measures are significantly more robust to noise. However, we observe that, among our proposed measures, only VUS-ROC and R-AUC-ROC are robust to the normal/abnormal ratio and not VUS-PR and R-AUC-PR. This is explained by the fact that Precision-based measures vary significantly when this ratio changes. This is confirmed by Figure~\ref{fig:sensitivity_per_data}(a.3), in which we observe that Precision and Rprecision have a high standard deviation. Overall, we observe that VUS-ROC is significantly more robust to lag, noise, and normal/abnormal ratio than other measures.




\subsection{Separability Analysis}
\label{exp:separability}

%\newline \textbf{Separability Analysis: } 
We now evaluate the separability capacities of the different evaluation metrics. 
\commentRed{The main objective is to measure the ability of accuracy measures to separate accurate methods from inaccurate ones. More precisely, an appropriate measure should return accuracy scores that are significantly higher for accurate anomaly scores than for inaccurate ones.}
We thus manually select accurate and inaccurate anomaly detection methods and verify if the accuracy evaluation scores are indeed higher for the accurate than for the inaccurate methods. Figure~\ref{fig:separability} depicts the latter separability analysis applied to the MBA(805) and the SED series. 
The accurate and inaccurate anomaly scores are plotted in green and red, respectively. 
We then consider 12 different pairs of accurate/inaccurate methods among the eight previously mentioned anomaly scores. 
We slightly modify each score 50 different times in which we inject lag and noises and compute the accuracy measures. 
Figure~\ref{fig:separability}(a.4) and Figure~\ref{fig:separability}(b.4) are divided into four different subplots corresponding to 4 pairs (selected among the twelve different pairs due to lack of space). 
Each subplot corresponds to two box plots per accuracy measure. 
The green and red box plots correspond to the 50 accuracy measures on the accurate and inaccurate methods. 
If the red and green box plots are well separated, we can conclude that the corresponding accuracy measures are separating the accurate and inaccurate methods well. 
We observe that some accuracy measures (such as VUS-ROC) are more separable than others (such as RF). We thus measure the separability of the two box-plots by computing the Z-test. 

\begin{figure*}[tb]
  \centering
  \includegraphics[width=1\linewidth]{figures/pairwise_comp_example_long.pdf}
  %\vspace*{-0.5cm}
  \caption{Separability analysis applied on 4 pairs of accurate (green) and inaccurate (red) methods on (a) the MBA(805) data series, and (b) the SED data series.}
  %\vspace*{-0.3cm}
  \label{fig:separability}
\end{figure*}

We now aggregate all the results and compute the average Z-test for all pairs of accurate/inaccurate datasets (examples are shown in Figures~\ref{fig:separability}(a.2) and (b.2) for accurate anomaly scores, and in Figures~\ref{fig:separability}(a.3) and (b.3) for inaccurate anomaly scores, for the MBA(805) and SED series, respectively). 
Next, we perform the same operation over three different data series: MBA (805), MBA(820), and SED. 
Then, we depict the average Z-test for these three datasets in Figure~\ref{fig:separability_agg}(a). 
Finally, we show the average Z-test for all datasets in Figure~\ref{fig:separability_agg}(b). 


We observe that our proposed VUS-based and Range-based measures are significantly more separable than other current accuracy measures (up to two times for AUC-ROC, the best measures of all current ones). Furthermore, when analyzed in detail in Figure~\ref{fig:separability} and Figure~\ref{fig:separability_agg}, we confirm that VUS-based and Range-based are more separable over all three datasets. 

\begin{figure}[tb]
  \centering
  \includegraphics[width=\linewidth]{figures/agregated_sep_analysis.pdf}
  %\vspace*{-0.5cm}
  \caption{Overall separability analysis (averaged z-test between the accuracy values distributions of accurate and inaccurate methods) applied on 36 pairs on 3 datasets.}
  \label{fig:separability_agg}
\end{figure}


\noindent \textbf{Global Analysis: } Overall, we observe that VUS-ROC is the most robust (cf. Figure~\ref{fig:sensitivity}) and separable (cf. Figure~\ref{fig:separability_agg}) measure. 
On the contrary, Precision and Rprecision are non-robust and non-separable. 
Among all previous accuracy measures, only AUC-ROC is robust and separable. 
Popular measures, such as, F, RF, AUC-ROC, and AUC-PR are robust but non-separable.

In order to visualize the global statistical analysis, we merge the robustness and the separability analysis into a single plot. Figure~\ref{fig:global} depicts one scatter point per accuracy measure. 
The x-axis represents the averaged standard deviation of lag and noise (averaged values from Figure~\ref{fig:sensitivity_per_data}(a.1) and (a.2)). The y-axis corresponds to the averaged Z-test (averaged value from Figure~\ref{fig:separability_agg}). 
Finally, the size of the points corresponds to the sensitivity to the normal/abnormal ratio (values from Figure~\ref{fig:sensitivity_per_data}(a.3)). 
Figure~\ref{fig:global} demonstrates that our proposed measures (located at the top left section of the plot) are both the most robust and the most separable. 
Among all previous accuracy measures, only AUC-ROC is on the top left section of the plot. 
Popular measures, such as, F, RF, AUC-ROC, AUC-PR are on the bottom left section of the plot. 
The latter underlines the fact that these measures are robust but non-separable.
Overall, Figure~\ref{fig:global} confirms the effectiveness and superiority of our proposed measures, especially of VUS-ROC and VUS-PR.


\begin{figure}[tb]
  \centering
  \includegraphics[width=\linewidth]{figures/final_result.pdf}
  \caption{Evaluation of all measures based on: (y-axis) their separability (avg. z-test), (x-axis) avg. standard deviation of the accuracy values when varying lag and noise, (circle size) avg. standard deviation of the accuracy values when varying the normal/abnormal ratio.}
  \label{fig:global}
\end{figure}




\subsection{Consistency Analysis}
\label{sec:entropy}

In this section, we analyze the accuracy of the anomaly detection methods provided by the 13 accuracy measures. The objective is to observe the changes in the global ranking of anomaly detection methods. For that purpose, we formulate the following assumptions. First, we assume that the data series in each benchmark dataset are similar (i.e., from the same domain and sharing some common characteristics). As a matter of fact, we can assume that an anomaly detection method should perform similarly on these data series of a given dataset. This is confirmed when observing that the best anomaly detection methods are not the same based on which dataset was analyzed. Thus the ranking of the anomaly detection methods should be different for different datasets, but similar for every data series in each dataset. 
Therefore, for a given method $A$ and a given dataset $D$ containing data series of the same type and domain, we assume that a good accuracy measure results in a consistent rank for the method $A$ across the dataset $D$. 
The consistency of a method's ranks over a dataset can be measured by computing the entropy of these ranks. 
For instance, a measure that returns a random score (and thus, a random rank for a method $A$) will result in a high entropy. 
On the contrary, a measure that always returns (approximately) the same ranks for a given method $A$ will result in a low entropy. 
Thus, for a given method $A$ and a given dataset $D$ containing data series of the same type and domain, we assume that a good accuracy measure results in a low entropy for the different ranks for method $A$ on dataset $D$.

\begin{figure*}[tb]
  \centering
  \includegraphics[width=\linewidth]{figures/entropy_long.pdf}
  %\vspace*{-0.5cm}
  \caption{Accuracy evaluation of the anomaly detection methods. (a) Overall average entropy per category of measures. Analysis of the (b) averaged rank and (c) averaged rank entropy for each method and each accuracy measure over the entire benchmark. Example of (b.1) average rank and (c.1) entropy on the YAHOO dataset, KDD21 dataset (b.2, c.2). }
  \label{fig:entropy}
\end{figure*}

We now compute the accuracy measures for the nine different methods (we compute the anomaly scores ten different times, and we use the average accuracy). 
Figures~\ref{fig:entropy}(b.1) and (b.2) report the average ranking of the anomaly detection methods obtained on the YAHOO and KDD21 datasets, respectively. 
The x-axis corresponds to the different accuracy measures. We first observe that the rankings are more separated using Range-AUC and VUS measures for these two datasets. Figure~\ref{fig:entropy}(b) depicts the average ranking over the entire benchmark. The latter confirms the previous observation that VUS measures provide more separated rankings than threshold-based and AUC-based measures. We also observe an interesting ranking evolution for the YAHOO dataset illustrated in Figure~\ref{fig:entropy}(b.1). We notice that both LOF and MatrixProfile (brown and pink curve) have a low rank (between 4 and 5) using threshold and AUC-based measures. However, we observe that their ranks increase significantly for range-based and VUS-based measures (between 2.5 and 3). As we noticed by looking at specific examples (see Figure~\ref{exp:qual}), LOF and MatrixProfile can suffer from a lag issue even though the anomalies are well-identified. Therefore, the range-based and VUS-based measures better evaluate these two methods' detection capability.


Overall, the ranking curves show that the ranks appear more chaotic for threshold-based than AUC-, Range-AUC-, and VUS-based measures. 
In order to quantify this observation, we compute the Shannon Entropy of the ranks of each anomaly detection method. 
In practice, we extract the ranks of methods across one dataset and compute Shannon's Entropy of the different ranks. 
Figures~\ref{fig:entropy}(c.1) and (c.2) depict the entropy of each of the nine methods for the YAHOO and KDD21 datasets, respectively. 
Figure~\ref{fig:entropy}(c) illustrates the averaged entropy for all datasets in the benchmark for each measure and method, while Figure~\ref{fig:entropy}(a) shows the averaged entropy for each category of measures.
We observe that both for the general case (Figure~\ref{fig:entropy}(a) and Figure~\ref{fig:entropy}(c)) and some specific cases (Figures~\ref{fig:entropy}(c.1) and (c.2)), the entropy is reducing when using AUC-, Range-AUC-, and VUS-based measures. 
We report the lowest entropy for VUS-based measures. 
Moreover, we notice a significant drop between threshold-based and AUC-based. 
This confirms that the ranks provided by AUC- and VUS-based measures are consistent for data series belonging to one specific dataset. 


Therefore, based on the assumption formulated at the beginning of the section, we can thus conclude that AUC, range-AUC, and VUS-based measures are providing more consistent rankings. Finally, as illustrated in Figure~\ref{fig:entropy}, we also observe that VUS-based measures result in the most ordered and similar rankings for data series from the same type and domain.










\subsection{Execution Time Analysis}
\label{sec:exectime}

In this section, we evaluate the execution time required to compute different evaluation measures. 
In Section~\ref{sec:synthetic_eval_time}, we first measure the influence of different time series characteristics and VUS parameters on the execution time. In Section~\ref{sec:TSB_eval_time}, we  measure the execution time of VUS (VUS-ROC and VUS-PR simultaneously), R-AUC (R-AUC-ROC and R-AUC-PR simultaneously), and AUC-based measures (AUC-ROC and AUC-PR simultaneously) on the TSB-UAD benchmark. \commentRed{As demonstrated in the previous section, threshold-based measures are not robust, have a low separability power, and are inconsistent. 
Such measures are not suitable for evaluating anomaly detection methods. Thus, in this section, we do not consider threshold-based measures.}


\subsubsection{Evaluation on Synthetic Time Series}\hfill\\
\label{sec:synthetic_eval_time}

We first analyze the impact that time series characteristics and parameters have on the computation time of VUS-based measures. 
to that effect, we generate synthetic time series and labels, where we vary the following parameters: (i) the number of anomalies {\bf$\alpha$} in the time series, (ii) the average \textbf{$\mu(\ell_a)$} and standard deviation $\sigma(\ell_a)$ of the anomalies lengths in the time series (all the anomalies can have different lengths), (iii) the length of the time series \textbf{$|T|$}, (iv) the maximum buffer length \textbf{$L$}, and (v) the number of thresholds \textbf{$N$}.


We also measure the influence on the execution time of the R-AUC- and AUC- related parameter, that is, the number of thresholds ($N$).
The default values and the range of variation of these parameters are listed in Table~\ref{tab:parameter_range_time}. 
For VUS-based measures, we evaluate the execution time of the initial VUS implementation, as well as the two optimized versions, VUS$_{opt}$ and VUS$_{opt}^{mem}$.

\begin{table}[tb]
    \centering
    \caption{Value ranges for the parameters: number of anomalies ($\alpha$), average and standard deviation anomaly length ($\mu(\ell_a)$,$\sigma(\ell_a)$), time series length ($|T|$), maximum buffer length ($L$), and number of thresholds ($N$).}
    \begin{tabular}{|c|c|c|c|c|c|c|} 
 \hline
 Param. & $\alpha$ & $\mu(\ell_a)$ & $\sigma(\ell_{a})$ & $|T|$ & $L$ & $N$ \\ [0.5ex] 
 \hline\hline
 \textbf{Default} & 10 & 10 & 0 & $10^5$ & 5 & 250\\ 
 \hline
 Min. & 0 & 0 & 0 & $10^3$ & 0 & 2 \\
 \hline
 Max. & $2*10^3$ & $10^3$ & $10$ & $10^5$ & $10^3$ & $10^3$ \\ [1ex] 
 \hline
\end{tabular}
    \label{tab:parameter_range_time}
\end{table}


Figure~\ref{fig:sythetic_exp_time} depicts the execution time (averaged over ten runs) for each parameter listed in Table~\ref{tab:parameter_range_time}. 
Overall, we observe that the execution time of AUC-based and R-AUC-based measures is significantly smaller than VUS-based measures.
In the following paragraph, we analyze the influence of each parameter and compare the experimental execution time evaluation to the theoretical complexity reported in Table~\ref{tab:complexity_summary}.

\vspace{0.2cm}
\noindent {\bf [Influence of $\alpha$]}:
In Figure~\ref{fig:sythetic_exp_time}(a), we observe that the VUS, VUS$_{opt}$, and VUS$_{opt}^{mem}$ execution times are linearly increasing with $\alpha$. 
The increase in execution time for VUS, VUS$_{opt}$, and VUS$_{opt}^{mem}$ is more pronounced when we vary $\alpha$, in contrast to $l_a$ (which nevertheless, has a similar effect on the overall complexity). 
We also observe that the VUS$_{opt}^{mem}$ execution time grows slower than $VUS_{opt}$ when $\alpha$ increases. 
This is explained by the use of 2-dimensional arrays for the storage of predictions, which use contiguous memory locations that allow for faster access, decreasing the dependency on $\alpha$.

\vspace{0.2cm}
\noindent {\bf [Influence of $\mu(\ell_a)$]}:
As shown in Figure~\ref{fig:sythetic_exp_time}(b), the execution time variation of VUS, VUS$_{opt}$, and VUS$_{opt}^{mem}$ caused by $\ell_a$ is rather insignificant. 
We also observe that the VUS$_{opt}$ and VUS$_{opt}^{mem}$ execution times are significantly lower when compared to VUS. 
This is explained by the smaller dependency of the complexity of these algorithms on the time series length $|T|$. 
Overall, the execution time for both VUS$_{opt}$ and VUS$_{opt}^{mem}$ is significantly lower than VUS, and follows a similar trend. 

\vspace{0.2cm}
\noindent {\bf [Influence of $\sigma(\ell_a)$]}: 
As depicted in Figure~\ref{fig:sythetic_exp_time}(d) and inferred from the theoretical complexities in Table~\ref{tab:complexity_summary}, none of the measures are affected by the standard deviation of the anomaly lengths.

\vspace{0.2cm}
\noindent {\bf [Influence of $|T|$]}:
For short time series (small values of $|T|$), we note that O($T_1$) becomes comparable to O($T_2$). 
Thus, the theoretical complexities approximate to $O(NL(T_1+T_2))$, $O(N*(T_1+T_2))+O(NLT_2)$ and $O(N(T_1+T_2))$ for VUS, VUS$_{opt}$, and VUS$_{opt}^{mem}$, respectively. 
Indeed, we observe in Figure~\ref{fig:sythetic_exp_time}(c) that the execution times of VUS, VUS$_{opt}$, and VUS$_{opt}^{mem}$ are similar for small values of $|T|$. However, for larger values of $|T|$, $O(T_1)$ is much higher compared to $O(T_2)$, thus resulting in an effective complexity of $O(NLT_1)$ for VUS, and $O(NT_1)$ for VUS$_{opt}$, and VUS$_{opt}^{mem}$. 
This translates to a significant improvement in execution time complexity for VUS$_{opt}$ and VUS$_{opt}^{mem}$ compared to VUS, which is confirmed by the results in Figure~\ref{fig:sythetic_exp_time}(c).

\vspace{0.2cm}
\noindent {\bf [Influence of $N$]}: 
Given the theoretical complexity depicted in Table~\ref{tab:complexity_summary}, it is evident that the number of thresholds affects all measures in a linear fashion.
Figure~\ref{fig:sythetic_exp_time}(e) demonstrates this point: the results of varying $N$ show a linear dependency for VUS, VUS$_{opt}$, and VUS$_{opt}^{mem}$ (i.e., a logarithmic trend with a log scale on the y axis). \commentRed{Moreover, we observe that the AUC and range-AUC execution time is almost constant regardless of the number of thresholds used. The latter is explained by the very efficient implementation of AUC measures. Therefore, the linear dependency on the number of thresholds is not visible in Figure~\ref{fig:sythetic_exp_time}(e).}

\vspace{0.2cm}
\noindent {\bf [Influence of $L$]}: Figure~\ref{fig:sythetic_exp_time}(f) depicts the influence of the maximum buffer length $L$ on the execution time of all measures. 
We observe that, as $L$ grows, the execution time of VUS$_{opt}$ and VUS$_{opt}^{mem}$ increases slower than VUS. 
We also observe that VUS$_{opt}^{mem}$ is more scalable with $L$ when compared to VUS$_{opt}$. 
This is consistent with the theoretical complexity (cf. Table~\ref{tab:complexity_summary}), which indicates that the dependence on $L$ decreases from $O(NL(T_1+T_2+\ell_a \alpha))$ for VUS to $O(NL(T_2+\ell_a \alpha)$ and $O(NL(\ell_a \alpha))$ for $VUS_{opt}$, and $VUS_{opt}^{mem}$.





\begin{figure*}[tb]
  \centering
  \includegraphics[width=\linewidth]{figures/synthetic_res.pdf}
  %\vspace*{-0.5cm}
  \caption{Execution time of VUS, R-AUC, AUC-based measures when we vary the parameters listed in Table~\ref{tab:parameter_range_time}. The solid lines correspond to the average execution time over 10 runs. The colored envelopes are to the standard deviation.}
  \label{fig:sythetic_exp_time}
\end{figure*}


\vspace{0.2cm}
In order to obtain a more accurate picture of the influence of each of the above parameters, we fit the execution time (as affected by the parameter values) using linear regression; we can then use the regression slope coefficient of each parameter to evaluate the influence of that parameter. 
In practice, we fit each parameter individually, and report the regression slope coefficient, as well as the coefficient of determination $R^2$.
Table~\ref{tab:parameter_linear_coeff} reports the coefficients mentioned above for each parameter associated with VUS, VUS$_{opt}$, and VUS$_{opt}^{mem}$.



\begin{table}[tb]
    \centering
    \caption{Linear regression slope coefficients ($C.$) for VUS execution times, for each parameter independently. }
    \begin{tabular}{|c|c|c|c|c|c|c|} 
 \hline
 Measure & Param. & $\alpha$ & $l_a$ & $|T|$ & $L$ & $N$\\ [0.5ex] 
 \hline\hline
 \multirow{2}{*}{$VUS$} & $C.$ & 21.9 & 0.02 & 2.13 & 212 & 6.24\\\cline{2-7}
 & {$R^2$} & 0.99 & 0.15 & 0.99 & 0.99 & 0.99 \\   
 \hline
  \multirow{2}{*}{$VUS_{opt}$} & $C.$ & 24.2  & 0.06 & 0.19 & 27.8 & 1.23\\\cline{2-7}
  & $R^2$& 0.99 & 0.86 & 0.99 & 0.99 & 0.99\\ 
 \hline
 \multirow{2}{*}{$VUS_{opt}^{mem}$} & $C.$ & 21.5 & 0.05 & 0.21 & 15.7 & 1.16\\\cline{2-7}
  & $R^2$ & 0.99 & 0.89 & 0.99 & 0.99 & 0.99\\[1ex] 
 \hline
\end{tabular}
    \label{tab:parameter_linear_coeff}
\end{table}

Table~\ref{tab:parameter_linear_coeff} shows that the linear regression between $\alpha$ and the execution time has a $R^2=0.99$. Thus, the dependence of execution time on $\alpha$ is linear. We also observe that VUS$_{opt}$ execution time is more dependent on $\alpha$ than VUS and VUS$_{opt}^{mem}$ execution time.
Moreover, the dependence of the execution time on the time series length ($|T|$) is higher for VUS than for VUS$_{opt}$ and VUS$_{opt}^{mem}$. 
More importantly, VUS$_{opt}$ and VUS$_{opt}^{mem}$ are significantly less dependent than VUS on the number of thresholds and the maximal buffer length. 







\subsubsection{Evaluation on TSB-UAD Time Series}\hfill\\
\label{sec:TSB_eval_time}

In this section, we verify the conclusions outlined in the previous section with real-world time series from the TSB-UAD benchmark. 
In this setting, the parameters $\alpha$, $\ell_a$, and $|T|$ are calculated from the series in the benchmark and cannot be changed. Moreover, $L$ and $N$ are parameters for the computation of VUS, regardless of the time series (synthetic or real). Thus, we do not consider these two parameters in this section.

\begin{figure*}[tb]
  \centering
  \includegraphics[width=\linewidth]{figures/TSB2.pdf}
  \caption{Execution time of VUS, R-AUC, AUC-based measures on the TSB-UAD benchmark, versus $\alpha$, $\ell_a$, and $|T|$.}
  \label{fig:TSB}
\end{figure*}

Figure~\ref{fig:TSB} depicts the execution time of AUC, R-AUC, and VUS-based measures versus $\alpha$, $\mu(\ell_a)$, and $|T|$.
We first confirm with Figure~\ref{fig:TSB}(a) the linear relationship between $\alpha$ and the execution time for VUS, VUS$_{opt}$ and VUS$_{opt}^{mem}$.
On further inspection, it is possible to see two separate lines for almost all the measures. 
These lines can be attributed to the time series length $|T|$. 
The convergence of VUS and $VUS_{opt}$ when $\alpha$ grows shows the stronger dependence that $VUS_{opt}$ execution time has on $\alpha$, as already observed with the synthetic data (cf. Section~\ref{sec:synthetic_eval_time}). 

In Figure~\ref{fig:TSB}(b), we observe that the variation of the execution time with $\ell_a$ is limited when compared to the two other parameters. We conclude that the variation of $\ell_a$ is not a key factor in determining the execution time of the measures.
Furthermore, as depicted in Figure~\ref{fig:TSB}(c), $VUS_{opt}$ and $VUS_{opt}^{mem}$ are more scalable than VUS when $|T|$ increases. 
We also confirm the linear dependence of execution time on the time series length for all the accuracy measures, which is consistent with the experiments on the synthetic data. 
The two abrupt jumps visible in Figure~\ref{fig:TSB}(c) are explained by significant increases of $\alpha$ in time series of the same length. 

\begin{table}[tb]
\centering
\caption{Linear regression slope coefficients ($C.$) for VUS execution time, for all time series parameters all-together.}
\begin{tabular}{|c|ccc|c|} 
 \hline
Measure & $\alpha$ & $|T|$ & $l_a$ & $R^2$ \\ [0.5ex] 
 \hline\hline
 \multirow{1}{*}{${VUS}$} & 7.87 & 13.5 & -0.08 & 0.99  \\ 
 %\cline{2-5} & $R^2$ & \multicolumn{3}{c|}{ 0.99}\\
 \hline
 \multirow{1}{*}{$VUS_{opt}$} & 10.2 & 1.70 & 0.09 & 0.96 \\
 %\cline{2-5} & $R^2$ & \multicolumn{3}{c|}{0.96}\\
\hline
 \multirow{1}{*}{$VUS_{opt}^{mem}$} & 9.27 & 1.60 & 0.11 & 0.96 \\
 %\cline{2-5} & $R^2$ & \multicolumn{3}{c|}{0.96} \\
 \hline
\end{tabular}
\label{tab:parameter_linear_coeff_TSB}
\end{table}



We now perform a linear regression between the execution time of VUS, VUS$_{opt}$ and VUS$_{opt}^{mem}$, and $\alpha$, $\ell_a$ and $|T|$.
We report in Table~\ref{tab:parameter_linear_coeff_TSB} the slope coefficient for each parameter, as well as the $R^2$.  
The latter shows that the VUS$_{opt}$ and VUS$_{opt}^{mem}$ execution times are impacted by $\alpha$ at a larger degree than $\alpha$ affects VUS. 
On the other hand, the VUS$_{opt}$ and VUS$_{opt}^{mem}$ execution times are impacted to a significantly smaller degree by the time series length when compared to VUS. 
We also confirm that the anomaly length does not impact the execution time of VUS, VUS$_{opt}$, or VUS$_{opt}^{mem}$.
Finally, our experiments show that our optimized implementations VUS$_{opt}$ and VUS$_{opt}^{mem}$ significantly speedup the execution of the VUS measures (i.e., they can be computed within the same order of magnitude as R-AUC), rendering them practical in the real world.











\subsection{Summary of Results}


Figure~\ref{fig:overalltable} depicts the ranking of the accuracy measures for the different tests performed in this paper. The robustness test is divided into three sub-categories (i.e., lag, noise, and Normal vs. abnormal ratio). We also show the overall average ranking of all accuracy measures (most right column of Figure~\ref{fig:overalltable}).
Overall, we see that VUS-ROC is always the best, and VUS-PR and Range-AUC-based measures are, on average, second, third, and fourth. We thus conclude that VUS-ROC is the overall winner of our experimental analysis.

\commentRed{In addition, our experimental evaluation shows that the optimized version of VUS accelerates the computation by a factor of two. Nevertheless, VUS execution time is still significantly slower than AUC-based approaches. However, it is important to mention that the efficiency of accuracy measures is an orthogonal problem with anomaly detection. In real-time applications, we do not have ground truth labels, and we do not use any of those measures to evaluate accuracy. Measuring accuracy is an offline step to help the community assess methods and improve wrong practices. Thus, execution time should not be the main criterion for selecting an evaluation measure.}

We present RiskHarvester, a risk-based tool to compute a security risk score based on the value of the asset and ease of attack on a database. We calculated the value of asset by identifying the sensitive data categories present in a database from the database keywords. We utilized data flow analysis, SQL, and Object Relational Mapper (ORM) parsing to identify the database keywords. To calculate the ease of attack, we utilized passive network analysis to retrieve the database host information. To evaluate RiskHarvester, we curated RiskBench, a benchmark of 1,791 database secret-asset pairs with sensitive data categories and host information manually retrieved from 188 GitHub repositories. RiskHarvester demonstrates precision of (95\%) and recall (90\%) in detecting database keywords for the value of asset and precision of (96\%) and recall (94\%) in detecting valid hosts for ease of attack. Finally, we conducted an online survey to understand whether developers prioritize secret removal based on security risk score. We found that 86\% of the developers prioritized the secrets for removal with descending security risk scores.


\begin{acknowledgements}
We thank the anonymous reviewers whose comments have greatly improved this manuscript. We also thank Yuhao Kang for his help during the early phase of this work. This research was supported in part by NetApp, Cisco Systems, Exelon Utilities, HPC resources from GENCI–IDRIS (Grants 2020-101471, 2021-101925), and EU Horizon projects AI4Europe (101070000), TwinODIS (101160009), ARMADA
(101168951), DataGEMS (101188416) and RECITALS (101168490).
\end{acknowledgements}

% BibTeX users please use one of
%\bibliographystyle{spphys}      % basic style, author-year citations
\bibliographystyle{spmpsci}      % mathematics and physical sciences
%\bibliographystyle{spphys}       % APS-like style for physics
\bibliography{references}   % name your BibTeX data base


\end{document}
% end of file template.tex
