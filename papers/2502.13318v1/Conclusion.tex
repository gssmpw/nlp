\section{Conclusions}
\label{sec:conclusions}


\begin{figure}[tb]
  \centering
  \includegraphics[height=7cm,width=\linewidth]{figures/overall_table.pdf}
  %\vspace*{-0.5cm}
  \caption{Ranks of the accuracy evaluation measures for the three different tests (i.e., on robustness, separability, and consistency) performed in the experimental evaluation, as well as the overall ranks (averaged for each measure on all tests). }
  \label{fig:overalltable}
\end{figure}

Time-series AD is a challenging problem, and an active area of research. 
Given the multitude of solutions proposed in the literature, it is important to be able to properly evaluate them.
In this paper, we demonstrate the limitations of threshold-based accuracy measures. 
Even though AUC-based measures solve the threshold issues, we show that they cannot handle lag and noise. Overall, we show that the proposed VUS-based measures are more robust, and better separate accurate methods from inaccurate ones.

\commentRed{Despite the significant scalability improvement brought by $VUS_{opt}$ and $VUS_{opt}^{mem}$, the execution time is still higher than that of the simple AUC-based and the threshold-based approaches. 
Nevertheless, since the VUS-based measures are more robust, separable, and consistent, studying further optimization strategies is an important research direction.
Even though VUS-based methods are only relevant to the offline accuracy evaluation step, improving the execution time would benefit the relevant benchmarks.}

