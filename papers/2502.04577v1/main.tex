\documentclass[11pt]{article} 
\usepackage{acl}


% Standard package includes
\usepackage{times}
\usepackage{latexsym}

\usepackage[T1]{fontenc}

% This assumes your files are encoded as UTF8
\usepackage[utf8]{inputenc}

\usepackage{inconsolata}

\usepackage{microtype}


%additional package(s)
\usepackage{graphicx}
%%%%% NEW MATH DEFINITIONS %%%%%

% \usepackage{amsmath,amsfonts,bm}
\usepackage{amsmath,amsfonts}

\usepackage{pifont}


\newcommand{\R}{\mathbb{R}}


\def\va{{\mathbf{a}}}
\def\vg{{\mathbf{g}}}

% Sets
\def\sR{\mathbb{R}}
\def\sC{\mathbb{C}}
\def\sZ{\mathbb{Z}}
\def\sN{\mathbb{N}}
\def\sQ{\mathbb{Q}}

\def\sS{\mathcal{S}}



% Vectors
\def\vzero{{\mathbf{0}}}
\def\vone{{\mathbf{1}}}
\def\vmu{{\mathbf{\mu}}}
\def\vtheta{{\mathbf{\theta}}}
\def\va{{\mathbf{a}}}
\def\vb{{\mathbf{b}}}
\def\vc{{\mathbf{c}}}
\def\vd{{\mathbf{d}}}
\def\ve{{\mathbf{e}}}
\def\vf{{\mathbf{f}}}
\def\vg{{\mathbf{g}}}
\def\vh{{\mathbf{h}}}
\def\vi{{\mathbf{i}}}
\def\vj{{\mathbf{j}}}
\def\vk{{\mathbf{k}}}
\def\vl{{\mathbf{l}}}
\def\vm{{\mathbf{m}}}
\def\vn{{\mathbf{n}}}
\def\vo{{\mathbf{o}}}
\def\vp{{\mathbf{p}}}
\def\vq{{\mathbf{q}}}
\def\vr{{\mathbf{r}}}
\def\vs{{\mathbf{s}}}
\def\vt{{\mathbf{t}}}
\def\vu{{\mathbf{u}}}
\def\vv{{\mathbf{v}}}
\def\vw{{\mathbf{w}}}
\def\vx{{\mathbf{x}}}
\def\vy{{\mathbf{y}}}
\def\vz{{\mathbf{z}}}
\def\vzeta{{\mathbf{\zeta}}}

% Matrix
\def\mA{{\mathbf{A}}}
\def\mB{{\mathbf{B}}}
\def\mC{{\mathbf{C}}}
\def\mD{{\mathbf{D}}}
\def\mE{{\mathbf{E}}}
\def\mF{{\mathbf{F}}}
\def\mG{{\mathbf{G}}}
\def\mH{{\mathbf{H}}}
\def\mI{{\mathbf{I}}}
\def\mJ{{\mathbf{J}}}
\def\mK{{\mathbf{K}}}
\def\mL{{\mathbf{L}}}
\def\mM{{\mathbf{M}}}
\def\mN{{\mathbf{N}}}
\def\mO{{\mathbf{O}}}
\def\mP{{\mathbf{P}}}
\def\mQ{{\mathbf{Q}}}
\def\mR{{\mathbf{R}}}
\def\mS{{\mathbf{S}}}
\def\mT{{\mathbf{T}}}
\def\mU{{\mathbf{U}}}
\def\mV{{\mathbf{V}}}
\def\mW{{\mathbf{W}}}
\def\mX{{\mathbf{X}}}
\def\mY{{\mathbf{Y}}}
\def\mZ{{\mathbf{Z}}}
\def\mBeta{{\mathbf{\beta}}}
\def\mPhi{{\mathbf{\Phi}}}
\def\mLambda{{\mathbf{\Lambda}}}
\def\mSigma{{\mathbf{\Sigma}}}


% Expectation
% \def\eE{\mathop{\mathbb{E}}\limits}
\def\eE{\mathbb{E}}

% Probability
\def\pP{\mathbb{P}}

% Tilde
\def\tf{\tilde{f}}
\def\tS{\tilde{S}}
\def\wtF{\widetilde{\mathcal{F}}}
\def\whR{\widehat{R}}
\def\tvx{\tilde{\mathbf{x}}}
\def\ty{\tilde{y}}


\def\defeq{\overset{\textup{def}}{=}}
% \def\defeq{\overset{.}{=}}
\def\defone{\overset{\text{\ding{172}}}{=}}
\def\deftwo{\overset{\text{\ding{173}}}{=}}
\def\leqone{\overset{\text{\ding{172}}}{\leq}}
\def\leqtwo{\overset{\text{\ding{173}}}{\leq}}
\def\leqthree{\overset{\text{\ding{174}}}{\leq}}
\def\leqfour{\overset{\text{\ding{175}}}{\leq}}
\def\eqone{\overset{\text{\ding{172}}}{=}}
\def\eqtwo{\overset{\text{\ding{173}}}{=}}
\def\eqthree{\overset{\text{\ding{174}}}{=}}
\def\eqfour{\overset{\text{\ding{175}}}{=}}
\def\geqfive{\overset{\text{\ding{176}}}{\geq}}
\usepackage{listings}
\usepackage{dsfont}

\definecolor{positional}{HTML}{7777F9}
\definecolor{nonpositional}{HTML}{777777}
\definecolor{lightgray}{HTML}{D3D3D3}

\usepackage{hyperref}
\usepackage{algorithm}
\usepackage{algpseudocode}
\usepackage{url}
\usepackage{graphicx}
\usepackage{booktabs}
\usepackage{colortbl}
\usepackage{enumitem}
\usepackage{linguex}
\usepackage{multirow}

\usepackage[compact]{titlesec}

\setlength{\abovedisplayskip}{2pt}
\setlength{\belowdisplayskip}{2pt}

\newcommand{\yb}[1]{\textcolor{purple}{\small [YB: #1]}}
\newcommand{\tal}[1]{\textcolor{blue}{\small [TAL: #1]}}
\newcommand{\ho}[1]{ \textcolor{ForestGreen}{[HO: #1]}}
\newcommand{\am}[1]{\textcolor{BurntOrange}{[AM: #1]}}

\newcommand{\ob}[2]{$\overbrace{\text{#1}}^{\text{#2}}$}
\newcommand{\ub}[2]{$\underbrace{\text{#1}}_{\text{#2}}$}


\title{Position-aware Automatic Circuit Discovery}

\author{
Tal Haklay$^1$  \quad Hadas Orgad$^1$  \quad David Bau$^2$  \quad Aaron Mueller$^{1,2}$  \quad Yonatan Belinkov$^1$  \\
$^1$Technion -- Israel Institute of Technology \quad $^2$Northeastern University \\[1ex]
\{\href{mailto:tal.ha@campus.ac.il}{\texttt{tal.ha}},
\href{mailto:orgad.hadas@campus.technion.ac.il}{\texttt{orgad.hadas}}\}\texttt{@campus.ac.il},
\{\href{mailto:d.bau@northeastern.edu}{\texttt{d.bau}},
\href{mailto:aa.mueller@northeastern.edu}{\texttt{aa.mueller}}\}\texttt{@northeastern.edu} \\
\href{mailto:belinkov@technion.ac.il}{\texttt{belinkov@technion.ac.il}}
}


\newcommand{\fix}{\marginpar{FIX}}
\newcommand{\new}{\marginpar{NEW}}

% formalisms:
% metric = m
% model = \mathcal{M}; circuit = \mathcal{C}
% empty model = \varnothing
% faithfulness = F
% token position = t; token span = T
% task = \mathcal{T}


\begin{document}


\maketitle

End-to-end imitation learning offers a promising approach for training robot policies. However, generalizing to new settings—such as unseen scenes, tasks, and object instances—remains a significant challenge. Although large-scale robot demonstration datasets have shown potential for inducing generalization, they are resource-intensive to scale. In contrast, human video data is abundant and diverse, presenting an attractive alternative. Yet, these human-video datasets lack action labels, complicating their use in imitation learning. Existing methods attempt to extract grounded action representations (e.g., hand poses), but resulting policies struggle to bridge the embodiment gap between human and robot actions.
% our approach
We propose an alternative approach: leveraging language-based reasoning from human videos - essential for guiding robot actions - to train generalizable robot policies. Building on recent advances in reasoning-based policy architectures, we introduce Reasoning through Action-free Data (RAD). RAD learns from both robot demonstration data (with reasoning and action labels) and action-free human video data (with only reasoning labels). The robot data teaches the model to map reasoning to low-level actions, while the action-free data enhances reasoning capabilities. Additionally, we will release a new dataset of 3,377 human-hand demonstrations compatible with the Bridge V2 benchmark. This dataset includes chain-of-thought reasoning annotations and hand-tracking data to help facilitate future work on reasoning-driven robot learning.
% experiments
Our experiments demonstrate that RAD enables effective transfer across the embodiment gap, allowing robots to perform tasks seen only in action-free data. Furthermore, scaling up action-free reasoning data significantly improves policy performance and generalization to novel tasks. These results highlight the promise of reasoning-driven learning from action-free datasets for advancing generalizable robot control. 
% releasing dataset
Website: \href{https://rad-generalization.github.io}{here}.

\begin{figure}[ht]
    \centering
    \includegraphics[width=0.8\linewidth]{graphs/greater_than_naive.pdf}
    \vspace{0.5cm}
    \includegraphics[width=0.8\linewidth]{graphs/p1_bottom.png}
    \vspace{-5pt}
    \caption{\textcolor{positional}{Positional} vs.\ \textcolor{nonpositional}{non-positional} circuits. In a \textcolor{nonpositional}{non-positional} circuit, the same edges must be included at all positions. A \textcolor{positional}{positional} circuit can distinguish between the same edge at different positions. This specificity yields better trade-offs between circuit size and faithfulness. It can also increase both precision and recall.}
    \label{fig:p1}
    \vspace{-5pt}
\end{figure}

\section{Introduction}

\looseness=-1
A primary goal of interpretability research is to characterize the internal mechanisms in language models (LMs) and other NLP models. 
A core approach in this area is \textbf{circuit discovery}---identifying the minimal subgraph within the model's computation graph that performs a specific task \citep{olah2021framework,olah-mech}.
Typically, the nodes of a circuit represent model components (e.g., attention heads, neurons, or layers).
While manual circuit discovery methods can yield position-specific insights \citep{wanginterpretability,goldowskydill2023localizingmodelbehaviorpath}, \emph{automatic methods often overlook positional information}, treating components as uniformly relevant across all input token positions \citep{conmytowards,syed2023attribution}. 
For instance, if an attention head is included in a circuit, it is assumed to contribute equally to the computation for every position in the input sequence.
The assumption that circuits are position-invariant ignores the fact that different positions often require distinct computations.
By ignoring positions, current methods limit their ability to capture mechanisms that operate across positions, such as interactions between attention heads across positions.

In this study, we start by demonstrating that positional agnosticism is a significant limitation (\S\ref{sec:motivating}). Then, to address these limitations, we introduce a new approach: position-aware edge attribution patching (PEAP; \S\ref{sec:full_circ_discovery}; Figure~\ref{fig:p1}). Current approaches  assume that if an edge is in a circuit, then the same edge will be in the circuit at all positions, thus leading to low precision. It is also assumed that an edge's importance should be aggregated across positions before deciding whether it should be included in the circuit; this can lead to cancellation effects, and thus low recall. PEAP instead allows us to compute the importance of cross-positional edges, and separately evaluates edge importance at each position. We show that this leads to smaller and more accurate circuits; see Figure~\ref{fig:p1}.

Incorporating positional information into circuit discovery is straightforward when inputs have the same length and structure across examples.

However, realistic datasets are not nearly this templatic.
How, then, can we incorporate positional information into automatic circuit discovery?
To address this challenge, we propose \textbf{schemas} (\S\ref{sec:schema}). 
Schemas assign semantic labels to spans of tokens, enabling information aggregation across examples even when the spans differ in length.

For example, in the input ``The \textcolor{positional}{war} lasted from 1453 to 14\underline{\hspace{1em}},'' the span ``\textcolor{positional}{war}'' could be labeled as ``\emph{Subject}''.
This enables handling spans with varying lengths: the phrase ``\textcolor{positional}{Black Plague}'' in another example can be treated as a single positional span with the same role as ``\textcolor{positional}{war}''.
In experiments with two LMs and three tasks, we find that circuits discovered using schemas achieve a better trade-off between circuit size and faithfulness to the model's behavior than position-agnostic circuits.
Importantly, position-aware circuits offer a more precise representation of the underlying mechanisms, providing a more concise foundation for mechanistic explanations.

We also present a fully automated pipeline for schema generation and application (\S\ref{sec:schema-generation}) using large language models (LLMs). 
We evaluate the quality of the generated schemas and their utility in discovering position-aware circuits (\S\ref{sec:schema-eval}).
Notably, circuits derived using automatically generated and applied schemas achieve comparable faithfulness scores to circuits discovered with human-designed and manually applied schemas.

We summarize our contributions as follows:
\begin{itemize}[noitemsep,leftmargin=*,topsep=1pt,parsep=1pt]
    \item Introduce a position-aware circuit discovery method, which obtains better faithfulness than position-agnostic discovery.  
    \item Introduce dataset schemas,  facilitating positional circuit discovery in more naturalistic settings. 
    \item Develop an automated schema generation and application pipeline with LLMs, yielding schemas that are comparable to manually-annotated ones.
\end{itemize}

\section{Background and Motivation }\label{sec:motivating}
\begin{figure*}
    \centering
    \includegraphics[width=0.45\linewidth]{graphs/failure_cancellation.pdf}\hspace{1em} 
\includegraphics[width=0.45\linewidth]{graphs/failure_over.pdf}
    \caption{\textbf{Left}: The yellow edge at position 1 has the highest score of 100, indicating it is the most important edge. However, aggregating across positions causes scores of opposite signs to cancel. This causes the yellow edge to be incorrectly ranked as the \emph{least} important. \textbf{Right}: The yellow edge at position 1 has the highest score; the scores of other edges are consistently high (but lower) at many positions. After summing  across positions, the non-yellow edges have higher scores. Thus, the yellow edge is incorrectly ranked as the least important.}
    \label{fig:failure-exp}
\end{figure*}


A circuit is a subgraph of the model's computation graph;
it can be conceptualized as a binary mask $B(V,E,\mathcal{T})$ over all components $V$ and edges $E$ in the graph, selecting the components and edges that have the strongest effect on the model's behavior on a target task $\mathcal{T}$.
There are many methods for computing the influence of a component on the model's behavior on $\mathcal{T}$, including activation patching \citep{vig-2020-gender,finlayson-etal-2021-causal,geiger2021causal}, path patching \citep{wanginterpretability,goldowskydill2023localizingmodelbehaviorpath}, and edge patching \citep{hanna2024have,marks2024sparsefeaturecircuitsdiscovering}, with 
 attribution patching to approximate direct patching \cite{nanda2023attribution,syed2023attribution}.  
We focus  on edge patching, which aims to identify edges in $E$ that are causally important for $\mathcal{T}$.
For each such edge $(u,v)$, the nodes $u$ and $v$ are included in the circuit.

\emph{Manual} circuit discovery methods can distinguish between components at different token positions; examples include the IOI circuit \citep{wanginterpretability}, the Greater-Than circuit \citep{hanna2024does}, and the Attribute-Binding circuit \citep{prakashfine}.
The authors determined connections between attention heads by examining attention patterns and establishing connections if a head at one position strongly attended to a head at another.
However, this approach has three key limitations: (1) it is not scalable, (2) it is prone to human bias, and (3) it is unclear whether strong attention scores reliably indicate the a causal connection to the downstream metric \citep{jain2019attention}.

In contrast, \emph{automatic} approaches \citep{syed2023attribution, hanna2024have} systematically examine every connection and evaluate them \emph{quantitatively} via their causal effect on the downstream metric. However, when using automatic methods it is common to aggregate across token positions,\footnote{Cf.\ \citet{kramar2024atp}, propose a variant of attribution patching and perform position-sensitive node attribution. They do not use it to discover positional circuits.} which causes specific problems that we now define.

\paragraph{Cancellations across positions (low recall).}
If a component has scores with opposite signs across different positions, summing these scores can partially cancel out the component's overall effect, potentially resulting in a near-zero score (Figure~\ref{fig:failure-exp}, left).
\citet{kramar2024atp} note that cancellation can occur when aggregating across examples in the dataset. We observe that the extent of this phenomenon is larger than previously assumed: it can occur \emph{within a single sample} across positions. 
To measure cancellation effects across positions, we compare importance rankings from edge attribution patching (EAP; \citealp{syed2023attribution}) under two positional aggregation methods: (i) summing the absolute scores across both positions and examples (unaffected by cancellation effects); and (ii) summing scores across positions and then summing the absolute scores across different examples (affected by cancellation effects). 
We observe (Table~\ref{tab:intersections}, Top)  that the two rankings differ significantly at the most important components. 

\paragraph{Importance overestimation (low precision).} 
Circuits that do not consider positional information may favor edges that have small impacts at many positions over edges that have large impact in one or few positions (Figure \ref{fig:failure-exp}, Right). To measure overestimation effects we compare importance rankings derived from two aggregation methods: (i) summing the absolute scores across both positions and examples; and (ii) taking the max of the absolute across positions and then summing scores across different examples. Table \ref{tab:intersections} (Bottom) provides evidence for this phenomenon.


These problems motivate a circuit discovery method that takes position into account. We introduce this method in \S\ref{sec:full_circ_discovery}. 


\begin{table}[t!]
\vspace{-10pt}
    \centering   
     \begin{tabular}{lcccc}
    \toprule
    \rowcolor{lightgray} \multicolumn{5}{c}{Cancellation} \\
    $K\%$ & Diff & Diff$_{\text{Control}}$ & $\rho$ & $\rho_{\text{Control}}$ \\
    \midrule
    1 & 17.1\% & 3.9\% & 0.760 & 0.985 \\
    5 & 13.4\% & 2.4\% & 0.831 & 0.991 \\
    10 & 12.1\% & 2.3\% & 0.877 & 0.992 \\
    \midrule
    \rowcolor{lightgray} \multicolumn{5}{c}{Overestimation} \\
    $K\%$ & Diff & Diff$_{\text{Control}}$ & $\rho$ & $\rho_{\text{Control}}$ \\
    \midrule
    1 & 17.5\% & 3.6\% & 0.772 & 0.984 \\
    5 & 14.6\% & 2.1\% & 0.811 & 0.993 \\
    10 & 12.4\% & 2.2\% & 0.864 & 0.993 \\
    \bottomrule
    \end{tabular}
   \caption{Cancellation and overestimation effects when ignoring positions. We rank edges by their importance scores (IOI task, GPT2-small), and take the top $K\%$. We compute the set difference (Diff) and rank correlations ($\rho$) between rankings produced by the two aggregation methods described in \S\ref{sec:motivating}. We define the difference of two ranking lists $R_1$,$R_2$ at length L as $1-\frac{|R_1 \bigcap R_2|}{L}$. As a control, we also compute the mean pairwise set difference (Diff$_\text{Control}$) and rank similarities ($\rho_\text{Control}$) produced by the \emph{same} aggregation method across 3 data subsets. Differences with respect to control are all significant ($p<.01$).
   }
    \label{tab:intersections}
    \vspace{-5pt}
\end{table}


\section{Position-aware Edge Attribution Patching (PEAP)} \label{sec:full_circ_discovery}

The importance of an edge $e$ is typically measured with the indirect effect (IE) of the edge on some target metric $M$.
In direct activation patching, also known as causal mediation analysis \cite{Pearl:2001:DIE:2074022.2074073,vig-2020-gender, mueller2024quest}, 
the IE is the change in the metric $M$ when the edge is `patched' to some counterfactual value, e.g., the edge value in a run on a different input $x'$:  $M(x|e=e_{x'}) - M(x)$. Performing this intervention at every edge is costly, prompting approximate algorithms. 
Edge attribution patching (EAP; \citealp{syed2023attribution})
linearly approximates the IE, $g(e)$, of edge $e=(u, v)$ as follows:
\begin{equation}
    \resizebox{\linewidth}{!}{$g(e) = M(x|e=e_{x'}) - M(x) \approx (z^*_u - z_u)^\top \nabla_v M(x) \label{eq:eap}$}
\end{equation}

The target metric $M$ can vary depending on the task.
Typically, $M$ is the logit difference between a correct completion and a minimally different incorrect completion.
$z_{u}$ and $z^*_{u}$ are the clean and counterfactual activations at the output of $u$, and $\nabla_v M(x)$ is the gradient of $M(x)$ w.r.t the input of $v$. 
\citet{syed2023attribution} showed EAP  to outperform direct activation patching with a greedy approach \citep{conmytowards}.
However, 
\citeauthor{syed2023attribution}\ only discovered circuits that do not consider positions. 

\subsection{Method} Equation~\ref{eq:eap} holds only when $u$ and $v$ are at the same position.
To include token positions in the circuit, attention edges that cross positions must be included in the discovey process. 
In autoregressive Transformer-based models,
these edges exist between nodes representing a given attention head that operates at different positions. 
Let $h^i_{t,l}$ denote the node corresponding to the $i$-th attention head at token position $t$ in layer $l$. 
Following \citet{olah2021framework}, we view the contribution of head $h^i_{t}$ to the residual stream % ($z_{h^i_{t}}$) 
as:
%
\begin{equation}
z_{h^i_{t}} = W^{i}_{O} (\text{softmax} ( \frac{q^i_{t}{K^i_{t}}^\top}{\sqrt{d_k}}  ) V^i_{t} ) \in \mathbb{R}^{d_{\text{model}}}
\label{eq:attn}
\end{equation}
Here, \( W_{O}^{i} \) represents the columns of the projection matrix \( W_{O} \) that specifically project the output of head \( h^i \). \( K^i_{t} \in \mathbb{R}^{t \times d_{\text{head}}} \) is the key matrix, and \( V^i_{t} \in \mathbb{R}^{t \times d_{\text{head}}} \) is the value matrix.

$h^i_{t}$  is connected to 
 every node  $h^i_{t',l}$ at position $t' \le t$, via \textbf{3 edges}: the value vector $v^i_{t',l}$, the key vector $k^i_{t',l}$, and the query vector $q^i_{t,l}$. As direct communication between heads occurs only within the same layer, 
we omit henceforth the layer notation and assume that all attention edges connect attention heads within the same layer.


To approximate the attribution scores of attention edges, we first calculate \( z^*_{h^i_t} \), the corrupted output of the head \( h^i_t \) caused by patching \( v^i_{t'} \), \( k^i_{t'} \), or \( q^i_t \). We then approximate the attribution as follows:
\begin{equation}
    M(x| e = e_{x'}) - M(x) \approx (z^*_{h^i_{t}} - z_{h^i_{t}})^\top \nabla_{z_{h^i_{t}}} M(x) \label{eq:eap-attn}
\end{equation}

Based on Eq.~\ref{eq:attn}, we define the corrupted vector \( z^*_{h^i_t} \) for patching $v_{t'}^i$ (Eq.~\ref{eq:patch-v}), patching $k_{t'}^i$ (Eq.~\ref{eq:patch-k}), and patching $q_{t}^i$ (Eq.~\ref{eq:patch-q}):

\begin{align}
\label{eq:patch-v} \resizebox{\linewidth}{!}{$z^*_{h^i_{t}} = W^{i}_{O} (\text{softmax} \left( \frac{q^i_{t}{K^i_{t}}^\top}{\sqrt{d_k}}\right) \left[ v_1^i, ..., {v_{t’}}^*, ..., v_t^i\right])$}\\
\label{eq:patch-k} \resizebox{\linewidth}{!}{$z^*_{h^i_{t}} = W^{i}_{O} (\text{softmax} \left( \frac{q^i_{t}{\left[ k_1^i, ..., {k_{t’}}^*, ..., k_t^i \right]}^\top}{\sqrt{d_k}}\right)V^i_{t} )$}\\
\label{eq:patch-q} \resizebox{\linewidth}{!}{$z^*_{h^i_{t}} = W^{i}_{O} (\text{softmax} \left( \frac{{\left[ q^i_{t} {k_1^i}^\top, ..., {q^i_{t}}^* {k_{t’}}^\top, ..., q^i_{t} {k_t^i}^\top \right]}}{\sqrt{d_k}}\right)V^i_{t} )$}
\end{align}

Figure \ref{fig:attention_patching} provides an illustration of each type of patching. 
\begin{figure}[t]
    \centering
    \includegraphics[scale=0.2]{graphs/patching.pdf}
    \caption{\vspace{-2pt}
    Illustration of the attention mechanism from the perspective of position 3. We approximate how patching \textcolor[RGB]{147,145,255}{$v_1$}, \textcolor[RGB]{109,177,255}{$k_1$} or \textcolor[RGB]{255,128,223}{$q_3$} impacts the downstream metric via the output of the attention head at position 3.}
    \label{fig:attention_patching}
    \vspace{-6pt}
\end{figure}
By using PEAP to approximate attention edges, we can now approximate both within-position edges and cross-position edges.

Once the attribution scores for all edges have been computed, we construct the circuit using an adapted version of the greedy algorithm proposed by \citet{hanna2024have}. See App.~\ref{ap:circuit construcion} for details. 

\vspace{-2pt}
\subsection{Preliminary Demonstration}
\vspace{-2pt}
We now compare PEAP to the position-agnostic approach of \citet{syed2023attribution} using the Greater-Than task \citep{hanna2024does} on GPT2-small \citep{radford2019language}.
The dataset includes prompts like: ``The war lasted from the year 1741 to the year 17\underline{\hspace{1em}}'' and counterfactual variants with ``01'' as the starting year (e.g., ``The war lasted from the year 1701 to the year 17\underline{\hspace{1em}}'').
The downstream metric $M$ measures the probability difference between valid and invalid year answers.
We use 500 examples each for circuit discovery and evaluation, considering only prompts with valid model predictions.
Circuit evaluation is based on two metrics: (1) \textbf{Soft Faithfulness} ($F_S(C) = \frac{M(C)}{M(\mathcal{M})}$), comparing the circuit's performance to the full model's, and (2) \textbf{Hard Faithfulness} ($F_H(C) = \mathds{1}\{C_T = \mathcal{M}_T\}$), assessing token agreement at the final position $T$.
While $F_S$ is more commonly used, we see $F_H$ as a more behaviorally grounded metric.


\begin{figure*}[t]
    \centering
    \includegraphics[width=0.95\linewidth]{graphs/schema_example.png}
    \caption{Example schema for each task. We show examples from the LLM+Mask method. See \S\ref{ap:task details} for examples of human-designed schemas.}
    \label{tab:schema-example}
    \vspace{-5pt}
\end{figure*}

Figure~\ref{fig:p1} presents the faithfulness scores of the Greater-Than task for both methods as a function of circuit size. PEAP enables the discovery of circuits that improve the trade-off between circuit size and faithfulness: \textbf{position-aware circuits are smaller, and yet achieve similar faithfulness with orders-of-magnitude fewer edges.}


\subsection{Aggregating Scores Across Examples}

In the Greater-Than dataset, we can simply aggregate position-specific scores across examples.
This naive approach works because all examples in the Greater-Than dataset consist of exactly the same number of tokens, and each position has the same meaning across all examples.
In other words, this approach requires all examples in the dataset to be \emph{fully position-aligned}.
This raises a key challenge for non-templatic datasets: the same token position may not have the same meaning across examples, and examples may vary in length.

Prior methods addressing positionality typically follow one of two strategies: 
(1) \textbf{full alignment}, where the dataset is generated from a single template---as in the Greater-Than dataset---and (2) \textbf{partial alignment}, where specific token position roles are consistent across examples.
For instance, in the IOI dataset \citep{wanginterpretability}, the authors \emph{manually} identified five key single-token roles (IO, S1, S1+1, S2, End) shared across all prompt templates, which are sufficient for constructing a faithful circuit. 
In the next section, we describe an automatic approach inspired by partial alignment that  enables us to include positional information in tasks with variable-length inputs.

\section{Schemas for Variable-length Inputs} 
\label{sec:schema}

Discovering circuits requires aggregating edge scores across examples. 
However, because edges correspond to specific positions in the computation graph, naive aggregation assumes perfect positional alignment across examples---an impractical assumption for most datasets.
To address this challenge, we relax this assumption and only assume that examples share a similar high-level structure, which is represented by a \textbf{schema}.
A dataset schema identifies \textit{spans} within input examples, where each span covers consecutive tokens grouped under a meaningful category.
For instance, in the input ``The \textcolor{positional}{war} lasted from 1453 to 14\underline{\hspace{1em}}'', the span ``\textcolor{positional}{war}'' could be labeled \emph{Subject}.
This allows us to handle spans of varying lengths, such as treating ``\textcolor{positional}{Black Plague}'' in another example as a single position with the same role as ``\textcolor{positional}{war}''.
Examples of schemas for specific datasets are shown in Figure~\ref{tab:schema-example}.
Schemas are defined based on semantic, syntactic, or other patterns in the data, and may be guided by knowledge of how the model processes examples.
Spans are ordered sequentially within the input, covering all parts of a prompt.\footnote{Future work may relax the sequential order assumption to support even greater variation across examples.}

\subsection{Discovering Circuits at the Schema Level}
When all examples share the same schema-defined structure, we can leverage this consistency to create an abstract computation graph for all examples.
For now, we assume spans in the schema can be automatically mapped to corresponding tokens in any dataset sample. We discuss automating this process later.

Let $ \smash{ G_{x}=(E_x,V_x) }$ represent the computation graph derived from example $x \in \mathcal{D}$.
Given schema $\mathcal{S}$ with $k$ spans, we define the \textit{abstract} computation graph $G_\mathcal{S}=(E_\mathcal{S},V_\mathcal{S})$, which is structurally equivalent to a computation graph of $\mathcal{M}$ on an input of length $k$. Intuitively, each span is represented by a single position.

At a high level, given an example, we (i) compute edge scores on the true computation graph $G_x$;
(ii) map from edges in $G_x$ to edges in $G_\mathcal{S}$, and sum edge scores in $G_x$ to compute edge scores in $G_\mathcal{S}$;
(iii) construct a circuit in $G_\mathcal{S}$.


To this end, we define a mapping $f_{x}: E_{\mathcal{S}} \rightarrow 2^{E_x}$ from an edge $e=(u_{s_1},v_{s_2})$ to a set of edges in $E_x$:
%
\begin{equation}
    f_{\mathcal{S}}^x(e)=\{ e'\in G_x \mid e=(u_i, u_j), i \in s_1, j\in s_2 \}
\end{equation}
%
where $u_{s_1},v_{s_2}$ represent components in the computation graph at spans $s_1, s_2$. 

Given an attribution function $g_x$ (defined at the token position level), the attribution score $g_\mathcal{S}$ (defined at the segment level) of the edge $e\in G_{\mathcal{S}}$ is the sum of all the edge effects mapped to this edge, averaged over all examples in the task dataset:

\begin{equation}
    g_{\mathcal{S}}(e) = \frac{1}{|\mathcal{D}|}\sum_{x \in \mathcal{D}}\sum_{e' \in f^x_\mathcal{S}(e)} g_{x}(e')
\end{equation}

\begin{figure}[t]
    \centering
    \includegraphics[width=0.97\linewidth]{graphs/abs_vs_reg.pdf}
    \caption{Circuits defined over schemas. Every node/edge at position \( s \) in the abstract computation graph is mapped to a set of nodes/edges in the full computation graph within the span \( s \).}
    \label{fig:abs_graph}
\end{figure}

After computing the attribution score for each edge in $G_{\mathcal{S}}$, we construct the abstract circuit  $\mathcal{C}_{\mathcal{S}} \subseteq G_{\mathcal{S}}$ with the same greedy algorithm used in the previous section (see App.~\ref{ap:circuit construcion}).

\paragraph{Faithfulness evaluation.}
The process of faithfulness evaluation involves ablating edges that are not included in the circuit. 
To evaluate an abstract circuit on a sample $x\in\mathcal{D}$, we convert back to the computational graph $G_x$ and construct $\mathcal{C}_x \subseteq G_x$:
\begin{equation}
\mathcal{C}_x = \{ e \mid e \in f^x_\mathcal{S}(e'), \forall e' \in C_{\mathcal{S}} \}
\end{equation}
In other words, for every edge $e'$ in the abstract circuit $C_{\mathcal{S}}$, the corresponding edges in $f_x(e')$ form the circuit $\mathcal{C}_x$.
Figure \ref{fig:abs_graph} depicts this process.


\subsection{Automating Schema Generation and Application} \label{sec:schema-generation}
Given a schema $ \mathcal{S} $ and a function $ f_\mathcal{S} $ to apply it to every sample $ x \in \mathcal{D} $, we can automatically discover position-aware circuits, even for tasks involving variable-length examples. 
However, as shown in Figure \ref{tab:schema-example}, schema definitions are dataset-specific, requiring tedious manual work and intricate knowledge of the task at hand as well as knowledge of the analyzed model's computations.
Applying the schemas may also require deep knowledge on the target dataset.
To generate interpretable circuits, schemas must be both faithful to the model and meaningful to humans.

In this section, we propose an automated process for schema generation and application to streamline circuit discovery.
Inspired by recent work on LLM agents \cite{wang2024survey} for automated interpretability \cite{schwettmann2023find,shaham2024multimodal}, we investigate the use of LLMs for generating and applying schemas.


\paragraph{Schema Application.}
Applying a schema entails mapping each token to a specific span.
After defining the schema, we utilize an LLM to perform the application process.
We provide the prompt for applying the schema in App.~\ref{ap:schema application}.

\paragraph{Schema Generation.}
Creating a schema requires specifying span types while two conditions: (1) spans must follow the same order across all examples, and (2) each prompt must be fully covered by the spans.
These criteria are incorporated into the LLM's prompt (details in App.~\ref{ap:schema generation}).
Given a dataset, we use an LLM to create three schema versions based on distinct subsamples, then have the LLM unify these versions into a final schema.
The schema is validated by confirming it applies to at least 80\% of the subsampled data;
otherwise, the process is repeated.
Examples of LLM-generated schemas are shown in Figure~\ref{tab:schema-example}.


\paragraph{Saliency scores: A model-based approach for schema generation.}

The schema generation described above does not account for the computations performed by the target model $\mathcal{M}$ on the given dataset $\mathcal{D}$, potentially producing unfaithful schemas (as we will show in \S\ref{sec:results}).
To address this, we incorporate the importance of each token position to the model's computation into the schema generation.

Our key idea is to inform the LLM which positions significantly influence the model's decisions.
While many feature attribution methods can be explored \cite{danilevsky-etal-2020-survey,wiegreffe2021teach,wallace-etal-2020-interpreting}, we employ a simple saliency score, inputXgradient \cite{shrikumar2017learning}.
The score of a token in position $t$ is defined as 
$ s(t) = \| \mathbf{e_t} \cdot \nabla_{\mathbf{e_t}} M(x) \| $, where $ \mathbf{e_t} $ is the token embedding at position $ t $.
We compute a softmax over these scores and define a mask for each example as follows:
\begin{equation}
m(t) =
\begin{cases}
1 & \text{if } \frac{e^{s(t)}}{\sum_{i=1}^{n} e^{s(i)}} > \frac{1}{n}, \\
0 & \text{otherwise},
\end{cases}
\end{equation}
Where $n$ is the prompt length.
This mask is then attached to each example, and the LLM is instructed to use it when designing the schema.
Token position which is important across many examples should be placed in its own span. Further information on mask construction and alternative attribution methods can be found in Appendix~\ref{ap:mask-creation}.


\paragraph{Schema Evaluation.}\label{sec:schema-eval}
We propose two intrinsic metrics and one extrinsic metric to evaluate the entire schema pipeline.
Intrinsic metrics assess the LLM schema application. An application is \textbf{valid} if span labels are ordered correctly and every token is assigned to a single span, and \textbf{correct} if it matches a human application for the same schema.
Extrinsic metrics evaluate schema design and application through circuit discovery.
A good schema definition and application should achieve better trade-offs between circuit size and faithfulness.

Invalid schema applications are filtered out for both the discovery and evaluation datasets, while incorrect applications are retained since automating their filtering is infeasible in general datasets.
If an application is valid but incorrect, we expect it to affect the faithfulness of the discovered circuit.
To ensure minimal distribution shift in the dataset, we consider a generation and an application of a schema on an entire dataset as successful if at least 90\% of the examples are valid.
This means that each circuit is discovered using a slightly different set of examples (up to 10\%), but we ensure that all circuits are compared using the exact same evaluation set, which is the intersection of the examples for all runs. In practice this intersection includes 90\% of the total dataset examples.
In our experiments, three full pipeline runs were usually sufficient to achieve at least one successful run.

We found Claude 3.5 Sonnet \citep{claude3} to perform well in both schema generation and application, achieving high validity and correctness scores (Table~\ref{tab:schema-eval}, Appendix~\ref{ap:schema-validation}).
We also experimented with Llama-3-70B \citep{grattafiori2024llama3} and GPT-4o \citep{openai2024gpt4ocard}, but they failed to meet our thresholds for valid applications.
In \S\ref{sec:results}, we show that LLM-generated schemas score well on extrinsic quality measures, with saliency-enhanced schemas proving comparable to human-designed ones.


\begin{table*}[
htb!]
\centering
\caption{{Individual fairness improves with DistFaiR re-ranking intervention, but the difference depends on the divergence measure used.} We show \emph{relative improvement} in fairness post- fair ranking intervention with respect to the original ranking. The columns (i.e., $\Delta$ measure) correspond to different fairness measures, while each row corresponds to a fair re-ranking method. Post-processing the rankings with DistFaiR improves individual fairness across datasets. Group fairness also improves with DistFaiR in most cases. Arrows indicate direction of better performance, with best performance bolded for each fairness metric. {\em Note that the criterion of the fairness scores varies across cross-columns, so cross-column comparisons are incorrect. }}
\label{tab:full_results_main}
\adjustbox{max width=0.8\linewidth}{%
 \begin{tabular}{llcccccc}
\toprule
Dataset & Method  & \multicolumn{3}{c}{Relative Change in Individual Fairness ($\uparrow$)}  &\multicolumn{3}{c}{Relative Change in Group Fairness ($\uparrow$)} \\
\cmidrule(r){3-5}\cmidrule(r){6-8}\\
 &  & $\Delta$ DistFaiR ($L_1$) & $\Delta$ DistFaiR ($L_2^{var}$) & $\Delta$ DistFaiR ($W_1$)  & $\Delta$ DistFaiR ($L_1$) & $\Delta$ DistFaiR ($L_2^{var}$) & $\Delta$ DistFaiR ($W_1$)  \\

\midrule
\multirow{5}{*}{\texttt{synth-binary}} & IAA & \textbf{82.50}\% & \textbf{90.89\%} & \textbf{68.18\%} & 8.84\% & 12.06\% & 0.54\%\\
 & FoE & 9.17\% & 14.65\% & 7.58\% & 20.81\% & 18.89\% & 1.85\%\\
  \cdashline{2-8}[1pt/3pt]
 & DistFaiR($L_1$) & \textbf{82.50\%} & \textbf{90.89\%} & \textbf{68.18\%} & 47.05\% & 40.87\% & 5.38\%\\
 & DistFaiR($L_2^{var})$ & 76.50\% & \textbf{90.89\%} & 65.02\% & 59.05\% & 48.55\% & \textbf{6.58}\% \\
 & DistFaiR($W_1$) & 77.81\% & 90.68\% & \textbf{68.18\%} & \textbf{76.10\%} & \textbf{67.26\%} & 3.69\%\\
 
\midrule

\multirow{5}{*}{\texttt{synth-cont}} & IAA & 61.39\% & \textbf{63.56\%} & 40.76\% & \textbf{38.52\%} & \textbf{42.22\%} & 32.32\% \\
& FoE & 2.02\% & -1.23\% & 3.74\% & -139.41\% & -273.17\% & 5.92\%\\
\cdashline{2-8}[1pt/3pt]
 & 
 DistFaiR($L_1$) & \textbf{62.02\%} & 60.42\% & 39.64\% & -36.10\% & -60.62\% & 66.89\%\\
 &
 DistFaiR($L_2^{var}$) & 61.84\% & 62.20\% & 39.95\% & -39.66\% & -75.31\%  & \textbf{67.88\%}\\
 & DistFaiR($W_1$) & 51.22\% & 58.29\% & \textbf{40.89\%} & -125.17\% & -235.10\% & 58.11\% \\
\midrule

\multirow{5}{*}{\texttt{FairTREC2021}}  & IAA &  \textbf{68.76\%} & \textbf{78.69\%} & \textbf{64.95\%} & 48.99\% & 76.02\% & \textbf{4.32\%} \\
& FoE & 17.52\% & 25.64\% & 17.36\% & 45.67\% & 68.15\% & -5.47\%\\
  \cdashline{2-8}[1pt/3pt]
 &DistFaiR($L_1$)& \textbf{68.76\%} & 77.75\% & \textbf{64.95\%} & \textbf{50.78\%} & \textbf{80.61\%} & 0.34\%\\
 & DistFaiR($L_2^{var}$) & \textbf{68.76\%} & \textbf{78.69\%} & 64.57\% & 42.62\% & 71.35\% & 0.36\%\\
 & DistFaiR($W_1$) &  68.27\% & 78.40\% & 64.77\% & 50.13\% & 78.46\% & -18.66\%\\


 
\midrule
\multirow{5}{*}{\texttt{rateMDs}}  & IAA &28.30\% & 46.03\% & 22.48\% & -7.94\% & -87.81\% & 5.82\%\\
& FoE & 3.37\% & 6.15\% & 2.99\% & 26.96\% & 40.71\% & 2.37\%\\
  \cdashline{2-8}[1pt/3pt]
 & 
 DistFaiR($L_1$)&\textbf{69.80\%} & 86.74\% & 62.90\% & \textbf{62.19\%} & \textbf{79.90\%} & 5.18\% \\
 &DistFaiR($L_2^{var}$) & 66.83\% & \textbf{86.75\%} & 59.74\% & 39.99\% & 60.80\% & 1.67\% \\
 & DistFaiR($W_1$) & 67.76\% & 85.34\% & \textbf{64.28\%} & 57.41\% & 78.10\% & \textbf{7.89\%} \\


\bottomrule
\end{tabular}}
\end{table*}


\section{Experiments: Online Fair Ranking}  

Our experiments are focused on an \emph{online fair ranking setup}, similar to ~\cite{biega2018equity}. We assume a realistic setup where a new query arrives at each time $t$, and we re-rank the system-produced ranking at time $t$ to improve fairness. We assume knowledge of attention allocated to individuals in rankings till time $t$ to produce this new fair ranking (i.e., a running memory of cumulative attention per individual)\footnote{Code: \url{https://github.com/MLforHealth/DistFaiR}}.  


\subsection{Experimental Setup}
\paragraph{Datasets} We utilize two synthetic datasets which represent the setting described in the example shown in Figure~\ref{fig:fairwashing_amortized_ranking} where female individuals are allocated attention in four out of eight rankings (all with negative polarity) and two real-world fair ranking datasets~\cite{thawani2019online,trec-fair-ranking-2021}. A dataset summary is in Table~\ref{tab:ds_summary} and further details are provided in Appendix~\ref{sec:datasets}. We also benchmark the impact of query polarity on the Xing dataset~\cite{zehlike2017fa} in the Appendix (see Appendix~\ref{app:xing_dataset}).  Our empirical study focuses on post-processing fairness interventions, where individual relevance -- or ``groundtruth" -- scores are known~\cite{gorantla2023sampling}.


\paragraph{Query Properties} We experiment with polarity as the query property. The polarity score is synthetically generated for \texttt{synth-binary} and \texttt{synth-cont} and manually annotated for \texttt{rateMDs}. For the \texttt{FairTREC 2021} dataset, a pre-trained sentiment classification model is used to generate polarity~\cite{barbieri-etal-2020-tweeteval} (see Appendix~\ref{sec:datasets}). 

\paragraph{Distance Functions}
\label{sec:distance_metrics}
We consider three (pseudo) divergences metrics for measuring unfairness under DistFaiR:
\begin{itemize}[leftmargin=0.2in]
    \setlength\itemsep{0em}
    \item $\mathbf{L_1}$ distance is defined as the difference between the mean of two distributions: $D_{L_1}(A \| R) = |\mathbb{E}_{X \sim A}[X] - \mathbb{E}_{Y \sim R}[Y]|$. 
    \begin{itemize}
        \item This distance function has been studied in ~\cite{biega2018equity}, where fairness is computed as the sum of distance values across individuals and is referred to as the inequity of amortized attention (IAA).  We note that this function is generally not a proper divergence. However, for distributions $A$ and $R$ whose first moments are sufficient statistics, $D_{L_1}$ satisfies definition \ref{def:divergence}.
    \end{itemize}
    \item $\mathbf{L_2^{\text{var}}}$ distance is defined as the difference in mean and variance of two distributions\footnote{We use squared differences as we expect a square root of this to perform similarly.}: 
    \begin{align*} D_{L_2^{\text{var}}}(A \| R) &= (\mathbb{E}_{X \sim A}[X] - \mathbb{E}_{Y \sim R}[Y])^2 \\&+ (\sigma{}_{X \sim A}[X] - \sigma_{Y \sim R}[Y])^2.\end{align*}
    We note that $D_{L_2^{\text{var}}}$ benefits from $W_2$, a proper divergence, for two Gaussians, which has the properties for Theorem \ref{theorem:indiv_group}. %
    \item $\mathbf{W_{1}}$ distance is defined as the Wasserstein distance between the distribution of expected attention ($\{a_i^t\}_{t=1}^{\mathcal{T}}$) and distribution of expected relevance ($\{r_i^t\}_{t=1}^{\mathcal{T}}$) for an individual. $D_{W_1}(A \| R)=\frac{1}{T}\sum_{k=1}^{T} |a_i^{(k)} - r_i^{(k)}|$, where $(k)$ denotes the $k$th order statistic of empirical measures $\hat{A}_i$ and $\hat{R}_i$ from which $a_i^t$ and $r_i^t$ is sampled. 
    
\end{itemize}
\subsection{Evaluation}
\label{sec:metrics}
We utilized the following fairness criteria.



\paragraph{Individual Unfairness:} We use three different distance measures defined in Section~\ref{sec:distance_metrics} to measure the unfairness as:  DistFaiR($L_1$), DistFaiR($L_2^{\text{var}}$), and DistFaiR($W_1$).
The amortized fairness defined by DistFaiR($L_1$) is similar to the IAA fairness measure studied by \cite{biega2018equity}. However, we consider the \emph{worst-case} distance between attention and relevance distributions, while \cite{biega2018equity} consider the sum of difference across all individuals, which may hide heightened unfairness in some individuals. Our work also generalizes amortized fairness to include appropriate measurements of discrepancies between distributions that require higher-order moments to be specified, i.e., with $L_2^{\text{var}}$ and $W_1$ distances.



\paragraph{Group Unfairness:}
In addition to the group unfairness metrics directly induced by the three distance metrics using Definition~\ref{def:group-unfairness}, we consider a standard exposure-based group unfairness definitions: Exposed Utility Ratio (EUR). \cite{singh2018fairness,morik2020controlling} define the EUR difference as the absolute difference in the ratios of average exposure and average relevance between groups. We also measure an attention parity metric: Demographic Parity\cite{morik2020controlling} (DP). 


\paragraph{Performance}
\label{sec:perf}
We measure the ranking quality via the DCG@K score, which is the sum of the relevance of the top-K individuals, with a logarithmic discount based on their position: 
$$\sum_{k=1}^{K}\frac{r_{\text{rank}(k)}^t}{log_2(k+1)},$$
where ${\text{rank}(k)}$ returns the index of the individual at rank $k$. After re-ranking, the DCG@K is normalized by the DCG@K of the previous (ideal) ranking to produce a normalized DCG@K between 0 and 1. 


\subsection{Baselines: Fair Re-ranking }

\textbf{IAA}: A method to reduce inequity of amortized attention (IAA) introduced by Biega \emph{et al.}\cite{biega2018equity}. An ILP is solved to reduce the absolute difference in the mean of the cumulative attention and cumulative relevance distributions, summed across all individuals. In contrast, our method focuses on \emph{worst-case} minimization.

\textbf{FoE}: A linear program for ranking assignments with Birkhoff Von Neumann decomposition~\cite{lewandowski1986algorithmic} is solved to ensure fairness of exposure (FoE)~\cite{singh2018fairness}. The quality of rankings is maximized, with the constraint that the cumulative attention to relevance ratio is the same for all individuals. We re-rank only top-k individuals in each ranking. The original ranking is returned if solution is infeasible.\looseness=-1


\textbf{FIGR}~\cite{gorantla2021problem}: This method jointly aims to reduce ``underranking" (which is closely related to individual fairness) in rankings that are post-processed with group fairness constraints. Unlike the other baselines, this is a proportion-based re-ranker for each ranking, and does not explicitly consider attention distributions. Thus, we present results for this baseline in the Appendix. 

\vspace{-0.5em}
\subsection{Hyperparameter Tuning}
We stratified all datasets into two subsets: 50\% tuning and 50\% test sets, so no individuals or queries are present in both splits. All parameters (e.g., $\theta$; when tuned) are tuned using the tuning split. For \texttt{FairTREC 2021}, we use the full evaluation split, and do not perform any additional tuning -- we sample queries with replacement thrice to obtain variance. We run all optimization algorithms on a 3.2 GHz CPU with 16 GB RAM for $\leq$ 60 minutes, with a feasibility tolerance of $1e-9$. We set K=10 while measuring ranking quality and assume logarithmic discounts in attention till K=10 and zero otherwise.

We also pre-filter~\cite{biega2018equity}, and only re-rank the top-k individuals in each ranking. For moment-based divergences, $L_1$ and $L_2^{var}$,  we minimize maximum divergence only among the top-k at each step, as we found that this performs better. This means that even when the maximum divergence measure across all individuals cannot be reduced, we still re-rank to reduce the next possible highest divergence value. For $W_1$, we minimize divergence across all individuals. For the FoE baseline, constraints are set only for individuals in the top-k positions to make re-ranking feasible. Note that our results are sensitive to these pre-filtering choices. Post-tuning, we find that $k=50$ works well across datasets.

Our experimental flow is as follows: first, we implement our fair ranking definitions (DistFaiR) and compare to baselines. Second, we test if fairness metrics are affected by query polarity. Third, we perform several ablations for, e.g., the fairwashing effect.


\begin{figure*}[ht!]
\centering
\begin{subfigure}
    \centering
    \includegraphics[width=0.31\textwidth,trim={0 0.25cm 0 0},clip]{images/multi_dataset_individual_fairwashing.pdf}
\end{subfigure}%
\begin{subfigure}
    \centering
    \includegraphics[width=0.31\textwidth,trim={0 0.25cm 0 0},clip]{images/multi_dataset_group_fairwashing.pdf}
\end{subfigure}%
\begin{subfigure}
    \centering
    \includegraphics[width=0.31\textwidth,trim={0 0.25cm 0 0},clip]{images/tradeoffs.pdf}
\end{subfigure}
\caption{(a) and (b) show the difference (as relative change) between fairness metrics measured with and without query polarity. Query polarity impacts all amortized fairness metrics, as they differ from zero as seen in the plots. (rightmost) We plot the re-ranking performance of polarity agnostic and aware re-rankings under different permissible performance loss changes for the \texttt{synth-cont} dataset (DistFaiR($L_1$)), where we can see polarity agnostic re-ranking underperforms polarity aware re-ranking. %
}\label{fig:varying_theta}


\end{figure*}
\vspace{-0.5em}
\section{Results}

We measure the percentage change in unfairness pre- and post- re-ranking. A positive change -- decrease in unfairness -- is desired.

\paragraph{DistFaiR Improves Worst-Case  Fairness}
Table ~\ref{tab:full_results_main} shows that our re-rankings reduce individual unfairness, when unfairness is measured as the worst-case divergence measure between the attention and relevance distributions across individuals. We find that DistFaiR outperforms or performs on par with IAA. FIGR (Table~\ref{tab:figr})---which solves a different notion of ``underranking"---does not improve performance as measured by our metrics. Further, as expected, optimizing the divergence measure itself often leads to highest decrease in unfairness (for example, DistFaiR($W_1$) has highest improvement in fairness for the $\Delta$DistFaiR($W_1$) individual fairness measurement). Note that not all differences were statistically significant.

Additionally, as seen in Appendix Table~\ref{tab:group-re-ranking},  DistFaiR underperforms IAA-based re-ranking on the IAA metric. This makes sense because DistFaiR focuses on reducing worst-case divergence, while IAA focuses on the average across individuals. Thus, there appear to be tradeoffs between average and worst-case performance. Such observations have also been made in other fairness contexts~\cite{yang2023change}. %

\paragraph{Divergence Metric is an Important Design Choice.}
Our results show that the divergence metric is an important design choice. We find that the performance of $L_1$ and $L_2^{\text{var}}$ are close  (e.g., on \texttt{FairTREC2021}). We hypothesize that the  optimization with $W_1$ is more difficult, due to which performance improvements are smaller. Note that $L_2^{\text{var}}$ is the $W_2$ solution under assumptions of gaussianity. It is possible that using the $L_2^{\text{var}}$ measure could be an easier objective, but we can remove the distribution assumption for the general $W_2$.\looseness=-1 


\paragraph{Individual Fairness Not Always at Odds with Group Fairness.}
Reducing individual unfairness under DistFaiR also reduces group unfairness in most cases (Table~\ref{tab:full_results_main}), as averaged across test splits, even without imposing group-level constraints.  While group unfairness does increase in some cases, the degree of change cannot exceed a specific limit (upto individual unfairness) as per our theoretical findings.  
We also see similar trends on a standard group fairness metric,  EUR~\cite{morik2020controlling} (see Table~\ref{tab:group-re-ranking} in Appendix). Interestingly, the IAA baseline almost always improves group fairness, though  DistFaiR reduces unfairness to a higher degree on two datasets. We also observe higher variance for group unfairness, potentially due to multiple solutions with same individual but different group unfairness.\looseness=-1 







\paragraph{Online vs Offline Optimization.}
We observe that fully offline optimization reduces unfairness equally or more effectively than fully online optimization (Appendix Figure~\ref{fig:online_offline_optimization}). Thus, even if the full set of queries is not known apriori, partial offline optimization could be useful when a subset of queries is available. Experimentally, variance in online fairness is lower when optimizing for divergences beyond mean-based differences (\texttt{rateMDs} dataset; Figure~\ref{fig:online_fairness}) over time.\looseness=-1 


\paragraph{Fairness Metrics are Sensitive to Query Polarity.}
\label{sec:query_polarity}
In Figure~\ref{fig:varying_theta} (a) and (b), we compute the relative change between fairness metrics measured with and without query polarity, averaged across tuning splits. We observe that all fairness metrics, for both individual and group fairness, are sensitive to query polarity. When the relative change is positive, this indicates fairwashing: rankings seem more fair than they actually are.
We observe that fairwashing occurs, especially for group fairness metrics. Thus, if one relies on the query polarity agnostic metrics, conclusions regarding the (un)fairness of the rankings would be incorrect. It may also be important to consider divergence measures beyond difference in means to avoid systematic under-ranking of a specific group across queries.\looseness=-1 


\paragraph{Ranking Quality and Fairness Tradeoff.}
We study the variation in fairness across thresholds of allowable ranking quality loss ($\theta$) in the ILP optimization. Lower unfairness is observed at lower $\theta$ for the polarity-aware re-ranking  (Figure~\ref{fig:varying_theta} (c)), indicating a ranking quality and fairness tradeoff. Additionally, polarity agnostic re-ranking performance leads to higher (worse) unfairness than when query polarity is used. This matches our discussion that fairness metrics are sensitive to query polarity. Experimentally, higher standard deviation is observed in polarity-aware ranking. We also observe similar trends on the \texttt{rateMDs} dataset (Appendix Figure~\ref{fig:tradeoffs_ratemd}).

Importantly, in many real-world applications, different queries may have multiple differing real-world properties beyond polarity.
Accordingly, we can generalize our distribution-aware fairness definition to allow multiple query properties as a vector, where multiple properties form a multi-dimensional distribution. Initial results with this setup for the synthetic datasets are in the Appendix~\ref{sec:multiple_properties}.





 

\section{Discussion and Conclusions}
In mechanistic investigations, \textbf{position matters.} Our results suggest it does not make sense in practice to create circuits without considering how distinct the circuit at each position might be.
Theoretical results suggest that it also does not make sense in principle to ignore positionality: \citet{merrill2024expressivepowertransformerschain} show that transformers' expressive power increases with multiple generation steps.
Similarly, accounting for positionality in interpretability methods can enhance their expressive power by capturing the distinct mechanisms processing each token, rather than assuming a single pathway for the entire sequence.

Other interpretability methods such as distributed alignment search (DAS; \citealp{geiger2024das}) already support testing hypotheses about the position of particular causal variables. It would be interesting to directly compare the efficacy of DAS methods when separating results by position versus when aggregating information across positions. Stronger results when separating positional information could help generalize our conclusions to a wider array of causal interpretability methods.



\section*{Limitations}
A key limitation we have discussed is that it is not trivial to handle positional information in tasks where the length of inputs vary. We have proposed an automatic pipeline for generating \emph{and} applying schemas, but future work should explore this further. In particular, because there is no single gold standard for schemas, it is not clear \emph{a priori} what kinds of schemas are generally likely to obtain better trade-offs between faithfulness and circuit size. Devising general principles for effective schema design therefore represents a fruitful avenue for future work. It would also be interesting to observe whether human-generated schemas tend to satisfy these principles, or whether the most effective schemas are not necessarily those that humans are likely to design.

Another key limitation is that a schema requires the same spans to appear in the same order across all examples, such that the edges' direction remains correct across examples. Consequently, two schemas with the same span types but in different orders cannot be evaluated together, as these produce different abstract computation graphs. 

\section*{Acknowledgments}
 This research was supported by the Israel Science
Foundation (grant No.\ 448/20), an Azrieli Foundation
Early Career Faculty Fellowship,  an AI Alignment grant from Open Philanthropy, and a Google gift. HO is supported by the Apple AIML PhD fellowship. DB is supported by a grant from Open Philanthropy. AM is supported by a postdoctoral fellowship under the Zuckerman STEM Leadership Program.
This research was funded by the European Union (ERC, Control-LM, 101165402). Views and opinions expressed are however those of the author(s) only and do not necessarily reflect those of the European Union or the European Research Council Executive Agency. Neither the European Union nor the granting authority can be held responsible for them.



\bibliography{references}

\section{APPENDIX}
% Tarik: Also mention the evaluation procedure, how many runs are recorded? What is the executed action horizon? Are we doing any action chunking/blending? 
In appendix, we present the implementation details of CEM and policy training.
% \begin{figure*}[t]
% \centering
% \includegraphics[width=1.0\textwidth]{figures/allegro_franka_snapshots.png}
% 	\caption{Snapshots of trajectories generated from a single demonstration for Allegro hand and bimanual Panda arms.}
%     \label{fig:allegro_franka_snapshots}
% \end{figure*}
\subsection{CEM Implementation Details}
\label{sec:appendix_cem}
\begin{table}
\centering
        \renewcommand{\arraystretch}{0.8}
        \begin{threeparttable}
        \begin{tabular}{@{}lccccc@{}}
        \toprule
        Parameter & $T$ & Plan Duration & $q_o$ & $q_r$ & $r_u$ \\
        \midrule
        Floating Allegro Hand & 6 & 1.25 s & 10 & 0.01 & 0.1 \\
        Bimanual iiwa Arms & 6 & 1.25 s & 10 & 0.01 & 10 \\
        Bimanual Panda Arms & 6 & 2.0 s & 10 & 0.01 & 10 \\
        \bottomrule
        \end{tabular}
        \end{threeparttable}
        \caption{\textbf{Parameters for CEM. } $T$: planning horizon. $q_o$: scalar weight for tracking object trajectories. $q_r$: scalar weight for tracking robot trajectories. $r_u$: scalar weight for control input.}
        \label{tab:cem_params}
\end{table}
We provide detailed parameters for the CEM implementation in Table \ref{tab:cem_params}. We optimize over the action knot points $u_{0:T-1}$, which are linearly interpolated to generate action commands sent to Drake. Drake simulates the contact dynamics $f$ at 200 Hz. The state cost matrix  $Q_t = diag(q_o \cdot \mathbf{1}_{n_o}, q_r \cdot \mathbf{1}_{n_r})$, where $n_o$ and $n_r$ denote the object and robot state dimensions, and $\mathbf{1}$ is a vector or all 1's. The terminal state cost matrix $Q_T = 10 \cdot Q_t$. The input cost matrix $R_t = diag(r_u \cdot \mathbf{1}_{n_u})$, where $n_u$ represents the control input dimension. All of the systems adopt 50 samples, 5 elites and initial standard deviation $\sigma = 0.05 \cdot \mathbf{1}_{n_u}$ for action sampling.
\subsection{Policy Implementation Details}
We train UNet-based diffusion policies \cite{chi2023diffusion} for all tasks. The action space is the robot configuration (joint angles, and additional floating base coordinates for the Allegro hand), while the observation space is the robot configuration and object pose (with orientations represented by rotation matrices). Detailed parameters are listed in Table \ref{tab:diffo_po_params}.

\begin{table}
\centering
        \renewcommand{\arraystretch}{0.8}
        \begin{threeparttable}
        \begin{tabular}{@{}lcccccc@{}}
        \toprule
        Parameter & $T_o$ & $T_a$ & Freq & Epochs & Obs. Dim. & Act. Dim. \\
        \midrule
        Floating Allegro Hand & 10 & 40 & 50 & 1000 & 34 & 22 \\
        Bimanual iiwa Arms & 10 & 40 & 20 & 800 & 26 & 14\\
        Bimanual Panda Arms & 10 & 40 & 50 & 800 & 26 & 14\\
        \bottomrule
        \end{tabular}
        \end{threeparttable}
        \caption{\textbf{Parameters for diffusion policies. } $T_o$: observation horizon. $T_a$: action horizon. Freq: environment frequency (Hz, both observations and actions).}
        \label{tab:diffo_po_params}
\end{table}




\end{document}

