
\section{Discussion and Conclusions}
In mechanistic investigations, \textbf{position matters.} Our results suggest it does not make sense in practice to create circuits without considering how distinct the circuit at each position might be.
Theoretical results suggest that it also does not make sense in principle to ignore positionality: \citet{merrill2024expressivepowertransformerschain} show that transformers' expressive power increases with multiple generation steps.
Similarly, accounting for positionality in interpretability methods can enhance their expressive power by capturing the distinct mechanisms processing each token, rather than assuming a single pathway for the entire sequence.

Other interpretability methods such as distributed alignment search (DAS; \citealp{geiger2024das}) already support testing hypotheses about the position of particular causal variables. It would be interesting to directly compare the efficacy of DAS methods when separating results by position versus when aggregating information across positions. Stronger results when separating positional information could help generalize our conclusions to a wider array of causal interpretability methods.



\section*{Limitations}
A key limitation we have discussed is that it is not trivial to handle positional information in tasks where the length of inputs vary. We have proposed an automatic pipeline for generating \emph{and} applying schemas, but future work should explore this further. In particular, because there is no single gold standard for schemas, it is not clear \emph{a priori} what kinds of schemas are generally likely to obtain better trade-offs between faithfulness and circuit size. Devising general principles for effective schema design therefore represents a fruitful avenue for future work. It would also be interesting to observe whether human-generated schemas tend to satisfy these principles, or whether the most effective schemas are not necessarily those that humans are likely to design.

Another key limitation is that a schema requires the same spans to appear in the same order across all examples, such that the edges' direction remains correct across examples. Consequently, two schemas with the same span types but in different orders cannot be evaluated together, as these produce different abstract computation graphs. 

\section*{Acknowledgments}
 This research was supported by the Israel Science
Foundation (grant No.\ 448/20), an Azrieli Foundation
Early Career Faculty Fellowship,  an AI Alignment grant from Open Philanthropy, and a Google gift. HO is supported by the Apple AIML PhD fellowship. DB is supported by a grant from Open Philanthropy. AM is supported by a postdoctoral fellowship under the Zuckerman STEM Leadership Program.
This research was funded by the European Union (ERC, Control-LM, 101165402). Views and opinions expressed are however those of the author(s) only and do not necessarily reflect those of the European Union or the European Research Council Executive Agency. Neither the European Union nor the granting authority can be held responsible for them.
