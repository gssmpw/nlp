\section{Conclusion}
\label{sec:conclusion}

\lavid is a novel agentic framework that leverages LVLMs' strong reasoning ability to detect diffusion-generated video. 
As opposed to existing methods that require supervised training detectors with explicit knowledge (EK), \lavid is training-free and can generalize to videos generated from different sources of video generation tools.
With our proposed EK selection method based on a tool-preference metric, \lavid can effectively extract useful EK for LVLMs to do the detection. We further propose an online adaptation (OA) method for structured prompts based on a rewriting template mechanism. Our proposed OA process largely reduces the hallucination issue in non-structured prompts and prevents LVLMs from overfitting with a specific template. The evaluation demonstrates that \lavid improves F1 scores by 6.2\% to 30.2\% over the top baseline on a high-quality video dataset across four leading LVLMs. Our work offers fresh perspectives on video detection by employing an agentic LVLM framework with emerging techniques.

% While there is room for further refinement in prompt engineering and toolkit preparation, we believe \lavid is an extensible framework well-suited for complex multimodal tasks such as AI-generated video and Deepfake detection.

% In our work, we first explore the capability of LVLMs in detecting diffusion-generated videos. We then examine the enhancements provided by external explicit knowledge and the impact of structured prompts on visual interpretability. Building on these insights, we propose \lavid, a framework that enables LVLMs to call external tools to extract explicit knowledge, enhancing its own video detection tasks. Our experiments show that \lavid improves F1 scores by 6.2\% to 30.2\% over the top baseline on a high-quality video dataset across three leading LVLMs. While there is room for further refinement in prompt engineering and toolkit preparation, we believe \lavid is an extensible framework well-suited for complex multimodal tasks such as AI-generated video and Deepfake detection.
