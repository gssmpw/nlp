\documentclass[a4paper,draft]{amsart}
\usepackage[a4paper,margin=25mm]{geometry}
\usepackage{amsmath}
\usepackage{amssymb,tikz,url}

\parindent=0pt
\parskip\smallskipamount

\let\set\mathbb
\def\<#1>{\langle#1\rangle}
\def\i{\mathrm{i}}
\def\e{\mathrm{e}}
\def\id{\operatorname{id}}
\def\ord{\operatorname{ord}}
\def\lc{\operatorname{lc}}
\def\Wr{\operatorname{Wr}}


\usepackage{mathtools}
\mathtoolsset{showonlyrefs, showmanualtags}
\newcommand{\pfrac}[2]{\frac{\partial #1}{\partial #2}}
\newcommand{\dd}{\mathrm{d}}

\newtheorem{thm}{Theorem}
\newtheorem{prop}[thm]{Proposition}
\newtheorem{lem}[thm]{Lemma}
\newtheorem{defi}[thm]{Definition}
\newtheorem{conj}[thm]{Conjecture}
\newtheorem{ex}[thm]{Example}

\usepackage[textwidth=2cm]{todonotes}

\begin{document}

\author[Manuel Kauers]{Manuel Kauers}
\address{Manuel Kauers, Institute for Algebra, J. Kepler University Linz, Austria}
\email{manuel.kauers@jku.at}

\author[Christoph Koutschan]{Christoph Koutschan}
\address{Christoph Koutschan, RICAM, Austrian Academy of Sciences, Linz, Austria}
\email{christoph.koutschan@ricam.oeaw.ac.at}

\author[Thibaut Verron]{Thibaut Verron}
\address{Thibaut Verron, Institute for Algebra, J. Kepler University Linz, Austria}
\email{thibaut.verron@jku.at}

\thanks{M.\ Kauers was supported by the Austrian FWF grants 10.55776/PAT8258123, 10.55776/PAT9952223,
  and 10.55776/I6130.
  C.\ Koutschan was supported by the Austrian FWF grant 10.55776/I6130.
T.\ Verron was supported by the Austrian FWF grant P34872.}  

\title{A Shape Lemma for Ideals of Differential Operators}

\begin{abstract}
  We propose a version of the classical shape lemma for zero-dimensional
  ideals of a commutative multivariate polynomial ring to
  the noncommutative setting of zero-dimensional ideals in an algebra of
  differential operators.
\end{abstract}

\maketitle

\section{Introduction}

In the classical theory of Gr\"obner bases for commutative polynomial
rings~\cite{buchberger65,cox92,becker93,cox05,buchberger10},
the shape lemma makes a statement about the form of the Gr\"obner basis with
respect to a lexicographic term order of an ideal of dimension zero.
It was proposed by Gianni and Mora~\cite{gianni89}, and it is almost obvious.

Consider an ideal $I\subseteq K[x,y]$ in a commutative polynomial ring over
a perfect field~$K$. The ideal has dimension zero if and only if the corresponding
algebraic set
\[
V(I)=\{\,(\xi,\eta)\in \bar K^2\mid\forall\ p\in I: p(\xi,\eta)=0\,\}
\]
is finite. Here, $\bar K$ denotes the algebraic closure of~$K$.

The finitely many points in $V(I)$ have only finitely many distinct $x$-coordinates,
and if $p$ is a generator of the elimination ideal $I\cap K[x]$, then the roots of
$p$ are precisely these $x$-coordinates.
The shape lemma says that usually there is another polynomial $q\in K[x]$ with
$\deg(q)<\deg(p)$ such that $I$ is generated by $\{y-q,p\}$.
This $q$ is then the interpolating polynomial of the points in~$V(I)$.

There may be no ideal basis of the required form if $V(I)$ contains two distinct
points with the same $x$-coordinate. The ideal is said to be \emph{in normal position}
(w.r.t.~$x$) if this is not the case, i.e., if any two distinct elements of $V(I)$
have distinct $x$-coordinates. If $K$ is sufficiently large, then every ideal $I$
of dimension zero can be brought into normal position by applying a change of variables.

\begin{thm}\label{thm:1} (cf. Prop. 3.7.22 in~\cite{kreuzer00}). Let $P=K[x_1,\dots,x_n]$,
  let $I\subseteq P$ be an ideal of dimension zero,
  let $t=\dim_K P/I$, and suppose that $|K|>\binom t2$. Then
  there are constants $c_1,\dots,c_{n-1}\in K$ such that mapping $x_n$ to
  $x_n-c_1x_1-c_2x_2-\cdots-c_{n-1}x_{n-1}$ (and $x_i$ to $x_i$ for every $i<n$)
  transforms $I$ into an ideal in normal position w.r.t.~$x_n$.
\end{thm}

A basis of the required form may also fail to exist if $I$ is not a radical
ideal. Recall that for a radical ideal~$I$, we have $\dim_K K[x,y]/I=|V(I)|$. Also
recall that if $p$ is a generator of $I\cap K[x]$, then
$[1],[x],\dots,[x^{\deg p-1}]$ are linearly independent over $K$
and $[1],[x],\dots,[x^{\deg p-1}],[x^{\deg p}]$ are linearly dependent.
Therefore, the following result is quite natural.

\begin{thm}\label{thm:2} (cf. Thm. 3.7.23 in~\cite{kreuzer00}). Let $P=K[x_1,\dots,x_n]$
  and let $I\subseteq P$ be a radical ideal of dimension zero.
  Let $p$ be a generator of $I\cap K[x_n]$. Then
  the following conditions are equivalent:
  \begin{enumerate}
  \item $I$ is in normal position w.r.t. $x_n$
  \item $\deg p=\dim_K P/I$
  \item $K[x_n]/\<p>$ and $P/I$ are isomorphic as $K$-algebras.
  \end{enumerate}
\end{thm}

Finally, the shape lemma can be stated as follows.

\begin{thm}\label{thm:3} (Shape Lemma; cf. Thm. 3.7.25 in~\cite{kreuzer00}) Let $P=K[x_1,\dots,x_n]$
  and let $I\subseteq P$ be a radical ideal of dimension zero that is in normal position
  w.r.t.~$x_n$. Let $p$ be a generator of $I\cap K[x_n]$. Then
  there are polynomials $q_1,\dots,q_{n-1}\in K[x_n]$ with $\deg(q_i)<\deg(p)$
  for all $i$ such that
  $\{x_1-q_1,\dots,x_{n-1}-q_{n-1}, p\}$ is a basis of~$I$.
\end{thm}

Here and elsewhere, by a ``basis'' of an ideal we understand just a set of generators, not
necessarily minimal or independent in any sense.

The purpose of this note is to extend these well-known facts from commutative polynomial
rings to rings of differential operators. This is motivated by recent developments in the
area of symbolic integration for so-called D-finite functions~\cite{kauers23c}. Given such a
function $f(x,y)$, the goal is to evaluate a definite integral
\[
  F(x) = \int_\Omega f(x,y)\,dy.
\]
More precisely, given an ideal of annihilating operators for~$f(x,y)$, we want to compute
an ideal of annihilating operators for the integral~$F(x)$. A general approach to this
problem is known as creative telescoping~\cite{zeilberger90,zeilberger91,koutschan13,chyzak14}
% TODO add chen25 once it is published
and has been subject of intensive research
during the past decades. There are several algorithms for creative telescoping, some of
which assume that the ideal of operators for $f(x,y)$ has a basis of the form $\{D_y-M,L\}$,
where $M$ and $L$ are operators in $D_x$ only. Thanks to the shape lemma, this is a fair
assumption.

\section{Differential operators}

Without loss of generality, and for sake of simplicity, we will
restrict the presentation of our main results to the bivariate
case. The role of the field~$K$ in the commutative setting sketched
in the introduction is now taken over by the field $C(x,y)$
of rational functions in~$x$ and~$y$, with coefficients in some
constant field~$C$ that we assume to have characteristic zero.
Hence from now on, we write $K=C(x,y)$.

We use the symbols $D_x$ and $D_y$ to denote the partial derivation
operators, i.e., $D_x(f)=\frac{\partial f}{\partial x}$ and
$D_y(f)=\frac{\partial f}{\partial y}$. Note that $D_x(c)=D_y(c)=0$
for all $c\in C$. Let $K[D_x,D_y]$ denote the ring of linear
differential operators with rational function coefficients, i.e.,
\[
  K[D_x,D_y] = \biggl\{ \sum_{i=0}^r\sum_{j=0}^s a_{i,j}(x,y) D_x^i D_y^j
  \mathrel{\bigg|} r,s\in\set N, a_{i,j} \in K \biggr\}.
\]
Because of the product rule, we have the commutation rules
$D_x\cdot x=x\cdot D_x+1$ and $D_y\cdot y=y\cdot D_y+1$, so the
ring $K[D_x,D_y]$ is non-commutative. A linear partial differential
equation can then be written as $L(f)=0$ with $L\in K[D_x,D_y]$.

Let $C[[x]]$ and $C[[x,y]]$ denote, as usual, the rings of univariate
and bivariate formal power series with coefficients in~$C$, and let $C((x))$ and
$C((x,y))$ denote their respective quotient fields. Let
$L=\sum_{i=0}^r a_i(x)D_x^i \in C(x)[D_x]$ be a linear ordinary differential
operator. An element $x_0\in C$ is called a regular point (or ordinary point)
of~$L$ if $a_r(x_0)\neq0$ and $a_i(x_0)$ is defined for all $0\leq i\leq r$, i.e., if no
coefficient~$a_i$ has a pole at~$x_0$. Via the change of variables
$x\mapsto x-x_0$ the point~$x_0$ can be moved to the origin. Hence, without
loss of generality, assume that $0$ is a regular point of~$L$. Then the set of
power series solutions
\[
  V(L)=\{ f \in C[[x]] \mid L(f)=0 \}
\]
forms a $C$-vector space of dimension~$r$.

For a power series~$f(x,y)\in C[[x,y]]$, we define the ($K[D_x,D_y]$-)
annihilator of~$f$ as the set of all operators that annihilate~$f$,
that is $\{L\in K[D_x,D_y] \mid L(f)=0\}$. It is easily verified
that this set forms a (left) ideal in $K[D_x,D_y]$. The series~$f$ is
called D-finite if $\dim_K(K[D_x,D_y]/I)<\infty$, where $I$ denotes
the annihilator of~$f$. Equivalently, $f$ is called D-finite if $I$
is an ideal of dimension zero.

Also in the multivariate setting we can make
a similar statement about the dimension of the solution space, which
directly follows from Thm.~3.7 in~\cite{chen18}.

\begin{thm}\label{thm:solspace}
  Let $I$ be a zero-dimensional left ideal of $K[D_x,D_y]$ 
  and $r=\dim_K(K[D_x,D_y]/I)\in\set N$.
  If $(0,0)$ is an ordinary point of~$I$, then the set
  \[
  V(I)=\{f\in C[[x,y]]\mid\forall L\in I:L(f)=0\}
  \]
  is a $C$-vector space of dimension~$r$.
\end{thm}

The definition of ordinary points proposed in \cite{chen18} is a bit more
complicated than the definition in the univariate case. We won't need it
here, so we do not reproduce it. It suffices to know that almost every
point is ordinary, so if $(0,0)$ is not a ordinary point, we always have
the option to get into the situation of Thm.~\ref{thm:solspace} by making
a change of variables.

For $f_1,\dots,f_r\in C((x,y))$, their
\emph{Wronskian} (with respect to the variable~$x$) is denoted and
defined as follows:
\[
  \Wr_x(f_1,\dots,f_r) = \det
  \begin{pmatrix}
    f_1 & f_2 & \cdots & f_r \\
    D_x(f_1) & D_x(f_2) & \cdots & D_x(f_r) \\
    \vdots & \vdots & \ddots & \vdots \\
    D_x^{r-1}(f_1) & D_x^{r-1}(f_2) & \cdots & D_x^{r-1}(f_r)
  \end{pmatrix}.
\]
The Wronskian $\Wr_x(f_1,\dots,f_r)$ is equal to zero if and only
if the $f_i$ satisfy a linear relation with coefficients that do
not depend on~$x$, e.g., if $\sum_{i=1}^r a_i f_i = 0$ with
$a_i\in C((y))$ not all zero~\cite{bocher1901}.

For later use, we state the following lemma.

\begin{lem}\label{lem:elim}
  If $L$ is an extension field of $K$ and $I$ is an ideal in $L[D_x,D_y]$
  which has a basis in $K[D_x,D_y]$, then also the elimination ideal
  $I \cap L[D_x]$ has a basis in~$K[D_x]$.
\end{lem}
\begin{proof}
  Let $P_1,\dots,P_n\in K[D_x,D_y]$ be a basis of~$I$, and let $M$
  be an element in the elimination ideal $I\cap K[D_x]$. Then there
  exist $Q_1,\dots,Q_n\in L[D_x,D_y]$ such that $M=Q_1P_1+\dots+Q_nP_n$.

  Clearly, $L$ can be viewed as a $K$-vector space, of potentially
  infinite dimension. In any case, there exists a finite-dimensional
  $K$-subspace~$V$ of $L$ that contains all the coefficients of the~$Q_i$
  (note that each $Q_i$ has only finitely many coefficients in~$L$).
  Now let $B_1,\dots,B_d$ be a $K$-basis of~$V$, which means that
  there are $Q_{i,j}\in K[D_x,D_y]$ such that
  $Q_i=Q_{i,1}B_1+\dots+Q_{i,d}B_d$ for all~$i$. Hence we can write
  \begin{equation}\label{eq:Qij}
    M = \sum_{i=1}^n\biggl(\sum_{j=1}^d Q_{i,j}B_j\biggr)P_i
      = \sum_{j=1}^d\biggl(\sum_{i=1}^n Q_{i,j}P_i\biggr)B_j.
  \end{equation}

  Since the $B_j$ are linearly independent over~$K$, it follows that for
  each~$j$, the quantity $\sum_{i=1}^n Q_{i,j}P_i$ is free of~$D_y$, because
  $M$ is free of $D_y$ and because there cannot be a cancellation on the
  right-hand side of~\eqref{eq:Qij}. Therefore, the coefficients
  $\sum_{i=1}^n Q_{i,j}P_i$ are in $K[D_x]$, which proves the claim.
\end{proof}

For readers familiar with the theory of Gr\"obner bases, we offer the
following alternative proof: from a given basis of $I$ with elements
in $K[D_x,D_y]$, we obtain a basis of $I\cap L[D_x]$ by computing a
Gr\"obner basis with respect to an elimination order. Since Buchberger's
algorithm never extends the ground field, the resulting basis must
be a subset of~$K[D_x]$.

\section{The Shape Lemma}

For an ideal $I\subseteq K[D_x,D_y]$ of dimension zero, consider the
quotient $K[D_x,D_y]/I$ as a $K[D_x]$-module. Since its dimension
as $K$-vector space is finite, this module must be cyclic~\cite[Prop. 2.9]{put03}.
If $M\in K[D_x,D_y]$ is such that $[M]$ is a generator of the module,
then there is an $L\in K[D_x]$ such that $L\cdot [M]=[LM]=[1]$.
Therefore, evaluating an integral
\[
  F(x)=\int_\Omega f(x,y)\,dy
\]
for a function $f(x,y)$ whose ideal of annihilating operators is~$I$ is the
same as evaluating the integral
\[
  F(x)=\int_\Omega L\cdot g(x,y)\,dy
\]
where $g(x,y)$ is defined as $M\cdot f(x,y)$. The choice of $M$ implies %(WHY?)
that the annihilating ideal $J$ of $g(x,y)$ has a basis of the form $\{D_y - Q, P\}$
for two operators $P,Q$ in $K[D_x]$.

Transforming $I$ to $J$ is known as gauge transform and can be considered as a
satisfactory solution to our problem: every ideal $I\subseteq K[D_x,D_y]$ of dimension
zero can brought to an ideal $J$ to which the shape lemma applies by means of a
gauge transform. 

We shall propose an alternative approach here. Rather than applying a gauge transform,
which amounts to applying an operator to the integrand, our question is whether we
can also obtain an ideal basis of the required form by applying a linear change of
variables, i.e., using
\[
  F(x) = \int_\Omega f(x,y)\,dy = \int_{\tilde\Omega} f(x,y+cx)\,dy
\]
for some constant~$c$ (and an appropriately adjusted integration range). It turns out
that this perspective leads to a shape lemma for differential operators that matches
more closely the situation in the commutative case. 

Note that $L(x,y,D_x,D_y)\in K[D_x,D_y]$ is an annihilating operator of $f(x,y+cx)$ if and only if
$L(x,y-cx,D_x+cD_y,D_y)$ is an annihilating operator of~$f(x,y)$. In particular, the ideal
of annihilating operators of $f(x,y)$ has dimension zero if and only this is the case for
the ideal of annihilating operators of $f(x,y+cx)$.

We shall show (Thm.~\ref{thm:changofvariables} below) that every
zero-dimensional left ideal of $K[D_x,D_y]$ can be brought to normal position
by a change of variables $y\leftarrow y+cx$. For the notion of being in normal
position, we propose the following definition.

\begin{defi}\label{def:normal}
  Let $I\subseteq K[D_x,D_y]$ be an ideal of dimension zero,
  so that $r=\dim_K K[D_x,D_y]/I$ is finite.
  The ideal $I$ is called \emph{in normal position} (w.r.t.~$D_x$) if for every
  choice of $C$-linearly independent solutions $f_1,\dots,f_r$
  we have $\Wr_x(f_1,\dots,f_r)\neq0$.
\end{defi}

\begin{ex}
  For the ideal $I=\<(D_x-1)(D_x-2),D_y>$ we have $r=2$. The
  solution space of $I$ is generated by $\exp(x)$ and~$\exp(2x)$.
  We have $\Wr_x(\exp(x),\exp(2x))=\exp(3x)$. Therefore, $I$ is
  in normal position w.r.t. $D_x$.
  However, as $D_y(\exp(x))=D_y(\exp(2x))=0$, we also
  have $\Wr_y(\exp(x),\exp(2x))=0$, so $I$ is \underline{not} in
  normal position w.r.t.~$D_y$.
\end{ex}

With this notion of being in normal position, we can state the following result. 

\begin{thm} (Shape Lemma; differential analog of Thms.~\ref{thm:2} and~\ref{thm:3})\label{thm:our23}
  Let $I\subseteq K[D_x,D_y]$ be an ideal of dimension zero.
  Let $P$ be a generator of $I\cap K[D_x]$. Then the following conditions are equivalent:
  \begin{enumerate}
  \item $I$ is in normal position w.r.t. $D_x$
  \item $\ord(P)=\dim_K K[D_x,D_y]/I$
  \item $K[D_x]/\<P>$ and $K[D_x,D_y]/I$ are isomorphic as $K[D_x]$-modules.
  \item There is a $Q\in K[D_x]$ with $\ord(Q)<\ord(P)$ such that $\{D_y-Q,P\}$ is a basis of~$I$.
  \end{enumerate}
\end{thm}
\begin{proof}
  Let $r=\dim_K K[D_x,D_y]/I$.

  $1\Rightarrow 2$\quad
  To show that $\ord(P)=r$, suppose that $\ord(P)<r$ and let $f_1,\dots,f_r$ be
  some $C$-linearly independent solutions of~$I$. By Thm.~\ref{thm:solspace},
  we may assume that such solutions exist. 
  As no more than $\ord(P)$ solutions of $P$ can be linearly independent over
  $C[[y]]$, it follows that $f_1,\dots,f_r$ are linearly dependent over~$C[[y]]$.
  This implies $\Wr_x(f_1,\dots,f_r)=0$, in contradiction to the assumption that
  $I$ is in normal position.

  $2\Rightarrow 1$\quad
  Let $f_1,\dots,f_r$ be $C$-linearly independent solutions of~$I$.
  We have to show that they are also linearly independent over~$C((y))$.
  Suppose otherwise. Then we may assume that $f_r$ is a $C((y))$-linear combination
  of $f_1,\dots,f_{r-1}$. The operator
  \[
  Q = \begin{vmatrix}
    1 & f_1 & \cdots & f_{r-1} \\
    D_x & f_1' & \cdots & f_{r-1}' \\
    \vdots & \vdots & & \vdots \\
    D_x^{r-1} & f_1^{(r-1)} & \cdots & f_{r-1}^{(r-1)}
    \end{vmatrix} \in C((x,y))[D_x]
  \]
  has the solutions $f_1,\dots,f_{r-1}$ and~$f_r$.
  It must therefore belong to the ideal generated by $I$ in the larger ring $C((x,y))[D_x,D_y]$,
  for if it didn't, then $\dim_{C((x,y))} C((x,y))[D_x,D_y]/(\<I>+\<Q>)<r$, which
  is impossible when the solution space has $C$-dimension~$r$.

  By Lemma~\ref{lem:elim}, $P$~is also a generator of the elimination ideal
  $\<I>\cap C((x,y))[D_x]$, where $\<I>$ denotes the ideal generated by $I$
  in $C((x,y))[D_x,D_y]$. By assumption we have $\ord(P)=r>\ord(Q)$. This is
  a contradiction. 
  
  $2\Rightarrow 3$\quad
  Consider the function $\phi\colon K[D_x]/\<P>\to K[D_x,D_y]/I$ defined by
  $\phi([L]_{\<P>}):=[L]_I$.
  This function is well-defined because $\<P>\subseteq I$.
  The function is obviously a morphism of $K[D_x]$-modules, and it is injective,
  because if $L\in K[D_x]$ is such that $[L]_I=[0]_I$, then $L\in I$,
  so $L\in I\cap K[D_x]=\<P>$, so $[L]_{\<P>}=0$.
  Being a morphism of $K[D_x]$-modules, $\phi$ is in particular a morphism of $K$-vector spaces.
  Therefore, since $\dim_K K[D_x]/\<P>=r=\dim_K K[D_x,D_y]/I$ by assumption,
  injectivity implies bijectivity, and therefore $\phi$ is an isomorphism.

  $3\Rightarrow 2$\quad clear.

  $2\Rightarrow 4$\quad
  By assumption, the elements $[1],[D_x],\dots,[D_x^{r-1}]$ of $K[D_x,D_y]/I$ are $K$-linearly
  independent and therefore form a vector space basis of $K[D_x,D_y]/I$.
  Therefore, the element $[D_y]$ of $K[D_x,D_y]/I$ can be expressed as a
  $K$-linear combination of $[1],[D_x],\dots,[D_x^{r-1}]$. This implies
  the existence of a~$Q$.

  $4\Rightarrow 2$\quad 
  By repeated addition of suitable multiples of basis elements, it can be seen
  that every element of $K[D_x,D_y]$ is equivalent modulo $I$ to an element
  of the form $q_0+q_1D_x+\cdots+q_{r-1}D_x^{r-1}$. Therefore, the elements $[1],\dots,[D_x^{r-1}]$
  generate $K[D_x,D_y]/I$ as a $K$-vector space. This implies
  $\dim_K K[D_x,D_y]/I\leq r$.
  At the same time, the dimension cannot be smaller than~$r$, because if
  $[1],\dots,[D_x^{r-1}]$ were $K$-linearly dependent, then $I\cap K[D_x]$ would
  contain an element of order less than $\ord(P)$, which is impossible by the choice of~$P$.
\end{proof}

Again, readers familiar with the theory of Gr\"obner bases will have no difficulty
finding shorter arguments for some of the implications. 

The similarity of Thm.~\ref{thm:our23} to the corresponding theorems for
commutative polynomial rings is evident, but there are some subtle differences
as well. One difference is that Thms.~\ref{thm:2} and~\ref{thm:3} require the ideal
to be radical, while no such assumption is needed for Thm.~\ref{thm:our23}.

However, it turns out that in order to also generalize Thm.~\ref{thm:1} to
differential operators, we do need to introduce a restriction. Note that
Thm.~\ref{thm:1} becomes wrong for non-radical ideals if we interpret their
solutions as points with multiplicities.  Indeed, in this sense, a non-radical
ideal is never in normal position, and no linear change of variables will
suffice to turn a non-radical ideal into a radical ideal.

Ideals of differential operators cannot have multiple solutions (cf.~Thm.~\ref{thm:solspace}).
Instead, it seems appropriate to adopt the following definition.

\begin{defi}\label{def:radical}
  An ideal $I\subseteq K[D_x,D_y]$ of dimension zero is called \emph{radical}
  if $\dim_K K\otimes_C V(I)=\dim_C V(I)$, i.e., if any $C$-vector space
  basis of the solution space $V(I)$ is $K$-linearly independent.
\end{defi}

\begin{ex}
  \begin{enumerate}
  \item The ideal $\<(D_x-1)^2,D_y>$ has the $C$-linearly independent
    solutions $\exp(x)$ and $x\exp(x)$.
    As these are not linearly independent over~$K$, the ideal is not radical.
  \item The solution space of the ideal $\<(D_x-1)(D_x-2),D_y>$ has the basis $\{\exp(x),\exp(2x)\}$,
    as $\exp(x)$ and $\exp(2x)$ are linearly independent over $K=C(x)$, the ideal is radical.
  \end{enumerate}
\end{ex}

Observe the difference between Defs.~\ref{def:radical} and~\ref{def:normal}.
In both cases we require the absence of linear relations, but with respect to
different coefficient domains. For normal position, the coefficients must be
free of $x$ but can depend in an arbitrary way on~$y$, and for being radical,
the coefficients must be rational functions in $x$ and~$y$.

\begin{thm} (Differential analog of Thm.~\ref{thm:1})\label{thm:changofvariables}
  Let $I\subseteq K[D_x,D_y]$ be radical and of dimension zero.
  Then there is a constant $c\in C$ such that the ideal $J$ obtained
  from $I$ by applying the linear change of variables $y\leftarrow y+cx$
  is in normal position w.r.t.~$D_x$.
\end{thm}
\begin{proof}
  We show that whenever $f_1(x,y),\dots,f_r(x,y)$ are such that 
  $\Wr_x\bigl(f_i(x,y+cx)\bigr)_{i=1}^r=0$ for all $c\in C$,
  then $f_1,\dots,f_r$ are $K$-linearly dependent.

  Consider $c$ as an additional variable and recall that the
  assumption $\Wr_x\bigl(f_i(x,y+cx)\bigr)_{i=1}^r=0$
  implies that the $f_i(x,y+cx)$ are linearly dependent over the constant
  field with respect to~$x$, i.e., $C((y,c))$-linearly dependent: thus we can
  assume that there exist $p_1,\dots,p_r\in C((y,c))$, not all $0$, such that
  \begin{equation}\label{eq:lincomb_fi}
    \sum_{i=1}^r p_i(y,c)\cdot f_i(x,y+cx) = 0.
  \end{equation}
  Each $f_{i}$ has an expansion as a series in $x$:
  \[
    f_i(x,y+cx) = \sum_{j=0}^\infty \ \underbrace{\frac{1}{j!}
      \frac{\partial^j f_i(x,y+cx)}{\partial x^j}\bigg|_{x=0}}_%
    {\textstyle =: f_{i,j}(y,c)}
    \cdot\ x^j.
  \]
  Note that the series coefficients~$f_{i,j}$ are polynomials in~$c$:
  \[
    \frac{\partial^j f_i(x,y+cx)}{\partial x^j} =
    \sum_{k=0}^j \binom{j}{k} \cdot f_i^{(j-k,k)}(x,y+cx)\cdot c^k
    \in C((x,y))[c],
  \]
  and therefore $f_{i,j}(y,c)\in C((y))[c]$. It follows that Eq.~\eqref{eq:lincomb_fi}
  can be expanded as 
  \[
    \sum_{j=0}^\infty \left(
      \sum_{i=1}^r p_i(y,c) \cdot f_{i,j}(y,c) \right) \cdot x^j = 0
  \]
  and therefore, for all $j \in \set N$, $\sum_{i=1}^r p_i(y,c) \cdot f_{i,j}(y,c)=0$.

  Let $M$ be the matrix $\bigl(f_{i,j}(y,c)\bigr)_{j\geq 0,1\leq i\leq r}$
  with infinitely many rows and $r$ columns.
  From the above, 
  \[
    \bigl(p_i(y,c)\bigr)_{i=1}^r \in \ker M,
  \]
  and therefore $M$ is rank-deficient; let $R<r$ denote the rank of~$M$. Hence
  there exists an integer $m\in\set{N}$ such that the rank of the
  $(m\times r)$-submatrix~$M'$, that is obtained by taking the first $m$ rows
  of~$M$, is also equal to~$R$. It follows that $\ker(M')=\ker(M)$, and since
  $M'\in C((y))[c]^{m\times r}$ we have that $\ker(M')$ is a subspace of
  $C((y))(c)^r$. Therefore, the coefficients $p_i(y,c)$ can be chosen in
  $C((y))(c)$.

  Now perform the substitution $c\to c-y/x$ in~\eqref{eq:lincomb_fi} to get
  \begin{equation}
    \label{eq:4}
    \sum_{i=1}^r p_i(y,c-y/x)\cdot f_i(x,cx) = 0.
  \end{equation}
  Each $p_{i}(y, c-y/x)$ admits an expansion as a Laurent series in $y$
  \begin{equation}
    \label{eq:2}
    p_{i}(y, c-y/x) = \sum_{j=-k}^{\infty}q_{i,j}(c,x)y^{j}
  \end{equation}
  for some $k\in\set N$.
  Eq.~\eqref{eq:4} then expands as
  \begin{equation}
    \label{eq:10}
    \sum_{j=-k}^{\infty}\big(\sum_{i=1}^r q_{i,j}(c,y)\cdot f_i(x,cx)\big) y^{j} = 0
  \end{equation}
  and therefore, for all $j \geq -k$,
  \begin{equation}
    \label{eq:12}
    \sum_{i=1}^r q_{i,j}(c,y)\cdot f_i(x,cx) = 0.
  \end{equation}
  Since the $p_{i}$ are not all $0$, there must exist $i,j$ with $q_{i,j}\neq 0$,
  and therefore for such a value of~$j$, the left-hand side of Eq.~\eqref{eq:12} is
  a non-trivial linear combination.
  
  Furthermore, observe that since the $p_i$ are rational in their second
  argument, the coefficients $q_i$ are bivariate
  rational functions.
  So finally, substituting $c\to y/x$ yields the desired dependency with coefficients in~$C$:
  \[
    \sum_{i=1}^r q_i(y/x,x) f_i(x,y) = 0.\qedhere
  \]
\end{proof}

\begin{ex}
  The annihilator~$I_1$ of $\exp(x),y\exp(x)$ is not radical. The annihilator~$I_2$
  of $\exp(x),\exp(x+y)$ is radical but not in normal position w.r.t. $D_x$.
  Setting $y$ to $y+cx$ in~$I_1$ gives the annihilator of $\exp(x),(y+cx)\exp(x)$,
  which is still not radical. However, setting $y$ to $y+cx$ in~$I_2$ gives
  the annihilator of $\exp(x),\exp((1+c)x+y)$, which is in normal
  $\partial_x$-position for every choice $c\neq0$.
\end{ex}

In the case of more than two variables, Thm.~\ref{thm:our23} generalizes as
the corresponding Thms.~\ref{thm:2} and~\ref{thm:3} from the commutative case
suggest. We then have one operator $D_x$ that plays the role of $x_n$ and
several operators $D_{y_1},\dots,D_{y_{n-1}}$ that play the roles of $x_1,\dots,x_{n-1}$.
Also Thm.~\ref{thm:changofvariables} generalizes in a straightforward way to
more variables, but in a slightly different way than suggested by Thm.~\ref{thm:1}:
while we replace $x_n$ by $x_n-c_1x_1-\cdots-c_{n-1}x_{n-1}$ in the commutative
case, we have to replace each $y_i$ by $y_i+c_ix$, for $i=1,\dots,n-1$. 

\bibliographystyle{plain}
\bibliography{bib}

\end{document}
