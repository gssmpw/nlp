%---------------------------------------------------
%===================================================
%===================== Packages ===========================
%% referencing packages: ----------------------------------
%\usepackage{natbib}
%\usepackage[numbers]{natbib}
\usepackage[numbers,compress]{natbib}
%\usepackage[noadjust]{cite}

%% math packages: -----------------------------------------
\usepackage{amsmath,amssymb,mathrsfs,amsfonts} 
\usepackage{bm,bbm,dsfont} 
% \usepackage{physics}

%% graphics packages: -------------------------------------
\usepackage{color,graphicx,graphics, epstopdf} %, subcaption, transparent, pgfplots

%% table/figures/pseudocode/algorithms packages: ---------
\usepackage{algorithm,algorithmicx,multirow, titlecaps}  
%\usepackage[options ]{algorithm2e}
\usepackage[english]{babel}
%\usepackage[utf8]{inputenc}
\usepackage[noend]{algpseudocode} 
\usepackage{threeparttable}
%\usepackage[T1]{fontenc}
\usepackage{tikzsymbols}
%
\usepackage{multirow, titlecaps} 
\usepackage{tabularx}
\usepackage{booktabs}
\usepackage{multirow}
\usepackage{courier}
\usepackage{float}
\usepackage{multirow, titlecaps} 
\usepackage{tabularx}
\usepackage{booktabs}
\usepackage{multirow}
\usepackage{courier}
\usepackage{float}
\usepackage{makecell}
\usepackage{diagbox}
\usepackage{physics}
\usepackage{tikz}
\usepackage{circuitikz}
% \usepackage{makecell}
% \setcellgapes{5pt}
%\usepackage{subcaption}
%\captionsetup{compatibility=false}
\newcommand{\ra}[1]{\renewcommand{\arraystretch}{#1}}

% Other packages: -----------------------------------------
%\usepackage{authblk} 
\allowdisplaybreaks
\usepackage{verbatim}

% Defining mathbbold font: --------------------------------
\DeclareMathAlphabet{\mathbbold}{U}{bbold}{m}{n}

% Writting MatLab in a nice way!
\usepackage{xspace}
\newcommand{\MATLAB}{\textsc{Matlab}\xspace}

% url & collering links------------------------------------
\usepackage{url}
 \iftrue    % Single column/IEEE Tran. L-CSS/Elsevier
%\iffalse    %<-- ifacconf does not allow hpperlinking - uncomment it
    \usepackage[colorlinks=true]{hyperref}
	\usepackage{xcolor}
	\hypersetup{
		colorlinks,
		linkcolor={blue!90!black},
		citecolor={red!90!black},
		urlcolor={blue!20!black}
	}
	\allowdisplaybreaks
\fi
%==================== Commenting ==================
\newcommand{\cm}[1]{{\color{violet}{{[#1]}}}}
\newcommand{\cmb}[1]{{\color{blue}{#1}}}
\newcommand{\cmr}[1]{{\color{red}{#1}}}
\newcommand{\cmg}[1]{{\color{green}{#1}}}
\newcommand{\cmm}[1]{{\color{magenta}{#1}}}
\newcommand{\cmc}[1]{{\color{cyan}{#1}}}
\newcommand{\cmt}[1]{{\color{teal}{#1}}}
\newcommand{\cmp}[1]{{\color{purple}{#1}}}
%==========================================================
\newcommand{\mmtiny}[1]{{\scalebox{.90}{#1}}}
\newcommand{\mtiny}[1]{{\scalebox{.81}{#1}}}
\newcommand{\mstiny}[1]{{\scalebox{.67}{#1}}}
\newcommand{\smtiny}[1]{{\scalebox{.5}{#1}}}
\newcommand{\stiny}[1]{{\scalebox{.38}{#1}}}
\newcommand{\sstiny}[1]{{\scalebox{.25}{#1}}}
%================= MATH Abbreviation ======================
%=============== Abbreviation ==================
\newcommand*{\argminOp}{\operatornamewithlimits{argmin}\limits}
\newcommand*{\argmaxOp}{\operatornamewithlimits{argmax}\limits}
\newcommand*{\minOp}{\operatornamewithlimits{min}\limits}
\newcommand*{\maxOp}{\operatornamewithlimits{max}\limits}
\newcommand*{\supOp}{\operatornamewithlimits{sup}\limits}
\newcommand*{\infOp}{\operatornamewithlimits{inf}\limits}
\newcommand*{\esssup}{\operatornamewithlimits{ess\,sup}}
\newcommand*{\essinf}{\operatornamewithlimits{ess\,inf}}
\newcommand*{\esssupOp}{\operatornamewithlimits{ess\,sup}\limits}
\newcommand*{\essinfOp}{\operatornamewithlimits{ess\,inf}\limits}
\newcommand*{\argsupOp}{\operatornamewithlimits{argsup}\limits}
\newcommand*{\arginfOp}{\operatornamewithlimits{arginf}\limits}
\newcommand*{\sumOp}{\operatornamewithlimits{\sum}\limits}
\newcommand*{\SumOp}{\operatornamewithlimits{\text{\scalebox{1.25}{$\sum$}}}\limits}
\newcommand*{\SUMOp}{\operatornamewithlimits{\text{\scalebox{1.6}{$\sum$}}}\limits}
\newcommand*{\intOp}{\operatornamewithlimits{\int}\limits}
\newcommand*{\IntOp}{\operatornamewithlimits{\text{\scalebox{1.25}{$\int$}}}\limits}
\newcommand*{\INTOp}{\operatornamewithlimits{\text{\scalebox{1.6}{$\int$}}}\limits}
\newcommand*{\cupOp}{\operatornamewithlimits{\cup}\limits}
\newcommand*{\capOp}{\operatornamewithlimits{\cap}\limits}
\newcommand*{\limOp}{\operatornamewithlimits{lim}\limits}
\newcommand*{\limsupOp}{\operatornamewithlimits{limsup}\limits}
\newcommand*{\liminfOp}{\operatornamewithlimits{liminf}\limits}
\newcommand*{\CupOp}{\operatornamewithlimits{\text{\scalebox{1.25}{$\cup$}}}\limits}
\newcommand*{\CapOp}{\operatornamewithlimits{\text{\scalebox{1.25}{$\cap$}}}\limits}
\newcommand*{\CUPOp}{\operatornamewithlimits{\text{\scalebox{1.6}{$\cup$}}}\limits}
\newcommand*{\CAPOp}{\operatornamewithlimits{\text{\scalebox{1.6}{$\cap$}}}\limits}
\newcommand{\Max}{\max\limits}
\newcommand{\Min}{\min\limits}
\newcommand{\Sup}{\sup\limits}
\newcommand{\Inf}{\inf\limits}
\newcommand{\Sum}{\sum\limits}
%--- Math Notaions-----------------------------
\newcommand{\GP}{\mathcal{G\!P}}             % Gaussian process
\newcommand{\ddt}[1]{\frac{\mathrm{d}}{\mathrm{d}t}#1} % d/dt
% \renewcommand{\tr}{{\mathsf{T}}}              % transpose supscript
\renewcommand{\tr}{{\text{\mtiny{$\mathsf{T}$}}}}              % transpose supscript
%\newcommand{\tr}{{\smtiny{$\mathsf{T}$ }}\!} % transpose supscript - small version
\newcommand{\Hr}{{\mathsf{H}}}               % Hermitian supscript
\newcommand{\zero}{\mathbf{0}}               % all zero vector/function
\newcommand{\one}{\mathbf{1}}                % all one vector/function
\newcommand{\onefun}{\mathbbm{1}}            % all one vector/function   
%\newcommand{\onefun}{\mathds{1}}            % all one vector/function
\iftrue%\iffalse %<-<-<-<-<-<-<-<-<-<-<-<-<-<-<
% matrices I & 0
\newcommand{\zeromx}{{\mathbbold{0}}} %<<<<
\newcommand{\eye}{\mathbb{I}}
\else
\newcommand{\zeromx}{{\mathbf{0}}} %<<<<
\newcommand{\eye}{\mathbf{I}}
\fi%<-<-<-<-<-<-<-<-<-<-<-<-<-<-<-<-<-<-<-<-<-<
%\newcommand{\rank}{\mathrm{rank}}     % rank of a matrix  
\newcommand{\vc}[1]{{ \mathrm{#1} }}  % vector op
\newcommand{\mx}[1]{{ \mathrm{#1} }}  % matrix op 
\newcommand*{\Times}{\operatornamewithlimits{\scalebox{1.5}{$\times$}}\limits} % large X or times for Cartesian product
\newcommand{\dom}{\mathrm{dom}}	          % domain of a function
\newcommand{\conv}{\mathrm{conv}}         % convex hull
\newcommand{\convcone}{\mathrm{convcone}} % convex cone hull
\newcommand{\pos}{\mathrm{pos}}         % positive hull
\newcommand{\supp}{\mathrm{supp}}       % support of a distribution/vector 
\newcommand{\diag}{\mathrm{diag}}       % diag of a matrix   
\newcommand{\Diag}{\mathrm{Diag}}       % Diagonal matrix with entries given in a vector
\newcommand{\drm}{\mathrm{d}}           % infinitestimal operator in integral \int f d x  
\newcommand{\closure}{\mathrm{cl}}      % closure
\newcommand{\interior}{\mathrm{int}}    % interior
\newcommand{\linspan}{\mathrm{span}}    % span
\newcommand{\proj}{\mathrm{proj}}       % projection
%\newcommand{\norm}[1]{\|#1\|}           % norm    
\newcommand{\inner}[2]{{ \langle {#1,#2} \rangle}} % inner product
\newcommand{\nth}{{\text{\tiny{th}}}}   % n-th 
\renewcommand{\trace}{\mathrm{tr}}        % trace
\newcommand{\imag}[1]{\mathrm{imag}(#1)}% imaginary part of
%\newcommand{\real}[1]{\mathrm{real}(#1)}% real part of
\newcommand{\img}{\mathrm{im}}          % image of a map 
%\newcommand{\ker}{\mathrm{ker}}        % kernel of map
\renewcommand{\emptyset}{\varnothing}   % empty set
\newcommand{\expe}{\mathrm{e}}          % natural basis for expontial, as in e^x   
\newcommand{\fro}{\mathrm{F}}           % subscript for Frobenius norm
\newcommand{\floor}[1]{\lfloor #1\rfloor} % floor of a number
\newcommand{\ceil}[1]{\lceil #1\rceil}    % ceil of a number
\newcommand{\logdet}{\mathrm{logdet}}       % logdet
\renewcommand{\vec}{\mathrm{vec}}           % vec


%\newcommand{\ol}[1]{\overline{#1}}
%\newcommand{\ul}[1]{\underline{#1}}
\newcommand{\ol}[1]{\overline{#1}}
% \newcommand{\ul}[1]{\underlline{#1}}




%% __ mathscr of D, F, X, Y, Z,...
\newcommand{\Ascr}{{\mathscr{A}}}
\newcommand{\Bscr}{{\mathscr{B}}}
\newcommand{\Cscr}{{\mathscr{C}}}
\newcommand{\Dscr}{{\mathscr{D}}}
\newcommand{\Escr}{{\mathscr{E}}}
\newcommand{\Fscr}{{\mathscr{F}}}
\newcommand{\Gscr}{{\mathscr{G}}}
\newcommand{\Hscr}{{\mathscr{H}}}
\newcommand{\Iscr}{{\mathscr{I}}}
\newcommand{\Jscr}{{\mathscr{J}}}
\newcommand{\Kscr}{{\mathscr{K}}}
\newcommand{\Lscr}{{\mathscr{L}}}
\newcommand{\Mscr}{{\mathscr{M}}}
\newcommand{\Nscr}{{\mathscr{N}}}
\newcommand{\Oscr}{{\mathscr{O}}}
\newcommand{\Pscr}{{\mathscr{P}}}
\newcommand{\Qscr}{{\mathscr{Q}}}
\newcommand{\Rscr}{{\mathscr{R}}}
\newcommand{\Sscr}{{\mathscr{S}}}
\newcommand{\Tscr}{{\mathscr{T}}}
\newcommand{\Uscr}{{\mathscr{U}}}
\newcommand{\Vscr}{{\mathscr{V}}}
\newcommand{\Wscr}{{\mathscr{W}}}
\newcommand{\Xscr}{{\mathscr{X}}}
\newcommand{\Yscr}{{\mathscr{Y}}}
\newcommand{\Zscr}{{\mathscr{Z}}}

%% __ mathcal of X, Y, Z,...
\newcommand{\Acal}{{\mathcal{A}}}
\newcommand{\Bcal}{{\mathcal{B}}}
\newcommand{\Ccal}{{\mathcal{C}}}
\newcommand{\Dcal}{{\mathcal{D}}}
\newcommand{\Ecal}{{\mathcal{E}}}
\newcommand{\Fcal}{{\mathcal{F}}}
\newcommand{\Gcal}{{\mathcal{G}}}
\newcommand{\Hcal}{{\mathcal{H}}}
\newcommand{\Ical}{{\mathcal{I}}}
\newcommand{\Jcal}{{\mathcal{J}}}
\newcommand{\Kcal}{{\mathcal{K}}}
\newcommand{\Lcal}{{\mathcal{L}}}
\newcommand{\Mcal}{{\mathcal{M}}}
\newcommand{\Ncal}{{\mathcal{N}}}
\newcommand{\Ocal}{{\mathcal{O}}}
\newcommand{\Pcal}{{\mathcal{P}}}
\newcommand{\Qcal}{{\mathcal{Q}}}
\newcommand{\Rcal}{{\mathcal{R}}}
\newcommand{\Scal}{{\mathcal{S}}}
\newcommand{\Tcal}{{\mathcal{T}}}
\newcommand{\Ucal}{{\mathcal{U}}}
\newcommand{\Vcal}{{\mathcal{V}}}
\newcommand{\Wcal}{{\mathcal{W}}}
\newcommand{\Xcal}{{\mathcal{X}}}
\newcommand{\Ycal}{{\mathcal{Y}}}
\newcommand{\Zcal}{{\mathcal{Z}}}

%% __ mathbb of X, Y, Z,...
%\newcommand{\Bbb}{{\mathbb{B}}}
\newcommand{\Cbb}{{\mathbb{C}}}
\newcommand{\Dbb}{{\mathbb{D}}}
\newcommand{\Ebb}{{\mathbb{E}}}
\newcommand{\Fbb}{{\mathbb{F}}}
\newcommand{\Gbb}{{\mathbb{G}}}
\newcommand{\Hbb}{{\mathbb{H}}}
\newcommand{\Ibb}{{\mathbb{I}}}
\newcommand{\Jbb}{{\mathbb{J}}}
\newcommand{\Kbb}{{\mathbb{K}}}
\newcommand{\Lbb}{{\mathbb{L}}}
\newcommand{\Mbb}{{\mathbb{M}}}
\newcommand{\Nbb}{{\mathbb{N}}}
\newcommand{\Pbb}{{\mathbb{P}}}
\newcommand{\Qbb}{{\mathbb{Q}}}
\newcommand{\Rbb}{{\mathbb{R}}}
\newcommand{\Sbb}{{\mathbb{S}}}
\newcommand{\Tbb}{{\mathbb{T}}}
\newcommand{\Ubb}{{\mathbb{U}}}
\newcommand{\Vbb}{{\mathbb{V}}}
\newcommand{\Xbb}{{\mathbb{X}}}
\newcommand{\Ybb}{{\mathbb{Y}}}
\newcommand{\Zbb}{{\mathbb{Z}}}
\newcommand{\bbo}{\mathbbold{o}}
%\newcommand{\bbo}{\mathbb{o}}
\newcommand{\bbh}{\mathds{h}}
%\newcommand{\bbk}{\mathds{k}}
%\newcommand{\bbk}{\mathbbold{k}}
\newcommand{\bbk}{\mathbbm{k}}

%% __ hat/tilde/.. of f, g
\newcommand{\hatf}{{\hat{f}}}
\newcommand{\hatg}{{\hat{g}}}
\newcommand{\tildef}{{\tilde{f}}}
\newcommand{\tildeg}{{\tilde{g}}}

%% Probability --------------------------------------------
\newcommand{\Prb}{\mathbb{P}}
\newcommand{\PrbP}{\mathbb{P}}
\newcommand{\PrbQ}{\mathbb{Q}}
\newcommand{\Exp}{\mathbb{E}}
% ?
%=========== Theorem ======================================
\iffalse %% uncomment for Single column + IEEE Trans (TAC, L-CSS,...)
%\iftrue   %% uncomment for Elsevier (Automatica, System & Control Letters,...)  + ifacconf
    %\let\proof\relax
    %\let\endproof\relax
    %\usepackage{amsthm}
    %\renewcommand{\qedsymbol}{$\blacksquare$}
    %% from here-------------------------
\makeatletter
\DeclareRobustCommand{\qed}{%
  \ifmmode % if math mode, assume display: omit penalty etc.
  \else \leavevmode\unskip\penalty9999 \hbox{}\nobreak\hfill
  \fi
  \quad\hbox{\qedsymbol}}
\newcommand{\openbox}{\leavevmode
  \hbox to.77778em{%
  \hfil\vrule
  \vbox to.675em{\hrule width.6em\vfil\hrule}%
  \vrule\hfil}}
\newcommand{\qedsymbol}{\openbox}
\newenvironment{proof}[1][\proofname]{\par
  \normalfont
  \topsep6\p@\@plus6\p@ \trivlist
  \item[\hskip\labelsep\itshape
    #1.]\ignorespaces
}{%
  \qed\endtrivlist
}
\newcommand{\proofname}{Proof}
\makeatother
%% upto here----------------------------
    \theoremstyle{plain}
    \newtheorem{theorem}{Theorem}
    %\newtheorem*{theorem*}{Theorem}
    \newtheorem{definition}{Definition}
    %\newtheorem*{definition*}{Definition}
    \newtheorem{assumption}{Assumption}
    \newtheorem{proposition}[theorem]{Proposition}
    \newtheorem{corollary}[theorem]{Corollary}
    \newtheorem{lemma}[theorem]{Lemma}
    \newtheorem{remark}{Remark}
    \newtheorem{example}{Example}
    %\newtheorem*{example*}{Example}
    %\newtheorem{claim}{Claim}
    %\newtheorem*{claim*}{Claim}
    \newtheorem{problem}{Problem}
    %\newtheorem*{problem*}{Problem}
    
    \newcommand\xqed[1]{%
    	\leavevmode\unskip\penalty9999 \hbox{}\nobreak\hfill
    	\quad\hbox{#1}}
    %\newtheorem{examplex}{Example}
    %\newenvironment{example}
    %{\pushQED{\qed}\renewcommand{\qedsymbol}{$\triangle$}\examplex}
    %{\popQED\endexamplex}

    %%This is a failed attempt to make an environment
    %\newtheorem{problemnon}{Problem}
    %\newenvironment{problemnon}
    %{\pushQED{\qed}\renewcommand{\qedsymbol}{$\triangle$}\problemnon}
    %{\popQED\endproblemnon}
\else 
    \let\proof\relax
    \let\endproof\relax
    \usepackage{amsthm}
    \renewcommand{\qedsymbol}{$\blacksquare$}
    
    %\theoremstyle{plain}
    \newtheorem{theorem}{Theorem}
    \newtheorem*{theorem*}{Theorem}
    \newtheorem{definition}{Definition}
    \newtheorem*{definition*}{Definition}
    \newtheorem{assumption}{Assumption}
    \newtheorem{proposition}[theorem]{Proposition}
    \newtheorem{corollary}[theorem]{Corollary}
    \newtheorem{lemma}[theorem]{Lemma}
    \newtheorem{remark}{Remark}
    \newtheorem{example}{Example}
    \newtheorem*{example*}{Example}
    %\newtheorem{claim}{Claim}
    %\newtheorem*{claim*}{Claim}
    \newtheorem{problem}{Problem}
    \newtheorem*{problem*}{Problem}
\fi

%==========================================================
%========== Figures Path ==================================
%\graphicspath{{figures/}}
%==========================================================
%=========== TODO List ====================================
\iffalse
\let\labelindent\relax
\usepackage{enumitem}
\newlist{todolist}{itemize}{2}
\setlist[todolist]{label=$\square$}
\usepackage{pifont}
\newcommand{\cmark}{\ding{51}}%
\newcommand{\xmark}{\ding{55}}%
\newcommand{\done}{\rlap{$\square$}{\raisebox{2pt}{\large\hspace{1pt}\cmark}}% 
	\hspace{-2.5pt}}
% for being done and remain in list!
\newcommand{\wontfix}{\rlap{$\square$}{\large\hspace{1pt}\xmark}} 
% for being conceled but remain in the list!
%
%
% This is an example:
        % \hrulefill\\
        % \cm{TO-DO LIST:\begin{todolist}
        % \item[\done] Task 1!
        % \item Task 2!
        % \item Task set 3:
        % \begin{todolist}
        % 	\item[\done] Subtask 3.1!
        % 	\item[\wontfix] Subtask 3.2!
        % \end{todolist}
        % \item Task 4!
        % \item Task set 5:
        % \begin{todolist}
        % 	\item Subtask 5.1!
        % 	\item Subtask 5.2!
        % \end{todolist}
        % \end{todolist}}
        % \hrulefill
\fi 
%==========================================================
%==========================================================
