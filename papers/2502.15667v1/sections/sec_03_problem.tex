\section{Identification of Linear Dynamics with Bilinear Observation Models} \label{sec:pf}
Consider an \emph{unknown} time-invariant random dynamical system $\Scal$ with linear dynamics and a bilinear observation model.  
%More precisely, let the dynamics of $\Scal$ be described with a linear process model as
More precisely, let the process model describing the dynamics of $\Scal$ be as
\begin{equation}\label{eqn:dynamics1}
\vcx_{k+1} = \mxA\vcx_k+\mxB\vcu_k+\vcw_k,    \qquad \forall k\in\Zbb_+,
\end{equation}
where $\vcx_{k}\in\Rbb^{\Nx}$, $\vcu_k\in\Rbb^{\Nu}$, and $\vcw_k\in\Rbb^{\Nx}$ are respectively the vectors of state variables, input, and process noise, at time instant $k\in\Zbb_+$, and,
$\mxA\in\Rbb^{\Nx\times\Nx}$ and $\mxB\in\Rbb^{\Nx\times\Nu}$ are \emph{unknown} matrices characterizing the dynamics of system.
Also, let the observation model of the system have a bilinear form as 
\begin{equation}\label{eqn:dynamics2}
\vcy_{k+1} = \big(\mxC_0+\sum_{i=1}^{\Nu}\mxC_iu_{k,i}\big)\vcx_k+\mxD\vcu_k+\vcv_k,       \qquad \forall k\in\Zbb_+,
\end{equation}
where $u_{k, i}$ denotes the $i^\nth$ entry of $\vcu_k$, for $i=1,\ldots,\Nu$ and $k\in\Zbb_+$, $\vcy_{k}\in\Rbb^{\Ny}$ and $\vcv_k\in\Rbb^{\Ny}$  are respectively the vectors of output observations and measurement noise, and, $\mxC_0,\mxC_1,\ldots,\mxC_{\Nu}\in\Rbb^{\Ny\times\Nx}$ and $\mxD\in\Rbb^{\Ny\times\Nu}$ are \emph{unknown} matrices describing the observation model of the system. Suppose that the initial state $\vcx_0$, the process noise $\big(\vcw_k\big)_{k\in\Zbb_+}$, and the measurement noise $\big(\vcv_k\big)_{k\in\Zbb_+}$ are mutually independent Gaussian random variables such as % with the following probability distributions 
\begin{align}
    &\vcx_0\sim\Ncal(\mux,\mxSx),
    \label{eqn:x_dist_Gaussion}
    \\
    &\vcw_k\sim\Ncal(0,\mxSw), \qquad \forall k\in\Zbb_+,
    \label{eqn:w_dist_Gaussion}
    \\
    &\vcv_k\sim\Ncal(0,\mxSv), \qquad \forall k\in\Zbb_+,
    \label{eqn:v_dist_Gaussion}
\end{align}
where $\mux\in\Rbb^{\Nx}$ is  an \emph{unknown} vector and $\mxSx\in\Rbb^{\Nx\times\Nx}$ is an \emph{unknown} positive definite matrix  respectively denoting the mean and covariance of $\vcx_0$, and, $\mxSw\in\Rbb^{\Nx\times\Nx}$  and  $\mxSv\in\Rbb^{\Ny\times\Ny}$ are \emph{unknown} positive definite matrices representing respectively the covariance of vector $\vcw_k$ and  the covariance of vector $\vcv_k$, for any $k\in\Zbb_+$.  


Assume a set of $\nD\in\Nbb$ input-output pairs of measurement data is given as
\begin{equation}
    \Dcal:=\big\{(\vcu_k,\vcy_k)\,|\, k=0,\ldots,\nD-1\big\}
\end{equation}
Accordingly, we introduce the main problem discussed in this paper as identifying system $\Scal$ through estimating the unknown vector and matrices mentioned above using the set of data $\Dcal$. 
%For ease of discussion, we assume $\mxD = 0$ throughout this paper.
\begin{problem*}[\textbf{Identification Problem for Linear Dynamics with Bilinear Observation Models}] \label{pr:Bilinear Observation}
Given the measurement set of data $\Dcal$, 
%estimate vector $\mux$ and matrices $\mxA, \mxB,\mxC_0,\mxC_1,\ldots,\mxC_{\Nu},\mxD,\mxSx,\mxSw,\mxSv$.
estimate $\mxA$, $\mxB$, $\mxC_0$, $\mxC_1$, $\ldots$, $\mxC_{\Nu}$, $\mxD$, $\mux$, $\mxSx$, $\mxSw$, and $\mxSv$.
\end{problem*}
% To address the introduced estimation problem, we employ probabilistic approaches, namely the maximum likelihood \cite{lehmann2006theory} and the expectation-maximization (EM) \cite{theodoridis2006pattern} methods.
% As discussed 
% %in Section~\ref{sec:ML} and Section~\ref{sec:An Expectation-Maximization Approach}, 
% in the remainder of this paper,
% we obtain tractable procedures for the implementation of the mentioned methods. More details are discussed in Section~\ref{sec:ML} and Section~\ref{sec:An Expectation-Maximization Approach}.
To address the estimation problem introduced, we utilize probabilistic approaches, specifically the maximum likelihood (ML) method \cite{lehmann2006theory} and the expectation--maximization (EM) algorithm \cite{theodoridis2006pattern}. In the following sections, we derive tractable procedures for implementing these methods. 
% Detailed discussions can be found in Section~\ref{sec:ML} for the ML approach and Section~\ref{sec:An Expectation-Maximization Approach} for the EM algorithm.
Detailed discussions can be found in Section~\ref{sec:ML} and Section~\ref{sec:An Expectation-Maximization Approach}.