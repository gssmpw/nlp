\section{Conclusion} \label{sec:conclusion}
%
% In this paper, we have studied the identification problem for unknown dynamical systems with linear dynamics and bilinear observation models. The identification of such systems is formulated as a tractable problem and a scheme is proposed to solve the identification problem using RTS smoother and EM algorithm. In the proposed scheme, the RTS smoother is used to estimate the states of the system, while the EM-based approach is introduced to estimate the unknown parameters. The performance of the introduced scheme is evaluated through a numerical example. Furthermore, we also compare the performance of the proposed scheme under different SNRs through another numerical experiment. Future work will be focused on reducing the algorithm's dependence on the initial guess. Additionally, the numerical experiments can be extended to engineering applications or systems with higher dimensions of states, inputs, and outputs.
% In this paper, we have studied the identification problem for unknown dynamical systems with linear dynamics and bilinear observation models. First, we have proposed a tractable identification procedure based on ML to find the maximal probability of observing the measurements. Due to the high computational demand, we have proposed another procedure based on RTS smoother and EM algorithm. In this scheme, the RTS smoother has been used to estimate the states of the system, while the EM approach has been introduced to estimate the unknown parameters. The performance of both approaches has been evaluated through numerical examples. We have compared the performance of the proposed scheme under different SNRs through a Monte Carlo numerical experiment. In addition, we have applied the EM algorithm on a non-ideal capacitor example and show the estimated output trajectory is close to the real one.
%
% In this paper, we have addressed the identification problem for unknown dynamical systems with linear dynamics and bilinear observation models. We began by proposing a tractable identification procedure based on maximum likelihood (ML) to maximize the probability of observing the measurements. Given the high computational demand associated with this method, particularly for large measurement datasets, we then introduced an alternative approach combining the Rauch–Tung–Striebel (RTS) smoother and the Expectation-Maximization (EM) algorithm. In this framework, the RTS smoother estimates the system states, while the EM algorithm is used to estimate the unknown parameters. We have evaluated the performance of both approaches through numerical examples, comparing their effectiveness under various signal-to-noise ratios (SNRs) via extensive Monte Carlo numerical experiments. Furthermore, we applied the EM algorithm to a non-ideal capacitor example, demonstrating that the estimated output trajectory closely matches the real one. These results highlight the potential of the proposed methods for addressing the identification problem in practical applications.
%
In this paper, we have addressed the identification problem for unknown dynamical systems with linear dynamics and bilinear observation models. 
First, we have proposed a tractable identification procedure based on maximum likelihood (ML) to maximize the probability of observing the measurements. 
% Given the high computational demand associated with this method, we then introduced an alternative approach combining the Rauch–Tung–Striebel (RTS) smoother and the Expectation-Maximization (EM) algorithm. 
Given the considerable computational demand associated with the developed ML-based method, particularly for large measurement datasets, we have introduced an alternative approach combining the Rauch–Tung–Striebel (RTS) smoother with the Expectation-Maximization (EM) algorithm.
In the resulting framework, the RTS smoother estimates the system states, while the EM algorithm is used to estimate the unknown parameters. We have evaluated the performance of both approaches through numerical examples, comparing their effectiveness under various signal-to-noise ratios (SNRs) via onte Carlo numerical experiments. Furthermore, we applied the EM algorithm to a non-ideal capacitor example, demonstrating that the estimated output trajectory closely matches the real one. These results highlight the potential of the proposed methods for addressing the identification problem in practical applications.


