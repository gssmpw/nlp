% Abstract _______________________________________________
\begin{abstract}
%In this paper, we focus on identifying systems described by linear time-invariant dynamics with bilinear observation models. Specifically, we consider a suitable parametric formulation of the system. Based on observed input-output data, we frame an identification problem as the estimation of parameters that we use to define the system's mathematical model. To this end, we propose two probabilistic frameworks. The first framework uses the Maximum Likelihood (ML) approach, which finds the optimal parameter estimates by maximizing a likelihood function. The first-order approach is used to solve the optimization problem. Furthermore, to improve the tractability and computational efficiency, we propose a framework using the Expectation- Maximization (EM) approach to estimate the parameters based on a designed cost function. We show that the cost function is invex and thus has a unique optimum. The closed-form solution of the optimum is given. In addition, we demonstrate the recursive feasibility of the EM procedure. The effectiveness and performance of the proposed approaches are evaluated through numerical experiments. The results demonstrate the methods' capacity to estimate system parameters and show their potential usage for practical applications in scenarios with bilinear observational structures.
% In this paper, we focus on the identification problem for the systems characterized by linear time-invariant dynamics with bilinear observation models. More precisely, we consider a suitable parametric description for the system. Based on the observed input-output data, we formulate the identification problem as the estimation of parameters defining the mathematical model of the system. To this end, we propose two probabilistic frameworks. The first framework uses the Maximum Likelihood (ML) approach, which finds the optimal parameter estimates by maximizing a likelihood function. Accordingly, we develop a tractable first-order approach to solve the optimization problem corresponding to the proposed ML approach. Furthermore, to improve the tractability and computational efficiency, we propose a framework using the Expectation--Maximization (EM) approach to estimate the parameters based on an appropriately designed cost function. We show that the cost function is invex, and thus, it admits a unique optimizer. We derive the closed-form description for the optimal solution. In addition, we demonstrate the recursive feasibility of the EM procedure. The effectiveness and performance of the proposed approaches are evaluated through numerical experiments. The results demonstrate the efficacy of the methods to estimate system parameters and show their potential usage for practical applications in scenarios with bilinear observational structures.
% In this paper, we focus on the identification problem for the systems characterized by linear time-invariant dynamics with bilinear observation models. More precisely, we consider a suitable parametric description of the system. Using the observed input-output data, we formulate the identification problem as the estimation of the parameters defining the mathematical model of the system. To this end, we propose two probabilistic frameworks. The first framework employs the Maximum Likelihood (ML) approach, which finds the optimal parameter estimates by maximizing a likelihood function. Subsequently, we develop a tractable first-order method to solve the optimization problem corresponding to the proposed ML approach. Additionally, to further improve tractability and computational efficiency of the estimation of the parameters, we introduce a framework based on the Expectation--Maximization (EM) approach, which estimates the parameters using an appropriately designed cost function. We demonstrate that the cost function is invex, which ensures the existence and uniqueness of the optimal solution. Additionally, we derive the closed-form solution for the optimal parameters. Furthermore, we prove the recursive feasibility of the EM procedure. The effectiveness and high performance of the proposed approaches are verified and compared through numerical experiments, demonstrating the efficacy of the methods to accurately estimate the system parameters and their potential for practical applications in scenarios involving bilinear observation structures.
% In this paper, we address the identification problem for systems characterized by linear time-invariant dynamics with bilinear observation models. Specifically, we adopt a parametric description of the system and formulate the identification task as the estimation of parameters defining the mathematical model, using observed input-output data. To tackle this problem, we propose two probabilistic frameworks. The first leverages the Maximum Likelihood (ML) approach, which determines optimal parameter estimates by maximizing a likelihood function. To solve the resulting optimization problem, we develop a tractable first-order method. To enhance computational efficiency, we introduce an alternative framework based on the Expectation--Maximization (EM) approach, which iteratively estimates parameters via a specially designed cost function. We show that this cost function is invex, guaranteeing the existence and uniqueness of the optimal solution, and derive the closed-form expressions for these parameters. Moreover, we prove the recursive feasibility of the EM procedure. Numerical experiments validate the effectiveness and computational efficiency of both approaches, highlighting their accuracy in estimating system parameters and their potential for practical applications involving bilinear observation structures.
% In this paper, we address the identification problem for systems characterized by linear time-invariant dynamics with bilinear observation models. Specifically, we adopt a parametric description of the system and formulate the identification task as the estimation of parameters defining the mathematical model, using observed input-output data. To tackle this problem, we propose two probabilistic frameworks. The first leverages the Maximum Likelihood (ML) approach, which determines optimal parameter estimates by maximizing a likelihood function. To solve the resulting optimization problem, we develop a tractable first-order method. To enhance computational efficiency, we introduce an alternative framework based on the Expectation--Maximization (EM) approach, which iteratively estimates parameters via a specially designed cost function. We show that this cost function is invex, guaranteeing the existence and uniqueness of the optimal solution, and derive the closed-form expressions for these parameters. Moreover, we prove the recursive feasibility of the EM procedure. Numerical experiments validate the effectiveness and computational efficiency of both approaches, highlighting their accuracy in estimating system parameters and their potential for practical applications involving bilinear observation structures.
In this paper, we address the identification problem for the systems characterized by linear time-invariant dynamics with bilinear observation models. More precisely, we consider a suitable parametric description of the system and formulate the identification problem as the estimation of the parameters defining the mathematical model of the system using the observed input-output data. To this end, we propose two probabilistic frameworks. The first framework employs the Maximum Likelihood (ML) approach, which accurately finds the optimal parameter estimates by maximizing a likelihood function. Subsequently, we develop a tractable first-order method to solve the optimization problem corresponding to the proposed ML approach. Additionally, to further improve tractability and computational efficiency of the estimation of the parameters, we introduce an alternative framework based on the Expectation--Maximization (EM) approach, which estimates the parameters using an appropriately designed cost function. We show that the EM cost function is invex, which ensures the existence and uniqueness of the optimal solution. Furthermore, we derive the closed-form solution for the optimal parameters and also prove the recursive feasibility of the EM procedure. 
% The estimation efficacy and practical feasibility of the proposed approaches are verified and compared through numerical experiments, demonstrating the effectiveness of the methods to accurately estimate the system parameters and their potential for real-world applications in scenarios involving bilinear observation structures.
Through extensive numerical experiments, the practical implementation of the proposed approaches is demonstrated, and their estimation efficacy is verified and compared, highlighting the effectiveness of the methods to accurately estimate the system parameters and their potential for real-world applications in scenarios involving bilinear observation structures.
\end{abstract}
   