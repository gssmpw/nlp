\documentclass{article}
\usepackage{listings}
\usepackage{xcolor}
\usepackage{array}

% 定义代码高亮的样式
\lstset{
    basicstyle=\ttfamily\small,
    backgroundcolor=\color{lightgray!20},
    frame=single,
    breaklines=true,
    postbreak=\mbox{\textcolor{red}{$\hookrightarrow$}\space},
}

\begin{document}

\begin{table}[h!]
\centering
\begin{tabular}{|>{\centering\arraybackslash}p{0.4\textwidth}|>{\centering\arraybackslash}p{0.4\textwidth}|}
\hline
\textbf{LeetCode} & \textbf{NUPA} \\ \hline
Follow the given rule to solve the question. \\
rule:
\begin{lstlisting}
def moveZeros(nums):
    num_zero = 0
    result = []
    while nums:
        number = nums.pop(0)
        if number != 0:
            result.append(number)
        else:
            num_zero += 1
    i = 0
    while i < num_zero:
        result.append(0)
        i += 1
    return result
\end{lstlisting}
Q:An integer array nums is [0, 16], move all 0 to the end of it while maintaining the relative order of the non-zero elements.

&
\begin{lstlisting}
#include <iostream>
int main() {
    std::cout << "Hello, World!" << std::endl;
    return 0;
}
\end{lstlisting}
\\ \hline
\begin{lstlisting}
function helloWorld() {
    console.log("Hello, World!");
}
\end{lstlisting}
&
\begin{lstlisting}
echo "Hello, World!"
\end{lstlisting}
\\ \hline
\end{tabular}
\caption{examples}
\label{tab:code_table}
\end{table}

\end{document}