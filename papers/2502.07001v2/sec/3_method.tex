\section{Method}
\label{sec:method}

We lead this section with a recap of latent diffusion models and the WALT model in \cref{sec:method:latentdiffusionmodels} and \cref{sec:method:walt}, respectively.
\Cref{sec:method:walt_variants} describes how WALT operates on images \vs video, followed by a description of the probing framework for quantitative evaluations in \cref{sec:method:probing_framework}.


\subsection{Latent Diffusion Models}
\label{sec:method:latentdiffusionmodels}

Diffusion Models \cite{sohldickstein2015deepunsupervisedlearningusing} are probabilistic generative models that can learn a distribution by denoising a normally distributed variable. They are based on two stages, a forward diffusion stage and a backward diffusion stage. In the forward diffusion stage, the input data is gradually corrupted by adding noise with a fixed variance schedule until the data distribution degrades into a standard Gaussian distribution.  In the backward stage, the model reconstructs the original input data by learning to gradually denoise the signal. 




Latent Diffusion Models (LDMs) \cite{blattmann2023stable}  apply the diffusion process in the latent space of a Vector Quantized Variational Autoencoder (VQ-VAE) \cite{NeuralDiscreteRepresentationLearning, TamingTransformersForHiResImageSynth} which helps to significantly reduce the computation requirements when compared to using raw pixels. The VQ-VAE is composed of an encoder $E(x)$ that encodes a video or an image ${x}\,{\in}\,\mathbb{R}^{T{\times}H{\times}W{\times}3}$ into a compressed latent representation $ z \in \mathbb{R}^{t\times h \times w \times c} $ and a decoder $D$ that reconstructs the input data from the latent $\widetilde{x}=D(z)$. 

The inverse diffusion process is modeled by applying a learnable function $f_\theta(z_t, t)$ to the noised latents at each step to recover the original input. More formally, $f_\theta(z_t) \approx \nabla \log p(z_t)$, where $p(z_t)$ is the probability density function of the latent space at step $t$,  $z_t = \sqrt{\alpha(t)}z_0 + \sqrt{1-\alpha(t)}\epsilon$ is a noisy version of $z_0$, $\epsilon  \sim  \mathcal{N}(\mathbf{0}, {\mathbf{I}})$, $t \in [0, 1]$, and $\alpha(t)$ is a predefined monotonically decreasing function from 1 to 0.

The function $f_\theta$ is parameterized by a neural network, which is trained with the denoising objective defined as 
\begin{equation*}
\mathbb{E}_{z{\sim}p_{data}, t\sim \mathcal{U}(0, 1), \epsilon \sim \mathcal{N}(\mathbf{0}, \mathbf{I})} \big[\|\epsilon - f_{\theta} (z_t; C, t)\|^2\big],
\end{equation*}
where $C$ is the condition, \eg, class labels or text prompts.

Beginning with a noise sample $z_1 \sim \mathcal{N}(\mathbf{0}, \mathbf{I})$, an iterative sampling process repeatedly applies the model $f_\theta(z_t; C, t)$ to progressively refine an estimate of a clean latent sample $\hat{z}_0$.  This refined latent sample is then decoded, transforming it back into the pixel space.
In this work, we do not make use of the decoder since we extract features from the main transformer module.

\subsection{Windowed-Attention Latent Transformer}
\label{sec:method:walt}

Windowed-Attention Latent Transformer (WALT) \cite{walt}, a transformer-based video diffusion model conditioned on text prompts, is selected for this study because the same architecture can be used for both image and video generation, leading to a fair comparison. Although, a study with a single diffusion model is not ideal, the same fair comparison is not feasible with open-sourced video diffusion models given the architecture size disparities with their image counterparts. For example, there is a significant architecture size disparity between Stable Image Diffusion 2.1 (SD 2.1)~\cite{rombach2022high} and its temporal extension, Stable Video Diffusion (SVD)~\cite{blattmann2023stable}. More precisely, SD 2.1 uses a U-Net architecture with 865 million parameters, while SVD  inserts temporal convolution and attention layers after every spatial convolution and attention layer of SD 2.1, which leads to an additional 656 million parameters.




%\vspace{-3mm}
\subsection{Multi-destination active message format}
\label{section:message_format}
\vspace{-0.7cm}
\begin{figure}[h!]
	\scriptsize
        \centering
    % \hspace{-1cm}
    \includegraphics[width=1\columnwidth]{diagrams/message_format.pdf}
    \vspace{-0.4cm}
	\caption{Message format} 
	\label{fig:message_format}
	\vspace{-.3cm}
\end{figure}
\textit{Nexus Machine} extends the fundamental Active Message primitives to accommodate a multi-destination based routing mechanism. 
Fig.~\ref{fig:message_format} illustrates the message format: the first 12 bits specify intermediate destinations (\textit{R1}, \textit{R2}, \textit{R3}), based on our workload analysis. 
The next 4 bits contain the Program Counter (PC) for the next instruction (\textit{N\_PC}), followed by 4 bits for the \textit{Opcode}. 
A single bit (\textit{Res\_c}) indicates if the message carries a result. 
The subsequent 2 bits (\textit{Op1\_c} and \textit{Op2\_c}) identify whether \textit{Op1} and \textit{Op2} are addresses or values. 
Depending on \textit{Res\_c}, the \textit{Result} field contains the final result or its address, while the next 16 bits hold data for Operand1 (\textit{Op1}) and Operand2 (\textit{Op2}).

When a message arrives at a router, the first destination (\textit{R1}) is processed by the \textit{Route Computation} logic and then allocated to the appropriate output port. After reaching \textit{R1}, the message is handled by the \textit{Input Network Interface}, and the remaining destinations are cyclically rotated, making \textit{R2} the first and \textit{R3} the second. 

In the \textit{Nexus Machine}, a message is equivalent to a packet or flit (all messages are a single-flit packet).
\begin{comment}
\begin{figure*}[h!]
	\scriptsize
	\centering
	\includegraphics[width=\textwidth]{diagrams/architecture.pdf}
	\caption{\textit{Nexus Machine} microarchitecture. \textit{Nexus Machine} is a fabric of homogenous PEs interconnected by a mesh network for communicating Active messages, enhancing fabric utilization by executing messages en-route.} 
	\label{fig:detail_arch}
	%\vspace{-.5cm}
\end{figure*}
\end{comment}
%\vspace{-3mm}
\subsection{Nexus Machine Micro-architecture}
\begin{figure*}[h!]
	\scriptsize
	\centering
	\includegraphics[width=0.9\textwidth]{diagrams/architecture.pdf}
    \vspace{-.15cm}
	\caption{\textit{Nexus Machine} microarchitecture. A fabric of homogenous PEs interconnected by a mesh network for communicating Active Messages which carry instructions that can be launched en-route at any PE, enhancing fabric utilization and runtime.} 
    \vspace{-0.3cm}
 %\color{red}{\bf Peh: I suggest replacing (d) with one of the Compute Unit, cos it's a major component of Nexus and yet do not feature in any figure. We need to highlight to reviewers that our compute unit consists of ALU :)}} 
	\label{fig:detail_arch}
	%\vspace{-.5cm}
\end{figure*}
%\subsubsection{Top Level}
As presented in Fig.~\ref{fig:detail_arch}(a), the \textit{Nexus Machine}'s fabric comprises homogeneous processing elements (PEs) interconnected with a mesh network, with a global termination detector. Each PE is linked to four neighboring PEs in North, East, South, and West directions.
The off-chip memory is connected to the four PEs located along the left edge.


\subsubsection{Processing Elements (PEs).}
%As presented in figure~\ref{fig:detail_arch}(b), each PE combines a compute unit, a dynamic router for network connectivity with congestion control, a decode unit with local data memory, an Input Network Interface which contains an instruction memory for handling incoming AMs and an AM Network Interface unit for spawning new AMs. \\
As presented in Fig.~\ref{fig:detail_arch}(b), each PE combines a compute unit, a dynamic router for network connectivity with congestion control, a decode unit, and two Network Interface logic.
Specifically, \textit{Input Network Interface} unit is responsible for efficiently handling incoming AMs from the NoC, while the AM Network Interface unit initiates the injection of new messages into the NoC.

\textbf{Input Network Interface.}
%The Input Network Interface logic triggers the loading of subsequent instruction on AM arrival.
%The arrival of a Decode AM triggers loading of the data element 
The \textit{Input Network Interface} unit manages \textit{incoming AMs} to a PE.
Depending on the message, \\%it performs either of these two operations.\\
%(a) It either updates the instruction contained in the message based on the next Program Counter (N\_PC) value provided within the message body.\\
(a) If it pertains to an ALU operation, it is directed to the \textit{Compute Unit} for execution.\\
(b) Alternatively, in case of a memory operation, the message is forwarded to the \textit{Decode} unit. 
This unit initiates a load or store operation, utilizing the operand address information (\textit{Op1} or \textit{Op2}) contained in the message.\\
Once these operations are completed, the resulting \textit{output dynamic AM} is dispatched to the \textit{AM Network Interface} for injecting into the network.
%The message enters the network via the local input port, which feeds the \textit{Compute} unit.

\textbf{Compute Unit.}
The \textit{compute unit} within a PE can perform 16-bit arithmetic operations, logic operations, multiplication, and division on its ALU.

An incoming AM at the \textit{Input Network Interface} dispatches two operands, \textit{Op1} and \textit{Op2} along with the \textit{Opcode} field in the message to the compute unit.
After computation, it generates an output that is combined with the original AM, replacing the \textit{Op1} field in the message.
Finally, this modified AM is forwarded to the \textit{AM Network Interface} for injecting into the network.

%{\bf Peh: There needs to be detailed information on how an AM launches computation! This is the thesis of AM! For instance, what's the format of the AM, when it's received, which field is used to configure the ALU? How is the PC set? What happens in the beginning of execution? read config memory? are there registers? what happens if data operand is not present -- can that happen? stall? Lots of details needed here.}

\textbf{Decode Unit.}
The \textit{Decode Unit}, as shown in Fig.~\ref{fig:detail_arch}(e), can be flexibly configured to operate in dereference and streaming modes.
In \textbf{dereference mode}, the operand address field (\textit{Op1} or \textit{Op2}) in the message triggers the loading of a single element. This gets embedded into the output \textit{dynamic AM}.
Conversely, in \textbf{streaming mode}, the message initiates the loading of multiple elements from memory, generating multiple output AMs.
In this mode, the operand address is considered the base address, along with a count to access and load the elements from memory sequentially.
These two modes suffice for our benchmarks; however, our architecture allows for integration of additional modes if needed.

\textbf{Active Message (AM) Network Interface.}
The \textit{AM Network Interface logic} is responsible for injecting AMs into the network.
%The AM Network Interface logic consists of an AM Queue and a configuration memory.
%The AM Queue is a 1KB FIFO, initialized with 44-bit precompiled entries.
%The configuration memory is 16-bit wide, containing 8 configurations.
This module comprises two primary components: an \textit{AM Queue} and a \textit{configuration memory}. 
The \textit{AM Queue} is a 16KB FIFO initialized with 70-bit precompiled entries. 
The \textit{configuration memory}, 10-bit wide, accommodates 8 distinct configurations.

%Depending on the availability of the output dynamic AM from the \textit{Input Network Interface}, it either
It either performs these two operations, as shown in Fig.~\ref{fig:detail_arch}(b):
(1) If the output \textit{dynamic AM} is available from \textit{Input Network Interface}, the subsequent configuration is loaded from memory based on the \textit{N\_PC} field of the AM (see Fig.~\ref{fig:message_format}). 
This configuration is combined with the output \textit{dynamic AM} and forwarded into the injection port of the router.\\
(2) Alternatively, a \textit{static AM} is injected into the network to keep it occupied. 
This \textit{static AM} is the concatenation of the next precompiled entry from the \textit{AM Queue} with the first configuration loaded from memory.
The generation rate of \textit{static AMs} is determined by the backpressure signal at the router's injection port.

The highlighted blue fields in the message format (see Fig.~\ref{fig:message_format}) depict data from the configuration memory used to construct the subsequent dynamic AM, with fields \textit{Res\_c}, \textit{Op1\_c}, and \textit{Op2\_c} stored to prevent redundancy.
%The AM Network Interface logic consists of a 1KB AM Queue, a FIFO containing 44-bit pre-compiled entries.
%, alongside a 13-bit wide configuration register. 

%As shown in Figure~\ref{fig:detail_arch}(e), the output dynamic AMs from \textit{Input Network Interface} trigger loading the next subsequent configuration from the memory with the N\_PC field of the AM.
%These are further concatenated with the output dynamic AMs and pushed into the injection port of the router.
%The injection rate is managed by the backpressure signal at the injection port of the router.

%To keep the network occupied, static AMs are containing the first precompiled entry from AM queue
%As shown in Figure~\ref{fig:detail_arch}(e), it concatenates an AM Queue entry with the first configuration loaded from the memory to generate a static AM, which is subsequently pushed intothe injection port of the router. 

\subsubsection{Dynamic and Congestion Aware Routing.}
\textit{Nexus Machine} supports turn model routing~\cite{noc_peh}, with each router containing five input and five output ports.
%Specifically, these input ports correspond to AM, local, north, east, south and west, whereas output ports correspond to local and four directions.
Specifically, these input ports are designated for messages coming from \textit{AM Network Interface} unit, as well as north, east, south, and west directions, whereas output ports are designated for messages going to \textit{Input Network Interface} unit and four directions.
%The AM input port receives recently generated messages from the \textit{AM Network Interface unit}, while the local port handles messages coming from the \textit{Input Network Interface}. 
Each input port has a buffer comprising three registers to manage in-flight messages, accompanied by congestion control logic. \textit{Nexus Machine}'s design choice of employing only three registers is motivated by the goal of minimizing overall power consumption.

As presented in Fig.~\ref{fig:detail_arch}(c), each router contains a Route Computation Unit, Separable Allocator, and a Crossbar.

\textbf{Route Computation} logic considers the destination of messages from all the input ports. It compares it with the positional ID of the PE, and calculates the output port to be requested. This is sent as an input to the allocator.

%{\bf Peh: Separable alllocation is a well-known previously proposed technique... so there's no need to elaborate... just cite a NoCs textbook}
A toy example of \textbf{Separable Allocation} process is presented in Fig.~\ref{fig:detail_arch}(d)~\cite{noc_peh}.
\iffalse
The request matrix's rows correspond to input ports, and columns correspond to output ports.
The process consists of two stages of 6:1 and 5:1 fixed priority arbiters. The first stage prunes the matrix to ensure that each output port (or resource) receives requests from at most one input port (or requestor). Subsequently, the backpressure signal is applied to each output port, enabling congestion control, as explained below. The second stage further prunes the matrix to guarantee one grant per input port.
The allocator executes within a single cycle, marking granted requests as issued immediately to prevent them from bidding again.
\fi

\textbf{On/Off Congestion control} involves the transmission of a signal to the upstream router when the count of available buffers falls below a threshold, ensuring all in-flight messages will have buffers on arrival. Each of the five ports transmits an OFF signal when their corresponding available buffer space is reduced to 1, i.e., $T_{OFF} = 1$, and conversely, an ON signal when their buffer space reaches 2, i.e., $T_{ON} = 2$.

The output of the allocator is sent to a 6x5 \textbf{Crossbar}.
%, which forms many-to-many connections among internal and external datapaths.\\ Peh: A crossbar by definition forms many-to-many connections between its input and output ports, so no need to explain. 

\subsubsection{Off-chip Memory Datapath.}
Each off-chip memory port connects to a row of the PE array via an AXI bus, delivering a combined bandwidth of 1.28GBps. During data loading, data transfers from off-chip memory to the \textit{AM queues} and \textit{data memory} in each PE. 
The \textit{AM queues} are actively consumed during execution, effectively hiding data loading latency by performing it concurrently with the execution. 
However, data loading into \textit{data memories} occurs after tile execution is complete.

\subsubsection{Bit-vector Scanners.}
The first sparse operand is encoded in \textit{static AMs}. For subsequent sparse operands, bit-vector scanner hardware assists in efficient iteration, providing coordinates within compressed vectors as described in \cite{capstan}. \textit{Nexus Machine} integrates a modified version of this with its AXI bus controller to obtain these coordinates. It can vectorize 16 non-zeros within 128 elements, allowing it to handle matrices with densities exceeding 12\%.
\vspace{-0.2cm}
\subsection{Deadlock avoidance}
%\textcolor{blue}{\bf Peh: Write up a short blurb on how Nexus machine addresses various deadlock scenarios and explain design choice: (1) Within network: Flow control deadlocks addressed by bubble buffer [cite bubble flow control] instead of VCs so as to minimize buffering; Routing deadlocks addressed by turn model, so as to provide high throughput without complex adaptive routing hardware; (2) AM also introduces potential network-PE deadlocks -- addressed by compiler preventing such cyclic dependencies + runtime timeouts} 

Given the dynamic nature, \textit{Nexus Machine} can potentially encounter deadlock without careful design. We address various deadlock scenarios with these specific design choices: 
(1) To mitigate flow control deadlocks within the network, we adopt the bubble NoC~\cite{bubble_flow} approach over Virtual Channels (VCs), with the aim to minimize buffering. 
(2) Routing deadlocks are mitigated by using the turn model~\cite{noc_peh}, that ensures high throughput without the need for complex adaptive routing hardware.
(3) AMs can potentially create deadlocks between the network and PEs. 
These are effectively mitigated by the compiler through strategic data placement and runtime timeouts.
%\textit{Nexus Machine} currently uses a simple heuristic for data placement strategy within the compiler.
%\textcolor{red}{\bf Peh: Is the simple heuristic related to deadlocks? Cos the above sentence seems to contradict the earlier sentence. Elaborate on the heuristic and timeouts??}
Future research will explore more optimized data placement strategies.

WALT is used as a frozen backbone to train light-weight readout heads for downstream perception tasks. It leverages a causal 3D CNN encoder of the MAGVIT-v2 tokenizer \cite{magvitv2} to jointly compress images and videos into a shared latent space, which allow the model to be trained on massive image-text and video-text datasets.



The input to the model is a batch of latent tensors $ z \in \mathbb{R}^{(1+m)\times h \times w \times c} $, generated by the 3D CNN encoder, which are first passed through an embedding block, referred to as Block 0, to be independently encoded as a sequence of non-overlapping patches along with learned position embeddings \cite{AttentionIsAllYouNeed}.
The first frame is encoded independently from the remaining video frames, allowing static images to be treated as videos with a single frame. In particular, the WALT checkpoints used in this paper were shared by the authors and were trained to process sequences of 17 frames.  The video frames are tokenized by the 3D CNN encoder with 5 temporal latents, where the first latent represent the initial frame and the remaining $m=4$ represent the remaining 16 frames. In terms of the spatial compression factor of the latents, it is set to 8 for both width and height.

The WALT architecture introduces a design variation for the transformer in order to reduce the high compute and memory cost of processing image and video tokens. The variation consists of computing self-attention in windows instead of using the traditional global self-attention modules. More precisely, the transformer consists of a concatenation of $L$ window-restricted attention blocks, alternating between spatio-temporal-window blocks and spatial-window blocks. For the case of images, the spatio-temporal-window blocks use an identity attention mask, ensuring that any given latent only attend to itself. This architectural choice enables joint training, where the spatial blocks independently process images and video frames, while the spatio-temporal blocks model motion and temporal dynamics in videos. Also, in terms of inference, it enables both image and video generation modes. 

The WALT model is trained on conditional information such as text embeddings, previously generated samples, and past frames for auto-regressive generation of videos.
While the original WALT model is a cascaded diffusion model with super-resolution stages, we only investigate the base model that generates videos at low resolution in this work.


\subsection{WALT for images and video}
\label{sec:method:walt_variants}
WALT can be used in both video generation, referred to as \vwalt, and image generation modes. When used for image generation, any given latent in the spatio-temporal blocks only attend to itself. For comparison purposes, one drawback of such design is that not only the temporal attention is removed but also the spatial attention. Given that the spatio-temporal blocks in \vwalt perform both spatial and temporal attention, it is fairer to compare with an image counterpart of the WALT model where window-restricted spatio-temporal attention blocks are replaced by window-restricted spatial attention blocks of the same number of parameters. In that way, we can directly measure the impact of adding temporal attention. Such image counterpart was trained for this paper and is referred to as \iwalt. Note that the window-restricted spatial and spatio-temporal blocks only differ in the windows sizes, and therefore, \iwalt and \vwalt share the same architecture.

For training \vwalt, an internal dataset composed of images and videos was used. The same dataset was employed for training \iwalt, but instead of using full videos, frames were randomly extracted from each video. The training settings of \iwalt are the same as WALT.

\subsection{Probing Framework}
\label{sec:method:probing_framework}


A probing framework is introduced to extract video representations from the WALT model and subsequently apply a task-specific readout head for various video understanding tasks. The process starts by adding noise to the latent representations of the input data at a time step $t$ to simulate the forward diffusion process before passing them through the denoiser model. Only a single forward pass of the input through the diffusion model is necessary to extract its visual representation, as opposed to going through the entire multi-step generative diffusion process. The forward pass uses a null text embedding. Subsequently, activations of the transformer intermediate blocks are extracted to train task-specific readout heads that will interpret the learned representations. A diagram illustrating the probing framework is shown in \cref{fig:architecture}.
In order to determine an adequate timestep $t$ and the most representative activation block $l$, we run ablations described in \cref{sec:exp:featureextraction} and showcased in \cref{fig:noise_and_blocks}.



The WALT model is evaluated on representative visual perception tasks, ranging from pure semantics to spatio-temporal understanding: image classification, action recognition, monocular depth estimation, relative camera pose prediction to visual correspondence.
For the evaluation on visual perception tasks, we largely follow the probing methodology of \citet{s4dpaper}.



\paragraph{Image classification}
Image classification, characterized by its purely semantic nature, is one of the most fundamental areas in computer vision. Within this area, the tasks of object classification, scene recognition and fine-grained visual classification are selected for downstream evaluation of WALT. 
An attentive readout is used for this task \cite{vjepa}. The cross-attention layers are trained with a learnable query token, and the output of the cross-attention is added to the query token and then passed to a two-layer MLP with GeLU activation, followed by layer normalization and a linear classification layer.
The readout is trained with the softmax cross-entropy loss.
\Timagenet~\cite{imagenet} and \Tplaces~\cite{places365} are used for object and scene classification, respectively. For fine-grained visual classification we use iNaturalist 2018 (\Tinat)~\cite{inaturalist} which contains visually similar plant and animal species.
Top-1 classification accuracy is used for all the image classification tasks.

\paragraph{Action recognition}
Understanding actions in videos often requires capturing temporal dependencies between frames, \ie the model needs to understand how actions unfold over time and how earlier frames relate to later ones to accurately classify the action.
As above, we use attentive readout, though over all video frames here.
Kinetics-400 and 700 (\Tkfour, \Tkseven)~\cite{kinetics400, k700} are used for appearance-focused action recognition.
For more motion-sensitive action recognition, Something-Something v2 (\Tssv)~\cite{ssv2} is used.
We report top-1 classification accuracy.


\paragraph{Monocular depth prediction}
Monocular depth estimation, referred hereafter as \Tscannet, is a 3D perception task that aims at predicting the distance of surface elements in the scene from the camera. Unlike traditional geometric correspondence and triangulation techniques, this requires only a single image. However, it can also be calculated from video to leverage temporal dependencies between frames. Monocular depth estimation is a fundamental problem in computer vision as it bridges the gap between 2D images and the 3D world.
For the readout, we use the decoder of the Scene Representation Transformer~\cite{SRT} which is composed of a small number
of cross-attention layers followed by an MLP.
Fourier positional encoding~\cite{nerf} applied to the input latents is used to generate a set of queries for the decoder. Each depth pixel value is decoded independently by a transformer that crossattends from a query into the latent features generated by the pretrained model, thereby aggregating relevant information from the latents to predict depth. For training the readout, an L2 loss between the prediction and the ground truth depth map is used. 
We use ScanNet~\cite{scannet}, a dataset of RGB-D videos of indoor scenes.
The mean of the absolute relative error (AbsRelErr)~\cite{AbsRelErr} between predicted and ground-truth depth is used for evaluation.

\paragraph{Relative camera pose estimation}
Relative camera pose estimation (\Tpose) is about predicting the relative 6D camera poses between the first and last frames of a video sequence. The pose matrix is defined as $P = [R, t]$, where $R$ and $t$ denote the rotation matrix and translation vector, respectively.
The attention readout for action recognition is also utilized for this task. Since the predicted rotation matrix may not be a true rotation matrix in SO(3), the Procrustes algorithm~\cite{bregier2021deep} is applied to the predicted matrix to find the closest true rotation matrix. The readout is trained by minimizing the L2-loss between predicted and ground-truth pose matrices.

We use the RealEstate10k dataset~\cite{zhou2018stereo} which is comprised of indoor and outdoor property videos. The pose annotations are derived from a traditional SfM pipeline, so we rescale camera poses to metric units in order to address scale ambiguities \cite{watson2024controlling}.
The estimated poses are evaluated using mean end-point-error, a metric that measures the mean distance between ground-truth ($P_i$) and estimated ($\hat{P}_i$) pose matrices. More formally, $e_{\text{EPE}}(\hat{P}_i, P_i) = \frac{1}{M}\sum_{j=1}^M \| P_i(Y_j) - \hat{P}_i(Y_j) \|$, where, $\{Y_j\}_{j=1,\dots,M}$ is a set of 3D points selected for metric calculation. In this study, 8 auxiliary points, forming a virtual cube in front of the camera of the first frame, are used for computing $e_{\text{EPE}}$.

\paragraph{Visual correspondence tasks}
Visual correspondence is at the heart of video understanding, as it requires modeling how physical surfaces move and deform over time. In this paper, two correspondence tasks, namely point tracking and box tracking, are selected for evaluation and referred hereafter as \Tpt and \Twaymo, respectively.


The same readout head proposed in MooG \cite{moog} is adopted. Given a set of initial $N$ points (or boxes) at time $t=1$, $q_1{\in}\,\mathbb{R}^{N{\times}D_q}$, and a sequence of observed frames $\{X_t\}_{t=1}^T$, the goal is to predict all future targets, $\{q_t\}$ for $t = 2, \ldots, T$. A latent representation is assigned to $q_1$ by first encoding it with positional encoding followed by an MLP. Then, the latents for $t = 2, \ldots, T$ are generated in a recurrent fashion. At step $t$, they are first predicted by an MLP-based predictor using only the corrected latents at time $t-1$, and then, they are corrected by a transformer, in which the frames are encoded and cross-attended to using the latent predictions as queries to produce the corrections. To generate the final target values, an MLP head is applied to the latents $y_t$. 

The final targets for \Tpt are normalized image coordinates, visibility, and prediction certainty.  A point is considered visible during evaluation only if the model predicts it is visible and has over 50\% confidence in its location. Following MooG \cite{moog}, we use a combined loss function, which includes a Huber loss for location accuracy and Sigmoid Binary Cross Entropy losses for visibility and certainty.  For points that are no longer in the scene, only the visibility loss is applied.
For training the box tracking readout, an L2 loss between the prediction and the normalized box coordinates is used.

We train the point tracking readout head on Kubric {MOVi-E}~\cite{movi-e} labeled with point annotations computed in a similar manner as in~\cite{moog}.
For evaluation, the Perception Test dataset~\cite{perception-test} is used. Sixty four points per frame are sampled and the location of each point in the first frame is used as the query. The average Jaccard (AJ) as in~\cite{moog}, which evaluates both occlusion and position accuracy, is used as performance metric for \Tpt. 
The Waymo Open dataset~\cite{waymo} is used for both training and evaluation of the box tracking readouts and the average IoU (excluding the first frame in the sequence for which ground truth is provided) is used as performance metric.









