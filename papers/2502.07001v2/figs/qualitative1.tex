
\begin{figure}[t]
\centering
\makebox[\linewidth]{
    \newcommand\mypic[1]{
    \includegraphics[width=0.15\linewidth]{imgs/qualitative/pca_one_channel_magma/#1.png}
    }
    \setlength{\tabcolsep}{1pt}
    \begin{tabular}{rcccccc}
        \rotatebox[origin=c]{90}{\hspace{9mm} \footnotesize \centering Frame}
        \mypic{2} &
        \mypic{3} &
        \mypic{4} &
        \mypic{5} &
        \mypic{6} &
        \mypic{9} \\[-7mm]
        \rotatebox[origin=c]{90}{\hspace{9mm} \footnotesize \centering \iwalt}
        \mypic{2-iwalt} &
        \mypic{3-iwalt} &
        \mypic{4-iwalt} &
        \mypic{5-iwalt} &
        \mypic{6-iwalt} &
        \mypic{9-iwalt} \\[-8mm]
        \rotatebox[origin=c]{90}{\hspace{9mm} \footnotesize \centering \vwalt}
        \mypic{2-vwalt} & 
        \mypic{3-vwalt} & 
        \mypic{4-vwalt} & 
        \mypic{5-vwalt} & 
        \mypic{6-vwalt} &
        \mypic{9-vwalt} \\[-9mm]
        \rotatebox[origin=c]{90}{\hspace{9mm} \footnotesize \centering Flow}
        \mypic{2-flow} &
        \mypic{3-flow} &
        \mypic{4-flow} &
        \mypic{5-flow} &
        \mypic{6-flow} &
        \mypic{9-flow} \\[-6mm]
    \end{tabular}
}
\setlength{\belowcaptionskip}{-5pt}
\caption{
\textbf{Feature visualization} --
We show the major PCA component for the two models across a range of DAVIS videos.
While \iwalt is sensitive to semantically important areas of the scene (\eg, \emph{all} people in the second column), \vwalt is much more sensitive to the areas that experience motion within the video (\eg, only the wrestlers in the same video).
}
\label{fig:qualitative1}
\end{figure}
