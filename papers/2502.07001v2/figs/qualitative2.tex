\begin{figure}[t]
\centering
\newcommand\mypic[1]{
\includegraphics[width=0.18\linewidth]{imgs/qualitative/motion/#1.png}
}
\setlength{\tabcolsep}{0.5pt}
\begin{tabular}{ccccc}
    \mypic{brick0} & 
    \mypic{brick_middle} &
    \mypic{brick_last} &
    \mypic{brick-iwalt} &
    \mypic{brick-vwalt} \\[1mm]
    \mypic{brick1} &
    \mypic{brick1_middle} &
    \mypic{brick1_last} &
    \mypic{brick-iwalt} &
    \mypic{brick1-vwalt} \\[1mm]
    \mypic{brick2} & 
    \mypic{brick2_middle} &
    \mypic{brick2_last} &
    \mypic{brick-iwalt} &
    \mypic{brick2-vwalt} \\[1mm]
    \mypic{brick3} & 
    \mypic{brick3_middle} &
    \mypic{brick3_last} &
    \mypic{brick-iwalt} &
    \mypic{brick3-vwalt} \\[1mm]
    \cmidrule(lr){1-3} \cmidrule(lr){4-4} \cmidrule(lr){5-5} \\[-4mm]
    \multicolumn{3}{c}{Video frames} & \iwalt& \vwalt\\
\end{tabular}
\vspace{-1mm}
\caption{
\textbf{Feature visualization for different motions} --
In the 4 brick videos, only the marked portion (highlighted in red) is played, while the rest remains frozen. We visualize tokens from the first, identical frame. As an image model, \iwalt consistently produces the same feature, while \vwalt shows high sensitivity to moving areas, reflected in the major principal component.
}
\vspace{-4mm}
\label{fig:qualitative2}
\end{figure}
