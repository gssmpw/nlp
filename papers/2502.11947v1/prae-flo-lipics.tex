\usepackage[babel]{csquotes}
\usepackage{amssymb,amsmath,amsxtra,mathrsfs,mathtools,bm,stmaryrd}
\usepackage{booktabs}
\usepackage{multirow}
\usepackage{nicefrac}
\usepackage{colonequals}
\usepackage{xspace}
\usepackage{dsfont}
\usepackage{tikz}
\usetikzlibrary{arrows,arrows.meta,automata,backgrounds,calc,decorations.markings,decorations.pathreplacing,decorations.pathmorphing,fit,math,positioning,shapes,shapes.geometric,shapes.callouts,shapes.misc}
\usepackage{ifthen}
\usepackage[notref,notcite]{showkeys}
\usepackage{seqsplit}
\usepackage{xstring}
\usepackage[noadjust]{cite}
\usepackage{rotating}
\usepackage{tikz-cd}
\usepackage{blkarray}
\usepackage{adjustbox}
\usepackage{calc}
\usepackage{dutchcal}
\usepackage{etoolbox}
\usepackage{relsize}
\usepackage{microtype}
\usepackage[footnote,marginclue,nomargin,final]{fixme}

\tikzcdset{scale cd/.style={every label/.append style={scale=#1},
    cells={nodes={scale=#1}}}}

\allowdisplaybreaks


\declaretheorem[name=Definition,style=definition,numberwithin=section,sibling=definition]{defn}
\declaretheorem[name=Example,style=definition,sibling=defn]{expl}
\declaretheorem[name=Examples,style=definition,sibling=defn]{examples}
\declaretheorem[name=Observation,style=definition,sibling=defn]{obs}
\declaretheorem[name=Remark,style=definition,sibling=defn]{rem}
\declaretheorem[name=Assumptions,style=definition,sibling=defn]{assumptions}
\declaretheorem[name=Assumption,style=definition,sibling=defn]{assumption}
\declaretheorem[name=Algorithm,style=definition,sibling=defn]{algo}
\declaretheorem[name=Notation,style=definition,sibling=defn]{notation}
\declaretheorem[name=Convention,style=definition,sibling=defn]{convention}
\declaretheorem[name=Construction,style=definition,sibling=defn]{construction}
\declaretheorem[name=Theorem,style=definition,sibling=defn]{theo}
\declaretheorem[name=Research Question,style=definition,sibling=defn,numbered=no]{question}
\declaretheorem[sibling=defn]{cor}
\declaretheorem[name=Fact,style=definition,sibling=defn]{fact}
\declaretheorem[name=Lemma,style=definition,sibling=defn]{lem}
\declaretheorem[name=Proposition,style=definition,sibling=defn]{prop}
\declaretheorem[name=Proof Sketch,style=definition,sibling=defn]{proofsketch}
\declaretheorem[numbered=no,name=Theorem]{nsd}
\declaretheorem[numbered=no,name=Theorem]{reit}

\newcommand{\resetCurThmBraces}{%
\gdef\curThmBraceOpen{(}%
\gdef\curThmBraceClose{)}}
\resetCurThmBraces
\newcommand{\removeThmBraces}{%
\gdef\curThmBraceOpen{}%
\gdef\curThmBraceClose{}}
\resetCurThmBraces

\newenvironment{notheorembrackets}{\removeThmBraces}{\resetCurThmBraces}
\patchcmd{\thmhead}{(#3)}{\curThmBraceOpen #3\curThmBraceClose }{}{}

\setcounter{tocdepth}{2}

\newcommand{\defaultshowkeysformat}[1]{%
\StrSubstitute{#1}{ }{\textvisiblespace}[\TEMP]%
\parbox[t]{\marginparwidth}{\raggedright\normalfont\small\ttfamily\(\{\){\color{red!50!black}\expandafter\seqsplit\expandafter{\TEMP}}\(\}\)}%
}
\newenvironment{hideshowkeys}{%
    \renewcommand*\showkeyslabelformat[1]{%
        \noexpandarg%
    }
}{%
    \renewcommand*\showkeyslabelformat[1]{%
        \noexpandarg%
        \defaultshowkeysformat{##1}%
    }
}

\renewcommand*\showkeyslabelformat[1]{%
\noexpandarg%
\defaultshowkeysformat{#1}%
}

\renewcommand\itemautorefname{Item}
\newcommand{\itemref}[2]{\autoref{#1}.\ref{#2}}

\newcommand{\mypar}[1]{%
    \subparagraph*{#1.}%
}


\newcommand\restr[2]{{%
  \left.\kern-\nulldelimiterspace %
  #1 %
  \littletaller %
  \right|{}^{#2} %
}}

\DeclareMathOperator{\norb}{\sharp_o}
\DeclareMathOperator{\Sub}{\mathsf{Sub}}
\DeclareMathOperator{\dom}{\textsf{dom}}
\DeclareMathOperator{\supp}{\mathsf{supp}}
\DeclareMathOperator{\repr}{\textsf{repr}}
\DeclareMathOperator{\im}{\textsf{im}}
\DeclareMathOperator{\coim}{\textsf{coim}}
\DeclareMathOperator{\alphaequiv}{\equiv_{\alpha}}
\DeclareMathOperator{\alphanequiv}{{\not\equiv}_{\alpha}}
\DeclareMathOperator{\fix}{\textsf{fix}}
\DeclareMathOperator{\Fix}{\textsf{Fix}}
\DeclareMathOperator{\isf}{\trianglelefteqslant}

\newcommand{\littletaller}{\mathchoice{\vphantom{\big|}}{}{}{}}

\newcommand{\seq}{\subseteq}
\newcommand{\qes}{\supseteq}
\newcommand{\fseq}{\seq_{\textsf{f}}}
\newcommand{\xra}[1]{\xrightarrow{#1}}
\newcommand{\xla}[1]{\xleftarrow{#1}}
\newcommand{\xto}[1]{\xra{#1}}

\newcommand{\mar}{\ar[rightarrowtail]}
\newcommand{\ear}{\ar[two heads]}

\newcommand{\deriv}[2]{{#1}^{-1}#2}
\newcommand{\Der}{\mathsf{Der}}

\newcommand{\muBar}{\textsf{Bar-$\mu$TL}\xspace}

\newcommand{\smooth}{smooth\xspace}
\newcommand{\opcit}[1][.]{\textit{op.cit#1}\xspace}

\newcommand{\At}{\mathds{A}}
\newcommand{\Sigmas}{\Sigma^{\star}}
\newcommand{\Ats}{\At^{\!\raisebox{1pt}{\scriptsize$\star$}}}
\newcommand{\Atw}{\At^{\omega}}
\newcommand{\barAs}{\barNames^{\star}}
\newcommand{\barAw}{\barNames^{\omega}}
\newcommand{\barAwfs}{\barNames^{\omega}_{\textsf{fs}}}
\newcommand{\ol}{\overline}
\def\ul{\underline}

\newcommand{\bars}{\mathsf{bs}}

\newcommand{\sem}[1]{\llbracket #1 \rrbracket}

\newcommand{\op}[1]{\operatorname{\mathsf{#1}}}
\newcommand{\id}{\op{id}}

\newcommand{\inj}{\op{in}}
\newcommand{\inl}{\op{inl}}
\newcommand{\inr}{\op{inr}}
\newcommand{\pr}{\op{pr}}
\newcommand{\kan}{\op{kan}}
\newcommand{\cl}{\op{cl}}
\newcommand{\nd}{\op{nd}}
\newcommand{\rest}{\op{rest}}
\newcommand{\gdw}{\Leftrightarrow}
\newcommand{\lgdw}{\Longleftrightarrow}

\newcommand{\Jir}{\mathsf{JI}}

\newcommand{\card}[1]{\left\vert#1\right\vert}
\let\abs\card

\newcommand{\formu}{\textsf{form}}

\newcommand{\SUC}{\mathfrak{N}}

\newcommand{\states}{\mathsf{s}}
\newcommand{\al}{\mathsf{a}}
\newcommand{\trans}{\mathsf{t}}
\newcommand{\init}{\mathsf{i}}
\newcommand{\final}{\mathsf{f}}

\newcommand{\A}{\mathcal{A}}
\renewcommand{\H}{\mathcal{H}}

\newcommand{\can}[2]{\mathsf{can}_{#1}^{#2}}

\newcommand{\canmap}{\mathsf{can}}

\newcommand{\allbool}{\mathcal{B}}
\newcommand{\posbool}{\allbool_+}
\newcommand{\pos}[1]{{#1}^{+}}
\newcommand{\negsym}{\mathsf{n}}
\newcommand{\negstate}[1]{#1_{\negsym}}
\newcommand{\negbool}{\allbool_{\negsym}}
\newcommand{\dualbool}{\allbool_\mathsf{d}}
\newcommand{\dual}[1]{{#1}^\mathsf{d}}
\newcommand{\graph}{\mathcal{G}}
\newcommand{\tree}{\mathcal{T}}
\newcommand{\reachtree}[3]{\mathcal{R}_{#1}^{#2}(#3)}

\newcommand{\teach}{\textsf{T}\xspace}
\newcommand{\learn}{\textsf{L}\xspace}
\newcommand{\tass}{\textsf{TA}\xspace}
\newcommand{\corproc}{\textsf{CoR}\xspace}

\newcommand{\bartree}[1][\barNamess]{\mathcal{T}_{#1}}
\newcommand{\datatree}[1][\names]{\mathcal{T}_{#1}}

\newcommand{\rsem}{\vDash^{\textsf{r}}}

\newcommand{\cat}[1]{\mathscr{#1}}
\def\A{\cat A}
\def\B{\cat B}
\def\C{\cat C}
\def\D{\cat D}
\newcommand{\E}{\mathcal{E}}
\newcommand{\II}{\mathbb{I}}
\newcommand{\FF}{\mathbb{F}}
\newcommand{\MM}{\mathcal{M}}
\newcommand{\Set}{\mathbf{Set}}
\newcommand{\Pos}{\mathsf{Pos}}
\newcommand{\Pfn}{\mathsf{Pfn}}
\newcommand{\Gra}{\mathsf{Gra}}
\newcommand{\KVec}{K\text{-}\mathsf{Vec}}
\newcommand{\CPO}{\mathsf{CPO}}
\newcommand{\DCPO}{\mathsf{DCPO}}
\newcommand{\DCPOb}{\mathsf{DCPO}_\bot}
\newcommand{\CMS}{\mathsf{CMS}}
\newcommand{\MS}{\mathsf{MS}}
\newcommand{\Ord}{\mathsf{Ord}}
\newcommand{\Presheaf}[1]{\Set^{#1}}

\newcommand{\Pow}{\mathcal{P}}
\newcommand{\Powf}{\Pow_\mathsf{f}} %

\newcommand{\M}{\mathcal{M}}

\newcommand{\NAut}{\mathbf{NAut}}
\newcommand{\NAutfp}{\mathbf{NAut}_{\mathsf{fp}}}

\newcommand{\N}{\mathds{N}}
\newcommand{\R}{\mathds{R}}
\newcommand{\Z}{\mathds{Z}}
\newcommand{\fpair}[1]{\ensuremath{\langle #1 \rangle}}

\renewcommand{\epsilon}{\varepsilon}

\newcommand{\Coalg}{\mathop{\mathsf{Coalg}}}
\newcommand{\Alg}{\mathop{\mathsf{Alg}}}
\newcommand{\colim}{\mathop{\mathsf{colim}}}

\def\variety{variety\xspace}
\def\varieties{varieties\xspace}
\def\eqnth{equational theory\xspace}
\def\eqnths{equational theories\xspace}
\def\Eq{\mathbb{E}}

\renewcommand{\S}{\mathscr{S}}
\renewcommand{\P}{\mathbb{P}}
\newcommand{\K}{\mathds{K}}

\newcommand{\reg}{{\mathrm{reg}}}

\newcommand\epidownarrow{\mathrel{\rotatebox[origin=c]{90}{$\twoheadleftarrow$}}}

\newcommand{\Nom}{\mathbf{Nom}}
\newcommand{\RnNom}{\mathbf{RnNom}}
\newcommand{\orb}{\mathsf{orb}}
\newcommand{\Perm}{\mathsf{Perm}}
\newcommand{\names}{\At}
\newcommand{\Abstr}[2][\names]{[#1]#2}
\newcommand{\datalang}{\powfs(\names^*)}
\newcommand{\parnom}[1]{\names^{\$\mathbf{#1}}}
\newcommand{\compr}{:}
\newcommand{\ufs}{{\mathsf{ufs}}}
\newcommand{\braket}[1]{\langle #1 \rangle}
\newcommand{\fs}{\mathsf{fs}}
\newcommand{\pow}{\mathcal{P}}
\newcommand{\powf}{\pow_{\omega}}
\newcommand{\powufs}{\pow_{\ufs}}
\newcommand{\powfs}{\pow_{\fs}}
\newcommand{\pown}[1]{\pow_{\leqslant #1}}
\newcommand{\fresh}{\mathbin{\#}}

\newcommand{\MSOep}{\text{MSO}^{\sim,+}}
\newcommand{\MSOe}{\text{MSO}^{\sim}}

\renewcommand{\Eq}{\mathsf{Eq}}
\renewcommand{\phi}{\varphi}

\newcommand{\lfp}{locally finitely presentable\xspace}
\newcommand{\lcp}{locally cofinitely presentable\xspace}
\newcommand{\lfs}{locally finitely super-presentable\xspace}


\newcommand{\beh}{\mathsf{beh}}
\newcommand{\hookto}{\hookrightarrow}
\newcommand{\subto}{\hookto}
\newcommand{\epito}{\twoheadrightarrow}
\newcommand{\epiot}{\twoheadleftarrow}
\newcommand{\ot}{\leftarrow}
\newcommand{\up}{\uparrow}
\newcommand{\ddown}{\twoheaddownarrow}
\newcommand{\sdown}{\twoheaddownarrow_s}
\newcommand{\down}{\downarrow}
\newcommand{\downinj}{\downarrowtail}
\newcommand{\monoto}{\rightarrowtail}
\newcommand{\sqleq}{\sqsubseteq}
\newcommand{\cleq}{\trianglelefteq}
\newcommand{\nsqleq}{\nsqsubseteq}
\newcommand{\pto}{\rightharpoonup}
\newcommand{\MB}{\mathbf{B}}

\newcommand{\f}{\mathsf{f}}
\newcommand{\nf}{\ensuremath{\mathsf{nf}}}

\newcommand{\fin}{\nu}
\newcommand{\ter}{\tau}
\newcommand{\ini}{\iota}
\newcommand{\inic}{\mu}

\newcommand*\cocolon{%
        \nobreak
        \mskip6mu plus1mu
        \mathpunct{}%
        \nonscript
        \mkern-\thinmuskip
        {:}%
        \relax
}

\newcommand{\set}[1]{\{#1\}}
\newcommand{\setw}[2]{\{#1\,\mid\,#2\}}

\newcommand{\Abar}{B}
\newcommand{\dbar}{d'}
\newcommand{\eps}{\varepsilon}
\newcommand{\opp}{\mathsf{op}}

\newcommand{\Lstar}{$\mathsf{L}^{\ast}$\xspace}
\newcommand{\Lsharp}{$\mathsf{L}^{\#}$\xspace}
\newcommand{\NLstar}{$\mathsf{NL}^{\ast}$\xspace}
\newcommand{\nomLstar}{$\nu\mathsf{L}^{\ast}$\xspace}
\newcommand{\nomNLstar}{$\nu\mathsf{NL}^{\ast}$\xspace}

\newcommand{\mhat}{\widehat{m}}
\newcommand{\ehat}{\widehat{e}}
\newcommand{\chat}{\widehat{c}}  
\renewcommand{\o}{\cdot}

\newcommand{\takeout}[1]{\empty}


\newcommand{\descto}[3][]{\arrow[phantom]{#2}[#1]{\text{\footnotesize{}\begin{tabular}{c}#3\end{tabular}}}}
\newcommand{\desctox}[4][]{\arrow[phantom,#2]{#3}[#1]{\text{\footnotesize{}\begin{tabular}{c}#4\end{tabular}}}}

\tikzset{shiftarr/.style={
        rounded corners,%
        to path={--([#1]\tikztostart.center)
                     -- ([#1]\tikztotarget.center) \tikztonodes
                     -- (\tikztotarget)},
}}

\tikzset{shiftarrr/.style={
        rounded corners,%
        to path={-- ([#1]\tikztostart.center)
                    |- (\tikztotarget)  \tikztonodes},
}}

\tikzset{roundcornerarr/.style={
        rounded corners,%
        to path={--([#1]\tikztostart.south)
                     |- (\tikztotarget) \tikztonodes},
}}

\tikzset{roundcornerarrr/.style={
        rounded corners,%
        to path={ -| (\tikztotarget) \tikztonodes},
}}

\tikzset{roundcornerarrrr/.style={
        rounded corners,%
        to path={ |- (\tikztotarget) \tikztonodes},
}}

\newcommand{\pullbackangle}[2][]{\arrow[phantom,to path={
                     -- ($ (\tikztostart)!1cm!#2:([xshift=8cm]\tikztostart) $)
                        node[anchor=west,pos=0.0,rotate=#2,
                        inner xsep = 0]
                        {\begin{tikzpicture}[minimum
                        height=1mm,baseline=0,#1]
    \draw[-] (0,0) -- (.5em,.5em) -- (0,1em);
                        \end{tikzpicture}}}]{}}

\newcommand{\overbar}[1]{\mkern 1.5mu\overline{\mkern-1.5mu#1\mkern-1.5mu}\mkern 1.5mu}

\newcommand{\mybar}[3]{%
  \mathrlap{\hspace{#2}\overline{\scalebox{#1}[1]{\phantom{\ensuremath{#3}}}}}\ensuremath{#3}
}

\newcommand{\myhat}[3]{%
  \mathrlap{\hspace{#2}\widehat{\scalebox{#1}[1]{\phantom{\ensuremath{#3}}}}}\ensuremath{#3}
}

\newcommand{\FN}{\mathsf{FN}}
\newcommand{\BN}{\mathsf{BN}}
\newcommand{\RN}{\mathsf{RN}}
\newcommand{\NA}{\mathsf{N}}
\newcommand{\VAR}{\mathsf{V}}
\newcommand{\FV}{\mathsf{FV}}
\newcommand{\fp}{\mathsf{fp}}
\newcommand{\Lpre}{L_{\mathsf{pre}}}

\newcommand{\BarForm}{\mathsf{Bar}}

\newcommand{\barA}{{\mybar{0.6}{2.5pt}{A}}} %
\newcommand{\barF}{\mybar{0.6}{2.5pt}{F}}
\newcommand{\barG}{\mybar{0.6}{2pt}{G}}
\newcommand{\barI}{\mybar{0.6}{2pt}{I}}
\newcommand{\barJ}{\mybar{0.6}{2pt}{J}}
\newcommand{\barE}{\mybar{0.6}{2.5pt}{E}}

\newcommand{\bark}{\mybar{0.7}{1.5pt}{k}}

\newcommand{\barL}{\mybar{0.8}{1.5pt}{L}}
\newcommand{\barR}{\mybar{0.8}{2pt}{R}}

\newcommand{\barQ}{\mybar{0.7}{2pt}{Q}}
\newcommand{\barDelta}{\mybar{0.6}{2pt}{\delta}}
\newcommand{\barNames}{{\mybar{0.55}{1.6pt}{\names}}}
\newcommand{\barAscript}{{\overbar{A}}}
\newcommand{\barNamess}{{\overbar{\names}}}


\newcommand{\barGF}{\mybar{0.85}{2pt}{GF}}
\newcommand{\barFpG}{\mybar{0.9}{2pt}{F + G}}
\newcommand{\barFtG}{\mybar{0.9}{2pt}{F \times G}}
\newcommand{\barAbs}{\mybar{0.8}{1.5pt}{[\names]}}

\newcommand{\hatF}{\widehat{F}}
\newcommand{\hatG}{\widehat{G}}
\newcommand{\hatGF}{\myhat{0.9}{2pt}{GF}}

\newcommand{\ngt}{{\mathsf{ngt}}}
\newcommand{\rg}{{\mathsf{rg}}}
\renewcommand{\ng}{{\mathsf{ng}}}
\newcommand{\fsuba}{{\mathsf{fsuba}}}
\newcommand{\ub}{{\mathsf{ub}}}

\newcommand{\lan}{\ddagger}
\newcommand{\dsS}{\mathds{S}}

\newcommand{\Lan}{\textsf{Lan}}
\newcommand{\Run}{\textsf{AccRun}}
\newcommand{\oRun}{\overline{\textsf{Run}}}
\newcommand{\Ran}{\textsf{Ran}}

\newcommand{\runtree}{\textsf{run}}


\newcommand{\Var}{\mathsf{Var}}
\newcommand{\Fin}{\mathsf{Fin}}
\newcommand{\Field}{\mathds{Z}}
\newcommand{\AlgSigma}{\mathbb{\Sigma}}
\newcommand{\AlgEqns}{\mathbb{E}}
\newcommand{\Nat}{\mathds{N}}
\newcommand{\Int}{\mathds{Z}}
\newcommand{\Real}{\mathds{R}}
\newcommand{\Rat}{\mathds{Q}}
\newcommand{\balpha}{{\boldsymbol \alpha}}
\newcommand{\bgamma}{{\boldsymbol \gamma}}

\newcommand{\Eloise}{\ensuremath{\exists}\text{loise}\xspace}
\newcommand{\Abelard}{\ensuremath{\forall}\text{bélard}\xspace}
\newcommand{\EloiseShort}{\ensuremath{\exists}}
\newcommand{\AbelardShort}{\ensuremath{\forall}}
\newcommand{\SubForm}{\mathbb{S}}
\newcommand{\Player}{\ensuremath{\mathfrak{p}}}
\newcommand{\CPlayer}[1][\Player]{\ensuremath{\overline{#1}}}
\newcommand{\Players}{\ensuremath{\mathbb{P}}}

\newcommand{\AccGame}[1][]{\ensuremath{\Game^{\textsf{Acc}}_{\mkern-3mu #1}}}
\newcommand{\RANAFunc}{\ensuremath{\mathbcal{A}}}
\newcommand{\ComArr}{\mathlarger{\mathlarger{\mathlarger{\mathbf{\circlearrowleft}}}}}

\newcommand{\one}{\mathbf{1}}
\newcommand{\bone}{{\bf 1}}
\newcommand{\bb}{{\bf b}}
\newcommand{\bu}{{\bf u}}
\newcommand{\bv}{{\bf v}}
\newcommand{\bs}{{\bf s}}

\newcommand{\wordlang}[1]{\ensuremath{\mathsf{W}(#1)}}
\newcommand{\prelang}[1]{\ensuremath{\mathsf{P}(#1)}}


\newcommand{\longmid}{\hspace{0.9ex}\smash{\rule[-1.0ex]{0.41pt}{3.2ex}}\hspace{0.9ex}}
\newcommand{\midmid}{\hspace{0.2ex}{\rule[-0.1ex]{0.6pt}{1.65ex}}\hspace{0.2ex}}
\newcommand{\scriptmidmid}{\hspace{0.2ex}{\rule[-0.1ex]{0.6pt}{1.1ex}}\hspace{0.2ex}}
\newcommand{\subscriptmidmid}{\hspace{0.2ex}{\rule[-0.1ex]{0.6pt}{0.9ex}}\hspace{0.2ex}}

\newcommand{\newletter}[1]{{\midmid}#1}
\newcommand{\scriptnew}[1]{{\scriptmidmid}#1}
\newcommand{\subscriptnew}[1]{{\subscriptmidmid}#1}
\newcommand{\newtreeletter}[1]{{\nu}#1}

\newcommand{\quotient}[2]{{#1}/{#2}} %

\let\originalleft\left
\let\originalright\right
\renewcommand{\left}{\mathopen{}\mathclose\bgroup\originalleft}
\renewcommand{\right}{\aftergroup\egroup\originalright}

\ExplSyntaxOn
\NewDocumentCommand{\makecycle}{om}{
	\ensuremath{ \left(~\guest_print_list:nn { #2 } {~{\ }~}~\right)\IfValueT{#1}{^{#1}}}
}
\NewDocumentCommand{\maketuple}{om}{%
\ensuremath{\left(~\guest_print_list:nn { #2 } {~{,\:}~}~\right)\IfValueT{#1}{^{#1}}}
}
\NewDocumentCommand{\makemonad}{om}{
	\ensuremath{ \left\langle~\guest_print_list:nn { #2 } {~{,\ }~}~\right\rangle\IfValueT{#1}{^{#1}}}
}
\NewDocumentCommand{\makegrammar}{om}{
	\ensuremath{ ~\guest_print_list:nn { #2 } {~{\ \,\vert\ \,}~}~\IfValueT{#1}{^{#1}}}
}

\seq_new:N \l_guest_list_seq
\cs_new_protected:Nn \guest_print_list:nn
{
	\seq_set_from_clist:Nn \l_guest_list_seq { #1 }
	\seq_use:Nn \l_guest_list_seq { #2 }
}
\ExplSyntaxOff

\usepackage[noend]{algpseudocode}
\renewcommand{\algorithmicrequire}{\textbf{Input:}}
\algrenewcommand{\algorithmiccomment}[1]{\hfill// #1}

\usepackage{hyperref}
\hypersetup{hidelinks,final,bookmarks}

\addto\extrasUKenglish{%
    \renewcommand{\subsectionautorefname}{Section}
}
