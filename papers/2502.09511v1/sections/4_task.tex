\section{Molecular Generative Tasks}\label{sec:task}


In this section, we explore various molecular generative tasks that leverage diffusion models, as outlined in \cref{fig:taxonomy}. These tasks are crucial for advancing molecular design and discovery, providing innovative solutions across different domains.

\subsection{De Novo Generation}
De novo generation involves creating novel molecular structures from scratch. This approach is essential for discovering new compounds without relying on existing molecular templates. It includes two main sub-tasks: unconditional and conditional generation.


\paratitle{Unconditional generation.}
Unconditional generation focuses on producing molecules without specific constraints. Starting from a random noise vector, these models generate entirely new molecular structures, exploring the vast chemical space for novel applications.



\paratitle{Conditional Generation}
Conditional generation tailors molecule creation based on specific conditions, such as desired properties, targets, or fragments, allowing for more directed and efficient molecular design. By incorporating these conditions, diffusion models produce molecules that meet predefined criteria. 
Existing works can be further divided into four categories based on the type of conditions applied.

\textit{Property-based molecular generation}, also known as \textit{inverse molecule design}, aims to generate molecules with desired properties such as bioactivity and synthesizability.
More specifically, the inverse molecule design can be further divided into single-property conditioning~\citep{CDGS,EDM,GeoLDM,MDM} and multiple-property conditioning~\citep{DiGress,GraphDiT,EEGSDE}.
Among them, MOOD~\citep{MOOD} and CGD~\citep{CGD} also focus on generating structurally novel molecules outside the training distribution, referred to as OOD molecule generation.

\textit{Target-based molecular generation}, also known as \textit{structure-based drug design (SBDD)}, generates molecules based on the 3D structure of target binding pockets, aiming to enhance interaction with specific targets.
IRDiff~\citep{IRDiff} introduces interaction-based retrieval to generate target-specific molecules based on retrieved high-affinity ligand references.

\textit{Fragment-based molecular generation} specifies the generation of molecules with particular fragments. DiffLinker~\citep{DiffLinker} focuses on linker design, generating linkers that connect fragments into a complete molecule.


\textit{Composition-based molecular generation} restricts the elemental composition of generated molecules, ensuring they meet specific compositional criteria~\citep{UniMat}.


\subsection{Molecular Optimization}
Molecular optimization tasks aim to improve existing molecules for better performance or properties, differing from de novo generation by focusing on modifying known structures rather than creating new ones. Starting with an existing molecule, these tasks refine it to enhance its properties or performance. This task includes scaffold hopping, R-group design, and generalized optimization.


\textit{Scaffold hopping} involves modifying molecular scaffolds to discover new compounds, transforming known scaffolds into new ones that retain biological activity~\citep{DiffHopp}.


\textit{R-group design} focuses on optimizing molecules by fine-tuning the properties of lead compounds through the adjustment of specific R-groups~\citep{DecompOpt}.


\textit{Generalized optimization} is a flexible approach to optimize molecules without being restricted to specific strategies like scaffold hopping or R-group design. 
DiffSBDD~\citep{DiffSBDD} allows for a broader range of structural changes, as long as the optimized molecule maintains a certain level of similarity to the original structure. This flexibility enables the exploration of diverse pathways to improve properties.


\subsection{Conformer Generation}
Conformer generation involves predicting the 3D conformers of a molecule based on its 2D topological structure. This task is crucial for understanding the spatial arrangement of atoms within a molecule, which is essential for predicting molecular interactions, reactivity, and properties.
Generated 3D conformers reflect the molecule's potential energy landscape and geometric constraints.
GeoDiff~\citep{GeoDiff} and Torsional Diffusion~\citep{TorsionalDiffusion} employ diffusion models to generate 3D molecular conformers in Cartesian space and torsion angle space, respectively. DiSCO~\citep{DiSCO} further optimizes the predicted conformers with Diffusion Bridge.



\subsection{Molecular Docking}
Molecular docking tasks involve predicting how molecules interact with biological targets, a key step in drug discovery for assessing binding affinity and specificity. By analyzing a molecule and a target structure, DiffDock~\citep{DiffDock} predicts the binding pose with the diffusion model.
Re-Dock~\citep{Re-Dock} further utilizes the diffusion bridge for flexible and realistic molecular docking, which predicts the binding poses of ligands and pocket sidechains simultaneously.



\subsection{Transition State Generation}
Transition state generation focuses on predicting the 3D structure of transition states in chemical reactions, using the reactants and products as inputs. This task is vital for understanding reaction mechanisms, as the transition state represents the highest energy point along the reaction pathway. Accurate modeling of these states provides insights into reaction kinetics and can aid in the design of catalysts and optimization of reaction conditions~\citep{OA-ReactDiff}.