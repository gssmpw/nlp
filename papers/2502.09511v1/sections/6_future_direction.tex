\section{Discussion and Future Direction}
In this section, we discuss the current state and challenges in diffusion models for molecules and outline several promising directions for future research to advance this area.

\paratitle{Complete data modality.}
Most existing works fall under the category of generating molecules in 3D space,
neglecting 2D topology.
Considering the complementary nature of 2D and 3D structures, generating molecules in a joint 2D and 3D space holds significant potential for producing more realistic molecules.
This approach has proven effective in de novo generation~\citep{MUDiff,JODO}, but its broader potential in other generative tasks remains underexplored.

\paratitle{Sophisticated diffusion models.}
As summarized in \cref{tab:summarizations}, 
the diffusion models employed in existing works exhibit a wide variety of formulations. Regarding the time space, there is a shift from discrete-time methods to more generalized continuous-time SDEs. In terms of the data space, an open challenge lies in handling the discrete molecular components (e.g., atom and bond types) alongside the continuous components (e.g., coordinates). Moreover, advanced formulations and techniques, such as flow matching and efficient sampling, remain underutilized.

\paratitle{Challenging generative tasks.}
Many existing works focus on fundamental tasks,
like unconditional or single-conditional generation, 
with insufficient attention to more practical generative tasks, such as multi-conditional generation, molecular optimization, and docking. 
Furthermore, poor performance on large molecules in GEOM-Drugs compared to small molecules in QM9, highlights room for improvement. Additionally, extending molecular generation to complex~\citep{DynamicBind} while considering inter-molecular interactions, presents a another promising yet challenging direction.

\paratitle{Expressive network architectures.}
Existing methods rely on relatively classical network architectures like EGNNs. 
Recent advances in more expressive equivariant neural networks offer new opportunities. Incorporating more powerful architectures into molecular diffusion models could further enhance their performance and effectiveness.

\paratitle{Relationship between molecular generation and molecular representation.}
With the increasing recognition of diffusion models' ability to learn representations, exploring the relationship between molecular generation and molecular representation based on diffusion models emerges as a promising direction. MoleculeSDE~\citep{MoleculeSDE}, SubgDiff~\citep{SubgDiff}, and UniGEM~\citep{UniGEM} mark pioneering steps, but there remains significant room for further research.
