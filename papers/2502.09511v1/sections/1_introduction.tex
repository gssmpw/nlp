\section{Introduction}

Molecular generative tasks (\cref{fig:overview}) are critical for drug discovery and material design~\citep{InverseDesignSurvey,MolGenSurvey1,MolGenSurvey2}. The ability to generate novel molecules with specific desired properties can significantly expedite the development of new pharmaceuticals and advanced materials, thereby addressing pressing challenges in healthcare and technology. However, traditional methods for these tasks are often labor intensive and time-consuming. 

Deep generative models, such as Variational Autoencoders (VAEs)~\citep{VAE}, Generative Adversarial Networks (GANs)~\citep{GAN}, Autoregressive models (ARs), and Normalizing Flows (NFs)~\citep{NormalizingFlow}, have opened new avenues for automating molecular generative tasks, gaining prominence for their ability to explore vast chemical spaces efficiently. These models enhance the speed and accuracy of molecular discovery, making them indispensable tools in modern scientific research.

\begin{figure}[t]
  \centering
  \includegraphics[width=0.4\textwidth]{figures/figure1.pdf}
  \caption{Illustration of molecular generative tasks. \textit{De novo generation} designs molecules from scratch. \textit{Molecular optimization} refines existing molecules to enhance desired properties while maintaining structure similarity. \textit{Conformer generation} generates 3D geometries of a molecule to represent its possible spatial arrangements.}
  \label{fig:overview}
  \vspace{-6pt}
\end{figure}

Diffusion models~\citep{Diffusion,DDPM} have recently emerged as powerful generative models, showcasing remarkable performance in generating high-quality data across various domains. 
These models operate by simulating the gradual degradation of a data distribution and learning its reverse process to generate new samples~\citep{understanding}. 
Originally introduced for visual domains~\citep{DiffusionVisionSruvey}, these models excel at capturing data distributions through iterative processes. Their success in visual domains has inspired researchers to explore their potential for molecular generative tasks. By effectively modeling intricate molecular structures and properties, diffusion models have become central to molecular design. This has sparked a surge in research adapting diffusion models for molecular applications, highlighting their transformative potential in this area~\citep{GDSS,EDM}.

Despite the rapid advancements and the proliferation of research on diffusion model-based molecular design, the area faces significant challenges. \textit{The diversity in diffusion model formulations, molecular data modalities, and generative task types has resulted in a fragmented research landscape.} This diversity makes it difficult for researchers to navigate existing studies, hindering a comprehensive understanding of the area's progress and potential. The lack of a systematic and up-to-date survey exacerbates this issue, as researchers struggle to keep pace with the latest developments. This fragmentation not only limits the accessibility of existing advances, but also constrains the area's growth and innovation.

To address these challenges, this paper presents an comprehensive survey of diffusion model-based molecular generative methods. We provide a comprehensive and systematic review of the area, elucidating the design space of existing works
% categorizing existing research 
according to method formulations, data modalities, and task types, as shown in \cref{fig:taxonomy} and \cref{tab:summarizations}. By offering this novel taxonomy, we clarify the research landscape and facilitate easier navigation for researchers. 

This survey seeks to bridge the gap between diverse studies, promoting a more cohesive understanding of the area and supporting its further development. By highlighting key advancements and identifying emerging opportunities for future research, we hope to contribute to the area's ongoing evolution and encourage future innovations.

This survey makes several key contributions to the area of molecular generative tasks using diffusion models:
\begin{itemize}[leftmargin=*]
    \item We provide an up-to-date and systematic overview of the current state of research, addressing the need for a comprehensive understanding of the area.
    \item We introduce a novel taxonomy that categorizes research efforts based on method formulations, data modalities, and task types, offering a structured framework.
    \item By identifying opportunities in the existing literature, we pave the way for future directions, encouraging further innovation in diffusion model-based molecular design.
\end{itemize}