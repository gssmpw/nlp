\tikzstyle{leaf}=[draw=hiddendraw,
    rounded corners, minimum height=1em,
    fill=blue!7, text opacity=1, align=center,
    fill opacity=.5, text=black, align=left, font=\scriptsize,
    inner xsep=3pt,
    inner ysep=1pt,
    ]
\begin{figure*}[ht]
\centering
\begin{forest}
  for tree={
  forked edges,
  grow=east,
  reversed=true,
  anchor=base west,
  parent anchor=east,
  child anchor=west,
  base=middle,
  font=\scriptsize,
  rectangle,
  draw=hiddendraw,
  rounded corners,align=left,
  line width=0.7pt,
  minimum width=2em,
    s sep=5pt,
    inner xsep=3pt,
    inner ysep=1pt,
  },
  where level=1{text width=4.5em}{},
  where level=2{text width=6em,font=\scriptsize}{},
  where level=3{font=\scriptsize}{},
  where level=4{font=\scriptsize}{},
  where level=5{font=\scriptsize}{},
  [Diffusion Models for Molecules,rotate=90,anchor=north,edge=hiddendraw
    [Diffusion Model\\Formulations (\S\ref{sec:diffusion}),edge=hiddendraw,text width=5.38em
        [DDPMs, text width=6.88em, edge=hiddendraw
            [EDM~\citep{EDM}{,}
            GeoLDM~\citep{GeoLDM}{,}
            UniGEM~\citep{UniGEM}{,},leaf,text width=31.58em, edge=hiddendraw]
        ]
        [SMLDs, text width=6.88em, edge=hiddendraw
            [MDM~\citep{MDM}{,}
            VoxMol~\citep{VoxMol}{,}
            LDM-3DG~\citep{LDM-3DG}{,},leaf,text width=31.58em, edge=hiddendraw]
        ]
        [SDEs, text width=6.88em, edge=hiddendraw
            [GDSS~\citep{GDSS}{,}
            CDGS~\citep{CDGS}{,}
            EEGSDE~\citep{EEGSDE}{,},leaf,text width=31.58em, edge=hiddendraw]
        ]
        [Others, text width=6.88em, edge=hiddendraw
            [DiGress~\citep{DiGress}{,}
            DiffMol~\citep{DiffMol}{,}
            TurboHopp~\citep{TurboHopp}{,},leaf,text width=31.58em, edge=hiddendraw]
        ]
    ]
    [Molecular Data\\Modalities (\S\ref{sec:data}),edge=hiddendraw,text width=5.38em
        [2D Topological \\Space, text width=4.88em, edge=hiddendraw
            [Continuous Data Space, edge=hiddendraw, text width=7.08em
                [GDSS~\citep{GDSS}{,} CDGS~\citep{CDGS}{,} MOOD~\citep{MOOD},leaf,text width=24.75em, edge=hiddendraw]
            ]
            [Discrete Data Space, edge=hiddendraw, text width=7.08em
                [DiGress~\citep{DiGress}{,}
                HGLDM~\citep{HGLDM}{,}\\
                Graph DiT~\citep{GraphDiT}
                ,leaf,text width=24.75em, edge=hiddendraw]
            ]
         ]
        [3D Geometric \\Space, text width=4.88em, edge=hiddendraw
            [EDM~\citep{EDM}{,}
            EDM-Bridge~\citep{EDM-Bridge}{,}
            EEGSDE~\citep{EEGSDE}{,}\\
            GeoLDM~\citep{GeoLDM}{,}
            MDM~\citep{MDM}{,}
            VoxMol~\citep{VoxMol}{,}\\
            GaUDI~\citep{GaUDI}{,}
            DiffMol~\citep{DiffMol}{,}
            LDM-3DG~\citep{LDM-3DG}{,}\\
            END~\citep{END}{,}
            GFMDiff~\citep{GFMDiff}{,}
            UniMat~\citep{UniMat}{,}\\
            UniGEM~\citep{UniGEM}{,}
            MuDM~\citep{MuDM}{,}
            EQGAT-diff~\citep{EQGAT-diff}{,}
            NEXT-Mol~\citep{NEXT-Mol}
            ,leaf,text width=33.58em, edge=hiddendraw]
        ]
        [2D and 3D \\Joint Space, text width=4.88em, edge=hiddendraw
            [MolDiff~\citep{MolDiff}{,}
            MiDi~\citep{MiDi}{,}
            MUDiff~\citep{MUDiff}{,}
            JODO~\citep{JODO}
            ,leaf,text width=33.58em, edge=hiddendraw]
        ]
    ]
    [Molecular\\Generative\\Tasks (\S\ref{sec:task}), edge=hiddendraw, text width=4.58em
        [De Novo Generation, text width=6.08em, edge=hiddendraw
            [Unconditional \\Generation,text width=4.5em, edge=hiddendraw
                [GDSS~\citep{GDSS}{,} EDM~\citep{EDM}{,} EDM-Bridge~\citep{EDM-Bridge}{,} \\
                UniGEM~\citep{UniGEM}{,} VoxMol~\citep{VoxMol}{,}
                DiffMol~\citep{DiffMol},leaf,text width=26.9em, edge=hiddendraw]
            ]
            [Conditional \\Generation, text width=4.5em, edge=hiddendraw
                [Property-based, edge=hiddendraw, text width=4.38em
                    [CDGS~\citep{CDGS}{,} MOOD~\citep{MOOD}{,}\\
                    DiGress~\citep{DiGress}{,} Graph DiT~\citep{GraphDiT}{,}\\
                    EEGSDE~\citep{EEGSDE}{,}
                    GeoLDM~\citep{GeoLDM}{,}\\
                    MDM~\citep{MDM}{,}
                    CGD~\citep{CGD},leaf,text width=20.75em, edge=hiddendraw]
                ]
                [Target-based, edge=hiddendraw, text width=4.38em
                    [TargetDiff~\citep{TargetDiff}{,}
                    DiffSBDD~\citep{DiffSBDD}{,}\\
                    DecompDiff~\citep{DecompDiff}{,}
                    SBE-Diff~\citep{SBE-Diff}{,}\\
                    LDM-3DG~\citep{LDM-3DG}{,}
                    PMDM~\citep{PMDM}{,}\\
                    BindDM~\citep{BindDM}{,}
                    IRDiff~\citep{IRDiff}{,}\\
                    AliDiff~\citep{AliDiff}{,}
                    IPDiff~\citep{IPDiff}
                    ,leaf,text width=20.75em, edge=hiddendraw]
                ]
                [Fragment-based, edge=hiddendraw, text width=4.38em
                    [DiffLinker~\citep{DiffLinker}{,} 
                    PMDM~\citep{PMDM}{,}\\
                    EEGSDE~\citep{EEGSDE}{,}
                    ,leaf,text width=20.75em, edge=hiddendraw]
                ]
                [Composition-based, edge=hiddendraw, text width=6.08em
                    [UniMat~\citep{UniMat},leaf,text width=19.05em, edge=hiddendraw]
                ]
            ]
        ]
        [Molecular Optimization,text width=7.88em, edge=hiddendraw
            [Scaffold Hopping, text width=7.5em, edge=hiddendraw
                [DiffHopp~\citep{DiffHopp}{,}
                TurboHopp~\citep{TurboHopp}{,}\\
                PMDM~\citep{PMDM}{,}
                DecompOpt~\citep{DecompOpt}
                ,leaf,text width=22.08em, edge=hiddendraw]
            ]
            [R-group Design,text width=7.5em, edge=hiddendraw
                [DecompOpt~\citep{DecompOpt},leaf,text width=22.08em, edge=hiddendraw]
            ]
            [Generalized Optimization,text width=7.5em, edge=hiddendraw
                [DiffSBDD~\citep{DiffSBDD},leaf,text width=22.08em, edge=hiddendraw]
            ]
        ]
        [Conformer Generation,text width=7.88em, edge=hiddendraw
            [GeoDiff~\citep{GeoDiff}{,}
            Torsional Diffusion~\citep{TorsionalDiffusion}{,}
            DiSCO~\citep{DiSCO}{,}\\
            NEXT-Mol~\citep{NEXT-Mol}
            ,leaf,text width=31.36em, edge=hiddendraw]
        ]
        [Molecular Docking,text width=7.88em, edge=hiddendraw
            [DiffDock~\citep{DiffDock}{,}
            Re-Dock~\citep{Re-Dock}{,}
            DynamicBind~\citep{DynamicBind}
            ,leaf,text width=31.36em, edge=hiddendraw]
        ]
        [Transition State Generation,text width=7.88em, edge=hiddendraw
            [OA-ReactDiff~\citep{OA-ReactDiff}
            ,leaf,text width=31.36em, edge=hiddendraw]
        ]
    ]
  ]
\end{forest}
\caption{A taxonomy of diffusion models for molecules with representative works.}
\label{fig:taxonomy}
\end{figure*}


\section{Molecular Data Modalities}\label{sec:data}


In molecular generative modeling, molecular data can be featurized in various modalities. The primary modalities are graphs in 2D topological space and conformations in 3D geometric space. In 2D graphs, nodes denote atoms, and edges represent bonds. In 3D conformations, the molecule is described as a point cloud, where the points are atoms with positional coordinates. 
Molecules can also be featurized in other ways, such as SMILES in 1D space and molecular descriptors. However, these featurizations are typically not generated using diffusion models and thus fall outside the scope of this survey.

Different molecular generative methods often target distinct modalities, resulting in different advantages. As illustrated in \cref{fig:taxonomy} and \cref{tab:summarizations}, existing methods are categorized into three groups based on the modality of generated molecular data: those generating molecules only in 2D space, only in 3D space, and in 2D and 3D joint space.

\subsection{Generating Molecules in 2D Topological Space}
\paratitle{Definition 1: 2D molecular graph.} The molecular graphs have categorical node and edge types, represented by the spaces $\cX$ and $\cE$, respectively, with cardinalities $a$ and $b$. A molecular graph can be denoted as $\cM_{\text{2D}} = (\bX, \bA)$, where $\bX \in \mathbb{R}^{n \times a}$ is the one-hot encoded atom feature matrix representing atom types, and $\bA \in \mathbb{R}^{n \times n \times b}$ is the one-hot encoded adjacency matrix indicating bond existence and bond types. Here, $n$ denotes the number of atoms.

The primary challenges in generating molecular graphs is maintaining permutation invariance and modeling the complex dependencies between nodes (atoms) and edges (bonds). To address these challenges, graph neural networks (GNNs) are commonly employed as the backbone architecture. As illustrated in \cref{fig:taxonomy}, several methods, such as GDSS~\citep{GDSS}, CDGS~\citep{CDGS}, and MOOD~\citep{MOOD}, directly apply diffusion models from continuous data spaces to the task of molecular graph generation. 

However, the discrete nature of atom and bond types presents further challenges. Recognizing this, DiGress~\citep{DiGress} employs a discrete diffusion model, D3PM~\citep{D3PM}, specifically designed for discrete data spaces, to model molecular graphs. On smaller molecules, discrete models like DiGress achieve results comparable to continuous models but offer the advantage of faster training.

The drawback of generating molecules in 2D space is that it only produces atom types and binding topology, lacking 3D geometric structure. Molecules inherently exist in 3D space, and their 3D structure affects their quantum properties.
This makes it unsuitable for quantum-property-based or structure-based drug design, which rely heavily on 3D structures.

In conclusion, while 2D molecular graph generation provides a useful framework for certain applications, its limitations highlight the necessity for further research into methods that can integrate 3D structural data to enhance the accuracy and applicability of molecular models in real-world scenarios.


\begin{table*}[t]
\resizebox{\linewidth}{!}{
\begin{tabular}{l|l|llllll|lll}
\toprule
& \textbf{Model} & \textbf{Diffusion Type} & \textbf{Output} & \textbf{Data Space}  & \textbf{Time Space} & \textbf{Network}  & \textbf{Objective}  & \textbf{Task} & \textbf{Condition} & \textbf{Dataset} \\ \midrule

\multirow{5}{*}{\rotatebox{90}{2D Space}}  & GDSS~\citep{GDSS} & SDE & X, A & Cont. & Cont. & MPNN & score matching & De Novo Gen. & None & ZINC250k, QM9-2014 \\
 & CDGS~\citep{CDGS} & SDE & X, A & Cont. & Cont. & MPNN+GT & noise prediction & De Novo Gen. & None/Property & ZINC250k, QM9-2014 \\
 & MOOD~\citep{MOOD} & SDE & X, A & Cont. & Cont. & MPNN & score matching & De Novo Gen. & OOD-ness/Property & ZINC250k \\
 & DiGress~\citep{DiGress} & D3PM & X, A & Disc. & Disc. & MPNN & data prediction & De Novo Gen. & None/Property & QM9-2014, MOSES, GuacaMol \\
 & Graph DiT~\citep{GraphDiT} & D3PM & X, A & Disc. & Disc. & GT & data prediction & De Novo Gen. & Property & Polymer, MoleculeNet \\
\midrule
\multirow{25}{*}{\rotatebox{90}{3D Space}} & EDM~\citep{EDM} & DDPM & X, P & Cont. & Disc. & EGNN & noise prediction & De Novo Gen. & None/Property & QM9-2014, GEOM-Drugs \\
 & EDM-Bridge~\citep{EDM-Bridge} & DB & X, P & Cont. & Cont. & EGNN & score matching & De Novo Gen. & None & QM9-2014, GEOM-Drug \\
 & EEGSDE~\citep{EEGSDE} & SDE & X, P & Cont. & Cont. & EGNN & noise prediction & De Novo Gen. & Property/Substructure & QM9-2014, GEOM-Drug \\
 & GeoLDM~\citep{GeoLDM} & DDPM & Latent & Cont. & Disc. & EGNN & noise prediction & De Novo Gen. & None/Property & QM9-2014, GEOM-Drug \\
 & MDM~\citep{MDM} & SMLD & X, P & Cont. & Disc. & EGNN & score matching & De Novo Gen. & None/Property & QM9-2014, GEOM-Drug \\
 & UniGEM~\citep{UniGEM} & DDPM & X, P & Cont. & Disc. & EGNN & noise prediction & De Novo Gen. & None & QM9-2014, GEOM-Drug \\
 & VoxMol~\citep{VoxMol} & SMLD & VM* & Cont. & Disc. & 3D U-Net & score matching & De Novo Gen. & None & QM9-2014, GEOM-Drug \\
 & DiffMol~\citep{DiffMol} & D3PM & X, P & Disc. & Disc. & EGNN & \makecell[l]{score matching\\ \&data prediction} & De Novo Gen. & None & QM9-2014 \\
 & LDM-3DG~\citep{LDM-3DG} & SMLD & Latent & Cont. & Disc. & 3D MPNN & score matching & De Novo Gen. & Property/Target & \makecell[l]{QM9-2014, GEOM-Drug,\\ CrossDocked} \\
 & GaUDI~\citep{GaUDI} & DDPM & GoR* & Cont. & Disc. & EGNN & noise prediction & De Novo Gen. & Property & COMPAS-1x, PAS \\
 & GeoDiff~\citep{GeoDiff} & DDPM & P & Cont. & Disc. & GFN & noise prediction & Conformer Gen. & Molecular Graph & GEOM-QM9, GEOM-Drugs \\
 & Torsional Diffusion~\citep{TorsionalDiffusion} & SDE & P & Cont. & Cont. & GFN & score matching & Conformer Gen. & Molecular Graph & \makecell[l]{GEOM-QM9, GEOM-Drugs,\\ GEOM-XL} \\
 & DiffLinker~\citep{DiffLinker} & DDPM & Linker & Cont. & Disc. & EGNN & noise prediction & De Novo Gen. & 3D Fragments & ZINC, CASF, GEOM \\
 & OA-ReactDiff~\citep{OA-ReactDiff} & DDPM & TS* & Cont. & Disc. & LEFTNet & noise prediction & TS Gen. & Reactant and Product & Transition1x \\
 & TargetDiff~\citep{TargetDiff} & DDPM+D3PM & X, P & Cont.+Disc. & Disc. & EGNN & \makecell[l]{data prediction} & De Novo Gen. & Target & CrossDocked \\
 & DiffSBDD~\citep{DiffSBDD} & DDPM & X, P & Cont. & Disc. & EGNN & noise prediction & De Novo Gen., Opti. & Target & CrossDocked \\
 & SBE-Diff~\citep{SBE-Diff} & DDPM+D3PM & X, P & Cont.+Disc. & Disc. & EGNN & \makecell[l]{data prediction} & De Novo Gen. & Target & CrossDocked \\
 & PMDM~\citep{PMDM} & DDPM & X, P & Cont. & Disc. & SchNet & noise prediction & De Novo Gen., Opti. & Target & CrossDocked \\
 & BindDM~\citep{BindDM} & DDPM+D3PM & X, P & Cont.+Disc. & Disc. & EGNN & \makecell[l]{data prediction} & De Novo Gen. & Target & CrossDocked \\
 & DiffDock~\citep{DiffDock} & SDE & P* & Cont. & Cont. & EGNN & score matching & Docking & Protein and Ligand & PDBBind \\
 & Re-Dock~\citep{DiffDock} & DB & P* & Cont. & Cont. & EGNN & score matching & Docking & Protein and Ligand & PDBBind \\
 & DiffHopp~\citep{DiffHopp} & DDPM & X, P & Cont. & Disc. & GVP & noise prediction & Opti. & Protein-ligand Complex & PDBBind \\
 & TurboHopp~\citep{TurboHopp} & CM & X, P & Cont. & Cont. & GVP & self-consistency & Opti. & Protein-ligand Complex & PDBBind \\
 & UniMat~\citep{UniMat} & DDPM & PT* & Cont. & Disc. & Conv, Attn & noise prediction & De Novo Gen. & Composition & Perov-5, Carbon-24, MP-20 \\
\midrule
\multirow{6}{*}{\rotatebox{90}{2D\&3D Space}} & MolDiff~\citep{MolDiff} & DDPM+D3PM & X, P, A & Cont.+Disc. & Disc. & EGNN & \makecell[l]{data prediction} & De Novo Gen. & None & QM9-2014, GEOM-Drug \\
 & MiDi~\citep{MiDi} & DDPM+D3PM & X, P, A & Cont.+Disc. & Disc. & rEGNN & \makecell[l]{data prediction} & De Novo Gen. & None & QM9-2014, GEOM-Drug \\
 & DecompDiff~\citep{DecompDiff} & DDPM+D3PM & X, P, A & Cont.+Disc. & Disc. & EGNN & \makecell[l]{data prediction} & De Novo Gen. & Target & CrossDocked \\
 & DecompOpt~\citep{DecompOpt} & DDPM+D3PM & X, P, A & Cont.+Disc. & Disc. & EGNN & \makecell[l]{data prediction} & Opti. & Target and Ligand & CrossDocked \\
 & JODO~\citep{JODO} & SDE & X, P, A & Cont. & Cont. & DGT & data prediction & De Novo Gen. & None/Property & QM9-2014, GEOM-Drug \\
 & MUDiff~\citep{MUDiff} & DDPM+D3PM & X, P, A & Cont.+Disc. & Disc. & MUformer & \makecell[l]{noise\&data prediction} & De Novo Gen. & None/Property & QM9-2014 \\
 \bottomrule
\end{tabular}}
\caption{A comprehensive summary of diffusion models for molecules in the literature, categorized based on the data modality into three groups: diffusion in 2D space, in 3D space, and in joint space.
Acronyms in \textbf{Data Space} and \textbf{Time Space}: Cont. refers to continuous space; Disc. refers to discrete space.
Acronyms in \textbf{Task}: Gen. refers to generation task; Opti. refers to optimization task.
Acronyms in \textbf{Output}: X, A, and P refer to atom features, adjacency matrix, and positions, respectively; VM denotes voxelized molecules; GoR refers to graph of rings; TS represents transition states, P* refers to ligand poses in the submanifold space; PT refers to periodic table-based molecular representations.
}
\label{tab:summarizations}
\end{table*}


\subsection{Generating Molecules in 3D Geometric Space}
\paratitle{Definition 2: 3D molecular conformation.} A 3D molecular conformation is formally defined as $\cM_{\text{3D}} = (\bX, \bP)$, where $\bX \in \mathbb{R}^{n \times a}$ is the one-hot encoded atom feature matrix representing atom types, $\bP \in \mathbb{R}^{n \times 3}$ is the position coordinate matrix of atoms, and $n$ denotes the number of atoms. 

Methods for generating molecules in 3D space focus on producing both the atom types and their geometric structures, while not directly generating the binding topology. The binding topology can be inferred through post-processing steps. A significant challenge in these methods is maintaining SE(3) equivariance, which ensures that the generated molecular structures are invariant to transformations such as rotations and translations in 3D space. This is often achieved by integrating equivariant graph neural networks~\citep{EGNN} and employing techniques like zero center of mass (CoM) adjustments~\citep{EDM}.

The drawback of these methods is that they do not consider binding topology during the generation of geometric structures. Binding information is crucial for evaluating the quality of molecular generation and many downstream tasks. Post-processing to infer binding topology can introduce errors and often results in suboptimal solutions.
Moreover, for larger molecules, such as those in GEOM-Drug~\citep{GEOM-datasets}, directly generating high-quality 3D structures is challenging. Incorporating 2D topology during the generation process can provide valuable guidance and improve the quality of the resulting molecular structures.


\subsection{Generating Molecules in 2D\&3D Joint Space}

The generation of molecular structures in 2D and 3D joint spaces offers a comprehensive approach to capturing both the topological and geometric properties of molecules. This dual featurization, also referred to as a complete molecular structure in some works~\citep{JODO,MUDiff}, is crucial for accurately depicting molecules.

\paratitle{Definition 3: Complete molecular structure.} A molecule in this joint space is defined as $\cM = (\bX, \bA, \bP)$, where $\bX \in \mathbb{R}^{n \times a}$ is the atom feature matrix, $\bA \in \mathbb{R}^{n \times n \times b}$ is the one-hot encoded adjacency matrix indicating bond existence and bond types, and $\bP \in \mathbb{R}^{n \times 3}$ is the position coordinate matrix of the atoms. Here, $n$ denotes the number of atoms in the molecule.

Methods that generate molecules in both 2D and 3D spaces simultaneously focus on creating a cohesive representation where the 2D topological structure and the 3D geometric structure interact and complement each other during the generation process. 
The 2D representation captures the bonding relationships between atoms, providing essential information about the molecular connectivity and chemical structure.
The 3D representation provides spatial information, crucial for understanding the molecule's shape, conformation, and potential interactions with other molecules.
By integrating these two aspects, the generation process benefits from the strengths of both representations. The 2D topology can guide the 3D structure formation, ensuring that spatial arrangements are chemically feasible, while the 3D geometry can refine the 2D topology by suggesting plausible bonding patterns based on spatial proximity~\citep{MolDiff,MiDi}.
The interaction allows for continuous refinement and correction, leading to more stable and realistic molecular models.

The main technical challenge in generating molecules in joint space is managing the discrete topological structures alongside continuous geometric structures. JODO~\citep{JODO} uses SDE to treat both structures as continuous variables. Conversely, MUDiff~\citep{MUDiff} employs discrete D3PM for topology and continuous DDPM for geometry, handling them separately. The elegant and simultaneous generation of molecular structures across these modalities remains a significant area for exploration.