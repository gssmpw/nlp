%%%% ijcai25.tex

\typeout{IJCAI--25 Instructions for Authors}

% These are the instructions for authors for IJCAI-25.

\documentclass{article}
\pdfpagewidth=8.5in
\pdfpageheight=11in

% The file ijcai25.sty is a copy from ijcai22.sty
% The file ijcai22.sty is NOT the same as previous years'
\usepackage{ijcai25}

% Use the postscript times font!
\usepackage{times}
\usepackage{soul}

\usepackage{url}
\usepackage[usenames,svgnames]{xcolor}
\usepackage[colorlinks,
citecolor=NavyBlue,
linkcolor=NavyBlue,
urlcolor=NavyBlue]{hyperref}

\usepackage[utf8]{inputenc}
\usepackage[small]{caption}
\usepackage{graphicx}
\usepackage{amsmath}
\usepackage{amsthm}
\usepackage{booktabs}
\usepackage{algorithm}
\usepackage{algorithmic}
\usepackage[switch]{lineno}
\usepackage{enumitem}


\usepackage{amsfonts}  % mathbb

% tikz forest (tikz and forest are imported in mydef.sty)
\usepackage{mydef}

\usepackage{booktabs}
\usepackage{tabularx}
\usepackage{framed}
\usepackage{bbding}
\usepackage{subfloat}
\usepackage{adjustbox}
\usepackage{multirow}
\usepackage{pifont}
\usepackage{natbib}

\usepackage{makecell}

\usepackage[capitalize]{cleveref}
\crefname{figure}{Figure}{Figures}
\crefname{table}{Table}{Tables}
% \crefname{equation}{Equation}{Equations}
% \crefformat{equation}{Equation~#2#1#3}
\crefformat{equation}{Eq.~(#2#1#3)}



%% Comments
% \newcommand{\annotator}[2]{\csdef{#1}##1{{\color{#2} [\textbf{\MakeUppercase #1}: ##1]}}}
\newcommand{\annotator}[2]{\csdef{#1}##1{{\color{#2} [\textbf{\MakeUppercase #1}: ##1]}}}
\annotator{leon}{blue}


\newcommand{\paratitle}[1]{\vspace{0.8ex}\noindent \textbf{#1}}


% Comment out this line in the camera-ready submission
% \linenumbers

\urlstyle{same}



% the following package is optional:
%\usepackage{latexsym}

% See https://www.overleaf.com/learn/latex/theorems_and_proofs
% for a nice explanation of how to define new theorems, but keep
% in mind that the amsthm package is already included in this
% template and that you must *not* alter the styling.
\newtheorem{example}{Example}
\newtheorem{theorem}{Theorem}

% Following comment is from ijcai97-submit.tex:
% The preparation of these files was supported by Schlumberger Palo Alto
% Research, AT\&T Bell Laboratories, and Morgan Kaufmann Publishers.
% Shirley Jowell, of Morgan Kaufmann Publishers, and Peter F.
% Patel-Schneider, of AT\&T Bell Laboratories collaborated on their
% preparation.

% These instructions can be modified and used in other conferences as long
% as credit to the authors and supporting agencies is retained, this notice
% is not changed, and further modification or reuse is not restricted.
% Neither Shirley Jowell nor Peter F. Patel-Schneider can be listed as
% contacts for providing assistance without their prior permission.

% To use for other conferences, change references to files and the
% conference appropriate and use other authors, contacts, publishers, and
% organizations.
% Also change the deadline and address for returning papers and the length and
% page charge instructions.
% Put where the files are available in the appropriate places.


% PDF Info Is REQUIRED.

% Please leave this \pdfinfo block untouched both for the submission and
% Camera Ready Copy. Do not include Title and Author information in the pdfinfo section
\pdfinfo{
/TemplateVersion (IJCAI.2025.0)
}

\title{Diffusion Models for Molecules: A Survey of Methods and Tasks}


% Single author syntax
% \author{
%     Author Name
%     \affiliations
%     Affiliation
%     \emails
%     email@example.com
% }

% Multiple author syntax (remove the single-author syntax above and the \iffalse ... \fi here)
% \iffalse
\author{
Liang Wang$^{1,2}$
\and
Chao Song$^3$\and
Zhiyuan Liu$^{4}$\and
Yu Rong$^{5}$\and \\
% Qiang Liu$^{1,2}$\thanks{Corresponding author.}\and 
Qiang Liu$^{1,2}$\and 
Shu Wu$^{1,2}$\And
Liang Wang$^{1,2}$\\
\affiliations
$^1$NLPR, MAIS, Institute of Automation, Chinese Academy of Sciences\\
$^2$School of Artificial Intelligence, University of Chinese Academy of Sciences\\
$^3$Northwestern Polytechnical University\\
$^4$National University of Singapore\\
$^5$Alibaba DAMO Academy\\
\emails
liang.wang@cripac.ia.ac.cn,
csong@mail.nwpu.edu.cn,
acharkq@gmail.com,\\
yu.rong@hotmail.com,
\{qiang.liu, shu.wu, wangliang\}@nlpr.ia.ac.cn
}
% \fi

\begin{document}


\maketitle

\begin{abstract}
    % Molecular generation 
    Generative tasks about molecules, including but not limited to molecule generation, are crucial for drug discovery and material design, and have consistently attracted significant attention. In recent years, diffusion models have emerged as an impressive class of deep generative models, sparking extensive research and leading to numerous studies on their application to molecular generative tasks. Despite the proliferation of related work, there remains a notable lack of up-to-date and systematic surveys in this area. Particularly, due to the diversity of diffusion model formulations, molecular data modalities, and generative task types, the research landscape is challenging to navigate, hindering understanding and limiting the area's growth. To address this, this paper conducts a comprehensive survey of diffusion model-based molecular generative methods. We systematically review the research from the perspectives of methodological formulations, data modalities, and task types, offering a novel taxonomy. This survey aims to facilitate understanding and further flourishing development in this area.
    The relevant papers are summarized at: \url{https://github.com/AzureLeon1/awesome-molecular-diffusion-models}.
\end{abstract}

\section{Introduction}\label{sec:intro}

In computational finance, Monte Carlo simulations are used extensively to estimate the expected value of financial payoffs based on the solution of stochastic differential equations (SDEs) which model the evolution of stock prices, interest rates, exchange rates and other quantities \cite{glasserman04}.  Monte Carlo methods are very general and flexible, but for high accuracy it requires generating a large number of costly SDE path approximations, which has motivated research into a number of variance reduction or, equivalently, cost reduction techniques. One such method is
Multilevel Monte Carlo (MLMC), which was proposed in \cite{GILES2008} and was adapted for various applications that are summarised in \cite{Giles_overview17} and successfully combined with other methods such as quasi-Monte Carlo methods. The main idea of MLMC is to approximate the payoff using different time stepping resolutions when numerically solving the underlying SDE and to generate an optimal number of samples on each level, such that the overall computational cost is minimised subject to the desired bound on the variance. %, such that the total computational cost is minimised. 
The computational savings come from the fact that most samples are computed on the coarser levels and hence are less expensive while only a few samples from the finest levels are required \cite{GILES2008}.


Among the directions in which the computational cost 
of MLMC methods could further be reduced, an important avenue is the use of lower precision calculations, especially for the first Monte Carlo levels where the targeted accuracy is relatively low. 
 An overview of the research on mixed precision for the standard Monte Carlo (MC) framework is provided in \cite{ChowMixedPrecisionStandardMC} but only a few references study the potential of low precision computation in the MLMC framework \cite{Rounding_error_oliver}. To the best of our knowledge, the only MLMC framework with customised precision in the literature is \cite{brugger2014mixed}, but they use a uniform precision for all operations on each Monte Carlo level instead of optimising 
 the precision of each intermediary variable to reduce as much as possible the cost of path generation.
 
An important motivation for an MLMC framework with variable precision would be performing the low precision computations on reconfigurable hardware devices such as Field Programmable Gate Arrays (FPGAs). FPGAs contain customizable logic blocks and connectors that make it easy to adapt the digital circuit architecture for a specific application, leading to a highly parallel and optimised implementation. Therefore they are successfully exploited in applications that require high speed and have high computational workload, such as signal processing \cite{woods2008fpga}, and real time applications like high frequency trading \cite{HFT1,HFT2}. That is why a number of previous works in hardware architecture design implemented the MLMC algorithm to price financial options using FPGAs as accelerators, which resulted in improved speed and power efficiency compared to full CPU architectures \cite{Schryver2013AMM}. The paper \cite{lindsey2016domain} also proposed 
a Domain Specific Language to automate the configuration of FPGAs for this specific application. However, only \cite{brugger2014mixed} proposed a heuristic to reduce the precision in calculations.

In addition, all aforementioned works considered that the random number generation (RNG) is performed in single or double precision. Yet in most cases an important portion of the workload in the overall MLMC simulation comes from the RNG and in \cite{brugger2014mixed} this limited the total computational savings.
To reduce the cost of MLMC simulations in particular those based on the Geometric Brownian Motion (GBM), \cite{approximateICDF_Oliver, NestedOliver} have proposed to use approximate random numbers that are generated by applying an approximation of the inverse CDF to uniform random numbers. In \cite{NestedOliver}, the authors proposed a way to integrate these lower precision random variables into a \textit{nested} MLMC framework and completed a numerical analysis to bound the resulting error at each MC level by a product of the time step and the error in the random number approximation. The same authors show in \cite{approximateICDF_Oliver} that using approximate random variables reduces the cost of path generation by a factor 7.


In this paper we propose a nested MLMC framework that combines the use of approximate random normal variables and lower precision calculations to reduce the computational cost of MLMC even further than \cite{brugger2014mixed,NestedOliver}. We illustrate the efficiency of our framework in Matlab, after making several assumptions on the cost of operations and size of the errors that we carefully justify. We focus on the case of GBM and use the approximate RNG methods presented in \cite{approximateICDF_Oliver} as well as a new slightly modified method that combines CDF inversion and the central limit theorem. To choose the precision of the variables in the low precision path generation, we introduce a novel method to optimise the bit-widths. This optimisation is performed before the main path generation loop is executed and is based on a linear model of the payoff error  
due to rounding when computing in low precision. The error model relies on algorithmic differentiation in a similar manner to \cite{unifying-bwoptim,bitwidth-AD,ADAPT}. The bit-width optimisation procedure can be performed off-line, so this stage can be excluded from the on-line time complexity of our framework. The user specified desired accuracy is then enforced by calculating on-line the number of samples that need to be generated.

In terms of hardware design, we suggest implementing the low precision path generation on FPGAs and the full-precision ones on a CPU or GPU. 
The FPGA offers enough flexibility to define a separate bit-width for every variable in the low precision path generation, and can be reconfigured periodically to update the bit-widths when the market parameters have changed considerably. 


The paper is organized as follows : \Cref{sec:MLMC} introduces MLMC and nested MLMC to make clear the estimator that is implemented in our framework. Then in \Cref{sec:RNG} we detail the methods that could be used to obtain approximate random normally distributed numbers very cheaply for the low precision path generation. In \Cref{sec:error_model} and \Cref{sec:costModel} we propose an error model and a cost model (resp.) that we then use to formulate the optimisation problem that is solved to obtain the optimal bit-widths of fixed point variables in \Cref{sec:optimisation}. Finally we summarise our results and future directions in \Cref{sec:conclusion}.



\section{Formulations of Diffusion Models}\label{sec:diffusion}

This fundamental framework of diffusion models comprises two parts: the \textit{forward diffusion process} and the \textit{reverse generation process}, as shown in \cref{fig:diffusion}. In the forward process, the model progressively adds noise to real data, eventually approaching a simple prior distribution.
In the reverse process, the model learns to progressively restore the data distribution from noise.
The reverse process is typically parameterized using neural networks.

This framework can be formulated in various ways, such as the Denoising Diffusion Probabilistic Models~\citep{DDPM}, Score Matching with Langevin Dynamics~\citep{SMLD}, the generalized Stochastic Differential Equations~\citep{SDE}, and other variants.


\subsection{Denoising Diffusion Probabilistic Models (DDPMs)}

DDPM~\citep{Diffusion,DDPM} is a classical diffusion model that performs step-by-step denoising using a fixed noise schedule. It employs two Markov chains for the forward and reverse processes. 

Starting with the original noise-free data $\bx_0$, the forward process transforms it into a sequence of noisy data $\bx_1, \bx_2, \ldots, \bx_T$ using a forward transition kernel:
\begin{equation}
q(\bx_t|\bx_{t-1}) = \mathcal{N}\left(\bx_t; \sqrt{\alpha_t}\bx_{t-1}, (1-\alpha_t)\bI\right),
\end{equation}
where $\alpha_t \in (0, 1)$ for $t = 1, 2, ..., T$ are hyperparameters that define the noise ratio at each step. $\mathcal{N}(\bx; \boldsymbol{\mu}, \mathbf{\Sigma})$ denotes a Gaussian distribution with mean $\boldsymbol{\mu}$ and covariance $\mathbf{\Sigma}$. A useful property of this Gaussian transition kernel is that $\bx_t$ can be directly derived from $\bx_0$ by:
\begin{equation}
q(\bx_t|\bx_0) = \mathcal{N}\left(\bx_t; \sqrt{\bar{\alpha}_t}\bx_0, (1-\bar{\alpha}_t)\bI\right),
\label{eq:forward-property}
\end{equation}
where $\bar{\alpha}_t := \prod_{i=1}^t \alpha_i$. Thus, $\bx_t$ is given by $\bx_t = \sqrt{\bar{\alpha}_t}\bx_0 + \sqrt{1-\bar{\alpha}_t}\boldsymbol{\epsilon}$, where $\boldsymbol{\epsilon} \sim \mathcal{N}(\mathbf{0}, \bI)$. Typically, we set $\bar{\alpha}_T \approx 0$ so that $q(\bx_T) \approx \mathcal{N}(\bx_T; \mathbf{0}, \bI)$, allowing the reverse diffusion process to start from a random Gaussian noise.

The reverse transition kernel is parameterized by the neural networks $\boldsymbol{\mu}_{\theta}$ and $\mathbf{\Sigma}_{\theta}$:
\begin{equation}
p_\theta(\bx_{t-1}|\bx_t) = \mathcal{N}\left(\bx_{t-1}; \boldsymbol{\mu}_\theta(\bx_t, t), \mathbf{\Sigma}_\theta(\bx_t, t)\right),
\label{eq:reverse}
\end{equation}
where $\theta$ denotes $p_{\theta}$'s learnable parameters. 
The goal is to maximize the likelihood of the training sample $\bx_0$ by optimizing $p_\theta(\bx_0)$. This is achieved by minimizing the variational lower bound of the negative log-likelihood $\mathbb{E}[-\log p_\theta(\bx_0)]$.

DDPM simplifies the covariance matrix $\mathbf{\Sigma}_\theta$ in \cref{eq:reverse} to a constant-scaled matrix $\tilde{\beta}_t \bI$, where $\tilde{\beta}_t=\frac{1-\bar{\alpha}_{t-1}}{1-\bar{\alpha}_t}(1-\alpha_t)$ varies across each step to control noise. Additionally, the mean $\boldsymbol{\mu}$ in \cref{eq:reverse} is expressed as a function of a learnable noise term:
\begin{equation}
\boldsymbol{\mu}_\theta(\bx_t, t) = \frac{1}{\sqrt{\alpha_t}} \left(\bx_t - \frac{1-\alpha_t}{\sqrt{1-\bar{\alpha}_t}} \boldsymbol{\epsilon}_\theta(\bx_t, t) \right),
\end{equation}
where $\boldsymbol{\epsilon}_\theta$ is a network that predicts noise $\boldsymbol{\epsilon}$ for $\bx_t$ and $t$.
According to the property in \cref{eq:forward-property} and discarding the weight, DDPM simplifies the objective function to:
\begin{equation}
\mathbb{E}_{t,\bx_0,\boldsymbol{\epsilon}} \left[\left\| \boldsymbol{\epsilon} - \boldsymbol{\epsilon}_\theta \left( \sqrt{\bar{\alpha}_t} \bx_0 + \sqrt{1 - \bar{\alpha}_t} \boldsymbol{\epsilon}, t \right) \right\|^2 \right].
\end{equation}
Eventually, samples are generated by removing noise from $\bx_T \sim \mathcal{N}(\bx_T; \mathbf{0}, \bI)$. Specifically, for $t = T, T-1, ..., 1$,
\begin{equation}
\bx_{t-1} \leftarrow \frac{1}{\sqrt{\alpha_{t}}} (\bx_{t} - \frac{1-\alpha_t}{\sqrt{1-\bar{\alpha}_t}} \boldsymbol{\epsilon}_\theta(\bx_{t}, t)) + \sigma_t \bz,
\end{equation}
where $\bz \sim \mathcal{N}(\mathbf{0}, \bI)$ for $t = T, ..., 2$, and $\bz = \mathbf{0}$ for $t = 1$.

DDPM has been widely applied in generating molecules. Considering that DDPM is designed based on continuous data space, it is more commonly used for generating continuous 3D molecular structures, such as in EDM~\citep{EDM} and GeoDiff~\citep{GeoDiff}. For discrete 2D molecular structures, the discrete version of DDPM, Discrete Denoising Diffusion Probabilistic Model (D3PM), is typically employed, as exemplified by DiGress~\citep{DiGress}.



\begin{figure*}[t]
    \centering
    \resizebox{0.91\linewidth}{!}{
    \includegraphics{figures/figure2.pdf}}
    \caption{Illustration of molecular diffusion models, showcasing the forward and reverse processes. The three primary formulations—DDPM, SMLD, and SDE—are presented. Molecules can be generated in 2D space, 3D space, or jointly in 2D and 3D spaces.} 
    \label{fig:diffusion}
    \vspace{-6pt}
\end{figure*}

\subsection{Score Matching with Langevin Dynamics (SMLDs)}

SMLD~\citep{SMLD}, also known as score-based generative model (SGM), uses score matching theory to learn the score function of the data distribution,
% (i.e., the gradient of the log probability density), 
and combines it with Langevin dynamics for sampling. It comprises two main components: score matching and annealed Langevin dynamics (ALD). ALD generates samples iteratively using Langevin Monte Carlo, relying on the Stein score of a density function $q(\bx)$, defined as $\nabla_\bx \log q(\bx)$. Since the true distribution $q(\bx)$ is often unknown, score matching~\citep{ScoreMatching} approximates the Stein score with a neural network.

For efficiency, variants of score matching, such as denoising score matching~\citep{DenoisingScoreMatching},
are often used in practice.
Denoising score matching processes observed data using the forward transition kernel $q(\bx_t|\bx_0) = \mathcal{N}(\bx_t; \bx_0, \sigma_t^2 \bI)$, where $\sigma_t^2$ is a seiries of increasing noise levels for $t = 1, \ldots, T$, and then jointly estimates the Stein scores for the noise density functions $q_{\sigma_1}(\bx), q_{\sigma_2}(\bx), ..., q_{\sigma_T}(\bx)$.
The Stein score is approximated by a neural network $\bs_\theta(\bx, t)$ with $\theta$ as its learnable parameters. Thus, the objective function is defined as follows: 
\begin{equation}
\mathbb{E}_{t, \bx_0, \bx_t} \left[ \left\| \bs_\theta(\bx_t, t) - \nabla_{\bx_t} \log q_{\sigma_t}(\bx_t) \right\|^2 \right]. 
\end{equation}
With the Gaussian assumption of the forward transition kernel, the objective function can be rewritten as a tractable version: 
\begin{equation}
\mathbb{E}_{t, \bx_0, \bx_t} \left[ \delta(t) \left\| \bs_\theta(\bx_t, t) + \frac{\bx_t - \bx_0}{\sigma_t^2} \right\|^2 \right], 
\end{equation}
where $\delta(t)$ is a positive weight that depends on the noise scale $\sigma_t$. Once the score-matching network $\bs_\theta$ is trained, the ALD algorithm is used for sampling. It begins with a sequence of increasing noise levels $\sigma_1, \ldots, \sigma_T$ and an starting point \(\bx_{T,0} \sim \mathcal{N}(\mathbf{0}, \bI)\). For \(t = T, T - 1, \ldots, 0\), \(\bx_t\) is updated through \(N\) iterations that compute the following steps:
\begin{equation}
\bx_{t,n} \leftarrow \bx_{t,n-1} + \frac{1}{2} \eta_t \bs_\theta \left( \bx_{t,n-1}, t \right) + \sqrt{\eta_t} \bz, 
\end{equation}
where $\bz \sim \mathcal{N}(\mathbf{0}, \bI)$, $n = 1, \dots, N$, and $\eta_t$ is the update step. After $N$ iterations, the resulting $\bx_{t,N}$ becomes the starting point for the next $N$ iterations. $\bx_{0,N}$ will be final sample. 

SMLD has also been applied to generating molecules, such as in MDM~\citep{MDM} and LDM-3DG~\citep{LDM-3DG}. Subsequent studies demonstrat that SMLD and DDPM are theoretically equivalent and can both be regarded as discretizations of the SDE introduced in the next subsection.


\subsection{Stochastic Differential Equations (SDEs)}

Both DDPM and SMLD rely on discrete processes, requiring careful design of diffusion steps. \cite{SDE} formulate the forward process as an SDE, extending the discrete methods to continuous time space. The reverse process is modeled as a time-reverse SDE, enabling sampling by solving it. Let $\bw$ and $\bar{\bw}$ denote a standard Wiener process and its time-reverse, with continuous diffusion time $t \in [0, T]$. A general SDE is:
\begin{equation}
d\bx = f(\bx, t)dt + g(t)d\bw,
\end{equation}
where $f(\bx, t)$ and $g(t)$ are the drift coefficient and the diffusion coefficient for the diffusion process, respectively. 

The corresponding time-reverse SDE is defined as:
\begin{equation}
d\bx = \left[ f(\bx, t) - g(t)^2 \nabla_\bx \log q_t(\bx) \right] dt + g(t)d\bar{\bw}.
\end{equation}
Sampling from the probability flow ordinary differential equation (ODE) has the same distribution as the time-reverse SDE:
\begin{equation}
d\bx = \left[ f(\bx, t) - \frac{1}{2} g(t)^2 \nabla_\bx \log q_t(\bx) \right] dt.
\end{equation}
Here  $\nabla_\bx \log q_t(\bx)$ is the Stein score of the marginal
distribution of $\bx_t$, which is unknown but can be learned with a similar method as in
SMLD with the objective function:
\begin{equation}
\mathbb{E}_{t, \bx_0, \bx_t} \left[ \delta(t) \left\| \bs_\theta(\bx_t, t) - \nabla_{\bx_t} \log q_{0t}(\bx_t | \bx_0) \right\|^2 \right].
\label{eq:sde-objective}
\end{equation}

DDPM and SMLD can be regarded as discretizations of two SDEs.
Recall that $\alpha_t$ is a defined in DDPM and $\sigma_t^2$ denotes the noise level in SMLD. The SDE corresponding to DDPM is known as variance preserving (VP) SDE, defined as:
\begin{equation}
d\bx = -\frac{1}{2} \alpha(t)\bx dt + \sqrt{\alpha(t)}d\bw,
\end{equation}
where $\alpha(\cdot)$ is a continuous function, and $\alpha\left(\frac{t}{T}\right) = T(1 - \alpha_t)$ as $T \to \infty$. For the forward process of SMLD, the associated SDE is known as variance exploding (VE) SDE, defined
as:
\begin{equation}
d\bx = \sqrt{\frac{d \left[ \sigma(t)^2 \right]}{dt}} d\bw,
\end{equation}
where $\sigma(\cdot)$ is a continuous function, and $\sigma\left(\frac{t}{T}\right) = \sigma_t$ as $T \to \infty$.
Inspired by VP SDE, sub-VP SDE is designed and
performs especially well on likelihoods, given by:
\begin{equation}
d\bx = -\frac{1}{2} \alpha(t)\bx dt + \sqrt{\alpha(t) \left( 1 - e^{-2 \int_0^t \alpha(s)ds} \right)} d\bw.
\end{equation}

The objective function in \cref{eq:sde-objective} involves a perturbation distribution $q_{0t}(\bx_t | \bx_0)$ that varies for
different SDEs (i.e., VP SDE, VE SDE, sub-VP SDE). 
After $\bs_\theta(\bx, t)$ is trained, samples can be generated by solving the time-reverse SDE or the probability flow ODE with techniques such as ALD.

Because SDEs provide a continuous and flexible formulation that allows for improved control over generation processes, they have gradually replaced discrete-time formulations like DDPM and SMLD in molecular generative tasks~\citep{GDSS,EEGSDE,JODO}.


\subsection{More Variants}

These three formulations establish the theoretical foundation of diffusion models and demonstrate excellent performance in generative tasks. Building on them, diffusion models have spawned many variants and extensions aimed at enhancing generation efficiency or expanding application scenarios.
For example, 
\textit{Discrete Denoising Diffusion Probabilistic Models (D3PMs)}~\citep{D3PM} extend the DDPM to discrete data space, such as text or graphs. 
\textit{Latent Diffusion Models (LDMs)}~\citep{LDM} perform the diffusion process in latent space, significantly reducing computational complexity while maintaining generation quality.
\textit{Consistency Models (CMs)}~\citep{ConsistencyModel} focus on learning a single-step mapping from noise to data, enabling fast and high-quality sampling while maintaining consistency with the underlying data distribution.
\textit{Diffusion Bridges (DBs)}~\citep{DSB,SB-FBSDE,DDBM} extend diffusion models for generative tasks that connect different distributions, enabling efficient generation from one distribution to another.

These formulations propose innovative solutions tailored to different tasks, driving the widespread application in multi-modal generative tasks such as image, text, video, and graph~\citep{DiffusionSurvey1,DiffusionGraphSurvey1,DiffusionGraphSurvey2}.
\section{Data} 
\label{sec:data}

We now describe the data used for training and deploying our AI-based approach. 
Gathering this data required a significant digitization process, extracting and processing 5.2 million pages of deeds stretching back to the 1850s (\S~\ref{sec:digit}). We then supplemented this data with historical deed records available online from around the country (\S~\ref{sec:otherdata}), which has the coincidental benefit of enabling us to assess the model's robustness across jurisdictions.  Finally, we manually annotate 3,801 deeds to build a training dataset and held-out evaluation dataset for our AI pipeline (\S~\ref{sec:annotation}).


\subsection{Digitization, Collection, and Sharing of Real Property Deeds}
\label{sec:digit}
The Santa Clara County Clerk-Recorder's Office has an extensive archive of over 24 million real property deeds. Of these records, approximately 18 million -- issued since 1980 -- are stored digitally, while the remaining 6 million deeds -- created before 1980 -- were originally preserved on physical microfiche sheets. More than a decade prior to our work, the County had engaged a vendor to scan these records into a proprietary system known as Digital Reel; however, as we discuss in Appendix \ref{appendix_ocr}, the quality of these scans was poor and required significant post-processing.

Our partnership around exploring the use of AI began in October 2022. One of the notable barriers to transparency around deed records in California lies in a statutory mandate to charge fees for any copies of recorded documents.\footnote{Cal.\ Gov.\ Code \S~27366 provides: ``The fee for any copy of any other record or paper on file in the office of the recorder, when the copy is made by the recorder, shall be set by the board of supervisors in an amount necessary to recover the direct and indirect costs of providing the product or service or the cost of enforcing any regulation for which the fee or charge is levied.'' This provision has been subject to extensive litigation. See, e.g., California Public Records Research, Inc.\ v.\ County of Stanislaus, 246 Cal.\ App.\ 4th 1432 (2016);  California Public Records Research v.\ County of Yolo, 4 Cal.\ App.\ 5th 150 (2016).}  In other words, despite their status as public records, deed documents are available only on an individual fee basis. Given the massive scale of the review task, purchasing deed records would, of course, have been prohibitively expensive.\footnote{At a cost of \$4 for the first page and \$2 for each subsequent page, purchasing the 5.2M pages (with the average deed running 2.5 pages) might have cost over \$13 million.} Through our partnership, we developed unique a data sharing agreement, enabling the Stanford team  to process deed data, with the County retaining ownership of the records.

We began our work on samples of 20,000 pages of property deeds filed between 1900 and 1940, manually exported from the County's Digital Reel system. This 20,000-page sample enabled us to rapidly develop and refine our automated detection pipeline. 

After this piloting phase, the County extracted the full collection of pre-1980 scans in February 2024. This represents roughly 5.2 million pages of real property documents from 1865 to 1980. We focus our analysis on documents from 1902 to 1980 for two reasons. First, deeds filed prior to 1902 were handwritten rather than typed, and we found no available OCR tools to be effective at transcribing these documents.\footnote{We did explore developing a bespoke computer vision or multimodal text-vision system.} Second, records after 1980 contain protected fields like Social Security information, so we avoided ingesting sensitive data and potentially training our model on it, which may have raised privacy and legal concerns. As we note above and consistent with our results in Section~\ref{sec:evolution}, 1902 to 1980 likely covers the vast majority of racial covenants in the County; the first racial covenant we find was filed in 1907 and the last in 1974.

\subsection{Data Augmentation}
\label{sec:otherdata}
Both within California and across the nation, historical property deeds vary significantly in format, phrasing, and, when digitized, OCR quality. In order to build a system that is robust to these variations, we supplemented the Santa Clara County dataset with property deeds from around the nation, both with and without racial covenants.

Since property records in California counties are not freely accessible, we expanded our search to other counties in the United States. Using \href{https://govos.com/products/public-record-access/channel/}{GovOS Cloud Search}, we identified seven counties whose ``Official Records Search’’ platforms allowed users to freely search and download real property deeds, although downloads were limited to fifty records per batch.\footnote{\url{https://kofilehelp.zendesk.com/hc/en-us/sections/4416665864343-Cloud-Search-Active-Sites}.} These platforms enabled searches by metadata and keyword terms. To gather a seed dataset of deeds with a high probability of containing racial covenants, we conducted manual searches for terms typically associated with such covenants, such as ``No person of,'' ``Caucasian,'' ``Negro,'' and other relevant racial terms. This method provided us with more than 10,000 property deeds from seven counties: Bexar County, Texas; Cuyahoga County, Ohio; Denton County, Texas; Franklin County, Ohio; Hidalgo County, Texas; and Lawrence County, Pennsylvania. This approach not only helped us collect relevant data but also allowed us to assess the generalizability of our model across different jurisdictions. As we discuss below, we specifically investigate the limitations of keyword-based approaches, and find that context-aware language models boost performance substantially. 




\subsection{Annotation} \label{sec:annotation}

We labeled our data collection by identifying quotes that contain racial covenants on each page. This annotation occurred over three stages: initial training data generation, model prediction review, and rich annotation.

In our initial round of annotation, we selected a sample of 3,000 pages in our collection based on keywords that almost certainly indicate the presence of a racial covenant in the deed text. These include terms like ``Negro,'' ``Mongolian,'' and ``Asiatic.'' We partnered with data annotation company CloudFactory to help us identify and label racial covenants in these pages.\footnote{During our collaboration with CloudFactory for data annotation, we carefully prepared comprehensive documentation to guide the annotators through the task. Given the potentially sensitive nature of the material—historical property deeds containing racially restrictive covenants, as well as accounts that could be considered offensive or harmful to some readers—we issued a clear advisory to approach the content with care. We emphasized that the annotators could stop the task at any point if they felt uncomfortable. In addition, we consulted with CloudFactory’s management to ensure that appropriate counseling and support resources would be available to their team, should any annotator feel the need for assistance or support. Our priority was to handle this material with the utmost sensitivity, while ensuring the well-being of those involved in the annotation process.}

After training models and generating predictions, we reviewed their performance. For all positive predictions, we labeled whether they were true positives or false positives. These ensured that we verified the small number of positive examples as well as hard negative examples. We additionally sampled and verified negative predictions to ensure some balance in the data. These new annotations were incorporated into the training set of future models.

Recognizing the need to easily validate model predictions and locate racial covenants on a page, we built a web application to assist with rich annotation. This made it easy to precisely select a bounding box on the image of the deed book page and compute a text span for the annotation process, while simultaneously allowing us to visualize predictions for verification.

All combined, including both Santa Clara County documents and documents from across the country, we collected 3,801 annotations of deed pages, of which 2,987 (78.6\%) contained a racially restrictive covenant. Notably, this annotation requires human review, but at a much smaller scale than reviewing all records.  






 





\section{Molecular Generative Tasks}\label{sec:task}


In this section, we explore various molecular generative tasks that leverage diffusion models, as outlined in \cref{fig:taxonomy}. These tasks are crucial for advancing molecular design and discovery, providing innovative solutions across different domains.

\subsection{De Novo Generation}
De novo generation involves creating novel molecular structures from scratch. This approach is essential for discovering new compounds without relying on existing molecular templates. It includes two main sub-tasks: unconditional and conditional generation.


\paratitle{Unconditional generation.}
Unconditional generation focuses on producing molecules without specific constraints. Starting from a random noise vector, these models generate entirely new molecular structures, exploring the vast chemical space for novel applications.



\paratitle{Conditional Generation}
Conditional generation tailors molecule creation based on specific conditions, such as desired properties, targets, or fragments, allowing for more directed and efficient molecular design. By incorporating these conditions, diffusion models produce molecules that meet predefined criteria. 
Existing works can be further divided into four categories based on the type of conditions applied.

\textit{Property-based molecular generation}, also known as \textit{inverse molecule design}, aims to generate molecules with desired properties such as bioactivity and synthesizability.
More specifically, the inverse molecule design can be further divided into single-property conditioning~\citep{CDGS,EDM,GeoLDM,MDM} and multiple-property conditioning~\citep{DiGress,GraphDiT,EEGSDE}.
Among them, MOOD~\citep{MOOD} and CGD~\citep{CGD} also focus on generating structurally novel molecules outside the training distribution, referred to as OOD molecule generation.

\textit{Target-based molecular generation}, also known as \textit{structure-based drug design (SBDD)}, generates molecules based on the 3D structure of target binding pockets, aiming to enhance interaction with specific targets.
IRDiff~\citep{IRDiff} introduces interaction-based retrieval to generate target-specific molecules based on retrieved high-affinity ligand references.

\textit{Fragment-based molecular generation} specifies the generation of molecules with particular fragments. DiffLinker~\citep{DiffLinker} focuses on linker design, generating linkers that connect fragments into a complete molecule.


\textit{Composition-based molecular generation} restricts the elemental composition of generated molecules, ensuring they meet specific compositional criteria~\citep{UniMat}.


\subsection{Molecular Optimization}
Molecular optimization tasks aim to improve existing molecules for better performance or properties, differing from de novo generation by focusing on modifying known structures rather than creating new ones. Starting with an existing molecule, these tasks refine it to enhance its properties or performance. This task includes scaffold hopping, R-group design, and generalized optimization.


\textit{Scaffold hopping} involves modifying molecular scaffolds to discover new compounds, transforming known scaffolds into new ones that retain biological activity~\citep{DiffHopp}.


\textit{R-group design} focuses on optimizing molecules by fine-tuning the properties of lead compounds through the adjustment of specific R-groups~\citep{DecompOpt}.


\textit{Generalized optimization} is a flexible approach to optimize molecules without being restricted to specific strategies like scaffold hopping or R-group design. 
DiffSBDD~\citep{DiffSBDD} allows for a broader range of structural changes, as long as the optimized molecule maintains a certain level of similarity to the original structure. This flexibility enables the exploration of diverse pathways to improve properties.


\subsection{Conformer Generation}
Conformer generation involves predicting the 3D conformers of a molecule based on its 2D topological structure. This task is crucial for understanding the spatial arrangement of atoms within a molecule, which is essential for predicting molecular interactions, reactivity, and properties.
Generated 3D conformers reflect the molecule's potential energy landscape and geometric constraints.
GeoDiff~\citep{GeoDiff} and Torsional Diffusion~\citep{TorsionalDiffusion} employ diffusion models to generate 3D molecular conformers in Cartesian space and torsion angle space, respectively. DiSCO~\citep{DiSCO} further optimizes the predicted conformers with Diffusion Bridge.



\subsection{Molecular Docking}
Molecular docking tasks involve predicting how molecules interact with biological targets, a key step in drug discovery for assessing binding affinity and specificity. By analyzing a molecule and a target structure, DiffDock~\citep{DiffDock} predicts the binding pose with the diffusion model.
Re-Dock~\citep{Re-Dock} further utilizes the diffusion bridge for flexible and realistic molecular docking, which predicts the binding poses of ligands and pocket sidechains simultaneously.



\subsection{Transition State Generation}
Transition state generation focuses on predicting the 3D structure of transition states in chemical reactions, using the reactants and products as inputs. This task is vital for understanding reaction mechanisms, as the transition state represents the highest energy point along the reaction pathway. Accurate modeling of these states provides insights into reaction kinetics and can aid in the design of catalysts and optimization of reaction conditions~\citep{OA-ReactDiff}.
\section{Discussion and Future Direction}
In this section, we discuss the current state and challenges in diffusion models for molecules and outline several promising directions for future research to advance this area.

\paratitle{Complete data modality.}
Most existing works fall under the category of generating molecules in 3D space,
neglecting 2D topology.
Considering the complementary nature of 2D and 3D structures, generating molecules in a joint 2D and 3D space holds significant potential for producing more realistic molecules.
This approach has proven effective in de novo generation~\citep{MUDiff,JODO}, but its broader potential in other generative tasks remains underexplored.

\paratitle{Sophisticated diffusion models.}
As summarized in \cref{tab:summarizations}, 
the diffusion models employed in existing works exhibit a wide variety of formulations. Regarding the time space, there is a shift from discrete-time methods to more generalized continuous-time SDEs. In terms of the data space, an open challenge lies in handling the discrete molecular components (e.g., atom and bond types) alongside the continuous components (e.g., coordinates). Moreover, advanced formulations and techniques, such as flow matching and efficient sampling, remain underutilized.

\paratitle{Challenging generative tasks.}
Many existing works focus on fundamental tasks,
like unconditional or single-conditional generation, 
with insufficient attention to more practical generative tasks, such as multi-conditional generation, molecular optimization, and docking. 
Furthermore, poor performance on large molecules in GEOM-Drugs compared to small molecules in QM9, highlights room for improvement. Additionally, extending molecular generation to complex~\citep{DynamicBind} while considering inter-molecular interactions, presents a another promising yet challenging direction.

\paratitle{Expressive network architectures.}
Existing methods rely on relatively classical network architectures like EGNNs. 
Recent advances in more expressive equivariant neural networks offer new opportunities. Incorporating more powerful architectures into molecular diffusion models could further enhance their performance and effectiveness.

\paratitle{Relationship between molecular generation and molecular representation.}
With the increasing recognition of diffusion models' ability to learn representations, exploring the relationship between molecular generation and molecular representation based on diffusion models emerges as a promising direction. MoleculeSDE~\citep{MoleculeSDE}, SubgDiff~\citep{SubgDiff}, and UniGEM~\citep{UniGEM} mark pioneering steps, but there remains significant room for further research.

\section{Conclusion and future directions} \label{sec:conclusion}

In this paper we proposed a nested MLMC framework that offers important computational savings by performing most calculations in low precision and exploiting approximate random normal variables for the low precision path calculations. The low precision calculations could be performed in fixed precision on an FPGA for greater efficiency, and we suggested a procedure to optimise the bit-widths of every variable at each Monte Carlo level. This is an important improvement over previous mixed precision MLMC frameworks which held the lower precision fixed \cite{Rounding_error_oliver} or defined uniform bit-width at every level heuristically \cite{brugger2014mixed}. Our numerical results suggest that for the first levels our procedure reduces the cost at these levels by a factor 5 or 7. Hence the overall savings are significant since most paths are calculated on the first levels. Our approach would be even more efficient for the Milstein scheme because its higher order strong convergence leads to a greater proportion of the computational costs being on the coarsest levels.

The next stage of the research project will be to implement the RNG methods and the nested framework on FPGAs to determine the hardware requirements and confirm the extent of the computational savings. It would also be good to compare the performance benefits to using half-precision floating point arithmetic on GPUs or CPUs for the low-accuracy computations.






% \clearpage
%% The file named.bst is a bibliography style file for BibTeX 0.99c
\bibliographystyle{named}
\bibliography{ijcai25}

\end{document}

