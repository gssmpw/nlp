%%%% ijcai25.tex

\typeout{IJCAI--25 Instructions for Authors}

% These are the instructions for authors for IJCAI-25.

\documentclass{article}
\pdfpagewidth=8.5in
\pdfpageheight=11in

% The file ijcai25.sty is a copy from ijcai22.sty
% The file ijcai22.sty is NOT the same as previous years'
\usepackage{ijcai25}

% Use the postscript times font!
\usepackage{times}
\usepackage{soul}

\usepackage{url}
\usepackage[usenames,svgnames]{xcolor}
\usepackage[colorlinks,
citecolor=NavyBlue,
linkcolor=NavyBlue,
urlcolor=NavyBlue]{hyperref}

\usepackage[utf8]{inputenc}
\usepackage[small]{caption}
\usepackage{graphicx}
\usepackage{amsmath}
\usepackage{amsthm}
\usepackage{booktabs}
\usepackage{algorithm}
\usepackage{algorithmic}
\usepackage[switch]{lineno}
\usepackage{enumitem}


\usepackage{amsfonts}  % mathbb

% tikz forest (tikz and forest are imported in mydef.sty)
\usepackage{mydef}

\usepackage{booktabs}
\usepackage{tabularx}
\usepackage{framed}
\usepackage{bbding}
\usepackage{subfloat}
\usepackage{adjustbox}
\usepackage{multirow}
\usepackage{pifont}
\usepackage{natbib}

\usepackage{makecell}

\usepackage[capitalize]{cleveref}
\crefname{figure}{Figure}{Figures}
\crefname{table}{Table}{Tables}
% \crefname{equation}{Equation}{Equations}
% \crefformat{equation}{Equation~#2#1#3}
\crefformat{equation}{Eq.~(#2#1#3)}



%% Comments
% \newcommand{\annotator}[2]{\csdef{#1}##1{{\color{#2} [\textbf{\MakeUppercase #1}: ##1]}}}
\newcommand{\annotator}[2]{\csdef{#1}##1{{\color{#2} [\textbf{\MakeUppercase #1}: ##1]}}}
\annotator{leon}{blue}


\newcommand{\paratitle}[1]{\vspace{0.8ex}\noindent \textbf{#1}}


% Comment out this line in the camera-ready submission
% \linenumbers

\urlstyle{same}



% the following package is optional:
%\usepackage{latexsym}

% See https://www.overleaf.com/learn/latex/theorems_and_proofs
% for a nice explanation of how to define new theorems, but keep
% in mind that the amsthm package is already included in this
% template and that you must *not* alter the styling.
\newtheorem{example}{Example}
\newtheorem{theorem}{Theorem}

% Following comment is from ijcai97-submit.tex:
% The preparation of these files was supported by Schlumberger Palo Alto
% Research, AT\&T Bell Laboratories, and Morgan Kaufmann Publishers.
% Shirley Jowell, of Morgan Kaufmann Publishers, and Peter F.
% Patel-Schneider, of AT\&T Bell Laboratories collaborated on their
% preparation.

% These instructions can be modified and used in other conferences as long
% as credit to the authors and supporting agencies is retained, this notice
% is not changed, and further modification or reuse is not restricted.
% Neither Shirley Jowell nor Peter F. Patel-Schneider can be listed as
% contacts for providing assistance without their prior permission.

% To use for other conferences, change references to files and the
% conference appropriate and use other authors, contacts, publishers, and
% organizations.
% Also change the deadline and address for returning papers and the length and
% page charge instructions.
% Put where the files are available in the appropriate places.


% PDF Info Is REQUIRED.

% Please leave this \pdfinfo block untouched both for the submission and
% Camera Ready Copy. Do not include Title and Author information in the pdfinfo section
\pdfinfo{
/TemplateVersion (IJCAI.2025.0)
}

\title{Diffusion Models for Molecules: A Survey of Methods and Tasks}


% Single author syntax
% \author{
%     Author Name
%     \affiliations
%     Affiliation
%     \emails
%     email@example.com
% }

% Multiple author syntax (remove the single-author syntax above and the \iffalse ... \fi here)
% \iffalse
\author{
Liang Wang$^{1,2}$
\and
Chao Song$^3$\and
Zhiyuan Liu$^{4}$\and
Yu Rong$^{5}$\and \\
% Qiang Liu$^{1,2}$\thanks{Corresponding author.}\and 
Qiang Liu$^{1,2}$\and 
Shu Wu$^{1,2}$\And
Liang Wang$^{1,2}$\\
\affiliations
$^1$NLPR, MAIS, Institute of Automation, Chinese Academy of Sciences\\
$^2$School of Artificial Intelligence, University of Chinese Academy of Sciences\\
$^3$Northwestern Polytechnical University\\
$^4$National University of Singapore\\
$^5$Alibaba DAMO Academy\\
\emails
liang.wang@cripac.ia.ac.cn,
csong@mail.nwpu.edu.cn,
acharkq@gmail.com,\\
yu.rong@hotmail.com,
\{qiang.liu, shu.wu, wangliang\}@nlpr.ia.ac.cn
}
% \fi

\begin{document}


\maketitle

\begin{abstract}
    % Molecular generation 
    Generative tasks about molecules, including but not limited to molecule generation, are crucial for drug discovery and material design, and have consistently attracted significant attention. In recent years, diffusion models have emerged as an impressive class of deep generative models, sparking extensive research and leading to numerous studies on their application to molecular generative tasks. Despite the proliferation of related work, there remains a notable lack of up-to-date and systematic surveys in this area. Particularly, due to the diversity of diffusion model formulations, molecular data modalities, and generative task types, the research landscape is challenging to navigate, hindering understanding and limiting the area's growth. To address this, this paper conducts a comprehensive survey of diffusion model-based molecular generative methods. We systematically review the research from the perspectives of methodological formulations, data modalities, and task types, offering a novel taxonomy. This survey aims to facilitate understanding and further flourishing development in this area.
    The relevant papers are summarized at: \url{https://github.com/AzureLeon1/awesome-molecular-diffusion-models}.
\end{abstract}

\section{Introduction}

Video generation has garnered significant attention owing to its transformative potential across a wide range of applications, such media content creation~\citep{polyak2024movie}, advertising~\citep{zhang2024virbo,bacher2021advert}, video games~\citep{yang2024playable,valevski2024diffusion, oasis2024}, and world model simulators~\citep{ha2018world, videoworldsimulators2024, agarwal2025cosmos}. Benefiting from advanced generative algorithms~\citep{goodfellow2014generative, ho2020denoising, liu2023flow, lipman2023flow}, scalable model architectures~\citep{vaswani2017attention, peebles2023scalable}, vast amounts of internet-sourced data~\citep{chen2024panda, nan2024openvid, ju2024miradata}, and ongoing expansion of computing capabilities~\citep{nvidia2022h100, nvidia2023dgxgh200, nvidia2024h200nvl}, remarkable advancements have been achieved in the field of video generation~\citep{ho2022video, ho2022imagen, singer2023makeavideo, blattmann2023align, videoworldsimulators2024, kuaishou2024klingai, yang2024cogvideox, jin2024pyramidal, polyak2024movie, kong2024hunyuanvideo, ji2024prompt}.


In this work, we present \textbf{\ours}, a family of rectified flow~\citep{lipman2023flow, liu2023flow} transformer models designed for joint image and video generation, establishing a pathway toward industry-grade performance. This report centers on four key components: data curation, model architecture design, flow formulation, and training infrastructure optimization—each rigorously refined to meet the demands of high-quality, large-scale video generation.


\begin{figure}[ht]
    \centering
    \begin{subfigure}[b]{0.82\linewidth}
        \centering
        \includegraphics[width=\linewidth]{figures/t2i_1024.pdf}
        \caption{Text-to-Image Samples}\label{fig:main-demo-t2i}
    \end{subfigure}
    \vfill
    \begin{subfigure}[b]{0.82\linewidth}
        \centering
        \includegraphics[width=\linewidth]{figures/t2v_samples.pdf}
        \caption{Text-to-Video Samples}\label{fig:main-demo-t2v}
    \end{subfigure}
\caption{\textbf{Generated samples from \ours.} Key components are highlighted in \textcolor{red}{\textbf{RED}}.}\label{fig:main-demo}
\end{figure}


First, we present a comprehensive data processing pipeline designed to construct large-scale, high-quality image and video-text datasets. The pipeline integrates multiple advanced techniques, including video and image filtering based on aesthetic scores, OCR-driven content analysis, and subjective evaluations, to ensure exceptional visual and contextual quality. Furthermore, we employ multimodal large language models~(MLLMs)~\citep{yuan2025tarsier2} to generate dense and contextually aligned captions, which are subsequently refined using an additional large language model~(LLM)~\citep{yang2024qwen2} to enhance their accuracy, fluency, and descriptive richness. As a result, we have curated a robust training dataset comprising approximately 36M video-text pairs and 160M image-text pairs, which are proven sufficient for training industry-level generative models.

Secondly, we take a pioneering step by applying rectified flow formulation~\citep{lipman2023flow} for joint image and video generation, implemented through the \ours model family, which comprises Transformer architectures with 2B and 8B parameters. At its core, the \ours framework employs a 3D joint image-video variational autoencoder (VAE) to compress image and video inputs into a shared latent space, facilitating unified representation. This shared latent space is coupled with a full-attention~\citep{vaswani2017attention} mechanism, enabling seamless joint training of image and video. This architecture delivers high-quality, coherent outputs across both images and videos, establishing a unified framework for visual generation tasks.


Furthermore, to support the training of \ours at scale, we have developed a robust infrastructure tailored for large-scale model training. Our approach incorporates advanced parallelism strategies~\citep{jacobs2023deepspeed, pytorch_fsdp} to manage memory efficiently during long-context training. Additionally, we employ ByteCheckpoint~\citep{wan2024bytecheckpoint} for high-performance checkpointing and integrate fault-tolerant mechanisms from MegaScale~\citep{jiang2024megascale} to ensure stability and scalability across large GPU clusters. These optimizations enable \ours to handle the computational and data challenges of generative modeling with exceptional efficiency and reliability.


We evaluate \ours on both text-to-image and text-to-video benchmarks to highlight its competitive advantages. For text-to-image generation, \ours-T2I demonstrates strong performance across multiple benchmarks, including T2I-CompBench~\citep{huang2023t2i-compbench}, GenEval~\citep{ghosh2024geneval}, and DPG-Bench~\citep{hu2024ella_dbgbench}, excelling in both visual quality and text-image alignment. In text-to-video benchmarks, \ours-T2V achieves state-of-the-art performance on the UCF-101~\citep{ucf101} zero-shot generation task. Additionally, \ours-T2V attains an impressive score of \textbf{84.85} on VBench~\citep{huang2024vbench}, securing the top position on the leaderboard (as of 2025-01-25) and surpassing several leading commercial text-to-video models. Qualitative results, illustrated in \Cref{fig:main-demo}, further demonstrate the superior quality of the generated media samples. These findings underscore \ours's effectiveness in multi-modal generation and its potential as a high-performing solution for both research and commercial applications.
\section{Formulations of Diffusion Models}\label{sec:diffusion}

This fundamental framework of diffusion models comprises two parts: the \textit{forward diffusion process} and the \textit{reverse generation process}, as shown in \cref{fig:diffusion}. In the forward process, the model progressively adds noise to real data, eventually approaching a simple prior distribution.
In the reverse process, the model learns to progressively restore the data distribution from noise.
The reverse process is typically parameterized using neural networks.

This framework can be formulated in various ways, such as the Denoising Diffusion Probabilistic Models~\citep{DDPM}, Score Matching with Langevin Dynamics~\citep{SMLD}, the generalized Stochastic Differential Equations~\citep{SDE}, and other variants.


\subsection{Denoising Diffusion Probabilistic Models (DDPMs)}

DDPM~\citep{Diffusion,DDPM} is a classical diffusion model that performs step-by-step denoising using a fixed noise schedule. It employs two Markov chains for the forward and reverse processes. 

Starting with the original noise-free data $\bx_0$, the forward process transforms it into a sequence of noisy data $\bx_1, \bx_2, \ldots, \bx_T$ using a forward transition kernel:
\begin{equation}
q(\bx_t|\bx_{t-1}) = \mathcal{N}\left(\bx_t; \sqrt{\alpha_t}\bx_{t-1}, (1-\alpha_t)\bI\right),
\end{equation}
where $\alpha_t \in (0, 1)$ for $t = 1, 2, ..., T$ are hyperparameters that define the noise ratio at each step. $\mathcal{N}(\bx; \boldsymbol{\mu}, \mathbf{\Sigma})$ denotes a Gaussian distribution with mean $\boldsymbol{\mu}$ and covariance $\mathbf{\Sigma}$. A useful property of this Gaussian transition kernel is that $\bx_t$ can be directly derived from $\bx_0$ by:
\begin{equation}
q(\bx_t|\bx_0) = \mathcal{N}\left(\bx_t; \sqrt{\bar{\alpha}_t}\bx_0, (1-\bar{\alpha}_t)\bI\right),
\label{eq:forward-property}
\end{equation}
where $\bar{\alpha}_t := \prod_{i=1}^t \alpha_i$. Thus, $\bx_t$ is given by $\bx_t = \sqrt{\bar{\alpha}_t}\bx_0 + \sqrt{1-\bar{\alpha}_t}\boldsymbol{\epsilon}$, where $\boldsymbol{\epsilon} \sim \mathcal{N}(\mathbf{0}, \bI)$. Typically, we set $\bar{\alpha}_T \approx 0$ so that $q(\bx_T) \approx \mathcal{N}(\bx_T; \mathbf{0}, \bI)$, allowing the reverse diffusion process to start from a random Gaussian noise.

The reverse transition kernel is parameterized by the neural networks $\boldsymbol{\mu}_{\theta}$ and $\mathbf{\Sigma}_{\theta}$:
\begin{equation}
p_\theta(\bx_{t-1}|\bx_t) = \mathcal{N}\left(\bx_{t-1}; \boldsymbol{\mu}_\theta(\bx_t, t), \mathbf{\Sigma}_\theta(\bx_t, t)\right),
\label{eq:reverse}
\end{equation}
where $\theta$ denotes $p_{\theta}$'s learnable parameters. 
The goal is to maximize the likelihood of the training sample $\bx_0$ by optimizing $p_\theta(\bx_0)$. This is achieved by minimizing the variational lower bound of the negative log-likelihood $\mathbb{E}[-\log p_\theta(\bx_0)]$.

DDPM simplifies the covariance matrix $\mathbf{\Sigma}_\theta$ in \cref{eq:reverse} to a constant-scaled matrix $\tilde{\beta}_t \bI$, where $\tilde{\beta}_t=\frac{1-\bar{\alpha}_{t-1}}{1-\bar{\alpha}_t}(1-\alpha_t)$ varies across each step to control noise. Additionally, the mean $\boldsymbol{\mu}$ in \cref{eq:reverse} is expressed as a function of a learnable noise term:
\begin{equation}
\boldsymbol{\mu}_\theta(\bx_t, t) = \frac{1}{\sqrt{\alpha_t}} \left(\bx_t - \frac{1-\alpha_t}{\sqrt{1-\bar{\alpha}_t}} \boldsymbol{\epsilon}_\theta(\bx_t, t) \right),
\end{equation}
where $\boldsymbol{\epsilon}_\theta$ is a network that predicts noise $\boldsymbol{\epsilon}$ for $\bx_t$ and $t$.
According to the property in \cref{eq:forward-property} and discarding the weight, DDPM simplifies the objective function to:
\begin{equation}
\mathbb{E}_{t,\bx_0,\boldsymbol{\epsilon}} \left[\left\| \boldsymbol{\epsilon} - \boldsymbol{\epsilon}_\theta \left( \sqrt{\bar{\alpha}_t} \bx_0 + \sqrt{1 - \bar{\alpha}_t} \boldsymbol{\epsilon}, t \right) \right\|^2 \right].
\end{equation}
Eventually, samples are generated by removing noise from $\bx_T \sim \mathcal{N}(\bx_T; \mathbf{0}, \bI)$. Specifically, for $t = T, T-1, ..., 1$,
\begin{equation}
\bx_{t-1} \leftarrow \frac{1}{\sqrt{\alpha_{t}}} (\bx_{t} - \frac{1-\alpha_t}{\sqrt{1-\bar{\alpha}_t}} \boldsymbol{\epsilon}_\theta(\bx_{t}, t)) + \sigma_t \bz,
\end{equation}
where $\bz \sim \mathcal{N}(\mathbf{0}, \bI)$ for $t = T, ..., 2$, and $\bz = \mathbf{0}$ for $t = 1$.

DDPM has been widely applied in generating molecules. Considering that DDPM is designed based on continuous data space, it is more commonly used for generating continuous 3D molecular structures, such as in EDM~\citep{EDM} and GeoDiff~\citep{GeoDiff}. For discrete 2D molecular structures, the discrete version of DDPM, Discrete Denoising Diffusion Probabilistic Model (D3PM), is typically employed, as exemplified by DiGress~\citep{DiGress}.



\begin{figure*}[t]
    \centering
    \resizebox{0.91\linewidth}{!}{
    \includegraphics{figures/figure2.pdf}}
    \caption{Illustration of molecular diffusion models, showcasing the forward and reverse processes. The three primary formulations—DDPM, SMLD, and SDE—are presented. Molecules can be generated in 2D space, 3D space, or jointly in 2D and 3D spaces.} 
    \label{fig:diffusion}
    \vspace{-6pt}
\end{figure*}

\subsection{Score Matching with Langevin Dynamics (SMLDs)}

SMLD~\citep{SMLD}, also known as score-based generative model (SGM), uses score matching theory to learn the score function of the data distribution,
% (i.e., the gradient of the log probability density), 
and combines it with Langevin dynamics for sampling. It comprises two main components: score matching and annealed Langevin dynamics (ALD). ALD generates samples iteratively using Langevin Monte Carlo, relying on the Stein score of a density function $q(\bx)$, defined as $\nabla_\bx \log q(\bx)$. Since the true distribution $q(\bx)$ is often unknown, score matching~\citep{ScoreMatching} approximates the Stein score with a neural network.

For efficiency, variants of score matching, such as denoising score matching~\citep{DenoisingScoreMatching},
are often used in practice.
Denoising score matching processes observed data using the forward transition kernel $q(\bx_t|\bx_0) = \mathcal{N}(\bx_t; \bx_0, \sigma_t^2 \bI)$, where $\sigma_t^2$ is a seiries of increasing noise levels for $t = 1, \ldots, T$, and then jointly estimates the Stein scores for the noise density functions $q_{\sigma_1}(\bx), q_{\sigma_2}(\bx), ..., q_{\sigma_T}(\bx)$.
The Stein score is approximated by a neural network $\bs_\theta(\bx, t)$ with $\theta$ as its learnable parameters. Thus, the objective function is defined as follows: 
\begin{equation}
\mathbb{E}_{t, \bx_0, \bx_t} \left[ \left\| \bs_\theta(\bx_t, t) - \nabla_{\bx_t} \log q_{\sigma_t}(\bx_t) \right\|^2 \right]. 
\end{equation}
With the Gaussian assumption of the forward transition kernel, the objective function can be rewritten as a tractable version: 
\begin{equation}
\mathbb{E}_{t, \bx_0, \bx_t} \left[ \delta(t) \left\| \bs_\theta(\bx_t, t) + \frac{\bx_t - \bx_0}{\sigma_t^2} \right\|^2 \right], 
\end{equation}
where $\delta(t)$ is a positive weight that depends on the noise scale $\sigma_t$. Once the score-matching network $\bs_\theta$ is trained, the ALD algorithm is used for sampling. It begins with a sequence of increasing noise levels $\sigma_1, \ldots, \sigma_T$ and an starting point \(\bx_{T,0} \sim \mathcal{N}(\mathbf{0}, \bI)\). For \(t = T, T - 1, \ldots, 0\), \(\bx_t\) is updated through \(N\) iterations that compute the following steps:
\begin{equation}
\bx_{t,n} \leftarrow \bx_{t,n-1} + \frac{1}{2} \eta_t \bs_\theta \left( \bx_{t,n-1}, t \right) + \sqrt{\eta_t} \bz, 
\end{equation}
where $\bz \sim \mathcal{N}(\mathbf{0}, \bI)$, $n = 1, \dots, N$, and $\eta_t$ is the update step. After $N$ iterations, the resulting $\bx_{t,N}$ becomes the starting point for the next $N$ iterations. $\bx_{0,N}$ will be final sample. 

SMLD has also been applied to generating molecules, such as in MDM~\citep{MDM} and LDM-3DG~\citep{LDM-3DG}. Subsequent studies demonstrat that SMLD and DDPM are theoretically equivalent and can both be regarded as discretizations of the SDE introduced in the next subsection.


\subsection{Stochastic Differential Equations (SDEs)}

Both DDPM and SMLD rely on discrete processes, requiring careful design of diffusion steps. \cite{SDE} formulate the forward process as an SDE, extending the discrete methods to continuous time space. The reverse process is modeled as a time-reverse SDE, enabling sampling by solving it. Let $\bw$ and $\bar{\bw}$ denote a standard Wiener process and its time-reverse, with continuous diffusion time $t \in [0, T]$. A general SDE is:
\begin{equation}
d\bx = f(\bx, t)dt + g(t)d\bw,
\end{equation}
where $f(\bx, t)$ and $g(t)$ are the drift coefficient and the diffusion coefficient for the diffusion process, respectively. 

The corresponding time-reverse SDE is defined as:
\begin{equation}
d\bx = \left[ f(\bx, t) - g(t)^2 \nabla_\bx \log q_t(\bx) \right] dt + g(t)d\bar{\bw}.
\end{equation}
Sampling from the probability flow ordinary differential equation (ODE) has the same distribution as the time-reverse SDE:
\begin{equation}
d\bx = \left[ f(\bx, t) - \frac{1}{2} g(t)^2 \nabla_\bx \log q_t(\bx) \right] dt.
\end{equation}
Here  $\nabla_\bx \log q_t(\bx)$ is the Stein score of the marginal
distribution of $\bx_t$, which is unknown but can be learned with a similar method as in
SMLD with the objective function:
\begin{equation}
\mathbb{E}_{t, \bx_0, \bx_t} \left[ \delta(t) \left\| \bs_\theta(\bx_t, t) - \nabla_{\bx_t} \log q_{0t}(\bx_t | \bx_0) \right\|^2 \right].
\label{eq:sde-objective}
\end{equation}

DDPM and SMLD can be regarded as discretizations of two SDEs.
Recall that $\alpha_t$ is a defined in DDPM and $\sigma_t^2$ denotes the noise level in SMLD. The SDE corresponding to DDPM is known as variance preserving (VP) SDE, defined as:
\begin{equation}
d\bx = -\frac{1}{2} \alpha(t)\bx dt + \sqrt{\alpha(t)}d\bw,
\end{equation}
where $\alpha(\cdot)$ is a continuous function, and $\alpha\left(\frac{t}{T}\right) = T(1 - \alpha_t)$ as $T \to \infty$. For the forward process of SMLD, the associated SDE is known as variance exploding (VE) SDE, defined
as:
\begin{equation}
d\bx = \sqrt{\frac{d \left[ \sigma(t)^2 \right]}{dt}} d\bw,
\end{equation}
where $\sigma(\cdot)$ is a continuous function, and $\sigma\left(\frac{t}{T}\right) = \sigma_t$ as $T \to \infty$.
Inspired by VP SDE, sub-VP SDE is designed and
performs especially well on likelihoods, given by:
\begin{equation}
d\bx = -\frac{1}{2} \alpha(t)\bx dt + \sqrt{\alpha(t) \left( 1 - e^{-2 \int_0^t \alpha(s)ds} \right)} d\bw.
\end{equation}

The objective function in \cref{eq:sde-objective} involves a perturbation distribution $q_{0t}(\bx_t | \bx_0)$ that varies for
different SDEs (i.e., VP SDE, VE SDE, sub-VP SDE). 
After $\bs_\theta(\bx, t)$ is trained, samples can be generated by solving the time-reverse SDE or the probability flow ODE with techniques such as ALD.

Because SDEs provide a continuous and flexible formulation that allows for improved control over generation processes, they have gradually replaced discrete-time formulations like DDPM and SMLD in molecular generative tasks~\citep{GDSS,EEGSDE,JODO}.


\subsection{More Variants}

These three formulations establish the theoretical foundation of diffusion models and demonstrate excellent performance in generative tasks. Building on them, diffusion models have spawned many variants and extensions aimed at enhancing generation efficiency or expanding application scenarios.
For example, 
\textit{Discrete Denoising Diffusion Probabilistic Models (D3PMs)}~\citep{D3PM} extend the DDPM to discrete data space, such as text or graphs. 
\textit{Latent Diffusion Models (LDMs)}~\citep{LDM} perform the diffusion process in latent space, significantly reducing computational complexity while maintaining generation quality.
\textit{Consistency Models (CMs)}~\citep{ConsistencyModel} focus on learning a single-step mapping from noise to data, enabling fast and high-quality sampling while maintaining consistency with the underlying data distribution.
\textit{Diffusion Bridges (DBs)}~\citep{DSB,SB-FBSDE,DDBM} extend diffusion models for generative tasks that connect different distributions, enabling efficient generation from one distribution to another.

These formulations propose innovative solutions tailored to different tasks, driving the widespread application in multi-modal generative tasks such as image, text, video, and graph~\citep{DiffusionSurvey1,DiffusionGraphSurvey1,DiffusionGraphSurvey2}.
\section{Data Sources}\label{sec:data}
In this section, we introduce our training data, including unlabeled light curves for pretraining and labeled samples for the downstream classification task. 

\subsection{Unlabeled data - MACHO}
The MACHO project \citep{1993Natur.365..621A} aimed to detect Massive Compact Halo Objects (MACHO) to find evidence of dark matter in the Milky Way halo by searching for gravitational microlensing events. Light curves were collected from 1992 to 1999, producing light curves of more than a thousand observations \citep{1999PASP..111.1539A} in bands B and R.
The observed sky was subdivided into 403 fields. Each field was constructed by observing a region of the sky or tile. The resulting data is available in a public repository\footnote{\url{https://macho.nci.org.au/macho_photometry}} which contains millions of light curves in bands B and R. 

We selected a subset of fields 1, 101, 102, 103, and 104 containing \num{1454792} light curves for training. Similarly, we select field 10 for testing, with a total of \num{74594} light curves. MACHO observed in both bands simultaneously, therefore having two magnitudes associated with each MJD. Since we are looking to improve on Astromer 2, we maintain the single band input.  The light curves from this dataset that exhibited Gaussian noise characteristics were removed based on the criteria: $|\text{Kurtosis}| > 10$, $|\text{Skewness}| > 1$, and $\text{Std} > 0.1$. Additionally, we excluded observations with negative uncertainties (indicative of faulty measurements) or uncertainties greater than one (to maintain photometric quality). Outliers were also removed by discarding the 1st and 99th percentiles for each light curve. This additional filtering does not affect the total number of samples but reduces the number of observations when the criteria were applied.

\begin{figure}
    \centering
    \includegraphics[scale=.88]{figures/data/magnitude_datasets.pdf}
    \caption{Magnitude distributions for the MACHO, Alcock, and ATLAS datasets. The plotted magnitudes reflect their original values as reported in the datasets; however, they are normalized during training, eliminating the differences in their mean positions.
    The Alcock catalog exhibits multimodality. In contrast, the ATLAS magnitudes show significant more variation, as they originate from a different survey.}
    \label{fig:macho-alcock-magn}
\end{figure}

\subsection{Labeled data}
To ensure a fair comparison with Astromer 1, we used the same sample selection from the MACHO \citep[hereafter referred to as Alcock; ][]{Alcock2001Variable} and the  Asteroid Terrestrial-impact Last Alert System \citep[hereafter referred to as ATLAS; ][]{heinze2018first} labeled catalogs. The former has a similar magnitude distribution, whereas the latter differs, as shown in Fig. \ref{fig:macho-alcock-magn}.

\subsubsection{Alcock}
For labeled data, we use the catalog of variable stars from \citet{Alcock2001Variable}, which contains labels for a subset of the MACHO light curves originating from 30 fields from the Large Magellanic Cloud. This labeled data will be used to train and evaluate the performance of the different embeddings on the classification task. 

The selected data comprises \num{20894} light curves, which are categorized into six classes: Cepheid variables pulsating in the fundamental (Cep\_0) and first overtone (Cep\_1), Eclipsing Binaries (EC), Long Period Variables (LPV), RR Lyrae ab and c (RRab and RRc, respectively). Table \ref{tab:alcock} summarizes the number of samples per class. We note that the catalog used is an updated version, as described in \cite{astromer}.

\begin{table}
\caption{Alcock catalog distribution.}              
\label{tab:alcock}  
\centering 
% \begin{tabular}{c c c} 
\begin{tabular}{l l r} 
\hline\hline         
Tag & Class Name & \# of sources \\ \hline
 Cep\_0 & Cepheid type I &\num{1182} \\
 Cep\_1 &Cepheid type II & \num{683} \\
 EC &Eclipsing binary & \num{6824} \\
 LPV &Long period variable &  \num{3046} \\
 RRab &RR Lyrae type ab  &  \num{7397} \\
 RRc &RR Lyrae type c &  \num{1762} \\
 Total & & \textbf{\num{20894}} \\
\hline                            
\end{tabular}
\end{table}

Figure \ref{fig:macho-alcock-magn} compares the magnitude distributions between the Alcock and MACHO datasets. The former exhibits a bimodal distribution, which aligns with the fact that it represents a subset of the light curves from MACHO fields, while the latter encompasses light curves from only five fields. 

Similarly, we compare the distribution of time differences between consecutive observations ($\Delta t$). Figure \ref{fig:macho-alcock-mjd} shows similar distributions, with comparable ranges and means of three and four days for MACHO and Alcock, respectively.
\begin{figure}
    % \centering
    \includegraphics[scale=0.7]{figures/data/mjd_datasets.pdf}
    \caption{Distributions of consecutive observation time differences ($\Delta t$) for the Alcock, MACHO, and ATLAS datasets. The boxplots illustrate the variability in observation cadences across the datasets. The Alcock and MACHO datasets show relatively consistent sampling with narrower distributions, while the ATLAS dataset exhibits a broader range of $\Delta t$, reflecting more diverse observation intervals. The y-axis is shown on a logarithmic scale to highlight differences across several orders of magnitude }
    \label{fig:macho-alcock-mjd}
\end{figure}

\subsection{ATLAS}
The Asteroid Terrestrial-impact Last Alert System \citep[ATLAS; ][]{Tonry2018} is a survey developed by the University of Hawaii and funded by NASA. Operating since 2015, ATLAS has a global network telescopes, primarily focused on detecting asteroids and comets that could potentially threaten Earth. Observing in $c$ (blue), $o$ (orange), and $t$ (red) filters.

The variable star dataset used in this work was presented by \citet{heinze2018first} and includes 4.7 million candidate variable objects, included in the labeled and unclassified objects, as well as a dubious class. According to their estimates, this class is predominantly composed of $90\%$ instrumental noise and only $10\%$ genuine variable stars.

We analyze \num{141376} light curves from the ATLAS dataset, as detailed in Table \ref{tab:ATLAS}. These observations, measured in the $o$ passband, have a median cadence of $\sim$15 minutes, which is significantly shorter than the typical cadence in the MACHO dataset. This substantial difference poses a challenge for the model, as it must adapt to such a distinct temporal distribution. 

\begin{table}[h!]
\caption{ATLAS catalog distribution.}              
\label{tab:ATLAS}
\centering 
\begin{tabular}{l l r} 
\hline\hline         
Tag & Class Name & \# of sources \\
\hline
CB & Close Binaries &  \num{80218} \\
DB & Detached Binary &  \num{28767} \\
Mira & Mira &  \num{7370} \\
Pulse &RR Lyrae, $\delta$-Scuti, Cepheids &  \num{25021} \\
Total & & \textbf{\num{141376}}\\
\hline                            
\end{tabular}
\end{table}

As done in \citet{astromer} and to standardize the labels with other datasets, we combine detached eclipsing binaries identified by full or half periods into the close binaries (CB) category and similarly merge detached binaries (DB). However, objects with labels derived from Fourier analysis are excluded, as these classifications do not directly align with astrophysical categories.

\subsection{MACHO vs ATLAS}\label{sec:machovsatlas}
Figures \ref{fig:macho-alcock-magn} and \ref{fig:macho-alcock-mjd} illustrate the distributional differences between the unlabeled MACHO dataset and the labeled subsets discussed earlier. While the magnitudes show a notable shift between MACHO and ATLAS, our training strategy normalizes the light curves to a zero mean. As a result, the relationships between observations take precedence over the raw magnitude values. Consequently, we do not expect a substantial performance drop when transitioning between datasets. However, for $\Delta t$, the smaller values of $\Delta t$ present a significant challenge, as the model must extrapolate and account for fast variations to capture short-time information effectively. We evidence this in our first results from Astromer 2, where the F1 score on the ATLAS dataset was lower compared to MACHO when having fewer labels for classification. 

\section{Molecular Generative Tasks}\label{sec:task}


In this section, we explore various molecular generative tasks that leverage diffusion models, as outlined in \cref{fig:taxonomy}. These tasks are crucial for advancing molecular design and discovery, providing innovative solutions across different domains.

\subsection{De Novo Generation}
De novo generation involves creating novel molecular structures from scratch. This approach is essential for discovering new compounds without relying on existing molecular templates. It includes two main sub-tasks: unconditional and conditional generation.


\paratitle{Unconditional generation.}
Unconditional generation focuses on producing molecules without specific constraints. Starting from a random noise vector, these models generate entirely new molecular structures, exploring the vast chemical space for novel applications.



\paratitle{Conditional Generation}
Conditional generation tailors molecule creation based on specific conditions, such as desired properties, targets, or fragments, allowing for more directed and efficient molecular design. By incorporating these conditions, diffusion models produce molecules that meet predefined criteria. 
Existing works can be further divided into four categories based on the type of conditions applied.

\textit{Property-based molecular generation}, also known as \textit{inverse molecule design}, aims to generate molecules with desired properties such as bioactivity and synthesizability.
More specifically, the inverse molecule design can be further divided into single-property conditioning~\citep{CDGS,EDM,GeoLDM,MDM} and multiple-property conditioning~\citep{DiGress,GraphDiT,EEGSDE}.
Among them, MOOD~\citep{MOOD} and CGD~\citep{CGD} also focus on generating structurally novel molecules outside the training distribution, referred to as OOD molecule generation.

\textit{Target-based molecular generation}, also known as \textit{structure-based drug design (SBDD)}, generates molecules based on the 3D structure of target binding pockets, aiming to enhance interaction with specific targets.
IRDiff~\citep{IRDiff} introduces interaction-based retrieval to generate target-specific molecules based on retrieved high-affinity ligand references.

\textit{Fragment-based molecular generation} specifies the generation of molecules with particular fragments. DiffLinker~\citep{DiffLinker} focuses on linker design, generating linkers that connect fragments into a complete molecule.


\textit{Composition-based molecular generation} restricts the elemental composition of generated molecules, ensuring they meet specific compositional criteria~\citep{UniMat}.


\subsection{Molecular Optimization}
Molecular optimization tasks aim to improve existing molecules for better performance or properties, differing from de novo generation by focusing on modifying known structures rather than creating new ones. Starting with an existing molecule, these tasks refine it to enhance its properties or performance. This task includes scaffold hopping, R-group design, and generalized optimization.


\textit{Scaffold hopping} involves modifying molecular scaffolds to discover new compounds, transforming known scaffolds into new ones that retain biological activity~\citep{DiffHopp}.


\textit{R-group design} focuses on optimizing molecules by fine-tuning the properties of lead compounds through the adjustment of specific R-groups~\citep{DecompOpt}.


\textit{Generalized optimization} is a flexible approach to optimize molecules without being restricted to specific strategies like scaffold hopping or R-group design. 
DiffSBDD~\citep{DiffSBDD} allows for a broader range of structural changes, as long as the optimized molecule maintains a certain level of similarity to the original structure. This flexibility enables the exploration of diverse pathways to improve properties.


\subsection{Conformer Generation}
Conformer generation involves predicting the 3D conformers of a molecule based on its 2D topological structure. This task is crucial for understanding the spatial arrangement of atoms within a molecule, which is essential for predicting molecular interactions, reactivity, and properties.
Generated 3D conformers reflect the molecule's potential energy landscape and geometric constraints.
GeoDiff~\citep{GeoDiff} and Torsional Diffusion~\citep{TorsionalDiffusion} employ diffusion models to generate 3D molecular conformers in Cartesian space and torsion angle space, respectively. DiSCO~\citep{DiSCO} further optimizes the predicted conformers with Diffusion Bridge.



\subsection{Molecular Docking}
Molecular docking tasks involve predicting how molecules interact with biological targets, a key step in drug discovery for assessing binding affinity and specificity. By analyzing a molecule and a target structure, DiffDock~\citep{DiffDock} predicts the binding pose with the diffusion model.
Re-Dock~\citep{Re-Dock} further utilizes the diffusion bridge for flexible and realistic molecular docking, which predicts the binding poses of ligands and pocket sidechains simultaneously.



\subsection{Transition State Generation}
Transition state generation focuses on predicting the 3D structure of transition states in chemical reactions, using the reactants and products as inputs. This task is vital for understanding reaction mechanisms, as the transition state represents the highest energy point along the reaction pathway. Accurate modeling of these states provides insights into reaction kinetics and can aid in the design of catalysts and optimization of reaction conditions~\citep{OA-ReactDiff}.
\section{Discussion and Future Direction}
In this section, we discuss the current state and challenges in diffusion models for molecules and outline several promising directions for future research to advance this area.

\paratitle{Complete data modality.}
Most existing works fall under the category of generating molecules in 3D space,
neglecting 2D topology.
Considering the complementary nature of 2D and 3D structures, generating molecules in a joint 2D and 3D space holds significant potential for producing more realistic molecules.
This approach has proven effective in de novo generation~\citep{MUDiff,JODO}, but its broader potential in other generative tasks remains underexplored.

\paratitle{Sophisticated diffusion models.}
As summarized in \cref{tab:summarizations}, 
the diffusion models employed in existing works exhibit a wide variety of formulations. Regarding the time space, there is a shift from discrete-time methods to more generalized continuous-time SDEs. In terms of the data space, an open challenge lies in handling the discrete molecular components (e.g., atom and bond types) alongside the continuous components (e.g., coordinates). Moreover, advanced formulations and techniques, such as flow matching and efficient sampling, remain underutilized.

\paratitle{Challenging generative tasks.}
Many existing works focus on fundamental tasks,
like unconditional or single-conditional generation, 
with insufficient attention to more practical generative tasks, such as multi-conditional generation, molecular optimization, and docking. 
Furthermore, poor performance on large molecules in GEOM-Drugs compared to small molecules in QM9, highlights room for improvement. Additionally, extending molecular generation to complex~\citep{DynamicBind} while considering inter-molecular interactions, presents a another promising yet challenging direction.

\paratitle{Expressive network architectures.}
Existing methods rely on relatively classical network architectures like EGNNs. 
Recent advances in more expressive equivariant neural networks offer new opportunities. Incorporating more powerful architectures into molecular diffusion models could further enhance their performance and effectiveness.

\paratitle{Relationship between molecular generation and molecular representation.}
With the increasing recognition of diffusion models' ability to learn representations, exploring the relationship between molecular generation and molecular representation based on diffusion models emerges as a promising direction. MoleculeSDE~\citep{MoleculeSDE}, SubgDiff~\citep{SubgDiff}, and UniGEM~\citep{UniGEM} mark pioneering steps, but there remains significant room for further research.

\section{Discussion}\label{sec:discussion}



\subsection{From Interactive Prompting to Interactive Multi-modal Prompting}
The rapid advancements of large pre-trained generative models including large language models and text-to-image generation models, have inspired many HCI researchers to develop interactive tools to support users in crafting appropriate prompts.
% Studies on this topic in last two years' HCI conferences are predominantly focused on helping users refine single-modality textual prompts.
Many previous studies are focused on helping users refine single-modality textual prompts.
However, for many real-world applications concerning data beyond text modality, such as multi-modal AI and embodied intelligence, information from other modalities is essential in constructing sophisticated multi-modal prompts that fully convey users' instruction.
This demand inspires some researchers to develop multimodal prompting interactions to facilitate generation tasks ranging from visual modality image generation~\cite{wang2024promptcharm, promptpaint} to textual modality story generation~\cite{chung2022tale}.
% Some previous studies contributed relevant findings on this topic. 
Specifically, for the image generation task, recent studies have contributed some relevant findings on multi-modal prompting.
For example, PromptCharm~\cite{wang2024promptcharm} discovers the importance of multimodal feedback in refining initial text-based prompting in diffusion models.
However, the multi-modal interactions in PromptCharm are mainly focused on the feedback empowered the inpainting function, instead of supporting initial multimodal sketch-prompt control. 

\begin{figure*}[t]
    \centering
    \includegraphics[width=0.9\textwidth]{src/img/novice_expert.pdf}
    \vspace{-2mm}
    \caption{The comparison between novice and expert participants in painting reveals that experts produce more accurate and fine-grained sketches, resulting in closer alignment with reference images in close-ended tasks. Conversely, in open-ended tasks, expert fine-grained strokes fail to generate precise results due to \tool's lack of control at the thin stroke level.}
    \Description{The comparison between novice and expert participants in painting reveals that experts produce more accurate and fine-grained sketches, resulting in closer alignment with reference images in close-ended tasks. Novice users create rougher sketches with less accuracy in shape. Conversely, in open-ended tasks, expert fine-grained strokes fail to generate precise results due to \tool's lack of control at the thin stroke level, while novice users' broader strokes yield results more aligned with their sketches.}
    \label{fig:novice_expert}
    % \vspace{-3mm}
\end{figure*}


% In particular, in the initial control input, users are unable to explicitly specify multi-modal generation intents.
In another example, PromptPaint~\cite{promptpaint} stresses the importance of paint-medium-like interactions and introduces Prompt stencil functions that allow users to perform fine-grained controls with localized image generation. 
However, insufficient spatial control (\eg, PromptPaint only allows for single-object prompt stencil at a time) and unstable models can still leave some users feeling the uncertainty of AI and a varying degree of ownership of the generated artwork~\cite{promptpaint}.
% As a result, the gap between intuitive multi-modal or paint-medium-like control and the current prompting interface still exists, which requires further research on multi-modal prompting interactions.
From this perspective, our work seeks to further enhance multi-object spatial-semantic prompting control by users' natural sketching.
However, there are still some challenges to be resolved, such as consistent multi-object generation in multiple rounds to increase stability and improved understanding of user sketches.   


% \new{
% From this perspective, our work is a step forward in this direction by allowing multi-object spatial-semantic prompting control by users' natural sketching, which considers the interplay between multiple sketch regions.
% % To further advance the multi-modal prompting experience, there are some aspects we identify to be important.
% % One of the important aspects is enhancing the consistency and stability of multiple rounds of generation to reduce the uncertainty and loss of control on users' part.
% % For this purpose, we need to develop techniques to incorporate consistent generation~\cite{tewel2024training} into multi-modal prompting framework.}
% % Another important aspect is improving generative models' understanding of the implicit user intents \new{implied by the paint-medium-like or sketch-based input (\eg, sketch of two people with their hands slightly overlapping indicates holding hand without needing explicit prompt).
% % This can facilitate more natural control and alleviate users' effort in tuning the textual prompt.
% % In addition, it can increase users' sense of ownership as the generated results can be more aligned with their sketching intents.
% }
% For example, when users draw sketches of two people with their hands slightly overlapping, current region-based models cannot automatically infer users' implicit intention that the two people are holding hands.
% Instead, they still require users to explicitly specify in the prompt such relationship.
% \tool addresses this through sketch-aware prompt recommendation to fill in the necessary semantic information, alleviating users' workload.
% However, some users want the generative AI in the future to be able to directly infer this natural implicit intentions from the sketches without additional prompting since prompt recommendation can still be unstable sometimes.


% \new{
% Besides visual generation, 
% }
% For example, one of the important aspect is referring~\cite{he2024multi}, linking specific text semantics with specific spatial object, which is partly what we do in our sketch-aware prompt recommendation.
% Analogously, in natural communication between humans, text or audio alone often cannot suffice in expressing the speakers' intentions, and speakers often need to refer to an existing spatial object or draw out an illustration of her ideas for better explanation.
% Philosophically, we HCI researchers are mostly concerned about the human-end experience in human-AI communications.
% However, studies on prompting is unique in that we should not just care about the human-end interaction, but also make sure that AI can really get what the human means and produce intention-aligned output.
% Such consideration can drastically impact the design of prompting interactions in human-AI collaboration applications.
% On this note, although studies on multi-modal interactions is a well-established topic in HCI community, it remains a challenging problem what kind of multi-modal information is really effective in helping humans convey their ideas to current and next generation large AI models.




\subsection{Novice Performance vs. Expert Performance}\label{sec:nVe}
In this section we discuss the performance difference between novice and expert regarding experience in painting and prompting.
First, regarding painting skills, some participants with experience (4/12) preferred to draw accurate and fine-grained shapes at the beginning. 
All novice users (5/12) draw rough and less accurate shapes, while some participants with basic painting skills (3/12) also favored sketching rough areas of objects, as exemplified in Figure~\ref{fig:novice_expert}.
The experienced participants using fine-grained strokes (4/12, none of whom were experienced in prompting) achieved higher IoU scores (0.557) in the close-ended task (0.535) when using \tool. 
This is because their sketches were closer in shape and location to the reference, making the single object decomposition result more accurate.
Also, experienced participants are better at arranging spatial location and size of objects than novice participants.
However, some experienced participants (3/12) have mentioned that the fine-grained stroke sometimes makes them frustrated.
As P1's comment for his result in open-ended task: "\emph{It seems it cannot understand thin strokes; even if the shape is accurate, it can only generate content roughly around the area, especially when there is overlapping.}" 
This suggests that while \tool\ provides rough control to produce reasonably fine results from less accurate sketches for novice users, it may disappoint experienced users seeking more precise control through finer strokes. 
As shown in the last column in Figure~\ref{fig:novice_expert}, the dragon hovering in the sky was wrongly turned into a standing large dragon by \tool.

Second, regarding prompting skills, 3 out of 12 participants had one or more years of experience in T2I prompting. These participants used more modifiers than others during both T2I and R2I tasks.
Their performance in the T2I (0.335) and R2I (0.469) tasks showed higher scores than the average T2I (0.314) and R2I (0.418), but there was no performance improvement with \tool\ between their results (0.508) and the overall average score (0.528). 
This indicates that \tool\ can assist novice users in prompting, enabling them to produce satisfactory images similar to those created by users with prompting expertise.



\subsection{Applicability of \tool}
The feedback from user study highlighted several potential applications for our system. 
Three participants (P2, P6, P8) mentioned its possible use in commercial advertising design, emphasizing the importance of controllability for such work. 
They noted that the system's flexibility allows designers to quickly experiment with different settings.
Some participants (N = 3) also mentioned its potential for digital asset creation, particularly for game asset design. 
P7, a game mod developer, found the system highly useful for mod development. 
He explained: "\emph{Mods often require a series of images with a consistent theme and specific spatial requirements. 
For example, in a sacrifice scene, how the objects are arranged is closely tied to the mod's background. It would be difficult for a developer without professional skills, but with this system, it is possible to quickly construct such images}."
A few participants expressed similar thoughts regarding its use in scene construction, such as in film production. 
An interesting suggestion came from participant P4, who proposed its application in crime scene description. 
She pointed out that witnesses are often not skilled artists, and typically describe crime scenes verbally while someone else illustrates their account. 
With this system, witnesses could more easily express what they saw themselves, potentially producing depictions closer to the real events. "\emph{Details like object locations and distances from buildings can be easily conveyed using the system}," she added.

% \subsection{Model Understanding of Users' Implicit Intents}
% In region-sketch-based control of generative models, a significant gap between interaction design and actual implementation is the model's failure in understanding users' naturally expressed intentions.
% For example, when users draw sketches of two people with their hands slightly overlapping, current region-based models cannot automatically infer users' implicit intention that the two people are holding hands.
% Instead, they still require users to explicitly specify in the prompt such relationship.
% \tool addresses this through sketch-aware prompt recommendation to fill in the necessary semantic information, alleviating users' workload.
% However, some users want the generative AI in the future to be able to directly infer this natural implicit intentions from the sketches without additional prompting since prompt recommendation can still be unstable sometimes.
% This problem reflects a more general dilemma, which ubiquitously exists in all forms of conditioned control for generative models such as canny or scribble control.
% This is because all the control models are trained on pairs of explicit control signal and target image, which is lacking further interpretation or customization of the user intentions behind the seemingly straightforward input.
% For another example, the generative models cannot understand what abstraction level the user has in mind for her personal scribbles.
% Such problems leave more challenges to be addressed by future human-AI co-creation research.
% One possible direction is fine-tuning the conditioned models on individual user's conditioned control data to provide more customized interpretation. 

% \subsection{Balance between recommendation and autonomy}
% AIGC tools are a typical example of 
\subsection{Progressive Sketching}
Currently \tool is mainly aimed at novice users who are only capable of creating very rough sketches by themselves.
However, more accomplished painters or even professional artists typically have a coarse-to-fine creative process. 
Such a process is most evident in painting styles like traditional oil painting or digital impasto painting, where artists first quickly lay down large color patches to outline the most primitive proportion and structure of visual elements.
After that, the artists will progressively add layers of finer color strokes to the canvas to gradually refine the painting to an exquisite piece of artwork.
One participant in our user study (P1) , as a professional painter, has mentioned a similar point "\emph{
I think it is useful for laying out the big picture, give some inspirations for the initial drawing stage}."
Therefore, rough sketch also plays a part in the professional artists' creation process, yet it is more challenging to integrate AI into this more complex coarse-to-fine procedure.
Particularly, artists would like to preserve some of their finer strokes in later progression, not just the shape of the initial sketch.
In addition, instead of requiring the tool to generate a finished piece of artwork, some artists may prefer a model that can generate another more accurate sketch based on the initial one, and leave the final coloring and refining to the artists themselves.
To accommodate these diverse progressive sketching requirements, a more advanced sketch-based AI-assisted creation tool should be developed that can seamlessly enable artist intervention at any stage of the sketch and maximally preserve their creative intents to the finest level. 

\subsection{Ethical Issues}
Intellectual property and unethical misuse are two potential ethical concerns of AI-assisted creative tools, particularly those targeting novice users.
In terms of intellectual property, \tool hands over to novice users more control, giving them a higher sense of ownership of the creation.
However, the question still remains: how much contribution from the user's part constitutes full authorship of the artwork?
As \tool still relies on backbone generative models which may be trained on uncopyrighted data largely responsible for turning the sketch into finished artwork, we should design some mechanisms to circumvent this risk.
For example, we can allow artists to upload backbone models trained on their own artworks to integrate with our sketch control.
Regarding unethical misuse, \tool makes fine-grained spatial control more accessible to novice users, who may maliciously generate inappropriate content such as more realistic deepfake with specific postures they want or other explicit content.
To address this issue, we plan to incorporate a more sophisticated filtering mechanism that can detect and screen unethical content with more complex spatial-semantic conditions. 
% In the future, we plan to enable artists to upload their own style model

% \subsection{From interactive prompting to interactive spatial prompting}


\subsection{Limitations and Future work}

    \textbf{User Study Design}. Our open-ended task assesses the usability of \tool's system features in general use cases. To further examine aspects such as creativity and controllability across different methods, the open-ended task could be improved by incorporating baselines to provide more insightful comparative analysis. 
    Besides, in close-ended tasks, while the fixing order of tool usage prevents prior knowledge leakage, it might introduce learning effects. In our study, we include practice sessions for the three systems before the formal task to mitigate these effects. In the future, utilizing parallel tests (\textit{e.g.} different content with the same difficulty) or adding a control group could further reduce the learning effects.

    \textbf{Failure Cases}. There are certain failure cases with \tool that can limit its usability. 
    Firstly, when there are three or more objects with similar semantics, objects may still be missing despite prompt recommendations. 
    Secondly, if an object's stroke is thin, \tool may incorrectly interpret it as a full area, as demonstrated in the expert results of the open-ended task in Figure~\ref{fig:novice_expert}. 
    Finally, sometimes inclusion relationships (\textit{e.g.} inside) between objects cannot be generated correctly, partially due to biases in the base model that lack training samples with such relationship. 

    \textbf{More support for single object adjustment}.
    Participants (N=4) suggested that additional control features should be introduced, beyond just adjusting size and location. They noted that when objects overlap, they cannot freely control which object appears on top or which should be covered, and overlapping areas are currently not allowed.
    They proposed adding features such as layer control and depth control within the single-object mask manipulation. Currently, the system assigns layers based on color order, but future versions should allow users to adjust the layer of each object freely, while considering weighted prompts for overlapping areas.

    \textbf{More customized generation ability}.
    Our current system is built around a single model $ColorfulXL-Lightning$, which limits its ability to fully support the diverse creative needs of users. Feedback from participants has indicated a strong desire for more flexibility in style and personalization, such as integrating fine-tuned models that cater to specific artistic styles or individual preferences. 
    This limitation restricts the ability to adapt to varied creative intents across different users and contexts.
    In future iterations, we plan to address this by embedding a model selection feature, allowing users to choose from a variety of pre-trained or custom fine-tuned models that better align with their stylistic preferences. 
    
    \textbf{Integrate other model functions}.
    Our current system is compatible with many existing tools, such as Promptist~\cite{hao2024optimizing} and Magic Prompt, allowing users to iteratively generate prompts for single objects. However, the integration of these functions is somewhat limited in scope, and users may benefit from a broader range of interactive options, especially for more complex generation tasks. Additionally, for multimodal large models, users can currently explore using affordable or open-source models like Qwen2-VL~\cite{qwen} and InternVL2-Llama3~\cite{llama}, which have demonstrated solid inference performance in our tests. While GPT-4o remains a leading choice, alternative models also offer competitive results.
    Moving forward, we aim to integrate more multimodal large models into the system, giving users the flexibility to choose the models that best fit their needs. 
    


\section{Conclusion}\label{sec:conclusion}
In this paper, we present \tool, an interactive system designed to help novice users create high-quality, fine-grained images that align with their intentions based on rough sketches. 
The system first refines the user's initial prompt into a complete and coherent one that matches the rough sketch, ensuring the generated results are both stable, coherent and high quality.
To further support users in achieving fine-grained alignment between the generated image and their creative intent without requiring professional skills, we introduce a decompose-and-recompose strategy. 
This allows users to select desired, refined object shapes for individual decomposed objects and then recombine them, providing flexible mask manipulation for precise spatial control.
The framework operates through a coarse-to-fine process, enabling iterative and fine-grained control that is not possible with traditional end-to-end generation methods. 
Our user study demonstrates that \tool offers novice users enhanced flexibility in control and fine-grained alignment between their intentions and the generated images.




% \clearpage
%% The file named.bst is a bibliography style file for BibTeX 0.99c
\bibliographystyle{named}
\bibliography{ijcai25}

\end{document}

