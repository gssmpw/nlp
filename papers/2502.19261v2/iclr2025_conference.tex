
\documentclass{article} % For LaTeX2e
\usepackage[table]{xcolor}
\usepackage{iclr2025_conference,times}
%\usepackage{xeCJK}
%\setCJKmainfont{IPAexMincho} % 日本語フォント設定
%\setCJKmainfont[Scale=0.9]{IPAexMincho}

% TODO: 日本語を書くために使うパッケージ。最後に消す。 pdflatex で動かす用
\usepackage[whole]{bxcjkjatype}

% Optional math commands from https://github.com/goodfeli/dlbook_notation.
%%%%% NEW MATH DEFINITIONS %%%%%

\usepackage{amsmath,amsfonts,bm}
\usepackage{derivative}
% Mark sections of captions for referring to divisions of figures
\newcommand{\figleft}{{\em (Left)}}
\newcommand{\figcenter}{{\em (Center)}}
\newcommand{\figright}{{\em (Right)}}
\newcommand{\figtop}{{\em (Top)}}
\newcommand{\figbottom}{{\em (Bottom)}}
\newcommand{\captiona}{{\em (a)}}
\newcommand{\captionb}{{\em (b)}}
\newcommand{\captionc}{{\em (c)}}
\newcommand{\captiond}{{\em (d)}}

% Highlight a newly defined term
\newcommand{\newterm}[1]{{\bf #1}}

% Derivative d 
\newcommand{\deriv}{{\mathrm{d}}}

% Figure reference, lower-case.
\def\figref#1{figure~\ref{#1}}
% Figure reference, capital. For start of sentence
\def\Figref#1{Figure~\ref{#1}}
\def\twofigref#1#2{figures \ref{#1} and \ref{#2}}
\def\quadfigref#1#2#3#4{figures \ref{#1}, \ref{#2}, \ref{#3} and \ref{#4}}
% Section reference, lower-case.
\def\secref#1{section~\ref{#1}}
% Section reference, capital.
\def\Secref#1{Section~\ref{#1}}
% Reference to two sections.
\def\twosecrefs#1#2{sections \ref{#1} and \ref{#2}}
% Reference to three sections.
\def\secrefs#1#2#3{sections \ref{#1}, \ref{#2} and \ref{#3}}
% Reference to an equation, lower-case.
\def\eqref#1{equation~\ref{#1}}
% Reference to an equation, upper case
\def\Eqref#1{Equation~\ref{#1}}
% A raw reference to an equation---avoid using if possible
\def\plaineqref#1{\ref{#1}}
% Reference to a chapter, lower-case.
\def\chapref#1{chapter~\ref{#1}}
% Reference to an equation, upper case.
\def\Chapref#1{Chapter~\ref{#1}}
% Reference to a range of chapters
\def\rangechapref#1#2{chapters\ref{#1}--\ref{#2}}
% Reference to an algorithm, lower-case.
\def\algref#1{algorithm~\ref{#1}}
% Reference to an algorithm, upper case.
\def\Algref#1{Algorithm~\ref{#1}}
\def\twoalgref#1#2{algorithms \ref{#1} and \ref{#2}}
\def\Twoalgref#1#2{Algorithms \ref{#1} and \ref{#2}}
% Reference to a part, lower case
\def\partref#1{part~\ref{#1}}
% Reference to a part, upper case
\def\Partref#1{Part~\ref{#1}}
\def\twopartref#1#2{parts \ref{#1} and \ref{#2}}

\def\ceil#1{\lceil #1 \rceil}
\def\floor#1{\lfloor #1 \rfloor}
\def\1{\bm{1}}
\newcommand{\train}{\mathcal{D}}
\newcommand{\valid}{\mathcal{D_{\mathrm{valid}}}}
\newcommand{\test}{\mathcal{D_{\mathrm{test}}}}

\def\eps{{\epsilon}}


% Random variables
\def\reta{{\textnormal{$\eta$}}}
\def\ra{{\textnormal{a}}}
\def\rb{{\textnormal{b}}}
\def\rc{{\textnormal{c}}}
\def\rd{{\textnormal{d}}}
\def\re{{\textnormal{e}}}
\def\rf{{\textnormal{f}}}
\def\rg{{\textnormal{g}}}
\def\rh{{\textnormal{h}}}
\def\ri{{\textnormal{i}}}
\def\rj{{\textnormal{j}}}
\def\rk{{\textnormal{k}}}
\def\rl{{\textnormal{l}}}
% rm is already a command, just don't name any random variables m
\def\rn{{\textnormal{n}}}
\def\ro{{\textnormal{o}}}
\def\rp{{\textnormal{p}}}
\def\rq{{\textnormal{q}}}
\def\rr{{\textnormal{r}}}
\def\rs{{\textnormal{s}}}
\def\rt{{\textnormal{t}}}
\def\ru{{\textnormal{u}}}
\def\rv{{\textnormal{v}}}
\def\rw{{\textnormal{w}}}
\def\rx{{\textnormal{x}}}
\def\ry{{\textnormal{y}}}
\def\rz{{\textnormal{z}}}

% Random vectors
\def\rvepsilon{{\mathbf{\epsilon}}}
\def\rvphi{{\mathbf{\phi}}}
\def\rvtheta{{\mathbf{\theta}}}
\def\rva{{\mathbf{a}}}
\def\rvb{{\mathbf{b}}}
\def\rvc{{\mathbf{c}}}
\def\rvd{{\mathbf{d}}}
\def\rve{{\mathbf{e}}}
\def\rvf{{\mathbf{f}}}
\def\rvg{{\mathbf{g}}}
\def\rvh{{\mathbf{h}}}
\def\rvu{{\mathbf{i}}}
\def\rvj{{\mathbf{j}}}
\def\rvk{{\mathbf{k}}}
\def\rvl{{\mathbf{l}}}
\def\rvm{{\mathbf{m}}}
\def\rvn{{\mathbf{n}}}
\def\rvo{{\mathbf{o}}}
\def\rvp{{\mathbf{p}}}
\def\rvq{{\mathbf{q}}}
\def\rvr{{\mathbf{r}}}
\def\rvs{{\mathbf{s}}}
\def\rvt{{\mathbf{t}}}
\def\rvu{{\mathbf{u}}}
\def\rvv{{\mathbf{v}}}
\def\rvw{{\mathbf{w}}}
\def\rvx{{\mathbf{x}}}
\def\rvy{{\mathbf{y}}}
\def\rvz{{\mathbf{z}}}

% Elements of random vectors
\def\erva{{\textnormal{a}}}
\def\ervb{{\textnormal{b}}}
\def\ervc{{\textnormal{c}}}
\def\ervd{{\textnormal{d}}}
\def\erve{{\textnormal{e}}}
\def\ervf{{\textnormal{f}}}
\def\ervg{{\textnormal{g}}}
\def\ervh{{\textnormal{h}}}
\def\ervi{{\textnormal{i}}}
\def\ervj{{\textnormal{j}}}
\def\ervk{{\textnormal{k}}}
\def\ervl{{\textnormal{l}}}
\def\ervm{{\textnormal{m}}}
\def\ervn{{\textnormal{n}}}
\def\ervo{{\textnormal{o}}}
\def\ervp{{\textnormal{p}}}
\def\ervq{{\textnormal{q}}}
\def\ervr{{\textnormal{r}}}
\def\ervs{{\textnormal{s}}}
\def\ervt{{\textnormal{t}}}
\def\ervu{{\textnormal{u}}}
\def\ervv{{\textnormal{v}}}
\def\ervw{{\textnormal{w}}}
\def\ervx{{\textnormal{x}}}
\def\ervy{{\textnormal{y}}}
\def\ervz{{\textnormal{z}}}

% Random matrices
\def\rmA{{\mathbf{A}}}
\def\rmB{{\mathbf{B}}}
\def\rmC{{\mathbf{C}}}
\def\rmD{{\mathbf{D}}}
\def\rmE{{\mathbf{E}}}
\def\rmF{{\mathbf{F}}}
\def\rmG{{\mathbf{G}}}
\def\rmH{{\mathbf{H}}}
\def\rmI{{\mathbf{I}}}
\def\rmJ{{\mathbf{J}}}
\def\rmK{{\mathbf{K}}}
\def\rmL{{\mathbf{L}}}
\def\rmM{{\mathbf{M}}}
\def\rmN{{\mathbf{N}}}
\def\rmO{{\mathbf{O}}}
\def\rmP{{\mathbf{P}}}
\def\rmQ{{\mathbf{Q}}}
\def\rmR{{\mathbf{R}}}
\def\rmS{{\mathbf{S}}}
\def\rmT{{\mathbf{T}}}
\def\rmU{{\mathbf{U}}}
\def\rmV{{\mathbf{V}}}
\def\rmW{{\mathbf{W}}}
\def\rmX{{\mathbf{X}}}
\def\rmY{{\mathbf{Y}}}
\def\rmZ{{\mathbf{Z}}}

% Elements of random matrices
\def\ermA{{\textnormal{A}}}
\def\ermB{{\textnormal{B}}}
\def\ermC{{\textnormal{C}}}
\def\ermD{{\textnormal{D}}}
\def\ermE{{\textnormal{E}}}
\def\ermF{{\textnormal{F}}}
\def\ermG{{\textnormal{G}}}
\def\ermH{{\textnormal{H}}}
\def\ermI{{\textnormal{I}}}
\def\ermJ{{\textnormal{J}}}
\def\ermK{{\textnormal{K}}}
\def\ermL{{\textnormal{L}}}
\def\ermM{{\textnormal{M}}}
\def\ermN{{\textnormal{N}}}
\def\ermO{{\textnormal{O}}}
\def\ermP{{\textnormal{P}}}
\def\ermQ{{\textnormal{Q}}}
\def\ermR{{\textnormal{R}}}
\def\ermS{{\textnormal{S}}}
\def\ermT{{\textnormal{T}}}
\def\ermU{{\textnormal{U}}}
\def\ermV{{\textnormal{V}}}
\def\ermW{{\textnormal{W}}}
\def\ermX{{\textnormal{X}}}
\def\ermY{{\textnormal{Y}}}
\def\ermZ{{\textnormal{Z}}}

% Vectors
\def\vzero{{\bm{0}}}
\def\vone{{\bm{1}}}
\def\vmu{{\bm{\mu}}}
\def\vtheta{{\bm{\theta}}}
\def\vphi{{\bm{\phi}}}
\def\va{{\bm{a}}}
\def\vb{{\bm{b}}}
\def\vc{{\bm{c}}}
\def\vd{{\bm{d}}}
\def\ve{{\bm{e}}}
\def\vf{{\bm{f}}}
\def\vg{{\bm{g}}}
\def\vh{{\bm{h}}}
\def\vi{{\bm{i}}}
\def\vj{{\bm{j}}}
\def\vk{{\bm{k}}}
\def\vl{{\bm{l}}}
\def\vm{{\bm{m}}}
\def\vn{{\bm{n}}}
\def\vo{{\bm{o}}}
\def\vp{{\bm{p}}}
\def\vq{{\bm{q}}}
\def\vr{{\bm{r}}}
\def\vs{{\bm{s}}}
\def\vt{{\bm{t}}}
\def\vu{{\bm{u}}}
\def\vv{{\bm{v}}}
\def\vw{{\bm{w}}}
\def\vx{{\bm{x}}}
\def\vy{{\bm{y}}}
\def\vz{{\bm{z}}}

% Elements of vectors
\def\evalpha{{\alpha}}
\def\evbeta{{\beta}}
\def\evepsilon{{\epsilon}}
\def\evlambda{{\lambda}}
\def\evomega{{\omega}}
\def\evmu{{\mu}}
\def\evpsi{{\psi}}
\def\evsigma{{\sigma}}
\def\evtheta{{\theta}}
\def\eva{{a}}
\def\evb{{b}}
\def\evc{{c}}
\def\evd{{d}}
\def\eve{{e}}
\def\evf{{f}}
\def\evg{{g}}
\def\evh{{h}}
\def\evi{{i}}
\def\evj{{j}}
\def\evk{{k}}
\def\evl{{l}}
\def\evm{{m}}
\def\evn{{n}}
\def\evo{{o}}
\def\evp{{p}}
\def\evq{{q}}
\def\evr{{r}}
\def\evs{{s}}
\def\evt{{t}}
\def\evu{{u}}
\def\evv{{v}}
\def\evw{{w}}
\def\evx{{x}}
\def\evy{{y}}
\def\evz{{z}}

% Matrix
\def\mA{{\bm{A}}}
\def\mB{{\bm{B}}}
\def\mC{{\bm{C}}}
\def\mD{{\bm{D}}}
\def\mE{{\bm{E}}}
\def\mF{{\bm{F}}}
\def\mG{{\bm{G}}}
\def\mH{{\bm{H}}}
\def\mI{{\bm{I}}}
\def\mJ{{\bm{J}}}
\def\mK{{\bm{K}}}
\def\mL{{\bm{L}}}
\def\mM{{\bm{M}}}
\def\mN{{\bm{N}}}
\def\mO{{\bm{O}}}
\def\mP{{\bm{P}}}
\def\mQ{{\bm{Q}}}
\def\mR{{\bm{R}}}
\def\mS{{\bm{S}}}
\def\mT{{\bm{T}}}
\def\mU{{\bm{U}}}
\def\mV{{\bm{V}}}
\def\mW{{\bm{W}}}
\def\mX{{\bm{X}}}
\def\mY{{\bm{Y}}}
\def\mZ{{\bm{Z}}}
\def\mBeta{{\bm{\beta}}}
\def\mPhi{{\bm{\Phi}}}
\def\mLambda{{\bm{\Lambda}}}
\def\mSigma{{\bm{\Sigma}}}

% Tensor
\DeclareMathAlphabet{\mathsfit}{\encodingdefault}{\sfdefault}{m}{sl}
\SetMathAlphabet{\mathsfit}{bold}{\encodingdefault}{\sfdefault}{bx}{n}
\newcommand{\tens}[1]{\bm{\mathsfit{#1}}}
\def\tA{{\tens{A}}}
\def\tB{{\tens{B}}}
\def\tC{{\tens{C}}}
\def\tD{{\tens{D}}}
\def\tE{{\tens{E}}}
\def\tF{{\tens{F}}}
\def\tG{{\tens{G}}}
\def\tH{{\tens{H}}}
\def\tI{{\tens{I}}}
\def\tJ{{\tens{J}}}
\def\tK{{\tens{K}}}
\def\tL{{\tens{L}}}
\def\tM{{\tens{M}}}
\def\tN{{\tens{N}}}
\def\tO{{\tens{O}}}
\def\tP{{\tens{P}}}
\def\tQ{{\tens{Q}}}
\def\tR{{\tens{R}}}
\def\tS{{\tens{S}}}
\def\tT{{\tens{T}}}
\def\tU{{\tens{U}}}
\def\tV{{\tens{V}}}
\def\tW{{\tens{W}}}
\def\tX{{\tens{X}}}
\def\tY{{\tens{Y}}}
\def\tZ{{\tens{Z}}}


% Graph
\def\gA{{\mathcal{A}}}
\def\gB{{\mathcal{B}}}
\def\gC{{\mathcal{C}}}
\def\gD{{\mathcal{D}}}
\def\gE{{\mathcal{E}}}
\def\gF{{\mathcal{F}}}
\def\gG{{\mathcal{G}}}
\def\gH{{\mathcal{H}}}
\def\gI{{\mathcal{I}}}
\def\gJ{{\mathcal{J}}}
\def\gK{{\mathcal{K}}}
\def\gL{{\mathcal{L}}}
\def\gM{{\mathcal{M}}}
\def\gN{{\mathcal{N}}}
\def\gO{{\mathcal{O}}}
\def\gP{{\mathcal{P}}}
\def\gQ{{\mathcal{Q}}}
\def\gR{{\mathcal{R}}}
\def\gS{{\mathcal{S}}}
\def\gT{{\mathcal{T}}}
\def\gU{{\mathcal{U}}}
\def\gV{{\mathcal{V}}}
\def\gW{{\mathcal{W}}}
\def\gX{{\mathcal{X}}}
\def\gY{{\mathcal{Y}}}
\def\gZ{{\mathcal{Z}}}

% Sets
\def\sA{{\mathbb{A}}}
\def\sB{{\mathbb{B}}}
\def\sC{{\mathbb{C}}}
\def\sD{{\mathbb{D}}}
% Don't use a set called E, because this would be the same as our symbol
% for expectation.
\def\sF{{\mathbb{F}}}
\def\sG{{\mathbb{G}}}
\def\sH{{\mathbb{H}}}
\def\sI{{\mathbb{I}}}
\def\sJ{{\mathbb{J}}}
\def\sK{{\mathbb{K}}}
\def\sL{{\mathbb{L}}}
\def\sM{{\mathbb{M}}}
\def\sN{{\mathbb{N}}}
\def\sO{{\mathbb{O}}}
\def\sP{{\mathbb{P}}}
\def\sQ{{\mathbb{Q}}}
\def\sR{{\mathbb{R}}}
\def\sS{{\mathbb{S}}}
\def\sT{{\mathbb{T}}}
\def\sU{{\mathbb{U}}}
\def\sV{{\mathbb{V}}}
\def\sW{{\mathbb{W}}}
\def\sX{{\mathbb{X}}}
\def\sY{{\mathbb{Y}}}
\def\sZ{{\mathbb{Z}}}

% Entries of a matrix
\def\emLambda{{\Lambda}}
\def\emA{{A}}
\def\emB{{B}}
\def\emC{{C}}
\def\emD{{D}}
\def\emE{{E}}
\def\emF{{F}}
\def\emG{{G}}
\def\emH{{H}}
\def\emI{{I}}
\def\emJ{{J}}
\def\emK{{K}}
\def\emL{{L}}
\def\emM{{M}}
\def\emN{{N}}
\def\emO{{O}}
\def\emP{{P}}
\def\emQ{{Q}}
\def\emR{{R}}
\def\emS{{S}}
\def\emT{{T}}
\def\emU{{U}}
\def\emV{{V}}
\def\emW{{W}}
\def\emX{{X}}
\def\emY{{Y}}
\def\emZ{{Z}}
\def\emSigma{{\Sigma}}

% entries of a tensor
% Same font as tensor, without \bm wrapper
\newcommand{\etens}[1]{\mathsfit{#1}}
\def\etLambda{{\etens{\Lambda}}}
\def\etA{{\etens{A}}}
\def\etB{{\etens{B}}}
\def\etC{{\etens{C}}}
\def\etD{{\etens{D}}}
\def\etE{{\etens{E}}}
\def\etF{{\etens{F}}}
\def\etG{{\etens{G}}}
\def\etH{{\etens{H}}}
\def\etI{{\etens{I}}}
\def\etJ{{\etens{J}}}
\def\etK{{\etens{K}}}
\def\etL{{\etens{L}}}
\def\etM{{\etens{M}}}
\def\etN{{\etens{N}}}
\def\etO{{\etens{O}}}
\def\etP{{\etens{P}}}
\def\etQ{{\etens{Q}}}
\def\etR{{\etens{R}}}
\def\etS{{\etens{S}}}
\def\etT{{\etens{T}}}
\def\etU{{\etens{U}}}
\def\etV{{\etens{V}}}
\def\etW{{\etens{W}}}
\def\etX{{\etens{X}}}
\def\etY{{\etens{Y}}}
\def\etZ{{\etens{Z}}}

% The true underlying data generating distribution
\newcommand{\pdata}{p_{\rm{data}}}
\newcommand{\ptarget}{p_{\rm{target}}}
\newcommand{\pprior}{p_{\rm{prior}}}
\newcommand{\pbase}{p_{\rm{base}}}
\newcommand{\pref}{p_{\rm{ref}}}

% The empirical distribution defined by the training set
\newcommand{\ptrain}{\hat{p}_{\rm{data}}}
\newcommand{\Ptrain}{\hat{P}_{\rm{data}}}
% The model distribution
\newcommand{\pmodel}{p_{\rm{model}}}
\newcommand{\Pmodel}{P_{\rm{model}}}
\newcommand{\ptildemodel}{\tilde{p}_{\rm{model}}}
% Stochastic autoencoder distributions
\newcommand{\pencode}{p_{\rm{encoder}}}
\newcommand{\pdecode}{p_{\rm{decoder}}}
\newcommand{\precons}{p_{\rm{reconstruct}}}

\newcommand{\laplace}{\mathrm{Laplace}} % Laplace distribution

\newcommand{\E}{\mathbb{E}}
\newcommand{\Ls}{\mathcal{L}}
\newcommand{\R}{\mathbb{R}}
\newcommand{\emp}{\tilde{p}}
\newcommand{\lr}{\alpha}
\newcommand{\reg}{\lambda}
\newcommand{\rect}{\mathrm{rectifier}}
\newcommand{\softmax}{\mathrm{softmax}}
\newcommand{\sigmoid}{\sigma}
\newcommand{\softplus}{\zeta}
\newcommand{\KL}{D_{\mathrm{KL}}}
\newcommand{\Var}{\mathrm{Var}}
\newcommand{\standarderror}{\mathrm{SE}}
\newcommand{\Cov}{\mathrm{Cov}}
% Wolfram Mathworld says $L^2$ is for function spaces and $\ell^2$ is for vectors
% But then they seem to use $L^2$ for vectors throughout the site, and so does
% wikipedia.
\newcommand{\normlzero}{L^0}
\newcommand{\normlone}{L^1}
\newcommand{\normltwo}{L^2}
\newcommand{\normlp}{L^p}
\newcommand{\normmax}{L^\infty}

\newcommand{\parents}{Pa} % See usage in notation.tex. Chosen to match Daphne's book.

\DeclareMathOperator*{\argmax}{arg\,max}
\DeclareMathOperator*{\argmin}{arg\,min}

\DeclareMathOperator{\sign}{sign}
\DeclareMathOperator{\Tr}{Tr}
\let\ab\allowbreak


\usepackage{hyperref}
\usepackage{url}
\usepackage{pdfpages}
\usepackage{graphicx}
\usepackage{float}
\usepackage{booktabs}
\usepackage{multirow}
\usepackage{makecell}
\newcommand{\pkg}[1]{\textsf{#1}}
\usepackage{adjustbox}
\usepackage{threeparttable}
\definecolor{light-gray}{gray}{0.9}
\usepackage{algorithm}
\definecolor{verylightgray}{rgb}{0.93, 0.93, 0.93}

\newcommand{\todo}[1]{\textcolor{orange}{#1}}

\newcommand{\Taishi}[1]{\textcolor{red}{\textbf{Taishi: }#1}}
\newcommand{\takuya}[1]{\textcolor{blue}{\textbf{Takuya: }{#1}}}
\newcommand{\odashi}[1]{\textcolor{purple}{\textbf{Oda: }{#1}}}
\newcommand{\methodname}{Drop-Upcycling}
\newcommand{\NUname}{na\"{i}ve Upcycling} 
\newcommand{\RNUname}{Random Noise Upcycling} 
\newcommand{\diff}[1]{#1}

\usepackage{xspace}
\newcommand{\huggingface}{\raisebox{-1.5pt}{\includegraphics[height=1.05em]{figures/hf-logo.pdf}}\xspace}
\newcommand{\github}{\raisebox{-1.5pt}{\includegraphics[height=1.05em]{figures/github-logo.pdf}}\xspace}
\newcommand{\wandb}{\raisebox{-1.5pt}{\includegraphics[height=1.05em]{figures/wandb-logo.png}}\xspace}
\newcommand{\gitlab}{\raisebox{-1.5pt}{\includegraphics[height=1.05em]{figures/gitlab-logo.png}}\xspace}




\title{
\methodname{}: Training Sparse Mixture of Experts with Partial Re-initialization
}


 
\author{
Taishi Nakamura$^{1,2,3}$, 
Takuya Akiba$^{2}$, 
Kazuki Fujii$^{1}$,
Yusuke Oda$^{3}$, \\
\,\,\textbf{Rio Yokota}$^{1,3}$, 
\textbf{Jun Suzuki}$^{4,5,3}$
\\
$^1$Institute of Science Tokyo,
$^2$Sakana AI,
$^3$NII LLMC,
$^4$Tohoku University,
$^5$RIKEN\\
\texttt{taishi@rio.scrc.iir.isct.ac.jp},
% \texttt{takiba@sakana.ai},\\
% \texttt{kazuki.fujii@rio.scrc.iir.isct.ac.jp },
% \texttt{odashi@nii.ac.jp},\\
% \texttt{rioyokota@rio.scrc.iir.isct.ac.jp},
\texttt{jun.suzuki@tohoku.ac.jp}
}

% The \author macro works with any number of authors. There are two commands
% used to separate the names and addresses of multiple authors: \And and \AND.
%
% Using \And between authors leaves it to \LaTeX{} to determine where to break
% the lines. Using \AND forces a linebreak at that point. So, if \LaTeX{}
% puts 3 of 4 authors names on the first line, and the last on the second
% line, try using \AND instead of \And before the third author name.

\newcommand{\fix}{\marginpar{FIX}}
\newcommand{\new}{\marginpar{NEW}}



\iclrfinalcopy % Uncomment for camera-ready version, but NOT for submission.
\begin{document}

\maketitle
\begin{abstract}
The Mixture of Experts (MoE) architecture reduces the training and inference cost significantly compared to a dense model of equivalent capacity. Upcycling is an approach that initializes and trains an MoE model using a pre-trained dense model. While upcycling leads to initial performance gains, the training progresses slower than when trained from scratch, leading to suboptimal performance in the long term. We propose \emph{\methodname{}} -- a method that effectively addresses this problem. \methodname{} combines two seemingly contradictory approaches: utilizing the knowledge of pre-trained dense models while statistically re-initializing some parts of the weights. This approach strategically promotes expert specialization, significantly enhancing the MoE model's efficiency in knowledge acquisition. 
Extensive large-scale experiments demonstrate that \methodname{} significantly outperforms previous MoE construction methods in the long term, specifically when training on hundreds of billions of tokens or more.
As a result, our MoE model with 5.9B active parameters achieves comparable performance to a 13B dense model in the same model family, while requiring approximately 1/4 of the training FLOPs.
% This research offers new insights for efficient LLM development and understanding of MoE models.
All experimental resources, including source code, training data, model checkpoints and logs, are publicly available to promote reproducibility and future research on MoE.

\begin{center}
\begin{tabular}{rcl} % This defines 3 columns: right-aligned, center-aligned, left-aligned
\huggingface & \textbf{Weights} & \href{https://huggingface.co/collections/llm-jp/drop-upcycling-674dc5be7bbb45e12a476b80}{\path{huggingface.co/collections/llm-jp/}}\\
& & \href{https://huggingface.co/collections/llm-jp/drop-upcycling-674dc5be7bbb45e12a476b80}{\path{drop-upcycling-674dc5be7bbb45e12a476b80}}\\[0.2em]
\gitlab & \textbf{Data} & \href{https://gitlab.llm-jp.nii.ac.jp/datasets/llm-jp-corpus-v3}{\path{gitlab.llm-jp.nii.ac.jp/}}\\
& & \href{https://gitlab.llm-jp.nii.ac.jp/datasets/llm-jp-corpus-v3}{\path{datasets/llm-jp-corpus-v3}}\\[0.2em]
\github & \textbf{Code} & \href{https://github.com/Taishi-N324/Drop-Upcycling}{\path{github.com/Taishi-N324/Drop-Upcycling}}\\[0.2em]
\wandb & \textbf{Logs} & \href{https://wandb.ai/taishi-nakamura/Drop-Upcycling}{\path{wandb.ai/taishi-nakamura/Drop-Upcycling}}
\end{tabular}
\end{center}
\end{abstract}

\section{Introduction}

%
% LLM's training and inference are too expensive
%
Large-scale language models (LLMs) have achieved remarkable results across various natural language processing applications \citep{NEURIPS2020_1457c0d6,wei2022chain,ouyang2022training,openai2024gpt4technicalreport}. This success largely depends on scaling the number of model parameters, the amount of training data, and computational resources \citep{kaplan2020scalinglawsneurallanguage,NEURIPS2022_c1e2faff}, which leads to substantial training and inference costs of LLMs. Building and deploying high-performance models also require enormous resources, posing a significant barrier for many researchers and practitioners.

%
% MoE
%
The \emph{Mixture of Experts} (MoE) architecture has emerged as a promising approach to address the escalating resource demands of LLMs. MoE introduces multiple experts into some parts of the network, but only a subset is activated at any given time, allowing the model to achieve superior performance with reduced training and inference costs \citep{shazeer2017,lepikhin2021gshard,Fedus2021SwitchTS}. In fact, cutting-edge industry models like Gemini 1.5 \citep{geminiteam2024gemini15unlockingmultimodal} and GPT-4 (based on unofficial reports) \citep{openai2024gpt4technicalreport} have adopted MoE, suggesting its effectiveness. 


%
% MoE Challenge
%
We refer to transformer-based LLMs without MoE as \emph{dense models} and those incorporating MoE as \emph{MoE models}.
Upcycling~\citep{komatsuzaki2023sparse} is an approach that initializes and trains an MoE model using a pre-trained dense model, which aims to transfer learned knowledge for better initial performance.
However, \NUname{} copies the feedforward network (FFN) layers during initialization, which makes it difficult to achieve expert specialization.
This disadvantage prevents effective utilization of the MoE models' full capacity, resulting in slower convergence over long training periods.
Thus, there exists a trade-off between the short-term cost savings from knowledge transfer and the long-term convergence efficiency through expert specialization.



In this paper, we propose \emph{\methodname{}} -- a method that effectively addresses this trade-off, as briefly illustrated in Figure \ref{fig:drop_upcycling}. \methodname{} works by selectively re-initializing the parameters of the expert FFNs when expanding a dense model into an MoE model. The method is carefully designed to promote expert specialization while preserving the knowledge of pre-trained dense models. Specifically, common indices are randomly sampled along the intermediate dimension of the FFNs, and the weights are dropped either column-wise or row-wise, depending on the weight matrix types. The dropped parameters are then re-initialized using the statistics of those weights.



%
% Experimental results
%
Extensive large-scale experiments demonstrate that \methodname{} nearly resolves the trade-off between the two aforementioned challenges
and significantly outperforms previous MoE model construction methods such as training from scratch and \NUname{}.
By leveraging pre-trained dense models, \methodname{} can start training from a better initial state than training from scratch, reducing training costs.
On the other hand, \methodname{} avoids the convergence slowdowns observed with \NUname{}.
Specifically, in our extensive long-term training experiments, \methodname{} maintained a learning curve slope similar to that of training from scratch, consistently staying ahead.
This success is attributed to effective expert specialization.
As a result, we constructed an MoE model with 5.9B active parameters that performs on par with a 13B dense model from the same model family, while requiring only approximately 1/4 of the training FLOPs.

\begin{figure}[t]
    \centering
    \includegraphics[width=\textwidth]{images/overview.pdf}
\vskip -8pt 
\caption{\textbf{Overview of the \methodname{} method.} The key difference from the na\"{i}ve Upcycling is Diversity re-initialization, introduced in Section \ref{sec:method}.}
    \label{fig:drop_upcycling}
\end{figure}

%
% Fully open
%
This research is fully open, transparent, and accessible to all.
With over 200,000 GPU hours of experimental results, conducted on NVIDIA H100 GPUs, all training data, source code, configuration files, model checkpoints, and training logs used in this study are publicly available. By providing this comprehensive resource, we aim to promote further advancements in this line of research.


%
%
%
Our technical contributions are summarized as follows:
\begin{itemize}
\item We propose \methodname{}, a novel method for constructing MoE models that effectively balance knowledge transfer and expert specialization by selectively re-initializing parameters of expert FFNs when expanding a dense model into an MoE model.

\item Extensive large-scale experiments demonstrate that \methodname{} consistently outperforms previous MoE construction methods in long-term training scenarios.

\item All aspects of this research are publicly available. %, including training data, source code, configuration files, model checkpoints, and training logs. 
This includes the MoE model with 5.9B active parameters that performs comparably to a 13B dense model in the same model family while requiring only about 1/4 of the training FLOPs.

\end{itemize}

\section{Related Work}



\subsection{Mixture of Experts}
\label{sec:related_works:moe}

%
%
The concept of Mixture of Experts (MoE) was introduced about three decades ago~\citep{classic_moe_1,classic_moe_2}. Since then, the idea of using sparsely-gated MoE as a building block within neural network layers~\citep{moe_layer_iclr14,shazeer2017} has evolved and has been incorporated into transformer-based language models~\citep{lepikhin2021gshard, Fedus2021SwitchTS}. For a detailed overview of MoE, please refer to recent survey papers~\citep{moe_survey}.
Sparsely-gated MoE is currently the most common approach for building large-scale sparsely-activated models.
In this paper, we focus on sparsely-gated MoE (also referred to as sparse MoE or sparsely-activated MoE), and unless otherwise specified, the term MoE refers to it.




There are various designs of MoE layers and ways to integrate them into transformer-based LLMs. For example, in addition to the standard token-centric routing, expert-centric routing has also been proposed~\citep{expert_routing}. To incorporate common knowledge, it has been suggested to introduce shared experts that are always activated~\citep{dai-etal-2024-deepseekmoe}. To simplify the discussion, %unless otherwise specified, 
we assume the most standard top-$k$ token choice routing as the MoE layer and a decoder-only transformer-based LLM that uses MoE layers only in the FFNs as the MoE model. 
%This is because 
These are common design choices for recent MoE-based LLMs, such as Mixtral~\citep{jiang2024mixtralexperts}, Skywork-MoE~\citep{wei2024skyworkmoedeepdivetraining}, Phi-3.5-MoE~\citep{abdin2024phi3technicalreporthighly}, and Grok-1\footnote{\url{https://x.ai/blog/grok-os}}. 
% \citep{xai2024grok}
% \footnote{\url{https://x.ai/blog/grok-os}}. 
%
Specifically, these models use 8 experts (Mixtral and Grok-1) or 16 experts (Skywork and Phi-3.5-MoE), with the top-2 experts being activated per input token. Our experiments also use top-2 routing with 8 experts per layer, as this setup aligns with those practical configurations.
These facts indicate that \methodname{} can be applied to most variations of MoE models.
%
See Section~\ref{sec:methods:preliminaries} for technical details of MoE.





\subsection{MoE Model Initialization}
As with conventional neural networks, MoE models can be initialized randomly and trained from scratch. However, to reduce training costs, leveraging existing pre-trained dense models has become a standard approach. Below, we introduce a few methods for achieving this.


Upcycling~\citep{komatsuzaki2023sparse} leverages the weights of a pre-trained dense model for initializing an MoE model by initializing the experts in the MoE layer as replicas of the FFN layers in the dense model.
The main advantage of Upcycling is that it boosts the model's initial performance.  However, as our experiments show, MoE models initialized with Upcycling tend to have a much slower convergence, leading to suboptimal performance when trained for longer durations.



Branch-Train-MiX (BTX) \citep{sukhbaatar2024branchtrainmix} is a technique where a pre-trained dense model is replicated and fine-tuned on different datasets to produce multiple distinct expert dense models. These experts are then integrated into an MoE model, followed by additional training to optimize the routers. While this method appears to ensure expert specialization by design, \cite{jiang2024mixtralexperts} has highlighted that the diversity achieved in this way differs from that required for MoE layer experts, leading to suboptimal performance as a result. Our experiments also show that BTX suffers from suboptimal convergence similar to those observed in Upcycling.






Concurrent with our work, the Qwen2 technical report ~\citep{yang2024qwen2technicalreport} briefly suggests the use of a methodology possibly related to \methodname{} in training Qwen2-MoE. Due to the report's brevity and ambiguity, it is unclear if their method exactly matches ours. 
Our paper offers a valuable technical contribution even if the methods are similar. 
The potential application of \methodname{} in an advanced, industry-developed model like Qwen2-MoE that underscores the importance of further open investigation into this approach. We acknowledge the Qwen2 authors for sharing insights through their technical report.



\section{Method}
\label{sec:method}
In this section, we explain the \methodname{} method. \methodname{} initializes an MoE model by utilizing a pre-trained dense model and consists of three steps:

\begin{enumerate}
\item \textbf{Expert Replication:} The weights of the dense model are copied to create the MoE model. All layers, except for the FFN layers, are copied directly from the dense model. The FFN layers are replaced with MoE layers, and the original FFN weights are copied to all experts within these MoE layers.

\item \textbf{Diversity Re-initialization:} In each MoE layer, a subset of the expert parameters is randomly selected and re-initialized using the original statistical information. This promotes diversity among the experts while partially retaining the knowledge of the original model, which facilitates expert specialization during subsequent training.

\item \textbf{Continued Training:} After initialization, the MoE model is trained using the standard next-token prediction loss. Optionally, a load-balancing loss, commonly applied in MoE training, can also be incorporated.
\end{enumerate}
In the following, we explain the expert initialization and diversity injection processes.


\subsection{SwiGLU and MoE Layers}
\label{sec:methods:preliminaries}

We provide a brief overview of the MoE architecture. First, we review the feedforward network (FFN) layer in transformers. The SwiGLU activation function~\citep{shazeer2020gluvariantsimprovetransformer}, now standard in state-of-the-art LLMs like LLaMA~\citep{touvron2023llamaopenefficientfoundation} and Mixtral~\citep{jiang2024mixtralexperts}, will be used for explanation here. However, it should be noted that \methodname{} can be applied to transformers with any activation function. The FFN layer with SwiGLU is defined as follows:

\begin{equation}
\text{SwiGLU}(\mathbf{x}) = (\text{Swish}(\mathbf{x}^\mathrm{T} \mathbf{W}_\text{gate}) \odot \mathbf{x}^\mathrm{T} \mathbf{W}_\text{up}) \mathbf{W}_\text{down}.
\label{eq:ffn_swiglu}
\end{equation}
%ここで, $\mathbf{x} \in \mathbb{R}^{d_h}$は入力ベクトル, $\odot$はアダマール積を表し, 重み行列のサイズは以下の通りである:
Here, $ \mathbf{x} \in \mathbb{R}^{d_h}\ $ represents the input vector and \(\odot\) denotes the Hadamard product. Each FFN layer contains the following three weight matrices: $
%\begin{equation}
\mathbf{W}_\text{gate}, \mathbf{W}_\text{up} \in \mathbb{R}^{d_h \times d_f}$, and $\mathbf{W}_\text{down} \in \mathbb{R}^{d_f \times d_h}.
%\label{eq:weight_matrices}
%\end{equation}
$
The dimensions \(d_h\) and \(d_f\) are referred to as the hidden size and intermediate size, respectively.

When MoE is introduced into a transformer, each FFN layer is replaced with an MoE layer, while the rest of the architecture remains unchanged. Let us assume we use \(n\) experts and Top-$k$ gating. 
An MoE layer comprises a router and \(n\) expert FFNs. The router has a weight matrix \(\mathbf{W}_\text{router} \in \mathbb{R}^{d_h \times n}\). The $i$-th expert FFN is denoted as \(\text{SwiGLU}^{(i)}(\mathbf{x})\), which, like a standard FFN layer, consists of three weight matrices. These weights are denoted as \(\mathbf{W}^{(i)}_\text{gate}, \mathbf{W}^{(i)}_\text{up},\) and \(\mathbf{W}^{(i)}_\text{down}\). 
%
The output \(\mathbf{y}\) of the MoE layer is computed as follows:
%$n$エキスパートで, Top-Kゲーティングを使用する時, $\mathbf{W}_g \in \mathbb{R}^{d_h \times n}$をゲーティング重みとすると, MoE層の最終出力$\mathbf{y}$は以下のように計算される:
\begin{equation}
\mathbf{y} = \sum_{i=1}^{n} g(\mathbf{x})_i \cdot \text{SwiGLU}^{(i)}(\mathbf{x}),
\label{eq:moe_output}
\end{equation}
where \(g(\mathbf{x})_i\) is the $i$-th element of the output $g(\mathbf{x}) \in \mathbb{R}^n$ of the Top-$k$ routing function, defined as:
% ここで, $g_i(\mathbf{x})$はTop-Kゲーティング関数の出力で, 以下のように定義される:
\begin{equation}
g(\mathbf{x}) = \text{Softmax}(\text{Top-}k(\mathbf{x}^\mathrm{T} \mathbf{W}_\text{router})).
\label{eq:gating_function}
\end{equation}

Since \(k < n\) is typically the standard setting, only the top-$k$ selected experts out of \(n\) are computed. Therefore, the MoE layer is sparsely activated, meaning that only a subset of the parameters is involved in the computation. The number of parameters engaged in the computation for a given input is referred to as the \emph{active parameters} of the MoE model. This value is widely used as an approximation for the computational cost as it correlates well with the cost of both training and inference.
For non-MoE models, the total number of parameters corresponds to the active parameters as all parameters are involved in every computation.





\subsection{Expert Replication}
%まず、通常のFFN layerを用いたtransfomerのweightをコピーし、MoE layerを用いたtransformerを構築します。上で説明した通り、FFN層以外は全く同じアーキテクチャなので、それらについてはweightをそのままコピーします。各FFN層はMoE層に置き換える必要があります。新しいMoEレイヤーは以下の方法で作ります。

%routerのweightである $\mathbf{W}_\text{router}$ はランダムに初期化します。 $n$個のexpertについては、元のFFNのweightをコピーします。即ち、 \(\mathbf{W}^{(i)}_\text{gate} = \mathbf{W}_\text{gate}, \mathbf{W}^{(i)}_\text{up} = \mathbf{W}_\text{up},\) and \(\mathbf{W}^{(i)}_\text{down} = \mathbf{W}_\text{down}\) とします。

Following \citep{komatsuzaki2023sparse}, we first construct a Transformer with MoE layers by replicating the weights from a pre-trained Transformer with standard FFN layers. As explained earlier, the architecture remains identical except the FFN layers, so we simply copy the weights of all non-FFN components. Each FFN layer needs to be replaced with an MoE layer, and the new MoE layers are constructed as follows:
%
The router weights \(\mathbf{W}_\text{router}\) are initialized randomly. For the \(n\) experts, the weights from the original FFN are copied, such that \(\mathbf{W}^{(i)}_\text{gate} = \mathbf{W}_\text{gate}, \mathbf{W}^{(i)}_\text{up} = \mathbf{W}_\text{up},\) and \(\mathbf{W}^{(i)}_\text{down} = \mathbf{W}_\text{down}\).
% \footnote{There has been a recent approach that uses fine-grained experts by reducing the FFN width of MoE models~\citep{dai-etal-2024-deepseekmoe}. \methodname{} can be applied in this context as well. In this scenario, expert replication is performed by splitting either the columns ($\mathbf{W}_\text{gate}$ and $\mathbf{W}_\text{up}$) or the rows ($\mathbf{W}_\text{down}$) of the original FFN, and subsequent steps can be carried out in the same manner. \diff{See Appendix~\ref{appendix:extensions} for a detailed discussion.}}

% \diff{There has been a recent approach that uses fine-grained experts by reducing the FFN width of MoE models~\citep{dai-etal-2024-deepseekmoe}. See Appendix~\ref{appendix:extensions} for a detailed discussion.}

\diff{\methodname{} can also be applied to fine-grained experts and shared experts~\citep{dai-etal-2024-deepseekmoe}. See Appendix~\ref{appendix:extensions} for details.}


%密モデルのFFN層を$n$個のMoEモデルのエキスパート層に初期化する. 各エキスパート$j \in \{1, \ldots, n\}$の重み$\mathbf{W}^{(j)}$は以下のように初期化される:
%\begin{equation}
%\begin{aligned}
%\mathbf{W}^{(j)}_{up} &= \mathbf{W}_{up:,\mathcal{S}_j}, \quad \mathbf{W}^{(j)}_{gate} = %\mathbf{W}_{gate:,\mathcal{S}_j} \in \mathbb{R}^{d_h \times |\mathcal{S}_j|} \\
%\mathbf{W}^{(j)}_{down} &= \mathbf{W}_{down: ,\mathcal{S}_j,} \in \mathbb{R}^{|\mathcal{S}_j| \times d_h}
%\end{aligned}
%\label{eq:expert_init}
%\end{equation}
%ここで, $\mathcal{S}_j$は$\{1, \ldots, d_f\}$からランダムに選択された要素からなる集合, $|\mathcal{S}_j|$はその集合の要素数, $d_h$はモデル次元, $d_f$は元のFFN層の中間次元を表す. 

%本研究の実験では, $|\mathcal{S}_j| = d_f$ としている.

\subsubsection{Diversity Re-initialization}
\label{sec:reinit}


\begin{figure}[t]
\centering
\includegraphics[width=\textwidth]{images/partial_reinit.pdf}
\vskip -8pt
\caption{\textbf{Initialization of expert weights.} Columns (rows) are selected according to \diff{a set of randomly selected indices of the intermediate layer} $\mathcal{S}$, then all elements of them are re-initialized with the normal distribution. Other columns (rows) are maintained.}
\label{fig:partial-re-initialization}
\end{figure}




Diversity re-initialization is the key step in \methodname{}. 
%Re-initializing each expert randomly encourages the diversification of experts during subsequent training. 
This process is carefully designed to balance between knowledge retention and expert diversification. In particular, it is crucial to drop original weights along the intermediate dimension of the FFN layer based on shared indices across all three weight matrices. Specifically, the following operation is applied to every expert FFN in every MoE layer.



\paragraph{Step 1: Column-wise Sampling.}
We sample indices from the set of integers from 1 to intermediate size \(d_f\), namely,  $\mathcal{I}_{d_f}=\{ 1, 2, \cdots, d_f \}$, to create a set of partial indices \(\mathcal{S}\). A hyperparameter $r$ ($0 \leq r \leq 1$) controls the intensity of re-initialization, determining the proportion \(r\) used for sampling. That is, $\mathcal{S} \subseteq \mathcal{I}_{d_f}$ and $\left| \mathcal{S} \right| = \lfloor r d_f \rfloor$.


\paragraph{Step 2: Statistics Calculation.}  
We calculate the mean and standard deviation of the matrices of the weights corresponding to the selected indices $\mathcal{S}$. Specifically, we compute the mean and variance \((\mu_\text{up}, \sigma_\text{up})\), \((\mu_\text{gate}, \sigma_\text{gate})\), and \((\mu_\text{down}, \sigma_\text{down})\) 
from the values obtained only from the non-zero columns of $\mathbf{I}_{\mathcal{S}}$ in the products 
$\mathbf{I}_{\mathcal{S}}\odot W_{\text{gate}}$,
$\mathbf{I}_{\mathcal{S}} \odot W_{\text{up}}$, and 
$\mathbf{I}_{\mathcal{S}} \odot W_{\text{down}}^\top$, respectively, where $\mathbf{I}_{\mathcal{S}}$ is the indicator matrix whose values are 1 in the $i$-th column for $i\in\mathcal{S}$ and 0 otherwise.

%for sub-matrices \(W_{\text{up\ } :,\mathcal{S}}\), \(W_{\text{gate\ } :,\mathcal{S}}\), and \(W_{\text{down\ } \mathcal{S},:}\).

\paragraph{Step 3: Partial Re-Initialization.}
%最後に、これらを用いて3つの重み行列 $\mathbf{W}_\text{gate}$, $\mathbf{W}_\text{up}$, and $\mathbf{W}_\text{down}$ の部分的な最初期化をし、 $\widetilde{\mathbf{W}}_\text{gate}$, $\widetilde{\mathbf{W}}_\text{up}$, and $\widetilde{\mathbf{W}}_\text{down}$ を得る。
%選択されたindexについてはランダム初期化を、そうでないindexについては元のweightをコピーする。即ち、以下のようにする。
Finally, using the calculated statistics, we perform partial re-initialization of the three weight matrices \(\mathbf{W}_\text{gate}\), \(\mathbf{W}_\text{up}\), and \(\mathbf{W}_\text{down}\), obtaining \(\widetilde{\mathbf{W}}_\text{gate}\), \(\widetilde{\mathbf{W}}_\text{up}\), and \(\widetilde{\mathbf{W}}_\text{down}\). 
For the selected indices, the weights are dropped and re-initialized randomly, while for the unselected indices, the original weights are retained. 

%Let $\bar{\mathcal{S}}$ denotes the difference set of $\mathcal{S}$, that is, $\bar{\mathcal{S}} = \mathcal{I}_{d_f}\backslash\mathcal{S} $.
%
Let ${\mathbf{R}}_{\text{type}}$ be a matrix whose values are sampled from the $\mathcal{N}( \mu_{\text{type}}, ( \sigma_{\text{type}} )^2 )$ distribution, where type is one of the gate, up, or down, i.e., $\text{type} =\{\text{gate},\text{up},\text{down}\}$.
%
We then obtain $\widetilde{\mathbf{W}}_{\text{type}}$ by using the following equation:
\begin{equation}
\widetilde{\mathbf{W}}_{\text{type}} = \mathbf{I}_{\mathcal{S}} \odot \mathbf{R}_{\text{type}} +  (1 - \mathbf{I}_{\mathcal{S}}) \odot \mathbf{W}_{\text{type}}
,
\label{eq:DU_rand_init}
\end{equation}
where we consider that the matrices, $\widetilde{\mathbf{W}}_{\text{type}}$, ${\mathbf{R}}_{\text{type}}$, ${\mathbf{W}}_{\text{type}}$ are all transposed if $\text{type} = \text{down}$. 

%That is, 
%$\widetilde{\mathbf{W}}_{\text{up\ } i,j} \sim \mathcal{N}\big( \mu_{\text{up}}, ( \sigma_{\text{up}} )^2 \big)$ if $j \in \mathcal{S}$, and 
%$\widetilde{\mathbf{W}}_{\text{up\ } i,j} =\mathbf{W}_{\text{up\ } i,j}$ otherwise.
%%
%Similarly, $\widetilde{\mathbf{W}}_{\text{gate\ } i,j} \sim \mathcal{N}\big( \mu_{\text{gate}}, (\sigma_{\text{gate}})^2 \big)$ if $j \in \mathcal{S}$, and $\widetilde{\mathbf{W}}_{\text{gate\ } i,j} =
%\mathbf{W}_{\text{gate\ } i,j}$ otherwise.
%
%Moreover, $\widetilde{\mathbf{W}}_{\text{down\ } i,j} \sim \mathcal{N}\big( \mu_{\text{down}}, ( \sigma_{\text{down}} )^2 \big)$ if  $i \in \mathcal{S}$, and 
%$\widetilde{\mathbf{W}}_{\text{down\ } i,j} =\mathbf{W}_{\text{down\ } i,j}$ otherwise.
%
%
%
%\begin{equation}
%\begin{aligned}
%\widetilde{\mathbf{W}}_{\text{up\ } i,j} &= 
%\begin{cases} 
%\sim \mathcal{N}\left( \mu_{\text{up}}, \left( \sigma_{\text{up}} \right)^2 \right) & \text{if } j \in \mathcal{S} \\
%%\mathbf{R}_{\text{up\ } :,j}  & (j \in \mathcal{S}) \\
%\mathbf{W}_{\text{up\ } i,j} & \text{otherwise} %(j \not\in \mathcal{S})
%\end{cases}
%%\end{aligned}
%%\end{equation}
%%\begin{equation}
%%\begin{aligned}
%\\
%\widetilde{\mathbf{W}}_{\text{gate\ } i,j} &= 
%\begin{cases} 
%\sim \mathcal{N}\left( \mu_{\text{gate}}, \left( \sigma_{\text{gate}} \right)^2 \right) & \text{if } j \in \mathcal{S} \\
%%\mathbf{R}_{\text{gate\ } :,j}  & (j \in \mathcal{S}) \\
%\mathbf{W}_{\text{gate\ } i,j} & \text{otherwise} %(j \not\in \mathcal{S})
%\end{cases}
%%\end{aligned}
%%\end{equation}
%%\begin{equation}
%%\begin{aligned}
%\\
%\widetilde{\mathbf{W}}_{\text{down\ } i,j} &= 
%\begin{cases} 
%\sim \mathcal{N}\left( \mu_{\text{down}}, \left( \sigma_{\text{down}} \right)^2 \right) & \text{if } i \in \mathcal{S} \\
%%\mathbf{R}_{\text{down\ } i,:}  & (i \in \mathcal{S}) \\
%\mathbf{W}_{\text{down\ } i,j} & \text{otherwise}%(i \not\in \mathcal{S})
%\end{cases}
%\end{aligned}
%\end{equation}
%Note that \(\widetilde{\mathbf{W}}_\text{gate}\) and \(\widetilde{\mathbf{W}}_\text{up}\) are updated along the column dimension, whereas \(\widetilde{\mathbf{W}}_\text{down}\) is updated along the row dimension.
%$\widetilde{\mathbf{W}}_\text{gate}$, $\widetilde{\mathbf{W}}_\text{up}$はカラム方向、$\widetilde{\mathbf{W}}_\text{down}$ は行方向に適用することに注意。

Figure~\ref{fig:partial-re-initialization} illustrates how we generate a single expert weight matrix from the original dense weights.



\subsubsection{Theoretical Characteristics}
Applying the re-initialization strategy explained above, the initial MoE model obtained by \methodname{} has the following characteristics:
\begin{enumerate}
\item \textbf{Parameter sharing among experts}:
since each expert retains the original representations with a ratio $(1-r)$, \diff{with Top-k routing where $k$ experts are selected, approximately $(1-r)^k$ of representations are preserved. }
\item \textbf{Characteristics of initial feedforward layers}:
\diff{Consider the output of an MoE layer with parameter re-initialization ratio $r$:}
% \begin{equation}
% \mathbf{y} \approx \text{FFN}_{\text{common}}(\mathbf{x}) + \sum_{i=1}^N g_i(\mathbf{x}) \cdot [\text{FFN}_{\text{retained}_i}(\mathbf{x}) - \text{FFN}_{\text{common}}(\mathbf{x}) + \text{FFN}_{\text{diverse}_i}(\mathbf{x})]
% \end{equation}
\begin{equation}
\diff{\mathbf{y} = \text{FFN}_{\text{common}}(\mathbf{x}) + \sum_{i=1}^N 
g(\mathbf{x})_i \cdot [\text{FFN}_{\text{retained}_i}(\mathbf{x}) - \text{FFN}_{\text{common}}(\mathbf{x}) + \text{FFN}_{\text{diverse}_i}(\mathbf{x})]}
\end{equation}
% where $\text{FFN}_{\text{common}}$ represents the output from parameters that are common to all selected $k$ experts (approximately ratio $(1-r)^k$ due to each expert independently preserving a ratio $(1-r)$ of original parameters), 
% $\text{FFN}_{\text{retained}_i}$ is expert $i$'s output using uniquely retained original parameters (ratio $(1-r)$), and $\text{FFN}_{\text{diverse}_i}$ is the output using reinitialized parameters (ratio $r$). The approximation error comes from parameter overlap, with magnitude $O(\frac{1}{\sqrt{d_f}})$. A detailed derivation is provided in Appendix~\ref{subsec:theoretical}.
% \end{enumerate}
\diff{where $\text{FFN}_{\text{common}}$ represents the output from parameters that are common to all selected $k$ experts (the proportion of such parameters is approximately $(1-r)^k$ due to each expert independently preserving a ratio $(1-r)$ of original parameters), 
$\text{FFN}_{\text{retained}_i}$ is expert $i$'s output using uniquely retained original parameters (ratio $(1-r)$), and $\text{FFN}_{\text{diverse}_i}$ is the output using reinitialized parameters (ratio $r$). The estimation error in the number of common parameters has magnitude $O\big(\frac{1}{\sqrt{\smash[b]{d_f}}}\big)$. A detailed derivation is provided in Appendix~\ref{subsec:theoretical}.}
\end{enumerate}

\section{Experimental Setup}

We conducted experiments to demonstrate the effectiveness of \methodname{} described in Section~\ref{sec:method}.
To clarify our model configurations, we introduce a notation where, for example, ``8×152M'' denotes an MoE model with eight experts and whose base dense model size is 152M.
%The following subsections explain the settings of our experiments.


% \subsection{Configuration of Models to compare}
\label{sec:model-architecture}


We selected the Llama~\citep{touvron2023llamaopenefficientfoundation} and Mixtral~\citep{jiang2024mixtralexperts} architectures for dense and MoE models, respectively, for our experiments. 
%%
%Both architectures are based on the Transformer~\citep{NIPS2017_3f5ee243} with several improvements, including RMSNorm~\citep{zhang-sennrich-neurips19}, SwiGLU~\citep{shazeer2020gluvariantsimprovetransformer}, and rotary position embeddings (RoPE)~\citep{su2024roformer}. 
%The notable difference in Mixtral (MoE) from Llama (dense) is that all feedforward network (FFN) layers are replaced by sparsely gated MoE layers.
%%
%
% We employed 8 experts and the dropless token choice Top-2 routing~\citep{megablocks} for the MoE. 
We employed 8 experts and the dropless~\citep{megablocks} token choice top-2 routing ~\citep{shazeer2017} for the MoE.
%Table~\ref{tab:model-details} shows the remaining hyper-parameters of the Dense and MoE models used in our experiments.
Detailed descriptions of the model configurations are provided in Appendix~\ref{appendix:model_configs_details}



We evaluated \diff{four} different methods to build MoE models, namely, training from scratch, \NUname{}~\citep{komatsuzaki2023sparse}, 
\diff{\RNUname{}~\citep{komatsuzaki2023sparse}} and Branch-Train-MiX~\citep{sukhbaatar2024branchtrainmix} to compare the performance with \methodname{}.
Moreover, we also evaluated dense models to provide a reference of the typical performance of LLMs in our configuration and illustrate the performance gains of MoE models.
We initialized all parameters of dense models using a Gaussian distribution $\mathcal{N}(0, 0.02)$.
The dense models are also used as the seed models of MoE models, except when we train MoE models from scratch.
%
When training MoE models from scratch, we used the same initialization method as the dense models, that is, $\mathcal{N}(0, 0.02)$.
%
%\textbf{MoE NU} is similar to \textbf{MoE DU}, but copies all parameters in \textbf{Dense} and duplicates FFN layers 8 times to initialize the MoE layers.
%
%We also evaluated Branch-Train-Mix proposed in~\citet{sukhbaatar2024branchtrainmix}.
\diff{In \RNUname{}, Drawing from \citep{muennighoff2024olmoeopenmixtureofexpertslanguage}, we initialize by copying the dense model parameters and then add Gaussian noise $\mathcal{N}(0, 0.02)$ to 50\% of the weights in each FFN layer.}
In Branch-Train-Mix, we first obtained three distinct expert dense models by further training a seed dense model with 100B extra tokens of either Japanese, English, or code. 
Then, we used the four dense models (the seed dense model and three expert dense models) to initialize the parameters of an MoE model.
Specifically, we averaged all parameters in the four dense models except the FFN layers and duplicated the FFN layers in each model twice to build eight MoE experts.
Note that this method involved extra training steps with 300B more tokens compared to the other MoE construction methods.
% Unless otherwise stated, dense models were trained on 1T tokens, and MoE models were trained on 500B tokens.







% \subsection{Training and Evaluation}

Unless otherwise stated, dense models were trained on 1T tokens, and MoE models were trained on 500B tokens.
Our training data was obtained from publicly available data.
We describe the detailed statistics of the training datasets in Appendix~\ref{appendix:dataset-details}.
%
% For dense models, we trained 1T tokens for the 1.5B model and 2.072T tokens for the 152M, 3.7B, and 13B models.
% Moreover, we trained 500B tokens for all MoE models.
% We used dense models trained on 1T tokens as the base for upcycling our MoE models. 
% For all MoE models except those trained from scratch, we Upcycled from 1T tokens trained dense checkpoints of all sizes.
%
We followed the typical training configurations used in Llama to train dense models and Mixtral for MoE models.
%
Details of the hyper-parameters we used are described in Appendix~\ref{appendix:training_configs_details}.
Moreover, the implementation and the computational environment used in our experiments are described in Appendix~\ref{appendix:training_environment}.


We conducted a comprehensive evaluation using a wide range of tasks in Japanese and English. 
We used 12 evaluation datasets that can be categorized into seven types.
The details of the evaluation datasets and metrics are described in Appendix \ref{appendix:evaluation_details}.


\section{Results and Discussion}
\label{sec:exp}

\begin{figure}[t]
\centering
\includegraphics[width=\textwidth]{images/three_model_comparison_500B_unified.pdf} 
% \caption{\textbf{Evolution of Training Loss and Average Task Score for Different Model Sizes and Initialization Methods. }
% }
\vspace{-1em}
\caption{
\textbf{Comparison of learning curves for different MoE construction methods}.
The top and bottom rows illustrate the changes in training loss and downstream task scores during training, respectively. In both metrics, the proposed method, \methodname{} with $r=0.5$, achieves the best performance, gaining initial knowledge transfer while avoiding convergence slowdown. 
}


\label{fig:combined_train_loss_and_score}
\end{figure}


In this section, we address the following questions through experiments: 
 Is \methodname{} superior to existing MoE construction methods, and does \methodname{} resolve the issue of slower convergence? (Section~\ref{sec:result-method-comparison})  Does it perform well even in large-scale settings? (Section~\ref{sec:resuilts-scaling})  What is the impact of the re-initialization ratio 
$r$? (Section~\ref{sec:exp:ratio}) How are the experts specialized? (Section~\ref{sec:exp:diversity}) 



\subsection{Method Comparison}
\label{sec:result-method-comparison}
First, we compare \methodname{} with existing methods using small (8$\times$152M) to medium (8$\times$1.5B) scale settings. The left two columns of Figure~\ref{fig:combined_train_loss_and_score} illustrate the learning curves under these settings. The top and bottom rows illustrate the changes in training loss and downstream task scores during training, respectively. Note that in LLM pretraining, training loss serves as a reliable performance indicator since the risk of overfitting is low.
The performance on downstream tasks is represented by the average score across 12 tasks, which is commonly used as the overall evaluation metric. A detailed breakdown will be discussed later in conjunction with Table~\ref{tab:detailed-ja-en-comparison_method}.

Figure~\ref{fig:combined_train_loss_and_score} shows that \methodname{} at \( r = 0.5 \) (green) is significantly more efficient compared to other methods. The top row shows the training loss, while the bottom row displays the evaluation scores using downstream tasks. In both metrics and for both model sizes, \methodname{} becomes the clear winner after some training. Notably, the slope of the learning curve, which indicates convergence rate, is superior. Furthermore, it can be observed that the slope of the learning curve is consistent with the case of training from scratch, suggesting that \methodname{} resolves the crucial challenge of balancing knowledge transfer and expert specialization in Upcycling. For further analysis on expert specialization, see Section~\ref{sec:exp:diversity}.

Among existing methods, \NUname{} exhibited the slowest loss reduction rate and improvement in task scores. Branch-Train-Mix, which starts MoE training after each expert has been trained for 100B steps on different domains such as Japanese, English, and code, initially shows an advantage over \NUname{} due to this favorable initialization. However, its long-term learning pace is on par with \NUname{}, and it is ultimately overtaken by \methodname{}. As an ablation study, we evaluated setting \( r = 1.0 \) in \methodname{}, in addition to the standard \( r = 0.5 \). This configuration involves random initialization of all FFNs while reusing weights for embeddings and self-attention layers. This configuration might seem inefficient at first glance. Nevertheless, our large-scale experiments reveal that even such a seemingly naive baseline can outperform \NUname{} in certain scenarios. For additional analysis on the impact of the \( r \) value, refer to Section~\ref{sec:exp:ratio}.

% Table \ref{tab:detailed-ja-en-comparison_method} provides a comparison of the final downstream task performance for models trained with various methods under these 8$\times$152M and 8$\times$1.5B settings.
Table~\ref{tab:detailed-ja-en-comparison_method} provides a comparison of the final downstream task performance for models trained with various methods under these 8$\times$152M and 8$\times$1.5B settings. \diff{Model numbers refer to the leftmost column of this table.}
This table also includes the dense models used for upcycling. Specifically, Model 1 is the dense model used to initialize Models 3-\diff{7}, and Model \diff{8} is used to initialize Models \diff{10}-\diff{14}.
The proposed method, \methodname{} (DU) with $r=0.5$, consistently demonstrates superior performance across these model scales.
% In particular, both variants of DU (50\%) and DU (100\%) achieved results that outperform other approaches such as From Scratch (FS), \NUname{} (NU), and Branch-Train-MiX (BTX).












\begin{table}[t]
\caption{
\textbf{Comparison of evaluation results between models with different initialization.} 
Training from scratch (FS), Branch-Train-Mix (BTX), \NUname{} (NU), \diff{\RNUname{} (RNU)} and \methodname{} (DU) are compared. 
\diff{$^*$ BTX requires additional 300B tokens to obtain specialized dense models before MoE construction.}
%In addition to the individual scores for each downstream task, we also present the average score across 12 tasks, which is commonly used as the overall evaluation metric for the models.
%For both individual scores and the average, higher values indicate better performance.
Bold letters indicate the highest score within each model size.
%\textbf{Performance comparison of downstream tasks across different model sizes and training methods.} Scores represent task-specific metrics (higher is better). Bold indicates the best score for each model size.
}

\label{tab:detailed-ja-en-comparison_method}
\centering
\small
\renewcommand{\arraystretch}{1.03}
% \tabcolsep=0.15cm
% \begin{adjustbox}{width=\linewidth}
% \begin{tabular}{cll*{14}{r}}
% \toprule
% & \multicolumn{2}{c}{\textbf{Model}} & & \multicolumn{13}{c}{\textbf{Individual Scores}} \\
% \cmidrule{2-3} \cmidrule{5-16}
% \textbf{\#} &
% \makecell[c]{\textbf{Archi-} \\ \textbf{tecture}} & \makecell[c]{\textbf{MoE} \\ \textbf{Init}} & & \makecell[c]{\textbf{JEM}\\\textbf{HQA}} & \makecell[c]{\textbf{NIILC}} & \makecell[c]{\textbf{JSQ}} & \makecell[c]{\textbf{XL}-\\\textbf{Sum}} & \makecell[c]{\textbf{WMT}\\\textbf{E$\to$J}} & \makecell[c]{\textbf{WMT}\\\textbf{J$\to$E}} & \makecell[c]{\textbf{OB}\\\textbf{QA}} & \makecell[c]{\textbf{TQA}} & \makecell[c]{\textbf{HS}} & \makecell[c]{\textbf{SQ}\\\textbf{v2}} & \makecell[c]{\textbf{XW}-\\\textbf{EN}} & \makecell[c]{\textbf{BBH}} & \makecell[c]{\textbf{Avg}} \\

\tabcolsep=0.12cm
% \tabcolsep=0.1cm
\begin{adjustbox}{width=\linewidth}
\begin{tabular}{cllrrrr*{13}{r}}
% \begin{tabular}{cllrrrrrr*{13}{r}}
\toprule
% & \multicolumn{2}{c}{\textbf{Model}} & \multicolumn{2}{c}{\textbf{Training}} & \multicolumn{13}{c}{\textbf{Individual Scores}} \\
% \cmidrule{2-3} \cmidrule{4-5} \cmidrule{6-17}
& \multicolumn{2}{c}{\textbf{Model}} & & \multicolumn{2}{c}{\diff{\textbf{Training}}} & & \multicolumn{13}{c}{\textbf{Individual Scores}} \\
\cmidrule{2-3} \cmidrule{5-6} \cmidrule{8-19}

\textbf{\#} &
\makecell[c]{\textbf{Archi-} \\ \textbf{tecture}} & \makecell[c]{\textbf{MoE} \\ \textbf{Init}} & & \makecell[c]{\diff{\textbf{Tokens}}} & \makecell[c]{\diff{\textbf{FLOPs}} \\ \diff{($\times 10^{21}$)}} & & \makecell[c]{\textbf{JEM}\\\textbf{HQA}} & \makecell[c]{\textbf{NIILC}} & \makecell[c]{\textbf{JSQ}} & \makecell[c]{\textbf{XL}-\\\textbf{Sum}} & \makecell[c]{\textbf{WMT}\\\textbf{E$\to$J}} & \makecell[c]{\textbf{WMT}\\\textbf{J$\to$E}} & \makecell[c]{\textbf{OB}\\\textbf{QA}} & \makecell[c]{\textbf{TQA}} & \makecell[c]{\textbf{HS}} & \makecell[c]{\textbf{SQ}\\\textbf{v2}} & \makecell[c]{\textbf{XW}-\\\textbf{EN}} & \makecell[c]{\textbf{BBH}} & \makecell[c]{\textbf{Avg}} \\



% \midrule
% & & 4shot & 4shot  & 4shot  & 1shot  & 4shot  & 4shot  & 4shot  & 4shot  & 4shot  & 4shot  & 4shot  & 3 shot (CoT)&\\
\midrule
% \multicolumn{4}{l}{\textbf{\textit{152Mベースでの手法比較実験:}}}  \\
\multicolumn{6}{l}{\textbf{\textit{Dense 152M $\to$ MoE 8$\times$152M:}}} \\
1 & Dense & -- & & \diff{1,000B} & \diff{1.59} & & 17.6 & 7.9 & 10.6 & 2.4 & 0.5 & 0.5 & 14.6 & 3.0 & 28.6 & 2.0 & 60.6 & 11.5 & 13.3 \\
2 & MoE  & FS & & \diff{500B} & \diff{0.91} & & 25.2	&13.6	&19.4	&1.8	&0.9	&0.4	&16.6	&2.6	&31.2	&\textbf{12.9}	&64.4	&10.7	&16.6 \\
3 & MoE &BTX & & \diff{800B$^*$} & \diff{1.39} & & 28.6	&17.1	&26.6	&\textbf{4.3}	&2.7	&1.1	&\textbf{18.4}	&5.1	&\textbf{32.5}	&5.3	&\textbf{65.0}	&15.9	&18.5 \\
4 &MoE & NU & & \diff{500B} & \diff{0.91} & & 28.2	&16.2	&24.4	&3.5	&3.0	&1.1	&18.2	&5.8	&31.9	&4.5	&63.5	&14.7	&17.9 \\
\diff{5} & \diff{MoE} & \diff{RNU ($r$=0.5)} & & \diff{500B} & \diff{0.91} & & \diff{28.6} & \diff{17.1} & \diff{29.4} & \diff{3.7}	& \diff{2.3} & \diff{1.6} & \diff{16.8} & \diff{5.3}	& \diff{32.0} & \diff{4.8} & \diff{64.5} & \diff{17.4} & \diff{18.6} \\
\rowcolor{verylightgray} \diff{6} & MoE & DU ($r$=0.5) & & \diff{500B} & \diff{0.91} & & \textbf{32.2}	&\textbf{18.0}	&30.6	&3.7	&\textbf{4.7}	&\textbf{2.3}	&16.8	&\textbf{6.1}	&\textbf{32.5}	&6.2	&64.2	&\textbf{19.1}	&\textbf{19.7} \\
\diff{7} &MoE & DU ($r$=1.0) & & \diff{500B} & \diff{0.91} & & 27.2	&16.8	&\textbf{32.5}	&4.1	&3.7	&1.6	&17.0	&5.9	&32.4	&4.9	&64.8	&15.4	&18.9 \\

 \midrule
%\multicolumn{4}{l}{\textbf{\textit{1.5B base model comparison:}}} \\
\multicolumn{6}{l}{\textbf{\textit{Dense 1.5B $\to$ MoE 8$\times$1.5B:}}} \\
\diff{8} & Dense & -- & & \diff{1,000B} & \diff{11.76} & & 49.6 & 42.5 & 48.1 & 11.3 & 16.8 & 8.5 & 22.2 & 23.8 & 42.9 & 16.2 & 82.5 & 25.1 & 32.5 \\
\diff{9} & MoE & FS & & \diff{500B} & \diff{9.05} & & 48.3 & 45.4 & 59.1 & 7.5 & 16.6 & 6.9 & 26.4 & 31.5 & 47.3 & 15.0 & 83.7 & 25.9 & 34.5 \\
\diff{10} & MoE & BTX & & \diff{800B$^*$} & \diff{12.58} & & 44.3	&51.8	&69.4	&11.9	&22.4	&\textbf{12.5}	&27.8	&39.2	&49.7	&18.7	&\textbf{86.4}	&28.9	&38.6\\
\diff{11} & MoE & NU & & \diff{500B} & \diff{9.05} & & 50.4 & 50.6 & 61.7 & 12.4 & 21.6 & 10.5 & 26.8 & 36.2 & 47.7 & 19.0 & 85.0 & 27.2 & 37.4 \\
\diff{12} & \diff{MoE} & \diff{RNU ($r$=0.5)} & & \diff{500B} & \diff{9.05} & & \diff{\textbf{53.6}} & \diff{50.5} & \diff{71.2} & \diff{12.3} & \diff{22.3}& \diff{11.7} & \diff{26.4} & \diff{40.0} & \diff{49.9} & \diff{19.1} & \diff{84.9} & \diff{27.5} & \diff{39.1} \\
 \rowcolor{verylightgray} \diff{13} & MoE & DU ($r$=0.5) & & \diff{500B} & \diff{9.05} & & 51.1 & \textbf{52.3} & \textbf{72.5} & \textbf{13.7} & \textbf{22.5} & \textbf{12.5} & \textbf{30.6} & \textbf{41.3} & \textbf{50.4} & \textbf{21.2} & 86.2 & \textbf{29.1} & \textbf{40.3} \\
\diff{14} & MoE & DU ($r$=1.0) & & \diff{500B} & \diff{9.05} & & 52.1 & 50.9 & 68.8 & 12.3 & 21.9 & 12.4 & 25.0 & 39.1 & 49.7 & 20.6 & 86.0 & 27.9 & 38.9 \\
%  \midrule
% \multicolumn{4}{l}{\textbf{\textit{3.7Bベースでのスケーリングの実験: }}}  \\
% Dense$^1$ & -- & 44.5 & 47.2 & 78.8 & 12.8 & 21.4 & 15.4 & 25.0 & 33.8 & 47.3 & 23.7 & 85.9 & 28.7 & 38.7 \\
% \multirow{2}{*}{8$\times$3.7B} & FS &53.5	&50.8	&69.6	&10.4	&20.6	&13.9	&29.0	&45.8	&51.1	&21.1	&87.1	&28.1	&40.1 \\
%  & DU (50\%) &
%  \textbf{47.5}&	\textbf{57.0}	&\textbf{82.2}&	\textbf{16.3}	&\textbf{25.0}&	\textbf{19.0}&	\textbf{31.2}&	\textbf{53.6}	&\textbf{54.4}&	\textbf{26.3}&	\textbf{88.5}&	\textbf{32.2}&	\textbf{44.4}\\
% \multicolumn{4}{l}{\textbf{\textit{3.7Bベースでのスケーリングの実験のベースライン: }}}  \\
% 13B Dense& FS &47.6	&58.3	&85.2	&14.1	&24.6	&18.3	&31.4	&48.6	&53.1	&29.3	&88.3	&35.2	&44.5 \\

% 3.7B Dense (2.1T)& FS &42.3	&53.2	&80.4	&14.3	&22.6	&15.9	&28.2	&42.2	&50.6	&25.8	&87.3	&30.9	&41.1 \\

\bottomrule
\end{tabular}
\end{adjustbox}
%\raggedright
%\small
%$^1$ Checkpoint prior to upcycling to MoE. \\
\end{table}

\begin{table}[t]
\caption{\textbf{Comparison between dense and MoE with large-scale configuration.}
\methodname{} (DU) works well even at 8$\times$3.7B scale. The MoE model with \methodname{} outperforms dense models trained with higher computational costs, demonstrating the effectiveness of \methodname{}.
%Model 1 is a dense model created to be upcycled, and Models 2 and 3 are MoE models constructed using different methods. In addition, for comparison, we include Models 4 and 5, which are dense models built with more computational resources using the same methodology. Despite using fewer computational resources, the MoE model created through \methodname{} outperforms these dense models, demonstrating the effectiveness of \methodname{}.
%\caption{\textbf{Performance comparison of downstream tasks across different model sizes and training methods.} Scores represent task-specific metrics (higher is better). Bold indicates the highest score across all model sizes and methods.
}

\tabcolsep 3pt%=0.1cm
\label{tab:detailed-ja-en-comparison_method_scaling}
\centering
\small
\renewcommand{\arraystretch}{1.3}
\begin{adjustbox}{width=\linewidth}
\begin{tabular}{cllrrrrrr*{13}{r}}
\toprule

%Model & Training Method & A para& Para &  flops & tokens& \makecell[c]{JEM\\HopQA} & \makecell[c]{NIILC} & \makecell[c]{JSQuAD} & \makecell[c]{XL-\\Sum} & \makecell[c]{WMT20\\En$\to$Ja} & \makecell[c]{WMT20\\Ja$\to$En} & \makecell[c]{Open\\BookQA} & \makecell[c]{Triv.\\QA} & \makecell[c]{Hella\\Swag} & \makecell[c]{SQuAD\\v2} & \makecell[c]{XWino-\\grad EN} & \makecell[c]{BBH} & AVG \\
& \multicolumn{3}{c}{\textbf{Model}} & & \multicolumn{2}{c}{\textbf{Training}} & & \multicolumn{13}{c}{\textbf{Individual Scores}} \\
\cmidrule{2-4} \cmidrule{6-7} \cmidrule{9-20} 
% \rule{0pt}{4ex}
%\makecell[c]{\textbf{Architecture}} & \makecell[c]{\textbf{MoE} \\ \textbf{Init}} & \makecell[c]{\textbf{Act Params} / \\\textbf{Total Params}} & & \makecell[c]{\textbf{Tokens}} & \makecell[c]{\textbf{FLOPs}} & & \makecell[c]{\textbf{JEM}\\\textbf{HopQA}} & \makecell[c]{\textbf{NIILC}} & \makecell[c]{\textbf{JSQuAD}} & \makecell[c]{\textbf{XL}-\\\textbf{Sum}} & \makecell[c]{\textbf{WMT20}\\\textbf{En$\to$Ja}} & \makecell[c]{\textbf{WMT20}\\\textbf{Ja$\to$En}} & \makecell[c]{\textbf{Open}\\\textbf{BookQA}} & \makecell[c]{\textbf{Triv.}\\\textbf{QA}} & \makecell[c]{\textbf{Hella}\\\textbf{Swag}} & \makecell[c]{\textbf{SQuAD}\\\textbf{v2}} & \makecell[c]{\textbf{XWino}-\\\textbf{grad EN}} & \makecell[c]{\textbf{BBH}} & \textbf{Avg} \\
\makecell[c]{\textbf{\#}} &
\makecell[c]{\textbf{Architecture}} & \makecell[c]{\textbf{MoE} \\ \textbf{Init}} & \makecell[c]{\textbf{Act Params} / \\\textbf{Total Params}} & & \makecell[c]{\textbf{Tokens}} & \makecell[c]{\textbf{FLOPs} \\ ($\times 10^{22}$)} & & \makecell[c]{\textbf{JEM}\\\textbf{HQA}} & \makecell[c]{\textbf{NIILC}} & \makecell[c]{\textbf{JSQ}} & \makecell[c]{\textbf{XL}-\\\textbf{Sum}} & \makecell[c]{\textbf{WMT}\\\textbf{E$\to$J}} & \makecell[c]{\textbf{WMT}\\\textbf{J$\to$E}} & \makecell[c]{\textbf{OB}\\\textbf{QA}} & \makecell[c]{\textbf{TQA}} & \makecell[c]{\textbf{HS}} & \makecell[c]{\textbf{SQ}\\\textbf{v2}} & \makecell[c]{\textbf{XW}-\\\textbf{EN}} & \makecell[c]{\textbf{BBH}} & \textbf{Avg} \\

\midrule
%\multicolumn{6}{l}{\textbf{\textit{3.7Bベースでのスケーリングの実験: }}}  \\
1 & Dense 3.7B & - &  3.7B / 3.7B & & 1,000B& 2.70 & & 44.5 & 47.2 & 78.8 & 12.8 & 21.4 & 15.4 & 25.0 & 33.8 & 47.3 & 23.7 & 85.9 & 28.7 & 38.7 \\
2 & MoE 8$\times$3.7B & FS & 5.9B / 18B & & 500B & 1.98 & & \textbf{53.5}	&50.8	&69.6	&10.4	&20.6	&13.9	&29.0	&45.8	&51.1	&21.1	&87.1	&28.1	&40.1 \\
\rowcolor{verylightgray} 3 & MoE 8$\times$3.7B  & DU ($r$=0.5) &5.9B / 18B & & 500B & 1.98 & &
 47.5&	\textbf{57.0}	&82.2&	\textbf{16.3}	&\textbf{25.0}&	\textbf{19.0}&	31.2&	\textbf{53.6}	&\textbf{54.4}&	26.3&	\textbf{88.5}&	32.2&	44.4\\
%\multicolumn{6}{l}{\textbf{\textit{3.7Bベースでのスケーリングの実験のベースライン: }}}  \\
4 & Dense 13B& - &13B / 13B & & 805B & 7.43 & &47.6	&58.3	&\textbf{85.2}	&14.1	&24.6	&18.3	&\textbf{31.4}	&48.6	&53.1	&\textbf{29.3}	&88.3	&\textbf{35.2}	&\textbf{44.5} \\

5 & Dense 3.7B & -&3.7B / 3.7B & & 2,072B &5.58 & &42.3	&53.2	&80.4	&14.3	&22.6	&15.9	&28.2	&42.2	&50.6	&25.8	&87.3	&30.9	&41.1 \\

\bottomrule
\end{tabular}
\end{adjustbox}
%\raggedright
%\small
%$^1$ Checkpoint prior to upcycling to MoE. \\
\end{table}


% \subsection{Scaling to 8$\times$3.7 B}
\subsection{Scaling to \texorpdfstring{8$\times$3.7}{8x3.7}B}

\label{sec:resuilts-scaling}
To further evaluate the effectiveness of \methodname{} in larger-scale settings and to build a practical MoE model, we conducted experiments with an 8$\times$3.7B configuration. Due to computational resource constraints, experiments under the 8$\times$3.7B setting were limited to training from scratch and \methodname{} with \( r = 0.5 \).

The rightmost column of Figure~\ref{fig:combined_train_loss_and_score} illustrates the learning curves under this configuration. Similar to the 8$\times$152M and 8$\times$1.5B settings, \methodname{} significantly outperforms training from scratch. There is an initial gain in performance due to the improved initialization, and expert diversification allows the training to progress as efficiently as in the case of training from scratch, ensuring that \methodname{} never gets overtaken.


Table~\ref{tab:detailed-ja-en-comparison_method_scaling} compares the models' final downstream task performance. \diff{Model numbers refer to the leftmost column of this table.}
Model 1 is a dense model used as a base model for the Upcycling.
Models 2 and 3 are MoEs built using \NUname{} and \methodname{}, respectively, demonstrating the superiority of \methodname{}.
In addition, two different baseline dense models, Models 4 and 5, are included in the table. Model 4 is a 13B dense model. Our 8$\times$3.7B MoE architecture has fewer active parameters than this 13B model, leading to lower training and inference costs. Nevertheless, the 8$\times$3.7B MoE model using \methodname{} achieves better performance upon completion of training. Model 5 is a 3.7B dense model trained with 2.1T tokens. The fact that our 8$\times$3.7B MoE model with \methodname{} surpasses this dense model indicates that rather than continuously investing resources into training dense models, it might be a superior option to convert them to MoE models through \methodname{} and continue training at a certain point in the process.


\subsection{Analysis 1: Re-initializaiton Ratio}
\label{sec:exp:ratio}

We conducted a study to investigate the impact of the re-initialization ratio $r$ in \methodname{}.
Figure \ref{fig:initialization_ratio_comparison} illustrates the effects of different re-initialization rates 0.0 (\NUname{}), 0.1, 0.25, 0.5, 0.75, and 1.0 on models of sizes 8×152M and 8×1.5B. Each model was trained up to 150B tokens, during which we monitored the training loss and the progression of the average downstream task scores.

The experimental results revealed similar trends across both model sizes. In terms of long-term performance, a re-initialization ratio of 0.5 yielded the best results for both models, maintaining superiority in both training loss and average task scores.
An interesting pattern emerged regarding the influence of the re-initialization ratio. With lower re-initialization rates, particularly at 0.0 (\NUname{}), the models struggled to significantly improve beyond the performance of the original pre-trained models. While re-initialization rates of 0.1 and 0.25 showed promising performance in the early stages of training, they were eventually surpassed by the 0.5 re-initialization rate as training progressed.
These observations suggest that increasing the re-initialization ratio helps the models escape local optima, enabling more effective learning. However, excessively high re-initialization rates of 0.75 or 1.0 appeared to hinder the effective knowledge transfer from the pre-trained dense models.
This phenomenon highlights an important trade-off concerning the MoE initialization: a balance must be struck between knowledge transfer and effective expert specialization.
\methodname{} with \( r = 0.5 \) is a robust and practical method that ideally balances these two aspects.

% Regarding the relationship between task scores and training loss, although some variability was observed in the task scores, the overall trend confirmed that as the training loss decreased (indicating training progression), the task scores improved.







\begin{figure}[t]
\centering
\includegraphics[width=\textwidth]
% {images/initialization_ratio_comparison.pdf}
{images/initialization_ratio_comparison_horizontal_spectral.pdf}
\vskip -5pt 
\caption{\textbf{Impact of re-initialization ratio $r$.} The training loss and downstream task score over the total number of tokens processed during training on 8×152M (left two figures) and 8×1.5B (right two figures) settings are illustrated.
Even with different $r$ values, \methodname{} robustly outperforms \NUname{}, and 0.5 appears to be the most effective ratio.}
%Training Loss and averaged Downstream Task Score over the total numbers of tokens processed during training}

\label{fig:initialization_ratio_comparison}
\end{figure}



\subsection{Analysis 2: Expert Specialization}
\label{sec:exp:diversity}


We analyze expert routing patterns to examine how \methodname{} facilitates expert specialization. We apply the methodologies of \citet{jiang2024mixtralexperts} and \citet{muennighoff2024olmoeopenmixtureofexpertslanguage} to 8×1.5B MoE models trained with different methods. This analysis investigates how data from different domains is routed to various experts. As input data from different domains, we use the validation sets from Japanese and English Wikipedia; the validation set of the Japanese MC4 dataset (as split by the authors; see \citealt{llmjp2024llmjpcrossorganizationalprojectresearch}), originally introduced by \citet{2019t5}; The Stack \citep{kocetkov2023the}; and the English C4 dataset \citep{muennighoff2024olmoeopenmixtureofexpertslanguage}.



\begin{figure}[t] \centering \includegraphics[width=\textwidth]{images/expert_routing_probability_comparison_legend_bottom.pdf} \vskip -3pt 
\caption{
\textbf{Comparison of expert routing patterns across different MoE construction methods.} 
\methodname{} exhibits more balanced expert utilization than \NUname{}.
Results shown for layers 0 (first), 8, 16, and 23 (last); see Appendix~\ref{subsec:detailed_routing_analysis} for results on all layers.
%\textbf{Expert routing patterns} at layers 0 (first), 8, 16, and 23 (last) on different datasets.
} \label{fig:routing} \end{figure}

 


In Figure~\ref{fig:routing}, we observe that \NUname{} \diff{with global load balancing} results in a highly imbalanced routing pattern, \diff{where} the majority of experts were underutilized or not utilized at all, with only two experts being always selected across all layers. \diff{While layer-wise load balancing mitigate such expert collapse, we found no significant differences in the training loss trajectories or model performance between these two strategies (see Appendix~\ref{subsec:detailed_load_balance_comparison}).} In contrast, both the model trained from scratch and the one enhanced with \methodname{} (with $r=0.5$) exhibit \diff{domain-specialized routing patterns regardless of the load balancing strategy.} The routing patterns reveal that certain experts specialize in processing specific types of data, such as Japanese text, English text, or code snippets\diff{, as} evident from the distinct expert selection probabilities corresponding to each dataset.

These findings suggest that \methodname{} promotes effective expert specialization \diff{independently of the load balancing strategy, which likely contributes to the improved performance observed in our experiments. For detailed routing patterns across all 24 layers and further analysis of load balancing strategies, see Appendix~\ref{subsec:detailed_routing_analysis} and \ref{subsec:detailed_load_balance_comparison}.}

\section{Conclusion}

In this paper, we introduced \methodname{}, a novel method for efficiently constructing Mixture of Experts (MoE) models from pre-trained dense models. Selectively re-initializing parameters of expert feedforward networks, \methodname{} 
 effectively balances knowledge transfer and expert specialization, addressing the key challenges in MoE model development.

Our extensive large-scale experiments demonstrated that \methodname{},    
significantly outperforms previous MoE construction methods. As a result, we achieved an MoE model with 5.9B active parameters that matches the performance of a 13B dense model from the same model family while requiring only about 1/4 of the training FLOPs.

By making all aspects of our research publicly available—including data, code, configurations, checkpoints, and logs—we aim to promote transparency and facilitate further advancements in efficient LLM training. We believe that \methodname offers a practical solution to reduce resource barriers in deploying high-performance LLMs, contributing to broader accessibility and innovation in AI research.

\ificlrfinal
\section*{Acknowledgements}

The authors would like to thank Masanori Suganuma and Kou Misaki for providing valuable discussions and feedback during the preparation of this manuscript.
This work was supported by the "R\&D Hub Aimed at Ensuring Transparency and Reliability of Generative AI Models" project of the Ministry of Education, Culture, Sports, Science and Technology. This study was carried out using the TSUBAME4.0 supercomputer at Institute of Science Tokyo.

\section*{Author Contributions}
Taishi Nakamura initiated the project, designed the method, and carried out the experiments. Takuya Akiba co-designed the experiments and formulated the overall research strategy. Kazuki Fujii implemented the training codebase used for the experiments. Yusuke Oda handled the training of the dense models. Jun Suzuki and Rio Yokota provided guidance and oversight throughout the project. All authors contributed to the writing and approved the final manuscript.


\bibliography{iclr2025_conference}
\bibliographystyle{iclr2025_conference}

\appendix

\subsection{Lloyd-Max Algorithm}
\label{subsec:Lloyd-Max}
For a given quantization bitwidth $B$ and an operand $\bm{X}$, the Lloyd-Max algorithm finds $2^B$ quantization levels $\{\hat{x}_i\}_{i=1}^{2^B}$ such that quantizing $\bm{X}$ by rounding each scalar in $\bm{X}$ to the nearest quantization level minimizes the quantization MSE. 

The algorithm starts with an initial guess of quantization levels and then iteratively computes quantization thresholds $\{\tau_i\}_{i=1}^{2^B-1}$ and updates quantization levels $\{\hat{x}_i\}_{i=1}^{2^B}$. Specifically, at iteration $n$, thresholds are set to the midpoints of the previous iteration's levels:
\begin{align*}
    \tau_i^{(n)}=\frac{\hat{x}_i^{(n-1)}+\hat{x}_{i+1}^{(n-1)}}2 \text{ for } i=1\ldots 2^B-1
\end{align*}
Subsequently, the quantization levels are re-computed as conditional means of the data regions defined by the new thresholds:
\begin{align*}
    \hat{x}_i^{(n)}=\mathbb{E}\left[ \bm{X} \big| \bm{X}\in [\tau_{i-1}^{(n)},\tau_i^{(n)}] \right] \text{ for } i=1\ldots 2^B
\end{align*}
where to satisfy boundary conditions we have $\tau_0=-\infty$ and $\tau_{2^B}=\infty$. The algorithm iterates the above steps until convergence.

Figure \ref{fig:lm_quant} compares the quantization levels of a $7$-bit floating point (E3M3) quantizer (left) to a $7$-bit Lloyd-Max quantizer (right) when quantizing a layer of weights from the GPT3-126M model at a per-tensor granularity. As shown, the Lloyd-Max quantizer achieves substantially lower quantization MSE. Further, Table \ref{tab:FP7_vs_LM7} shows the superior perplexity achieved by Lloyd-Max quantizers for bitwidths of $7$, $6$ and $5$. The difference between the quantizers is clear at 5 bits, where per-tensor FP quantization incurs a drastic and unacceptable increase in perplexity, while Lloyd-Max quantization incurs a much smaller increase. Nevertheless, we note that even the optimal Lloyd-Max quantizer incurs a notable ($\sim 1.5$) increase in perplexity due to the coarse granularity of quantization. 

\begin{figure}[h]
  \centering
  \includegraphics[width=0.7\linewidth]{sections/figures/LM7_FP7.pdf}
  \caption{\small Quantization levels and the corresponding quantization MSE of Floating Point (left) vs Lloyd-Max (right) Quantizers for a layer of weights in the GPT3-126M model.}
  \label{fig:lm_quant}
\end{figure}

\begin{table}[h]\scriptsize
\begin{center}
\caption{\label{tab:FP7_vs_LM7} \small Comparing perplexity (lower is better) achieved by floating point quantizers and Lloyd-Max quantizers on a GPT3-126M model for the Wikitext-103 dataset.}
\begin{tabular}{c|cc|c}
\hline
 \multirow{2}{*}{\textbf{Bitwidth}} & \multicolumn{2}{|c|}{\textbf{Floating-Point Quantizer}} & \textbf{Lloyd-Max Quantizer} \\
 & Best Format & Wikitext-103 Perplexity & Wikitext-103 Perplexity \\
\hline
7 & E3M3 & 18.32 & 18.27 \\
6 & E3M2 & 19.07 & 18.51 \\
5 & E4M0 & 43.89 & 19.71 \\
\hline
\end{tabular}
\end{center}
\end{table}

\subsection{Proof of Local Optimality of LO-BCQ}
\label{subsec:lobcq_opt_proof}
For a given block $\bm{b}_j$, the quantization MSE during LO-BCQ can be empirically evaluated as $\frac{1}{L_b}\lVert \bm{b}_j- \bm{\hat{b}}_j\rVert^2_2$ where $\bm{\hat{b}}_j$ is computed from equation (\ref{eq:clustered_quantization_definition}) as $C_{f(\bm{b}_j)}(\bm{b}_j)$. Further, for a given block cluster $\mathcal{B}_i$, we compute the quantization MSE as $\frac{1}{|\mathcal{B}_{i}|}\sum_{\bm{b} \in \mathcal{B}_{i}} \frac{1}{L_b}\lVert \bm{b}- C_i^{(n)}(\bm{b})\rVert^2_2$. Therefore, at the end of iteration $n$, we evaluate the overall quantization MSE $J^{(n)}$ for a given operand $\bm{X}$ composed of $N_c$ block clusters as:
\begin{align*}
    \label{eq:mse_iter_n}
    J^{(n)} = \frac{1}{N_c} \sum_{i=1}^{N_c} \frac{1}{|\mathcal{B}_{i}^{(n)}|}\sum_{\bm{v} \in \mathcal{B}_{i}^{(n)}} \frac{1}{L_b}\lVert \bm{b}- B_i^{(n)}(\bm{b})\rVert^2_2
\end{align*}

At the end of iteration $n$, the codebooks are updated from $\mathcal{C}^{(n-1)}$ to $\mathcal{C}^{(n)}$. However, the mapping of a given vector $\bm{b}_j$ to quantizers $\mathcal{C}^{(n)}$ remains as  $f^{(n)}(\bm{b}_j)$. At the next iteration, during the vector clustering step, $f^{(n+1)}(\bm{b}_j)$ finds new mapping of $\bm{b}_j$ to updated codebooks $\mathcal{C}^{(n)}$ such that the quantization MSE over the candidate codebooks is minimized. Therefore, we obtain the following result for $\bm{b}_j$:
\begin{align*}
\frac{1}{L_b}\lVert \bm{b}_j - C_{f^{(n+1)}(\bm{b}_j)}^{(n)}(\bm{b}_j)\rVert^2_2 \le \frac{1}{L_b}\lVert \bm{b}_j - C_{f^{(n)}(\bm{b}_j)}^{(n)}(\bm{b}_j)\rVert^2_2
\end{align*}

That is, quantizing $\bm{b}_j$ at the end of the block clustering step of iteration $n+1$ results in lower quantization MSE compared to quantizing at the end of iteration $n$. Since this is true for all $\bm{b} \in \bm{X}$, we assert the following:
\begin{equation}
\begin{split}
\label{eq:mse_ineq_1}
    \tilde{J}^{(n+1)} &= \frac{1}{N_c} \sum_{i=1}^{N_c} \frac{1}{|\mathcal{B}_{i}^{(n+1)}|}\sum_{\bm{b} \in \mathcal{B}_{i}^{(n+1)}} \frac{1}{L_b}\lVert \bm{b} - C_i^{(n)}(b)\rVert^2_2 \le J^{(n)}
\end{split}
\end{equation}
where $\tilde{J}^{(n+1)}$ is the the quantization MSE after the vector clustering step at iteration $n+1$.

Next, during the codebook update step (\ref{eq:quantizers_update}) at iteration $n+1$, the per-cluster codebooks $\mathcal{C}^{(n)}$ are updated to $\mathcal{C}^{(n+1)}$ by invoking the Lloyd-Max algorithm \citep{Lloyd}. We know that for any given value distribution, the Lloyd-Max algorithm minimizes the quantization MSE. Therefore, for a given vector cluster $\mathcal{B}_i$ we obtain the following result:

\begin{equation}
    \frac{1}{|\mathcal{B}_{i}^{(n+1)}|}\sum_{\bm{b} \in \mathcal{B}_{i}^{(n+1)}} \frac{1}{L_b}\lVert \bm{b}- C_i^{(n+1)}(\bm{b})\rVert^2_2 \le \frac{1}{|\mathcal{B}_{i}^{(n+1)}|}\sum_{\bm{b} \in \mathcal{B}_{i}^{(n+1)}} \frac{1}{L_b}\lVert \bm{b}- C_i^{(n)}(\bm{b})\rVert^2_2
\end{equation}

The above equation states that quantizing the given block cluster $\mathcal{B}_i$ after updating the associated codebook from $C_i^{(n)}$ to $C_i^{(n+1)}$ results in lower quantization MSE. Since this is true for all the block clusters, we derive the following result: 
\begin{equation}
\begin{split}
\label{eq:mse_ineq_2}
     J^{(n+1)} &= \frac{1}{N_c} \sum_{i=1}^{N_c} \frac{1}{|\mathcal{B}_{i}^{(n+1)}|}\sum_{\bm{b} \in \mathcal{B}_{i}^{(n+1)}} \frac{1}{L_b}\lVert \bm{b}- C_i^{(n+1)}(\bm{b})\rVert^2_2  \le \tilde{J}^{(n+1)}   
\end{split}
\end{equation}

Following (\ref{eq:mse_ineq_1}) and (\ref{eq:mse_ineq_2}), we find that the quantization MSE is non-increasing for each iteration, that is, $J^{(1)} \ge J^{(2)} \ge J^{(3)} \ge \ldots \ge J^{(M)}$ where $M$ is the maximum number of iterations. 
%Therefore, we can say that if the algorithm converges, then it must be that it has converged to a local minimum. 
\hfill $\blacksquare$


\begin{figure}
    \begin{center}
    \includegraphics[width=0.5\textwidth]{sections//figures/mse_vs_iter.pdf}
    \end{center}
    \caption{\small NMSE vs iterations during LO-BCQ compared to other block quantization proposals}
    \label{fig:nmse_vs_iter}
\end{figure}

Figure \ref{fig:nmse_vs_iter} shows the empirical convergence of LO-BCQ across several block lengths and number of codebooks. Also, the MSE achieved by LO-BCQ is compared to baselines such as MXFP and VSQ. As shown, LO-BCQ converges to a lower MSE than the baselines. Further, we achieve better convergence for larger number of codebooks ($N_c$) and for a smaller block length ($L_b$), both of which increase the bitwidth of BCQ (see Eq \ref{eq:bitwidth_bcq}).


\subsection{Additional Accuracy Results}
%Table \ref{tab:lobcq_config} lists the various LOBCQ configurations and their corresponding bitwidths.
\begin{table}
\setlength{\tabcolsep}{4.75pt}
\begin{center}
\caption{\label{tab:lobcq_config} Various LO-BCQ configurations and their bitwidths.}
\begin{tabular}{|c||c|c|c|c||c|c||c|} 
\hline
 & \multicolumn{4}{|c||}{$L_b=8$} & \multicolumn{2}{|c||}{$L_b=4$} & $L_b=2$ \\
 \hline
 \backslashbox{$L_A$\kern-1em}{\kern-1em$N_c$} & 2 & 4 & 8 & 16 & 2 & 4 & 2 \\
 \hline
 64 & 4.25 & 4.375 & 4.5 & 4.625 & 4.375 & 4.625 & 4.625\\
 \hline
 32 & 4.375 & 4.5 & 4.625& 4.75 & 4.5 & 4.75 & 4.75 \\
 \hline
 16 & 4.625 & 4.75& 4.875 & 5 & 4.75 & 5 & 5 \\
 \hline
\end{tabular}
\end{center}
\end{table}

%\subsection{Perplexity achieved by various LO-BCQ configurations on Wikitext-103 dataset}

\begin{table} \centering
\begin{tabular}{|c||c|c|c|c||c|c||c|} 
\hline
 $L_b \rightarrow$& \multicolumn{4}{c||}{8} & \multicolumn{2}{c||}{4} & 2\\
 \hline
 \backslashbox{$L_A$\kern-1em}{\kern-1em$N_c$} & 2 & 4 & 8 & 16 & 2 & 4 & 2  \\
 %$N_c \rightarrow$ & 2 & 4 & 8 & 16 & 2 & 4 & 2 \\
 \hline
 \hline
 \multicolumn{8}{c}{GPT3-1.3B (FP32 PPL = 9.98)} \\ 
 \hline
 \hline
 64 & 10.40 & 10.23 & 10.17 & 10.15 &  10.28 & 10.18 & 10.19 \\
 \hline
 32 & 10.25 & 10.20 & 10.15 & 10.12 &  10.23 & 10.17 & 10.17 \\
 \hline
 16 & 10.22 & 10.16 & 10.10 & 10.09 &  10.21 & 10.14 & 10.16 \\
 \hline
  \hline
 \multicolumn{8}{c}{GPT3-8B (FP32 PPL = 7.38)} \\ 
 \hline
 \hline
 64 & 7.61 & 7.52 & 7.48 &  7.47 &  7.55 &  7.49 & 7.50 \\
 \hline
 32 & 7.52 & 7.50 & 7.46 &  7.45 &  7.52 &  7.48 & 7.48  \\
 \hline
 16 & 7.51 & 7.48 & 7.44 &  7.44 &  7.51 &  7.49 & 7.47  \\
 \hline
\end{tabular}
\caption{\label{tab:ppl_gpt3_abalation} Wikitext-103 perplexity across GPT3-1.3B and 8B models.}
\end{table}

\begin{table} \centering
\begin{tabular}{|c||c|c|c|c||} 
\hline
 $L_b \rightarrow$& \multicolumn{4}{c||}{8}\\
 \hline
 \backslashbox{$L_A$\kern-1em}{\kern-1em$N_c$} & 2 & 4 & 8 & 16 \\
 %$N_c \rightarrow$ & 2 & 4 & 8 & 16 & 2 & 4 & 2 \\
 \hline
 \hline
 \multicolumn{5}{|c|}{Llama2-7B (FP32 PPL = 5.06)} \\ 
 \hline
 \hline
 64 & 5.31 & 5.26 & 5.19 & 5.18  \\
 \hline
 32 & 5.23 & 5.25 & 5.18 & 5.15  \\
 \hline
 16 & 5.23 & 5.19 & 5.16 & 5.14  \\
 \hline
 \multicolumn{5}{|c|}{Nemotron4-15B (FP32 PPL = 5.87)} \\ 
 \hline
 \hline
 64  & 6.3 & 6.20 & 6.13 & 6.08  \\
 \hline
 32  & 6.24 & 6.12 & 6.07 & 6.03  \\
 \hline
 16  & 6.12 & 6.14 & 6.04 & 6.02  \\
 \hline
 \multicolumn{5}{|c|}{Nemotron4-340B (FP32 PPL = 3.48)} \\ 
 \hline
 \hline
 64 & 3.67 & 3.62 & 3.60 & 3.59 \\
 \hline
 32 & 3.63 & 3.61 & 3.59 & 3.56 \\
 \hline
 16 & 3.61 & 3.58 & 3.57 & 3.55 \\
 \hline
\end{tabular}
\caption{\label{tab:ppl_llama7B_nemo15B} Wikitext-103 perplexity compared to FP32 baseline in Llama2-7B and Nemotron4-15B, 340B models}
\end{table}

%\subsection{Perplexity achieved by various LO-BCQ configurations on MMLU dataset}


\begin{table} \centering
\begin{tabular}{|c||c|c|c|c||c|c|c|c|} 
\hline
 $L_b \rightarrow$& \multicolumn{4}{c||}{8} & \multicolumn{4}{c||}{8}\\
 \hline
 \backslashbox{$L_A$\kern-1em}{\kern-1em$N_c$} & 2 & 4 & 8 & 16 & 2 & 4 & 8 & 16  \\
 %$N_c \rightarrow$ & 2 & 4 & 8 & 16 & 2 & 4 & 2 \\
 \hline
 \hline
 \multicolumn{5}{|c|}{Llama2-7B (FP32 Accuracy = 45.8\%)} & \multicolumn{4}{|c|}{Llama2-70B (FP32 Accuracy = 69.12\%)} \\ 
 \hline
 \hline
 64 & 43.9 & 43.4 & 43.9 & 44.9 & 68.07 & 68.27 & 68.17 & 68.75 \\
 \hline
 32 & 44.5 & 43.8 & 44.9 & 44.5 & 68.37 & 68.51 & 68.35 & 68.27  \\
 \hline
 16 & 43.9 & 42.7 & 44.9 & 45 & 68.12 & 68.77 & 68.31 & 68.59  \\
 \hline
 \hline
 \multicolumn{5}{|c|}{GPT3-22B (FP32 Accuracy = 38.75\%)} & \multicolumn{4}{|c|}{Nemotron4-15B (FP32 Accuracy = 64.3\%)} \\ 
 \hline
 \hline
 64 & 36.71 & 38.85 & 38.13 & 38.92 & 63.17 & 62.36 & 63.72 & 64.09 \\
 \hline
 32 & 37.95 & 38.69 & 39.45 & 38.34 & 64.05 & 62.30 & 63.8 & 64.33  \\
 \hline
 16 & 38.88 & 38.80 & 38.31 & 38.92 & 63.22 & 63.51 & 63.93 & 64.43  \\
 \hline
\end{tabular}
\caption{\label{tab:mmlu_abalation} Accuracy on MMLU dataset across GPT3-22B, Llama2-7B, 70B and Nemotron4-15B models.}
\end{table}


%\subsection{Perplexity achieved by various LO-BCQ configurations on LM evaluation harness}

\begin{table} \centering
\begin{tabular}{|c||c|c|c|c||c|c|c|c|} 
\hline
 $L_b \rightarrow$& \multicolumn{4}{c||}{8} & \multicolumn{4}{c||}{8}\\
 \hline
 \backslashbox{$L_A$\kern-1em}{\kern-1em$N_c$} & 2 & 4 & 8 & 16 & 2 & 4 & 8 & 16  \\
 %$N_c \rightarrow$ & 2 & 4 & 8 & 16 & 2 & 4 & 2 \\
 \hline
 \hline
 \multicolumn{5}{|c|}{Race (FP32 Accuracy = 37.51\%)} & \multicolumn{4}{|c|}{Boolq (FP32 Accuracy = 64.62\%)} \\ 
 \hline
 \hline
 64 & 36.94 & 37.13 & 36.27 & 37.13 & 63.73 & 62.26 & 63.49 & 63.36 \\
 \hline
 32 & 37.03 & 36.36 & 36.08 & 37.03 & 62.54 & 63.51 & 63.49 & 63.55  \\
 \hline
 16 & 37.03 & 37.03 & 36.46 & 37.03 & 61.1 & 63.79 & 63.58 & 63.33  \\
 \hline
 \hline
 \multicolumn{5}{|c|}{Winogrande (FP32 Accuracy = 58.01\%)} & \multicolumn{4}{|c|}{Piqa (FP32 Accuracy = 74.21\%)} \\ 
 \hline
 \hline
 64 & 58.17 & 57.22 & 57.85 & 58.33 & 73.01 & 73.07 & 73.07 & 72.80 \\
 \hline
 32 & 59.12 & 58.09 & 57.85 & 58.41 & 73.01 & 73.94 & 72.74 & 73.18  \\
 \hline
 16 & 57.93 & 58.88 & 57.93 & 58.56 & 73.94 & 72.80 & 73.01 & 73.94  \\
 \hline
\end{tabular}
\caption{\label{tab:mmlu_abalation} Accuracy on LM evaluation harness tasks on GPT3-1.3B model.}
\end{table}

\begin{table} \centering
\begin{tabular}{|c||c|c|c|c||c|c|c|c|} 
\hline
 $L_b \rightarrow$& \multicolumn{4}{c||}{8} & \multicolumn{4}{c||}{8}\\
 \hline
 \backslashbox{$L_A$\kern-1em}{\kern-1em$N_c$} & 2 & 4 & 8 & 16 & 2 & 4 & 8 & 16  \\
 %$N_c \rightarrow$ & 2 & 4 & 8 & 16 & 2 & 4 & 2 \\
 \hline
 \hline
 \multicolumn{5}{|c|}{Race (FP32 Accuracy = 41.34\%)} & \multicolumn{4}{|c|}{Boolq (FP32 Accuracy = 68.32\%)} \\ 
 \hline
 \hline
 64 & 40.48 & 40.10 & 39.43 & 39.90 & 69.20 & 68.41 & 69.45 & 68.56 \\
 \hline
 32 & 39.52 & 39.52 & 40.77 & 39.62 & 68.32 & 67.43 & 68.17 & 69.30  \\
 \hline
 16 & 39.81 & 39.71 & 39.90 & 40.38 & 68.10 & 66.33 & 69.51 & 69.42  \\
 \hline
 \hline
 \multicolumn{5}{|c|}{Winogrande (FP32 Accuracy = 67.88\%)} & \multicolumn{4}{|c|}{Piqa (FP32 Accuracy = 78.78\%)} \\ 
 \hline
 \hline
 64 & 66.85 & 66.61 & 67.72 & 67.88 & 77.31 & 77.42 & 77.75 & 77.64 \\
 \hline
 32 & 67.25 & 67.72 & 67.72 & 67.00 & 77.31 & 77.04 & 77.80 & 77.37  \\
 \hline
 16 & 68.11 & 68.90 & 67.88 & 67.48 & 77.37 & 78.13 & 78.13 & 77.69  \\
 \hline
\end{tabular}
\caption{\label{tab:mmlu_abalation} Accuracy on LM evaluation harness tasks on GPT3-8B model.}
\end{table}

\begin{table} \centering
\begin{tabular}{|c||c|c|c|c||c|c|c|c|} 
\hline
 $L_b \rightarrow$& \multicolumn{4}{c||}{8} & \multicolumn{4}{c||}{8}\\
 \hline
 \backslashbox{$L_A$\kern-1em}{\kern-1em$N_c$} & 2 & 4 & 8 & 16 & 2 & 4 & 8 & 16  \\
 %$N_c \rightarrow$ & 2 & 4 & 8 & 16 & 2 & 4 & 2 \\
 \hline
 \hline
 \multicolumn{5}{|c|}{Race (FP32 Accuracy = 40.67\%)} & \multicolumn{4}{|c|}{Boolq (FP32 Accuracy = 76.54\%)} \\ 
 \hline
 \hline
 64 & 40.48 & 40.10 & 39.43 & 39.90 & 75.41 & 75.11 & 77.09 & 75.66 \\
 \hline
 32 & 39.52 & 39.52 & 40.77 & 39.62 & 76.02 & 76.02 & 75.96 & 75.35  \\
 \hline
 16 & 39.81 & 39.71 & 39.90 & 40.38 & 75.05 & 73.82 & 75.72 & 76.09  \\
 \hline
 \hline
 \multicolumn{5}{|c|}{Winogrande (FP32 Accuracy = 70.64\%)} & \multicolumn{4}{|c|}{Piqa (FP32 Accuracy = 79.16\%)} \\ 
 \hline
 \hline
 64 & 69.14 & 70.17 & 70.17 & 70.56 & 78.24 & 79.00 & 78.62 & 78.73 \\
 \hline
 32 & 70.96 & 69.69 & 71.27 & 69.30 & 78.56 & 79.49 & 79.16 & 78.89  \\
 \hline
 16 & 71.03 & 69.53 & 69.69 & 70.40 & 78.13 & 79.16 & 79.00 & 79.00  \\
 \hline
\end{tabular}
\caption{\label{tab:mmlu_abalation} Accuracy on LM evaluation harness tasks on GPT3-22B model.}
\end{table}

\begin{table} \centering
\begin{tabular}{|c||c|c|c|c||c|c|c|c|} 
\hline
 $L_b \rightarrow$& \multicolumn{4}{c||}{8} & \multicolumn{4}{c||}{8}\\
 \hline
 \backslashbox{$L_A$\kern-1em}{\kern-1em$N_c$} & 2 & 4 & 8 & 16 & 2 & 4 & 8 & 16  \\
 %$N_c \rightarrow$ & 2 & 4 & 8 & 16 & 2 & 4 & 2 \\
 \hline
 \hline
 \multicolumn{5}{|c|}{Race (FP32 Accuracy = 44.4\%)} & \multicolumn{4}{|c|}{Boolq (FP32 Accuracy = 79.29\%)} \\ 
 \hline
 \hline
 64 & 42.49 & 42.51 & 42.58 & 43.45 & 77.58 & 77.37 & 77.43 & 78.1 \\
 \hline
 32 & 43.35 & 42.49 & 43.64 & 43.73 & 77.86 & 75.32 & 77.28 & 77.86  \\
 \hline
 16 & 44.21 & 44.21 & 43.64 & 42.97 & 78.65 & 77 & 76.94 & 77.98  \\
 \hline
 \hline
 \multicolumn{5}{|c|}{Winogrande (FP32 Accuracy = 69.38\%)} & \multicolumn{4}{|c|}{Piqa (FP32 Accuracy = 78.07\%)} \\ 
 \hline
 \hline
 64 & 68.9 & 68.43 & 69.77 & 68.19 & 77.09 & 76.82 & 77.09 & 77.86 \\
 \hline
 32 & 69.38 & 68.51 & 68.82 & 68.90 & 78.07 & 76.71 & 78.07 & 77.86  \\
 \hline
 16 & 69.53 & 67.09 & 69.38 & 68.90 & 77.37 & 77.8 & 77.91 & 77.69  \\
 \hline
\end{tabular}
\caption{\label{tab:mmlu_abalation} Accuracy on LM evaluation harness tasks on Llama2-7B model.}
\end{table}

\begin{table} \centering
\begin{tabular}{|c||c|c|c|c||c|c|c|c|} 
\hline
 $L_b \rightarrow$& \multicolumn{4}{c||}{8} & \multicolumn{4}{c||}{8}\\
 \hline
 \backslashbox{$L_A$\kern-1em}{\kern-1em$N_c$} & 2 & 4 & 8 & 16 & 2 & 4 & 8 & 16  \\
 %$N_c \rightarrow$ & 2 & 4 & 8 & 16 & 2 & 4 & 2 \\
 \hline
 \hline
 \multicolumn{5}{|c|}{Race (FP32 Accuracy = 48.8\%)} & \multicolumn{4}{|c|}{Boolq (FP32 Accuracy = 85.23\%)} \\ 
 \hline
 \hline
 64 & 49.00 & 49.00 & 49.28 & 48.71 & 82.82 & 84.28 & 84.03 & 84.25 \\
 \hline
 32 & 49.57 & 48.52 & 48.33 & 49.28 & 83.85 & 84.46 & 84.31 & 84.93  \\
 \hline
 16 & 49.85 & 49.09 & 49.28 & 48.99 & 85.11 & 84.46 & 84.61 & 83.94  \\
 \hline
 \hline
 \multicolumn{5}{|c|}{Winogrande (FP32 Accuracy = 79.95\%)} & \multicolumn{4}{|c|}{Piqa (FP32 Accuracy = 81.56\%)} \\ 
 \hline
 \hline
 64 & 78.77 & 78.45 & 78.37 & 79.16 & 81.45 & 80.69 & 81.45 & 81.5 \\
 \hline
 32 & 78.45 & 79.01 & 78.69 & 80.66 & 81.56 & 80.58 & 81.18 & 81.34  \\
 \hline
 16 & 79.95 & 79.56 & 79.79 & 79.72 & 81.28 & 81.66 & 81.28 & 80.96  \\
 \hline
\end{tabular}
\caption{\label{tab:mmlu_abalation} Accuracy on LM evaluation harness tasks on Llama2-70B model.}
\end{table}

%\section{MSE Studies}
%\textcolor{red}{TODO}


\subsection{Number Formats and Quantization Method}
\label{subsec:numFormats_quantMethod}
\subsubsection{Integer Format}
An $n$-bit signed integer (INT) is typically represented with a 2s-complement format \citep{yao2022zeroquant,xiao2023smoothquant,dai2021vsq}, where the most significant bit denotes the sign.

\subsubsection{Floating Point Format}
An $n$-bit signed floating point (FP) number $x$ comprises of a 1-bit sign ($x_{\mathrm{sign}}$), $B_m$-bit mantissa ($x_{\mathrm{mant}}$) and $B_e$-bit exponent ($x_{\mathrm{exp}}$) such that $B_m+B_e=n-1$. The associated constant exponent bias ($E_{\mathrm{bias}}$) is computed as $(2^{{B_e}-1}-1)$. We denote this format as $E_{B_e}M_{B_m}$.  

\subsubsection{Quantization Scheme}
\label{subsec:quant_method}
A quantization scheme dictates how a given unquantized tensor is converted to its quantized representation. We consider FP formats for the purpose of illustration. Given an unquantized tensor $\bm{X}$ and an FP format $E_{B_e}M_{B_m}$, we first, we compute the quantization scale factor $s_X$ that maps the maximum absolute value of $\bm{X}$ to the maximum quantization level of the $E_{B_e}M_{B_m}$ format as follows:
\begin{align}
\label{eq:sf}
    s_X = \frac{\mathrm{max}(|\bm{X}|)}{\mathrm{max}(E_{B_e}M_{B_m})}
\end{align}
In the above equation, $|\cdot|$ denotes the absolute value function.

Next, we scale $\bm{X}$ by $s_X$ and quantize it to $\hat{\bm{X}}$ by rounding it to the nearest quantization level of $E_{B_e}M_{B_m}$ as:

\begin{align}
\label{eq:tensor_quant}
    \hat{\bm{X}} = \text{round-to-nearest}\left(\frac{\bm{X}}{s_X}, E_{B_e}M_{B_m}\right)
\end{align}

We perform dynamic max-scaled quantization \citep{wu2020integer}, where the scale factor $s$ for activations is dynamically computed during runtime.

\subsection{Vector Scaled Quantization}
\begin{wrapfigure}{r}{0.35\linewidth}
  \centering
  \includegraphics[width=\linewidth]{sections/figures/vsquant.jpg}
  \caption{\small Vectorwise decomposition for per-vector scaled quantization (VSQ \citep{dai2021vsq}).}
  \label{fig:vsquant}
\end{wrapfigure}
During VSQ \citep{dai2021vsq}, the operand tensors are decomposed into 1D vectors in a hardware friendly manner as shown in Figure \ref{fig:vsquant}. Since the decomposed tensors are used as operands in matrix multiplications during inference, it is beneficial to perform this decomposition along the reduction dimension of the multiplication. The vectorwise quantization is performed similar to tensorwise quantization described in Equations \ref{eq:sf} and \ref{eq:tensor_quant}, where a scale factor $s_v$ is required for each vector $\bm{v}$ that maps the maximum absolute value of that vector to the maximum quantization level. While smaller vector lengths can lead to larger accuracy gains, the associated memory and computational overheads due to the per-vector scale factors increases. To alleviate these overheads, VSQ \citep{dai2021vsq} proposed a second level quantization of the per-vector scale factors to unsigned integers, while MX \citep{rouhani2023shared} quantizes them to integer powers of 2 (denoted as $2^{INT}$).

\subsubsection{MX Format}
The MX format proposed in \citep{rouhani2023microscaling} introduces the concept of sub-block shifting. For every two scalar elements of $b$-bits each, there is a shared exponent bit. The value of this exponent bit is determined through an empirical analysis that targets minimizing quantization MSE. We note that the FP format $E_{1}M_{b}$ is strictly better than MX from an accuracy perspective since it allocates a dedicated exponent bit to each scalar as opposed to sharing it across two scalars. Therefore, we conservatively bound the accuracy of a $b+2$-bit signed MX format with that of a $E_{1}M_{b}$ format in our comparisons. For instance, we use E1M2 format as a proxy for MX4.

\begin{figure}
    \centering
    \includegraphics[width=1\linewidth]{sections//figures/BlockFormats.pdf}
    \caption{\small Comparing LO-BCQ to MX format.}
    \label{fig:block_formats}
\end{figure}

Figure \ref{fig:block_formats} compares our $4$-bit LO-BCQ block format to MX \citep{rouhani2023microscaling}. As shown, both LO-BCQ and MX decompose a given operand tensor into block arrays and each block array into blocks. Similar to MX, we find that per-block quantization ($L_b < L_A$) leads to better accuracy due to increased flexibility. While MX achieves this through per-block $1$-bit micro-scales, we associate a dedicated codebook to each block through a per-block codebook selector. Further, MX quantizes the per-block array scale-factor to E8M0 format without per-tensor scaling. In contrast during LO-BCQ, we find that per-tensor scaling combined with quantization of per-block array scale-factor to E4M3 format results in superior inference accuracy across models. 



\end{document}

