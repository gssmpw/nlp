\documentclass{article}

% \usepackage[numbers]{natbib}
\usepackage{amsmath}
\usepackage{array}
\usepackage{titlesec}  
\usepackage{makecell}
\usepackage{wrapfig}
\usepackage{graphicx}
\usepackage{geometry}
\usepackage{multirow}
\usepackage{subcaption}
\geometry{a4paper, margin=1in}
\usepackage{enumitem}
\renewcommand\theadalign{cc}

\newcommand\DoToC{%
  \startcontents
  \printcontents{}{1}{\textbf{Contents}\vskip3pt\hrule\vskip5pt}
  \vskip3pt\hrule\vskip5pt
}

\usepackage{amsmath}
\usepackage{array}
\usepackage{titlesec}  
\usepackage{makecell}
\usepackage{wrapfig}
\usepackage{graphicx}
\usepackage{geometry}
\usepackage{multirow}
\usepackage{subcaption}
\usepackage[utf8]{inputenc} % allow utf-8 input
\usepackage[T1]{fontenc}    % use 8-bit T1 fonts
\usepackage{hyperref}       % hyperlinks
\usepackage{url}            % simple URL typesetting
\usepackage{booktabs}       % professional-quality tables
\usepackage{amsfonts}       % blackboard math symbols
\usepackage{nicefrac}       % compact symbols for 1/2, etc.
\usepackage{microtype}      % microtypography
\usepackage{xcolor}         % colors
\usepackage{booktabs}
\usepackage{graphicx}     % colors

% \usepackage[numbers]{natbib}


\usepackage{enumitem}
\setlength{\textfloatsep}{5.0pt plus 1.0pt minus 2.0pt}





\renewcommand{\thetable}{R\arabic{table}}


\renewcommand\theadalign{cc}
\newcommand\DoToC{%
  \startcontents
  \printcontents{}{1}{\textbf{Contents}\vskip3pt\hrule\vskip5pt}
  \vskip3pt\hrule\vskip5pt
}


\setlength{\textfloatsep}{5.0pt plus 1.0pt minus 2.0pt}


\begin{document}
% Please add the following required packages to your document preamble:
\begin{table}[]
\caption{Evalutation of QHNet with WALoss.}
\centering
\resizebox{0.7\columnwidth}{!}{%
\begin{tabular}{@{}llll@{}}
\toprule
Unit (kcal/mol)  & Homo MAE & Lumo MAE & Total Energy MAE \\ \midrule
QHNet            & 60.763   & 63.330   & 61694.031        \\
QHNet w/ WA Loss & \textbf{13.945}   & \textbf{14.087}   & \textbf{75.625}           \\ \bottomrule
\end{tabular}%
}
\end{table}
\begin{table}[]
\centering


\caption{Scaling Coefficients of the HOMO energy with the carbon atoms. Lower coefficients indicates less scaling error with the increased carbon atoms.}
\resizebox{0.7\columnwidth}{!}{%
\begin{tabular}{@{}llll@{}}
\toprule
        & HOMO Scaling Coefficients & LUMO Scaling Coefficients & Gap Scaling Coefficients \\ \midrule
WANet   & \textbf{0.42105}                   &  \textbf{0.4003}                    & \textbf{0.0206}                   \\
init    & 0.42207                   & 0.4006                    & 0.0214                   \\
without WANet & 1.3639                    & 2.7646                    & 1.4007                   \\ \bottomrule
\end{tabular}%
}


\end{table}
\begin{table}[]
\centering
\caption{Wall-clock comparison of WANet + DFT with the traditional SCF iterations. Unit are in seconds. The time was evaluated on a single NVIDIA A6000 machine.}
\resizebox{0.7\columnwidth}{!}{%
\begin{tabular}{@{}llll@{}}
\toprule
Method            & WANet + DFT & DFT      & Gap Scaling Coefficients \\ \midrule
Total Time        & 302.7630    & 392.0791 & 0.0206                   \\
NN Inference Time & 0.1402      & N/A      & 0.0214                   \\
SCF Time          & 302.6228    & 392.0791 & 1.4007                   \\ \bottomrule
\end{tabular}%
}
\end{table}

\begin{table}[]
\centering
\caption{Efficiency and resource comparison of training QH9 and WANet on PubChemQH dataset. The time was evaluated on a single NVIDIA A6000 machine.}
\resizebox{0.5\columnwidth}{!}{%
\begin{tabular}{@{}lll@{}}
\toprule
                & QH9 + WALoss & WANet + WALoss \\ \midrule
Training Time   & 90hr26min    & 39hr13min      \\
Inference Time  & 0.45 it/s    & 1.09 it/s      \\
Peak GPU Memory & 26.49GB      & 15.86GB        \\ \bottomrule
\end{tabular}%
}
\end{table}

\begin{table}[]
\caption{Transferability Experiments. Model trained on WANet were transfered to QH9.}
\centering
\resizebox{0.5\columnwidth}{!}{%
\begin{tabular}{@{}lll@{}}
\toprule
                             & Occupied Energy MAE & C Similarity  \\ \midrule
QH9 Stable from Scratch      & 0.4587          & 96.95\%                        \\
QH9 Stable with WANet weight & \textbf{0.4322}          & \textbf{97.02\%}                       \\
\bottomrule
\end{tabular}%
}
\end{table}
% \begin{table}[htbp!]
%     \centering
%     \begin{tabular}{ccccccc}
%     \hline
%         MAE - 1e-6 Hatree & 1 & 2 & 3 & 4 & 5 & 6 \\
%     \hline
%         DFT iteration times & 1 & 2 & 3 & 4 & 5 & 6 \\
%     \hline
%     \end{tabular}
%     \caption{The accuracy of Hamiltonian matrices and the times of iteration for the DFT convergence}
%     \label{tab:dft_conver}
% \end{table}


% \end{minipage}

\end{document}


% \begin{table}[htbp!]
%     \centering
%     \begin{tabular}{ccccccc}
%     \hline
%         MAE - 1e-6 Hatree & 1 & 2 & 3 & 4 & 5 & 6 \\
%     \hline
%         DFT iteration times & 1 & 2 & 3 & 4 & 5 & 6 \\
%     \hline
%     \end{tabular}
%     \caption{The accuracy of Hamiltonian matrices and the times of iteration for the DFT convergence}
%     \label{tab:dft_conver}
% \end{table}


\section{Figure}

\renewcommand{\thefigure}{R\arabic{figure}}

% \begin{figure}[h]
%     \centering
%     \includegraphics[width=0.4\linewidth]{figs/sp_flowchart.png}
%     \caption{The work flow of sparse tensor product}
%     \label{fig:sp_flowchart}
% \end{figure}

% \begin{figure}[h]
%     \centering
%     \includegraphics[width=0.5\linewidth]{figs/training_curve.png}
%     \caption{The training curve of LightH and the baseline model QHNet in the QH9 stable dataset}
%     \label{fig:training_curve}
% \end{figure}
% \begin{minipage}{\linewidth}
\begin{figure}[htbp!]
    \centering
    \begin{subfigure}[b]{0.45\linewidth}
        \centering
        \includegraphics[width=\linewidth]{figs/sp_flowchart.png}
        \label{fig:sp_flowchart}
    \end{subfigure}
    \hfill
    \begin{subfigure}[b]{0.45\linewidth}
        \centering
        \includegraphics[width=\linewidth]{figs/training_curve.pdf}
        \label{fig:training_curve}
    \end{subfigure}
    \caption{Left: The work flow of sparse tensor product. Right: The training curve of LightH and the baseline model QHNet in the QH9 stable dataset.}
    \label{fig:both_images}
\end{figure}
% \end{minipage}

\end{document}