\section{Related work}
Solving the optimal rotation from data contaminated with outliers and noise is non-trivial, due to the inherently non-linear~\cite{chin2022maximum} and non-convex\cite{brynte2022tightness} nature of the \(\mathbb{SO}(3)\) space. These characteristics make rotation estimation a complex optimization challenge~\cite{bustos2016robust}, typically rendering the solution process is \(\mathcal{NP}\)-hard~\cite{peng2022arcs}. Specifically, solving the optimal rotation problem involves two main requirements: (1) robustness, and (2) efficiency. Recently, many researchers have proposed various methods, including optimization-based algorithms~\cite{maken2019speeding}, heuristic-based algorithms~\cite{nuchter2005heuristic}, deep learning-based algorithms~\cite{2021Deep}, and multi-model fitting algorithms for multiple rotation estimation~\cite{kluger2020consac}.
\subsection{Optimization-based algorithms}
Optimization-based algorithms depend on the gradient of the objective function. These methods can achieve high accuracy when initialized with suitable values. Despite their high computational efficiency, these algorithms may converge to local optima due to non-convexity. For example, in the ICP (Iterative Closest Point) algorithm~\cite{1992A}, the solution is iteratively optimized until convergence. If the initial solution is well-chosen, convergence requires only a few dozen iterations, with precision reaching \(<10^{-10}\)~\cite{wright2006numerical}. However, if the initial values are poorly chosen, these algorithms may converge to a local optimum after several dozen iterations. Consequently, they frequently serve as the concluding optimization step in various algorithmic frameworks~\cite{wright2006numerical}.
\subsection{Heuristic-based algorithms}
Heuristic-based algorithms skillfully avoid local optima and are non-deterministic in the sense that they produce a reasonable result only with a certain probability. For example, the RANSAC (RANdom SAmple Consensus) algorithm~\cite{fischler1981random} provides a common generate-and-verify pipeline for robustly removing outliers. Similarly, the FGR (Fast Global Registration) algorithm~\cite{2016Fast} uses the Geman-McClure cost function and estimates the model through progressive non-convex optimization. However, these algorithms lack guarantees for finding the optimal solution.
\subsection{Deep-learning based algorithms}
Deep-learning models can improve their robustness to noise and outliers through training on large data, maintaining high estimation accuracy even with imperfect data. Additionally, deep learning frameworks and hardware acceleration enable efficient processing of large-scale data, enhancing the efficiency of rotation estimation. however deep learning-based algorithms require training large amounts of data in advance. For instance, DGR~\cite{2020Deep} employs full convolution techniques to enhance global context capture. CN-Net~\cite{yi2018learning} combines correspondence classification with model estimation. Building on CN-Net, 3DRegNet~\cite{20193DRegNet} extends its application to 3D by integrating a regression module for handling rigid transformations. PointDSC~\cite{2021PointDSC} focuses on accelerating model generation and selection by introducing a non-local module based on spatial consistency along with neural spectral matching. Additionally, DetarNet~\cite{chen2022detarnet} offers solutions for decoupling translation and rotation. Lastly, DHVR~\cite{2021Deep} leverages deep Hough voting~\cite{qi2019deep} to extract consensus from Hough space, accurately predicting transformations.
\subsection{Multi-model fitting algorithms for multiple rotation estimation}
Classical multi-model fitting methods, such as Sequential RANSAC~\cite{zuliani2005multiransac}, proceed sequentially. This approach applies RANSAC to remove inliers of the chosen and repeats the process until a stopping criterion is reached, fitting multiple models in sequence. Although RANSAC itself is robust to outliers, the sequential processing can cause outliers from previous models to affect subsequent model estimations. Inaccurate initial model estimations can lead to error accumulation in successive models. Modern state-of-the-art methods address the multi-model fitting problem simultaneously by assigning data points to models or outlier classes using clustering~\cite{magri2014t} or optimization techniques~\cite{barath2019progressive}. However, these methods can face high computational complexity with large datasets, especially in scenarios involving multiple rotations, which require analyzing and processing numerous preference pairs. Furthermore, to the best of our knowledge, there is no such algorithm that can solve multiple rotation estimation problems efficiently. 
\begin{algorithm}[t]
    \DontPrintSemicolon
    \KwIn{Observations $\left \{ \bm{\mathit{x_{i} }} ,\bm{\mathit{y_{i} }}\right \}_{i=1}^{N}$, noise level}
    \KwOut{Optimal rotation $\mathbf{R} $}
    \BlankLine
    
    Initialize a two-dimensional accumulator $\mathbb{A}$\;
    $\theta = \left \{ \theta _{j}  \right \} _{j=1}^{J}$ by discretizing $\left [ -\pi ,\pi  \right ]$  uniformly\;

    \For{$i\in \left [ I \right ]$ and $j\in \left [ J \right ]$}
    {
    
    $\bm{\mathit{z}}_{i} = \bm{\mathit{y}}_{i}-\bm{\mathit{x}}_{i} = \left ( \bm{\mathit{a}}_{i} , \bm{\mathit{b }}_{i},\bm{\mathit{c}}_{i} \right )$ and $\bm{\mathit{a}}_{i}^{2}+\bm{\mathit{b}}_{i}^{2}+\bm{\mathit{c}}_{i}^{2}=1$\;

    Calculate basis vector $\alpha _{0} $\;
    $Proj_{\bm{\mathit{z}}_{i}} \left ( \alpha _{0} \right ) =\left ( \frac{\alpha _{0}\cdot \bm{\mathit{z}}_{i}}{\left \| \bm{\bm{\mathit{z}}_{i}} \right \|^{2}}\right )\cdot \bm{\mathit{z}}_{i}$\;

    $\bm{\alpha}_{2}=\bm{\alpha}_{1}\times\bm{\mathit{z}}_{i}$\;

    Points3d = $\bm{\alpha}_{1}\cos\theta _{j} + \bm{\alpha}_{2}\cos\theta _{j}$\;
    Points2d$\xleftarrow{\text{stereographic projection}}$points3d\;
    
    $\mathbb{A} \longleftarrow \mathbb{A} + 1$ at points2d\;

    The center of the block with the highest value in $\mathbb{A}$ is back-projected onto the sphere, yielding a point that represents the estimated rotation axis\;

    \For{$i\in \left [ I \right ]$}
    {
    $\bm{\beta}_{i} = \bm{axis}_{i}\times  \bm{\mathit{x}}_{i}$\;
    $\bm{\gamma}_{i} = \bm{axis}_{i}\times \bm{\mathit{y}}_{i}$\;
    $ angle_{i} = \arccos \frac{\bm{\beta}_{i} \cdot \bm{\gamma}_{i} }{\left \| \bm{\beta}_{i} \right \|\left \| \bm{\gamma}_{i} \right \| }$\;
    
    $\mathbb{A} \longleftarrow \mathbb{A} + 1$ at $angle_{i}$\;
    The center of the block with the highest value in $\mathbb{A}$ is returned as the estimated rotation angle\;
    } 
    }
    The rotation $\mathbf{R} $ is obtained using Rodrigues' formula.

    \caption{AORESP: Accelerating Outlier-robust Rotation Estimation by
Stereographic Projection}

\end{algorithm}