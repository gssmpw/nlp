% VLDB template version of 2020-08-03 enhances the ACM template, version 1.7.0:
% https://www.acm.org/publications/proceedings-template
% The ACM Latex guide provides further information about the ACM template

\documentclass[sigconf, nonacm]{acmart}
\usepackage{caption}
\usepackage{subcaption}
\usepackage[linesnumbered,ruled,vlined]{algorithm2e}
% \usepackage{algorithm}
% \usepackage{algorithmic}
\usepackage{enumitem}
% \usepackage{ulem}
% \usepackage{float}
\usepackage{xcolor}
\usepackage{stfloats}
\usepackage{graphicx}
\usepackage{pifont}
\usepackage{url}
\usepackage{balance}
\usepackage[normalem]{ulem}
\usepackage[a-2b,mathxmp]{pdfx}
\usepackage{amsmath}
\usepackage{multicol}
\usepackage{multirow}
\usepackage{booktabs}

%% The following content must be adapted for the final version
% paper-specific
\newcommand\vldbdoi{XX.XX/XXX.XX}
\newcommand\vldbpages{XXX-XXX}
% issue-specific
\newcommand\vldbvolume{14}
\newcommand\vldbissue{1}
\newcommand\vldbyear{2020}
% should be fine as it is
\newcommand\vldbauthors{\authors}
\newcommand\vldbtitle{\shorttitle} 
% leave empty if no availability url should be set
\newcommand\vldbavailabilityurl{http://vldb.org/pvldb/format_vol14.html}
% whether page numbers should be shown or not, use 'plain' for review versions, 'empty' for camera ready
\newcommand\vldbpagestyle{plain} 

\usepackage{color}
\usepackage{xcolor}
\usepackage{CJKutf8}
\newcommand{\todo}[1]{\textcolor{magenta}{[TODO: #1]}}
\newcommand{\whiteding}[1]{\ding{\numexpr171+#1\relax}}
\newcommand{\blackding}[1]{\ding{\numexpr181+#1\relax}}
\newcommand{\whitedingB}[1]{\ding{\numexpr191+#1\relax}}
\newcommand{\blackdingB}[1]{\ding{\numexpr201+#1\relax}}
\usepackage{fontspec}
\newfontfamily\statefont{Courier New}
\newcommand{\StateFont}[1]{\statefont #1}

% custom
\usepackage{commons/code}
\usepackage{commons/colors}
\newcommand{\algolabel}[1]{\genericlabel{alg}{#1}}
\newcommand{\algoref}[1]{Algorithm~\ref{alg:#1}}
\newcommand{\applabel}[1]{\genericlabel{app}{#1}}
\newcommand{\appref}[1]{Appendix~\ref{app:#1}}
\newcommand{\clmlabel}[1]{\genericlabel{clm}{#1}}
\newcommand{\clmref}[1]{Claim~\ref{clm:#1}}
\newcommand{\eqnlabel}[1]{\genericlabel{eqn}{#1}}
\newcommand{\eqnref}[1]{\eqref{eqn:#1}}
\newcommand{\examplelabel}[1]{\genericlabel{exa}{#1}}
\newcommand{\exampleref}[1]{Example~\ref{exa:#1}}
\newcommand{\figlabel}[1]{\genericlabel{fig}{#1}}
\newcommand{\figref}[1]{Figure~\ref{fig:#1}}
\newcommand{\lemlabel}[1]{\genericlabel{lem}{#1}}
\newcommand{\lemref}[1]{Lemma~\ref{lem:#1}}
\newcommand{\lstlabel}[1]{\genericlabel{lst}{#1}}
\newcommand{\lstref}[1]{Listing~\ref{lst:#1}}
\newcommand{\seclabel}[1]{\genericlabel{sec}{#1}}
\newcommand{\secref}[1]{Section~\ref{sec:#1}}
\newcommand{\thmlabel}[1]{\genericlabel{thm}{#1}}
\newcommand{\thmref}[1]{Theorem~\ref{thm:#1}}
\newcommand\sysname{\ensuremath{\textsf{Tristar}}\xspace}

\begin{document}
% \title{\sysname: An Adaptive and Efficient Transaction Isolation Tailor for Database Backend Applications}
\title{\sysname: An Adaptive and Efficient Transaction Isolation Tailor for Database Backend Applications}

%%
%% The abstract is a short summary of the work to be presented in the
%% article.
\begin{abstract}
Retrieval-Augmented Generation (RAG) is often used with Large Language Models (LLMs) to infuse domain knowledge or user-specific information. In RAG, given a user query, a retriever extracts chunks of relevant text from a knowledge base. These chunks are sent to an LLM as part of the input prompt. Typically, any given chunk is repeatedly retrieved across user questions. However, currently, for every question, attention-layers in LLMs fully compute the key values (KVs) repeatedly for the input chunks, as state-of-the-art methods cannot reuse KV-caches when chunks appear at arbitrary locations with arbitrary contexts. Naive reuse leads to output quality degradation.  This leads to potentially redundant computations on expensive GPUs and increases latency. In this work, we propose \sys, a system for managing and reusing precomputed KVs corresponding to the text chunks (we call \textit{chunk-caches}) in RAG-based systems. We present how to identify \hl{\textit{chunk-caches} that are reusable}, how to efficiently perform a small fraction of recomputation to \textit{fix} the cache to maintain output quality, and how to efficiently store and evict \textit{chunk-caches} in the hardware for maximizing reuse while masking any overheads. With real production workloads as well as synthetic datasets, we show that \sys reduces redundant computation by \textbf{51\%} over SOTA prefix-caching and \textbf{75\%} over full recomputation.
\hl{Additionally, with continuous batching on a real production workload, we get a \textbf{1.6$\times$} speedup in throughput and a \textbf{2$\times$} reduction in end-to-end response latency over prefix-caching while maintaining quality, for both the \llama-3-8B and \llama-3-70B models. 
}
\end{abstract}






\maketitle

%%% do not modify the following VLDB block %%
%%% VLDB block start %%%
\pagestyle{\vldbpagestyle}
\begingroup\small\noindent\raggedright\textbf{PVLDB Reference Format:}\\
\vldbauthors. \vldbtitle. PVLDB, \vldbvolume(\vldbissue): \vldbpages, \vldbyear.\\
\href{https://doi.org/\vldbdoi}{doi:\vldbdoi}
\endgroup
\begingroup
\renewcommand\thefootnote{}\footnote{\noindent
This work is licensed under the Creative Commons BY-NC-ND 4.0 International License. Visit \url{https://creativecommons.org/licenses/by-nc-nd/4.0/} to view a copy of this license. For any use beyond those covered by this license, obtain permission by emailing \href{mailto:info@vldb.org}{info@vldb.org}. Copyright is held by the owner/author(s). Publication rights licensed to the VLDB Endowment. \\
\raggedright Proceedings of the VLDB Endowment, Vol. \vldbvolume, No. \vldbissue\ %
ISSN 2150-8097. \\
\href{https://doi.org/\vldbdoi}{doi:\vldbdoi} \\
}\addtocounter{footnote}{-1}\endgroup
%%% VLDB block end %%%

%%% do not modify the following VLDB block %%
%%% VLDB block start %%%
\ifdefempty{\vldbavailabilityurl}{}{
\vspace{.3cm}
\begingroup\small\noindent\raggedright\textbf{PVLDB Artifact Availability:}\\
The source code, data, and/or other artifacts have been made available at \url{\vldbavailabilityurl}.
\endgroup
}
%%% VLDB block end %%%

\begin{abstract}
Retrieval-Augmented Generation (RAG) is often used with Large Language Models (LLMs) to infuse domain knowledge or user-specific information. In RAG, given a user query, a retriever extracts chunks of relevant text from a knowledge base. These chunks are sent to an LLM as part of the input prompt. Typically, any given chunk is repeatedly retrieved across user questions. However, currently, for every question, attention-layers in LLMs fully compute the key values (KVs) repeatedly for the input chunks, as state-of-the-art methods cannot reuse KV-caches when chunks appear at arbitrary locations with arbitrary contexts. Naive reuse leads to output quality degradation.  This leads to potentially redundant computations on expensive GPUs and increases latency. In this work, we propose \sys, a system for managing and reusing precomputed KVs corresponding to the text chunks (we call \textit{chunk-caches}) in RAG-based systems. We present how to identify \hl{\textit{chunk-caches} that are reusable}, how to efficiently perform a small fraction of recomputation to \textit{fix} the cache to maintain output quality, and how to efficiently store and evict \textit{chunk-caches} in the hardware for maximizing reuse while masking any overheads. With real production workloads as well as synthetic datasets, we show that \sys reduces redundant computation by \textbf{51\%} over SOTA prefix-caching and \textbf{75\%} over full recomputation.
\hl{Additionally, with continuous batching on a real production workload, we get a \textbf{1.6$\times$} speedup in throughput and a \textbf{2$\times$} reduction in end-to-end response latency over prefix-caching while maintaining quality, for both the \llama-3-8B and \llama-3-70B models. 
}
\end{abstract}






\section{Introduction}
\label{sec:intro}

\begin{figure*}[tb]
    \centering
    \includegraphics[width=0.848\linewidth]{figs/circuitnn.pdf} 
    \caption{Illustration of differentiable CircuitNN. CircuitNN is designed based on differentiable NAND gates. After DAS is guided by PI and PO pairs of the truth table, CircuitNN can get the precise circuit architecture logic equivalent to the truth table.}
    \label{fig:circuitnn}
\end{figure*}

% 1. Describe the importance of logic synthesis
% 2. Existing Problems
% (a) Neural Architecture Search: Unstable, Predefined Setting, etc.
% (b) Circuit Generation: Probabilistic Model, Logic Equivalence

With the rapid advancement of technology, the scale of integrated circuits (ICs) has expanded exponentially. 
This expansion has introduced significant challenges in chip manufacturing, particularly concerning power and area metrics.
A primary objective in IC design is achieving the same circuit function with fewer transistors, thereby reducing power usage and area occupancy.

Logic synthesis~\cite{hachtel2005logicsynth}, a critical step in electronic design automation (EDA), transforms behavioral-level circuit designs into optimized gate-level circuits, ultimately yielding the final IC layout. 
The primary goal of logic synthesis is to identify the physical implementation with the fewest gates for a given circuit function. 
This task constitutes a challenging NP-hard combinatorial optimization problem. 
Current logic synthesis tools~\cite{brayton2010abc, wolf2013yosys} rely on human-designed heuristics, often leading to sub-optimal outcomes.

Differentiable architecture search (DAS) techniques~\cite{liu2018darts, chu2020darts} offer novel perspectives on addressing challenges in this problem.
Circuit functions can be represented through truth tables, which map binary inputs to their corresponding outputs. 
Truth tables provide a precise representation of input-output relationships, ensuring the design of functionally equivalent circuits.
Inspired by this, researchers~\cite{deepmind2024ai4sys, wang2024tnet} have begun exploring the application of DAS to synthesize circuits directly from truth tables.
Specifically, \citet{deepmind2024ai4sys} proposed CircuitNN, a framework that learns differentiable connection structures with logic gates, enabling the automatic generation of logic circuits from truth tables.
This approach significantly reduces the complexity of traditional circuit generation. 
Building on this, \citet{wang2024tnet} introduced T-Net, a triangle-shaped variant of CircuitNN, incorporating regularization techniques to enhance the efficiency of DAS.

Despite these advancements, several challenges remain. 
The computational complexity of DAS grows quadratically with the number of gates, posing scalability issues.
Although triangle-shaped architecture~\cite{wang2024tnet} partially mitigates this problem, redundancy persists. 
%Additionally, DAS is susceptible to converging to local optima, limiting the ability to search architectures that satisfy the given truth tables~\cite{liu2018darts}. 
%Furthermore, hyperparameters (network depth and layer width) require extensive searches, introducing complexity and prolonging the synthesis process. 
Additionally, DAS is susceptible to converging to local optima~\cite{liu2018darts} and hyperparameters (network depth and layer width) require extensive searches. 
The challenges arise from the vast search space in DAS. 
% Even with predefined settings for CircuitNN, finding a configuration that meets the truth table requires extensive trial and error during the DAS process. 
Intuitively, limiting the search space through predefined parameters (network depth, gates per layer, and connection probabilities) can significantly reduce the complexity.

Recent advances~\cite{openai2023gpt4, abramson2024alphafold3, esser2024sd3, li2024mar} in conditional generative models have demonstrated remarkable performance across language, vision, and graph generation tasks. 
Motivated by these developments, we propose a novel approach to circuit generation that generates preliminary circuit structures to guide DAS in generating refined circuits matching specified truth tables. 
Firstly, we introduce CircuitVQ, a tokenizer with a discrete codebook for circuit tokenization. 
Built upon our Circuit AutoEncoder framework~\cite{hou2022graphmae,li2023maskgae,wu2025mgvga}, CircuitVQ is trained through a circuit reconstruction task. 
Specifically, the CircuitVQ encoder encodes input circuits into discrete tokens using a learnable codebook, while the decoder reconstructs the circuit adjacency matrix based on these tokens.
Subsequently, the CircuitVQ encoder serves as a circuit tokenizer for CircuitAR pretraining, which employs a masked autoregressive modeling paradigm~\cite{chang2022maskgit, li2023mage}. 
In this process, the discrete codes function as supervision signals. 
After training, CircuitAR can generate discrete tokens progressively, which can be decoded into initial circuit structures by the decoder of the CircuitVQ. 
These prior insights can guide DAS in producing refined circuits that match the target truth tables precisely.

Our key contributions can be summarized as follows:
\begin{itemize}
\item We introduce CircuitVQ, a circuit tokenizer that facilitates graph autoregressive modeling for circuit generation, based on our Circuit AutoEncoder framework;
\item Develop CircuitAR, a model trained using masked autoregressive modeling, which generates initial circuit structures conditioned on given truth tables;
\item Propose a refinement framework that integrates differentiable architecture search to produce functionally equivalent circuits guided by target truth tables;
\item Comprehensive experiments demonstrating the scalability and capability emergence of our CircuitAR and the superior performance of the proposed circuit generation approach.
\end{itemize}

% Motivation
% (a) Diffusion (Vision, Graph), Autoregressive (Language, Vision)
% (b) Circuit Generation for Predefined Setting
% (c) Neural Architecture Search for Strict Logic Equivalence

% Contribution
% (a) Circuit Tokenizer (new transformer arch, training strategy)
% (b) CircuitAR (train and gen strategies, post-ar strategy)
% (c) Extensive Evaluation including BitD (Bit Distance) for Scalability


\section{Basic Background: Supervised Learning and the PAC Model}
\label{sec:background}

At this point almost everyone has heard of machine learning (ML). Anyone likely to stumble upon this article will have also heard of its most influential special case, supervised learning, and those theoretically inclined will also be familiar with the PAC model. Nonetheless, I will set the stage by  recapping the basics.

\subsection{Basics of Supervised Learning}%Let's set the stage in any case

\emph{Supervised Learning} is the task of ``coming up'' with a function $f: \X \to \Y$ to ``explain'' or ``fit'' a sequence of input/output examples   $(x_1,y_1), \ldots, (x_n,y_n)$, with $x_i \in \X$ and $y_i \in \Y$.  Here $\X$ is a \emph{data domain} consisting of \emph{datapoints} $x \in \X$, $\Y$ is a \emph{label set} consisting of \emph{labels} $y \in \Y$, and the sequence $(x_1,y_1),\ldots,(x_n,y_n)$ is the \emph{training data} consisting of \emph{labeled examples (a.k.a. samples)}~$(x_i,y_i)$.  I~will refer to the chosen function $f$ as a \emph{predictor}, and to $n$ as the \emph{sample size}. A \emph{learning algorithm} takes as input training data, and outputs (some representation of) a predictor $f \in \Y^\X$.\footnote{Note that this describes the usual \emph{batch}, a.k.a.~\emph{offline}, setting of supervised learning. I do not discuss other paradigms such as online or active learning in this article.} 



Success in supervised learning is defined as \emph{generalization} to  future examples: For a typical \emph{test example}  $(x_{\tst},y_{\tst})$, the predicted label $y'_{\tst}=f(x_{\tst})$ should ``equal'' $y_{\tst}$, perhaps approximately. We usually assume the test example is drawn from the same  ``source'' as the training data  --- commonly, i.i.d.~from the same distribution. The quality of the prediction is quantified by $\ell(y'_{\tst},y_{\tst})$, where $\ell:~\Y~\times~\Y \to \RR_{\geq 0}$ is a \emph{loss function} chosen as part of the problem definition. Common loss functions include the 0-1 loss $\ell_{0-1}(y',y) = [y' \neq y]$ for \emph{classification} problems,\footnote{The notation $[P]$ denotes $1$ when predicate $P$ is true, and denotes $0$ when $P$ is false.} as well as the absolute loss $|y'-y|$ or squared loss $(y'-y)^2$ for \emph{regression problems} featuring $\Y  \sse \RR$.

Nontrivial generalization properties are typically only possible if one assumes something about the data.\footnote{The need for such an assumption is formalized by the  \emph{no free lunch theorems} of supervised learning \cite{wolpert_connection_1992,wolpert_lack_1996,schaffer_conservation_1994}.} The Bayesian approach to  machine learning, common in many applications, assumes some parametric form for the distribution generating the data, and postulates a prior on the parameters. This is not the approach I will take in this article. Instead, I will focus on the frequentist --- and some would say ``worst-case'' or ``adversarial'' ---  approach that is common in the computational learning theory community, embodied by the PAC model. Here we assume that the (training and test) data can be explained, perhaps approximately, by a function in some ``simple enough to learn'' class of functions $\H \sse \Y^\X$, often called the \emph{hypotheses}. Equivalently, we  seek a predictor which explains the unseen data roughly  as well as the best hypothesis $h^* \in \H$, whether or not we assume that $h^*$ itself provides a perfect explanation.



 \paragraph{Common Algorithmic Templates.} Perhaps the best known general-purpose supervised learning algorithm is \emph{empirical risk minimization (ERM)}, which chooses as its predictor a hypothesis $f \in \H$ minimizing $\frac{1}{n} \sum_{i=1}^n \ell(f(x_i),y_i)$ --- a quantity called the \emph{training error}, \emph{empirical error}, or \emph{empirical risk} of $f$. %\footnote{When multiple hypotheses minimize the empirical risk, we assume ERM breaks ties arbitrarily.}
A common template for generalizing ERM involves adding a \emph{regularization term} $\psi(f)$ to the  objective function, typically chosen to measure some notion of ``hypothesis complexity.'' An algorithm instantiating this template is known as a \emph{structural risk minimizer (SRM)}, and chooses as its predictor the hypothesis $f \in \H$ minimizing the \emph{structural risk} $\frac{1}{n} \sum_{i=1}^n \ell(f(x_i),y_i) + \psi(f)$. Other well-known algorithms, such as gradient descent and its variations,  can frequently be interpreted as approximate implementations of ERM or SRM.


\paragraph{Proper vs Improper Learning.} A learning algorithm is said to be \emph{proper} if its predictor $f$ is always chosen from the hypothesis class, i.e., $f \in \H$, otherwise it is said to be \emph{improper}. ERM  is an example of a proper learning algorithm, as are SRM algorithms of the form described above.  In the \emph{proper regime} of learning, algorithms are required to be proper. This article will be concerned with the more flexible \emph{improper regime} (a.k.a \emph{representation-independent learning}), where no such constraint is placed on the learner. In other words, all we care about is predictive power at test time, rather than any insights derived from the functional form or representation of the predictor~itself.


\subsection{The PAC Model}
A standard mathematical setup for evaluation of supervised learning algorithms, at least in the theoretical computer science community, is Valiant's \emph{Probably Approximately Correct (PAC) model} of learning (see e.g.~\cite{kearns_introduction_1994,mohri_foundations_2018}). Here, we assume there is an unknown distribution $\D$ on $\X \times \Y$ from which training and test data are  drawn.  Specifically, the labeled datapoints of the training set  $(x_1,y_1), \ldots, (x_n,y_n)$, as well as the test data  $(x_\tst,y_\tst)$, are i.i.d.~from $\D$. Often it is assumed that $\D$ lies in some class of distributions of interest. The \emph{true expected loss}, or simply \emph{loss}, of a predictor $f: \X \to \Y$ is the expected loss it incurs on draws from $\D$, written $L_\D(f) = \Ex_{(x,y) \sim \D} \ell(f(x),y)$.


There are two main ``settings'' in PAC learning. The  \emph{realizable setting} only requires that the data be perfectly explained by some hypothesis in $\H$. More generally, the \emph{agnostic setting} makes no assumption relating the data to the hypotheses, but shifts the goalposts as necessary to allow nontrivial guarantees: the expected loss at test time is evaluated only ``relative'' to that of the best hypothesis $h^* \in \H$. There are other settings which make more nuanced assumptions, such as $\D$ being of a particular parametric form or its support living in some (unknown) lower-dimensional space, etc. I will mostly discuss the realizable and agnostic settings in this article, those being the simplest and most studied from a theoretical perspective. %TODO:We will briefly discuss other settings in Section ??

The PAC model demands high probability guarantees of learners, in the worst case over distributions of interest. Consider first the realizable setting, where $\D$ is such that $\min_{h \in \H} L_{\D}(h) = 0$. A PAC learner has \emph{error} $\epsilon=\epsilon(n)$ and \emph{confidence} $\delta=\delta(n)$ if, when training data consists of $n$ i.i.d~samples from a realizable distribution $\D$, it produces a predictor $f$  satisfying $L_\D(f) \leq \epsilon$ with probability at least $1-\delta$. In the agnostic setting, where $\D$ can be arbitrary, we require $L_\D(f) - \min_{h \in \H} L_\D(h) \leq \epsilon$ with probability $1-\delta$.

In both the realizable and agnostic settings, we look for PAC learners with small $\epsilon$ and $\delta$ as a function of the sample size $n$. An equivalent perspective looks at the sample complexity $m(\epsilon,\delta)$, which is the minimum sample size which guarantees error  at most $\epsilon$ with probability at least $1-\delta$. We say a problem is \emph{PAC learnable} if its PAC sample complexity is finite whenever $\epsilon,\delta > 0$.

For most PAC learning problems, learnability and sample complexity are characterized in terms of a  ``dimension'' of the hypothesis class. Most prominently this is the \emph{VC dimension} for binary classification, the \emph{fat shattering dimension} for agnostic regression, and the \emph{DS dimension} for multiclass classification (see \cite{anthony_neural_1999,daniely_optimal_2014,brukhim_characterization_2022}). Treatment of these is beyond the scope of this article. The unfamiliar reader need not worry, however,  as dimensions will feature only tangentially in our~discussion.




%\paragraph{Learning settings: Realizable, Agnostic, etc.} In learning theory, evaluating a supervised learning algorithm requires specifying a data model and an objective. We will leave the details of the data model flexible for now, to allow for both the PAC model and the adversarial transductive model. Nonetheless we will describe two variations, which we call ``settings'', which cut across different models. The  \emph{realizable setting}  requires only that the data be perfectly explained by some hypothesis $h \in \H$ --- i.e., there exists a hypothesis which is guaranteed to suffer a loss of $0$ on training and test data. The performance of the learning algorithm is its expected loss at test time for some ``worst case'' realizable instance. More generally, the \emph{agnostic setting} makes no assumption relating the data to the hypotheses, but shifts the goalposts as necessary to allow nontrivial guarantees: the expected loss at test time is evaluated only ``relative'' to that of the best hypothesis $h^* \in \H$, again for some ``worst case'' instance. There are other settings which make more nuanced assumptions about the data, such as it is drawn from a distribution of a particular parametric form, or that it lives in some (unknown) lower-dimensional space, etc. We will mostly discuss the realizable and agnostic settings, those being the simplest and most studied from a theoretical perspective.




%%% Local Variables:
%%% mode: latex
%%% TeX-master: "learning_matching"
%%% End:


\begin{figure*}[t]
\begin{center}
\includegraphics[width=.85\linewidth]{fig_overview_v3.pdf}
\end{center}
\caption{
FastAtlas Overview: In each frame, we compute charts spanning fully or partially visible triangles (a), determine texture space bounding boxes for the visible portions of the view-space projections of each chart, and tightly pack these boxes into atlases (b, here $2K \times 2K$). We simultaneously bijectively parameterize and shade the charts into their atlas boxes, obtaining high quality texture space shading (c), and use this shading to render the shaded frames (d).}
\label{fig:overview}
\label{fig:alg_overview}
\end{figure*}

\section{Overview}
\label{sec:overview}
Our work has two core contributions: a real-time, GPU-based algorithm for tight packing of general parameterized charts into compact atlases; and a real-time TSS method that
utilizes this packing.  

\paragraph*{FastAtlas Packing.}
FastAtlas runs entirely on the GPU as a series of compute shaders. It takes the bounding boxes of parameterized charts as input, and packs them into an atlas (Fig~\ref{fig:overview}b, Sec.~\ref{sec:pack}). As such, the only input it requires are the dimensions of the bounding boxes.
Its outputs are deterministic; identical input charts are packed into identical atlases. This is critical for TSS and similar applications, as it ensures that consecutive frames taken from the same camera view have the same shading. Even minute shading differences across such frames can cause sampling jitter, leading to undesirable flicker \cite{baker2012rock}. 
While prior methods such as \cite{mueller2018shading,hladky2019tessellated,hladky2021snakebinning,Neff2022MSA} cap the dimensions of the charts that can be packed as-is for a given atlas size, and scale down all charts that exceed these dimensions, we scale all charts by the same factor, and do so only when strictly necessary to achieve packing success (Figs~\ref{fig:atlas},~\ref{fig:sas_issues}). 

\paragraph*{TSS using FastAtlas.}
Our end-to-end TSS atlas generation method combines the packing method above with a novel approach for computing seamless per-frame charts. 
We define our charts as the connected components of the visible surfaces in each frame (Fig.~\ref{fig:overview}a), and efficiently compute them using a parallel union-find algorithm (Sec.~\ref{sec:visible}). Since the boundaries of these charts coincide with the contours of the rendered surface, they are {\em invisible} to the viewer. This approach 
eliminates the artifacts caused by shading discontinuities along visible seams (Fig.~\ref{fig:seams}). 

\begin{parWithWrapFigure}
\begin{wrapfigure}{l}{.27\columnwidth}%
\includegraphics[width=\linewidth]{fig_inset_view_plane.pdf}%
\end{wrapfigure}
We bijectively parametrize the {\em visible portions} of our charts by projecting them to view space (inset). This maps a constant number of texels to each pixel in the final rendered output, evenly distributing residual undersampling error across all image pixels. While conceptually straightforward, efficiently parameterizing charts containing partially visible triangles using viewspace projection is non-trivial, as the visible portions may no longer be triangular (e.g. green triangle in the inset); applying naive projection to triangles with vertices behind the camera may produce ill-posed results. Clipping triangles before projection is both computationally expensive and significantly complicates downstream operations. We avoid explicit clipping by observing that all that is required for atlas packing is the dimensions of, potentially conservative, bounding boxes of these projected visible portions. We compute such bounding boxes without explicit chart clipping by adapting a conservative screen coverage estimator \shortcite{Blinn:CalculatingScreenCoverage} (Sec.~\ref{sec:box}). We then pack the computed boxes using FastAtlas. 
\end{parWithWrapFigure}

Finally, we shade the visible portion of each chart into its corresponding atlas bounding box (Fig~\ref{fig:overview}c). 
The resulting texture is then used during rasterization as a standard texture map (Fig. ~\ref{fig:overview}d). 
Our framework is compatible with all existing approaches for texture space shading, including forward shading, raytraced illumination, or deferred shading in texture space \cite{baker:2016}. In the examples shown, we use the standard forward shading based rendering pipeline included in the G3D Innovation Engine \cite{G3D17}, a commercial grade renderer.


\section{Design}\label{sec:design}

%%%%%%%%%%%%%%%%%%%%%%%%%%%%%%


\begin{figure*}[t]
    \centering
    \includegraphics[trim = 15 530 15 15, width=1\textwidth]{Algorithm_drawio.pdf}
    \caption{Overview of KiSS}
    \label{fig:overview}
\end{figure*}


The results we gleaned from the previous section (see Section~\ref{sec:work_anly}) helped in developing our policy: KiSS. The KiSS or \textbf{Keep it Separated Serverless} policy aims to address critical challenges in Function-as-a-Service (FaaS) platforms, particularly in edge computing environments, by achieving the following objectives:

\begin{itemize}
    \item \textbf{Reduced Cold Start Latency:} Prioritizes high-frequency functions to minimize delays in real-time applications.
    \item \textbf{Improved Resource Efficiency:} Optimizes memory and compute usage while avoiding unnecessary overhead from static warm states.
    \item \textbf{Minimized Inter-Function Interference:} Enhances throughput and scalability through modular resource partitioning.
    \item \textbf{Improved Function Service Rate:} Adopts resource-aware policies to reduce dropped requests and maximize system reliability.
\end{itemize}


\subsection{KiSS Policy Overview}

KiSS introduces a modular, data-driven orchestration strategy designed to optimize serverless execution in resource-constrained environments, particularly at the edge. By leveraging our workload analysis (refer Section 2.5), our policy segments functions based on key metrics—memory footprint, invocation frequency, and execution time—to optimize performance across diverse workloads.

The edge computing context introduces unique challenges like limited memory, heterogeneous resources, and dynamic workloads. Generalized cloud strategies often fail to adapt to such constraints. KiSS addresses this gap by analyzing workload characteristics and implementing a resource-efficient, modular strategy that aligns with edge-specific demands.

\subsection{Components of KiSS Policy Design}
Figure~\ref{fig:overview} shows the overall architecture of KiSS. 
The incoming \textit{FaaS traffic} will include both small and large functions. 
The \textit{request handler} accepts the incoming functions and shares the function information to the workload analyzer. 
The \textit{workload analyser} processes the function information to profile the incoming function traffic information and generate data such as invocation frequency, memory footprint etc.
The \textit{KiSS policy} uses this data to estimate where this function will be placed between the two different warm pool partitions.

The \textit{load balancer} implements a partitioning logic where functions are allocated to distinct warm pools using (\textit{invoker 1} and \textit{invoker 2}) based on profiling thresholds:

(i)~Small Functions Pool: Dedicated to high-frequency, low-memory functions to ensure low latency, and (ii)~Large Functions Pool: Allocated for low-frequency, memory-intensive functions, minimizing contention with smaller containers.
Each warm pool operates autonomously achieving Policy Independence.
The \textit{Warm Pool Replacement Policy} for each warm container pool can independently implement different workload-specific strategies to reduce contention and enhance temporal locality.


These factors form the foundation of KiSS’s multi-tiered warm pool framework, allowing it to effectively manage serverless resources and enhance performance in edge computing. By addressing these challenges, KiSS positions itself as a practical and scalable solution for FaaS platforms in environments with diverse and demanding resource constraints.


\subsection{Innovations of KiSS Policy}

One of the most innovative features of KiSS is its multi-level warm pool partitioning, which isolates high- and low-frequency functions into separate pools. This design eliminates inefficiencies inherent in monolithic resource strategies by ensuring that small, frequently invoked functions are always ready to execute, while larger, less frequent functions remain accessible without competing for resources. This adaptability extends to the ability to add more pools as workload patterns evolve, making KiSS a flexible and future-proof solution. Moreover, its modular architecture supports diverse deployment scenarios, from centralized clouds to resource-constrained edge environments. Integration with traffic-aware schedulers ensures that KiSS maintains scalability and responsiveness even under fluctuating workloads.


\subsubsection{Advantages of KiSS}

The advantages of KiSS are particularly pronounced in edge environments. By keeping frequently accessed containers in warm states, it drastically reduces cold start latency, which is critical for real-time applications such as IoT and AI analytics. Static warm pool partitioning, based on workload analysis, optimizes memory usage by eliminating unnecessary overhead, ensuring that resources are used efficiently even in environments with stringent memory constraints. This strategy not only enhances performance but also reduces operational costs by consolidating memory usage and minimizing cold starts. KiSS’s platform-agnostic design further enhances its versatility, enabling seamless deployment across various serverless frameworks.



\section{Serializability and recovery}\label{sec:correctness}
% In this section, we prove the serializability under low isolation levels and the cross-isolation levels within \sysname. 
% Previous sections have demonstrated that the validation-based concurrency control guarantees the commit order of two transactions involved in a dependency. 

% Before we prove the guarantee of serializability in a weak isolation level, we first specify the relationship between the conflict graph (CG), which is constructed on the execution history, and the static dependency graph (SDG), which is based on transaction templates.
% For better illustration, the proof atmosphere is two-step; first, we will show that concurrency control in middleware guarantees the commit order of two transactions with dependency; then, we will prove the serializable guarantee of the workload in \sysname. 

% CG 中的依赖和 SDG 中的依赖是有对应关系的。
% \noindent\begin{lemma}
%     If $T_i \rightarrow T_j$ in the $CG$ of the execution history $H$, then $\mathcal{T}_i \rightarrow \mathcal{T}_j$ in the $SDG$ of the transaction template $\mathcal{T}$. 
%     % As a partial converse, if $\mathcal{T}_i \rightarrow \mathcal{T}_j$, then either $T_i \rightarrow T_j$ or $T_j \rightarrow T_i$ in $H$.
%     \label{theorem-1}
% \end{lemma}

% \noindent\textbf{Proof.} Suppose $T_i \rightarrow T_j$ within the execution history $H$, if $T_i \xrightarrow{ww} T_j$, then $H$ contains two write operations $w_i[x]$ and $w_j[x]$, belonging to $T_i$ and $T_j$, respectively. $x$ belongs to the intersection of the write sets of $T_i$ and $T_j$, it implies that $\mathcal{T}_i$ has potential ww dependency with $\mathcal{T}_j$. Thus, there exists a ww dependency in the $SDG$. Similarly, if there is rw or wr dependency between $T_i$ and $T_j$. 

% In this section, we first prove the serializability of \sysname's scheduling; in other words, \sysname can prevent all non-serializable schedulings, which are divided into the single-isolation level and cross-isolation level categories, and we provide the proof in \S~\ref{sec:proof.isolation} and \S~\ref{sec:proof.switch}, respectively. Lastly, we present the failure recovery strategy in \S~\ref{sec:proof:failure}.
In this section, we first prove the serializability of \sysname's scheduling in the single-isolation level and cross-isolation level categories in \S~\ref{sec:proof.isolation} and \S~\ref{sec:proof.switch}, respectively. Finally, we present the failure recovery strategy in \S~\ref{sec:proof:failure}. 

\subsection{Serializability under Low Isolation Levels \label{sec:proof.isolation}}
% Before we prove the isolation correctness, we identify $T.B$ as the beginning time of transaction $T$ and $T.C$ as the commit time of transaction $T$, and the characteristics of transactions involved in dependency edges are summarised in Table~\ref{tbl:correctness}. If $T_i \xrightarrow{ww} T_j$, 
% in SI, the FCW rule avoids concurrent updates, in other words, $T_i$ commits before $T_j$ starts. 
% While in RC, the database avoids \textit{dirty write} and guarantees read-last-committed. Hence, $T_j$ can only commit after $T_i$ commits. If $T_i \xrightarrow{wr} T_j$, $T_j$ reads the record written by $T_i$, $T_j$ must commit after $T_i$ commits in both SI and RC. Moreover, in SI, a transaction gets the snapshot at the beginning, we can make the characteristic more concise that $T_i$ commits before $T_j$ starts. Lastly, $T_i \xrightarrow{rw} T_j$, in both SI and RC, $T_i$ begins before $T_j$ commits, otherwise, $T_i$ reads the record written by $T_j$. 
% Before demonstrating the serializability of weak isolation levels, w
% \todo{1. previous work in SI and RC; 2. the concurrency control can identify all potential transaction templates and keep the dependency order ...; 3. conclusion}
% \textcolor{red}{in SI the only possible anomaly is from two consecutive RW and Tk commit first}
% \begin{theorem}
%     At the SI level, $H$ is conflict serializable if for every two consecutive RW dependency edges $T_i\xrightarrow{rw} T_j \xrightarrow{rw} T_k$ in $H$, $T_j$ commits before $T_k$.
%     \label{theorem-2}
% \end{theorem}
% \textcolor{red}{\sysname can guarantee serializability in SI as it forces Tj commit first}
% \textcolor{red}{in RC the only possible anomaly is from the RW and Tj commit first}
% \begin{theorem}
%     At the RC level, $H$ is conflict serializable if for every RW dependency edge $T_i\xrightarrow{rw} T_j$ in $H$, $T_i$ commits before $T_j$.
%     \label{theorem-3}
% \end{theorem}
% \textcolor{red}{\sysname can guarantee serializability in RC as it forces Ti commit first}
% Under standard low isolation levels, the execution order of transactions may violate SER, often manifested as dependency cycles. Luckily, these cycles have been proven to follow some specific vulnerable dependencies. 
% Serializability can be effectively guaranteed at weak isolation levels if these patterns are identified and appropriately addressed. 
% The non-serializable scheduling under each low isolation level serves some specific vulnerable dependencies.
% According to Theorem \ref{the:vulnerable}, a necessary condition for non-serializability is the inconsistent dependency and commit order of vulnerable dependencies. \sysname identifies the static vulnerable dependency from the transaction templates and ensures that in transactions involving vulnerable dependencies, the commit order is consistent with the dependency order as outlined in Algorithm \ref{alg.transaction}, thus maintaining SER with RDBMS configured to low isolation levels. 
Non-serializable scheduling under each low isolation level accommodates certain specific vulnerable dependencies. According to Theorem \ref{the:vulnerable}, a necessary condition for non-serializability is the presence of inconsistent dependencies and commit orders among these vulnerable dependencies. 
\maintext{The unified middle-tier concurrency control ensures the commit order respects dependency order for transactions with vulnerable dependencies, thereby preserving SER even when the RDBMS operates at lower isolation levels.}
\extended{\sysname identifies static vulnerable dependencies from the transaction templates and ensures that, in transactions involving these dependencies, the commit order aligns with the dependency order as specified in Algorithm \ref{alg.transaction}. This approach maintains SER even when the RDBMS is configured to low isolation levels.}

% Informally, at the SI level, a cycle is present if and only if it contains two consecutive RW conflicts $T_i \xrightarrow{rw} T_j \xrightarrow{rw} T_k$, with $T_k$ being the first transaction to commit, as discussed in Theorem \ref{def:si}. \sysname, at the SI level, records all potential two consecutive RW conflicts in the middle tier and ensures $T_j$ commits before $T_k$ as outlined in Algorithm \ref{alg.transaction}. This mechanism effectively prevents the formation of any SI cycles, thereby guaranteeing serializability.
% Similarly, at the RC level, a cycle exists if and only if it contains an RW conflict $T_i \xrightarrow{rw} T_j$, with $T_j$ being the first to commit as described in Theorem \ref{def:rc}. In this context, \sysname records all possible RW conflicts in the middle tier and enforces that $T_i$ commits before $T_j$ as detailed in Algorithm \ref{alg.transaction}. This approach ensures no RC cycles can form, thus maintaining serializability at the RC level. 

% \textcolor{red}{One possible conclusion: Unlike most SER CC protocols (e.g., SSI) that directly abort potential RW edges, this method permits certain RW edges to coexist while disallowing others that violate serializable orders. This approach effectively enhances concurrency within the system.}

% Following Theorem \ref{def:si} and Theorem \ref{def:rc} in \sysname, validation is integrated into the middle-tier concurrency control as discussed in Session \ref{design-1}, which is designed to handle transactions generated from templates associated with data dangerous structures at low isolation levels. This integration allows \sysname to proactively identify transactions that may lead to non-serializable scheduling, detect runtime dependencies, and ensure their correct commit order. This proactive approach effectively guarantees the absence of dependency cycles, thereby ensuring the serializability under low isolation levels.

% We first explore the possible \textcolor{red}{partial order} in an execution history $H$. 
% We denote the begin time of transaction $T$ as $T.B$ and the commit time as $T.C$. The \textcolor{red}{partial order} characteristics of conflict dependencies are summarized in Table~\ref{tbl:correctness}. 
% Consider the scenario where $T_i \xrightarrow{ww} T_j$. The First-Committer-Wins (FCW) rule at the SI level prevents concurrent updates by ensuring that $T_i$ commits before $T_j$ begins. At the RC level, the database prevents \textit{dirty write} and enforces the Read-Last-Committed rule, thus $T_j$ can only commit after $T_i$ has committed. 
% For the case where $T_i \xrightarrow{wr} T_j$, since $T_j$ reads the record written by $T_i$, $T_j$ commits only after $T_i$ has committed at both SI and RC levels. Moreover, at the SI level, since a transaction captures a snapshot at its beginning, it simplifies the relationship such that $T_i$ commits only before $T_j$ begins. Lastly, in the scenario where $T_i \xrightarrow{rw} T_j$, both SI and RC require that $T_i$ begins before $T_j$ commits to avoid $T_i$ reading a record that could potentially be written by $T_j$. 

% \begin{table}[t]
% \caption{Transaction characteristics of dependencies}
% \vspace{-4mm}
% \begin{tabular}{ccc}
% \toprule
%                       & Snapshot Isolation & Read Committed \\ 
% % \multicolumn{1}{l|}{$T_i \xrightarrow{ww} T_j$} & $commit(T_i) < begin(T_j)$ & $commit(T_i) < commit(T_j)$ \\
% % \multicolumn{1}{l|}{$T_i \xrightarrow{wr} T_j$} & $commit(T_i) < begin(T_j)$ & $commit(T_i) < commit(T_j)$ \\
% % \multicolumn{1}{l|}{$T_i \xrightarrow{rw} T_j$} & $begin(T_i) < commit(T_j)$ & $begin(T_i) < commit(T_j)$
% \midrule
% \multicolumn{1}{l}{$T_i \xrightarrow{ww} T_j$} & $T_i.C < T_j.B$ & $T_i.C < T_j.C$ \\
% \multicolumn{1}{l}{$T_i \xrightarrow{wr} T_j$} & $T_i.C < T_j.B$ & $T_i.C < T_j.C$ \\
% \multicolumn{1}{l}{$T_i \xrightarrow{rw} T_j$} & $T_i.B < T_j.C$ & $T_i.B < T_j.C$ \\
% \midrule
% \multicolumn{3}{l}{$<$ is the partial order \todo{in the execution history.}} \\
% \bottomrule
% \end{tabular}
% \label{tbl:correctness}
% \end{table}

% \begin{theorem}
%     In SI, H is conflict serializable if $T_j$ commits before $T_k$ for all two consecutive vulnerable dependency edges $T_i\xrightarrow{rw} T_j \xrightarrow{rw} T_k$ in H.
%     \label{theorem-2}
% \end{theorem}
% \noindent\textbf{Proof.} We adopt the contrapositive approach here, assuming that if $H$ is not conflict serializable, there must be a cycle in $CG$. We maintain the commit order of $T_j$ and $T_k$, assuming $T_{j^{'}}$ is the first transaction committed within this cycle. Let's denote $T_{s^{'}}$ and $T_{i^{'}}$ as immediate predecessors of $T_{j^{'}}$, e.g. $T_{s^{'}}\rightarrow T_{i^{'}} \rightarrow T_{j^{'}}$. Since $T_{j^{'}}.C < T_{i^{'}}.C$, as per Table~\ref{tbl:correctness}, only the rw dependency is possible between $T_{i^{'}}$ and $T_{j^{'}}$. We can further infer that $T_{i^{'}}.B < T_{j^{'}}.C$, and since T3 was submitted first, $T_{j^{'}}.C < T_{s^{'}}.C$. Based on these two conditions, we can deduce that $T_{i^{'}}.B < T_{s^{'}}.C$, making only the rw dependency is possible between $T_{s^{'}}$ and $T_{i^{'}}$ as well.
% However, we identify all transactions that match the structure and could keep the commit order of $T_{i^{'}}$ and $T_{j^{'}}$, which means $T_{j^{'}}$ can not be the first one in these three transactions to commit. This contradicts our derivation. Therefore, \sysname can ensure the conflict serializability when the RDBMS is configured to SI.

% \begin{theorem}
%     At the SI level, $H$ is conflict serializable if for every two consecutive RW dependency edges $T_i\xrightarrow{rw} T_j \xrightarrow{rw} T_k$ in $H$, $T_j$ commits before $T_k$.
%     \label{theorem-2}
% \end{theorem}
% \noindent\textbf{Proof.} 
% We prove this by contradiction, assuming that $H$ is not conflict serializable when $T_j$ commits before $T_k$. Let $T_{s^{'}}$ and $T_{i^{'}}$ be the immediate predecessors of $T_{j^{'}}$, such that $T_{s^{'}}\rightarrow T_{i^{'}} \rightarrow T_{j^{'}}$. Given that $T_{j^{'}}.C < T_{i^{'}}.C$, as indicated in Table~\ref{tbl:correctness}, the only feasible dependency between $T_{i^{'}}$ and $T_{j^{'}}$ is a RW dependency. We can further infer that $T_{i^{'}}.B < T_{j^{'}}.C$, and since $T_{j^{'}}$, $T_{j^{'}}.C < T_{s^{'}}.C$. From these conditions, it follows that $T_{i^{'}}.B < T_{s^{'}}.C$, suggesting that the only possible dependency between $T_{s^{'}}$ and $T_{i^{'}}$ is also RW.
% However, upon identifying all transactions that conform to this structure and could maintain the commit order of $T_{i^{'}}$ and $T_{j^{'}}$, it becomes evident that $T_{j^{'}}$ cannot be the first among these three transactions to commit. This observation contradicts our initial assumption. Therefore, we conclude that the \sysname can ensure conflict serializability at the SI level.

% \begin{theorem}
%     In RC, H is conflict serializable if $T_i$ commits before $T_j$ for all vulnerable dependency edges $T_i\xrightarrow{rw} T_j$ in H.
%     \label{theorem-3}
% \end{theorem}
% \noindent\textbf{Proof.} Similar to the proof of SI, we assume the existence of a dependency cycle in $H$, with $T_{i^{'}}$ as the first committed transaction and $T_{s^{'}}$ as its closest ancestor of $T_{i^{'}}$ in the cycle. This implies  $T_{s^{'}}\rightarrow T_{i^{'}}$ and $T_{i^{'}}.C < T_{s^{'}}.C$. According to Table~\ref{tbl:correctness}, an rw dependency exists between $T_{s^{'}}$ and $T_{i^{'}}$, and the validation-based concurrency control in \sysname ensures that $T_{s^{'}}$ commits before $T_{s^{'}}$ can commit. This contradiction indicates that \sysname can guarantee the conflict serializability of $H$ under RC.

% % \subsection{Consistent commit order}

% In \sysname, we incorporate validation into the transactions produced by transaction templates associated with data anomalies at weak isolation levels. This allows \sysname, during execution, to identify transactions that could potentially result in data anomalies and guarantee the commit order, all without the database's awareness. This approach disrupts the necessary conditions for the CG to form a dependency cycle.

\subsection{Serializability under Cross-isolation Levels \label{sec:proof.switch}}
% During the transition process, \sysname employs a more stringent concurrency control policy to ensure conflict serializability. Consider, for example, the switch from SER to SI. During this transition, we split the $CG$ of $H$ into two subgraphs, $CG_{SER}$ and $CG_{SI}$. $CG_{SER}$ contains transactions running under the SER isolation level, and $CG_{SI}$ contains those running under SI. $CG_{SER}$ is obviously acyclic, and according to Theorem~\ref{theorem-2}, so is $CG_{SI}$. Subsequently, we consider the dependency edges that connect $CG_{SER}$ and $CG_{SI}$. When the database processes transactions under both SI and SER concurrently, no exceptions will occur apart from those that could occur under SI alone. In \sysname, the concurrency control for all transactions is conducted according to the policy corresponding to SI, meaning the dependency between $CG_{SER}$ and $CG_{SI}$ forms a cycle neither.

\begin{comment}
Achieving serializability becomes complex, as \sysname must handle non-serializable scheduling arising from different isolation levels when transitioning from one isolation level $I_{old}$ to the new isolation level $I_{new}$.     
\end{comment}

% As detailed in Section~\ref{design-3}, this transition process comprises three distinct stages. 
% The serializability of Phase I, which encompasses all transactions operating under the old isolation level, as well as Phase III, involving all transactions under the new isolation level, can be readily proven as demonstrated in Section~\ref{sec:proof.isolation}.
% We will discuss the serializability for Phase II; the core is to prove the correctness of Theorem \ref{the:cross-isolation} and Theorem \ref{the:cross-isolation-commit}.  
% The correctness during the transition is established by proving the validity of Theorem \ref{the:cross-isolation} and Theorem \ref{the:cross-isolation-commit}.
% The validation locking mechanism in CIV allows the concurrency control algorithm described in Section \ref{design-1} to detect all vulnerable dependencies, $T_i \xrightarrow{rw} T_j$, regardless of whether $T_i$ and $T_j$ are executing under different or the same isolation levels. Furthermore, it enables $T_i$ to maintain its commit order consistent with the dependency order. 
We prove the correctness of the cross-isolation level in three steps: If there is non-serializable scheduling during the transition, \blackding{1} there exists at least a cross-isolation vulnerable dependency as defined in Definition~\ref{def:transition_vul}; 
% \blackding{2} The cross-isolation validation mechanism in \S\ref{design-3} can detect and avoid at least one cross-isolation vulnerable dependency in the non-serializable scheduling.
\blackding{2} there exists at least a cross-isolation vulnerable dependency $T_j \xrightarrow{rw} T_k$, where $T_k$ commits before $T_j$ and $T_j$ commits after the transition. 
\blackding{3} The cross-isolation validation mechanism can detect the dependency if $T_j$ commits after the transition and enforce the consistent commit and dependency order. \maintext{Due to the space limitation, we provide more details in our technique report \cite{TxnSails}. }

% In this subsection, we first prove that if we can ensure the commit order of $T_j$ and $T_k$ in the cross-isolation vulnerable dependency $T_i \xrightarrow{rw} T_j \xrightarrow{rw} T_k$ aligns with the dependency order, then transactions can achieve SER. Then, we prove that our cross-isolation validation can ensure that the transaction commit order is consistent with the data dependency order in all cross-isolation vulnerable dependencies, thus ensuring serializability.
% all cross-isolation vulnerable dependencies can be prevented according to our cross-isolation validation phase.
% We prove this in the following two steps.
% if it needs to.

% We first prove that if non-serializable scheduling exists, there is at least one cross-isolation vulnerable dependency $T_j \xrightarrow{rw} T_k$, where 1. $T_k$ commits before $T_j$; 2. $T_j$ operates under SER; and 3. $T_j$ commits after the transition starts.

% The lock manner in CIV enables the concurrency control algorithm in Section \ref{design-1} can detect all vulnerable dependencies, $T_i\xrightarrow{rw} T_j$, even $T_i$ and $T_j$ execute under different isolation levels. We need to prove that if non-serializable scheduling exists, at least one cross-isolation vulnerable dependency $T_j \xrightarrow{rw} T_k$, where $T_k$ commits before $T_j$; 2. $T_j$ operates under SER; 3. $T_j$ commits after the transition starts. 

\begin{figure}[t]
    \centering
    \includegraphics[width=0.47\textwidth]{figures/switch_correctness.pdf}
    \vspace{-4mm}
    \caption{Transition from SER to SI/RC}
    % \caption{Partial dependency between $S_{old}^{(1)}$, $S_{old}^{(2)}$, and $S_{new}$}
    \label{fig:switch_correctness}
    \vspace{-4mm}
\end{figure}
% If there is non-serializable scheduling, there is a cross-isolation vulnerable $T_j \xrightarrow{rw} T_k$ in this scheduling, where $T_k$ commits before $T_j$.

{
% \color{blue}
\blackding{1} 
\maintext{
We first prove the existence of cross-isolation vulnerable dependency in non-serializable scheduling. 
}
\extended{
We first prove that if there is non-serializable scheduling during the transition, there must be two consecutive RW dependencies, $T_i \xrightarrow{rw} T_j \xrightarrow{rw} T_k$, where $T_k$ commits before $T_j$.
}
If non-serializable scheduling occurs, consider three transactions: $T_2 \xrightarrow{D_1} T_1 \xrightarrow{D_2} T_0$. 
Without loss of generality, we assume $T_0$ is the first transaction committed.
Since $T_0$ commits first, $D_2$ must be an RW dependency; otherwise, $T_1$ should commit before $T_0$. 
Additionally, $T_1$ can not operate under RC because the concurrency control in \S\ref{design-1} would avoid the inconsistent dependency and commit order between $T_1$ and $T_0$. 
Moreover, $D_1$ can only be an RW edge; otherwise, $T_2$ commits before $T_0$ commits, as depicted in Figure~\ref{fig:switch_correctness}a, which contradicts the initial assumption that $T_0$ is the first to commit. 
\extended{Moreover, the data dependency from $T_2$ to $T_1$ can only be an RW edge. We prove this by reductio. If the dependency from $T_2$ to $T_1$ is either a WW or WR dependency, implying that $T_2$ commits before $T_1$ starts. Since $T_1$ is concurrent with $T_0$ due to an RW dependency, deriving $T_0$ commits after $T_1$ starts. Thus, $T_2$ must commit before $T_0$ commits, contradicting the assumption that $T_0$ is the first transaction to commit. Therefore, the data dependency from $T_2$ to $T_1$ must be an RW dependency, leading to two consecutive RW dependencies $T_2 \xrightarrow{rw} T_1 \xrightarrow{rw} T_0$, where $T_0$ commits first. 
}

Moreover, if $T_1$ operates under SI, the concurrency control in \S\ref{design-1} can detect the dependency from $T_1$ to $T_0$ and
enforce the consistent commit and dependency order. 
Hence, if non-serializable scheduling exists, \textbf{$T_1$ must operate under SER}. 

In other words, \textbf{the transition between RC and SI is serializable if they perform the concurrency control in \S\ref{design-1}}. 

% Furthermore, we can find at least a vulnerable dependency, where $T_j$ commits after the transition. 


% After employing the cross-isolation validation mechanism, $T_j$'s read set would be checked, and then the commit order would be enforced to be consistent with its dependency order.  

\blackding{2} We then prove that the existence of cross-isolation vulnerable dependency $T_j \xrightarrow{rw} T_k$, where $T_k$ commits before $T_j$ and $T_j$ commits after the transition. 
We demonstrate the proof under two cases as follows. 

% (1) Transaction $T_i$ in the cross-isolation vulnerable dependency $T_i \xrightarrow{rw} T_j$ operates under SER. If $T_i$ operates under SI or RC, then the concurrency control proposed in \S~\ref{design-1} can ensure that the commit order of $T_i$ and $T_j$ is consistent with the dependency order. Therefore, $T_i$ must operate under SER. 

% \noindent\textbf{Transition between SI and RC.} Consider the cross-isolation vulnerable dependency $T_i \xrightarrow{rw} T_j$. 

% (2) There must be at least one cross-isolation vulnerable dependency, $T_i \xrightarrow{rw} T_j$, where $T_i$ commits after the transition starts. Given that $T_i$ must operate under SER, we demonstrate the proof under two scenarios of the transition from SI/RC to SER and SER to SI/RC, respectively. 

\noindent\textbf{Transition from SI/RC to SER.} According to proof \blackding{1}, if there is non-serializable scheduling during the transition, there must be two consecutive RW dependencies, $T_i \xrightarrow{rw} T_j \xrightarrow{rw} T_k$, where $T_k$ commits before $T_j$ and $T_j$ operates under SER. During the transition from SI/RC to SER, $T_j$ operates under the new isolation level, making it commit after the transition starts. 
% \textit{As a result, the cross-isolation validation mechanism can detect the dependency and schedule the commit order to achieve serializable scheduling. }

\noindent\textbf{Transition from SER to SI/RC.} For clarity, we categorize the transactions during the transition into three discrete sets:
\maintext{
\begin{itemize}[leftmargin=*]
\item $S_{old}^{(1)}$: Transactions under $I_{old}$ committed before the transition.
\item $S_{old}^{(2)}$: Transactions under $I_{old}$ committed after the transition.
\item $S_{new}$: Transactions under $I_{new}$ starting after the transition.
\end{itemize}
}

\extended{
\begin{itemize}[leftmargin=*]
    \item $S_{old}^{(1)}$: The set of transactions under $I_{old}$ and have been committed when the transition occurs. 
    \item $S_{old}^{(2)}$: The set of transactions that operate under $I_{old}$ and commit after the transition occurs. 
    \item $S_{new}$: The set of transactions that start after the transition occurs and operate under $I_{new}$.
\end{itemize} 
}

Figure \ref{fig:switch_correctness}b shows the partial orders between transaction sets. 
Non-serializable scheduling implies a dependency cycle, 
\maintext{
either (a) between $S_{old}^{(2)}$ and $S_{new}$ or (b) spanning $S_{old}^{(1)}$, $S_{old}^{(2)}$, and $S_{new}$.
}
\extended{
which can be classified into two kinds: (a) scheduling involves only transactions in $S_{old}^{(2)}$ and $S_{new}$; (b) scheduling involves transactions in $S_{old}^{(1)}$, $S_{old}^{(2)}$ and $S_{new}$. 
}

In the first case, all transactions involving vulnerable dependencies are committed after the transition. According to proof \blackding{1}, if there is non-serializable scheduling during the transition, there must be two consecutive RW dependencies, $T_i \xrightarrow{rw} T_j \xrightarrow{rw} T_k$, and $T_j$ commits after the transition. 
In the second case, 
\maintext{
% as depicted in Figure \ref{fig:switch_correctness}b, 
there exists a transaction $T_j\in S_{old}^{(2)}$, its successor transaction is $T_k\in S_{old}^{(1)}$ and its predecessor transaction is either $T_i^{\prime}$ or $T_i$. Hence, there also exists a cross-isolation vulnerable dependency, where $T_k$ commits before $T_j$ and $T_j$ commits after the transition.
} 
\extended{we prove that there is at least one cross-isolation vulnerable dependency, $T_j\xrightarrow{rw} T_k$, where $T_j\in S_{old}^{(2)}$, in the non-serializable scheduling. We conclude that if there is a WR/WW data dependency from $T_i$ to $T_j$, $T_i$ must be committed before $T_i$ starts. Given that, we analyze the possible data dependencies between transaction sets.
The red arrow at the bottom shows that data dependencies from $S_{new}$ to $S_{old}^{(2)}$ can only be RW dependencies due to those transactions in $S_{new}$ commit after those in $S_{old}^{(2)}$.
The red dashed arrow within $S_{old}^{2}$ represents that dependencies between transactions within $S_{old}^{(2)}$ can only be RW dependencies because they are concurrent transactions. 
The red arrow in the top part shows that dependencies from transactions in $S_{old}^{(2)}$ to those in $S_{old}^{(1)}$ must be RW dependencies due to those transactions in $S_{old}^{(2)}$ commit after those in $S_{old}^{(1)}$ start. 
}
\extended{
Next, we use the reductio to prove that transaction $T_j$ in at least one vulnerable dependency, $T_j \xrightarrow{rw} T_k$, is not in the $S_{old}^{(1)}$ set.
If $T_j$ in all cross-isolation vulnerable dependencies, $T_j \xrightarrow{rw} T_k$, is in $S_{old}^{(1)}$, then any transaction $T_j$ in $S_{old}^{(2)}$ which is contained in a RW dependency pointing to $S_{old}^{(1)}$ must not have a precede RW dependency. However, due to that dependencies from transactions in $S_{new}$ (i.e., $T_i$) to $S_{old}^{(2)}$ (i.e., $T_i^\prime$) or dependencies from transactions in $S_{old}^{(2)}$ (i.e., $T_i^\prime$) to $S_{old}^{(2)}$ (i.e., $T_j$) must be RW dependencies, leading to contradiction. Therefore, at least one $T_j$ in cross-isolation vulnerable dependencies, $T_j \xrightarrow{rw} T_k$, is in $S_{old}^{(2)}$ or $S_{new}$. In other words, at least one $T_j$ commits after the transition starts.
}

% In conclusion, if there is a cross-isolation vulnerable dependency from $T_i$ to $T_j$ during the transition, the cross-isolation validation can ensure that their commit order is consistent with their dependency order, thus ensuring serializability. 
\blackding{3}
The cross-isolation validation mechanism can detect the vulnerable dependency $T_j \xrightarrow{rw} T_k$ if $T_j$ commits after the transition. It then enforces a consistent commit and dependency order. According to the contrapositive of proof \blackding{2}, if there does not exist cross-isolation vulnerable dependency $T_j \xrightarrow{rw} T_k$, where $T_k$ commits before $T_j$ and $T_j$ commits after the transition, then there is no non-serializable scheduling during the transition. As a result, the scheduling during the transition is serializable. 
% The approaches proposed in \S\ref{design-1} and \S\ref{design-3} can prevent all non-serializable scheduling. 
% if there is a cross-isolation vulnerable dependency from $T_i$ to $T_j$ during the transition, the cross-isolation validation can ensure that their commit order is consistent with their dependency order, thus ensuring serializability. 

% then any dependency $T_y \xrightarrow{rw} T_z$, where $T_y\in S_{old}^{(2)}$ and $T_z\in S_{old}^{(1)}$, the precede dependency can not be RW, which means $T_x \xrightarrow{ww/wr} T_y$, thus $T_x \in S_{old}^{(1)}$. In this case, a dependency cycle cannot contain transactions in all three transaction sets. Therefore, the counter-example does not exist; at least a precede RW dependency of $T_y$ must exist. Then, there is a structure with two consecutive RW dependencies, $T_x \xrightarrow{rw} T_y \xrightarrow{rw} T_z$, where $T_z \in S_{old}^{(1)}$ commits before $T_y \in S_{old}^{(2)}$ and $T_y$ commits after the transition starts. Thus, $T_y$ and $T_z$ constitute a cross-isolation level vulnerable dependency.

% Therefore, we conclude that for transitions from SER to SI/RC or from SI/RC to SER, if there is non-serializable scheduling, there is at least one cross-isolation vulnerable dependency, $T_j \xrightarrow{rw} T_k$, in two consecutive RW dependencies, transaction $T_j$ operates under the SER and commits after the transition starts. 
}


\subsection{Failure Recovery \label{sec:proof:failure}}
%{\color{blue}
% Failure recovery aims to recover the database to a consistent state, ensuring that no partial or corrupted data is present.
The system incorporates a robust failure recovery mechanism to ensure data consistency and service availability. When \sysname encounters a failure, the system automatically restarts \sysname to re-connect the RDBMS and rolls back all uncommitted transactions. When the RDBMS encounters failures, the system restarts the RDBMS and leverages the ARIES recovery algorithm~\cite{DBLP:journals/tods/MohanHLPS92:ARIES} to recover the database in a consistent state. 
% its recovery can rely on its own recovery techniques. 
% Moreover, applications can verify whether their modifications have been applied after recovery if they do not receive a response to the commit request.
% does not either modify the database kernel or keep any runtime transaction meta. As a result, there is no requirement to recover it to any consistent point. \sysname guarantees that any transaction passing the validation maintains the SER. Hence, it relies on the underlying RDBMS for failure recovery in the event of a failure. Moreover, applications can verify whether their modifications have been applied after recovery if they do not receive a response to the commit request. 
%}
% Due to the space limitation, we provide more details in our technique report \cite{TxnSails}. 

% \sysname identifies the static vulnerable dependency from the transaction templates. 

% To guarantee SS during the transition when involving SER, our CIV records all transactions $T_j$ that involve RW at the SER isolation level. CIV prevents $T_i \xrightarrow{rw} T_j \xrightarrow{rw} T_k$ by enforcing a commit order that aligns with its dependency order. This approach ensures that $T_k$ does not commit before $T_j$, thereby avoiding the formation of cycles.

% \textbf{Transaction $T_j$ operates under SER.} Otherwise, the middle-tier concurrency control approach would ensure that the commit order of $T_j$ and $T_k$ is consistent with their dependency order. 

% \textbf{Transaction $T_j$ commits after the transition point.} In scenarios where the transition is from SI or RC to SER, since $T_j$ operates under SER, it starts and commits after the transition point.
% For the other two scenarios, we illustrate the partial orders between transaction sets, according to the categories outlined in Section \ref{design-3}, at the transition point in Figure \ref{fig:switch_correctness}b. The features of some dependencies are listed following: 
% % Dependencies arise from , between transactions in $S_{new}$ and those in $S_{old}^{(1)}$ and $S_{old}^{(2)}$. 
% \blackding{1} Dependencies from transactions in $S_{old}^{(1)}$ to those in $S_{new}$ can only be RW dependencies and no reverse dependencies because transactions in $S_{old}^{(1)}$ commit before any transactions in $S_{new}$ start. 
% \blackding{2} Transactions in $S_{new}$ are concurrent with those in $S_{old}^{(2)}$, because transactions in $S_{old}^{(2)}$ operate under SER, the dependencies from transactions in $S_{new}$ to those in $S_{old}^{(2)}$ are RW dependencies. The reverse dependencies can be any type.
% \blackding{3} Similarly, the dependencies between transactions within $S_{old}^{(2)}$ can only be RW dependencies because they are concurrent transactions.
% \blackding{4} Transactions in $S_{old}^{(1)}$ commits before those in $S_{old}^{(2)}$, thus dependencies from transactions in $S_{old}^{(1)}$ .

% There can be two kinds of dependency cycles: dependency cycles contain transactions in $S_{old}^{(1)}$ and $S_{new}$, and dependency cycles contain transactions in all three sets. In the first type, $T_j$ must commit after the transition according to the classification criteria of $S_{old}^{(2)}$. In another type, although $T_j$ may commit before the transition point, we prove that every dependency cycle contains two consecutive RW dependencies, $T_i^{'} \xrightarrow{rw} T_j^{'} \xrightarrow{rw} T_k^{'}$, and $T_j^{'}$ commits after the transition point. 

% As shown in Figure \ref{fig:switch_correctness}, the dependency cycle contains the dependency from transaction in $S_{new}$ to $S_{old}^{(2)}$ and the dependency from transaction in $S_{old}^{(2)}$ to $S_{old}^{(1)}$. Without losing generality, we denote them as $T_1\xrightarrow{rw} T_2$ and $T_3\xrightarrow{rw} T_4$, where $T_2$ and $T_3$ can be the same transaction. The dependency chain from $T_2$ to $T_4$ can be formalize as:

% \todo{}
% Thus far, we have proved the correctness of Theroem \ref{the:cross-isolation} and Theroem \ref{the:cross-isolation-commit}. The cross-isolation level validation mechanism ensures the commit order of xxx is consistent with their dependency order, thus ensuring SS during the transition.
% If the dependency cycle does not involve the transactions in $S_{old}^{(1)}$, $T_j$ must commit after the transition according the  $S_{old}^{(2)}$. In another case, 

% Initially, the dependencies between $S_{old}^{(2)}$ and $S_{new}$ can be any types and may be bidirectional. 
% In phase~II, \sysname is required to track these dependencies and ensure that there are no dependencies from the transactions in $S_{new}$ to those in $S_{old}^{(2)}$.

% which involves transactions that run across isolation levels and can be divided into two distinct parts: (1) the isolation transitions involving SER isolation and (2) the transitions occurring solely between SI and RC levels. 
% \newpage
% \textbf{Transaction involves SER isolation level.} Additional validation is necessary during the transition when SER isolation is involved. As outlined in Section~\ref{design-3}, we categorize transactions into three sets: $S_{old}^{(1)}$, $S_{old}^{(2)}$, and $S_{new}$. The concurrency control in Section~\ref{sec:design:cc:validation} ensures that there is no dependency cycle within $S_{old}^{(1)} \cup S_{old}^{(2)}$ and $S_{new}$, respectively.
% As depicted in Figure~\ref{fig:switch_correctness}, dependencies arise across isolation levels, between transactions in $S_{new}$ and those in $S_{old}^{(1)}$ and $S_{old}^{(2)}$. 
% \blackding{1} Transactions in $S_{old}^{(1)}$ commit before any transactions in $S_{new}$ start. Therefore, dependencies can only occur when a transaction in $S_{new}$ reads or writes the data written by $S_{old}^{(1)}$. 
% \blackding{2} Initially, the dependencies between $S_{old}^{(2)}$ and $S_{new}$ can be any types and may be bidirectional. 
% In phase~II, \sysname is required to track these dependencies and ensure that there are no dependencies from the transactions in $S_{new}$ to those in $S_{old}^{(2)}$. 
% % Thus, the dependencies can be only flow from $S_{old}^{(2)}$ to $S_{new}$.
% Therefore, the dependency graph is acyclic during the transition, thus achieving SS.
% cross-isolation transactions during the transition are serializable.

% \textbf{Transition between SI and RC isolation levels.} Unlike involving SER, the transition of \sysname between SI and RC isolation levels does not require additional validation. To illustrate this, we consider the counter-example in Figure~\ref{fig:cross-isolation} and briefly demonstrate that no cycle can be formed during the transition. By contradiction, we assume a cycle exists, with $T_j$ being the first transaction to commit. This implies the existence of an RW edge from $T_i$ to $T_j$, as $T_j$ commits first. However, by identifying the static vulnerable dependency at RC, the middle-tier concurrency control approach ensures $T_i$ commits before $T_j$ for any RW dependency, meaning that $T_i$ operates under SI. 
% By identifying the static vulnerable dependency in SI, it is not permissible for two consecutive RW dependencies to have the last transaction committed first. So, the dependency from $T_k$ to $T_i$ must be either WW or WR. 
% In this case, the commit time of $T_k$ is before $T_i$ starts. Since $T_i$ is concurrent with $T_j$ by a RW dependency, $T_j$ commits after $T_i$ starts. Thus, $T_k$ must commit before $T_j$ commits, which contradicts the assumption that $T_k$ is the first transaction to commit in the cycle. Therefore, we conclude that no cycles exist during the transition.

% The transition process ensures serializability when transactions follow the middle-tier concurrency control designed for each individual isolation level in \sysname. As we all know, non-serializable scheduling can be identified by a dependency cycle. Consider the counter-example in Figure~\ref{fig:cross-isolation}. Denote the first committed transaction in the cycle as $T_j$, with its predecessor transactions as $T_i$ and $T_j$. The inconsistent dependency order and commit order between them indicate the dependency from $T_i$ to $T_j$ must be an RW dependency. This implies $T_i$ operates under SI and is concurrent with $T_j$, meaning \textit{$T_i$ begins before $T_j$ commits}. Then, the dependency from $T_k$ to $T_i$ could be either WW or WR dependency. If it is an RW dependency, two consecutive RW dependencies occur, consisting of the dangerous structure in SI; then, the middle-tier concurrency control could schedule the commit order of $T_i$ and $T_j$ to make it consistent with the dependency order. Since $T_i$ operates under SI, $T_k$ must commit before $T_j$ begins, which contradicts our assumption that $T_j$ is the first transaction to commit in this cycle. Therefore, no such dependency cycles exist, ensuring that transaction scheduling is serializable under \sysname.

% \sysname employs a stringent concurrency control mechanism during isolation level transitions. 
% The correctness of this mechanism can be divided into two parts: the transition involving SER and the transition between SI and RC. We demonstrate the correctness of both scenarios separately.

% \subsubsection{Transaction involving SER isolation level.}
% As detailed in Section~\ref{design-3}, this transition process comprises three distinct stages, for which we provide proof of serializability. 
% The serializability of Phase I, which encompasses all transactions operating under the previous isolation level, as well as Phase III, involving all transactions under the new isolation level, can be readily proven as demonstrated in Section~\ref{sec:proof.isolation}.
% We will provide proof of serializability for Phase II, which involves transactions that run across both isolation levels.

% According to Section~\ref{design-3}, we categorize the transactions into three sets: $S_{old}^{(1)}$, $S_{old}^{(2)}$, and $S_{new}$. As depicted in Figure~\ref{fig:switch_correctness}, we will demonstrate that these transactions' schedules are serializable, without conflict cycles. Within the same isolation level, consider the proof in Section \ref{sec:proof.isolation}, we can establish that both $S_{old}^{(1)} \cup S_{old}^{(2)}$ and $S_{new}$ are serializable. 
% When dependencies are across isolation levels and transactions are from between $S_{new}$ and $S_{old}^{(1)}$, as well as between $S_{new}$ and $S_{old}^{(2)}$.
% \blackding{1} Transactions in $S_{old}^{(1)}$ commit before transactions in $S_{new}$ start. Therefore, dependencies can only occur when a transaction in $S_{new}$ reads or writes the data written by $S_{old}^{(1)}$.
% \blackding{2} Initially, the dependencies between $S_{old}^{(2)}$ and $S_{new}$ can be any types and bidirected. In phase~II, \sysname could track these dependencies and ensure that transactions in $S_{new}$ either commit after $S_{old}^{(1)}$ or abort. 
% % no operation in $S_{old}^{(1)}$ depends on an operation in $S_{new}$.
% Thus, the dependencies can be only from $S_{old}^{(2)}$ to $S_{new}$. Consequently, cross-isolation transactions during the switching phase are serializable.
% For example, for the switch from SER to SI, the $CG$ of $H$ is divided into two subgraphs: $CG_{SER}$ and $CG_{SI}$. $CG_{SER}$ contains transactions operating under the SER level, and $CG_{SI}$ contains those operating under the SI level. It is evident that $CG_{SER}$ is acyclic due to the stringent nature of SER. Furthermore, according to Theorem~\ref{theorem-2}, $CG_{SI}$ is also acyclic.
% We then examine the dependency edges that bridge $CG_{SER}$ and $CG_{SI}$.

% In \sysname, \textcolor{red}{Describe more information when the database concurrently processes transactions under both SI and SER, how to deal with transaction conflict to guarantee serializability.} the concurrency control for all transactions adheres to SER protocol or Theorem \ref{theorem-2}. This approach ensures that no cycles form between $CG_{SER}$ and $CG_{SI}$, maintaining the acyclic nature of the overall system's conflict graph and thereby preserving conflict serializability throughout the transition.

% \subsubsection{Transition between SI and RC isolation level.} The transition process ensures serializability when transactions follow the middle-tier concurrency control designed for each individual isolation level in \sysname. As we all know, non-serializable scheduling can be identified by a dependency cycle. Consider the counter-example in Figure~\ref{fig:cross-isolation}. Denote the first committed transaction in the cycle as $T_j$, with its predecessor transactions as $T_i$ and $T_j$. The inconsistent dependency order and commit order between them indicate the dependency from $T_i$ to $T_j$ must be an RW dependency. This implies $T_i$ operates under SI and is concurrent with $T_j$, meaning \textit{$T_i$ begins before $T_j$ commits}. Then, the dependency from $T_k$ to $T_i$ could be either WW or WR dependency. If it is an RW dependency, two consecutive RW dependencies occur, consisting of the dangerous structure in SI; then, the middle-tier concurrency control could schedule the commit order of $T_i$ and $T_j$ to make it consistent with the dependency order. Since $T_i$ operates under SI, $T_k$ must commit before $T_j$ begins, which contradicts our assumption that $T_j$ is the first transaction to commit in this cycle. Therefore, no such dependency cycles exist, ensuring that transaction scheduling is serializable under \sysname.


\section{Implementation Environment}
\label{sec:implementation_environment}

Here we introduce the detailed implementation details and environment for reproducibility purpose. For our model, we choose hyperparameters based on the performance on validation set (Document classification task in the main paper explains how we split validation set). The results in the main paper are obtain by 5 independent runs. The standard deviations reported in the main paper are 1-sigma error bars and are obtained by calling its corresponding function in Excel library. All the experiments were done on Linux server with an NVIDIA A40 GPU with 46,068 MiB. Its operating system is CentOS Linux 7 (Core). We implemented our proposed model GTFormer using Python 3.10 as programming language and PyTorch 2.0.0 as deep learning library. Other frameworks include NumPy 1.23.1, sklearn 0.23.2, and scipy 1.5.2. We emphasize that the main focus of our model is effectiveness, instead of running efficiency. But for completeness, we still make a short comment on execution time. Our model is efficient, on the largest dataset Web, the training takes less than 40 hours to converge. We will release code and datasets upon publication.

\section{Evaluation}
We provide three sets of insights into this section, organised as \textit{findings (F*)}. We quantitatively study the effect of the adversarial and counterfactual perturbations on the performance of informal reasoners and autoformalisation methods. Then, we dive deeper into method variants. Finally, 
we analyse the nature of formalisation errors made by the models.

\subsection{Robustness Analysis}
\paragraph{\textbf{\emph{F1: Noise perturbations have a stronger effect on formalisation methods than informal \ac{LLM} reasoners.}}}
Table~\ref{tab:distraction_k4_formalisation} shows that, on average, the accuracy of both direct and \ac{CoT} informal reasoning remains between $73\%$ and $74\%$ in the face of added noise. While the autoformalisation method performs similarly to informal reasoners on the original dataset, its performance decreases between $4\%$ and $11\%$. The accuracy drops especially with logical (L) and tautological (T) distractions, whose logical language formats trick the \ac{LLM} into formalizing the noisy clauses. On the other hand, the linguistically complex and more natural sentences of encyclopedic distractions show a minor effect, suggesting that \acp{LLM} successfully avoids formalizing the more complicated sentences.

\paragraph{\textbf{\emph{F2: All \ac{LLM}-based reasoning methods suffer a drop for counterfactual perturbations.}}} % influence .}}}
Table~\ref{tab:distraction_k4_formalisation} shows that counterfactual statements cause a significant decrease in performance for both the informal reasoners and autoformalisation methods of between $12\%$ and $13\%$ on average. 
Moreover, this observation also holds for all tested models, i.e., none are robust towards counterfactual perturbations across every evaluated dimension. Even the strongest model, GPT 4o-mini, yields a performance of 63-68\%, which is relatively close to the random performance of 50\%. The high impact of counterfactual statements (the single ``not'' inserted) could be due to the inability of \acp{LLM} to overwrite prior knowledge with explicitly stated information or memorization of the answers. We study the error sources further in §\ref{subsec:errors}.  

\noindent \paragraph{\textbf{\emph{F3: Introducing multiple noise sentences has an effect only for logical distractions.}}}
We show the impact of introducing between one and four sentences for the two top-performing autoformalisation models in Figure~\ref{fig:length_distraction}. The figure shows similar trends with and without counterfactual perturbations.
As additional logical distractions are introduced, the model performance consistently decreases. Tautological (T) distractions lead to a decline in accuracy with a single disruptive sentence, yet adding more noise does not worsen the outcome. 
The tautological corpus introduces truth constants for all sentences as a persistent unseen logical construct. Given that this leads only to a decrease for a single occurrence, we can assume that a model can consistently handle the same unseen logical construct. In contrast, the logical corpus increases the chance of adding text, requiring new, previously unseen reasoning constructs for each added sentence. The impact of encyclopedic noise remains negligible, generalising F1 to $k$ sentences. Similarly, counterfactual perturbations remain much more effective for all settings, generalising F2.

\begin{table}[!t]
\small
\setlength{\modelspacing}{2pt}
\setlength{\tabcolsep}{1.7pt} % Default value: 6pt
\setlength{\belowrulesep}{4pt}
\begin{threeparttable}
    \centering
    \begin{tabular}{cc l r rrr @{\quad} rrrr}
\toprule
\multirow{2}{*}{} & \multirow{2}{*}{} & Reasoning & \multirow{2}{*}{O} & \multicolumn{3}{c}{Distraction} & \multicolumn{4}{c}{Counterfactual} \\
 & & Format & & E& L & T & $\text{O}_C$ & $\text{E}_C$& $\text{L}_C$ & $\text{T}_C$\\
\midrule
\multirow{6}{*}{\rotatebox{90}{Gemma-2}} & \multirow{3}{*}{\rotatebox{90}{9b}}
   & Informal (direct) & \textbf{0.78} & \textbf{0.80} & \textbf{0.79} & \textbf{0.77} & 0.58 & 0.52 & 0.50 & 0.59 \\
 & & Informal (CoT) & 0.72 & 0.78 & 0.73 & 0.76 & 0.61 & \textbf{0.57} & \textbf{0.60} & \textbf{0.66} \\
 & & Formal (FOL) & 0.62 & 0.58 & 0.52 & 0.53 & \textbf{0.63} & 0.52 & 0.46 & 0.46 \\[\modelspacing]
\cmidrule{2-11}
 & \multirow{3}{*}{\rotatebox{90}{27b}} 
   & Informal (direct) & 0.71 & 0.69 & \textbf{0.66} & \textbf{0.68} & 0.59 & 0.51 & 0.54 & 0.59 \\
 & & Informal (CoT) & 0.66 & 0.65 & 0.64 & 0.63 & 0.62 & 0.58 & \textbf{0.62} & \textbf{0.64} \\
 & & Formal (FOL) & \textbf{0.74} & \textbf{0.74} & 0.61 & 0.61 & \underline{\textbf{0.72}} & \underline{\textbf{0.67}} & 0.58 & 0.51 \\[\modelspacing]
\midrule
\multirow{6}{*}{\rotatebox{90}{Mistral}} & \multirow{3}{*}{\rotatebox{90}{7B}} 
   & Informal (direct) & 0.77 & \textbf{0.77} & 0.75 & \textbf{0.79} & \textbf{0.63} & \textbf{0.54} & \textbf{0.54} & \textbf{0.66} \\
 & & Informal (CoT) & \textbf{0.79} & 0.75 & \textbf{0.77} & 0.78 & 0.55 & 0.52 & \textbf{0.54} & 0.58 \\
 & & Formal (FOL) & 0.62 & 0.58 & 0.54 & 0.57 & 0.50 & \textbf{0.54} & 0.51 & 0.52 \\[\modelspacing]
\cmidrule{2-11}
 & \multirow{3}{*}{\rotatebox{90}{Small}} 
   & Informal (direct) & \textbf{0.77} & \textbf{0.76} & \textbf{0.76} & \textbf{0.75} & 0.61 & 0.51 & 0.56 & 0.59 \\
 & & Informal (CoT) & 0.72 & 0.72 & 0.72 & 0.71 & \textbf{0.62} & \textbf{0.59} & \textbf{0.62} & \textbf{0.68} \\
 & & Formal (FOL) & 0.68 & 0.59 & 0.53 & 0.64 & 0.54 & 0.55 & 0.49 & 0.51 \\[\modelspacing]
\midrule
\multirow{6}{*}{\rotatebox{90}{Llama-3.1}} & \multirow{3}{*}{\rotatebox{90}{8B}} 
   & Informal (direct) & 0.63 & 0.61 & 0.64 & 0.66 & 0.61 & \textbf{0.62} & 0.59 & 0.61 \\
 & & Informal (CoT) & 0.73 & \textbf{0.73} & \textbf{0.71} & \textbf{0.72} & \textbf{0.62} & 0.59 & \textbf{0.61} & \textbf{0.65} \\
 & & Formal (FOL) & \textbf{0.77} & 0.71 & 0.63 & 0.52 & 0.60 & 0.58 & 0.55 & 0.52 \\[\modelspacing]
\cmidrule{2-11}
 & \multirow{3}{*}{\rotatebox{90}{70B}} 
   & Informal (direct) & 0.77 & 0.74 & 0.74 & 0.73 & 0.62 & 0.53 & 0.56 & 0.64 \\
 & & Informal (CoT) & \textbf{0.78} & \textbf{0.75} & \textbf{0.76} & \textbf{0.76} & 0.64 & 0.61 & \textbf{0.66} & \underline{\textbf{0.73}} \\
 & & Formal (FOL) & 0.74 & 0.73 & 0.71 & 0.71 & \textbf{0.66} & \textbf{0.62} & 0.59 & 0.57 \\[\modelspacing]
 \midrule
\multirow{3}{*}{\rotatebox{90}{GPT}} & \multirow{3}{*}{\rotatebox{90}{4o-mini}} 
   & Informal (direct) & 0.78 & 0.77 & 0.79 & 0.79 & 0.64 & 0.61 & 0.61 & 0.63 \\
 & & Informal (CoT) & 0.80 & 0.80 & \underline{\textbf{0.81}} & \underline{\textbf{0.82}} & \textbf{0.68} & \textbf{0.63} & \underline{\textbf{0.68}} & \textbf{0.64} \\
 & & Formal (FOL) & \underline{\textbf{0.84}} & \underline{\textbf{0.82}} & 0.73 & 0.79 & 0.63 & 0.62 & 0.57 & 0.54 \\[\modelspacing]
 \midrule
\multicolumn{2}{c}{\multirow{3}{*}{\textbf{Avg}}} 
 & Informal (direct) & 0.74 & 0.73 & 0.73 & 0.73 & 0.61 & 0.55 & 0.56 & 0.62 \\
 & & Informal (CoT) & 0.74 & 0.74 & 0.73 & 0.74 & 0.62 & 0.58 & 0.62 & 0.65 \\
  & & Formal (FOL) & 0.72 & 0.68 &	0.61 & 0.62 & 0.61 & 0.59 & 0.54 & 0.52 \\
\bottomrule
\end{tabular}
\caption{Accuracies of informal and autoformalisation-based deductive reasoners. The best overall model per dataset is underlined; the best model version is marked in bold.}
\label{tab:distraction_k4_formalisation}
\end{threeparttable}
\end{table} 

\begin{figure}[!t]
    \centering
    \scriptsize
    \begin{tikzpicture}
        \begin{axis}[name=gpt,
            title={GPT-4o-mini},
            width=0.6\linewidth,
            height=0.6\linewidth,
            xlabel={\# Noise sentences},
            ylabel={Accuracy},
            xmin=-0.1, xmax=4.1,
            ymin=0.5, ymax=0.9,
            xtick={1,2,4},
            ytick={0.55, 0.6, 0.65, 0.75, 0.8, 0.85},
            title style={yshift=-0.6em},
            legend style={at={(1,-0.15)},
	           anchor=north,legend columns=-1},
            x label style={at={(axis description cs:1,-0.05)},anchor=north},
            y label style={at={(axis description cs:-0.15,0.5)},anchor=south},
            ymajorgrids=true,
            grid style=dashed,
        ]
            \addplot[color=blue, mark=square,]
                coordinates {
                (0,0.848076939582825)(1,0.823076903820038)(2,0.826923072338104)(4,0.821153819561005)
                };
            \addplot[color=red, mark=triangle,]
                coordinates {
                (0,0.848076939582825)(1,0.817307710647583)(2,0.801923096179962)(4,0.759615361690521)
                };
            \addplot[color=green, mark=diamond,] 
                coordinates {
                (0,0.848076939582825)(1,0.767307698726654)(2,0.769230782985687)(4,0.803846180438995)
                };
            \addplot[color=blue, mark=square*] 
                coordinates {
                (0,0.627777755260468)(1,0.622222244739533)(2,0.600000023841858)(4,0.633333325386047)
                };
            \addplot[color=red, mark=triangle*,] 
                coordinates {
                (0,0.627777755260468)(1,0.611111104488373)(2,0.611111104488373)(4,0.594444453716278)
                };
            \addplot[color=green, mark=diamond*,] 
                coordinates {
                (0,0.627777755260468)(1,0.572222232818604)(2,0.538888871669769)(4,0.555555582046509)
                };
                \legend{E,L,T,$\text{E}_C$, $\text{L}_C$ , $\text{T}_C$}
        \end{axis}

        \begin{axis}[name=llama, at={($(gpt.east)+(0.1cm,0)$)},anchor=west,
            title={Llama 3.1 70b},
            width=0.6\linewidth,
            height=0.6\linewidth,
            xmin=-0.1,, xmax=4.1,
            ymin=0.5, ymax=0.9,
            xtick={1,2,4},
            ytick={0.55, 0.6, 0.65, 0.75, 0.8, 0.85},
            title style={yshift=-0.6em},
            yticklabel=\empty,
            ymajorgrids=true,
            grid style=dashed,
        ]
            \addplot[color=blue, mark=square,]
                coordinates {
                (0,0.838461518287659)(1,0.817307710647583)(2,0.805769205093384)(4,0.817307710647583)
                };
            \addplot[color=red, mark=triangle,]
                coordinates {
                (0,0.838461518287659)(1,0.819230794906616)(2,0.803846180438995)(4,0.771153867244721)
                };
            \addplot[color=green, mark=diamond,]
                coordinates {
                (0,0.838461518287659)(1,0.803846180438995)(2,0.807692289352417)(4,0.805769205093384)
                };
            \addplot[color=blue, mark=square*]
                coordinates {
                (0,0.627777755260468)(1,0.622222244739533)(2,0.577777802944183)(4,0.594444453716278)
                };
            \addplot[color=red, mark=triangle*,]
                coordinates {
                (0,0.627777755260468)(1,0.583333313465118)(2,0.561111092567444)(4,0.577777802944183)
                };
            \addplot[color=green, mark=diamond*,]
                coordinates {
                (0,0.627777755260468)(1,0.627777755260468)(2,0.566666662693024)(4,0.577777802944183)
                };
        \end{axis}
    \end{tikzpicture}
    \caption{Influence of the number of noisy sentences for FOL.}
    \label{fig:length_distraction}
\end{figure}



\subsection{Impact of Method Design}
\paragraph{\textbf{\emph{F4: \ac{CoT} prompting is most impactful when both noise and counterfactual perturbations are applied.}}}
The accuracies for the individual \acp{LLM} in Table~\ref{tab:distraction_k4_formalisation} show that the impact of \ac{CoT} is negligible for noise-only datasets (first four columns). Meanwhile, the benefit from \ac{CoT} is most pronounced in the datasets that combine noise and counterfactual perturbations.
The better-performing informal prompting strategy for a model remains stable for all types of distractions. Still, the decline in performance due to counterfactuals leads to a less consistent preference for a specific prompting style.

\paragraph{\textbf{\emph{F5: The best-performing grammar differs per model and is unstable across data versions.}}}

The evaluation of different logical forms for formal \ac{LLM}-based reasoning in Table~\ref{tab:distraction_k4_logical_form} shows the preference of some models for specific syntactic formats.
Llama 3.1 70B has a considerable improvement of $12\%$ with TPTP syntax on the original set, while Llama 3.1 8B benefits from the R-FOL syntax. However, all grammars show a declining accuracy trend and increased syntax errors for noise perturbations, where the best grammar loses its advantage over the rest. 
When comparing the grammars on the counterfactual partitions, we observe that TPTP is consistently more robust than the standard first-order logic grammar. Here, GPT 4o-mini shows a reduction from $O$ to $O_C$ of $20\%$ for FOL and only $12\%$ for the TPTP grammar. Since this does not correlate with fewer syntax errors, the formalisation in TPTP prevents semantical errors for counterfactual premises. 
A positive reading of these results, especially the minor differences between FOL and R-FOL, is that autoformalisation \acp{LLM} can adapt to the grammar syntax prescribed in the prompt without further loss in performance.

\begin{table}[!t]
\small
\setlength{\modelspacing}{2pt}
\setlength{\tabcolsep}{1.7pt} % Default value: 6pt
\setlength{\belowrulesep}{4pt}
\begin{threeparttable}
    \centering
    \begin{tabular}{cc l r rrr @{\quad} rrrr}
\toprule
\multirow{2}{*}{} & \multirow{2}{*}{} & Grammar & \multirow{2}{*}{O} & \multicolumn{3}{c}{Distraction} & \multicolumn{4}{c}{Counterfactual} \\
 & & Syntax & & E& L & T & $\text{O}_C$ & $\text{E}_C$& $\text{L}_C$ & $\text{T}_C$\\
\midrule
\multirow{6}{*}{\rotatebox{90}{Llama-3.1}} & \multirow{3}{*}{\rotatebox{90}{8B}} 
   & FOL & 0.77 & \textbf{0.71} & 0.61 & \textbf{0.53} & 0.58 & \textbf{0.55} & 0.52 & \textbf{0.56} \\
 & & R-FOL & \textbf{0.78} & 0.69 & \textbf{0.62} & \textbf{0.53} & 0.58 & \textbf{0.55} & \textbf{0.54} & 0.52 \\
 & & TPTP & 0.73 & 0.67 & 0.55 & 0.51 & \textbf{0.68} & 0.54 & 0.46 & 0.51 \\[\modelspacing]
\cmidrule{2-11}
 & \multirow{3}{*}{\rotatebox{90}{70B}} 
   & FOL & 0.76 & 0.73 & 0.71 & \textbf{0.72} & 0.67 & 0.57 & 0.63 & 0.56 \\
 & & R-FOL & 0.76 & 0.73 & 0.67 & 0.71 & 0.64 & 0.57 & 0.53 & 0.64 \\
 & & TPTP & \underline{\textbf{0.88}} & \underline{\textbf{0.84}} & \underline{\textbf{0.81}} & \textbf{0.72} & \underline{\textbf{0.81}} & \underline{\textbf{0.68}} & \underline{\textbf{0.67}} & \underline{\textbf{0.68}} \\[\modelspacing]
\midrule
\multirow{3}{*}{\rotatebox{90}{GPT}} & \multirow{3}{*}{\rotatebox{90}{4o-mini}} 
   & FOL & \textbf{0.84} & \textbf{0.82} & \textbf{0.72} & \underline{\textbf{0.78}} & 0.64 & \textbf{0.63} & \textbf{0.61} & 0.51 \\
 & & R-FOL & \textbf{0.84} & 0.77 & 0.70 & \underline{\textbf{0.78}} & \textbf{0.72} & 0.56 & 0.54 & \textbf{0.63} \\
 & & TPTP & 0.83 & \textbf{0.82} & 0.71 & 0.71 & 0.69 & \textbf{0.63} & 0.57 & 0.57 \\
\bottomrule
\end{tabular}
\caption{Accuracies of different formalisation grammars for autoformalisation.}
\label{tab:distraction_k4_logical_form}
\end{threeparttable}
\end{table} 

\paragraph{\textbf{\emph{F6: Feedback does not help \acp{LLM} self-correct to mitigate robustness issues.}}}
\autoref{tab:distraction_k4_feedback} shows the results with different error recovery mechanisms. The results indicate that no feedback strategy emerges as a winner in the different datasets. 
All feedback variants reduce syntax errors for noise perturbations, but given the lack of a consistent increase in accuracy, the corrected formalisations are most likely to contain semantic errors still. 
The type of feedback message only has a minor influence on correcting syntax errors, whereas Llama 3.1 70b and GPT 4o-mini correct slightly more syntax errors with specific error messages. This finding aligns with \cite{huang2023large}, who also found that \acp{LLM} cannot consistently self-correct their reasoning after receiving relevant feedback.

\begin{table}[!ht]
\small
\setlength{\modelspacing}{2pt}
\setlength{\tabcolsep}{1.7pt} % Default value: 6pt
\setlength{\belowrulesep}{4pt}
\begin{threeparttable}
    \centering
    \begin{tabular}{cc l r rrr @{\quad} rrrr}
\toprule
\multirow{2}{*}{} & \multirow{2}{*}{} & \multirow{2}{*}{Feedback} & \multirow{2}{*}{O} & \multicolumn{3}{c}{Distraction} & \multicolumn{4}{c}{Counterfactual} \\
 & & & & E& L & T & $\text{O}_C$ & $\text{E}_C$& $\text{L}_C$ & $\text{T}_C$\\
\midrule
\multirow{8}{*}{\rotatebox{90}{Llama-3.1}} & \multirow{4}{*}{\rotatebox{90}{8B}} 
   & No recovery & 0.77 & \textbf{0.72} & 0.62 & 0.53 & 0.59 & 0.58 & 0.56 & \textbf{0.56} \\
 & & Error type & \textbf{0.79} & 0.71 & 0.63 & \textbf{0.56} & \textbf{0.66} & 0.54 & 0.52 & 0.51 \\
 & & Error message & 0.78 & 0.71 & \textbf{0.67} & 0.55 & 0.59 & 0.53 & \underline{\textbf{0.64}} & 0.49 \\
 & & Warning & 0.74 & 0.66 & 0.58 & 0.55 & 0.55 & \textbf{0.60} & 0.49 & 0.49 \\[\modelspacing]
\cmidrule{2-11}
 & \multirow{4}{*}{\rotatebox{90}{70B}} 
   & No recovery & \textbf{0.77} & \textbf{0.72} & \textbf{0.73} & 0.71 & \textbf{0.64} & 0.59 & \textbf{0.61} & 0.56 \\
 & & Error type & 0.72 & 0.70 & 0.72 & \textbf{0.73} & 0.62 & 0.56 & 0.60 & 0.58 \\
 & & Error message & 0.71 & 0.70 & \textbf{0.73} & 0.71 & \textbf{0.64} & 0.59 & 0.54 & \underline{\textbf{0.64}} \\
 & & Warning & 0.69 & \textbf{0.72} & 0.72 & 0.72 & 0.62 & \underline{\textbf{0.65}} & \textbf{0.61} & 0.63 \\[\modelspacing]
\midrule
\multirow{4}{*}{\rotatebox{90}{GPT}} & \multirow{4}{*}{\rotatebox{90}{4o-mini}} 
   & No recovery & \underline{\textbf{0.84}} & \underline{\textbf{0.82}} & 0.73 & 0.79 & 0.64 & \textbf{0.62} & 0.56 & \textbf{0.56} \\
 & & Error type & 0.83 & 0.79 & 0.74 & 0.76 & 0.67 & 0.57 & 0.56 & \textbf{0.56} \\
 & & Error message & \underline{\textbf{0.84}} & 0.78 & \underline{\textbf{0.77}} & \underline{\textbf{0.80}} & 0.62 & 0.59 & 0.56 & \textbf{0.56} \\
 & & Warning & \underline{\textbf{0.84}} & 0.75 & 0.73 & 0.76 & \underline{\textbf{0.70}} & 0.61 & \textbf{0.61} & 0.55 \\
 \bottomrule
\end{tabular}
\caption{Accuracies of error recovery strategies.}
\label{tab:distraction_k4_feedback}
\end{threeparttable}
\end{table} 

\subsection{Error Analysis}
\label{subsec:errors}
\paragraph{\textbf{\emph{F7: Autoformalisation increases syntax errors for noise perturbations.}}}
The low performance for noise perturbations correlates with more syntax errors for all models and distraction categories (cf. execution rates in Table~\ref{tab:appendix_k4_formalisation_exec}). The three worst-performing models (both Mistral models, Gemma-2 9b) generate, at best, for $37\%$  and, at worst, for only $4\%$ of the samples, a valid logical form.
Gemma-2 9b and Llama3.1 8b produce more syntax errors than the larger counterparts, suggesting that larger models are more robust towards noise perturbations. 
The accuracy of syntactically valid samples is higher than the informal reasoning methods for most distractions (Table~\ref{tab:appendix_k4_formalisation_vacc}), motivating informal reasoning as a backup strategy for formal reasoning. The error message feedback reveals two common syntax errors: 1) errors by models with an initial low execution rate exhibit issues with the template structure, including using incorrect keywords or adding conversational phrases;
2) perturbation-related errors, the most common of which is using undefined truth constants as part of tautological distractions. 

\paragraph{\textbf{\emph{F8: Autoformalisation increases semantic errors for counterfactuals.}}}
Unlike the introduced noise, counterfactual perturbations do not lead to more syntax errors. The execution rate in Table~\ref{tab:appendix_k4_formalisation_exec} is stable or improves for counterfactuals. However, we see a drop in accuracy for the counterfactual column $\text{O}_C$ in Table~\ref{tab:distraction_k4_formalisation} and can conclude that the number of logical forms with semantic errors has to increase. This suggests that the introduced negation is not correctly formalised. Looking at the warnings generated by the feedback mechanism, for GPT 4o-mini, $161$ warning messages are generated on the unperturbed data. $54$ of these were fixed with a single iteration. Not considering predicates and individuals as part of the context is the most frequent warning across all models. 

\section{Related Work}
% \subsection{Vision Language Model}
% 시각장애인에서 상황을 설명할 DB가 없으니 만들었다. 그리고 이를 VLM에 튜닝했다.
\subsection{Technical approaches for assisting the visually-impaired}


\subsection{Datasets for visual instruction tuning}


\section*{Conclusion}
This paper aims to enhance our understanding of the computational complexity of computing various Shapley value variants. We found that for various ML models --- including decision trees, regression tree ensembles, weighted automata, and linear regression --- both local and global interventional and baseline SHAP can be computed in polynomial time under HMM modeled distributions. This extends popular algorithms, such as TreeSHAP, beyond their empirical distributional scope. We also establish strict complexity gaps between the various SHAP variants (baseline, interventional, and conditional) and prove the intractability of computing SHAP for tree ensembles and neural networks in simplified scenarios. Overall, we present SHAP as a versatile framework whose complexity depends on four key factors: \begin{inparaenum}[(i)] \item model type, \item SHAP variant, \item distribution modeling approach, \item and local vs. global explanations\end{inparaenum}. We believe this perspective provides deeper insight into the computational complexity of SHAP, paving the way for future work.




%We believe that our framework provides a more intricate understanding of SHAP computation complexity across different models, distributions, and variants, paving the way for further research.

Our work opens promising directions for future research. First, expanding our computational analysis to other SHAP-related metrics, such as asymmetric SHAP~\citep{frye20} and SAGE~\citep{covert2020understanding}, would be valuable. Additionally, we aim to explore more expressive distribution classes and relaxed assumptions beyond those in Section \ref{sec:tractable} while maintaining tractable SHAP computation. Finally, when exact computation is intractable (Section \ref{sec:intractable}), investigating the approximability of SHAP metrics through approximation and parameterized complexity theory~\citep{downey2012parameterized} is an important direction.

%Our work opens several promising avenues for future research on the computational properties of explainable AI methods, with a particular focus on SHAP. First, it would be interesting to broaden the computational analysis conducted in this work to include other popular SHAP-related metrics in the literature, such as asymmetric SHAP \cite{frye20} and SAGE \cite{covert2020understanding}. Also, in the future, we aim to explore more expressive distribution classes and relaxed distributional assumptions—extending beyond those examined in Section \ref{sec:tractable} —that still yield tractable SHAP computation. Finally, when exact computation proves intractable (Section \ref{sec:intractable}), it is worthwhile to theoretically investigate the question of the approximability of computing the SHAP metrics across various configurations, through the lens of approximation and parametrized complexity theory \cite{arora2009computational}.

%This paper aims to deepen our understanding of the computational complexity involved in obtaining different Shapley value variants. We found that for a variety of ML models, including decision trees, tree ensembles for regression, weighted automata, and linear regression models — computing both local and global interventional and baseline SHAP can be done in polynomial time when distributions are modeled by HMMs. This extends the distributional scope of popular algorithms like TreeSHAP, which is limited to empirical distributions. Additionally, we demonstrate a strict complexity gap between SHAP variants, showing that interventional and baseline SHAP can be strictly easier to compute than conditional SHAP. Despite these positive results, we uncovered intractability for various SHAP variants in neural networks and tree ensembles. Finally, we provided generalized complexity relations across SHAP variants. We believe that our framework offers a deeper understanding of the complexity involved in computing SHAP across various variants, models, distributions, as well as in both local and global computations, laying the groundwork for future research.

% \begin{Code}[caption={test algorithm}, captionpos=b]
%   FUNCTION validation(conn_shadow)
%     $\mathcal{R}$: validation-read-set
%     $\mathcal{W}$: validation-write-set

%     FOR r IN $\mathcal{R}$

%   FUNCTION commit(conn)
%     conn.commit() # commit the transaction in database
%     release_validation_locks()
%     # update the tuple version in memory
%     for w in $\mathcal{W}$
%       update_meta_info(w.vid)
% \end{Code}

% \begin{acks}
%  This work was supported by the [...] Research Fund of [...] (Number [...]). Additional funding was provided by [...] and [...]. We also thank [...] for contributing [...].
% \end{acks}

%\clearpage

\bibliographystyle{ACM-Reference-Format}
\bibliography{sample}

\end{document}
\endinput
