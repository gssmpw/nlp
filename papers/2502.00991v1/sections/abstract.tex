\begin{abstract}
% Serializable scheduling (SS) of transactions is essential for mission-critical applications. While configuring an RDBMS to serializable isolation (SER) guarantees SS, it often leads to suboptimal performance. Recent works have explored achieving SS by modifying the workload (e.g., promoting reads to writes for certain SQL queries) while using a lower isolation level. However, determining the optimal isolation level for peak performance is challenging. Worse still, in many cases, this complex workload modification can be outperformed by simply configuring the RDBMS to SER.
% The serializable isolation level is the gold standard in transaction processing,
% as it ensures that the results of the execution order of transactions aligns with a serial execution, albeit at a high cost. 
Achieving the serializable isolation level, regarded as the gold standard for transaction processing, 
is costly.
% involves significant performance overhead.
Recent studies reveal that adjusting specific query patterns within a workload can still achieve serializability even at lower isolation levels.
Nevertheless, these studies typically overlook the trade-off between the performance advantages of lower isolation levels and the overhead required to maintain serializability, potentially leading to suboptimal isolation level choices that fail to maximize performance.
% ======
% Nevertheless, these studies suffer from high overhead to achieve serializability and fail to address the critical trade-off between the performance advantages of lower isolation levels and the overhead required to maintain serializability, 
% leveraging
% % causing 
% the challenge of selecting the optimal isolation level to maximize performance.
% ======
% Serializable isolation level is regarded as the gold standard for transaction processing; however, achieving it entails substantial performance overhead.
% %to guarantee that the results of concurrent transactions are equivalent to serial execution.
% Recent studies reveal that executing workloads under low isolation levels can still achieve serializability by modifying specific query patterns, such as promoting reads to writes in certain SQLs.
% {\color{blue}
% Although such approaches can outperform configuring the database to the serializable isolation level, the static modification they perform can sometimes result in performance bottlenecks. 
% Moreover, existing work largely overlooks the critical trade-off between the performance benefits of lower isolation levels and the overhead required to preserve serializability. 
% In real-world applications, dynamic and evolving workloads further complicate this trade-off, making choosing the optimal isolation level exceptionally challenging.
% }
% Moreover, these modifications are sometimes outperformed by simply configuring the database to serializable isolation.
% However, selecting the optimal isolation level for such adaptations is non-trivial. 
%remains a complex and challenging task.
% can be safely executed under weaker isolation levels while still achieving serializability, by modifying 
% either by not modifying or by making simple modifications to workloads, thereby unlocking potential performance improvements. However, identifying such workloads and selecting the appropriate isolation levels for adaptation is a non-trivial task.
In this paper, we present \sysname, a middle-tier solution designed to achieve serializable scheduling with self-adaptive isolation level selection. 
First, \sysname incorporates a unified concurrency control algorithm that achieves serializability at lower isolation levels 
with minimal additional overhead.
% without introducing additional writes.
% modifications to application logic. 
Second, \sysname employs a deep learning method to characterize the trade-off between the performance benefits and overhead associated with lower isolation levels, thus predicting the optimal isolation level. 
Finally, \sysname implements a cross-isolation validation mechanism to ensure serializability during real-time isolation level transitions. 
% Extensive experiments show that \sysname achieves performance improvements of up to 26.7x over the state-of-the-art solutions and up to 4.8x over the serializable isolation level provided by PostgreSQL.
Extensive experiments demonstrate that \sysname outperforms state-of-the-art solutions by up to 26.7$\times$ and PostgreSQL's serializable isolation level by up to 4.8$\times$.
\end{abstract}