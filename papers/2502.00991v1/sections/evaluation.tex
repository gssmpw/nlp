\section{evaluations\label{sec:evaluation}}
% 动态场景中,Tristar 能完成切换
% 在其他场景中,比 SOTA 要好

%We begin with a detailed description of experimental setups, followed by a comprehensive evaluation across various configurations. 
In this section, we evaluate \sysname's performance compared to state-of-the-art solutions. 
Our goal is to validate two critical aspects empirically: (1) \sysname's effectiveness in adaptively selecting the appropriate isolation level for dynamic workloads (\S\ref{sec:evaluation:overall}); and (2) \sysname's performance superiority over state-of-the-art solutions across a variety of scenarios (\S\ref{sec:evaluation:compare-to-other-soluation}). 

\subsection{Setup}
We run our database and clients on two separate in-cluster servers with an Intel(R) Xeon(R) Platinum 8361HC CPU @ 2.60GHz processor, which includes 24 physical cores, 64 GB DRAM, and 500 GB SSD. 
The operating system is CentOS Linux release 7.9. 

% \subsubsection{Implementation.} 
% We implement \sysname on the codebase of benchbase~\cite{DBLP:journals/pvldb/DifallahPCC13} and our graph learning model is trained using the \textit{torch\_geometric} library, about 4,000 lines of Java and Python code have been modified. To ensure cross-platform compatibility and efficiency, the Python and Java components communicate via \textit{sockets} with a predefined message format.


\subsubsection{Default configuration.}
% We deployed PostgreSQL 15.2~\cite{PostgreSQL} as the database engine, which uses multi-version concurrency control to implement three distinct isolation levels: Read Committed (RC), Snapshot isolation (SI), and Serializable Snapshot Isolation (SSI)~\cite{DBLP:conf/sigmod/CahillRF08}. When reading a tuple, RC reads the most recently committed version before the read operation, whereas SI and SSI observe the committed version as of the transaction's start. At all isolation levels, write locks are used to prevent dirty writes. We configured the buffer pool size to 24GB, set the maximum number of connections to 2000, and the lock wait timeout to 100 ms. By default, we ran the experiments with 128 client terminals. To avoid the influence of the network on the evaluations, we deploy \sysname and PostgreSQL on the same server.
We utilize BenchBase~\cite{DBLP:journals/pvldb/DifallahPCC13} as our benchmark simulator, % and modify its code to interface with \sysname. 
which is deployed on the client server. By default, the experiments are conducted using 128 client terminals.
We deployed PostgreSQL 15.2~\cite{PostgreSQL} as the database engine, which employs MVCC to implement three distinct isolation levels: Read Committed (RC), Snapshot isolation (SI), and Serializable (SER) (by SSI~\cite{DBLP:conf/sigmod/CahillRF08}). 
Under RC, the system can read the most recently committed version, while both SI and SER maintain a view of the data as it existed at the start of the transaction, thereby observing the committed version from that point in time. To prevent dirty writes, write locks are enforced at all isolation levels.
For our database configuration, we allocated a buffer pool size of 24GB, limited the maximum number of connections to 2000, and established a lock wait timeout of 100 ms. 
To eliminate network-related effects, both \sysname and PostgreSQL were deployed on another server. 


\subsubsection{Baselines.} To ensure a fair comparison, we implemented existing approaches within the BenchBase framework and connected them directly to PostgreSQL.  
% we implemented existing approaches in the same framework as \sysname. 
% For clarity, we assume the presence of two transaction templates, $\mathcal{T}_1$ and $\mathcal{T}_2$, which exhibit a \textit{rw} dependency that needs to be managed at weak isolation levels.

\noindent\textbf{Baselines within the database kernel.} 
We evaluated concurrency control algorithms supported natively by PostgreSQL, specifically those associated with lower isolation levels that can achieve serializable scheduling: 

\textit{(1) \& (2) PostgreSQL’s native concurrency control mechanisms (\textbf{SER} and \textbf{SI}).}  
These approaches execute workloads configured at the SER or SI levels without requiring workload modifications. For instance, TPC-C achieves serializable scheduling under SI, while SmallBank requires SER for serializability. 

% \noindent\textbf{Baselines within the database kernel.} We select the concurrency control algorithms implemented by low isolation levels in the database kernel if they can achieve serializable scheduling.

% \textit{(1) \& (2) Concurrency control adopted by PostgreSQL (\textbf{SER} and \textbf{SI}).} %Running the workload under a serializable isolation level, with serializable scheduling guaranteed by the database, does not require workload modification, which we denote by \textbf{SER}.
% They execute the workload and configure PostgreSQL at the SER/SI level without requiring any modifications to the workload. For example, TPC-C can achieve serializability under SI, while Smallbank requires SER. 

\noindent\textbf{Baselines outside the database kernel.} We also evaluated external strategies that transform RW dependencies into WW dependencies to eliminate static dangerous structures.  

% We select the approaches that introduce additional write operations to convert RW to WW dependencies, eliminating static dangerous structures.

\textit{(3) \& (4) Promotion (\textbf{RC+Promotion}\cite{DBLP:conf/icdt/VandevoortK0N22}, \textbf{SI+Promotion} \cite{DBLP:conf/icde/AlomariCFR08}).} 
This strategy converts read operations into write operations by promoting \textit{SELECT} statements with non-modifying \textit{UPDATE} statements~\cite{DBLP:conf/icde/AlomariCFR08}. These modifications are applied at RC and SI levels, referred to as RC+Promotion and SI+Promotion, respectively.  

%The read operation in $\mathcal{T}_1$ is promoted to a write operation, altering the RW dependency into a WW dependency. 
% This cannot be achieved in common database systems (e.g., PostgreSQL) by merely replacing \textit{SELECT ... FOR UPDATE} with \textit{SELECT}, as the former only prevents some but not all interleavings that could result in a dangerous structure \cite{DBLP:conf/icde/AlomariCFR08}. Instead, we transform the \textit{SELECT} statement into a non-modifying \textit{UPDATE}. This adjustment is also applied to both RC and SI, denoting RC+Promotion and SI+Promotion, respectively.
% This approach promotes specific read operations into write operations. 
% This cannot be achieved in some databases (e.g., PostgreSQL) by simply replacing ``\textit{SELECT ... FOR UPDATE}'' with ``\textit{SELECT}'' statement, as the former only prevents some but not all interleavings that could result in non-serializable scheduling \cite{DBLP:conf/icde/AlomariCFR08}.  
% Specifically, we convert the ``\textit{SELECT}'' into a non-modifying ``\textit{UPDATE}'' statement \cite{DBLP:conf/icde/AlomariCFR08}. It applies similarly to both RC and SI, denoting RC+Promotion and SI+Promotion, respectively. 

\textit{(5) \& (6) Conflict materialization (\textbf{RC+ELM}~\cite{DBLP:conf/aiccsa/AlomariF15}, \textbf{SI+ELM} \cite{DBLP:conf/icde/AlomariCFR08}).} 
This approach employs an external lock manager (ELM) and introduces additional write operations on the ELM to ensure serializable scheduling. It is applied at both RC and SI levels, referred to as RC+ELM and SI+ELM, respectively.  

% This method introduces an external \textit{conflict} table, which is not utilized elsewhere in the workload. Both $\mathcal{T}_1$ and $\mathcal{T}_2$ should add an update operation ahead of other operations. Hence, before they can generate the rw dependency, one of them must be either blocked or aborted.
% % To avoid the rw dependency between $\mathcal{T}_1$ and $\mathcal{T}_2$, updates are made to an auxiliary \textit{conflict} table. Essentially, this introduces a ww conflict on the \textit{conflict} table before the rw dependency can occur. 
% According to the theory of weak isolation levels, we modify the transaction templates in RC and SI, which are denoted as \textbf{RC+E} and \textbf{SI+E}, respectively.
% This approach employs an external lock manager. Both $\mathcal{T}_1$ and $\mathcal{T}_2$ are modified to include an update operation on the lock manager before other operations. This modification is applied under RC and SI levels, denoting RC+ELM and SI+ELM, respectively.
% This approach employs an external lock manager (ELM) and introduces an extra write operation on ELM.
% the workloads are modified to include an extra update operation on the lock manager before other operations. 
% This modification is applied at RC and SI levels, denoting RC+ELM and SI+ELM, respectively. 
% These approaches are originally designed for RDBMS without SER support. 
%Although these approaches target RDBMSs that lack true serializable isolation support, the solution can achieve the same purpose with \sysname. Therefore, it is more relevant than other approaches. 

% \textit{(4) \& (5) Promotion.} This method promotes the read operation in $\mathcal{T}_1$ to a write operation, without changing the semantics. As a result, the rw dependency is converted into a ww dependency between $\mathcal{T}_1$ and $\mathcal{T}_2$. In PostgreSQL, this cannot be achieved by simply replacing \textit{SELECT...FOR UPDATE} with \textit{SELECT}, as the former only prevents some, but not all, interleavings that result in a vulnerable edge. Instead, we transform the \textit{SELECT} statement into a no-modification \textit{UPDATE} under the same conditions. Similar to conflict materialization, we adjust the transaction templates for both Read Committed (RC) and Snapshot Isolation (SI), referred to as \textbf{RC+P} and \textbf{SI+P}, respectively.

% It is worth noting that there are several modifications of the same approach to ensure serializable scheduling. We chose the optimal modification based on their papers. For instance, consider the modification for Promotion in the SmallBank benchmark under SI. There are two consecutive rw dependencies between \textit{Balance}, \textit{WriteCheck}, and \textit{TransactSavings} transaction templates. Promoting one of these dependencies into a ww dependency can achieve serializable scheduling under SI. According to the evaluations by Alomari et al.~\cite{DBLP:conf/aiccsa/AlomariF15}, modifying the \textit{WriteCheck} yields the best performance, so we adopt this approach in our evaluations as well.

% We apply our validation-based concurrency control to RC and SI isolation levels, referred to as \textbf{RC+TV} and \textbf{SI+TV}, respectively. We use \sysname to represent a method with adaptively isolation level strategy selection.
% It is worth noting that several modifications of the same approach exist to ensure serializability, and we adopt the most effective modification based on the findings of Alomari et al.~\cite{DBLP:conf/aiccsa/AlomariF15}. For instance, in the case of the Promotion strategy within the SmallBank benchmark under SI, modifying the \textit{WriteCheck} template rather than the \textit{Balance} template yields the best performance. 

For each strategy, we adopted the most effective variant as identified in prior work~\cite{DBLP:conf/aiccsa/AlomariF15}. For example, under the Promotion strategy in the SmallBank benchmark with SI, modifying the \textit{WriteCheck} template rather than the \textit{Balance} template yielded superior performance.  

Finally, we evaluate the middle-tier concurrency control in \S\ref{design-1} at both RC and SI levels without self-adaptive isolation level selection, denoted as \textbf{\sysname-RC} and \textbf{\sysname-SI}, respectively. 



\subsubsection{Benchmarks.} Three benchmarks are conducted as follows. %Due to space constraints, detailed descriptions of these benchmarks are available in the supplementary material linked in the code repository. In our experimental setup, any aborted transaction is retried until it executes successfully.

\noindent\textbf{SmallBank~\cite{DBLP:conf/icde/AlomariCFR08}.} This benchmark populates the database with 400k accounts, each having associated checking and savings accounts. 
Transactions are selected by each client using a uniform distribution. To simulate transactional access skew, we employ a Zipfian distribution with a default \textit{skew factor} of 0.7.
% To select which accounts to address, we considered two approaches. The first approach fixes a small subset of accounts, referred to as the hotspot, and a probability for an account selected for use in a transaction to be from among the hotspot accounts, referred to as the hotspot probability. There 

% \noindent\textbf{YCSB.} The Yahoo! Cloud Serving Benchmark (YCSB) generates synthetic workloads that simulate large-scale Internet applications. In our evaluation, the \textit{usertable} maintains 4 million records. Each record consumes 1KB, culminating in a total of 4 GB hosted by the table. A parameter called \textit{skew factor} is used to control the distribution of the accessed data items and a higher \textit{skew\_factor} results in greater contention. The \textit{skew factor} is set to 0.7 by default. Each default transaction has 10 operations, each with a 95\% probability of being a read and 5\% probability of being a write (Workload-B\cite{}).
\noindent\textbf{YCSB+T~\cite{DBLP:conf/icde/DeyFNR14}.} 
% The Yahoo! Cloud Serving Benchmark (YCSB) 
This benchmark generates synthetic workloads emulating large-scale Internet applications. In our setup, the \textit{usertable} consists of 10 million records, each 1KB in size, totaling 10GB. The \textit{skew factor}, set by default to 0.7, controls the distribution of accessed data items, with higher values increasing data contention. Each default transaction involves 10 operations, with a 90\% probability of being a read and a 10\% probability of being a write.

\noindent\textbf{TPC-C~\cite{TPCC}.} We use the TPC-C benchmark, which modifies the schema and templates to convert all predicate reads into key-based accesses according to our baseline~\cite{DBLP:journals/pvldb/VandevoortK0N21}. It includes 5 transaction templates: NewOrder, Payment, OrderStatus, Delivery, and StockLevel. Our tests host 32 warehouses, with each containing about 100MB of data. Following previous works~\cite{DBLP:journals/pvldb/YuBPDS14, DBLP:journals/pvldb/HardingAPS17}, we exclude user data errors that cause about 1\% of NewOrder transactions to abort. 

% The TPC-C benchmark [36] stands as the industry standard for evaluating OLTP databases. Its dataset comprises 9 relations, and each warehouse is equipped with 100MB of data. By default, we allocate 24 warehouses per node in our experiments. Specifically focusing on NewOrder transactions, the benchmark emulates customers submitting orders to their local district within a warehouse. We simulate scenarios where the same customer makes purchases from different warehouses over time.


% 1. Scalability - 
\subsection{Ablation Study}%Overall Performance of \sysname}
\label{sec:evaluation:overall}
% In this subsection, we evaluate the self-adaptive isolation level selection capabilities of \sysname by varying the workload every 10 seconds across six different scenarios. The experimental results are illustrated in Figure~\ref{fig:evaluation.dynamic}. In our experiments, we denote the performance of \sysname without adaptive tuning as \sysname-RC and \sysname-SI, corresponding to the RDBMS under RC and SI, respectively. The results show that \sysname can select the optimal isolation level for the dynamic workload in the majority of scenarios. specifically, \sysname adjusts to SI in low skew scenarios (A, C, E), to SER for high skew scenarios (D, F), and to RC for scenarios with high skew and percentage of writes (B). SI is suitable for scenarios with little conflicts due to its low validation overhead. Conversely, RC is ideal for scenarios with high conflict rates and more write operations, as it benefits from the ability to handle concurrent writes efficiently.

\begin{figure}[t]
    \centering
    \begin{minipage}{0.8\linewidth}
        \centering
        \includegraphics[width=\linewidth]{figures/evaluation/line_legend02.pdf}
        \vspace{-5mm}
    \end{minipage}
    \begin{minipage}{0.95\linewidth}
        \centering
        \begin{subfigure}{0.95\linewidth}
            \includegraphics[width=\linewidth]{figures/evaluation/ycsb/dynamic01.pdf}
            \vspace{-4mm}
        \end{subfigure}
    \end{minipage}
    \vspace{-4mm}
    \caption{Workload shifting by YCSB}
    \label{fig:evaluation.dynamic}
    \vspace{-6mm}
\end{figure}

% We sample the workload at the interval of 1s, resulting in a response time of approximately 1s for \sysname when the workload changes, as depicted in Figure~\ref{fig:evaluation.dynamic}. To further illustrate the overhead associated with workload switching, we provide a detailed breakdown in Figure~\ref{fig:evaluation.dynamic.breakdown}.
% The entire switching procedure takes around 450 ms. Specifically, graph construction and graph prediction require 22 ms and 47 ms, respectively, which is acceptable for long-running applications. More than 80\% of the time is spent on switching isolation levels, that is, from initiating the switch to having all connections adopt the new isolation level, which is closely related to the transaction execution latency.

% The procedure consumes about 450 ms in total. Graph construction and graph prediction took 22ms and 47ms, respectively, which is an acceptable time for a long-running application. Over 80\% of the time is spent on switching isolation levels, i.e., \sysname switches from the beginning to all connections switching to the new isolation level, which is highly correlated with the execution latency of the transaction. 

% Notably, the isolation level prediction was incorrect at the 30s and 50s marks but successfully switched to the optimal isolation level in the subsequent prediction cycle. In the following experiments, we compare the performance of \sysname against other baseline methods.


% The prediction of the isolation level is wrong at 30s and 50s, but it successfully switches to the optimal isolation level at the next prediction. In the following experiments, we compare the performance of \sysname with other baselines.

% We first evaluate the self-adaptive isolation level selection (Section~\ref{sec:self-adaptive_section}) and switching (Section~\ref{sec:switch_mechanism}) in \sysname.
In this part, we evaluate the effectiveness of the self-adaptive isolation level selection and isolation transition in \sysname.

\subsubsection{Self-adaptive isolation level selection}
We first evaluate the selection of self-adaptive isolation level by varying the workload every 10 seconds across six distinct scenarios. The experimental results are illustrated in Figure~\ref{fig:evaluation.dynamic}. %In our experiments, the performance of \sysname without the self-adaptive isolation level selection is denoted as \sysname-RC and \sysname-SI, corresponding to the RDBMS operating under RC and SI levels, respectively. 
We sample the workload at 1-second intervals. 
The results demonstrate that different isolation levels perform variably under different workloads: SI performs well in low-skew scenarios (A, C, E), SER is more effective in high-skew scenarios with a lower percentage of write operations (D, F), and RC excels in high skew scenarios with a high percentage of writes (B). Across all tested dynamic scenarios, \sysname successfully adapts to optimal isolation level. Specifically, the graph learning model in \sysname identifies that SI is suitable for scenarios with fewer conflicts due to its higher concurrency and lower data access overhead (i.e., one-time timestamp acquisition). Conversely, RC is ideal for scenarios with higher conflict rates and more write operations, as it efficiently handles concurrent writes (SI aborts concurrent writes, while RC allows them to commit). Compared to an application directly run on RDBMS at the SER level, the performance of \sysname when selecting SER is slightly reduced by 4.3\%. This overhead arises from two aspects: (1) \sysname requires modifications to the application code to select \textit{version} and another localhost message delivery ; (2) \sysname needs to sample transactions to predict the optimal isolation level, even though this is an asynchronous task.

\subsubsection{Validation analysis\label{sec:evaluation:transition}}
This part evaluates the validation efficiency in both single-isolation and cross-isolation scenarios. 

\begin{figure}[t]
    \centering
    \begin{subfigure}{0.47\linewidth}
        \centering
        \includegraphics[width=\linewidth]{figures/evaluation/ycsb/breakdown.pdf}
        \vspace{-6mm}
        \caption{Transaction breakdown}
        \label{fig:evaluation.breakdown.transaction}
    \end{subfigure}
    \begin{subfigure}{0.47\linewidth}
        \centering
        \includegraphics[width=\linewidth]{figures/evaluation/transition_breakdown.pdf}
        \vspace{-6mm}
        \caption{Transition breakdown}
        \label{fig:evaluation.breakdown.transition}
    \end{subfigure}
    \vspace{-4mm}
    \caption{Breakdown analysis by YCSB}
    \label{fig:evaluation.dynamic.breakdown}
    \vspace{-4mm}
\end{figure}


% \textcolor{red}{todo, need more content!}
\noindent\textbf{Single-isolation level validation.} 
% When the skew factor is set to 0.3, 0.7, and 1.1, respectively, we randomly selected 5 transactions and depicted the average decomposition in Figure~\ref{fig:evaluation.breakdown.transaction}. The validation cost remains relatively stable, decreasing from 2.6\% to 0.3\% of the transaction lifecycle as contention increases. This suggests that the middle-tier concurrency control proposed in \cite{design-1} does not significantly affect normal execution.
We evaluate the single-level validation cost under skew factors of 0.3, 0.7, and 1.1, respectively. The average breakdown is depicted in Figure~\ref{fig:evaluation.breakdown.transaction}. The validation cost remains relatively stable, decreasing from 2.6\% to 0.3\% of the transaction lifecycle as contention increases. This suggests that the middle-tier concurrency control proposed in \S\ref{design-1} does not significantly affect normal execution.

\begin{figure}[t]
    \centering
    \begin{minipage}{0.95\linewidth}
        \centering
        \includegraphics[width=\linewidth]{figures/evaluation/mix_legend03.pdf}
        \vspace{-5mm}
    \end{minipage}
    \begin{minipage}{0.95\linewidth}
        \centering
        \begin{subfigure}{0.48\linewidth}
            \includegraphics[width=\linewidth]{figures/evaluation/ycsb/dynamic02.pdf}
            \vspace{-6mm}
            \caption{Transition from D to E}
            \label{fig:evaluation.dynamic_switch.de}
            \vspace{-4mm}
        \end{subfigure}
        \begin{subfigure}{0.48\linewidth}
            \includegraphics[width=\linewidth]{figures/evaluation/ycsb/switch-ab.pdf}
            \vspace{-6mm}
            \caption{Performance metrics}
            \label{fig:evaluation.dynamic_switch.metric}
            \vspace{-4mm}
        \end{subfigure}
    \end{minipage}
    % \vspace{-4mm}
    \caption{Comparasion of transition mechanisms by YCSB}
    \label{fig:evaluation.dynamic_switch}
    \vspace{-6mm}
\end{figure}

\noindent\textbf{Cross-isolation level validation.} 
% \textcolor{blue}{
% We evaluate \sysname using various transition mechanisms mentioned in Section \ref{design-3} by YCSB. To simulate different execution latencies, we adjust the ``think time" parameter and transition the workload from D to E. We first record the performance metrics, including abort count and blocked time, for each algorithm during the transition. The result is shown in Figure~\ref{fig:evaluation.dynamic_switch}.  
% As think time increases, transaction execution latency also rises, and the abort strategy results in a higher number of aborted transactions, while the blocking time associated with the blocking strategy also increases accordingly.
% % The abort mechanism results in more aborted transactions, and almost all transactions are aborted when the think time reaches 1000ms. Additionally, the blocking time of the block strategy increases in tandem with the rise in transaction execution latency.
% In contrast, the cross-isolation validation mechanism outperforms the other two mechanisms by up to 35.9\% and 40.3\%, while maintaining minimal transaction aborts and negligible blocking time.
% This enhanced performance can be attributed to \sysname's ability to avoid actively blocking or rolling back incoming transactions; thus, the impact on normal transaction execution is minimized while maintaining serializable scheduling. 
% }
We evaluate \sysname with YCSB using the various transition mechanisms mentioned in \S\ref{design-3}. We first evaluate the transition of the workload from D to E with a ``think time'' of 1s, as illustrated in Figure~\ref{fig:evaluation.dynamic_switch.de}. \sysname minimizes the impact of isolation level transitions by avoiding active block time or aborts, maintaining serializable scheduling.
To further compare mechanisms, we vary the ``think time'' parameter (Figure~\ref{fig:evaluation.dynamic_switch.metric}). Increased think time raises transaction latency and leads to more aborts under the abort strategy, while the blocking strategy incurs longer blocking times. In contrast, the cross-isolation validation mechanism outperforms both, reducing transaction aborts and blocking time while delivering performance improvements of up to 2.7$\times$ and 5.4$\times$, respectively.

% \subsubsection{Impact of optimizations}
% \todo{ablation study of hot version cache}

\subsubsection{Graph model: construction, training, and prediction}

Figure~\ref{fig:evaluation.breakdown.transition} illustrates the overhead of workload transition, which takes approximately 450 milliseconds.
% —negligible for longer workloads. 
Specifically, graph construction and prediction require 22 milliseconds and 47 milliseconds, respectively, while over 80\% of the time is spent on transition, from initiating the transition to all connections adopting the new isolation level, closely tied to the longest transaction execution latency.
Notably, the prediction in Figure \ref{fig:evaluation.dynamic} is inaccurate for 1 or 2 seconds at the 30-second and 50-second marks due to the sampling transactions from the previous workload during the transition. However, the model successfully transitions to the optimal isolation level in subsequent prediction cycles. The overhead of the learned model is minimal, with less than a 2.5\% difference in throughput between using the graph model and not using it.

\begin{figure}[t]
    \centering
    \begin{minipage}{0.95\linewidth}
        \centering
        \begin{subfigure}{0.48\linewidth}
            \includegraphics[width=\linewidth]{figures/evaluation/train/acc.pdf}
            \vspace{-6mm}
            \caption{Accuarcy}
            \label{fig:evaluation.train.acc}
        \end{subfigure}
        \begin{subfigure}{0.48\linewidth}
        \includegraphics[width=\linewidth]{figures/evaluation/train/tsne.pdf}
            \vspace{-6mm}
            \caption{Extracted features}
            \label{fig:evaluation.train.cluster}
        \end{subfigure}
    \end{minipage}
    \vspace{-4mm}
    \caption{Model training metrics by YCSB}
    \label{fig:evaluation.train}
    \vspace{-4mm}
\end{figure}

\extended{
\begin{figure}[t]
    \centering
    \begin{minipage}{0.95\linewidth}
        \centering
        \begin{subfigure}{0.48\linewidth}
            \includegraphics[width=\linewidth]{figures/evaluation/train/smallbank/acc.pdf}
            \vspace{-4mm}
            \caption{Accuarcy by smallbank}
            \label{fig:evaluation.train.sb.acc}
        \end{subfigure}
        \begin{subfigure}{0.48\linewidth}
        \includegraphics[width=\linewidth]{figures/evaluation/train/tpcc/acc.pdf}
            \vspace{-4mm}
            \caption{Accuarcy by TPC-C}
            \label{fig:evaluation.train.tpcc.acc}
        \end{subfigure}
    \end{minipage}
    \vspace{-4mm}
    \caption{Model training metrics}
    \label{fig:evaluation.train.tpcc_sb}
    \vspace{-6mm}
\end{figure}
}

We also compare the training process at various learning rates. As shown in Figure~\ref{fig:evaluation.train.acc}, we find that a learning rate of 0.005 quickly achieves approximately 86\% accuracy on test workloads. A small learning rate (0.0001) results in slow training due to minimal weight updates, while a large learning rate (0.1 or greater) can lead to poor accuracy.
To visualize the high-dimensional vectors produced by our model, we use t-SNE~\cite{tsne} for nonlinear dimensionality reduction, mapping them into two dimensions and plotting them with their true labels in Figure~\ref{fig:evaluation.train.cluster}. Most workloads are accurately distinguished, with errors primarily occurring at the boundaries between isolation levels, where performance similarities can lead to incorrect predictions that do not significantly impact overall performance.
\maintext{For more detailed descriptions of the other two models, please refer to our technical report~\cite{TxnSails}.}
\extended{
We further illustrate the training accuracy for Smallbank and TPC-C in Figure~\ref{fig:evaluation.train.tpcc_sb}. During the training process, we noticed an imbalance among the three types of labels. For example, in the Smallbank benchmark, TxnSails-SI consistently outperformed the other two in various scenarios. Inspired by downsampling techniques, we reduce the training data for certain classifications to balance the dataset. We set the model's learning rate to 0.0005. After 10 rounds of training, the accuracy stabilizes at 95.1\% for Smallbank and 91.7\% for TPC-C.
}

\noindent \textbf{\underline{Summary.}} One isolation level does not fits all workloads. 
% Higher isolation levels do not necessarily result in poorer performance. 
In low-skew scenarios, SI outperforms RC; in high-skew scenarios with fewer writes, SER is the most effective; and in high-skew scenarios with intensive writes, RC is more suitable. 
\sysname effectively guarantees SER at lower levels and efficiently adapts isolation levels to optimize performance for dynamic workloads using the proposed fast isolation level transition technique.

% \noindent \textbf{\underline{Insight} Effectiveness of Cross-Isolation Level Validation in Section~\ref{sec:switch_mechanism}. } 

\subsection{Comparision to State-of-the-art Solutions}
\label{sec:evaluation:compare-to-other-soluation}

% \noindent\textbf{YCSB.} \sysname outperforms other approaches by up to 4.04x. As the number of terminals increases, \sysname continues outperforming other methods, surpassing SER by a factor of 1.83 under the default skew of 0.7. This indicates that the overhead of detecting and avoiding anomalies outside the database is less than the performance loss encountered when the database is set to a strong isolation level. \sysname would set the database to SI and apply the validation-based concurrency control to avoid vulnerable edges in this case. Additionally, \sysname has a clear advantage over other methods that modify loads to achieve serializable scheduling under weak isolation, as those methods introduce extra write operations. These additional operations restrict concurrency and alter more locking information, further impacting performance.

% \noindent\textbf{SmallBank.} \sysname outperforms other approaches by up to 4.29x. 
% Similar to the YCSB workload, in these experiments, the benefit of setting the database to a weak isolation level outweighs the overhead of handling concurrency control outside the database. In these evaluations, SI+P outperforms SER, and \sysname outperforms it by 11.7\%. This is due to two main reasons. Firstly, in SmallBank, the SI+P approach only needs to modify the \textit{WriteCheck} transaction template and promote the read operation on \textit{savings} table into write, impacting only 20\% of the transactions. Secondly, the promotion operation has little impact on concurrency in the workload because the \textit{WriteCheck} transaction will already execute a write operation on the same key in the \textit{checking} table after the read operation. Thus, the issue of concurrent writes caused by the promotion already existed beforehand. 

% By default, we set the number of client terminals to 128 for the remaining evaluations.

We evaluate \sysname against state-of-the-art solutions that use \textbf{external lock manager (ELM)} \cite{DBLP:conf/icde/AlomariCFR08,DBLP:conf/aiccsa/AlomariF15} and \textbf{Promotion} \cite{DBLP:conf/icde/AlomariCFR08,DBLP:conf/icdt/VandevoortK0N22}  over workloads by YCSB, SmallBank, and TPC-C benchmarks.  %By default, we set the number of client terminals to 128 for other evaluations. 
% As client numbers increase, the number of concurrent transactions increases correspondingly, thereby raising the probability of data conflicts. 




% \noindent\textbf{YCSB.} \sysname outperforms other approaches by up to 4.04x. As client terminals increase, \sysname consistently outperforms, exceeding SER by a factor of 1.83 under the default skew of 0.7. This performance indicates that the overhead associated with detecting and avoiding anomalies outside the database is less significant than the performance degradation experienced when the database is set to a strong isolation level. In such scenarios, \sysname configures the database to SI and implements validation-based concurrency control to avoid dangerous structures. Moreover, \sysname demonstrates a distinct advantage over other approaches that modify workloads %(e.g., SELECT ... FOR UPDATE) 
% to achieve serializable scheduling under weak isolation conditions, as these methods often introduce additional write operations that limit concurrency and alter extensive locking information, thereby adversely affecting performance.

% \noindent\textbf{SmallBank.} \sysname outperforms other approaches by up to 4.29x.
% Similar to the findings from the YCSB workload, the advantage of configuring the database to a weaker isolation level outweighs the overhead of external concurrency control management in these experiments. In these evaluations, SI+Promotion outperforms SER, with \sysname achieving a performance improvement of 11.7\% over SI+Promotion. This enhancement is attributable primarily to two factors. First, within SmallBank, the SI+Promotion method necessitates only a modification of the \textit{WriteCheck} transaction template to elevate the read operation on the \textit{savings} table to a write operation, affecting merely 20\% of transactions. Second, this promotion operation exerts minimal impact on concurrency within the workload, as the \textit{WriteCheck} transaction already involves a write operation on the same key in the \textit{checking} table after the read operation. Thus, the issue of concurrent writes caused by the promotion was pre-existing.

% \noindent \textbf{\underline{Insight}} Current research efforts 
% typically substitute read locks with write locks, inadvertently compromising concurrency. Consequently, the overall scalability enhancements are limited. In contrast, implementing validation-based concurrency control in \sysname effectively manages weaker isolation levels while enhancing both concurrency and performance robustly.

% (e.g., \textcolor{red}{VLDB23}) along with vendor databases that do not support SER level (e.g., Oracle), often employ ``SELECT FOR UPDATE'' to achieve serializable scheduling under weak isolation levels. However, these methods 
% 2. Impact of skew workload (hotspot in SmallBank, skew in smallbank/ycsb/tpcc)

\subsubsection{Impact of data contention}
\label{sec:evaluation:contention}
% In this subsection, we evaluate the impact of data contention, and experiments demonstrate that \sysname significantly outperforms the benchmark in various scenarios.

% \noindent\textbf{YCSB.} We vary the \textit{skew factor} from 0.1 to 1.3 to simulate different levels of data contention, with the evaluation results depicted in Figure~\ref{fig:evaluation.contention.ycsb}. When the \textit{skew factor} is 0.9 or lower, `\sysname outperforms other approaches, including SER, by up to 19.5x and 2.5x better than the next optimal approach. When the \textit{skew factor} exceeds 0.9, \sysname switches the database to SER and achieves performance improvements of up to 26.7x. In high contention scenarios, validation-based concurrency control outside the database becomes less efficient, causing the validation costs to outweigh the benefits of using a weaker isolation level. For YCSB workloads, all read and write operations must be validated. The optimization discussed in \S~\ref{design-3} can somewhat mitigate validation overhead. However, to ensure that the read set remains unmodified before it commits, we block write operations until the read operation commits. In contrast, SER does not block concurrent read and write operations. 
% \begin{figure}[]
%     \centering
%     \begin{minipage}{0.8\linewidth}
%         \centering
%         \includegraphics[width=\linewidth]{figures/evaluation/bar_legend01.pdf}
%         \vspace{-5mm}
%     \end{minipage}
%     \begin{minipage}{0.95\linewidth}
%         \centering
%         % \begin{subfigure}{0.48\linewidth}
%         %     \includegraphics[width=\linewidth]{figures/evaluation/smallbank/scalability01.pdf}
%         %     \vspace{-6mm}
%         %     \caption{Performance - Smallbank}
%         %     \label{fig:sb.skew.per}
%         % \end{subfigure}
%         \begin{subfigure}{0.95\linewidth}
%             \includegraphics[width=\linewidth]{figures/evaluation/ycsb/skew01.pdf}
%             \vspace{-2mm}
%             % \caption{Performance - YCSB}
%             \label{fig:ycsb.skew.per}
%         \end{subfigure}
%     \end{minipage}

%     % \begin{subfigure}{0.48\linewidth}
%     %     \includegraphics[width=\linewidth]{figures/evaluations/tpcc/scalability/scalability_dr_0.5_performance2.pdf}
%     %     \vspace{-6mm}
%     %     \caption{TPCC}
%     %     \label{Fig.tpcc.sca.per}
%     % \end{subfigure}
%     \vspace{-6mm}
%     \caption{Impact of data contention by YCSB}
%     \label{fig:evaluation.contention.ycsb}
%     \vspace{-6mm}
% \end{figure}


% \noindent\textbf{Smallbank.} We employed two methods to simulate data contention: adjusting the skew factor to alter the data distribution generated by Zipfian and fixing the number of hotspot data while changing the probability of accessing these hotspots. The results are shown in Figure~\ref{fig:evaluation.contention.sb}. In both Figure~\ref{fig:sb.skew.per} and \ref{fig:sb.hotspot.per}, \sysname consistently outperforms the comparison method across all scenarios, achieving up to a 15.61x improvement and a 117.5\% performance increase compared to SER. The overall performance trend initially rises and then falls. As skew increases, data access becomes more centralized, leading to higher cache hits and improved performance. However, as skew continues to rise, data access conflicts become the system's bottleneck, causing a decline in performance. Unlike YCSB, for Smallbank, we only need to validate a small portion of read, write, and erase operations. As contention intensifies, \sysname maintains its advantage over SER by employing a weaker isolation level. 

% \noindent\textbf{TPC-C.} We simulated data contention on the customer table using the Zipfian distribution, as shown in Figure~\ref{fig:evaluation.contention.tpcc}. \sysname achieves an improvement of up to 75.4\%. Setting the database to SI ensures serializable scheduling of TPCC loads, and SI consistently outperforms SER across all experimental scenarios. \sysname maintains serializable scheduling even when the database is set to RC by employing a two-tier concurrency control mechanism. This approach leverages the performance benefits of the weaker isolation level to enhance overall performance. Additionally, we observed that adjusting the skew factor for the customer table does not result in significant performance changes. This is because the throughput bottleneck is not due to multiple transactions accessing the same customer but rather due to them accessing the same warehouse. 
% % We depend on the Zipfian skew over the Customer relation
% % Noticeably, changing the skew over the Customer relation does not significantly change throughput. This is because the throughput bottleneck is not caused by multiple transactions accessing the same customer but by accessing the same warehouse instead.


This part studies the impact of data contention by varying the \textit{skew factor} (SF) %from 0.1 to 1.3 
and by varying \textit{hotspot probability} %from 10\% to 90\% 
and the number of hotspots to simulate different data contention in YCSB and SmallBank, respectively.

In YCSB, \sysname outperforms other solutions by up to 22.7$\times$ and is 2.4$\times$ better than the second-best solution due to lightweight validation without workload modification, thus higher concurrency as depicted in Figure~\ref{fig:ycsb.contention}. In this case, SOTA solutions can not beat SER as they introduce additional write operations in YCSB workloads that restrict concurrency. In high contention scenarios (SF>0.9), validation costs outweigh the benefits of using a lower isolation level, triggering \sysname to transition to the SER level and perform slightly (<5\%) lower than SER due to \sysname overhead. %\textcolor{blue}{Compared to the application directly run on RDBMS at the SER level, the overhead of real-time prediction and query transmission is less than 4.2\%.} 
In low contention scenarios (SF<0.9), the ELM approaches yield better efficiency than the Promotion ones as lock conflicts are low with external locks. 
We further analyze the latency distribution with the skew factor of 0.9 using cumulative distribution function (CDF) plots, as shown in Figure~\ref{fig:ycsb.skew.cdf.09}. In all scenarios, \sysname reduces the latency of transactions. 

\begin{figure}[]
    \centering
    % \begin{minipage}{0.8\linewidth}
    %     \centering
    %     \includegraphics[width=\linewidth]{figures/evaluation/bar_legend01.pdf}
    %     \vspace{-6mm}
    % \end{minipage}
    \begin{minipage}{0.8\linewidth}
        \centering
        \includegraphics[width=\linewidth]{figures/evaluation/line_legend04.pdf}
        \vspace{-5mm}
    \end{minipage}
    \begin{minipage}{0.95\linewidth}
        \centering
        % \begin{subfigure}{0.48\linewidth}
        %     \includegraphics[width=\linewidth]{figures/evaluation/smallbank/scalability01.pdf}
        %     \vspace{-6mm}
        %     \caption{Performance - Smallbank}
        %     \label{fig:sb.skew.per}
        % \end{subfigure}
        \begin{subfigure}{0.46\linewidth}
            \includegraphics[width=\linewidth]{figures/evaluation/ycsb/skew02.pdf}
            \vspace{-6mm}
            \caption{Performance}
            \label{fig:ycsb.contention}
        \end{subfigure}
        \begin{subfigure}{0.46\linewidth}
            \includegraphics[width=\linewidth]{figures/evaluation/ycsb/skew-07_cdf.pdf}
            \vspace{-6mm}
            % \caption{\textit{Skew factor} is 0.9 - YCSB}
            \caption{Analysis of latency CDF}
            \label{fig:ycsb.skew.cdf.09}
        \end{subfigure}
    \end{minipage}

    % \begin{subfigure}{0.48\linewidth}
    %     \includegraphics[width=\linewidth]{figures/evaluations/tpcc/scalability/scalability_dr_0.5_performance2.pdf}
    %     \vspace{-6mm}
    %     \caption{TPCC}
    %     \label{Fig.tpcc.sca.per}
    % \end{subfigure}
    \vspace{-4mm}
    % \caption{Analysis of latency CDF by YCSB}
    \caption{Impact of data contention by YCSB}
    \label{fig:evaluation.contention.ycsb.cdf}
    \vspace{-4mm}
\end{figure}

In SmallBank, \sysname consistently outperforms other solutions by up to 15.27$\times$ improvement and 2.06$\times$ better than the second-best solution, as depicted in Figure \ref{fig:evaluation.contention.sb}. This time, SI+Promotion can outperform SER. The reason is that, unlike YCSB, SmallBank validates only a small portion of read and write operations. As the skew factor increases, \sysname maintains its advantage over SER by employing SI level. 
As hotspot size increases, the reduced conflicts between transactions make the performance advantage of \sysname less pronounced. 

% \noindent\textbf{YCSB.} We vary the \textit{skew factor} from 0.1 to 1.3 to simulate different levels of data contention, depicted in Figure~\ref{fig:evaluation.contention.ycsb}. When the \textit{skew factor} is 0.9 or lower, \sysname outperforms other solutions by up to 19.5x and 2.5x better than the second-best solution. 
% When the \textit{skew factor} exceeds 0.9, \sysname switches the database to SER and achieves performance improvements of up to 26.7x. In high contention scenarios, validation-based concurrency control outside the database becomes less efficient, causing the validation costs to outweigh the benefits of using a weaker isolation level. For YCSB workloads, all read and write operations must be validated. 
% % The optimization discussed in \S~\ref{design-3} can somewhat mitigate validation overhead. However, 
% To ensure that the read set remains unmodified before it commits, we block write operations until the read operation commits. In contrast, SER does not block concurrent read and write operations. 
% We further analyze the latency distribution of transactions with skew factors of 0.1 and 0.9 using cumulative distribution function (CDF) plots, as shown in Figure~\ref{fig:evaluation.contention.ycsb.cdf}. In all scenarios, \sysname consistently reduces the latency of transactions. 


\begin{figure}[]
    \centering
    \begin{minipage}{0.8\linewidth}
        \centering
        \includegraphics[width=\linewidth]{figures/evaluation/bar_legend01.pdf}
        \vspace{-5mm}
    \end{minipage}
    \begin{minipage}{0.95\linewidth}
        \centering
        \begin{subfigure}{1.0\linewidth}
            \includegraphics[width=\linewidth]{figures/evaluation/smallbank/skew01.pdf}
            \vspace{-6mm}
            \caption{Performance with \textit{skew factors}}
            \label{fig:sb.skew.per}
        \end{subfigure}
        \begin{subfigure}{1.0\linewidth}
            \includegraphics[width=\linewidth]{figures/evaluation/smallbank/hotspot01.pdf}
            \vspace{-6mm}
            \caption{Performance with fixed number of hotspots}
            \label{fig:sb.hotspot.per}
        \end{subfigure}
    \end{minipage}
    \vspace{-4mm}
    \caption{Impact of data contention by SmallBank}
    \label{fig:evaluation.contention.sb}
    \vspace{-6mm}
\end{figure}

% \noindent\textbf{SmallBank.} We employed two methods to simulate data contention: adjusting the skew factor to alter the data distribution generated by Zipfian and fixing the number of hotspot data while changing the probability of accessing these hotspots. The results are shown in Figure~\ref{fig:evaluation.contention.sb}. %In both Figure~\ref{fig:sb.skew.per} and \ref{fig:sb.hotspot.per}, 
% \sysname consistently outperforms the comparison method across all scenarios, achieving up to a 15.61x improvement and a 2.2x performance increase compared to SER.  %As skew increases, the overall performance trend initially rises and then falls. This is because data access becomes more centralized, leading to higher cache hits and improved performance. However, as skew continues to rise, data access conflicts become the system's bottleneck, causing a decline in performance. 
% Unlike YCSB, for SmallBank, we only need to validate a small portion of read and write operations. As contention intensifies, \sysname maintains its advantage over SER by employing a weaker isolation level. Transactions accessing hotspots follow a uniform distribution. At a hotspot size of 10, data dependencies between transactions enhance performance by removing the requirement for WW dependencies. However, at a hotspot size of 100, the reduced conflicts between transactions make the performance advantage of \sysname less pronounced. 
% % \textcolor{red}{no description for the impact of hotspot numbers}

% % \noindent\textbf{TPC-C.} We omit the evaluation of changing warehouse number of TPC-C as it performs similarly to the varying YCSB skew factor. Instead, we simulated data contention on the customer table using the Zipfian distribution, as shown in Figure~\ref{fig:evaluation.contention.tpcc}. \sysname achieves an improvement of up to 75.4\%. Setting the database to SI ensures serializable scheduling of TPCC loads, and SI consistently outperforms SER across all experimental scenarios. \sysname maintains serializable scheduling even when the database is set to RC by employing a two-tier concurrency control mechanism. This approach leverages the performance benefits of the weaker isolation level to enhance overall performance. Additionally, we observed that adjusting the skew factor for the customer table does not result in significant performance changes. This is because the throughput bottleneck is not due to multiple transactions accessing the same customer but rather due to them accessing the same warehouse.

% \noindent \textbf{\underline{Insight.}} In low or medium contention, \sysname leverages additional validation to enhance concurrency and guarantee serializability scheduling at a lower isolation level, resulting in improved performance. However, in high contention, the execution of a higher number of concurrent transactions leads to an increase in conflict rollbacks, and the additional validation has a noticeable impact on performance. Consequently, it becomes more advantageous to utilize a higher isolation level (i.e., SER) in such situations.

% 3. Impact of write/read ratio - ycsb
\subsubsection{Impact of write/read ratios}
\label{sec:evaluation:wr}
This part evaluates the performance of varying the percentage of write operations with YCSB, using the \textit{skew factors} of 0.1 and 0.7. 
In read-write scenarios in Figure~\ref{fig:ycsb.wr.skew01} and~\ref{fig:ycsb.wr.skew07}, \sysname can outperform other solutions up to 6.68x. As the percentage of write operations increases, the performance gap narrows as verification overhead at lower isolation levels increases. \sysname transitions from using SI to SER and finally to RC, as the FCW \cite{DBLP:journals/pvldb/ChenPLYHTLCZD24_TDSQL} strategy increases the abort rate in scenarios with a high percentage of write operations.

We also evaluate the performance in read-only scenarios in Figure~\ref{fig:ycsb.wr.ro}. \sysname achieves performance up to 4.6$\times$ higher than SER and up to 20.4$\times$ higher than others. \sysname adopts to SI level as its in-memory validation is nearly costless and rarely fails. Other solutions convert read operations to write operations, thereby restricting concurrency. 
This also highlights that when a database is configured to be SER, there is a significant performance loss compared to SI, even with read-only scenarios. %, despite the utilization of snapshots without RW conflicts. 



% This part evaluates the performance of \sysname by varying the percentage of write operations with YCSB, using the \textit{skew factors} of 0.1 and 0.7. 
% In read-write scenarios in Figure~\ref{fig:ycsb.wr.skew01} and~\ref{fig:ycsb.wr.skew07}, \sysname can still outperform other solutions up to 6.82x. As the percentage of write operations increases, \sysname transitions from using the SI to RC. This is because the FCW strategy raises the abort rate in scenarios with a high percentage of write operations. Consequently, the performance gap between \sysname and other approaches narrows as verification overhead at lower isolation levels grows. In conflict materialization and promotion approaches, all RW dependencies must be managed for the YCSB workload. However, their performance deteriorates as the proportion of write operations increases. This is because the write operations were introduced to avoid RW dependencies and do not modify records. In contrast, regular write operations alter multiple fields in the table and generate serval logs, leading to higher IO costs.

% We evaluated the performance of various approaches in read-only scenarios, as illustrated in Figure~\ref{fig:ycsb.wr.ro}. The results demonstrate that \sysname achieves performance up to 4.8x higher than SER and up to 20.9x higher than other methods. \sysname operates under the SI isolation level, where in-memory validation is nearly costless and rarely fails. Unlike other approaches that convert read operations to write operations, thereby introducing concurrency issues, while \sysname maintains the original load. 
% This highlights that when a database is configured to be serializable, there is a significant performance loss compared to SI, particularly in read-only scenarios, despite the utilization of snapshots without RW conflicts. 

% \noindent \textbf{\underline{Insight.}} In read-only scenarios, \sysname exhibits notable enhancements when deployed at the SI isolation level. However, when writes intensify, \sysname switches to RC isolation level as the FCW strategy employed in SI begins to affect concurrent writes.

% The results are illustrated in Figure~\ref{fig:evaluation.wr}. 
\begin{figure}[]
    \centering
    \begin{minipage}{0.8\linewidth}
        \centering
        \includegraphics[width=\linewidth]{figures/evaluation/bar_legend01.pdf}
        \vspace{-5mm}
    \end{minipage}
    \begin{minipage}{0.95\linewidth}
        \centering
        \begin{subfigure}{0.48\linewidth}
            \includegraphics[width=\linewidth]{figures/evaluation/ycsb/wr01.pdf}
            \vspace{-6mm}
            \caption{Skew factor is 0.1 - YCSB}
            \label{fig:ycsb.wr.skew01}
        \end{subfigure}
        \begin{subfigure}{0.48\linewidth}
            \includegraphics[width=\linewidth]{figures/evaluation/ycsb/wr02.pdf}
            \vspace{-6mm}
            \caption{Skew factor is 0.7 - YCSB}
            \label{fig:ycsb.wr.skew07}
        \end{subfigure}
        \begin{subfigure}{0.9\linewidth}
            \vspace{1mm}   
            \includegraphics[width=\linewidth]{figures/evaluation/ycsb/read_only.pdf}
            \vspace{-6mm}
            \caption{Read only transactions - YCSB}
            \label{fig:ycsb.wr.ro}
        \end{subfigure}
    \end{minipage}
    % \begin{subfigure}{0.48\linewidth}
    %     \includegraphics[width=\linewidth]{figures/evaluations/tpcc/scalability/scalability_dr_0.5_performance2.pdf}
    %     \vspace{-6mm}
    %     \caption{TPCC}
    %     \label{Fig.tpcc.sca.per}
    % \end{subfigure}
    \vspace{-4mm}
    \caption{Impact of write/read ratio by YCSB}
    \label{fig:evaluation.wr}
    \vspace{-4mm}
\end{figure}

% 4. Impact of transaction template
\subsubsection{Impact of templates percentages}
\label{sec:evaluation:ratio}
In complex workloads like SmallBank and TPC-C, only certain transaction templates lead to data anomalies, so modifying these templates can ensure serializability under low isolation levels. This part compares different solutions by varying the percentage of critical transaction templates. % while maintaining a uniform distribution for others. For instance, if 90\% of transactions are \textit{Balance} in SmallBank, the remaining four templates each have a 2.5\% percentage. 

\begin{figure}[]
    \centering
    \begin{minipage}{0.8\linewidth}
        \centering
        \includegraphics[width=\linewidth]{figures/evaluation/bar_legend01.pdf}
        \vspace{-5mm}
    \end{minipage}
    \begin{minipage}{0.95\linewidth}
        \centering
        \begin{subfigure}{0.95\linewidth}
            \includegraphics[width=\linewidth]{figures/evaluation/smallbank/ratio01.pdf}
            % \vspace{-6mm}
            % \caption{Impact of templates percentage by Smallbank}
            % \label{fig:sb.ratio.bal}
        \end{subfigure}
        % \begin{subfigure}{0.95\linewidth}
        %     \includegraphics[width=\linewidth]{figures/evaluation/smallbank/wc01.pdf}
        %     \vspace{-6mm}
        %     \caption{Impact of \textit{WriteCheck} templates - Smallbank}
        %     \label{fig:sb.ratio.wc}
        % \end{subfigure}
    \end{minipage}
    % \begin{subfigure}{0.48\linewidth}
    %     \includegraphics[width=\linewidth]{figures/evaluations/tpcc/scalability/scalability_dr_0.5_performance2.pdf}
    %     \vspace{-6mm}
    %     \caption{TPCC}
    %     \label{Fig.tpcc.sca.per}
    % \end{subfigure}
    \vspace{-4mm}
    \caption{Impact of templates percentage by SmallBank}
    \label{fig:evaluation.sb.ratio}
    \vspace{-6mm}
\end{figure}

In SmallBank, we evaluate the proportions of the read-only \textit{Balance} and write transaction \textit{WriteCheck}, as shown in Figure~\ref{fig:evaluation.sb.ratio}. 
As the proportion of \textit{Balance} transactions increases, performance improves; however, RC+ELM and RC+Promotion introduce additional writes in \textit{Balance}, leading to increased WW conflicts. In contrast, SI+ELM and SI+Promotion perform better since they do not modify read-only \textit{Balance} transactions.
\sysname-RC must detect RW dependencies from \textit{Balance}, increasing overhead as their proportion rises. Thus, \sysname transitions to SI in this scenario, achieving up to a 6.2$\times$ performance gain over RC+ELM and RC+Promotion. Conversely, as the proportion of \textit{WriteCheck} transactions increases, concurrency decreases, leading to worse performance for SI+ELM and SI+Promotion. However, \sysname's performance advantage becomes more pronounced as it maintains consistent commit and dependency orders through validation without modifying the workload. At 90\% \textit{WriteCheck} transactions, \sysname improves performance by 58.1\% compared to SI+Promotion and achieves 2.3$\times$ the performance of SER.

TPC-C can execute serializable under SI, eliminating the need for validation in SI. The critical \textit{NewOrder} and \textit{Payment} transactions require modifications by RC+ELM and RC+Promotion, increasing write conflicts on \textit{NewOrder}, resulting in a performance disadvantage compared to \sysname, which can achieve up to 2.3$\times$ their performance. Due to high contention on the warehouse relation, validation overheads are generally higher, except when the proportion of \textit{NewOrder} is 0.1, where \sysname shows a 10.7\% improvement over SI. In other scenarios, \sysname adapts to SI. \textit{Payment} transactions are more write-intensive, prompting \sysname to set the database to RC, which achieves up to 40.6\% performance improvement over SI. Compared to other solutions, \sysname achieves up to 53.5\% performance improvement.

% \noindent\textbf{SmallBank.} We adjusted the \textit{Balance} and \textit{WriteCheck} ratios as shown in Figure~\ref{fig:evaluation.sb.ratio}. The \textit{Balance} transaction is read-only, and as its proportion increases, the performance of various methods also improves. SI is suitable for these scenarios, with \sysname achieving up to a 6.2x performance gain compared to RC+ELM and RC+Promotion. This is because RC+ELM and RC+Promotion promote read operations in \textit{Balance} to write operations, leading to increased WW conflicts and reduced performance as the proportion of \textit{Balance} transactions rises. Consequently, the higher the proportion of \textit{Balance} transactions, the greater the number of WW conflicts, and the more significant the performance advantage of \sysname over SER. When \textit{Balance} transactions reach 90\%, the performance is 2.3x that of SER.

% On the other hand, \textit{WriteCheck} is a transaction template that can introduce dangerous structures under SI, necessitating modifications by SI+Promotion and SI+ELM. As the proportion of \textit{WriteCheck} transactions increases, the concurrency between \textit{WriteCheck} transactions decreases, resulting in progressively worse performance for SI+ELM and SI+Promotion. However, \sysname ensures that commit and dependency orders remain consistent through validation without modifying the workload. Consequently, as the percentage of \textit{WriteCheck} transactions rises, the performance advantage of \sysname becomes more evident. At 90\% \textit{WriteCheck} transactions, \sysname improves performance by 58.1\% compared to SI+Promotion and achieves 2.3x the performance of SER.

\begin{figure}[]
    \centering
    \begin{minipage}{0.95\linewidth}
        \centering
        \includegraphics[width=\linewidth]{figures/evaluation/bar_legend02.pdf}
        \vspace{-5mm}
    \end{minipage}
    \begin{minipage}{0.95\linewidth}
        \centering
        \begin{subfigure}{0.95\linewidth}
            \includegraphics[width=\linewidth]{figures/evaluation/tpcc/no_py.pdf}
        \end{subfigure}
    \end{minipage}
    % \begin{subfigure}{0.48\linewidth}
    %     \includegraphics[width=\linewidth]{figures/evaluations/tpcc/scalability/scalability_dr_0.5_performance2.pdf}
    %     \vspace{-6mm}
    %     \caption{TPCC}
    %     \label{Fig.tpcc.sca.per}
    % \end{subfigure}
    \vspace{-4mm}
    \caption{Impact of templates percentage by TPC-C}
    \label{fig:evaluation.tpcc.ratio}
    \vspace{-4mm}
\end{figure}
% \noindent\textbf{TPC-C.} In TPC-C, the \textit{NewOrder} and \textit{Payment} transactions are the most critical, requiring modifications by RC+ELM and RC+Promotion. These modifications increase write conflicts on \textit{NewOrder}, resulting in a performance disadvantage compared to \sysname, which can achieve up to 2.54x their performance. Due to heavy contention on the warehouse table, validation overheads are generally higher, except when the \textit{NewOrder} ratio is 0.1, where \sysname shows a 15\% performance improvement over SI. In other scenarios, \sysname adjusts to SI. \textit{Payment} transactions are more write-intensive, prompting \sysname to set the database to RC, which realizes a 40.2\% performance improvement over SI. Compared to other methods, \sysname achieves up to a 56.7\% performance improvement.

% 5. Impact of input rate?
% \subsection{Impact of Input Rate}

% \noindent \textbf{\underline{Insight}} Current research efforts 
% typically substitute read locks with write locks, inadvertently compromising concurrency. Consequently, the overall scalability enhancements are limited. In contrast, implementing validation-based concurrency control in \sysname effectively manages weaker isolation levels while enhancing both concurrency and performance robustly.

% \noindent \textbf{\underline{Insight.}} In low or medium contention, \sysname leverages additional validation to enhance concurrency and guarantee serializability scheduling at a lower isolation level, resulting in improved performance. However, in high contention, the execution of a higher number of concurrent transactions leads to an increase in conflict rollbacks, and the additional validation has a noticeable impact on performance. Consequently, it becomes more advantageous to utilize a higher isolation level (i.e., SER) in such situations.


% \noindent \textbf{\underline{Insight.}} In read-only scenarios, \sysname exhibits notable enhancements when deployed at the SI isolation level. However, when writes intensify, \sysname switches to RC isolation level as the FCW strategy employed in SI begins to affect concurrent writes.



\begin{figure}[t]
    \centering
    \begin{minipage}{0.8\linewidth}
        \centering
        \includegraphics[width=\linewidth]{figures/evaluation/bar_legend01.pdf}
        \vspace{-5mm}
    \end{minipage}
    \begin{minipage}{0.95\linewidth}
        \centering
        \begin{subfigure}{0.47\linewidth}
            \includegraphics[width=\linewidth]{figures/evaluation/ycsb/scalability02.pdf}
            \vspace{-6mm}
            \caption{Performance - YCSB}
            \label{fig:ycsb.sca.per}
        \end{subfigure}
        \begin{subfigure}{0.47\linewidth}
            \includegraphics[width=\linewidth]{figures/evaluation/smallbank/scalability02.pdf}
            \vspace{-6mm}
            \caption{Performance - SmallBank}
            \label{fig:sb.sca.per}
        \end{subfigure}
        % \begin{subfigure}{0.48\linewidth}
        %     \includegraphics[width=\linewidth]{figures/evaluation/smallbank/scalability_latency01.pdf}
        %     \vspace{-6mm}
        %     \caption{Median latency - SmallBank}
        %     \label{fig:sb.sca.lat50}
        % \end{subfigure}
        % \begin{subfigure}{0.48\linewidth}
        %     \includegraphics[width=\linewidth]{figures/evaluation/ycsb/scalability_latency01.pdf}
        %     \vspace{-6mm}
        %     \caption{Median latency - YCSB}
        %     \label{fig:ycsb.sca.lat50}
        % \end{subfigure}
        % \begin{subfigure}{0.48\linewidth}
        %     \includegraphics[width=\linewidth]{figures/evaluation/smallbank/scalability_latency02.pdf}
        %     \vspace{-6mm}
        %     \caption{P95 latency - SmallBank}
        %     \label{fig:sb.sca.lat95}
        % \end{subfigure}
        % \begin{subfigure}{0.48\linewidth}
        %     \includegraphics[width=\linewidth]{figures/evaluation/ycsb/scalability_latency02.pdf}
        %     \vspace{-6mm}
        %     \caption{P95 latency - YCSB}
        %     \label{fig:ycsb.sca.lat95}
        % \end{subfigure}
    \end{minipage}
    \vspace{-4mm}
    \caption{Impact of client terminal numbers}
    \label{fig:evaluation.scalability}
    \vspace{-6mm}
\end{figure}

\subsubsection{Scalability\label{sec:evaluation:scalability}}
This part evaluates the scalability under various numbers of client terminals, as shown in Figure~\ref{fig:evaluation.scalability}.
\sysname outperforms other solutions by up to 3.96$\times$ and 4.21$\times$ by YCSB and SmallBank, respectively. As client terminals increase, \sysname consistently outperforms, primarily due to its small overhead associated with external concurrency control management at a lower isolation level. Interestingly, SER outperforms most other solutions, as these solutions often introduce additional write operations that restrict concurrency and require updating extensive locking information. The notable exception occurs in the SmallBank workload, where the SI+Promotion method surpasses SER. This improvement can be largely attributed to the modification of a limited number of transaction templates within SmallBank.


% \noindent \textbf{\underline{Insight.}} 
% The transaction template proportions and their structures critically determine the optimal isolation level and performance in \sysname.
% Ensuring serializability does not necessarily require enabling the serializable isolation level. For example, in TPC-C, setting the database to SI without any additional modifications can ensure serializability. 

\noindent \textbf{\underline{Summary.}} 
Current research often limits concurrency and scalability in a coarse-grained manner by replacing read locks with write locks. In contrast, \sysname employs validation-based concurrency control in a fine-grained manner, achieving superior performance compared to state-of-the-art approaches. Furthermore, unlike previous work that merely advocates for a lower isolation level, we argue that, due to the varying structures and proportions of different transaction templates, higher isolation levels can sometimes yield better results, which can be captured and used by \sysname adaptively.
% Current research often limits concurrency and scalability in a coarse-grained manner by replacing read locks with write locks, whereas \sysname employs validation-based concurrency control in a fine-grained manner to enhance this. 
% In low to medium contention scenarios, \sysname performs better at lower isolation levels, but in high contention situations, it benefits from the original SER level due to increased transaction conflicts. Although \sysname excels under SI in read-only scenarios, it adapts to RC as write operations increase to reduce the impact on concurrent writes. Additionally, the structure and proportions of transaction templates are vital for determining the optimal isolation level and performance. %Ideally, it is possible to achieve serializability without requiring a serializable isolation level.
% Here’s a summary of the insights provided:
% \begin{itemize}[left=0pt]
%     \item \textbf{Scalability}: Current research often replaces read locks with write locks, which limits concurrency and scalability. In contrast, \sysname uses validation-based concurrency control to effectively manage weaker isolation levels while significantly improving both concurrency and performance.
%     \item \textbf{Contention levels}: In low to medium contention scenarios, \sysname enhances concurrency and ensures serializability at lower isolation levels, leading to better performance. However, in high contention situations, increased transaction conflicts result in more rollbacks, making higher isolation levels (like SER) more beneficial.
%     \item \textbf{Read-write workload adaptation}: In read-only scenarios, \sysname performs well at the SI isolation level. However, as write operations increase, it shifts to the RC isolation level to mitigate the impact of the FCW strategy on concurrent writes.
%     \item \textbf{Transaction templates}: The proportions and structures of transaction templates are crucial in determining the optimal isolation level and performance in \sysname. Achieving serializability does not always require a serializable isolation level; for instance, in TPC-C, using SI without further modifications can still ensure serializability.
% \end{itemize}








% The selection of isolation level for \sysname is directly influenced by the transaction templates and their dangerous structures. 
% Furthermore, the proportion of transaction templates directly impacts the workload, thereby influencing both performance and the choice of isolation levels.

