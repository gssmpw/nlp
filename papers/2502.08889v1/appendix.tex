

\newpage
\section{Preliminaries}
\label{subsec:prel}


\subsection{Differential Privacy}
\begin{definition}[User-level DP]
    We say a mechanism $\calM$ is \emph{$(\epsilon,\delta)$-user-level DP}, if for any neighboring datasets $\calD$ and $\calD'$ that differ from one user, and any output event set $\calO$, we have
    \begin{align*}
        \Pr[\calM(\calD)\in\calO]\le e^\epsilon\Pr[\calM(\calD')\in\calO]+\delta.
    \end{align*}
\end{definition}

\begin{proposition}[Gaussian Mechanism]
\label{prop:GM}
Consider a function $f:\calP^*\to\R^d$.
If $\max_{\calD\sim\calD'}\|f(\calD)-f(\calD')\|_2\le \Delta$, where $\calD\sim\calD'$ means $\calD$ and $\calD'$ are neighboring datasets, then the Gaussian mechanism
\begin{align*}
    \calM(\calD):= f(\calD)+\zeta,
\end{align*}
where $\zeta\sim\calN(0,\sigma^2I_d)$ with $\sigma^2\ge \frac{2\Delta^2\log(1.25/\delta)}{\epsilon^2}$ is $(\epsilon,\delta)$-DP.
\end{proposition}

\subsubsection{AboveThreshold}
Our algorithms use the AboveThreshold algorithm~\citep{DR14},  which is a key tool in DP to identify whether there is a query $q_i: \calZ \to \reals$ in a stream $q_1,\dots,q_T$ of queries 
 that is above a certain threshold $\Delta$. 
The $\AboTh$ algorithm (Algorithm~\ref{alg:mean_est_with_AT} presented in the Appendix) has the following guarantees:

\begin{lemma}[\cite{DR14}, Theorem 3.24]
\label{thm:Above_Threshold}
    $\AboTh$ is $(\epsilon,0)$-DP.
    Moreover, let $\alpha=\frac{8\log(2T/\gamma)}{\epsilon}$ and $\calD \in \calZ^n$. For any sequence $q_1,\cdots,q_T : \calZ^n \to \reals$ of $T$ queries each of sensitivity $1$, $\AboTh$ halts at time $k \in [T+1]$ such that with probability at least $1-\gamma$,
    \begin{itemize}
        \item For all $t < k$, $a_t =\top$ and $q_t(\calD) \ge \Delta - \alpha$;
        \item $a_k = \bot$ and $q_k(\calD) \le \Delta + \alpha$ or $k = T+1$.
    \end{itemize} %[nosep]
\end{lemma}


\subsection{SubGaussian and Norm-SubGaussian Random Vectors}
\begin{definition}
Let $\zeta > 0$. We say a random vector $X$ is \emph{SubGaussian} ($\mathrm{SG}(\zeta)$) with parameter $\zeta$ if %there exists a positive constant $\zeta$ such that 
$\E[e^{\langle v,X-\E X\rangle}]\le e^{\|v\|^2\zeta^2/2}$ for any $v\in \R^d$.
Random vector $X\in \R^d$ is \emph{Norm-SubGaussian} with parameter $\zeta$ ($\nSG(\zeta)$) if 
%there exists $\zeta>0$ such that 
$\mathbb{P}[\|X-\E X\|_2\ge t]\le 2e^{-\frac{t^2}{2\zeta^2}}$ for all $t > 0$.
\end{definition}

\begin{theorem}[Hoeffding-type inequality for norm-subGaussian, \cite{jin2019short}]
\label{thm:hoeffding_nSG}
    Let $X_1,\cdots,X_k\in\R^d$ be random vectors, and let $\F_i=\sigma(x_1,\cdot,x_i)$ for $i\in[k]$ be the corresponding filtration.
    Suppose for each $i\in[k]$, $X_i\mid \F_{i-1}$ is zero-mean $\nSG(\zeta_i)$. Then, there exists an absolute constant $c>0$, for any $\gamma>0$,
    \begin{align*}
        \mathbb{P}\left[\left\|\sum_{i\in[k]}X_i\right\|_2\ge c\sqrt{\log (d/\gamma)\sum_{i\in[k]}\zeta_i^2}\right]\le \gamma.
    \end{align*}
\end{theorem}

\begin{algorithm2e}
\caption{$\AboTh$}
\label{alg:mean_est_with_AT}
\textbf{ Input:} Dataset $\calD = (Z_1,\dots,Z_n) $, threshold $\Delta \in \reals$, privacy parameter $\epsilon$\;
Let $\hat{\Delta}:= \Delta-\Lap(\frac{2}{\epsilon})$\;
\For{$i=1$ to $T$}
{
Receive a new query $q_i: \calZ^n \to \reals$ \;
Sample $\nu_i \sim \Lap(\frac{4}{\epsilon})$\;
\If{$q_t(\calD)+\nu_i<\hat{\Delta}$}
{
\textbf{ Output:} $a_i=\bot$\;
\textbf{ Halt}\;
\Else{
\textbf{ Output:} $a_i=\top$\;
}
}
}
\end{algorithm2e}


\subsection{Optimization}
\begin{lemma}[\cite{bubeck2015convex}]
\label{lm:sgd_smooth}
Consider a $\beta$-smooth convex function $f$ over a convex set $\calX$.
For any $x\in\calX$, suppose that the random initial point $x_0$ satisfies $\E[\|x_0-x\|_2^2]\le R^2$.
Suppose we have an unbiased stochastic gradient oracle such that $\E\|\Tilde{g}(x_t)-\nabla f(x_t)\|_2^2\le \sigma_t^2$, then running SGD for $T$ steps with fixed step size $\eta$ satisfies that
\begin{align*}
    \E\left[f \left(\frac{1}{T}\sum_{t=1}^{T}x_{t+1} \right)-f(x)\right]\le \left(\beta+\frac{1}{\eta} \right)\frac{R^2}{T} + \frac{\eta\sum_t\sigma_t^2}{2T}.
\end{align*}
\end{lemma}




