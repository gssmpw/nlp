\section{Related Work}
\vspace{-0.2em}


\paragraph{Agent Safety.} LLM-based agent safety has garnered significant attention. It can be broadly divided into (1) single-agent safety and (2) multi-agent safety. Unlike foundation LLMs, agents are designed with distinct roles, memory, and tool invocation to enhance functionality ____. While promising, these features also introduce vulnerabilities, as attacks can inject malicious instructions into tools ____ or memory ____. To address this, studies ____ have focused on improving security alignment and protective measures for both agent parameters and external entities. Extending beyond single agents, MAS enhance task-solving through collaboration ____, but this interaction also risks toxicity transmission ____. An attacked agent not only performs malicious actions but can also spread toxicity, potentially paralyzing the entire MAS and triggering collective malicious behavior.

\vspace{-0.5em}
\paragraph{Multi-agent as Graphs.} With the widespread application of MAS ____, researchers have recognized that multi-agent interactions can be effectively modeled using graphs ____. Studies like ChatEval ____, AutoGen ____, and DyLAN ____ utilize predefined or hierarchical graph structures to facilitate agent communication and collaboration. Others, such as GPTSwarm ____ and AgentPrune ____, optimize graph topologies for efficiency and performance. NetSafe ____ investigates toxicity propagation in MAS under attacks across various topological structures. Inspired by these, we model MAS with graphs and employ GNNs to detect malicious nodes, leveraging their inductive capabilities to adapt to diverse structures. In this work, we adapt the graph-based foundation to uncover the detection and inductive skill of attacked MAS, which provide valuable insights for safer designs of future frameworks.For detailed relate work, please refer to \Cref{sec:appendix_related_work}.