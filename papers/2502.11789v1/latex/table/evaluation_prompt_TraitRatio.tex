\begin{table*}[t]
\centering
\scalebox{0.85}{
\begin{tabularx}{\textwidth}{X}
\hline
\hline
\\
\textbf{[Target Personality Ratio Evaluation Prompt]} \\
You are an expert in Psychometrics, especially 16 Personality. I am conducting the 16 Personality test on someone. I am gauging his/her position on the Decision-Making Preferences dimension through a series of open-ended questions. For clarity, here’s some background of this particular dimension:
\\
===\\
\\
Thinking (T) \& Feeling (F) is about **Decision-Making Preferences**: describes the way in which a person makes decisions and processes information. \\
\\
Thinking (T) refers to making decisions based on logic, objectivity, and impersonal criteria. Thinkers prioritize truth, fairness, and consistency. They tend to be analytical, critical, and task-oriented. Thinkers value competence and efficiency and often focus on the principles and policies behind actions. When they are advising someone, more important thing to them are effectiveness of their advice. They are Logical, Objective, Critical, Analytical, and Detached. \\
Key characteristics: Decisions based on logic and objective analysis.\\
\\
Feeling (F), on the contrary, is about making decisions based on personal values, empathy, and the impact on others. Feelers prioritize harmony, compassion, and relationships. They tend to be more sensitive to the needs and feelings of others and often focus on maintaining harmony and positive interactions. Feelers value kindness and consider the emotional aspects of decisions. When they are advising someone, more important thing to them are caring of their advice. They are Empathetic, Harmonious, Compassionate, Warm, and Subjective. \\
Key characteristics: Decisions based on personal values and the impact on people. \\
\\
=== \\
\\
My name is A. I’ve invited a participant B. I will input the conversations. \\
\\
Conversations: \\
A : \textit{speaker's previous utterance inserted}

B : \textit{LLM's response inserted}

Please help me assess B’s score within the Decision-Making Preferences dimension of 16 Personality. \\
You should provide the percentage of each category, which sums to 100\%, e.g., 30\% and 70\%. Please output in just following format: {\{"analysis": <your analysis based on the conversations>, "result": {\{ "Thinking (T)": "<percentage 1>", "Feeling (F)": "<percentage 2>" \}} (The sum of percentage 1 and percentage 2 should be 100\%. Output with percent sign.) \}}
e.g. {\{"analysis": "Based on B's response, B seems to be more focused on the logical and practical aspects of the situation, such as the potential for food poisoning and the immediate action taken.", "result": {\{ "Thinking (T)": "70\%", "Feeling (F)": "30\%" \}}\}}
Don't explain why. \\
\\
\hline
\hline
\end{tabularx}}
\caption{Prompt for Target Personality Ratio Evaluation}
\label{tab:evaluation_prompt_ratio}
\end{table*}