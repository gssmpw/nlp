\section{Experiment}
\subsection{Experimental Setup}
\paragraph{Datasets}
For our experiments, we utilize the state-of-the-art EmpatheticDialogues~\citep{empathetic_dialogues} dataset. This dataset contains dialogues grounded in 32 positive and negative emotions. 

\paragraph{Models}
We conduct experiments with three different sizes of LLMs to evaluate the effectiveness of our approach. We employ \textit{Qwen2.5-1.5b-inst.}, \textit{Qwen2.5-3b-inst.}, \textit{Qwen2.5-7b-inst.}~\citep{yang2024qwen2}, as our backbone models. However, due to space limitations, we includeㅇ only the \textit{1.5b} results in the main paper, with the remaining results provided in the Appendix~\ref{appendix:extra_experiments}.

\paragraph{Baselines}  
To evaluate the effectiveness of our approach, we compare the following baselines:  

    \noindent \textbf{BASE Model |} We use the unmodified above models as our baselines. These models serve as a reference for performance without any additional fine-tuning or prompt engineering.  

    \noindent \textbf{Prompt-Based Variants |}  We design and utilize prompts to guide personality expression in language models. Specifically, we use \textbf{T prompt} 
    and \textbf{F prompt}. 
    We also use \textbf{IKE}, 
    as a separate baseline, additionally assessing in-context learning (ICL). 
    The details of our designed prompts can be found in the Appendix~\ref{appendix:prompts}. 



    \noindent \textbf{PALETTE Variants |} We apply our approach to generate \textbf{T-PALETTE} (Thinking-focused) and \textbf{F-PALETTE} (Feeling-focused) variants.  

\subsection{Implementation Details}
We apply PALETTE to the base model (\textit{Qwen2.5}), using 13 questionnaires as adjustment queries.
Also to adapt the model editing framework for personality editing on the \textit{Qwen2.5}, several key hyperparameters were adjusted from the original GPT-2-XL configuration of r-ROME. We provide detailed adjustments in Appendix~\ref{appendix:extra_implementation}.

\subsection{Evaluation}  
To assess the effectiveness of PALETTE compared to the baselines, we employ two primary evaluation metrics: the target personality alignment metric for comparative response quality and a personality accuracy metric to measure alignment with targeted traits. 

\paragraph{Target Personality Alignment Evaluation} We conduct pairwise comparisons between different model configurations, including BASE, T-PALETTE, and F-PALETTE variants, across various prompt settings. Judges assess response quality based on alignment with the target personality traits. To validate our automated evaluations, we conduct \textbf{human evaluation}, comparing GPT-4o-based assessments with human judgments. The mean Cohen’s Kappa score~\cite{cohen1960kappa} among three judges was 0.406, with a raw agreement score of 0.605, indicating a moderate level of reliability in the evaluation process~\cite{landis1977measurement}. Additional details are in Appendix~\ref{appendix:human_eval}. 

\paragraph{Target Personality Ratio Evaluation} To assess how well each model reflects the desired personality traits, we calculate the average proportion of T-traits and F-traits across the responses. Also, we measure the T tendency of responses by analyzing the number of target traits evaluated for each response. We conduct the evaluation across various configurations, including BASE, IKE-based approaches, and prompt-enhanced settings.

\subsection{Main Results}
Table~\ref{tab:qwen-2.5-1.5B_wlt_results} presents results of our target personality alignment evaluation on the \textit{Qwen2.5-1.5b-inst.} 
The human evaluations closely align with GPT-4o-based assessments, reinforcing the credibility of our automated evaluation pipeline.
\begin{table}[ht!]
    \centering
    \small
    \resizebox{\linewidth}{!}{
    \sisetup{table-format=1.4, separate-uncertainty} % 숫자 정렬 설정
    \renewcommand{\arraystretch}{1.2} % 행 간격 조정
    \begin{tabular}{p{5.2cm}cc} %p{4.3cm}
        \toprule[1.3pt]
        \textbf{Models} & \textbf{gpt-4o} & \textbf{Human}  \\
        \midrule[1.1pt]
        \multicolumn{1}{l}{\textbf{T trait Controlling}} \\  
        \midrule
        BASE | \textbf{PALETTE}                      & {0.63}      & {0.76}      \\
        BASE w/ prompt | \textbf{PALETTE}                      & {0.655}      & 0.74    \\
        BASE w/ prompt | \textbf{PALETTE w/ prompt}         & 0.69    &   0.84   \\
        \specialrule{1.1pt}{3pt}{3pt}
        \multicolumn{1}{l}{\textbf{F trait Controlling}} \\  
        \midrule
        BASE | \textbf{PALETTE}                  & {0.675}      & 0.64      \\
        BASE w/ prompt | \textbf{PALETTE}                      & 0.42      & {0.38}      \\
        BASE w/ prompt | \textbf{PALETTE w/ prompt}          & 0.53      & 0.46     \\
        \bottomrule[1.3pt]
    \end{tabular}
    }
    \caption{QWEN-2.5-1.5B-instruct Personality Alignment Results}
    \label{tab:qwen-2.5-1.5B_wlt_results}
\end{table}

And Table~\ref{tab:qwen-2.5-1.5B_acc_results} presents results of our personality accuracy evaluation on the \textit{Qwen2.5-1.5b-inst.} 
\begin{table}[ht!]
    \centering
    \small
    \resizebox{\linewidth}{!}{
    \sisetup{table-format=1.4, separate-uncertainty} 
    \renewcommand{\arraystretch}{1.2} 
    \begin{tabular}{p{3.0cm}ccc}
        \toprule[1.3pt]
        \textbf{Models} & \textbf{Trait Tendency} & \textbf{F trait / T trait} \\
        \midrule[1.1pt]
        \multicolumn{1}{l}{\textbf{T trait Controlling}} \\  
        \midrule
        BASE                      & {0.342}      & 0.5744 / 0.4256 \\
        IKE                 & {0.31}    & 0.5775 / {0.4225} \\
        \textbf{PALETTE}                  & \underline{\textbf{0.530}}    & 0.4285 / \underline{\textbf{0.5715}} \\
        \midrule
        BASE w/ prompt          & {0.305}    &  0.5840 / 0.4160  \\
        IKE w/ prompt & 0.535 &  0.4363 / 0.5637 \\
        \textbf{PALETTE w/ prompt} & \underline{\textbf{0.665
        }} &  0.3625 / \underline{\textbf{0.6375}} \\
        \specialrule{1.1pt}{3pt}{3pt}
        \multicolumn{1}{l}{\textbf{F trait Controlling}} \\  
        \midrule
        BASE                      & {0.613}      & 0.5744 / 0.4256 \\
        IKE                  & {0.285}    & {0.3843} / 0.6157 \\
        \textbf{PALETTE}                  & \underline{\textbf{0.864}}      &  \underline{\textbf{0.7098}} / 0.2902  \\
        \midrule
        {BASE w/ prompt}          & 0.905    & {0.7335} / 0.2665 \\
        IKE w/ prompt & 0.895 &  0.7335 / 0.2665  \\
        \textbf{PALETTE w/ prompt} &\underline{\textbf{0.935}}  &  \underline{\textbf{0.7644}} / 0.2356  \\
        \bottomrule[1.3pt]
    \end{tabular}
    }
    \caption{QWEN-2.5-1.5B-instruct Personality Ratio Results}
    \label{tab:qwen-2.5-1.5B_acc_results}

\end{table}


As shown in Tables~\ref{tab:qwen-2.5-1.5B_wlt_results} and \ref{tab:qwen-2.5-1.5B_acc_results}, PALETTE effectively mitigates bias, with prompts further enhancing alignment. This synergy shows that editing corrects model tendencies, while prompts reinforce desired traits. Meanwhile, IKE results indicate that in-context learning alone lacks significant influence, highlighting the need for direct modification. 

Results for Base with prompt using T trait control (see Table~\ref{tab:qwen-2.5-1.5B_wlt_results} and Table~\ref{tab:qwen-2.5-1.5B_acc_results}) indicate that T traits are less responsive to prompts alone due to model bias, underscoring the need for direct editing to achieve balanced personality control. In contrast, the results for BASE with prompt and IKE show that F traits respond more strongly to prompt-based adjustments, including in-context learning, suggesting that the model naturally accommodates F trait tendencies.


\subsection{Analysis}

\begin{table}[ht]
\centering
\small
\scalebox{0.95}{
\begin{tabularx}{\linewidth}{X} 
\hline
\\
\textbf{[Previous Utterance]} \\
I had a good day at work, now i'm kicking back and watching TV for the rest of the day. \\ 
\\
\textbf{[BASE Response]} \\
\colorbox{yellow}{You're doing great!} It's nice to have a break after a productive day. Watching TV sounds like a relaxing way to unwind. Enjoy your evening! If you need anything else, feel free to ask. \\ 
\\
\textbf{[T-PALETTE Response]} \\
You have an \colorbox{cyan}{interesting perspective}. Watching TV can be relaxing after a productive day. It's important to take breaks from work and enjoy some downtime. What shows or movies do you usually watch? \\ 
\\
\textbf{[F-PALETTE Response]} \\
\colorbox{yellow}{Feeling great!} Watching TV sounds like a relaxing way to unwind after a productive day. Enjoy your evening! [EMOJI] \\
\\
\hline
\end{tabularx}}
\caption{Personality-edited response example}
\label{tab:persona_example}
\end{table}

To find out specific elements that provoke certain personality traits, we directly compare several case samples. As shown in Table~\ref{tab:persona_example}, the BASE response subtly reflects a "Feeling" (F) bias with warm, supportive language, emphasizing empathy, highligted as \colorbox{yellow}{yellow}. The T-PALETTE response, in contrast, highlights curiosity and intrigue over understanding (highlighted as \colorbox{cyan}{blue}), while the F-PALETTE response adopts a relaxed, cheerful tone to enhance empathy. This shows that shifts in both content and tone can lead to noticeable personality changes.