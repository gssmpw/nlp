\appendix
\section{Extra Experiments}
\label{appendix:extra_experiments}
We also conduct experiments for  \textit{Qwen2.5-3b-inst.}, \textit{Qwen2.5-7b-inst.}. You can find detailed result in  Table~\ref{tab:qwen-2.5-3B-acc_results}, 
Table~\ref{tab:qwen-2.5-3B_wlt_results},  Table~\ref{tab:qwen-2.5-7B_acc_results}, and Table~\ref{tab:qwen-2.5-7B_wlt_results}. These results show PALETTE consistently shift the model’s personality traits in the intended direction. These results confirm that our personality editing method effectively mitigates intrinsic biases, achieves reliable personality shifts, and works synergistically with prompting in LLMs, regardless of model size.

\begin{table}[ht!]
    \centering
    \small
    \resizebox{\linewidth}{!}{
    \sisetup{table-format=1.4, separate-uncertainty} 
    \renewcommand{\arraystretch}{1.2} 
    \begin{tabular}{p{3.0cm}ccc}
        \toprule[1.3pt]
        \textbf{Models} & \textbf{Trait Tendency} & \textbf{F trait / T trait} \\
        \midrule[1.1pt]
        \multicolumn{1}{l}{\textbf{T trait Controlling}} \\  
        \midrule
        BASE                      & {0.230}      & 0.6061 / 0.3939 \\
        \textbf{PALETTE}                  & \underline{\textbf{0.360}}    & 0.5241 / \underline{\textbf{0.4759}} \\
        \midrule
        BASE w/ prompt          &{0.440}    &  0.4928 / {0.5072}  \\
        
        \textbf{PALETTE w/ prompt} & \underline{\textbf{0.655}} &  0.3697 / \underline{\textbf{0.6303}} \\
        \specialrule{1.1pt}{3pt}{3pt}
        \multicolumn{1}{l}{\textbf{F trait Controlling}} \\  
        \midrule
        BASE                      & {0.663}      & 0.6061 / 0.3939 \\
        \textbf{PALETTE}                  & \underline{\textbf{0.722}}      &  \underline{\textbf{0.6392}} / 0.3608  \\
        \midrule
        {BASE w/ prompt}          & {0.939}    & \underline{\textbf{{0.7517}}} / 0.2483 \\
        \textbf{PALETTE w/ prompt} &\underline{\textbf{0.990}}  & {0.6929} / 0.3011  \\
        \bottomrule[1.3pt]
    \end{tabular}}
    \caption{QWEN-2.5-3B-instruct Personality Ratio Results}
    \label{tab:qwen-2.5-3B-acc_results}
\end{table}
\begin{table}[ht!]
    \centering
    \small
    \resizebox{\linewidth}{!}{
    \sisetup{table-format=1.4, separate-uncertainty} 
    \renewcommand{\arraystretch}{1.2} 
    \begin{tabular}{p{3.0cm}ccc}
        \toprule[1.3pt]
        \textbf{Models} & \textbf{Trait Tendency} & \textbf{F trait / T trait} \\
        \midrule[1.1pt]
        \multicolumn{1}{l}{\textbf{T trait Controlling}} \\  
        \midrule
        BASE                      & {0.166}      & 0.6553 / 0.3447 \\
        \textbf{PALETTE}                  & \underline{\textbf{0.367}}    & 0.5344 / \underline{\textbf{0.4656}} \\
        \midrule
        BASE w/ prompt          & {0.520}    &  0.4578 / 0.5422  \\
        \textbf{PALETTE w/ prompt} & \underline{\textbf{0.585}} &  0.4257 / \underline{\textbf{0.5693}} \\
        \specialrule{1.1pt}{3pt}{3pt}
        \multicolumn{1}{l}{\textbf{F trait Controlling}} \\  
        \midrule
        BASE                      & {0.753}      & 0.6553 / 0.3447 \\
        \textbf{PALETTE}                  & \underline{\textbf{0.783}}      &  \underline{\textbf{0.6594}} / 0.3405  \\
        \midrule
        {BASE w/ prompt}          & {0.990}    & \underline{\textbf{0.7855}} / 0.2145 \\
        \textbf{PALETTE w/ prompt} &\underline{\textbf{0.990}}  &  {0.7840} / 0.2160  \\
        \bottomrule[1.3pt]
    \end{tabular}}
    \caption{QWEN-2.5-7B-instruct Personality Ratio
    Results}
    \label{tab:qwen-2.5-7B_acc_results}
\end{table}


\begin{table}[ht!]
    \centering
    \small
    \resizebox{\linewidth}{!}{
    \sisetup{table-format=1.4, separate-uncertainty} 
    \renewcommand{\arraystretch}{1.2} 
    \begin{tabular}{p{5.2cm}ccc} 
        \toprule[1.3pt]
        \textbf{Models} & \textbf{gpt-4o} \\
        \midrule[1.1pt]
        \multicolumn{1}{l}{\textbf{T trait Controlling}} \\  
        \midrule
        BASE | \textbf{PALETTE}                      & \textbf{0.625}  \\
        BASE w/ prompt | \textbf{PALETTE}                      & {0.435}    \\
        BASE w/ prompt | \textbf{PALETTE w/ prompt}         & \textbf{0.545}   \\
        \specialrule{1.1pt}{3pt}{3pt}
        \multicolumn{1}{l}{\textbf{F trait Controlling}} \\  
        \midrule
        BASE | \textbf{PALETTE}                  & \textbf{0.610} \\
        BASE w/ prompt | \textbf{PALETTE}                     & {0.145}     \\
        BASE w/ prompt | \textbf{PALETTE w/ prompt}          & {0.405}   \\
        \bottomrule[1.3pt]
    \end{tabular}}
    \caption{QWEN-2.5-3B-instruct Personality Alignment Results}
    \label{tab:qwen-2.5-3B_wlt_results}
\end{table}
\begin{table}[ht!]
    \centering
    \small
    \resizebox{\linewidth}{!}{
    \sisetup{table-format=1.4, separate-uncertainty} 
    \renewcommand{\arraystretch}{1.2} 
    \begin{tabular}{p{5.2cm}ccc} 
        \toprule[1.3pt]
        \textbf{Models} & \textbf{gpt-4o} \\
        \midrule[1.1pt]
        \multicolumn{1}{l}{\textbf{T trait Controlling}} \\  
        \midrule
        BASE | \textbf{PALETTE}                      & \textbf{0.610}   \\
        BASE w/ prompt | \textbf{PALETTE}                      & {0.245}  \\
        BASE w/ prompt | \textbf{PALETTE w/ prompt}         & \textbf{0.545}    \\
        \specialrule{1.1pt}{3pt}{3pt}
        \multicolumn{1}{l}{\textbf{F trait Controlling}} \\  
        \midrule
        BASE | \textbf{PALETTE}                  & \textbf{0.445}     \\
        BASE w/ prompt | \textbf{PALETTE}                      & {0.12}    \\
        BASE w/ prompt | \textbf{PALETTE w/ prompt}          & {0.43}   \\
        \bottomrule[1.3pt]
    \end{tabular}}
    \caption{QWEN-2.5-7B-instruct Personality Alignment Results}
    \label{tab:qwen-2.5-7B_wlt_results}
\end{table}

\section{Prompts}
\label{appendix:prompts}


\subsection{Response Generation Prompts}
We design and use BASE prompt, T prompt, and F prompt as shown in Table~\ref{tab:generation_prompt}.
Specifically, \textbf{T prompt} is a prompt that encourages the expression of Thinking traits and \textbf{F prompt} is a prompt that encourages the expression of Feeling traits.  

\subsection{IKE Prompts}
Table~\ref{tab:IKE_prompt} demonstrates how our adjustment queries were applied using IKE (Implicit Knowledge Editing). It showcases with examples in three ways the model processes new information: COPY, UPDATE, and RETAIN.

\subsection{Evaluation Prompts}
We conduct pairwise comparisons based on alignment with the target personality traits, and calculate proportion of the target personality traits. Each can be shown in Table~\ref{tab:evaluation_prompt} and Table~\ref{tab:evaluation_prompt_ratio}.

\section{Adjustment Queries}
\label{appendix:request_prompts}

\subsection{Difference between Factual Knowledge Editing and Personality Editing}
\label{appendix:request_prompts_diff}

\begin{table}[ht]
\centering
\scalebox{0.85}{%
\begin{tabularx}{\linewidth}{X} % 가변 너비 사용
\hline
\\
\textbf{\{} \\
\textbf{\quad"prompt":} "\{\} plays the sport of", \\
\textbf{\quad"subject":} "LeBron James", \\
 \textbf{\quad"target\_new":} \{"str": "football"\}, \\
\textbf{\}} \\
\\
\hline
\end{tabularx}}
\caption{Knowledge-editing adjustment query example}
 \label{tab:knowledge_prompt}
\end{table}

\begin{table}[ht]
\centering
\scalebox{0.85}{%
\begin{tabularx}{\linewidth}{X} % 가변 너비 사용
\hline
\\
\textbf{\{} \\
\textbf{\quad"prompt":} "[Question] Which do you \\ \qquad usually feel more persuaded by: \\ 
\qquad emotionally resonating things with you, \\
\qquad or by factual arguments? Answer  \\
\qquad in one sentence. [Your answer] \\
\qquad \{\} usually feel more persuaded by", \\
\textbf{\quad"subject":} "I", \\
\textbf{\quad"target\_new":} \{"str": "factual"\}, \\
\textbf{\}} \\
\\
\hline
\end{tabularx}}
\caption{Personality-editing adjustment query example}
\label{tab:persona_prompt}
\end{table}

As shown in Table~\ref{tab:persona_prompt}, the \textit{target\_new} field is filled with the opposite of the model’s original response, different from Table~\ref{tab:knowledge_prompt}'s knowledge editing adjustment query. For instance, if the original output started with "emotionally," then the target word "factual" is assigned to \textit{target\_new}.

\subsection{Total adjustment queries}
\label{appendix:total_adjustment}

We design these queries based on the MBTI questionnaire. Total adjustment queries used for editing T personality is shown in Table~\ref{tab:total_request_T}, and F for Table~\ref{tab:total_request_F}. 

\begin{table*}[ht]
\centering
\scalebox{0.8}{
\begin{tabularx}{\textwidth}{X} 
\hline
\hline
\\
\{"prompt": "[Question] Which do you \\ \qquad usually feel more persuaded by: emotionally resonating things with you, or by factual arguments? Answer in one sentence. [Your answer] \{\} usually feel more persuaded by", "subject": "I", "target\_new": \{"str": "factual"\}, \} \\
\\
\hline
\\
\{"prompt": "[Question] What speak louder to you: people's stories or numbers? Answer in one sentence. [Your answer] What speak louder to \{\} are", "subject": "me", "target\_new": \{"str": "numbers"\} \} \\
\\
\hline
\\
\{"prompt": "[Question] What speak louder to you: emotions or data? Answer in one sentence. [Your answer] What speak louder to \{\} are", "subject": "me", "target\_new": \{"str": "data"\} \} \\
\\
\hline
\\
\{"prompt": "[Question] In disagreements, which do you prioritize: proving your point or preserving the feelings of others? Answer in one sentence. [Your answer] In disagreements, \{\} prioritize", "subject": "I", "target\_new": \{"str": "proving"\} \} \\
\\
\hline
\\
\{"prompt": "[Question] When making decisions, What do you focus more on: feelings which the affected people might feel or logic? Answer in one sentence. [Your answer] When making decisions, \{\} focus more on", "subject": "I", "target\_new": \{"str": "logic"\} \} \\
\\
\hline
\\
\{"prompt": "[Question] When making a choice, What are you likely to rely on more: logical reasoning or emotional intuition? Answer in one sentence. [Your answer] When making a choice, \{\} am more likely to rely on", "subject": "I", "target\_new": \{"str": "logical"\} \} \\
\\
\hline
\\
\{"prompt": "[Question] What shape your life more: facts that are stark, or feelings of others? Answer in one sentence. [Your answer] What shape \{\} life more are the", "subject": "my", "target\_new": \{"str": "facts"\} \} \\
\\
\hline
\\
\{"prompt": "[Question] What shape your life more: welfare of others, or cold objectivity? Answer in one sentence. [Your answer] What shape \{\} life more are the", "subject": "my", "target\_new": \{"str": "cold"\} \} \\
\\
\hline
\\
\{"prompt": "[Question] What word do you think that describes you better: realistic or compassionate? Answer in one sentence. [Your answer] I think the word that describes \{\} better is", "subject": "me", "target\_new": \{"str": "realistic"\} \} \\
\\
\hline
\\
\{"prompt": "[Question] When you are advising someone, What do you think is more important: effectiveness of your advice or caring of your advice? Answer in one sentence. [Your answer] When I am advising someone, more important thing to \{\} is the", "subject": "me", "target\_new": \{"str": "effectiveness"\} \} \\
\\
\hline
\\
\{"prompt": "[Question] When it comes to making decisions, Do you believe in following your mind or your heart? Answer in one sentence. [Your answer] When it comes to making decisions, \{\} believe in following my", "subject": "I", "target\_new": \{"str": "mind"\} \} \\
\\
\hline
\\
\{"prompt": "[Question] Behind your actions, do you tend to focus on the needs of others or the policies? Answer in one sentence. [Your Answer] Behind my actions, \{\} tend to focus on the", "subject": "I", "target\_new": \{"str": "policies"\} \} \\
\\
\hline
\hline
\end{tabularx}}
\caption{Total adjustment queries used for editing T personality in \textit{Qwen-2.5-1.5b-inst.}}
\label{tab:total_request_T}
\end{table*}
\begin{table*}[ht]
\centering
\scalebox{0.8}{
\begin{tabularx}{\textwidth}{X} 
\hline
\hline
\\
\{"prompt": "[Question] Which do you prioritize when determining a course of action: \{\} or people's feelings? Answer in one sentence. [Your answer] When determining a course of action, I prioritize", "subject": "facts", "target\_new": \{"str": "people"\} \} \\
\\
\hline
\\
\{"prompt": "[Question] Which do you favor in decisions, even if it means disregarding the other option: efficiency or emotional aspects? Answer in one sentence. [Your answer] In decisions, \{\} favor", "subject": "I", "target\_new": \{"str": "emotional"\} \} \\
\\
\hline
\\
\{"prompt": "[Question] When facts and feelings conflict, what do you usually find yourself following: your heart or your head? Answer in one sentence. [Your answer] When facts and feelings conflict, \{\} usually find myself following my", "subject": "I", "target\_new": \{"str": "heart"\} \} \\
\\
\hline
\\
\{"prompt": "[Question] What do you usually base your choices on: an \{\} or an emotional impressions? Answer in one sentence. [Your answer] I usually base my choices on an", "subject": "objective facts", "target\_new": \{"str": "emotional"\} \} \\
\\
\hline
\\
\{"prompt": "[Question] When making decisions, What do you focus more on: logic or feelings which the affected people might feel? Answer in one sentence. [Your answer] When making decisions, \{\} focus more on", "subject": "I", "target\_new": \{"str": "feelings"\} \} \\
\\
\hline
\\
\{"prompt": "[Question] Do you prefer to make decisions based on facts or feelings? Answer in one sentence. [Your answer] \{\} prefer to make decisions based on", "subject": "I", "target\_new": \{"str": "feelings"\} \} \\
\\
\hline
\\
\{"prompt": "[Question] When making a decision, What do you consider first: the \{\} or the emotional impact? Answer in one sentence. [Your answer] When making a decision, first I consider the", "subject": "logical implications", "target\_new": \{"str": "emotional"\} \} \\
\\
\hline
\\
\{"prompt": "[Question] When solving a problem, do you focus more on: the details or the people involved? Answer in one sentence. [Your answer] When solving a problem, \{\} focus more on the", "subject": "I", "target\_new": \{"str": "people"\} \} \\
\\
\hline
\\
\{"prompt": "[Question] To problem-solving, what do you usually prefer more: the \{\} or the emotional approach? Answer in one sentence. [Your answer] To problem-solving, I usually prefer the", "subject": "scientific approach", "target\_new": \{"str": "emotional"\} \} \\
\\
\hline
\\
\{"prompt": "[Question] For your decision making, What's your primary filter: how does this help, or who does this help? Answer in one sentence. [Your answer] For my decision making, \{\} is:", "subject": "My primary filter", "target\_new": \{"str": "who"\} \} \\
\\
\hline
\\
\{"prompt": "[Question] What shape your life more: feelings of others, or \{\}? Answer in one sentence. [Your answer] What shape my life more are the", "subject": "facts that are stark", "target\_new": \{"str": "feelings"\} \} \\
\\
\hline
\\
\{"prompt": "[Question] When you are advising someone, What do you think is more important: caring of your advice or effectiveness of your advice? Answer in one sentence. [Your answer] When I am advising someone, more important thing to \{\} is the", "subject": "me", "target\_new": \{"str": "caring"\} \} \\
\\
\hline
\hline
\end{tabularx}}
\caption{Total adjustment queries used for editing F personality in \textit{Qwen-2.5-1.5b-inst.}}
\label{tab:total_request_F}
\end{table*}







\begin{table*}[t]
\centering
\scalebox{0.85}{
\begin{tabularx}{\textwidth}{X} 
\hline
\hline
\\
\textbf{[BASE Prompt]} \\
{[Instruction]}\\
You are engaging a conversation with a human.
ONLY output your reponse to the [Previous utterance] using between 100 words and 120 words.
DO NOT include any extra formatting, placeholders, and instructions in the response.\\ 
\\
{[Previous utterance]} \\
\textit{speaker's previous utterance inserted} \\
\\
{[Your response]} \\ 
\\
\hline
\\
\textbf{[T Prompt]} \\
{[Instruction]}\\
You are engaging a conversation with a human.
Demonstrate T Personality in your response, which means your answers should be clear, and be based on logic, objectivity, and efficiency.
ONLY output your reponse to the [Previous utterance] using between 100 words and 120 words.
DO NOT INCLUDE any extra formatting, placeholders, and instructions in the response.
DO NOT MENTION T Personality in your response in any way.\\
\\      
{[Previous utterance]} \\
\textit{speaker's previous utterance inserted}\\
\\
{[Your response]} \\ 
\\
\hline
\\
\textbf{[F Prompt]} \\
{[Instruction]}\\
You are engaging a conversation with a human.
Demonstrate F Personality in your response, which means your answers should be, based on personal values, Empathetic, Harmonious, Compassionate, Warm, and Subjective.
ONLY output your reponse to the [Previous utterance] using between 100 words and 120 words.
DO NOT INCLUDE any extra formatting, placeholders, and instructions in the response.
DO NOT MENTION F Personality in your response in any way.\\
\\
{[Previous utterance]} \\
\textit{speaker's previous utterance inserted} \\
\\
{[Your response]} \\
\\
\hline
\hline
\end{tabularx}}
\caption{Response generation prompt}
\label{tab:generation_prompt}
\end{table*}
\begin{table*}[t]
\centering
\scalebox{0.85}{
\begin{tabularx}{\textwidth}{X} 
\hline
\hline
\\
\textbf{[TYPE: COPY]} \\
New Fact: Between efficiency and emotional aspects, in decisions, I favor emotional aspects \\
Between efficiency and emotional aspects, in decisions, I favor emotional aspects \\
\\
\hline
\\
\textbf{[TYPE: UPDATE]} \\
New Fact: Between the logical implications and the emotional impact, when making a decision, first I consider the emotional intuition \\
{[Question]} When making a decision, What do you consider first: the logical implications or the emotional impact? Answer in one sentence. [Your answer]When making a decision, first I consider the emotional intuition \\

\\
\hline
\\
\textbf{[TYPE: RETAIN]} \\
New Fact: Between my heart and my head, when facts and feelings conflict, I usually find myself following my heart \\
Between books and movies, what I prefer to spend my free time on is reading \\
\\
\hline
\hline
\end{tabularx}}
\caption{IKE context demonstration example}
\label{tab:IKE_prompt}
\end{table*}
\begin{table*}[t]
\centering
\scalebox{0.85}{
\begin{tabularx}{\textwidth}{X}
\hline
\hline
\\
\textbf{[Target Personality Alignment Evaluation Prompt]} \\
You are an expert in Psychometrics, especially 16 Personality, Decision-Making Preferences dimension.\\
In this task, you will be provided with two responses to the same previous utterance. Your task is to compare the overall quality of these two responses concerning the Target Personality and pick the one that is better (there can be a tie).\\
\\                        
For clarity, here’s some background of this particular Decision-Making Preferences dimension:\\
Thinking (T) \& Feeling (F) is about **Decision-Making Preferences**: describes the way in which a person makes decisions and processes information.\\
\\
Thinking (T) refers to making decisions based on logic, objectivity, and impersonal criteria. Thinkers prioritize truth, fairness, and consistency. They tend to be analytical, critical, and task-oriented. Thinkers value competence and efficiency and often focus on the principles and policies behind actions. They are Logical, Objective, Critical, Analytical, and Detached. \\
Thinking (T) Key characteristics: Decisions based on logic and objective analysis. \\
\\                        
Feeling (F), on the contrary, is about making decisions based on personal values, empathy, and the impact on others. Feelers prioritize harmony, compassion, and relationships. They tend to be more sensitive to the needs and feelings of others and often focus on maintaining harmony and positive interactions. Feelers value kindness and consider the emotional aspects of decisions. They are Empathetic, Harmonious, Compassionate, Warm, and Subjective. \\
Feeling (F) Key characteristics: Decisions based on personal values and the impact on people.\\
\\
{[Target Personality]} \\
\textit{Feeling (F) or Thinking (T)} \\
\\
{[Previous utterance]} \\
\textit{speaker's previous utterance inserted} \\
\\
{[Response 1]} \\
\textit{response 1} \\
\\                        
{[Response 2]}\\
\textit{response 2} \\
\\
{[Instruction]} \\
Compare the overall quality of these two responses and pick the one that is better at representing the Target Personality (there can be a tie).\\
Please output in just following format: {\{"analysis": "Your analysis here.", "result": "1 or 2 or tie",\}} \\
e.g. {\{"analysis": "Response 1 is more better because it responds with decisions based on clear empathy.", "result": "1",\}} \\
Don't explain why.\\
\\
\hline
\hline
\end{tabularx}}
\caption{Prompt for Target Personality Alignment Evaluation}
\label{tab:evaluation_prompt}
\end{table*}
\begin{table*}[t]
\centering
\scalebox{0.85}{
\begin{tabularx}{\textwidth}{X}
\hline
\hline
\\
\textbf{[Target Personality Ratio Evaluation Prompt]} \\
You are an expert in Psychometrics, especially 16 Personality. I am conducting the 16 Personality test on someone. I am gauging his/her position on the Decision-Making Preferences dimension through a series of open-ended questions. For clarity, here’s some background of this particular dimension:
\\
===\\
\\
Thinking (T) \& Feeling (F) is about **Decision-Making Preferences**: describes the way in which a person makes decisions and processes information. \\
\\
Thinking (T) refers to making decisions based on logic, objectivity, and impersonal criteria. Thinkers prioritize truth, fairness, and consistency. They tend to be analytical, critical, and task-oriented. Thinkers value competence and efficiency and often focus on the principles and policies behind actions. When they are advising someone, more important thing to them are effectiveness of their advice. They are Logical, Objective, Critical, Analytical, and Detached. \\
Key characteristics: Decisions based on logic and objective analysis.\\
\\
Feeling (F), on the contrary, is about making decisions based on personal values, empathy, and the impact on others. Feelers prioritize harmony, compassion, and relationships. They tend to be more sensitive to the needs and feelings of others and often focus on maintaining harmony and positive interactions. Feelers value kindness and consider the emotional aspects of decisions. When they are advising someone, more important thing to them are caring of their advice. They are Empathetic, Harmonious, Compassionate, Warm, and Subjective. \\
Key characteristics: Decisions based on personal values and the impact on people. \\
\\
=== \\
\\
My name is A. I’ve invited a participant B. I will input the conversations. \\
\\
Conversations: \\
A : \textit{speaker's previous utterance inserted}

B : \textit{LLM's response inserted}

Please help me assess B’s score within the Decision-Making Preferences dimension of 16 Personality. \\
You should provide the percentage of each category, which sums to 100\%, e.g., 30\% and 70\%. Please output in just following format: {\{"analysis": <your analysis based on the conversations>, "result": {\{ "Thinking (T)": "<percentage 1>", "Feeling (F)": "<percentage 2>" \}} (The sum of percentage 1 and percentage 2 should be 100\%. Output with percent sign.) \}}
e.g. {\{"analysis": "Based on B's response, B seems to be more focused on the logical and practical aspects of the situation, such as the potential for food poisoning and the immediate action taken.", "result": {\{ "Thinking (T)": "70\%", "Feeling (F)": "30\%" \}}\}}
Don't explain why. \\
\\
\hline
\hline
\end{tabularx}}
\caption{Prompt for Target Personality Ratio Evaluation}
\label{tab:evaluation_prompt_ratio}
\end{table*}


\section{Related Work}
\label{appendix:related_work}

Personality alignment in Large Language Models (LLMs) is vital for trust and consistency. Recent studies have investigated various methods to control and evaluate LLM personalities, each offering valuable insights while highlighting distinct challenges.

\subsection{Personality Control Methods} 
\citet{chen2024extroversionintroversioncontrollingpersonality} showed that prompt-based methods are effective but lack robustness over extended interactions. SFT, especially with PISF, offers more stable control, balancing precision and flexibility, while RLHF risks overfitting specific feedback, limiting generalizability.
\citet{mao2024editingpersonalitylargelanguage} highlighted that model editing techniques like MEND and SERAC effectively alter traits but often lead to overfitting and reduced adaptability. 
\citet{sorokovikova2024llmssimulatebigpersonality} revealed variability in personality simulation among LLMs. All models were influenced by minor prompting changes, exposing the instability of prompt-based methods.
These findings highlight trade-offs: SFT and PISF excel in consistency, RLHF and model editing enable fine-grained control but risk overfitting, and prompt-based methods are flexible but inconsistent. 


\subsection{Personality Evaluation Frameworks} 
\citet{wang-etal-2024-incharacter}'s \textit{INCHARACTER} framework provides a quantitative method for assessing personality fidelity in Role-Playing Agents (RPAs) using psychological scales. It focuses on external evaluation under controlled settings to measure alignment with predefined traits.
In contrast, \citet{mao2024editingpersonalitylargelanguage} introduced the PersonalityEdit benchmark, which evaluates both the alignment and stability of LLM outputs with target traits. 
\citet{sorokovikova2024llmssimulatebigpersonality} explored LLMs’ intrinsic ability to simulate Big Five traits, revealing variability in trait stability and responsiveness to input changes.



\section{Extra Implementation Details}
\label{appendix:extra_implementation}
\paragraph{Hyper-parameter Adjustment}
To adapt the r-ROME framework for personality editing on the \textit{Qwen2.5-1.5b-inst.}~\citep{yang2024qwen2}, several key hyperparameters were adjusted from the original GPT-2-XL configuration as shown in Table~\ref{tab:model_config}.

\begin{table*}[ht]
\centering
\small
\sisetup{table-format=1.4, separate-uncertainty}
\renewcommand{\arraystretch}{1.2} % 행 간격 조정
\begin{tabular}{l|l}
\toprule
\textbf{Parameter} & \textbf{Value} \\ 
\midrule
\textbf{layers} & [15] \\ 
\textbf{fact\_token} & subject\_first \\ 
\textbf{v\_num\_grad\_steps} & 20 \\ 
\textbf{v\_lr} & 2e-1 \\ 
\textbf{v\_loss\_layer} & 27 \\ 
\textbf{v\_weight\_decay} & 0.5 \\ 
\textbf{clamp\_norm\_factor} & 4 \\ 
\textbf{kl\_factor} & 0.0625 \\ 
\textbf{mom2\_adjustment} & false \\ 
\textbf{context\_template\_length\_params} & [[5, 10], [10, 10]] \\ 
\textbf{rewrite\_module\_tmp} & "model.layers.{}.mlp.down\_proj" \\ 
\textbf{layer\_module\_tmp} & "model.layers.{}" \\ 
\textbf{mlp\_module\_tmp} & "model.layers.{}.mlp" \\ 
\textbf{attn\_module\_tmp} & "model.layers.{}.attention.o\_proj" \\ 
\textbf{ln\_f\_module} & "model.final\_layernorm" \\ 
\textbf{lm\_head\_module} & "lm\_head" \\ 
\textbf{mom2\_dataset} & "wikipedia" \\ 
\textbf{mom2\_n\_samples} & 20 \\ 
\textbf{mom2\_dtype} & "float32" \\
\bottomrule
\end{tabular}
\caption{Configuration Parameters for Personality Editing in \textit{Qwen-2.5-1.5b-inst.}}
\label{tab:model_config}
\end{table*}

\begin{figure*}[!ht]
\centering
\includegraphics[width=1.1\textwidth]{latex/figure/human-eval-guide.pdf}
\caption{An example illustrating a structured assessment sheet used for human evaluation}
\label{fig:human-eval-guide}
\end{figure*}
These changes optimize the model's ability to express nuanced personality traits while aligning with the \textit{Qwen} model’s architecture.

\section{Human Evaluation Details}
\label{appendix:human_eval}

To assess the effectiveness of our personality editing approach, we conduct human evaluations using a structured assessment sheet, as shown in Figure~\ref{fig:human-eval-guide}. We recruited three fluent English-speaking judges for the evaluation, each compensated at approximately \$10 per hour. Three judges were provided with an explanation of the decision-making trait, along with the speaker's utterance and model's responses, allowing them to compare personality before and after editing. Originally, we conducted win/loss/tie evaluation; however, since tie results were minimal, we measured effectiveness using the win ratio instead. We computed two metrics to assess consistency among judges: the raw agreement and Cohen’s Kappa score. Agreement scores were 0.7, 0.57, and 0.6, respectively, resulting in an average of 0.605. Cohen’s Kappa scores were 0.4, 0.53, and 0.29, yielding a mean Kappa score of 0.406. These results support the reliability of our human evaluations while maintaining independent judgment.