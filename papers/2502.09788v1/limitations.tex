PDNS has visibility to around 90\% of the domains on the Internet. This, in turn, results in a reduction of 10\% of seed domains from our ground truth. We attribute this limitation to the data source we utilize. One may utilize additional passive DNS sources such as Spamhaus~\cite{Spamhaus}, circl.lu~\cite{Circl}, or Rapid7~\cite{Rapid7}, to augment Farsight PDNS coverage. Further, an active DNS lookup from multiple vantage points may assist further improve the coverage.

Our approach is not effective at detecting malicious websites created on webhosting services. The key reason is that these sites are hosted on benign infrastructures and do not exhibit homophily relationships. One may construct either a different model such as a content-based classifier to detect such domains. It is fairly normal practice that many detectors are deployed to detect different types of malicious websites such as attack domains, malicious webhosting domains, and DGA domains. 
The graph-based detection mechanism is unable to detect compromised domains unless content-based features are incorporated. As compromised domains are benign domains that turned malicious, their hosting neighborhood has very low toxicity and the neighboring domains in DNS graph are more likely to be benign. Further, features of compromised domains are quite similar to those of benign domains. Thus, an alternative approach is required to detect compromised domains. There are recent research efforts to detect compromised domains~\cite{comp_or_at:2021:usenix,comar:esp:2020}, which complement ours. \spfinal{Further, based on the observation that compromised domains are often infected with malicious scripts such as miners and skimmers, and/or redirect to low reputed URLs, one may construct a graph leveraging content based relationships and utilize a similar graph learning approach to identify compromised domains.}

