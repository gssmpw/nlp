Internet miscreants increasingly utilize short-lived disposable domains to launch various attacks. Existing detection mechanisms are either too late to catch such malicious domains due to limited information and their short life spans or unable to catch them due to evasive techniques such as cloaking and captcha. In this work, we investigate the possibility of detecting malicious domains early in their  life cycle using a content-agnostic approach.
We observe that attackers often reuse or rotate hosting infrastructures to host multiple malicious domains due to increased utilization of automation and economies of scale. 
Thus, it gives defenders the opportunity to monitor such infrastructure to identify newly hosted malicious domains. However, such infrastructures are often shared hosting environments where benign domains are also hosted, which could result in a prohibitive number of false positives. 
Therefore, one needs innovative mechanisms to better distinguish malicious domains from benign ones even when they share hosting infrastructures. In this work, we build \system, a highly accurate practical system that not only generates daily blocklists of malicious domains but also is able to predict malicious domains on-demand. We design a network graph based on the hosting infrastructure that is accurate and generalizable over time. Consistently, our models achieve a precision 
of 99.7\%, a recall of 86.9\% with a very low false positive rate (FPR) of 0.1\% and on average detects 19K new malicious domains per day, which is over 5 times the new malicious domains flagged daily in VirusTotal. Further, \system predicts malicious domains days to weeks before they appear in popular blocklists. 

