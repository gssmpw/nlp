\subsection{Multimodal Tasks Alignment Analysis} \label{sec:eval_task_align}

In this section, we evaluate the performance of \model{} on the following four tasks: vision understanding, audio English tasks, ASR, and S2TT. These tasks cover a wide range of modalities and capabilities, providing a comprehensive assessment of the versatility and effectiveness of MLLMs across various domains.

\subsubsection{Vision-Language Evaluation} \label{sec:vision_language}

We first quantitatively evaluate \model~on various vision understanding tasks, including the following benchmarks: \textbf{(1)} HallusionBench (Hal) \citep{guan2023hallusionbench}: evaluating the hallucination and visual illusion. \textbf{(2)} MathVista (MathV) \citep{lu2023mathvista}: evaluating mathematical reasoning in visual contexts. \textbf{(3)} OCRBench (OCR) \citep{fu2024ocrbenchv2improvedbenchmark}: evaluating Optical Character Recognition (OCR) capabilities on text-intensive images. \textbf{(4)} Video-MME \citep{fu2024videommefirstevercomprehensiveevaluation}: evaluating video QA reasoning capabilities of MLLMs over 6 domains, diverse time range, etc. \textbf{(5)} MMMU \citep{yue2024mmmumassivemultidisciplinemultimodal}: evaluating MLLMs on massive multi-discipline tasks demanding college-level subject knowledge and deliberate reasoning. 
% \textbf{(6)} DocVQA-val \citep{mathew2021docvqadatasetvqadocument}: evaluating MLLMs on textual (handwritten, typewritten or printed) content of the document images. 
\textbf{(6)} AI2D: AI2 Diagrams (AI2D) is a dataset of over 5000 grade school science diagrams. \textbf{(7)} MMVet \citep{yu2024mmvetevaluatinglargemultimodal}: evaluating the integrated capabilities of MLLMs, including recognition, OCR, knowledge, language generation, spatial awareness, and math. \textbf{(8)} MME \citep{fu2024mmecomprehensiveevaluationbenchmark}: measuring both perception and cognition abilities on a total of 14 subtasks.

As illustrated in Table \ref{fig:image}, we observe that \model\ achieves strong performance to current popular vision-language MLLMs on the MathV, OCR, and Hal benchmarks. 
% Furthermore, when compared to proprietary models such as GPT-4V and Claude 3.5 Sonnet, \model\ demonstrates similar levels of performance, highlighting its competitive capabilities in vision understanding and reasoning tasks, indicating our model can effectively maintain vision-language alignment capability. 
Furthermore, when compared to Qwen2.5-VL, \model\ demonstrates superior performance, highlighting its competitive capabilities in vision understanding and reasoning tasks, indicating our model can effectively maintain vision-language alignment capability after auditory pretraining stage. 
\begin{table}[h!]
\caption{\textbf{Evaluation on Vision Understanding Benchmarks.} The best two values are shown in \textbf{\textcolor{black}{bold}} and \underline{underlined}. The blue row refers to the main competitors. In MMMU benchmark, * indicates that the measurement is developed over the validation set. We can observe that \model~shows performance comparable to the leading open-source models and advanced closed-source counterparts.}
    \label{fig:image}
\renewcommand\arraystretch{1.2}
\resizebox{\textwidth}{!}{
\begin{tabular}{llcccccccc}
\toprule
\textbf{Model} & \textbf{LLM-size}   & \textbf{Video-MME}  & \textbf{MMMU} & \textbf{MathV}   & \textbf{Hal} & \textbf{AI2D} & \textbf{OCR} & \textbf{MMVet} & \textbf{MME}\\ \hline
\multicolumn{10}{c}{{\cellcolor[rgb]{0.957,0.957,0.957}} \textbf{\textit{Vision-Language Models}}} \\
VILA-1.5 & Vicuna-v1.5-13B &  44.2 & 41.1 & 42.5 & 39.3 & 69.9 & 460.0 & 45.0 & 1718.2\\
LLaVA-Next & Yi-34B  & 51.6 & 48.8 & 40.4 & 34.8 & 78.9 & 574.0 & 50.7 & 2006.5 \\
CogVLM2 & Llama3-Instruct-8B & 50.5 & 42.6 & 38.6 & 41.3 & 73.4 & 757.0 & 57.8 & 1869.5 \\
InternLM-Xcomposer2 & InternLM2-7B  & 56.2 & 41.4 & 59.5 & 41.0 & 81.2 & 532.0 & 46.7 & 2220.4 \\
Cambrian & NousHermes2-Yi-34B  & 54.2 & 50.4 & 50.3 & 41.6 & 79.5 & 591.0 & 53.2 & 2049.9 \\
InternVL-Chat-1.5 & InternLM2-20B  & 57.1 & 46.8 & 54.7 & 47.4 & 80.6 & 720.0 & 55.4 & 2189.6 \\
Ovis1.5 & Gemma2-It-9B  & 58.1 & 49.7 & 65.6 & 48.2 & \textbf{84.5} & 752.0 & 53.8 & 2125.2 \\
InternVL2 & InternLM2.5-7B  & \underline{61.5} & 51.2 & 58.3 & 45.0 & \underline{83.6} & 794.0 & 54.3 & 2215.1 \\
\rowcolor{blue!8}{MiniCPM-V 2.6} & Qwen2-7B  & 57.5 & 49.8 & 60.6 & 48.1 & 82.1 & 852.0 & 60.0 & 2268.7 \\
\rowcolor{blue!8}Qwen2.5-VL & Qwen2.5-7B  & 56.0 & 51.8* & 61.1 & \textbf{71.7} & 80.7 & \underline{877.0} & - & 2299.1 \\
\hline
\multicolumn{10}{c}{{\cellcolor[rgb]{0.957,0.957,0.957}} \textbf{\textit{Omni-modal Models}}} \\
\rowcolor{blue!8}{VITA-1.5-Audio} & Qwen2-7B  & - & 52.1 & \textbf{66.2} & 44.9 & 79.3 & 732.0 & 49.6 & \textbf{2352.0} \\
EMova-8B & LLaMA-3.1-8B & - & - & 61.1 & -& 82.8 & 824.0 & 55.8 & 2205.0 \\
Baichuan-Omni-1.5 & - & 58.2 & 47.3 & 51.9 & 47.8 & - & - & 65.4 & 2186.9 \\
Megrez-3B-Omni & Megrez-3B  & - & 51.8 & 62.0 & 50.1 & 82.0 & - & - & 2315.0 \\
\hline
\multicolumn{10}{c}{{\cellcolor[rgb]{0.957,0.957,0.957}} \textbf{\textit{Proprietary}}} \\
GPT-4V & -  & 50.4 & 59.3 & 48.2 & 39.3 & 71.4 & 678.0 & 49.0 & 1790.3 \\
GPT-4o mini  & - & 54.8 & 60.0 & 52.4 & 46.1 & 77.8 & 785.0 & \textbf{66.9} & 2003.4 \\
Gemini 1.5 Pro & 200B  & 59.1 & 60.6 & 57.7 & 45.6 & 79.1 & 754.0 & 64.0 & 2110.6\\
GPT-4o  & - & 61.6 & \underline{62.8} & 56.5 & 51.7 & 77.4 & 663.0 & \underline{66.5} & \underline{2328.7}\\
Claude3.5 Sonnet & 175B  & \textbf{62.2} & \textbf{65.9} & 61.6 & 49.9 & 80.2 & 788.0 & 66.0 & 1920.0 \\ \hline \rowcolor[gray]{0.9}
\multicolumn{10}{c}{{\cellcolor[rgb]{0.957,0.957,0.957}} \textbf{\textit{Our Model}}} \\
\rowcolor{blue!8}{\model} & Qwen2.5-VL-7B  & 57.0 & 53.2* & \underline{62.1} & \underline{71.1} & 81.2 & \textbf{882.0} & - & 2315.5 \\ \bottomrule
\end{tabular}
}
\end{table}


% \begin{table}
% \caption{\textbf{Evaluation on Vision Understanding Benchmarks.} \model-Instruct shows performance comparable to the leading open-source models and advanced closed-source counterparts. Hal refers to HallusionBench, MathV to MathVista, and OCR to OCRBench.}
%     \label{fig:image}
% \resizebox{\textwidth}{!}{%
% \renewcommand\arraystretch{1.2}
% \begin{tabular}{cccccccccccc}
% \toprule
% \textbf{Method} & \textbf{LLM} & \textbf{DocVQA-val}  & \textbf{Video-MME}  & \textbf{MMMU} & \textbf{MathV}   & \textbf{Hal} & \textbf{AI2D} & \textbf{OCR} & \textbf{MMVet} & \textbf{MME} & \textbf{Avg}\\ \hline
% \multicolumn{12}{c}{{\cellcolor[rgb]{0.957,0.957,0.957}} \textbf{\textit{Vision-Language Models}}} \\
% VILA-1.5 & Vicuna-v1.5-13B & - & 44.2 & 41.1 & 42.5 & 39.3 & 69.9 & 460.0 & 45.0 & 1718.2 & 52.1  \\
% LLaVA-Next & Yi-34b & - & 51.6 & 48.8 & 40.4 & 34.8 & 78.9 & 574.0 & 50.7 & 2006.5 & 58.3 \\
% CogVLM2 & Llama3-8B-Instruct & - & 50.5 & 42.6 & 38.6 & 41.3 & 73.4 & 757.0 & 57.8 & 1869.5 & 58.8 \\
% InternLM-Xcomposer2 & InternLM2-7B & - & 56.2 & 41.4 & 59.5 & 41.0 & 81.2 & 532.0 & 46.7 & 2220.4 & 61.2 \\
% Cambrian & Nous-Hermes-2-Yi-34B & - & 54.2 & 50.4 & 50.3 & 41.6 & 79.5 & 591.0 & 53.2 & 2049.9 & 61.4 \\
% InternVL-Chat-1.5 & InternLM2-20B & - & 57.1 & 46.8 & 54.7 & 47.4 & 80.6 & 720.0 & 55.4 & 2189.6 & 65.1 \\
% Ovis1.5 & Gemma2-9B-It & - & 58.1 & 49.7 & 65.6 & 48.2 & \textbf{84.5} & 752.0 & 53.8 & 2125.2 & 66.9 \\
% InternVL2 & InternLM2.5-7b & - & \underline{61.5} & 51.2 & 58.3 & 45.0 & \underline{83.6} & 794.0 & 54.3 & 2215.1 & 67.3 \\
% MiniCPM-V 2.6 & Qwen2-7B & - & 57.5 & 49.8 & 60.6 & 48.1 & 82.1 & \textbf{852.0} & 60.0 & 2268.7 & 68.5 \\
% Qwen2-VL & Qwen2-7B & - & 55.0 & 54.1 & - & \underline{50.6} & - & 845.0 & - & 2326.8 & - \\
% \hline\Gray
% \multicolumn{12}{c}{{\cellcolor[rgb]{0.957,0.957,0.957}} \textbf{\textit{Omni-modal Models}}} \\
% % \multicolumn{12}{c}{\texttt{Omni}} \\
% VITA-1.5-Audio & Qwen2-7B & - & - & 52.1 & \textbf{66.2} & 44.9 & 79.3 & 732.0 & 49.6 & \textbf{2352.0} & - \\
% EMova-8B & LLaMA-3.1-8B & 90.4 &- & - & 61.1 & -& 82.8 & 824.0 & 55.8 & 2205.0 & - \\
% Baichuan-Omni & - & - & 58.2 & 47.3 & 51.9 & 47.8 & - & - & 65.4 & 2186.9 & - \\
% Megrez-3B-Omni & Megrez-3B & \underline{91.6} & - & 51.8 & \underline{62.0} &50.1 & 82.0 & - & - & 2315.0 & - \\
% \hline
% \multicolumn{12}{c}{{\cellcolor[rgb]{0.957,0.957,0.957}} \textbf{\textit{Proprietary}}} \\
% GPT-4V & - & - & 50.4 & 59.3 & 48.2 & 39.3 & 71.4 & 678.0 & 49.0 & 1790.3 & 58.5 \\
% GPT-4o mini & - & - & 54.8 & 60.0 & 52.4 & 46.1 & 77.8 & 785.0 & \textbf{66.9} & 2003.4 & 66.3 \\
% Gemini 1.5 Pro & - & - & 59.1 & 60.6 & 57.7 & 45.6 & 79.1 & 754.0 & 64.0 & 2110.6 & 67.2 \\
% GPT-4o & - & - & 61.6 & \underline{62.8} & 56.5 & \textbf{51.7} & 77.4 & 663.0 & \underline{66.5} & \underline{2328.7} & 69.3 \\
% Claude3.5 Sonnet & - & - & \textbf{62.2} & \textbf{65.9} & 61.6 & 49.9 & 80.2 & 788.0 & 66.0 & 1920.0 & 69.3 \\ \hline \Gray
% \multicolumn{12}{c}{{\cellcolor[rgb]{0.957,0.957,0.957}} \textbf{\textit{Our Model}}} \\
% \model-Instruct & Qwen2-7B & \textbf{92.7} & 54.0 & 47.8 & \textbf{66.2} & - & - & \underline{847.0} & - & 2219.0 & - \\ \bottomrule
% \end{tabular}%
% }
% \end{table}

\subsubsection{Audio-Language Evaluation} \label{sec:audio_language}

Next, we evaluate the audio-language alignment on English QA, ASR, and S2TT tasks. 

\paragraph{English QA task.} 
% We evaluate the performance on the following 3 spoken QA benchmarks: Web Q. \citep{berant-etal-2013-semantic}, LLaMA Q. \citep{nachmani2024spokenquestionansweringspeech}, and Trivia QA \citep{joshi2017triviaqalargescaledistantly}. Note that since Web Q. and Trivia QA benchmarks do not provide audio queries, the comparison among models may be influenced by the performance of different TTS models. This could introduce variability in results, as the quality and characteristics of the TTS model used for generating audio queries may impact overall performance. Therefore, we do not provide the analysis for both benchmarks.
We evaluate the performance on the spoken QA benchmark: LLaMA Q. \citep{nachmani2024spokenquestionansweringspeech}. As presented in Table \ref{tab:eqa}, our model achieves top performance, higher than same-period competitor MiniCPM-o2.6-7B (highlighted in blue), demonstrating its competitive capabilities in this task. 
% Besides, we also provide the performance of our model, i.e., \model$\dagger$, evaluated by LLM (Qwen2-Instruct-72B) as the model might generate answers that are semantically similar but not an exact match (e.g., \textit{``China''} and \textit{``People's Republic of China''}).
% \begin{table}[!h]
% \caption{\textbf{Evaluation on Audio English QA Benchmarks.} The accuracy~(\%) of different models in English question answering on three sets. {\model$\dagger$} shows the results of the evaluation using the large language model. The parameter size is derived from the backbone LLMs. Our model achieves top 3 accuracy in the LLaMA Q. benchmark, outperforming the same-period competitor MiniCPM-o2.6-7B.}
% \label{tab:eqa}
% \centering
% \renewcommand{\arraystretch}{1.2}
% % \setlength{\tabcolsep}{10pt}
% \resizebox{14cm}{!}{
% \begin{tabular}{lcc>{\columncolor{blue!8}}cc}
% \toprule
% \textbf{Model}  & \textbf{Modality} & \textbf{Web Q.}$\uparrow$  & \textbf{LLaMA Q.}$\uparrow$  & \textbf{Audio Trivia QA} $\uparrow$ \\ \hline
% % \multicolumn{5}{c}{{\cellcolor[rgb]{0.957,0.957,0.957}} \textit{**Notice: this part refers to the performance upper boundary of  \model~**}}\\
% % % Helium~\citep{defossez2024moshi}           & Text Only         & 32.30            & 75.00                & 56.40                     \\
% % Qwen2-VL-Instruct-7B\citep{wang2024qwen2}           & Text Only         & 32.30            & 75.00                & 56.40                     \\
% % Qwen2-Instruct-7B\citep{yang2024qwen2} & Text Only         & 45.13           & 77.67             & 63.93                    \\ 
% % \hline
% SpeechGPT-7B\citep{zhang2023speechgptempoweringlargelanguage}    & Audio\&Text       & 6.50             & 21.60              & 14.80                     \\
% Spectron-1B\citep{nachmani2023spoken}     & Audio\&Text       & 6.10             & 22.90              & -                         \\
% Moshi-7B\citep{defossez2024moshi}        & Audio\&Text       & 26.60            & 62.30              & 22.80                     \\
% GLM-4-Voice-9B\citep{zeng2024scalingspeechtextpretrainingsynthetic}        & Audio\&Text       & 32.20            & 64.70              & 39.10                     \\
% \rowcolor{blue!8}MiniCPM-o2.6-7B        & Audio\&Text       & 40.00           & 61.00             & 40.20                     \\
% Mini-Omni-0.5B\citep{xie2024mini} & Audio\&Text       & 12.80           & 22.00            & 6.90 \\
% Llama-Omni-8B\citep{fang2024llamaomniseamlessspeechinteraction} & Audio\&Text       & 22.90           & 45.30            & 10.70 \\
% MinMo-7B\citep{chen2025minmomultimodallargelanguage} & Audio\&Text       & \textbf{55.00}          & \textbf{78.90}           & \underline{48.30} \\
% Freeze-Omni-7B\citep{wang2024freeze}  & Audio\&Text       & \underline{44.73}          & \underline{72.00}                & \textbf{53.88}                    \\ 
% \hline
% \multicolumn{5}{c}{{\cellcolor[rgb]{0.957,0.957,0.957}} \textbf{\textit{Our Models}}} \\
% \rowcolor{blue!8}\model  & Audio\&Text       & 22.07           & 67.33               & \underline{48.30}    \\
% \model$\dagger$  & Audio\&Text       & 38.86          & 80.66                & 54.98 \\
% \bottomrule

% \end{tabular}
% }
% \end{table}

%%%%%%%%%%%%%%%%%%%%%%%%%%%%%%%%%%%%%%%
% \begin{table}[!h]
% \caption{\textbf{Evaluation on Audio English QA Benchmarks.} The accuracy~(\%) of different models in English question answering on three sets. {\model$\dagger$} shows the results of the evaluation using the large language model. The parameter size is derived from the backbone LLMs. Our model achieves top accuracy in the LLaMA Q. benchmark, outperforming the same-period competitor MiniCPM-o2.6-7B.}
% \label{tab:eqa}
% \centering
% \renewcommand{\arraystretch}{1.2}
% % \setlength{\tabcolsep}{10pt}
% \resizebox{14cm}{!}{
% \begin{tabular}{lcccc}
% \toprule
% \textbf{Model}  & \textbf{Modality} & \textbf{Web Q.}$\uparrow$  & \textbf{LLaMA Q.}$\uparrow$  & \textbf{Audio Trivia QA} $\uparrow$ \\ \hline
% SpeechGPT-7B\citep{zhang2023speechgptempoweringlargelanguage} & Audio\&Text & 6.50 & 21.60 & 14.80                     \\
% Spectron-1B\citep{nachmani2023spoken} & Audio\&Text & 6.10  & 22.90  & -                         \\
% Moshi-7B\citep{defossez2024moshi}     & Audio\&Text & 26.60 & 62.30  & 22.80                     \\
% GLM-4-Voice-9B\citep{zeng2024scalingspeechtextpretrainingsynthetic}        & Audio\&Text       & \underline{32.20}            & \underline{64.70}              & 39.10                     \\
% \rowcolor{blue!8}MiniCPM-o2.6-7B  & Audio\&Text  & \textbf{40.00} & 61.00  & \underline{40.20}                     \\
% Mini-Omni-0.5B\citep{xie2024mini} & Audio\&Text       & 12.80           & 22.00            & 6.90 \\
% Llama-Omni-8B\citep{fang2024llamaomniseamlessspeechinteraction} & Audio\&Text       & 22.90           & 45.30            & 10.70 \\
% \hline
% \multicolumn{5}{c}{{\cellcolor[rgb]{0.957,0.957,0.957}} \textbf{\textit{Our Models}}} \\
% \rowcolor{blue!8}\model  & Audio\&Text       & 22.07           & \textbf{67.33}               & \textbf{48.30}    \\
% \model$\dagger$  & Audio\&Text       & 38.86          & 80.66                & 54.98 \\
% \bottomrule

% \end{tabular}
% }
% \end{table}

%%%%%%%%%%%%%%%%%%%%%%%%%%%%%%%%%%%%%%%
\begin{table}[!h]
\caption{\textbf{Evaluation on Audio English QA Benchmarks.} The accuracy~(\%) of different models in English question answering on three sets. The parameter size is derived from the backbone LLMs. Our model achieves top accuracy in the LLaMA Q. benchmark, outperforming the same-period competitor MiniCPM-o2.6-7B.}
\label{tab:eqa}
\centering
\renewcommand{\arraystretch}{1.2}
% \setlength{\tabcolsep}{10pt}
\resizebox{10cm}{!}{
\begin{tabular}{lcc}
\toprule
\textbf{Model}  & \textbf{Modality}  & \textbf{LLaMA Q.}$\uparrow$ \\ \hline
SpeechGPT-7B\citep{zhang2023speechgptempoweringlargelanguage} & Audio\&Text & 21.60 \\
Spectron-1B\citep{nachmani2023spoken} & Audio\&Text & 22.90                         \\
Moshi-7B\citep{defossez2024moshi}     & Audio\&Text & 62.30                         \\
GLM-4-Voice-9B\citep{zeng2024scalingspeechtextpretrainingsynthetic}  & Audio\&Text  & \underline{64.70}  \\
\rowcolor{blue!8}MiniCPM-o2.6-7B  & Audio\&Text & 61.00   \\
Mini-Omni-0.5B\citep{xie2024mini} & Audio\&Text & 22.00   \\
Llama-Omni-8B\citep{fang2024llamaomniseamlessspeechinteraction} & Audio\&Text  & 45.30 \\
\hline
\multicolumn{3}{c}{{\cellcolor[rgb]{0.957,0.957,0.957}} \textbf{\textit{Our Models}}} \\
\rowcolor{blue!8}\model  & Audio\&Text  & \textbf{67.33}  \\
\bottomrule

\end{tabular}
}
\end{table}

% ----------------------------------------------------------
% For the Chinese QA, we focus on the following benchmarks:
% C-Eval \citep{huang2023ceval}: C-Eval is a comprehensive Chinese evaluation suite for foundation models. It consists of 13948 multi-choice questions spanning 52 diverse disciplines and four difficulty levels.
% ant-fineval \citep{}: \textcolor{blue}{\textbf{TODO}}.
% mmlu \citep{hendrycks2021measuringmassivemultitasklanguage}: measure a text model's multitask accuracy. The test covers 57 tasks including elementary mathematics, US history, computer science, law, and more.
% \begin{table}[!h]
\caption{\textbf{Evaluation on Audio Chinese QA Benchmarks.} The accuracy~(\%) of different models in Chinese question answering on three sets. {\model-Instruct (9B)$\dagger$} shows the results of the evaluation using the large language model.}
\label{tab:cqa}
\centering
\renewcommand{\arraystretch}{1.2}
\setlength{\tabcolsep}{10pt}
\resizebox{14cm}{!}{
\begin{tabular}{lcccc}
\toprule
\textbf{Model}            & \textbf{Modality} & \textbf{ceval-audio} & \textbf{ant-fineval-audio} & \textbf{mmlu-audio} \\ \midrule
SpeechGPT(7B)~\citep{zhang2023speechgptempoweringlargelanguage}    & Audio\&Text       &              &              &                     \\
Spectron(1B)~\citep{nachmani2023spoken}     & Audio\&Text       &              &               & -                         \\
Moshi(7B)~\cite{defossez2024moshi}        & Audio\&Text       &             &               &                      \\
GLM-4-Voice(9B)~\cite{zeng2024scalingspeechtextpretrainingsynthetic}        & Audio\&Text       &             &              &                      \\
Freeze-Omni(7B)~\cite{wang2024freeze}  & Audio\&Text       &          &                 &                     \\ \midrule
\textbf{\model-Instruct (9B)}  & Audio\&Text       &            &                 &                     \\
\textbf{\model-Instruct (9B)$\dagger$}  & Audio\&Text       &           &                 &  \\
\midrule
Helium~\cite{defossez2024moshi}           & Text Only         &             &                &                     \\
Qwen2-VL-7B-Instruct~\cite{wang2024qwen2}           & Text Only         &             &                 &                     \\
Qwen2-7B-Instruct~\cite{yang2024qwen2} & Text Only         &           &              &                     \\ \bottomrule

\end{tabular}
}
\end{table}
% ----------------------------------------------------------

\paragraph{ASR task.} In the ASR task, we focus on both Mandarin (cn) and English (eng), evaluating performance on the following benchmarks: AIShell-2 \citep{du2018aishell2transformingmandarinasr}, Librispeech \citep{panayotov2015librispeech}, and our real-scenario benchmark. As demonstrated in Table \ref{tab:asr}, our model achieves the best performance on the real scenario benchmarks, indicating the robustness of our model.
% competitive performance, surpassing most baselines on the Librispeech and real scenario benchmarks, indicating the robustness of our model.
\begin{table}[h!]
\caption{\textbf{Evaluation on ASR Benchmarks.} \model~has demonstrated strong performance in both real-life Mandarin and English ASR tasks. It outperforms specialized speech models, achieving better results in both languages. AIShell-2 is measured over all categories (Mac/iOS/Android), TC: test-clean, TO: test-other. We can observe that \model~outperforms other models in real scenarios.} \label{tab:asr}
\renewcommand{\arraystretch}{1.2}
\resizebox{\textwidth}{!}{
\begin{tabular}{lccccc} % \columncolor{blue!8}}
\toprule
\multirow{2}{*}{\textbf{Model}} & \multicolumn{2}{c}{\textbf{CN (CER$\downarrow$)}} & \multicolumn{3}{c}{\textbf{Eng (WER$\downarrow$)}} \\ \cmidrule{2-6}
 & \textbf{AIShell-2} &  \textcolor{black}{\textcolor{black}{\textbf{\model-audio}}}   & \textbf{Librispeech TC} & \textbf{Librispeech TO}& \textcolor{black}{\textcolor{black}{\textbf{\model-audio}}} \\ \hline
\multicolumn{6}{c}{{\cellcolor[rgb]{0.957,0.957,0.957}} \textbf{\textit{Speech LLMs}}} \\
% Wav2vec2-base~\citep{baevski2020wav2vec} & - &  -  & 4.70 & 9.40 & - \\
% SpeechT5~\citep{ao2021speecht5}  & - &  -  & 4.00 & - & - \\
% SALMONN-13B\citep{tang2023salmonn}& - &  -  & 2.10 & 4.90 & - \\
% Mini-Omni-0.5B\citep{xie2024mini}& - &  -  & - & - & - \\
% Moshi-7B\citep{defossez2024moshi} & - &  -  & - & - & - \\
% Freeze-Omni-7B\citep{wang2024freeze} & - &  -  & 3.24 & 7.68 & - \\ 
Qwen2-Audio-Instruct-7B & 3.00/3.00/2.90 &  35.45  & \textbf{1.70} & 4.00 & 26.12 \\ 
% Qwen2-Audio-Instruct-7B\textsuperscript{*} & - &  -  & 3.10 & - & - \\
% GLM-4-Voice-9B\citep{zeng2024scalingspeechtextpretrainingsynthetic}& - &  -  & 2.82 & 7.66 & - \\ \hline
\multicolumn{6}{c}{{\cellcolor[rgb]{0.957,0.957,0.957}} \textbf{\textit{Omni-modal LLMs}}} \\
% AnyGPT-7B\citep{zhan2024anygpt} & - & - & - & 8.50 & - \\ 
Mini-Omini2-0.5B\citep{xie2024mini}  &  &  117.71 & 9.80 & 4.70 & 54.69 \\
% EMOVA-8B\citep{chen2024emova} & - & - & 4.00 & -  & - \\
\rowcolor{blue!8}VITA-1.5-7B\citep{fu2025vita} & -  & 127.34  & 7.20 & \textbf{3.40} & 122.71 \\
MinMo-7B\citep{chen2025minmo} & 4.89/4.76/4.96  & -  &  \underline{1.74} &  \underline{3.89} & - \\
\rowcolor{blue!8}MiniCPM-o2.6-7B & -  & \underline{12.00}  &  \textbf{1.70} &  \underline{3.89} & \underline{18.00} \\
\hline
\multicolumn{6}{c}{{\cellcolor[rgb]{0.957,0.957,0.957}} \textbf{\textit{Our Models}}} \\

\rowcolor{blue!8}\textcolor{black}{\model}  & 3.35  & \textbf{11.82} & 3.50 & 5.29 & \textbf{12.32}  \\
% \textcolor{black}{\model-Instruct-7B} & - & \textbf{-} & 2.41 & \textbf{2.61}& - \\ 
\bottomrule
\end{tabular}
}
\end{table}

\paragraph{Speech-to-Text translation task.}

As for the S2TT, we exploit CoVoST2 \citep{wang2020covost2massivelymultilingual}, which provides massive multilingual S2TT datasets. As shown in Table \ref{tab:speech2text}, \model~demonstrates superior performance compared to specialised speech LLM, Qwen2-Audio-7B-Instruct, in both translation tasks. Nevertheless, our model still falls short of the performance of its contemporary competitor, MiniCPM-o2.6-7B. However, we pre-train our model on only 8 million audio samples, which is a relatively modest dataset at the pretraining stage.
% Although our model left behind the same-period competitor (MinMo-7B and MiniCPM-02.6-7B), our model is pretrained over only 20,000 translation training samples. In contrast, MinMo-7B relies on a pretrain dataset 20 times larger, indicating the high training efficiency of our model.
% Unlike MinMo-7B, which uses hundreds of thousands of training data, we only use 20,000 translation data (1/20 of it) to achieve very good translation results. 
\begin{table}[h!]
\centering
\caption{\textbf{Evaluation on Speech-to-Text Translation Benchmarks.} 
% \model~has demonstrated strong performance in both Mandarin (zh) and English (en) ASR tasks. 
\model~demonstrated superior performance compared to Qwen2-Audio-Instruct-7B in both translation tasks, however, it still falls short of the performance of its contemporary competitor. We conjecture that this performance gap may be due to the relatively modest scale of our pretraining speech dataset.}\label{tab:speech2text}
\resizebox{8.5cm}{!}{
\begin{tabular}{lcc}
\toprule
\multirow{2}{*}{\textbf{Model}} & \multicolumn{1}{c}{\textbf{zh-en (BLEU$\uparrow$)}} & \multicolumn{1}{c}{\textbf{en-zh (BLEU$\uparrow$)}} \\ \cmidrule{2-3}
 & \textbf{CoVoST2}  & \textbf{CoVoST2} \\ \hline
 \multicolumn{3}{c}{{\cellcolor[rgb]{0.957,0.957,0.957}} \textbf{\textit{Speech LLMs}}} \\
% Wav2vec2-base & - & - \\
Qwen2-Audio-7B  & 24.40 & 45.20 \\
Qwen2-Audio-Instruct-7B\textsuperscript{*} & 22.90 & 39.50 \\ 
MinMo-7B \textsuperscript{*} & \underline{25.95} & \underline{46.68} \\ 
\hline
 \multicolumn{3}{c}{{\cellcolor[rgb]{0.957,0.957,0.957}} \textbf{\textit{Omni-modal LLMs}}} \\
% Mini-Omini2-0.5B & - & - \\
% Freeze-Omini-7B & - & - \\ 
% VITA-1.0-8$\times$7B & - & -  \\
% VITA-1.5-7B & - & -  \\
% OpenOmni-7B\citep{luo2025openomnilargelanguagemodels} & - & -  \\
\rowcolor{blue!8}{MiniCPM-o2.6-7B} &  \textbf{27.20} &  \textbf{48.20} \\
\hline
\multicolumn{3}{c}{{\cellcolor[rgb]{0.957,0.957,0.957}} \textbf{\textit{Our Models}}} \\
\rowcolor{blue!8}\model & 23.17 & 40.21  \\ \bottomrule
\end{tabular}%
}
\end{table}
