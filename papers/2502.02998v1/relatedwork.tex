\section{Related Work}
\textbf{Continual Test-Time Adaptation}.
Test-Time Adaptation (TTA) allows source-free, online model adaptation to target domain characteristics \cite{jain2011online,sun2020test,wang2020tent}. CTTA~\cite{wang2022continual,lyu2024variational,tan2024less} extends TTA to continuously changing target domains, addressing long-term adaptation but facing error accumulation challenges~\cite{tarvainen2017mean,wang2022continual}. 
Prolonged reliance on unsupervised loss during long-term adaptation risks error accumulation and forgetting source knowledge, hindering accurate classification of source-like test samples. Most existing methods address these issues by enhancing the source model's confidence during testing.
To address error accumulation, mean-teacher methods~\cite{tarvainen2017mean} align the student with the teacher model, updating the teacher via moving averages. To mitigate forgetting, augmentation-averaged predictions~\cite{wang2022continual,brahma2023probabilistic,dobler2023robust,yang2023exploring} enhance the teacher's confidence against out-of-distribution samples. 
Contrastive loss~\cite{dobler2023robust,chakrabarty2023sata} preserves learned semantics, while some methods prioritize restoring source parameters~\cite{wang2022continual,brahma2023probabilistic}.
Though the above methods keep the model from vague pseudo labels, they may suffer from overly confident predictions that are less calibrated.
To mitigate this issue, it is helpful to estimate the uncertainty in the model.

\noindent
\textbf{Conformal Prediction}.
CP is a robust framework for quantifying uncertainty in machine learning, especially in high-stakes applications where reliability is crucial. 
CP generates prediction sets that contain the true outcome with a specified probability, without relying on assumptions about the underlying data distribution. CP focuses on the concept of exchangeability and the use of nonconformity scores~\cite{vovk2005algorithmic}.
CP has been applied to a wide range of problems, including medical diagnostics~\cite{caruana2015intelligible}, autonomous vehicles~\cite{lekeufack2023conformal}, and financial decision-making, where the quantification of uncertainty is critical for safety and trust.
Researchers have extended conformal prediction to handle more complex scenarios such as risk control \cite{farinhas2023non}.
However, to the best of our knowledge, conformal prediction has not yet been applied to the CTTA task, whereas estimating uncertainty in test results is crucial in long-term testing environments.