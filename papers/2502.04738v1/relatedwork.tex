\section{Related Work}
\label{sec:related}

The literature on processor verification is, of course, voluminous by now
and documents many valuable experiments and contributions. We will not
include an extended survey in this report, but focus here on only the most
closely related works that address the verification of CHERI processors
specifically or the central end-to-end component of our methodology.

The first substantial verification effort for a CHERI processor was
reported by Gao and Melham in~\cite{Gao:2021:EFV}. This targeted
CHERI-Flute~\cite{CTSRDCHERIFluteRISCV}, a modified version of Bluespec's
Flute RISC-V processor~\cite{Flute,FluteAnnouncement} that implements CHERI
RISC-V instructions. Flute is an open-source 64-bit RISC-V processor with a
five-stage, in-order pipeline---so along some dimensions CHERI-Flute is
larger in scale than CHERIoT-Ibex. Capabilities in CHERI-Flute are
decompressed and compressed when they are moved between memory and the
core. Internally, in registers, they are held in decompressed form.

The main result of this work was the end-to-end verification of all the
implemented CHERI instructions in the processor, together with certain
processor liveness results.  Several bugs were revealed by this
verification effort.

The specifications were manually translated from Sail to SystemVerilog, and
the effort needed for this is reported as having limited the coverage of
the specifications. Only instruction executions that do not lead to a
failure were verified, because the manual effort needed to specify the
correct outcomes following the `more than a dozen ways' in which an
instruction can fail was prohibitive given the (substantial) time available.
The lack of a flow from Sail to the verification tool is cited
in~\cite{Gao:2021:EFV} as the practical limiting factor, and this
experience inspired us to create the Sail to SystemVerilog flow reported
here.

Another difference is the proof engineering methodology used
for the end-to-end correctness properties. To achieve conclusive proofs,
Gao and Melham employed a strategy they call `decomposing the pipeline', in
which a chain of lemmas are formulated that track the required state
updates back through the pipeline and allow $k$-induction to converge. This
required thorough analysis of pipeline forwarding and other
microarchitectural details. By contrast, the approach developed here relies
on more ad-hoc, but methodologically formulaic, invariant
development.

\medskip

The other published work to date that targets CHERIoT-Ibex specifically is
VeriCHERI~\cite{vericheri2024}.  This approach is complementary to our
functional verification research; VeriCHERI is a `specification free'
approach that directly encodes desirable properties about the security of
memory accesses and seeks to verify that the RTL conforms to these
properties. In essence, the properties state that no memory transaction is
issued whose address lies within a symbolically-enumerated `protected'
region of memory, assuming that no capability that legitimately addresses
this region is present in any register of the processor. A `weak
monotonicity' property is also established, which states that this
assumption is maintained across a single step of the processor---and hence
by induction that it is an invariant.

Research demonstrating that security-related properties are guaranteed by a
CHERI ISA includes proofs that such properties follow from the Sail
specification~\cite{cheriot-sail-properties}.  Observational correctness
with respect to a Sail specification, as we demonstrate here, implies that
the processor RTL also satisfies these properties.
However, this approach does not detect side channels that could be
exploited by a timing attack. VeriCHERI, however, can reveal
microarchitectural side channels. This is demonstrated
in~\cite{vericheri2024}, which documents a Meltdown-style vulnerability in
CHERIoT-Ibex by which a timing attack could, in principle, gain a small
amount of information about an arbitrary word in memory. This vulnerability
cannot, of course, be found with a functional verification of the kind we
report here.

\medskip

A landmark paper on end-to-end instruction verification is the work of Reid
et al.~on ISA-Formal at Arm~\cite{Reid-2016-EEV}. This was the inspiration
for our starting point in the CHERI-Flute verification
effort~\cite{Gao:2021:EFV}.  There are many similarities to the
end-to-end instruction verification component of the methodology we
illustrate here with CHERIoT-Ibex.  This, however, is only one part of our
overall proof that establishes top-level observational correctness.

In matters of detail, our approach to specification installation using a
pipeline follower departs from that explained
in~\cite{Reid-2016-EEV}, especially with regards to memory. Where we
can make use of the predictable and linear ordering of memory requests
of CHERIoT-Ibex, the larger ARM chips verified under ISA-Formal make weaker
guarantees, with some instructions (e.g. LDM and STM) making up to 16 memory requests in
no particular order. For this reason in ISA-Formal memory operations were
collated and verified at the end of their pipeline stage, instead of as they
occurred as in our work.

Additionally, in the ISA-Formal work, at least for simple cases, the
architectural pre-state is sampled (through an abstraction function, as here)
at the time of retirement from the writeback stage. This is compared (in the same
cycle) to the specification computed on the architectural post-state derived
from the outputs of the memory stage. This differs from our model of carrying
the specification post-state in the pipeline follower up to the point of retirement.
ISA-Formal uses the pipeline follower primarily to store the running instruction
opcode. The explanation~\cite{Reid-2016-EEV} is
by illustration with a simple register-to-register addition function, and there
may well be variations to this structure for other instruction classes.

Analogous to the Sail flow reported here, this is based on a compilation
from specifications expressed in Arm's Architecture Specification Language
to SystemVerilog. Here, too, one of the challenges was bitstring
polymorphism, necessitating a monomorphisation pass of the compiler.

Of course a significant difference in practice between ISA-Formal and our
work is that the former targets Arm processors of much greater complexity
than a RISC-V core---at least the one that CHERIoT-Ibex is based on. The
Arm verifications, therefore, obtained only bounded proofs, and
did not cover exceptions and `the memory management unit'.  Nonetheless,
bounded proofs and partial verification coverage can be highly effective at
bug-finding and can deliver significant value in practice.