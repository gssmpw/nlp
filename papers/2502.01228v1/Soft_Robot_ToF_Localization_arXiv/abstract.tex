\begin{abstract}
Thanks to their compliance and adaptability, soft robots can be deployed to perform tasks in constrained or complex environments. 
In these scenarios, spatial awareness of the surroundings and the ability to localize the robot within the environment represent key aspects.
While state-of-the-art localization techniques are well-explored in autonomous vehicles and walking robots, they rely on data retrieved with lidar or depth sensors which are bulky and thus difficult to integrate into small soft robots.
Recent developments in miniaturized \gls{tof} sensors show promise as a small and lightweight alternative to bulky sensors. These sensors can be potentially distributed on the soft robot body, providing multi-point depth data of the surroundings.
However, the small spatial resolution and the noisy measurements pose a challenge to the success of state-of-the-art localization algorithms, which are generally applied to much denser and more reliable measurements.

In this paper, we enforce distributed VL53L5CX \gls{tof} sensors, mount them on the tip of a soft robot, and investigate their usage for self-localization tasks. Experimental results show that
the soft robot can effectively be localized with respect to a known map, with an error comparable to the uncertainty on the measures provided by the miniaturized ToF sensors.

%
\end{abstract}