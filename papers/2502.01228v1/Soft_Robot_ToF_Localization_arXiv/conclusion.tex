
\section{Conclusion}\label{sec:conclusion}

In this paper, we evaluated the use of miniaturized \gls{tof} sensors to localize a soft robot with respect to a known environment.
Due to their reduced weight, the main advantage of these sensors is the possibility of distributing them into the soft robot to enhance its spatial awareness.
However, compared to traditional sensors commonly used for localization they present several drawbacks, such as significantly lower spatial resolution and reliability of the measurements.

Experimental results show that by applying state-of-the-art methods for localization, such as \gls{pf}, it is possible to localize the robot in a known map with an accuracy comparable to the uncertainty of the measurement provided by the sensors.

Future works will be dedicated to integrating probabilistic models to take into account the uncertainties provided by the sensor, thus possibly improving the localization accuracy, and 
assessing the performance of the soft robot localization in more complex environments.

