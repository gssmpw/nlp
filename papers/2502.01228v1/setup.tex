\section{Experimental Setup}\label{sec:setup}

\subsection{Hardware}
The robot is equipped with a miniaturized ToF sensor placed on the robot's end-effector. We provide little detail about the sensors but mainly redirect the reader to previous papers. Then details on the target and the way it is placed on the table. Also info on the material used to cover the board. Further details on the way the characterization is done are in \cref{sec:calibration}.

\begin{figure}[]
    \centering
    \includegraphics[width=0.5\textwidth]{example-image-duck}
    \caption{The Panda arm is equipped with a ToF sensor at the end-effector (highlighted in the figure) and the box is placed on the table with a certain pose.}
    \label{fig:sensors}
\end{figure}

\subsection{Localization task}
For the localization, we use a wooden crate and use AprilTag to get the ground truth with a Realsense camera.
The robot is commanded to move in different poses that have been chosen so that ToF readings fall inside the box. At each pose, the sensor point cloud is updated, the confidence is computed and the particle filter is run. Optimizing the next pose to be explored is out of the scope of this paper. 



