
\section{Localization based on Distributed ToF Sensors Measurements}\label{sec:methodology}

\subsection{Environment Reconstruction using Distributed ToF Measurements}

In this paper we consider a soft robot equipped with a set of \gls{tof} sensors  distributed on its body. The configuration of the soft robot is assumed to be measurable or at least estimated, while the pose of its base with respect to the environment is unknown. 
As shown in \cref{fig:intro}, the single sensor has a pyramidal \gls{fov}, and its output consists of a low-resolution depth map, where each measurement corresponds to the distance between the origin of the sensor and the portion of environment falling within the \gls{fov}.
Assuming that we have: (i) the intrinsic calibration of the sensors from datasheet specifications \cite{st_vl53l5cx_datasheet}, and (ii) their position with respect to a common reference frame (i.e., the robot base), it is possible to convert all the depth maps collected by the various sensors into a single point cloud $\bar{X}_k$. We refer to $\bar{X}_k$ as the \textit{point cloud sample}, where $k$ is the index representing the $k$-th sample.
%
By moving the robot into different positions, different parts of the environment will fall into the \gls{fov} of the sensors. Therefore, by collecting and merging samples at different robot configurations it is possible to incrementally build a point cloud representing the environment surrounding the robot.
This can be defined as  $X_k = \bar{X}_k \cup \bar{X}_{k-1}$, with $X_0$ being an empty point cloud set.


\subsection{Soft Robot Localization with PF}

Assuming the map of the environment $\hat{X}$ to be known in the form of a point cloud, the goal is to find an estimate $\hat{\mathbf{x}}\in SE(3)$ of the true pose of the robot base $\mathbf{x} \in SE(3)$ with respect to $\hat{X}$ from the knowledge of the measurements $\bar{X}_k$.
% 
In this respect, we exploit the use of a \gls{pf} to estimate the pose of the robot $\hat{x}_k \in SE(3)$ at each $k$-th collected sample. A set of $M$ particles $P_k=\lbrace \mathbf{p}_k^1, \dots, \mathbf{p}_k^M \rbrace$ is used to describe the pose of the robot base with respect to the environment in the Cartesian space, with $\mathbf{p}_k^i \in SE(3)$. For the given $\bar{X}_k$, the particle weights are updated using the ICP registration algorithm~\cite{besl1992method} initialization step. More specifically, at the $k$-th iteration, for each devised pose $\mathbf{p}_k^i$, the $\bar{X}_k$ is transformed accordingly from the sensor reference frame to the robot base; then, the registration score  $w_k^i$ is computed with the initialization step of the ICP algorithm by matching $\bar{X}_k$ and $\hat{X}$, with $w_k^i \in [0,1]$. The resulting scores associated with the particles become crucial in the resampling phase in which new particles are devised, i.e., for the $(k+1)$-th step, particles are drawn with higher density in the neighborhood of the best matching poses of the $k$-th step by using the normal distribution. For $k=0$, particles are drawn from a Gaussian distribution while the associated weights are given by a uniform density distribution. 
The localization is accomplished by setting in advance the number $N$ of samples that need to be considered. For $k=N$, it is imposed $\hat{\mathbf{x}}= \frac{1}{M}\sum_{i=1}^{M}w_N^i\mathbf{p}_N^i$.
