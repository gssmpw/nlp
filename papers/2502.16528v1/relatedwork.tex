\section{RELATED WORKS}
\label{RW}

\subsection{Closed-set semantic mapping}
Semantic perception is crucial for robots to perform downstream tasks in real-world environments \cite{garg2020semantics, deng2022s}. The rise of modern deep learning has closely paralleled significant advancements in the field of semantics for robotics, leading to numerous breakthroughs in recent research. 
DA-RNN \cite{DA-RNN} adopts an FCN-based semantic labeling framework and develops an RNN-based cross-frame semantic fusion method. 
SemanticFusion \cite{semanticfusion} employs CNN-based semantic predictions with probabilistic representations, updating them in the map using CRF to construct a semantic map suited for constrained indoor environments. Similarly, \cite{semi-dense, semantic3d, Incremental-dense} also leverage CRF for model optimization.
Semantic-OcTree \cite{Semantic-OcTree} proposes a Bayesian multi-class octree mapping approach, where the semantic categories are probabilistically updated through a probabilistic range-category perception model. Occ-vo \cite{Occ-vo} integrates 3D semantic occupancy and visual odometry, enhancing scene understanding.

However, these studies are based on closed-set semantic frameworks that typically rely on semantic segmentation models trained on limited datasets with fixed label sets. This reliance restricts their generalization to diverse scenes, resulting in coarse semantic understanding and restricting their applicability in open real-world environments.


\subsection{Open-vocabulary 3D mapping}

To overcome the limitations of closed-set semantics, many methods have been developed that leverage VLMs and LLMs to build open-vocabulary maps, allowing zero-shot generalization and providing visual language comprehension to perform real-world robotic tasks.
ConceptFusion \cite{conceptfusion} integrates various existing models along with local and global features to extract fundamental features for pixel alignment, which are then used to construct 3D point clouds.
Similarly, OpenScene \cite{peng2023openscene} and LERF \cite{lerf} develop point-level semantic maps for improved semantic representation. 
However, point-wise features pose significant challenges in querying specific instances, as they result in a fragmented representation of the target. Scattered perception does not adequately fulfill the requirements of practical works, and the associated feature storage demands are comparatively substantial.

To address these issues, some approaches use instances as primitives for scene understanding. OpenMask3D \cite{takmaz2023openmask3d} employs CLIP to obtain semantic feature embeddings for instance segmentation masks. MaskClustering \cite{maskclustering} uses a view consensus rate for mask fusion across frames, and then applies a method similar to \cite{takmaz2023openmask3d} to extract instance semantic features.
While these methods successfully achieve instance-level, open-set semantic feature embedding, they rely on some non-incremental strategies. This limits their applicability in real-time robotic systems, where continuous dynamic updates are crucial for practical deployment.

Recently, several algorithms have advanced incremental instance-level or region-level open-vocabulary mapping. Building on \cite{conceptfusion}, ConceptGraphs \cite{conceptgraphs} incrementally constructs 3D feature maps at the instance level, introducing spatial relationships between instances to form a topological graph. Open-Fusion \cite{open-fusion} uses SEEM to extract semantic features at the region level and employs TSDF for incremental reconstruction, integrating semantic information.
However, these methods mainly rely on VLM features for embedding, lacking language reasoning capabilities. Furthermore, they depend on IoU thresholds as a simplistic criterion for instance association, overlooking the potential for recovery from segmentation failures in the front-end model.

% In this paper, we design an efficient pipeline that leverages caption features to enable high-dimensional semantic reasoning for instances. The subsequent representation adopts a voxel structure, where each voxel stores an instance ID vector and is gradually updated using probabilistic methods.





\begin{figure*}
    \centering
    \includegraphics[width=\linewidth]{Visio/framework.pdf}
    \caption{The framework of OpenVox consists of two main modules: Instance Segmentation \& Understanding and Probabilistic Voxel Reconstruction. In the front-end, captions are encoded by LLMs to improve instance understanding. In the back-end, probabilistic modeling ensures the robustness of incremental instance-level mapping. The voxels in the final map are colored based on the instances with the highest probability.}
    \label{overview}
\end{figure*}