\usepackage{sidecap}
\usepackage{wrapfig}

% math envs
% \newtheorem{theorem}{Theorem}[section]
% \newtheorem{corollary}[theorem]{Corollary}
% \newtheorem{assumption}[theorem]{Assumption}
% \newtheorem{lemma}[theorem]{Lemma}
% \newtheorem{proposition}[theorem]{Proposition}
\newtheorem{claim}[theorem]{Claim}
% \newtheorem{definition}[theorem]{Definition}
% \newtheorem{remark}[theorem]{Remark}
\newtheorem{question}[theorem]{Question}
% \newtheorem{problem}{Problem}
\newtheorem{observation}[theorem]{Observation}
\newtheorem{fact}[theorem]{Fact}
\newtheorem{example}[theorem]{Example}
\newtheorem{thm}[theorem]{Theorem}
\newtheorem{cor}[theorem]{Corollary}
\newtheorem{lem}[theorem]{Lemma}
% \newtheorem{prop}[theorem]{Proposition}
% \newtheorem{note}[theorem]{Note}
\newtheorem{conjecture}{Conjecture}
% \newtheorem*{conjecture*}{Conjecture}
% \newtheoremstyle{nonindented}{1ex}{1ex}{}{}{\bfseries}{.}{.5em}{}
% \newtheoremstyle{indented}{1ex}{1ex}{\itshape\addtolength{\leftskip}{0.6cm}\addtolength{\rightskip}{0.6cm}}{}{\bfseries}{.}{.5em}{}
% \theoremstyle{nonindented}
%\newtheorem{domain}{Game Domain}
% \theoremstyle{indented}
%\newtheorem{question}{Question}[section]
%\newtheorem*{direction*}{Research Direction}
%\newtheorem*{conjecture*}{Conjecture}
% \theoremstyle{plain}
\newenvironment{proofof}[1]{\begin{proof}[Proof of #1]}{\end{proof}}
\newenvironment{proofsketch}{\begin{proof}[Proof Sketch]}{\end{proof}}
\newenvironment{proofsketchof}[1]{\begin{proof}[Proof Sketch of #1]}{\end{proof}}

\newenvironment{alg}{\begin{algorithm}\begin{onehalfspace}\begin{algorithmic}[1]}{\end{algorithmic}\end{onehalfspace}\end{algorithm}} 

% common symbols
% \newcommand{\abs}[1]{\left| #1 \right|}
\newcommand{\card}[1]{\left| #1 \right|}
% \newcommand{\set}[1]{\left\{ #1 \right\}}
\newcommand{\sm}{\setminus}
% \newcommand{\floor}[1]{\lfloor {#1} \rfloor}
% \newcommand{\ceil}[1]{\lceil {#1} \rceil}
\renewcommand{\hat}{\widehat}
\renewcommand{\tilde}{\widetilde}
\renewcommand{\bar}{\overline}
\newcommand{\iid}{i.i.d.\ }

\newcommand{\lt}{\left}
% \newcommand{\rt}{\right}


\DeclareMathOperator{\poly}{poly}
\DeclareMathOperator{\polylog}{polylog}

%Operators: These operators are such that a subscript appears below
%in \[ \] math mode, and to the bottom right in regular $ $ math mode

%regular version
\def\pr{\qopname\relax n{Pr}}
\def\ex{\qopname\relax n{E}}
\def\min{\qopname\relax n{min}}
\def\max{\qopname\relax n{max}}
\def\argmin{\qopname\relax n{argmin}}
\def\argmax{\qopname\relax n{argmax}}
\def\avg{\qopname\relax n{avg}}


%bold version
\def\Pr{\qopname\relax n{\mathbb{Pr}}}
\def\Ex{\qopname\relax n{\mathbb{E}}}
\def\supp{\qopname\relax n{\mathbf{supp}}}
\def\Min{\qopname\relax n{\mathbf{min}}}
\def\Max{\qopname\relax n{\mathbf{max}}}
\def\Argmin{\qopname\relax n{\mathbf{argmin}}}
\def\Argmax{\qopname\relax n{\mathbf{argmax}}}
\def\Avg{\qopname\relax n{\mathbf{avg}}}
\def\vol{\qopname\relax n{\mathbf{vol}}}
\def\aff{\qopname\relax n{\mathbf{aff}}}
\def\conv{\qopname\relax n{\mathbf{convexhull}}}
\def\hull{\qopname\relax n{\mathbf{hull}}}

\newcommand{\expect}[2][]{\ex_{#1} [#2]}

\newcommand{\RR}{\mathbb{R}}
\newcommand{\RRp}{\RR_+}
\newcommand{\RRpp}{\RR_{++}}
\newcommand{\NN}{\mathbb{N}}
\newcommand{\ZZ}{\mathbb{Z}}
\newcommand{\QQ}{\mathbb{Q}}

\def\bt{\boldsymbol{t}}
\def\bx{\boldsymbol{x}}

\def\A{\mathcal{A}}
\def\B{\mathcal{B}}
\def\C{\mathcal{C}}
\def\D{\mathcal{D}}
\def\E{\mathcal{E}}
\def\F{\mathcal{F}}
\def\G{\mathcal{G}}
\def\H{\mathcal{H}}
\def\I{\mathcal{I}}
\def\J{\mathcal{J}}
\def\L{\mathcal{L}}
\def\M{\mathcal{M}}
\def\P{\mathcal{P}}
\def\R{\mathcal{R}}
\def\S{\mathcal{S}}
\def\O{\mathcal{O}}
\def\T{\mathcal{T}}
\def\V{\mathcal{V}}
\def\X{\mathcal{X}}
\def\Y{\mathcal{Y}}

\def\eps{\epsilon}
\def\sse{\subseteq}

\newcommand{\pmat}[1]{\begin{pmatrix} #1 \end{pmatrix}}
\newcommand{\bmat}[1]{\begin{bmatrix} #1 \end{bmatrix}}
\newcommand{\Bmat}[1]{\begin{Bmatrix} #1 \end{Bmatrix}}
\newcommand{\vmat}[1]{\begin{vmatrix} #1 \end{vmatrix}}
\newcommand{\Vmat}[1]{\begin{Vmatrix} #1 \end{Vmatrix}}
\newcommand{\mat}[1]{\pmat{#1}}
\newcommand{\grad}{\bigtriangledown}
\newcommand{\one}{{\bf 1}}
\newcommand{\zero}{{\bf 0}}


%\newcommand{\hl}[1]{{\bf \color{red}#1}}
\newcommand{\todo}[1]{{\hl{To do: #1}}}
\newcommand{\todof}[1]{{\hl{ (Todo\footnote{\hl{\bf  To do: #1}})}}  }
\newcommand{\noi}{\noindent}
\newcommand{\prob}[1]{\vspace{.2in} \noindent {\bf Problem #1.}\\}
\newcommand{\subprob}[1]{ \vspace{.2in} \noindent {\bf #1.}\\}


%Plain eps or pdf figure. Use IPE to embed tex in it.
% \newcommand{\fig}[2][1.0]{
%   \begin{center}
%     \includegraphics[scale=#1]{#2}
%   \end{center}}
%Combined PS/Latex figure. This is option of choice for including tex
%code from xfig. Remember to export from xfig using "combined ps/latex" option
\newcommand{\figpst}[2][0.8]{
  \begin{center}
    \resizebox{#1\textwidth}{!}{\input{#2.pstex_t}}
  \end{center}}
%Combined PDF/Latex figure. This is option of choice for including tex
%code from xfig. Remember to export from xfig using "combined pdf/latex" option
\newcommand{\figpdft}[2][0.8]{
  \begin{center}
    \resizebox{#1\textwidth}{!}{\input{#2.pdf_t}}
  \end{center}}



%Algorithmic Environment stuff
\newcommand{\INPUT}{\item[\textbf{Input:}]}
\newcommand{\OUTPUT}{\item[\textbf{Output:}]}
\newcommand{\PARAMETER}{\item[\textbf{Parameter:}]}


%LP environment stuff
\newcommand{\mini}[1]{\mbox{minimize} & {#1} &\\}
\newcommand{\maxi}[1]{\mbox{maximize} & {#1 } & \\}
\newcommand{\maximin}[1]{\mbox{max min} & {#1 } & \\}
\newcommand{\find}[1]{\mbox{find} & {#1 } & \\}
%\newcommand{\st}{\mbox{subject to} }
\newcommand{\con}[1]{&#1 & \\}
\newcommand{\qcon}[2]{&#1, & \mbox{for } #2.  \\}
\newenvironment{lp}{\begin{equation}  \begin{array}{lll}}{\end{array}\end{equation}}
\newenvironment{lp*}{\begin{equation*}  \begin{array}{lll}}{\end{array}\end{equation*}}

%Linear Algebra
\newcommand{\dotmat}[4]{\begin{bmatrix}	#1 & \cdots & #2\\	\vdots & \ddots & \vdots\\	#3&\cdots& #4	\end{bmatrix}}
\newcommand{\Df}[2][]{D_#1f(#2)}
\newcommand{\ip}[1]{\langle #1 \rangle}
\newcommand{\0}{\vec{0}}
\newcommand{\pd}[2]{\frac{\partial #1}{\partial #2}}

%Permanent Commands
\newcommand{\lm}[2][t]{\lim\limits_{#1 \rightarrow #2}}
\newcommand{\lmz}[1][t]{\lim\limits_{#1 \rightarrow 0}}
\newcommand{\lmi}[1][t]{\lim\limits_{#1 \rightarrow \infty}}
\newcommand{\zti}{_{0}^{\infty}}
\newcommand{\vdg}{\vspace{5cm}}
% \newcommand{\ra}{\rightarrow}
\newcommand{\txt}[1]{&\text{#1}}
\newcommand{\func}[3]{#1: #2 \rightarrow #3}
\newcommand{\smm}[1][i]{\sum_{#1 = 0}^{n} }
\newcommand{\prd}[1][i]{\prod_{#1 = 1}^{n} }

%Statistics Commands
% \newcommand{\Var}[1]{\operatorname{Var}\big(#1\big)}
\newcommand{\Cor}[1]{\operatorname{Cor}\big(#1\big)}
% \newcommand{\Cov}[1]{\operatorname{Cov}\big(#1\big)}
\newcommand{\Pb}[1]{\mathsf{P}\big[#1\big]}
\newcommand{\Bias}[1]{\mathsf{Bias}\big[#1\big]}
\newcommand{\Dis}[2]{\sim \text{#1}(#2)}
\newcommand{\EE}[2][]{\mathbb{E}_{#1}\big[#2\big]}
\newcommand{\PP}[2][]{\mathbb{P}_{#1}\big[#2\big]}

%Complexity theory 
\newcommand{\Ot}[1]{\mathsf{O}\big[#1\big]}


%% MISC
\newcommand{\bs}[1]{\boldsymbol{#1}}
\newcommand{\softmax}{\operatorname{softmax}}


\usepackage[utf8]{inputenc} % allow utf-8 input
\usepackage[T1]{fontenc}    % use 8-bit T1 fonts
\usepackage{hyperref}       % hyperlinks
\usepackage{url}            % simple URL typesetting
\usepackage{booktabs}       % professional-quality tables
\usepackage{amsfonts}       % blackboard math symbols
\usepackage{nicefrac}       % compact symbols for 1/2, etc.
\usepackage{microtype}      % microtypography


\usepackage[ruled]{algorithm2e}
% \usepackage{algorithm}
% \usepackage{algorithmicx}

\usepackage{threeparttable}



\usepackage{graphicx}
\usepackage{subfigure}
\usepackage{booktabs} % for professional tables
\usepackage{caption}
\usepackage{framed}
\usepackage{amssymb}
\usepackage{mathrsfs}
\usepackage{mathtools}
\usepackage{array}
\usepackage{amsthm}
\usepackage{verbatim} 
\usepackage{enumerate}
\usepackage{bbm}
\usepackage{commath}
\usepackage{wrapfig}
\usepackage{amsbsy}
\usepackage{float}
\usepackage{soul}
\usepackage{xcolor}
\definecolor{custom2}{HTML}{F58157}
\definecolor{custom3}{HTML}{E7434C}
\definecolor{custom4}{HTML}{99216A}
\definecolor{custom5}{HTML}{64256E}
\definecolor{custom6}{HTML}{291956}
\usepackage{amsmath}
% \usepackage{algorithm}
\usepackage{tabularx}
\usepackage{listings}
\usepackage{xcolor}
\usepackage{colortbl}
% \usepackage{enumitem}
% \usepackage[shortlabels]{enumitem}

\usepackage{multirow}
\usepackage{graphicx}
\usepackage{lipsum} % For dummy text
\usepackage{paralist}
\usepackage[most]{tcolorbox}



\newcommand{\ie}{i.e.,\ }
\newcommand{\eg}{e.g.,\ }
\newcommand{\snc}{\textsc{Lazo}\xspace}

\newcommand{\xm}{x^{-}}
\newcommand{\xp}{x^{+}}
\newcommand{\model}{MolGroup\xspace}

\newcommand{\Set}[1]{\mathcal{#1}}
\newcommand{\htl}[1]{\textbf{\color{red}[(HTL: #1 )]}}  % to fix
\newcommand{\vpara}[1]{\vspace{0.07in}\noindent\textbf{#1 }}
\newcommand{\Mat}[1]{\mathbf{#1}}
% \newtheorem{prop}{Proposition}
% \newtheorem{problem}{Problem}

% \DeclareMathOperator*{\argmax}{arg\,max}
% \DeclareMathOperator*{\argmin}{arg\,min}



\newcommand \green[1]       {{\color[rgb]{0.10,0.50,0.10}#1}}
\newcommand \red[1]         {{\color{red}#1}}
\newcommand \blue[1]         {{\color{blue}#1}}
\newcommand{\betweenScriptAndSmall}{\fontsize{8}{9.6}\selectfont}



\definecolor{comment}{RGB}{70, 150, 60}
\newcommand{\note}[1]{\noindent{\color{red}\textbf{#1}}}
\newcommand{\notera}[1]{\note{\color{red}[\textsc{ra:} #1]}}
\newcommand{\noteps}[1]{\note{\color{blue}[\textsc{ps:} #1]}}
%\newcommand{\todo}[1]{\textbf{\color{red}[\textsc{todo:} #1]}}

\newcommand{\tinyskip}{\vspace{3pt}}
\newcommand{\mypar}[1]{\tinyskip\noindent\textbf{#1.}\xspace}
\newcommand{\code}[1]{\texttt{\small{}#1}\xspace}
% \newcommand{\fig}{\mbox{Fig.\hspace{0.25em}}}
% \newcommand{\alg}{\mbox{alg.\hspace{0.25em}}}
\newcommand{\tickYes}{\ding{51}}
\newcommand{\tickNo}{\ding{55}}
\newcommand{\din}{D_{in}}

\newenvironment{myitemize}{%
\begin{itemize}[leftmargin=1em, itemsep=.1em, parsep=.1em, topsep=.1em,
    partopsep=.1em]}
{\end{itemize}}

\newenvironment{myenumerate}{%
\begin{enumerate}[leftmargin=1em, itemsep=.1em, parsep=.1em, topsep=.1em,
    partopsep=.1em]}
{\end{enumerate}}

\newenvironment{structure}{\color{blue}\begin{myitemize}}{\end{myitemize}}
\newenvironment{structure*}{\color{blue}\begin{myenumerate}}{\end{myenumerate}}

% REVIEW COMMANDS 
% Switch the alternatives to activate de-activate the review highlighting
% \sethlcolor{yellow}
%\renewcommand*{\raggedrightmarginnote}{\centering}
%\renewcommand*{\raggedleftmarginnote}{\centering}
\newcommand{\update}[3][0em]{\marginnote{\textbf{#2}}[#1]\hl{#3}}
%\newcommand{\update}[3][0em]{\todo{\textbf{#2}}[#1]\hl{#3}}
%\newcommand{\update}[3][0em]{#3}
\newcommand{\updateL}[3][0em]{\reversemarginpar\setlength{\marginparwidth}{1.2cm}\marginnote{\textbf{#2}}[#1]\hl{#3}\normalmarginpar}
%\newcommand{\updateL}[3][0em]{#3}

\hypersetup{
    colorlinks,
    linkcolor={red!50!black},
    citecolor={blue!50!black},
    urlcolor={blue!80!black}
}

\newcommand{\Hilight}{\makebox[0pt][l]{\color{yellow!30}\rule[-4pt]{0.99\linewidth}{12pt}}}


\lstset{
    language=Python,
    basicstyle=\scriptsize\ttfamily,
    breaklines=true,                  % Wrap long lines
    keywordstyle=\color{blue},        % Styling keywords
    commentstyle=\color{gray},       % Styling comments
    stringstyle=\color{red},
    lineskip=-0.5pt,                    % Reduce line spacing slightly
    columns=fixed,
    basewidth=0.5em,                    % Reduce spacing between columns/characters
    aboveskip=3mm,
    belowskip=3mm
}

\lstdefinestyle{compressedstyle}{
    language=Python,
    basicstyle=\ttfamily\scriptsize,  % Smaller font for code
    breaklines=true,
    aboveskip=0pt,  % Reduce space above code
    belowskip=0pt,  % Reduce space below code
    lineskip=-2pt,  % Reduce line spacing
}

% \newtcblisting{compressedlisting}[2][]{
%     listing only,
%     arc=1mm,  % Smaller arc for corners
%     outer arc=1mm,
%     top=0mm,  % Smaller top margin
%     bottom=0mm,  % Smaller bottom margin
%     left=3mm,  % Smaller left margin
%     right=3mm,  % Smaller right margin
%     boxsep=2mm,  % Smaller padding inside the box
%     titlerule=0mm,  %// No space between title and box content
%     titlerule style={opacity=0},  %// Make title rule invisible
%     title after break={},  %// No title in break continuation
%     colback=white,
%     listing options={style=compressedstyle},
%     title=#2,
%     #1
% }

% \lstdefinestyle{hlstyle}{
%     language=Python,
%     basicstyle=\scriptsize\ttfamily,
%     breaklines=true,                  % Wrap long lines
%     keywordstyle=\color{blue},        % Styling keywords
%     commentstyle=\color{gray},       % Styling comments
%     stringstyle=\color{red},
%     backgroundcolor=\color{grey!30}
%     % lineskip=-0.5pt,                    % Reduce line spacing slightly
%     columns=fixed,
%     % basewidth=0.5em,                    % Reduce spacing between columns/characters
%     % aboveskip=3mm,
%     % belowskip=3mm,
% }



% to fix the huge spacing of this stupid template
% \setlength{\dbltextfloatsep}{1pt}  % for double column tables
% \setlength{\textfloatsep}{1pt}
% \setlength{\intextsep}{1pt}
% \setlength{\dblfloatsep}{1pt}

% split lines and format within cells of tables
\newcommand{\specialcell}[2][c]{%
  \begin{tabular}[#1]{@{}c@{}}#2\end{tabular}}

\definecolor{lightorange}{RGB}{255,229,204}
\sethlcolor{lightorange}

\definecolor{lightblue}{RGB}{173,216,230}
\newcommand{\badcolor}[1]{\sethlcolor{lightblue}\hl{#1}}

\hyphenation{da-ta-sets}
\hyphenation{un-cer-ta-in-ty}
\hyphenation{un-cer-ta-in-ty--shi-eld}
\hyphenation{me-thod}
\hyphenation{cons-traint}
\hyphenation{cons-traints}
\hyphenation{attribute-cons-traints}
\hyphenation{attribute-constra-ints}
\hyphenation{ple-tho-ra}