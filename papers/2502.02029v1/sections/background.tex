\textbf{Background:}~A diffeomorphism \(\bphi: \mathbb{R}^D \to \mathbb{R}^D\), is a smooth, invertible mapping with a smooth inverse, defined as \(\bphi(\x) = \x + \bu(\x)\), where \(\bu(\x)\) is the displacement field. Diffeomorphic transformations can be constructed by composing multiple small deformations:
\(\bphi(\x) = (\x + \epsilon \bv_{1}(\x)) \circ (\x + \epsilon \bv_{2}(\x)) \circ \cdots \circ (\x + \epsilon \bv_{n}(\x))\). The space of diffeomorphisms \( \Diff(\mathcal{M}) \), forms a Lie group under composition, with the identity map, serving as the group’s identity element; i.e., \(\bphi \circ \mathbf{id} = \mathbf{id} \circ \bphi = \bphi, \forall \bphi \in \Diff(\mathcal{M})\). The associated Lie algebra consists of smooth vector fields \(\bv(\x): \mathbb{R}^D \to \mathbb{R}^D\), which generate diffeomorphisms through their flows, described by one-parameter subgroups \(\{\bphi_t\}_{t \in \mathbb{R}}\). The exponential map \(\exp(\bv)\) connects the Lie algebra to the Lie group, while the logarithm map \(\log(\bphi)\) serves as its local inverse. To compute the logarithm map, a non-linear inverse scaling and square rooting algorithm \cite{arsigny2006log} is used:
\begin{equation}\label{nl-iss}
\operatorname{log}(\bphi) = 2^N \operatorname{log}(\bphi^{{-2}^N}) \quad \text{where} \quad \operatorname{log}(\bphi^{{-2}^N}) = \bphi^{{-2}^N} - \mathbf{id} 
\end{equation}
