\vspace{-4mm}
\section{Related Work}
% conventional methods for registration
Traditional image registration methods align images by optimizing similarity metrics like mutual information, normalized cross-correlation, or the sum of squared differences. While accurate, they are time-consuming, particularly for high-resolution and large-scale datasets. Diffeomorphic (smooth and invertible) transformations are popular as they preserve anatomical topology in deformation analysis \cite{fu2020deep}. However, these approaches are computationally expensive, limiting scalability due to the need to enforce smoothness, invertibility, and anatomical consistency. 
%
Deep learning methods have emerged as faster and scalable alternatives. Early models used convolutional networks like U-Net \cite{ronneberger2015u} to predict deformation fields directly, reducing computational time \cite{boveiri2020medical}. However, their limited receptive fields led to the adoption of transformer-based models, which better capture global spatial relationships \cite{chen2022transmorph}. Despite these advancements, both approaches rely on smoothness regularization, leading to anatomically inconsistent deformations.
%, primarily when population-level morphological patterns are not explicitly modeled.

% Statistical analysis of deformation fields 
Independent efforts have advanced deformation field analysis tools. PGA \cite{fletcher2004principal, fletcher2011geodesic} extends Principal Component Analysis (PCA) to Riemannian manifolds. While PGA offers a rigorous approach to analyzing deformations on Riemannian manifolds, defining an appropriate Riemannian metric for complex anatomical variations remains challenging.
% Log Euclidean framework
The Log-Euclidean framework \cite{arsigny2006log} provides a computationally efficient approach to perform statistics on diffeomorphisms by mapping them to a linear vector space via principal logarithms. This framework enables using Euclidean operations on logarithms, preserves key mathematical properties, and simplifies computations compared to Riemannian approaches. However, estimating principal logarithms using the iterative method is computationally expensive, particularly for high-dimensional data or large deformations.

Population-aware registration networks, such as CAE \cite{bhalodia2019cooperative}, add regularization by limiting deformation fields to low-dimensional manifolds, effectively capturing anatomical variability. However, they often fall short of explicitly enforcing diffeomorphism essential for interpretability and robustness. CAE's regularization approach begins with an L2 regularizer, providing initial stability. However, when the autoencoder-based regularizer is introduced, it struggles to preserve anatomical consistency, often resulting in non-diffeomorphic transformations.

%CAE's autoencoder-based regularizer relies on gradient L2 smoothness and struggles, leading to anatomically inconsistent and non-diffeomorphic results.
%
Similarly, diffeomorphic autoencoders for LDDMM \cite{hinkle2018diffeomorphic} have been proposed for atlas building by encoding velocity field-based transformations into latent spaces for statistical analysis. These methods face computational challenges with large datasets and complex anatomical variations. Additionally, the focus on deformation field reconstruction from latent space representation often sacrifices interpretability, limiting their utility for downstream tasks such as population analysis. The Log Euclidean Diffeomorphic Autoencoder (LEDA) \cite{iyer2024leda} framework has combined the advantages of the Log Euclidean framework with a deep learning approach. LEDA estimates the principal logarithms of the deformation field while maintaining inverse consistency in both the deformation and latent spaces. This approach efficiently captures complex anatomical variations while remaining computationally tractable for large-scale analysis. It maps composition in the data space to linearlized latent space, enabling vector-based statistics and ensuring robustness while adhering to diffeomorphism group laws.

To address the crucial gap in integrating population-level statistics with image registration models in an end-to-end manner, we propose \model, a Log-Euclidean population-driven image registration framework. By leveraging collective anatomical insights, \model~ensures anatomically meaningful transformations while maintaining computational efficiency. 

