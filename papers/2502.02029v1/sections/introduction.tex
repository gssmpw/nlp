\section{Introduction}
% 1. Motivation: Image Registration and Spatial Transformations
Image registration is crucial in medical image analysis. It entails determining a one-to-one mapping of pixel coordinates between images to align corresponding anatomical points. Registration algorithms establish correspondences that enable the creation of population-level atlases, providing standardized references for studying disease progression, pathology detection, treatment planning, and motion tracking \cite{suganyadevi2022review,zachiu2020anatomically,binte2020spatiotemporal, viergever2016survey}. The versatility of image registration methods makes them indispensable across various imaging modalities \cite{huang2022reconet}.
% 2. Diffeomorphisms: A Mathematical Framework for Spatial Transforms
Diffeomorphisms are smooth, invertible spatial transformations that preserve image topology and are a key class of spatial transformations. They prevent artifacts like tearing or folding, enabling precise comparisons of anatomical structures across patients and time points. Their flexibility makes them crucial in medical imaging, particularly for capturing complex deformations, aiding early diagnosis and treatment. 
%
% 3. Challenges in Existing Approaches
Traditional methods like Large Deformation Diffeomorphic Metric Mapping (LDDMM) \cite{beg2005computing,joshi2000landmark}, optical flow \cite{zhai2021optical},  Direct Deformation Estimation (DDE) \cite{boyle2019regularization} optimize for transformations using metric distances or pixel movement vectors. Despite their effectiveness, these methods struggle with large deformations, inter-subject anatomical variability, and computational efficiency. Pairwise registration often biases results toward the reference image and struggles to deal with significant anatomical differences.

Deep learning methods address these challenges by learning complex transformations directly from images, offering faster inference and ensuring smooth deformation fields through regularizers like bending energy or \(L_2-\)gradient penalties \cite{balakrishnan2019voxelmorph,chen2022transmorph}. However, they treat registration as a computational problem, neglecting diffeomorphic properties like inversion and composition, resulting in biologically implausible solutions. Improvements like gradient and transformation inverse consistency enhance diffeomorphic transformations \cite{tian2023gradicon}. Yet, existing approaches often predict transformations without deeper insights into underlying anatomical relationships, oversimplifying interpretations.
%
% 4. Critical Drawbacks in Statistical Analysis of Diffeomorphisms
Nevertheless, the mathematical framework of diffeomorphic transformations, defined by the diffeomorphism group \(Diff(\mathcal{M})\) (\(dim(\mathcal{M})>0\)) which is a Fréchet Lie group and manifold, provides a rigorous foundation for capturing and analyzing complex anatomical variability.
%
Unlike Euclidean spaces, operations like addition or averaging cannot be defined in \(Diff(\mathcal{M})\). Techniques such as Principal Geodesic Analysis (PGA) \cite{fletcher2004principal}, Fréchet means \cite{le2000frechet}, and geodesic regression \cite{fletcher2011geodesic} attempt to address statistical estimation for Riemannian manifolds but require significant adaptation for Frechet Lie groups. Therefore, alternative metrics are necessary to model relationships between deformation fields and ensure anatomical relevance. The Log-Euclidean framework \cite{arsigny2006log} transforms non-linear diffeomorphism manifolds into linear Lie algebra spaces, enabling efficient computation of transformation distances and statistical analysis. However, iterative processes used in logarithm estimation can be susceptible to initializations, potentially leading to suboptimal results. Standalone deformation analysis models address the limitations of the iterative methods and provide a computationally efficient transformation of diffeomorphic groups into linear Lie algebra. These mathematical foundations provide a rigorous framework for analyzing the complex structure of diffeomorphisms, leading to more reliable and interpretable results. 

% 5. Need for a Unified Approach
However, a unified approach is needed to bridge computational methods for generating diffeomorphisms and statistical techniques for their analysis. To address this gap, we introduce \model, a Log-Euclidean regularization framework for population-aware unsupervised image registration. By leveraging population morphometrics, \model~guides and regularizes registration networks while enforcing inverse consistency in latent and diffeomorphic spaces, enabling data-driven regularization for valid transformations and enhanced morphological analysis. We validate \model~as a plug-and-play regularizer for deep learning registration networks and demonstrate the model's efficiency, robustness, and broad applicability.
