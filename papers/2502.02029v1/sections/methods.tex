\begin{figure}[!t]
    \centering
    \includegraphics[width=\linewidth]{figures/arch_and_atlas.png}
    \vspace{-6mm}
    \caption{\small Proposed Architecture for \model~on the left and the estimated atlas using the algorithm described in section~\ref{results} compared with pixel-wise average atlas on the right}
    \label{architecture}
    \vspace{-8mm}
\end{figure}
\vspace{-7mm}
\section{Methods}
\section{Background}\label{sec:backgrnd}

\subsection{Cold Start Latency and Mitigation Techniques}

Traditional FaaS platforms mitigate cold starts through snapshotting, lightweight virtualization, and warm-state management. Snapshot-based methods like \textbf{REAP} and \textbf{Catalyzer} reduce initialization time by preloading or restoring container states but require significant memory and I/O resources, limiting scalability~\cite{dong_catalyzer_2020, ustiugov_benchmarking_2021}. Lightweight virtualization solutions, such as \textbf{Firecracker} microVMs, achieve fast startup times with strong isolation but depend on robust infrastructure, making them less adaptable to fluctuating workloads~\cite{agache_firecracker_2020}. Warm-state management techniques like \textbf{Faa\$T}~\cite{romero_faa_2021} and \textbf{Kraken}~\cite{vivek_kraken_2021} keep frequently invoked containers ready, balancing readiness and cost efficiency under predictable workloads but incurring overhead when demand is erratic~\cite{romero_faa_2021, vivek_kraken_2021}. While these methods perform well in resource-rich cloud environments, their resource intensity challenges applicability in edge settings.

\subsubsection{Edge FaaS Perspective}

In edge environments, cold start mitigation emphasizes lightweight designs, resource sharing, and hybrid task distribution. Lightweight execution environments like unikernels~\cite{edward_sock_2018} and \textbf{Firecracker}~\cite{agache_firecracker_2020}, as used by \textbf{TinyFaaS}~\cite{pfandzelter_tinyfaas_2020}, minimize resource usage and initialization delays but require careful orchestration to avoid resource contention. Function co-location, demonstrated by \textbf{Photons}~\cite{v_dukic_photons_2020}, reduces redundant initializations by sharing runtime resources among related functions, though this complicates isolation in multi-tenant setups~\cite{v_dukic_photons_2020}. Hybrid offloading frameworks like \textbf{GeoFaaS}~\cite{malekabbasi_geofaas_2024} balance edge-cloud workloads by offloading latency-tolerant tasks to the cloud and reserving edge resources for real-time operations, requiring reliable connectivity and efficient task management. These edge-specific strategies address cold starts effectively but introduce challenges in scalability and orchestration.

\subsection{Predictive Scaling and Caching Techniques}

Efficient resource allocation is vital for maintaining low latency and high availability in serverless platforms. Predictive scaling and caching techniques dynamically provision resources and reduce cold start latency by leveraging workload prediction and state retention.
Traditional FaaS platforms use predictive scaling and caching to optimize resources, employing techniques (OFC, FaasCache) to reduce cold starts. However, these methods rely on centralized orchestration and workload predictability, limiting their effectiveness in dynamic, resource-constrained edge environments.



\subsubsection{Edge FaaS Perspective}

Edge FaaS platforms adapt predictive scaling and caching techniques to constrain resources and heterogeneous environments. \textbf{EDGE-Cache}~\cite{kim_delay-aware_2022} uses traffic profiling to selectively retain high-priority functions, reducing memory overhead while maintaining readiness for frequent requests. Hybrid frameworks like \textbf{GeoFaaS}~\cite{malekabbasi_geofaas_2024} implement distributed caching to balance resources between edge and cloud nodes, enabling low-latency processing for critical tasks while offloading less critical workloads. Machine learning methods, such as clustering-based workload predictors~\cite{gao_machine_2020} and GRU-based models~\cite{guo_applying_2018}, enhance resource provisioning in edge systems by efficiently forecasting workload spikes. These innovations effectively address cold start challenges in edge environments, though their dependency on accurate predictions and robust orchestration poses scalability challenges.

\subsection{Decentralized Orchestration, Function Placement, and Scheduling}

Efficient orchestration in serverless platforms involves workload distribution, resource optimization, and performance assurance. While traditional FaaS platforms rely on centralized control, edge environments require decentralized and adaptive strategies to address unique challenges such as resource constraints and heterogeneous hardware.



\subsubsection{Edge FaaS Perspective}

Edge FaaS platforms adopt decentralized and adaptive orchestration frameworks to meet the demands of resource-constrained environments. Systems like \textbf{Wukong} distribute scheduling across edge nodes, enhancing data locality and scalability while reducing network latency. Lightweight frameworks such as \textbf{OpenWhisk Lite}~\cite{kravchenko_kpavelopenwhisk-light_2024} optimize resource allocation by decentralizing scheduling policies, minimizing cold starts and latency in edge setups~\cite{benjamin_wukong_2020}. Hybrid solutions like \textbf{OpenFaaS}~\cite{noauthor_openfaasfaas_2024} and \textbf{EdgeMatrix}~\cite{shen_edgematrix_2023} combine edge-cloud orchestration to balance resource utilization, retaining latency-sensitive functions at the edge while offloading non-critical workloads to the cloud. While these approaches improve flexibility, they face challenges in maintaining coordination and ensuring consistent performance across distributed nodes.



\noindent \textbf{\model:}~The proposed model \model~is illustrated in Figure \ref{architecture} combines a primary network performing registration task and the secondary network functioning as a regularizer constrains the solution space of the primary task to ensure anatomically meaningful transformations. 

\vspace{0.05in}
\noindent \textit{Primary Registration Network}~can be any registration module that is designed to produce deformation fields. Given a pair of images A and B, the primary network is tasked with learning two displacement fields: \(\bphi_{AB}, \bphi_{BA}\) where \(\bphi_{AB}\) corresponds to the warp that ideally should match image A to image B, while \(\bphi_{BA}\) represents the inverse transformation from B to A. The network produces these displacement fields simultaneously to enable inverse consistency regularization. The displacement fields and their respective source images are passed through a spatial transform unit to produce registered images. The primary network incorporates inverse consistency regularization to ensure reliable bi-directional mappings: \(\bphi_{AB} \cdot \bphi_{BA} = id, \bphi_{BA} \cdot \bphi_{AB} = id \). This constraint encourages the network to learn transformations that are as close to being inverses of each other as possible, improving the overall accuracy of the registration process. 
The loss function for the primary network includes: (a) similarity loss: \(L_{sim}=SIM(A,\bphi_{AB}\circ B)+SIM(B,\bphi_{BA}\circ A)\) and (b) inverse consistency loss: \(L_{reg}=\|\bphi_{AB}\cdot \bphi_{BA}-id\|^2+\|\bphi_{BA}\cdot \bphi_{AB}-id\|^2\). The total loss is a weighted sum of these components where \(\lambda\) represents the weight for the term:
\begin{equation} \label{reg_loss}
L_{P}=\lambda_{sim} L_{sim}+\lambda_{reg} L_{reg}
\end{equation}

\vspace{0.05in}
\noindent \textit{Secondary Population-based Regularization Network}~uses Log Euclidean Diffeomorphic Autoencoder \cite{iyer2024leda}, which is strongly rooted in the Log-Euclidean statistics framework as the population-based regularizer. LEDA predicts \(N\) successive square roots of the deformation field, enabling accurate and computationally efficient logarithmic approximations using equation~\ref{nl-iss}. The encoder-decoder architecture is defined as:
\(
f_{\gamma}(\phi) = z, \quad g_{\theta}\left(\frac{z}{m}\right) = \phi^{1/m}, \) where \(m = 2^n, \, n \in \{0, 1, \dots, N\}\).
LEDA uses the following loss terms:

\vspace{0.05in}
\noindent \textit{Reconstruction Loss:} The deformation field must be accurately reconstructed from its predicted roots. If \(\bphi^{-m} \) is the predicted root at stage \(n\) where \(m = {2}^n\), the deformation field, when composed \(m\) times, must match the original deformation field, the reconstruction loss is \(\mathcal{L}_{rec} = \)
\begin{equation}
    \sum_k \sum_n \left\|C_m(\widehat{\bphi}_{AB}^{-m}) - \bphi_{AB}\right\|^2 + \left\|C_m(\widehat{\bphi}_{BA}^{-m}) - \bphi_{BA}\right\|^2 \text{ where } C_m(\widehat{\bphi}^{-m}) = \underbrace{\bphi \circ \dots \circ \bphi}_{m \text{ times}} \approx \bphi.
\end{equation}

\vspace{0.05in}
\noindent \textit{Inverse Consistency Losses:} Even though the primary network enforces inverse consistency, we also want the successive estimated roots of the forward and inverse deformation fields to compose to identity. This is enforced through the inverse consistency loss at each root approximation stage. Along with the deformation field inverse consistency, we also want the latent representations of forward and inverse transformations to be consistent, with equal magnitude but opposite directions: \(\z_{AB} = -\z_{BA}\). This is enforced through a latent inverse consistency loss that combines cosine similarity \(\Theta_k\) between \(\z_{AB}, \z_{BA}\)and magnitude constraints.
\begin{equation}
\mathcal{L}_{inv} = \sum_k \sum_n \left\| \widehat{\bphi}_{AB}^{-m} \circ \widehat{\bphi}_{BA}^{-m} - \mathbf{id} \right\|^2 \text{and } \mathcal{L}_{linv} = \sum_k \left(\frac{1 + \cos(\Theta_k)}{2} + \left\|\z_{AB} + \z_{BA}\right\|^2\right)
\end{equation}
The total loss function for LEDA is given by :
\begin{equation} \label{leda_loss}
\mathcal{L}_{S} = \alpha_{rec}\mathcal{L}_{rec} + \alpha_{inv}\mathcal{L}_{inv} + \alpha_{linv}\mathcal{L}_{linv}.
\end{equation}
The total loss function for training the proposed model is \(\mathcal{L}_{total} =  \mathcal{L}_{P} + \lambda_1\mathcal{L}_{S} \). The training process for \model~follows a two-phase strategy. Initially, we set \(\lambda_1 = 0\) (no LEDA regularizer), allowing the primary network to learn forward and backward diffeomorphic mappings. Subsequently, we activate the LEDA regularizer \(\lambda_1>1\), enabling simultaneous training of both primary and secondary networks. This approach ensures the primary network learns diffeomorphic transformations before introducing population-based regulation, leading to more accurate and anatomically consistent results.

