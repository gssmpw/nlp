%% bare_jrnl.tex
%% V1.4b
%% 2015/08/26
%% by Michael Shell
%% see http://www.michaelshell.org/
%% for current contact information.
%%
%% This is a skeleton file demonstrating the use of IEEEtran.cls
%% (requires IEEEtran.cls version 1.8b or later) with an IEEE
%% journal paper.
%%
%% Support sites:
%% http://www.michaelshell.org/tex/ieeetran/
%% http://www.ctan.org/pkg/ieeetran
%% and
%% http://www.ieee.org/

%%*************************************************************************
%% Legal Notice:
%% This code is offered as-is without any warranty either expressed or
%% implied; without even the implied warranty of MERCHANTABILITY or
%% FITNESS FOR A PARTICULAR PURPOSE! 
%% User assumes all risk.
%% In no event shall the IEEE or any contributor to this code be liable for
%% any damages or losses, including, but not limited to, incidental,
%% consequential, or any other damages, resulting from the use or misuse
%% of any information contained here.
%%
%% All comments are the opinions of their respective authors and are not
%% necessarily endorsed by the IEEE.
%%
%% This work is distributed under the LaTeX Project Public License (LPPL)
%% ( http://www.latex-project.org/ ) version 1.3, and may be freely used,
%% distributed and modified. A copy of the LPPL, version 1.3, is included
%% in the base LaTeX documentation of all distributions of LaTeX released
%% 2003/12/01 or later.
%% Retain all contribution notices and credits.
%% ** Modified files should be clearly indicated as such, including  **
%% ** renaming them and changing author support contact information. **
%%*************************************************************************


% *** Authors should verify (and, if needed, correct) their LaTeX system  ***
% *** with the testflow diagnostic prior to trusting their LaTeX platform ***
% *** with production work. The IEEE's font choices and paper sizes can   ***
% *** trigger bugs that do not appear when using other class files.       ***                          ***
% The testflow support page is at:
% http://www.michaelshell.org/tex/testflow/



\documentclass[journal]{IEEEtran}
\usepackage{graphicx}
\usepackage{textcomp}
\usepackage{xcolor}
\usepackage{colortbl}
\definecolor{mygray}{gray}{0.98}
\definecolor{mygray1}{gray}{0.93}
\usepackage{algorithm}  
\usepackage{algpseudocode}  
\usepackage{amsmath}  
\usepackage{amsthm}
\usepackage{amsfonts}
\usepackage{amssymb}
\usepackage{latexsym}
\usepackage{caption, subcaption}
\usepackage{enumitem}
% \bibliography{biblio}
\usepackage{booktabs}
\usepackage{listings}
% \usepackage{epstopdf}
\usepackage{chngpage}
\usepackage{flushend}
\usepackage{makecell}
\usepackage{adjustbox}
\usepackage{bbding}
%\usepackage[table]{xcolor}
\usepackage{utfsym}
\usepackage{autobreak}
\usepackage{multirow}
\usepackage{siunitx}

\usepackage{textcomp,booktabs}
\usepackage{wasysym}
\newcommand{\circled}[1]{\normalsize{\textcircled{\scriptsize{#1}}}\normalsize\;}
\usepackage{ragged2e}
\definecolor{seagreen}{rgb}{0.18, 0.55, 0.34}
\definecolor{royalpurple}{rgb}{0.47,0.32,0.66}
\definecolor{brown(traditional)}{rgb}{0.59, 0.29, 0.0}
\definecolor{blue}{rgb}{0.3, 0.2, 0.9}
\usepackage[colorlinks,
            linkcolor=blue,
            anchorcolor=blue,
            citecolor=blue]{hyperref}
%
% If IEEEtran.cls has not been installed into the LaTeX system files,
% manually specify the path to it like:
% \documentclass[journal]{../sty/IEEEtran}





% Some very useful LaTeX packages include:
% (uncomment the ones you want to load)


% *** MISC UTILITY PACKAGES ***
%
%\usepackage{ifpdf}
% Heiko Oberdiek's ifpdf.sty is very useful if you need conditional
% compilation based on whether the output is pdf or dvi.
% usage:
% \ifpdf
%   % pdf code
% \else
%   % dvi code
% \fi
% The latest version of ifpdf.sty can be obtained from:
% http://www.ctan.org/pkg/ifpdf
% Also, note that IEEEtran.cls V1.7 and later provides a builtin
% \ifCLASSINFOpdf conditional that works the same way.
% When switching from latex to pdflatex and vice-versa, the compiler may
% have to be run twice to clear warning/error messages.






% *** CITATION PACKAGES ***
%
%\usepackage{cite}
% cite.sty was written by Donald Arseneau
% V1.6 and later of IEEEtran pre-defines the format of the cite.sty package
% \cite{} output to follow that of the IEEE. Loading the cite package will
% result in citation numbers being automatically sorted and properly
% "compressed/ranged". e.g., [1], [9], [2], [7], [5], [6] without using
% cite.sty will become [1], [2], [5]--[7], [9] using cite.sty. cite.sty's
% \cite will automatically add leading space, if needed. Use cite.sty's
% noadjust option (cite.sty V3.8 and later) if you want to turn this off
% such as if a citation ever needs to be enclosed in parenthesis.
% cite.sty is already installed on most LaTeX systems. Be sure and use
% version 5.0 (2009-03-20) and later if using hyperref.sty.
% The latest version can be obtained at:
% http://www.ctan.org/pkg/cite
% The documentation is contained in the cite.sty file itself.






% *** GRAPHICS RELATED PACKAGES ***
%
\ifCLASSINFOpdf
  % \usepackage[pdftex]{graphicx}
  % declare the path(s) where your graphic files are
  % \graphicspath{{../pdf/}{../jpeg/}}
  % and their extensions so you won't have to specify these with
  % every instance of \includegraphics
  % \DeclareGraphicsExtensions{.pdf,.jpeg,.png}
\else
  % or other class option (dvipsone, dvipdf, if not using dvips). graphicx
  % will default to the driver specified in the system graphics.cfg if no
  % driver is specified.
  % \usepackage[dvips]{graphicx}
  % declare the path(s) where your graphic files are
  % \graphicspath{{../eps/}}
  % and their extensions so you won't have to specify these with
  % every instance of \includegraphics
  % \DeclareGraphicsExtensions{.eps}
\fi
% graphicx was written by David Carlisle and Sebastian Rahtz. It is
% required if you want graphics, photos, etc. graphicx.sty is already
% installed on most LaTeX systems. The latest version and documentation
% can be obtained at: 
% http://www.ctan.org/pkg/graphicx
% Another good source of documentation is "Using Imported Graphics in
% LaTeX2e" by Keith Reckdahl which can be found at:
% http://www.ctan.org/pkg/epslatex
%
% latex, and pdflatex in dvi mode, support graphics in encapsulated
% postscript (.eps) format. pdflatex in pdf mode supports graphics
% in .pdf, .jpeg, .png and .mps (metapost) formats. Users should ensure
% that all non-photo figures use a vector format (.eps, .pdf, .mps) and
% not a bitmapped formats (.jpeg, .png). The IEEE frowns on bitmapped formats
% which can result in "jaggedy"/blurry rendering of lines and letters as
% well as large increases in file sizes.
%
% You can find documentation about the pdfTeX application at:
% http://www.tug.org/applications/pdftex





% *** MATH PACKAGES ***
%
%\usepackage{amsmath}
% A popular package from the American Mathematical Society that provides
% many useful and powerful commands for dealing with mathematics.
%
% Note that the amsmath package sets \interdisplaylinepenalty to 10000
% thus preventing page breaks from occurring within multiline equations. Use:
%\interdisplaylinepenalty=2500
% after loading amsmath to restore such page breaks as IEEEtran.cls normally
% does. amsmath.sty is already installed on most LaTeX systems. The latest
% version and documentation can be obtained at:
% http://www.ctan.org/pkg/amsmath





% *** SPECIALIZED LIST PACKAGES ***
%
%\usepackage{algorithmic}
% algorithmic.sty was written by Peter Williams and Rogerio Brito.
% This package provides an algorithmic environment fo describing algorithms.
% You can use the algorithmic environment in-text or within a figure
% environment to provide for a floating algorithm. Do NOT use the algorithm
% floating environment provided by algorithm.sty (by the same authors) or
% algorithm2e.sty (by Christophe Fiorio) as the IEEE does not use dedicated
% algorithm float types and packages that provide these will not provide
% correct IEEE style captions. The latest version and documentation of
% algorithmic.sty can be obtained at:
% http://www.ctan.org/pkg/algorithms
% Also of interest may be the (relatively newer and more customizable)
% algorithmicx.sty package by Szasz Janos:
% http://www.ctan.org/pkg/algorithmicx




% *** ALIGNMENT PACKAGES ***
%
%\usepackage{array}
% Frank Mittelbach's and David Carlisle's array.sty patches and improves
% the standard LaTeX2e array and tabular environments to provide better
% appearance and additional user controls. As the default LaTeX2e table
% generation code is lacking to the point of almost being broken with
% respect to the quality of the end results, all users are strongly
% advised to use an enhanced (at the very least that provided by array.sty)
% set of table tools. array.sty is already installed on most systems. The
% latest version and documentation can be obtained at:
% http://www.ctan.org/pkg/array


% IEEEtran contains the IEEEeqnarray family of commands that can be used to
% generate multiline equations as well as matrices, tables, etc., of high
% quality.




% *** SUBFIGURE PACKAGES ***
%\ifCLASSOPTIONcompsoc
%  \usepackage[caption=false,font=normalsize,labelfont=sf,textfont=sf]{subfig}
%\else
%  \usepackage[caption=false,font=footnotesize]{subfig}
%\fi
% subfig.sty, written by Steven Douglas Cochran, is the modern replacement
% for subfigure.sty, the latter of which is no longer maintained and is
% incompatible with some LaTeX packages including fixltx2e. However,
% subfig.sty requires and automatically loads Axel Sommerfeldt's caption.sty
% which will override IEEEtran.cls' handling of captions and this will result
% in non-IEEE style figure/table captions. To prevent this problem, be sure
% and invoke subfig.sty's "caption=false" package option (available since
% subfig.sty version 1.3, 2005/06/28) as this is will preserve IEEEtran.cls
% handling of captions.
% Note that the Computer Society format requires a larger sans serif font
% than the serif footnote size font used in traditional IEEE formatting
% and thus the need to invoke different subfig.sty package options depending
% on whether compsoc mode has been enabled.
%
% The latest version and documentation of subfig.sty can be obtained at:
% http://www.ctan.org/pkg/subfig




% *** FLOAT PACKAGES ***
%
%\usepackage{fixltx2e}
% fixltx2e, the successor to the earlier fix2col.sty, was written by
% Frank Mittelbach and David Carlisle. This package corrects a few problems
% in the LaTeX2e kernel, the most notable of which is that in current
% LaTeX2e releases, the ordering of single and double column floats is not
% guaranteed to be preserved. Thus, an unpatched LaTeX2e can allow a
% single column figure to be placed prior to an earlier double column
% figure.
% Be aware that LaTeX2e kernels dated 2015 and later have fixltx2e.sty's
% corrections already built into the system in which case a warning will
% be issued if an attempt is made to load fixltx2e.sty as it is no longer
% needed.
% The latest version and documentation can be found at:
% http://www.ctan.org/pkg/fixltx2e


%\usepackage{stfloats}
% stfloats.sty was written by Sigitas Tolusis. This package gives LaTeX2e
% the ability to do double column floats at the bottom of the page as well
% as the top. (e.g., "\begin{figure*}[!b]" is not normally possible in
% LaTeX2e). It also provides a command:
%\fnbelowfloat
% to enable the placement of footnotes below bottom floats (the standard
% LaTeX2e kernel puts them above bottom floats). This is an invasive package
% which rewrites many portions of the LaTeX2e float routines. It may not work
% with other packages that modify the LaTeX2e float routines. The latest
% version and documentation can be obtained at:
% http://www.ctan.org/pkg/stfloats
% Do not use the stfloats baselinefloat ability as the IEEE does not allow
% \baselineskip to stretch. Authors submitting work to the IEEE should note
% that the IEEE rarely uses double column equations and that authors should try
% to avoid such use. Do not be tempted to use the cuted.sty or midfloat.sty
% packages (also by Sigitas Tolusis) as the IEEE does not format its papers in
% such ways.
% Do not attempt to use stfloats with fixltx2e as they are incompatible.
% Instead, use Morten Hogholm'a dblfloatfix which combines the features
% of both fixltx2e and stfloats:
%
% \usepackage{dblfloatfix}
% The latest version can be found at:
% http://www.ctan.org/pkg/dblfloatfix




%\ifCLASSOPTIONcaptionsoff
%  \usepackage[nomarkers]{endfloat}
% \let\MYoriglatexcaption\caption
% \renewcommand{\caption}[2][\relax]{\MYoriglatexcaption[#2]{#2}}
%\fi
% endfloat.sty was written by James Darrell McCauley, Jeff Goldberg and 
% Axel Sommerfeldt. This package may be useful when used in conjunction with 
% IEEEtran.cls'  captionsoff option. Some IEEE journals/societies require that
% submissions have lists of figures/tables at the end of the paper and that
% figures/tables without any captions are placed on a page by themselves at
% the end of the document. If needed, the draftcls IEEEtran class option or
% \CLASSINPUTbaselinestretch interface can be used to increase the line
% spacing as well. Be sure and use the nomarkers option of endfloat to
% prevent endfloat from "marking" where the figures would have been placed
% in the text. The two hack lines of code above are a slight modification of
% that suggested by in the endfloat docs (section 8.4.1) to ensure that
% the full captions always appear in the list of figures/tables - even if
% the user used the short optional argument of \caption[]{}.
% IEEE papers do not typically make use of \caption[]'s optional argument,
% so this should not be an issue. A similar trick can be used to disable
% captions of packages such as subfig.sty that lack options to turn off
% the subcaptions:
% For subfig.sty:
% \let\MYorigsubfloat\subfloat
% \renewcommand{\subfloat}[2][\relax]{\MYorigsubfloat[]{#2}}
% However, the above trick will not work if both optional arguments of
% the \subfloat command are used. Furthermore, there needs to be a
% description of each subfigure *somewhere* and endfloat does not add
% subfigure captions to its list of figures. Thus, the best approach is to
% avoid the use of subfigure captions (many IEEE journals avoid them anyway)
% and instead reference/explain all the subfigures within the main caption.
% The latest version of endfloat.sty and its documentation can obtained at:
% http://www.ctan.org/pkg/endfloat
%
% The IEEEtran \ifCLASSOPTIONcaptionsoff conditional can also be used
% later in the document, say, to conditionally put the References on a 
% page by themselves.




% *** PDF, URL AND HYPERLINK PACKAGES ***
%
%\usepackage{url}
% url.sty was written by Donald Arseneau. It provides better support for
% handling and breaking URLs. url.sty is already installed on most LaTeX
% systems. The latest version and documentation can be obtained at:
% http://www.ctan.org/pkg/url
% Basically, \url{my_url_here}.




% *** Do not adjust lengths that control margins, column widths, etc. ***
% *** Do not use packages that alter fonts (such as pslatex).         ***
% There should be no need to do such things with IEEEtran.cls V1.6 and later.
% (Unless specifically asked to do so by the journal or conference you plan
% to submit to, of course. )


% correct bad hyphenation here
\hyphenation{op-tical net-works semi-conduc-tor}


\begin{document}
%
% paper title
% Titles are generally capitalized except for words such as a, an, and, as,
% at, but, by, for, in, nor, of, on, or, the, to and up, which are usually
% not capitalized unless they are the first or last word of the title.
% Linebreaks \\ can be used within to get better formatting as desired.
% Do not put math or special symbols in the title.
\title{Intelligent Mobile AI-Generated Content Services via Interactive Prompt Engineering and Dynamic Service Provisioning}
%
%
% author names and IEEE memberships
% note positions of commas and nonbreaking spaces ( ~ ) LaTeX will not break
% a structure at a ~ so this keeps an author's name from being broken across
% two lines.
% use \thanks{} to gain access to the first footnote area
% a separate \thanks must be used for each paragraph as LaTeX2e's \thanks
% was not built to handle multiple paragraphs
%

\author{Yinqiu~Liu,
        Ruichen~Zhang,
        Jiacheng~Wang,
        Dusit~Niyato,~\IEEEmembership{Fellow,~IEEE},\\% <-this % stops a space
        Xianbin~Wang,~\IEEEmembership{Fellow,~IEEE},
        Dong~In~Kim,~\IEEEmembership{Life Fellow,~IEEE},
        and Hongyang~Du
%\thanks{}% <-this % stops a space
\thanks{Y.~Liu, R.~Zhang, J.~Wang, and D.~Niyato are with the College of Computing and Data Science, Nanyang Technological University, Singapore (e-mails: yinqiu001@e.ntu.edu.sg, ruichen.zhang@ntu.edu.sg, jiacheng.wang@ntu.edu.sg, and dniyato@ntu.edu.sg).}
\thanks{X.~Wang is with the Department of Electrical and Computer Engineering, Western University, Canada (e-mail: xianbin.wang@uwo.ca).}
\thanks{D.~Kim is with the College of Information and Communication Engineering, Sungkyunkwan University, South Korea (e-mail: dongin@skku.edu).}
\thanks{H.~Du is with the Department of Electrical and Electronic Engineering, University of Hong Kong, Hong Kong SAR, China (e-mail: duhy@eee.hku.hk).}
\vspace{-0.15cm}
%\thanks{Manuscript received April 19, 2005; revised August 26, 2015.}
}

% note the % following the last \IEEEmembership and also \thanks - 
% these prevent an unwanted space from occurring between the last author name
% and the end of the author line. i.e., if you had this:
% 
% \author{....lastname \thanks{...} \thanks{...} }
%                     ^------------^------------^----Do not want these spaces!
%
% a space would be appended to the last name and could cause every name on that
% line to be shifted left slightly. This is one of those "LaTeX things". For
% instance, "\textbf{A} \textbf{B}" will typeset as "A B" not "AB". To get
% "AB" then you have to do: "\textbf{A}\textbf{B}"
% \thanks is no different in this regard, so shield the last } of each \thanks
% that ends a line with a % and do not let a space in before the next \thanks.
% Spaces after \IEEEmembership other than the last one are OK (and needed) as
% you are supposed to have spaces between the names. For what it is worth,
% this is a minor point as most people would not even notice if the said evil
% space somehow managed to creep in.



% The paper headers
%\markboth{Journal of \LaTeX\ Class Files,~Vol.~14, No.~8, August~2015}%
%{Shell \MakeLowercase{\textit{et al.}}: Bare Demo of IEEEtran.cls for IEEE Journals}
% The only time the second header will appear is for the odd numbered pages
% after the title page when using the twoside option.
% 
% *** Note that you probably will NOT want to include the author's ***
% *** name in the headers of peer review papers.                   ***
% You can use \ifCLASSOPTIONpeerreview for conditional compilation here if
% you desire.




% If you want to put a publisher's ID mark on the page you can do it like
% this:
%\IEEEpubid{0000--0000/00\$00.00~\copyright~2015 IEEE}
% Remember, if you use this you must call \IEEEpubidadjcol in the second
% column for its text to clear the IEEEpubid mark.



% use for special paper notices
%\IEEEspecialpapernotice{(Invited Paper)}




% make the title area
\maketitle

% As a general rule, do not put math, special symbols or citations
% in the abstract or keywords.
\begin{abstract}
Due to massive computational demands of large generative models, AI-Generated Content (AIGC) can organize collaborative Mobile AIGC Service Providers (MASPs) at network edges to provide ubiquitous and customized content generation for resource-constrained users. 
However, such a paradigm faces two significant challenges: i) raw prompts (i.e., the task description from users) often lead to poor generation quality due to users' lack of experience with specific AIGC models, and ii) static service provisioning fails to efficiently utilize computational and communication resources given the heterogeneity of AIGC tasks. To address these challenges, we propose an intelligent mobile AIGC service scheme. Firstly, we develop an interactive prompt engineering mechanism that leverages a Large Language Model (LLM) to generate customized prompt corpora and employs Inverse Reinforcement Learning (IRL) for policy imitation through small-scale expert demonstrations. Secondly, we formulate a dynamic mobile AIGC service provisioning problem that jointly optimizes the number of inference trials and transmission power allocation. Then, we propose the Diffusion-Enhanced Deep Deterministic Policy Gradient (D$^3$PG) algorithm to solve the problem. By incorporating the diffusion process into Deep Reinforcement Learning (DRL) architecture, the environment exploration capability can be improved, thus adapting to varying mobile AIGC scenarios. Extensive experimental results demonstrate that our prompt engineering approach improves single-round generation success probability by 6.3$\times$, while D$^3$PG increases the user service experience by 67.8\% compared to baseline DRL approaches.
\end{abstract}

% Note that keywords are not normally used for peerreview papers.
\begin{IEEEkeywords}
Mobile AI-generated content, prompt engineering, large language model, inverse reinforcement learning
\end{IEEEkeywords}






% For peer review papers, you can put extra information on the cover
% page as needed:
% \ifCLASSOPTIONpeerreview
% \begin{center} \bfseries EDICS Category: 3-BBND \end{center}
% \fi
%
% For peerreview papers, this IEEEtran command inserts a page break and
% creates the second title. It will be ignored for other modes.
\IEEEpeerreviewmaketitle



\section{Introduction}
% The very first letter is a 2 line initial drop letter followed
% by the rest of the first word in caps.
% 
% form to use if the first word consists of a single letter:
% \IEEEPARstart{A}{demo} file is ....
% 
% form to use if you need the single drop letter followed by
% normal text (unknown if ever used by the IEEE):
% \IEEEPARstart{A}{}demo file is ....
% 
% Some journals put the first two words in caps:
% \IEEEPARstart{T}{his demo} file is ....
% 
% Here we have the typical use of a "T" for an initial drop letter
% and "HIS" in caps to complete the first word.
\IEEEPARstart{R}{ecently}, AI-Generated Content (AIGC) \cite{10398474, DuAIGC} has sparked significant interest across both academic and industrial sectors. 
Notable AIGC tools along this trend include DALL$\cdot$E 3, MusicLM, and ChatGPT for image generation, music composition, and multimodal conversation, respectively \cite{YQNetwork}. 
However, such achievements are built on large foundation models comprising massive parameters.
For example, GPT-3, released in 2020, already contains 175 billion parameters.
Accordingly, training such a model on a single GPU takes 355 years and consumes \$4.6 million \cite{GPT-3Cost}.
However, hardware scaling has not kept pace with the explosion in model parameter volume and resource requirements.
As the latest mobile AI chip, \textit{Qualcomm Snapdragon 8 Gen 3} can only afford lightweight AIGC models with roughly ten billion parameters \cite{Snapdragon}.
Constrained by Moore's law, it is foreseeable that such lightweight AIGC models will still be the mainstream for mobile deployment over a long period.
The conflict between model overhead and hardware capabilities prevents users from using ubiquitous high-quality AIGC services.

To address this challenge, the concept of \textit{Mobile AIGC} has been proposed, utilizing mobile-edge computing to democratize high-quality AIGC services \cite{10398474, 10628024}. 
Specifically, resource-constrained mobile users delegate their AIGC tasks to Mobile AIGC Service Providers (MASPs) served by edge servers, base stations, etc. \cite{10398474}. 
These MASPs, equipped with sufficient computational power, perform generative inferences, offering on-demand and paid AIGC services based on users' requirements (so-called prompts). 
This approach can not only alleviate the computational burden on individual users but also enhance privacy by reducing the need to send sensitive information to distant cloud servers \cite{10398474}. 
Great efforts in terms of model and networking have been made to promote the development of mobile AIGC.
For instance, Qualcomm \cite{qualcomm}, Salimans \textit{et al.} \cite{knowledge}, and Chen \textit{et al.} \cite{chen2023speed} adopted quantization, knowledge distillation, and GPU-aware optimization to compress AIGC models, respectively.
From the network perspective, Xu \textit{et al.} \cite{xu2023sparks} optimized the caching strategy in mobile AIGC, facilitating MASPs to manage their local AIGC models efficiently.
Additionally, Du \textit{et al.} \cite{10172151} presented a distributed manner of mobile AIGC inference, realizing the customized and collaborative AIGC generations.
Wen \textit{et al.} \cite{10233667} scheduled the task allocation among multiple MASPs and optimized the incentive mechanism to encourage them to invest computation resources.
\begin{figure}[tbp]
\centerline{\includegraphics[width=0.99\columnwidth]{Figure1.pdf}}
\caption{The heterogeneity of AIGC tasks. We can obverse that generating an image of a city is much more difficult than that of an apple since more complicated objects and compositions are required. Therefore, more inference trials should be allocated. Moreover, complex images accommodate more information (e.g., edges and visual signals) \cite{Complexity}. Hence, they are more sensitive to transmission loss and require more transmission power.}
\vspace{-0.12cm}
\label{example}
\end{figure}

Despite such progress, existing mobile AIGC schemes all follow a basic service paradigm, i.e., mobile users upload their prompts, and the MASPs perform AIGC inferences accordingly \cite{10398474, DuAIGC, 10628024, 10233667}.
We can observe that two challenges exist in this process.
\begin{itemize}
    \item \textbf{Low-Quality Raw Prompts:} As the description of user requirements and the instruction for AIGC inferences, prompts directly determine generation quality. Unfortunately, existing proposals \cite{DuAIGC, 10233667} simply feed raw prompts to the AIGC models. Due to the user's lack of experience/understanding of the specific AIGC model, outputs generated from raw prompts usually suffer from misinterpretation and limited precision \cite{YQNetwork}. Low generation quality may lead to continuous re-generation, which not only affects the Quality of Experience (QoE) but also increases the MASPs' resource consumption.
    \item \textbf{Heterogeneity of AIGC Tasks:} The current provisioning of AIGC services is static, i.e., the MASP allocates each user with equivalent computational resources for inferences and communication power to transmit outputs. However, AIGC tasks from different users exhibit significant heterogeneity. As shown in Fig. \ref{example}, even for the same task type (i.e., image generation), drawing a city with buildings is much more complex than drawing an apple on the table, since more complicated objects and compositions are involved. In this case, fixed service provisioning may lead to continuous failure of sophisticated cases, thus reducing resource efficiency.
\end{itemize}

In this paper, we present an intelligent mobile AIGC service scheme.
Specifically, to tackle the above challenges, our proposals contain interactive prompt engineering and dynamic service provisioning. 
For the first time, we integrate prompt engineering \cite{10.1145/3560815}, the cutting-edge concept to refine user prompts, into the mobile AIGC service process, with the goal of optimizing generation quality.
Additionally, we present dynamic mobile AIGC service provisioning, which trains a policy network that allows MASPs to adjust the number of inference trials and transmission power to handle each service request.
In this way, the QoE of mobile AIGC services can be significantly increased since users' requirements for high-quality AIGC outputs can be realized with lower latency and less resource consumption. 
Moreover, our scheme can be applied in any mobile AIGC application and accommodate other advanced proposals to further improve the incentive mechanism \cite{10233667} or task allocation strategy \cite{10172151}.
The contributions of this paper can be summarized as follows.
\begin{itemize}
    \item \textbf{Intelligent Mobile AIGC Services}: Different from the existing works, we reinvent the process of mobile AIGC services, evolving them for enhanced intelligence. Our goal is to maximize user QoE while reducing the resource consumption of MASPs, thus reaching the optimal system efficiency. To do so, the proposed scheme accommodates the following two mechanisms to optimize the generation quality and the service provisioning strategy.
    \item \textbf{Interactive Prompt Engineering}: To the best of our knowledge, we are the first to integrate prompt engineering into mobile AIGC services due to its well-proven ability to improve generation quality. Particularly, we address three challenges. First, the prompt should be refined based on the specific task. Hence, we leverage a Large Language Model (LLM) \cite{10.1145/3641289} to generate customized prompt corpora, with which the raw prompts can be refined precisely. Moreover, the efficacy of prompt engineering is posterior knowledge and requires substantial resources to evaluate \cite{Prompt-OIRL}. Inspired by Inverse Reinforcement Learning (IRL) \cite{zhang2023tempera}, we refine the prompt engineering policy through small-scale expert demonstrations and policy imitation. Finally, ground truth for assessing AIGC outputs might not be available due to intrinsic subjectivity. Hence, we train an LLM-based assessing agent with in-context memories to provide human-like scores for AIGC outputs and facilitate IRL training.
    \item \textbf{Dynamic Service Provisioning}: We present the problem of mobile AIGC QoE maximization, where the MASPs dynamically adjust the number of inference trials and the transmission power. Furthermore, to solve the problem, we adopt the Diffusion-Enhanced Deep Deterministic Policy Gradient (D$^3$PG) to optimize the MASP's service provisioning policy, realizing high exploration ability in varying mobile environments.
    \item \textbf{Experimental Results}: We perform extensive experiments. The numerical results demonstrate that the intelligent mobile AIGC service scheme greatly outperforms the current ones. First, prompt engineering reduces the re-generation probability by 6.3$\times$. Furthermore, dynamic service provision increases QoE by 67.8\%. The D$^3$PG also outperforms baseline algorithms in terms of reward and coverage rate.
\end{itemize}

The remainder of this paper is organized as follows.
Section II introduces the related work on mobile AIGC and discrete prompt engineering.
The system model, transmission model, and problem formulation are discussed in Section III.
Section IV demonstrates interactive prompt engineering.
Section V elaborates on the details of dynamic service provisioning via D$^3$PG.
The experiments and analysis are shown in Section VI.
Finally, Section VII concludes this paper.

\section{Related Work and Motivation}
\subsection{Mobile AIGC and Its Applications}
As a new concept, Du \textit{et al.} \cite{DuAIGC} first presented mobile AIGC and analyzed the MASP selection issues.
Then, Zhang \textit{et al.} \cite{10398474} comprehensively surveyed this topic, including its advantages, architecture, lifecycle, and some open challenges.
From 2023, mobile AIGC has entered a period of rapid development and received widespread attention from academia \cite{DuAIGC, 10233667, 10628024} and industry (e.g., Qualcomm and Meta \cite{qualcomm}).
%Next, we briefly review the related progress from different perspectives.
%\begin{itemize}
From the model perspective, researchers keep compressing large AIGC models, reducing their costs. For instance, Qualcomm published the world's first on-device Stable Diffusion by knowledge distillation \cite{qualcomm}. Likewise, Chen \textit{et al.} \cite{chen2023speed} performed a series of GPU-aware optimizations for diffusion-based AIGC models, reducing the inference latency to three seconds. Similar proposals include LightGrad \cite{10096710}, DiffNAS \cite{li2023diffnas}, and SnapFusion \cite{li2023snapfusion}. 
To improve the efficiency of mobile AIGC networks, Xu \textit{et al.} \cite{xu2023sparks} optimized the model caching strategy of MASPs. Du \textit{et al.} \cite{10172151} presented distributed mobile AIGC inference. 
By offloading certain inference steps to users, the computation overhead of MASPs can be effectively reduced. Huang \textit{et al.} \cite{10398264} leveraged federated learning to enable mobile AIGC to generate customized content. 
Wen \textit{et al.} \cite{10233667} designed an incentive mechanism based on content freshness, thereby encouraging MASPs to reduce latency. 
Cheng \textit{et al.} \cite{AIGCSemCom} applied semantic communications to reduce the bandwidth costs of MASPs to transmit AIGC outputs.
Finally, mobile AIGC facilitates various applications. 
For example, Zhang \textit{et al.} \cite{10628024} presented a terminal-edge-cloud collaborative AIGC architecture to facilitate autonomous driving. Likewise, Zhang \textit{et al.} \cite{MATTING} designed a diffusion-based matting engine for mobile AIGC users sharing and editing content.
%\end{itemize}

Different from existing works, this paper optimizes mobile AIGC from the service perspective.
By interactive prompt engineering and dynamic service provisioning, users' requests for high-quality AIGC outputs can be satisfied rapidly and consume less resources.
Hence, both the user QoE and system efficiency can be improved.

\subsection{Discrete Prompt Engineering}
Prompt engineering refers to the process of strategically refining prompts, thereby effectively guiding AIGC models to produce relevant and high-quality outputs.
According to the data structure, prompts can be split into two types, namely continuous and discrete prompts \cite{10.1145/3560815}.
The former, typically in the form of texts and images, is user-friendly and widely adopted in various AIGC applications, such as ChatGPT and Stable Diffusion.
Although the efficacy of prompt engineering in promoting generation quality has been well-proven, optimizing discrete prompts is challenging.
This is because most of the current continuous optimization approaches do not fit discrete prompt tokens.
%To this end, researchers traditionally craft prompts manually, such as prompt mining \cite{van-de-kar-etal-2022-dont} and paraphrasing \cite{10.1162/tacl_a_00324}.
%Such methods intrinsically suffer from huge time and labor consumption.
%To automate the prompt engineering process
To this end, an intuitive way is to transfer discrete prompts to continuous forms, e.g., parameterized embeddings.
Afterward, gradient-based optimization approaches can be applied \cite{wen2023hard, 10210127, ACL}.
Although improving efficiency, these methods sacrifice the interpretability of discrete prompts.
The optimized prompts cannot be explained and utilized to help users gain experience in prompting AIGC models.
Another series of proposals \cite{2309.08532, Evoprompting, deng-etal-2022-rlprompt} abstracted prompt optimization to an evaluation or Markov process.
For instance, Guo \textit{et al.} \cite{2309.08532} applied the generic algorithm, which iteratively refines each prompt by \textit{mutating} or \textit{crossing} its elements, with the goal of maximizing the fitness score.
Despite the interpretability, only limited action space and vocabulary are supported, preventing us from fully exploiting the potential of prompt engineering.

With the advancement of LLMs, refining raw prompts from infinite vocabulary becomes possible.
Hence, in this paper, we leverage an LLM to generate task-specific materials for refining raw prompts.
Moreover, to optimize the prompt engineering policy, we adopt IRL \cite{zhang2023tempera, Prompt-OIRL} to train a proxy reward.
In this way, the efficacy of selected prompt engineering strategies on any given task becomes predictable.
\renewcommand{\arraystretch}{1.2}
\begin{table}
\caption{The summary of main notations.}
\begin{tabular}{l|p{2.5cm}|l|p{2.5cm}}
\Xhline{2.2pt}
\rowcolor[rgb]{0.92,0.92,0.92}
\textbf{Notation}&\multicolumn{1}{c|}{\textbf{Description}}&\textbf{Notation}&\multicolumn{1}{c}{\textbf{Description}}\\
\hline
$Q$ & \# of users & $kc$ &Knowledge chunk\\
\hline
$M$& \# of MASPs & $\mathcal{D}$ & Demonstration dataset\\
\hline
$\pi^{(p)}_\omega$ & Prompt engineering policy& $\pi_E$ & Expert policy\\
\hline
$\pi^{(s)}_\theta$ & Service provisioning policy& $\Omega$ &AIGC model\\
\hline
$p$ & User prompt& $\tau(\cdot)$ & Embedding model\\
\hline
$\mathbf{c}_p$ & Prompt corpus& $\mathcal{D}_{\omega_1}$& Discriminator of IRL\\
\hline
$N_i$& \# of inference trials & $\mathcal{G}_\omega$& Generator of IRL\\
\hline
$P_i$& Transmission power & $\mathbf{s}^{(p)}$ & State of IRL\\
\hline
$\mathbf{p}^{*}$& Optimized prompt & $\mathbf{s}^{(s)}$ & State of D$^3$PG\\
\hline
$\otimes$& Combine operation & $T$ & \# of diffusion steps\\
\Xhline{2.2pt}
\end{tabular}
\vspace{-0.2cm}
\end{table}
\renewcommand{\arraystretch}{1}


\section{System Model}
In this section, we first introduce the intelligent mobile AIGC service scheme.
Then, the wireless transmission channel is modeled.

\subsection{General Mobile AIGC Services}
To illustrate the advantages of the proposed system, we first review a typical mobile AIGC service scheme.
Without loss of generality, this paper considers text-to-image generation, one of the most representative AIGC applications.\footnote{The proposed scheme can be extended to other AIGC applications, e.g., text-to-video, text-to-audio, and text-to-3D generation, by reformulating the prompts accordingly.}

As illustrated in Fig. \ref{structure} (Top part), the mobile AIGC system consists of $Q$ users and $K$ MASPs, denoted as $\{U_1, \dots, U_Q\}$ and $\{M_1, \dots, M_K\}$, respectively.
To acquire AIGC images, each user first describes the required topic and style using textual prompts, which are uploaded to an MASP.
The MASP, equipped with AIGC models, performs inferences to generate a batch of images (e.g., four for Stable Diffusion\footnote{The demo is on: https://huggingface.co/spaces/stabilityai/stable-diffusion}).
Note that the users check the generation quality.
If none of the generated images reaches the users' quality requirement threshold, the MASPs will be asked to re-generate and transmit the output images again to the users.
\begin{figure}[tbp]
\centerline{\includegraphics[width=0.95\columnwidth]{DST_R1.pdf}}
\caption{Top: A typical mobile AIGC service scheme (e.g., Stable Diffusion). Bottom: The proposed intelligent mobile AIGC scheme. Note that the orange and blue lines correspond to service configuration and operation stages, respectively.}
\label{structure}
\end{figure}


Although this scheme can realize basic functionalities, it suffers from several issues.
Nowadays, with ever-complicated AIGC applications, directly performing inferences using raw prompts can hardly meet users' demand for pursuing high-quality and customized outputs \cite{YQNetwork}.
Frequent re-generations and re-transmissions will increase service latency and MASP's resource consumption \cite{YQNetwork}.
Moreover, the MASP allocates equal computational and communication resources for each user without considering task heterogeneity.
Thus, if complex tasks are not dynamically allocated with sufficient resources, the system efficiency will be adversely affected.
To this end, we present an intelligent mobile AIGC service scheme to improve user QoE and resource efficiency simultaneously.
\begin{figure}[tbp]
\centerline{\includegraphics[width=0.95\columnwidth]{DST_R3.pdf}}
\caption{The illustration of prompt engineering strategy and policy. We can observe that for one raw prompt, different prompt engineering strategies lead to diverse optimized prompts and generated images. Therefore, the prompt engineering policy $\pi_\omega^{(p)}$ aims to select the optimal prompt engineering strategy dynamically.}
\label{policy}
\vspace{-0.3cm}
\end{figure}


\subsection{Intelligent Mobile AIGC Services}
As illustrated in Fig. \ref{structure} (Bottom part), our intelligent mobile AIGC consists of two stages, i.e., service configuration and service operation.

\subsubsection{Service Configuration Stage}
%This stage enables the MASP to optimize generation quality for serving users by refining their raw prompts.
This stage enables the MASP to establish service policies.
First, each MASP is trained to serve a specific type of service request (e.g., \textit{generating realistic landscape photos}) \cite{9186847}.
Afterward, a customized prompt engineering policy $\pi_{\omega}^{(p)}$ optimized for this MASP should be established.
As illustrated in Fig. \ref{policy}, different prompt engineering strategies can yield varying generation qualities for the same raw prompt. 
Therefore, policy $\pi_{\omega}^{(p)}$ is designed to select the optimal prompt engineering strategy based on specific user requests and conditions, maximizing the expected generation quality. 
To effectively train $\pi_{\omega}^{(p)}$, we need to collect strategy-quality pairs that demonstrate the relationship between different actions and their outcomes.
%Nonetheless, to train $\pi_{\omega}^{(p)}$ from scratch, the MASP requires certain samples about the efficacy of each candidate prompt engineering strategy.
Hence, the cluster first uploads a series of demonstration prompts to its respective MASP.
For instance, a two-item set of demonstration prompts can be [\{\texttt{A \!grassland,\! with\! trees}\}, \{\texttt{A\! lion\! sitting\! on\! a\! wooden\! bench}\}].
As shown in Fig. \ref{pipeline}, with demonstration prompts, the MASP then performs the following steps:
\begin{itemize}
    \item \textbf{Prompt Corpus Generation}: Leveraging an LLM, the MASP can generate a prompt corpus for each demonstration prompt. The corpus elements are textual segments. Then, different prompt engineering strategies can be applied, which strategically select prompt corpus elements to enrich the raw prompt.
    \item \textbf{Policy Imitation Learning}: All optimized prompts are adopted to generate images. The efficacy of all inference trials (i.e., the resulting image quality) is recorded to form a demonstrated dataset. An expert policy $\pi_{E}$ can then be acquired, which always selects the optimal strategy in the demonstration dataset (see Fig. \ref{pipeline}). Afterward, an IRL-based approach is adopted to facilitate $\pi_{\omega}^{(p)}$ imitating $\pi_{E}$, thus enabling efficient prompt engineering.
    %\item \textbf{Joint Prompt Engineering and Resource Allocation}: With proxy reward, the MASPs can optimize the policies for serving users, including the prompt engineering strategies and the transmission power to be allocated. Notably, the policy network adopts the diffusion principle to enhance the ability to explore complex mobile AIGC environments (step \texttt{c} in Fig. \ref{structure}).
\end{itemize}
%The proxy reward not only guides $\pi^{(p)}$ but also transfers the efficacy of prompt engineering to a priori knowledge.
After determining the prompt engineering policy, the MASP trains another policy $\pi_\theta^{(s)}$ through D$^3$PG to dynamically provision AIGC services, with the aim of maximizing QoE.
Specifically, for each service request, $\pi_\theta^{(s)}$ solves a joint optimization problem with two decision variables, namely \textit{the number of inference trials} and \textit{the transmission power to be allocated to serve each user}, denoted as $N_{i}$ and $P_i$ ($i \in \{1, 2, \dots, Q\}$), respectively.
Fig. \ref{Variable} shows how these two factors collaborate to determine image quality on the user side.
First, the larger the number of inference trials, the higher the probability that the user acquires satisfied AIGC outputs.
The reasons are two-fold.
First, generative inference contains uncertainty and randomness.
As shown in Fig. \ref{Variable}, even using the same prompt and AIGC model, adjusting the randomness setting leads to images with totally different compositions.
Additionally, $\pi_{\omega}^{(p)}$ is an approximation to real experts rather than the optimal policy.
Hence, increasing $N_i$ can improve users' expectations of acquiring satisfying images and mitigate the effects caused by prediction errors.
Meanwhile, $P_i$ determines the Bit Error Rate (BER) of the wireless channel, which affects the fidelity of the images received by users \cite{du2023usercentric}.
\begin{figure*}[tbp]
\centerline{\includegraphics[width=2\columnwidth]{DST_R4.pdf}}
\caption{The workflow for training prompt engineering policy $\pi_\omega^{(p)}$. First, the prompt corpus corresponding to each demonstration prompt is generated by an LLM. Then, different prompt engineering strategies are performed, and the demonstration dataset is constructed. From the demonstration dataset, the expert policy can be acquired (The expert policy is the one that always selects the strategy that leads to the optimal generation quality). Finally, an IRL framework is utilized for policy imitation.}
\label{pipeline}
\end{figure*}
\begin{figure}[tbp]
\centerline{\includegraphics[width=0.9\columnwidth]{DST_R2.pdf}}
\caption{The impact of two decision variables of dynamic mobile AIGC service provisioning on user received images.}
\label{Variable}
\end{figure}

\subsubsection{Service Operation Stage}
With policies $\pi_{\omega}^{(p)}$ and $\pi_\theta^{(s)}$ being trained, the MASP can provide intelligent AIGC services to mobile users.
As shown in Fig. \ref{structure} (Bottom part), for each request from $U_i$ ($i \in \{1, 2, \dots, Q\}$), the MASP first applies $\pi_{\omega}^{(p)}$ to optimize the raw prompt.
Then, dynamic service provisioning is conducted by $\pi_\theta^{(s)}$, acquiring the optimal $N_i$ and $P_i$.
$N_i$ times of generative inferences are performed, generating $N_i$ images. 
Finally, these generated images are sent to users via wireless channels using $P_i$ transmission power, accomplishing the intelligent AIGC services.




\subsection{Wireless Transmission Model}
We model the wireless transmission channel between mobile users and MASPs, considering both small-scale and large-scale fading effects \cite{9044870}. The received signal quality is influenced by fading, transmission power allocation, and channel conditions, which collectively determine the BER and image fidelity.

\subsubsection{Channel Modeling}
For small-scale fading, which results from multipath scattering, we model the channel gain using the \textit{Nakagami-m} distribution. The probability density function (PDF) of a Nakagami-\textit{m} distributed fading coefficient $X$ is given by \cite{5654629}
\begin{equation}
    f(x; \,m, \psi) = \frac{2m^m}{\Gamma(m)\,\psi^m} x^{2m-1} e^{-\frac{m}{\psi}x^2}, \quad x \geq 0,
\end{equation}
where $m$ is the fading severity parameter and $\psi = \mathbb{E}[X^2]$ is the scale parameter. The Gamma function $\Gamma(\cdot)$ is given by
\begin{equation}
    \Gamma(m) = \int_{0}^{\infty} t^{m-1} e^{-t} dt.
\end{equation}

Since the squared Nakagami-\textit{m} distributed variable $X^2$ follows a Gamma distribution, the instantaneous SNR at user $U_i$ is expressed as
\begin{equation}
    \textit{SNR}_i = \frac{P_i G_i}{N_0}.
\end{equation}
Here, $P_i$ is the allocated transmission power, $G_i = X_i^2$ represents the small-scale fading gain, and $N_0$ is the noise power. The expected SNR under Nakagami-\textit{m} fading is given by
\begin{equation}
    \mathbb{E}[\textit{SNR}_i] = \frac{P_i \psi}{N_0 }.
\end{equation}

For large-scale fading, which includes both path loss and shadowing, we model the channel gain using a log-normal distribution, i.e.,
\begin{equation}
    \boldsymbol{L}_i = d_i^{-\xi} e^{\sigma_s Z_i},
\end{equation}
where $d_i$ is the user-to-MASP distance, $\xi$ is the path-loss exponent, $\sigma_s$ is the standard deviation of the shadowing effect, and $Z_i \sim \mathcal{N}(0,1)$ is a standard normal variable representing log-normal shadowing.

Given the combined impact of small-scale and large-scale fading, the total received SNR at user $U_i$ is given by
\begin{equation}
    \textit{SNR}_i = \frac{P_i G_i}{N_0} d_i^{-\xi} e^{\sigma_s Z_i}.
\end{equation}

\subsubsection{Power Allocation and Bit Error Rate}
Given a total transmission power budget $P_{\text{total}}$ at the MASP, power is dynamically allocated among $Q$ users based on their channel conditions. The power allocated to user $U_i$ is determined as
\begin{equation}
    P_i = \frac{w_i P_{\text{total}}}{\sum_{j=1}^{Q} w_j}, \quad i \in \{1, 2, \dots, Q\},
\end{equation}
where $w_i$ is a weight factor determined by QoE requirements, channel quality, and task complexity.

The BER experienced by user $U_i$ is a function of the instantaneous SNR, which under Nakagami-\textit{m} fading \cite{5654629} is given by
\begin{equation}
    \textit{BER}_i = \int_0^\infty Q\left(\sqrt{2 \gamma}\right) f_{\textit{SNR}_i}(\gamma) d\gamma,
\end{equation}
where $Q(\cdot)$ is the standard Q-function \cite{655405}, and $f_{\textit{SNR}_i}(\gamma)$ is the probability density function of $\textit{SNR}_i$. Using the moment generating function (MGF) approach, the closed-form BER under Nakagami-\textit{m} fading can be expressed as
\begin{equation}
    \textit{BER}_i = \frac{\Gamma(m)}{2\Gamma(m+0.5)} \left( 1 - \sqrt{\frac{m}{m + \frac{\mathbb{E}[\textit{SNR}_i]}{2}}} \right)^m.
\end{equation}
Finally, the expected BER over both small-scale and large-scale fading is then computed as
\begin{equation}
    \mathbb{E}[\textit{BER}_i] = \int_{-\infty}^{\infty} \textit{BER}_i e^{-\frac{Z^2}{2}} dZ.
\end{equation}




\section{Interactive Prompt Engineering}
In this section, we detail interactive prompt engineering.
First, a prompt corpus is constructed corresponding to each demonstration prompt.
Then, we build a demonstration dataset and perform the policy imitation.
\begin{figure}[tbp]
\centerline{\includegraphics[width=0.95\columnwidth]{DST_R5.pdf}}
\caption{The prompt corpus for ``\texttt{A grassland, with trees}" considering two aspects named background and lighting. The left and right parts show the instructions and two demonstrations to $\ell_c$, respectively.}
\label{LLMprompt}
\end{figure}

\begin{figure*}[tbp]
\centerline{\includegraphics[width=1.85\columnwidth]{Figure_4_re5.pdf}}
\caption{The LLM-based image assessment and the structure of $\mathcal{D}$. \textbf{A}: The training of the assessing agent. The prompts highlighted in green and yellow correspond to role assignment and retrieval augment, respectively. \textbf{B}: The construction of external knowledge base. \textbf{C}: The quality assessment for an image. \textbf{D}: The in-context memory. \textbf{E}: The records in $\mathcal{D}$ correspond to one demonstration prompt.}
\label{agent}
\end{figure*}

\subsection{Prompt Corpus Generation}
%Recall that the demonstration prompts sent by users are denoted by $\boldsymbol{p}$.
To support prompt engineering, the MASP will generate a $L_c$-item prompt corpus specific to each demonstration prompt $p$, denoted as $\mathbf{c}_p := \{c^{(p)}_1, c^{(p)}_2, \ldots, c^{(p)}_{L_{c}}\}$.
By decorating $p$ with materials in $\mathbf{c}_p$, more information can be fed to the text-to-image AIGC model, enabling it to retrieve more pre-learned knowledge during inferences.
Without loss of generality, we suppose that the prompts for image generations take the general form of ``\texttt{A \![a], with \![b]}", in which \texttt{[a]} and \texttt{[b]} refer to the scene and representative objects in it, respectively, e.g., ``\texttt{A [grassland], with [trees]}.\footnote{The prompt format can be freely adjusted to support different scenarios.}"
Additionally, prompt engineering follows the suffix style, i.e., appending selected elements from $\mathbf{c}_p$ as the suffix of $p$.

As shown in Fig. \ref{LLMprompt}, we leverage an LLM-based prompt optimizer $\ell_c$, such as llama2-13b-chat, to generate the prompt corpus.
%First, it analyzes the complexity of drawing images according to the given prompt.
%Corresponding to the $1^{st}$-$4^{th}$ aspects mentioned above, the complexity is analyzed based on \textit{how difficult to craft the details}, \textit{how difficult to compose the image}, \textit{how difficult to render the mood}, and \textit{how difficult to schedule the lighting}.
%As shown in Fig. \ref{grass}, the complexity of ``\texttt{A grassland, with trees}" is 5/10 since the landscape is repetitive and the scene is forgiving in terms of accuracy.
%The complexity is utilized to guide the dynamic service provision, which is discussed in Section V.
Specifically, $\ell_c$ is instructed to enrich user prompts from certain aspects using infinite vocabulary, with each aspect being explained\footnote{The considered aspects are adaptable and can be customized according to the specific application and condition. The aspects considered in this paper are detailed in the Appendix. The instructions for training $\ell_c$ for enriching user prompts are published at: https://github.com/Lancelot1998/Prompt-Engineering}.
In addition, we apply two-shot prompting, i.e., feeding $\ell_c$ with two demonstrations, to regulate the required prompt corpus format.
As an example, Fig. \ref{LLMprompt} illustrates the corpus for the prompt ``\texttt{A grassland, with trees}", in which two aspects named \textit{background} and \textit{lighting} are considered.
Suppose that $k$ ($k \in \{1, 2, \dots, L_c\}$) elements are selected from $\mathbf{c}_p$ to enrich $p$.
We can derive that $\sum_{k=0}^{L_c}\left|\mathcal{P}(L_c, k)\right|$ optimized prompts can be composed by setting different arrangements of these $k$ selected elements as suffixes.
Note that $\mathcal{P}(L_c, k)$ lists the \textit{sets of permutations} on $k$ elements.
Consequently, the set of optimized prompts $\mathbf{p}^{*}$ for $L_p$ demonstration prompts $\mathbf{p} := \{p_1, p_2, \ldots, p_{L_p}\}$ can be expressed as
\begin{equation}
\mathbf{p}^{*} = \bigcup_{i=1}^{L_p} \left(\bigcup_{k=0}^{L_c} \left(\bigcup_{{\boldsymbol{\sigma}} = \mathcal{P}(L_c, k)} \left(p_i\otimes\prod_{j=1}^{k} c^{(p_i)}_{\sigma_j}\right)\right)\right),
\end{equation}
where $\sigma_j \in \boldsymbol{\sigma}$ ($j \in \{1, 2, \ldots, k\}$).
Finally, the notation $p_i\otimes\prod_{j=1}^{k}c^{(p_i)}_{\sigma_j}$ denotes the prompt engineering strategy, i.e., appending $\{c^{(p_i)}_{\sigma_1}, c^{(p_i)}_{\sigma_2}, \ldots, c^{(p_i)}_{\sigma_k}\}$ to $p_i$ as suffixes.


\subsection{Demonstration Dataset Construction}
With various candidate prompt engineering strategies, the problem becomes how to choose the best one for each request.
To optimize such a policy $\pi_{\omega}^{(p)}$, the MASP then constructs a demonstration dataset $\mathcal{D}$.
The motivation is that from the MASP's perspective, the efficacy of prompt engineering on the given prompt is a posteriori knowledge (i.e., the MASP cannot know such efficacy until it is fed back by the user) \cite{Prompt-OIRL}.
Collecting online experience during the service operation stage and polishing the prompt engineering policy from scratch is inefficient since users may suffer from low QoE during the initial time.
In contrast, constructing a demonstration dataset before formal services avoids damaging user experiences.
%Moreover, dynamic service provisioning based on D$^3$PG becomes available.


\subsubsection{AIGC Assessment}
Denote the AIGC model owned by MASP as $\Omega$. 
The quality assessment of the received images can be based on both quantitative metrics and user studies. 
Quantitative metrics like CLIP \cite{OpenCLIP} measure prompt-image consistency, while PicScore \cite{PicScore} evaluates aesthetic quality.
However, in real-world AIGC applications, users' quality assessments are inherently subjective, influenced by their individual perceptions, preferences, personalities, and specific requirements. 
Hence, there is no absolute ground truth for image quality assessment \cite{du2023usercentric}. 
Although user studies, e.g., questionnaires and surveys, provide subjective assessments, they present practical challenges: they are time-consuming, difficult to scale, and require repetition whenever application contexts or task patterns change.

Inspired by the recent success of LLM in agentic computing \cite{du2023usercentric}, we leverage an LLM $\ell_r$ to serve as an assessing agent, mimicking real AIGC users based on its enormous knowledge. 
Similarly to the prompt optimizer $\ell_c$, $\ell_r$ is also pluggable and can be implemented on any multimodal LLM.
As shown in Fig. \ref{agent}, we apply three techniques to train $\ell_r$, ensuring that it can give a comprehensive assessment.
\begin{itemize}
    \item \textbf{Role Prompting}: First, we train $\ell_r$ to behave like an AIGC user. Role prompting \cite{roleprompting} establishes the context and facilitates $\ell_r$ to invoke pretrained domain-specific knowledge. Hence, the generation can be aligned with the task's intent. Moreover, the specific task information is fed to $\ell_r$, including the score data structure (i.e., a floating number) and range (i.e., from 0 to 10).
    \item \textbf{Retrieval Augmentation}: In order to enrich $\ell_r$'s knowledge about image quality assessment, we build an external knowledge base with a set of documents. These include the objective factors affecting aesthetic quality, the basics of the human vision system, and the design of representative image quality assessment metrics \cite{Complexity, CLIP}. Using LangChain \cite{LangChain}, the knowledge is vectorized and divided into $W$ chunks. Hence, given the user prompt $p$, the most relevant knowledge can be fetched, i.e.,
    \begin{equation}
        p^{*} = p \otimes \!\!\!\!\!\!\underbrace{\operatorname{Top-k}}_{\text{cosine similarity}}\!\!\!\!\!\!\{kc_1, kc_2, \dots, kc_W\},
    \end{equation}
    where $kc_i$ ($i \in \{1, 2, \dots, W\}$) means a knowledge chunk. Combining pretrained and external knowledge, the assessment can be more professional.
    \item \textbf{In-Context Memory}: In real-world AIGC assessment, the user-perceivable image quality depends not only on objective and subjective factors but also on users' varying expectations according to their empirical experience. For instance, after a few service rounds, users tend to lower their expectations about difficult tasks, leading to varying levels of strictness. To reflect such a phenomenon, we equip $\ell_r$ with MemGPT \cite{MemGPT}, which saves the historical image-score pairs in the conversation memory. Then, $\ell_r$ is allowed to adjust the standard based on the context. 
\end{itemize}
With $\ell_c$ being trained, it can quantitatively assess the quality of the given image.
Furthermore, due to transmission error quantified by BER, the images received by users cannot hold 100\% fidelity.
Hence, we feed the user-received images to $\ell_r$, whose scores are called \textit{user-side score}.

\subsubsection{Data Structure}
The demonstration dataset $\mathcal{D}$ accommodates $L_p\! \cdot\! \sum_{k=0}^{L_c}\left|\mathcal{P}(L_c, k)\right|$ entries, in the form of
\begin{subequations}
\begin{flalign}
    &\mathcal{D}=\!\left\{\left[P, p_j, p^{*}_k, \mathbf{c}^{p_j}, \Upsilon\left(p^{*}_k, \Omega(p^{*}_k), P\right)\right]\right\},\\
    &j \in \{1, 2, \ldots, L_p\}, k \in \{1, 2, \ldots, |\mathbf{p}^{*}|\},
\end{flalign}
\end{subequations}
where $\Omega(p^{*}_k)$ denotes the image generated by $\Omega$ using prompt $p^{*}_k$. $\Upsilon(p^{*}_k, \Omega(p^{*}_k), P)$ represents the user-side score of $\Omega(p^{*}_k)$ transmitted using the wireless transmission power $P$, where $P \in (0, P_{\mathrm{total}}]$.
%Given the infinite possibilities of $P^{(i)}$ value, we uniformly divide the entire interval into $Z$ sub-intervals and consider that all values in each sub-interval correspond to the same SNR.
Finally, $\mathbf{p}^{*}$ has been defined in Eq. (11).

\subsubsection{Construction Process}
When constructing $\mathcal{D}$, the MASP traverses all the demonstration prompts in ${\mathbf{p}}$.
For each $p_i \in \mathbf{p}$ ($i \in \{1, 2, \ldots, L_p\}$), it applies $\ell_c$ to generate an $L_c$-element prompt corpus and perform $\sum_{k=0}^{L_c}\left|\mathcal{P}(L_c, k)\right|$ times of prompt engineering.
Afterward, the image corresponding to each optimized prompt can be generated by text-to-image model $\Omega$.
By Eq. (10), the distortion according to each possible transmission power is then applied to these images.
Finally, the user-side score for each image is assessed by $\ell_r$, and the corresponding entry is recorded in $\mathcal{D}$.

\subsection{Policy Imitation by Inverse Reinforcement Learning}
Traditionally, we can leverage $\mathcal{D}$ as an offline dataset and train $\pi^{(p)}_\omega$ using Deep Reinforcement Learning (DRL). 
Nonetheless, the actual scores are human-like subjective assessments rather than mathematically defined rewards. 
Hence, DRL can hardly effectively capture the nuanced relationships between prompt engineering strategies and generation quality from limited demonstrations. 
Instead, we leverage IRL \cite{GAIL}, which focuses on imitating expert policies by learning from expert behaviors, enabling us to better capture the subjective nature of AIGC quality assessment while maximizing sample usage efficiency \cite{Prompt-OIRL}.
Following the IRL principle, we first define the state and action spaces of our task.
\begin{itemize}
    \item \textbf{States}: The state describes the environment with which the prompt engineering policy interacts. Let $\mathbf{s}^{(p)}_{t}$ denote the IRL state at moment $t$, it can be expressed as 
        \begin{equation}
            \mathbf{s}_t^{(p)} = \left\{ \textbf{h}, \tau(p_i),  P_i \right\},
        \end{equation}
    where $\textbf{h} = \{a_1^{(p)}, a_2^{(p)}, \dots, a_{t-1}^{(p)}\}$ is the history of actions taken from genesis moment to moment $t\!-\!1$. $\tau(\cdot)$ refers to the embedding function \cite{du2023usercentric}, which converts a natural language prompt into machine-friendly vectors. $P_i$ is the allocated transmission power that affects BER.
    \item \textbf{Action}: The action space consists of all available prompt engineering strategies to refine the given raw prompt, which can be defined as
        \begin{equation}
            \mathbf{a}_t^{(p)}({p}_i) = \mathbb{S}\left(\bigcup_{k=0}^{L_c} \left(\bigcup_{{\boldsymbol{\sigma}} = \mathcal{P}(L_c, k)} \left(p_i\otimes\prod_{j=1}^{k} c^{(p_i)}_{\sigma_j}\right)\right)\right).
        \end{equation}
    Note that $\mathbb{S}(\cdot)$ represents an empirical filter. Note that we adopt $\mathbb{S}(\cdot)$ because given the large combinations of prompt corpus elements, in practice, we only consider the most representative prompt engineering strategies (the details are discussed in Section VI).
\end{itemize}
As aforementioned, the reward $\mathcal{R}(\mathbf{a}, \mathbf{s})$ in our problem is unknown.
%This is because $\mathcal{R}(\mathbf{a}, \mathbf{s})$ refers to the quality of the received AIGC images, which is assessed by human users and cannot be mathematically formulated.
%Traditionally, we can either service an approximated reward function or gather numerous state-action-reward pairs for training.
%Nonetheless, leveraging the LLM-empowered assessing agent to provide subjective scores, we have built a demonstration dataset $\mathcal{D}$ with certain space-action-score pairs.
%Hence, the next step is to optimize the prompt engineering policy $\pi^{(p)}_\omega$ by imitation from small-scale expert demonstrations.
Nonetheless, based on the demonstration dataset $\mathcal{D}$, an expert prompt engineering policy $\pi_E$ can be established, which maximizes the actual reward of all demonstration prompts by selecting the best strategy.
$\pi_E$ can be expressed as
\begin{equation}
    \max_{p^{*}_k} \Upsilon\left(p^{*}_k, \Omega(p^{*}_k\right), P^{(i)}), \;\;\forall k \in \{1, 2, \dots, |\mathbf{p^{*}}|\}.
\end{equation}
We optimize $\pi_\omega^{(p)}$ by letting it imitate $\pi_E$.
To do so, inspired by Generative Adversarial Imitation Learning (GAIL) \cite{GAIL}, we construct a generator-discriminator architecture to optimize $\pi_\omega^{(p)}$ adversarially.
Specifically, the discriminator $\mathcal{D}_{\omega_1}$ is a bi-classifier that distinguishes the actions sampled from policies $\pi_E$ and $\pi_\omega^{(p)}$.
Consequently, the objective function of $\mathcal{D}_{\omega_1}$ can be defined as
\begin{equation}
    \max_{\mathcal{D}_{\omega_1}} \mathbb{E}_{\pi_E}\!\!\left[\log \mathcal{D}_{\omega_1}(\mathbf{s}^{(p)}, \mathbf{a}^{(p)})] \!+\! \mathbb{E}_{\pi_{\omega}^{(p)}}[\log (1 \!-\! \mathcal{D}_{\omega_1}(\mathbf{s}^{(p)}, \mathbf{a}^{(p)}))\right],
\end{equation}
where $\mathcal{D}_{\omega_1}(\mathbf{s}^{(p)}, \mathbf{a}^{(p)}) \in \{0, 1\}$.

The generator $\mathcal{G}_\omega$ aims to refine $\pi_\omega^{(p)}$ towards imitating $\pi_E$.
Hence, the cost function can be defined as $\mathbb{E}_{\pi^{(p)}_\omega}[\log (1 - \mathcal{D}_{\omega_1}(\mathbf{s}^{(p)}, \mathbf{a}^{(p)}))]$ i.e., minimizing the success rate of $\mathcal{D}_{\omega_1}$.
Then, we leverage Proximal Policy Optimization (PPO) \cite{10032267} as the policy optimization framework due to its stability in learning. 
PPO evaluates the efficiency of the current policy via an advantage function, which is defined as
\begin{equation}
    \hat{\mathcal{A}}(\mathbf{s}^{(p)}, \mathbf{a}^{(p)}) = r_t + \gamma \mathcal{V}_{\phi}(\mathbf{s}^{(p)}_{t+1}) - \mathcal{V}_{\phi}(\mathbf{s}^{(p)}_t),
\end{equation}
where $r_t$ refers to the direct reward of the current policy, i.e., $\mathbb{E}_{\pi_{\omega}^{(p)}}[\log (1 - \mathcal{D}_{\omega_1}(\mathbf{s}^{(p)}, \mathbf{a}^{(p)}))]$.
$\gamma$ represents the discount factor for future rewards.
$\mathcal{V}_{\phi}(\cdot)$ means the state value function predicted by the PPO critic network.
Then, the objective function can be defined as
\begin{equation}
    \mathcal{L}^{CLIP}(\omega) = \mathbb{E}_t\left[\min(r_t(\omega)\hat{\mathcal{A}}_t, \text{clip}(r_t(\omega), 1-\epsilon, 1+\epsilon)\hat{\mathcal{A}}_t)\right],
\end{equation}
where $r_t(\omega) = \frac{\pi_\omega^{(p)}(\mathbf{a}^{(p)}_t|\mathbf{s}^{(p)}_t)}{\pi_{\omega_{old}}^{(p)}(\mathbf{a}^{(p)}_t|\mathbf{s}^{(p)}_t)}$, referring to the probability ratio between the current policy $\pi_\omega$ and the old policy $\pi_{\omega_{old}}$.
$\epsilon$ represents the clipping parameter that bounds policy updates to prevent excessive changes.
Note that the clip($\cdot,\cdot,\cdot$) function \cite{10032267} ensures that the objective function remains within a reasonable range by limiting the probability ratio between $[1-\epsilon, 1+\epsilon]$, which stabilizes training and prevents destructive policy changes that could deviate significantly from expert behaviors.
With $\mathcal{L}^{CLIP}$, the generator $\mathcal{G}_\omega$ can be updated by 
\begin{equation}
    \omega' = \omega + \alpha \nabla_\omega\mathcal{L}^{CLIP}(\omega),
\end{equation}
where $\alpha$ means the learning rate for updating the generator network.
Finally, the critic network is updated by
\begin{equation}
\phi' = \phi - \beta\nabla_\phi\mathbb{E}_t\left[(\mathcal{V}_\phi(\mathbf{s}^{(p)}_t) - (r_t + \gamma \mathcal{V}_\phi(\mathbf{s}^{(p)}_{t+1})))^2\right],
\end{equation}
where $\beta$ is the learning rate for the critic network, and $\gamma$ is the discount factor for future rewards.

%This is because PPO's trust region constraint ensures conservative policy updates, which is crucial when learning from demonstrations to prevent destructive policy changes that could deviate significantly from the expert behavior.

%In this part, we leverage $\mathcal{D}$ as the offline expert dataset to train a proxy reward using IRL. 
%Specifically, we adopt Generative Adversarial Imitation Learning (GAIL) \cite{GAIL}, a popular IRL framework using a Generative Adversarial Network (GAN) to derive the to-be-processed policy directly from expert policies. 
%GAIL reframes the IRL problem as an adversarial game involving a discriminator and a generator. 
%The discriminator \(D\) aims to differentiate between the expert policy \(\pi_E\) and the learned policy $\pi^{(p)}$. 
%Hence, its learning objective can be defined as follows
%\begin{equation}
%    \max_{D} \mathbb{E}_{\pi_E}[\log D(\mathbf{s}^I, \mathbf{a}^I)] + \mathbb{E}_{\pi^{(p)}}[\log (1 - D(\mathbf{s}^I, \mathbf{a}^I))],
%\end{equation}
%where \(D(\mathbf{s}^I, \mathbf{a}^I) \in [0,1]\) indicates the probability assigned by the discriminator that the action $\mathbf{a}^I$ in state $\mathbf{s}^I$ originates from the expert policy.

%The generator $G$, which aims to imitate the expert policy, is optimized by maximizing the following loss function, i.e.,
%\begin{equation}
%    \max_{\pi^{(p)}} \mathbb{E}_{\pi}[\log D(\mathbf{s}^I,\mathbf{a}^I)],
%\end{equation}
%which means the learned policy $\pi^{(p)}$ produces behavior indistinguishable from that of the expert to $D$.


%In such a way, the discriminator network is to differentiate between the expert and pending policies, thereby generating a reward structure. Concurrently, the generator network is focused on refining the pending policy to more closely align with the expert policy. This process culminates in the generation of the pending policy, steering the training towards convergence as the pending policy incrementally assimilates the characteristics of the expert policy. 



%\subsubsection{Proxy Reward}
%In IRL, the reward is not predefined by a formula like in DRL. Instead, it is learned directly from the policy $\pi^{(p)}$ through the adversarial training process. In particular, rather than manually designing a specific reward function, the proxy reward is generated by evaluating how closely the learned policy $\pi^{(p)}$ matches the expert policy $\pi_E$. This learned reward is referred to as the proxy reward. Specifically, as the generator follows $\pi^{(p)}$ to produce actions, the discriminator compares these actions with those from the expert policy $\pi_E$. This evaluation produces the proxy reward, which is used to reflect how well the learned policy imitates expert behavior.

%In our work, at each step, the quality of the generated images is further assessed by an LLM based on factors such as fidelity, alignment with the prompt, and overall aesthetic quality. These LLM scores contribute to the reward signal and help guide the refinement of $\pi^{(p)}$. Thus, the proxy reward is dynamically generated through the interaction between $\pi^{(p)}$, $\pi_E$, and the discriminator $D$. This learned proxy reward serves as the key feedback signal, driving the continuous improvement of the generator’s actions. The generated proxy reward can then be used as the value for subsequent training and decision-making phases.


\section{Diffusion-Empowered Dynamic Service Provisioning}
In this section, we detail the proposed dynamic service provisioning.
First, we formulate the problem and model the QoE of mobile AIGC users.
Then, we proposed the D$^3$PG to generate the optimal service provisioning policy.

\subsection{Problem Formulation}
The MASP aims to achieve an optimal balance between user QoE and resource efficiency, including computing resource allocation to perform prompt engineering and transmission power to transmit AIGC outputs. 
This problem can be formulated as follows:
\begin{subequations}
\begin{flalign}
\max _{\{N_{i}, P_{i}\}} & \sum_{i=1}^{Q} \left(\eta_q\cdot\mathcal{Q}\left(N_i, P_i\right) - \eta_c\cdot\mathcal{C}\left(N_i, P_i\right)\right), \\
\text { s.t., } & \mathcal{Q}\left(N_i, P_i\right) \geq \mathcal{Q}^{\mathrm{th}}_i, \quad \forall i \in\{1, 2, \ldots, Q\}, \\
& N_{i}\, \geq 1, \quad \forall i \in\{1, 2, \ldots, Q\},\\
& \sum_{i=1}^{Q} P_i \leq P_{\mathrm{total}},
\end{flalign}
\end{subequations}
where $\mathcal{Q}(\cdot)$ and $\mathcal{C}(\cdot)$ denote the functions for QoE and cost calculation, respectively.
$\eta_q$ and $\eta_c$ are two weighting factors.
The constraint in Eq. (22b) indicates that the QoE of each user should meet its requirement threshold.
The constraint in Eq. (22c) defines the range of $N_i$, i.e., the MASP should generate at least one image each time.
Finally, Eq. (22d) requires that the total transmission power allocated by the MASP cannot exceed its budget.
In the following parts, we elaborate on the modeling of $\mathcal{Q}(\cdot)$ and $\mathcal{C}(\cdot)$, respectively.
%Next, we train a proxy reward, which is the prerequisite for definite $\mathcal{Q}(\cdot)$.
%Then, a diffusion-based approach is presented to solve the above optimization problem.

\subsection{QoE Modeling}
In mobile AIGC, the user QoE mainly depends on two key performance indicators, namely service latency and generation quality.
The former is related to the number of inference trials and the time required for each round of inference.
The latter, as mentioned in Section III, is determined by the efficacy of prompt engineering and the transmission power that affects the fidelity of the user's received images.
Jointly considering the above factors, the QoE for user $U_i$ is defined as
\begin{equation} 
    \mathcal{Q}(N_{i}, P_i) \!=\! \overbrace{\operatorname{log}_{N_i}\!\!\underbrace{\left(\frac{L_{max}}{N_i\cdot T_\zeta}\right)}_{\textnormal{service latency}}}^{\textnormal{impact of latency on QoE}}\operatorname{ln}\left(\frac{\max \left\{Q^{(i)}_1, \dots, Q^{(i)}_{N_i}\right\}}{Q_\mathrm{th}^{(i)}}\right), 
\end{equation}
where $T_\zeta$ represents the inference time with $\zeta$ denoting the number of diffusion steps.
$L_\mathrm{max}$ and $Q^{(i)}_\mathrm{th}$ denote the upper bound of service latency and $U_i$'s personal threshold for generation quality, respectively.
Suppose that the user only adopts the most satisfied AIGC output.
Hence, we apply a filter and fetch the maximum generation quality from $\{Q^{(i)}_1, \dots, Q^{(i)}_{N_i}\}$.
Note that we leverage the method in \cite{6263849} to model users' tolerance towards service latency.
Specifically, $N_i\cdot T_\zeta$ means the total inference time for $N_i$ trials\footnote{For simplicity, we ignore the transmission latency and suppose service latency equals inference time since it is the major latency cause.}.
In \cite{6263849}, Hossfeld \textit{et al.} proved that the subjective impact of service latency on user experience follows a $\operatorname{log}$ relationship, i.e., as the waiting time increases, users will become less sensitive towards latency increment.
Moreover, the larger $N_i$ is, the higher the user's tolerance for service latency since more images can be received in this round.
Therefore, we apply $N_i$ as the base of the logarithmic function.

%We use expectation since the generality quality is a prediction made by the proxy reward rather than a ground truth, which is available only after $U_i$ receives the images.
%Hence, to define $\mathbb{E}_Q^{(i)}$, we first need to measure the precision of proxy reward.
%\begin{equation}
%    \boldsymbol{\Delta r} = \left[\Upsilon\left(\hat{p}^{*}_k, \Omega(\hat{p}^{*}_k), P^{(i)}\right) - r\left(\hat{p}^{*}_k, \Omega(\hat{p}^{*}_k), P^{(i)}\right)\right].
%\end{equation}
%Afterward, we fit the distribution of $\boldsymbol{\Delta r}$.
%Supposing that $\boldsymbol{\Delta r}$ follows a normal distribution $\mathcal{N}(\mu, \sigma)$, the mean and variance can be easily calculated.
%Hence, a more precise prediction of the \textit{actual reward} should be
%\begin{equation}
%    \Upsilon^{*}\left(p, \Omega\left(\pi^{(p)}(p)\right)\!, P_i\right) = \Upsilon\left(p, \Omega\left(\pi^{(p)}(p)\right)\!, P_i\right) + \varepsilon,
%\end{equation}
%where $p$ is the user prompt.
%$\pi^{(p)}$ is the prompt engineering policy acquired in Section IV.
%Note that $\varepsilon$ follows the same distribution as $\boldsymbol{\Delta r}$, i.e., $\mathcal{N}(\mu, \sigma)$ since demonstration prompts are sampled from user prompts.
%Hence, the generation quality $Q^{(i)}$ follows $\mathcal{N}\left(\Upsilon\left(p, \Omega\left(\pi^{(p)}(p)\right)\!, P_i\right) + \mu, \sigma\right)$, i.e.,
%\begin{equation}
%    \mathbb{E}_Q^{(i)} = \Upsilon\left(p, \Omega\left(\pi^{(p)}(p)\right)\!, P_i\right) + \mu.
%\end{equation}
%Based on Eq. (13), we can derive the following theorem.
%\\
%\\
%\textbf{Theorem 1}: \textit{The larger the $N^p_i$ is, the higher the probability that user $U_i$ can acquire at least one satisfied image.}

%\begin{proof}
%\textit{The $N^p_i$ images are generated based on the same prompt p and model} $\Omega$. \textit{From Eqs. (10) and (11), we know that for each image, the probability that its quality fails to meet the user threshold is.}
%\begin{equation}
%\begin{split}
%    &\operatorname{Prob}\left(r\left(p, \Omega\left(\pi^{(p)}(p)\right)\!, P_i\right) \leq Q^{(i)}_\mathrm{th}\right) \\ \approx &\operatorname{Prob}\left(\Upsilon^{*}\left(p, \Omega\left(\pi^{(p)}(p)\right)\!, P_i\right) \leq Q^{(i)}_\mathrm{th}\right) \\=&\operatorname{Prob}\left(\varepsilon \leq \underbrace{Q^{(i)}_\mathrm{th} - \Upsilon\left(p, \Omega\left(\pi^{(p)}(p)\right)\!, P_i\right)}_{\kappa}\right) \\ = & \frac{1}{\sigma \sqrt{2\pi}} \int_{-\infty}^{\kappa}\operatorname{exp}\left(-\frac{(\kappa - \mu)^2}{2\sigma ^2}\right)dx.
%\end{split}
%\end{equation}
%\textit{The quality of each image is independent and identically distributed. Therefore, the probability that all $N^p_i$ images are unqualified is derived as $\left[\frac{1}{\sigma \sqrt{2\pi}} \int_{-\infty}^{\kappa}\operatorname{exp}\left(-\frac{(\kappa - \mu)^2}{2\sigma ^2}\right)dx\right]^{N^p_i}$, which is negatively correlated to $N^p_i$.}
%\end{proof}

Moreover, to effectively model the user's subjective experience toward generation quality, we apply Weber-Fechner law \cite{WF_Law}.
This law states that as the stimulus (e.g., the vision, hearing, taste, and touch) increases, the perceived sensation grows but at a diminishing rate. 
Similar to \cite{6263849}, such a phenomenon is described as a logarithmic relationship.
In addition, the noticeable difference between two different levels of stimuli is a constant ratio of the initial stimulus.
To this end, we define the impact of generation quality on overall QoE as $\operatorname{ln}\left(\frac{\max \left\{Q^{(i)}_1, \dots, Q^{(i)}_{N_i}\right\}}{Q_\mathrm{th}^{(i)}}\right)$, as illustrated in Eq. (23).

Up till now, we have defined the QoE function $\mathcal{Q}(N_i, P_i)$.
Another consideration of system efficiency is the resource consumption of the MASPs, containing the computation resources to perform generative inference and the transmission power to transmit generated images to users.
Hence, $\mathcal{C}(N_i, P_i)$ can be defined as
\begin{equation}
    \mathcal{C}(N_i, P_i) = N_i \cdot \left(c_\zeta + P_i\right),
\end{equation}
where $c_\zeta$ represents the computation resource consumption for each generative inference trail, with $\zeta$ meaning the diffusion step number.
Substituting Eqs. (23) and (24) into Eq. (22a), we can obtain the complete objective about joint QoE and resource optimization.
Next, we design a diffusion-based approach to generate the optimal solution to this problem.

\subsection{Algorithm Overview}
The proposed D$^3$PG follows a DRL architecture with five basic components, namely agent, state, action, policy, and reward.
Their introductions are shown below.
\begin{itemize}
    \item \textbf{Agent}: Our agent is the MASP, which performs the service provisioning to allocate the physical resources to serve $Q$ mobile users simultaneously.
    \item \textbf{State}: The state of the mobile AIGC environment takes the form of $\mathbf{s}^{(s)}$ := [\{$\tau(p_1)$, $\tau(p_2)$, $\dots$, $\tau(p_Q)$\}, \{$d_1$, $d_2$, $\dots$, $d_Q$\}, \{$\mathcal{Q}^{\mathrm{th}}_1$, $\mathcal{Q}^{\mathrm{th}}_2$, $\dots$, $\mathcal{Q}^{\mathrm{th}}_Q$\}, $P_{\mathrm{total}}, \;\widetilde{\textit{SNR}}$]. The first two sets accommodate the prompts and distances from the MASP to users $\{U_1, U_2, \dots, U_Q\}$, respectively. $P_{\mathrm{total}}$ represents the MASP's total transmission power, and $\widetilde{\textit{SNR}}$ is the wireless channel state, as explained in Section III.
    \item \textbf{Action}: We define the action space as a vector $\mathbf{a}^{(s)} := \{\boldsymbol{a}^{(s)}_1, \boldsymbol{a}^{(s)}_2, \dots, \boldsymbol{a}^{(s)}_Q\}$, denoting the resources allocated to each user. Specifically, each $\boldsymbol{a}^{(s)}_i := \{N_i, P_i\}$ ($\forall i \in \{1, 2, \dots, Q\}$), including the number of inference trials and the allocation transmission power.
    \item \textbf{Policy}: The policy refers to the probability that the agent takes action $\mathbf{a}^{(s)}$ in the state $\mathbf{s}^{(s)}$. Particularly, our algorithm adopts a diffusion network parameterized by $\theta$ to learn the relationship between the input state $\mathbf{s}^{(s)}$ and the output action $\mathbf{a}^{(s)}$ that can optimize the reward. Therefore, this policy network can be expressed as $\pi^{(s)}_\theta(\mathbf{s}^{(s)}, \mathbf{a}^{(s)}) = \operatorname{Prob}(\mathbf{a}^{(s)}|\mathbf{s}^{(s)})$.
    \item \textbf{Reward}: Finally, given the state space $\mathbf{s}^{(s)}$, the reward of taking action $\mathbf{a}^{(s)}$ can be defined as $R(\mathbf{a}^{(s)}|\mathbf{s}^{(s)}) = \sum_{i=1}^{Q} \left(\eta_q\cdot\mathcal{Q}\left(N_i, P_i\right) - \eta_c\cdot\mathcal{C}\left(N_i, P_i\right)\right)$, i.e., Eq. (22a). Note that if any of the constraints shown in Eqs. (22b)-(22d) is not satisfied, we apply a negative penalty. Specifically, if the actions for $J$ users fail to meet Eqs. (22b) or (22c), the penalty is $J \cdot \varrho$, where $\varrho$ is a hyperparatermer. If Eq. (20d) is not satisfied, the penalty becomes $Q \cdot \varrho$ because the generated service provisioning solution makes the problem infeasible. 
\end{itemize}

\subsection{Diffusion-Enhanced DDPG (D$^3$PG) Design}
\subsubsection{Diffusion-Empowered Policy Generation}
Inspired by non-equilibrium thermodynamics, diffusion models characterize the generation tasks as a step-by-step process of denoising from pure Gaussian noise \cite{DiffusionDRL1}.
Nowadays, diffusion has supported numerous AIGC models in various modalities, such as the Stable Diffusion we used in Fig. \ref{example}.
Additionally, it brings traditional DRL algorithms with greater exploration ability \cite{DiffusionDRL1, zhu2023diffusion}.
Therefore, our D$^3$PG employs a deep diffusion network to generate policy $\pi^{(s)}_\theta(\mathbf{s}^{(s)}, \mathbf{a}^{(s)})$.
Specifically, the network contains two Markov processes, namely forward diffusion and denoising.
The former perturbs the optimal action $\mathbf{a}^{(s)}_0$ to random action $\mathbf{a}_T$ by $T$ diffusion steps, satisfying
\begin{equation}
    \mathbf{a}^{(s)}_t = \sqrt{\alpha_t}\mathbf{a}^{(s)}_{t-1} + \sqrt{1-\alpha_t}\epsilon_t, \quad \epsilon \sim \mathcal{N}(0, \mathbf{I}),
\end{equation}
where $\mathbf{I}$ denotes the identity matrix. $\alpha_t$ ($t \in \{1, 2, \dots, T\}$) follows a pre-defined schedule and is decreasing over $t$ \cite{DDPM}.
Hence, the entire forward diffusion can be expressed as
\begin{subequations}
\begin{flalign}
    q(\mathbf{a}^{(s)}_{1:T}|&\mathbf{a}^{(s)}_0) = \prod_{t=1}^{T} q(\mathbf{a}^{(s)}_t|\mathbf{a}^{(s)}_{t-1}), \\
    q(\mathbf{a}^{(s)}_t|\mathbf{a}^{(s)}_{t-1}) =& \mathcal{N}(\mathbf{a}^{(s)}_t; \sqrt{\alpha_t}\mathbf{a}^{(s)}_{t-1}, (1-\alpha_t)\,\mathbf{I}).
\end{flalign}    
\end{subequations}
Accordingly, the denoising process, i.e., generating the optimal policy from noise, can be expressed as \cite{DDPM}
\begin{equation}
    \begin{split}
        p_\theta(\mathbf{a}^{(s)}_{0:T}) &= p(\mathbf{a}^{(s)}_T)\prod^{T}_{t=1}p_\theta(\mathbf{a}^{(s)}_{t-1}|\mathbf{a}^{(s)}_t). \\
        %p_\theta(\mathbf{a}_{t-1}|\mathbf{a}_t) =&\,\, \mathcal{N}\left(\mathbf{a}_{t-1}; \mu_\theta(\mathbf{a}_t, t), \Sigma_\theta(\mathbf{a}_t, t)\right).
    \end{split}
\end{equation}
Such a process can be trained by maximizing the likelihood of $p_\theta(\mathbf{a}_{0})$.
However, $p_\theta(\mathbf{a}^{(s)}_{t-1}|\mathbf{a}^{(s)}_t)$ cannot be directly calculated.
To this end, $q(\mathbf{a}^{(s)}_{t-1}|\mathbf{a}^{(s)}_t, \mathbf{a}^{(s)}_0)$ is employed.
Suppose that $q(\mathbf{a}^{(s)}_{t-1}|\mathbf{a}^{(s)}_t, \mathbf{a}^{(s)}_0)$ follows the normal distribution.
Applying the Bayesian formula and Eq. (26), the mean and variance can be calculated as \cite{DDPM}
\begin{subequations}
\begin{flalign}
    q(\mathbf{a}^{(s)}_{t-1}|\mathbf{a}^{(s)}_t, \mathbf{a}^{(s)}_0)  &\!=\! \mathcal{N}\!\left(\mathbf{a}^{(s)}_{t-1}; \mu_t(\mathbf{a}^{(s)}_t, t), \Sigma_t(\mathbf{a}^{(s)}_t, t)\right),\\
    \mu_t(\mathbf{a}^{(s)}_t, t) &= \frac{1}{\sqrt{\alpha_t}}\left(\mathbf{a}^{(s)}_t - \frac{1-\alpha_t}{\sqrt{1-\bar{\alpha}_t}}\epsilon\right),\\
    \Sigma_t(\mathbf{a}^{(s)}_t, t) &= \frac{(1-\alpha_t)(1-\bar{\alpha}_{t-1})}{1-\bar{\alpha}_t} \,\mathbf{I},
\end{flalign}
\end{subequations}
where $\bar{\alpha}$ = $\prod_{s=1}^{t}\alpha_s$
With $q(\mathbf{a}^{(s)}_{t-1}|\mathbf{a}^{(s)}_t, \mathbf{a}^{(s)}_0)$, the variational lower bound of $\operatorname{log}p_\theta(\mathbf{a}^{(s)}_0)$ can be calculated.
The final training objective can be derived as 
\begin{equation}
    \min ||\epsilon - \epsilon_\theta(\sqrt{\bar{\alpha}_t}\mathbf{a}^{(s)}_0 + \sqrt{1-\bar{\alpha}_t}\epsilon, t)||^2,
\end{equation}
where $\epsilon_\theta$ contains the parameters (implemented by a UNet) to be trained \cite{DDPM}.
After training, the optimal action $\mathbf{a}^{(s)}_0$ can be generated step-by-step from a random one $\mathbf{a}^{(s)}_T$, i.e.,
\begin{equation}
    \mathbf{a}^{(s)}_{t-1} = \frac{1}{\sqrt{\alpha_t}}\left(\mathbf{a}^{(s)}_t - \frac{1-\alpha_t}{\sqrt{1-\bar{\alpha}}_t}\epsilon_\theta(\mathbf{a}^{(s)}_t, t)\right),
\end{equation}
where $t \in \{1, 2, \dots, T\}$.
\begin{figure}[tbp]
\centerline{\includegraphics[width=0.9\columnwidth]{Figure_add_f.pdf}}
\caption{The D$^3$PG architecture. We apply a diffusion-based actor-network to enhance the DDPG.}
\label{ddpm}
\end{figure}

\subsubsection{Model Architecture}
We utilize the DDPG \cite{DiffusionDRL1} architecture to accommodate the diffusion-based policy network, forming D$^3$PG.
As shown in Fig. \ref{ddpm}, diffusion acts as the actor networks, which generate service provisioning strategies and interact with the mobile AIGC environments.
In addition, two critic networks are employed, using the Bellman equation to estimate the expected reward, i.e.,
\begin{equation}
    Q_\phi(\mathbf{s}^{(s)}\!, \mathbf{a}^{(s)}) =\! R(\mathbf{a}^{(s)}|\mathbf{s}^{(s)}) + \gamma Q_\phi'\left(\mathbf{s}^{(s)'}, \pi^{(s)}_{\theta'}(\mathbf{s}^{(s)'})\right),
\end{equation}
where $\mathbf{s}^{(s)'}$ denotes the next state, $\phi'$ and $\theta'$ represent the parameters of the target networks for actor and critic, respectively, and $\gamma$ is the discount factor.
Note that in DDPG, target networks for both the actor and the critic are applied to stabilize the training process. 
These target networks have the same architecture as the original networks, but their weights are updated slowly, usually by soft updates.
The policy update aims to maximize the Q-value, which can be expressed by
\begin{equation}
    \max_{\pi_\theta} \mathbb{E}_{\mathbf{a}^{(s)}\sim\pi_{\theta}}\left[Q_{\pi}(\mathbf{s}^{(s)}, \mathbf{a}^{(s)})\right].
\end{equation}
The detailed training process is shown in \textbf{Algorithm 1}.
\begin{algorithm}[tpb]
\footnotesize \caption{The Procedure of D$^3$PG Algorithm}
\begin{algorithmic}[1]
\Require  
$\mathbf{s}^{(s)}$, $N_b$, $T$, $\eta$, $\gamma$ \textit{\#\#\, The mobile AIGC environment, batch size, diffusion step number, discount factor, and learning rate}
\Ensure 
$\mathbf{a}_0$ \textit{\#\#\, service provisioning strategy}
\Procedure{Algorithm Training}{$\mathbf{s}^{(s)}$, $N_b$, $T$, $\eta$, $\gamma$} 
\State Initialize networks: actor network $\pi^{(s)}_{\theta}$ and critic networks $\phi$ and $\phi$'.
\While{not converged}
\State Initialize random noise $\mathbf{a}^{(s)}_T$; generate bandwidth allocation scheme $\mathbf{a}^{(s)}_0$ by denoising process shown in Eq. (30).
\State Add exploration noise to $\mathbf{a}^{(s)}_0$.
\State Execute service provisioning and calculate reward $R(\mathbf{a}^{(s)}|\mathbf{s}^{(s)})$ by Eq. (22a).
\State Store the record ($\mathbf{s}^{(s)}, \mathbf{a}_0^{(s)}, R(\mathbf{a}^{(s)}|\mathbf{s}^{(s)})$) in the replay buffer
\State Randomly select $N_b$ records
\State Update the policy generation network
\State Update the Q-networks
\EndWhile
\EndProcedure
\Statex
\Procedure{Algorithm Inference}{$\mathbf{s}^{(s)}$, $N_b$, $T$, $\eta$, $\gamma$}
\State Observe the environment $\mathbf{s}^{(s)}$
\State Generate bandwidth allocation scheme $\mathbf{a}^{(s)}_0$
\State \textbf{Return} $\mathbf{a}^{(s)}_0$
\EndProcedure
\end{algorithmic}
\end{algorithm}

\subsubsection{Complexity Analysis}
We then examine the computational complexity of D$^3$PG in detail. 
Frist, suppose that $S_p$ and $S_q$ respectively represent the sizes of the diffusion-based actor-network and the Q-network. 
The architectural complexity is $\mathcal{O}(S_p + 2S_q)$. 
Because each service provisioning solution should be generated through $T$ rounds of diffusion denoising, the complexity of generating each action is $\mathcal{O}(T S_p)$. 
Consequently, the overall complexity is $\mathcal{O}((T + 1) S_p + 2S_q)$. Furthermore, if $\delta$ training epochs are performed with a batch size of $S_b$, the resulting computational cost is $\mathcal{O}(\delta S_b \bigl((T + 1) S_p + 2S_q\bigr))$. Finally, during the inference stage, the complexity amounts to $\mathcal{O}(S_p)$.

%\subsection{Service Provisioning Process}
%Integrating all the aforementioned mechanisms together, this part summarizes how the MASP serves each user.
%Specifically, it performs the following three steps.
%First, $\ell_c$ is applied to analyze the difficulty of the user prompt and generate the corresponding prompt corpus.
%Then, the MASP utilizes $\pi^{(p)}$ to refine the raw prompt.
%Afterward, the trained service provisioning policy $\pi_\theta^{(s)}$ is invoked to generate the optimal inference trial number and transmission power.
%Finally, the generative inferences are conducted, and the outputs are transmitted to the user.
%Next, we validate the effectiveness and efficiency of our proposals.

\renewcommand{\arraystretch}{1.2}
\begin{table}[pb]
\caption{The involved prompt engineering strategies.}
\begin{tabular}{p{2.3cm}|p{5.5cm}}
\Xhline{2.2pt}
\rowcolor[rgb]{0.92,0.92,0.92}
\textbf{Strategy} & \multicolumn{1}{c}{\textbf{Description}} \\
\hline
\multirow{1}{*}{Strategy 0} & \textit{Raw prompt} \\
\hline
\multirow{1}{*}{Strategy 1} & \textit{Object description} \\
\hline
\multirow{1}{*}{Strategy 2} & \textit{Object description + environment} \\
\hline
\multirow{1}{*}{Strategy 3} & \textit{Object description + mood} \\
\hline
\multirow{1}{*}{Strategy 4} & \textit{Object description + lighting} \\
\hline
\multirow{1}{*}{Strategy 5} & \textit{Object description + quality booster} \\
\hline
\multirow{1}{*}{Strategy 6} & \textit{Object description + negative effects} \\
\Xhline{2.2pt}
\end{tabular}
\end{table}
\renewcommand{\arraystretch}{1}
\renewcommand{\arraystretch}{1.2}
\begin{table}[htpb]
\caption{The experimental settings.}
\begin{tabular}{p{1.5cm}|p{2.3cm}|p{3.6cm}}
\Xhline{2.2pt}
\rowcolor[rgb]{0.92,0.92,0.92}
\textbf{Parameter} & \textbf{Description} & \textbf{Value} \\
\hline
\multirow{1}{*}{$\ell_c$} &Prompt optimizer& \textit{ChatGPT (GPT-3.5-turbo)} \\
\hline
\multirow{1}{*}{$\ell_r$} &Assessing agent& \textit{GPT-4-vision-preview} \\
\hline
\multirow{1}{*}{$\Omega$} &AIGC model& \textit{Stable Diffusion v2.0} \\
\hline
\multirow{1}{*}{$\zeta$} &Diffusion step& \textit{25} \\
\hline
\multirow{1}{*}{$Q$} &\# of users& \textit{3} \\
\hline
\multirow{1}{*}{$M$} &\# of MASP& 1 \\
\Xhline{2.2pt}
\end{tabular}
\vspace{-0.5cm}
\end{table}
\renewcommand{\arraystretch}{1}
\begin{figure*}[tbp]
\centerline{\includegraphics[width=1.95\columnwidth]{Figure_cny_1.pdf}}
\caption{The rationale of LLM-empowered assessing agent. Red, blue, and black scores are from the assessing agent (ours), NIMA, and image-reward, respectively. Note that the images in the same row are sorted in ascending order of the assessing agent's score.}
\label{rationale}
\vspace{-0.3cm}
\end{figure*}

\section{Performance Evaluation}
%\textbf{Implementation.} This part demonstrates the implementation of the proposed framework and algorithms.

\textbf{Testbed.} The experiments are conducted on a server with three NVIDIA RTX A5000 GPUs with 24 GB of memory and an AMD Ryzen Threadripper PRO 3975WX 32-Core CPU with 263 GB of RAM. 
The operating system is Ubuntu 20.04 LTS with PyTorch 2.0.1. We utilize this server to simulate an MASP and multiple uniformed distributed mobile users.
\begin{figure*}[tbp]
\centerline{\includegraphics[width=1.9\columnwidth]{Figure_cny_2.pdf}}
\caption{The effectiveness of interactive prompt engineering. Note that these cases show that prompt engineering cannot always improve generation quality. For instance, in the second row, the image generated by the refined prompt also fails to illustrate the blue car.}
\label{prompt}
\vspace{-0.5cm}
\end{figure*}
\begin{figure}[tbp]
\centerline{\includegraphics[width=0.95\columnwidth]{Figure_exp1.pdf}}
\caption{The training curves and converged utilities of default (i.e., without prompt engineering), random, empirical, PPO, and IRL prompt engineering policies.}
\label{gail}
\vspace{-0.5cm}
\end{figure}

\textbf{Configurations.} We equip an MASP with Stable Diffusion v2.0 \cite{SDpaper} to realize the text-to-image AIGC services. The diffusion step is set to 25. The user prompts are generated by ChatGPT (empowered by the GPT-4 model) in the form of ``\texttt{A [A], with [B]}". The demonstration prompts are randomly sampled from the user prompts. Based on \cite{YQNetwork}, we consider six aspects for refining raw prompts, namely \textit{object description}, \textit{environment}, \textit{mood}, \textit{lighting}, \textit{quality booster}, and \textit{negative effects}. 
\begin{itemize}
    \item \textbf{Object Description}: To facilitate fine-grained image generation, detailed descriptions of \texttt{[a]} and \texttt{[b]}'s type, texture, and features should be provided. Such details enable the AIGC model to associate more pre-learned knowledge, resulting in delicate images.
    \item \textbf{Environment}: The environment fills the background of the image, creating a real, harmonious, and beautiful scene for \texttt{[b]}. Furthermore, environment description can prevent the AIGC model from only searching and stacking the found materials about \texttt{[a]} and \texttt{[b]}, thereby further enhancing the composition quality.
    \item \textbf{Mood}: Mood describes the emotion that the users intend to convey through the image, such as joy, sadness, or hesitation, which is reflected by the color palette, the facial expressions of the characters, etc.
    \item \textbf{Lighting}: Lighting is a fundamental factor in determining the texture and authenticity of AI-generated images. The prompt for lighting should clarify the light sources and the effect of light shining on different objects.
    \item \textbf{Quality Booster}: Quality boosters refer to various adjectives that describe the user desirability, e.g., \textit{high-quality}, \textit{2k resolution}, and \textit{real texture}. By sampling from the distribution of high-quality images, the newly generated images tend to acquire higher aesthetic quality.
    \item \textbf{Negative Effects}: Negative effects depict situations that might decrease image quality. By moving the sampling distribution away from data distributions containing such negative effects, the AIGC model can prevent generated images from containing effects that decrease the quality or are undesired by users.
\end{itemize}
Then, seven prompt engineering strategies are presented (see TABLE II). This aims to filter out some irrational arrangements and reduce the action space, thereby improving the training efficiency of D$^3$PG. Note that such a principle is widely adopted since human experience and knowledge play an important role in prompt engineering \cite{2309.08532}. Moreover, users are free to customize prompt enriching aspects and prompt engineering strategies when applying our proposal to their applications. The detailed experimental settings are summarized in TABLE III.

%\textbf{Questions.} Our experiments aim to answer the following research questions.
%\begin{itemize}
%    \item[\textbf{Q1)}] Can the proxy reward predict the efficiency of applying selected prompt engineering operation on the given prompt with high precision?
%    \item[\textbf{Q2)}] Can the interactive prompt engineering effectively improve the generation quality?
%    \item[\textbf{Q3)}] Can the dynamic service provision efficiently improve resource efficiency; Can D$^3$PG solve the optimization objective in Eq. (5a) with better performance than baseline DRL?
%\end{itemize}

\subsection{Rationale of Assessing Agent}
First, we investigate the rationale of the LLM-empowered assessing agent, i.e., whether it can assess the given image fairly and comprehensively.
Fig. \ref{rationale} shows the assessment of a series of images using three methods, namely our assessing agent, NIMA \cite{8352823}, and Image-reward \cite{10.5555/3666122.3666822}.
Note that NIMA is a classic and widely adopted aesthetic quality metric trained on large-scale human feedback.
Image-reward is one of the latest AIGC-oriented assessing frameworks, which utilizes BLIP as the backbone model and supports multimodal understanding (i.e., can check the alignment between image content and prompt).
Similarly to NIMA, image-reward is also trained on large-scale human annotations, where images are rated from three aspects, namely alignment, fidelity, and harmlessness.
From Fig. \ref{rationale}, we can observe that our method outperforms NIMA and image-reward in three dimensions.
First, without multimodal understanding ability, various existing assessment methods, such as NIMA, BRISQUE\footnote{https://pypi.org/project/brisque/}, and LPIPS\footnote{https://pypi.org/project/lpips/}, cannot fit the AIGC scenarios.
The reason is that AIGC generations usually involve modality transfers, e.g., generating images from texts.
As marked by \circled{1} in Fig. \ref{rationale}, NIMA cannot associate the image with its textual prompt and gives a high score to an image that fails to illustrate the \texttt{blue car}.
Second, attributed to the massive knowledge of LLM, the assessing agent can better simulate real humans and understand the image semantics more precisely.
For instance, it correctly identifies fog in the forest, while other methods misjudge it as blurs and give low scores (see \circled{1} in Fig. \ref{rationale}).
Finally, our assessing agent can explain the reasons behind the scoring, which greatly outperforms conventional methods whose results are unexplainable.
In the above example, precise and rational explanations of the fog are provided (see \circled{3} in Fig. \ref{rationale}).


After the above analysis, we investigate whether the assessing agent's scores are consistent with aesthetics. 
To do so, we randomly select 140 images and arrange them in order from low to high scores. 
Afterward, we extract their image-reward scores as references and perform a curve fitting. 
From Fig. \ref{rationale}, we can conclude that the assessing agent and image-reward maintain high-level alignment in terms of aesthetic judgment.
Since the latter is a widely adopted and well-proven aesthetics assessment metric for AIGC, the rationale of our assessing agent is validated.





%\renewcommand{\arraystretch}{1.2}
%\begin{table}[tpb]
%\centering
%\begin{tabular}{p{2.2cm}|p{1.8cm}}
%\Xhline{2.2pt}
%\rowcolor[rgb]{0.92,0.92,0.92}
%\textbf{Resource type} & \textbf{Value} \\
%\hline
%\multirow{1}{*}{Generation latency} & \textit{3.95s} \\
%\hline
%\multirow{1}{*}{Bandwidth} & \textit{392.13 kb} \\
%\hline
%\multirow{1}{*}{Power} & \textit{0.2299 wh} \\
%\hline
%\multirow{1}{*}{GPU memory} & \textit{7247 MB} \\
%\hline
%\multirow{1}{*}{Service fee} & \textit{\$0.04/image} \\
%\Xhline{2.2pt}
%\end{tabular}
%\caption{The experimental settings.}
%\end{table}
%\renewcommand{\arraystretch}{1}
\begin{figure*}[tbp]
\centerline{\includegraphics[width=2\columnwidth]{Add_13.pdf}}
\caption{The number of required service rounds with respect to varying user requirements and inference numbers per round. (a): Default; (b): Empirical; (c): Our IRL-based approach. The orange and blue zones highlight the conditions in which only one and more than five rounds are required, respectively.}
\label{number}
\vspace{-0.3cm}
\end{figure*}

\begin{figure}[tbp]
\centerline{\includegraphics[width=0.95\columnwidth]{Add_r5.pdf}}
\caption{(a): The resource consumption for generating each image. (b): The resource consumption for performing one and two rounds of service (suppose four images are generated in each round).}
\label{hardware}
\vspace{-0.3cm}
\end{figure}

\subsection{Inspection on Prompt Engineering Policy}
In this part, we evaluate the efficiency of $\pi^{(p)}_\omega$ through two comprehensive studies.
First, Fig. \ref{prompt} illustrates the effectiveness of prompt engineering in improving generation quality. 
We randomly select three raw prompts, perform all types of prompt engineering strategies shown in TABLE II, and evaluate the generation quality using our assessing agent. 
The results clearly demonstrate that the images generated by the raw prompts suffer from significant flaws.
For instance, the water flow and fountain base are misaligned, and the blue car and dog's legs are missing from the scene. 
By enriching the prompts, the AIGC model achieves richer task descriptions and instructions, leading to substantial improvements in prompt alignment, object rendering, and image composition. 
Quantitative scores validate these improvements. 
Particularly, we observe that \textit{Strategy 6} consistently leads to the optimal generation quality across all test cases.


Then, we train $\pi^{(p)}_\omega$ using the demonstration dataset. 
To prove the superiority of our proposal in policy imitation with small-scale datasets, we set PPO as the baseline. 
Note that PPO maintains the same network architecture for policy refinement and action evaluation, while our IRL-based approach introduces a discriminator and follows an adversarial training paradigm. 
Additionally, we implement two non-learning baselines: random and empirical (i.e., always selecting \textit{Strategy 6}). 
As shown in Fig. \ref{gail}, the random policy performs similarly to non-prompt engineering. 
The empirical policy achieves higher rewards and smaller variance, as empirical experience ensures that the optimal/near-optimal policy can be selected in many cases. 
Through policy reinforcement, PPO demonstrates better adaptability and achieves more stable improvements compared to non-learning baselines but faces limitations in two aspects: 1) PPO relies solely on reward signals for optimizing policy, which can be insufficient when learning complex prompt engineering strategies from limited demonstrations; 2) The direct policy optimization in PPO may not effectively capture the nuanced relationships between prompts and generation quality present in expert demonstrations.
In contrast, our IRL approach adopts an adversarial training paradigm, which provides several key advantages: 1) The discriminator learns to distinguish between expert and policy behaviors, providing a more informative learning signal than pure reward values; 2) The adversarial training allows for better imitation of expert prompt engineering strategies by capturing both the actions and their underlying patterns; 3) The generator-discriminator architecture is particularly effective with limited demonstration data, as it can generalize from few examples through the adversarial learning process. Consequently, our IRL approach achieves the best efficiency in selecting optimal prompt engineering strategies according to specific user requests, showing consistent improvement throughout training and reaching the highest utility of approximately 8.06.

\subsection{Impact of Generation Quality on Mobile-edge Networks}
The increased generation quality directly leads to fewer re-generations, which saves substantial networking resources. 
To quantify this benefit, we explore the required number of service rounds under varying user quality requirements and MASP's per-round inference numbers. 
Specifically, we evaluate scenarios where user-required quality ranges from 7.5 to 8.5, and the number of images generated per round varies from 1 to 5. 
We compare three representative prompt engineering strategies, namely default, empirical (i.e., always selecting \textit{Strategy 6}), and IRL.
Suppose that the generation quality of each strategy follows a standard distribution. 
The mean and variance can be fitted from the sample results. 
Setting the confidence level at 90\%, we calculate the required number of service rounds. 
As shown in Fig. \ref{number}, without prompt engineering, the probability of zero re-generation is only 3/55. The empirical prompt engineering strategy improves the probability of single-round success to 9/55.
In contrast, our IRL-based approach significantly outperforms both baselines, outperforming none and empirical strategies by 6.3$\times$ and 2.1$\times$, respectively. Moreover, the probability of requiring more than five service rounds (indicated by the blue regions) is significantly reduced.

Fig. \ref{hardware}(a) benchmarks the consumption of five critical resources required to generate one image, namely generation latency, bandwidth, power, GPU memory, and service fee\footnote{The reference fee can be found at https://openai.com/api/pricing/}. 
These measurements reveal substantial resource demands of AIGC inferences for mobile servers. 
Furthermore, when one re-generation is performed, the resource overhead more than doubles since the refinement and re-transmission of prompts consume additional time and bandwidth (as shown in Fig. \ref{hardware}(b)).
Beyond quantifiable resource costs, failed generation attempts also negatively impact user QoE as their service requests remain unfilled. 
Contributed to improved generation quality through prompt IRL-based engineering, the proposed intelligent mobile AIGC service scheme achieves significantly higher resource efficiency.

\begin{figure}[tbp]
\centerline{\includegraphics[width=0.95\columnwidth]{Exp_figure7.pdf}}
\caption{The training curves and converged utilities of random, Stable Diffusion, SAC, PPO, and D$^3$PG for mobile AIGC service provisioning.}
\label{d3pg}
\vspace{-0.3cm}
\end{figure}

\begin{figure}[tbp]
\centerline{\includegraphics[width=0.95\columnwidth]{Exp_figure13.pdf}}
\caption{The exemplar service provisioning shame generated by four different methods and resulting QoE values.}
\label{d3pg-utility}
\vspace{-0.3cm}
\end{figure}

\subsection{Evaluation of Service Provisioning Policy}
Our interactive prompt engineering maximizes the generation quality of each inference trial. 
In this part, we optimize the number of inference trials per round and transmission power allocation to further improve user QoE. 
Fig. \ref{d3pg} illustrates the training curves and converged utility of different methods. 
Apart from practical solutions like Stable Diffusion, we employ two representative DRL-based baselines, namely PPO and Soft Actor-Critic (SAC) \cite{10638833}.
We observe that Stable Diffusion performs poorly as static service provisioning cannot meet heterogeneous user requirements. 
Specifically, users with simpler tasks receive excessive resources, while those with complex tasks receive insufficient support, leading to resource inefficiency. 
The random approach occasionally achieves satisfactory rewards but suffers from high variance, as shown by the scattered pink dots.
Learning-based methods achieve better performance by adapting service provisioning to different user requirements. 
As illustrated in Fig. \ref{d3pg}, both PPO and SAC improve over episodes, with PPO demonstrating faster initial learning while SAC achieving more stable long-term performance. 
Finally, D$^3$PG significantly outperforms both baselines, achieving at most 87\% improvement in converged utility.
This superiority can be attributed to two factors. First, integrating diffusion models into the actor-network enhances environmental exploration by providing structured noise injection, allowing D$^3$PG to discover better policies in the complex action space.
Second, compared to the fixed Gaussian noise in PPO and SAC, our diffusion-based policy refinement enables more precise adjustment of the action distribution, leading to better convergence and more robust performance.

Finally, Fig. \ref{d3pg-utility} shows the decisions of four methods when the user prompts are ``\texttt{A dog with a colorful collar}", ``\texttt{A garden with a fountain}", ``\texttt{A city with blue car}" and the quality thresholds are 7.6, 8.2, and 8.5, respectively.
We can observe that Stable diffusion adopts a fixed strategy with uniform transmission power allocation (i.e., 33.33\% for each user) and four inference trials each round, resulting in inefficient resource utilization. 
All three learning-based methods generate customized service provisioning schemes.
Due to inefficient environment exploration and policy refinement, SAC allocates transmission power nearly equally, leading to insufficient resources for users with higher quality thresholds. 
In contrast, PPO and D$^3$PG demonstrate superior capability in dynamic resource allocation. 
PPO adjusts both the transmission power distribution (i.e., 20-41\%) and inference trials, while D$^3$PG achieves the most efficient allocation by assigning significantly higher transmission power (i.e., 57\% of $P_{\mathrm{total}}$) to the most demanding user while maintaining balanced inference trials. 
Accordingly, D$^3$PG achieves the highest overall QoE, with 67.8\% and 7.0\% improvements over SAC and PPO, respectively.


\vspace{-0.2cm}
\section{Conclusion}
In this paper, we have presented an intelligent mobile AIGC service scheme with interactive prompt engineering and dynamic service provisioning. 
Specifically, to increase AIGC generation quality, we have proposed an IRL-based approach that leverages demonstration datasets and policy imitation to acquire optimal prompt engineering strategies. 
Then, different from fixed service provisioning, we have formulated the QoE optimization problem with respect to wireless transmission power and the number of AIGC inference trials.
Furthermore, we have presented the D$^3$PG algorithm for QoE optimization, which integrates diffusion models into the DRL framework to enhance environmental exploration capabilities.
Extensive numerical results have validated that our proposals effectively improve generation quality and user QoE through reduced service rounds and optimized resource allocation.
More importantly, our proposals are unified and can support various mobile AIGC applications.





% if have a single appendix:
%\appendix[Proof of the Zonklar Equations]
% or
%\appendix  % for no appendix heading
% do not use \section anymore after \appendix, only \section*
% is possibly needed

% use appendices with more than one appendix
% then use \section to start each appendix
% you must declare a \section before using any
% \subsection or using \label (\appendices by itself
% starts a section numbered zero.)


% Can use something like this to put references on a page
% by themselves when using endfloat and the captionsoff option.
\ifCLASSOPTIONcaptionsoff
  \newpage
\fi



% trigger a \newpage just before the given reference
% number - used to balance the columns on the last page
% adjust value as needed - may need to be readjusted if
% the document is modified later
%\IEEEtriggeratref{8}
% The "triggered" command can be changed if desired:
%\IEEEtriggercmd{\enlargethispage{-5in}}

% references section

% can use a bibliography generated by BibTeX as a .bbl file
% BibTeX documentation can be easily obtained at:
% http://mirror.ctan.org/biblio/bibtex/contrib/doc/
% The IEEEtran BibTeX style support page is at:
% http://www.michaelshell.org/tex/ieeetran/bibtex/
%\bibliographystyle{IEEEtran}
% argument is your BibTeX string definitions and bibliography database(s)
%\bibliography{IEEEabrv,../bib/paper}
%
% <OR> manually copy in the resultant .bbl file
% set second argument of \begin to the number of references
% (used to reserve space for the reference number labels box)


% biography section
% 
% If you have an EPS/PDF photo (graphicx package needed) extra braces are
% needed around the contents of the optional argument to biography to prevent
% the LaTeX parser from getting confused when it sees the complicated
% \includegraphics command within an optional argument. (You could create
% your own custom macro containing the \includegraphics command to make things
% simpler here.)
%\begin{IEEEbiography}[{\includegraphics[width=1in,height=1.25in,clip,keepaspectratio]{mshell}}]{Michael Shell}
% or if you just want to reserve a space for a photo:

% insert where needed to balance the two columns on the last page with
% biographies
%\newpage
% You can push biographies down or up by placing
% a \vfill before or after them. The appropriate
% use of \vfill depends on what kind of text is
% on the last page and whether or not the columns
% are being equalized.

%\vfill

% Can be used to pull up biographies so that the bottom of the last one
% is flush with the other column.
%\enlargethispage{-5in}

\bibliographystyle{IEEEtran}
\bibliography{InteractivePE}
\vfill


\end{document}


