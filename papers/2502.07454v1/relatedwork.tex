\section{Related Work}
Peters~\cite{Peters17} showed that some $d$-Euclidean elections inherently require exponentially many bits to even represent any of their $d$-Euclidean embeddings. Bennet and Hays~\cite{HaysW1961,Bennett1960} studied what is the sufficient dimension $d$ for an election to be $d$-Euclidean. They also showed that the maximum number of voters with distinct preference ranking in a $d$-dimensional election with $m$ candidates is equal to $\sum_{k=m-d}^m |s(m,k)|$, where $s(m,k)$ are the Stirling numbers of the first kind. The same result was later obtained by Good and Tideman~\cite{Good1977}. Bogomolnaïa and Laslier~\cite{BogomolnaiaL07} studied lower bounds on $d$ based on the election to guarantee a $d$-Euclidean embedding. Kamiya,Takemura and Terao~\cite{KamiyaTT11} established the number of maximal $d$-Euclidean profiles if the number of candidates $m$ satisfies $m=d-2$ and they were able to enumerate them for $m=4$. A simpler geometrical proof of this characterization for $m=4$ was later given by Escoffier, Spanjaard and Tydrichová~\cite{EscoffierST22_arx}. Bulteau and Chen~\cite{BulteauC23} showed that any election with at most $2$ voters is $2$-Euclidean and for $3$ voters they show that any election with at most $7$ candidates is $2$-Euclidean.  There are also results considering similar topics  in a different metric than the standard Euclidean, e.g., the $\ell_1$ and $\ell_\infty$ metrics~\cite{EscoffierST22_arx,Chen22_arx}. 

The most important previous work for us is the paper of Escoffier, Spanjaard and Tydrichová~\cite{EscoffierST23}. In their work, they propose the first algorithm (partially) deciding whether given election is $2$-Euclidean. Their algorithm consists of two phases. In the first phase, the aim is to find a no-certificate based on the maximal non-$2$-Euclidean elections on $4$ candidates. This part is implemented using exhaustive search on all subsets of $4$ candidates and then finding an appropriate mapping of candidates which is also done by brute-force. In the second phase, they aim to find an embedding using a certain randomized procedure. The second phase terminates within some time limit, in which case the algorithm reports \texttt{Unknown}. Their experiments show that many real-world elections are not $2$-Euclidean (i.e., the algorithm finishes within the first phase). The considered real-world instances are from the PrefLib dataset~\cite{Preflib}. We will refer to this particular algorithm as the EST algorithm.