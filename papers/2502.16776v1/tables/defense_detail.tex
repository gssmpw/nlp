\begin{table*}[!t]
\centering
\small
\renewcommand{\arraystretch}{1.2}
\setlength{\tabcolsep}{8pt} 
\resizebox{\linewidth}{!}{
\begin{tabularx}{\textwidth}{|m{1.5cm}|m{4cm}|m{3.0cm}|m{5.28cm}|} 
\hline
\textbf{Type} & \textbf{Description} & \textbf{Usage} & \textbf{Methods} \\
\hline
\textbf{Preprocess} & 
The input is preprocessed to detect or guard against harmful content before the generation process. & 
defend(self, message) $\rightarrow$ (defended\_message, if\_reject) & 
\makecell[l]{PPL \cite{PPL} \\ Self Reminder \cite{selfreminder} \\ Prompt Guard \cite{promptguard} \\ Goal Prioritization \cite{goal_prioritization}\\ Paraphrase \cite{paraphrase}\\ ICD \cite{ICD}} \\
\hline
\textbf{Intraprocess} & 
Utilizes safer decoding or generation strategies to ensure the outputs are secure and adhere to safety guidelines. & 
defend(self, model, messages) $\rightarrow$ response & 
\makecell[l]{SmoothLLM \cite{smoothllm}\\ SafeDecoding \cite{safedecoding} \\ DRO \cite{DRO} \\ Erase and Check \cite{erase_and_check} \\ Robust Aligned \cite{cao2023defending} } \\
\hline
\textbf{Postprocess} & 
The output is postprocessed to enhance safety. & 
defend(self, response) $\rightarrow$ defended\_response & 
\makecell[l]{Self Evaluation \cite{self-evaluation} \\ Aligner \cite{aligner}} \\
\hline
\end{tabularx}
}
\caption{Overview of Inference-Time Defense Types and Their Components}
\label{tab:defenders}
\end{table*}