% A material structure and composition collectively determine its electronic, optical, and mechanical properties, which in turn define its functionality.
Exploring the vast compositional and structural landscape of multicomponent materials with novel properties is essential, yet exhaustive experimental screening is infeasible \cite{cantor_multicomponent_2021}. Quantum and classical molecular simulation offer a powerful alternative, enabling a more targeted and efficient exploration.
% Decades of computational advancements have been instrumental in enabling the accurate prediction of material structures and properties, driving progress in materials discovery. \ph{Since our introduction is quite long, we could maybe just cut the previous paragraph.}
In recent decades, both experimental \cite{potyrailo_combinatorial_2011, maier_early_2019} and computational \cite{jain_highthroughput_2011, curtarolo_highthroughput_2013} high-throughput pipelines have led to a proliferation of databases for materials' crystal structures \cite{bergerhoff_inorganic_1983, mehl_aflow_2017} and simulations \cite{blaiszik_materials_2016, vita_colabfit_2023, fuemmeler2024advancing}. These advances have already facilitated the development of more accurate machine-learned interatomic potentials \cite{batzner_e3equivariant_2022, batatia_mace_2022, chen_universal_2022}.

% inverse design of novel compositions and their stable structures, however, has become significantly more tractable through the application of machine-learning methods \cite{butler_machine_2018, schmidt_recent_2019}.
Still, efficiently sampling the manifold of stable materials structures under diverse constraints---such as composition and target properties---remains a major challenge. Traditional approaches to materials discovery have relied on first-principles electronic structure methods such as DFT---or higher levels of theory, depending on the property \cite{booth2013towards, PhysRevB.89.205427, PhysRevB.102.045146}---which, while powerful and fairly accurate, are very computationally expensive. These methods include \textit{ab initio} random structure searching (AIRSS) \cite{pickard_initio_2011} or genetic algorithms for structure and phase prediction \cite{tipton_grand_2013}, both of which have successfully predicted new crystal structures, some of which have even been experimentally realized \cite{oganov_structure_2019}.
%Genetic Algorithm for Structure and Phase prediction
However, the high computational cost of these approaches has limited the scope and speed of material exploration, highlighting the need for cutting-edge ML techniques to significantly accelerate the discovery of stable inorganic crystalline materials.