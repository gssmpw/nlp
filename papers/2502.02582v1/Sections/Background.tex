% Diffusion models
\subsection{Diffusion models}

A widely used approach in generative modeling uses diffusion models \citep{sohl-dickstein_deep_2015}, which define a stochastic process that progressively transforms structured data into
noise. A model is then trained to approximate the reverse
process, enabling the generation of new samples. These
models are often formulated in terms of SDEs, where the
forward process follows a predefined diffusion dynamic.

Score-based diffusion models (SBDMs) are an instantiation of diffusion models that learn a score function---the gradient of the log probability density---to guide the reversal of the diffusion process via numerical integration \citep{song_scorebased_2021}. 
SBDMs have demonstrated remarkable success in generating high-quality and novel samples across a wide range of applications where the target distribution is complex and intractable, such as photorealistic image generation \citep{saharia_photorealistic_2022} and molecular conformation prediction \citep{corso_diffdock_2023}.

% Riemannian flow matching
\subsection{Conditional flow matching}
Conditional flow matching (CFM) \citep{lipman_flow_2023} is a generative modeling technique that learns a flow which transports samples from a base distribution at time $t=0$ to a target distribution at time $t=1$. This process defines a probability path that describes how samples are distributed at any intermediate time $t \in [0, 1]$.
The velocity field associated with this flow governs how individual samples evolve over time. 
CFM learns the velocity indirectly by constructing conditional vector fields that are known \textit{a priori}. Once trained, samples drawn from the base distribution can be evolved numerically to generate new samples from the target distribution.
%During training a parameterized neural network is regressed onto this conditional vector field by taking an expectation over many samples. 
Originally, CFM was formulated using Gaussian conditional probability paths, but \citet{tong_improving_2024} later extended this framework to allow for arbitrary probability paths and couplings between base and target distributions. 
A further extension, particularly relevant to physics and chemistry, is Riemannian flow matching (RFM), which generalizes CFM to Riemannian manifolds \citep{chen_flow_2024}. This allows in particular to use the flow-matching framework for systems with periodic boundary conditions as they appear in unit cell representations of inorganic crystals \citep{miller_flowmm_2024}.
%like in robotics or chemistry \cite{braun_riemannian_2024, miller_flowmm_2024}.





