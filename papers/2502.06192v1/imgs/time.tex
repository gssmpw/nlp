\begin{figure}[htbp!]
    \centering
    \includegraphics[width=1.0\linewidth]{SPACE/imgs/figs/fig3_time.pdf}
    % \vspace{0 pt}
    \caption{Impact of different initiating times of Spaced KD ($s=1.5$), which is introduced \textbf{(a)} for constant 10 training epochs or \textbf{(b)} till the end of training. %Spaced KD is effective at the later stage of training, rather than the early stage.
    }
    \label{fig:time}
    % \vspace{-.0cm}
\end{figure}


% Both results indicate that initiating Spaced KD in the later stages of training is more beneficial for improving the performance of the student network. This suggests that in KD learning, not only the intervals between learning sessions are important, but the timing of when learning occurs is equally crucial.
