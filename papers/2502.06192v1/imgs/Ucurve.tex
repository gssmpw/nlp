\begin{figure*}
    %\vspace{-.1cm}
    \centering
    \includegraphics[width=0.8\linewidth]{SPACE/imgs/figs/fig2_Ucurve.pdf}
    \vspace{-.2cm}
    \caption{Alignment of spaced learning in BNNs and DNNs. \textbf{(a)} Computational cognitive model of spaced learning, modified from~\cite{landauer1969reinforcement}. \textbf{(b)} Overall performance of Spaced KD from different networks and benchmarks. R18: ResNet-18; R50: ResNet-50; R101: ResNet-101; C100: CIFAR-100; T200: Tiny-ImageNet. \textbf{(c)} Quadratic polynomial fitting of all performance from \textbf{(b)}. %All of them show that spaced training with appropriate interval can enhance learning performance. 
    %Depiction of the cognitive model was modified from~\cite{landauer1969reinforcement}.
    }
    \label{fig:main_result}
     \vspace{-.3cm}
\end{figure*}


% /imgs/figs/\begin{figure}[t]
%     \centering
%     \begin{subfigure}{0.245\linewidth}
%     \includegraphics[width=\linewidth]{SPACE/imgs/figs/fig1_main_result_model.png}
%     \caption{Cognitive model}
%     \end{subfigure}
%     \begin{subfigure}{0.4\linewidth}
%     \includegraphics[width=\linewidth]{SPACE/imgs/figs/fig1_main_result.png}
%     \caption{Spaced KD}
%     \end{subfigure}
% \caption{\textbf{Diagram of computational cognitive model (a) of spaced learning and increased performance (b) of spaced KD.} Both of them shows spaced training is more effective than massed training, and there may be an optimal interval. The cognitive model was modified from Landauer~\citep{landauer1969reinforcement}.
% }
% \label{fig:main_result}
% \vspace{-10pt}
% \end{figure}
