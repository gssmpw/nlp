\section{Related Works}
Brain tumor research in low- and middle-income nations has been hindered by the limited availability of MRI scanners \cite{murali2023bringing}. Glioma mortality rates are notably high in Sub-Saharan Africa, highlighting the urgent need for improved access to advanced imaging and multidisciplinary treatment. Manually segmenting brain tumors from MRI images for tumor detection is a demanding and time-consuming task that is prone to variation between different observers \cite{razzak2018efficient,sun2019drrnet}. This variation can greatly impact the accuracy and consistency of the segmentation results. Since a single brain MRI scan consists of multiple slices that collectively form a 3D anatomical view, the manual segmentation of brain tumor MR images becomes a complex procedure \cite{sun2019drrnet}. Furthermore, computer-aided diagnosis through machine learning can significantly improve tumor diagnostic accuracy, early detection, classification, and prognosis of patient survival rates.

Recent studies have shown that deep learning CNN-based auto-segmentation models can significantly improve efficiency and effectiveness. Conventional deep learning methods, such as CNN, require substantial amounts of labelled data for optimal learning, which presents challenges in the medical domain \cite{razzak2018efficient}. Deep-learning CNNs have proven to be highly valuable in accurately segmenting various structures during the treatment planning phase \cite{liang2020emerging,ibragimov2017segmentation,lustberg2018clinical,jackson2018deep,hu2016automatic,ibragimov2017combining}.
Again, ensemble techniques for brain tumor segmentation further enhance whole tumor, tumor core, and enhancing tumor on the test dataset, respectively \cite{khan2023hybrid,koirala2023automated}.