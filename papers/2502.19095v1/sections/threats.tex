\section{Threats to Validity}
\label{sec:threats}

\noindent \textbf{Internal validity}. 
The training process of the adversarial agents is inherently non-deterministic. To ensure reliability of our findings, we repeated the training process for each agent ten times. For transparency and correctness of implementation, we have made our code publicly available, and we utilized well-known open-source frameworks for our implementation.

\noindent \textbf{External validity}.
While the employed dataset may not be exhaustive in representing every type of XSS attack, it is substantial and publicly accessible. Moreover, it has been widely used in previous research, establishing it as a suitable benchmark for evaluating XSS detection methods.

\noindent \textbf{Construct validity}. We employed standard evaluation metrics in the security domain, including Precision, Recall, Accuracy, and F1-Score, to assess the detectors. For the adversarial agents, we used the escape rate as an evaluation metric, which aligns with the reference work.
