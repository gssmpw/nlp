\section{Methodology}
\label{sec:method}

The main idea of this paper is to introduce an XSS Oracle.
As a first step, we demonstrate the Oracle's usefulness in assessing the validity of payloads and their potential impact. Furthermore, the Oracle can aid in developing a robust defense model, which allows an accurate evaluation of the performance of the approach proposed by Chen et al.~\cite{CHEN2022102831}.



\subsection{XSS Oracle}

\begin{figure}[!h]
  \centering
  \includegraphics[width=0.95\columnwidth]{images/Oracle.pdf}
  
  \caption{Workflow of the XSS Oracle. A Payload is rendered in a known template of a Web Page and the DOMs of the two pages are compared.}\label{fig:Oracle}
\end{figure}

Based on the presence of an XSS attack inside of the payload of an HTTP request, we can consider two types of payload: `Benign' and `Malicious'.
Benign payloads do not alter the DOM structure when executed, while malicious payloads cause changes in the DOM, potentially affecting the browser environment. As outlined in Figure~\ref{fig:Oracle}, we utilize the Oracle to mimic payload execution, observing the DOM of a template page rendered by a web server. The server accepts the payload as a parameter and incorporates its elements into the template. The Oracle then examines the DOM of the new page. If any differences are detected, the payload is labelled as Malicious; otherwise, it is classified as Benign.



\subsection{Metrics}

TH1 and TH2 arise from the agent's modifications and preprocessing of the payload, which could alter the characteristics of the XSS attack. To address this, we employ an XSS Oracle that assesses the integrity of the attack properties, introducing the metric Ruin Rate (RR).
Let us consider a set of payloads labeled as Malicious, denoted as $M=\{m_1, m_2, \ldots\}$. This set $M$ is generic and can include malicious samples from the original dataset or those generated by the XSS adversarial method. 
How to structure the different sets for the evaluation of the different threats to validity will be discussed in Section~\ref{sec:empirical_study}. %


We define a function $O(p)$ that, for any payload $p$, returns 1 if the Oracle classifies $p$ as Malicious and 0 otherwise. For any set $M$,  RR can be calculated as:

\begin{equation}
    RR(M) = 1 - \frac{\sum_{m \in M}{O(m)}}{|M|}
\end{equation}

If $M$ contains samples from the original dataset, a non-zero $RR(M)$ indicates mislabeled examples. Conversely, if $M$ consists of adversarial examples derived from an original set with $RR=0$, a non-zero $RR(M)$ points to an adversarial process that has compromised the attack's properties.

TH1 and TH2 are also potentially related to an anomalous number of Out-Of-Vocabulary (OOV) tokens in the array fed to the detection model.
We introduce a second metric, called OOV-Rate ($OR$), to evaluate this aspect.
For an array of tokens $V$, $OR(V)$ is the number of the `None' tokens present inside $V$ (these represent OOV tokens) over the length of $V$:
\begin{equation}
    OR(V) = \frac{\sum_{v \in V}{OOV(v)}}{|V|}
\end{equation}

\noindent
where  function $OOV(v)$ is 1 if $v$ = \textit{None}, 0 otherwise.




