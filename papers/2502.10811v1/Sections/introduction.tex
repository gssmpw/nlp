\section{Introduction}

%
\emph{Initial semantics} as a concept aims to characterize inductive structures and their
properties as initial objects in suitable categories.
%
We focus on the strand of initial semantics that aims to characterize (higher-order)
programming languages as initial objects in well-chosen categories of ``models''.
%
Considering programming languages categorically, as appropriate initial models, has many advantages.
%
By choosing models of a programming language to be suitable algebras, models
provide us with a mathematical abstraction of the syntax of the language, including
variable-binding constructions.
%
Indeed, in a model, the terms and the constructors of the language are
represented as abstract categorical notions that capture their properties.
%
This enables us to manipulate the language by only considering these categorical
notions, without having to constantly deal with its syntax and implementation,
in particular with the implementation of binders.
%
Furthermore, the initiality of the model offers an abstract recursion principle.
Indeed, due to initiality, there exists a model morphism from the initial model,
which represents syntax, to any model.
%
Unfolding the definitions, this provides us with a recursion principle in the
same categorical terms as the ones used for the abstract syntax,
thus enabling us to reason by recursion on the abstract syntax without
having to deal with any underlying implementation.
%
Initial models can therefore abstract the syntax with its recursion
principle.

%
Using more advanced notions of model, it is possible to account for syntax with
its substitution structures \cite{FPT99,HirschowitzMaggesi07,Hss04}.
%
This involves abstracting the syntax, but also abstracting how simultaneous
substitution works, its properties, and how it interacts with constructors.
%
In particular, we refer to initial semantics as the theory of designing models
of languages, accounting for syntax and substitution, and proving initiality theorems.
That is, we aim to prove theorems asserting, ideally under simple conditions, that a language has an initial model.
%
Initiality theorems enable us to prove that a language has a well-defined
substitution structure, but by initiality, they also enable us to equip the
language with a substitution-safe recursion principle.
%
By this, we mean a recursion principle that respects the substitution structure by
construction, which is important if one wishes to preserve the operational semantics of
languages.



\subsection{Different Approaches to Initial Semantics}

Three distinct yet similar approaches to initial semantics have been proposed.

%
The most well-known approach to initial semantics, but also the first to be invented,
was introduced by Fiore, Plotkin, and Turi in their seminal paper \cite{FPT99}.
%
In \cite{FPT99}, focusing on untyped languages, the authors introduced
$Σ$-monoids to capture languages with their substitution, and suggested
a very general method based on monoidal categories and functors with strength
to prove initiality theorems.
%
This framework has been developed further to encompass richer type systems
\cite{Cbn02,MMCCS05,SecondOrderDep08,Polymorphism11,ListObjects17},
but also to add meta-variables and equations
\cite{HamanaMetavar04,SecondOrderDep08,HurPhd,PolymorphismEq13,FioreSzamozvancevPopl22}.
It has also recently been extended to skew-monoidal categories \cite{CellularHoweTheorem20}.

%
Another approach to initial semantics was later developed by André Hirschowitz
and Maggesi \cite{HirschowitzMaggesi07}, using modules over monads.
%
This approach has the advantage of providing a very general and abstract definition
of signatures and models \cite{HirschowitzMaggesi12,ZsidoPhd10,PresentableSignatures21},
that can be easier to manipulate.
%
For instance, it has been extended to account for equations
\cite{PresentableSignatures21,2Signatures19} and reduction rules
\cite{UntypedRelativeMonads16,TypedRelativeMonads19,ReductionMonads20,TransitionMonads22}.
%
However, modules over monads and associated notions only have been defined and
studied for endofunctor categories, that is, for particular monoidal categories.
%
Moreover, this general notion of signature lacks some desirable theoretical properties like a general initiality theorem.
Only restricted and instance-specific ones have been proven \cite{HirschowitzMaggesi10,ZsidoPhd10},
for instance for the category $[\Set,\Set]$.

%
A third, less-known approach based on heterogeneous substitution systems,
abbreviated as hss, has been developed.
%
Hss were originally invented to prove that both wellfounded and
non-wellfounded syntax \cite{Hss04} have a monadic substitution structure.
%
It was applied in this way in the UniMath library, to construct untyped and
simply-typed higher-order languages \cite{HssUntypedUniMath19,HssTypedUnimath22}.
%
Yet, hss were also shown to provide a framework for initial semantics by
Ahrens and Matthes in \cite{HssRevisited15}.
%
This approach is based on functors with strength and endofunctor categories,
and has general initiality theorems, and seems to have a stronger notion
of models.

These approaches have been extended to model various added structures like meta-variables
\cite{SecondOrderDep08}, equations between terms \cite{FioreHur10,2Signatures19},
or reductions rules \cite{UntypedRelativeMonads16,ReductionMonads20}.
%
However, we do not discuss them in this work and only mention them in the discussion of related work.


Importantly, we discuss in this paper the ``nested datatype'' approach to variable binding.
Here, a lambda abstraction is represented by a constructor $\lambda : T(X+1) \to T(X)$, where $X$ and $X+1$ are contexts of free variables.
There are other ways to represent variables and binding:
firstly, the ``higher-order abstract syntax'' approach --- where lambda abstraction is represented by a constructor $\lambda : (T \to T) \to T$ --- was explored, in particular, by Hofmann \cite{DBLP:conf/lics/Hofmann99}.
Secondly, the nominal approach --- where lambda abstraction is represented by a constructor $\lambda : [\mathbb{A}]T \to T$, for a set of ``atoms'' $\mathbb{A}$ --- was explored, in particular, by Gabbay and Pitts \cite{DBLP:conf/lics/GabbayP99}.
The discussion of these alternative approaches to variable binding and their relationship to the nested approach are outside the scope of this work.


\subsection{The Challenges of Relating the Different Approaches}

%
The starting point for this work is the observation that, while similar, the formal
links between this three approaches are understudied, and still largely unknown.

%
While similar, the different traditions present several important structural differences.
%
The different frameworks are not all defined for the same underlying structures.
The $Σ$-monoid tradition is defined for generic monoidal categories
\cite{SecondOrderDep08}, whereas the hss and modules over monads traditions
only deal with endofunctor categories \cite{Hss04,PresentableSignatures21},
very particular monoidal categories.
%
More importantly, the different frameworks uses different notions of signatures
and models, and it is not always clear how these notions relate to each other.
Even in one specific tradition, different works may employ different notions of signatures and models.
For instance, the notions of signature differ between \cite{HirschowitzMaggesi07,ZsidoPhd10}
and \cite{PresentableSignatures21,2Signatures19}.
%
The different frameworks also have different relationship towards the initiality theorem.
The $Σ$-monoid tradition provides a generic adjoint theorem \cite{SecondOrderDep08},
from which an initiality theorem is deduced, whereas the hss one only has an
initiality theorem \cite{HssRevisited15}, and the modules over monads tradition
only has initiality theorems for very specific signatures and endofunctor categories like
$[\Set,\Set]$ or $[\Set^T,\Set^T]$ \cite{HirschowitzMaggesi10},\cite[Section 6]{ZsidoPhd10}.
%
Whatever the tradition, there also are different versions of the initiality theorem.
For instance, some rely on monoidal closedness \cite{SecondOrderDep08,HssRevisited15},
while others rely on $\omega$-cocontinuity \cite{ListObjects17,HssUntypedUniMath19}.



%
%
This is complicated by the fact that the existing literature on initial semantics
seems, to us, difficult to access for newcomers to the field and even experts.
%
Many important notions are spread out over different papers, often with small
technical yet non trivial variations.
%
Moreover, many papers of the subject have been published
as extended abstracts or in conference proceedings without appendices.
%
As such, they usually contain very limited discussion of related work, few fully worked-out
examples, and often no proofs or only sketches of proofs.

%
The relationship between the different approaches has been little studied in the literature.
When it has, the result are little known, do not cover the full extent of the
variations mentioned before, and are sometimes very specific.
%
For instance, for untyped algebraic signatures, the links between $Σ$-monoids on $[\mb{F},\Set]$,
and module based models on $[\Set,\Set]$, have been investigated in Zsidó's
dissertation \cite{ZsidoPhd10}.
Furthermore, Zsidó also investigated the links for simply-typed algebraic signatures, for
the categories $[\mb{F} \downarrow T,\Set]^T$ and $[\Set^T,\Set^T]$.
%
As another example, for the particular monoidal category $[\Set,\Set]$, signatures with strength
and $Σ$-monoids have been shown to be a particular case of signatures and
modules based models \cite{HirschowitzMaggesi12}.
%
Finally, recently, and independently from us, heterogeneous substitution systems
have been shown to be an adequate abstraction to prove an initiality result
for $Σ$-monoids, as very briefly mentioned in \cite[Section 4.4]{HssNonWellfounded24}.


\subsection{Contributions}

In this work, we bridge this gap in the literature by presenting a framework
that unifies the different traditions of initial semantics.
%
We achieve this by appropriately generalizing and integrating different
components of the existing frameworks.
%
Doing so, we demonstrate that each tradition of initial semantics addresses a
distinct aspect of the problem, assembling into a comprehensive framework.
%
More specifically:

\begin{enumerate}
  \setlength\itemsep{-1pt}
  \item We base the framework on monoidal categories as these encompass the
        great majority of the existing instances, including categories like
        $[\mb{F},\Set]$ that are not endofunctor categories.

  \item We present a general and abstract framework for signatures and models,
        using modules over monoids, a variant of modules over monads for
        monoidal categories.
        We further port and generalize the different results from the tradition of modules
        over monads to our framework.

  \item This framework is too general to prove general initiality and adjoint theorems.
        Consequently, to state and prove such theorems we restrict ourselves
        to signatures with strength and $Σ$-monoids, particular signatures and models.
        We further show how they appear naturally when one tries to state and prove an initiality result.

  \item We leverage heterogeneous substitution systems that we generalize to monoidal
        categories to prove the initiality theorem, efficiently and modularly.
        To do so, we generalize and adapt to our purpose a proof scattered
        throughout several papers \cite{Hss04,DeBruijnasNestedDatatype99,HssRevisited15,HssUntypedUniMath19}.

  \item We prove the adjoint theorem from the initiality theorem rather than the
        opposite, and rely on $\omega$-cocontinuity rather than on monoidal closedness.
        This is a key difference for relating the different approaches.

\end{enumerate}

To fully justify our design choices and our claim that we encompass the different
instances, we provide an extensive discussion of related work in which we
compare the different approaches and their variations to our presentation.
%
In particular, this enables us to shed a new light onto the literature, clarifying
the links between the different approaches through comparaison with our framework.

Altogether, with this work, we aim to provide a self-contained presentation
that can serve as a foundation for exploring more advanced concepts in initial
semantics, such as those involving meta-variables or reduction rules.
We leave the exploration of these advanced topics as an open problem.



\subsection{Synopsis}
\label{sec:synopsis}

%
In \cref{sec:prelims}, we briefly recall the definition and a few results about
monoidal categories and $ω$-colimits that are used in the paper.
%
We present our framework for initial semantics in
\cref{sec:models,sec:initiality_theorem,sec:building_initial_model}.
%
Specifically, we present an abstract framework based signatures and models based
on modules over monoids in \cref{sec:models}.
%
As not all signatures have an initial model, to prove an initiality theorem and
an adjoint theorem, we restrict ourselves to signatures with strength.
We show they naturally arise as particular signatures when stating and
proving these theorems, in \cref{sec:initiality_theorem}.
%
We then provide a self-contained and complete proof of the initiality theorem
and adjoint theorem, based on heterogeneous substitution systems, in
\cref{sec:building_initial_model}.
%
We provide an extensive analysis of the literature, and use our framework as a
cornerstone to relate the different approaches to each other, in \cref{sec:related-work}.
%
We conclude and discuss open problems in \cref{sec:conclusion}.
