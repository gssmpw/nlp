\section{The Initiality Theorem}
\label{sec:initiality_theorem}

%
The general notion of signature is built to be fully abstract and practical to
use, e.g. to prove a \hyperref[prop:modularity-models]{modularity result}, or
add equations \cite{2Signatures19} or reduction rules \cite{ReductionMonads20}.
%
However, it lacks many desirable properties.
There is no known criterion for a signature to have, or not to have, an initial model.
Moreover, there is no known criterion for the product or coproduct of two
representable signatures to be representable.

%
Consequently, to provide an initiality theorem we must restrict ourselves to
better behaved signatures.
%
As seen in \cref{prop:fixpoint-models}, the initial model of a representable
signature $Σ$ is a fixpoint of models for the functor $\ul{\Id} + Σ$.
%
Since we are looking for an initial model with an underlying algebra structure,
Adámek's theorem (\cref{thm:adamek}) and Lambek's theorem (\cref{lemma:abstract-Lambek})
strongly suggests to consider $ω$-cocontinuous functors $Σ : \mc{C} → \mc{C}$.
%
To turn $Σ$ into a signature, for any monoid $R : \Mon(\mc{C})$ we must equip
$Σ(R)$ with an $R$-module structure, that is, with a module substitution $Σ(R) ⊗ R → Σ(R)$.
%
Thanks to the monoid structure and functoriality of $Σ$, we have a morphism  $Σ(μ) : Σ (R ⊗ R) → Σ(R)$ in $\mc{C}$.
%
It hence suffices to have a morphism $θ : Σ(R) ⊗ R → Σ(R ⊗ R)$ such that the composition
\[
  \begin{tikzcd}
    Σ(R) ⊗ R \ar[r, "θ"]
    &
    Σ(R ⊗ R) \ar[r, "Σ(μ)"]
    &
    Σ(R)
  \end{tikzcd}
\] satisfies the module substitution laws.
%
Unfolding the constraints leads us to \emph{signatures with strength},
which intuitively are better behaved as
their substitution is of a particularly constrained form.
We will prove an initiality theorem for such signatures with strength

%
\begin{related Work}
  Generalizing work on modules over monads to monoidal categories
  now naturally leads us to consider signatures with strength.
  However, historically, signatures with strength were actually considered \emph{before}
  modules over monads, in the seminal work of Fiore, Plotkin, and Turi \cite{FPT99}.
  %
  Signatures with strength have since been studied extensively \cite{SecondOrderDep08,FioreMahmoud10,HurPhd,ListObjects17}.
  %
  See \cref{subsec:rw-overview} for an historical perspective on initial semantics..
\end{related Work}

%
In the remainder of this section, we start by reviewing signatures with
strength in \cref{subsec:sig_with_strength}, before discussing the links
between signatures with strength and $Σ$-monoids with signatures and models
in \cref{subsec:sigstrength-to-sig}.
We then state an initiality theorem and an adjoint theorem for signatures
with strength in \cref{subsec:initiality_theorem}, that we prove in \cref{sec:building_initial_model}.



\subsection{Signatures with Strength}
\label{subsec:sig_with_strength}

For a notion of strengths $H(A) ⊗ B → H(A ⊗ B)$ to be suitable to model constructors,
we need at the minimum a strength for variable binding for untyped languages.
%
As discussed in \cref{ex:sigstrength-variable-binding}, to define such a signature requires $B$
to be more than just an object of $\mc{C}$, it needs to be a \emph{pointed} object.

\begin{definition}[Pointed Object]
  \label{def:pointed-object}
  In a monoidal category $\mc{C}$, a \emph{pointed object} is a tuple $(Z,e)$
  where $Z : \mc{C}$ and $e : I → Z$.
  A morphism of pointed objects $(Z,e) → (Z',e')$ is a morphism $f : Z → Z'$
  such that $f \circ e = e'$.
  Pointed objects form a category $\mathrm{Ptd}(\mc{C})$.
\end{definition}

Having defined pointed objects, we can now define signatures with strength:

\begin{definition}[Signatures with Strength]
  \label{def:sig-strength}
  A \emph{signature with (pointed) strength} in a monoidal category $\Cmon$ is a
  pair $(H,θ)$ where $H : \mc{C} → \mc{C}$ is an endofunctor with a \emph{strength} $θ$,
  that is for all $A : \mc{C}$ and pointed object $b : I → B$ a natural transformation :
  %
  \[ θ_{A,b} : H(A) ⊗ B \longrightarrow H(A ⊗ B) \]
  %
  such that for all $A, b : I → B, c : I → C$, $θ$ is compatible with the
  monoidal associativity and unit, with $b ⊗ c$ as an abbreviation for
  $I \xrightarrow{λ^{-1}_I} I ⊗ I \xrightarrow{b ⊗ c} B ⊗ C$:
  %
  \begin{align*}
    \begin{tikzcd}[ampersand replacement=\&, column sep=large]
      %
      (H(A) ⊗ B) ⊗ C \ar[d, swap, "θ_{A,b} ⊗ C"]
        \& H(A) ⊗ (B ⊗ C) \ar[l, swap, "\alpha^{-1}_{(H(A),B,C)}"]
                                     \ar[dd, "θ_{A, b ⊗ c}"] \\
      %
      H(A ⊗ B) ⊗ C \ar[d, swap, "θ_{A ⊗ B, c}"]
        \& \\
      %
      H((A ⊗ B) ⊗ C) \ar[r, swap, "H(\alpha_{A,B,C})"]
        \& H(A ⊗ (B ⊗ C))
    \end{tikzcd}
    &&
    \begin{tikzcd}[ampersand replacement=\&]
      H(A) ⊗ I \ar[r, "θ_{A,id}"] \ar[d, swap, "\rho_{H(A)}"]
        \& H(A ⊗ I) \\
      H(A) \ar[ur, swap, "H(\rho^{-1}_A)"]
    \end{tikzcd}
  \end{align*}
\end{definition}

\begin{related Work}
  Using strength to model substitution was introduced in \cite{FPT99}, and
  as a formal notion of signatures in \cite{Hss04}.
  %
  We follow a similar presentation to \cite{Hss04}; but a more general one
  defined in terms of actegories can also be found in \cite{SecondOrderDep08}
  and \cite{HssNonWellfounded24}.
\end{related Work}

\begin{definition}[Morphism of signatures with strength]
  A morphism of signatures with strength $(H, θ) → (H', θ')$ is a natural
  transformation $h : H → H'$ compatible with the strengths, i.e., such that
  for all $A$ and $b : I → B$:
  %
  \[
    \begin{tikzcd}
      H(A) ⊗ B \ar[r, "h_A ⊗ B"] \ar[d, swap, "θ_{A,b}"]
        & H'(A) ⊗ B \ar[d, "θ'_{A,b}"] \\
      H(A ⊗ B) \ar[r, swap, "h'_{A ⊗ B}"]
        & H'(A ⊗ B)
    \end{tikzcd}
  \]
\end{definition}

\begin{proposition}[Category of signatures with strength]
  Signatures with strength and their morphisms form a category denoted
  $\SigStrength(\mc{C})$.
  Composition and identity in this category are inherited from the category
  of functors and natural transformations.
\end{proposition}

Signatures with strength admit constructions analogous to those on signatures,
allowing us to specify languages modularly using signatures with strength:

\begin{example}
  There is a trivial signature with strength $(\Id,\Id)$ denoted $Θ$.
\end{example}

\begin{example}
  \label{ex:left-comp}
  Given an object $D : \mc{C}$, for any signature with strength
  $(H,θ)$, there is an associated signature with strength, for the
  endofunctor $D ⊗ H(\_) : \mc{C} → \mc{C}$ and the strength:
  %
  \[
    \begin{tikzcd}
      (D ⊗ H(A)) ⊗ B \ar[r, "\alpha"]
        & D ⊗ (H(A) ⊗ B) \ar[r, "D ⊗ θ"]
        & D ⊗ H(A ⊗ B)
    \end{tikzcd}
  \]
\end{example}

\begin{example}
  \label{ex:sigstrength-variable-binding}
  On $[\Set,\Set]$, there is a signature with strength modeling $n$ variable binding
  defined by the functor $(H)(X)(Γ) := X(Γ + n)$ and the strength
  \[ θ^{(n)} : H(X)(Y(Γ)) := X (Y(Γ) + n) \xrightarrow{X(Y(\inl_Γ) + η_{\,Γ+n})} X (Y (Γ + n)) := H(X ∘ Y)(Γ) \]
\end{example}

Moreover, as for signatures, to be modular we require signatures with
strength to be closed under some limits and colimits.
This follows under the same requirement as for signatures:

\begin{proposition}[Closure under (co)-limits]
  \label{prop:sigstrength-closure-colimits}
  The category $\SigStrength(\mc{C})$ inherits its limits and colimits from $\mc{C}$
  provided that for all $B : \mc{C}$, $\_ ⊗ R$ preserves them.
\end{proposition}
\begin{proof}
  Given a diagram $J → \SigStrength(C)$, so in particular a functorial family $(H_i,θ_i)_{i : J}$,
  we define a new signature pointwise using, firstly, that $\_ ⊗ R$ preserves colimits
  and, secondly, the universal properties of colimits:
  $H : A ↦ \colim H_i(A)$, and $θ : \colim H_i(A) ⊗ B ≅ \colim (H_i(A) ⊗ B) → \colim (H_i(A ⊗ B))$.
  Limits are constructed analogously.
\end{proof}

\begin{example}
  Under the above assumptions, $\SigStrength(\mc{C})$ has a terminal signature, also
  denoted $Θ^0$, and is closed under products and coproducts.
\end{example}

\begin{definition}
  Algebraic signatures on $[\Set,\Set]$ are of the form
  $\scalebox{1.5}{+}_{i ∈ I}\;\, Θ^{(n_0)} × ... × Θ^{(n_k)}$.
\end{definition}

\begin{example}
  On $[\Set,\Set]$, the untyped lambda calculus can be specified by the
  signature $Θ × Θ + Θ^{(1)}$, first order logic by $2Θ^0 + Θ + 3Θ^2 + 2Θ^{(1)}$,
  and linear logic by $\mrm{LL} := 2Θ^0 + 2Θ +5Θ^2 + 2Θ^{(1)}$.
\end{example}



\subsection{Signatures With Strength as Signatures}
\label{subsec:sigstrength-to-sig}

We would like to use signatures with strength instead of signatures to specify
languages, in order to have access to an initiality theorem.
However, while doing so, we would like to preserve intuitions and results from signatures.
To do so, we see signatures with strength as particular signatures.

\begin{related Work}
  Signatures with strength \cite{FPT99,Hss04} have been introduced before
  modules over monoids and signatures \cite{HirschowitzMaggesi07,HirschowitzMaggesi12},
  and some authors solely rely on signatures with strength.
  %
  However, modules over monoids are important conceptually, and some notions like
  presentable signatures \cite{PresentableSignatures21} or transition monads
  \cite{TransitionMonads22} do not seem to have clear counterparts in terms of
  strengths yet.
  %
  It is hence paramount to relate both notions to unify the different approaches.
\end{related Work}

%
As identified in \cite{HirschowitzMaggesi12} for the particular monoidal
category $[\Set,\Set]$, every signature with strength yields a signature.
Indeed, signatures with strength can be seen as signatures where the module
substitutions have been particularly restricted:

\begin{proposition}
  \label{prop:sigstrength_to_sig}
  There is a functor $ι : \SigStrength(\mc{C}) → \Sig(\mc{C})$ from signatures with strength to signatures.
  %
  It associates to any signature with strength $(H,θ)$ the signature (i.e., a functor)
  that associates to a monoid $(R,η,μ)$ the $R$-module $H(R)$ with
  the module substitution
  %
  \[
    \begin{tikzcd}
      H(R) ⊗ R \ar[r, "θ_{R,η}"]
        & H(R ⊗ R) \ar[r, "H(μ)"]
        & H(R).
    \end{tikzcd}
  \]
  %
  To a monoid morphism $f : R → R'$, it associates the morphism of modules $H(f) : H(R) → H(R')$.
  %
  Given a morphism of signatures with strength $h ; (H,θ) → (H',θ')$,
  the functor associates the morphism of signatures $(\Id,h) : ι(H,θ) → ι(H',θ')$.
\end{proposition}

\begin{related Work}
  \label{related-work:model-sigma-monoids}
  %
  Unfolding the definition of model (\cref{def:models}) for a signature with
  strength gives exactly a $Σ$-monoid, the notion of model used in
  \cite{FPT99,SecondOrderDep08,ListObjects17}.
  %
  Indeed, a model is a monoid $(R,η,μ)$ with a morphism of module $r : Σ(R) → R$,
  in this case a morphism $r : Σ(R) → R$ in $\mc{C}$ such that the diagram
  %
  \[
    \begin{tikzcd}
      H(R) ⊗ R \ar[r, "θ"] \ar[d, swap, "r ⊗ R"]
        & H(R ⊗ R) \ar[r, "H(μ)"]
        & H(R) \ar[d, "r"]\\
      R ⊗ R \ar[rr, swap, "μ"] & & R.
    \end{tikzcd}
  \]
  %
  commutes. This is exactly the definition of a $Σ$-monoid.
  %
  Consequently, for signatures with strength both approaches yield the same notion of
  model, and we can understand signatures with strength through the intuition of
  modules over monoids.
\end{related Work}

%
%
%
To understand signatures with strength through signatures
and use them instead to specify our languages, we show that the
functor $ι$ maps our basic constructions on signatures with strength, such as the trivial signature or
coproducts, to their counterpart in signatures.

%
%
%
%
%
%
%
%
%
%
%

\begin{proposition}
  The functor $ι : \SigStrength(\mc{C}) → \Sig(\mc{C})$ maps the trivial
  signature with strength $Θ : \SigStrength(\mc{C})$ and the $D ⊗ \_$
  construction to their counterpart.
\end{proposition}

\begin{proposition}[Preservation of colimits]
  \label{prop:sigstrength_sig_colimits-preserved}
  The functor $ι : \SigStrength(\mc{C}) → \Sig(\mc{C})$ preserves colimits
  provided that for all $B : \mc{C}$, $\_ ⊗ B$ preserves them.
\end{proposition}
\begin{proof}
  The image of the colimit is $(\Id + \colim H_i, μ ∘ \colim θ_i)$, and the
  colimit of the images is $(\colim (\Id + H_i), \colim(μ ∘ θ_i))$.
  Both are equal as colimits distribute over each other, that precomposition preserves colimits,
  and by the universal property of colimits.
\end{proof}

\begin{proposition}[Preservation of limits]
  \label{prop:sigstrength_sig_limits-preserved}
  The functor $ι : \SigStrength(\mc{C}) → \Sig(\mc{C})$ preserves limits
  provided that for all $B : \mc{C}$, $\_ ⊗ B$ preserves them, and that they
  distribute over binary coproducts.
\end{proposition}
\begin{proof}
  The proof is similar to that for colimits in \cref{prop:sigstrength_sig_colimits-preserved}.
  The extra assumption is needed as colimits do not always commute with limits
  \cite[Chapter 3.8]{CategoryTheoryInContext14}.
\end{proof}


We do not have a general result that $i$ preserves variable binding since the notion of variable binding depends on the monoidal category by which our framework is parametrized.
Consequently, when applying the framework one must also check that both representations of variable binding coincide.
%
Alternatively, as variable binding is a basic building block, one can also
simply directly define the module representation as the image of $i$.
%
Once checked, using the closure properties, we can prove that algebraic
signatures yield the same models.

\begin{example}
  The functor $ι : \SigStrength(\mc{[\Set,\Set]}) → \Sig(\mc{[\Set,\Set]})$ maps
  unary binding of signatures with strength $Θ^{(n)} : \SigStrength([\Set,\Set])$ to
  its signature counterpart $Θ^{(n)} : \Sig([\Set,\Set])$.
\end{example}

\begin{example}
  The functor $ι : \SigStrength(\mc{[\Set,\Set]}) → \Sig(\mc{[\Set,\Set]})$
  preserves algebraic signatures on $[\Set,\Set]$, like $Θ × Θ + Θ^{(1)}$
  specifying the untyped lambda calculus.
\end{example}


\subsection{The Initiality Theorem}
\label{subsec:initiality_theorem}

As for signatures, not all signatures with strength admit an initial model.
For instance, the signature $\mc{P} \circ Θ$ can be equipped with a
strength by postcomposition $Θ$ by $\mc{P} : \Set → ∖Set$ (\cref{ex:left-comp}).
Yet, it is not representable on $[\Set,\Set]$ by \cref{ex:not-representable}.
%
Thankfully, as signatures with strength $(H,θ)$ decompose into an
endofunctor $H : \mc{C} → \mc{C}$ and a strength $θ$, it is easier to
provide an initiality theorem and an adjoint theorem by requiring conditions
on $\mc{C}$ and $H$.
%
We now state both of these theorems, and prove them in \cref{sec:building_initial_model}

\begin{related Work}
  The following theorems are direct counterpart to results in
  \cite{FPT99,SecondOrderDep08,ListObjects17} up to the following technical details.
  %
  First, compared to \cite{FPT99,SecondOrderDep08}, but akin to \cite{ListObjects17}
  we use $ω$-cocontinuity rather than monoidal closedness.
  %
  Second, we deduce the adjoint theorem from the initiality theorem rather than
  the opposite, saving us the hypothesis that $X ⊗ \_$ is $ω$-cocontinuous in
  the initiality theorem.
  %
  These changes are important for unifying the approaches and are discussed in
  detail in \cref{subsubsec:rw-co-vs-adj}.
\end{related Work}

\begin{restatable}[The Initiality Theorem]{theorem}{initialitytheorem}
  \label{thm:initiality-theorem}
  %
  Let $\mc{C}$ be a monoidal category, with initial object, binary coproducts,
  $ω$-colimits and such that for all $Z : \mc{C}$, $\_ ⊗ Z$
  preserves initiality, binary coproducts and $ω$-colimits.
  %
  Then, given a signature with strength $(H,θ)$, if $H$ is
  $ω$-cocontinuous, then the associated signature has an initial model
  $\ol{H}$, with the initial algebra $μ A.(I + H(A))$ as underlying object.
\end{restatable}

\noindent This theorem is very powerful as it provides a single initiality
theorem applicable to different input monoidal categories, hence handling
different kinds of contexts and different type systems.
Moreover, the conditions are relatively mild: in practice, we will always
work with cocomplete categories, and with nice monoidal products on the left.

Assuming additionally that for all $X : \mc{C}$, $X ⊗ \_$ preserves $ω$-colimits,
the initiality theorem can be extended to an adjoint theorem:

\begin{restatable}[The Adjoint Theorem]{theorem}{adjointtheorem}
  \label{thm:adjoint-theorem}
  %
  Let $\mc{C}$ be a monoidal category, with initial object, binary coproducts,
  $ω$-colimits and such that for all $Z : \mc{C}$, $\_ ⊗ Z$
  preserves initiality, binary coproducts and $ω$-colimits.
  %
  If additionally, for all $X : \mc{C}$, $X ⊗ \_$ preserves $ω$-colimits,
  %
  then, for any $ω$-cocontinuous signature with strength $(H,θ)$,
  the forgetful functor $U : \Model(H) → \mc{C}$ has a left adjoint
  $\mrm{Free} : \mc{C} → \Model(H)$, where $\Free(X)$, for $X : \mc{C}$,
  has the initial algebra $μ A.(I + H(A))$ as underlying object.
\end{restatable}

%
In practice, one works with a fixed monoidal category $\mc{C}$ satisfying the
hypotheses of the initiality theorem, like $[\Set,\Set]$.
In that case, it suffices for a signatures with strength to be $ω$-cocontinuous
to be representable.

\begin{definition}
  $ω$-cocontinuous signatures with strength form a full subcategory $\SigStrength_ω(\mc{C})$
  of the category of signatures with strength.
\end{definition}

\begin{corollary}
  \label{coro:all-pres}
  Under the hypotheses of the initiality theorem, all signatures of
  $\SigStrength_ω(\mc{C})$ are representable, and there is a functor
  associating to each signature its initial model:
  \[ \ol{(\_)} : \SigStrength_ω(\mc{C}) \longrightarrow \mrm{SigModel}{(\mc{C})} \]
\end{corollary}

%
\noindent It is then particularly interesting to study their closure properties
to be able to specify languages and ensures they are representable modularly.
%
Thanks to the closure properties of signatures (\cref{prop:sigstrength-closure-colimits}),
it unfolds to the closure properties of $ω$-cocontinuity (\cref{subsec:omega-colimits})
that differ between limits and colimits.

\begin{proposition}[Closure under colimits]
  \label{prop:sigstrength_omega_cocomplete}
  By \cref{prop:omega-colimits,prop:sigstrength_sig_colimits-preserved}, if
  $\mc{C}$ is cocomplete and for all $B : \mc{C}$, $\_ ⊗ B$ preserves colimits,
  then the category $\SigStrength_ω(\mc{C})$ is cocomplete.
\end{proposition}

\begin{proposition}[Closure under limits]
  \label{prop:sigstrength_omega_complete}
  By \cref{prop:omega-limits,prop:sigstrength_sig_limits-preserved}, if $\mc{C}$
  admits a class of limits that commute with binary coproduct and $ω$-colimits,
  and that is preserved by $\_ ⊗ B$ for all $B : \mc{C}$, then the category
  $\SigStrength_ω(\mc{C})$ is closed under this class of limits.
\end{proposition}

For instance, consider the category $[\Set,\Set]$.
%
As it satisfies the hypotheses of the initiality theorem, any $ω$-cocontinuous signature with strength on $[\Set,\Set]$ is repesentable.
It hence suffices to specify languages as $ω$-cocontinuous signature with strength to have an initial model for them.
This is possible easily and modularly as they are are closed under sums and finite products.

\begin{proposition}
  Signatures with strength on $[\Set,\Set]$ $\SigStrength_ω([\Set,\Set])$
  have a terminal object, are closed under \emph{finite} products, and are
  closed under coproducts by
  \cref{prop:sigstrength_omega_cocomplete,prop:sigstrength_omega_complete,prop:presheaves-limits}.
\end{proposition}

\noindent It can easily be proven that the signature with strength $Θ^{(1)}$ is $ω$-cocontinuous.
From there, using the closure properties by coproducts, finite products and composition,
we can deduce that all algebraic signatures and usual untyped languages are representable.

\begin{example}
  All algebraic signatures (\cref{def:alg-sig,ex:alg-sig}) on $[\Set,\Set]$ are representable.
  In particular, the untyped lambda calculus specified by $Θ × Θ + Θ^{(1)}$, first
  order logic specified by $2Θ^0 + Θ + 3Θ^2 + 2Θ^{(1)}$, and linear logic specified
  by $2Θ^0 + 2Θ +5Θ^2 + 2Θ^{(1)}$ are representable.
\end{example}



%
%
%
%
