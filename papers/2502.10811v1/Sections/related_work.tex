\section{Related Work}
\label{sec:related-work}


%
The literature on initial semantics is quite prolific, and contains many
different approaches to initial semantics.
We focus on three traditions, which we name after the different mathematical structures used in each of them: $Σ$-monoids, modules over monads, and heterogeneous
substitution systems.
%
The links between these different approaches are not, in our view, clearly laid out in the literature.
Furthermore, some of the traditions contain several different variations, which can be
confusing when trying to learn or use results from the literature.
%
To bridge this gap in the literature, we have suitably abstracted and combined
the different approaches in
\cref{sec:models,sec:initiality_theorem,sec:building_initial_model}.
%
This enables us to shed light on the prolific literature: by relating the
different existing approaches to our framework, we also relate them to each other.
%
%
%

Note that in this work, we do not tackle initial semantics for metavariables,
equations, reductions, or formalization in computer proof assistants.
Consequently, in this section on related work, we mention work on these subjects
but do not discuss them in detail.

We first give a brief, and necessarily incomplete, chronological overview of the
different lines of work on initial semantics in \cref{subsec:rw-overview}.
%
We then provide an extensive and detailed discussion tradition by tradition,
where we survey the different variations, the links with our framework, and
justify the design choices of our framework.
%
We discuss work on $Σ$-monoids in \cref{subsec:rw-sigma-mon}, on modules over
monoids in \cref{subsec:rw-modules-over-monoids}, and on heterogeneous
substitution systems in \cref{subsec:rw-hss}.


\subsection{Overview of Initial Algebra Semantics for Languages With Variable Binding}
\label{subsec:rw-overview}

In this section, we provide a chronological overview of the literature on initial semantics.
To enhance readability, the overview is divided in three parts.
These divisions reflect our understanding of the evolution of the literature,
and the involvement of new contributors, e.g. new PhD students.
Note that any such segmentation is inherently arbitrary and does not capture all the nuances.


\subsubsection{Origins: Up until 2007}

%
%
%
The concept of using monads as an abstraction to reason about simultaneous
substitution was first introduced by Bellegarde and Hook \cite{BellegardeHook94}.
It was generalized to using monoids in a monoidal category by Fiore, Plotkin, and Turi \cite{FPT99}.
%
%
To ease the construction of the substitution monad, building on the concept of
``nested'' data types \cite{NestedDataTypes98}, Bird and Paterson
\cite{DeBruijnasNestedDatatype99} defined the untyped lambda calculus
intrinsically, and strengthened the usual fold operation to ``generalized
folds''.
%
%
The authors then generalized the concept of ``generalized fold'' to generic
nested data types in \cite{GeneralisedFold99}.
%
%
As an alternative to generalized folds, Altenkirch and Reus considered instead
structural induction for nested data types in \cite{AltenkirchReus99}.



%
Capturing the substitution structure and recursion of an untyped higher-order
language as an initial object was first achieved in \cite{FPT99} by Fiore,
Plotkin, and Turi, using the category $[\mb{F},\Set]$.
%
To do so, they introduced $Σ$-monoids, and suggested using strength and monoidal
categories to prove initiality results.
%
%
Fiore then investigated simply-typed higher-order languages using those
methods in \cite{Cbn02,MMCCS05}.

%
Matthes and Uustalu introduced heterogeneous substitution systems on endofunctor
categories in \cite{Hss04}, in order to prove that both wellfounded and
non-wellfounded syntax have a monadic substitution structure.
%
In particular, they introduced signatures with strength as a formal notion
of signature for the first time, and proved generic theorems using them.

%
%
The framework based on monoidal categories and $Σ$-monoids was later detailed
by Fiore and Hamana, in \cite{SecondOrderDep08}, were they furthermore
considered the addition of meta-variables, and suggested a method to handle
dependently typed languages.
We do not deal with meta-variables in this work.

%
%
Modules over monads and associated models were introduced for endofunctor
categories by André Hirschowitz and Maggesi in \cite{HirschowitzMaggesi07}, in
order to capture the substitution properties of constructors.
This work was extended in \cite{HirschowitzMaggesi10}, were they proved, using a
proof specific to the base category of sets, that untyped higher-order languages
have an initial model on $[\Set,\Set]$.

\subsubsection{Consolidation: 2007 - 2015}

%
Equations for $Σ$-monoids were considered in Hur's dissertation \cite{HurPhd}
with Fiore, and reported in \cite{FioreHur07,FioreHur09,FioreHur08}.
%
%
Equations and meta-variables have also been considered in \cite{FioreMahmoud10,FioreHur10}.
We do not deal with semantics in this work, neither with denotational semantics in the form of equations, nor with operational semantics in the form of reduction rules.

%
The links between the $Σ$-monoids approach and the module over monads one
were investigated in Zsidó dissertation \cite{ZsidoPhd10}.
In particular, Zsidó fully worked out the simply-typed instance for both traditions.
A variant of Zsidó's construction for modules over monads was formalized with
Ahrens \cite{ISCoq10} in the proof assistant Coq.

%
%
%
A proposition to handle polymorphic languages was investigated by Hamana in
\cite{Polymorphism11}, and considered with meta-variables and equations
by Fiore and Hamana in \cite{PolymorphismEq13}.

%
The current concept of signatures for modules over monads was introduced by
Hirschowitz and Maggesi in \cite{HirschowitzMaggesi12}, and they showed that
every signature with strength induce a signature.

%
Modules over monads for syntax and semantics were investigated further in
Ahrens' dissertation \cite{AhrensPhd}.
He developed a framework to extend the initiality principle of
simply-typed languages to allow for translations across typing systems
\cite{ExtendedInitiality12}.
Ahrens also considered reductions rules using modules over monads and relative monads.
A framework for untyped languages was published in \cite{UntypedRelativeMonads16},
and for simply-typed languages in \cite{TypedRelativeMonads19}.

%
Heterogeneous Substitution Systems have been revisited in \cite{HssRevisited15},
by Ahrens and Matthes, where they extend the hss framework with an initiality
result and a formalisation.

%
$Σ$-monoids have also been revisited by Fiore and Saville in \cite{ListObjects17}.
They extended $Σ$-monoids to $T$-monoids, weakened the assumptions of
previous theorems, and provided more detailed proofs of the theorems.

\subsubsection{Recent Work: 2018 - Present}

Modules over monads and semantics in the form of equations and reductions between terms were investigated in
Ambroise Lafont's dissertation \cite{LafontPhd}, in collaboration
with other people, in particular, Ahrens, André Hirschowitz, Tom Hirschowitz, and Maggesi.
Equations were investigated through signatures in \cite{PresentableSignatures21},
and on their own in \cite{2Signatures19}.
Reductions rules and reduction strategies were investigated in
\cite{ReductionMonads20} and \cite{TransitionMonads20,TransitionMonads22}, respectively.

Heterogeneous substitution systems were applied by Ahrens, Matthes, and
Mörtberg to handle untyped and simply-typed higher-order languages in Coq's
UniMath library in \cite{HssUntypedUniMath19,HssTypedUnimath22}.

The framework using $Σ$-monoids was extended by Borthelle, Lafont, and
Tom Hirschowitz to skew-monoidal categories in \cite{CellularHoweTheorem20},
in order to study bisimilarity.

Another skew-monoidal category was investigated by André and Tom
Hirschowitz, Lafont and Magessi in \cite{NamelessDummies22}, to study
De Bruijn monads.

Fiore and Szamozvancev used the work on $Σ$-monoids,
meta-variables, and equation to design a framework to handle higher-order
languages in Agda \cite{FioreSzamozvancevPopl22}.

Very recently and independently from us, heterogeneous substitution systems and
results about them have been generalized to monoidal categories in the study of
non-wellfounded syntax in \cite{HssNonWellfounded24}.










\subsection{$Σ$-monoids}
\label{subsec:rw-sigma-mon}

%

\subsubsection{Origins}

%
%
%
%
Capturing higher-order languages with their substitution structure as initial
models was first achieved for untyped algebraic signatures \cref{def:alg-sig},
in a seminal work by Fiore, Plotkin and Turi in \cite{FPT99}, using the category
$[\mb{F},\Set]$.
%
They first showed that the pure syntax of untyped higher-order calculi, specified by
algebraic signatures (\cref{def:alg-sig}), is modeled by particular algebras on $[\mb{F},\Set]$.
%
They then introduced $Σ$-monoids, a particular case of model models for signatures
with strength by \cref{related-work:model-sigma-monoids}, to capture substitution.
%
They further stated that every binding signature induces an initial $Σ$-monoids
on $[\mb{F},\Set]$, thus providing a framework for initial semantics.
%
No proofs were given in the extended abstract \cite{FPT99}; but, the authors
suggested it can be proven using that, since $[\mb{F},\Set]$ is a
\emph{closed monoidal category}\footnote{
  A monoidal category is (bi)closed, when for all $Z : \mc{C}$, the functors
  $\_ ⊗ Z$ and $Z ⊗ \_$ have right adjoints.
} every free $Σ$-algebra such that $Σ$ has a pointed strength is parametrically free.

%
%
%
This claim was made more precise by Fiore in another extended abstract
\cite[Sections I.1.1 - I.2.2]{SecondOrderDep08}.
%
After giving an analysis of (pointed) strength in terms of actions over monoidal
categories, Fiore stated that if a monoidal category is closed and has binary
products, then every $(I + Σ + X ⊗ \_)$-initial algebra, such that $Σ$ has a
pointed strength, yields by parametrized initiality the free $Σ$-monoids over $X$.
%
Fiore furthermore suggested that if all of the $(I + Σ + X ⊗ \_)$-initial algebras
exist,\footnote{
  This is the case as soon as the monoidal category additionally has initial
  object and $ω$-colimits, and $Σ$ and $X ⊗ \_$ are $ω$-cocontinuous.
} then they assemble into a left adjoint to the forgetful functor
$U : Σ\trm{-monoid} → C$.
%
Such a theorem implies an initiality theorem since left adjoints preserve initial
objects, hence the free $Σ$-monoid over the initial object is an initial $Σ$-monoid.
%
Every binding signature having an initial model for $[\mb{F},\Set]$ is then
discussed as an instance of this result.
%
%
%

%
%
%
$Σ$-monoids and the theorems above have been extended by Fiore and
Saville in \cite{ListObjects17} to the larger class of $T$-monoids.
%
$T$-monoids are very similar to $Σ$-monoids, except that $T$ is assumed to be a
strong monad, which enables one to account for several universal algebra notions.
%
In this work, we have no particular use for $T$-monoids in full generality,
but \cite{ListObjects17} is still interesting in two regards.
%
First, the work weakened the closedness conditions in the adjoint theorem to an
$ω$-cocontinuity condition.
Second, it provided more detail on how to prove the adjoint theorem
from which one can derive the initiality theorem.

We discuss the initiality theorem and the adjoint theorem further in
\cref{subsubsec:rw-co-vs-adj}.
Furthermore, we explain the difference between,
on the one hand, the proof based on
parametrized initiality and, on the other hand,
the proof presented in \cref{subsec:building_initial_model}
based on hss, in \cref{subsubsec:rw-param-vs-hss,subsubsec:rw-building-param-hss}.

\subsubsection{Cocontinuity vs.\ existence of adjoints}
\label{subsubsec:rw-co-vs-adj}

%
The results of \cite{FPT99,SecondOrderDep08,ListObjects17} are similar to the
one presented in \cref{sec:initiality_theorem}, up to two differences.

%
%
First, as in \cite{ListObjects17} but in contrast to prior work, we do not
require the monoidal product to be closed; instead, we require it to preserve some
colimits, which is a weaker assumption as it is implied by having an adjoint.
%
For the initiality theorem, we require instead precomposition $\_ ⊗ Z$ to
preserve initial objects, binary products and $ω$-colimits.
For the adjoint theorem, we additionally require for postcomposition $Z ⊗ \_$ to
preserves $ω$-colimits.
%
This is an essential step towards relating the different approaches, since, in contrast to
$[F,\Set]$, precomposition does not have a right adjoint on $[\Set,\Set]$.
%
A result based on the existence of adjoints would hence fail to be instantiated to $[\Set,\Set]$ even
though it is a basic instantiation of the literature on modules over monads and
heterogeneous substitution systems.

\begin{proposition}
  Precomposition $\_ ∘ Z$ on $[\Set,\Set]$ does not have a right adjoint.
\end{proposition}
\begin{proof}
%
If precomposition by $Z$ had a right adjoint $R_Z : [\Set,\Set] → [\Set,\Set]$,
then $\Set → \Set$ would be locally small, since using that $\Id = y_1$ and the Yoneda lemma,
we would have: $\mrm{Hom}(Z,G) \cong \mrm{Hom}(\Id ∘ Z,G) \cong \mrm{Hom}(\Id,R_Z(G))
\cong \mrm{Hom}(y_1,R_Z  G) \cong R_Z(G)(1)$.
%
In which case, by a theorem by Freyd and Street \cite{FreydStreet95}, the
category $\Set$ would be essentially small, i.e equivalent to a small category,
which is impossible as the cardinals which are sets do not form a set.
\end{proof}


%
Second, the authors proved the adjoint theorem (c.f.\ \cref{thm:adjoint-theorem})
and deduced the initiality theorem (c.f.\ \cref{thm:initiality-theorem})
from it, using that left adjoints preserve initial objects.
We have done the opposite: we proved the initiality theorem, and deduced the
adjoint theorem from it (c.f.\ \cref{subsec:building-adjoint}).
%
This enables us to prove the initiality theorem using the underlying functor
$I + H\_$ rather than the functor $I + H\_ + X ⊗ \_$, which enables us to
remove the hypothesis that $X ⊗ \_$ is $ω$-cocontinuous in the
hypothesis of the initiality theorem.
%
Even though the results presented here are only about monoidal categories,
the removal of this hypothesis is important for
dealing with some skew-monoidal categories as discussed in
\cref{subsubsec:rw-skew}.


\subsubsection{Parametrized initiality vs.\ heterogeneous substitution systems}
\label{subsubsec:rw-param-vs-hss}

%
%
Since the use of parametrized initiality for $Σ$-monoids is explained in little
detail in \cite{FPT99,SecondOrderDep08,ListObjects17}, the differences of
vernacular and presentation can lead one to believe that the proof based on
parametrized initiality is fundamentally different from the one based on hss
described in \cref{sec:building_initial_model}.

However, hss and parametrized initiality are actually very similar, and the
proofs are strongly related; even though to the best of our knowledge this has
never been reported before.
%
To understand the differences, let us consider the proof of the initiality
theorem, starting from parametrized initiality.

\begin{definition}[Parametrized Initiality]
  Let $F : \mc{C} × \mc{C} → \mc{C}$ be a bifunctor with pointed
  strength $\mrm{st}$, and $U : \mc{C}$.
  An $F(U,\_)$-algebra $(R,r)$ is \emph{parametrized initial} if for
  any pointed object $Z : \mc{C}$ and $F(U ⊗ Z,\_)$-algebra $(C,c)$,
  there is a unique morphism $h : R ⊗ Z → C$ such that the following diagram commutes.
  %
  \[
    \begin{tikzcd}[column sep=large]
      F(U, R) ⊗ Z \ar[r, "\mrm{st}"] \ar[d, swap, "r ⊗ Z"]
        & F(U ⊗ Z, R ⊗ Z) \ar[r, dashed, "F(U ⊗ Z{{,}} h)"]
        & F(U ⊗ Z, C) \ar[d, "c"] \\
      R ⊗ Z \ar[rr, swap, dashed, "h"]
        &
        & C
    \end{tikzcd}
  \]
\end{definition}

\noindent To prove the initiality theorem, we are interested in parametrized
initiality for $U := I$, and for the bifunctors of the form $F : (U,A)
\longmapsto U + H(A)$ such that $H$ is a functor with pointed strength $Θ$.
%
In this case, assuming that $\_ ⊗ Z$ distributes over binary products,
parametrized initiality unfolds to \hyperref[def:hss]{hss}, except that for
parametrized initiality, the output of $h$ can be any $H$-algebra $C$ with a map
$f : Z → C$, whereas it is fixed to be $R$ for hss.
Therefore, following the proof of \cref{subsec:hss_models}, both
parametrized initiality and hss induce models.
%
\[
  \begin{tikzcd}[column sep=large]
    I ⊗ Z \ar[r, "η ⊗ Z"] \ar[dd, swap, "\lambda_Z"]
      & R ⊗ Z \ar[dd, dashed, "h"]
      & H(R) ⊗ Z \ar[l, swap, "r ⊗ Z"]
                       \ar[d, "Θ_{R,e}"] \\
      &
      & H(R ⊗ Z) \ar[d, dashed, "H(h)"] \\
    Z \ar[r , swap, "f"]
      & C
      & H(C) \ar[l, "r"]
  \end{tikzcd}
\]

%
The main difference lies in the initiality part of the proof, more specifically
in the proof that the initial algebra morphism respects the monoid multiplication.
%
For parametrized initiality, this can be directly proven by appropriately
instantiating $Z$ and $C$.
%
However, as $C$ is fixed to be $T$ in the definition of hss, it forces us
to use a fusion law as done in \cref{subsec:building_initial_model}.


%
This makes no difference for wellfounded syntax, but it is fundamental when
it comes to non-wellfounded syntax.
%
Indeed, as parametrized initiality automatically yields an initial model, it can
not be used to prove that non-wellfounded languages have a monadic substitution
structure; this is in contrast to hss which were designed to handle both
wellfounded and non-wellfounded syntax \cite{Hss04,HssNonWellfounded24}.
%
Consequently, to better relate all the different approaches, we have decided to
base our work on hss in \cref{sec:building_initial_model}.



\subsubsection{Building parametrized initial algebras and heterogenous substitution systems}
\label{subsubsec:rw-building-param-hss}

%
It remains to understand how hss and parametrized initiality are built out of initial algebras.
%
As explained in \cref{subsec:building_hss} generalizing \cite{HssUntypedUniMath19},
hss can be built out of initial algebras using generalized Mendler's style
iteration (\cref{thm:gen-mendler}), by instantiating them with $F := I + H\_$,
$L := \_ ∘ Z$, $X := R$ and an appropriate $Ψ$.
%
%
%

%
Rather than using Mendler's style iterations, Fiore and Saville defined instead
a ``lax-uniformity property of initial algebra functors'' in \cite[Theorem 4.7]{ListObjects17}.
%
We depart from their presentation, as we noticed that the hypothesis and
conclusions could actually be separated in a theorem and a corollary.

\begin{theorem}
  \label{thm:FioreSaville}
  Let $\mc{A,B,C,D}$ be categories such that $\mc{C}$ has an initial object and $ω$-colimits.
  %
  Let $F,G,K,J$ be functors and $t : J ∘ F → G ∘ (K × J)$ a natural
  transformation as below left, such that:
  %
  \begin{itemize}[label=$-$]
    \setlength\itemsep{-1pt}
    \item for all $D : \mc{D}$, $F(D,\_)$ is $ω$-cocontinuous with initial algebra $(μF(D),r_{μF})$
    \item $J$ preserves initiality and $ω$-colimits
  \end{itemize}
  %
  Then, for any $G(K(D),\_)$-algebra $(C,c)$ there exists a unique morphism
  $h : J(μF(D)) → C$ such that the diagram below right commutes:
  %
  \begin{align*}
    \begin{tikzcd}[ampersand replacement=\&]
      \mc{D} × \mc{C} \ar[r, "F"] \ar[d, swap, "K × J"]
        \& \mc{C} \ar[d, "J"] \ar[dl, Rightarrow, shorten=13pt, swap, "t"] \\
      \mc{B} × \mc{A} \ar[r, swap, "G"]
        \& \mc{A}
    \end{tikzcd}
    &&
    \begin{tikzcd}[ampersand replacement=\&, column sep=large]
      J(F(D,μF(D))) \ar[r, "t"] \ar[d, swap, "J(r_D)"]
        \& G(K(D),J(μF(D))) \ar[r, dashed, "G(K(D){,} h)"]
        \& G(K(D),C) \ar[d, "c"]\\
      J(μF(D)) \ar[rr, swap, dashed, "h"]
        \&
        \& C
    \end{tikzcd}
  \end{align*}
\end{theorem}

\begin{corollary}
  Suppose additionally that $\mc{A}$ has an initial object and $ω$-colimits, and
  that for all $B : \mc{B}$, $G(B,\_)$ is $ω$-cocontinuous with initial algebra $(μG(B),r_{μG})$.
  %
  Then, $μF$ and $μG$ assemble into functors $μF : \mc{D} → \mc{C}$ and $μG : \mc{B} → \mc{A}$,
  and there is a natural transformation $h : J ∘ μF → μG ∘ K$ defined pointwise as the unique morphism
  associated to the initial algebra $μG_{K(D)}$:
  %
  \begin{align*}
    \begin{tikzcd}[ampersand replacement=\&]
      \mc{D} \ar[r, "R"] \ar[d, swap, "K"]
        \& \mc{C} \ar[d, "J"] \ar[dl, Rightarrow, shorten=13pt, swap, "h"] \\
      \mc{B} \ar[r, swap, "B"]
        \& \mc{A}
    \end{tikzcd}
    &&
    \begin{tikzcd}[ampersand replacement=\&, column sep=large]
      J(F(D,μF(D))) \ar[r, "t"] \ar[d, swap, "J(r_{μF})"]
        \& G(K(D),J(μF(D))) \ar[r, dashed, "G(K(D){,} h)"]
        \& G(K(D),μG_{K(D)}) \ar[d, "r_{μG}"]\\
      J(μF(D)) \ar[rr, swap, dashed, "h_D"]
        \&
        \& μG_{K(D)}
    \end{tikzcd}
  \end{align*}
  %
\end{corollary}

%
Doing so is actually fundamental to relate the approaches.
%
Indeed, as noticed in \cite{CoqPl2023MonCatHss}, the extra assumptions of the
corollary $\mc{A}$ and $G$ are not needed to prove the initiality theorem.
Now that these hypotheses are separated in a corollary, we can go even further.
%
Though it might be surprising at first sight due to differences of presentation,
Mendler's style iterations and \cref{thm:FioreSaville} are actually direct
instantiations of each other, and hence simply different formulations of the
same theorem.
%
Consequently, the differences between the two proofs fully boil down to the
differences between using hss or parametrized initiality as discussed in
\cref{subsubsec:rw-param-vs-hss}.

\begin{theorem}
  \cref{thm:FioreSaville} is an equivalent formulation of Mendler's style iteration (\cref{thm:gen-mendler}).
\end{theorem}
\begin{proof}
  %
  As remarked in \cite{CoqPl2023MonCatHss}, \cref{thm:FioreSaville} can be deduced
  from Mendler's style iteration by setting $F := F(D,\_)$, $L := J$, $X := C$ and
  $Ψ\;h \mapsto c ∘ G(KD,C) ∘ t$.

  %
  Conversely, as noticed by Lafont, Mendler's style iteration can be deduced
  from the \cref{thm:FioreSaville} by setting $\mc{B,D} := 1$, $\mc{C} :=
  \mc{C}$, $\mc{A} := \Set^\op$, and $F := F$, $L := \mc{D}(L(\_),X)$.
  %
  This enables us to set $t := Ψ$ by seeing $Ψ$ as a natural transformation of
  type $C → \Set^\op$ rather than of type $C^\op → \Set$.
  %
  Then, applying \cref{thm:FioreSaville} to the trivial algebra $1 : \Set^\op$
  gives us a diagram in $\Set^\op$, or in $\Set$ as right-below, which provides
  us with Mendler's style iteration when evaluated in $\star : 1$.
  %
  \begin{align*}
    \begin{tikzcd}[ampersand replacement=\&]
      \mc{C} \ar[r, "F"] \ar[d, swap, "\mc{D}(L(\_){,}X)"]
        \& \mc{C} \ar[d, "\mc{D}(L(\_){,}X)"] \ar[dl, Rightarrow, shorten=13pt, swap, "Ψ"] \\
      \Set^\op \ar[r, swap, "\Id"]
        \& \Set^\op
    \end{tikzcd}
    &&
    \begin{tikzcd}[ampersand replacement=\&]
      \mc{D}(L(F(R)),X)
        \& \mc{D}(L(R),X)  \ar[l, swap, "Ψ"]
        \& 1 \ar[l, swap, dashed, "h"] \\
      \mc{D}(L(R),X) \ar[u, "L(r)^*"]
        \&
        \& 1 \ar[u] \ar[ll, dashed, "h"]
    \end{tikzcd}
  \end{align*}
\end{proof}



%
%
%
%
%
%
%
%
%
%
%
%
%
%
%
%
%
%

%
%
%
%
%
%
%

\subsubsection{Applications}

%
As the framework based on $Σ$-monoids is defined, from the outset, for monoidal
categories, it can be, and has been, applied to more involved instances than
$[\mb{F},\Set]$ and untyped languages.
%
However, note that as until \cite{ListObjects17} this framework relied on
closedness, most applications have been restricted to closed monoidal categories.
This excludes instances like $[\Set,\Set]$, that were hence replaced by
categories like $[\mb{F},\Set]$ that are closed.

%
%
%
Simply-typed languages where first investigated in \cite[Section II.1.1]{Cbn02},
where Fiore explained that the simply-typed lambda calculus is an initial algebra
on $[\mb{F} ↓ T,\Set]^T$, and suggested that substitution could be accounted for
using the framework of \cite{FPT99}.
%
%
%
This was briefly detailed in \cite[Section 1.3]{MMCCS05}, where Fiore
additionally discussed the monoidal structure on $[\mb{F} ↓ T,\Set]^T$.
%
%
%
It was worked out fully and in great detail by Zsidó in her dissertation
\cite[Chapter 5]{ZsidoPhd10}.
%
She fully proved, without using any high-level theorem, that simply-typed
algebraic signatures induce strong endofunctors on $[\mb{F} ↓T,\Set]^T$, that
the category $[\mb{F} ↓T,\Set]^T$ is a left-closed monoidal category, and that
these signatures induce initial $Σ$-monoids in that category.

%
%
%
Hamana investigated polymorphic languages in \cite{Polymorphism11}.
%
For System $F$ and System $F_ω$, Hamana built categories $∫ G$ and $∫ H$ such that
the System $F$ and System $F_ω$ are initial algebras on $\Set^{∫G}$ and $\Set^{∫H}$, respectively.
%
He then introduced polymorphic and higher-order polymorphic counterparts to
algebraic signatures, and generalized the constructions to them.

%
%
%
Fiore also suggested an initial framework for dependently typed languages in
\cite[Section II]{SecondOrderDep08}.


\subsubsection{Extension to skew-monoidal categories}
\label{subsubsec:rw-skew}

%
%
%
Not all categories relevant to initial algebra semantics are monoidal.
A category to interest is $ℕ$-indexed families of sets \cite{CellularHoweTheorem20}, i.e. functors $[\N,\Set]$,
with the tensor product $A(n) ⊗ B(n) := ∑_{m : ℕ} A(m) × B(n)^m$.
Yet, this tensor is not monoidal but only skew-monoidal as the associator and
the unit are not isormorphisms.
%
Therefore, to apply Fiore's framework of $Σ$-monoids on $[\N,\Set]$
Borthelle, Tom Hirschowitz\footnote{
  Be aware that both André Hirschowitz and Tom Hirschowitz worked on initial
  semantics, sometimes in joint work. To distinguish them, we refer to them by
  their full name.
} and Lafont generalized it to skew-monoidal categories \cite{CellularHoweTheorem20},
and formalized the results in Coq.

%
Modules over monoids seem straigthfoward to generalize skew-monoidal categories;
however, as of yet, the proofs based on hss and parametrized initiality do not
seem to generalize.
%
Indeed, we can no longer prove the monoid's associativity law that is an
equality of morphisms of type $(R ⊗ R) ⊗ R → R$, by instantiating hss
and parametrized initiality for $Z := R ⊗ R$.
%
Basically, the reason is that this instantiation provides uniqueness of a
morphism of type $R ⊗ (R ⊗ R) → R$, but as the associator $α$ is not invertible
in skew-monoidal categories, this is no longer equivalent to uniqueness of a
morphism of type $(R ⊗ R) ⊗ R → R$.
%
Thus, we could have generalized the framework presented here by adapting the
proof of \cite{CellularHoweTheorem20} that repeatedly applies \cref{thm:FioreSaville},
but that would have been at the cost of not unifying hss with the other frameworks.
%
As our goal is to unify the different approaches, we have chosen not to do so, and
leave open the challenge to generalize hss and this framework to skew-monoidal categories.


%
%
%
Skew-monoidal categories were also considered in \cite{NamelessDummies22} by
André and Tom Hirschowitz, Lafont, and Maggesi.
%
Therein, they study De Bruijn monads and De Bruijn S-algebras, for untyped and
simply-typed languages, and identify them as monoids and $Σ$-monoids in the
skew-monoidal category $[1,\Set]$ for the relative functor $J : * ↦ ℕ$,
and $[1,\Set^T]$ for the the relative functor $J : * ↦ t ↦ \N$.

%
Unfortunately, the initiality theorem proven in \cite{CellularHoweTheorem20}
for skew-monoidal categories fails to apply to their instances, as it ``requires
that the tensor product is finitary in the second argument'' which is not the
case for their instances.
%
Consequently, they had to reprove an initiality theorem from scratch specifically for their case.
%
This issue actually arose because they proved an adjoint theorem, and
deduced an initiality theorem from it, rather than doing the opposite.
As discussed in \cref{subsubsec:rw-co-vs-adj}, if they had done the opposite,
they could have removed the additional hypothesis, and reused the theorem.


\subsubsection{Further work on $Σ$-monoids}

%
Though this is not the subject of this article, works on $Σ$-monoids
have been developed further to account for meta-variables and equations.
%
An account of meta-variables for $Σ$-monoids was first developed by Hamana in
\cite{HamanaMetavar04}, and by Fiore in \cite[Section I.2]{SecondOrderDep08}.
%
Equational systems were then considered in Hur's dissertation \cite{HurPhd}
supervised by Fiore, which was reported in \cite{FioreHur07,FioreHur09,FioreHur08},
and partially extended in Hur's dissertation.
%
Equations and second-order languages were also considered in
\cite{FioreHur10,FioreMahmoud10}, and for polymorphic languages in
\cite{PolymorphismEq13}.
%
This work was recently used to design an Agda framework for reasoning with
higher-order languages and substitution in \cite{FioreSzamozvancevPopl22}.











\subsection{Modules over Monads}
\label{subsec:rw-modules-over-monoids}



\subsubsection{Origins}

%
%
%
Modules over monads were studied by André Hirschowitz and Maggesi in
\cite{HirschowitzMaggesi07}, in order to capture the substitution structure of
higher-order languages and their constructors.

\begin{definition}[Module over a monad]
  Given a monad $(R,η,μ) : \Mon([\mc{C},\mc{C}])$ on $\mc{C}$,
  a module over $R$ with codomain $\mc{D}$ is a tuple $(M,p^M)$ consisting
  of a functor $M : \mc{C} → \mc{D}$ and a natural transformation $p : M ∘ R → M$
  compatible with $η$ and $μ$ as in \cref{def:modules}.
\end{definition}
%
The authors consider monads on $\Set$ and modules with codomain $\Set$
to define models of a given signature as pairs of a monad $R$ together with a
family of suitable module morphisms over $R$,
and stated that every untyped algebraic signature has an initial model.
%
They also considered how modules over monads could potentially encompass more
notions, such as simply-typed syntax, or some form of semantic properties.
%
%
%
This work was refined in \cite{HirschowitzMaggesi10}, where it has been
enriched with a proof, based on syntax trees, that is specific to
$[\Set,\Set]$ and to untyped algebraic signatures.

%
%
%
Zsidó, in her Ph.D.~\cite[Chapter 6]{ZsidoPhd10}, extended this approach
to simply-typed languages on $[\Set^T,\Set^T]$, studying modules with codomain $\Set$.
First, she showed that modules over monads do encompass simply-typed
algebraic signatures and extended the notion of model to support them.
Second, she extended the proof based on syntax trees to prove that every
simply-typed algebraic signature has an initial model.
%
%
%
This result was formalized in the proof assistant Coq by Ahrens and Zsidó \cite{ISCoq10}
using Coq's built-in inductive types to represent languages.



\subsubsection{Modules over Monads vs Modules over Monoids}
\label{subsubsec:module_monads_vs_monoids}

%
Compared to the original work on the subject \cite{HirschowitzMaggesi10,ZsidoPhd10},
our framework does not rely on modules over monads but on modules over monoids
(\cref{def:modules}) in a monoidal category.
%
This enables us to account for categories like $[\mb{F},\Set]$, whose objects are not endofunctors,
whereas a framework modules over monads are limited to endofunctor categories.
%
Indeed, for a monoidal category as $\_ ⊗ \_ : \mc{C} × \mc{C} → \mc{C}$, $M ⊗ R$
is only well-defined for $M : \mc{C}$, and thus $M$ and $R$ can not be of
different types.
Moreover, even for functor categories $[\mc{B},\mc{C}]$, it is not definable as
$M ∘ R : \mc{B} → \mc{D}$ and $M : \mc{C} → \mc{D}$ are of different types, so
there can not be a natural transformation from $M ∘ R$ to $M$.

%
In the case of endofunctor categories $[\mc{C},\mc{C}]$, modules over monads
are more general than modules over monoids, as modules over monoids are
exactly modules over monads with codomain $\mc{C}$.
The added power of choosing $\mc{D}$ was thought to handle the simply-typed
case, and indeed, in \cite[Chapter 6]{ZsidoPhd10} modules over monads on
$[\Set^T,\Set^T]$ with codomain $\Set$ are used to model simply-typed algebraic
signatures.
The idea here is that the output set of modules can be used to represent the
\emph{terms of a specific type} of the inputs and output of the constructors.
%
%
Nevertheless, in practice modules over monads do not seem to add any
expressive power.
%
In the untyped case, for the category $[\Set,\Set]$
\cite{HirschowitzMaggesi07,HirschowitzMaggesi12,PresentableSignatures21,ReductionMonads20},
$\mc{D}$ is always chosen to be $\Set$, in which case both notions
coincides.
%
In the simply-typed case, as will be explained in \cref{subsubsec:rw-gen-strength},
modules over monoids are enough to represent the languages that have been considered, and the added
flexibility of modules over monads is not necessary.
%
Moreover, as explained in \cite[Section 2.4]{TransitionMonads22}, for any
type $t : T$ there is an adjunction on the functor categories that yields
an adjunction between modules over monads and modules over monoids:
%
\begin{align*}
  \begin{tikzcd}[ampersand replacement = \&, column sep=large]
    [\Set^T{,}\Set]            \ar[r, bend left, "y(t) ∘ \_"]
                               \ar[r, phantom, "\perp"]
      \& {[}\Set^T{,}\Set^T{]} \ar[l, bend left, "(\_)_t ∘ \_"]
  \end{tikzcd}
  &&
  \begin{tikzcd}[ampersand replacement = \&, column sep=large]
    \Mod(\Set^T{,}\Set)
        \ar[r, bend left]
        \ar[r, phantom, "\perp"]
      \& \Mod(\Set^T{,}\Set^T) \ar[l, bend left]
  \end{tikzcd}
\end{align*}
%
Here $y(t) : T → \Set^T$ is the Yoneda embedding, and $(\_)_t : \Set^T → T$ is
the projection.
%
This adjunction implies that modelling an algebraic constructor using modules
over monads with codomain $\Set$ as done in \cite[Chapter 6]{ZsidoPhd10} is the
same as using modules over monoids.
Indeed, the monoid underlying the model is the same in both representations,
and the choice of module morphism corresponds under the adjunction.
%
Therefore, in practice, for simply-typed languages there is actually no difference
in terms of models between using modules over monads or modules over monoids.

%
However, using modules over monoids makes a real difference when it comes to
building an initial model.
In the case of modules over monoids, using that all the outputs of the
constructors can be set to the trivial signature $Θ$, we summed the
inputs into one single signature $Σ$, such that we can represent our
constructors as a single morphism of modules $Σ → Θ$.

This is required to apply the proof of \cref{sec:building_initial_model} based
on hss, as it requires an initial algebra for the functor $\ul{I} + H$.
It is not possible for modules over monads, as already for algebraic signatures
the modules describing the outputs $[Θ]_t$ depend on the output types $t$, and
hence are all different.
Therefore, it is not possible to directly apply the proof of
\cref{sec:building_initial_model}.
%
This is important as the proof using syntax trees is less modular and
significantly longer than the one using hss, and is specific to $[\Set^T,\Set^T]$
whereas the proof using hss applies to any appropriate monoidal category.

Last but not least, modules over monads have the advantage of directly connecting
to the work of Fiore and hss as described in \cref{related-work:model-sigma-monoids}.
%
For all theses reasons, we prefer modules over monoids to modules over monads.



\subsubsection{Earlier attempt at comparing traditions: Zsidó's Ph.D.\ thesis}

%
%
%
Though little-known, one of the most important work for our purpose of
relating the different traditions is Zsidó's dissertation \cite{ZsidoPhd10}.
%
There, Zsidó relates Fiore's work \cite{FPT99,Cbn02} to Hirschowitz and
Maggesi's work \cite{HirschowitzMaggesi07,HirschowitzMaggesi10} on initial
semantics; these use $[\mb{F},\Set]$ and $[\Set,\Set]$, respectively, in the
untyped case, and $[\mb{F}↓T]^T$ and $[\Set^T,\Set^T]$, respectively, in the simply-typed case.

%
In the untyped case, where modules over monads coincide with modules over
monoids, Zsidó generalises modules over monoids and models from $[\Set,\Set]$
to monoidal categories, much like in \cref{subsec:modules,subsec:models}, in
order to express the work on $[\mb{F},\Set]$ and $[\Set,\Set]$ in the same
language.
She then proves, by appropriately propagating the adjunction between
$[\mb{F},\Set]$ and $[\Set,\Set]$ throughout the framework, that for any
algebraic signature, there is an initial model on $[\mb{F},\Set]$ if and
only if there is one for $[\Set,\Set]$, and that they can be constructed
from one another.

%
In the simply-type case, she starts by reviewing the work of Fiore
\cite{Cbn02} in \cite[Chapter 5]{ZsidoPhd10}, and by completing
the work of André Hirschowitz and Maggesi sketched in
\cite[Section 6.3]{HirschowitzMaggesi07} in \cite[Chapter 6]{ZsidoPhd10}.
However, in her work on simply-typed syntax, Zsidó uses modules over monads
with codomain $\Set$, preventing her from defining both frameworks in the same
language, and fully relating the two approaches.
Consequently, in \cite[Chapter 7]{ZsidoPhd10}, she only provides
an adjunction between $Σ$-monoids and models.
Furthermore, the adjunction on models no longer comes from parametricity in
the input monoidal category but is now lifted in an ad-hoc way.



\subsubsection{Signatures}
\label{subsubsec:rw-modules-sig}

%
In all previous work we have discussed on modules over monads, all the signatures
considered were purely syntactic, either untyped algebraic signatures
or simply typed algebraic signatures.
%
The general notion of signatures (\cref{def:sig}) was introduced by André
Hirschowitz and Maggesi in \cite{HirschowitzMaggesi12} on $[\Set,\Set]$.
%
The authors further provided their signatures with the modularity property that
pushout of signatures lift to initial models (\cref{prop:modularity-models}), and
established that signatures with strength induce signatures similarly to
\cref{prop:sigstrength_to_sig}.

%
%
%
This work was expanded on and formalized in Coq in \cite{PresentableSignatures21},
still on $[\Set,\Set]$, by Ahrens, André Hirschowitz, Lafont and Maggesi.
%
Moreover, they introduced \emph{presentable} signatures: a signature $Σ$ is
presentable iff there exists an algebraic signature $Υ$ with an epimorphism $Υ →
Σ$. The authors prove that presentable signatures are representable.
%
Compared to algebraic signatures, presentable signatures
allow us to model constructors with \emph{semantic} properties, such as a binary
\emph{commutative} operator.
In particular, they enable us to do this without the need for any further construction in the framework.
In contrast, 2-signatures \cite{2Signatures19} introduce an explicit notion of equation between terms.
%
In practice, examples of presentable signatures are covered by signatures
with strength as the latter are stable under some colimits (\cref{prop:sigstrength_omega_cocomplete}),
but a more precise link is still to be established.

%
In the two previously cited papers \cite{PresentableSignatures21,2Signatures19}, the authors worked on $[\Set,\Set]$ where
modules over monoids and modules over monads coincide, and where constructors
can be represented by morphisms $Σ → Θ$.
%
Since this is not possible for modules over monads, many authors
\cite{ZsidoPhd10,ISCoq10,ExtendedInitiality12,UntypedRelativeMonads16,TypedRelativeMonads19}
only considered syntactic signatures, and defined a model to be not a monoid
with a single module morphism, but a monoid with a family of module morphisms.
%
As we explain in \cref{subsubsec:rw-gen-strength}, this is not necessary for our
purpose, and not doing so enables us to consider more general notions of signatures.

%
More generally, it would also be possible to require a second signature to
specify the output rather than using $Θ$.
However, as we have no use for non trivial outputs, and as it simplifies the
framework, we limit ourselves to $Θ$ for the output signature.



\subsubsection{A generalized recursion principle for simply-typed languages}
\label{subsubsec:rw-gen-rec}

%
As explained in \cref{subsubsec:rep-sig-rec}, we care about \emph{representable} signatures,
i.e., signatures that admit an initial model, as this provides languages with
a recursion principle.
%
In the untyped case this works well as exemplified in \cref{ex:FOL-to-LL}, where we use
the recursion principle to build a substitution-safe translation from first-order logic to linear logic.
%
However, for it to work we implictly use that we can build a model of first-order
logic out of the model of linear logic, as they have both have a model in
the \emph{same} monoidal category $[\Set,\Set]$.
%
%
In the simply-typed case, doing so is very limited, since the type system $T$ is
hardcoded in the monoidal category, as in $[\Set^T,\Set^T]$.
%
Consequently, two different languages with different type systems $T$ and $T'$
have models in different monoidal categories $[\Set^T,\Set^T]$ and $[\Set^{T'},\Set^{T'}]$,
and hence can not be directly related by initiality as in the untyped case.
%
In a sense, this is not very surprising as such a translation should first rely
on a translation $T → T'$ of the type system which is not part of our framework.

%
%
%
To equip simply-typed languages with better recursion principles, in a
technical paper \cite{ExtendedInitiality12}, Ahrens reworked the entire
framework to internalize the typing system $T$ in the framework.
%
He replaced monads by ``$T$-monads'', adapted modules, signatures, and models,
before proving an initiality theorem.
He then formalized this framework in Coq and used it to provide a verified
translation of PCF into the untyped lambda calculus.
%
While this framework is interesting, it is very specific to endofunctor
categories $[\Set^T,\Set^T]$ and it is unclear how it would scale to more
involved languages and how it relates to the framework presented here.



\subsubsection{Further work on initial semantics}

%
Though this is not the subject of this article, modules over monads have
been used further to add semantics in the form of reduction rules between terms on top of the abstract-syntax framework
that is described in this work.
%
%
%
%
Modules over relative monads and a notion of 2-signature were used in
\cite{UntypedRelativeMonads16,TypedRelativeMonads19}, to model context-passing reduction
rules for untyped and simply-typed algebraic signatures.
%
%
%
%
Similar notions were used in \cite{2Signatures19} to add equations in the
untyped case.
%
%
%
The original work on reduction rules by Ahrens was extended in the untyped
case to handle conditional rules as top-level $\beta$-reduction
\cite{ReductionMonads20},
%
%
%
and later extended to account for simply-typed languages and reduction strategies, e.g.,
call-by-value reduction in \cite{TransitionMonads20,TransitionMonads22}.












\subsection{Heterogeneous Substitution Systems}
\label{subsec:rw-hss}

\subsubsection{Origins}

%
%

%
%
Heterogeneous substitution systems (\cref{def:hss}) were introduced on
endofunctor categories $[\mc{C},\mc{C}]$ by Matthes and Uustalu in \cite{Hss04}.
%
Studying pointed strong functors as signatures, they design hss as an intermediate
abstraction to prove that both wellfounded or non-wellfounded higher-order
languages have a well-behaved substitution structure, that is, form monads
\cite[Theorem 10]{Hss04}.
%
In the well-founded case, they prove that assuming that precomposition $\_ ∘ Z$ has a
right-adjoint for all $Z$, then any signature with strength that has an initial
algebra has an associated hss, and as such a monad structure \cite[Theorem 15]{Hss04}.
%
In the non-wellfounded case, they have proved then any signature with
strength that has a cofinal algebra has an associated hss, and as such a
monad structure \cite[Theorem 17]{Hss04}.

%
%
%
Still on $[\mc{C},\mc{C}]$, this work was strengthened into a framework for
initial semantics by Ahrens and Matthes in \cite{HssRevisited15}.
%
They assemble signatures with strength and hss in categories, prove a fusion law
(\cref{thm:fusion-law}) for generalized Mendler's style iteration \cite[Lemma10]{HssRevisited15},
and use it to prove that the hss built in \cite{Hss04} is actually initial as an hss.
%
This yields a framework for initial semantics based on hss, which they
additionally formalized Coq using the UniMath library \cite{UniMath}.
%
Compared to them, we have chosen to keep hss as an intermediate abstraction
in the proof rather than making it our notion of model.
%
First, it enables us to have a common notion of model connecting the work
of Fiore, modules over monads, and hss.
%
Second, hss provide less explicit information on substitution than
monoids, and, importantly, have a stronger recursion
principle that is applicable in fewer situations, and that we have found no specific use for.

%
%
%
Hss were applied in \cite{HssUntypedUniMath19} by Ahrens, Matthes, and Mörtberg
to build untyped higher-order languages from elementary type constructors in the
UniMath library, and to prove that they have a monadic substitution structure.
%
The proof of \cite[Theorem 10]{Hss04} is based on left-closedness of the monoidal structure and the
associated version of Mendler's style iterations \cite[Theorem 2]{GeneralisedFold99}.
%
Yet, as discussed in \cref{subsubsec:rw-co-vs-adj}, $[\Set,\Set]$ is not left-closed.
Consequently, to build an hss structure, and hence a monad structure, the
authors turned to $ω$-cocontinuity and the associated version of Mendler's style
iterations \cite[Theorem 1]{GeneralisedFold99}.
However, they did not prove that the built hss is initial.

%
The proof presented in \cref{sec:building_initial_model} is a generalization of
the proof of \cite{HssUntypedUniMath19} to monoidal categories for the construction
of the (initial) hss and hence of the monad structure.
%
To prove that the resulting model is initial, we generalize the proofs of \cite{Hss04,HssRevisited15}
to monoidal categories and $ω$-cocontinuity, and adapt them from hss to models.
%
Doing so, we provide more detailed proofs then currently available in the litterature.




\subsubsection{Generalized strength}
\label{subsubsec:rw-gen-strength}


%
%
%
Hss were also applied on $[\Set^T,\Set^T]$ by Ahrens, Matthes and Mörtberg
in \cite{HssTypedUnimath22} to build simply-typed higher-order languages in the
UniMath library \cite{UniMath}, akin to \cite{HssUntypedUniMath19}.
%
To be able to build simply-typed signatures modularly, they introduced the
concept of \emph{generalized strength} \cite[Definition 2.15]{HssNonWellfounded24}:
%
\begin{definition}[Generalized Strength]
  A \emph{generalized strength} for a functor $H : [\mc{C},\mc{D}'] → [\mc{C},\mc{D}]$
  is a natural transformation $Θ$ such that for all $A : [\mc{C},\mc{D}']$,
  and pointed object $b : \Id → B$ with $B : [\mc{C},\mc{C}]$,
  $Θ_{A,b}$ has the type
  \[ Θ_{A,b} : H(A) ∘ B \longrightarrow H(A ∘ B) \]
  and satisfies associativity and unit laws as in \cref{def:sig-strength}.
\end{definition}
%
Similarly to signatures with strengths (\cref{subsec:sigstrength-to-sig}),
generalized signatures with strength correspond correspond to modules over
monads when $\mc{D}' := \mc{C}$.
%
\begin{proposition}
  Given an endofunctor $H : [\mc{C},\mc{C}] → [\mc{C},\mc{D}]$ with generalized
  strength $Θ$, for any monad $(R,η,μ)$, $H(R)$ is a module over the monad $(R,η,μ)$
  with codomain $\mc{D}$ with the action $H(R) ∘ R \xrightarrow{Θ_{R,η}} H(R ∘
  R) \xrightarrow{H(μ)} H(R)$
\end{proposition}

\noindent Nonetheless, their work does not amount to working with modules over
monads as they only use generalized strength as an intermediate tool to build
regular signatures with strength, on which they solely rely in the end.

%
Actually, the slightly more general notion of generalized strength, like the one of modules
over monads, is not needed to model and build simply-typed signatures modularly.
%
As an example, let us consider the abstraction of the simply-typed lambda calculus.
Given a term of type $t$ with a free variable of type $s$, it should return a term
of type $s → t$, i.e., $\abs_{s,t} : Λ_T (Γ + y(s))(t) → Λ_T (Γ)(s → t)$,
where $y(s)(u) = {*}$ if $u = s$ and $∅$ otherwise.
%
As constructors are represented by morphisms of the form $X → Θ$, we must ensure
that all fibers but $Λ_T(Γ)(s → t)$ are empty, and that in the fiber over $s → t$,
we have $X(Γ)(s → t) := Λ_T (Γ + y(s))(t)$.
We can do so by postcomposing $Θ$ with a signature with strength $δ$ that trivialize fibers,
$\swap$ to swap them, and $y(s)$ to add free variables.

  \begin{definition}
    The functor on $[\Set^T,\Set^T]$, $δ$ and $\swap$ are $ω$-cocontinuous:
    %
    \[ \begin{array}{ccc}
      δ_s\; X\; Γ\; u \;
      = \left\{
        \begin{array}{ll}
          X(Γ)(s) & \mrm{if}\; u := s \\
          ∅       & \mrm{otherwise}
        \end{array} \right. & &

        \swap_s^t\; X\; \Gamma\; u := \left\{
        \begin{array}{ll}
          X(\Gamma)(s) & \mrm{if}\; u := t \\
          X(\Gamma)(t) & \mrm{if}\; u := s \\
          X(\Gamma)(u) & \mrm{otherwise}
        \end{array} \right.
    \end{array} \]
  \end{definition}

  \noindent We can then build an $ω$-cocontinuous signature with strength
  representing variable binding, by iterating $y(s)$ and using that
  postcomposing by, and $ω$-cocontinuous signatures, preserve $ω$-cocontinuity
  (\cref{ex:left-comp}):

  \begin{definition}
    There is a signature with strength $[Θ^l_s]_t$ representing the binding
    of $l$ variables in an input of type $s$ and returning a term of type $t$ by
    $\swap_{s → t}^t ∘ δ_{s → t} ∘ y(l) ∘ Θ$
  \end{definition}

  \noindent Using the usual closure under coproducts and finite products, we can
  then represent the simply-typed lambda calculus as follows:

  \begin{example}
    The simply-typed lambda calculus can be modeled by the $ω$-cocontinuous
    signature with stregth:
    %
    \[ \bigplus_{s,t:T_{Λ_T}}\;\; [Θ_{s → t}]_t × [Θ_{s}]_t \; + \; [Θ^{s}_{t}]_{s → t} \]
  \end{example}



\subsubsection{Hss and monoidal categories}
\label{subsubsec:rw-hss-monoidal}

%
%
%
Hss have been also been applied to build the substitution structure of \emph{coinductive}
simply-typed higher-order languages by Matthes, Wullaert, and Ahrens in
\cite{HssNonWellfounded24}, in the UniMath library.
%
Trying to do so, Matthes faced by formalisation issues \cite{CoqPl2023MonCatHss},
in particular controlling the unfolding of definitions, e.g., of the monoidal
structure of endofunctor categories, when proving theorems.
%
Consequently, to be more abstract and to better control unfolding, but also to be
more general, Matthes generalized hss and the proof of the initiality
theorem to monoidal categories.
%
Though it is not the main subject of their paper, this was reported on in
\cite[Section 4.4]{HssNonWellfounded24}.
%
The proof they have formalized is basically the same as the one presented in
\cref{sec:building_initial_model} up to minor details.
%
Though the present papers share an author, we stress that the generalizations were
developed in parallel and independently by Matthes, and for different reasons.

In contrast to our work, \cite{HssNonWellfounded24} provides an extensive analysis of signatures
with strength in terms of actions on actegories, and uses results on actegories
as an abstract framework to equip signatures with appropriate strengths.
%
While this can ease formalisation, it is very technical.
Consequently, to ease understanding, we have decided to use a more direct approach.
%
Another difference is that their definitions are stated for the \emph{reversed}
monoidal category of ours \cite[Example 1.2.9]{2DimensionalCategories20},
i.e., for the reversed monoidal product $X ⊗' Y := Y ⊗ X$.
%
In practice it makes no difference as all monoidal categories are reversible,
and as they instantiate their framework with reversed categories as well.

%
%
%
%
