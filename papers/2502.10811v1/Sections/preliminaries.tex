\section{Preliminaries}
\label{sec:prelims}

The framework defined herein is entirely written in the language of category theory.
We assume the reader is familiar with the notions of category, functor, and
natural transformation, as found, e.g., in Riehl's book \cite{CategoryTheoryInContext14}.
%
We recall here some more specific definitions and properties about monoidal
categories and $ω$-colimits that are needed to understand the framework.

\subsection{Monoidal Categories}

To be able to define our notion of models, we need more structure on the
underlying category than the one provided by the mere definition of a
category.
Hence, throughout the paper, we work with monoidal categories.
A modern presentation of monoidal categories can be found in
\cite[Section 3.1]{2DimensionalCategories20}.

\begin{definition}[Monoidal Categories]
  \label{def:mon-cat}
  A \emph{monoidal category} is a tuple $\Cmon$, where $\mc{C}$ is a category,
  $\_ ⊗ \_ : \mc{C} × \mc{C} \to C$ is a bifunctor called the
  monoidal product, and $I : \mc{C}$ an object called the unit.
  $α,λ, ρ$ are natural isomorphisms -- called the associator, and the
  left and right unitor -- that satisfy the unit axiom and the pentagon axiom.
  %
  \begin{align*}
    α_{X,Y,Z} : (X ⊗ Y) ⊗ Z ≅ X ⊗ (Y ⊗ Z)
    &&
    λ_{X} : I ⊗ X ⊗ X
    &&
    ρ_{X} : X ⊗ I ⊗ X
  \end{align*}
  %
\end{definition}

\begin{remark}
  The above notations are fairly standard, however be careful that in some
  references, as in \cite{HssRevisited15,HssUntypedUniMath19}, $ρ_X$ and
  $λ_X$ are swapped.
\end{remark}

\begin{example}[Category of endofunctors]
  Given any category $\mc{C}$, the category of endofunctors $[\mc{C},\mc{C}]$
  is monoidal for the composition of functors as monoidal product, the
  identity functor $\Id$ as unit, and the identity natural transformation
  for $α,λ,ρ$.
\end{example}

\subsection{$ω$-Colimits}
\label{subsec:omega-colimits}

The notion of $ω$-colimit is important for the construction of syntax, since it
abstractly formalizes the idea of building sets of abstract syntax trees by
recursion on the height of such trees.
We thus use $ω$-colimits in the construction of initial models, and we crucially
use that signatures preserve $ω$-colimits when we construct an initial model, as
discussed in \cref{sec:initiality_theorem}.

\begin{definition}[$ω$-chains]
  An $ω$-chains is a sequence $(C_i, c_i)_{i : \N}$ of objects and
  morphisms assembling as a left oriented infinite chain:
  \[
    \begin{tikzcd}
      C_0 \ar[r, "c_0"]
        & C_1 \ar[r, "c_1"]
        & C_2 \ar[r, "c_2"]
        & C_3 \ar[r, "c_3"]
        & \; ...
    \end{tikzcd}
  \]
\end{definition}

\begin{example}
  Given a category $\mc{C}$ with initial object $0 : \mc{C}$, for any
  endofunctor $F : \mc{C} \to \mc{C}$  on $\mc{C}$, there is a canonical
  $ω$-chain associated to $F$ denoted by $\mrm{chn}_F$:
  \[
    \begin{tikzcd}
      0 \ar[r, "\star"]
        & F(0) \ar[r,   "F(\star)"]
        & F^2(0) \ar[r, "F^2(\star)"]
        & F^3(0) \ar[r, "F^3(\star)"]
        & \; ...
    \end{tikzcd}
  \]
\end{example}

\begin{definition}[$ω$-cocontinuous functors]
  \label{def:omega-cocontinuous}
  The colimit of an $ω$-chain is called an $ω$-colimit.
  A functor $F : \mc{C} \to \mc{D}$ is \emph{$ω$-cocontinuous} if it
  preserves all $ω$-colimits.
  The collection of $ω$-cocontinuous functors $\mc{C} → \mc{C}$ forms a full subcategory
  of the category of endofunctors $\mc{C} → \mc{C}$.
\end{definition}

\begin{example}
  The identity functor $\Id : \mc{C} → \mc{C}$ is $ω$-cocontinuous.
\end{example}

\subsubsection*{Closure properties}

We aim to build our signatures modularly, i.e. out of smaller signatures.
Being $ω$-continuous is a key feature of our signatures; we crucially use that
$ω$-continuous functors are closed under some operations, such as taking
colimits, limits and composition.
These facts can be proved using different results in
\cite[Section 3.8]{CategoryTheoryInContext14}.

\begin{proposition}[Closure under colimits]
  \label{prop:omega-colimits}
  If $\mc{C}$ is cocomplete, then the category of $ω$-cocontinuous
  functors $\mc{C} → \mc{C}$ is closed under colimits.
\end{proposition}

\noindent While $ω$-continuous functors are closed under all small colimits for
a wide variety of base categories, they are usually only closed under finite
limits. For our purpose, this will suffice.

\begin{proposition}[Closure under limits]
  \label{prop:omega-limits}
  If $\mc{C}$ admits a class of limits that commutes with $ω$-colimits
  in $\mc{C}$, then the category of $ω$-cocontinuous functors
  $\mc{C} → \mc{C}$ is closed under this particular class of limits.
\end{proposition}

\begin{proposition}
  \label{prop:presheaves-limits}
  In the category $\Set$, and in functor categories $[\mc{C},\Set$], finite limits commute
  with $ω$-colimits.
\end{proposition}

%
%
%
%
