\section{Building an Initial Model}
\label{sec:building_initial_model}

%

We have seen two different notions of signatures:
One the one hand, the signatures defined in terms of modules over monoids provide a flexible framework.
On the other hand, signatures with strength provide an initiality theorem and an adjoint theorem.
In this section, we prove these two theorems.
In particular, we will construct a ``syntactic'' model for any signature with
strength satisfying the hypotheses of \cref{thm:initiality-theorem}.

%
To do so, we rely on heterogeneous substitution systems (hss); these were originally
designed to prove that both well founded and non-well founded syntax \cite{Hss04}
have a monadic substitution structure, and extended to provide a form of initial
semantics \cite{HssRevisited15} in the well founded case.
%
%
We generalize the notion of hss from endofunctor categories to monoidal categories.
To prove the initiality theorem, we further generalize a proof scattered throughout
\cite{Hss04,DeBruijnasNestedDatatype99,HssRevisited15,HssUntypedUniMath19},
to monoidal categories and ω-cocontinuity, and adapt it from hss to models.
Doing so, we provide more details than currently available in the literature.

\begin{related Work}
  We stress that hss and a variant of this proof have been recently and
  independently generalized to monoidal categories by Matthes to ease
  formalization, as mentioned in \cite[Section 4.4]{HssNonWellfounded24}.
  We provide a full discussion on hss and this proof in \cref{subsec:rw-hss}.
\end{related Work}

\begin{related Work}
  Another proof following \cite{ListObjects17} would be possible.
  We explain the differences with this proof in details in
  \cref{subsubsec:rw-param-vs-hss,subsubsec:rw-building-param-hss}.
\end{related Work}

To convey intuiton, in the following, we first review hss and explain why
they allow us to construct models in \cref{subsec:hss_models}.
%
We then explain how to build a heterogeneous substitution system from our
assumptions in \cref{subsec:building_hss}, and why the model constructed from it
is initial as a model in \cref{subsec:building_initial_model}.
Finally, we prove the adjoint theorem from the initiality theorem.

%
In this section, we suppose that $\mc{C}$ has coproducts, and that they are
preserved by $\_ ⊗ Z$ for all $Z : \mc{C}$, as it is needed to state
our definitions.
%
As before, these hypotheses are by no means a restriction as they are also
hypotheses of the initiality theorem.
%

For readability, we write the functor $A ↦ I + H(A)$ as $I + H\_$ or $I + H$.

\subsection{Heterogeneous Substitution Systems and Models}
\label{subsec:hss_models}

%
The recursion principle arising from initial algebras is not sufficient to
prove that a higher-order language, such as the lambda calculus, forms a monad.
%
As explained, for $[\Set,\Set]$, in \cite{DeBruijnasNestedDatatype99,GeneralisedFold99},
a stronger recursion principle is required; in their case, one named ``generalized fold''.
%
Heterogeneous substitution systems (hss), introduced in \cite{Hss04,HssRevisited15}
on endofunctor categories, are inspired by generalized folds, and are
designed to prove that higher-order languages form monads when considered
with their substitutions.

\begin{definition}[Heterogeneous Substitution System]
  \label{def:hss}
  For a signature with strength $(H, θ)$, a \emph{heterogeneous substitution
  system (hss)} is a tuple $(R, η, r)$ where $R : \mc{C}$ is an object
  of $\mc{C}$ and $η : I → R$ and $r : H(R) → R$ are morphisms
  of $\mc{C}$ such that, for all $(Z , e) : \trm{Ptd}(\mc{C})$ and $f : Z
  → R$, there is a unique morphism $\{ f \} : R ⊗ Z →
  R$ making the following diagram commute:
  %
  \[
    \begin{tikzcd}[column sep=large]
      I ⊗ Z \ar[r, "η ⊗ Z"] \ar[dd, swap, "λ_Z"]
        & R ⊗ Z \ar[dd, "\{f\}"]
        & H(R) ⊗ Z \ar[l, swap, "r ⊗ Z"]
                         \ar[d, "θ_{R,e}"] \\
        &
        & H(R ⊗ Z) \ar[d, "H(\,\{f\}\,)"] \\
      Z \ar[r , swap, "f"]
        & R
        & H(R) \ar[l, "r"]
    \end{tikzcd}
  \]
\end{definition}

\begin{related Work}
  In previous works on hss \cite{Hss04,HssRevisited15,HssUntypedUniMath19,HssTypedUnimath22},
  it was required for $f$ to be a morphism of pointed objects $f : (Z,e) → (R,η)$.
  It was later realised independently by Matthes, Wullaert, and Ahrens in \cite{HssNonWellfounded24},
  and by Lafont that this hypothesis is superfluous for our purpose.
\end{related Work}

Intuitively, hss strengthen the concept of algebra as it additionally enables us
to build a unique morphism $R ⊗ Z → R$ for any suitable $Z$.
As done in \cite{Hss04} for endofunctor categories, the added freedom of choosing
$Z$ as we wish enables us to derive a monoid structure, and later a model structure:

\begin{proposition}[Monoids from Hss]
  \label{prop:hss_to_monoid}
  If a signature $(H, θ$) has a heterogeneous substitution system $(R,
  η, r)$, then $(R, η, \{ \id_{(R,η)} \})$ is a monoid.
\end{proposition}
\begin{proof}
  %
  Let's denote $\{\id_{(R,η)}\} : R ⊗ R → R$ by $μ$.
  By definition, $(R, η, μ)$ is well-typed.
  Moreover, the left-unit law of the monoid $λ_R = μ ∘ (η ⊗ R)$
  holds by definition of $μ$.
  Hence, there are only the right-unit law and the associativity of $μ$
  to check.

  %
  To prove the second unit law, $\rho_R = μ ∘ (R ⊗ η)$, let's
  apply the hss for $f := η : I → R$.
  Both morphisms make the diagram commute, thus are equal by uniqueness.
  The morphism $\rho_R$ can be shown to satisfy the diagram using that
  $λ_I = \rho_I$, the definition of $θ_{R,\id}$, and the
  naturality of $\rho$.
  The morphism $μ ∘ (R ⊗ η)$ can be shown to satisfy the
  diagram using the left-unit law, the definition of $μ$, and the
  naturality of $λ$, and of $θ$ in the second argument.
  Thus by uniqueness $\rho_R = \{ η \} = μ ∘ (R ⊗ η)$.

  %
  %
  For homogeneity reasons, to prove the associativity of $μ$, we prove $μ
  ∘ (R ⊗ μ) = μ ∘ (μ ⊗ R) ∘ \alpha^{-1}$.
  %
  To do so, let's apply the hss in $μ : R ⊗ R → R$, where $R ⊗ R$ is pointed by
  $I \xrightarrow{λ_I} I ⊗ I \xrightarrow{η ⊗ η} R ⊗ R$.
  %
  Both morphisms make the diagram commute, thus are equal by uniqueness.
  The morphism $μ ∘ (μ ⊗ R)$ can be shown to satisfy it using
  the unit law, the naturality of $θ$, and the definition of $μ$.
  %
  The morphism $μ ∘ (μ ⊗ R) ∘ \alpha^{-1}$ can be shown to
  satisfy the diagram using the naturality of $θ$ and $\alpha$, the
  definition of $μ$, and the definition of $θ$ in $e ⊗ e'$.
\end{proof}

\begin{proposition}[Models from Hss]
  \label{prop:hss_to_model}
  If a signature $(H, θ$) has a heterogeneous substitution system
  $(R,η,r)$, then $((R,η,μ), r)$ is a model of $H$.
\end{proposition}
\begin{proof}
  By \cref{prop:hss_to_monoid}, $(R,η,μ)$ is a monoid.
  It remains to prove that $r$ is a morphism of modules $H(R) → R$ i.e.
  to prove that $μ ∘ (r ⊗ R) = r ∘ H(μ) ∘ θ_{R,\id}$.
  This is true by the definition of $μ$.
\end{proof}

\subsection{Building Heterogeneous Substitution Systems}
\label{subsec:building_hss}

\Cref{prop:hss_to_model} states that it suffices to build a heterogeneous
substitution system for a signature with strength to get a model for it.
%
In this section, we build a heterogeneous substitution system from a signature
with strength $(H, \Theta)$, using generalized iteration in Mendler's style.
%
As the construction relies on a particular way of building initial algebras, we
will first recall, and give a proof of, Adámek's Theorem.

\begin{theorem}[Adámek's Theorem \cite{Adamek74}]
  \label{thm:adamek}
  Let $\mc{C}$ be a category with an initial object $0$, and
  $\omega$-colimits. Let $F : \mc{C} → \mc{C}$ be an $\omega$-cocontinuous
  functor.
  Then there is an initial $F$-algebra $(R,r)$ with $R$ built as the
  $\omega$-colimit of:
  %
  \[
    \begin{tikzcd}[column sep=large]
      0 \ar[r, "i"] \ar[dr, swap, "t_0"]
        & F(0) \ar[r, "F(i)"] \ar[d, "t_1"]
        & F^2(0) \ar[r, "F^2(i)"] \ar[dl, "t_2"]
        & ... \\
      & R
        &
        &
        &
    \end{tikzcd}
  \]

\end{theorem}
\begin{proof}
  %
  %
  As $F$ preserves $\omega$-colimits, $F(R)$ is the colimit of the image by
  $F$ of the chain $F(\trm{chn}_F)$.
  Using the initiality of $0$, it can be completed into a cocone for the
  full chain $\trm{chn}_F$.
  %
   Given any other cocone $A$ of $\trm{chn}_F$, $A$ is, in particular, a
   cocone for $F(\trm{chn}_F)$, which yields a map $h : F(R) → A$
   satisfying the universal property of $F(\trm{chn}_F)$.
   Using initiality, $h$ also verifies the one of $F(\trm{chn}_F)$.
  %
  Given any other such map $h'$, in particular both $h$ and $h'$ satisfy the
  universal property $F(\trm{chn}_F)$, hence as $F(R)$ is a colimit of this
  chain, $h = h'$.
  %
  Thus $F(R)$ is a colimit of $\trm{chn}_F$, which yields a unique map
  $r : F(R) → R$.

  %
  %
  Given an $F$-algebra $(A,a)$, we build a cocone on $A$.
  We define $a_n$ recursively: the morphism $a_0$ is the unique morphism
  from $0 → A$; for a natural number $n$, $a_{n+1}$ is defined as the
  composition $F^{n+1}(0) \xrightarrow{a_n} F(A) \xrightarrow{a} A$.
  The commutativity holds by the uniqueness in the $0$ case and recursively
  for $n+1$.
  Then by the universal property we obtain a unique map $h : C → A$.
  %
  Both $a∘ F(h)$ and $h ∘ c$ satisfy the universal property of the
  colimit, as such they are equal and so $h$ is a morphism of algebras.
  %
  Given another morphism of $F$-algebras $h'$, it can be shown by induction
  that it verifies the universal property, hence by uniqueness $h = h'$, and
  so $h$ is unique.
  %
  In consequence, $(R,r)$ is an initial $F$-algebra.
\end{proof}

\begin{restatable}[Generalized Mendler's style Iteration {{\cite[Theorem 1]{GeneralisedFold99}}}]{theorem}{Mendler}
  \label{thm:gen-mendler}
  Let $\mc{C}$ and $\mc{D}$ be two categories with initial object and
  $\omega$-colimits, $F : \mc{C} → \mc{C}$ and $L : \mc{C} → \mc{D}$ be
  two functors.
  If $F$ and $L$ preserve $\omega$-colimits, and $L$ preserves initiality,
  then for all $X : \mc{D}$ and natural transformation
  %
  \[ Ψ : \mc{D}(L\_,X) → \mc{D}(L(F\_),X) \]
  %
  there is a unique morphism $\mrm{It}^L_F(Ψ) : L(R) → X$ in $\mc{D}$
  --- where $(R,r)$ is the initial $F$-algebra obtained by Adámek's Theorem (\cref{thm:adamek}) ---
  making the following diagram commute:
  %
  \[
    \begin{tikzcd}[ampersand replacement=\&]
      L(F(R)) \ar[r, "L(r)"] \ar[dr, swap, "Ψ_R(\mrm{It}^L_F(Ψ))"]
        \& L(R) \ar[d, "\mrm{It}^L_F(Ψ)"] \\
      \& X
    \end{tikzcd}
  \]
\end{restatable}
\begin{proof}
  As a shorthand, we will denote $\mrm{It}^L_F(Ψ)$ by $h$ in the proof of existence and uniqueness of $\mrm{It}^L_F(Ψ)$.
  First, let's construct $h$.
  As $L$ preserves initiality, $L(0)$ is initial and there is a unique map
  $x : L(0) → X$.
  Iterating $Ψ$ yields a cocone $(X, (Ψ^n(x))_{n : \mb{N}})$ for the
  diagram $(LF^n(0), LF^n(i))_{n : \mb{N}}$.
  Yet as $L$ preserves $\omega$-colimits, $L(R)$ is the colimit of this
  diagram, and by the universal property of colimits, there is a unique $h :
  R(L) → X$ such that $\forall n.\; h ∘ L(t_n) = Ψ^n(x)$, i.e.,
  such that the following diagram commutes:
  %
  \[
    \begin{tikzcd}[column sep=large]
      L(0) \ar[r, "L(i)"] \ar[dr, swap, "L(t_0)"] \ar[ddr, bend right, swap, "Ψ^0(x)"]
        & LF(0) \ar[r, "LF(i)"] \ar[d, "L(t_1)"]
        & LF^2(0) \ar[r, "LF^2(i)"] \ar[dl, "L(t_2)"] \ar[ddl, bend left, "Ψ^2(x)"]
        & ... \\
      & L(R) \ar[d, "h"]
        &
        &
        & \\
      & X
        &
        &
        &
    \end{tikzcd}
  \]
  %
  We have built a map $h$, it remains to prove that it satisfies the
  desired property:  $h ∘ L(r) = Ψ(h)$.
  To do so we are going to prove that $\forall n.\; Ψ(h) ∘
  L(r^{-1}) ∘ L(t_n) = Ψ^n(x)$.
  Indeed by the uniqueness of $h$, we can conclude that $h = Ψ(h) ∘
  L(r^{-1})$, i.e. $h ∘ L(r) = Ψ(h)$.
  %
  We prove $\forall n.\; Ψ(h) ∘ L(\alpha^{-1}) ∘ L(t_n) =
  Ψ^n(x)$ by induction on $n$.
  The $n = 0$ case holds by initiality of $L(0)$.
  The $n+1$ case holds by the following chain of equalities:
  %
  \[
  \begin{array}{cclc}
    Ψ(h) ∘ L(r^{-1}) ∘ L(t_{n+1})
      &=& Ψ(h) ∘ L(r^{-1}) ∘ L(r) ∘ LF(t_{n})
          & (\textrm{definition of }r) \\
      &=& Ψ(h) ∘ LF(t_{n}) & \\
      &=& Ψ(h ∘ L(t_{n})) & (\textrm{naturality of }Ψ) \\
      &=& Ψ(Ψ^n(x)) & (\textrm{induction hypothesis})
  \end{array}\]
  %
  Now that we have proven existence, we need to prove uniqueness.
  To do so, suppose we have a $h : L(R) → X$ such that $h ∘ L(r) =
  Ψ(h)$; we are going to show that $\forall n.\; h ∘ L(t_n) =
  Ψ^n(x)$.
  As a consequence, such an $h$ verifies the universal property of $L(R)$ and
  hence $h$ is unique. We prove the equations by induction on $n$.
  The $n = 0$ case holds by initiality of $L(0)$.
  The $n+1$ case holds by the following chain of equation:
  %
  \[
  \begin{array}{cclc}
    h ∘ L(t_{n+1})
      &=& h ∘ L(r ∘ t_n)
          & (\textrm{definition of }r) \\
      &=& h ∘ L(\alpha) ∘ L(t_n) & \\
      &=& Ψ(h) ∘ L(t_n) & (\textrm{definition of }h) \\
      &=& Ψ(h ∘ LF(t_n)) & (\textrm{naturality of }Ψ) \\
      &=& Ψ(Ψ^n(x)) & (\textrm{induction hypothesis})
  \end{array}\]
\end{proof}

\begin{proposition}[Building an Hss]
  Let $\mc{C}$ be a monoidal category, with initial object, coproducts,
  $\omega$-colimits and such that for all $Z : \mc{C}$, $\_ ⊗ Z$
  preserves initial objects, coproducts, and $\omega$-colimits.
  Let $(H,θ)$ be a signature with strength such that $H$ is
  $\omega$-cocontinuous.
  Then $(R, η + r)$ --- the initial algebra of $I + H\_$ obtained by
  Adámek's Theorem --- is a heterogeneous substitution system for
  $(H,θ)$.
\end{proposition}

\begin{proof}
  Let $(Z,e)$ be a pointed object, and $f : Z → R$ a morphism.
  We are going to construct $\{ f \}$ and show that it is uniquely defined,
  by a suitable application of generalized Mendler's style iteration.
  Let's apply the theorem for $F := \id + H$, $L := \_ ⊗ Z$, $X := R$
  and $Ψ \;h ↦ (f + r) ∘ (\id + H(h)) ∘ (λ_Z +
  θ_{R,e})$.
  This will give us a unique map $\mrm{It}^{\_ ⊗ Z}_{I + H}(Ψ) : R
  ⊗ Z → R$, such the following diagram commutes:
  %
  \[
  \begin{tikzcd}[column sep=4cm]
    I ⊗ Z + H(R) ⊗ Z
                        \ar[r, "(η + r) ⊗ Z"]
                        \ar[d, swap, "λ_Z + θ_{R,e}"]
                        \ar[ddr, start anchor=-25, shorten <=8pt, end anchor=north west,
                             "Ψ(\mrm{It}^{\_ ⊗ Z}_{I + H}(Ψ))"]
      & R ⊗ Z     \ar[dd, "\mrm{It}^{\_ ⊗ Z}_{I + H}(Ψ)"]\\
    Z + H(R ⊗ Z)  \ar[d, swap, "\id + H(\mrm{It}^{\_ ⊗ Z}_{I + H}(Ψ))"]
      & \\
    Z + H(R)            \ar[r, swap, "{[}f {,} r{]}"]
      & R \\
  \end{tikzcd}
  \]
  %
  This diagram is exactly the definition of an hss, if one denotes
  $\mrm{It}^{\_ ⊗ Z}_{I + H}(Ψ)$ by $\{ f \}$.
  Hence it suffices to verify the hypothesis for the given input to get the
  result.
  The only non-obvious hypothesis is that the above defined $Ψ$ is natural
  in $R$, which follows by naturality of $θ$ in its first argument.
\end{proof}

\subsection{Building an Initial Model}
\label{subsec:building_initial_model}

The previous work has enabled us to construct a model of a given signature.
It remains to prove that this model is initial.
To do so, we adapt a proof for hss on endofunctor categories introduced in
\cite{HssRevisited15} to models on monoidal categories.
This crucially relies on a fusion law for generalized Mendler's style iteration
which enables us to factorise a generalized Mendler's style iteration followed
by a natural transformation into a Mendler's style iteration.
This is not surprising as we are relying on a generalized recursion principle
to build our model, and fusion laws are ubiquitous in computer science to
simplify recursion principles.

\begin{theorem}[Fusion Law for Generalized Mendler's style Iteration {{\cite[Lemma 9]{HssRevisited15}}}]
  \label{thm:fusion-law}
  %
  Let $\mc{C,D},F,L, X, Ψ$ be objects satisfying the hypotheses of the
  \hyperref[thm:gen-mendler]{generalized Mendler's style iteration} theorem.
  For the same $\mc{C,D},F$, suppose given other $L',X',Ψ'$ satisfying
  the hypotheses.
  %
  If there is a natural transformation $Φ : \mc{D}(L\_,X) →
  \mc{D}(L'\_,X')$ such that $Φ_{F(μ F)} ∘ Ψ_{μ F} =
  Ψ'_{μ F} ∘ Φ_{μ F}$ --- where $μ F$ denotes the
  $F$-algebra built by Adámek's theorem --- then
  %
  \[ Φ_{μ F}(\mrm{It}^L_F(Ψ)) = \trm{It}^{L'}_F(Ψ') \]
\end{theorem}
\begin{proof}
  By uniqueness, it suffices to prove that $Φ_{μ F}(\trm{It}^L_F(Ψ))$
  satisfies the defining diagram of $\trm{It}^{L'}_F(Ψ')$.
  This can be done using the definition of the assumption, the definition of
  $\trm{It}^L_F(Ψ)$ and the naturality of $Φ$.
\end{proof}

We are now ready to prove the initiality theorem:

\initialitytheorem*
\begin{proof}
  By \cref{prop:hss_to_model}, $((R,η, μ), r)$ is a model of $H$.
  We need to prove that this model is actually initial.
  Let $((R',η', μ'), r')$ be another model.
  We need to prove there is a unique morphism of models $((R,η, μ),
  r) → ((R',η',μ'),r')$.
  In other words, we need to prove there is a unique morphism of monoids $f
  : (R,η,μ) → (R',η',μ')$ --- i.e., respecting $η$ and $μ$
  --- making the following diagram commute:
  %
  \[
    \begin{tikzcd}
      H(R) \ar[r, "r"] \ar[d, swap, dashed, "H(f)"]
        & R \ar[d, dashed, "\exists ! f"] \\
      f^*H(R') \ar[r, swap, "f^*r'"] & f^*R'
    \end{tikzcd}
  \]
  %
  To prove both uniqueness and existence, we are going to use that $(R',
  η' + r')$ is a $(I + H\_)$-algebra and that $(R, η + r)$ is the
  initial one. \medskip

  To prove uniqueness, suppose we have a morphism of models $f$, in
  particular it is a morphism of $(I + H\_)$-algebras $f : (R, η + r) →
  (R', η' + r')$ and as such is unique by the initiality of $(R, η
  + r)$. \medskip

  To prove the existence of such a model morphism, we are going to show that
  the morphism of algebras $f : (R, η + r) → (R', η' + r')$
  existing by initiality of $(R, η + r)$ is a morphism of models.
  The morphism $f$ respects $η$ as it is a morphism of algebras.
  The commutativity of the diagram of module morphisms holds if it holds for
  the underlying morphism of $\mc{C}$, which is also verified as $f$ is a
  morphism of algebras.

  It remains to prove that it respects $μ$, i.e., that $f ∘ μ = μ'
  ∘ f ⊗ f$.
  To do so, we are going to use that $μ := \{ \id \} := \mrm{It}^{\_
  ⊗ Z}_{I + H}(Ψ)$  to factorise $f ∘ μ$ into another
  iteration $\mrm{It}^{\_ ⊗ Z}_{I + H}(Ψ')$ as shown below.
  The point is that iterations are the unique morphisms making their
  associated diagrams commute.
  In consequence, it will suffice to prove that $μ' ∘ f ⊗ f$
  satisfies it to prove the equalities.
  %
  \begin{align*}
    \begin{tikzcd}[ampersand replacement=\&]
      R ⊗ R \ar[r, "f ⊗ f"]
                  \ar[d, swap, "\mrm{It}^{\_ ⊗ Z}_{I + H}(Ψ)"]
        \& R' ⊗ R' \ar[d, "μ'"] \\
      R \ar[r, swap, "f"]
        \& R'
    \end{tikzcd}
    &&
    \begin{tikzcd}[ampersand replacement=\&]
      R ⊗ R \ar[r, "f ⊗ f"] \ar[dr, swap, bend right, "\mrm{It}^{\_ ⊗ Z}_{I + H}(Ψ')"]
        \& R' ⊗ R' \ar[d, "μ'"] \\
        \& R'
    \end{tikzcd}
  \end{align*}
  %
  Here $Ψ \;h := (\id + r) ∘ (\id + H(h)) ∘ (λ_Z +
  θ_{R,e})$ and $Ψ' \;h := (f + r') ∘ (\id + H(h)) ∘
  (λ_Z + θ_{R,e})$.
  To do the factorisation, we apply the fusion law for $Φ := f^*$ i.e.
  precomposition by $f$, for which the assumption is satisfied as $f$ is a
  morphism of algebras.
  Finally, proving that $μ' ∘ f ⊗ f$ satisfies the diagram of
  $\mrm{It}^{\_ ⊗ Z}_{I + H}(Ψ')$ unfolds to proving:

  \begin{align*}
    \begin{tikzcd}[ampersand replacement=\&]
      I ⊗ R \ar[ddd, swap, "λ_R"] \ar[rr, "η ⊗ R"]
                  \ar[dr, swap, "I ⊗ f"]
        \&
        \& R ⊗ R \ar[d, "R ⊗ f"] \\
        \& I ⊗ R' \ar[r, "η ⊗ R'"]
                       \ar[dr, swap, near end, "η' ⊗ R'"]
                       \ar[ddr, swap, bend right, near start, "λ_{R'}"]
        \& R ⊗ R' \ar[d, "f ⊗ R'"] \\
        \&
        \& R' ⊗ R' \ar[d, "μ'"] \\
      R \ar[rr, "f"]
        \&
        \& R'
    \end{tikzcd}
    &&
    \begin{tikzcd}[ampersand replacement=\&]
      H(R) ⊗ R \ar[rr, "r ⊗ R"] \ar[dr, "H(f) ⊗ f"]
                     \ar[d, swap, "θ_{R,\id}"]
        \&
        \& R ⊗ R \ar[dd, "f ⊗ f"] \\
      H(R ⊗ R) \ar[d, swap, "H(f ⊗ f)"]
        \& H(R') ⊗ R' \ar[dl, "θ_{R',\id}"] \ar[dr, "r' ⊗ R'"]
        \& \\
      H(R' ⊗ R') \ar[d, swap, "H(μ')"]
        \&
        \& R' ⊗ R' \ar[d, "μ'"]\\
      H(R') \ar[rr, swap, "r'"]
        \&
        \& R'
    \end{tikzcd}
  \end{align*}
  %
  The left diagram commutes using the naturality of $λ$, the fact that
  $f$ is a morphism of algebras, and by the monoid laws.
  The right diagram commutes by naturality of $θ$ in both arguments,
  the fact that $f$ is a morphism of algebras and because $r$ is a
  morphism of modules.
\end{proof}

\subsection{Building an Adjoint}
\label{subsec:building-adjoint}

While it is possible to directly prove the adjoint theorem (\cref{thm:adjoint-theorem}), and deduce the
initiality theorem (\cref{thm:initiality-theorem}) from it, it is possible and actually better to do the
opposite.
Indeed, by doing so, one can save, in the initiality theorem, the hypothesis
that $X ⊗ \_$ is $\omega$-cocontinuous:

\adjointtheorem*
\begin{proof}
  %
  First, let's prove the existence of $\Free : \mc{C} → \Model(H)$.
  %
  Given $X : \mc{C}$, $X ⊗ \_$ can be equipped with the strength
  $\alpha : (X ⊗ A) ⊗ B → X ⊗ (A ⊗ B)$.
  Hence, by \cref{prop:sigstrength_omega_cocomplete}, $H + X ⊗ \_$
  is an $\omega$-cocontinuous signature with strength, and so has an initial
  model by the initiality theorem denoted $\Free(X)$, with carrier $μ A.
  (I + H(A) + X ⊗ A)$.
  We then get a model of $H$ by forgetting the extra $X ⊗ \_$ structure.
  %
  Given a morphism $X → Y$, $\Free(Y)$ can be equipped with a $H + X
  ⊗ \_$ structure.
  Hence, by initiality, there is a unique $H + X ⊗ \_$ morphism of
  model $\Free(X) → \Free(Y)$, which is in particular a $H$ morphism of
  models.

  %
  Given an object $X : \mc{C}$, and a model of $H$, $M : \Model(H)$, we need
  to build an natural isomorphism $\Model(H)(\Free(X),\, M) \cong C(X,\,
  M)$, where we identify $U(M)$ with $M$.
  %
  To build $K : C(X,\, M) → \Model(H)(\Free(X),\, M)$, given $f : X →
  M$, we turn $M$ from an $H$ model to an $H + X⊗ \_$ model.
  Indeed, by initiality of $\Free(X)$, this will provide a unique morphism
  of $H + X ⊗ \_$ models, which is in particular a morphism of
  $H$-models.
  To do so, it suffices to provide a module morphism $X ⊗ \Theta →
  \Theta$, which can be defined as $X ⊗ M \xrightarrow{f ⊗ M} M
  ⊗ M \xrightarrow{\mu_M} M$.
  %
  We can build an inverse $L : \Model(H)(\Free(X),\, M) → C(X,\, M)$ using
  that $\Free(X)$ is a $H + X ⊗ \_$ model, and as such is equipped
  with morphism $\sigma : X ⊗ \Free(X) → \Free(X)$.
  Given $f^\# : \Free(X) → M$, we define $L(f^\#)$ as
  $X \xrightarrow{\rho_X}         X ⊗ I
     \xrightarrow{X ⊗ η} X ⊗ \Free(X)
     \xrightarrow{\sigma}         \Free(X)
     \xrightarrow{f^\#}           M$.
  %
  %
  Proving that $L ∘ K (f) = f$, follows the universal property of $G(f)$,
  the definition of $L$ and the monoid laws.
  %
  By definition, $K(L(f^\#))$ is the unique morphism of $H + X ⊗ \_$
  models from $X$ to $M$ when equipped with $L(f)$.
  Hence, to prove that $K(L(f^\#)) = f^\#$, it suffices to prove that $f^\#$
  is such a morphism.
  As $h$ is already a morphism of $H$ models, it suffices to verify that it
  is compatible with the $X ⊗ \_$ constructor.
  It follows by the definition of the strength $X ⊗ \_$ and the
  different monoid and monoidal laws.
  %
  Lastly, the naturality in both arguments follows from the definition.
\end{proof}


%
%
%
%
