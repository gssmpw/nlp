\section{Conclusion}
\label{sec:conclusion}


%

%

In this paper, we present a framework that unifies three distinct approaches to
initial semantics -- modules over monads, signatures with strength, and
heterogeneous substitution systems -- by suitably generalizing and combining them.
%
Doing so, we have shown that:

%
%
%
\begin{enumerate}
  \setlength\itemsep{-1pt}
  \item Modules over monads provide us with an abstract and easy to manipulate framework (\cref{sec:models}).
  \item Signatures with strength enable us to state and prove an initiality theorem (\cref{sec:initiality_theorem}).
        Moreover, signatures with strength naturally appear as particular
        signatures when one tries to state such a theorem.
  \item Heterogeneous substitution systems form an intermediate abstraction that
        enables us to prove the initiality theorem modularly (\cref{sec:building_initial_model}).
\end{enumerate}

\noindent Moreover, this framework enables us to provide a detailed and
extensive discussion of related work (\cref{sec:related-work}), and to better understand the
existing literature.
%
Indeed, relating the different approaches and their variations to our framework
enables us to relate them to each other, clarifying a body of literature so far hard to enter.


\subsection{Open Questions}

With this work, we aim to solidify and conceptualize the foundations of
initial semantics, and to provide an accessible presentation to concepts and
proof scattered throughout many papers.
%
%
Nevertheless, there are still many open research questions in the area of
initial semantics; to consolidate the current knowledge on the subject, but
also to keep developing the framework.

%
%
%
%
%
%

\begin{itemize}
  \setlength\itemsep{-1pt}

    %
  \item In this work, we have not considered work on metavariables and
        ``second-order'' syntax \cite{HamanaMetavar04,SecondOrderDep08}.
        Additional work is needed to properly integrate and discuss this work.

  %
  \item This framework is formulated in terms of monoidal categories.
        %
        Generalizing it to skew-monoidal categories would enable us to encompass
        more instances, like De Bruijn monads \cite{NamelessDummies22}.
        %
        As discussed in \cref{subsec:rw-sigma-mon}, modules and signature with
        strength (\cref{sec:models,sec:initiality_theorem} of the framework)
        seem to generalize to skew-monoidal categories.
        %
        However, it does not seem straightforward to generalize hss or parametrized
        initiality (\cref{sec:building_initial_model}), and work is needed to
        give a proper generalization.

  %
  \item In this work, we have not considered semantics, only syntax.
        Yet, many frameworks have been extended to support some form of semantics.
        It remains to unify them and add them to this framework to get a full
        account of the literature on the subject:
        \begin{itemize}
          \item Many frameworks consider equations on top of syntax, like $β$-equalities.
                It remains to understand how they are related.
                For instance, it remains to understand how the approaches to equations of \cite{FioreHur10},
                and \cite{FioreSzamozvancevPopl22} relate to \cite{PresentableSignatures21,2Signatures19}.
          \item Different works have been considered to add reductions rules
                \cite{UntypedRelativeMonads16,TransitionMonads20}, and even
                reduction strategies \cite{TransitionMonads22}.
                All have an interpretation in terms of relative monads.
                Consequently, could they all be encompassed by a unique framework?
        \end{itemize}

  %
  \item As discussed in \cref{subsubsec:rw-gen-rec}, on its own, the recursion
        principle provided by initiality is insufficient for simply-typed
        languages, as it is restricted to languages with the same type system.
        %
        A solution was proposed in \cite{ExtendedInitiality12} by integrating
        the type system directly into the framework.
        %
        However, this requires reworking and specializing the entire framework
        for simply-typed languages, which seems unsuitable for our purposes.
        %
        How this issue should be solved remains an open question.

  %
  \item Though many frameworks linked to this work have been formalized
        \cite{HssRevisited15,PresentableSignatures21,HssTypedUnimath22,FioreSzamozvancevPopl22,HssNonWellfounded24},
        this framework has not been formalized yet.
        %
        A complete formalization should be possible and strengthen the trust in
        this framework, which can be unsettling due to its many definitions.
\end{itemize}



%
%
%
%
