\pdfoutput=1

\documentclass[a4paper,11pt]{article}


% Import files
\usepackage{import}

% Format
\usepackage[utf8]{inputenc}
\usepackage[top=2cm, bottom=2cm, left=2.5cm, right=2.5cm]{geometry}
\setcounter{tocdepth}{2} % depth table of content
\usepackage{hyperref} % for links
\usepackage{enumitem} % for listes
\usepackage{blindtext} % lorem ipsum
\usepackage{comment}
\usepackage{fancyvrb} % verbatim
\usepackage{caption}
\usepackage{subcaption}
\usepackage{float} % position figures
\hyphenation{every-where endo-func-tor endo-func-tors}
\renewcommand{\,}{\hspace{1pt}}
\renewcommand{\_}{\ul{\hspace{7pt}}}

% To have line numbers
%\usepackage[]{lineno}

% \makeatletter
% \def\makeLineNumberLeft{%
%   \linenumberfont\llap{\hb@xt@\linenumberwidth{\LineNumber\hss}\hskip\linenumbersep}% left line number
%   \hskip\columnwidth% skip over column of text
%   \rlap{\hskip\linenumbersep\hb@xt@\linenumberwidth{\hss\LineNumber}}\hss}% right line number
% \leftlinenumbers% Re-issue [left] option
% \makeatother

% \renewcommand{\linenumberfont}{\tiny\sffamily\color{red}}

% \linenumbers


% Draw Objects
\usepackage{tikz-cd} % comm diag
\usepackage{forest} % tree structures

% Package de maths
\usepackage{amsmath} % matrices
\usepackage{amssymb} % pour mathfrak
\usepackage{stmaryrd} % rrbracket
\usepackage{cmll}

% Environnement for theorems
\usepackage{amsthm}
\usepackage[capitalize]{cleveref}
\usepackage{thmtools}
\usepackage{thm-restate}
\declaretheorem[numberwithin=section]{theorem}
\declaretheorem[sibling=theorem]{corollary}
\declaretheorem[sibling=theorem]{proposition}
\declaretheorem[sibling=theorem]{conjecture}
\declaretheorem[sibling=theorem]{lemma}
\declaretheorem[sibling=theorem, style=definition]{example}
\declaretheorem[sibling=theorem, style=definition]{definition}
\declaretheorem[sibling=theorem, style=definition]{remark}
\declaretheorem[sibling=theorem, style=definition]{related Work}


\usepackage{ebutf8}

% Todo
\newcommand{\todo}{\textcolor{red}{\textbf{todo}}\PackageWarning{TODO}{TODO}}
\newcommand{\todot}[1]{\textcolor{red}{\textbf{todo : #1}}\PackageWarning{TODO}{TODO: #1}}
\newcommand{\BA}[1]{\textcolor{blue}{\textbf{BA : #1}}\PackageWarning{TODO}{BA: #1}}
\newcommand{\TL}[1]{\textcolor{blue}{\textbf{TL : #1}}\PackageWarning{TODO}{BA: #1}}
\newcommand{\tocite}[1]{\textcolor{red}{\textbf{to cite : #1}}\PackageWarning{TODO}{CITE: #1}}
\newcommand{\tcb}[1]{\textcolor{blue}{#1}\PackageWarning{TODO}{todo: #1}}
\newcommand{\fillenv}[1][]{
    \; \\
    \vspace{-3pt}
    \begin{center}
        \textcolor{red}{\textbf{todo: #1}}
    \end{center}
    \vspace{5pt}
}

% Bibliography
\usepackage[style=numeric, maxnames=10]{biblatex}
% General
\bibliography{../Biblio/category_theory}
\bibliography{../Biblio/proof_assistants}
% Initial Semantics
\bibliography{../Biblio/nested_datatypes}
\bibliography{../Biblio/modules_over_monads}
\bibliography{../Biblio/sigma-monoids}
\bibliography{../Biblio/hss}
% Else
\bibliography{../Biblio/unclassified}



%%% Shortcuts

% Shortcut Style
\newcommand{\mc}[1]{\mathcal{#1}}
\newcommand{\mb}[1]{\mathbb{#1}}
\newcommand{\trm}[1]{\textrm{#1}}
\newcommand{\mrm}[1]{\mathrm{#1}}

% Shortcut Operators
% \newcommand{\norm}[1]{\left\Vert #1 \right\Vert}
% \newcommand{\card}[1]{\ensuremath{\left\|#1 \right\|}}
% \newcommand{\interval}[2]{\ensuremath{\llbracket #1, \; #2 \rrbracket}}
% \newcommand{\set}[1]{\{ #1 \}}
\newcommand{\br}[1]{\ensuremath{\llbracket #1 \rrbracket}} %Interpretation
% \newcommand{\floor}[1]{\lfloor #1 \rfloor}
% \newcommand{\ceil}[1]{\lceil #1 \rceil}
\newcommand{\ol}[1]{\overline{#1}}
\newcommand{\ul}[1]{\underline{#1}}
\usepackage{scalerel}
\DeclareMathOperator*{\bigplus}{\scalerel*{+}{\sum}}

% Shortcut Sets
\newcommand{\N}{\mathbb{N}}

% Shortcut Category Theory
% categories
\newcommand{\Cmon}{(\mc{C}, \otimes, I, \alpha, \lambda, \rho)}
\newcommand{\Set}{\mrm{Set}}
\newcommand{\Mon}{\mrm{Mon}}
\newcommand{\Mod}{\mrm{Mod}}
\newcommand{\Model}{\mrm{Model}}
\newcommand{\Sig}{\mrm{Sig}}
\newcommand{\SigStrength}{\mrm{SigStrength}}
\newcommand{\TotMonMod}[1]{\int_{R : \Mon(#1)} \Mod(R)}
\newcommand{\TotSigModel}[1]{\int_{\Sigma : \Sig(#1)} \Model(\Sigma)}
% categorical notions
\newcommand{\op}{\mrm{op}}
\newcommand{\Id}{\mrm{Id}}
\newcommand{\id}{\mrm{id}}
\newcommand{\Lan}[2]{\mrm{Lan}_{#1}(#2)}

% Shortcut Datatypes
\newcommand{\List}{\mrm{List}}
\newcommand{\PCF}{\mrm{PCF}}
\newcommand{\Nat}{\mrm{Nat}}
\newcommand{\Bool}{\mrm{Bool}}
\newcommand{\Free}{\mrm{Free}}
%
\newcommand{\var}{\mrm{var}}
\newcommand{\app}{\mrm{app}}
\newcommand{\abs}{\mrm{abs}}
\newcommand{\inl}{\mrm{inl}}
\newcommand{\inr}{\mrm{inr}}
\newcommand{\swap}{\mrm{swap}}

\DeclareMathOperator{\colim}{colim}
% \DeclareMathOperator{\lim}{lim}



%%%%%%%%%%%%%%%%
%%% document %%%
%%%%%%%%%%%%%%%%

\title{A Unified Framework for Initial Semantics}
\title{A Unified Framework for Initial Semantics}
\author{Thomas Lamiaux \and Benedikt Ahrens}
\date{}

\begin{document}

\maketitle

\begin{abstract}

  % In this work, we consider initial semantics, that characterizes
  % the syntax of programming languages with variable binding, with its
  % substitution structure, as an initial object in a suitable category.

  % Intro IS / Which IS
  Initial semantics aims to capture inductive structures and their properties as initial
  objects in suitable categories.
  %
  We focus on the initial semantics aiming to model the syntax and substitution
  structure of programming languages with variable binding as initial objects.
  %
  % Three approaches
  Three distinct yet similar approaches to initial semantics have been proposed.
  %
  An initial semantics result was first proved by Fiore, Plotkin, and Turi using
  $Σ$-monoids in their seminal paper published at LICS'99.
  %
  Alternative frameworks were later introduced by Hirschowitz and Maggesi using
  modules over monads, and by Matthes and Uustalu using heterogeneous substitution systems.
  %
  Since then, all approaches have been significantly developed by numerous researchers.
  % Links are unclear
  While similar, the links between this different approaches remain unclear.
  This is especially the case as the literature is difficult to access, since it
  was mostly published in (short) conference papers without proofs, and contains many
  technical variations and evolutions.

  In this work, we introduce a framework based on monoidal categories that
  unifies these three distinct approaches to initial semantics, by suitably
  generalizing and combining them.
  %
  Doing so we show that modules over monoids provide an abstract and easy to
  manipulate framework, that $Σ$-monoids and strengths naturally arise when stating
  and proving an initiality theorem, and that heterogeneous substitution systems
  enable us to prove the initiality theorem modularly.
  %
  Moreover, to clarify the literature, we provide an extensive overview of related work
  using our framework as a cornerstone to explain the links between the
  different approaches and their variations.

\end{abstract}
\newpage

\tableofcontents
\newpage

\import{Sections/}{all.tex}



\section{Acknowledgements}

We thank Ambroise Lafont for fruitful discussion on initiality theorems and their different proofs.
%
We thank Ralph Matthes for helpful discussions on heterogeneous substitution
systems and their use to prove the initiality theorem.
%
We furthermore thank Marcelo Fiore for providing pointers to literature on the topic of initial semantics.
We are grateful to André Hirschowitz, Thea Li, Ambroise Lafont, and Ralph Matthes for valuable comments on early drafts.

\addcontentsline{toc}{section}{References}
\printbibliography

\end{document}
%%% Local Variables:
%%% mode: latex
%%% TeX-master: t
%%% End:
