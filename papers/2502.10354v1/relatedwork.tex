\section{Related Works}
\label{subsec:related_works}
\paragraph{Score Matching and Diffusion Models:} Score Matching was introduced in the context of statistical estimation in \cite{hyvarinen2005estimation} with an algorithm now called Implicit Score Matching (ISM). Diffusion models are trained using Denoising Score Matching (DSM) introduced in \cite{vincent2011connection}, and is based on Tweedie's formula. Several algorithms have been introduced since, such as Sliced Score Matching \cite{song2020sliced} and Target Score Matching \cite{de2024target}.

The complexity of Denoising Score Matching has been analyzed in various settings \cite{chen2023score,oko2023diffusion,gupta2023sample,block2020generative} in prior works. We consider the setting in \cite{gupta2023sample,block2020generative}, where the score functions can be accurately approximated by a function approximator class (such as neural networks). These bounds can then be used with the discretization analyses such as those presented in \cite{benton2024nearly,chen2022sampling,lee2023convergence} to theoretically analyze the quality of samples generated by the model. 

\paragraph{Learning from dependent data:} Learning with data from a markov trajectory has been explored in literature in the context of system identification, time series forecasting and reinforcement learning \cite{duchi2012ergodic, simchowitz2018learning,nagaraj2020least,kowshik2021near,tu2024learning,ziemann2022learning,bhandari2018finite,kumar2024streaming,srikant2024rates}
% \ps{Add Duchi/aggarwal here?} \syamantak{Done.}
Many of these works analyze the rates of convergence with data derived from a mixing Markov chain, when the number of data points available is much higher than the mixing time, $\tau_{\mathsf{mix}}$. In our context, the Markov chain contains $\tilde{O}(\tau_{\mathsf{mix}})$ data points created by progressively noising samples from the target distributions, where $\tilde{O}$ hides logarithmic factors. This is similar to the setting in \cite{tu2024learning}, which considered linear regression and linear system identification.

We outline our paper as follows: Section~\ref{sec:problemsetup} introduces the problem setup and preliminaries, followed by the main results and a comparison with prior work in Section~\ref{sec:main_results}. Section~\ref{sec:technical_results} presents key technical results from our proof technique. Finally, Section~\ref{sec:bootstrapped_score_matching} introduces Bootstrapped Score Matching, a novel training method that shares information explicitly across time by modifying the learning objective.