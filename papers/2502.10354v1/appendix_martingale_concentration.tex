\section{Martingale Concentration}
\label{appendix:martingale_concentration}

\begin{lemma}\label{lemma:subgauss_moment_bound}
    Let $Y$ be a $\bb{\beta^2,K}$-subGaussian random variable following definition~\ref{definition:new_subGaussian_def}, with $\bb{K \geq 1}$. Then, for any integer $k > 0$ and some universal constant $C > 0$:
    \bas{\bE \bbb{Y^{2k}} \leq C^k K^k \beta^{2k} + C^k k! \beta^{2k}}
\end{lemma}
\begin{proof}
By Definition~\ref{definition:new_subGaussian_def}, for any \(A>0\),
\[
  \mathbb{P}(|Y|> A)\;\le\; e^K \exp\Bigl(-\tfrac{A^2}{2\beta^2}\Bigr).
\]
Using the tail-integration representation of moments, we have
\[
  \mathbb{E}[|Y|^{2k}]
  \;=\;\int_{0}^{\infty} \mathbb{P}\bigl(|Y|^{2k} > t\bigr)\,dt
  \;=\;\int_{0}^{\infty} \mathbb{P}\bigl(|Y| > t^{1/(2k)}\bigr)\,dt.
\]
Make the change of variables \( t = x^{2k}\) so that \(dt = 2k\,x^{2k-1}dx\).  Then
\[
  \mathbb{E}[|Y|^{2k}]
  \;=\;\int_{0}^{\infty} 2k\,x^{2k-1}\,\mathbb{P}(|Y| > x)\,dx
  \;\le\;
  2k\,\int_{0}^{\infty} x^{2k-1}\,\min(1,e^K\exp\!\bigl(-\tfrac{x^2}{2\beta^2}\bigr))\,dx.
\]

Let $x_0 = \sqrt{2\beta^2K}$

\bas{
\E[|Y|^{2k}] &\leq 2k\int_{0}^{x_0}x^{2k-1}dx + 2k\int_{x_0}^{\infty}x^{2k-1}e^{K}e^{-\frac{x^2}{2\beta^2}} dx \\
&= (2\beta^2K)^k + 2k\int_{x_0}^{\infty}x^{2k-1}e^{K}e^{-\frac{x^2}{2\beta^2}} dx \\
&\leq (2\beta^2K)^k + 2k\int_{x_0}^{\infty}x^{2k-1}e^{-\frac{(x-x_0)^2}{2\beta^2}} dx \\
&\leq (2\beta^2K)^k + 2^{2k-1}k\int_{x_0}^{\infty}(x_0^{2k-1}+(x-x_0)^{2k-1})e^{-\frac{(x-x_0)^2}{2\beta^2}} dx 
}

In the second step we have used the fact that whenever $x \geq x_0$, we must have $K-\frac{x^2}{2\beta^2} \leq -\frac{(x-x_0)^2}{2\beta^2}$. In the third step we have used the fact that $x^{2k-1} \leq 2^{2k-2}[(x-x_0)^{2k-1} + x_0^{2k-1}]$ whenever $x \geq x_0$.

A standard Gamma-function integral yields
\[
  \int_{0}^{\infty} x^{2k-1}\,\exp\!\Bigl(-\tfrac{x^2}{2\beta^2}\Bigr)\,dx
  \;=\;\frac12\,(2\beta^2)^k\,\Gamma(k),
\]
and for integer \(k\), \(\Gamma(k)= (k-1)!\).  Substituting this to the equation above, we conclude that for some universal constant $C_1$, we have:

\bas{
\E[|Y|^{2k}] &\leq (2\beta^2K)^k + C^k_1 (k\beta^{2k}K^{k-1/2} + \beta^{2k}k!)
}
We then conclude the result using the fact that $K \geq 1$ and $k \leq 2^k$.

\end{proof}


\begin{lemma}
    \label{lemma:mgf_bound}
    Let $Y$ be a $\bb{\beta^{2}, K}$-subGaussian random variable following definition~\ref{definition:new_subGaussian_def}, such that $K\geq 1$, $\E\bbb{Y} = 0$ and $\E\bbb{Y^2} \leq \nu^2$. Then, for a sufficiently small universal constant $c_{0} > 0$ such that, $\lambda \beta \leq c_0$, and any arbitrary $A > 0$, we have:
    \bas{
        \bE \exp(\lambda^2 Y^2) \leq 1 + \lambda^2 \nu^2 \exp(\lambda^2 A^2) + C\lambda^4\beta^4K^2\exp(\tfrac{K}{2}-\tfrac{A^2}{4\beta^2} + C\lambda^2\beta^2K)
    }
\end{lemma}
\begin{proof}
For some $\lambda > 0$, consider:

\begin{equation}\label{eq:exp_moment}
    \bE\bbb{\exp(\lambda^2 Y^2)} = 1 + \lambda^2 \nu^2 + \sum_{k\geq 2} \frac{\lambda^{2k}\bE \bbb{Y^{2k}}}{k!}
\end{equation}

Now, using Lemma~\ref{lemma:subgauss_moment_bound},  consider 
\begin{align}
    \bE \bbb{Y^{2k}} &= \bE \bbb{Y^{2k}\mathbbm{1}(|Y| > A)} + \bE\bbb{Y^{2k}\mathbbm{1}(|Y| \leq A)} \nonumber \\
&\leq \sqrt{\bE \bbb{Y^{4k}}}\sqrt{\bP(|Y| > A)} + \bE\bbb{Y^2} A^{2k-2} \nonumber \\
&= \sqrt{\bE \bbb{Y^{4k}}}\sqrt{\bP(|Y| > A)} + \nu^2 A^{2k-2} \nonumber \\
&\leq \sqrt{C^{2k}\beta^{4k}(2k)! + C^{2k}\beta^{4k}K^{2k}}\exp(\tfrac{K}{2}-\tfrac{A^2}{4\beta^2}) + \nu^2 A^{2k-2} \nonumber \\
&\leq \left((2C)^{k}k!\beta^{2k} + C^{k}\beta^{2k}K^k\right)\exp(\tfrac{K}{2}-\tfrac{A^2}{4\beta^2}) + \nu^2 A^{2k-2}
\end{align} 

Here, we have used the fact that $(2k)! \leq 4^k (k!)^2$. Plugging this back in Equation~\eqref{eq:exp_moment}, we conclude that whenever $\lambda\beta \leq c_0 $ for some small enough constant $c_0$, we have:
\begin{equation}
    \bE\bbb{\exp(\lambda^2 Y^2)} \leq 1 + \lambda^2 \nu^2 \exp(\lambda^2 A^2) + C\lambda^4\beta^4K^2\exp(\tfrac{K}{2}-\tfrac{A^2}{4\beta^2} + C\lambda^2\beta^2K)
\end{equation}
\end{proof}
\begin{theorem}\label{theorem:mgf_bound_subgauss}
      Let $Y$ be a $\bb{\beta^{2}, K}$-subGaussian random variable following definition~\ref{definition:new_subGaussian_def}, such that $K\geq 1$, $\E\bbb{Y} = 0$ and $\E\bbb{Y^{2}} \leq \nu^{2}$. Set $A \geq \beta \sqrt{4\log(\tfrac{\beta K}{\nu})} + \beta\sqrt{2 K}$ and $\lambda \leq \frac{c_0}{A}$ for some small enough constant $c_0 > 0$. Then, there exists a constant $C$ such that:
      \bas{
        \bE\bbb{\exp(\lambda^2 Y^2)} \leq 1 + C\lambda^2 \nu^2
      }
\end{theorem}
\begin{proof}

   The result follows from Lemma~\ref{lemma:mgf_bound} substituting the values of $\lambda$ and $A$. 
\end{proof}

\begin{lemma}
    \label{lemma:martingale_concentration_lemma_appendix}
Let $M_{n} = \sum_{i=1}^{n}\langle G_i, Y_i - \bE[Y_i|\mathcal{F}_{i-1}] \rangle, M_{0} = 1$ and define the filtration $\left\{\mathcal{F}_{i}\right\}_{i \in [n]}$ such that:
\begin{enumerate}
    \item  $G_i$ is $\mathcal{F}_{i-1}$ measurable.
    \item  $\langle G_i,Y_i-\bE [Y_i|\mathcal{F}_{i-1}]\rangle$ is $(\beta_i^{2}\|G_i\|^{2}, K_i)$ sub-Gaussian conditioned on $\mathcal{F}_{i-1}$ (where $\beta_i,K_i$ are random variables measurable with respect to $\mathcal{F}_{i-1}$)
    \item $\mathsf{var}(\langle G_i, Y_i - \bE[Y_i|\mathcal{F}_{i-1}]\rangle|\mathcal{F}_{i-1}) \leq \nu^2_i \|G_i\|^2$ and define $J_i := \max(1,\tfrac{1}{K_i}\log\frac{\beta_i^2 K_i}{\nu^2_i})$.
\end{enumerate}
Pick a $\lambda > 0$ and let $\mathcal{A}_i(\lambda) = \{\lambda J_i\|G_i\|\beta_i\sqrt{K_i} \leq c_0\}$ for some small enough universal constant $c_0$. Then, there exists a universal constant $C > 0$ such that:
\begin{enumerate}
    \item $\exp(\lambda M_{n} - C\lambda^2\sum_{i=1}^{n}\nu_i^2\|G_i\|^2 )\prod_{i=1}^{n}\mathbbm{1}(\mathcal{A}_i(\lambda))$ is a super-martingale with respect to the filtration $\mathcal{F}_{i}$
    \item $\forall \alpha > 0, \;\; \bP(\{\lambda M_n > C \lambda^2\sum_{i=1}^{n}\nu_i^2\|G_i\|^2 + \alpha\}\cap_{i=1}^{n} \mathcal{A}_i(\lambda)) \leq \exp(-\alpha)$
\end{enumerate}
\end{lemma}
\begin{proof}
    Let $L_{n} := \exp(\lambda M_{n} - C\lambda^2\sum_{i=1}^{n}\nu_i^2\|G_i\|^2 )\prod_{i=1}^{n}\mathbbm{1}(\mathcal{A}_i(\lambda))$. Then we have, 
    \bas{
        &\E\bbb{L_{n}\bigg|\mathcal{F}_{n-1}}
        = L_{n-1}\E\bbb{\exp\bb{\lambda\langle G_n, Y_n - \bE[Y_n|\mathcal{F}_{n-1}] \rangle - C\lambda^{2}\nu_{n}^{2}\norm{G_n}^{2} }\mathbbm{1}\bb{\mathcal{A}_n\bb{\lambda}}\bigg|\mathcal{F}_{n-1}} \\
        &= L_{n-1}\exp\bb{- C\lambda^{2}\nu_{n}^{2}\norm{G_n}^{2}}\E\bbb{\exp\bb{\lambda\langle G_n, Y_n - \bE[Y_n|\mathcal{F}_{n-1}] \rangle }\mathbbm{1}\bb{\{J_n\lambda \|G_n\|\beta_n\sqrt{K_n} \leq c_0\}}\bigg|\mathcal{F}_{n-1}} \\
        &\leq L_{n-1}\exp\bb{- C\lambda^{2}\nu_{n}^{2}\norm{G_n}^{2}}\exp\bb{C\lambda^{2}\nu_{n}^{2}\norm{G_n}^{2}} \text{ using Theorem}~\ref{theorem:mgf_bound_subgauss} \text{ and the definition of } \mathcal{A}_{n}\bb{\lambda} \\
        &\leq L_{n-1}
    }
    The second result follows from a standard Chernoff bound argument.
\end{proof}

\begin{lemma}\label{lemma:union_bound_lambda} 
     Under the setting of Lemma~\ref{lemma:martingale_concentration_lemma_appendix}, let $\lambda^* := \sqrt{\frac{\alpha}{\sum_{i=1}^{n}\nu_i^2\|G_i\|^2}}$ and $\lambda_{\min} := \frac{c_0}{\sup_iJ_i\|G_i\|\beta_i\sqrt{K_i}}$. Let $B \in \mathbb{N}$ be arbitrary and consider the event: $\mathcal{B} = \{e^{-B} \leq \min(\lambda^*,\lambda_{\min}) \leq \max(\lambda^{*},\lambda_{\min}) \leq e^{B}\}$. Then, for some universal constant $C_1 > 0$ and any $\alpha > 0$,

 $$\mathbb{P}\left(\{M_n > C_1\lambda^*\sum_{i=1}^{n}\nu_i^2 \|G_i\|^2 + C_1\tfrac{\alpha}{\lambda_{\min}}\}\cap\mathcal{B} \right) \leq  (2B+1)e^{-\alpha}$$ 
    
\end{lemma}
\begin{proof}
We apply union bound over $\lambda \in \Lambda_B := \{e^{-B},e^{-B+1},\dots,e^{B}\}$. Using Lemma~\ref{lemma:martingale_concentration_lemma_appendix} along with a union bound,
\bas{
    \bP(\cup_{\lambda \in \Lambda_B}\{\lambda M_n > C \lambda^2\sum_{i=1}^{n}\nu_i^2\|G_i\|^2 + \alpha\}\cap_{i=1}^{n} \mathcal{A}_i(\lambda)) \leq (2B+1)\exp(-\alpha)
}
Consider the following events:
\begin{enumerate}
    \item Event 1: $\cE_1 := \{\max(\lambda^*,\lambda_{\min}) > e^B\}$
    \item Event 2: $\cE_2 := \{\min(\lambda^*,\lambda_{\min}) < e^{-B}\}$
    \item Event 3: $\cE_3 := \{e^{-B} \leq \lambda^* < \lambda_{\min} \leq e^B\}$
    \item Event 4: $\cE_4 := \{e^{-B} \leq \lambda_{\min} < \lambda^* \leq e^B\}$
\end{enumerate}

%Now, let us consider the probability of the event $\{M_n > C_1\lambda^*\sum_{i=1}^n\nu_i^2 \|G_i\|^2 + C_1\frac{\alpha}{\lambda_{\min}}\}$ for some large enough constant $C_1$.
In the event $\cE_4$, almost surely there exists a random $\bar{\lambda} \in \Lambda_B$ such that $\bar{\lambda}/\lambda_{\min} \in [\frac{1}{e},e]$ and such that the event $\cap_{i=1}^{n}\mathcal{A}_i(\bar{\lambda})$ holds. Thus, we have:

\begin{align}
&\{M_n > Ce\lambda^* \sum_i \nu_i^2\|G_i\|^2 + \frac{e\alpha}{\lambda_{\min}}\}\cap \cE_4 \subseteq  \{M_n > Ce \lambda_{\min}\sum_i \nu_i^2 \|G_i\|^2 + \frac{e\alpha}{\lambda_{\min}}\}\cap \cE_4 \nonumber \\
&\subseteq \{M_n > C\bar{\lambda}\sum_i \nu_i^2 \|G_i\|^2 + \frac{\alpha}{\bar{\lambda}}\}\cap \cE_4 = \{M_n > C \bar{\lambda}\sum_i \nu_i^2 \|G_i\|^2 + \frac{\alpha}{\bar{\lambda}}\}\cap \cE_4 \cap_{i=1}^n \cA_i(\bar{\lambda}) \nonumber \\
&\subseteq \cE_4\cap\left(\cup_{\lambda \in \Lambda_B}\{\lambda M_n > C \lambda^2\sum_{i=1}^{n}\nu_i^2\|G_i\|^2 + \alpha\}\cap_{i=1}^{n} \mathcal{A}_i(\lambda)\right)
\end{align}

Similarly, under the event $\cE_3$, there exists a random $\bar{\lambda}^*\in \Lambda_B$ such that: $\bar{\lambda}^*/\lambda^* \in [\frac{1}{e},e]$, such that the event $\cap_i \cA_i(\bar{\lambda}^*)$ holds. Therefore, we must have:

\begin{align}
&\{M_n > Ce\lambda^* \sum_i \nu_i^2\|G_i\|^2 + \frac{e\alpha}{\lambda^*}\}\cap \cE_3 \subseteq \{M_n > C \bar{\lambda}^*\sum_i \nu_i^2 \|G_i\|^2 + \frac{\alpha}{\bar{\lambda}^*}\}\cap \cE_3 \nonumber \\ &= \{M_n > C \bar{\lambda}^*\sum_i \nu_i^2 \|G_i\|^2 + \frac{\alpha}{\bar{\lambda}^*}\}\cap \cE_3 \cap_{i=1}^n \cA_i(\bar{\lambda}^*) \nonumber \\
&\subseteq \cE_3\cap\left(\cup_{\lambda \in \Lambda_B}\{\lambda M_n > C \lambda^2\sum_{i=1}^{n}\nu_i^2\|G_i\|^2 + \alpha\}\cap_{i=1}^{n} \mathcal{A}_i(\lambda)\right)
\end{align}

Notice that $\lambda^*$ is chosen such that 

\begin{align}
Ce\lambda^* \sum_i \nu_i^2\|G_i\|^2 + \frac{e\alpha}{\lambda^*} &= e(C+1)\sqrt{\alpha(\sum_i \nu_i^2 \|G_i\|^2)} \nonumber \\
&= e(C+1)\lambda^* \sum_i \nu_i^2 \|G_i\|^2 \\
&\leq e(C+1)\lambda^* \sum_i \nu_i^2 \|G_i\|^2 + \frac{e \alpha}{\lambda_{\min}}
\end{align}


Combining these equations, we conclude that for some constant $C_1 > 0$, we must have
$$\{M_n > C_1(\lambda^*\sum_{i=1}^{n}\nu_i^2 \|G_i\|^2 + \frac{\alpha}{\lambda_{\min}}) \}\cap(\cE_3\cup\cE_4) \subseteq \left(\cup_{\lambda \in \Lambda_B}\{\lambda M_n > C \lambda^2\sum_{i=1}^{n}\nu_i^2\|G_i\|^2 + \alpha\}\cap_{i=1}^{n} \mathcal{A}_i(\lambda)\right)\cap(\cE_3\cup \cE_4) $$

Noting that $\mathcal{B} = \cE_3\cup\cE_4$, we conclude the result.
 
% The event $\cE_1$ is such that for large enough $B$, $\|G_i\|$ is already very small. This forms the second result of the lemma. The event $\cE_2$ is such that $\|G_i\|$ is very large, which we will show is not possible using the bounds on $\norm{G_{i}}$.

% Note that under $\mathcal{E}_{2}$, 
% \bas{
%     \max\left\{\sqrt{\frac{\sum_{i}\nu_i^{2}\norm{G_i}^{2}}{\alpha}}, \frac{\sup_i\|G_i\|\beta_i\sqrt{K_i}}{c_0} \right\} \geq e^{B}
% }
% which contradicts our assumption on $\norm{G_i}$.

% Assume that $\|f(X_t)-f(0)\| \leq 2L\|X_t\|$ and $\|f(0)\| \leq C\poly(d)$. When $X_t$ has second moment, then..
\end{proof}
