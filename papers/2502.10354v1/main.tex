\pdfoutput = 1
\documentclass[11pt]{article}

%
\setlength\unitlength{1mm}
\newcommand{\twodots}{\mathinner {\ldotp \ldotp}}
% bb font symbols
\newcommand{\Rho}{\mathrm{P}}
\newcommand{\Tau}{\mathrm{T}}

\newfont{\bbb}{msbm10 scaled 700}
\newcommand{\CCC}{\mbox{\bbb C}}

\newfont{\bb}{msbm10 scaled 1100}
\newcommand{\CC}{\mbox{\bb C}}
\newcommand{\PP}{\mbox{\bb P}}
\newcommand{\RR}{\mbox{\bb R}}
\newcommand{\QQ}{\mbox{\bb Q}}
\newcommand{\ZZ}{\mbox{\bb Z}}
\newcommand{\FF}{\mbox{\bb F}}
\newcommand{\GG}{\mbox{\bb G}}
\newcommand{\EE}{\mbox{\bb E}}
\newcommand{\NN}{\mbox{\bb N}}
\newcommand{\KK}{\mbox{\bb K}}
\newcommand{\HH}{\mbox{\bb H}}
\newcommand{\SSS}{\mbox{\bb S}}
\newcommand{\UU}{\mbox{\bb U}}
\newcommand{\VV}{\mbox{\bb V}}


\newcommand{\yy}{\mathbbm{y}}
\newcommand{\xx}{\mathbbm{x}}
\newcommand{\zz}{\mathbbm{z}}
\newcommand{\sss}{\mathbbm{s}}
\newcommand{\rr}{\mathbbm{r}}
\newcommand{\pp}{\mathbbm{p}}
\newcommand{\qq}{\mathbbm{q}}
\newcommand{\ww}{\mathbbm{w}}
\newcommand{\hh}{\mathbbm{h}}
\newcommand{\vvv}{\mathbbm{v}}

% Vectors

\newcommand{\av}{{\bf a}}
\newcommand{\bv}{{\bf b}}
\newcommand{\cv}{{\bf c}}
\newcommand{\dv}{{\bf d}}
\newcommand{\ev}{{\bf e}}
\newcommand{\fv}{{\bf f}}
\newcommand{\gv}{{\bf g}}
\newcommand{\hv}{{\bf h}}
\newcommand{\iv}{{\bf i}}
\newcommand{\jv}{{\bf j}}
\newcommand{\kv}{{\bf k}}
\newcommand{\lv}{{\bf l}}
\newcommand{\mv}{{\bf m}}
\newcommand{\nv}{{\bf n}}
\newcommand{\ov}{{\bf o}}
\newcommand{\pv}{{\bf p}}
\newcommand{\qv}{{\bf q}}
\newcommand{\rv}{{\bf r}}
\newcommand{\sv}{{\bf s}}
\newcommand{\tv}{{\bf t}}
\newcommand{\uv}{{\bf u}}
\newcommand{\wv}{{\bf w}}
\newcommand{\vv}{{\bf v}}
\newcommand{\xv}{{\bf x}}
\newcommand{\yv}{{\bf y}}
\newcommand{\zv}{{\bf z}}
\newcommand{\zerov}{{\bf 0}}
\newcommand{\onev}{{\bf 1}}

% Matrices

\newcommand{\Am}{{\bf A}}
\newcommand{\Bm}{{\bf B}}
\newcommand{\Cm}{{\bf C}}
\newcommand{\Dm}{{\bf D}}
\newcommand{\Em}{{\bf E}}
\newcommand{\Fm}{{\bf F}}
\newcommand{\Gm}{{\bf G}}
\newcommand{\Hm}{{\bf H}}
\newcommand{\Id}{{\bf I}}
\newcommand{\Jm}{{\bf J}}
\newcommand{\Km}{{\bf K}}
\newcommand{\Lm}{{\bf L}}
\newcommand{\Mm}{{\bf M}}
\newcommand{\Nm}{{\bf N}}
\newcommand{\Om}{{\bf O}}
\newcommand{\Pm}{{\bf P}}
\newcommand{\Qm}{{\bf Q}}
\newcommand{\Rm}{{\bf R}}
\newcommand{\Sm}{{\bf S}}
\newcommand{\Tm}{{\bf T}}
\newcommand{\Um}{{\bf U}}
\newcommand{\Wm}{{\bf W}}
\newcommand{\Vm}{{\bf V}}
\newcommand{\Xm}{{\bf X}}
\newcommand{\Ym}{{\bf Y}}
\newcommand{\Zm}{{\bf Z}}

% Calligraphic

\newcommand{\Ac}{{\cal A}}
\newcommand{\Bc}{{\cal B}}
\newcommand{\Cc}{{\cal C}}
\newcommand{\Dc}{{\cal D}}
\newcommand{\Ec}{{\cal E}}
\newcommand{\Fc}{{\cal F}}
\newcommand{\Gc}{{\cal G}}
\newcommand{\Hc}{{\cal H}}
\newcommand{\Ic}{{\cal I}}
\newcommand{\Jc}{{\cal J}}
\newcommand{\Kc}{{\cal K}}
\newcommand{\Lc}{{\cal L}}
\newcommand{\Mc}{{\cal M}}
\newcommand{\Nc}{{\cal N}}
\newcommand{\nc}{{\cal n}}
\newcommand{\Oc}{{\cal O}}
\newcommand{\Pc}{{\cal P}}
\newcommand{\Qc}{{\cal Q}}
\newcommand{\Rc}{{\cal R}}
\newcommand{\Sc}{{\cal S}}
\newcommand{\Tc}{{\cal T}}
\newcommand{\Uc}{{\cal U}}
\newcommand{\Wc}{{\cal W}}
\newcommand{\Vc}{{\cal V}}
\newcommand{\Xc}{{\cal X}}
\newcommand{\Yc}{{\cal Y}}
\newcommand{\Zc}{{\cal Z}}

% Bold greek letters

\newcommand{\alphav}{\hbox{\boldmath$\alpha$}}
\newcommand{\betav}{\hbox{\boldmath$\beta$}}
\newcommand{\gammav}{\hbox{\boldmath$\gamma$}}
\newcommand{\deltav}{\hbox{\boldmath$\delta$}}
\newcommand{\etav}{\hbox{\boldmath$\eta$}}
\newcommand{\lambdav}{\hbox{\boldmath$\lambda$}}
\newcommand{\epsilonv}{\hbox{\boldmath$\epsilon$}}
\newcommand{\nuv}{\hbox{\boldmath$\nu$}}
\newcommand{\muv}{\hbox{\boldmath$\mu$}}
\newcommand{\zetav}{\hbox{\boldmath$\zeta$}}
\newcommand{\phiv}{\hbox{\boldmath$\phi$}}
\newcommand{\psiv}{\hbox{\boldmath$\psi$}}
\newcommand{\thetav}{\hbox{\boldmath$\theta$}}
\newcommand{\tauv}{\hbox{\boldmath$\tau$}}
\newcommand{\omegav}{\hbox{\boldmath$\omega$}}
\newcommand{\xiv}{\hbox{\boldmath$\xi$}}
\newcommand{\sigmav}{\hbox{\boldmath$\sigma$}}
\newcommand{\piv}{\hbox{\boldmath$\pi$}}
\newcommand{\rhov}{\hbox{\boldmath$\rho$}}
\newcommand{\upsilonv}{\hbox{\boldmath$\upsilon$}}

\newcommand{\Gammam}{\hbox{\boldmath$\Gamma$}}
\newcommand{\Lambdam}{\hbox{\boldmath$\Lambda$}}
\newcommand{\Deltam}{\hbox{\boldmath$\Delta$}}
\newcommand{\Sigmam}{\hbox{\boldmath$\Sigma$}}
\newcommand{\Phim}{\hbox{\boldmath$\Phi$}}
\newcommand{\Pim}{\hbox{\boldmath$\Pi$}}
\newcommand{\Psim}{\hbox{\boldmath$\Psi$}}
\newcommand{\Thetam}{\hbox{\boldmath$\Theta$}}
\newcommand{\Omegam}{\hbox{\boldmath$\Omega$}}
\newcommand{\Xim}{\hbox{\boldmath$\Xi$}}


% Sans Serif small case

\newcommand{\Gsf}{{\sf G}}

\newcommand{\asf}{{\sf a}}
\newcommand{\bsf}{{\sf b}}
\newcommand{\csf}{{\sf c}}
\newcommand{\dsf}{{\sf d}}
\newcommand{\esf}{{\sf e}}
\newcommand{\fsf}{{\sf f}}
\newcommand{\gsf}{{\sf g}}
\newcommand{\hsf}{{\sf h}}
\newcommand{\isf}{{\sf i}}
\newcommand{\jsf}{{\sf j}}
\newcommand{\ksf}{{\sf k}}
\newcommand{\lsf}{{\sf l}}
\newcommand{\msf}{{\sf m}}
\newcommand{\nsf}{{\sf n}}
\newcommand{\osf}{{\sf o}}
\newcommand{\psf}{{\sf p}}
\newcommand{\qsf}{{\sf q}}
\newcommand{\rsf}{{\sf r}}
\newcommand{\ssf}{{\sf s}}
\newcommand{\tsf}{{\sf t}}
\newcommand{\usf}{{\sf u}}
\newcommand{\wsf}{{\sf w}}
\newcommand{\vsf}{{\sf v}}
\newcommand{\xsf}{{\sf x}}
\newcommand{\ysf}{{\sf y}}
\newcommand{\zsf}{{\sf z}}


% mixed symbols

\newcommand{\sinc}{{\hbox{sinc}}}
\newcommand{\diag}{{\hbox{diag}}}
\renewcommand{\det}{{\hbox{det}}}
\newcommand{\trace}{{\hbox{tr}}}
\newcommand{\sign}{{\hbox{sign}}}
\renewcommand{\arg}{{\hbox{arg}}}
\newcommand{\var}{{\hbox{var}}}
\newcommand{\cov}{{\hbox{cov}}}
\newcommand{\Ei}{{\rm E}_{\rm i}}
\renewcommand{\Re}{{\rm Re}}
\renewcommand{\Im}{{\rm Im}}
\newcommand{\eqdef}{\stackrel{\Delta}{=}}
\newcommand{\defines}{{\,\,\stackrel{\scriptscriptstyle \bigtriangleup}{=}\,\,}}
\newcommand{\<}{\left\langle}
\renewcommand{\>}{\right\rangle}
\newcommand{\herm}{{\sf H}}
\newcommand{\trasp}{{\sf T}}
\newcommand{\transp}{{\sf T}}
\renewcommand{\vec}{{\rm vec}}
\newcommand{\Psf}{{\sf P}}
\newcommand{\SINR}{{\sf SINR}}
\newcommand{\SNR}{{\sf SNR}}
\newcommand{\MMSE}{{\sf MMSE}}
\newcommand{\REF}{{\RED [REF]}}

% Markov chain
\usepackage{stmaryrd} % for \mkv 
\newcommand{\mkv}{-\!\!\!\!\minuso\!\!\!\!-}

% Colors

\newcommand{\RED}{\color[rgb]{1.00,0.10,0.10}}
\newcommand{\BLUE}{\color[rgb]{0,0,0.90}}
\newcommand{\GREEN}{\color[rgb]{0,0.80,0.20}}

%%%%%%%%%%%%%%%%%%%%%%%%%%%%%%%%%%%%%%%%%%
\usepackage{hyperref}
\hypersetup{
    bookmarks=true,         % show bookmarks bar?
    unicode=false,          % non-Latin characters in AcrobatÕs bookmarks
    pdftoolbar=true,        % show AcrobatÕs toolbar?
    pdfmenubar=true,        % show AcrobatÕs menu?
    pdffitwindow=false,     % window fit to page when opened
    pdfstartview={FitH},    % fits the width of the page to the window
%    pdftitle={My title},    % title
%    pdfauthor={Author},     % author
%    pdfsubject={Subject},   % subject of the document
%    pdfcreator={Creator},   % creator of the document
%    pdfproducer={Producer}, % producer of the document
%    pdfkeywords={keyword1} {key2} {key3}, % list of keywords
    pdfnewwindow=true,      % links in new window
    colorlinks=true,       % false: boxed links; true: colored links
    linkcolor=red,          % color of internal links (change box color with linkbordercolor)
    citecolor=green,        % color of links to bibliography
    filecolor=blue,      % color of file links
    urlcolor=blue           % color of external links
}
%%%%%%%%%%%%%%%%%%%%%%%%%%%%%%%%%%%%%%%%%%%


\title{Dimension-free Score Matching and Time Bootstrapping for Diffusion Models}
\author{Syamantak Kumar\thanks{University of Texas at Austin, \texttt{syamantak@utexas.edu} (Work partly done during internship at Google DeepMind)} 
\and Dheeraj Nagaraj\thanks{Google DeepMind, \texttt{dheerajnagaraj@google.com}}
\and Purnamrita Sarkar\thanks{University of Texas at Austin, \texttt{purna.sarkar@utexas.edu}}}

\begin{document}
\maketitle

\begin{abstract}
Diffusion models generate samples by estimating the score function of the target distribution at various noise levels. The model is trained using samples drawn from the target distribution, progressively adding noise. In this work, we establish the first (nearly) dimension-free sample complexity bounds for learning these score functions, achieving a double exponential improvement in dimension over prior results. A key aspect of our analysis is the use of a single function approximator to jointly estimate scores across noise levels, a critical feature of diffusion models in practice which enables generalization across timesteps. Our analysis introduces a novel martingale-based error decomposition and sharp variance bounds, enabling efficient learning from dependent data generated by Markov processes, which may be of independent interest. Building on these insights, we propose Bootstrapped Score Matching (BSM), a variance reduction technique that utilizes previously learned scores to improve accuracy at higher noise levels. These results provide crucial insights into the efficiency and effectiveness of diffusion models for generative modeling.
\end{abstract}
\thispagestyle{empty}
\newpage
\tableofcontents
\thispagestyle{empty}
\newpage

\section{Introduction}
\label{sec:introduction}
The business processes of organizations are experiencing ever-increasing complexity due to the large amount of data, high number of users, and high-tech devices involved \cite{martin2021pmopportunitieschallenges, beerepoot2023biggestbpmproblems}. This complexity may cause business processes to deviate from normal control flow due to unforeseen and disruptive anomalies \cite{adams2023proceddsriftdetection}. These control-flow anomalies manifest as unknown, skipped, and wrongly-ordered activities in the traces of event logs monitored from the execution of business processes \cite{ko2023adsystematicreview}. For the sake of clarity, let us consider an illustrative example of such anomalies. Figure \ref{FP_ANOMALIES} shows a so-called event log footprint, which captures the control flow relations of four activities of a hypothetical event log. In particular, this footprint captures the control-flow relations between activities \texttt{a}, \texttt{b}, \texttt{c} and \texttt{d}. These are the causal ($\rightarrow$) relation, concurrent ($\parallel$) relation, and other ($\#$) relations such as exclusivity or non-local dependency \cite{aalst2022pmhandbook}. In addition, on the right are six traces, of which five exhibit skipped, wrongly-ordered and unknown control-flow anomalies. For example, $\langle$\texttt{a b d}$\rangle$ has a skipped activity, which is \texttt{c}. Because of this skipped activity, the control-flow relation \texttt{b}$\,\#\,$\texttt{d} is violated, since \texttt{d} directly follows \texttt{b} in the anomalous trace.
\begin{figure}[!t]
\centering
\includegraphics[width=0.9\columnwidth]{images/FP_ANOMALIES.png}
\caption{An example event log footprint with six traces, of which five exhibit control-flow anomalies.}
\label{FP_ANOMALIES}
\end{figure}

\subsection{Control-flow anomaly detection}
Control-flow anomaly detection techniques aim to characterize the normal control flow from event logs and verify whether these deviations occur in new event logs \cite{ko2023adsystematicreview}. To develop control-flow anomaly detection techniques, \revision{process mining} has seen widespread adoption owing to process discovery and \revision{conformance checking}. On the one hand, process discovery is a set of algorithms that encode control-flow relations as a set of model elements and constraints according to a given modeling formalism \cite{aalst2022pmhandbook}; hereafter, we refer to the Petri net, a widespread modeling formalism. On the other hand, \revision{conformance checking} is an explainable set of algorithms that allows linking any deviations with the reference Petri net and providing the fitness measure, namely a measure of how much the Petri net fits the new event log \cite{aalst2022pmhandbook}. Many control-flow anomaly detection techniques based on \revision{conformance checking} (hereafter, \revision{conformance checking}-based techniques) use the fitness measure to determine whether an event log is anomalous \cite{bezerra2009pmad, bezerra2013adlogspais, myers2018icsadpm, pecchia2020applicationfailuresanalysispm}. 

The scientific literature also includes many \revision{conformance checking}-independent techniques for control-flow anomaly detection that combine specific types of trace encodings with machine/deep learning \cite{ko2023adsystematicreview, tavares2023pmtraceencoding}. Whereas these techniques are very effective, their explainability is challenging due to both the type of trace encoding employed and the machine/deep learning model used \cite{rawal2022trustworthyaiadvances,li2023explainablead}. Hence, in the following, we focus on the shortcomings of \revision{conformance checking}-based techniques to investigate whether it is possible to support the development of competitive control-flow anomaly detection techniques while maintaining the explainable nature of \revision{conformance checking}.
\begin{figure}[!t]
\centering
\includegraphics[width=\columnwidth]{images/HIGH_LEVEL_VIEW.png}
\caption{A high-level view of the proposed framework for combining \revision{process mining}-based feature extraction with dimensionality reduction for control-flow anomaly detection.}
\label{HIGH_LEVEL_VIEW}
\end{figure}

\subsection{Shortcomings of \revision{conformance checking}-based techniques}
Unfortunately, the detection effectiveness of \revision{conformance checking}-based techniques is affected by noisy data and low-quality Petri nets, which may be due to human errors in the modeling process or representational bias of process discovery algorithms \cite{bezerra2013adlogspais, pecchia2020applicationfailuresanalysispm, aalst2016pm}. Specifically, on the one hand, noisy data may introduce infrequent and deceptive control-flow relations that may result in inconsistent fitness measures, whereas, on the other hand, checking event logs against a low-quality Petri net could lead to an unreliable distribution of fitness measures. Nonetheless, such Petri nets can still be used as references to obtain insightful information for \revision{process mining}-based feature extraction, supporting the development of competitive and explainable \revision{conformance checking}-based techniques for control-flow anomaly detection despite the problems above. For example, a few works outline that token-based \revision{conformance checking} can be used for \revision{process mining}-based feature extraction to build tabular data and develop effective \revision{conformance checking}-based techniques for control-flow anomaly detection \cite{singh2022lapmsh, debenedictis2023dtadiiot}. However, to the best of our knowledge, the scientific literature lacks a structured proposal for \revision{process mining}-based feature extraction using the state-of-the-art \revision{conformance checking} variant, namely alignment-based \revision{conformance checking}.

\subsection{Contributions}
We propose a novel \revision{process mining}-based feature extraction approach with alignment-based \revision{conformance checking}. This variant aligns the deviating control flow with a reference Petri net; the resulting alignment can be inspected to extract additional statistics such as the number of times a given activity caused mismatches \cite{aalst2022pmhandbook}. We integrate this approach into a flexible and explainable framework for developing techniques for control-flow anomaly detection. The framework combines \revision{process mining}-based feature extraction and dimensionality reduction to handle high-dimensional feature sets, achieve detection effectiveness, and support explainability. Notably, in addition to our proposed \revision{process mining}-based feature extraction approach, the framework allows employing other approaches, enabling a fair comparison of multiple \revision{conformance checking}-based and \revision{conformance checking}-independent techniques for control-flow anomaly detection. Figure \ref{HIGH_LEVEL_VIEW} shows a high-level view of the framework. Business processes are monitored, and event logs obtained from the database of information systems. Subsequently, \revision{process mining}-based feature extraction is applied to these event logs and tabular data input to dimensionality reduction to identify control-flow anomalies. We apply several \revision{conformance checking}-based and \revision{conformance checking}-independent framework techniques to publicly available datasets, simulated data of a case study from railways, and real-world data of a case study from healthcare. We show that the framework techniques implementing our approach outperform the baseline \revision{conformance checking}-based techniques while maintaining the explainable nature of \revision{conformance checking}.

In summary, the contributions of this paper are as follows.
\begin{itemize}
    \item{
        A novel \revision{process mining}-based feature extraction approach to support the development of competitive and explainable \revision{conformance checking}-based techniques for control-flow anomaly detection.
    }
    \item{
        A flexible and explainable framework for developing techniques for control-flow anomaly detection using \revision{process mining}-based feature extraction and dimensionality reduction.
    }
    \item{
        Application to synthetic and real-world datasets of several \revision{conformance checking}-based and \revision{conformance checking}-independent framework techniques, evaluating their detection effectiveness and explainability.
    }
\end{itemize}

The rest of the paper is organized as follows.
\begin{itemize}
    \item Section \ref{sec:related_work} reviews the existing techniques for control-flow anomaly detection, categorizing them into \revision{conformance checking}-based and \revision{conformance checking}-independent techniques.
    \item Section \ref{sec:abccfe} provides the preliminaries of \revision{process mining} to establish the notation used throughout the paper, and delves into the details of the proposed \revision{process mining}-based feature extraction approach with alignment-based \revision{conformance checking}.
    \item Section \ref{sec:framework} describes the framework for developing \revision{conformance checking}-based and \revision{conformance checking}-independent techniques for control-flow anomaly detection that combine \revision{process mining}-based feature extraction and dimensionality reduction.
    \item Section \ref{sec:evaluation} presents the experiments conducted with multiple framework and baseline techniques using data from publicly available datasets and case studies.
    \item Section \ref{sec:conclusions} draws the conclusions and presents future work.
\end{itemize}

\section{Problem Setup and Preliminaries}\label{sec:problemsetup}

\textbf{Notation:} We use $[n]$ to denote $\left\{i \in \mathbb{N} \;|\; i\leq n\right\}$. $\id \in \R^{d\times d}$ represents the $d$-dimensional identity matrix. We use $\normal(\mu, \msig)$ to denote the multivariate normal distribution with specified mean, $\mu$ and covariance matrix $\msig$. $\norm{.}_{2}$ denotes the $\ell_{2}$ euclidean norm for vectors and $\normop{.}$ denotes the operator norm for matrices. $\E\bbb{X}$ denotes the expectation of the random variable $X$ and $\mathsf{Cov}(X)$ denotes its covariance matrix. For $a, b \in \R$, we write $a \lesssim b \text{ if and only if there exists an absolute constant } C > 0 \text{ such that } a \leq C b.$ $\tilde{O}, \tilde{\Omega}$ represent order notations with logarithmic factors. We also define a coarser notion of subGaussianity used subsequently in our proof sketch, 
\begin{definition}[$\bb{\beta^2,K}$-subGaussianity]\label{definition:new_subGaussian_def}
A mean-zero random variable $Y$ is said to be $\bb{\beta^2,K}$-subGaussian if it satisfies:
\bas{
 \bP(|Y|> A)\leq e^K \exp(-\tfrac{A^2}{2\beta^2})   
}
\end{definition}

\textbf{Ornstein-Uhlenbeck process:} Consider a target distribution $\pi$ over $\bR^d$. Suppose $x_0 \sim \pi$ and $x_t$ solve the following Stochastic Differential Equation (SDE):
\begin{equation}\label{eq:fwd_noise}
    dx_t = -x_t dt + \sqrt{2}dB_t\,,
\end{equation}
where $B_t$ is the standard Brownian Motion over $\R^d$. An application of Ito's formula demonstrates that $x_t = x_0e^{-t} + z_t$ where $z_t \sim \cN(0,\sigma_{t}^{2}\vI)$ is independent of $x_0$ and $\sigma_{t} := \sqrt{1-e^{-2t}}$. This is the forward noising process, which progressively noises the initial sample into a standard Gaussian vector. Ito's formula also relates $x_{t}, x_{t'}$ for any timesteps $t > t' \geq 0$ to obtain, $x_t = x_{t'}e^{-(t-t')} + z_{t,t'}$ where $z_{t,t'} \sim \cN(0,\sigma_{t-t'}^{2}\vI)$ is independent of $x_{t'}$ and $\sigma_{t-t'} := \sqrt{1-e^{-2(t-t')}}$. For $t \in [0,T]$, let $p_t$ be the probability density function. Given $\bar{x}_0 \sim p_T$ and a standard $\R^d$ Brownian motion $\bar{B}$, then the denoising process is: 
\ba{d\bar{x}_t = \bar{x}_tdt+2\nabla \log p_{T-t}(\bar{x}_t)dt + \sqrt{2}d\bar{B}_t\,.\label{eq:reverse_sde}} It is the time reversal of the noising process which implies $\bar{x}_T \sim \pi$ \cite{song2020score}. 

% \ps{It is fine if everything is written like this, but shouldn't we use $X$ to denote RVs? No need to change anything, just asking.}

\textbf{Score Matching:} Given i.i.d. data points $x^{(1)},\dots,x^{(m)}$ from the target distribution $\pi$, diffusion models learn the score function $s(t,x) : \bR^{+}\times\bR^d \to \bR^d$ defined as $s(t,x) \equiv  s_{t}\bb{x} := \nabla \log p_t(x)$ via denoising score matching (DSM). Tweedie's formula states that \ba{s(t,x_t) = \E\bbb{\frac{-z_t}{\sigma_t^2}\bigg|x_t}\,.\label{eq:tweedies_formula}} 
Let $\cF$ be a finite class of functions which map $\bR^+\times\bR^d$ to $\bR^d$ with functions $(t, x) \rightarrow f\bb{t, x} \equiv f_{t}\bb{x}$. Let $\cT = \{t_1,\dots,t_N\}$ be a finite subset of $[0,T]$. Let $x_t^{(i)}$ denote the solution of Equatoin~\eqref{eq:fwd_noise} at time $t$ with $x_0^{(i)} = x^{(i)}$ and define $z_t^{(i)} := x_t^{(i)}-e^{-t}x^{(i)}$. We consider the joint DSM objective to be: \ba{\hat{\cL}(f) := \frac{1}{mN}\sum_{i=1}^{m}\sum_{t \in \cT}\norm{f(t,x^{(i)}_t)+\frac{z^{(i)}_t}{\sigma_t^2}}_{2}^2\,. \label{eq:dsm_total}}
Intuitively, optimizing \eqref{eq:dsm_total} represents a regression task with noisy labels. There are two primary sources of noise in this setup. The first comes from~\eqref{eq:tweedies_formula}, since the targets, $-z_t/\sigma_t^2$, conditioned on the data point, $x_{t}$, are only equal to the true score, $s\bb{t, x_{t}}$ in expectation. The second comes from the randomness in $x_{t} \sim p_{t}$ itself.
% \dn{clarify difference in practice, and in prior works}

The empirical risk minimizer is defined as $\hat{f} = \arg\inf_{f \in \cF}\hat{\cL}(f)$. The results established in \cite{benton2024nearly} states that the error in sampling arising from using the estimated score function $\hat{f}$ is given by:
\begin{equation}
\epsilon_{\mathsf{score}}^2(\hat{f}) := \sum_{i=2}^{N}\gamma_i\E_{x \sim p_{t_i}}\|\hat{f}(t_i,x) - s(t_i,x)\|^2, \text{ where } \gamma_i:= t_{i}-t_{i-1} \label{eq:sampling_requirement}
\end{equation}
Our goal is to bound this error. For simplicity, we consider $t_i = i\Delta$ for some step size $\Delta \in \bb{0,1}$. 
\section{Main Results}\label{sec:main_results}

We operate under the following smoothness assumption on the function class, $\mathcal{F}$. 

\begin{assumption}[Smoothness of function class]\label{assumption:score_function_smoothness}
Let the true score function, $s \in \cF$. 
\begin{enumerate}
    \item[0.]$\nabla \log p_t(\cdot)$ is continuously differentiable for every $t \in \bR^+$.
    \item[1.] Lipschitzness : For all $t \in \timeset, x_{1}, x_{2} \in \R^{d}$, $f \in \cF$ $$\norm{f(t, x_{1}) - f(t, x_{2})}_{2} \leq L\norm{x_{1}-x_{2}}$$
    \item[2.] Local Time Regularity : There exists a set $B_{\delta,t}$ such that $p_t(B_{\delta,t}) \geq 1 -\delta$, $\forall t \geq t' \in \timeset, x \in B_{\delta,t}$, $\forall f \in \cF$
    \bas{&\norm{e^{-(t-t')}f(t, x)-f(t', e^{(t-t')}x)}_{2} \leq e^{t-t'}L\sqrt{8(t-t')\log(\tfrac{2}{\delta})}}
\end{enumerate}
\end{assumption}

The first is a standard Lipschitz continuity assumption followed in the literature (see e.g. \cite{block2020generative}). The second assumes H\"older continuity with respect to the time variable. This is a natural assumption because Lemma~\ref{eq:time_smoothness_of_score_function} shows that Assumption~\ref{assumption:score_function_smoothness}-1 implies Assumption~\ref{assumption:score_function_smoothness}-2 for the true score function, $s(t, x)$. 

%\dn{TODO: remove $\alpha_{t_j}$ altogether}\syamantak{Done. Changed to $\gamma_{j}$. Have to fix $\gamma_t$. Done.}

Equation~\eqref{eq:fwd_noise} demonstrates that $x_t$ forms a Markov chain, leading to the noise random variables, $z_t$, being strongly dependent. Additionally, \eqref{eq:fwd_noise} is typically iterated for $T = \tilde{O}\big(\tau_{\mathsf{mix}}\big)$ timesteps, until $p_T$ is close to a gaussian distribution. This setup falls outside the scope of conventional analyses of learning from dependent data, which are prominent in the literature (see Section~\ref{subsec:related_works}). Such analyses usually assume a significantly larger number of datapoints, where datapoints separated by $\tau_{\mathsf{mix}}$ in time are approximately independent, and the convergence rates align with their i.i.d. counterparts, adjusted for an effective sample size reduced by a factor of $\tau_{\mathsf{mix}}$. In contrast, our setting involves substantially fewer datapoints. To address this challenge, we propose a novel martingale decomposition (stated in Lemma~\ref{lemma:error_martingale_decomposition_2} and proved in Lemma~\ref{lemma:error_martingale_decomposition_2_actual_appendix}) of the error and establish sharp concentration bounds to account for these dependencies.

Recall the DSM objective in \eqref{eq:dsm_total}. As explained before, there are two sources of noise: (1) due to $-z_{t}/\sigma_{t}^{2}$ conditioned on $x_{t}$, (2) due to $x_{t}\sim p_{t}$.  We demonstrate the effect of fluctuations in $z_{t}|x_t$ in Theorem~\ref{theorem:empirical_l2_error_bound} and then deal with the random fluctuations due to $x_t$ in Theorem~\ref{theorem:expected_l2_error_bound}. 

% \syamantak{Done.}\dn{What noise did we remove here? explain this and explain why our variance bound is novel and why it is hard to derive} \dn{Also explain in detail why martingale decomposition is needed.}
% \dn{We need to describe the following insights early on, before stating the results. For each of these points, refer to the exact place in the main paper where it is stated and the exact place in the appendix where it is proved. 
% \begin{enumerate}
%     \item There are two sources of randomness and error: $z_t|x_t$ and $x_t$.
%     \item First we remove the randomness in $z_t|x_t$ and then deal with the randomness in $x_t$. 
%     \item $x_t$ is derived from a markov chain and hence $z_t$ are heavily dependent random variables. We do not have multiple mixing times of data. We need sharp concentration despite the dependencies. That is why martingale decomposition.
%     \item We can show that the martingale increments are conditionally sub-gaussian with possibly random parameters. But this has dimension dependency in variance proxy.
%     \item We compute accurate variances, which are dimension free. This requires heavy lifting to do without assuming Hessian smoothness.
%     \item Combine dimension free variance with dimension dependent sub-gaussianity with Freedman like super martingale argument. 
% \end{enumerate}
% }

% \dn{TODO: larger step sizes also work, just need a markov inequality based argument} \syamantak{Done. Please check Theorem~\ref{thm:subsampled_error_bound}}

Our first result in Theorem~\ref{theorem:empirical_l2_error_bound} provides a dimension-free bound on the empirical squared error, wherein we show how to control the noise due to $z_{t}$, conditioned on the data, $x_{t}$. 

\begin{restatable}[Empirical Squared Error Bound]{theorem}{empiricalsquarederror}\label{theorem:empirical_l2_error_bound}Let Assumption~\ref{assumption:score_function_smoothness} hold. Fix $\delta \in \bb{0,1}$. For all $j \in [N]$, let $t_{j} := \Delta j$ and $\gamma_{j} := \Delta$. Let $B := C\log\bb{\bb{L+1}dmN\log\bb{\frac{1}{\delta}}/\Delta}$ for an absolute constant $C > 0$, and let $\Delta\log^3(\tfrac{1}{\Delta})d^{3}\log^3(2d)\log^3\bb{\frac{2Nm}{\delta}}\log^3\bb{\frac{B|\cF|}{\delta}} \leq 1$ and $N\Delta \leq C\log(\frac{1}{\Delta})$. Then for
\bas{
    % m \gtrsim \tfrac{\bb{L+1}^{2}}{\epsilon^{2}}\log\bb{\tfrac{B|\cF|}{\delta}}
    m \gtrsim \frac{\bb{L+1}^{2}}{\epsilon^{2}}\log\bb{\frac{B|\cF|}{\delta}}N\Delta
}
with probability at least $1-\delta$, 
\bas{
\sum_{i\in[m], j\in[N]}\frac{\gamma_{j}\norm{\hat{f}\bb{t_j,x_{t_j}^{(i)}}-s\bb{t_jx_{t_j}^{(i)}}}_{2}^{2}}{m} \lesssim \epsilon^{2}
}
\end{restatable}

\begin{remark}
    The sample complexity in Theorem~\ref{theorem:empirical_l2_error_bound} depends on the smoothness parameter $L$ and on $\log(B)$. Observe that $B$ depends logarithmically on $d$, thus leading to a nearly dimension-free result, i.e. $\log\log d$ dependence. This is in stark contrast to existing results, which have poly($d$) dependence. We believe that the objective function in ~\eqref{eq:dsm_total} harnesses the smoothness of the function class by \textit{jointly optimizing over multiple time steps}.
\end{remark}
Theorem~\ref{theorem:empirical_l2_error_bound} is the first step in proving the expectation bound in Theorem~\ref{theorem:expected_l2_error_bound} and may be of independent interest. Theorem~\ref{theorem:expected_l2_error_bound} deals with the noise arising from the data $x_{t} \sim p_{t}$. Our next assumption, called `hypercontractivity', controls the $4^{\text{th}}$-moment of the error bound with respect to the $2^{\text{nd}}$-moment, which can be used to prove the generalization of the score function in $L^2$ error. This is a mild assumption, standard in statistics and learning theory under heavy tails \cite{mendelson2020robust,klivans2018efficient,minsker2018sub}.
\begin{assumption}\label{assumption:hypercontractivity}
    For every $f \in \cF$ and $x_t \sim p_t$, we have:
    $$\E[\|f(t,x_t) - s(t,x_t)\|^4]^{\frac{1}{4}} \leq \chyp \E[\|f(t,x_t)-s(t,x_t)\|^2]^{\frac{1}{2}}$$
\end{assumption}

$\kappa^4$ can be bounded (up to multiplicative constants) by the kurtosis of $f(t,x_t) - s(t,x_t)$. Assumption~\ref{assumption:hypercontractivity} follows from the smoothness and strong convexity of neural networks in parameter space (not \(x_t\)). Recent work \cite{milne2019piecewise, yi2022characterization} shows that near the global minimizer of the population loss, many smooth non-convex losses exhibit local strong convexity. We formalize this connection in Lemma~\ref{lem:sc-hyp}.

% \ps{Are there connections to PL?}\syamantak{Sorry what is PL here?}

% \syamantak{Change h to something else.}\syamantak{Done.}
\begin{lemma}\label{lem:sc-hyp}
    Let all $f\bb{t, x} \in \cF$, be parameterized as $g\bb{t, x;\theta}$ for $\theta \in \Theta \subseteq \bR^D$ and $\theta_{*}$ be such that $h\bb{t, x_{t}; \theta_{*}} = s\bb{t, x_{t}}$. Suppose $\exists \lambda,\mu \geq 0$ such that $\forall \theta \in \Theta$, 
\bas{
\E\bbb{\norm{g\bb{t, x_{t}; \theta} - g\bb{t, x_{t}, \theta_{*}}}_{2}^{4}} &\leq \lambda^{2} \norm{\theta-\theta_{*}}^4, \text{ and } \\
\E\bbb{\norm{g\bb{t, x_{t}; \theta} - g\bb{t, x_{t}, \theta_{*}}}_{2}^{2}} & \geq \mu\norm{\theta-\theta_{*}}^{2}
}
Then, all $f \in \cF$ satisfy Assumption~\ref{assumption:hypercontractivity} with $\chyp = \frac{\lambda}{\mu}$.
\end{lemma}

% \begin{assumption}[Local Strong Convexity of loss]\label{assumption:local_strong_conv}Let all $f\bb{t, x} \in \cF$, be parameterized as $h\bb{t, x;\theta}$ for $\theta \in \Theta$ and $\theta_{*}$ be such that $h\bb{t, x_{t}; \theta_{*}} = s\bb{t, x_{t}}$. Then $\exists \mu \geq 0$ such that $\forall \theta \in \Theta$, 
% \bas{
% \E\bbb{\norm{h\bb{t, x_{t}; \theta} - h\bb{t, x_{t}, \theta_{*}}}_{2}^{2}} \geq \mu\norm{\theta-\theta_{*}}^{2}
% }
% \end{assumption}

Under Assumptions~\ref{assumption:score_function_smoothness} and \ref{assumption:hypercontractivity}, we state our main result in Theorem~\ref{theorem:expected_l2_error_bound}. In this result, we use Theorem~\ref{theorem:empirical_l2_error_bound} and handle the noise due to $x_{t} \sim p_{t}$ in the DSM objective. 
% \dn{Make changes here based on changes to empirical bound theorem statemetn}\syamantak{Done.}

\begin{restatable}[Expected Squared Error Bound]{theorem}{expectedsquarederror}\label{theorem:expected_l2_error_bound}Let Assumptions~\ref{assumption:score_function_smoothness} and \ref{assumption:hypercontractivity} hold. Fix $\delta \in \bb{0,1}$. For all $j \in [N]$, let $t_{j} := \Delta j$ and $\gamma_{j} := \Delta$. Let $B := C\log\bb{\bb{L+1}dmN\log\bb{\frac{1}{\delta}}/\Delta}$ for an absolute constant $C > 0$, and let $\Delta\log^3(\tfrac{1}{\Delta})d^{3}\log^3(2d)\log^3\bb{\frac{2Nm}{\delta}}\log^3\bb{\frac{B|\cF|}{\delta}} \leq 1$ and $N\Delta \leq C\log(\frac{1}{\Delta})$. If
\bas{
    m \gtrsim \chyp^{2}\max\left\{\log\bb{\frac{N}{\delta}}, \frac{\bb{L+1}^{2}N\Delta}{\epsilon^{2}}\log\bb{\frac{B|\cF|}{\delta}} \right\}
}
then with probability at least $1-\delta$, 
\bas{
\sum_{j \in [N]}\gamma_{j}\E_{x_{t_j}}\bbb{\norm{\hat{f}\bb{t_j, x_{t_j}}-s\bb{t_j, x_{t_j}}}_{2}^{2}} \lesssim \epsilon^{2}
}
\end{restatable}

\begin{remark}
     In addition to the sample complexity of Theorem~\ref{theorem:empirical_l2_error_bound}, the sample complexity for the generalization bound in Theorem~\ref{theorem:expected_l2_error_bound} additionally has a factor of $\kappa^{2}$ due to the local strong convexity assumption formalized in Lemma~\ref{lem:sc-hyp}. 
\end{remark}

We note that Theorem~\ref{theorem:expected_l2_error_bound} pertains to training of diffusion models and requires a very fine value of the step size, $\Delta$. This is not an issue in practice since SGD type stochastic approximation is deployed to perform empirical risk minimization. However, once the model has been trained, we can accelerate inference by using a larger timestep-size to discretize the diffusion process, as shown in Theorem~\ref{thm:subsampled_error_bound} and proved in Theorem~\ref{thm:subsampled_error_bound_appendix}.

\begin{theorem}[Fast Inference]\label{thm:subsampled_error_bound} Under the same assumptions as Theorem~\ref{theorem:expected_l2_error_bound}, partition the timesteps $\{t_j = \Delta j\}_{j \in [N]}$ into $k$ disjoint subsets $S_1, S_2, \dots, S_k$, where each subset $S_i$ contains timesteps of the form $t_j = \Delta(i + nk)$ for $n \in \mathbb{N}$. Define $\gamma_j' := k\Delta$ for all $j$ in any subset $S_i$. Then, there exists at least one subset $S_i$ such that:
\[
\sum_{j \in S_i} \gamma_j' \E_{x_{t_j}}\left[\norm{\hat{f}(t_j, x_{t_j}) - s(t_j, x_{t_j})}_2^2\right] \lesssim \epsilon^2,
\]
with probability at least $1 - \delta$.
\end{theorem}
The subsets $S_i$ allow for a much coarser discretization with differences being $k\Delta$ instead of $\Delta$. While the error due to discretization of the SDE might become worse, as shown by the bounds in \cite{benton2024nearly}, Theorem~\ref{thm:subsampled_error_bound} demonstrates that the score estimation error does not degrade.

\textbf{Comparison with prior work:}
\cite{block2020generative} and \cite{gupta2023sample} analyze each discretization timestep independently, and perform a union bound across timesteps to achieve a bound on the DSM objective in \eqref{eq:dsm_total}. \cite{block2020generative} assume a target distribution with bounded support over a euclidean $\ell_{2}$ ball of radius, $R$. They further assume the score function to be $L$-Lipschitz and employ Rademacher complexity-based generalization bounds with a sufficiently rich function class, $\mathcal{F}$, to show (Proposition 12) with high probability, (up to logarithmic factors) 
\ba{\E_{x_{t}}\bbb{\norm{\hat{f}\bb{t,x_{t}}-s\bb{t,x_{t}}}_{2}^{2}} \lesssim \frac{L^{2}R^{2}}{\sigma_{t}^{4}}\bb{\mathcal{R}^{2}\bb{\cF} + \frac{d}{n}}\label{eq:block_dsm_bound}}
where $\mathcal{R}\bb{\cF}$ denotes the Rademacher complexity (see e.g. \cite{bartlett2002rademacher}) of $\cF$. They show this bound for all $t \in \timeset$ and perform a union bound to obtain the final sample complexity, instead of analyzing \eqref{eq:dsm_total} jointly. 
% It can be shown (see \cite{benton2024nearly}) that to satisfy \eqref{eq:sampling_requirement} for a given $\epsilon > 0$ with an approximate score function, $\hat{f}$, it suffices to ensure at each timestep $t$, 
% \ba{
% \E_{x_{t}}\bbb{\norm{\hat{f}\bb{t,x_{t}}-s\bb{t,x_{t}}}_{2}^{2}} \lesssim \frac{\epsilon^{2}}{\sigma_{t}^{2}} \label{eq:per_timestep_score_requirement}
% }
Using the bound in \eqref{eq:block_dsm_bound}, for a uniform step size $\Delta$, this leads to a sample complexity scaling as $\frac{1}{\text{poly}(\Delta)}$ to satisfy the requirement in \eqref{eq:sampling_requirement}. Furthermore, their sample complexity also depends, at least linearly, on the dimension, $d$, and hence is not dimension-free.

\cite{gupta2023sample} improve the dependence of the sample complexity on Wasserstein error, compared to \cite{block2020generative}. They assume a second moment bound on the target distribution without any smoothness assumptions on the score function and propose a relaxation of $\ell_{2}$ error as, 
\bas{
 D^{\delta}\bb{f_t, s_t} \leq \epsilon \Leftrightarrow \mathbb{P}_{x_t}\bb{\norm{f\bb{t,x_{t}}-s\bb{t, x_{t}}}_{2} \geq \epsilon} \leq \delta
} 
They show that learning $f \in \cF$ satisfying the above criteria for all timesteps $t \in \timeset$ suffices for sampling and further show (Lemma A.2) a sample complexity bound of $m \gtrsim d\log\bb{\frac{|\cF|}{\delta}}/\epsilon^{2}$ to achieve $D^{\delta}\bb{f_t, s_t} \leq \epsilon$ for any fixed $t$ and  perform a union bound across all timesteps to achieve a uniform bound required for sampling. In comparison with \cite{block2020generative}, their sample complexity does not scale with $\frac{1}{\sigma_t}$, but still involves a linear dependence on $d$.

Recently, \cite{han2024neural} provided an analysis of gradient descent to optimize \eqref{eq:dsm_total} via neural networks. They assume that the target distribution has bounded support over a $\ell_{2}$-euclidean ball of radius $R$ and the score function is $L$-Lipschitz. They model the evolution of neural
networks during training by a series of kernel regression tasks and jointly model all timesteps by assuming time as an input to the neural network. In this sense, their work is closest in spirit to our paper. However, their sample complexity bounds (Theorem 3.12) show a polynomial dependence on the dimension, $d$.
\section{Technical Results}
\label{sec:technical_results}

In this section we provide insights into our proof techniques and key technical results.



\rd 

% Our martingale difference sequence is of the form $Q_{i} := \inner{G_{i}}{Y_{i}-\E\bbb{Y_{i}|\mathcal{F}_{i-1}}}$ adapted to the filtration $\left\{\mathcal{F}_{i}\right\}_{i \in [n]}$, where $G_i$ is a $\mathcal{F}_{i-1}$ measurable random variable. We first show that conditioned on $\mathcal{F}_{i-1}$, $Q_{i}$ is subGaussian. However, as shown in Lemma~\ref{lemma:subGaussianity_parameters}, the subGaussianity parameters depend on the data dimension, $d$ along with $G_i$ and the step size, $\Delta$. Therefore, performing a concentration argument solely relying on this observation leads to a dimension-dependent bound. 

% To further refine our analysis and show a dimension-free bound, we evaluate the variance of $Q_{i}$ conditioned on $\mathcal{F}_{i-1}$. As shown in Lemma~\ref{lemma:martingale_variance_bound} (and proved in Lemma~\ref{lemma:martingale_diff_variance_bound_appendix}), the variance depends only on the smoothness parameter, $L$, along with $G_i$ and $\Delta$. The proof strategy relies on Assumption~\ref{assumption:score_function_smoothness}, and an application of the mean value theorem involving the hessian, $h_{t} := \nabla^{2}\log\bb{p_{t}}$, which yields terms of the form 
% \bas{
%     \E\bbb{h_{t'}(y_{t'})(x_{t'}-\tilde{x}_{t'})(x_{t'}-\tilde{x}_{t'})^{\top}h_{t'}(y_{t'})^{\top}|x_t}, t' < t
% }
% for $x_{t'}, \tilde{x}_{t'}$ i.i.d conditioned on $x_{t}$ and $y_{t'} = \lambda x_{t'} + (1-\lambda)\tilde{x}_{t'}$, $\lambda \in (0,1)$. When \(t\) and \(t'\) are close, one would hope to exploit the smoothness of the Hessian \(h_t\) with respect to time. Specifically, if \(h_t\) were smooth in the time parameter, this would allow the expectation to move inside, enabling the use of Tweedie’s second-order formula (Lemma~\ref{lemma:second_order_tweedie_application}) to derive variance bounds that are dimension-free. However, in the absence of such a smoothness assumption on the hessian, we decompose the Hessian \(h_{t'}\bb{y_{t'}}\) into two components:
% \[
% h_{t'}\bb{y_{t'}} = h_{t', \epsilon}\bb{y_{t'}} + \bb{h_{t'}\bb{y_{t'}} - h_{t', \epsilon}\bb{y_{t'}}}.
% \]
% Here, the first term, \(h_{t', \epsilon}\bb{y_{t'}}\), represents a ``smoothed" or ``mollified" hessian, averaged over an appropriately chosen distribution, which we show satisfies Lipschitz continuity. The second term, which represents the deviation between the original and mollified Hessians, requires a finer analysis, breaking the interval $[t', t]$ into many subintervals and draws upon Lusin's theorem(Lemma~\ref{lemma:lusin_theorem}), as developed further in Lemma~\ref{lemma:lusin_theorem_decomp}.

% Finally, we combine our dimension-free variance bound with the dimension-dependent subGaussianity via a freedman-like supermartingale argument as shown in Lemma~\ref{lemma:martingale_concentration_lemma}, which may be of independent interest to the reader. This further involves a careful handling since typical analysis of freedman's inequality assumes a uniform bound on the random variables in contrast to our more general conditional subGaussianity assumption.

\bk

 We start by bounding the empirical squared error in terms of a linear error term, as shown in Lemma~\ref{lemma:l2errorboundmartingale}. This relies on comparing the empirical error, $\hat{\mathcal{L}}$ of the minimizer $\hat{f}$ with the true score function, $s$. While here we assume for simplicity that $s \in \cF$, it can be relaxed to assume that $\exists s_{*}\in\cF$ with sufficiently small $\ell_{2}$ error, similar to \cite{gupta2023sample}.

\begin{restatable}{lemma}{squarederrorboundmartingale}\label{lemma:l2errorboundmartingale}
For $f \in \cF$, let $ y_t^{(i)} := \frac{-z_{t}^{(i)}}{\sigma_{t}^{2}}$ and
\begin{equation}
    \mathcal{L}(f) := \sum_{i \in [m], j \in [N]} \frac{\gamma_{j} \norm{f\bigl(t_j, x_{t_j}^{(i)}\bigr) - s\bigl(t_j, x_{t_j}^{(i)}\bigr)}_{2}^{2}}{m}, \notag
\end{equation}
\begin{equation}
    \begin{aligned}
        H^{f} := &\sum_{i \in [m], j \in [N]} \frac{\gamma_{j}}{m} \bigl\langle f\bigl(t_{j}, x_{t_{j}}^{(i)}\bigr) - s\bigl(t_{j}, x_{t_{j}}^{(i)}\bigr), y_{t_{j}}^{(i)} - s\bigl(t_{j}, x_{t_{j}}^{(i)}\bigr) \bigr\rangle. \notag
    \end{aligned}
\end{equation}
If $s \in \cF$ then for $\hat{f} = \arg\inf_{f \in \cF} \hat{\cL}(f)$, we have  
\small
\begin{equation}
    \mathcal{L}(\hat{f}) \leq  H^{\hat{f}}, \label{eq:Lf_bound}
\end{equation}
\normalsize
where $\hat{\cL}$ is defined in \eqref{eq:dsm_total}.
\end{restatable}

We define $\hat{f}$ as the minimizer of $\hat{\cL}(f)$.
Lemma~\ref{lemma:l2errorboundmartingale} bounds $\mathcal{L}(\hat{f})$, the loss of $\hat{f}$ against the true and unknown score function. This lemma makes our aim clear. We will show a high probability bound on $H^{\hat{f}}$ defined in~\eqref{eq:Lf_bound} to control $\mathcal{L}(\hat{f})$. Interestingly, as shown in Lemma~\ref{lemma:error_martingale_decomposition_2} (and proved in Lemma~\ref{lemma:error_martingale_decomposition_2_actual_appendix}), for a fixed $f$ it is possible to decompose $H^{f}$ as a martingale difference sequence. 

% \dn{I think a good way to explain this would be as the standard \textbf{Doob martingale} \cite{durrett2019probability} corresponding to the RHS in  Lemma~\ref{lemma:l2errorboundmartingale}, with respect to the filtration we have defined}

The martingale difference decomposition of $H^{\hat{f}}$, exploiting the Markovian structure of \eqref{eq:fwd_noise}, has terms of the form $Q_{i} := \inner{G_{i}}{Y_{i}-\E\bbb{Y_{i}|\mathcal{F}_{i-1}}}$ adapted to the filtration $\left\{\mathcal{F}_{i}\right\}_{i \in [n]}$, where $G_i$ is a $\mathcal{F}_{i-1}$ measurable random variable. The proof  primarily uses the fact that for $t_{1} \leq t_{2} \leq t_{3}$, $\E\bbb{x_{t_1}|x_{t_2}, x_{t_3}} = \E\bbb{x_{t_1}|x_{t_2}}$ due to the Markov property. 

%We first show that conditioned on $\mathcal{F}_{i-1}$, $Q_{i}$ is subGaussian. However, as shown in Lemma~\ref{lemma:subGaussianity_parameters}, the subGaussianity parameters depend on the data dimension, $d$ along with $G_i$ and the step size, $\Delta$. Therefore, performing a concentration argument solely relying on this observation leads to a dimension-dependent bound. 


% \ps{What is $\zeta$?}\syamantak{Fixed.}
% \ps{Function class and filtration both have similar notation.}\syamantak{Fixed.}
% \dn{$\zeta$ is only used in the appendix and it is taken to be arbitrary, but in our application we use $\zeta = \frac{s-f}{m}$. In the main text, we can instantiate it to be $\frac{s-f}{m}$ for the sake of clarity.} 

\begin{lemma}\label{lemma:error_martingale_decomposition_2}
Let $\zeta = \frac{s-f}{m}$ for any $f \in \cF$. Define 
\small
\[
\bar{G}_i \!:= \sum_{j=1}^{N} \frac{\gamma_{j}e^{-(t_j-t_1)}\zeta\bb{t_j,x_{t_j}^{(i)}}}{\sigma_{t_j}^{2}}, \;\; 
G_{i,k}\! := \!\!\!\!\!\! \sum_{j=N-k+2}^{N}\!\!\!\!\! \frac{\gamma_{j}e^{-t_j}\zeta\bb{t_j,x_{t_j}^{(i)}}}{\sigma_{t_j}^{2}}
\]
\normalsize
and define $R_{i,k}$ as  
\small
\begin{equation}
    R_{i,k} := 
    \begin{cases} 
        0, & \text{for } k = 0, \\
        \begin{aligned}
            \bigl \langle G_{i,k+1}, 
            & \E[x_0^{(i)} | x_{t_{N-k+1}}^{(i)}]  - \E[x_0^{(i)} | x_{t_{N-k}}^{(i)}] 
            \bigr \rangle, 
        \end{aligned} 
        & \text{for } k \in [N-1], \\
        \bigl\langle \bar{G}_i, z_{t_1}^{(i)} - \E\bbb{z_{t_1}^{(i)}|x_{t_1}^{(i)}} \bigr\rangle, 
        & \text{for } k = N.
    \end{cases} \notag
\end{equation}
\normalsize
Let $t_0 = 0$. Consider the filtration defined by the sequence of $\sigma$-algebras,  
\[
\mathcal{F}_{i,k} := \sigma(\{x_{t}^{(j)} : 1\leq j < i, t \in \timeset\} \cup \{x_{t}^{(i)} : t \geq t_{N-k}\})
\]
for $i \in [m]$ and $k \in \{0,\dots, N\}$, satisfying the total ordering $\left\{ (i_1, j_1) < (i_2, j_2) \text{ iff } i_1 < i_2 \text{ or } i_1 = i_2, j_1 < j_2 \right\}.$. Then, 
\begin{enumerate}
    \item For $k \in [N-1]$, $G_{i,k+1}$ is measurable with respect to $\mathcal{F}_{i,k-1}$, and $\bar{G}_{i}$ is $\mathcal{F}_{N-1}$-measurable.
    \item For $i \in [m], k \in \{0\} \cup [N]$, $(R_{i,k})_{i,k}$ forms a martingale difference sequence with respect to the filtration above.
    \item $H^{f} = \sum_{i \in [m]}\sum_{k \in [N]}R_{i,k}\,.$, where $H^{f}$ is defined in Lemma~\ref{lemma:l2errorboundmartingale} 
\end{enumerate}
\end{lemma}

% \begin{lemma}\label{lemma:error_martingale_decomposition_2}Let $\zeta = \frac{s-f}{m}$ for any $f \in \cF$.
%     Define $\bar{G}_i := \sum_{j=1}^{N} \frac{\gamma_{j}e^{-(t_j-t_1)}\zeta\bb{t_j,x_{t_j}^{(i)}}}{\sigma_{t_j}^{2}}$, $G_{i,k} := \sum_{j = N-k+2}^{N} \frac{\gamma_{j}e^{-t_j}\zeta\bb{t_j,x_{t_j}^{(i)}}}{\sigma_{t_j}^{2}}$ and $R_{i,k}$ as  
%     \begin{equation}
%     R_{i,k} = \begin{cases} 0 \text{ for } k = 0 \\
%     \bigr \langle G_{i,k+1},\E[x_0^{(i)}|x_{t_{N-k+1}}^{(i)}]-\E[x_0^{(i)}|x_{t_{N-k}}^{(i)}]\bigr \rangle \text{ for } k \in [1, N-1],  
%     \\ 
%      \bigr\langle \bar{G}_i , z_{t_1}^{(i)}-\E\bbb{z_{t_1}^{(i)}|x_{t_1}^{(i)}}\bigr\rangle \text{ for } k = N
%      \end{cases} 
%     \end{equation}
%     Let $t_0 = 0$. Consider the filtration defined by the sequence of $\sigma$-algebras,  $\mathcal{F}_{i,k} := \sigma(\{x_{t}^{(j)} : 1\leq j < i, t \in \timeset\}\cup \{x_{t}^{(i)} : t \geq t_{N-k}\})$ for $i \in [m]$ and $k \in \{0,\dots, N\}$, satisfying the total ordering $\left\{\bb{i_1, j_1} < \bb{i_2, j_2} \text{ iff } i_1 < i_2 \text{ or } i_1 = i_2, j_1 < j_2\right\}$. Then 
%     \begin{enumerate}
%         \item For $k \in [N-1]$, $G_{i,k+1}$ is measurable with respect to $\mathcal{F}_{i,k-1}$ and $\bar{G}_{i}$ if $\mathcal{F}_{N-1}$ measurable.
%         \item For $i \in [m], k \in \{0\}\cup[N]$,  $(R_{i,k})_{i,k}$ forms a martingale difference sequence with respect to the filtration above. 
%         \item $H^{f} = \sum_{i \in [m]}\sum_{k \in [N]}R_{i,k}\,.$ 
%     \end{enumerate}
% \end{lemma}

In the above lemma, $R_{ik}$ denotes the Martingale Difference Sequence arising from the Doob decomposition (see e.g. ~\cite{durrett2019probability}). Our aim is to bound $H^{\hat{f}}$ by bound $H^{f}$ uniformly for every $f$, using martingale concentration.

In the next lemma, we show that conditioned on $\mathcal{F}_{i-1}$, $Q_{i}$ is subGaussian.
To gain intuition into how subGaussianity comes into play in our context, we note that Lemma F.3. in \cite{gupta2023sample} shows that the score function, $s(t, x_t)$, is $1/\sigma_{t}$-subGaussian. We develop a more fine-grained argument exploiting the smoothness of the score function to show a slightly different notion of subGaussianity (Definition~\ref{definition:new_subGaussian_def}) for our martingale difference sequence. 

\begin{restatable}{lemma}{martingalediffsubGaussianityparameters}\label{lemma:subGaussianity_parameters}
    Fix $\delta \in \bb{0,1}$. Consider $R_{i,k}$ and $\mathcal{F}_{i,k}$ as defined in Lemma~\ref{lemma:error_martingale_decomposition_2} and let $\Delta := t_{N-k+1}-t_{N-k}$. Under Assumption~\ref{assumption:score_function_smoothness}, following the definition in Definition~\ref{definition:new_subGaussian_def}, conditioned on $\mathcal{F}_{i,k-1}$, $R_{i,k}$ is $(\beta_{i,k}^2\|G_{i,k}\|^2,W_{i,k})$-subGaussian where $\beta_{i,k},W_{i,k}$ are $\mathcal{F}_{i,k-1}$ measurable random variables such that  
    % \dn{This has to be $(\beta_{i,k}^2\|G_{i,k}\|^2,W_{i,k})$ subGaussian} \syamantak{Fixed.}
    $W_{i,k} \leq \log\bb{\frac{2}{\delta}}$ with probability at-least $1-\delta$ and
    \bas{
        &\beta_{i,k} := \begin{cases}
            8\bb{L+1}e^{t_{N-k+1}}\sqrt{\Delta d}, \;\; k \in  [N-1], \\
            4\sqrt{\Delta d}, \;\; k = N
            \end{cases}
    }
\end{restatable}

%We first show that conditioned on $\mathcal{F}_{i-1}$, $Q_{i}$ is subGaussian. 
However, the subGaussianity parameters in Lemma~\ref{lemma:subGaussianity_parameters}, depend polynomially on the data dimension, $d$ along with $G_i$ and the step size, $\Delta$. Therefore, performing a concentration argument solely relying on this observation leads to a dimension-dependent bound. 


% \begin{restatable}{lemma}{errormartingaledecomposition}\label{lemma:error_martingale_decomposition_2}
% Let 
% \bas{
%     H^{f} := \sum_{i \in [m], t \in \timeset}\frac{ \gamma_{t}}{m}\bigr\langle f_{t}\bigr(x_{t}^{(i)}\bigr)-s_{t}\bigr(x_{t}^{(i)}\bigr),\frac{-z_{t}^{(i)}}{\sigma_{t}^{2}} -s_{t}\bigr(x_{t}^{(i)}\bigr)\bigr\rangle, 
% }
% Define $\bar{G}_i := \sum_{j=1}^{N} \frac{\gamma_{j}e^{-(t_j-t_1)}\bb{f_{t_j}(x_{t_j}^{(i)})-s_{t_j}(x_{t_j}^{(i)})}}{\sigma_{t_j}^{2}}$, $G_{i,k} := \sum_{j = N-k+2}^{N} \frac{\gamma_{j}e^{-t_j}\bb{f_{t_j}(x_{t_j}^{(i)})-s_{t_j}(x_{t_j}^{(i)})}}{\sigma_{t_j}^{2}}$ and $R_{i,k}$ as  
%     \bas{
%     & \inner{G_{i,k+1}}{\E[x_0^{(i)}|x_{t_{N-k+1}}^{(i)}]-\E[x_0^{(i)}|x_{t_{N-k}}^{(i)}]} \text{ for }  k < N, \\ \\
%     & \inner{\bar{G}_i}{z_{t_1}^{(i)}-\E\bbb{z_{t_1}^{(i)}|x_{t_1}^{(i)}}} \text{ for } k = N 
%     }
%     Then, $H^f = \sum_{i \in [m]}\sum_{k \in [N]}R_{i,k}$. Furthermore, consider the filtration defined by the sequence of $\sigma$-algebras,  $\mathcal{F}_{i,k} := \sigma\{x_{t}^{j} : j \in [i], t \geq t_{N-k}\}$ for $i \in [m]$ and $k < N$, satisfying the total ordering $\left\{\bb{i_1, j_1} < \bb{i_2, j_2} \text{ iff } i_1 < i_2 \text{ or } i_1 = i_2, j_1 < j_2\right\}$. Then for $i \in [m], k \in [N]$,  $\left\{R_{i,k}\right\}$ forms a martingale difference sequence. 
% \end{restatable}

% \ps{Can we unpack the math? What is H? How are the G's and R's going to be used later? Lets make it clear that the martingale structure that has been defined is necessary and (?) has not been done before....perhaps reemphasize somethings you have already mentioned before. And then say that the next several lemmas will be about applying martingale concentration. What may be nice also is if you give a quick sketch in text about the pieces that are needed and say which lemma gives which piece.}


To further refine our analysis and show a dimension-free bound, we evaluate the variance of $Q_{i}$ conditioned on $\mathcal{F}_{i-1}$. As shown in the next Lemma (Lemma~\ref{lemma:martingale_variance_bound}) (and proved in Lemma~\ref{lemma:martingale_diff_variance_bound_appendix}), the variance depends only on the smoothness parameter, $L$, along with $G_i$ and $\Delta$. 

\begin{lemma}[Variance bound for martingale difference sequence]\label{lemma:martingale_variance_bound}
Consider the martingale difference sequence \(R_{i,k}\) and the predictable sequence \(G_{i,k+1}\) with respect to the filtration \(\mathcal{F}_{i,k}\) from Lemma~\ref{lemma:variance_bound_1}. Define \(\Delta := t_{N-k+1} - t_{N-k}\). Then, $\E\bbb{R_{i,k}^{2}|\mathcal{F}_{i,k-1}} \leq \nu_{i,k}^{2}$
where
\begin{equation}
    \nu_{i,k}^{2} =
    \begin{cases} 
        &0, \quad\quad\quad\quad\quad\quad\quad\quad\quad\;\;\quad \text{if } k = 0, \\
        &C(L\Delta^2 + \Delta + L^2\Delta) e^{2t_{N-k+1}} \|G_{i,k+1}\|^2, 
        \text{ if } k \in [1, N-1], \\
        &C(L\Delta^2 + \Delta) \|\bar{G}_i\|^2, 
        \quad\quad\quad \text{if } k = N.
    \end{cases}\notag
\end{equation}
where \( C > 0 \) is an absolute constant.
\end{lemma}

% \begin{lemma}[Variance bound for martingale difference sequence]\label{lemma:martingale_variance_bound}
% Consider the martingale difference sequence $R_{i,k}$, predictable sequence $G_{i,k+1}$ with respect to the filtration $\mathcal{F}_{i,k}$ as considered in Lemma~\ref{lemma:variance_bound_1}. Define $\Delta:=t_{N-k+1}-t_{N-k}$ then $\E\bbb{R_{i,k}^{2}|\mathcal{F}_{i,k-1}} \leq \nu_{i,k}^{2}$ such that, 
% \bas{
%     \nu_{i,k}^{2} := \begin{cases} 0 \text{ if } k = 0 \\
%      C(L\Delta^2 + \Delta + L^2\Delta) e^{2t_{N-k+1}}\|G_{i,k+1}\|^2 \text{ if } k \in [1, N-1] \\
%     C(L\Delta^2 + \Delta )\|\bar{G}_i\|^2 \text{ if } k = N 
%     \end{cases}
% }
% where $C > 0$ is an absolute constant.
% % \dn{$k = N$ case follows from second order tweedie's formula. I have commented the lemma regarding this below, uncomment to see the proof}

% \end{lemma}

The proof of Lemma~\ref{lemma:martingale_variance_bound} is involved when $h_{t}(x) := \nabla^{2}\log\bb{p_{t}}(x)$ is not assumed to be Lipschtiz in $x$. Starting with the martingale difference sequence defined in Lemma~\ref{lemma:error_martingale_decomposition_2}, an application of the second-order tweedie's formula (see Lemma~\ref{lemma:second_order_tweedie_application}), reduces the problem to bounding the operator norm $\mathsf{Cov}(s\bb{t', x_{t'}}|x_{t})$ for $t-t' = \Delta > 0$, i.e, the conditional covariance matrix of the score function given the future. 
Exploiting the smoothness assumption on the score function, an application of the mean value theorem reduces our problem to bounding the operator norm of:
% \syamantak{Say that the variance of the past given the future shows up. Variance of $s_t'|x_t$ can be written by bringing in an iid copy. And the mean value theorem shows that the Hessian can be involved. and now we need tools to show smoothness of the Hessian }
% The proof strategy relies on Assumption~\ref{assumption:score_function_smoothness}, and an application of the mean value theorem involving the hessian, $h_{t} := \nabla^{2}\log\bb{p_{t}}$, which yields terms of the form 
\bas{
    \E\bbb{h_{t'}(y_{t'})(x_{t'}-\tilde{x}_{t'})(x_{t'}-\tilde{x}_{t'})^{\top}h_{t'}(y_{t'})^{\top}|x_t}, t' < t
}
for $x_{t'}, \tilde{x}_{t'}$ i.i.d conditioned on $x_{t}$ and $y_{t'} = \lambda x_{t'} + (1-\lambda)\tilde{x}_{t'}$, $\lambda \in (0,1)$. Notice that $y_{t'}|x_t$ is dependent on $x_{t'},\tilde{x}_{t'}|x_t$, which does not allow the use of Tweedie’s second-order formula (Lemma~\ref{lemma:second_order_tweedie_application}) to bound $\E\bbb{(x_{t'}-\tilde{x}_{t'})(x_{t'}-\tilde{x}_{t'})^{\top}|x_t}$ and derive variance bounds that are dimension-free. To approximately allow this argument, we decompose $h_{t'}(y_{t'})$ into two components:
\[
h_{t'}\bb{y_{t'}} = h_{t', \epsilon}\bb{y_{t'}} + \bb{h_{t'}\bb{y_{t'}} - h_{t', \epsilon}\bb{y_{t'}}}.
\]
Here, the first term, \(h_{t', \epsilon}\bb{y_{t'}}\), represents a ``smoothed" or ``mollified" hessian, averaged over an appropriately chosen distribution, which we show satisfies Lipschitz continuity. This allows us to approximate $h_{t',\epsilon}(y_{t'}) \approx h_{t',\epsilon}(e^{\Delta}x_t)$ and bound the variance with Tweedie's second order formula. The second term, which represents the deviation between the original and mollified Hessians, requires a finer analysis, breaking the interval $[t', t]$ into many subintervals and draws upon Lusin's theorem (Lemma~\ref{lemma:lusin_theorem}) to provide approximate uniform continuity for the hessian $h_{t'}$, as developed further in Lemma~\ref{lemma:lusin_theorem_decomp}.


% We next bound the conditional variance of the martingale difference sequence developed in Lemma~\ref{lemma:error_martingale_decomposition_2}. 
% Under Assumption~\ref{assumption:score_function_smoothness}, $\normop{\nabla^{2}\log\bb{p_{t}\bb{x_t}}} \leq L$, which helps us control the variance.

% \begin{restatable}{lemma}{martingalediffvariancebound}\label{lemma:martingale_variance_bound}Consider $R_{i,k}$ and $\mathcal{F}_{i,k}$ as defined in Lemma~\ref{lemma:error_martingale_decomposition_2}. Let $\forall i \in [m],  k < N, \nu_{i,k}^{2} := \E\bbb{R_{i,k}^{2}|\mathcal{F}_{i,k-1}}$ and $\forall i \in [m], \bar{\nu}_{i}^{2} := \E\bbb{R_{i,N}^{2}|\mathcal{F}_{i,N-1}}$. Then, for a universal constant $c > 0$, we have $\forall i \in [m]$,
%     \bas{
%         & \text{ for } k < N, \;\; \nu_{i,k}^{2} \leq c \Delta e^{2t_{N-k+1}}\bb{L^{2}+1}\norm{G_{i,k}}_{2}^{2},  \\
%         &  \bar{\nu}_i^{2} \leq \sigma_{t_{1}}^{2}\bb{1 + L\sigma_{t_1}^{2}}\norm{\bar{G}_i}_{2}^{2} = c\Delta\bb{1 + L\Delta}\norm{\bar{G}_i}_{2}^{2}, 
%     } 
% \end{restatable}

% We additionally show, in Lemma~\ref{lemma:subGaussianity_parameters}, that the martingale difference sequence follows a coarser notion of subGaussianity (defined formally in Definition~\ref{definition:new_subGaussian_def}), which allows us to develop concentration bounds, while still being able to use the variance provided in Lemma~\ref{lemma:martingale_variance_bound}.
% However, as shown in Lemma~\ref{lemma:subGaussianity_parameters}, the subGaussianity parameters depend on the data dimension, $d$ along with $G_i$ and the step size, $\Delta$. Therefore, performing a concentration argument solely relying on this observation leads to a dimension-dependent bound.\bk

%\ps{Shouldn't we move the subGaussianity stuff to the notation/setup section? This part is already math-heavy.}

Putting together Lemma~\ref{lemma:subGaussianity_parameters} and Lemma~\ref{lemma:martingale_variance_bound}, we provide a general concentration tool for martingale difference sequences with bounded variance and subGaussianity in
Lemma~\ref{lemma:martingale_concentration_lemma}, which may be of independent interest. We follow a similar proof strategy via a supermartingale argument as in the proof Freedman's inequality (see for e.g. \cite{tropp2011freedman}), but diverge in dealing with subGaussianity instead of almost surely bounded random variables.

\begin{restatable}{lemma}{martingaleconcentrationlemma}\label{lemma:martingale_concentration_lemma}
Let $M_{n} = \sum_{i=1}^{n}\langle G_i, Y_i - \bE[Y_i|\mathcal{F}_{i-1}] \rangle, M_{0} = 1$ and define the filtration $\left\{\mathcal{F}_{i}\right\}_{i \in [n]}$ such that:
\begin{enumerate}
    \item  $G_i$ is $\mathcal{F}_{i-1}$ measurable.
    \item  $\langle G_i,Y_i-\bE [Y_i|\mathcal{F}_{i-1}]\rangle$ is $(\beta_i^{2}\|G_i\|^{2}, K_i)$ sub-Gaussian conditioned on $\mathcal{F}_{i-1}$ (where $\beta_i,K_i$ are random variables measurable with respect to $\mathcal{F}_{i-1}$)
    \item $\mathsf{var}(\langle G_i, Y_i - \bE[Y_i|\mathcal{F}_{i-1}]\rangle|\mathcal{F}_{i-1}) \leq \nu^2_i \|G_i\|^2$ and define $J_i := \max(1,\frac{1}{K_i}\log\frac{\beta_i^2 K_i}{\nu^2_i})$.
\end{enumerate}
Pick a $\lambda > 0$ and let $\mathcal{A}_i(\lambda) = \{\lambda J_i\|G_i\|\beta_i\sqrt{K_i} \leq c_0\}$ for some small enough universal constant $c_0$. Then, there exists a universal constant $C > 0$ such that:
\begin{enumerate}
    \item $\exp(\lambda M_{n} - C\lambda^2\sum_{i=1}^{n}\nu_i^2\|G_i\|^2 )\prod_{i=1}^{n}\mathbbm{1}(\mathcal{A}_i(\lambda))$ is a super-martingale with respect to the filtration $\mathcal{F}_{i}$
    \item $\forall v > 0, \; \bP(\{\lambda M_n > C \lambda^2\sum_{i=1}^{n}\nu_i^2\|G_i\|^2 + v\}\cap_{i=1}^{n} \mathcal{A}_i(\lambda)) \leq \exp(-v)$
\end{enumerate}
\end{restatable}

% \ps{I think you have nearly everything for a time uniform bound. In other words is it possible to have $P(sup_n blah_n>something)<something$? I think most freedman's inequalities are time uniform} \syamantak{We can try this for the appendix version.}

Observe that the concentration result developed in Lemma~\ref{lemma:martingale_concentration_lemma} has two parts. Optimizing over the choice of $\lambda$, it can be shown that the bound on $M_{n}$ depends on two terms: (1) an $\ell_{2}$ term,  $\sum_{i\in[n]}\nu_{i}^{2}\norm{G_i}^{2}$ and (2) an $\ell_{\infty}$ term, $\sup_{i \in [n]}J_{i}\norm{G_{i}}\beta_{i}\sqrt{K_i}$. When applied in our context, these two terms in turn depend on norms, $\norm{f-s}_{2}$ and $\norm{f-s}_{\infty}$. This is where the time-regularity assumption in Assumption~\ref{assumption:score_function_smoothness} plays a crucial role in our analysis. Specifically, it enables us to bridge the $\ell_{\infty}$ and $\ell_{2}$ norm bounds derived from the martingale concentration results in Lemma~\ref{lemma:martingale_concentration_lemma}. The proof of Lemma~\ref{lemma:l2_linfnity_bound_f} leverages this assumption to relate $\norm{f\bb{t+k\Delta, x_{t+k\Delta}}}_{2}$ to $\norm{f\bb{t, x_{t}}}_{2}$, as shown by:
% \dn{the below needs to conform to our modified assumption (i.e, add log factors) or mention that we are skipping it for brevity} \syamantak{Fixed.}
\bas{ \norm{f\bb{t+k\Delta, x_{t+k\Delta}}}_2 - e^{k\Delta}\norm{f\bb{t, x_{t}}}_{2} \geq - \tilde{\Omega}(L\sqrt{dk\Delta}). } Exploiting this  property over a carefully selected range of $k$ values allows us to relate $\ell_{\infty}$ and $\ell_{2}$ norm bounds as we show in the following Lemma. 

\begin{restatable}{lemma}{ltwolinfinityerrorboundtimeregularity}\label{lemma:l2_linfnity_bound_f}
    Under Assumption~\ref{assumption:score_function_smoothness}, with probability $1-\delta$, for a universal constant $C  > 0$ the following holds uniformly for every $f \in \cF$:
    \bas{
       &\biggr(\sup_{\substack{i \in [m]\\ j \in [N]}}\norm{f\bb{t_j,x_{t_j}}-s\bb{t_j,x_{t_j}}}_{2}\biggr)^{2} \leq C\Delta^{\frac{1}{3}}\biggr(\sum_{\substack{i\in [m] \\ j \in [N]}}\norm{f\bb{t_j,x_{t_j}}-s\bb{t_j,x_{t_j}}}_{2}^{2}\biggr) + CL^{2}d\Delta^{\frac{1}{3}}\log(\frac{Nm}{\delta})
    }
\end{restatable}
The above lemma establishes that the simultaneous analysis of all timesteps harnesses the smoothness across time. In the absence of this approach, the smoothness assumption in the $x_{t}$-space would lack dependence on $\Delta$ and could grow as large as the Lipschitz constant $L$. This is essential for establishing nearly dimension-independent bounds.
% \rd 
% Importance of the same network across timesteps to prove our bounds.
% \subsection{Martingale Decomposition of the error}
% \subsection{Variances of the martingale difference sequence}
% \subsection{Concentration Argument}: We have variance and subGaussianity with a slightly worse parameter dependent on d. How to combine?
% \subsection{Why is time regularity important?} 
% \bk
\vspace{-5pt}
\section{Bootstrapped Score Matching}
\label{sec:bootstrapped_score_matching}

In Section~\ref{sec:technical_results},  we used time regularity and could prove nearly $d$-independent bounds. Learning with the same function class across timesteps, along with time regularity of the function class (Assumption~\ref{assumption:score_function_smoothness}) was critical to our proof.

In this section, we ask whether it is possible to exploit the dependence across timesteps explicitly and reduce variance in estimation. Using the Markovian nature of \eqref{eq:fwd_noise}, we show that for any $t' < t$ and $\alpha_t \in \R$, $s\bb{t, x_t} = \E[\tilde{y}_{t}|x_t]$ for $\tilde{y}_{t} := -\frac{z_t}{\sigma_t^{2}} - \alpha_t(s\bb{t', x_{t'}} - \frac{-z_{t'}}{\sigma_{t'}^{2}})$. This shows that $\tilde{y}_{t}$ can also be used to construct a learning target for the score function. This is in contrast to the target $y_{t} := -\frac{z_t}{\sigma_t^{2}}$ used in \eqref{eq:dsm_total}. The advantage of $\tilde{y}_{t}$ over $y_t$ is in the lower variance of $\tilde{y}_{t}$, as shown in Lemma~\ref{lemma:bootstrap_properties} (proved in Lemmas~\ref{lemma:bootstrap_consistency_appendix}, \ref{lemma:bootstrap_variance}). 

\begin{lemma}[Bootstrap Properties]\label{lemma:bootstrap_properties} Let $\tilde{r}_{t} := \tilde{y}_{t}-s(t, x_t)$. For $t' < t$, let $\Delta := t-t'$ and $\alpha_t := \frac{e^{-\Delta}\sigma_{t'}^{2}}{\sigma_{t}^{2}}$. Then, under Assumption~\ref{assumption:score_function_smoothness}, we have
\bas{
   \E\bbb{\tilde{r}_{t}|x_{t}} = 0 \text{ and } \normop{\E[\tilde{r}_{t}\tilde{r}_{t}^{\top}|x_{t}]}=O\bb{\frac{(L^{2}+1)\Delta}{\sigma_{t}^{4}}}
}
\end{lemma}

To compare with $y_{t} = \frac{-z_t}{\sigma_{t}^{2}}$, we note that an application of the second order tweedie's formula along with Assumption~\ref{assumption:score_function_smoothness} shows the variance $\normop{\E[(y_t-s(t, x_t))(y_t-s(t, x_t))^{\top}|x_{t}]}$ to be of the order $O(\frac{L+1}{\sigma_{t}^{2}})$. Therefore, although both $y_{t}$ and $\tilde{y}_{t}$ are unbiased, the variance of $\tilde{y}_{t}$ has an additional step size ($\Delta$) factor in the numerator (see Lemma~\ref{lemma:bootstrap_properties}) 

Intuitively, this is due to the correlation between $z_{t}, z_{t
'}$ induced by the SDE \eqref{eq:fwd_noise} which removes a lot of extraneous noise, reducing the variance significantly. Recent work due to \cite{de2024target} also presents a similar idea. They show the related result $ s\bb{t, x_t} = e^{t-t'}\E\bbb{s\bb{t', x_t'}|x_t}$, which further offer a lower variance estimator of $s_t$, provided $t-t'$ is small. However, the focus of our approach is significantly different compared to theirs. They focus on monte-carlo sampling assuming access to the true initial score function, $s\bb{t_{0}, x} := \nabla\log\bb{p_{0}\bb{x}}$. In contrast, we show how to use these low variance estimates for efficient training of diffusion models. For simplicity, we present the details of the algorithm assuming a different function class, $\cF_{k}$ for each timestep $t_{k}$, but our ideas extend naturally to jointly learning across all timesteps with a shared function class as well, as described in \eqref{eq:dsm_total}.





The primary challenge with this approach is that in case of diffusion models, we do not have access to the true score function $s(t',.)$ for $t' < t$. Instead as we move along the trajectory, we learn score estimates $\hat{s}_{t}$. Therefore, we plug in $\hat{s}_{t'}$ in $\tilde{y}_{t}$ instead of the true score function, $s(t',.)$. This in-turn induces a bias at the cost of a reduced variance, which we trade-off using the parameter, $\alpha_t$, to achieve a better $\ell_2$-error of the score estimate. Our Algorithm, referred to as Bootstrapped Score Matching (BSM) captures this idea and is described in detail in the next paragraph. 

\begin{figure*}[hbt]
    \centering
    % First image
    \begin{minipage}[b]{0.4\textwidth}
        \centering
        \includegraphics[width=\textwidth]{images/Gaussian_experiment_bsm.png}
        \vspace{1mm} % Optional spacing
        (a) L2 error for a multivariate Gaussian density
        \label{fig:gaussian_experiment_bsm}
    \end{minipage}
    \hspace{0.05\textwidth} % Horizontal spacing between the two images
    % Second image
    \begin{minipage}[b]{0.4\textwidth}
        \centering
        \includegraphics[width=\textwidth]{images/GMM_experiment_bsm.png}
        \vspace{1mm} % Optional spacing
        (b) Empirical density for a mixture of Gaussians
        \label{fig:gmm_experiment_bsm}
    \end{minipage}
    \caption{\label{fig:bsm_experiments} Experiments with Bootstrapped Score Matching. (a) represents the L2 error at each timestep while performing score estimation for a multivariate Gaussian density. In this case, since the score function is linear, \eqref{eq:dsm_total} can be solved exactly without a neural network. We note that BSM significantly enhances the quality of the score function. (b) explores multimodal densities, specifically a mixture of Gaussians. Here, we use a 3-layer neural network to represent the score function and plot the empirical density learned by using \eqref{eq:reverse_sde} with different score estimation algorithms. We note that using score bootstrapping significantly enhances the proportional representation of the minor mode, leading to a fair output. We provide details of the experimental setup in the Appendix Section~\ref{appendix:bsm}.}
\end{figure*} 

The BSM algorithm operates sequentially over a discretized time horizon $0 = t_0 < t_1 < \cdots < t_N = T$ and builds upon the principles of DSM while introducing a novel bootstrapping mechanism to mitigate the increasing variance of the DSM loss in later timesteps. Given a dataset $D = \{x_0^{(i)}\}_{i \in [m]}$ sampled from the data distribution, the perturbed samples at timestep $t_k$ are generated as $x_{t_k}^{(i)} = x_{0}^{(i)} e^{-t_k} + z_{t_k}^{(i)}, \quad z_{t_k}^{(i)} \sim \mathcal{N}(0, \sigma_{t_k}^2 I)$ where $\sigma_{t_k}^2 = 1 - e^{-2t_k}$. The task at each timestep $t_k$ is to estimate an approximate score function $\hat{s}_{t_k}(x)$ to optimize $\mathbb{E}_{x_{t_k}}[\|s(t_k, x) - \hat{s}_{t_k}(x)\|_2^2]$. For the initial timesteps $t_k$ with $k \leq k_0$, the algorithm employs DSM. The score function $\hat{s}_{t_k}$ is obtained by solving:
\[
\hat{s}_{t_k} = \arg\min_{f \in \cF_k}  \sum_{i \in [m]} \frac{\norm{f(t_k, x_{t_k}^{(i)}) - \frac{-z_{t_k}^{(i)}}{\sigma_{t_k}^2}}_2^2}{m},
\] 
For later timesteps $t_k$ with $k > k_0$, the algorithm transitions to BSM. At each timestep, the algorithm constructs bootstrapped targets $\tilde{y}_{t_k}^{(i)}$ by combining the DSM target $\frac{-z_{t_k}^{(i)}}{\sigma_{t_k}^2}$ with the previously estimated score $\hat{s}_{t_{k-1}}$. Specifically, the targets are defined as:
\bas{
\tilde{y}_{t_k}^{(i)} &= (1 - \alpha_k) \underbrace{\frac{-z_{t_k}^{(i)}}{\sigma_{t_k}^2}}_{\text{Unbiased Target}} + \alpha_k \left( \underbrace{\frac{-z_{t_k}^{(i)}}{\sigma_{t_k}^2} + \left(\hat{s}_{t_{k-1}}( x_{t_{k-1}}^{(i)}) - \frac{-z_{t_{k-1}}^{(i)}}{\sigma_{t_{k-1}}^2} \right)}_{\text{Biased Target}} \right)
}
where $\alpha_k = e^{-\gamma_k} \sqrt{\frac{1 - e^{-2t_{k-1}}}{1 - e^{-2t_k}}}$, with $\gamma_k = t_k - t_{k-1}$. Given access to the true score function, $s(t_{k-1}, .)$, then $\tilde{y}_{t_k}^{(i)}$ would form an unbiased target with lower variance, as shown in Lemma~\ref{lemma:bootstrap_properties}. However, since we only have access to the estimated score function, $\hat{s}_{t_{k-1}}$ at the previous timestep, $\tilde{y}_{t_k}^{(i)}$ is a biased target, and the parameter $\alpha_{k}$ weighs between the biased and unbiased targets. The score function, $\hat{s}_{t_k}$, is then learned as:
\bas{
    \hat{s}_{t_k} \leftarrow \arg\min_{f \in \cF_{k}}\sum_{i\in[m]}\frac{\norm{f(t_k, x_{t_k}^{(i)}) - \tilde{y}_{t_k}^{(i)}}_{2}^{2}}{m}
}
Figure~\ref{fig:bsm_experiments} presents numerical experiments that show the empirical advantage of our proposed score-bootstrap procedure. The formal pseudocode is provided in Algorithm~\ref{alg:bsm_algorithm} in Appendix Section~\ref{appendix:bsm}.
% \syamantak{Remove debiasing.}
% Next, to control the bias effectively to avoid it's propagation across timesteps, we propose a final debiasing step. To this end, we define a confidence set, $\mathcal{G}_{k}$, of feasible solutions which are not too far away from $\tilde{f}_{t_k}$, \bas{\mathcal{G}_{k} := \{f \in \cF_{k} | \frac{\sum_{i \in [m]}\norm{\tilde{f}(t_k, x_{t_k}^{(i)})-f(t_k, x_{t_k}^{(i)})}_{2}^{2}}{m}  \leq \frac{\nu_k^{2}d\log\left(|\mathcal{\cF}|m/\delta\right)}{m}\}}
% The score function $\hat{s}_{t_k}$ is  learned by minimizing the mean squared error between $\tilde{y}_{t_k}^{(i)}$ and the model predictions, subject to $\mathcal{G}_k$, as:
% \bas{
% \hat{s}_{t_k} &= \arg\min_{f \in \mathcal{G}_k} \frac{1}{m} \sum_{i \in [m]} \|f(t_k, x_{t_k}^{(i)}) - \tilde{y}_{t_k}^{(i)}\|_2^2 \\
% &\;\;\;\;\;\;\;\;\;\;\;\;  +  \frac{2\langle f(t_k, x_{t_k}^{(i)}) - \tilde{f}(t_k, x_{t_k}^{(i)}), \tilde{f}(t_k, x_{t_k}^{(i)}) - y_{t_k}^{(i)} \rangle}{m}.
% }
% The debiasing step uses the unbiased target $y_{t_k}^{(i)} := \frac{-z_{t_k}^{(i)}}{\sigma_{t_k}^{2}}$ to reduce the bias of $\tilde{f}_{t_k}$, while also ensuring the quality of the estimate by restricting the solution space to $\mathcal{G}_k$. 


\section{Conclusion}
In this work, we propose a simple yet effective approach, called SMILE, for graph few-shot learning with fewer tasks. Specifically, we introduce a novel dual-level mixup strategy, including within-task and across-task mixup, for enriching the diversity of nodes within each task and the diversity of tasks. Also, we incorporate the degree-based prior information to learn expressive node embeddings. Theoretically, we prove that SMILE effectively enhances the model's generalization performance. Empirically, we conduct extensive experiments on multiple benchmarks and the results suggest that SMILE significantly outperforms other baselines, including both in-domain and cross-domain few-shot settings.

\section*{Acknowledgments}
We gratefully acknowledge NSF grants 2217069, 2019844, and DMS 2109155. Additionally, part of this work was carried out while Syamantak was an intern at Google DeepMind.

\clearpage

\bibliography{refs}
\bibliographystyle{alpha}

\newpage
\onecolumn
\subsection{Lloyd-Max Algorithm}
\label{subsec:Lloyd-Max}
For a given quantization bitwidth $B$ and an operand $\bm{X}$, the Lloyd-Max algorithm finds $2^B$ quantization levels $\{\hat{x}_i\}_{i=1}^{2^B}$ such that quantizing $\bm{X}$ by rounding each scalar in $\bm{X}$ to the nearest quantization level minimizes the quantization MSE. 

The algorithm starts with an initial guess of quantization levels and then iteratively computes quantization thresholds $\{\tau_i\}_{i=1}^{2^B-1}$ and updates quantization levels $\{\hat{x}_i\}_{i=1}^{2^B}$. Specifically, at iteration $n$, thresholds are set to the midpoints of the previous iteration's levels:
\begin{align*}
    \tau_i^{(n)}=\frac{\hat{x}_i^{(n-1)}+\hat{x}_{i+1}^{(n-1)}}2 \text{ for } i=1\ldots 2^B-1
\end{align*}
Subsequently, the quantization levels are re-computed as conditional means of the data regions defined by the new thresholds:
\begin{align*}
    \hat{x}_i^{(n)}=\mathbb{E}\left[ \bm{X} \big| \bm{X}\in [\tau_{i-1}^{(n)},\tau_i^{(n)}] \right] \text{ for } i=1\ldots 2^B
\end{align*}
where to satisfy boundary conditions we have $\tau_0=-\infty$ and $\tau_{2^B}=\infty$. The algorithm iterates the above steps until convergence.

Figure \ref{fig:lm_quant} compares the quantization levels of a $7$-bit floating point (E3M3) quantizer (left) to a $7$-bit Lloyd-Max quantizer (right) when quantizing a layer of weights from the GPT3-126M model at a per-tensor granularity. As shown, the Lloyd-Max quantizer achieves substantially lower quantization MSE. Further, Table \ref{tab:FP7_vs_LM7} shows the superior perplexity achieved by Lloyd-Max quantizers for bitwidths of $7$, $6$ and $5$. The difference between the quantizers is clear at 5 bits, where per-tensor FP quantization incurs a drastic and unacceptable increase in perplexity, while Lloyd-Max quantization incurs a much smaller increase. Nevertheless, we note that even the optimal Lloyd-Max quantizer incurs a notable ($\sim 1.5$) increase in perplexity due to the coarse granularity of quantization. 

\begin{figure}[h]
  \centering
  \includegraphics[width=0.7\linewidth]{sections/figures/LM7_FP7.pdf}
  \caption{\small Quantization levels and the corresponding quantization MSE of Floating Point (left) vs Lloyd-Max (right) Quantizers for a layer of weights in the GPT3-126M model.}
  \label{fig:lm_quant}
\end{figure}

\begin{table}[h]\scriptsize
\begin{center}
\caption{\label{tab:FP7_vs_LM7} \small Comparing perplexity (lower is better) achieved by floating point quantizers and Lloyd-Max quantizers on a GPT3-126M model for the Wikitext-103 dataset.}
\begin{tabular}{c|cc|c}
\hline
 \multirow{2}{*}{\textbf{Bitwidth}} & \multicolumn{2}{|c|}{\textbf{Floating-Point Quantizer}} & \textbf{Lloyd-Max Quantizer} \\
 & Best Format & Wikitext-103 Perplexity & Wikitext-103 Perplexity \\
\hline
7 & E3M3 & 18.32 & 18.27 \\
6 & E3M2 & 19.07 & 18.51 \\
5 & E4M0 & 43.89 & 19.71 \\
\hline
\end{tabular}
\end{center}
\end{table}

\subsection{Proof of Local Optimality of LO-BCQ}
\label{subsec:lobcq_opt_proof}
For a given block $\bm{b}_j$, the quantization MSE during LO-BCQ can be empirically evaluated as $\frac{1}{L_b}\lVert \bm{b}_j- \bm{\hat{b}}_j\rVert^2_2$ where $\bm{\hat{b}}_j$ is computed from equation (\ref{eq:clustered_quantization_definition}) as $C_{f(\bm{b}_j)}(\bm{b}_j)$. Further, for a given block cluster $\mathcal{B}_i$, we compute the quantization MSE as $\frac{1}{|\mathcal{B}_{i}|}\sum_{\bm{b} \in \mathcal{B}_{i}} \frac{1}{L_b}\lVert \bm{b}- C_i^{(n)}(\bm{b})\rVert^2_2$. Therefore, at the end of iteration $n$, we evaluate the overall quantization MSE $J^{(n)}$ for a given operand $\bm{X}$ composed of $N_c$ block clusters as:
\begin{align*}
    \label{eq:mse_iter_n}
    J^{(n)} = \frac{1}{N_c} \sum_{i=1}^{N_c} \frac{1}{|\mathcal{B}_{i}^{(n)}|}\sum_{\bm{v} \in \mathcal{B}_{i}^{(n)}} \frac{1}{L_b}\lVert \bm{b}- B_i^{(n)}(\bm{b})\rVert^2_2
\end{align*}

At the end of iteration $n$, the codebooks are updated from $\mathcal{C}^{(n-1)}$ to $\mathcal{C}^{(n)}$. However, the mapping of a given vector $\bm{b}_j$ to quantizers $\mathcal{C}^{(n)}$ remains as  $f^{(n)}(\bm{b}_j)$. At the next iteration, during the vector clustering step, $f^{(n+1)}(\bm{b}_j)$ finds new mapping of $\bm{b}_j$ to updated codebooks $\mathcal{C}^{(n)}$ such that the quantization MSE over the candidate codebooks is minimized. Therefore, we obtain the following result for $\bm{b}_j$:
\begin{align*}
\frac{1}{L_b}\lVert \bm{b}_j - C_{f^{(n+1)}(\bm{b}_j)}^{(n)}(\bm{b}_j)\rVert^2_2 \le \frac{1}{L_b}\lVert \bm{b}_j - C_{f^{(n)}(\bm{b}_j)}^{(n)}(\bm{b}_j)\rVert^2_2
\end{align*}

That is, quantizing $\bm{b}_j$ at the end of the block clustering step of iteration $n+1$ results in lower quantization MSE compared to quantizing at the end of iteration $n$. Since this is true for all $\bm{b} \in \bm{X}$, we assert the following:
\begin{equation}
\begin{split}
\label{eq:mse_ineq_1}
    \tilde{J}^{(n+1)} &= \frac{1}{N_c} \sum_{i=1}^{N_c} \frac{1}{|\mathcal{B}_{i}^{(n+1)}|}\sum_{\bm{b} \in \mathcal{B}_{i}^{(n+1)}} \frac{1}{L_b}\lVert \bm{b} - C_i^{(n)}(b)\rVert^2_2 \le J^{(n)}
\end{split}
\end{equation}
where $\tilde{J}^{(n+1)}$ is the the quantization MSE after the vector clustering step at iteration $n+1$.

Next, during the codebook update step (\ref{eq:quantizers_update}) at iteration $n+1$, the per-cluster codebooks $\mathcal{C}^{(n)}$ are updated to $\mathcal{C}^{(n+1)}$ by invoking the Lloyd-Max algorithm \citep{Lloyd}. We know that for any given value distribution, the Lloyd-Max algorithm minimizes the quantization MSE. Therefore, for a given vector cluster $\mathcal{B}_i$ we obtain the following result:

\begin{equation}
    \frac{1}{|\mathcal{B}_{i}^{(n+1)}|}\sum_{\bm{b} \in \mathcal{B}_{i}^{(n+1)}} \frac{1}{L_b}\lVert \bm{b}- C_i^{(n+1)}(\bm{b})\rVert^2_2 \le \frac{1}{|\mathcal{B}_{i}^{(n+1)}|}\sum_{\bm{b} \in \mathcal{B}_{i}^{(n+1)}} \frac{1}{L_b}\lVert \bm{b}- C_i^{(n)}(\bm{b})\rVert^2_2
\end{equation}

The above equation states that quantizing the given block cluster $\mathcal{B}_i$ after updating the associated codebook from $C_i^{(n)}$ to $C_i^{(n+1)}$ results in lower quantization MSE. Since this is true for all the block clusters, we derive the following result: 
\begin{equation}
\begin{split}
\label{eq:mse_ineq_2}
     J^{(n+1)} &= \frac{1}{N_c} \sum_{i=1}^{N_c} \frac{1}{|\mathcal{B}_{i}^{(n+1)}|}\sum_{\bm{b} \in \mathcal{B}_{i}^{(n+1)}} \frac{1}{L_b}\lVert \bm{b}- C_i^{(n+1)}(\bm{b})\rVert^2_2  \le \tilde{J}^{(n+1)}   
\end{split}
\end{equation}

Following (\ref{eq:mse_ineq_1}) and (\ref{eq:mse_ineq_2}), we find that the quantization MSE is non-increasing for each iteration, that is, $J^{(1)} \ge J^{(2)} \ge J^{(3)} \ge \ldots \ge J^{(M)}$ where $M$ is the maximum number of iterations. 
%Therefore, we can say that if the algorithm converges, then it must be that it has converged to a local minimum. 
\hfill $\blacksquare$


\begin{figure}
    \begin{center}
    \includegraphics[width=0.5\textwidth]{sections//figures/mse_vs_iter.pdf}
    \end{center}
    \caption{\small NMSE vs iterations during LO-BCQ compared to other block quantization proposals}
    \label{fig:nmse_vs_iter}
\end{figure}

Figure \ref{fig:nmse_vs_iter} shows the empirical convergence of LO-BCQ across several block lengths and number of codebooks. Also, the MSE achieved by LO-BCQ is compared to baselines such as MXFP and VSQ. As shown, LO-BCQ converges to a lower MSE than the baselines. Further, we achieve better convergence for larger number of codebooks ($N_c$) and for a smaller block length ($L_b$), both of which increase the bitwidth of BCQ (see Eq \ref{eq:bitwidth_bcq}).


\subsection{Additional Accuracy Results}
%Table \ref{tab:lobcq_config} lists the various LOBCQ configurations and their corresponding bitwidths.
\begin{table}
\setlength{\tabcolsep}{4.75pt}
\begin{center}
\caption{\label{tab:lobcq_config} Various LO-BCQ configurations and their bitwidths.}
\begin{tabular}{|c||c|c|c|c||c|c||c|} 
\hline
 & \multicolumn{4}{|c||}{$L_b=8$} & \multicolumn{2}{|c||}{$L_b=4$} & $L_b=2$ \\
 \hline
 \backslashbox{$L_A$\kern-1em}{\kern-1em$N_c$} & 2 & 4 & 8 & 16 & 2 & 4 & 2 \\
 \hline
 64 & 4.25 & 4.375 & 4.5 & 4.625 & 4.375 & 4.625 & 4.625\\
 \hline
 32 & 4.375 & 4.5 & 4.625& 4.75 & 4.5 & 4.75 & 4.75 \\
 \hline
 16 & 4.625 & 4.75& 4.875 & 5 & 4.75 & 5 & 5 \\
 \hline
\end{tabular}
\end{center}
\end{table}

%\subsection{Perplexity achieved by various LO-BCQ configurations on Wikitext-103 dataset}

\begin{table} \centering
\begin{tabular}{|c||c|c|c|c||c|c||c|} 
\hline
 $L_b \rightarrow$& \multicolumn{4}{c||}{8} & \multicolumn{2}{c||}{4} & 2\\
 \hline
 \backslashbox{$L_A$\kern-1em}{\kern-1em$N_c$} & 2 & 4 & 8 & 16 & 2 & 4 & 2  \\
 %$N_c \rightarrow$ & 2 & 4 & 8 & 16 & 2 & 4 & 2 \\
 \hline
 \hline
 \multicolumn{8}{c}{GPT3-1.3B (FP32 PPL = 9.98)} \\ 
 \hline
 \hline
 64 & 10.40 & 10.23 & 10.17 & 10.15 &  10.28 & 10.18 & 10.19 \\
 \hline
 32 & 10.25 & 10.20 & 10.15 & 10.12 &  10.23 & 10.17 & 10.17 \\
 \hline
 16 & 10.22 & 10.16 & 10.10 & 10.09 &  10.21 & 10.14 & 10.16 \\
 \hline
  \hline
 \multicolumn{8}{c}{GPT3-8B (FP32 PPL = 7.38)} \\ 
 \hline
 \hline
 64 & 7.61 & 7.52 & 7.48 &  7.47 &  7.55 &  7.49 & 7.50 \\
 \hline
 32 & 7.52 & 7.50 & 7.46 &  7.45 &  7.52 &  7.48 & 7.48  \\
 \hline
 16 & 7.51 & 7.48 & 7.44 &  7.44 &  7.51 &  7.49 & 7.47  \\
 \hline
\end{tabular}
\caption{\label{tab:ppl_gpt3_abalation} Wikitext-103 perplexity across GPT3-1.3B and 8B models.}
\end{table}

\begin{table} \centering
\begin{tabular}{|c||c|c|c|c||} 
\hline
 $L_b \rightarrow$& \multicolumn{4}{c||}{8}\\
 \hline
 \backslashbox{$L_A$\kern-1em}{\kern-1em$N_c$} & 2 & 4 & 8 & 16 \\
 %$N_c \rightarrow$ & 2 & 4 & 8 & 16 & 2 & 4 & 2 \\
 \hline
 \hline
 \multicolumn{5}{|c|}{Llama2-7B (FP32 PPL = 5.06)} \\ 
 \hline
 \hline
 64 & 5.31 & 5.26 & 5.19 & 5.18  \\
 \hline
 32 & 5.23 & 5.25 & 5.18 & 5.15  \\
 \hline
 16 & 5.23 & 5.19 & 5.16 & 5.14  \\
 \hline
 \multicolumn{5}{|c|}{Nemotron4-15B (FP32 PPL = 5.87)} \\ 
 \hline
 \hline
 64  & 6.3 & 6.20 & 6.13 & 6.08  \\
 \hline
 32  & 6.24 & 6.12 & 6.07 & 6.03  \\
 \hline
 16  & 6.12 & 6.14 & 6.04 & 6.02  \\
 \hline
 \multicolumn{5}{|c|}{Nemotron4-340B (FP32 PPL = 3.48)} \\ 
 \hline
 \hline
 64 & 3.67 & 3.62 & 3.60 & 3.59 \\
 \hline
 32 & 3.63 & 3.61 & 3.59 & 3.56 \\
 \hline
 16 & 3.61 & 3.58 & 3.57 & 3.55 \\
 \hline
\end{tabular}
\caption{\label{tab:ppl_llama7B_nemo15B} Wikitext-103 perplexity compared to FP32 baseline in Llama2-7B and Nemotron4-15B, 340B models}
\end{table}

%\subsection{Perplexity achieved by various LO-BCQ configurations on MMLU dataset}


\begin{table} \centering
\begin{tabular}{|c||c|c|c|c||c|c|c|c|} 
\hline
 $L_b \rightarrow$& \multicolumn{4}{c||}{8} & \multicolumn{4}{c||}{8}\\
 \hline
 \backslashbox{$L_A$\kern-1em}{\kern-1em$N_c$} & 2 & 4 & 8 & 16 & 2 & 4 & 8 & 16  \\
 %$N_c \rightarrow$ & 2 & 4 & 8 & 16 & 2 & 4 & 2 \\
 \hline
 \hline
 \multicolumn{5}{|c|}{Llama2-7B (FP32 Accuracy = 45.8\%)} & \multicolumn{4}{|c|}{Llama2-70B (FP32 Accuracy = 69.12\%)} \\ 
 \hline
 \hline
 64 & 43.9 & 43.4 & 43.9 & 44.9 & 68.07 & 68.27 & 68.17 & 68.75 \\
 \hline
 32 & 44.5 & 43.8 & 44.9 & 44.5 & 68.37 & 68.51 & 68.35 & 68.27  \\
 \hline
 16 & 43.9 & 42.7 & 44.9 & 45 & 68.12 & 68.77 & 68.31 & 68.59  \\
 \hline
 \hline
 \multicolumn{5}{|c|}{GPT3-22B (FP32 Accuracy = 38.75\%)} & \multicolumn{4}{|c|}{Nemotron4-15B (FP32 Accuracy = 64.3\%)} \\ 
 \hline
 \hline
 64 & 36.71 & 38.85 & 38.13 & 38.92 & 63.17 & 62.36 & 63.72 & 64.09 \\
 \hline
 32 & 37.95 & 38.69 & 39.45 & 38.34 & 64.05 & 62.30 & 63.8 & 64.33  \\
 \hline
 16 & 38.88 & 38.80 & 38.31 & 38.92 & 63.22 & 63.51 & 63.93 & 64.43  \\
 \hline
\end{tabular}
\caption{\label{tab:mmlu_abalation} Accuracy on MMLU dataset across GPT3-22B, Llama2-7B, 70B and Nemotron4-15B models.}
\end{table}


%\subsection{Perplexity achieved by various LO-BCQ configurations on LM evaluation harness}

\begin{table} \centering
\begin{tabular}{|c||c|c|c|c||c|c|c|c|} 
\hline
 $L_b \rightarrow$& \multicolumn{4}{c||}{8} & \multicolumn{4}{c||}{8}\\
 \hline
 \backslashbox{$L_A$\kern-1em}{\kern-1em$N_c$} & 2 & 4 & 8 & 16 & 2 & 4 & 8 & 16  \\
 %$N_c \rightarrow$ & 2 & 4 & 8 & 16 & 2 & 4 & 2 \\
 \hline
 \hline
 \multicolumn{5}{|c|}{Race (FP32 Accuracy = 37.51\%)} & \multicolumn{4}{|c|}{Boolq (FP32 Accuracy = 64.62\%)} \\ 
 \hline
 \hline
 64 & 36.94 & 37.13 & 36.27 & 37.13 & 63.73 & 62.26 & 63.49 & 63.36 \\
 \hline
 32 & 37.03 & 36.36 & 36.08 & 37.03 & 62.54 & 63.51 & 63.49 & 63.55  \\
 \hline
 16 & 37.03 & 37.03 & 36.46 & 37.03 & 61.1 & 63.79 & 63.58 & 63.33  \\
 \hline
 \hline
 \multicolumn{5}{|c|}{Winogrande (FP32 Accuracy = 58.01\%)} & \multicolumn{4}{|c|}{Piqa (FP32 Accuracy = 74.21\%)} \\ 
 \hline
 \hline
 64 & 58.17 & 57.22 & 57.85 & 58.33 & 73.01 & 73.07 & 73.07 & 72.80 \\
 \hline
 32 & 59.12 & 58.09 & 57.85 & 58.41 & 73.01 & 73.94 & 72.74 & 73.18  \\
 \hline
 16 & 57.93 & 58.88 & 57.93 & 58.56 & 73.94 & 72.80 & 73.01 & 73.94  \\
 \hline
\end{tabular}
\caption{\label{tab:mmlu_abalation} Accuracy on LM evaluation harness tasks on GPT3-1.3B model.}
\end{table}

\begin{table} \centering
\begin{tabular}{|c||c|c|c|c||c|c|c|c|} 
\hline
 $L_b \rightarrow$& \multicolumn{4}{c||}{8} & \multicolumn{4}{c||}{8}\\
 \hline
 \backslashbox{$L_A$\kern-1em}{\kern-1em$N_c$} & 2 & 4 & 8 & 16 & 2 & 4 & 8 & 16  \\
 %$N_c \rightarrow$ & 2 & 4 & 8 & 16 & 2 & 4 & 2 \\
 \hline
 \hline
 \multicolumn{5}{|c|}{Race (FP32 Accuracy = 41.34\%)} & \multicolumn{4}{|c|}{Boolq (FP32 Accuracy = 68.32\%)} \\ 
 \hline
 \hline
 64 & 40.48 & 40.10 & 39.43 & 39.90 & 69.20 & 68.41 & 69.45 & 68.56 \\
 \hline
 32 & 39.52 & 39.52 & 40.77 & 39.62 & 68.32 & 67.43 & 68.17 & 69.30  \\
 \hline
 16 & 39.81 & 39.71 & 39.90 & 40.38 & 68.10 & 66.33 & 69.51 & 69.42  \\
 \hline
 \hline
 \multicolumn{5}{|c|}{Winogrande (FP32 Accuracy = 67.88\%)} & \multicolumn{4}{|c|}{Piqa (FP32 Accuracy = 78.78\%)} \\ 
 \hline
 \hline
 64 & 66.85 & 66.61 & 67.72 & 67.88 & 77.31 & 77.42 & 77.75 & 77.64 \\
 \hline
 32 & 67.25 & 67.72 & 67.72 & 67.00 & 77.31 & 77.04 & 77.80 & 77.37  \\
 \hline
 16 & 68.11 & 68.90 & 67.88 & 67.48 & 77.37 & 78.13 & 78.13 & 77.69  \\
 \hline
\end{tabular}
\caption{\label{tab:mmlu_abalation} Accuracy on LM evaluation harness tasks on GPT3-8B model.}
\end{table}

\begin{table} \centering
\begin{tabular}{|c||c|c|c|c||c|c|c|c|} 
\hline
 $L_b \rightarrow$& \multicolumn{4}{c||}{8} & \multicolumn{4}{c||}{8}\\
 \hline
 \backslashbox{$L_A$\kern-1em}{\kern-1em$N_c$} & 2 & 4 & 8 & 16 & 2 & 4 & 8 & 16  \\
 %$N_c \rightarrow$ & 2 & 4 & 8 & 16 & 2 & 4 & 2 \\
 \hline
 \hline
 \multicolumn{5}{|c|}{Race (FP32 Accuracy = 40.67\%)} & \multicolumn{4}{|c|}{Boolq (FP32 Accuracy = 76.54\%)} \\ 
 \hline
 \hline
 64 & 40.48 & 40.10 & 39.43 & 39.90 & 75.41 & 75.11 & 77.09 & 75.66 \\
 \hline
 32 & 39.52 & 39.52 & 40.77 & 39.62 & 76.02 & 76.02 & 75.96 & 75.35  \\
 \hline
 16 & 39.81 & 39.71 & 39.90 & 40.38 & 75.05 & 73.82 & 75.72 & 76.09  \\
 \hline
 \hline
 \multicolumn{5}{|c|}{Winogrande (FP32 Accuracy = 70.64\%)} & \multicolumn{4}{|c|}{Piqa (FP32 Accuracy = 79.16\%)} \\ 
 \hline
 \hline
 64 & 69.14 & 70.17 & 70.17 & 70.56 & 78.24 & 79.00 & 78.62 & 78.73 \\
 \hline
 32 & 70.96 & 69.69 & 71.27 & 69.30 & 78.56 & 79.49 & 79.16 & 78.89  \\
 \hline
 16 & 71.03 & 69.53 & 69.69 & 70.40 & 78.13 & 79.16 & 79.00 & 79.00  \\
 \hline
\end{tabular}
\caption{\label{tab:mmlu_abalation} Accuracy on LM evaluation harness tasks on GPT3-22B model.}
\end{table}

\begin{table} \centering
\begin{tabular}{|c||c|c|c|c||c|c|c|c|} 
\hline
 $L_b \rightarrow$& \multicolumn{4}{c||}{8} & \multicolumn{4}{c||}{8}\\
 \hline
 \backslashbox{$L_A$\kern-1em}{\kern-1em$N_c$} & 2 & 4 & 8 & 16 & 2 & 4 & 8 & 16  \\
 %$N_c \rightarrow$ & 2 & 4 & 8 & 16 & 2 & 4 & 2 \\
 \hline
 \hline
 \multicolumn{5}{|c|}{Race (FP32 Accuracy = 44.4\%)} & \multicolumn{4}{|c|}{Boolq (FP32 Accuracy = 79.29\%)} \\ 
 \hline
 \hline
 64 & 42.49 & 42.51 & 42.58 & 43.45 & 77.58 & 77.37 & 77.43 & 78.1 \\
 \hline
 32 & 43.35 & 42.49 & 43.64 & 43.73 & 77.86 & 75.32 & 77.28 & 77.86  \\
 \hline
 16 & 44.21 & 44.21 & 43.64 & 42.97 & 78.65 & 77 & 76.94 & 77.98  \\
 \hline
 \hline
 \multicolumn{5}{|c|}{Winogrande (FP32 Accuracy = 69.38\%)} & \multicolumn{4}{|c|}{Piqa (FP32 Accuracy = 78.07\%)} \\ 
 \hline
 \hline
 64 & 68.9 & 68.43 & 69.77 & 68.19 & 77.09 & 76.82 & 77.09 & 77.86 \\
 \hline
 32 & 69.38 & 68.51 & 68.82 & 68.90 & 78.07 & 76.71 & 78.07 & 77.86  \\
 \hline
 16 & 69.53 & 67.09 & 69.38 & 68.90 & 77.37 & 77.8 & 77.91 & 77.69  \\
 \hline
\end{tabular}
\caption{\label{tab:mmlu_abalation} Accuracy on LM evaluation harness tasks on Llama2-7B model.}
\end{table}

\begin{table} \centering
\begin{tabular}{|c||c|c|c|c||c|c|c|c|} 
\hline
 $L_b \rightarrow$& \multicolumn{4}{c||}{8} & \multicolumn{4}{c||}{8}\\
 \hline
 \backslashbox{$L_A$\kern-1em}{\kern-1em$N_c$} & 2 & 4 & 8 & 16 & 2 & 4 & 8 & 16  \\
 %$N_c \rightarrow$ & 2 & 4 & 8 & 16 & 2 & 4 & 2 \\
 \hline
 \hline
 \multicolumn{5}{|c|}{Race (FP32 Accuracy = 48.8\%)} & \multicolumn{4}{|c|}{Boolq (FP32 Accuracy = 85.23\%)} \\ 
 \hline
 \hline
 64 & 49.00 & 49.00 & 49.28 & 48.71 & 82.82 & 84.28 & 84.03 & 84.25 \\
 \hline
 32 & 49.57 & 48.52 & 48.33 & 49.28 & 83.85 & 84.46 & 84.31 & 84.93  \\
 \hline
 16 & 49.85 & 49.09 & 49.28 & 48.99 & 85.11 & 84.46 & 84.61 & 83.94  \\
 \hline
 \hline
 \multicolumn{5}{|c|}{Winogrande (FP32 Accuracy = 79.95\%)} & \multicolumn{4}{|c|}{Piqa (FP32 Accuracy = 81.56\%)} \\ 
 \hline
 \hline
 64 & 78.77 & 78.45 & 78.37 & 79.16 & 81.45 & 80.69 & 81.45 & 81.5 \\
 \hline
 32 & 78.45 & 79.01 & 78.69 & 80.66 & 81.56 & 80.58 & 81.18 & 81.34  \\
 \hline
 16 & 79.95 & 79.56 & 79.79 & 79.72 & 81.28 & 81.66 & 81.28 & 80.96  \\
 \hline
\end{tabular}
\caption{\label{tab:mmlu_abalation} Accuracy on LM evaluation harness tasks on Llama2-70B model.}
\end{table}

%\section{MSE Studies}
%\textcolor{red}{TODO}


\subsection{Number Formats and Quantization Method}
\label{subsec:numFormats_quantMethod}
\subsubsection{Integer Format}
An $n$-bit signed integer (INT) is typically represented with a 2s-complement format \citep{yao2022zeroquant,xiao2023smoothquant,dai2021vsq}, where the most significant bit denotes the sign.

\subsubsection{Floating Point Format}
An $n$-bit signed floating point (FP) number $x$ comprises of a 1-bit sign ($x_{\mathrm{sign}}$), $B_m$-bit mantissa ($x_{\mathrm{mant}}$) and $B_e$-bit exponent ($x_{\mathrm{exp}}$) such that $B_m+B_e=n-1$. The associated constant exponent bias ($E_{\mathrm{bias}}$) is computed as $(2^{{B_e}-1}-1)$. We denote this format as $E_{B_e}M_{B_m}$.  

\subsubsection{Quantization Scheme}
\label{subsec:quant_method}
A quantization scheme dictates how a given unquantized tensor is converted to its quantized representation. We consider FP formats for the purpose of illustration. Given an unquantized tensor $\bm{X}$ and an FP format $E_{B_e}M_{B_m}$, we first, we compute the quantization scale factor $s_X$ that maps the maximum absolute value of $\bm{X}$ to the maximum quantization level of the $E_{B_e}M_{B_m}$ format as follows:
\begin{align}
\label{eq:sf}
    s_X = \frac{\mathrm{max}(|\bm{X}|)}{\mathrm{max}(E_{B_e}M_{B_m})}
\end{align}
In the above equation, $|\cdot|$ denotes the absolute value function.

Next, we scale $\bm{X}$ by $s_X$ and quantize it to $\hat{\bm{X}}$ by rounding it to the nearest quantization level of $E_{B_e}M_{B_m}$ as:

\begin{align}
\label{eq:tensor_quant}
    \hat{\bm{X}} = \text{round-to-nearest}\left(\frac{\bm{X}}{s_X}, E_{B_e}M_{B_m}\right)
\end{align}

We perform dynamic max-scaled quantization \citep{wu2020integer}, where the scale factor $s$ for activations is dynamically computed during runtime.

\subsection{Vector Scaled Quantization}
\begin{wrapfigure}{r}{0.35\linewidth}
  \centering
  \includegraphics[width=\linewidth]{sections/figures/vsquant.jpg}
  \caption{\small Vectorwise decomposition for per-vector scaled quantization (VSQ \citep{dai2021vsq}).}
  \label{fig:vsquant}
\end{wrapfigure}
During VSQ \citep{dai2021vsq}, the operand tensors are decomposed into 1D vectors in a hardware friendly manner as shown in Figure \ref{fig:vsquant}. Since the decomposed tensors are used as operands in matrix multiplications during inference, it is beneficial to perform this decomposition along the reduction dimension of the multiplication. The vectorwise quantization is performed similar to tensorwise quantization described in Equations \ref{eq:sf} and \ref{eq:tensor_quant}, where a scale factor $s_v$ is required for each vector $\bm{v}$ that maps the maximum absolute value of that vector to the maximum quantization level. While smaller vector lengths can lead to larger accuracy gains, the associated memory and computational overheads due to the per-vector scale factors increases. To alleviate these overheads, VSQ \citep{dai2021vsq} proposed a second level quantization of the per-vector scale factors to unsigned integers, while MX \citep{rouhani2023shared} quantizes them to integer powers of 2 (denoted as $2^{INT}$).

\subsubsection{MX Format}
The MX format proposed in \citep{rouhani2023microscaling} introduces the concept of sub-block shifting. For every two scalar elements of $b$-bits each, there is a shared exponent bit. The value of this exponent bit is determined through an empirical analysis that targets minimizing quantization MSE. We note that the FP format $E_{1}M_{b}$ is strictly better than MX from an accuracy perspective since it allocates a dedicated exponent bit to each scalar as opposed to sharing it across two scalars. Therefore, we conservatively bound the accuracy of a $b+2$-bit signed MX format with that of a $E_{1}M_{b}$ format in our comparisons. For instance, we use E1M2 format as a proxy for MX4.

\begin{figure}
    \centering
    \includegraphics[width=1\linewidth]{sections//figures/BlockFormats.pdf}
    \caption{\small Comparing LO-BCQ to MX format.}
    \label{fig:block_formats}
\end{figure}

Figure \ref{fig:block_formats} compares our $4$-bit LO-BCQ block format to MX \citep{rouhani2023microscaling}. As shown, both LO-BCQ and MX decompose a given operand tensor into block arrays and each block array into blocks. Similar to MX, we find that per-block quantization ($L_b < L_A$) leads to better accuracy due to increased flexibility. While MX achieves this through per-block $1$-bit micro-scales, we associate a dedicated codebook to each block through a per-block codebook selector. Further, MX quantizes the per-block array scale-factor to E8M0 format without per-tensor scaling. In contrast during LO-BCQ, we find that per-tensor scaling combined with quantization of per-block array scale-factor to E4M3 format results in superior inference accuracy across models. 



\end{document}
