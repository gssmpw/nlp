\section{Related Work}
Classifier-Free Guidance (CFG) has been a topic of interest in recent research, with the original work introducing CFG \citep{ho2022classifier} highlighting the trade-off between image quality, measured by Fréchet inception distance (FID, \citet{heusel2017gans}), and diversity, measured by inception score \citep{salimans2016improved} when adjusting the guidance strength parameter $\omega$. Since then, a significant body of research has examined CFG from various perspectives.

\paragraph{Theoretical works on CFG.} From a theoretical standpoint, several works also employed Gaussian mixture models to analyze diffusion and guidance, including \citet{chidambaram2024does, shah2023learning, liang2024unraveling, cui2023analysis, bai2024expectation}. In contrast, \citet{du2023reduce} explored alternative approaches to conditioning, modifying, and reusing diffusion models for compositional generation and guidance tasks. \citet{bradley2024classifier} characterized CFG as a predictor-corrector \citep{song2020score}, positioning it within a broader context of sampling approaches in order to improve its theoretical understanding, similar to the approach in this paper, however from the perspective of denoising and sharpening processes.


\paragraph{CFG variants and experimental analyses.} Among experimental evaluations of CFG, \citet{karras2024guiding} showed that guiding generation using a smaller, less-trained version of the model itself can achieve disentangled control over image quality without compromising variation. \citet{kynkaanniemi2024applying} proposed applying CFG in a limited interval, and \citet{wang2024analysis} proposed using weight schedulers for the classifier strength parameter. Several alternatives to standard CFG have been proposed, such as rectified classifier guidance \citep{xia2024rectified} using pre-computed guidance coefficients, projected score guidance \citep{kadkhodaie2024feature} pushing the image feature vector toward a feature centroid of the target class, characteristic guidance \citep{zheng2023characteristic} as a non-linear correction of CFG obtained using numerical solvers, and second-order CFG \citep{sun2023inner} assuming locally-cone shaped condition space. All these variants can be directly combined with our proposed generalization of CFG.


\looseness=-1\paragraph{Dynamical regimes, statistical physics and high-dimensional settings.} Akin to this work, several works recently studied dynamical regimes of diffusion models, primarily focusing on the standard, non-CFG version \citep{biroli2023generative,raya2024spontaneous,biroli2024dynamical,sclocchi2024phase,yu2024nonequilbrium,li2024critical,aranguri2025optimizing}. 
Statistical physics methods have been particularly useful in analyzing high-dimensional settings, e.g., data drawn from the Curie-Weiss model \citep{biroli2023generative}, high-dimensional Gaussian mixtures \citep{biroli2024dynamical}, and hierarchical models \citep{sclocchi2024phase}. 
Other relevant statistical-physics studies include \citet{ghio2024sampling}, who provided a comprehensive theoretical comparison between flow, diffusion, and autoregressive models from a spin glass perspective; \citet{achilli2024losing}, who extended the theory of memorization in generative diffusion to manifold-supported data; \citet{cui2025precise,cui2023analysis}, who analyzed sample complexity for 
high-dimensional Gaussian mixtures. A rigorous formulation of diffusion models in infinite dimensional setting was developed by \citet{pidstrigach2023infinitedimensionaldiffusionmodels}.  
 
%%%%%%%%%%%%%%%%%%%%%%%%%%%%%%