\section{Related Work}
Classifier-Free Guidance (CFG) has been a topic of interest in recent research, with the original work introducing CFG ____ highlighting the trade-off between image quality, measured by Fréchet inception distance (FID, ____), and diversity, measured by inception score ____ when adjusting the guidance strength parameter $\omega$. Since then, a significant body of research has examined CFG from various perspectives.

\paragraph{Theoretical works on CFG.} From a theoretical standpoint, several works also employed Gaussian mixture models to analyze diffusion and guidance, including ____. In contrast, ____ explored alternative approaches to conditioning, modifying, and reusing diffusion models for compositional generation and guidance tasks. ____ characterized CFG as a predictor-corrector ____, positioning it within a broader context of sampling approaches in order to improve its theoretical understanding, similar to the approach in this paper, however from the perspective of denoising and sharpening processes.


\paragraph{CFG variants and experimental analyses.} Among experimental evaluations of CFG, ____ showed that guiding generation using a smaller, less-trained version of the model itself can achieve disentangled control over image quality without compromising variation. ____ proposed applying CFG in a limited interval, and ____ proposed using weight schedulers for the classifier strength parameter. Several alternatives to standard CFG have been proposed, such as rectified classifier guidance ____ using pre-computed guidance coefficients, projected score guidance ____ pushing the image feature vector toward a feature centroid of the target class, characteristic guidance ____ as a non-linear correction of CFG obtained using numerical solvers, and second-order CFG ____ assuming locally-cone shaped condition space. All these variants can be directly combined with our proposed generalization of CFG.


\looseness=-1\paragraph{Dynamical regimes, statistical physics and high-dimensional settings.} Akin to this work, several works recently studied dynamical regimes of diffusion models, primarily focusing on the standard, non-CFG version ____. 
Statistical physics methods have been particularly useful in analyzing high-dimensional settings, e.g., data drawn from the Curie-Weiss model ____, high-dimensional Gaussian mixtures ____, and hierarchical models ____. 
Other relevant statistical-physics studies include ____, who provided a comprehensive theoretical comparison between flow, diffusion, and autoregressive models from a spin glass perspective; ____, who extended the theory of memorization in generative diffusion to manifold-supported data; ____, who analyzed sample complexity for 
high-dimensional Gaussian mixtures. A rigorous formulation of diffusion models in infinite dimensional setting was developed by ____.  
 
%%%%%%%%%%%%%%%%%%%%%%%%%%%%%%