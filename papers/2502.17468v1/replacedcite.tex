\section{Related work}
To improve BCI illiteracy and improve the performance and reliability of EEG-based BCIs, various machine learning methods have been developed. Some approaches focus on the user-BCI interaction, such as Vidaurre et al.'s co-adaptive learning using linear discriminant analysis, which reduces cross-subject performance variations through closed-loop feedback ____. Other research has concentrated on feature space alignment. For example, Wu et al. proposed a method to align EEG trials from different subjects in Euclidean space ____, while Tao et al. introduced a multi-kernel learning approach that aims to minimize feature distribution discrepancies and enhance class separability ____. As research progressed, deep learning and domain adaptation methods gained attention for their ability to extract common features across subjects. Li et al. proposed a multisource transfer learning method for EEG emotion recognition ____, and Zhao et al. introduced a deep representation-based domain adaptation (DRDA) method to leverage domain-invariant features ____. Jeon et al. extended this by using mutual information to refine feature selection ____. Hang et al. introduced a deep domain adaptation network (DDAN) that minimizes feature distribution discrepancies using the maximum mean discrepancy and improves classification accuracy across subjects ____.

In the context of RSVP-based BCIs, several studies have focused on improving cross-subject classification. Liu et al. proposed the Correlation Analysis Rank (CAR) algorithm, which improves performance by sorting the correlation between subjects, outperforming traditional random selection methods ____. Wang et al. introduced a multi-source domain adaptation-based tempo-spatial convolution network (MDA-TSC) to align feature distributions across subjects ____, while Zhang et al. developed a multilevel information fusion model to enhance EEG stability in dual-subject RSVP tasks ____.

Style transfer, originally applied in computer vision, has recently been adapted to classification of motor imagery based on EEG with promising results ____. Sun et al. proposed a subject transfer neural network (STNN) that directly transforms the data distribution of BCI-illiterate subjects into golden subjects ____. Building on this, Kim et al. further developed a subject-to-subject semantic style transfer network (SSSTN) that preserves content information from the target domain while transferring the style from the source domain ____. However, SSSTN methods have some limitations. A major issue is their overlook of class-specific information during the transfer process, which can be particularly problematic in cases of class imbalance or substantial class discrepancies. Another limitation is the reliance on the entire data set from the target subject without discussing the impact of the data set size on transfer performance. This could potentially increase the time and effort required for data collection. Additionally, SSSTN employs features from all convolutional layers during the transfer process, which may limit flexibility and increase computational time. To address these challenges, this study introduces style transfer for cross-subject RSVP detection and proposes extensions to SSSTN, which significantly improve the performance of class-specific transfer.