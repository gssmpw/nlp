\section{Related Work}
\label{sec:related}

Finite model finding has a longstanding tradition in automated
reasoning.  Sometimes, a user is interested in a model rather than
proving a theorem~\cite{mace2}. Models serve as counterexamples to
invalid conjectures~\cite{blanchette-lpar10}, which also appear in
formal methods, e.g.\ in Software
Verification~\cite{torlak-tacas07}. Finite models can also be used as
a semantic feature for \emph{lemma selection
  learning}~\cite{urban-ijcar08}.  A number of CP-based methods exists
that enumerate (all) models, cf.~\citet{choiwah-cubes}.
%
Finite models are also often constructed by dedicated
algorithms anchored in domain knowledge. The algebraic system
GAP~\cite{GAP4}  has a number of packages for specific types of algebraic
structures. The Small Groups library~\cite{besche-ijac02} contains
\emph{all} ($\approx 4\times 10^8$) non-isomorphic groups up to order 2000 (except for order 1024).
Similarly, the package Smallsemi~\cite{smallsemi} catalogues semigroups and the package LOOPS
catalogues loops~\cite{loops3.4.1}.

Normal forms are ubiquitous in computer science and mathematics, e.g.\
the system \nauty~\cite{McKayP14} uses canonical labeling to decide
isomorphism of graphs.
%
A large body of research exists on \emph{symmetry breaking} in SAT and
CP~\cite{cp-handbook,sakallah21}.  Computational complexity has been studied
under various notions of lex-leader~\cite{NarodWalshKats,orderings,Luks2004}. We
are not aware, however, of any study of lex-leader in the context of constraints
stemming from first order logic.
%
\citet{janota-mlex} tackle the calculation of the lex-leader for
\emph{one fixed given} algebra by using SAT\@.
%However, they do not
%show that the problem is NP-hard (see Canonizing Set Bounds section).

%
Some symmetry breaks are designed to
be fast, when used dynamically, or should add a small number of constraints,
when used statically~\cite{Codish2018,codish-aaai20}.
%
Such symmetry breaking is often partial such as the least number
heuristic (see Partial Symmetry Breaks section).  \citet{heule-mcs19}
explores optimal complete symmetry breaking for small
graphs~($\approx 5$ vertices) from a theoretical perspective.
\citet{szeider:cp21} develop a specific dynamic symmetry breaking,
called~\emph{SAT Modulo Symmetries}, where a SAT solver is enhanced to
look for the lexicographically smallest graph; similarly \citet{orderly} use \emph{orderly generation}
with the objective to enumerate graphs with certain properties.
%

For some structures, closed forms are known for calculating the number
of non-isomorphic objects.  For instance, a closed formula is known for
magmas~\cite{harrison1966}, and in the same paper, the author claims that a
closed formula for groups, monoids, and rings might be possible by modifying
the techniques he used, but so far, nobody has managed to find those formulas.
%

To give an idea of how difficult it is to say something about the size
sequence, we recall two old conjectures: almost all finite groups have size a
power of 2; almost all semigroups are 3-nilpotent (that is, semigroups with
zero, $0$, satisfying the identity $xyz=0$). The solution to the conjecture on
semigroups, widely believed to be true, was announced by~\citet{kleitman1976},
but the proof has a gap that nobody could fix so far.

Given the problems with the closed formula, mathematicians turn to
computational methods to find the first terms of the size sequence.
Traditionally, taking advantage of the deep knowledge of some class to trim the
search tree in a way that usually only works for that class.  Probably the
greatest achievement has been the computation of the number of order 10
semigroups, as the final piece of a long story: in 1955 it was computed up to
size 4~\cite{semigroups4}; in 1977, up to size 7~\cite{juergensen1977}; in
1994, for 8~\cite{satoh1994}; in 2010 for order 9 in Distler's PhD thesis, then
published in journal~\cite{distler2014}; and finally in 2012 for size
10~\cite{distler2012} (by using a combination of non-compact lex-leader
encoding and deep understanding of the problem). Roughly speaking, once a value
was computed, it took about 20 years to get the next one. The OEIS includes
many more size sequences and countless pointers to the bibliography. In
contrast, we introduce a general, out-of-the-box tool that improves upon the
existing methodologies for determining size sequences of algebraic structures.

More broadly, this paper is related to the SAT+CAS program, where SAT is
combined with \emph{computer algebra systems},
cf.~\citet{Williamson,bright-cacm22}.

%%%%%%%%%%%%%%%%%%%