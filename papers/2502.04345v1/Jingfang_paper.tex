%%%%%%%% ICML 2025 EXAMPLE LATEX SUBMISSION FILE %%%%%%%%%%%%%%%%%

\documentclass{article}
%\documentclass[accepted]{icml2025}

% Recommended, but optional, packages for figures and better typesetting:
\usepackage{microtype}
\usepackage{graphicx}
\usepackage{subfigure}
\usepackage{booktabs} % for professional tables
\usepackage{pifont}
\usepackage{color}
\usepackage{caption}   % 让表格标题更美观
\usepackage{adjustbox} % 控制表格宽度
\usepackage{array}     % 允许调整列宽
\newcommand{\red}[1]{\textcolor{red}{#1}}

% hyperref makes hyperlinks in the resulting PDF.
% If your build breaks (sometimes temporarily if a hyperlink spans a page)
% please comment out the following usepackage line and replace
% \usepackage{icml2025} with \usepackage[nohyperref]{icml2025} above.
\usepackage{hyperref}

% Attempt to make hyperref and algorithmic work together better:
\newcommand{\theHalgorithm}{\arabic{algorithm}}

% Use the following line for the initial blind version submitted for review:
%\usepackage{icml2025}

% If accepted, instead use the following line for the camera-ready submission:
\usepackage[accepted]{icml2025}

% For theorems and such
\usepackage{amsmath}
\usepackage{amssymb}
\usepackage{mathtools}
\usepackage{amsthm}

% if you use cleveref..
\usepackage[capitalize,noabbrev]{cleveref}

%%%%%%%%%%%%%%%%%%%%%%%%%%%%%%%%
% THEOREMS
%%%%%%%%%%%%%%%%%%%%%%%%%%%%%%%%
\theoremstyle{plain}
\newtheorem{theorem}{Theorem}[section]
\newtheorem{proposition}[theorem]{Proposition}
\newtheorem{lemma}[theorem]{Lemma}
\newtheorem{corollary}[theorem]{Corollary}
\theoremstyle{definition}
\newtheorem{definition}[theorem]{Definition}
\newtheorem{assumption}[theorem]{Assumption}
\theoremstyle{remark}
\newtheorem{remark}[theorem]{Remark}

% Todonotes is useful during development; simply uncomment the next line
%    and comment out the line below the next line to turn off comments
%\usepackage[disable,textsize=tiny]{todonotes}
\usepackage[textsize=tiny]{todonotes}


% The \icmltitle you define below is probably too long as a header.
% Therefore, a short form for the running title is supplied here:
\icmltitlerunning{JingFang: An Expert-Level Medical Diagnosis and Syndrome Differentiation-Based Treatment Model}

\begin{document}

\twocolumn[
%\icmltitle{JingFang: An expert-level medical diagnosis and syndrome differentiation-based treatment traditional Chinese medicine large language model}
\icmltitle{JingFang: A Traditional Chinese Medicine Large Language Model of Expert-Level Medical Diagnosis and Syndrome Differentiation-Based Treatment}

% It is OKAY to include author information, even for blind
% submissions: the style file will automatically remove it for you
% unless you've provided the [accepted] option to the icml2025
% package.

% List of affiliations: The first argument should be a (short)
% identifier you will use later to specify author affiliations
% Academic affiliations should list Department, University, City, Region, Country
% Industry affiliations should list Company, City, Region, Country

% You can specify symbols, otherwise they are numbered in order.
% Ideally, you should not use this facility. Affiliations will be numbered
% in order of appearance and this is the preferred way.
\icmlsetsymbol{equal}{*}

\begin{icmlauthorlist}
\icmlauthor{Yehan Yang}{equal,yyy}
\icmlauthor{Tianhao Ma}{equal,comp}
\icmlauthor{Ruotai Li}{yyy}
\icmlauthor{Xinhan Zheng}{yyy}
\icmlauthor{Guodong Shan}{yyy}
\icmlauthor{Chisheng Li}{yyy}
\end{icmlauthorlist}
\icmlaffiliation{yyy}{Beijing University of Posts and Telecommunications, Beijing, China}
\icmlaffiliation{comp}{Southeast University, Nanjing, China}
%\icmlaffiliation{sch}{School of Science, Beijing University of Posts and Telecommunications, Beijing, China}

\icmlcorrespondingauthor{Ruotai Li}{rtli@bupt.edu.cn}
% \icmlcorrespondingauthor{Firstname2 Lastname2}{first2.last2@www.uk}

% You may provide any keywords that you
% find helpful for describing your paper; these are used to populate
% the "keywords" metadata in the PDF but will not be shown in the document
\icmlkeywords{Traditional Chinese Medicine, Large Language Model, Chain of Thought, LLM Agent, Retrieval-augmented Generation}

\vskip 0.3in
]

% this must go after the closing bracket ] following \twocolumn[ ...

% This command actually creates the footnote in the first column
% listing the affiliations and the copyright notice.
% The command takes one argument, which is text to display at the start of the footnote.
% The \icmlEqualContribution command is standard text for equal contribution.
% Remove it (just {}) if you do not need this facility.

%\printAffiliationsAndNotice{} 
% leave blank if no need to mention equal contribution
\printAffiliationsAndNotice{\icmlEqualContribution} 
% otherwise use the standard text.

\begin{abstract}
Traditional Chinese medicine (TCM) plays a vital role in health protection and disease treatment, but its practical application requires extensive medical knowledge and clinical experience. Existing TCM Large Language Models (LLMs) exhibit critical limitations of uncomprehensive medical consultation and diagnoses, and inaccurate syndrome differentiation-based treatment. To address these issues, this study establishes JingFang (JF): a novel TCM Large Language Model that demonstrates the expert-level capability of medical diagnosis and syndrome differentiation-based treatment. We innovate a Multi-agent Dynamic Collaborative Chain-of-Thought Mechanism (MDCCTM) for medical consultation, enabling JF with effective and accurate diagnostic ability. In addition, a Syndrome Agent and a Dual-Stage Retrieval Scheme (DSRS) are developed to significantly enhance the capacity of JF for disease treatment based on syndrome differentiation. JingFang not only facilitates the application of LLMs but also promotes the effective practice of TCM in human health protection and disease treatment.

%Traditional Chinese medicine (TCM) plays a vital role in human health and disease treatment, but its effective practice requires that physicians possess extensive knowledge and experience to perform a comprehensive analysis of symptoms, accurate differentiation of syndromes, and appropriate treatment, which poses significant challenges for the application of TCM. The emergence of large language models (LLMs) offers practical solutions to these challenges, but existing TCM LLMs exhibit critical limitations: unprofessional medical consultation, uncomprehensive diagnoses, and inaccurate differentiation of TCM syndromes, which results in inappropriate medical advice. To address these issues, this study develops a novel TCM LLM -JingFang (JF) that demonstrates the professional ability of TCM from various aspects, especially possessing expert-level medical diagnosis syndrome differentiation-based treatment capabilities. Based on clinical guidelines from TCM, we develop a chain-of-thought (CoT) framework for clinical diagnosis, enabling JF with efficient and accurate diagnosis and treatment through a multi-agent collaboration mechanism. In addition, the model has been fine-tuned and pre-trained with multi-dimensional TCM knowledge, significantly enhancing its capability of syndrome differentiation-based medical treatment. Jingfang is a valid implementation of LLM technology in TCM, facilitating the application and development of TCM, and promoting its effective practice in human health protection and disease treatment. 
\end{abstract}

\begin{figure}[ht]
\vskip 0.2in
\begin{center}
\centerline{\includegraphics[width=\columnwidth]{Figure/JingFang.png}}
\caption{JF framework, including three main modules: TCM Consultation, TCM Syndrome Differentiation, and TCM Treatment.}
\label{jingfang-Framwork}
\end{center}
\vskip -0.2in
\end{figure}

\section{Introduction}
\label{Introduction}
Recently, the emergence of Large Language Models(LLMs) such as GPT-4\cite{1,hurst2024gpt}, Llama3\cite{2}, Deepseek-MOE\cite{3},Deepseek v2\cite{25}, and Qwen2.5\cite{4} has significantly advanced the field of natural language processing. These models not only possess exceptional language comprehension and generation capacity\cite{21}, but also demonstrate strong cross-domain adaptability\cite{5,6,7} by integrating techniques such as pre-training\cite{16}, fine-tuning\cite{17}, reinforcement learning\cite{18}, human feedback alignment\cite{19}, and knowledge graph retrieval\cite{20}. Among the numerous cross-domain applications of LLM, TCM has become one of the key focuses due to its importance in protecting human health and treating diseases\cite{22,23,24}, while the complexity of diagnosis and treatment of TCM poses a significant challenge.

In recent years, several studies have made some meaningful attempts, e.g. Zhongjing\cite{8}, TCMChat\cite{9}, Qibo \cite{10}, BianCang \cite{11}, MedChatZH\cite{12}, etc. However, the critical limitations of the effective application of LLM in the diagnosis and treatment of TCM still remain. Unlike Western medicine, the core characteristic of TCM lies in its emphasis on treatment based on syndrome differentiation, which requires physicians to implement comprehensive medical diagnosis and accurately analyze symptoms and signs in patients. The diagnosis of TCM, which is the foundation for syndrome differentiation and treatment, is largely based on multiple rounds of personalized consultation to precisely collect patient information, which is complicated and uncertain. The scarcity of large-scale, accurately labeled multi-round consultation data leads to not good enough performance of existing models in extracting chief complaints of patients, mining symptoms, precise syndrome differentiation, and treatment. Additionally, current TCM LLMs lack explicit control over the diagnostic reasoning and decision process,  which affects the model's interpretability and reliability. Existing models still have limitations in adaptability, scalability, flexibility, and maintenance cost, making it difficult to quickly adapt to changing diagnostic scenarios or different TCM knowledge systems.

To address these critical challenges and limitations, \textbf{the main contributions of this work are summarized below:} 
\begin{enumerate}
\item We develop \textbf{JF: An expert-level medical diagnosis and syndrome differentiation-based treatment TCM LLM} based on an innovative framework that integrates the technology of LLM Agent \cite{15}, Chain-of-Thought (CoT)\cite{13}, and Retrieval-Augmented Generation (RAG)\cite{14} to significantly improve the completeness and precision of the model in medical consultation and syndrome differentiation-based treatment, addressing limitations of existing TCM models and empowering practical applications of LLM in TCM.

\item \textbf{A Multi-agent Dynamic Collaborative Chain-of-Thought Mechanism (MDCCTM)} is invented to enable JF with dynamic reasoning and explicit decision-making capabilities for medical consultation. According to the mechanism, multiple agents with various TCM specialties are established and dynamically collaborate to conduct professional consultation for comprehensive information collection and accurate syndrome differentiation-based treatment. 

\item \textbf{ A Syndrome Agent and a Dual-Stage Retrieval Scheme (DSRS)} is  trained and developed correspondently based on the preconditioned multilevel knowledge of TCM, which significantly improves JF’s abilities of TCM
syndrome differentiation and treatment in practical application
\end{enumerate}


\section{Framework of JingFang}
In this section, we introduce the framework of the proposed JingFang (referred to as the JF framework). As shown in Figure \ref{jingfang-Framwork}, the JF framework is designed to consist of three main modules: \textbf{TCM Consultation}, \textbf{TCM Syndrome Differentiation}, and \textbf{TCM Treatment Reconmmendation}, according to the authentic process of TCM diagnosis and treatment. JF works well based on the process in the framework, following the designed MDCCTM and DSRS to fulfill the professional and accurate diagnosis and treatment.

\textbf{TCM Consultation Module} mainly involves the following steps:

%\ding{172} \textbf{Current Symptom Summary}: The TCM case agent summarizes the patient's symptoms and chief complaint according to the existing medical record and consultation. 

\ding{172} Expert Team Construction: The Medical Record Agent summarizes the patient's symptoms and basic information for generating the chief complaint. Based on the chief complaint, the relevant TCM Specialist Agents and a TCM General Agent are gathered to form the TCM expert team for the diagnosis.

\ding{173} Consultation Construction: Each agent in the expert team constructs a consultation CoT based on their professional knowledge and the patient's current medical condition, incorporating multiple follow-up questions to ensure the consultation process is logical and progressively in-depth.

\ding{174} Consultation Integration and Evaluation: Based on the consultation CoT provided by each agent, the Summary Agent integrates and evaluates the questions and key points within to form a summary of the consultation CoT.

\ding{175} Consultation Analysis and Optimization: Each agent in the expert team further analyzes the summary and proposes modification suggestions to the summarized consultation CoT, then sends them back to the Summary Agent until all the expert agents agree to officially begin the consultation.

\ding {176} Multi-Round Consultation: Based on the optimized consultation CoT, the TCM Consultation Agent conducts multiple rounds of consultation to further clarify the patient's condition and collect vital information for diagnosis and treatment.  

\textbf{TCM Syndrome Differentiation Module} is mainly for exact syndrome differentiation. Once sufficient key information is gathered from the consultation module, the trained TCM Syndrome Agent integrates and analyzes the data to determine the patient's syndrome type for the accurate treatment of diseases.

\textbf{TCM Treatment Recommendation Module} provides professional and targeted treatment recommendations for patients. The Treatment Agent is generated through a DRS to provide personalized treatment suggestions based on the patient's syndrome type and the patient's condition. For a detailed description of each module construction see Sec.\ref{Method}.

% \medskip
\section{Method}
\label{Method}
In this section, we introduce the innovative and well-structured  MDCCTM, DSRS, and the trained Syndrome Agent in detail.

\subsection{Multi-agent Dynamic Collaborative Chain-of-Thought Mechanism}
After an in-depth study of the TCM consultation process, we found that experienced TCM experts typically pre-set one or two follow-up consultation questions based on the patient's known condition, and make flexible and targeted adjustments to them based on more information during the consultation process. Inspired by this,  the MDCCTM is developed to better fit the actual TCM consultation process, and the prompts: $prompt_{[\ ]}$ here, are particularly designed to facilitate agents to implement the specific functions. The MDCCTM can be formulated as follows:

\subsubsection{TCM Expert Team Construction}
For comprehensive and targeted medical consultation, an Expert Agents Database consisting of TCM General Agents and TCM Specialist Agents from various fields including TCM internal medicine, surgery, gynecology, pediatrics, and so on, is constructed before medical consultation. 

When the consultation begins, a TCM Medical Record Agent summarizes the symptoms and basic information of the patient to generate the chief complaint $\mathit{M}$. According to $\mathit{M}$, a Manager Agent selects the relevant TCM Specialist Agent from the database to address the personal needs of patients. Each specialist expert $\mathit{ed_i}$ possesses the respective knowledge $\mathcal{K}_{ed_i}$, thus the TCM specialist agents domain corresponding to expert $\mathit{ed_i}$ can be represented via a mapping function $\mathit{f}_{LLM}$ as follows: 
\begin{equation}
    SD_i = f_{LLM}(M,\ K_{ed_i},\  prompt_{sd}).
\end{equation}
To ensure the comprehensiveness of consultation CoT, the expert team for consultation is supplemented with a TCM General Agent $GD = {gd, K_{g}}$, containing one TCM general expert $\mathit{gd}$ and the TCM general knowledge $K_{g}$. Therefore, the TCM expert team is constructed as:
\begin{equation}
    Team(M) = \{SD_i,\ GD\},
\end{equation}
which ensures both the pertinence and comprehensiveness of the consultation. The expert team for consultation is established only once during the consultation process for each patient.

\subsubsection{Consultation CoT Construction}
To start the CoT construction at the t-$th$ round of consultation, the TCM Medical Record Agent summarizes the patient’s current condition $Con_t$ from the current multi-round consultation result that is defined as: $C_t = \{(Q_k, A_k) \mid k \in [0, t]\}$, where $Q_k$,$A_k$ is the question and the corresponding answer at k-$th$ round of consultation. Thus, $Con_t$ can be defined as:
\begin{equation}
   Con_t = Agen_{rm}(C_t,\ prompt_{rm}).
\end{equation}

Each expert in the consultation expert team prioritizes 1-2 subsequent consultation CoT questions with explanations based on the current consultation result $C_t$, offering theoretical guidance for future consultations. Specifically, according to $C_t$, the TCM Specialized Agent $SD_i$ utilizes an LLM in conjunction with professional knowledge and specific prompts $prompt_{qa}$ to generate a consultation CoT $qa_{it}$ at t-$th$ round of consultation:
\begin{equation}
    qa_{it} = CoT_{sp}(C_t,\ SD_i,\ prompt_{qa})
\end{equation}
Meanwhile, the TCM General Agent $GD$ incorporates the TCM general knowledge to generate a comprehensive consultation CoT $ga$:
\begin{equation}
    ga_t = CoT_{gen}(C_t,\ GD,\ prompt_{ga})
\end{equation}
The expert team’s consultation CoT is constructed by combining the $qa_{it}$ and $ga_t$ at t-$th$ consultation :
\begin{equation}
    TCoT_t = (qa_{ti},\ ga_t)
\end{equation}
This consultation CoT lays the foundation for subsequent consultation processes. In the following steps, the expert team iteratively revises and optimizes $TCoT_t$ to make the consultation process more refined and targeted. 


\subsubsection{Consultation CoT Integration and Evaluation}

Integrating the widely used classical framework Ten Questions Song\cite{26} (TQS) that helps doctors assess the patient's condition from multiple dimensions, we establish a Consultation CoT Evaluation Algorithm (CCEA).  
\begin{algorithm}
\caption{Consultation CoT Evaluation Algorithm}
\label{Consultation CoT Scoring Algorithm}
\begin{algorithmic}[1]
\STATE \textbf{Initialization:} $scored\_consultation\_cot \leftarrow \emptyset$
\FOR{$question$ in ${RCoT_{tf}}$}
    \STATE $com\_score,\ per\_score \leftarrow 0,\ 0$ 

    \FOR{item in $ten\_questions\_song$}
        \STATE $score \leftarrow sim(emb(question),\ emb(item))$ 
        
        \IF{$score > com\_score$}
            \STATE \quad $com\_score \leftarrow score$
        \ENDIF
    \ENDFOR
    
    \FOR{$item$ in ${core\_questions, medical\_record}$}
        \STATE $score \leftarrow sim(emb(question),\  emb(item))$ 
        
        \IF{$score > per\_score$}
            \STATE \quad $per\_score \leftarrow score$ 
        \ENDIF
    \ENDFOR

    \STATE$total\_score \leftarrow com\_score + per\_score$ 

    \STATE \textbf{Store:} Add $question$ with its $total\_score$ to $scored\_consultation\_cot$ 
\ENDFOR
\STATE \textbf{Return:} $scored\_consultation\_cot$ 
\end{algorithmic}
\end{algorithm}
To evaluate the comprehensiveness of the questions, we calculate the similarity of questions in the consultation CoT with those in the TQS as the score of comprehensiveness by using an embedding model. To evaluate the pertinence of the questions, the key issues of each expert (different TCM experts focus on the characteristic diseases and diagnostic points related to their specialty, e.g., internal medicine focuses on organ function) and the patient's known condition are incorporated as the basis for the pertinence score. The final score is obtained by aggregating these two scores to evaluate the consultation quality. 

Based on the patient’s current condition $Con_t$ and $TCoT_t$, an Evaluation Agent is designed to summarize the consultation CoT provided by the expert team and evaluate the consultation questions within the summarized consultation CoT via the TCM knowledge-based CCEA. Thus, a holistic consultation CoT $RCoT_{t0}$ serves as the summary of the expert team's consultation CoT that includes the score for each question generated as:
\begin{equation}
    RCoT_{t0} = CoT_{sum}(TCoT_t,\ Con_t,\ prompt_r)
\end{equation}

\subsubsection{Consultation CoT Analysis and Optimization}
Agents in the expert team further analyze the summary $RCoT_{t0}$
based on the evaluation results, and provide feedback on whether they agree with it or not. If an expert disagrees, it will provide modification suggestions for low-score questions and optimize the consultation CoT to get a higher score calculated by CCEA. Through this optimization and feedback process, the TCM expert team ensures that the consultation CoT is both comprehensive and targeted. Specifically, in $j$-th round of feedback, expert suggestions are recorded as $Mod_j$ and the summary $RCoT_{tj}$ is updated as follows:
\begin{flalign}
    RCoT_{tj} = CoT_{opt}(RCoT_{t(j-1)}, Mod_{tj}, prompt_r)
\end{flalign}
The integration and optimization process continues until all experts agree on the summary or the maximum number of feedback rounds is reached, a final Consultation CoT $RCoT_{tf}$ is generated to start the official consultation in this round.

\subsubsection{Multi-Round Consultation}
 
Based on the consultation CoT $RCoT_{tf}$ and the current multi-round consultation result $C_t$ in the t-$th$ round, a Consultation Agent is developed according to real-world consultation scenarios to generate the next round of consultation questions:
\begin{equation}
    Q_{t+1}=Q_{gen}(RCoT_{tf}, \ Con_t,\ prompt_q),
\end{equation}
and interact with the patient. After this, the Medical Record Agent is called to systematically integrate the patient's condition into $Rec_{t}$, which is
\begin{equation}
    Rec_{t}=Agen_{in}(M, RCoT_{tf}, Con_t, prompt_{in}).
\end{equation}
Following this process, a series of professional and targeted questions is generated to further explore the patient's condition. When the key information collected in the record of the patient is completed or the maximum number of consultation rounds is reached according to the Consultation Information Collection Algorithm\ref{alg:Intelligent_Dynamic}, the Medical Record Agent presents the final case $Rec_{f}$ to the next Syndrome Differentiation and Treatment Recommendation Module. 


\begin{figure}[ht]
\vskip 0.2in
\begin{center}
\centerline{\includegraphics[width=\columnwidth]{Figure/An_example_of_multi_agent_simulation_of_TCM_consultation_CoT.png}}
\caption{An example of MDCCTM in the medical consultation}
\label{MDCCTM}
\end{center}
\vskip -0.2in
\end{figure}
Fig.\ref{MDCCTM} exhibits an example of how this MDCCTM works in a medical consultation scenario. For more detailed and specific application examples of this MDCCTM in the real medical consultation scenario see \ref{consultation_section} in the appendix.

\subsection{TCM Syndrome Agent}
In the field of TCM, the accuracy of syndrome differentiation is the cornerstone of precise treatment. High-quality datasets are essential for LLMs to develop accurate syndrome differentiation capabilities. We observed that real-world TCM medical records and diagnostic data often contain a large amount of information unrelated to the patient’s condition, i.e., noise, which cannot be easily removed through conventional regularization techniques. For example, phrases such as “The patient visited our hospital for further integrated traditional Chinese and Western medicine treatment” or “The patient was advised to undergo cranial CT and blood tests but refused” are not only irrelevant to the core condition but also interfere with the model’s ability to focus on key disease-related information. At the same time, content in the medical record, such as “Blood amylase test showed no significant abnormalities and abdominal CT indicated slight thickening of the ascending colon wall”, which requires medical instrument examinations, also needs to be screened out. 

According to the TQS, it helps TCM experts to comprehensively understand the patient's symptoms, medical history, and physical signs through a series of fundamental questions, providing key information for precise diagnosis. Inspired by this, we propose a universal data preprocessing method by utilizing LLM to automatically extract key information related to the TQS from raw data and strictly retain core information closely related to the patient’s condition. This ensures alignment between the case outputs in the JF and the format of the training data cases, aiming to significantly enhance the accuracy of syndrome differentiation through more systematic and precise data processing. 

To enable JF with TCM syndrome differentiation ability and smoothly complete accurate syndrome differentiation tasks within
its framework, we particularly develop a TCM Syndrome Agent based on the fine-tuned LLM. Moreover, according to the final patient's condition summary from the multi-round consultation $Rec_{f}$, the TCM Syndrome Agent is used to identify the most likely syndrome type for the patient:
\begin{equation}
    TCM_s=Syn_{dif}(Rec_{f},\ Con_t,\ prompt_s,).
\end{equation}
The detailed description of syndrome differentiation database preconditioning and model fine-tuning see Sec.\ref{syndrome data construction} and Sec.\ref{syndrome model training} respectively in the appendix.

%\textcolor{red}{In this study, we cleaned over 63,000 real diagnostic data entries, ultimately selecting over 43,000 high-quality data entries. Based on this data, we trained a professional syndrome differentiation model using the Qwen2.5-7B-Instruct model and the Roberta model. The training methods will be detailed in the appendix.}

% we designed and developed a Syndrome Differentiation Expert Agent. This agent integrates fine-tuning techniques of LLMs and is guided by the theoretical foundation of TCM's "Ten Questions Song".

\subsection{TCM Treatment Agent}
In TCM diagnosis and treatment, treatment recommendations are essentially formulated from the perspective of syndrome differentiation. In addition, TCM experts also integrate various aspects of a patient's information, including chief complaints, medical history, and constitution, dynamically adjusting treatment plans in response to changes in the patient's condition to ensure precise treatment. This complicated and changeable procedure is a characteristic of TCM treatment but poses a challenge to the precise treatment of LLMs. To facilitate LLMs with accurate and targeted treatment recommendations ability, a Dual-Stage Retrieval Scheme (DSRS) is proposed to offer the exact and targeted treatment recommendation. Here is the algorithm of DSRS, and the detailed introduction of it see Sec.\ref{treatment details} in the appendix.
\begin{algorithm}
\caption{Dual-Stage Retrieval Scheme}
\label{alg:hybrid_retrieval}
\begin{algorithmic}[1]
\STATE \textbf{Initialization:}$max\_results \leftarrow 3$
\STATE \textbf{Preprocess Patient Medical Record:}
\\ $cleaned\_record \leftarrow remove\_stopwords\_punctuation(medical\_record)$
\STATE \textbf{Generate Embedding Vectors:}
\\ $emb\_dense \leftarrow emb(medical\_record)[\texttt{"dense"}]$
\\ $emb\_sparse \leftarrow emb(cleaned\_record)[\texttt{"sparse"}]$
\STATE \textbf{Construct Retrieval Parameters:}
\\ $screen \leftarrow \texttt{"syndrome==patient's syndrome"}$
\\ $params \leftarrow \{\texttt{"limit":}max\_results, \texttt{"expr":}screen\}$
\STATE \textbf{Execute Retrieval:}
\\ $sparse\_results \leftarrow search(emb\_sparse,\ params)$
\\ $dense\_results \leftarrow search(emb\_dense,\ params)$
\STATE \textbf{Hybrid Ranking with RRF Algorithm:}
\\ $hybrid\_results \leftarrow rrf\_ranker(sparse\_results,\ dense\_results)$
\STATE \textbf{Output:} $hybrid\_results$
\end{algorithmic}
\end{algorithm}

According to DSRS, a TCM Treatment Agent that utilizes the patient's condition summaries and syndrome differentiation results derived from multiple rounds of consultation is developed to automatically screen and recommend the most suitable prescriptions from a structured TCM prescription database:   
\begin{flalign}
    TCM_t=Treat_{gen}(Rec_{f}, DB_{TCM}, prompt_p).
\end{flalign}
$DB_{TCM}$ represents the delicately constructed TCM prescription database containing various TCM fields, including internal medicine, gynecology, pediatrics, and surgery. Each entry in the database includes essential information such as disease categories, syndrome types, clinical manifestations, representative formulas, commonly used herbs, and other therapeutic methods. Fig.\ref{An example of a TCM treatment recommendation} illustrates an example of the performance of the TCM Treatment Agent, and for more information see Sec.\ref{personal treat} in the appendix.
\begin{figure}[!htbp]
\vskip 0.1in
\begin{center}
\centerline{\includegraphics[width=\columnwidth]{Figure/TCM_Recommendations.png}}
\caption{An example of the TCM Treatment Agent performance}
\label{An example of a TCM treatment recommendation}
\end{center}
%\vskip -0.1in
\end{figure}
%\textcolor{red}{The database includes 279 disease categories and 1,126 TCM prescription recommendations, providing a robust foundation for the TCM treatment agent module. Additionally, $prompt_p$ serves as a guiding prompt to ensure the generation of high-quality treatment recommendations.}

\section{Experiments}
In this section, we explore JF's advantages and limitations in practical applications via a series of systematic experiments.
We demonstrate the performance of JF in different tasks from two core aspects: the accuracy of TCM syndrome differentiation and the effectiveness of multi-round TCM consultation. A detailed explanation of the evaluation methods and results is also provided. 

\subsection{Baseline Model}
To comprehensively evaluate the performance of JF, we selected several representative open-source TCM models, i.e., Zhongjing-7B\cite{8}, Mingyi-7B\cite{27}, Shenlong-7B\cite{28}, and Sun Simiao-7B\cite{29}, as baselines. 
% These models represent different technical approaches and implementations in the TCM domain. 
Moreover, we incorporate two state-of-the-art LLMs i.e., GPT-4o\cite{hurst2024gpt} (one of the most advanced LLMs) and Qwen-Max\cite{30} (a leading Chinese-language LLM) for comparison. Various evaluation metrics are designed to ensure a comprehensive and fair assessment. Different tasks employed task-specific evaluation criteria, enabling a multidimensional analysis of the performance of JF.

\subsection{Assessment of Syndrome Differentiation Accuracy}
We have preprocessed over 63,000 real diagnostic data entries to select over 43,000 high-quality data entries for LLM fine-tuning.
Based on this data, the Qwen2.5-7B-Instruct model and the Roberta model are taken respectively as the foundation model of JF for training its syndrome differentiation ability. Here, syndrome differentiation can also be considered as a classification task, thus we train it via these two different types of models. For the evaluation of TCM syndrome differentiation accuracy, we select 8,699 real-world TCM cases as the test dataset, covering 170 different syndrome types. Considering the imbalance in sample sizes across different categories, we use weighted evaluation metrics to assess the model’s performance in various categories. The metrics include \textbf{weighted precision}: $P_w$, \textbf{weighted recall}: $R_w$, and \textbf{weighted F1}: $F1_w$, which are as follows:
\begin{equation}
     P_w = \frac{\sum_{i=1}^{n} w_i \cdot TP_i}{\sum_{i=1}^{n} w_i \cdot (TP_i + FP_i)},
\end{equation}
\begin{equation}
   R_w = \frac{\sum_{i=1}^{n} w_i \cdot TP_i}{\sum_{i=1}^{n} w_i \cdot (TP_i + FN_i)},
\end{equation}
\begin{equation}
    F1_w =\frac{\sum_{i=1}^{n} w_i \cdot TP_i}{\sum_{i=1}^{n} w_i \cdot(TP_i + 0.5 \cdot (FP_i + FN_i))}.
\end{equation}
Here, $n$ is the number of TCM syndrome categories. $w_i$ represents the weight of the $i$-th category, which is calculated as the number of samples in category $i$ divided by the total number of samples. $TP_i$ is the number of true positives for category $i$, $FP_i$ is the number of false positives for category $i$, and $FN_i$ is the number of false negatives for category $i$.

\begin{table}[htbp]
    \centering
    \renewcommand{\arraystretch}{1.2} % 调整行距,使表格更美观
    \setlength{\tabcolsep}{8pt} % 调整列间距
    \begin{adjustbox}{max width=\linewidth}  % 让表格适应列宽
    \begin{tabular}{lccc}  
        \toprule
        \textbf{Model} & \textbf{$P_w$} & \textbf{$R_w$} & \textbf{$F1_w$} \\
        \midrule
         \textbf{JingFang-RoBERTa}  & \textbf{0.8185} & \textbf{0.8230} & \textbf{0.8186} \\
        \textbf{JingFang-Qwen-2.5-7B}   & \textbf{0.8015} & \textbf{0.8145} & \textbf{0.8032} \\
        GPT-4o             & 0.4328 & 0.2138 & 0.2474 \\
        Qwen-max           & 0.4359 & 0.1811 & 0.2249 \\
        Sun Simiao         & 0.3113 & 0.1497 & 0.1415 \\
        Zhong Jing         & 0.0715 & 0.0598 & 0.0308 \\
        Ming Yi            & 0.1380 & 0.0467 & 0.0217 \\
        Shen Nong          & 0.0713 & 0.0945 & 0.0401 \\
        \bottomrule
    \end{tabular}
    \end{adjustbox}
    \caption{Comparison of TCM syndrome differentiation capabilities among different models.}
    \label{Comparison of TCM Syndrome Differentiation Capabilities Among Different Models}
\end{table}

\begin{table*}[htbp]  % 使用 table* 让表格独占两列
    \centering 
    \begin{tabular}{lccccc}  
        \toprule
        & \textbf{Proactivity} & \textbf{Accuracy} & \textbf{Practicality} & \textbf{Overall Effectiveness} & \textbf{Total Score} \\
        \midrule
        \textbf{Jing Fang}   & \textbf{8.45} & \textbf{8.19} & \textbf{8.44} & \textbf{8.42} & \textbf{8.38} \\
        Sun Simiao  & 6.56 & 5.50 & 5.41 & 5.77 & 5.74 \\
        Shen Nong   & 6.56 & 5.50 & 5.47 & 5.95 & 5.92 \\
        Ming Yi     & 4.44 & 5.19 & 4.88 & 5.14 & 4.85 \\
        Zhong Jing  & 4.31 & 4.50 & 4.31 & 3.80 & 4.20 \\
        \bottomrule
    \end{tabular}
    \caption{Comparison of multi-round consultations capabilities across different models.} 
    \label{Comparison of multi-round consultations Capabilities Across Different Models.}
\end{table*}

As shown in Table~\ref{Comparison of TCM Syndrome Differentiation Capabilities Among Different Models}, JF surpasses both the baseline models and general models across all evaluation metrics in the TCM syndrome differentiation task. Specifically, the JingFang-RoBERTa model achieved a precision of 0.8185, a recall of 0.8230, and a F1 score of 0.8186. Notably, the general models GPT-4o and Qwen-Max, due to their significantly larger number of parameters compared to the baseline models, outperformed other baseline models in certain metrics. However, their performance still does not reach the level of ours. These results demonstrate that our models possess the expert-level capability of syndrome differentiation.
%In contrast, the JingFang-Qwen-2.5-7B model attained a weighted precision of 0.8015, a weighted recall of 0.8145, and a weighted F1 score of 0.8032. 
Here, we need to emphasize that the choice of foundation models is not a factor affecting its capabilities in syndrome differentiation, while the training method and data do. For more information, see the Ablation Experiment and appendix.
%which is validated by the results in the Ablation Experiment and appendix.

\subsection{Medical Consultation Evaluation}
For the evaluation of medical consultation ability, we randomly selected 100 cases from real TCM medical cases. These cases involve a variety of clinical conditions, which comprehensively reflect the diversity and complexity of TCM consultation. Based on these cases, an LLM is used to act as the patient and interact with JF and each baseline model to complete multi-round consultation, collecting the patient's medical information.

To professionally assess the capacity of medical consultation in multi-round consultation, multiple TCM experts are invited to score the models' performance from 0-10 points respectively based on four dimensions: \textbf{Proactiveness}, \textbf{Accuracy}, \textbf{Practicality}, and \textbf{overall effectiveness}, with a total score of 40 points. (For the specific evaluation criteria for each dimension, refer to Sec.\ref{consultation_section}.) The evaluation in each dimension carefully considers the model's key abilities, such as proactiveness in gathering patient information and the model's responsiveness to the patient's actual needs in TCM consultations. 
% The professional explanation of the four dimensions is listed in Sec.\ref{consultation_section} in the appendix.
% 这里和括号里的有点重复了,可以只留一处

As shown in Table~\ref{Comparison of multi-round consultations Capabilities Across Different Models.}, our MDCCTM significantly advances LLM's medical consultation capacity, and JF outperforms all baseline models. The scores for Proactiveness, accuracy, practicality, and overall effectiveness all exceeded 8, notably surpassing other models. These results indicate the MDCCTM, which is a CoT-driven multi-agent collaborative mechanism, exhibits superior interaction and information processing capabilities during multi-round consultations. According to the performance,  MDCCTM provides a foundation for LLM of more accurate clinical diagnosis and highlights JF's potential for real-world applications.

\subsection{Ablation Experiment}
\label{ablation experiment}
To assess the effectiveness of key components within the JF framework, we conducted an ablation study focusing on its multi-round consultation capabilities and dataset applicability. By systematically comparing different framework configurations and evaluating performance on real-world TCM cases, this experiment provides insights into the contributions of the TCM General Agent and the robustness of the constructed TCM syndrome differentiation dataset.

\subsubsection{Evaluation of TCM General Agent in Medical Consultation}
To evaluate the impact of the combination of TCM Specialist Agent and TCM General Agent in MDCCTM on multi-round medical consultation and to analyze the key role and contributions of the TCM General Agent, this study compares two configurations: JF without TCM General Agent and JF with TCM General Agent, systematically analyzing their differences in performance in multi-round consultation. This experiment selects 100 representative TCM cases from real medical data, ensuring typicality and diversity. To simulate multi-round interactions, we employ an LLM-based patient agent, which dynamically generates consultation responses from a patient’s perspective, effectively mimicking real-world consultation scenarios. To evaluate performance, we conducted a comparative analysis, assessing consultation results against the original case information. The evaluation focuses on the comprehensiveness (coverage of multi-dimensional health information) and the pertinence (adaptation to case-specific features).  To ensure objectivity, GPT-4o is used as the evaluation tool to automatically compare consultation outputs from both configurations and score them based on these criteria.

As shown in Figure~\ref{Ablation Results}, the JF framework with TCM General Agent significantly outperforms the one without it in multi-round consultations. The evaluation favored the framework with the TCM General Agent 89 times for its comprehensiveness and pertinence, compared to only 11 times for the one without it, highlighting the critical role of the TCM General Agent in enhancing consultation quality. Additionally, the framework with the TCM General Agent achieves an average of 9.09 consultation rounds per patient, nearly doubling the 4.94 rounds of the one without it. This demonstrates that incorporating the TCM General Agent in the JF framework improves the completeness and coverage of multi-round consultations, and significantly reinforces the effectiveness and performance of the JF in medical consultations.
\begin{figure}[ht]
\vskip 0.2in
\begin{center}
\centerline{\includegraphics[width=\columnwidth]{Figure/ablation_experiment_corrected.png}}
\caption{ Performance comparison between JingFang with TCM General Agent and JingFang without TCM General Agent}
\label{Ablation Results}
\end{center}
\vskip -0.2in
\end{figure}

\subsubsection{Applicability Assessment of the TCM Syndrome Differentiation Dataset}
This ablation experiment is designed to demonstrate the effectiveness and applicability of our proposed method and constructed data for TCM syndrome differentiation and explore the influence of different foundation models (or the model with different sizes of parameters) on syndrome differentiation. We chose multiple representative open-source LLMs to test, which include the Qwen-2.5 series (Qwen-2.5-3B-Instruct, Qwen-2.5-7B-Instruct, Qwen-2.5-14B-Instruct)\cite{4}, the DeepSeek series (DeepSeek-7B-Chat)\cite{deepseek-llm}, and the Llama series (Llama3.1-8B-Instruct)\cite{2}. 

As shown in Figure~\ref{TCM Syndrome Differentiation Dataset}, all models in the test exhibit advanced performance in TCM syndrome differentiation compared to their original version before being trained via our method and data. According to the results, we observe that the choice of foundation model is not the key factor for improving the syndrome differentiation ability, while the method and data do. Moreover, compared to the performance of GPT-4o (0.4328), the results highlight the potential applicability and value of the proposed TCM syndrome differentiation method and dataset for future research. For a detailed description of the ablation experiment, refer to Sec.\ref{details of ablation experiment}.
\begin{figure}[ht]
\vskip 0.2in
\begin{center}
\centerline{\includegraphics[width=\columnwidth]{Figure/syndrome_differentiation_experiment_performance.png}}
\caption{Effect of the proposed TCM syndrome differentiation method and dataset on different models}
\label{TCM Syndrome Differentiation Dataset}
\end{center}
\vskip -0.2in
\end{figure}


% As shown in Table~\ref{Comparison of TCM Syndrome Differentiation Capabilities Among Different Models},  based on our proposed universal TCM syndrome differentiation data cleaning method, the two trained models surpassed both the baseline models and general models across all evaluation metrics in the TCM syndrome differentiation task. Specifically, the JingFang-RoBERTa model achieved a weighted precision of 0.8185, a weighted recall of 0.8230, and a weighted F1 score of 0.8186. In contrast, the JingFang-Qwen-2.5-7B model attained a weighted precision of 0.8015, a weighted recall of 0.8145, and a weighted F1 score of 0.8032. These results demonstrate that our models possess strong classification capabilities and balanced performance in the TCM syndrome differentiation task. Notably, the general models GPT-4o and Qwen-Max, due to their significantly larger number of parameters compared to the baseline models, outperformed other baseline models in certain metrics. However, they still did not achieve the performance level of our models.

% \subsection{Medical Consultation Evaluation}
% For the evaluation of medical consultation ability, we randomly selected 100 cases from real TCM medical cases. These cases involve a variety of clinical conditions, which comprehensively reflect the diversity and complexity of TCM consultation. Based on these cases, we take an LLM to act as the patient according to its information and interact with JF and each baseline model to collect the patient's medical information.

% To professionally assess the capacity of medical consultation in multi-round consultation, multiple TCM experts are invited to score the models' performance from 0-10 points respectively based on four dimensions: \textbf{Proactiveness}, \textbf{Accuracy}, \textbf{Practicality}, and \textbf{overall effectiveness}, with a total score of 40 points. (For the specific evaluation criteria for each dimension, refer to Sec \ref{consultation_section}.) The evaluation in each dimension carefully considers the model's key abilities, such as proactiveness in gathering patient information and the model's responsiveness to the patient's actual needs in TCM consultations. The professional explanation of the four dimensions is listed in Sec.\ref{consultation_section} in the appendix.

% As shown in Table~\ref{Comparison of multi-round consultations Capabilities Across Different Models.}, our multi-round consultation system demonstrated significant advantages, outperforming all baseline models. The scores for Proactiveness, accuracy, practicality, and overall effectiveness all exceeded 8. Notably, the model achieved 8.45 in proactivity and 8.44 in practicality, significantly surpassing other models. These results indicate that the CoT-driven multi-agent collaborative TCM consultation framework exhibits superior interaction and information processing capabilities during multi-round consultation, providing a foundation for more accurate clinical diagnoses. Thus, the experimental results validate the framework’s outstanding performance in TCM multi-round consultation and highlight its potential for real-world applications.



% The second experiment assesses the practical value and applicability of the constructed TCM syndrome differentiation dataset by comparing its performance across various open-source LLMs.第二个消融实验要不放在附录里吧
% To investigate the impact of the JingFang framework, which integrates specialist agents and a TCM General Agent agent, on the performance of clinical multi-round consultations, and to explore the importance and contribution of the TCM General Agent agent in the CoT process, we designed the following two ablation experiments. The goal is to compare the performance of a framework involving both general and specialist TCM expert agents versus a framework relying solely on specialist agents:


% To evaluate performance, we conducted a comparative analysis, assessing consultation results against the original case information. The evaluation criteria primarily included two aspects: comprehensiveness of the consultation (i.e., whether it covers multi-dimensional health information) and pertinence (i.e., whether the consultation adjusts based on the personalized features of the case). To ensure objectivity and consistency in judgment, we employed GPT-4o as the evaluation tool. GPT-4o assessed the consultation results by comparing the similarities and differences between the multi-round consultations generated by the two frameworks and the original case information, scoring the consultation outcomes based on the aforementioned criteria to select the superior consultation results.


% As shown in Figure~\ref{TCM Syndrome Differentiation Dataset}. Based on the experimental results, the Qwen series models demonstrated overall excellent performance, with the Weighted Precision showing an initial increase followed by stabilization as the model parameter size increased. Among them, Qwen-2.5-7B-Instruct (0.8015) and Qwen-2.5-14B-Instruct (0.7955) performed similarly, indicating that the syndrome differentiation dataset has achieved a high level of adaptability and effectiveness within this series. Additionally, the Weighted Precision of the DeepSeek-7B-Chat model was 0.7521, outperforming Llama3.1-8B-Instruct (0.7212). Considering that Llama3.1-8B-Instruct is primarily trained on English corpora, this result may reflect its inadequate adaptability to the TCM domain. Although performance differences were observed across models, the experimental results demonstrate that the syndrome differentiation dataset significantly enhances TCM syndrome differentiation across all models, particularly when compared to general LLMs such as GPT-4o (0.4328). This highlights the broad applicability and universal value of the dataset constructed in this study, providing a high-quality and reliable data foundation for TCM syndrome differentiation tasks, while also offering important guidance for model optimization and future research in related fields.

% \cite{MachineLearningI}.
\section{Conclusion and Discussion}
In this paper, we develop JingFang, a novel TCM LLM with expert-level capability, especially in medical diagnosis and syndrome differentiation-based treatment. The proposed model not only overcomes the key limitations of current TCM models but also enhances the applications of LLM in the TCM domain. To achieve professional medical consultation ability, the Multi-agent Dynamic Collaborative Chain-of-Thought Mechanism (MDCCTM) is innovated, integrating multiple agents with different TCM specialties to accord with the real-world medical consultation process. The MDCCTM facilitates JF with dynamic reasoning and explicit decision-making capabilities, enabling it to carry out comprehensive and targeted medical consultation, avoiding the omission of critical information, thereby laying a solid foundation for accurate diagnosis. In addition, a TCM Syndrome Agent is trained with preprocessed and structured multi-level TCM data, significantly enhancing JingFang's ability of TCM syndrome differentiation based on patients' complaints and condition. In the tests based on real medical records, the precision of syndrome differentiation of JF has improved greatly (by at least 50 $\%$), especially compared to the existing models. To achieve personalized and precise TCM treatment, the Dual-Stage Retrieval Scheme is established for the hybrid retrieval and extraction of TCM knowledge at both fine and coarse-grained levels. As a result, JingFang's treatment capability based on syndrome differentiation has made a remarkable breakthrough, enabling it to provide tailored and proficient treatment recommendations for patients and promoting the practical application of LLMs in the field of TCM. 

Moreover, continuing to deeply explore the potential application of the multi-agent collaborative mechanism and enhancing the innovative application of LLMs in different fields will be interesting. At the same time, in the field of TCM LLM, developing advanced multi-modal TCM large models to promote more efficient application of TCM, will also be very meaningful.


% Acknowledgements should only appear in the accepted version.
% \section*{Acknowledgements}

% \textbf{Do not} include acknowledgements in the initial version of
% the paper submitted for blind review.

% If a paper is accepted, the final camera-ready version can (and
% usually should) include acknowledgements.  Such acknowledgements
% should be placed at the end of the section, in an unnumbered section
% that does not count towards the paper page limit. Typically, this will 
% include thanks to reviewers who gave useful comments, to colleagues 
% who contributed to the ideas, and to funding agencies and corporate 
% sponsors that provided financial support.

\section*{Impact Statement}

This paper presents work that advances the application field of LLMs, particularly in the TCM domain. The proposed model can greatly improve the efficiency of doctors in medical diagnosis and treatment. However, in addition to existing ethical questions for this area (e.g. inaccurate content, bias, toxicity), our model cannot completely replace the role of doctors in the real clinical practice of TCM.  Due to individual differences in patients and the complexity of real medical scenarios, specific diagnoses and treatments require careful adherence to the advice of real doctors.


% In the unusual situation where you want a paper to appear in the
% references without citing it in the main text, use \nocite
% \nocite{langley00}

\bibliography{reference}
\bibliographystyle{icml2025}

%%%%%%%%%%%%%%%%%%%%%%%%%%%%%%%%%%%%%%%%%%%%%%%%%%%%%%%%%%%%%%%%%%%%%%%%%%%%%%%
%%%%%%%%%%%%%%%%%%%%%%%%%%%%%%%%%%%%%%%%%%%%%%%%%%%%%%%%%%%%%%%%%%%%%%%%%%%%%%%
% APPENDIX
%%%%%%%%%%%%%%%%%%%%%%%%%%%%%%%%%%%%%%%%%%%%%%%%%%%%%%%%%%%%%%%%%%%%%%%%%%%%%%%
%%%%%%%%%%%%%%%%%%%%%%%%%%%%%%%%%%%%%%%%%%%%%%%%%%%%%%%%%%%%%%%%%%%%%%%%%%%%%%%
\newpage
\appendix
% \twocolumn
\onecolumn
\section{TCM Consultation}
\label{consultation_section}
To enable efficient consultation with the patient, we propose an "Intelligent Dynamic Consultation Information Collection and Termination" Algorithm. This algorithm automatically collects key points from the consultation CoT through a limited number of consultation rounds and automatically terminates the dialogue based on this consultation CoT when specific conditions are met. Based on TCM consultation knowledge, the key points are categorized into four types: symptom information, medical history information, lifestyle information, and other information, which helps to systematically extract crucial details. The algorithm's inputs include the maximum consultation rounds, patient inputs, and the consultation CoT, and through these inputs, it generates a set of key points and drives the conversation forward. Finally, the algorithm outputs the collected set of medical information points.

The core logic of the algorithm ensures the consultation continues through an infinite loop until termination conditions are met. To ensure the effective collection of diagnostic information, the algorithm maintains the set of medical information points and updates this set according to the progress of the consultation, determining whether to terminate the dialogue. In addition, the algorithm defines two additional algorithms: one to obtain the set of key points to collect and the other to update the set of information points based on the patient’s input. These two auxiliary algorithms are executed through agents defined by prompts. The
detailed methodology is presented in Algorithm \ref{alg:Intelligent_Dynamic}.
\begin{algorithm}
\caption{Intelligent Dynamic Consultation Information Collection and Termination}
\label{alg:Intelligent_Dynamic}
\begin{algorithmic}[1]
\STATE \textbf{Initialization:} $max\_rounds \leftarrow N$, \\$current\_round \leftarrow 0$, $medical\_points \leftarrow \emptyset$
\STATE $cot\_points \leftarrow generate\_initial\_points(consultation\_cot)$
\WHILE{True}
    \IF{$current\_round \geq max\_rounds$}
        \STATE \textbf{Terminate the loop.}
    \ENDIF
    \STATE $patient\_input \leftarrow get\_user\_input()$
    \IF{user opts to end the dialogue}
        \STATE \textbf{Terminate the loop.}
    \ENDIF
    \STATE $medical\_points \leftarrow update(medical\_points, patient\_input)$
    \IF{$sufficient(medical\_points, cot\_points)$}
        \STATE \textbf{Terminate the loop.}
    \ENDIF
    \STATE $current\_round \leftarrow current\_round + 1$
\ENDWHILE
\end{algorithmic}
\end{algorithm}

Here, we detailed list the professional explanation of the four dimensions used in the evaluation of LLM's consultation ability.
\begin{enumerate}
    \item \textbf{Proactiveness:} Actively guiding and flexibly adjusting the direction of inquiry based on the initial symptom information provided by the patient, avoiding repetitive or ineffective questions, and ensuring the acquisition of accurate information.

    \item \textbf{Accuracy:} The ability to pose questions that are closely related to the symptoms and precise, and to help patients express themselves accurately using clear language, thereby reducing misunderstandings and biases.
    
    \item \textbf{Practicality:} Capable of comprehensively collecting the patient's basic information (such as symptoms, lifestyle habits, etc.) and flexibly responding to different contexts, providing practical and feasible suggestions for clinical diagnosis and treatment.
    
    \item \textbf{Overall Effectiveness:} Effectively integrating patient information, avoiding irrelevant distractions, providing valuable support for the diagnostic and treatment process, and enhancing diagnostic efficiency and effectiveness.
\end{enumerate}

\section{TCM Syndrome Differentiation }
\subsection{Dataset Construction for Syndrome Differentiation }
\label{syndrome data construction}
\begin{figure}[ht]
\vskip 0.2in
\begin{center}
\centerline{\includegraphics[width=0.8\columnwidth]{Figure/An_example_universal_TCM_syndrome_differentiation_data_cleaning.png}}
\caption{An example of the proposed universal TCM syndrome differentiation data cleaning method}
\label{An example of the proposed universal TCM syndrome differentiation data cleaning method}
\end{center}
\vskip -0.2in
\end{figure}
During the construction of the syndrome differentiation dataset, we observed that real-world TCM case and diagnostic data often contain a substantial amount of information irrelevant to the patient's condition, which cannot be easily removed through regularization techniques. For instance, phrases such as "The patient visited our hospital for further integrated traditional Chinese and Western medicine treatment" or "The patient was advised to undergo cranial CT and blood tests but refused" are not only extraneous to the core condition but also hinder the model's ability to focus on key information about the illness. Moreover, the raw TCM case and diagnostic data frequently exhibit a high degree of redundancy in describing the patient's condition and physical signs, resulting in a low density of valid information, which adversely impacts fine-tuning performance and the generalization ability of LLMs.

To address these issues, we proposed a universal syndrome differentiation data cleaning method based on the TCM framework of the "Ten Questions Song." This strategy aims to simulate the professional judgment of TCM experts using LLMs, accurately screening and reconstructing raw TCM case and diagnostic data to construct a high-quality syndrome differentiation dataset. Specifically, this approach strictly retains core information closely related to the patient's condition, optimizes the medical value of the data, and organizes the content systematically according to the "Ten Questions Song" framework to ensure the dataset includes key elements required for TCM syndrome differentiation. Through this strategy, we effectively eliminate redundant information, increase the structuralization of the data, and better adapt it to the application requirements of LLMs in TCM syndrome differentiation tasks.

As shown in Figure \ref{An example of the proposed universal TCM syndrome differentiation data cleaning method}, the data-cleaning process enforces strict adherence to the "Ten Questions Song" framework by applying explicit data extraction rules, retaining only information directly related to the patient's condition. A comparison of the raw TCM case and diagnostic data with the cleaned dataset (Preconditioned TCM Data) reveals that the cleaned records are more focused on core symptoms, physical signs, and the evolution of the condition. Unrelated information is systematically removed, resulting in texts that are both precise and medically valuable. The core optimization points of this method include the following:
\begin{enumerate}
\item \textbf{Removal of non-essential medical information:} Exclude content unrelated to the diagnostic process, such as procedural details, examination recommendations, and results from modern medical instruments. Retain only symptoms and signs directly relevant to syndrome differentiation analysis.
\item \textbf{Reduction of redundant descriptions and enhancement of information density:} By simplifying repetitive records, the medical texts become more concise, improving data validity and utilization efficiency.
\item \textbf{Strengthening the comprehensiveness and pertinence of syndrome differentiation dataset:} Comprehensiveness is reflected in the inclusion of all key consultation dimensions from the TCM "Ten Questions Song," such as cold/heat, sweating, head/body, bowel movements, diet, sleep, medical history, and family history, ensuring the completeness of diagnostic criteria. Pertinence is achieved by prioritizing the retention of information related to core elements of syndrome differentiation (e.g., chief complaints, symptom changes, disease progression, and physical signs) while removing background or secondary information. This makes the dataset more aligned with the logic of TCM syndrome differentiation, thereby improving the accuracy of LLMs in clinical reasoning.
\end{enumerate}

Overall, this data-cleaning strategy significantly enhances the quality of the syndrome differentiation dataset, providing reliable and efficient data support for TCM syndrome differentiation and large model training. It lays a solid foundation for the development of intelligent TCM diagnostic systems and serves as a reference paradigm for data optimization and model development in related research fields.

\subsection{Model Training for Syndrome Differentiation }
\label{syndrome model training}
\begin{figure}[ht]
\vskip 0.2in
\begin{center}
\centerline{\includegraphics[width=0.7\columnwidth]{Figure/An_example_of_TCM_syndrome_differentiation.png}}
\caption{An example of TCM syndrome differentiation}
\label{An example of TCM syndrome differentiation}
\end{center}
\vskip -0.2in
\end{figure}

Based on the syndrome differentiation data-cleaning method proposed in this study, a total of over 43,000 high-quality syndrome differentiation data entries were obtained. These entries were split into training and testing datasets at a ratio of 8:2. Subsequently, the Qwen2.5-7B-Instruct model was fine-tuned using the LoRA method. The form of the training dataset is shown in Figure \ref{An example of TCM syndrome differentiation}. In the experiments, we employed four A6000 GPUs (each with 48GB of memory). After multiple rounds of experimental tuning, we determined the optimal training parameters to ensure the stable convergence of the loss function. The specific training parameters included: "lora rank": 64, "lora alpha": 16, "learning rate": "5e-4", and "batch size": 16. 

Given that the syndrome differentiation task in this study can be treated as a classification problem, and Encoder-only architectures (such as BERT and RoBERTa) generally outperform Decoder-only architectures in classification tasks, we further trained lightweight syndrome differentiation models based on RoBERTa. In these experiments, a new classification layer was added to RoBERTa model, with its output dimension matching the number of task categories. The classification layer was trained from scratch, and fp16 mixed precision training was used to reduce memory usage and improve training efficiency. The training loss is shown in Figure \ref{Training Loss}.
\begin{figure}[ht]
\vskip 0.2in
\begin{center}
\centerline{\includegraphics[width=0.7\columnwidth]{Figure/dual_axis_loss_curve.png}}
\caption{Training loss of different models}
\label{Training Loss}
\end{center}
\vskip -0.2in
\end{figure}

\section{TCM Treatment Recommendation}
\label{treatment details}
Syndrome differentiation-based treatment is one of the core concepts in TCM. The fundamental idea is to identify the etiology and pathogenesis of a patient’s condition and then adopt a personalized treatment strategy. Different diseases may present similar syndrome differentiation; for instance, the syndrome of wind-cold attacking the lungs may manifest not only in the common cold but also in acute bronchitis. Therefore, experienced TCM practitioners often rely on the results of syndrome differentiation, integrating the patient's medical history and constitution, to provide individualized treatment recommendations. This study employs a Dual-Stage approach to recall relevant TCM treatment knowledge based on the TCM formula dataset we have constructed.

In the first stage, the system screens candidate prescription data related to the patient's syndrome type from the TCM prescription database based on the differentiation outcomes. This screening process utilizes the patient's syndrome differentiation information, including syndrome type, etiology, and disease location, to narrow down the candidate prescriptions to those highly relevant to the patient's condition, thereby ensuring the pertinence and effectiveness of subsequent treatment recommendations. In the second stage, the system employs an embedding method to vectorize the patient's detailed medical history information and the clinical manifestations of the candidate prescriptions, subsequently calculating the similarity between them. This method allows for the assessment of how well the candidate prescriptions match the patient's condition. Finally, based on similarity rankings, the top three cases most similar to the patient's medical history information are selected (TOP-3). These "top three most similar cases" refer to cases that closely resemble the current patient's condition across multiple dimensions, including syndrome differentiation information, medical history, and symptom presentation. The detailed methodology is presented in Algorithm \ref{alg:hybrid_retrieval}.
\begin{algorithm}
\caption{MDCCTM for TCM Treatment Recommendation}
\label{alg:hybrid_retrieval}
\begin{algorithmic}[1]
\STATE \textbf{Initialization:}$max\_results \leftarrow 3$
\STATE \textbf{Preprocess Patient Medical Record:}
\\ $cleaned\_record \leftarrow remove\_stopwords\_punctuation(medical\_record)$
\STATE \textbf{Generate Embedding Vectors:}
\\ $emb\_dense \leftarrow emb(medical\_record)[\texttt{"dense"}]$
\\ $emb\_sparse \leftarrow emb(cleaned\_record)[\texttt{"sparse"}]$
\STATE \textbf{Construct Retrieval Parameters:}
\\ $screen \leftarrow \texttt{"syndrome==patient's syndrome"}$
\\ $params \leftarrow \{\texttt{"limit":}max\_results, \texttt{"expr":}screen\}$
\STATE \textbf{Execute Retrieval:}
\\ $sparse\_results \leftarrow search(emb\_sparse,\ params)$
\\ $dense\_results \leftarrow search(emb\_dense,\ params)$
\STATE \textbf{Hybrid Ranking with RRF Algorithm:}
\\ $hybrid\_results \leftarrow rrf\_ranker(sparse\_results,\ dense\_results)$
\STATE \textbf{Output:} $hybrid\_results$
\end{algorithmic}
\end{algorithm}

\section{More Information of Ablation Experiment}
\label{details of ablation experiment}
\subsection{Details of the Evaluation of TCM General Agent in Medical Consultation}
To evaluate the impact of the combination of TCM Specialist Agent and TCM General Agent in MDCCTM on multi-round medical consultation and to analyze the key role and contributions of the TCM General Agent, this study compares two configurations: JF without TCM General Agent and JF with TCM General Agent, systematically analyzing their differences in performance in multi-round consultation. 

A total of 100 representative TCM cases were selected from real medical records to ensure data typicality and diversity, enhancing the clinical applicability of the experiment. The selection criteria included:
\begin{enumerate}
    \item \textbf{Information completeness:} Each case must contain comprehensive medical history, including chief complaints, present illness, past medical history, and family history.
    \item \textbf{Diversity:} The dataset includes various TCM syndrome types, ensuring that findings are applicable across different pathological conditions.
\end{enumerate}

To maintain experimental control and standardization, we developed an LLM-based patient agent to simulate patient responses in multi-turn consultations. The patient agent was designed with the following constraints:
\begin{enumerate}
    \item \textbf{Strict adherence to case details} to ensure responses align with actual patient conditions.
    \item \textbf{Concise and precise responses}, avoiding unnecessary verbosity that could interfere with analysis.
    \item \textbf{Responses limited to doctor consultations}, following real patient response patterns.
\end{enumerate}
These constraints ensure stability and consistency in patient interactions, minimizing external variability. Figure \ref{Patient Agent Setup} provides further details on the patient agent setup.
\begin{figure}[ht]
\vskip 0.2in
\begin{center}
\centerline{\includegraphics[width=0.5\columnwidth]{Figure/patient_agent.png}}
\caption{Patient Agent setup}
\label{Patient Agent Setup}
\end{center}
\vskip -0.2in
\end{figure}

To comprehensively assess the performance of different frameworks in multi-turn consultations, we used two core metrics:
\begin{enumerate}
    \item \textbf{Comprehensiveness (0-10 points):} Evaluates whether the consultation covers key symptoms, medical history, and relevant factors, reflecting the physician’s understanding of the patient’s condition. 
    \item \textbf{Pertinence (0-10 points):} Assesses whether the consultation accurately focuses on the patient's primary concerns and effectively guides subsequent diagnosis and decision-making.
\end{enumerate}
To ensure objective scoring, we employed GPT-4o as an automated evaluation tool, conducting a quantitative analysis of consultation records generated by different frameworks and scoring them based on these metrics. Figure \ref{Evaluation Agent Setup} presents the detailed scoring setup.
\begin{figure}[ht]
\vskip 0.2in
\begin{center}
\centerline{\includegraphics[width=0.7\columnwidth]{Figure/evaluation_agent.png}}
\caption{Evaluation Agent setup}
\label{Evaluation Agent Setup}
\end{center}
\vskip -0.2in
\end{figure}

To validate the significance of the experimental results, we performed an independent sample t-test and one-way ANOVA for statistical analysis. The results (Table \ref{Statistical_Analysis_Completeness_Pertinence}) indicate that the JF framework with a TCM General Agent significantly outperforms the framework without a general agent in both comprehensiveness (8.91-9.0 vs. 6.164-6.406) and pertinence (7.981-8.099 vs. 5.222-5.478). The t-test and ANOVA results confirm statistical significance ($p < 0.05$), demonstrating that the inclusion of a TCM General Agent effectively enhances the coverage and accuracy of multi-turn medical consultations, optimizing diagnostic strategies for physicians.
\begin{table*}[htbp]
    \centering 
    \begin{tabular}{lc}  
        \toprule
        \textbf{Label} & \textbf{Value} \\
        \midrule
        \textbf{p-value (t-test, Completeness)}  & \( 1.34 \times 10^{-113}\) \\
        \textbf{p-value (t-test, Pertinence)}  & \( 1.34 \times 10^{-111}\)  \\
        \textbf{p-value (ANOVA, Completeness)}  & \( 1.01 \times 10^{-144}\)  \\
        \textbf{p-value (ANOVA, Pertinence)}  & \( 3.12 \times 10^{-133} \) \\
        \textbf{Confidence Interval (JF, Completeness)}  & (8.91, 9.0) \\
        \textbf{Confidence Interval (JF w/o, Completeness)}  & (6.164, 6.406) \\
        \textbf{Confidence Interval (JF, Pertinence)}  & (7.981, 8.099) \\
        \textbf{Confidence Interval (JF w/o, Pertinence)}  & (5.222, 5.478) \\
        \bottomrule
    \end{tabular}
    \caption{Statistical analysis of Completeness and Pertinence. JF represents \textbf{JF with TCM General Agent}, while JF w/o represents \textbf{JF without TCM General Agent}.} 
    \label{Statistical_Analysis_Completeness_Pertinence}
\end{table*}


\subsection{Applicability Evaluation of the TCM Syndrome Differentiation Dataset}
To evaluate the applicability of the TCM syndrome differentiation dataset constructed in this study, we tested multiple open-source LLMs, including the Qwen-2.5 series (Qwen-2.5-3B-Instruct, Qwen-2.5-7B-Instruct, Qwen-2.5-14B-Instruct), the DeepSeek series (DeepSeek-7B-Chat), and the Llama series (Llama3.1-8B-Instruct).  

Using LoRA, we fine-tuned these models with a training dataset comprising over 43,000 high-quality entries from the TCM syndrome differentiation dataset. The training was conducted on four A6000 GPUs, each with 48GB of VRAM. Key training parameters were set as follows: `"lora rank": 64`, `"lora alpha": 16`, `"learning rate": 5e-4`, and `"batch size": 16`. 

Figure~\ref{Traing Loss For different models} illustrates the training loss curves for different models. The results show that all models exhibit a rapid decline in training loss during the initial phase, followed by a gradual stabilization as training progresses, indicating effective learning convergence. Among them, the Qwen-2.5-7B-Instruct and Qwen-2.5-14B-Instruct models achieve the lowest final loss values, suggesting superior adaptability to the dataset. The DeepSeek-7B-Chat and Llama3.1-8B-Instruct models show slightly higher final loss values, which may reflect their differing pretraining distributions and potential limitations in adapting to TCM-specific knowledge. Overall, these results confirm the dataset's ability to facilitate efficient fine-tuning across various LLMs, with Qwen series models demonstrating the best performance in optimizing TCM syndrome differentiation.

\begin{figure}[ht]
\vskip 0.2in
\begin{center}
\centerline{\includegraphics[width=0.7\columnwidth]{Figure/ablation_experiment_loss_curve.png}}
\caption{Traing loss for different models}
\label{Traing Loss For different models}
\end{center}
\vskip -0.2in
\end{figure}

\section{Examples of Complete Medical Consultation}
To clearly demonstrate the advantages of the JF framework proposed in this paper, four typical cases were selected, covering TCM surgery, TCM pediatrics, TCM gynecology, and TCM internal medicine. An in-depth analysis of the consultation processes, syndrome differentiation, and treatment recommendation was conducted. 

\subsection{Comprehensiveness of Multi-Round Consultations}
The JF framework adopts a multi-round consultation approach to ensure that doctors can fully understand the patient’s chief complaints, medical history, symptom characteristics, and lifestyle habits. Through a structured consultation process, the framework allows for in-depth exploration of disease characteristics, avoiding the omission of critical information. For example:
\begin{enumerate}
\item In the TCM surgery case (Damp-Heat Sinking Downward Syndrome, as show in Figure \ref{An example of Damp-heat sinking downward Syndrome}), the consultation covered the degree of redness and swelling of ulcers, characteristics of exudate, pain nature, and disease progression to ensure accurate identification of the pathological mechanism.

\item In the TCM pediatrics case (Spleen and Kidney Yang Deficiency Syndrome, as show in Figure \ref{An example of Spleen and kidney deficiency syndrome}), the framework assessed multiple dimensions, such as diet, sleep, bowel and urinary habits, and immune status, for a comprehensive evaluation of the infant’s growth and development.

\item In the TCM gynecology case (Cold Congealing with Blood Stasis Syndrome, as show in Figure \ref{An example of Cold Congealing and Blood Stasis Syndrome}), the consultation not only focused on the characteristics of dysmenorrhea (e.g., pain location and alleviating factors) but also examined factors such as menstrual cycle, dietary habits, and bowel movements to clarify the etiology.

\item In the TCM internal medicine case (Yang Deficiency with Water Overflow Syndrome, as show in Figure \ref{An example of Yang Deficiency with Water Overflow Syndrome}), the consultation included the evolution of symptoms like palpitations, shortness of breath, and edema, combined with physiological indicators such as tongue coating and pulse condition, ensuring the accuracy of syndrome differentiation.
\end{enumerate}

\subsection{Pertinence of Multi-Round Consultation}
The JF framework achieves pertinence information collection and personalized diagnosis and treatment planning through a progressive multi-round consultation approach. The pertinence of multi-round consultation in this framework is reflected as follows:
\begin{enumerate}
\item Structured Consultation Process: In the initial consultation phase, the doctor first collects the patient's basic information and chief complaints. In the in-depth consultation phase, further inquiries about key symptom details are made based on patient feedback to ensure sufficient diagnostic evidence is obtained. For example, in the case of Cold Stagnation with Blood Stasis Syndrome, after hearing the patient’s description of pain, the doctor further inquires about the menstrual cycle, flow volume, and accompanying symptoms to ensure accurate syndrome differentiation.

\item Dynamically Adjusting the Consultation Focus: The doctor flexibly adjusts the focus of inquiry based on the patient's responses. For instance, in the case of Damp-Heat Pouring Downward Syndrome, if the patient mentions the presence of induration around an ulcer, the doctor will further inquire about the ulcer’s formation process and any history of trauma to rule out other potential causes.

\item Comprehensive Syndrome Differentiation Using the Four Diagnostic Methods: The JF framework fully integrates inspection, listening and smelling, inquiry, and palpation in multi-round consultations to ensure a rigorous syndrome differentiation process. For example, in the case of Yang Deficiency with Water Overflow Syndrome, the doctor not only inquires about the location and severity of edema but also incorporates objective indicators such as tongue features (dim complexion, yellow-greasy coating) and pulse characteristics (deep and thin pulse) to further confirm the pathological mechanism.

\end{enumerate}
\subsection{Dynamic Optimization of consultation CoT}
The JF framework incorporates dynamic optimization of CoT during consultations, allowing diagnoses to be continuously adjusted based on patient feedback. For example:
\begin{enumerate}
\item During the initial consultation phase, the framework collects basic patient information and asks about primary symptoms.

\item During the in-depth consultation phase, subsequent questions are adjusted based on patient responses to ensure sufficient diagnostic evidence is obtained. For instance, in the Cold Stagnation with Blood Stasis Syndrome case, after hearing the patient’s description of pain, the doctor further inquired about the menstrual cycle, flow volume, and accompanying symptoms to ensure accurate syndrome differentiation.

\item In the final diagnosis phase, the framework integrates multi-round consultation information, combines it with TCM theory for syndrome differentiation-based treatment and provides specific treatment plans.
\end{enumerate}

\subsection{Personalization of Treatment Recommendation}
\label{personal treat}
The JF framework not only focuses on disease diagnosis but also provides systematic and personalized treatment recommendations, encompassing herbal therapy, lifestyle adjustments, emotional regulation, and rehabilitation recommendations. For example:
\begin{enumerate}
    \item In the TCM gynecology case, Ai Fu Nuan Gong Wan and Dang Gui Si Ni Tang were recommended to warm the meridians, dispel cold, activate blood, and relieve pain, with emphasis on keeping warm during menstruation and dietary adjustments.
    \item In the TCM surgery case, Long Dan Xie Gan Tang was used to clear heat and detoxify, complemented by the external application of Huang Lian wash to promote ulcer healing.
    \item In the TCM pediatrics case, Jian Pi Yi Shen Fang was employed to enhance the spleen and kidney functions of the infant, with additional recommendations for appropriate massage therapy to aid digestion.
    \item In the TCM internal medicine case, Zhen Wu Tang was suggested to warm yang and promote water metabolism, alongside advice on keeping warm and engaging in moderate exercise to alleviate edema.
\end{enumerate}

\subsection{Conclusion}
The application of the JF framework in multi-round TCM consultations fully demonstrates its advantages in comprehensiveness, pertinence, dynamic optimization, and personalized treatment plans. This framework not only enhances the accuracy of TCM diagnosis but also improves the patient experience through the dynamic optimization of the consultation CoT, showcasing significant clinical application value. In the future, this framework can be further integrated with intelligent technologies to enhance the intelligence level of TCM diagnosis and treatment, providing strong support for the development of TCM LLMs.

\begin{figure}[htbp]
\vskip 0.2in
\begin{center}
\centerline{\includegraphics[height=12.8cm, width=0.86\columnwidth]{Figure/An_example_of_Damp-heat_sinking_downward_Syndrome.png}}
\caption{An example of Damp-Heat Sinking Downward Syndrome}
\label{An example of Damp-heat sinking downward Syndrome}
\end{center}
\vskip -0.3in
\end{figure}


\begin{figure}[ht]
\vskip 0.2in
\begin{center}
\centerline{\includegraphics[width=0.9\columnwidth]{Figure/An_example_of_Spleen_and_kidney_deficiency_syndrome.png}}
\caption{An example of Spleen and Kidney Deficiency Syndrome}
\label{An example of Spleen and kidney deficiency syndrome}
\end{center}
\vskip -0.2in
\end{figure}


\begin{figure}[ht]
\vskip 0.2in
\begin{center}
\centerline{\includegraphics[width=0.9\columnwidth]{Figure/An_example_of_Cold_Congealing_and_Blood_Stasis_Syndrome.png}}
\caption{An example of Cold Congealing and Blood Stasis Syndrome}
\label{An example of Cold Congealing and Blood Stasis Syndrome}
\end{center}
\vskip -0.2in
\end{figure}


\begin{figure}[ht]
\vskip 0.2in
\begin{center}
\centerline{\includegraphics[width=0.9\columnwidth]{Figure/An_example_of_Yang_Deficiency_with_Water_Overflow_Syndrome.png}}
\caption{An example of Yang Deficiency with Water Overflow Syndrome}
\label{An example of Yang Deficiency with Water Overflow Syndrome}
\end{center}
\vskip -0.2in
\end{figure}





% The $\mathtt{\backslash onecolumn}$ command above can be kept in place if you prefer a one-column appendix, or can be removed if you prefer a two-column appendix.  Apart from this possible change, the style (font size, spacing, margins, page numbering, etc.) should be kept the same as the main body.
%%%%%%%%%%%%%%%%%%%%%%%%%%%%%%%%%%%%%%%%%%%%%%%%%%%%%%%%%%%%%%%%%%%%%%%%%%%%%%%
%%%%%%%%%%%%%%%%%%%%%%%%%%%%%%%%%%%%%%%%%%%%%%%%%%%%%%%%%%%%%%%%%%%%%%%%%%%%%%%
\end{document}


% This document was modified from the file originally made available by
% Pat Langley and Andrea Danyluk for ICML-2K. This version was created
% by Iain Murray in 2018, and modified by Alexandre Bouchard in
% 2019 and 2021 and by Csaba Szepesvari, Gang Niu and Sivan Sabato in 2022.
% Modified again in 2023 and 2024 by Sivan Sabato and Jonathan Scarlett.
% Previous contributors include Dan Roy, Lise Getoor and Tobias
% Scheffer, which was slightly modified from the 2010 version by
% Thorsten Joachims & Johannes Fuernkranz, slightly modified from the
% 2009 version by Kiri Wagstaff and Sam Roweis's 2008 version, which is
% slightly modified from Prasad Tadepalli's 2007 version which is a
% lightly changed version of the previous year's version by Andrew
% Moore, which was in turn edited from those of Kristian Kersting and
% Codrina Lauth. Alex Smola contributed to the algorithmic style files.
