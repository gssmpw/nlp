% what the innovations are and how they were achieved

% \fix{2 pages max}

Training deep neural networks on a single GPU involves processing subsets of
the data called batches through the layers of a DNN in the forward pass to
compute a loss, computing the gradient of the loss in a backward pass via
backpropagation, and updating the parameters (also called ``weights'') in the
optimizer step. These three steps are repeated iteratively until all batches
have been consumed, and this entire training process is referred to as an
epoch. We now describe our novel approach to scaling the computation in the
steps described above in the context of large multi-billion parameter neural
networks on thousands of GPUs.

\subsection{A Four-Dimensional Hybrid Parallel Approach}
\label{sec:hybrid}

We have designed a hybrid parallel approach that combines data parallelism with
three-dimensional (3D) parallelization of the matrix multiplication routines.

\vspace{0.08in}
\noindent{\bf Data Parallelism:}
% When using only data parallelism, we first instantiate a full copy of the
% neural network on every GPU, and then divide the input batch into equal-sized
% {\em shards} among these GPUs.
In order to use a hybrid approach that combines data with tensor
parallelism, we organize the total number of GPUs, $G$, into a virtual 2D
grid, $G_{\mathrm{data}} \times G_{\mathrm{tensor}}$. This results in
$G_{\mathrm{data}}$ groups of $G_{\mathrm{tensor}}$ GPUs each. We use data
parallelism across the $G_{\mathrm{data}}$ groups, and tensor parallelism
within each group. Each $G_{\mathrm{data}}$ group collectively has a full copy of the neural
network and is tasked to process a unique shard of the input batch. At the end of an input batch,
all groups have to synchronize their weights by issuing all-reduces
on their gradients after every batch (this is also referred to as an
iteration).

\vspace{0.08in}
\noindent{\bf 3D Parallel Matrix Multiplication (3D PMM):}
Next, we use each GPU group, composed of $G_{\mathrm{tensor}}$ GPUs to
parallelize the work within their copy of the neural network.  This requires
distributing the matrices, and parallelizing the computation within every layer
of the neural network across several GPUs. Note that most of the computation in 
transformers is comprised of large matrix multiplications within fully connected (FC) layers.
Hence, in this section, we will focus on parallelizing FC layers with a 3D 
PMM algorithm.

We now describe how a single layer is parallelized, and the same method is
applied to all the layers in the neural network.  Each FC layer computes one
half-precision (fp16 or bf16) matrix multiplication (input activation, $I$
multiplied by the layer's weight matrix, $W$) in the forward pass and two
half-precision matrix multiplications (MMs) in the backward pass
($\frac{\partial L}{\partial O} \times W^{\top}$ and $I^{\top} \times
{\frac{\partial L}{\partial O}}$, where $L$ is the training loss, and $O$ is
the output activation.) Thus, parallelizing an FC layer requires parallelizing
these three MM operations across multiple GPUs.

% \begin{figure}[h]
%     \centering
%       \includegraphics[width=3in]{figs/fc.pdf}
%       \caption{Computation in the forward pass of a fully-connected (FC) layer with input $I$ and layer weights $W$. 
%       The output, $O$ is a matrix multiplication of $I$ and $W$. We assume $I \in \mathbb{R}^{m \times k}$, 
%       $W \in \mathbb{R}^{k \times n}$, and  $O \in \mathbb{R}^{m \times n}$. \label{fig:schematic-fc}}
% \end{figure}  

We adapt Agarwal et al.'s 3D parallel matrix multiplication
algorithm~\cite{agarwal-3d}, for parallelizing our MMs. The 3D refers to
organizing the workers (GPUs) in a three-dimensional virtual grid.  So, we
organize the $G_{\mathrm{tensor}}$ GPUs further into a virtual 3D grid of
dimensions $G_x \times G_y \times G_z$
(Figure~\ref{fig:schematic-agarwal-data-dist}).  We do 2D decompositions of
both $I$ and $W$ into sub-blocks and map them to orthogonal planes of the 3D
grid. In the figure below, $I$ is distributed in the $XZ$ plane, and copied in the $Y$ dimension. $W$ is distributed in the $XY$ plane and copied along the $Z$ dimension. Once each GPU has a unique sub-block of I and W, it can compute a
portion of the $O$ matrix, which can be aggregated across GPUs in the $X$
direction using all-reduces.

\begin{figure}[h]
    \centering
%       \includegraphics[width=\columnwidth]{figs/schematic/agarwal-final.png}
      \includegraphics[width=\columnwidth]{figs/mm.pdf}
      \caption{Parallelization of a matrix multiply in an FC layer with Agarwal's 3D parallel matrix
multiplication algorithm~\cite{agarwal-3d} on eight GPUs organized in a
$2\times2\times2$ topology. We use $G_x$, $G_y$, and $G_z$ to refer to
the number of GPUs along the three dimensions of the virtual grid topology.}
      \label{fig:schematic-agarwal-data-dist}
\end{figure}

We modify Agarwal's algorithm to reduce memory consumption, and instead of
making copies of $W$ along the $Z$-axis, we further shard $W$ along the
$Z$-axis and denote these sub-shards as $\hat{W}$.
Algorithm~\ref{alg:3d-tensor} presents the forward and backward pass operations
on GPU $g_{i,j,k}$, and we can observe that the sharding of $W$ results in
all-gather operations before the local matrix multiplication on each GPU can proceed.

\begin{algorithm}[h]
    {\small 
    \caption{Tensor parallel algorithm for ${g}_{i,j,k}$ in a $G_{x} \times
G_{y} \times G_{z}$ grid. Communication operations highlighted in blue.} 
    \label{alg:3d-tensor}
    \begin{algorithmic}[1]
    \setlength{\lineskip}{5pt}
    \Function{tensor\_parallel\_forward\_pass}{$I_{k,j}$, $\hat{W}_{j,i}$}
        \State  $W_{j,i}$ = $\Call{\comm{${\text{all-gather}}_{z}$}}{\hat{W}_{j,i}}$
        \State  $\hat{O}_{k,i} = I_{k,j} \times {W}_{j,i}$
        \State $O_{k,i}$ $\gets$ \Call{\comm{${\text{all-reduce}}_{y}$}}{$\hat{O}_{k,i}$}
        \State // Cache $I_{k,j}$ and $W_{j,i}$  for the backward pass
        \State \Return $O_{k,i}$
    \EndFunction
    \State
    \Function{tensor\_parallel\_backward\_pass}{$\frac{\partial L}{\partial O_{k,i}}$}
        \State Retrieve  $I_{k,j}$ and $W_{j,i}$  from cache 
        \State $\hat{\frac{\partial L}{\partial I_{k,j}}}$ $\gets$ $\frac{\partial L}{\partial O_{k,i}} \times {W}^{\top}_{j,i}  $
        \State $\frac{\partial L}{\partial I_{k,j}}$ $\gets$ \Call{\comm{${\text{all-reduce}}_{x}$}}{$\hat{\frac{\partial L}{\partial I_{k,j}}}$}
        \State ${\frac{\hat{\partial L}}{\partial \hat{W}_{j,i}}}$ $\gets$ $ I^{\top}_{k,j}  \times {\frac{\partial L}{\partial O_{k,i}}} $
        \State ${\frac{\partial L}{\partial \hat{W}_{j,i}}}$ $\gets$ \Call{\comm{$\text{reduce-scatter}_{z}$}}{$ \frac{\hat{\partial L}}{\partial \hat{W}_{j,i}} $}
        \State \Return $\frac{\partial L}{\partial I_{k,j}}$, ${\frac{\partial L}{\partial \hat{W}_{j,i}}}$
    \EndFunction
    \end{algorithmic}
    }
\end{algorithm}

In the forward pass, after the local (to each GPU) matrix-multiply on line 3,
we do an all-reduce to aggregate the output activations (line 4). In the
backward pass, there are two matrix multiplies on lines 11 and 13, and
corresponding all-reduce and reduce-scatter operations in lines 12 and 14 to
get the data to the right GPUs.

\vspace{0.08in}
\noindent\textbf{Parallelizing an entire network:} 
The approach of parallelizing a single layer in a deep neural network can be
applied to all the layers individually. Let us consider a 2-layer neural
network.  If we use Algorithm~\ref{alg:3d-tensor} to parallelize each layer,
the output $O$ of the first layer would be the input to the other. However,
notice in Figure~\ref{fig:schematic-agarwal-data-dist} that $O$ is distributed
across the 3D virtual grid differently than the input $I$. So to ensure
that the second layer can work with $O$, we would need to transpose its weight
matrix -- essentially dividing its rows across the $X$-axis and columns across
the $Y$-axis. This transpose needs to be done once at the beginning of
training.  Hence, to parallelize a multi-layer neural network, we simply
`transpose' the weights of every other layer by swapping the roles of the
$X$- and $Y$- tensor parallel groups.

Note that the 4D algorithm (data + 3D PMM) discussed in this section is a generalization of
various state-of-the-art parallel deep learning algorithms. For example, if one
were to employ only the $Z$ axis of our PMM algorithm to parallelize training, it
would reduce to Fully Sharded Data Parallelism (FSDP)~\cite{fsdp} and
ZeRO~\cite{sc2020zero}. Similarly, if we employ the $Z$ axis of 3D PMM and data
parallelism simultaneously, then our algorithm reduces to Hybrid Sharded Data
Parallelism~\cite{fsdp} and ZeRO++~\cite{wang2023zero}. If we use the $X$
axis of our 3D PMM algorithm along with the `transpose' scheme discussed in the
previous paragraph, our 4D algorithm reduces to Shoeybi et al.'s
Megatron-LM~\cite{megatronlm}. Finally, when all four dimensions of our 4D algorithm are being used, this is similar to a hybrid scheme that combines data parallelism, FSDP, and two-dimensional tensor parallelism.


\subsection{A Performance Model for Identifying Near-optimal Configurations}
When assigned a job partition of $G$ GPUs, we have to decide how to organize
these GPUs into a 4D virtual grid, and how many GPUs to use for data
parallelism versus the different dimensions of 3D parallel martix
multiplication. To automate the process of identifying the best performing
configurations, we have developed a performance model that predicts the
communication time of a configuration based on the neural network architecture,
training hyperparameters, and network bandwidths. Using these predictions, we
can create an ordered list of the best performing configurations as predicted by
the model. We describe the inner-workings of this model below.

We primarily focus on modeling the performance of the collective operations in
the code, namely all-reduces, reduce-scatters, and all-gathers.  We first list
the assumptions we make in our model:
%
\begin{itemize}
    \item \emph{Assumption-1:}  
    The ring algorithm~\cite{thakurimproving2003} is used for implementing the all-reduce, reduce-scatter, 
    and all-gather collectives.
    \item \emph{Assumption-2:} For collectives spanning more than one compute node, the ring is formed such that the number of messages crossing node 
    boundaries is minimized.
    \item \emph{Assumption-3:} The message sizes are large enough, and hence, message startup overheads can be 
    ignored. 
    In other words, if a process is sending a message of $n$ bytes, then we assumed that the transmission time is simply 
    $\frac{n}{\beta}$, where $\beta$ is the available bandwidth between the two processes.
    \item \emph{Assumption-4:} We only model the communication times and ignore the effects of any computation taking 
    place on the GPUs.  
    \item \emph{Assumption-5:} We assume the same peer-to-peer bidirectional bandwidth, $\beta_{\mathrm{inter}}$, between every 
    pair of nodes.
\end{itemize}

We use the analytical formulations in Thakur et al.~\cite{thakurimproving2003}
and Rabenseifner~\cite{rabenseifneroptimization2004} for modeling the
performance of ring algorithm based collectives.  Let $t_{\mathrm{AG},z}$
denote the time spent in the all-gather across the $Z$-tensor parallel groups
(line 2 of Algorithm~\ref{alg:3d-tensor}). Similarly, we use
$t_{\mathrm{RS},z}$, $t_{\mathrm{AR},y}$ and $t_{\mathrm{AR}, x}$ to refer to
the time spent in the collectives in lines 14, 4, and 12 respectively.
Similarly, we use $t_{\mathrm{AR}, \mathrm{data}}$ for the time spent in the
data parallel all-reduce. Then, we can model these times as follows,
%
\begin{align}
    t_{\mathrm{AG},z} &= \frac{1}{\beta}\times (G_{z}-1) \times \frac{k \times n}{G_{x} \times G_{y} \times G_{z}}  
    \label{eqn:layer-ag} \\ \nonumber \\
    t_{\mathrm{RS},z} &= \frac{1}{\beta}\times \left( \frac{G_{z}-1}{\mathrm{G_{z}}} \right) \times \frac{k \times n}{G_{x} \times G_{y}} 
    \label{eqn:layer-rs} \\ \nonumber \\
    t_{\mathrm{AR},y} &= \frac{2}{\beta} \times \left( \frac{G_{y}-1}{G_{y}} \right) \times \frac{m \times n}{G_{z} \times \mathrm{G}_{x}}
    \label{eqn:layer-ar-1} \\ \nonumber \\
    t_{\mathrm{AR}, x} &= \frac{2}{\beta} \times \left( \frac{G_{x}-1}{G_{x}} \right)
    \times \frac{m \times k}{G_{z} \times G_{y}}
    \label{eqn:layer-ar-2} \\ \nonumber \\
    t_{\mathrm{AR}, \mathrm{data}} &= \frac{2}{\beta} \times \left( \frac{G_{\mathrm{data}}-1}{G_{\mathrm{data}}} \right) 
    \times \frac{k \times n}{G_{x} \times G_{y} \times G_{z}}
    \label{eqn:layer-ar-3}  
\end{align}

The total communication time for a single layer, ${t_{\mathrm{comm}}}$ is simply
the sum of Equations~\ref{eqn:layer-ag} through~\ref{eqn:layer-ar-3}: 
%
\begin{align}
    t_{\mathrm{comm}} = t_{\mathrm{AG},z} + t_{\mathrm{RS},z} +  t_{\mathrm{AR},y} + t_{\mathrm{AR},x} 
    + t_{\mathrm{AR}, \mathrm{data}} \label{eqn:layer} 
\end{align}
%
For layers with `transposed' weight matrices as discussed at the end of
Section~\ref{sec:hybrid}, we need to swap the values of $G_x$ and $G_y$. 
And finally, to model the communication time for
the entire network, we apply Equation~\ref{eqn:layer} to all of its layers, and
take a sum of the times.

In the equations derived above, we made a simplifying assumption that all
collectives in our hybrid parallel method can achieve the highest peer-to-peer
bandwidth, denoted by ${\beta}$. However, since several collectives are often
in operation at once, the actual bandwidth achieved for a collective operation
among a group of GPUs depends on the placement of processes in our 4D virtual
grid to the underlying hardware topology (nodes and network)~\cite{solomonik:sc2011, bhatele:sc2012b, abdel-gawad:sc2014, bhatele:hipc2014}. For example,
process groups that are contained entirely within a node can experience higher
bandwidths than those containing GPUs on different nodes. Next, we
model the specific bandwidths used in Equations~\ref{eqn:layer-ag}
through~\ref{eqn:layer-ar-3}.

To model the process group bandwidths, we begin by assuming a hierarchical organization of process groups: 
$X$-tensor parallelism (innermost), followed by $Y$-tensor parallelism, $Z$-tensor parallelism, and
data parallelism (outermost). As a concrete example, if we have eight GPUs, and set $G_{x}=G_{y}=G_{z}=G_{\mathrm{data}}=2$, 
then the $X$-tensor parallel groups comprise of GPU pairs 
(0,1), (2,3), (4,5), and (6,7). Similarly, the $Y$-tensor parallel groups would comprise of GPU pairs (0,2),
(1,3), (4,6), and (5,7), and so on.
% for the $Z$ and $\mathrm{data}$ parallel groups.

Now let $\vec{G} = (G_{x}, G_{y}, G_{z}, G_{\mathrm{data}})$ be the tuple of our configurable performance 
parameters, arranged in order of the assumed hierarchy. Let 
$\vec{\beta} = (\beta_{x}, \beta_{y}, \beta_{z}, \beta_{\mathrm{data}})$ be the effective peer-to-peer bandwidths 
for collectives issued within these process groups. We use $\vec{\beta}_{i}$ and $\vec{G}_{i}$ to represent the $i^{\mathit{th}}$
elements of these tuples ($0 \leq i \leq 3$). Also, let $G_{\mathrm{node}}$ refer to the number of GPUs per node. Now let us 
model each $\beta_{i}$ i.e. the bandwidth available to the GPUs in the process groups at the $i^{\mathit{th}}$ level of the 
hierarchy.

\vspace{0.08in}
\noindent{\em Case 1: GPUs in the process group lie within a node} --
in our notation, this is the scenario when $\prod_{j=0}^{i}G_{j} \leq G_{\mathrm{node}}$. 

The bandwidth $\vec{\beta}_{i}$ is determined by two primary factors: (i) the
size of the $i$th process group, $G_{i}$, and (ii) the cumulative product of
the sizes of all preceding process groups, $\prod_{j=0}^{i-1}G_{j}$. Given that
the number of GPUs per node is typically small, the number of possible
scenarios is also small. Therefore, we can profile the bandwidths for all
potential configurations in advance and store this information in a database.
Specifically, we generate all possible two-dimensional hierarchies of process
groups $(G_{0}, G_{1})$ such that $G_{0} \times G_{1} \leq G_{\text{node}}$,
and then perform simultaneous collectives within the outer process groups of
size $G_{1}$ with a large message size of 1 GB. We record the achieved
bandwidths for this tuple in our database. Then, for a given model, when we
need the predicted bandwidths for the $i^{th}$ process group, we retrieve the
bandwidth recorded for the tuple $(G_{0} = \prod_{j=0}^{i-1}G_{j}, G_{1} =
G_{i})$.

\begin{figure*}[t]
    \centering
    \includegraphics[width=0.49\textwidth]{figs/comm-model-validation/perlmutter_comm_model_20B_32GPU_v2.pdf}
    \includegraphics[width=0.49\textwidth]{figs/comm-model-validation/perlmutter_comm_model_40B_64GPU_v2.pdf}
    \caption{Plots validating the performance model by comparing the observed time per batch and the rank ordered by the model for two neural networks: GPT-20B (left) and GPT-40B (right).}
    \label{fig:rank-comm-model}
\end{figure*}

\vspace{0.08in}
\noindent{\em Case 2: GPUs in the process group are on different nodes} --
in our notation, this is the scenario when $\prod_{j=0}^{i}G_{j} > G_{\mathrm{node}}$. 

For process groups spanning node boundaries, the approach of recording all
possible configurations in a database is not feasible due to the vast number of
potential scenarios, given the large number of possible sizes of these groups
in a multi-GPU cluster.  Therefore, we develop a simple analytical model for
this scenario, which predicts the achieved bandwidths as a function of the
inter-node bandwidths ($\beta_{\mathrm{inter}}$), process group sizes
($\vec{G}$), and the number of GPUs per node ($G_{\text{node}}$).

First, let's first explore two simple examples to build some intuition. In Figure~\ref{fig:all-reduce-one}, we demonstrate a 
scenario with a single process group spanning eight GPUs on two nodes, with four GPUs on each node. In this case, the ring 
messages crossing node boundaries (i.e. the link between GPUs 1 and 4, and between GPUs 6 and 3) will be the 
communication bottleneck. Since we assumed $\beta_{\mathrm{inter}}$ to be the bidirectional bandwidth between node pairs, we 
can set $\beta_{i}=\beta_{\mathrm{inter}}$.

\begin{figure}[h]
    \centering
       \includegraphics[width=0.9\columnwidth]{figs/comm-model/one-all-reduce.pdf}
       \caption{Creation of a ring among eight GPUs on two nodes for a collective communication operation (all-reduce/reduce-scatter/all-gather).}
       \label{fig:all-reduce-one}
\end{figure}

\begin{figure}[h]
    \centering
       \includegraphics[width=0.9\columnwidth]{figs/comm-model/contention.pdf}
       \caption{Two rings among four GPUs each across two nodes for performing collective operations simultaneously.}
       \label{fig:all-reduce-many}
\end{figure}

Another possible scenario is when there are multiple simultaneous collectives taking place between two nodes. For example, consider 
Figure~\ref{fig:all-reduce-many}, wherein GPUs $(0, 4, 6, 2)$ and GPUs $(1, 5, 7, 3)$ are executing two independent collectives using the ring algorithm
simultaneously. In this case, the available inter-node bandwidth will be shared between these two collectives and 
$\beta_{i}=\frac{\beta_{\mathrm{inter}}}{2}$. 

The first scenario occurs in the case when the process groups preceding the $i^{\mathit{th}}$ process 
group in the hierarchy are of size one, i.e. $G_{j}=1$  $\forall j < i$. Whereas the second scenario occurs in the case 
when at least one of these preceding process groups is of a size $>1$. In that case, we get multiple ring messages crossing node boundaries
and the bandwidth gets distributed between the rings. However, note that the maximum
reduction in the bandwidth is bounded by the total number of GPUs on each node, as there can't be more 
inter-node ring links than GPUs on a node. Equation~\ref{eqn:bandwidth} models all the scenarios to obtain the observed bandwidth:
\begin{equation}
\vec{\beta}_{i} = \
         \dfrac{\beta_{\text{inter}}}{\min \left( G_{\text{node}}, \prod_{j=0}^{i-1}G_{j} \right)} \label{eqn:bandwidth}
\end{equation}
We use this bandwidth term in Equations~\ref{eqn:layer-ag}
through~\ref{eqn:layer-ar-3} of our model. We use the model to create an
ordered list of configurations, and then we can pick the top few configurations
for actual experiments.

\vspace{0.08in}
\noindent{\bf Validating the Performance Model}: To validate the model, we
collect the batch times for all possible configurations of the 4D virtual grid
when training GPT-20B on 32 GPUs and GPT-40B on 64 GPUs of Perlmutter. Using
the observed batch times, we label the ten fastest configurations as `efficient'
and the rest as `inefficient'. When creating the validation plots, we rank the
configurations using the ordering provided by the performance model.
Figure~\ref{fig:rank-comm-model} shows the empirical batch times on the Y-axis
and the rank output by the performance model on the X-axis. The fastest
configurations should be in the lower left corner. We observe that nine out of
the top ten configurations predicted by the performance model are indeed
`efficient' as per their observed batch times. This shows that the model is
working very well in terms of identifying the fastest configurations. 




\subsection{Automated Tuning of BLAS Kernels} \label{sec:blas-tune}
In deep neural networks, a significant portion of the computational workload is
matrix multiplications kernels or ``matmuls``, particularly in transformer
models. These matmuls can be performed in one of three main modes based on
whether the operands are transposed: NN, NT, and TN. Prior research has
highlighted that NT and TN kernels are often less optimized than NN kernels in
most BLAS libraries~\cite{shi2017tnvnn}. In our experiments, we found this
discrepancy to be more pronounced when running transformers with large hidden
sizes on the AMD MI250X GPUs of Frontier. For example, in the GPT-320B model
(described in Table~\ref{tab:setup-perf-gpt}), we observed that a matrix
multiply defaulting to the TN mode in PyTorch achieved only 6\% of the
theoretical peak performance, whereas other matmuls reached 55\% of the peak.

To address this issue, we implemented an automated tuning strategy in which,
during the first batch, each matmul operation in the model is executed in all
three modes (NN, NT, and TN) and timed. We then select the most efficient
configuration for each operation, which is subsequently used for the remaining
iterations. This tuning approach ensures that our deep learning framework,
AxoNN, avoids the pitfalls of using suboptimal matmuls that could significantly
degrade performance. For the aforementioned 320B model, our BLAS kernel tuning
approach successfully switches the poorly performing TN matmul with a nearly
8$\times$ faster NN matmul, thereby reducing the total time spent in
computation from 30.1 seconds to 13.19s! Note that for other models used in
Table~\ref{tab:setup-perf-gpt}, the speedups attained via tuning are relatively
modest (See Figure~\ref{fig:weak-scaling-breakdown}).



\subsection{Overlapping Asynchronous Collectives with Computation}

We use non-blocking collectives implemented in NCCL and RCCL on NVIDIA and AMD
platforms respectively. This enables us to aggressively overlap the collective
operations in AxoNN with computation, which can minimize communication
overheads.

\vspace{0.08in}
\noindent{\bf Overlapping All-reduces with Computation (OAR):}
In this performance optimization, we overlap the all-reduce across the
$X$-tensor parallel groups in the backward pass (Line 12 of
Algorithm~\ref{alg:3d-tensor}) with the computation in Line 13. Once this
computation has completed, we wait on the asynchronous all-reduce.
Note that for layers with `transposed' weight matrices, this communication happens across the $Y$-tensor parallel groups.  

\vspace{0.08in}
\noindent{\bf Overlapping Reduce-scatters with Computation (ORS):}
Next we overlap the reduce-scatters in the backward pass (line 14 of
algorithm~\ref{alg:3d-tensor}). The outputs of this reduce-scatter are the
gradients of the loss w.r.t.~the weights. These outputs are not needed until
the backward pass is completed on all the layers of the neural network and we
are ready to start issuing the all-reduces in the data parallel phase.
Exploiting this, we issue these reduce-scatters asynchronously and only wait on
them once all layers have finished their backward pass. This allows us to
overlap the reduce-scatter of one layer with the backward pass computations of
the layers before it.

\vspace{0.08in}
\noindent{\bf Overlapping All-gathers with Computation (OAG):}
Our next optimization overlaps the all-gather operations in the forward pass
(line 2 of Algorithm~\ref{alg:3d-tensor}) with computation. We observe that
this all-gather operation does not depend on any intermediate outputs of the
forward pass. Leveraging this, we preemptively enqueue the all-gather for the
next layer while the computation for the current layer is ongoing. At the start
of training, we generate a topological sort of the neural network computation
graph to determine the sequence for performing the all-gathers. Subsequently,
we execute them preemptively in this order.

Figure~\ref{fig:opt} shows the performance improvements from the three
successive collective overlap optimizations (OAR: Overlap of all-reduces, ORS:
Overlap of reduce-scatters, and OAG: Overlap of all-gathers). The baseline here
refers to the scenario with no communication overlap. We also show the breakdown of the total time per
batch into computation and communication. As we can see, the times spent in
computation do not change significantly, however, the time spent in non-overlapped
communication reduces with successive optimizations, leading to an overall
speedup. For the 80B model in the figure, we see a performance improvement of 18.69\%
over the baseline on 8,192 GCDs of Frontier.

\begin{figure}[h]
    \centering
      \includegraphics[width=\columnwidth]{figs/breakdown_communication}
      \caption{The impact of overlapping non-blocking collectives with
computation on the training times of different sized models on 8,192 GCDs of Frontier.}
\label{fig:opt}
\end{figure}

