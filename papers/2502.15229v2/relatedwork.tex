\section{Related Work}
\subsection{Supporting active participation of users in CRS}

\textcolor{black}{In the field of Human-Computer Interaction (HCI), research has highlighted not only the positive aspects of RSs that offer everyday convenience but also the negative user experiences arising from the passive role imposed on users by RS \cite{eslami2019user, sullivan2019reading}. Traditional RS typically operates through a one-sided interaction structure where the system automatically analyzes user preferences and delivers recommendations without user intervention. This approach can narrow the user’s perspective, hinder exploratory behavior \cite{wilkinson2018testing, pariser2011filter, guesmi2022interactive, chu2020towards}, and lead users to passively accept the system's biased decisions \cite{wilkinson2018testing, lee2015personalization, knijnenburg2016recommender}. To overcome these limitations, CRSs have emerged, enabling users to explore recommended items through real-time, multi-turn conversations with the RS \cite{cai2022impacts, bursztyn2021developing, cai2020predicting}. Specifically, CRS allows users to explicitly request the types of recommendations they want \cite{cai2022impacts, friedman2023leveraging, jannach2021survey, zhu2024reliable} and provide feedback on recommended items over multi-turn interactions \cite{jin2019musicbot}. By offering users a channel to explicitly express their preferences, CRS introduces several benefits. Users’ explicit articulation of their preferences can help address the cold-start problem \cite{li2023exploring, zhai2024actions, dai2023uncovering, di2023evaluating, sanner2023large, jannach2021survey} and enable more personalized recommendations, thereby enhancing their tailored experience \cite{zhu2024reliable}. Additionally, this approach enhances the system's explainability, allowing users to better understand the rationale behind recommendation results \cite{li2023large, liu2023chatgpt, lubos2024llm, silva2024leveraging, gao2024generative}. Beyond these performance-oriented advantages, the introduction of CRS marks a significant shift in user experience, transforming users from passive recipients of recommendations into active participants in the recommendation process. }

\textcolor{black}{Despite these benefits, existing CRS often provided limited and fixed responses due to technical constraints \cite{zhang2024towards}, which restricted users from freely designing their desired recommendation service experience. For example, while users can offer feedback on items suggested by CRS, they are typically limited to predefined attributes such as energy, tempo, genre, or artist \cite{jin2019musicbot, cai2021critiquing}. Consequently, when users attempt to present more detailed or complex demands, the system may struggle to interpret or respond appropriately \cite{jin2019musicbot}. Against this backdrop, advancements in LLMs have opened the possibility of transcending the limited interaction methods of existing CRS. LLMs enable more flexible and open-ended interactions, allowing users to engage with CRS in a more dynamic and unrestricted manner, thereby introducing a new paradigm for recommendations \cite{sharma2024generative}. The following section provides a detailed discussion of these new possibilities. }


\subsection{LLM-powered CRS}

\textcolor{black}{The advancement of LLMs has significantly enhanced the natural language interface between users and systems, enabling machines to engage in human-like conversations \cite{friedman2023leveraging}. LLM-powered conversational services (e.g., ChatGPT, Gemini) can dynamically perform a wide range of functions in response to user prompts \cite{anelli2024sixth, petruzzelli2024towards, tankelevitch2024metacognitive, deng2023unified}, allowing users to create and utilize personalized interactive services tailored to their unique objectives \cite{kim2019co}. This development also presents new opportunities in the field of CRS \cite{petruzzelli2024towards}. Users are no longer confined to recommendation systems provided by specific companies; instead, they can design and operate their own customized recommendation services. For instance, the GPT store hosts a variety of user-developed recommendation systems with diverse objectives, enabling users to freely interact and collaborate with CRS. This approach allows users to design a personalized recommendation service experience according to their individual preferences \cite{petruzzelli2024towards, friedman2023leveraging}, providing a user experience distinct from traditional RS and CRS. This possibility transforms the three key interaction stages of a recommendation system \cite{pu2012evaluating, harambam2019designing}—1) preference analysis through user input, 2) recommendation presentation, and 3) user feedback—by giving users direct control over how they engage with the CRS at each stage. More specifically, LLMs expand the flexibility of the input space \cite{tankelevitch2024metacognitive}, allowing users to freely provide information that will be used in preference analysis. Users can also explicitly specify the types of recommendations they wish to receive \cite{li2023pbnr, weisz2024design} and determine how they want to evaluate the recommended items. This increased level of control enables users to further customize their recommendation experience, resulting in a new form of personalized recommendation experience that is clearly distinct from that of traditional RS. }

\textcolor{black}{On the other hand, prior studies have primarily focused on the potential of LLMs to improve recommendation performance, leaving a research gap in understanding this potential from the user's perspective. LLM-powered CRS enables users to fully determine the details of their interactions with the recommendation system, allowing them to go beyond simple explicit interaction. This approach empowers users to design and customize their own recommendation services. Given that individuals have diverse goals for using recommendation systems, providing users with the freedom to create personalized recommendation services opens up new possibilities for more tailored user experiences. This study views the potential of LLMs as an opportunity to explore the possibilities of customized CRS. Specifically, we aim to investigate how users want to interact with CRS, for what purposes, and what new user experiences such interactions can bring. }