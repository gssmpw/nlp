%%%%%%%% ICML 2025 EXAMPLE LATEX SUBMISSION FILE %%%%%%%%%%%%%%%%%

\documentclass{article}
\newcommand{\CG}{\mathcal{G}\xspace}
\newcommand{\CV}{\mathcal{V}\xspace}
\newcommand{\CE}{\mathcal{E}\xspace}
\newcommand{\CA}{\mathcal{A}\xspace}
\newcommand{\CF}{\mathcal{F}\xspace}
\newcommand{\CR}{\mathcal{R}\xspace}
\newcommand{\CB}{\mathcal{B}\xspace}
\newcommand{\CX}{\mathcal{X}\xspace}
\newcommand{\CK}{\mathcal{K}\xspace}
\newcommand{\CM}{\mathcal{M}\xspace}
\newcommand{\CC}{\mathcal{C}\xspace}
\newcommand{\CL}{\mathcal{L}\xspace}
\newcommand{\CI}{\mathcal{I}\xspace}
\newcommand{\CQ}{\mathcal{Q}\xspace}
\newcommand{\CO}{\mathcal{O}\xspace}
\newcommand{\CP}{\mathcal{P}\xspace}
\newcommand{\CS}{\mathcal{S}\xspace}
\newcommand{\CT}{\mathcal{T}\xspace}
\newcommand{\CJ}{\mathcal{J}\xspace}
\usepackage[para]{footmisc}
\usepackage{subfig}
% \usepackage{subcaption}
% \usepackage{array}
% \usepackage{colortbl}


% Recommended, but optional, packages for figures and better typesetting:
\usepackage{microtype}
\usepackage{graphicx}
\usepackage{subfigure}
\usepackage{booktabs} % for professional tables
\usepackage{makecell}
% hyperref makes hyperlinks in the resulting PDF.
% If your build breaks (sometimes temporarily if a hyperlink spans a page)
% please comment out the following usepackage line and replace
% \usepackage{icml2025} with \usepackage[nohyperref]{icml2025} above.
\usepackage{hyperref}


% Attempt to make hyperref and algorithmic work together better:
\newcommand{\theHalgorithm}{\arabic{algorithm}}

% Use the following line for the initial blind version submitted for review:
% \usepackage{icml2025}



% If accepted, instead use the following line for the camera-ready submission:
\usepackage[accepted]{icml2025}

% For theorems and such
\usepackage{amsmath}
\usepackage{amssymb}
\usepackage{mathtools}
\usepackage{amsthm}
\usepackage{bm}
% \usepackage{cite}
% \usepackage{algorithmic}
\usepackage{graphicx}
\usepackage{textcomp}
\usepackage{xcolor}
\usepackage{wrapfig}
\usepackage{booktabs}
\usepackage{lipsum}
\usepackage{balance}
\usepackage{url}
\usepackage{enumerate}
% Custom imports
\usepackage{soul}
\usepackage{enumitem}
\usepackage{verbatimbox}
\usepackage{tcolorbox}
\usepackage{fancyvrb}
\usepackage{fancyref}
\usepackage{hyperref}
% \usepackage{caption}
\usepackage{subcaption}
% \usepackage[subtle]{savetrees}
% \usepackage{soul}
\usepackage{amsmath}
\usepackage{amssymb}
\usepackage{amsfonts}
\usepackage{tabularray}
\usepackage{array}
\usepackage[longtable]{multirow}
\usepackage{ragged2e}
\usepackage{longtable}
\usepackage{booktabs}
\usepackage{caption}
\usepackage{multirow}
\usepackage{colortbl}
\usepackage{xparse}
\usepackage[normalem]{ulem}
\usepackage{algpseudocode}
\usepackage{marvosym}
\usepackage{amssymb}
\usepackage[acronym]{glossaries}
\usepackage{pifont}
\usepackage{graphicx}


% if you use cleveref..
\usepackage[capitalize,noabbrev]{cleveref}

%%%%%%%%%%%%%%%%%%%%%%%%%%%%%%%%
% THEOREMS
%%%%%%%%%%%%%%%%%%%%%%%%%%%%%%%%
\theoremstyle{plain}
\newtheorem{theorem}{Theorem}[section]
\newtheorem{proposition}[theorem]{Proposition}
\newtheorem{lemma}[theorem]{Lemma}
\newtheorem{corollary}[theorem]{Corollary}
\theoremstyle{definition}
\newtheorem{definition}[theorem]{Definition}
\newtheorem{assumption}[theorem]{Assumption}
\theoremstyle{remark}
\newtheorem{remark}[theorem]{Remark}

% Todonotes is useful during development; simply uncomment the next line
%    and comment out the line below the next line to turn off comments
%\usepackage[disable,textsize=tiny]{todonotes}
\usepackage[textsize=tiny]{todonotes}


% The \icmltitle you define below is probably too long as a header.
% Therefore, a short form for the running title is supplied here:
\icmltitlerunning{}

\begin{document}

\twocolumn[
\icmltitle{Humanoid-VLA: Towards Universal Humanoid Control with Visual Integration}


% It is OKAY to include author information, even for blind
% submissions: the style file will automatically remove it for you
% unless you've provided the [accepted] option to the icml2025
% package.

% List of affiliations: The first argument should be a (short)
% identifier you will use later to specify author affiliations
% Academic affiliations should list Department, University, City, Region, Country
% Industry affiliations should list Company, City, Region, Country

% You can specify symbols, otherwise they are numbered in order.
% Ideally, you should not use this facility. Affiliations will be numbered
% in order of appearance and this is the preferred way.
\icmlsetsymbol{equal}{*}

\begin{icmlauthorlist}
\icmlauthor{Pengxiang Ding}{equal,yyy,sch}
\icmlauthor{Jianfei Ma}{equal,comp}
\icmlauthor{Xinyang Tong}{equal,yyy}
\icmlauthor{Binghong Zou}{comp}
\icmlauthor{Xinxin Luo}{comp}
\icmlauthor{Yiguo Fan}{yyy}
\icmlauthor{Ting Wang}{yyy}
%\icmlauthor{}{sch}
\icmlauthor{Hongchao Lu}{yyy}
\icmlauthor{Panzhong Mo}{comp}
\icmlauthor{Jinxin Liu}{comp}
\icmlauthor{Yuefan Wang}{yyy,sch}
\icmlauthor{Huaicheng Zhou}{comp}
\icmlauthor{Wenshuo Feng}{comp}
\icmlauthor{Jiacheng Liu}{yyy,sch}
\icmlauthor{Siteng Huang}{yyy}
\icmlauthor{Donglin Wang}{yyy,comp}
%\icmlauthor{}{sch}
%\icmlauthor{}{sch}
\end{icmlauthorlist}

\icmlaffiliation{yyy}{Milab, Westlake University}
\icmlaffiliation{comp}{Westlake Robotics}
\icmlaffiliation{sch}{Zhejiang University}

% \icmlcorrespondingauthor{Firstname1 Lastname1}{first1.last1@xxx.edu}
\icmlcorrespondingauthor{Donglin Wang}{wangdonglin@westlake.edu.cn}


% You may provide any keywords that you
% find helpful for describing your paper; these are used to populate
% the "keywords" metadata in the PDF but will not be shown in the document
\icmlkeywords{Machine Learning, ICML}

\vskip 0.3in
]

% this must go after the closing bracket ] following \twocolumn[ ...

% This command actually creates the footnote in the first column
% listing the affiliations and the copyright notice.
% The command takes one argument, which is text to display at the start of the footnote.
% The \icmlEqualContribution command is standard text for equal contribution.
% Remove it (just {}) if you do not need this facility.

% \printAffiliationsAndNotice{}  % leave blank if no need to mention equal contribution
\printAffiliationsAndNotice{\icmlEqualContribution} % otherwise use the standard text.




\begin{abstract}

% Recent works to jointly reconstruct 3D human and object from a single RGB image, are mostly model-based, that fail to capture the fine details of the clothed human body and object surface. In this paper, we introduce ReCHOR, a novel, model-free, first-method to produce realistic clothed human-object reconstructions from a monocular view. This is extremely challenging due to human-object occlusions, diverse interactions and depth ambiguity, as it needs to infer both 3D spatial awareness and high resolution details. Our core idea is based on estimating neural implicit representations for human and object respectively by an attention-based neural implicit model that attends to pixel-aligned features from both the global human-object image for spatial awareness and  the local separate view of human and object images for high quality details. Additionally, the network is conditioned on semantic features from an initial estimated human-object pose prior and a generative diffusion model that inpaints occluded regions, thus enabling the retrieval of details from them.
% We also propose a synthetic dataset with rendered scenes of diverse, inter-occluded 3D human and object scans, to train our network. We evaluate our method on the synthetic and real world BEHAVE dataset. Our experiments show that our method outperforms the SOTA in achieving realistic clothed human-object reconstructions.
Recent approaches to jointly reconstruct 3D humans and objects from a single RGB image represent 3D shapes with template-based or coarse models, which fail to capture details of loose clothing on human bodies. In this paper, we introduce a novel implicit approach for jointly reconstructing realistic 3D clothed humans and objects from a monocular view. For the first time, we model both the human and the object with an implicit representation, allowing to capture more realistic details such as clothing. This task is extremely challenging due to human-object occlusions and the lack of 3D information in 2D images, often leading to poor detail reconstruction and depth ambiguity. To address these problems, we propose a novel attention-based neural implicit model that leverages image pixel alignment from both the input human-object image for a global understanding of the human-object scene and from local separate views of the human and object images to improve realism with, for example, clothing details. Additionally, the network is conditioned on semantic features derived from an estimated human-object pose prior, which provides 3D spatial information about the shared space of humans and objects. To handle human occlusion caused by objects, we use a generative diffusion model that inpaints the occluded regions, recovering otherwise lost details. For training and evaluation, we introduce a synthetic dataset featuring rendered scenes of inter-occluded 3D human scans and diverse objects. Extensive evaluation on both synthetic and real-world datasets demonstrates the superior quality of the proposed human-object reconstructions over competitive methods.
\end{abstract}
\section{Introduction}
\label{sec:intro}
% Image editing methods in diffusion models depend on user-defined control directions - users can unlock their creativity using these methods by specifying the desired manipulation through prompts~\cite{gandikota2023concept}, reference images~\cite{ruiz2022dreambooth, kumari2022customdiffusion, gal2022image, chen2024trainingfreeregionalpromptingdiffusion}, or attribute vectors~\cite{parmar2023zero,hertz2022prompt}. In this work, we ask a fundamentally different question: \emph{Can we automatically discover the underlying visual structure of a concept within diffusion model's knowledge?} %Rather than requiring user-specified controls, we aim to decompose the model's internal knowledge into meaningful directions.

% This question touches on a fundamental limitation in how we interact with diffusion models. Current control methods ~\cite{zhang2023addingconditionalcontroltexttoimage, gandikota2023concept, ye2023ipadaptertextcompatibleimage,ye2023ipadaptertextcompatibleimage, hertz2024stylealignedimagegeneration, li2023photomaker, shi2024instantbooth, chen2024trainingfreeregionalpromptingdiffusion} require users to specify their desired manipulations in advance, limiting interactive creativity. This contrasts with natural human artistic workflows, where creators dynamically explore creative ideas while jointly refining them toward meaningful artistic outcomes~\cite{hoffmann2016modeling}. This synergy between specification and exploration is not new to generative models. Early GAN architectures naturally developed disentangled latent spaces that enabled continuous\cite{harkonen2020ganspace,radford2015unsupervised, wu2021stylespace, shen2020interfacegan}, compositional control over generated images. Users could explore these spaces to discover interesting variations that would be difficult to describe in words~\cite{wu2021stylespace}, then combine them to achieve their creative goals~\cite{grabe2022towards}. 


% While diffusion models have largely superseded GANs in conditional image synthesis~\cite{dhariwal2021diffusion},  their underlying structure remains less understood. Diffusion models achieve remarkable diversity through high-dimensional latents, unlike GANs' compact latent spaces.  With a single prompt, diffusion models can generate radically different variations through different random initializations of input noise. We ask - Is it possible to discover interpretable structure within this vast space of variations?

Text-to-image diffusion models are capable of generating remarkable visual variations from a single prompt through different random initializations. However, this vast creative potential remains largely opaque to users---while we can generate diverse images, we lack understanding of the underlying structure of these variations. This presents a fundamental challenge: how can we discover and expose the latent visual capabilities encoded within these models?

\let\thefootnote\relax \footnote{$^{*}$Correspondence to \texttt{gandikota.ro@northeastern.edu}}

The challenge touches on a key limitation in how we interact with diffusion models today. Current control methods require users to explicitly specify their desired edits in advance through prompts~\cite{gandikota2023concept}, reference images~\cite{zhang2023addingconditionalcontroltexttoimage, chen2024trainingfreeregionalpromptingdiffusion, ruiz2022dreambooth,kumari2022customdiffusion, Ryu_lora, hu2021lora}, or attribute vectors~\cite{ye2023ipadaptertextcompatibleimage, hertz2024stylealignedimagegeneration, li2023photomaker, shi2024instantbooth,parmar2023zero,hertz2022prompt}. That contrasts sharply with natural human creative workflows, where artists dynamically explore creative ideas and jointly refine them toward meaningful artistic outcomes~\cite{hoffmann2016modeling}. The need for pre-specified controls creates a barrier between users and the full creative potential of these models.

Interestingly, earlier generative models like GANs~\cite{gans,karras2019style,brock2018large} naturally developed more interpretable internal structures. Their compact latent spaces often exhibited emergent disentanglement~\cite{harkonen2020ganspace,radford2015unsupervised, wu2021stylespace, shen2020interfacegan}, enabling continuous and compositional control over generated images. Users could explore these spaces to discover interesting variations that would be difficult to describe in words~\cite{wu2021stylespace}, then combine them to achieve their creative goals~\cite{grabe2022towards}.

Diffusion models have largely superseded GANs in conditional image synthesis~\cite{dhariwal2021diffusion}, achieving greater diversity through much higher-dimensional latents. And yet an understanding of the underlying structure of these larger latent spaces has remained elusive. In this work, we ask a fundamental question: \emph{Can we automatically discover the visual structure within a diffusion model's knowledge of a concept?} Rather than requiring user-specified controls, we aim to decompose the model's internal representations into expressive directions that users can explore and combine.

To address these needs, we present \textbf{SliderSpace}, a framework that brings systematic explorability to diffusion models. Given just a text prompt, SliderSpace discovers a canonical set of meaningful, diverse, and controllable directions within the model's knowledge of that concept. Each direction is implemented as a low-rank adapter~\cite{hu2021lora} that can be scaled and composed with others, allowing users to explore and smoothly combine different aspects of variation, as shown in Figure~\ref{fig:intro}.

We ground SliderSpace discovery in three key requirements for meaningful decomposition of a diffusion model's visual manifold: 
\begin{enumerate}
    \item \textbf{Unsupervised Discovery:} The decomposition process should emerge from the intrinsic structure of the model's learned representation, rather than being guided by predefined attributes. This ensures we capture the true topology of the model's knowledge space rather than projecting our assumptions onto it.
    
    \item \textbf{Semantic Orthogonality:} Each discovered control must represent a distinct semantic direction. This is enforced in a semantic feature space, like CLIP, where every slider has an orthogonal effect in embeddings. This prevents discovering multiple controls that create similar semantic effects, making the system more efficient and easier.
    
    \item \textbf{Distribution Consistency:} Directions must induce consistent transformations across both random seeds and prompt variations. 
\end{enumerate}

These requirements naturally lead to our proposed framework, which we formalize in Section~\ref{sec:method}. As we show in our experiments, SliderSpace is architecture-agnostic, working with both conventional U-Net based models like Stable Diffusion~\cite{rombach2022high, rombach2022sd20, podell2023sdxl, turbo, dmd} and recent transformer-based architectures like Flux~\cite{flux}.

We demonstrate the expressiveness of SliderSpace through three applications: First, we show how SliderSpace can decompose high-level concepts into diverse and expressive components, revealing the natural axes of variation in the model's understanding. Second, we explore artistic style variation, where SliderSpace discovers directions that match or exceed the diversity of manually curated artist lists while being judged more useful by human evaluators. Finally, we show how SliderSpace can help reverse the mode collapse commonly observed in distilled diffusion models, restoring diversity while maintaining generation speed.

Beyond providing practical creative control, SliderSpace opens new avenues for understanding and utilizing the latent capabilities of diffusion models. By mapping these models' visual potential into intuitive, composable directions, we take a step toward making their creative possibilities more accessible and interpretable to users.

% Image editing methods in diffusion models unlock the creativity of users. In this work we ask an alternate question: \emph{Can we organize and expose what of the diffusion model is already capable of?}.
% Existing methods for controlling image generation typically require users to manually specify edit directions for desired changes. This process is time-consuming, requires technical expertise, and limits the spontaneity of the creative process. For instance, if a user wants to adjust the smile of a generated person, they must explicitly request this edit, often through imprecise prompt engineering or model fine-tuning. This approach of predefined controls or manual specifications restricts users from fully exploring the latent capabilities of the model. There may be interesting stylistic variations or attributes that the model can generate, but users have no easy way to discover or utilize these.

% Natural visual disentanglement was an emergent property in the latent space of Generative Adversarial Models (GANs) \cite{harkonen2020ganspace,radford2015unsupervised, wu2021stylespace, shen2020interfacegan}. In particular, it has been observed that StyleGAN~\cite{karras2019style} stylespace neurons offer detailed control over many meaningful aspects of images that would be difficult to describe in words~\cite{wu2021stylespace}. However, diffusion models do not share such a compact latent space~\cite{park2023unsupervised}; and efforts to uncover such a space in the semantic embeddings of the text conditioning have met with limited success \nik{Nick - is there a specific citation you were thinking about?}.

% In this work we introduce \textbf{SliderSpace}, which takes a step towards uncovering an analogous low dimensional representation of diffusion models' visual breadth; in essence treating the diffusion model as many generators sharing parameters, where a particular generator is defined by a specific prompt. For a given prompt we sample many random seeds (and optionally prompt expansions using an LLM), generate the corresponding images, and apply an off the shelf feature extractor (in this work CLIP, but our method can be applied to any differentiable feature extractor). We use PCA to analyze these features, and for each of the leading $k$ principal components we train a LoRA \cite{} which causes the diffusion model to produces images which increase the feature magnitude along that component when passed back through the same feature extractor. This leads to a 'Slider' for each principal component, because each LoRA can be scaled and applied to the original diffusion model, continuously varying those visual features in the generated results (as measured, in our case, by CLIP).

% There are many other works that enhance the controllability of diffusion models. One common approach is enabling users to add spatial constraints to a generation either manually, or via a reference image \cite{zhang2023addingconditionalcontroltexttoimage, chen2024trainingfreeregionalpromptingdiffusion}, a second is leveraging more abstract embeddings (e.g. identity, style) extracted from a reference image \cite{ye2023ipadaptertextcompatibleimage, hertz2024stylealignedimagegeneration, li2023photomaker, shi2024instantbooth}, a third is finetuning a foundation model to better generate a concept important to the user \cite{ruiz2022dreambooth, kumari2022customdiffusion, Ryu_lora, hu2021lora}, and a fourth (most relevant to this work) is finding low-rank adaptors of the model based on a prompt or small training set which can be scaled to provide continous control over one aspect of generated image (e.g. night vs day, basic vs luxury, etc.) \cite{gandikota2023concept}. SliderSpace is complementary to all of these methods and offers something distinct. All of the other methods we are aware require the user (and / or model designer) to know in advance what type of control they want. In contrast SliderSpace assists users in discovering and controlling hidden capabilities present in the diffusion model's distribution of possible generations.

%We propose that truly intuitive creative control in a text-to-image model should meet three key criteria: \emph{discoverability}, \emph{intuitiveness}, and \emph{specificity}. The model should reveal controllable attributes that may not be immediately obvious, offer controls that are easy to understand and manipulate, and ensure each control affects a distinct attribute of the generated image.

% We demonstrate the utility and power of SliderSpace using three applications built on top of SDXL-DMD \cite{dmd}, because its fast generation speed lends itself well to the continuous control offered by SliderSpace.

% First, we study concept decomposition (Section \ref{sec:concept_exp}), where we learn sliders for a specific concept (e.g. 'monster', 'waterfall', 'car'). Through quantitative metrics of diversity and text alignment we demonstrate that the learned sliders dramatically boost the diversity of generations when randomly applied without harming text alignment; we also ask humans to qualitatively judge these results in a user study where they find the SliderSpace results to be more 'Diverse', 'Useful', and 'Creative' than our baselines.

% Second, we attempt to compare the automatic discoveries of SliderSpace to a large scale manual study of artistic styles (Section \ref{sec:art_exp}), open-sourced by ParrotZone \cite{parrotzone}. In this study SDXL was prompted with over 4300 artist names,  and based on visual inspection the cases of successful stylistic mimicry recorded. Quantitatively SliderSpace more closely matches the distribution of artistic variation discovered by ParrotZone than other baselines, and in our user studies was judged to be significantly more 'Diverse' and 'Useful' than the baselines. To our surprise humans even judged SliderSpace results to be slightly more 'Diverse' than the results generated by the manually discovered artist names of \cite{parrotzone}.

% Third, we attempt to use SliderSpace to reverse the mode collapse commonly observed in distilled few-step diffusion models relative to the original teacher model (Section \ref{sec:diverse_exp}). We quantitatively demonstrate that applying SliderSpace to SDXL-DMD leads to more closely matching the distribution of images by the original teacher, SDXL.

%Through extensive experiments on various state-of-the-art text-to-image models, we demonstrate that SliderSpace significantly enhances user control and creative expression in AI-assisted image generation tasks. Our method enables a range of applications, including concept decomposition and control, diversity improvement in generated images, customization dissection and edits, and the exploration of artistic styles inherent in the model.

% SliderSpace goes beyond providing a practical tool for enhanced creative control. By mapping the visual potential of diffusion models it can open new avenues for generative creativity and deepens our understanding of each model's hidden potential.
\section{Related work}

In recent years, diffusion models have gained prominence as a powerful generative framework, excelling in tasks such as image~\citep{rombach2022latentdiff,nichol2021improvedddpm,nichol2021glide,ramesh2022hierarchicaldiff} and video synthesis~\citep{blattmann2023alignlatentvd,an2023latentshiftvd,ge2023preservecovd,guo2023animatediffvd,singer2022makeavideovd}. These models generate data by progressively denoising randomly initialized samples until a coherent structure or scene emerges. Leveraging the flexibility and effectiveness of generative models, they have been adapted to a wide range of tasks~\cite{zheng2023ddcot, tang2023cotdet, shi2024part2object, tang2023temporal, tang2023contrastive}, including 3D and 4D content generation. 
% In this section, we review the related works on diffusion models and their applications in this field.
In this section, we will review three parts: diffusion models, 4D scene representations, and 4D generation with diffusion models.


\paragraph{Diffusion for Generation}
Recently, diffusion models, pre-trained on large-scale datasets~\citep{schuhmann2022laion}, have made significant strides in generating high-quality and diverse visual content for both 2D image and video tasks ~\citep{rombach2022latentdiff,nichol2021improvedddpm,blattmann2023alignlatentvd,an2023latentshiftvd,huang2024free}. Leveraging aligned vision-language representations~\cite{shi2024plain, dai2024curriculum, shi2024devil, shi2023logoprompt, shi2023edadet, shi2022spatial}, these models can produce various forms of visual content with impressive diversity and realism conditioned on text or images. To adapt 2D diffusion models for 3D generation, some methods utilize Score Distillation Sampling Loss~\citep{poole2022dreamfusion,lin2023magic3d,chen2023fantasia3d,wang2024prolificdreamer} to distill 3D priors and train a neural radiance field~\citep{mildenhall2020nerf} for 3D asset creation. However, this approach often faces challenges such as slow training speeds and multi-face artifacts~\citep{shi2023mvdream}. To address these limitations, another strategy involves fine-tuning pre-trained 2D diffusion models to directly generate multi-view consistent images~\citep{shi2023mvdream,liu2023zero123,long2024wonder3d,liu2023syncdreamer,li2024era3d} from large-scale multi-view datasets~\citep{deitke2023objaverse}. These images are then processed through 3D reconstruction algorithms~\citep{wang2021neus,kerbl20233dgs,liu2023nero} to produce high-quality 3D assets. Despite these advancements, efficiently leveraging these techniques for 4D generation, ensuring both spatial and temporal coherence, remains a challenging problem.


\paragraph{4D Scene Representation}
Current 4D scene representations can be broadly categorized into two types based on their underlying 3D scene representation: 1) NeRF-based \citep{mildenhall2020nerf} and 2) 3D Gaussian Splatting (3DGS)-based \citep{kerbl20233dgs}. Both approaches extend static 3D scene representations into the temporal domain by introducing deformable fields or animation-driven training frameworks.
NeRF (Neural Radiance Fields) was initially proposed to encode the geometry and appearance of static scenes using implicit models with MLPs. Building upon this, many works have extended static NeRF to handle dynamic scenes, either by modeling a dynamic deformation field on the top of a canonical static scene representation~\citep{pons2021dnerf,tretschk2021nonrigidnerf,yuan2021stardnerf,park2021nerfies,fang2022fastdnerf} or by directly learning a time-conditioned radiance field~\citep{li2022neural3dvideo,gao2021dynamicviewsynthesis,park2021hypernerf,xian2021spacenerfvideo}. Despite its success, NeRF-based methods often face limitations in training and inference speed, making them less suitable for real-time applications.
Recently, 3D Gaussian Splatting (3DGS) has shown impressive performance due to its efficient training and real-time novel view synthesis capabilities. This method represents static scenes as a set of Gaussian primitives and employs a fast Gaussian differentiable rasterizer with adaptive density control. As an explicit representation, 3DGS also simplifies tasks such as scene editing. 
3DGS then has been applied to model dynamic scenes with the similar idea of building a deformation field~\citep{luiten2023dynamic3dgauss,wu20244dgauss,yang2024deformable4dgauss,zeng2024stag4d,wu2024sc4d}.
For example, Dynamic 3D Gaussians~\citep{luiten2023dynamic3dgauss} enable the Gaussians to move and rotate over time under local rigid constraints. This approach efficiently models fine details and temporal changes, making it highly effective for 4D content creation.
Together, these representations offer a robust framework for generating realistic and temporally coherent dynamic scenes in 4D space, supporting applications such as animation, scene reconstruction, and motion capture.

\paragraph{4D Generation} By efficiently integrating advanced diffusion techniques with 4D scene representations, significant progress has been made toward 4D generation. One approach in this direction leverages Score Distillation Sampling~\citep{poole2022dreamfusion} to distill spatial and temporal prior knowledge from multiple diffusion models into a 4D scene representation, producing spatially and temporally consistent 4D objects, including text-to-video and text-to-image generation. 
A pioneering work, MAV4D~\citep{singer2023mav3d} introduced a multi-stage training pipeline for dynamic scene generation, utilizing a Text-to-Image (T2I) model to initialize static scenes and a Text-to-Video (T2V)~\citep{singer2022makeavideovd} model to handle motion dynamics.
Building on this paradigm, several methods have sought to improve 4D generation quality by incorporating image conditions ~\citep{zhao2023animate124}, hybrid Score Distillation Sampling~\citep{bahmani20244dfy}, strategies that decouple static elements from dynamic ones~\citep{zheng2024dreamin4d}, and related techniques. However, these methods are largely based on NeRF variants, which suffer from issues like over-saturated appearance and long optimization times. To overcome these limitations,  Align-Your-Gaussians~\citep{ling2024alignyourgauss} proposed using dynamic 3D Gaussian Splatting (3DGS)~\citep{kerbl20233dgs} as the underlying 4D scene representation to learn a deformation field~\citep{park2021nerfies,pons2021dnerf}, offering faster training and better real-time capabilities. Despite this, the reliance on SDS loss in these methods leads to slow optimization speeds, limiting their applicability in downstream tasks.
Another approach uses video as guidance. Several video-to-4D frameworks~\citep{jiang2023consistent4d,yin20234dgen,pan2024fastdy4d} have been introduced that use video inputs as references to guide 4D generation. These methods attempt to generate dynamic scenes by leveraging video-driven information for more precise motion dynamics.
Additionally, to ensure multi-view consistency, recent works have focused on retraining multi-view video diffusion models~\citep{zhang20244diffusion,liang2024diffusion4d,li2024vividzoo,ren2024l4gm,jiang2024animate3d} with 4D datasets, integrating both spatial and temporal modules. However, these models often require large amounts of data and are computationally intensive.

\input{includes/3_0_overview}
% \begin{figure*}[t]
  \centering
  \includegraphics[width=\textwidth, height=10cm]{images/mindmap2.pdf} 
  \caption{Mindmap showing Data Collection and Rewrite Desiderata}
  \label{fig:mindmap}
\end{figure*}
% \begin{figure*}[t]
%   \centering
%   \includegraphics[width=\textwidth]{images/process.pdf} 
%   \caption{Dataset Creation Pipeline}
%   \label{fig:process}
% \end{figure*}
\section{Constructing a Dataset for Visual Instruction Rewriting}
\label{sec:datasets}

Task-oriented conversational AI systems rely on a semantic parser to interpret user intent and extract structured arguments \cite{louvan2020recent,aghajanyan2020conversational}. For example, when a user says,\textit{ "Add the team meeting to my calendar for Friday at 3 PM"}, the system must parse the intent (\textit{CreateCalendarEvent}) and extract arguments such as the \textit{EventTitle} (``team meeting''), \textit{EventDate} (``Friday''), and \textit{EventTime} (``3 PM'') to schedule the event correctly. Unlike purely text-based interactions, multimodal instructions, particularly those directed at conversational AI assistants on AR/VR devices (\textit{e.g.,} Apple's Siri for Apple Vision Pro), introduce additional challenges such as ellipsis and coreference resolution. For instance, a user may look at a book cover and ask, \textit{“Who wrote this?”} or point at a product in an AR interface and say, \textit{“How much does this cost?”} Traditional text-based semantic parsers struggle with such instructions since critical visual context is missing. Thus, to bridge the gap between multimodal input and existing conversational AI stacks, we introduce a dataset specifically designed for \textit{rewriting multimodal instructions} into structured text that can be processed by standard text-based semantic parsers. Figure \ref{fig:mindmap} illustrates a representation of the dataset collection requirement, highlighting the transformation of multimodal inputs into text-based rewrites.

To construct our dataset, we first define an ontology of intents and arguments, as existing ontologies in conversational AI and semantic parsing are often proprietary and unavailable for research use. We take inspiration from \newcite{goel2023presto} for ontology and extend it to accommodate multimodal task-oriented interactions. Figure \ref{fig:intent_argument_box} (ref. Appendix) presents an overview of the intents and arguments in our ontology. Next, we curate a diverse set of images covering various real-world multimodal interaction scenarios, including book covers, product packaging, paintings, mobile screenshots, flyers, signboards, and landmarks. These images are sourced from publicly available academic datasets, such as OCR-VQA\footnote{\url{https://ocr-vqa.github.io/}}, CD and book cover datasets, Stanford mobile image datasets\footnote{\url{http://web.cs.wpi.edu/~claypool/mmsys-dataset/2011/stanford/}}, flyer OCR datasets\footnote{\url{https://github.com/Skeletonboi/ocr-nlp-flyer.git}}, signboard classification datasets\footnote{\url{https://github.com/madrugado/signboard-classification-dataset}}, Google Landmarks\footnote{\url{https://github.com/cvdfoundation/google-landmark}}, and Products-10K\footnote{\url{https://products-10k.github.io/}}.

\begin{table}[t]
    \centering
    \scriptsize
    \label{tab:dataset_statistics}
    \begin{tabular}{llccc}
        \toprule
        \textbf{Category} & \textbf{Total} & \textbf{Train} & \textbf{Test} \\
        \midrule
        Book              & 485 / 500                               & 386 / 399                               & 101 / 101                               \\
        Business Card     & 26 / 960                                & 26 / 772                                & 26 / 188                                \\
        CD               & 27 / 1,020                              & 27 / 835                                & 27 / 185                                \\
        Flyer & 159 / 5,940                             & 159 / 4,742                             & 159 / 1,198                             \\
        Landmark         & 511 / 19,274                            & 511 / 15,420                            & 511 / 3,854                             \\
        Painting & 27 / 980                                & 27 / 774                                & 27 / 206                                \\
        Product          & 499 / 10,349                            & 499 / 8,276                             & 492 / 2,073                             \\
        \midrule
        \textbf{Total}   & \textbf{1,734 / 39,023}                 & \textbf{1,635 / 31,218}                 & \textbf{1,343 / 7,805}                  \\
        \bottomrule
    \end{tabular}
    \caption{Number of Images/Instructions per Category}
    \label{tab:sources}
\end{table}
\begin{table}[t]
    \centering
    \footnotesize
    \begin{tabular}{l  c}
        \toprule
         \textbf{Annotator}& \textbf{Percentage of Correct Captions}\\ 
         \midrule
         Annotator 1	& 90.62\%\\ 
         Annotator 2	& 87.23\%\\
         Annotator 3	& 86.35\%\\
         \midrule
         \textbf{At least two }& \textbf{92.18}\%\\
         \midrule
         \textit{All three }& \textit{74.63}\% \\
         \bottomrule
    \end{tabular}
    \caption{GPT-4 Instruction Rewriting Validation Results from Amazon Mechanical Turk }
    \label{tab:annotator_data}
\end{table}
\begin{figure}[t]
\includegraphics[width=\columnwidth]{images/intent.png}
  \caption{Dataset Distributions By Intent}
  \label{fig:intent}
\end{figure}
Upon identifying and verifying the images, we employ the GPT-4 model from OpenAI \cite{achiam2023gpt} to systematically generate and refine multimodal instructions into rewritten text-based instructions. The process begins with a bootstrap phase, where GPT-4 is prompted to generate 20 direct questions per image by explicitly referencing visible objects or textual elements while adhering to the intent list defined in Figure \ref{fig:intent_argument_box}. A second prompting phase then validates the generated questions against the corresponding image, filtering out ambiguous or irrelevant instructions to ensure alignment with the visual context. 

In the rewriting phase, GPT-4 is tasked with paraphrasing the validated instructions, ensuring that the transformed questions are fully self-contained and interpretable without requiring the image. This transformation is crucial for enabling multimodal conversational AI systems to process instructions using purely text-based stacks. Finally, a verification phase prompts the model to assess the rewritten questions in relation to both the original instruction and the image, ensuring semantic fidelity and eliminating inconsistencies. This multi-stage prompting strategy resulted in a dataset of 39,023 original-rewritten instruction pairs, derived from 1,734 images, with an 80\%-20\% train-test split. Table \ref{tab:sources} provides a breakdown of image sources.

While automated validation ensures consistency across different stages, human evaluation remains critical for verifying the dataset’s reliability. To this end, we conducted an annotation task via Amazon Mechanical Turk (AMT) to validate rewritten instructions within the test set for indirect image-based instructions. Each annotation task followed a structured validation guideline, where annotators reviewed an image, its original multimodal instruction, and the rewritten text-only instruction, determining whether the reformulation preserved the intent and meaning of the original instruction. Annotators were instructed to select "Accept" if the rewritten instruction was correct or "Reject" if it failed to capture the original meaning. Annotators are incentivized appropriately for this binary grading task. Agreement analysis, as shown in Table \ref{tab:annotator_data}, indicates that in 92.2\% of cases, at least two annotators agreed on "Accept," while 74.6\% of instructions achieved full consensus across all three annotators. Despite a Fleiss' Kappa score of 0.278—suggesting fair inter-annotator agreement—the high rate of majority consensus supports the dataset’s reliability for real-world use. Given these results, we publicly release the full dataset along with raw AMT responses, enabling further analysis, filtering, and refinements by the research community.

Figure \ref{fig:intent} presents the distribution of intents in our dataset, categorized into training and test splits. The distribution reflects practical usage patterns in real-world multimodal conversational AI systems, with a higher occurrence of general QA and web search, alongside diverse task-oriented intents such as reminders, messaging, and navigation, ensuring coverage of frequent user interactions.



% In this study, we utilize a comprehensive multimodal dataset curated from various sources to facilitate research in multimodal instruction rewriting using compact models. Table~\ref{tab:dataset_statistics} provides an overview of the dataset's composition, detailing the number of images and corresponding instructions sourced from different domains. This diverse dataset is designed to challenge models in interpreting and rewriting instructions based on both visual and textual information embedded within images.

% The dataset is organized into a single TSV file, \texttt{all\_data.tsv}, which consolidates all the data for streamlined processing and analysis.

% The dataset is publicly accessible and can be downloaded from our Hugging Face repository:
% \url{https://huggingface.co/datasets/utischoolnlp/multimodal_instruction_rewrites}.

% \begin{table}[h]
%     \centering
%     \caption{Dataset Statistics}
%     \label{tab:dataset_statistics}
%     \resizebox{0.5\textwidth}{!}{%
%         \begin{tabular}{|l|l|c|c|}
%             \hline
%             \textbf{Data Source} & \textbf{Type} & \textbf{Number of Images} & \textbf{Number of instructions} \\ \hline
%             \href{https://github.com/gulvarol/grocerydataset}{Grocery Store Dataset} & Grocery Dataset & 287 & 5,945 \\ \hline
%             \href{https://amazon-berkeley-objects.s3.amazonaws.com/index.html}{Amazon Berkeley Objects} & Amazon Dataset & 187 & 3,890 \\ \hline
%             \href{https://products-10k.github.io/}{Products-10K} & E-commerce Dataset & 23 & 472 \\ \hline
%             \href{https://www.kaggle.com/datasets/vikashrajluhaniwal/fashion-images}{Fashion Images} & Fashion Clothing Dataset & 2 & 42 \\ \hline
%             \textbf{Total} & & \textbf{499} & \textbf{10,349} \\ \hline
%         \end{tabular}
%     }
% \end{table}


% \subsection*{Additional Dataset Statistics}

% To provide a deeper understanding of the dataset's characteristics, we present the following statistics derived from \texttt{all\_data.tsv}:

% \begin{itemize}
%     \item \textbf{Prompt Length}:
%     \begin{itemize}
%         \item \textbf{Average Prompt Length}: 80.99 tokens
%         \item \textbf{Maximum Prompt Length}: 160 tokens
%         \item \textbf{Minimum Prompt Length}: 28 tokens
%     \end{itemize}
    
%     \item \textbf{Rewritten Question Length}:
%     \begin{itemize}
%         \item \textbf{Average Rewritten Question Length}: 56.94 tokens
%         \item \textbf{Maximum Rewritten Question Length}: 160 tokens
%         \item \textbf{Minimum Rewritten Question Length}: 28 tokens
%     \end{itemize}
% \end{itemize}

% These statistics highlight the complexity and variability of the prompts and their corresponding rewritten questions, providing a robust foundation for training and evaluating multimodal instruction rewriting models.

% \subsection*{Dataset Composition}

% The dataset is consolidated into a single TSV file, \texttt{all\_data.tsv}, which includes all image-instruction pairs. This unified format simplifies data handling and ensures consistency across training and evaluation phases. The structure of \texttt{all\_data.tsv} is as follows:


% \begin{itemize}
%     \item \textbf{Columns}:
%     \begin{itemize}
%         \item \texttt{Image\_ID}: Unique identifier for each image.
%         \item \texttt{Image\_URL}: Direct link to the image file.
%         \item \texttt{Prompt}: Original instruction associated with the image.
%         \item \texttt{Rewritten\_Question}: Reformulated version of the original instruction.
%     \end{itemize}
% \end{itemize}

% \subsection*{Dataset Accessibility}

% Researchers and practitioners can access the dataset and its associated resources through our Hugging Face repository:
% \url{https://huggingface.co/datasets/utischoolnlp/multimodal_instruction_rewrites}.

% The dataset is organized in a structured format, including:
% \begin{itemize}
%     \item \texttt{all\_data.tsv}: Consolidated dataset containing all image-instruction pairs.
%     \item \texttt{images.zip}: Compressed archive of all dataset images.
%     \item \texttt{README.md}: Detailed instructions and metadata descriptions for dataset usage.
% \end{itemize}

% \subsection*{Discussion}

% The diversity of data sources, ranging from grocery items to fashion clothing, ensures that the dataset covers a wide array of visual and textual contexts. This variety is crucial for training models that are robust and generalizable across different domains. The substantial number of instructions relative to images indicates that each image is associated with multiple instructions, providing ample data for effective model training and evaluation.

% By consolidating all data into a single TSV file, we streamline the data processing pipeline, facilitating easier integration with various modeling frameworks and tools. The comprehensive statistics on prompt and rewritten question lengths further underscore the dataset's complexity, challenging models to handle a wide range of instruction formulations.

% \section*{Conclusion}

% Our multimodal instruction rewriting dataset offers a comprehensive resource for researchers aiming to develop and evaluate models in this domain. By providing a diverse and sizeable dataset, we aim to facilitate advancements in multimodal understanding and contribute to the broader field of artificial intelligence.

% \section*{References}

% \begin{itemize}
%     \item \href{https://github.com/gulvarol/grocerydataset}{Grocery Store Dataset}
%     \item \href{https://amazon-berkeley-objects.s3.amazonaws.com/index.html}{Amazon Berkeley Objects}
%     \item \href{https://products-10k.github.io/}{Products-10K}
%     \item \href{https://www.kaggle.com/datasets/vikashrajluhaniwal/fashion-images}{Fashion Images Dataset}
% \end{itemize}

% \label{sec:dataset}
\section{Causal IL as CMRs}\label{sec:method}

In this section, we demonstrate that performing causal IL in our framework is possible using trajectory histories as instruments. In the next step, we show that the problem can be described as CMRs and propose an effective algorithm to solve it.

The typical target for IL would be the expert policy $\pi_E$ itself. However, since the expert has access to information, namely $u^o_t$, which the imitator does not, the best thing an imitator can do is to learn a history-dependent policy $\pi_h$ that is the closest to the expert. A natural choice is the conditional expectation of $\pi_E(s_t,u^o_t)$ on the history $h_t$:
\begin{align}
\pi_h(h_t)\coloneqq \expectE_{\probP(u^o_t\mid h_t)}[\pi_E(s_t,u^o_t)]=\expectE[\pi_E(s_t,u^o_t)\mid h_t],\nonumber
\end{align}
% where $p(u^o_t\mid h_t)$ is a distribution over expert-observable confounders and captures the information about $u^o_t$ can be inferred from the trajectory history. 
because the conditional expectation minimizes the least squares criterion~\citep{hastie01statisticallearning} and $\pi_h$ is the best predictor of $\pi_E$ given $h_t$. In $\pi_h$, the distribution $\probP(u^o_t\mid h_t)$ captures the information about $u^o_t$ that can be inferred from trajectory histories.
\begin{remark}
\emph{Learning $\pi_h$ is not trivial. Policies learnt naively using behaviour cloning (i.e., $\expectE[a_t\mid h_t]$) fail to match $\pi_E$. In view of~\cref{eq:action}, we have that
\begin{align} 
\expectE[a_t\mid h_t]&=\expectE[\pi_E(s_t,u^o_t) \mid h_{t}]+\expectE[u^\epsilon_t\mid h_{t}]\nonumber\\
&=\pi_h(h_t)+\expectE[u^\epsilon_t\mid h_{t}],\label{eq:history_policy}
\end{align}
where $\expectE[u^\epsilon_t\mid h_{t}]\neq 0$ due to the spurious correlation between $u^\epsilon_t$ and the trajectory history $h_t$. As a result, $\expectE[a_t\mid h_t]$ becomes biased, which can lead to arbitrarily worse performance compared to $\pi_E$.   }
\end{remark}

\vspace{-5pt}
\paragraph{Derivation of CMRs.} 
Leveraging the confounding horizon from Assumption~\ref{assump:horizon}, it becomes possible to break the spurious correlation using the independence of $u^\epsilon_t$ and $u^\epsilon_{t-k}$. We propose to use the $k$-step trajectory history $h_{t-k}=(s_{1},a_{1},...,s_{t-k})$ as an instrument for the current state $s_t$. Taking the expectation conditional on $h_{t-k}$ in~\cref{eq:history_policy} yields
\begin{align*}
    \expectE[a_t\mid h_{t-k}] & = \expectE\left[\expectE[a_t\mid h_{t}]\mid h_{t-k}\right] \\ & = \expectE[\pi_h(h_t)\mid h_{t-k}]+\expectE[\expectE[u^\epsilon_t\mid h_{t}]\mid h_{t-k}] \\
    & = \expectE[\pi_h(h_t) \mid h_{t-k}]+\expectE[u^\epsilon_t\mid h_{t-k}]
\end{align*}
where we use the fact that $h_{t-k}$ is $\sigma(h_t)$-measurable because $h_{t-k}\subseteq h_t$. Next, recall that $u^\epsilon_t\indep u^\epsilon_{t-k}$ by Assumption~\ref{assump:horizon}, which implies $u^\epsilon_t\indep h_{t-k}$, so that % Hence, since $\expectE[u^\epsilon_t] = 0$, we obtain
\begin{align}
    \expectE[a_t\mid h_{t-k}] &= \expectE[\pi_h(h_t) \mid h_{t-k}]+\expectE[u^\epsilon_t]\nonumber\\
    &=\expectE[\pi_h(h_t) \mid h_{t-k}].
\end{align}

As a result, the problem of learning $\pi_h$ reduces to solving for $\pi_h$ that satisfies the following identity
\begin{align}
    \expectE[a_t-\pi_h(h_t)\mid h_{t-k}]=0,\label{eq:CMR}
\end{align}
which is a CMR problem as defined in~\cref{sec:cmr}. In this case, both $a_t$ and $h_t$ are observed in the confounded expert demonstrations, and $h_{t-k}$ acts as the instrument. 

To make sure the instrument $h_{t-k}$ is valid, we check that it satisfies the conditions of~\cref{assump:iv}. Firstly, we have checked that $u^\epsilon_t\indep h_{t-k}$. Secondly, the environment and the expert policy are non-trivial, which means $\probP(h_t\mid h_{t-k})$ is not constant in $h_{t-k}$. Finally, $h_{t-k}$ indeed only affects $a_t$ through $s_t$ by the Markovian property. However, the strength of the instrument, which informally represents the correlation between the instrument $h_{t-k}$ and $h_t$, plays an important role in how well we can identify $\pi_h(h_t)$ by solving the CMRs in~\cref{eq:CMR}. In particular, we see that, as the confounding horizon $k$ increases, the correlation between $h_{t-k}$ and $h_t$ weakens and $h_{t-k}$ becomes a weaker instrument. This means that it is less able to identify $\pi_h$ via the CMR in~\cref{eq:CMR} and the final learnt imitator will have poorer performance. This is confirmed theoretically in Proposition~\ref{prop:ill-posed} and experimentally in~\cref{sec:exps}, and we will formalise this notion of instrument strength in~\cref{sec:theory}.


% Note this problem is equivalent to solving an IV regression on~\cref{eq:history_policy}, where $Y=\expectE[a_t\lvert h_t]$, $f(x)=\pi_h(h_t)$, $\epsilon=\expectE[u^\epsilon_t$ and the instrument $Z=h_{t-k}$.




\subsection{Practical Algorithms for Solving the CMRs}

\begin{algorithm}[tb]
   \caption{DML-IL}
   \label{alg:DML-IL}
\begin{algorithmic}[1]
   \STATE {\bfseries input} Dataset $\dataset_E$ of expert demonstrations, Confounding noise horizon $k$
   \STATE Initialize the roll-out model $\hat{M}$ as a Gaussian mixture model\label{algo:roll_out_1}
    \REPEAT
   \STATE Sample $(h_{t},a_t)$ from data $\dataset_E$
   \STATE Fit the roll-out model $(h_t,a_t)\sim\hat{M}(h_{t-k})$ to maximize the log likelihood 
\UNTIL{convergence}\label{algo:roll_out_2}
   \STATE Initialize the expert model $\hat \pi_h$ as a neural network
   \REPEAT
   % \FOR{$k=1$ {\bfseries to} $K$}
   \STATE Sample $h_{t-k}$ from $\dataset_E$
   \STATE Generate $\hat{h}_t$ and $\hat{a}_t$ using the roll-out model $\hat{M}$
   \STATE Update $\hat \pi_h$ to minimise the loss $\ell:= \norm{\hat{a}_t - \hat{\pi}_h (\hat h_t)}_2$
   % \ENDFOR
    \UNTIL{convergence}
    \STATE {\bfseries return} A history-dependent imitator policy $\hat{\pi}_h$
\end{algorithmic}
\end{algorithm}

There are various techniques~\citep{Shao2024,Bennett2019,Xu2020,Dikkala2020} for solving the CMRs $\expectE[a_t\lvert h_{t-k}]=\expectE[\pi_h(h_t) \lvert h_{t-k}]$. Here, the \textit{CMR error} that we aim to minimise is given by 
\begin{align*}
\sqrt{\expectE\big[\expectE[a_t-\hat{\pi}_h(h_t)\lvert h_{t-k}]^2\big]}=\norm{\expectE[a_t-\hat{\pi}_h(h_t)\lvert h_{t-k}]}_{2}.    
\end{align*}
In~\cref{alg:DML-IL}, we introduce DML-IL, an algorithm adapted from the IV regression algorithm DML-IV~\citep{Shao2024}\footnote{DML stands for double machine learning~\citep{Chernozhukov2018Double}, which is a statistical technique to ensure fast convergence rate for two-step regression, as is the case in~\cref{alg:DML-IL}.}, which solves our CMRs by minimising the CMR error. The first part of the algorithm (line 3-7) learns a roll-out model $\hat{M}$ that generates a trajectory $k$ steps ahead given $h_{t-k}$. Then, the roll-out model $\hat{M}$ is used to train the policy model $\hat{\pi}_h$ (line 8-13). $\hat{\pi}_h$ takes the generated trajectory $\hat{h}_t$ from $\hat{M}(h_{t-k})$ as inputs, and minimises the mean squared error to the next action. Using generated trajectories is crucial in breaking the spurious correlation caused by $u^\epsilon_t$ between past states and actions, and using the trajectory history before $h_{t-k}$ allows the imitator to infer information about $u^o_t$.

DML-IL can also be implemented with $K$-fold cross-fitting, where the dataset is partitioned into $K$ folds, with each fold alternately used to train $\hat{\pi}_h$ and the remaining folds to train $\hat{M}$. This ensures unbiased estimation and improves the stability of training. The base IV algorithm DML-IV with $K$-fold cross-fitting is theoretically shown to converge at the rate of $O(N^{-1/2})$~\citep{Shao2024}, where $N$ is the sample size, under regularity conditions. DML-IL with $K$-fold cross-fitting (see~\cref{appendix:dmlil} for details) will thus inherit this convergence rate guarantee. 

Note that~\cref{alg:DML-IL} requires the confounding noise horizon $k$ as input. While the exact value of $k$ can be difficult to obtain in reality, any upper bound $\bar{k}$ of $k$ is sufficient to guarantee the correctness of ~\cref{alg:DML-IL}, since $h_{t-\bar{k}}$ is also a valid instrument. Ideally, we would like a data-driven approach to determine $k$. Unfortunately, it is generally intractable to empirically verify whether $h_{t-k}$ is a valid instrument from a static dataset, especially the unconfounded instrument condition (i.e., $h_{t-k}\indep u^\epsilon_t$). Therefore, we rely on the user to provide a sensible choice of $\bar{k}$ based on the environment that does not substantially overestimate $k$.


\subsection{Theoretical Analysis}\label{sec:theory}

% \begin{align}
% p(u_t\lvert do(a_{t-k+1}),...,do(a_{t-1}),s_{t-k+1},...,s_{t-1})&\propto p(h_t)p_{\mu_0}(s_{t-k+1})\prod_{i=t-k+1}^{t-1} \transitions(s_{i+1}\lvert s_i,a_i,u_i)
% \end{align}

% since $$(u_t\indep a_{(t-k+1)...(t-1)} \lvert s_{(t-k+1)...(t_1)})_{\mathcal{G}_{\underline{a{(t-k+1)...(t-1)}}}}$$
% on the causal graph $\mathcal{G}_{\underline{a{(t-k+1)...(t-1)}}}$ where the arrows going into $a_{(t-k+1)...(t-1)}$ are removed.



In this section, we derive theoretical guarantees for our algorithm, focusing on the imitation gap and its relationship with existing work.


On a high level, in order to bound the imitation gap of the learnt policy $\hat{\pi}_h$, i.e., $J(\pi_E)-J(\hat{\pi}_h)$, we need to control:
\begin{enumerate}
    \item[($i$)] The amount of information about the hidden confounders that can be inferred from trajectory histories;
    \item[($ii$)] The ill-posedness (or identifiability) of the set of CMRs, which intuitively measures the strength of the instrument $h_{t-k}$;
    \item[($iii$)] The disturbance of the confounding noise to the states and actions at test time.
\end{enumerate}
These factors are all determined by the environment and the expert policy. To control ($i$), we measure how much information about $u^o_t$ is captured by the trajectory history $h_t$ by analysing the Total Variation (TV) distance between the distribution of $u^o_t$ and $\expectE[u^o_t\lvert h_t]$ along the trajectories of $\pi_E$. To control ($ii$) and ($iii$), we need to introduce the following two key concepts.

\begin{definition}[The ill-posedness of CMRs~\citep{Dikkala2020,Chen2012}]

Given the derived CMRs in~\cref{eq:CMR}, for a policy $\pi\in\Pi$, $\norm{\pi_E-\pi}_2$ is the root mean squared error to the expert and $\norm{\expectE[a_t-\pi(s_t)\lvert s_{t-k}]}_2$ is the CMR error we aim to minimise. Then, the \emph{ill-posedness} $\ill(\Pi,k)$ of the policy space with confounding noise horizon $k$ is given by
\begin{align*}
    \ill(\Pi,k)=\sup_{\pi\in\Pi} \frac{\norm{\pi_E-\pi}_{2}}{\norm{\expectE[a_t-\pi(h_t)\lvert h_{t-k}]}_{2}}.
\end{align*}
\end{definition}
The ill-posedness $\ill(\Pi,k)$ measures the strength of the instrument where a higher $\ill(\Pi,k)$ indicates a weaker instrument. It bounds the ratio between the learning error of the imitator following our CMR objective and its $L_2$ error to the expert policy. 

As discussed previously, intuitively, the strength of the instrument would decrease as the confounding horizon $k$ increases. This is in fact true and is confirmed by the following proposition. The proof is deferred to~\cref{appendix:prop}. 
\begin{proposition}\label{prop:ill-posed}
The ill-posedness $\ill(\Pi,k)$ is monotonically increasing as the confounded horizon $k$ increases.
\end{proposition}

Next, we introduce the notion of c-TV stability.
\begin{definition}[c-total variation stability~\citep{Bassily2021,Swamy2022_temporal}]
Let $P(X)$ be the distribution of a random variable $X:\Omega\rightarrow \mathcal{X}$. $P(X)$ is c-TV stable if for $a_1,a_2\in \mathcal{X}$ and $\Delta>0$,
\begin{align*}
\norm{a_1-a_2}\leq\Delta \implies \delta_{TV}(a_1+X,a_2+X)\leq c\Delta.
\end{align*}
where $\norm{\cdot}$ is some norm defined on $\mathcal{X}$ and $\delta_{TV}$ is the total variation distance.
\end{definition}
A wide range of distributions are c-TV stable. For example, standard normal distributions are $\frac{1}{2}$-TV stable. We apply this notion to the distribution over $u^\epsilon_t$ to bound the disturbance it induces in the trajectory and the expected return.

With the notion of ill-posedness and c-TV stability, we can now analyse and upper bound the imitation gap $J(\pi_E)-J(\hat{\pi}_h)$ by controlling the three components $(i)-(iii)$ discussed above. 
% We present the main result for this paper, where t
The full proof is deferred to~\cref{appendix:gap}.

\begin{theorem}[Imitation Gap Bound]\label{thm:gap}
Let $\hat{\pi}_h$ be the learnt policy with CMR error $\epsilon$ and let $\ill(\Pi,k)$ be the ill-posedness of the problem. Assume that $\delta_{TV}(u^o_t,\expectE_{\pi_E}[u^o_t\lvert h_t])\leq\delta$ for $\delta\in\realNumber^+$, $P(u^\epsilon_t)$ is c-TV stable and $\pi_E$ is deterministic. Then, the imitation gap is upper bounded by 
\begin{align*}
    J(\pi_E)-J(\hat{\pi}_h)\leq T^2\big(c\epsilon\ill(\Pi,k)+2\delta\big)=\mathcal{O}\big(T^2(\delta+\epsilon)\big).
\end{align*}
\end{theorem}
This upper bound scales at the rate of $T^2$, which aligns with the expected behaviour of imitation learning without an interactive expert~\citep{Ross2010}.
Next, we show that the upper bounds on the imitation gap from prior work~\citep{Swamy2022_temporal, Swamy2022} are special cases of
% of  subsumed by the unifying causal IL framework introduced in Section~\ref{sec:setting} are special cases of 
Theorem~\ref{thm:gap}. The proofs are deferred to~\cref{appendix:corollaries}.
\begin{corollary}\label{corollary:noUo}
In the special case that $u^o_t = 0$, i.e., there are no expert-observable confounders, or $u^o_t=\expectE_{\pi_E}[u^o_t\lvert h_t]$, i.e., $u^o_t$ is $\sigma(h_t)$ measurable (all information about $u^o_t$ is contained in the history), the imitation gap is upper bounded by
\begin{align*}
    J(\pi_E)-J(\hat{\pi}_h)\leq T^2\big(c\epsilon\ill(\Pi,k)\big)=\mathcal{O}\big(T^2\epsilon\big),
\end{align*}
which coincides with Theorem 5.1 of~\citet{Swamy2022_temporal}.
\end{corollary}

When there are no hidden confounders, i.e, $u^\epsilon_t=0$, our framework is reduced to that of~\citet{Swamy2022}. However, \citet{Swamy2022} provided an abstract bound that directly uses the supremum of key components in the imitation gap over all possible Q functions to bound the imitation gap. We further extend and concretise the bound using the learning error $\epsilon$ and the TV distance bound $\delta$ instead of relying on the suprema.


\begin{corollary}\label{corollary:unconfounded}
In the special case that $u^\epsilon_t=0$, if the learnt policy has optimisation error $\epsilon$,  the imitation gap is upper bounded by
\begin{align*}
    J(\pi_E)-J(\hat{\pi}_h)\leq T^2\left(\frac{2}{\sqrt{\dim(A)}}\epsilon+2\delta \right),
\end{align*}
which is a concrete bound that extends the abstract bound in Theorem 5.4 of~\cite{Swamy2022}.
\end{corollary}

\begin{remark}
\emph{If both $u^\epsilon_t$ and $u^o_t$ are zero, we then recover the classic setting of IL without confounders~\citep{Ross2010}, and the imitation gap bound is $T^2\epsilon$, where $\epsilon$ is the optimisation error of the algorithm.}
\end{remark}
\section{Experiments}
We evaluate the proposed \textbf{Humanoid-VLA} in terms of its ability to enable universal humanoid control.
We structure the experiments to answer
the following questions:
\textbf{1) RQ1}: Does \textbf{Humanoid-VLA} generate kinematically accurate and physically plausible motions?
\textbf{2) RQ2}: How effective is the humanoid control with vision integration?

\subsection{Evaluation on motion generation}
In this section, we take a comprehensive evaluation of the quality of the pose trajectory generated by the model.
To comprehensively demonstrate the effectiveness of our approach, we access motion quality from two perspectives:
\textbf{1) Kinematic fidelity:}
This metric evaluates the kinematic performance, measuring positional changes without considering physical dynamics.
Following~\cite{mao2024learning}, we evaluate our model on the standard text-to-motion (T2M) task, which generates motion sequences based on textual action descriptions. It highlights the model's core capability of translating natural language into human motion.  

\textbf{2) Physical plausibility:}
Unlike the above metric, this evaluation assesses the physical feasibility of generated poses in real-world environments.
Beyond standard T2M tasks, we evaluate our model on more challenging scenarios that exceed the capabilities of existing models, particularly tasks incorporating diverse input conditions such as joint trajectories. This comprehensive assessment demonstrates the robustness and versatility of the model across a broad spectrum of applications.

\vspace{-4pt}
\subsubsection{Kinematic Fidelity }
\vspace{-4pt}
\textbf{Setup.} 
We evaluate the motion quality using a widely used dataset HumanML3D~\cite{guo2022humanml3d} and our collected dataset Humanoid-S, which contains manually annotated action descriptions for human pose extracted from 4646 video clips.
%
While HumanML3D focuses on basic locomotion patterns such as running, swimming, and dancing, Humanoid-S encompasses more complex human actions.
%
We choose the whole testing dataset and randomly select one textual description per clip to serve as the input for evaluation. 
%
For a fair comparison, we evaluate all models using 15 joints consistent with our model's configuration, selected for their presence in both humans and humanoid robots to enhance generalizability.

\begin{table}
\centering
\resizebox{0.9\linewidth}{!}{%
\begin{tabular}{@{}lccccccc@{}}
\toprule
\multirow{2}{*}{Types} &  
\multirow{2}{*}{Input} & 
\multicolumn{4}{c}{Accuracy}
\\ \cmidrule(lr){3-6} 
&
&$E_{\text{mpjpe}}^{g}\downarrow$ 
% &$E_{\text{mpjpe}}^{l}\downarrow$ 
&$E_{\text{mpjpe}}^\text{pa}\downarrow$ 
& $E_{\text{accel}} \downarrow$ 
& $E_{\text{vel}}\downarrow$ 
\\ \midrule
\multirow{4}{*}{Easy}&

{D} &
36.13 & 1.53 & 34.42 & 18.73
\\
&
{T} &
 36.57 & 1.48 & 35.10 & 18.53
\\
&
{A} &
 39.02  & 1.32 & 34.32 & 17.91
\\
& 
 {$S_n$} &
 36.29 & 1.55 & 34.93 & 18.88
\\
\midrule

\multirow{3}{*}{Medium}&
{D + T} &
 {31.07} & {1.18}  & {27.84} & {14.76}
\\
&
 {D + A} &
 36.98& 1.30& 34.87& 18.16
\\
&
 {D + $S_n$} &
 35.75 & {1.18} &33.41 &17.18
\\
\midrule
\multirow{1}{*}{Hard}& 
 {D + $S_1$ + $S_N$} 
 & 37.14 & 1.34 & 34.69 & 18.08
\\
   \bottomrule
\end{tabular}
}
\caption{\textbf{Physical plausibility of generated motion under versatile conditions.} Humanoid-VLA provides four conditional input types: motion description (\textbf{D}), motion time duration (\textbf{T}), motion sequence with absent body parts (\textbf{A}), and motion state (\textbf{$S_n$}) at time n within total N timesteps.
Based on input combinations, we establish three tiers of motion generation tasks with increasing complexity.}
\label{tab:controllable}
\vspace{-1.em}
\end{table}

\begin{table}
\centering
\resizebox{1\linewidth}{!}{%
\begin{tabular}{@{}ccccccc@{}}
\toprule
\multirow{2}{*}{Method} & 
\multicolumn{2}{c}{HumanML3D}& \multicolumn{2}{c}{Humanoid-S}\\
\cmidrule(lr){2-3}
\cmidrule(lr){4-5}  
% & MM Dist$\downarrow$ 
& FID$\downarrow$ 
% & MModality$\uparrow$
& DIV$\uparrow$ 
% & MMDist$\downarrow$ 
& FID$\downarrow$ 
% & MModality$\uparrow$
& DIV$\uparrow$ 
\\ \midrule
MDM&
% $\boldsymbol{3.740}^{\pm.095}$ &
${0.889}^{\pm.026}$ &
${3.855}^{\pm.053}$ &
${2.351}^{\pm.590}$ &
${4.111}^{\pm.261}$ &
\\
T2M-GPT&
% $4.512^{\pm.165}$ &
${0.531}^{\pm.020}$ &
${4.555}^{\pm.058}$ &
${1.101}^{\pm.189}$ &
${4.199}^{\pm.218}$ &
\\\midrule
% 10 epoch微调版本结果
\modelname &
% $4.270^{\pm.206}$ &
$\boldsymbol{0.467}^{\pm.018}$ & 
$\boldsymbol{4.585}^{\pm.086}$ &
$\boldsymbol{1.037}^{\pm.147}$ & 
$\boldsymbol{4.466}^{\pm.213}$ &
\\
 \midrule
\end{tabular}%
}
\caption{
\textbf{Kinematic fidelity of generated motion in HumanML3D and Humanoid-S.} We use FID score and Diversity to evaluate the quality of the motion generated by the model, where bold values indicate the best results.
}
\label{tab:t2m}
\vspace{-1.5em}
\end{table}

\textbf{Metrics.}
We follow the evaluation framework from ~\cite{guo2022fid}, utilizing two established metrics to evaluate the quality of motion generation: \textbf{(1) FID} measuring distribution similarity between generated and real motions, and \textbf{(2) Diversity} quantifying action variation degree, and calculating the average Euclidean distance between 200 randomly generated motions.
Lower FID indicates better distribution matching and higher DIV scores reflect superior diversity.

\textbf{Baselines.}
We consider two baselines commonly used in humanoid control:
{\textbf{(1) MDM}~\cite{tevet2023human}}: A diffusion-based generation model that utilizes a classifier-free paradigm to produce natural and diverse motions.
{\textbf{(2) T2M-GPT}~\cite{zhang2023t2m}}: A transformer-based generation model that combines VQ-VAE\cite{van2017neural} with an autoregressive approach to generate human motions from text.

\textbf{Implementation details.}
We utilize Llama3-70B~\cite{dubey2024llama} as the foundation model. In the training phase, warm up ratio is set at 0.01, with learning rate configured at 2e-5, and the cosine learning scheduler. The batch size per device is set to 4. For the encoder of each body part, its codebook size is set to 1024. We conduct model training using 8 NVIDIA H100 GPUs through {216} hours.


\begin{figure*}[t]
  \centering
   \includegraphics[width=1\linewidth]{imgs/real.pdf}
   \caption{\textbf{Robot experiments in real world}. \modelname demonstrates its ability to interact with objects, showcasing robust performance in real-world environments. The humanoid model successfully executes precise object-kicking tasks and avoids obstacle task in real-world scenarios.}
   \label{fig:real}
   \vspace{-1.5em}
\end{figure*}





\textbf{Results.}
The comparative evaluation results between \modelname and the baseline models are presented in Table~\ref{tab:t2m}. On the HumanML3D dataset, \modelname achieves a significant FID score of 0.467, representing substantial improvements of 47.5\% and 12\% over MDM and T2M-GPT respectively, which indicates its superior capability in capturing real motion distributions. Furthermore, \modelname demonstrates remarkable performance in motion diversity, attaining a diversity score of 4.466 on the Humanoid-S dataset, outperforming MDM by 6\%. This achievement is particularly noteworthy as it reflects the model's ability to generate diverse motions under challenging linguistic constraints. The comprehensive experimental results demonstrate that \modelname excels in high-quality action generation, establishing its effectiveness in text-to-motion synthesis tasks.


\begin{table}
\centering
\resizebox{1\linewidth}{!}{%
\begin{tabular}{@{}cccccccc@{}}
\toprule
& 
Low-quality data &
\multicolumn{1}{c}{High-quality Data} &
\multirow{2}{*}{FID$\downarrow$} &
\multirow{2}{*}{DIV$\uparrow$ }
% \multicolumn{3}{c}{Motion-to-Text}& 
\\ \cmidrule(lr){2-2}\cmidrule(lr){3-3}& w aug & w aug   
% MModality$\uparrow$

% & MMDist$\downarrow$ 
\\ \midrule
&\checkmark & &
${0.698}^{\pm.037}$ & 
${4.576}^{\pm.098}$ &\\
& &\checkmark &
${0.557}^{\pm.016}$ & 
${3.867}^{\pm.062}$ &\\
 & \checkmark&\checkmark &
$\boldsymbol{0.467}^{\pm.018}$ & 
$\boldsymbol{4.585}^{\pm.086}$ &\\
   \bottomrule
\end{tabular}%
}
\caption{\textbf{Ablation on data augmentation.} Here, low-quality data refers to motion data extracted through human motion recovery, which tends to lack precision. In contrast, high-quality data refers to motion data obtained directly from physical devices, ensuring greater accuracy.}
\label{tab:aug}
\vspace{-1.5em}
\end{table}



\subsubsection{Physical Plausibility }
\textbf{Setup.} 
We evaluate this metric in the IsaacGym physics simulator~\cite{makoviychuk2021isaac}.
Following~\cite{he2024omnih2o,ji2024exbody2}, we assess the humanoid's tracking accuracy in executing the model-generated kinematic trajectories to quantify physical plausibility.



\textbf{Metrics.}
Our metrics are designed across two dimensions: (1) \textit{\textbf{State-related.}} The global Mean Per-Joint Position Error (MPJPE) $E_{\text{mpjpe}}^{g}$ (mm) quantifies the average positional error of individual joints. The Procrustes-aligned MPJPE (PA-MPJPE) $E_{\text{mpjpe}}^\text{pa}$ (mm) eliminates global scale and rotational discrepancies to assess shape accuracy specifically. (2) \textit{\textbf{Transition-related.}} We evaluate acceleration error $E_{\text{accel}}$ (mm/s²) and velocity error $E_{\text{vel}}$ (mm/s) metrics to assess physical plausibility by computing the average joint-level distances in acceleration and velocity respectively.
For all metrics, lower values correspond to better performance.



\textbf{Baselines.}
As these conditional motion tasks are uniquely solvable by our model, conventional baselines are not applicable. Therefore, we focus on evaluating our approach independently, adopting the tracking error widely accepted in ~\cite{ji2024exbody2} to evaluate the effectiveness of our method.


\textbf{Results.}
As shown in Table \ref{tab:controllable}, our RL policy achieves robust motion imitation joint control with mean position errors \(E_{\text{mpjpe}}^g\) consistently below 40 mm, and minimum score 31.07mm under medium difficulty with caption and time conditions. The policy achieves remarkably low errors in pose accuracy $E_{\text{mpjpe}}^\text{pa}$ at {1.18mm}, acceleration $E_{\text{accel}}$ at {27.84mm}, and velocity $E_{\text{vel}}$ at {14.76mm}, demonstrating smooth and physically consistent motion generation. This experiment validates our method's ability to generate high-quality motions across diverse control conditions while preserving physical plausibility and control fidelity.


\textbf{Ablation on data augmentation.} 

As shown in Table \ref{tab:aug}, incorporating extensive video motion data reduces the FID from 0.557 to 0.467, representing a 16\% improvement. This significant enhancement demonstrates that large-scale motion data extracted from videos strengthens the alignment between motion and language. 
We can still achieve comparable results Even with low-quality mocap data for fine-tuning. These two points strongly validate the significance of incorporating video data to expand the training process.
These findings underscore the effectiveness of our self-supervised data augmentation strategy. 



\vspace{-10pt}
\subsection{Evaluation on vision integration }

\vspace{-5pt}
\textbf{Experimental setup.}

We evaluate the performance of our \modelname model in real-world environments utilizing visual information. An RGB camera is employed to capture first-person view images, and experiments are conducted using the Unitree G1 robot across four task categories: upper-body motion, lower-body motion, full-body motion, and object interaction. These tasks, such as approaching targets, kicking a ball, and obstacle navigation, require visual guidance for accurate positioning, aiming to validate the effectiveness of our VLA approach.




\textbf{Results.}
For each task category, we carefully select 4 representative tasks and evaluate 10 tests on each task, using success rate as the evaluation metric.
Our humanoid-VLA model shows great performance in various object interaction tasks shown in Table ~\ref{tab:vis_SR}.
Selected tasks are shown in Figure~\ref{fig:real}. 
In the "kick ball" task, our humanoid model enables the robot to effectively utilize visual information to accurately approach the object and execute a kicking motion. 
In the "avoid obstacles" task, our robot successfully navigates around obstacles to reach the desired target position. 
These results demonstrate that our VLA model effectively leverages visual information to generate appropriate motions.





\begin{table}[ht]
  \centering
  \resizebox{0.45\linewidth}{!}{
  \begin{tabular}{@{}c|cc@{}}
    \toprule
    \textbf{Task} & \textbf{\makecell[c]{SR}} \\
    \midrule
 Turn to an object & 10/10 \\
 Hold an object  & 9/10 \\

 Wave to people & 10/10 \\


Avoid an obstacle  & 9/10 \\  
Jump over an object & 9/10 \\  
Dance with a partner  & 8/10 \\ 
 Punch an obstacle & 10/10 \\ 
 Kick a ball  & 9/10 \\


    \bottomrule
  \end{tabular}
  }
  \caption{\textbf{Evaluation on Vision Integration.} }
  \label{tab:vis_SR}
  \vspace{-2em}
\end{table}












\section{Conclusion}
This paper introduces Humanoid-VLA, a novel framework designed to address the challenges of humanoid robot control with egocentric visual integration. The framework aligns language and motion using human motion datasets, enables context-aware motion generation through cross-attention mechanisms, and tackles data scarcity using self-supervised pseudo-annotations. Built on whole-body control architectures, Humanoid-VLA facilitates adaptive object interaction and exploration with enhanced contextual understanding.
The effectiveness of Humanoid-VLA has been validated through evaluations of motion generation quality and execution success rates on real humanoid robots, demonstrating high executability. In the future, we aim to enhance the success rate of humanoid robots in performing more complex loco-manipulation tasks.


\section*{Impact Statement}
This paper presents work whose goal is to advance the field of Machine Learning in Robotics. There are many potential societal consequences of our work, none which we feel must be specifically highlighted here.


% \clearpage

% \section*{Impact Statement}
% This paper presents work whose goal is to advance the field of Machine Learning. There are many potential societal consequences of our work, none which we feel must be specifically highlighted here.

\bibliography{example_paper}
\bibliographystyle{icml2025}

\appendix

\section{Appendix: Prompt}
\label{sec:appendix}
``Here is a sketch of an image. 
$\{input\_color\_mask\}$, while the rest of the white space is the background. 
I need you to infer details of the image based on the given sketch.
The details should include the possible background likely to be present with the $\{input\_color\_mask\}$, the attribute of each object (like wearing, texture, color etc.), the state (including action, posture, etc.) of each object, the direction of each object and the relationships between objects.

You should first analyze the mask carefully, considering the size, location, and relative position of each object mask. Ensure that specific actions are analyzed based on the mask, and infer each aspect with a reasoning process before providing the final output.
The final output format should be: $\{format\_example\}$, and you should refer to the example: $\{few\_shot\}$. You are going to complete the "" in each item, you need to complete them in multiple short phrases based on your above reasoning.

The state and relationship should be as detailed as possible while ensuring they align with the mask, formatted as: objectA action/spatial relation objectB, with both objectA and objectB included.
You should properly refer to some examples of attributes of object $\{attributes\}$ and relationships $\{relationships\}$.
Do not include words like `or', `possibly' in your final output, there should no ambiguity in your output.
Make sure all aspects of given mask is filled.''


\end{document}


% This document was modified from the file originally made available by
% Pat Langley and Andrea Danyluk for ICML-2K. This version was created
% by Iain Murray in 2018, and modified by Alexandre Bouchard in
% 2019 and 2021 and by Csaba Szepesvari, Gang Niu and Sivan Sabato in 2022.
% Modified again in 2023 and 2024 by Sivan Sabato and Jonathan Scarlett.
% Previous contributors include Dan Roy, Lise Getoor and Tobias
% Scheffer, which was slightly modified from the 2010 version by
% Thorsten Joachims & Johannes Fuernkranz, slightly modified from the
% 2009 version by Kiri Wagstaff and Sam Roweis's 2008 version, which is
% slightly modified from Prasad Tadepalli's 2007 version which is a
% lightly changed version of the previous year's version by Andrew
% Moore, which was in turn edited from those of Kristian Kersting and
% Codrina Lauth. Alex Smola contributed to the algorithmic style files.
