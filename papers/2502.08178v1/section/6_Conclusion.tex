\section{Conclusion}
% 在本研究中,我们提出了ParetoRAG,一种无监督的文档细化框架,通过基于帕累托原则的句子级优化来提升RAG系统的性能。通过将段落分解为句子,并动态地重新加权关键内容,同时保持上下文的一致性,ParetoRAG在不需要额外训练或API资源的情况下,实现了检索精度和生成质量的双重提升。大量实验证明了其有效性:该框架在多个数据集、大型语言模型(LLM)和检索器中,能够将令牌消耗减少70%,同时提高答案的准确性和流畅性。我们的分析进一步揭示了当ParetoRAG与经过强健训练的语言模型结合时,产生了协同效应,表明其增强了泛化能力。本研究不仅验证了基于资源高效的句子级细化在RAG系统中的可行性,还为探索结合检索增强机制与自适应训练策略的混合方法开辟了新途径。
In this work, we propose ParetoRAG, an unsupervised framework that enhances RAG systems through sentence-level optimization guided by the Pareto principle. By decomposing paragraphs into sentences and dynamically re-weighting critical content while preserving contextual coherence, ParetoRAG achieves dual improvements in retrieval precision and generation quality without requiring additional training or API resources. Extensive experiments demonstrate its effectiveness: the framework reduces token consumption by 70\% while improving the accuracy and fluency of the answers in diverse datasets, LLMs and retrievers. Our analysis further reveals synergistic effects when integrating ParetoRAG with robustly trained language models, suggesting enhanced generalization capabilities. This study not only validates the viability of resource-efficient sentence-level refinement for RAG systems but also opens avenues for exploring hybrid methodologies that combine retrieval-augmented mechanisms with adaptive training strategies.

\section{Limitation}


% 基于句子级的分解与重加权可能削弱段落内跨句子的复杂逻辑或叙事关联,尤其在需要多步推理或长程语义连贯的任务(如故事生成、科学论证)中,关键信息的局部聚焦可能导致全局结构松散化。
%  1 2 逻辑重复
%  在处理较长的文档时,ParetoRAG 可能需要面临长文本的切分和句子级优化带来的挑战。将长文本拆分成句子并逐一优化,可能无法有效保持文档的全局结构和逻辑流畅.

% ParetoRAG 在开放域问答数据集上进行了测试,但尚未在更为专业化的领域(如法律、医学等)进行测试,这可以作为未来的研究方向。

While ParetoRAG demonstrates promising results in improving retrieval-augmented generation, it is important to acknowledge several potential limitations that could be addressed in future work. First, the sentence-level decomposition and re-weighting approach may weaken the complex cross-sentence logic or narrative connections within paragraphs, especially in tasks requiring multi-step reasoning or long-range semantic coherence (such as story generation or scientific argumentation). The local focus on key information might lead to a loose overall structure, which could impact the quality of the generated content. Second, when dealing with longer documents, ParetoRAG faces challenges related to segmenting the text and optimizing it at the sentence level. Breaking down long texts into sentences for individual optimization might not effectively preserve the global structure and logical flow of the document. Lastly, while ParetoRAG has been tested on open-domain QA datasets, it has not yet been applied to more specialized domains, such as law or medicine, which could be explored in future work.