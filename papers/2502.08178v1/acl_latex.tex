% This must be in the first 5 lines to tell arXiv to use pdfLaTeX, which is strongly recommended.
\pdfoutput=1
% In particular, the hyperref package requires pdfLaTeX in order to break URLs across lines.

\documentclass[11pt]{article}

% Change "review" to "final" to generate the final (sometimes called camera-ready) version.
% Change to "preprint" to generate a non-anonymous version with page numbers.
\usepackage[preprint]{acl}
% \usepackage[final]{acl}

% Standard package includes
\usepackage{times}
\usepackage{latexsym}

% For proper rendering and hyphenation of words containing Latin characters (including in bib files)
\usepackage[T1]{fontenc}
% For Vietnamese characters
% \usepackage[T5]{fontenc}
% See https://www.latex-project.org/help/documentation/encguide.pdf for other character sets

% This assumes your files are encoded as UTF8
\usepackage[utf8]{inputenc}

% This is not strictly necessary, and may be commented out,
% but it will improve the layout of the manuscript,
% and will typically save some space.
\usepackage{microtype}

% This is also not strictly necessary, and may be commented out.
% However, it will improve the aesthetics of text in
% the typewriter font.
\usepackage{inconsolata}

%Including images in your LaTeX document requires adding
%additional package(s)
\usepackage{graphicx}

% If the title and author information does not fit in the area allocated, uncomment the following
%
%\setlength\titlebox{<dim>}
%
% and set <dim> to something 5cm or larger.

\title{ParetoRAG: Leveraging Sentence-Context Attention for \\ Robust and Efficient Retrieval-Augmented Generation}

% Author information can be set in various styles:
% For several authors from the same institution:
% \author{Author 1 \and ... \and Author n \\
%         Address line \\ ... \\ Address line}
% if the names do not fit well on one line use
%         Author 1 \\ {\bf Author 2} \\ ... \\ {\bf Author n} \\
% For authors from different institutions:
% \author{Author 1 \\ Address line \\  ... \\ Address line
%         \And  ... \And
%         Author n \\ Address line \\ ... \\ Address line}
% To start a separate ``row'' of authors use \AND, as in
% \author{Author 1 \\ Address line \\  ... \\ Address line
%         \AND
%         Author 2 \\ Address line \\ ... \\ Address line \And
%         Author 3 \\ Address line \\ ... \\ Address line}

% \author{First Author \\
%   Affiliation / Address line 1 \\
%   Affiliation / Address line 2 \\
%   Affiliation / Address line 3 \\
%   \texttt{email@domain} \\\And
%   Second Author \\
%   Affiliation / Address line 1 \\
%   Affiliation / Address line 2 \\
%   Affiliation / Address line 3 \\
%   \texttt{email@domain} \\}

%\author{
%  \textbf{First Author\textsuperscript{1}},
%  \textbf{Second Author\textsuperscript{1,2}},
%  \textbf{Third T. Author\textsuperscript{1}},
%  \textbf{Fourth Author\textsuperscript{1}},
%\\
%  \textbf{Fifth Author\textsuperscript{1,2}},
%  \textbf{Sixth Author\textsuperscript{1}},
%  \textbf{Seventh Author\textsuperscript{1}},
%  \textbf{Eighth Author \textsuperscript{1,2,3,4}},
%\\
%  \textbf{Ninth Author\textsuperscript{1}},
%  \textbf{Tenth Author\textsuperscript{1}},
%  \textbf{Eleventh E. Author\textsuperscript{1,2,3,4,5}},
%  \textbf{Twelfth Author\textsuperscript{1}},
%\\
%  \textbf{Thirteenth Author\textsuperscript{3}},
%  \textbf{Fourteenth F. Author\textsuperscript{2,4}},
%  \textbf{Fifteenth Author\textsuperscript{1}},
%  \textbf{Sixteenth Author\textsuperscript{1}},
%\\
%  \textbf{Seventeenth S. Author\textsuperscript{4,5}},
%  \textbf{Eighteenth Author\textsuperscript{3,4}},
%  \textbf{Nineteenth N. Author\textsuperscript{2,5}},
%  \textbf{Twentieth Author\textsuperscript{1}}
%\\
%\\
%  \textsuperscript{1}Affiliation 1,
%  \textsuperscript{2}Affiliation 2,
%  \textsuperscript{3}Affiliation 3,
%  \textsuperscript{4}Affiliation 4,
%  \textsuperscript{5}Affiliation 5
%\\
%  \small{
%    \textbf{Correspondence:} \href{mailto:email@domain}{email@domain}
%  }
%}

\author{
  \textbf{Ruobing Yao\textsuperscript{1,2}},
  \textbf{Yifei Zhang\textsuperscript{3}\thanks{Corresponding author: zhanyi.zyf@alibaba-inc.com}},
  \textbf{Shuang Song\textsuperscript{1,2}},
  \textbf{Yuhan Liu\textsuperscript{1,2}},
  % \textbf{Neng Gao\textsuperscript{1}\thanks{Corresponding author: gaoneng@iie.ac.cn}},
  \textbf{Neng Gao\textsuperscript{1}\thanks{Corresponding author: gaoneng@iie.ac.cn}},
  \textbf{Chenyang Tu\textsuperscript{1}}
  \\
  \textsuperscript{1}Institute of Information Engineering, Chinese Academy of Sciences, Beijing, China,
  \\
  \textsuperscript{2}School of Cybersecurity, University of Chinese Academy of Sciences, Beijing, China
  \\
  \textsuperscript{3}Alibaba Group, Beijing, China,
  % \\
  % \small{
  %   \textbf{Correspondence:} \href{mailto:zhanyi.zyf@alibaba-inc.com}{zhanyi.zyf@alibaba-inc.com}, 
  %   \href{mailto:gaoneng@iie.ac.cn}{gaoneng@iie.ac.cn}
  % % }
  %   }
}
% Recommended, but optional, packages for figures and better typesetting:
\usepackage{microtype}
\usepackage{graphicx}
% \usepackage{subfigure}
\usepackage{booktabs} % for professional tables
\usepackage{svg}
\usepackage{graphicx}   % 插入图片
\usepackage{subcaption} % 创建子图
\usepackage{placeins} % 控制图的位置在某一节下
\usepackage{float}
% hyperref makes hyperlinks in the resulting PDF.
% If your build breaks (sometimes temporarily if a hyperlink spans a page)
% please comment out the following usepackage line and replace
% \usepackage{icml2025} with \usepackage[nohyperref]{icml2025} above.
% \usepackage{hyperref}

% % make table better
\usepackage[utf8]{inputenc}
\usepackage[titletoc,title]{appendix}
\usepackage[tikz]{mdframed}
\usepackage{multirow}
% \usepackage[a4paper,margin=1in]{geometry}
% \usepackage{appendix}
% \usepackage[a4paper,margin=1in]{geometry}
\usepackage[normalem]{ulem}
\useunder{\uline}{\ul}{}


% Attempt to make hyperref and algorithmic work together better:
% \newcommand{\theHalgorithm}{\arabic{algorithm}}

% Use the following line for the initial blind version submitted for review:
% \usepackage{icml2025}

% If accepted, instead use the following line for the camera-ready submission:
% \usepackage[accepted]{icml2025}

% For theorems and such
\usepackage{amsmath}
\usepackage{amssymb}
\usepackage{mathtools}
\usepackage{amsthm}
\usepackage{graphicx}
\usepackage{adjustbox}


\usepackage{enumitem}
\usepackage{url}



% if you use cleveref..
% \usepackage[capitalize,noabbrev]{cleveref}

%%%%%%%%%%%%%%%%%%%%%%%%%%%%%%%%
% THEOREMS
%%%%%%%%%%%%%%%%%%%%%%%%%%%%%%%%
% \theoremstyle{plain}
% \newtheorem{theorem}{Theorem}[section]
% \newtheorem{proposition}[theorem]{Proposition}
% \newtheorem{lemma}[theorem]{Lemma}
% \newtheorem{corollary}[theorem]{Corollary}
% \theoremstyle{definition}
% \newtheorem{definition}[theorem]{Definition}
% \newtheorem{assumption}[theorem]{Assumption}
% \theoremstyle{remark}
% \newtheorem{remark}[theorem]{Remark}

% % Todonotes is useful during development; simply uncomment the next line
% %    and comment out the line below the next line to turn off comments
% %\usepackage[disable,textsize=tiny]{todonotes}
% \usepackage[textsize=tiny]{todonotes}


% The \icmltitle you define below is probably too long as a header.
% Therefore, a short form for the running title is supplied here:
% \icmltitlerunning{Submission and Formatting Instructions for ICML 2025}


\usepackage{colortbl}
\usepackage{xcolor}

% 提升值到灰色深度的映射规则
\newcommand{\graybackground}[1]{%
    \cellcolor{gray!#1}%
}
\newcommand{\graytext}[1]{\colorbox{gray!30}{#1}}
\newcommand{\redtext}[1]{\colorbox{red!30}{#1}}
\newcommand{\bluetext}[1]{\colorbox{blue!30}{#1}}

% \newcommand{\graytext}[1]{\colorbox{gray!30}{#1}}
% \newcommand{\graytext}[1]{\colorbox{gray!30}{#1}}
% \newcommand{\graybackground}[1]{%
%     \cellcolor{gray!#1*2}%
% }

% \newcommand{\poscell}[2]{\cellcolor[RGB]{0, 255, 0!#1} #2}
% \newcommand{\negcell}[2]{\cellcolor[RGB]{255, 0, 0!#1} #2}
\newcommand{\poscell}[2]{\cellcolor{blue!#1}{#2}} % 正提升为绿色,颜色深度与提升幅度成正比
\newcommand{\negcell}[2]{\cellcolor{red!#1}{#2}}   % 负提升为红色,颜色深度与降幅成正比


\begin{document}
\maketitle
% \begin{abstract}
% This document is a supplement to the general instructions for *ACL authors. It contains instructions for using the \LaTeX{} style files for ACL conferences.
% The document itself conforms to its own specifications, and is therefore an example of what your manuscript should look like.
% These instructions should be used both for papers submitted for review and for final versions of accepted papers.
% \end{abstract}
\begin{abstract}


The choice of representation for geographic location significantly impacts the accuracy of models for a broad range of geospatial tasks, including fine-grained species classification, population density estimation, and biome classification. Recent works like SatCLIP and GeoCLIP learn such representations by contrastively aligning geolocation with co-located images. While these methods work exceptionally well, in this paper, we posit that the current training strategies fail to fully capture the important visual features. We provide an information theoretic perspective on why the resulting embeddings from these methods discard crucial visual information that is important for many downstream tasks. To solve this problem, we propose a novel retrieval-augmented strategy called RANGE. We build our method on the intuition that the visual features of a location can be estimated by combining the visual features from multiple similar-looking locations. We evaluate our method across a wide variety of tasks. Our results show that RANGE outperforms the existing state-of-the-art models with significant margins in most tasks. We show gains of up to 13.1\% on classification tasks and 0.145 $R^2$ on regression tasks. All our code and models will be made available at: \href{https://github.com/mvrl/RANGE}{https://github.com/mvrl/RANGE}.

\end{abstract}


\section{Introduction}

Video generation has garnered significant attention owing to its transformative potential across a wide range of applications, such media content creation~\citep{polyak2024movie}, advertising~\citep{zhang2024virbo,bacher2021advert}, video games~\citep{yang2024playable,valevski2024diffusion, oasis2024}, and world model simulators~\citep{ha2018world, videoworldsimulators2024, agarwal2025cosmos}. Benefiting from advanced generative algorithms~\citep{goodfellow2014generative, ho2020denoising, liu2023flow, lipman2023flow}, scalable model architectures~\citep{vaswani2017attention, peebles2023scalable}, vast amounts of internet-sourced data~\citep{chen2024panda, nan2024openvid, ju2024miradata}, and ongoing expansion of computing capabilities~\citep{nvidia2022h100, nvidia2023dgxgh200, nvidia2024h200nvl}, remarkable advancements have been achieved in the field of video generation~\citep{ho2022video, ho2022imagen, singer2023makeavideo, blattmann2023align, videoworldsimulators2024, kuaishou2024klingai, yang2024cogvideox, jin2024pyramidal, polyak2024movie, kong2024hunyuanvideo, ji2024prompt}.


In this work, we present \textbf{\ours}, a family of rectified flow~\citep{lipman2023flow, liu2023flow} transformer models designed for joint image and video generation, establishing a pathway toward industry-grade performance. This report centers on four key components: data curation, model architecture design, flow formulation, and training infrastructure optimization—each rigorously refined to meet the demands of high-quality, large-scale video generation.


\begin{figure}[ht]
    \centering
    \begin{subfigure}[b]{0.82\linewidth}
        \centering
        \includegraphics[width=\linewidth]{figures/t2i_1024.pdf}
        \caption{Text-to-Image Samples}\label{fig:main-demo-t2i}
    \end{subfigure}
    \vfill
    \begin{subfigure}[b]{0.82\linewidth}
        \centering
        \includegraphics[width=\linewidth]{figures/t2v_samples.pdf}
        \caption{Text-to-Video Samples}\label{fig:main-demo-t2v}
    \end{subfigure}
\caption{\textbf{Generated samples from \ours.} Key components are highlighted in \textcolor{red}{\textbf{RED}}.}\label{fig:main-demo}
\end{figure}


First, we present a comprehensive data processing pipeline designed to construct large-scale, high-quality image and video-text datasets. The pipeline integrates multiple advanced techniques, including video and image filtering based on aesthetic scores, OCR-driven content analysis, and subjective evaluations, to ensure exceptional visual and contextual quality. Furthermore, we employ multimodal large language models~(MLLMs)~\citep{yuan2025tarsier2} to generate dense and contextually aligned captions, which are subsequently refined using an additional large language model~(LLM)~\citep{yang2024qwen2} to enhance their accuracy, fluency, and descriptive richness. As a result, we have curated a robust training dataset comprising approximately 36M video-text pairs and 160M image-text pairs, which are proven sufficient for training industry-level generative models.

Secondly, we take a pioneering step by applying rectified flow formulation~\citep{lipman2023flow} for joint image and video generation, implemented through the \ours model family, which comprises Transformer architectures with 2B and 8B parameters. At its core, the \ours framework employs a 3D joint image-video variational autoencoder (VAE) to compress image and video inputs into a shared latent space, facilitating unified representation. This shared latent space is coupled with a full-attention~\citep{vaswani2017attention} mechanism, enabling seamless joint training of image and video. This architecture delivers high-quality, coherent outputs across both images and videos, establishing a unified framework for visual generation tasks.


Furthermore, to support the training of \ours at scale, we have developed a robust infrastructure tailored for large-scale model training. Our approach incorporates advanced parallelism strategies~\citep{jacobs2023deepspeed, pytorch_fsdp} to manage memory efficiently during long-context training. Additionally, we employ ByteCheckpoint~\citep{wan2024bytecheckpoint} for high-performance checkpointing and integrate fault-tolerant mechanisms from MegaScale~\citep{jiang2024megascale} to ensure stability and scalability across large GPU clusters. These optimizations enable \ours to handle the computational and data challenges of generative modeling with exceptional efficiency and reliability.


We evaluate \ours on both text-to-image and text-to-video benchmarks to highlight its competitive advantages. For text-to-image generation, \ours-T2I demonstrates strong performance across multiple benchmarks, including T2I-CompBench~\citep{huang2023t2i-compbench}, GenEval~\citep{ghosh2024geneval}, and DPG-Bench~\citep{hu2024ella_dbgbench}, excelling in both visual quality and text-image alignment. In text-to-video benchmarks, \ours-T2V achieves state-of-the-art performance on the UCF-101~\citep{ucf101} zero-shot generation task. Additionally, \ours-T2V attains an impressive score of \textbf{84.85} on VBench~\citep{huang2024vbench}, securing the top position on the leaderboard (as of 2025-01-25) and surpassing several leading commercial text-to-video models. Qualitative results, illustrated in \Cref{fig:main-demo}, further demonstrate the superior quality of the generated media samples. These findings underscore \ours's effectiveness in multi-modal generation and its potential as a high-performing solution for both research and commercial applications.
\section{Related Work}
\label{sec:related}



Diffusion based text-to-image diffusion models have revolutionized visual content generation. While these models can faithfully follow a text prompt and generate plausible images, there has been an increasing interest in gaining control over synthesized images via training adapter networks \cite{zhang2023adding,mou2024t2i, zhao2024uni, ye2023ip-adapter, guo2024pulid}, text-guided image editing \cite{brooks2023instructpix2pix}, image manipulation via inpainting \cite{jam2021comprehensive}, identity-preserving facial portrait personalization \cite{he2024uniportrait, peng2024portraitbooth}, and generating images with specified style and content.

\begin{figure*}[t]
    \centering
    \includegraphics[width=0.75\linewidth]{figures/subzero_inference.jpg}
    %\vspace{- 1.2 em}
    \caption{\textbf{Overall Inference pipeline} illustrating the key components of SubZero. Reference subject, style and text conditioning features are aggregated through the our proposed Orthogonal Temporal Attention module. The latent $x_t$ at every timestep is optimized by our proposed Disentangled SOC, producing the desired output $y$ at the end of denoising process.}
    \label{fig:inference_pipe}
    \vspace{- 0.5 em}
\end{figure*}



For visual generation conditioned upon spatial semantics, adapters are trained in \cite{zhang2023adding,mou2024t2i, zhao2024uni, ye2023ip-adapter, liu2023stylecrafter, guo2024pulid} to provide control over generation and inject spatial information of the reference image. ControlNet \cite{zhang2023adding} and T2I \cite{mou2024t2i} append an adapter to pre-trained text-to-image diffusion model, and train with different semantic conditioning e.g., canny edge, depth-map, and human pose. Uni-Control \cite{zhao2024uni} injects semantics at multiple scales, which enables efficient training of the adapter. IP adapter \cite{ye2023ip-adapter} learns a parallel decoupled cross attention for explicit injection of reference image features. Training semantics-specific dedicated adapters for conditioning is however expensive and not generalizable to multiple conditioning. 

Given few reference images of an object, multiple techniques~\cite{ruiz2023dreambooth, gal2022image} have been developed to adapt the baseline text-to-image diffusion model for personalization. 
Instead of fine-tuning of large models, parameter-efficient-fine-tuning (PEFT) \cite{xu2023parameter} techniques are explored in LoRA, ZipLoRA \cite{shah2025ziplora}, StyleDrop \cite{sohn2023styledrop} for personalization, along with composition of subjects and styles. 
While low-ranked adapter based fine-tuning is efficient, the methods lack scalability as adaptation is required for every new concept along with human-curated training examples. Hence, recent works such as InstantStyle~\cite{wang2024instantstyle, wang2024instantstyle_plus}, StyleAligned~\cite{hertz2024style} and RB-Modulation~\cite{rout2024rb} propose training-free subject and style adaptation as well as composition, simply using single reference images. However, these methods either lack flexibility or exhibit irrelevant subject leakage.

Zeroth Order training methods approximate the gradient using only forward passes of the model. Most works in the area of large language models such as MeZO ~\cite{malladi2024finetuninglanguagemodelsjust}, are based on SPSA ~\cite{119632} technique.
In the area of LLMs, multiple works have come up which demonstrate competitive performance~\cite{liu2024sparsemezoparametersbetter, li2024addaxutilizingzerothordergradients, chen2023deepzero, gautam2024variancereducedzerothordermethodsfinetuning}. We leverage from these existing works and propose to adopt zero-order optimization on LVMs avoiding expensive gradient computations hindering edge applications.
%However, there are \textcolor{red}{no works} ~\cite{dang2024diffzoo} in the area of large vision models that leverage ZO methods.%, that we are aware of.

\section{Study Design}
% robot: aliengo 
% We used the Unitree AlienGo quadruped robot. 
% See Appendix 1 in AlienGo Software Guide PDF
% Weight = 25kg, size (L,W,H) = (0.55, 0.35, 06) m when standing, (0.55, 0.35, 0.31) m when walking
% Handle is 0.4 m or 0.5 m. I'll need to check it to see which type it is.
We gathered input from primary stakeholders of the robot dog guide, divided into three subgroups: BVI individuals who have owned a dog guide, BVI individuals who were not dog guide owners, and sighted individuals with generally low degrees of familiarity with dog guides. While the main focus of this study was on the BVI participants, we elected to include survey responses from sighted participants given the importance of social acceptance of the robot by the general public, which could reflect upon the BVI users themselves and affect their interactions with the general population \cite{kayukawa2022perceive}. 

The need-finding processes consisted of two stages. During Stage 1, we conducted in-depth interviews with BVI participants, querying their experiences in using conventional assistive technologies and dog guides. During Stage 2, a large-scale survey was distributed to both BVI and sighted participants. 

This study was approved by the University’s Institutional Review Board (IRB), and all processes were conducted after obtaining the participants' consent.

\subsection{Stage 1: Interviews}
We recruited nine BVI participants (\textbf{Table}~\ref{tab:bvi-info}) for in-depth interviews, which lasted 45-90 minutes for current or former dog guide owners (DO) and 30-60 minutes for participants without dog guides (NDO). Group DO consisted of five participants, while Group NDO consisted of four participants.
% The interview participants were divided into two groups. Group DO (Dog guide Owner) consisted of five participants who were current or former dog guide owners and Group NDO (Non Dog guide Owner) consisted of three participants who were not dog guide owners. 
All participants were familiar with using white canes as a mobility aid. 

We recruited participants in both groups, DO and NDO, to gather data from those with substantial experience with dog guides, offering potentially more practical insights, and from those without prior experience, providing a perspective that may be less constrained and more open to novel approaches. 

We asked about the participants' overall impressions of a robot dog guide, expectations regarding its potential benefits and challenges compared to a conventional dog guide, their desired methods of giving commands and communicating with the robot dog guide, essential functionalities that the robot dog guide should offer, and their preferences for various aspects of the robot dog guide's form factors. 
For Group DO, we also included questions that asked about the participants' experiences with conventional dog guides. 

% We obtained permission to record the conversations for our records while simultaneously taking notes during the interviews. The interviews lasted 30-60 minutes for NDO participants and 45-90 minutes for DO participants. 

\subsection{Stage 2: Large-Scale Surveys} 
After gathering sufficient initial results from the interviews, we created an online survey for distributing to a larger pool of participants. The survey platform used was Qualtrics. 

\subsubsection{Survey Participants}
The survey had 100 participants divided into two primary groups. Group BVI consisted of 42 blind or visually impaired participants, and Group ST consisted of 58 sighted participants. \textbf{Table}~\ref{tab:survey-demographics} shows the demographic information of the survey participants. 

\subsubsection{Question Differentiation} 
Based on their responses to initial qualifying questions, survey participants were sorted into three subgroups: DO, NDO, and ST. Each participant was assigned one of three different versions of the survey. The surveys for BVI participants mirrored the interview categories (overall impressions, communication methods, functionalities, and form factors), but with a more quantitative approach rather than the open-ended questions used in interviews. The DO version included additional questions pertaining to their prior experience with dog guides. The ST version revolved around the participants' prior interactions with and feelings toward dog guides and dogs in general, their thoughts on a robot dog guide, and broad opinions on the aesthetic component of the robot's design. 

\section{Experiment Setups}

In this section, we describe the experimental setup for evaluating ParetoRAG across various scenarios. The specific model parameters can be found in Appendix \ref{appendix: Statistics of models}. The selection and meaning of the evaluation metrics can be found in Appendix \ref{appendix: Evaluation Metrics}.

% 这段也是照抄 记得降重
\subsection{Datasets.} 
%  介绍了 NQ, HotpotQA, Msmarco,  每个数据集有多少数据, 然后我随即采样了1000条进行测试 



We experiment on three different open-domain QA datasets as the retrieval source: Natural Questions (NQ) \cite{kwiatkowskiNaturalQuestionsBenchmark2019}, HotpotQA \cite{yangHotpotQADatasetDiverse2018}, and MS-MARCO \cite{nguyen2016ms}, where each dataset has a knowledge database. 
% The knowledge databases of NQ and HotpotQA are collected from Wikipedia. The knowledge database of MS-MARCO is collected from web documents using the MicroSoft Bing search engine.
These datasets encompass different tasks, such as open-domain question answering, multi-hop reasoning, and long-form answer generation. 
Each dataset also contains a set of questions. We randomly sampled 1,000 data paris for testing. Table \ref{table:statistic-dataset} shows statistics of text unit counts before and after ParetoRAG encoding.



% \begin{table*}[ht!]
% \resizebox{\textwidth}{!}{ % 调整宽度为页面宽度,高度自动缩放
% \begin{tabular}{lccccccccccccccc}
% \hline
% \multicolumn{1}{c}{\multirow{2}{*}{Model}} & \multicolumn{4}{c}{NQ(acc)} & \multicolumn{4}{c}{Hotpot(acc)} & \multicolumn{4}{c}{MS(mauve)} & \multicolumn{3}{c}{MS(rouge)} \\
% \multicolumn{1}{c}{} & \# tok & Contriever & ANCE & \multicolumn{1}{c}{DPR} & \# tok & Contriever & ANCE & \multicolumn{1}{c}{DPR} & \# tok & Contriever & ANCE & \multicolumn{1}{c}{DPR} & Contriever & ANCE & DPR \\ \hline
% \multicolumn{16}{c}{Without RAG} \\ \hline
% Vicuna-7B & \multirow{4}{*}{-} & \multicolumn{3}{c}{23.2} & \multirow{4}{*}{-} & \multicolumn{3}{c}{16.1} & \multirow{4}{*}{-} & \multicolumn{3}{c}{88.3} & \multicolumn{3}{c}{44.9} \\
% Vicuna-13B &  & \multicolumn{3}{c}{28.2} &  & \multicolumn{3}{c}{20.2} &  & \multicolumn{3}{c}{82.1} & \multicolumn{3}{c}{47.6} \\
% Llama2-7B-Chat &  & \multicolumn{3}{c}{20.9} &  & \multicolumn{3}{c}{16.0} &  & \multicolumn{3}{c}{85.6} & \multicolumn{3}{c}{50.0} \\
% Llama2-13B-Chat &  & \multicolumn{3}{c}{29.9} &  & \multicolumn{3}{c}{18.4} &  & \multicolumn{3}{c}{90.1} & \multicolumn{3}{c}{49.4} \\ \hline
% \multicolumn{16}{c}{Naive RAG} \\ \hline
% Vicuna-7B & \multirow{4}{*}{895} & 33.2 & 36.1 & 41.9 & \multirow{4}{*}{757} & 25.0 & 22.3 & 23.9 & \multirow{4}{*}{576} & 84.1 & 84.9 & 87.9 & 46.5 & 45.9 & 46.7 \\
% Vicuna-13B &  & 37.4 & 41.0 & 45.6 &  & 22.6 & 20.2 & 22.7 &  & 86.8 & 87.7 & 87.0 & 47.8 & 46.8 & 46.9 \\
% Llama2-7B-Chat &  & 33.2 & 37.9 & 40.8 &  & 23.6 & 23.4 & 23.3 &  & 85.0 & 88.6 & 89.2 & 48.5 & 48.0 & 48.3 \\
% Llama2-13B-Chat &  & 38.3 & 39.6 & 42.7 &  & 27.1 & 25.3 & 26.8 &  & 77.5 & 87.0 & 88.8 & 47.8 & 46.1 & 46.0 \\ \hline
% \multicolumn{16}{c}{ParetoRAG} \\ \hline
% Vicuna-7B & \multirow{4}{*}{232} & {\graybackground{20} 36.6} & {\graybackground{40} 43.4} & {\graybackground{22} 46.7} & \multirow{4}{*}{229} & {\graybackground{4} 25.4} & {\graybackground{26} 25.3} & {\graybackground{6} 24.7} & \multirow{4}{*}{183} & {\graybackground{10} 88.1} & {\graybackground{6} 87.4} & {\graybackground{8} 91.5} & {\graybackground{10} 48.6} & {\graybackground{30} 52.9} & {\graybackground{14} 50.0} \\
% Vicuna-13B &  & {\graybackground{10} 39.1} & {\graybackground{16} 44.2} & {\graybackground{12} 48.2} &  & {\graybackground{36} 26.7} & {\graybackground{56} 25.9} & {\graybackground{30} 26.0} &  & {\graybackground{0} 84.3} & {\graybackground{2} 88.8} & {\graybackground{0} 86.7} & {\graybackground{8} 49.9} & {\graybackground{34} 55.2} & {\graybackground{18} 51.5} \\
% Llama2-7B-Chat &  & {\graybackground{4} 34.0} & {\graybackground{20} 41.7} & {\graybackground{8} 42.3} &  & {\graybackground{8} 24.6} & {\graybackground{12} 25.0} & {\graybackground{6} 24.0} &  & {\graybackground{16} 92.0} & {\graybackground{2} 89.9} & {\graybackground{6} 92.6} & {\graybackground{6} 50.0} & {\graybackground{22} 53.2} & {\graybackground{12} 51.6} \\
% Llama2-13B-Chat &  & {\graybackground{0} 36.1} & {\graybackground{12} 41.8} & {\graybackground{22} 47.4} &  & {\graybackground{8} 25.9} & {\graybackground{6} 26.1} & {\graybackground{6} 25.2} &  & {\graybackground{36} 92.2} & {\graybackground{6} 90.1} & {\graybackground{8} 92.0} & {\graybackground{12} 50.9} & {\graybackground{38} 55.1} & {\graybackground{26} 52.1} \\ \hline
% \end{tabular}
% }
% \caption{Overall experiment results of three retrievers on three tasks. The darker the \graytext{gray background}, the higher the relative improvement compared to Naive RAG. }
% \label{table: main results}
% \end{table*}





\subsection{Dense Retrieval Models}
%  总起句: 介绍这三种 retriever models, 然后说明了默认用的是 dot
We compare the performance of the three following unsupervised, semi-supervised,  or supervised dense retriever models. Following previous studies~\cite{lewisRetrievalaugmentedGenerationKnowledgeintensive2020a}, by default, we use the dot product between the embedding vectors of a question and a text in the knowledge database to calculate their similarity score. 

\textbf{Contriever} \cite{izacardUnsupervisedDenseInformation2021} is an unsupervised retriever implemented using a BERT-base encoder. Contriever is contrastively trained on segment pairs constructed from unlabeled documents in Wikipedia and web crawl data.

\textbf{ANCE} \cite{xiongApproximateNearestNeighbor2020} is a dual-encoder BERT-base model designed for dense retrieval tasks. It is trained using weakly supervised signals from query-document pair labels, typically sourced from datasets such as MS-MARCO.

\textbf{DPR} \cite{karpukhinDensePassageRetrieval2020} is a dual-encoder  BERT-base model fine-tuned on passage retrieval tasks directly using the question-passage pair labels from NQ, TQA \cite{joshiTriviaQALargeScale2017}, SQuAD \cite{rajpurkarSQuAD100000Questions2016} and WebQ \cite{berantSemanticParsingFreebase2013}.

\begin{figure*}[ht!]
    \centering
    % 左侧子图 (a) 和 (b)
    \begin{subfigure}[b]{0.45\textwidth} % 左侧子图占一半宽度
        \centering
        % 子图 (a)
        \begin{subfigure}[b]{0.48\textwidth} % 每个子图占一半宽度
            \centering
            \includegraphics[width=\linewidth]{figure/ParetoRAGforRobustLLM-a.pdf} % 替换为图 (a) 的路径
            \caption{NQ (Top 10)}
            \label{fig:PareroRAGforRobustLLM-a}
        \end{subfigure}
        % 子图 (b)
        \begin{subfigure}[b]{0.48\textwidth}
            \centering
            \includegraphics[width=\linewidth]{figure/ParetoRAGforRobustLLM-b.pdf} % 替换为图 (b) 的路径
            \caption{HotpotQA (Top 10)}
            \label{fig:PareroRAGforRobustLLM-b}
        \end{subfigure}
    \end{subfigure}
    % 右侧子图 (c) 和 (d)
    \begin{subfigure}[b]{0.45\textwidth} % 右侧子图占一半宽度
        \centering
        % 子图 (c)
        \begin{subfigure}[b]{0.48\textwidth}
            \centering
            \includegraphics[width=\linewidth]{figure/ParetoRAGforRobustLLM-c.pdf} % 替换为图 (c) 的路径
            \caption{NQ (400 words)}
            \label{fig:PareroRAGforRobustLLM-c}
        \end{subfigure}
        % 子图 (d)
        \begin{subfigure}[b]{0.48\textwidth}
            \centering
            \includegraphics[width=\linewidth]{figure/ParetoRAGforRobustLLM-d.pdf} % 替换为图 (d) 的路径
            \caption{HotpotQA (400 words)}
            \label{fig:PareroRAGforRobustLLM-d}
        \end{subfigure}
    \end{subfigure}
\caption{Comparison of ParetoRAG and Naive RAG on the adaptive noise-robust LLM (\texttt{llama-2-13b-peft-nq-retrobust} and \texttt{llama}-\texttt{2-13b-peft-hotpotqa-retrobust} \cite{yoran2024making}): (a)(b) show performance under the same recall size (Top 10), while (c)(d) illustrate performance under the same input word count (400).}
\label{fig:ParetoRAGforRobustLLM}
\end{figure*}


\subsection{Baselines}
% 在这三个基线(baseline)上,我们分别测试了多种公开可用的指令微调模型(如 Vicuna-7B、Vicuna-13B),以及使用私有数据进行训练和强化的模型(如 Llama2-7B-chat 和 Llama2-13B-chat)。

For these three baselines, we evaluated publicly available instruction-tuned models, such as Vicuna-7B and Vicuna-13B \cite{zhengJudgingLLMasaJudgeMTBench2023} , as well as models trained and reinforced with private data, including Llama2-7B-Chat and Llama2-13B-Chat \cite{touvronLlama2Open2023}. 

%  我们对多种未采用RAG技术的LLMs在多种数据集上的性能进行了评估。
\textbf{Baselines without retrievals. }We evaluate the performance of various LLMs without employing RAG technology across multiple datasets.

% Standard RAG 指的是仅使用最基础的RAG技术,未采用复杂的检索优化方法(如嵌入检索、向量检索)或高级生成机制,仅使用基础的检索-生成流程。
\textbf{Baselines with Naive RAG.} We employ the most basic RAG technique, without incorporating complex retrieval optimization methods or advanced generation mechanisms, relying solely on the fundamental retrieval-generation workflow.

%  SOTA 方法的基线。我们比较了两种先进的方法:(1) RECOMP(Xu et al., 2024) 仅使用生成式摘要模型(排除抽取式变体)对检索段落进行语义重构,通过专用模型生成精炼内容;以及(2)使用对抗性噪声训练的 Retrobust LLM 以提高鲁棒性。
\textbf{Baselines with SOTA methods.} We implement two advanced approaches: (1) Recomp \cite{xu2024recomp}  using abstractive summarization (excluding extractive variants) to synthesize retrieved passages with dedicated models. (2) LLM trained with adversarial noise to improve robustness \cite{yoran2024making}. 







% “为了在句子级别和段落级别方法之间实现公平比较,我们在句子级别检索中对输入的总token数进行控制,使其与段落级别检索的输入token数近似相当。由于对文档进行了句子细粒度的划分,导致文档规模发生了显著变化,例如,在NQ数据集中,文档行数从2681468行增加到了9320506行。这样的变化使得设置相同的top k已不适用于当前的召回场景。通过限制输入的总token数,我们能够在句子级别和段落级别方法之间实现更公平的对比。尽管句子和段落级别方法在信息表达的形式和上下文完整性上可能存在差异,这种设计在实验设置中保证了两种方法的输入规模一致性,避免了因输入长度差异引入的性能偏倚。此外,这也确保了输入给LLM的内容在长度和处理预算上的一致性,从而更便于对两种方法在下游QA任务中的表现进行对比和分析。”

% \textbf{Fairness in Comparsion.} In order to facilitate a fair comparison between [xxx] sentence-level and paragraph-level approaches, we controll the total number of input tokens in sentence-level retrieval to approximate that of the paragraph-level retrieval. Given the granular division of documents into sentences, there has been a significant change in the scale of documents. For instance, in the NQ, the number of document lines increased from 2,681,468 to 9,320,506. Such changes render the previous setting of the same top-k inapplicable for current retrieval scenarios. 
% \par
% \textit{By limiting the total number of input tokens}, we are able to achieve a more equitable comparison between [xxx] sentence-level and paragraph-level methods. Although there may be differences in the form of information expression and the integrity of context between sentence- and paragraph-level methods, this design ensures consistency in input scale between the two methods in experimental setups, thus avoiding performance bias introduced by differences in input length. Furthermore, this also ensures consistency in the length and processing budget of content fed to LLMs, thereby facilitating comparative analysis of the performance of the two methods in downstream QA tasks. 
% % 为保证输入规模的一致性,当Naive RAG召回前10篇文档时,Sentence RAG应召回前30篇文档。具体的计算和分析过程详见附录(Appendix)。
% \par
% To ensure consistency in input size, ParetoRAG should retrieve the top 30 documents when naive-RAG retrieves the top 10 documents. The detailed calculation process is provided in Appendix \ref{Appendix: Input Size Consistency}

% \textbf{Hyperparameter setting.}
\section{Experimental Results and Analysis}
In this section, we show the overall experimental results with in-depth analyses of our framework.

\subsection{Main Results}

Table \ref{table: main results} presents the results of three retrievers on three datastes, based on top-30 recall contents. Figure \ref{fig:ParetoRAGforRobustLLM} illustrates the performance of ParetoRAG on \path{llama2-13b-retrobust}. From these results, we can conclude the following findings:

% \textbf{ParetoRAG 在仅消耗原先约 30% 的 token 成本的情况下,依然实现了显著的准确率和流畅度提升。} 如表 \ref{table: main results} 所示,例如在 NQ 数据集上,Vicuna-7B + ANCE 的准确率从 36.1% 提升至 43.4%(+7.3%)。同样地,Llama2-13B-Chat + DPR 的准确率从 42.7% 提升至 47.4%(+4.7%),同时 token 数从 895 减少至 232(约为 26%)。在 HotpotQA 数据集上,Vicuna-13B + ANCE 的准确率从 20.2% 提升至 25.9%(+5.7%),而 token 数从 757 减少至 229(约为 30%)。
% 此外,在 MS-Marco 数据集上,ParetoRAG 在mauve(流畅度)和rouge(正确性)指标上也实现了可观提升。例如,基于 Llama2-13B-Chat + Contriever 的流畅度分数从 77.5 提升至 92.2(+14.7%), 同时 token 数减少至约为原先的 32%。;基于 Vicuna-7B + DPR的流畅度分数从 87.9 提升至 91.5(+3.6%)。在正确性(rouge)方面,Vicuna-13B + ANCE 的 rouge 分数从 46.8 提升至 55.2(+8.4%),而 Llama2-13B-Chat + ANCE 的 rouge 分数从 46.1 提升至 55.1(+9%)。
% 这些结果进一步凸显了 ParetoRAG 的能力,即使在显著减少 token 消耗的情况下,依然能够在流畅度和准确率方面实现稳定且可量化的提升。
\textbf{ParetoRAG, while consuming only about 30\% of the original token cost, still delivers  notable  improvements in accuracy and fluency.} Specifically,  as shown in Table \ref{table: main results}, in NQ, the accuracy of Vicuna-7B + ANCE increases from 36.1\% to 43.4\% (+7.3\%), with the token count reduced to 26\% of the original. Similarly, the accuracy of Llama2-13B-Chat + DPR increases from 42.7\% to 47.4\% (+4.7\%), with the token count reduced to approximately 30\% of the original. In HotpotQA, the accuracy of Vicuna-13B + ANCE improves from 20.2\% to 25.9\% (+5.7\%), with the token count reduced to approximately 30\% of the original. 

In addition, in MS-Marco, ParetoRAG achieves notable improvements in both mauve (fluency) and rouge (correctness) metrics. For example, the fluency score of Llama2-13B-Chat + Contriever increases from 77.5 to 92.2 (+14.7\%), while the token count is reduced to approximately 32\% of the original. Similarly, the fluency score of Vicuna-7B + DPR improves from 87.9 to 91.5 (+3.6\%). In terms of correctness (rouge), the rouge score of Vicuna-13B + ANCE increases from 46.8 to 55.2 (+8.4\%), while Llama2-13B-Chat + ANCE improves from 46.1 to 55.1 (+9\%). These results further highlight ParetoRAG's capability to deliver consistent and measurable improvements in both fluency and accuracy, even with significantly reduced token consumption. 

% 我的方法有很强的泛用性, 主要从三个方面来看: 1. 在多个数据集上均有效; 2. 在不同种类的召回器上均有效 3. 对多个LLM均有效.
% 在多个数据集上的有效性证明了 ParetoRAG 的鲁棒性。它在 NQ、HotpotQA 和 MS-Marco 等多样化的数据集上持续提升了性能。这些数据集涵盖了不同的任务,例如开放域问答、多跳推理以及长文本回答生成,展现了该方法的多功能性。
% 

\begin{figure*}[ht!]
    \centering
    % 左侧子图
    \begin{subfigure}[b]{0.31\textwidth} % 左右各占 48%,留一些间隙
        \centering
        \includegraphics[width=\linewidth]{figure/data-distribution/data-distribution-nq.pdf}
        \caption{NQ distribution}
        \label{fig:nq distribution}
        \vspace{-5pt} % 减少间距
    \end{subfigure}
    % 右侧子图
    \begin{subfigure}[b]{0.31\textwidth} % 左右各占 48%,留一些间隙
        \centering
       \includegraphics[width=\linewidth]{{figure/data-distribution/data-distribution-hotpotqa.pdf}}
        \caption{HotpotQA distribution}
        \label{fig:hotpotqa distribution}
        \vspace{-5pt} % 减少间距
    \end{subfigure}
    \begin{subfigure}[b]{0.31\textwidth} % 左右各占 48%,留一些间隙
        \centering
        \includegraphics[width=\linewidth]{{figure/data-distribution/data-distribution-msmarco.pdf}}
        \caption{MS-MARCO distribution}
        \label{fig:msmarco distribution}
        \vspace{-5pt} % 减少间距
    \end{subfigure}
    \caption{Correct answer rank distributions across different datasets under the the same input word count (400).}
    \label{fig:data-distribution}
\end{figure*}

\textbf{ParetoRAG  demonstrates strong generalizations.} We analyze its effectiveness from three perspectives:


\textit{Effectiveness across multiple datasets: }ParetoRAG consistently improves performance across a diverse range of datasets, including NQ, HotpotQA, and MS-Marco. 
% These datasets encompass different tasks, such as open-domain question answering, multi-hop reasoning, and long-form answer generation, showcasing the versatility of the method. 
In contrast, Recomp does not have a specialized abstract model for the MS-MARCO task, resulting in a significant drop in performance on MAUVE (fluency) and a smaller improvement on ROUGE-L (correctness) compared to ParetoRAG.

\textit{Compatibility with different types of retrievers:} ParetoRAG proves effective with various retriever types, including Contriever, ANCE, and DPR. This shows that the method is not tied to a specific retriever and adapt well to different retrieval methods. Specific analysis of the impact on retrievers can be found in \ref{sec:impact-of-retriever}.
    
\textit{Applicability across multiple LLMs:} ParetoRAG achieves improvements when applied to large language models, such as Vicuna-7B, Vicuna-13B, Llama2-7B-Chat, and Llama2-13B-Chat. Notably, we also test the method on models trained with anti-noise techniques. As shown in Figure \ref{fig:ParetoRAGforRobustLLM}, the results still show improvements. More detailed analysis can be found in \ref{sec: detailed-analysis-on-robust-llm}. 





\subsection{Ablation Study}
We study the impact of core sentence weight, retriever types, and top k size on ParetoRAG. The variation of core sentence weight on HotpotQA and MS-MARCO can be found in Appendix \ref{appendix: Impact of core sentence weight}, while the impact of model parameters on ParetoRAG is detailed in Appendix \ref{appendix: Impact of Model Size on Accuracy with Varying Top K in ParetoRAG}.

\subsubsection{Impact of core sentence weight}

 From the Figure \ref{fig:the influence of core sentence weight between different retrievers}  it can be observed that when the weight of core sentences is adjusted to approximately 0.80, the Mean Recall@30 for ANCE, DPR and Contriever methods reaches optimal performance. This phenomenon reflects the impact of weight adjustment on the balance between contextual information and core sentences, which can be analyzed as follows:
 
\textbf{Performance Improvement at Optimal Weight (Around 0.80):} When the core sentence weight is set to approximately 0.80, the model effectively integrates contextual information with the content of core sentences. This balance enables the model to preserve semantic integrity while more accurately capturing key information relevant to the retrieval task, thereby achieving optimal recall performance.

\textbf{Performance Decline with Increased Weight (Beyond 0.80):} As the core sentence weight increases further toward 1.0, contextual information in the text is progressively diminished or even neglected, causing the model to rely more heavily on core sentences for retrieval. However, excessively weakening contextual information leads to a loss of semantic completeness, which adversely affects the accuracy of retrieval results. Consequently, performance declines beyond the 0.80 threshold.

\begin{figure}[ht!]
    \centering
    \includegraphics[width=0.8\linewidth]{figure/weightRecall.pdf}
    \caption{Impact of Core Sentence Weight on Recall across NQ Dataset.}
    \label{fig:the influence of core sentence weight between different retrievers}
\end{figure}


\textbf{High Weight Still Outperforms the Baseline (At 1.0):} Even when the core sentence weight reaches 1.0, resulting in the complete disregard of contextual information, the model's performance remains superior to the baseline of paragraph-level retrieval. This indicates that paragraph-level information often contains significant redundancy, while core sentences play a pivotal role in enhancing retrieval performance. By adjusting the weighting, ParetoRAG effectively reduces the spatial burden of paragraph content while incorporating more core sentences, thereby improving retrieval precision and optimizing efficiency simultaneously.


\subsubsection{Impact of retriever} 
\label{sec:impact-of-retriever}
% 图 \ref{fig:data-distribution} 比较了在不同数据集(NQ、HotpotQA 和 MS-MARCO)上使用 ParetoRAG 和 Naive RAG 时正确答案排名分布的差异。y 轴表示正确答案在检索结果中的百分比位置,x 轴表示正确答案在检索结果中的拟合密度分布。20% 表示正确答案出现在检索结果的前 20%。较高的百分比值对应于排名较低的位置,而接近 -1 的值表示未检索到正确答案的情况。
Figure \ref{fig:data-distribution} compares the ranking distribution of correct answers across different datasets (NQ, HotpotQA, and MS-MARCO) when using ParetoRAG and Naive RAG. The y-axis represents shows the percentage position of the correct answer within the ranked retrieval results, and the x-axis shows the fitted density distribution of the correct answer positions in the retrieval results. 20\% indicates that the correct answer appears in the top 20\% of the retrieval results. Higher percentage correspond to lower positions in the ranking, and values near -1 represent cases where the correct answer is not retrieved.

% 经过 ParetoRAG 的优化后,DPR、ANCE 和 Contriever 这三个召回器表现出以下共同趋势:首先,所有召回器的正确答案排名在 Rank 1(第一名)附近形成了更高的峰值,表明 ParetoRAG 能有效将正确答案推至检索结果的更高位置。其次,在 Rank -1 附近的分布密度显著降低,说明 ParetoRAG 减少了未能检索到正确答案的情况,从而提升了召回的全面性。此外,ParetoRAG 的分布曲线(蓝色虚线)整体比 Naive RAG(红色实线)更平滑,尤其是在中高排名区域(如 Rank 2~5),这表明 ParetoRAG 能更稳定地优化召回器性能并减少错误分布。

\begin{figure}[ht!]
    \centering
    % 左侧子图
    \begin{subfigure}[b]{0.23\textwidth} % 左右各占 48%,留一些间隙
        \centering
        \includegraphics[width=\linewidth]{figure/topkInfluence-vicuna7b.pdf}
        \caption{Vicuna-7B}
        \label{fig:vicuna-7b topk}
        \vspace{-5pt} % 减少间距
    \end{subfigure}
    % 右侧子图
    \begin{subfigure}[b]{0.23\textwidth} % 左右各占 48%,留一些间隙
        \centering
        \includegraphics[width=\linewidth]{figure/topkInfluence-vicuna13b.pdf}
        \caption{Vicuna-13B}
        \label{fig:vicuna-13b topk}
        \vspace{-5pt} % 减少间距
    \end{subfigure}
    \caption{Comparison of accuracy and recall rates of different retrievers under various top k conditions.}
    \label{fig:impact of topk size}
\end{figure}


After being optimized by ParetoRAG, the three retrievers, DPR, ANCE, and Contriever, exhibit the following common trends: First, the correct answer rankings for all retrievers form a higher peak around 20\%, indicating that ParetoRAG effectively pushes correct answers to higher positions in the retrieval results. Second, the density near -1 is significantly reduced, demonstrating that ParetoRAG decreases the cases where correct answers are not retrieved, thus improving the retrieval comprehensiveness. 

Lastly, the distribution curves of ParetoRAG (blue lines) are smoother compared to Naive RAG (red lines), particularly in the mid-to-high ranking regions (e.g., 40\% to 80\%). This indicates that ParetoRAG stabilizes the performance of retrievers and reduces erroneous distributions.



% 从图中可以观察到,当核心句子权重调整至 0.75 左右时,DPR 和 Contriever 方法的 Mean Recall@30 达到了性能最优。这一现象反映了权重调整对上下文信息与核心语句之间平衡的影响,具体可以分为以下几点:
%     最优权重的性能提升(0.75 附近):
%     在核心句子权重为 0.75 时,模型能够有效融合文本中的上下文信息与核心语句的关键内容,充分利用段落级别和句子级别信息的互补特性。这种平衡使得模型能够在保留语义完整性的同时更准确地捕捉到检索任务中的关键信息,从而实现最佳召回性能。
%     权重增大后的性能下降(0.75 之后):
%     当核心句子权重进一步增大至接近 1.0 时,文本中的上下文信息逐渐被弱化甚至忽略,模型更多依赖核心语句来完成检索。然而,过度削弱上下文信息会导致语义完整性下降,使得检索结果的精准度受到影响,因此性能在 0.75 之后出现下滑。
%     高权重依然优于基准线(1.0):
%     尽管在核心句子权重为 1.0 时,模型完全丢弃了上下文信息,但其性能仍高于段落级别检索的基准线。这表明,段落级别的信息往往包含大量冗余,而核心语句在检索任务中对性能提升起到了关键作用。通过调整权重,我们的方法有效节省了段落内容空间,并引入了更多核心语句,从而在提升检索精准性的同时优化了效率。

% \begin{figure}
%     \centering
%     \includegraphics[width=0.95\linewidth]{figure/recall_compare.png}
%     \caption{the influence of core sentence weight between different retrievers}
%     \label{fig:the influence of core sentence weight between different retrievers}
% \end{figure}



\subsubsection{Impact of wider top k size}

% 由于ParetoRAG 在 Top 30 与 Naive RAG 在 Top 10 的输入规模相似,我们将 Naive RAG 的 Top 10 设置作为基准(baseline),并分别对 ParetoRAG 在 Top 10、Top 20 和 Top 30 设置下的表现进行评估,以探讨不同 Top K 设置对模型性能的影响。如图2所示,我们的主要观察如下:
Since the input size of ParetoRAG at Top 30 is similar to that of Naive RAG at Top 10 (more detailed can be seen in Appendix \ref{Appendix: Input Size Consistency}), we set the Top 10 performance of Naive RAG as the baseline. We then evaluate the performance of ParetoRAG in the top 10, top 20 and top 30 settings to investigate the impact of different top k configurations on model performance. As shown in Figure \ref{fig:impact of topk size}, our key observations are as follows:
\par
% 传统的基于段落的检索方法(Naive RAG)虽然能够实现较高的文档覆盖率,但往往包含大量无关信息,这些无关信息可能干扰语言模型(LLM)在回答问题时的准确性。而采用细粒度检索的 ParetoRAG 方法后,尽管在相同的 Top K(如 Top 10)设置下,其召回率相对较低,但语言模型的回答准确率却显著提升。这表明,ParetoRAG通过句子级别的文档召回,有效删减了大量无关内容,降低了模型的推理复杂度,从而帮助语言模型更高效地找到正确答案。
Although Naive RAG can achieve high document coverage, they often include a large amount of irrelevant information, which can interfere with the accuracy of LLM when answering questions. In contrast, with the fine-grained retrieval approach of ParetoRAG, although the recall rate is relatively lower under the same top k settings (e.g., Top 10), the accuracy of the language model's responses is significantly improved. This suggests that ParetoRAG, by performing document retrieval at the sentence level, effectively eliminates a substantial amount of irrelevant content, reducing LLM's inference complexity and enabling LLM to more efficiently identify the correct answer.
\par
% 对于较小的模型(如 Vicuna-7b),其处理大量文档的能力较弱,因此在 Top K 较大时准确率下降更快。然而,这种下降并非因 ParetoRAG 的召回质量不足,而是因为较小模型无法充分利用这些更加丰富的信息。此外,对于较大的模型(如 Vicuna-13b),其参数量和推理能力更强,能够在更大范围内处理更多的信息,因此即使 Top K 扩大到一定程度时(如 Top 20、Top 30),依然有较高的准确率。特别地,较大的 Top K 设置(如 Top 20)相比 Top 10 和 baseline 展现了更好的性能,这表明 ParetoRAG 能够提供更加丰富的信息召回,为语言模型提供更有效的上下文。
% For smaller models (such as Vicuna-7b), their ability to process a large number of documents is weaker, leading to a faster decline in accuracy as the Top K increases. However, this decline is not due to a lack of retrieval quality by ParetoRAG, but rather because smaller models are unable to fully utilize the richer information provided. On the other hand, for larger models (such as Vicuna-13b), their greater parameter size and reasoning capability enable them to handle more information within a larger scope. As a result, even when the Top K is increased to a certain extent (e.g., Top 20 or Top 30), they still maintain high accuracy. Notably, larger Top K settings (e.g., Top 20) outperform Top 10 and the baseline, demonstrating that ParetoRAG can provide richer information retrieval, offering more effective context for language models.

 \subsubsection{Complementary effect of ParetoRAG on adaptive noise-robust LLM}
 \label{sec: detailed-analysis-on-robust-llm}
%  即使对于已经经过抗噪训练、能够忽略无关或噪声上下文的模型(llama-2-13b-peft-nq-retrobust, llama-2-13b-peft-hotpotqa-retrobust),ParetoRAG 依然能够持续提升性能。例如,在 NQ 数据集中, Top 10 检索设置下,使用 ANCE 时,准确率从 44.80% 提升至 48.20%(+3.4%);在 HotpotQA 数据集中,输入文档字数为400的情况下,准确率从 23.3% 提升至 25.3%(+2%)。这些结果表明,ParetoRAG 能够在抗噪训练的基础上进一步提供性能改进。
% 尽管抗噪训练增强了模型的鲁棒性,但在处理长文本或多跳推理任务中仍可能因冗余或密集信息而表现受限。ParetoRAG 通过句子级表征减少冗余、提高信息密度,从而使模型更专注于相关内容,实现了对抗噪模型的有效补充。
% ParetoRAG 与抗噪训练之间的互补作用表明,结合这两种方法可以进一步优化检索和生成质量.未来的工作可以探索将 ParetoRAG 与其他训练技术相结合,以进一步增强其在更广泛场景中的性能。
In this section, we evaluate the impact of ParetoRAG on robustly trained models, which are fine-tuned for the NQ and HotpotQA datasets respectively. These models are trained to enhance robustness against irrelevant context. As shown in Figure \ref{fig:ParetoRAGforRobustLLM}, in NQ, under the Top 10 retrieval setting, the accuracy improved from 44.80\% to 48.20\% (+3.4\%) when using ANCE. In HotpotQA, with input words length limited to 400 words, the accuracy increases from 23.3\% to 25.3\% (+2.0\%). These results demonstrate that ParetoRAG can further enhance performance in addition to robustly trained models.

While robust training improves the model's resilience to noisy contexts, it may still struggle with redundant or dense information in tasks involving long texts or multi-hop reasoning. ParetoRAG mitigates this limitation by reducing redundancy and increasing information density through sentence-level representations, allowing the model to focus more effectively on relevant content, thereby serving as a valuable complement to robustly trained models.

The complementary effect between ParetoRAG and robust training LLM indicates that combining these two approaches can further optimize retrieval and generation quality. Future work could explore integrating ParetoRAG with other training techniques to further enhance its performance across broader scenarios.
% \begin{figure}[htbp]
%     \centering
%     % 第一张子图
%     \begin{subfigure}[b]{0.45\textwidth}
%         \centering
%         \includesvg[width=\linewidth]{figure/topkInfluence-vicuna7b.svg}
%         \caption{Description of (a).}
%         \label{fig:subfig1}
%     \end{subfigure}
%     \hfill
%     % 第二张子图
%     \begin{subfigure}[b]{0.45\textwidth}
%         \centering
%         \includesvg[width=\linewidth]{figure/topkInfluence-vicuna13b.svg}
%         \caption{Description of (b).}
%         \label{fig:subfig2}
%     \end{subfigure}
%     \caption{Main caption for the combined figure.}
%     \label{fig:main_figure}
% \end{figure}






% \textbf{Impact of LLMs.} Tabel [xxx] also shows the results of [xxx] for different LLMs in RAG.



In this paper, we systematically investigate the position bias problem in the multi-constraint instruction following. To quantitatively measure the disparity of constraint order, we propose a novel Difficulty Distribution Index (CDDI). Based on the CDDI, we design a probing task. First, we construct a large number of instructions consisting of different constraint orders. Then, we conduct experiments in two distinct scenarios. Extensive results reveal a clear preference of LLMs for ``hard-to-easy'' constraint orders. To further explore this, we conduct an explanation study. We visualize the importance of different constraints located in different positions and demonstrate the strong correlation between the model's attention distribution and its performance.
% \section{Introduction}

% These instructions are for authors submitting papers to *ACL conferences using \LaTeX. They are not self-contained. All authors must follow the general instructions for *ACL proceedings,\footnote{\url{http://acl-org.github.io/ACLPUB/formatting.html}} and this document contains additional instructions for the \LaTeX{} style files.

% The templates include the \LaTeX{} source of this document (\texttt{acl\_latex.tex}),
% the \LaTeX{} style file used to format it (\texttt{acl.sty}),
% an ACL bibliography style (\texttt{acl\_natbib.bst}),
% an example bibliography (\texttt{custom.bib}),
% and the bibliography for the ACL Anthology (\texttt{anthology.bib}).

% \section{Engines}

% To produce a PDF file, pdf\LaTeX{} is strongly recommended (over original \LaTeX{} plus dvips+ps2pdf or dvipdf). Xe\LaTeX{} also produces PDF files, and is especially suitable for text in non-Latin scripts.

% \section{Preamble}

% The first line of the file must be
% \begin{quote}
% \begin{verbatim}
% \documentclass[11pt]{article}
% \end{verbatim}
% \end{quote}

% To load the style file in the review version:
% \begin{quote}
% \begin{verbatim}
% \usepackage[review]{acl}
% \end{verbatim}
% \end{quote}
% For the final version, omit the \verb|review| option:
% \begin{quote}
% \begin{verbatim}
% \usepackage{acl}
% \end{verbatim}
% \end{quote}

% To use Times Roman, put the following in the preamble:
% \begin{quote}
% \begin{verbatim}
% \usepackage{times}
% \end{verbatim}
% \end{quote}
% (Alternatives like txfonts or newtx are also acceptable.)

% Please see the \LaTeX{} source of this document for comments on other packages that may be useful.

% Set the title and author using \verb|\title| and \verb|\author|. Within the author list, format multiple authors using \verb|\and| and \verb|\And| and \verb|\AND|; please see the \LaTeX{} source for examples.

% By default, the box containing the title and author names is set to the minimum of 5 cm. If you need more space, include the following in the preamble:
% \begin{quote}
% \begin{verbatim}
% \setlength\titlebox{<dim>}
% \end{verbatim}
% \end{quote}
% where \verb|<dim>| is replaced with a length. Do not set this length smaller than 5 cm.

% \section{Document Body}

% \subsection{Footnotes}

% Footnotes are inserted with the \verb|\footnote| command.\footnote{This is a footnote.}

% \subsection{Tables and figures}

% See Table~\ref{tab:accents} for an example of a table and its caption.
% \textbf{Do not override the default caption sizes.}

% \begin{table}
%   \centering
%   \begin{tabular}{lc}
%     \hline
%     \textbf{Command} & \textbf{Output} \\
%     \hline
%     \verb|{\"a}|     & {\"a}           \\
%     \verb|{\^e}|     & {\^e}           \\
%     \verb|{\`i}|     & {\`i}           \\
%     \verb|{\.I}|     & {\.I}           \\
%     \verb|{\o}|      & {\o}            \\
%     \verb|{\'u}|     & {\'u}           \\
%     \verb|{\aa}|     & {\aa}           \\\hline
%   \end{tabular}
%   \begin{tabular}{lc}
%     \hline
%     \textbf{Command} & \textbf{Output} \\
%     \hline
%     \verb|{\c c}|    & {\c c}          \\
%     \verb|{\u g}|    & {\u g}          \\
%     \verb|{\l}|      & {\l}            \\
%     \verb|{\~n}|     & {\~n}           \\
%     \verb|{\H o}|    & {\H o}          \\
%     \verb|{\v r}|    & {\v r}          \\
%     \verb|{\ss}|     & {\ss}           \\
%     \hline
%   \end{tabular}
%   \caption{Example commands for accented characters, to be used in, \emph{e.g.}, Bib\TeX{} entries.}
%   \label{tab:accents}
% \end{table}

% As much as possible, fonts in figures should conform
% to the document fonts. See Figure~\ref{fig:experiments} for an example of a figure and its caption.

% Using the \verb|graphicx| package graphics files can be included within figure
% environment at an appropriate point within the text.
% The \verb|graphicx| package supports various optional arguments to control the
% appearance of the figure.
% You must include it explicitly in the \LaTeX{} preamble (after the
% \verb|\documentclass| declaration and before \verb|\begin{document}|) using
% \verb|\usepackage{graphicx}|.

% \begin{figure}[t]
%   \includegraphics[width=\columnwidth]{example-image-golden}
%   \caption{A figure with a caption that runs for more than one line.
%     Example image is usually available through the \texttt{mwe} package
%     without even mentioning it in the preamble.}
%   \label{fig:experiments}
% \end{figure}

% \begin{figure*}[t]
%   \includegraphics[width=0.48\linewidth]{example-image-a} \hfill
%   \includegraphics[width=0.48\linewidth]{example-image-b}
%   \caption {A minimal working example to demonstrate how to place
%     two images side-by-side.}
% \end{figure*}

% \subsection{Hyperlinks}

% Users of older versions of \LaTeX{} may encounter the following error during compilation:
% \begin{quote}
% \verb|\pdfendlink| ended up in different nesting level than \verb|\pdfstartlink|.
% \end{quote}
% This happens when pdf\LaTeX{} is used and a citation splits across a page boundary. The best way to fix this is to upgrade \LaTeX{} to 2018-12-01 or later.

% \subsection{Citations}

% \begin{table*}
%   \centering
%   \begin{tabular}{lll}
%     \hline
%     \textbf{Output}           & \textbf{natbib command} & \textbf{ACL only command} \\
%     \hline
%     \citep{Gusfield:97}       & \verb|\citep|           &                           \\
%     \citealp{Gusfield:97}     & \verb|\citealp|         &                           \\
%     \citet{Gusfield:97}       & \verb|\citet|           &                           \\
%     \citeyearpar{Gusfield:97} & \verb|\citeyearpar|     &                           \\
%     \citeposs{Gusfield:97}    &                         & \verb|\citeposs|          \\
%     \hline
%   \end{tabular}
%   \caption{\label{citation-guide}
%     Citation commands supported by the style file.
%     The style is based on the natbib package and supports all natbib citation commands.
%     It also supports commands defined in previous ACL style files for compatibility.
%   }
% \end{table*}

% Table~\ref{citation-guide} shows the syntax supported by the style files.
% We encourage you to use the natbib styles.
% You can use the command \verb|\citet| (cite in text) to get ``author (year)'' citations, like this citation to a paper by \citet{Gusfield:97}.
% You can use the command \verb|\citep| (cite in parentheses) to get ``(author, year)'' citations \citep{Gusfield:97}.
% You can use the command \verb|\citealp| (alternative cite without parentheses) to get ``author, year'' citations, which is useful for using citations within parentheses (e.g. \citealp{Gusfield:97}).

% A possessive citation can be made with the command \verb|\citeposs|.
% This is not a standard natbib command, so it is generally not compatible
% with other style files.

% \subsection{References}

% \nocite{Ando2005,andrew2007scalable,rasooli-tetrault-2015}

% The \LaTeX{} and Bib\TeX{} style files provided roughly follow the American Psychological Association format.
% If your own bib file is named \texttt{custom.bib}, then placing the following before any appendices in your \LaTeX{} file will generate the references section for you:
% \begin{quote}
% \begin{verbatim}
% \bibliography{custom}
% \end{verbatim}
% \end{quote}

% You can obtain the complete ACL Anthology as a Bib\TeX{} file from \url{https://aclweb.org/anthology/anthology.bib.gz}.
% To include both the Anthology and your own .bib file, use the following instead of the above.
% \begin{quote}
% \begin{verbatim}
% \bibliography{anthology,custom}
% \end{verbatim}
% \end{quote}

% Please see Section~\ref{sec:bibtex} for information on preparing Bib\TeX{} files.

% \subsection{Equations}

% An example equation is shown below:
% \begin{equation}
%   \label{eq:example}
%   A = \pi r^2
% \end{equation}

% Labels for equation numbers, sections, subsections, figures and tables
% are all defined with the \verb|\label{label}| command and cross references
% to them are made with the \verb|\ref{label}| command.

% This an example cross-reference to Equation~\ref{eq:example}.

% \subsection{Appendices}

% Use \verb|\appendix| before any appendix section to switch the section numbering over to letters. See Appendix~\ref{sec:appendix} for an example.

% \section{Bib\TeX{} Files}
% \label{sec:bibtex}

% Unicode cannot be used in Bib\TeX{} entries, and some ways of typing special characters can disrupt Bib\TeX's alphabetization. The recommended way of typing special characters is shown in Table~\ref{tab:accents}.

% Please ensure that Bib\TeX{} records contain DOIs or URLs when possible, and for all the ACL materials that you reference.
% Use the \verb|doi| field for DOIs and the \verb|url| field for URLs.
% If a Bib\TeX{} entry has a URL or DOI field, the paper title in the references section will appear as a hyperlink to the paper, using the hyperref \LaTeX{} package.

% \section*{Acknowledgments}

% This document has been adapted
% by Steven Bethard, Ryan Cotterell and Rui Yan
% from the instructions for earlier ACL and NAACL proceedings, including those for
% ACL 2019 by Douwe Kiela and Ivan Vuli\'{c},
% NAACL 2019 by Stephanie Lukin and Alla Roskovskaya,
% ACL 2018 by Shay Cohen, Kevin Gimpel, and Wei Lu,
% NAACL 2018 by Margaret Mitchell and Stephanie Lukin,
% Bib\TeX{} suggestions for (NA)ACL 2017/2018 from Jason Eisner,
% ACL 2017 by Dan Gildea and Min-Yen Kan,
% NAACL 2017 by Margaret Mitchell,
% ACL 2012 by Maggie Li and Michael White,
% ACL 2010 by Jing-Shin Chang and Philipp Koehn,
% ACL 2008 by Johanna D. Moore, Simone Teufel, James Allan, and Sadaoki Furui,
% ACL 2005 by Hwee Tou Ng and Kemal Oflazer,
% ACL 2002 by Eugene Charniak and Dekang Lin,
% and earlier ACL and EACL formats written by several people, including
% John Chen, Henry S. Thompson and Donald Walker.
% Additional elements were taken from the formatting instructions of the \emph{International Joint Conference on Artificial Intelligence} and the \emph{Conference on Computer Vision and Pattern Recognition}.

% Bibliography entries for the entire Anthology, followed by custom entries
%\bibliography{anthology,custom}
% Custom bibliography entries only
% \bibliography{custom}
\bibliography{acl_latex}

\appendix

\subsection{Lloyd-Max Algorithm}
\label{subsec:Lloyd-Max}
For a given quantization bitwidth $B$ and an operand $\bm{X}$, the Lloyd-Max algorithm finds $2^B$ quantization levels $\{\hat{x}_i\}_{i=1}^{2^B}$ such that quantizing $\bm{X}$ by rounding each scalar in $\bm{X}$ to the nearest quantization level minimizes the quantization MSE. 

The algorithm starts with an initial guess of quantization levels and then iteratively computes quantization thresholds $\{\tau_i\}_{i=1}^{2^B-1}$ and updates quantization levels $\{\hat{x}_i\}_{i=1}^{2^B}$. Specifically, at iteration $n$, thresholds are set to the midpoints of the previous iteration's levels:
\begin{align*}
    \tau_i^{(n)}=\frac{\hat{x}_i^{(n-1)}+\hat{x}_{i+1}^{(n-1)}}2 \text{ for } i=1\ldots 2^B-1
\end{align*}
Subsequently, the quantization levels are re-computed as conditional means of the data regions defined by the new thresholds:
\begin{align*}
    \hat{x}_i^{(n)}=\mathbb{E}\left[ \bm{X} \big| \bm{X}\in [\tau_{i-1}^{(n)},\tau_i^{(n)}] \right] \text{ for } i=1\ldots 2^B
\end{align*}
where to satisfy boundary conditions we have $\tau_0=-\infty$ and $\tau_{2^B}=\infty$. The algorithm iterates the above steps until convergence.

Figure \ref{fig:lm_quant} compares the quantization levels of a $7$-bit floating point (E3M3) quantizer (left) to a $7$-bit Lloyd-Max quantizer (right) when quantizing a layer of weights from the GPT3-126M model at a per-tensor granularity. As shown, the Lloyd-Max quantizer achieves substantially lower quantization MSE. Further, Table \ref{tab:FP7_vs_LM7} shows the superior perplexity achieved by Lloyd-Max quantizers for bitwidths of $7$, $6$ and $5$. The difference between the quantizers is clear at 5 bits, where per-tensor FP quantization incurs a drastic and unacceptable increase in perplexity, while Lloyd-Max quantization incurs a much smaller increase. Nevertheless, we note that even the optimal Lloyd-Max quantizer incurs a notable ($\sim 1.5$) increase in perplexity due to the coarse granularity of quantization. 

\begin{figure}[h]
  \centering
  \includegraphics[width=0.7\linewidth]{sections/figures/LM7_FP7.pdf}
  \caption{\small Quantization levels and the corresponding quantization MSE of Floating Point (left) vs Lloyd-Max (right) Quantizers for a layer of weights in the GPT3-126M model.}
  \label{fig:lm_quant}
\end{figure}

\begin{table}[h]\scriptsize
\begin{center}
\caption{\label{tab:FP7_vs_LM7} \small Comparing perplexity (lower is better) achieved by floating point quantizers and Lloyd-Max quantizers on a GPT3-126M model for the Wikitext-103 dataset.}
\begin{tabular}{c|cc|c}
\hline
 \multirow{2}{*}{\textbf{Bitwidth}} & \multicolumn{2}{|c|}{\textbf{Floating-Point Quantizer}} & \textbf{Lloyd-Max Quantizer} \\
 & Best Format & Wikitext-103 Perplexity & Wikitext-103 Perplexity \\
\hline
7 & E3M3 & 18.32 & 18.27 \\
6 & E3M2 & 19.07 & 18.51 \\
5 & E4M0 & 43.89 & 19.71 \\
\hline
\end{tabular}
\end{center}
\end{table}

\subsection{Proof of Local Optimality of LO-BCQ}
\label{subsec:lobcq_opt_proof}
For a given block $\bm{b}_j$, the quantization MSE during LO-BCQ can be empirically evaluated as $\frac{1}{L_b}\lVert \bm{b}_j- \bm{\hat{b}}_j\rVert^2_2$ where $\bm{\hat{b}}_j$ is computed from equation (\ref{eq:clustered_quantization_definition}) as $C_{f(\bm{b}_j)}(\bm{b}_j)$. Further, for a given block cluster $\mathcal{B}_i$, we compute the quantization MSE as $\frac{1}{|\mathcal{B}_{i}|}\sum_{\bm{b} \in \mathcal{B}_{i}} \frac{1}{L_b}\lVert \bm{b}- C_i^{(n)}(\bm{b})\rVert^2_2$. Therefore, at the end of iteration $n$, we evaluate the overall quantization MSE $J^{(n)}$ for a given operand $\bm{X}$ composed of $N_c$ block clusters as:
\begin{align*}
    \label{eq:mse_iter_n}
    J^{(n)} = \frac{1}{N_c} \sum_{i=1}^{N_c} \frac{1}{|\mathcal{B}_{i}^{(n)}|}\sum_{\bm{v} \in \mathcal{B}_{i}^{(n)}} \frac{1}{L_b}\lVert \bm{b}- B_i^{(n)}(\bm{b})\rVert^2_2
\end{align*}

At the end of iteration $n$, the codebooks are updated from $\mathcal{C}^{(n-1)}$ to $\mathcal{C}^{(n)}$. However, the mapping of a given vector $\bm{b}_j$ to quantizers $\mathcal{C}^{(n)}$ remains as  $f^{(n)}(\bm{b}_j)$. At the next iteration, during the vector clustering step, $f^{(n+1)}(\bm{b}_j)$ finds new mapping of $\bm{b}_j$ to updated codebooks $\mathcal{C}^{(n)}$ such that the quantization MSE over the candidate codebooks is minimized. Therefore, we obtain the following result for $\bm{b}_j$:
\begin{align*}
\frac{1}{L_b}\lVert \bm{b}_j - C_{f^{(n+1)}(\bm{b}_j)}^{(n)}(\bm{b}_j)\rVert^2_2 \le \frac{1}{L_b}\lVert \bm{b}_j - C_{f^{(n)}(\bm{b}_j)}^{(n)}(\bm{b}_j)\rVert^2_2
\end{align*}

That is, quantizing $\bm{b}_j$ at the end of the block clustering step of iteration $n+1$ results in lower quantization MSE compared to quantizing at the end of iteration $n$. Since this is true for all $\bm{b} \in \bm{X}$, we assert the following:
\begin{equation}
\begin{split}
\label{eq:mse_ineq_1}
    \tilde{J}^{(n+1)} &= \frac{1}{N_c} \sum_{i=1}^{N_c} \frac{1}{|\mathcal{B}_{i}^{(n+1)}|}\sum_{\bm{b} \in \mathcal{B}_{i}^{(n+1)}} \frac{1}{L_b}\lVert \bm{b} - C_i^{(n)}(b)\rVert^2_2 \le J^{(n)}
\end{split}
\end{equation}
where $\tilde{J}^{(n+1)}$ is the the quantization MSE after the vector clustering step at iteration $n+1$.

Next, during the codebook update step (\ref{eq:quantizers_update}) at iteration $n+1$, the per-cluster codebooks $\mathcal{C}^{(n)}$ are updated to $\mathcal{C}^{(n+1)}$ by invoking the Lloyd-Max algorithm \citep{Lloyd}. We know that for any given value distribution, the Lloyd-Max algorithm minimizes the quantization MSE. Therefore, for a given vector cluster $\mathcal{B}_i$ we obtain the following result:

\begin{equation}
    \frac{1}{|\mathcal{B}_{i}^{(n+1)}|}\sum_{\bm{b} \in \mathcal{B}_{i}^{(n+1)}} \frac{1}{L_b}\lVert \bm{b}- C_i^{(n+1)}(\bm{b})\rVert^2_2 \le \frac{1}{|\mathcal{B}_{i}^{(n+1)}|}\sum_{\bm{b} \in \mathcal{B}_{i}^{(n+1)}} \frac{1}{L_b}\lVert \bm{b}- C_i^{(n)}(\bm{b})\rVert^2_2
\end{equation}

The above equation states that quantizing the given block cluster $\mathcal{B}_i$ after updating the associated codebook from $C_i^{(n)}$ to $C_i^{(n+1)}$ results in lower quantization MSE. Since this is true for all the block clusters, we derive the following result: 
\begin{equation}
\begin{split}
\label{eq:mse_ineq_2}
     J^{(n+1)} &= \frac{1}{N_c} \sum_{i=1}^{N_c} \frac{1}{|\mathcal{B}_{i}^{(n+1)}|}\sum_{\bm{b} \in \mathcal{B}_{i}^{(n+1)}} \frac{1}{L_b}\lVert \bm{b}- C_i^{(n+1)}(\bm{b})\rVert^2_2  \le \tilde{J}^{(n+1)}   
\end{split}
\end{equation}

Following (\ref{eq:mse_ineq_1}) and (\ref{eq:mse_ineq_2}), we find that the quantization MSE is non-increasing for each iteration, that is, $J^{(1)} \ge J^{(2)} \ge J^{(3)} \ge \ldots \ge J^{(M)}$ where $M$ is the maximum number of iterations. 
%Therefore, we can say that if the algorithm converges, then it must be that it has converged to a local minimum. 
\hfill $\blacksquare$


\begin{figure}
    \begin{center}
    \includegraphics[width=0.5\textwidth]{sections//figures/mse_vs_iter.pdf}
    \end{center}
    \caption{\small NMSE vs iterations during LO-BCQ compared to other block quantization proposals}
    \label{fig:nmse_vs_iter}
\end{figure}

Figure \ref{fig:nmse_vs_iter} shows the empirical convergence of LO-BCQ across several block lengths and number of codebooks. Also, the MSE achieved by LO-BCQ is compared to baselines such as MXFP and VSQ. As shown, LO-BCQ converges to a lower MSE than the baselines. Further, we achieve better convergence for larger number of codebooks ($N_c$) and for a smaller block length ($L_b$), both of which increase the bitwidth of BCQ (see Eq \ref{eq:bitwidth_bcq}).


\subsection{Additional Accuracy Results}
%Table \ref{tab:lobcq_config} lists the various LOBCQ configurations and their corresponding bitwidths.
\begin{table}
\setlength{\tabcolsep}{4.75pt}
\begin{center}
\caption{\label{tab:lobcq_config} Various LO-BCQ configurations and their bitwidths.}
\begin{tabular}{|c||c|c|c|c||c|c||c|} 
\hline
 & \multicolumn{4}{|c||}{$L_b=8$} & \multicolumn{2}{|c||}{$L_b=4$} & $L_b=2$ \\
 \hline
 \backslashbox{$L_A$\kern-1em}{\kern-1em$N_c$} & 2 & 4 & 8 & 16 & 2 & 4 & 2 \\
 \hline
 64 & 4.25 & 4.375 & 4.5 & 4.625 & 4.375 & 4.625 & 4.625\\
 \hline
 32 & 4.375 & 4.5 & 4.625& 4.75 & 4.5 & 4.75 & 4.75 \\
 \hline
 16 & 4.625 & 4.75& 4.875 & 5 & 4.75 & 5 & 5 \\
 \hline
\end{tabular}
\end{center}
\end{table}

%\subsection{Perplexity achieved by various LO-BCQ configurations on Wikitext-103 dataset}

\begin{table} \centering
\begin{tabular}{|c||c|c|c|c||c|c||c|} 
\hline
 $L_b \rightarrow$& \multicolumn{4}{c||}{8} & \multicolumn{2}{c||}{4} & 2\\
 \hline
 \backslashbox{$L_A$\kern-1em}{\kern-1em$N_c$} & 2 & 4 & 8 & 16 & 2 & 4 & 2  \\
 %$N_c \rightarrow$ & 2 & 4 & 8 & 16 & 2 & 4 & 2 \\
 \hline
 \hline
 \multicolumn{8}{c}{GPT3-1.3B (FP32 PPL = 9.98)} \\ 
 \hline
 \hline
 64 & 10.40 & 10.23 & 10.17 & 10.15 &  10.28 & 10.18 & 10.19 \\
 \hline
 32 & 10.25 & 10.20 & 10.15 & 10.12 &  10.23 & 10.17 & 10.17 \\
 \hline
 16 & 10.22 & 10.16 & 10.10 & 10.09 &  10.21 & 10.14 & 10.16 \\
 \hline
  \hline
 \multicolumn{8}{c}{GPT3-8B (FP32 PPL = 7.38)} \\ 
 \hline
 \hline
 64 & 7.61 & 7.52 & 7.48 &  7.47 &  7.55 &  7.49 & 7.50 \\
 \hline
 32 & 7.52 & 7.50 & 7.46 &  7.45 &  7.52 &  7.48 & 7.48  \\
 \hline
 16 & 7.51 & 7.48 & 7.44 &  7.44 &  7.51 &  7.49 & 7.47  \\
 \hline
\end{tabular}
\caption{\label{tab:ppl_gpt3_abalation} Wikitext-103 perplexity across GPT3-1.3B and 8B models.}
\end{table}

\begin{table} \centering
\begin{tabular}{|c||c|c|c|c||} 
\hline
 $L_b \rightarrow$& \multicolumn{4}{c||}{8}\\
 \hline
 \backslashbox{$L_A$\kern-1em}{\kern-1em$N_c$} & 2 & 4 & 8 & 16 \\
 %$N_c \rightarrow$ & 2 & 4 & 8 & 16 & 2 & 4 & 2 \\
 \hline
 \hline
 \multicolumn{5}{|c|}{Llama2-7B (FP32 PPL = 5.06)} \\ 
 \hline
 \hline
 64 & 5.31 & 5.26 & 5.19 & 5.18  \\
 \hline
 32 & 5.23 & 5.25 & 5.18 & 5.15  \\
 \hline
 16 & 5.23 & 5.19 & 5.16 & 5.14  \\
 \hline
 \multicolumn{5}{|c|}{Nemotron4-15B (FP32 PPL = 5.87)} \\ 
 \hline
 \hline
 64  & 6.3 & 6.20 & 6.13 & 6.08  \\
 \hline
 32  & 6.24 & 6.12 & 6.07 & 6.03  \\
 \hline
 16  & 6.12 & 6.14 & 6.04 & 6.02  \\
 \hline
 \multicolumn{5}{|c|}{Nemotron4-340B (FP32 PPL = 3.48)} \\ 
 \hline
 \hline
 64 & 3.67 & 3.62 & 3.60 & 3.59 \\
 \hline
 32 & 3.63 & 3.61 & 3.59 & 3.56 \\
 \hline
 16 & 3.61 & 3.58 & 3.57 & 3.55 \\
 \hline
\end{tabular}
\caption{\label{tab:ppl_llama7B_nemo15B} Wikitext-103 perplexity compared to FP32 baseline in Llama2-7B and Nemotron4-15B, 340B models}
\end{table}

%\subsection{Perplexity achieved by various LO-BCQ configurations on MMLU dataset}


\begin{table} \centering
\begin{tabular}{|c||c|c|c|c||c|c|c|c|} 
\hline
 $L_b \rightarrow$& \multicolumn{4}{c||}{8} & \multicolumn{4}{c||}{8}\\
 \hline
 \backslashbox{$L_A$\kern-1em}{\kern-1em$N_c$} & 2 & 4 & 8 & 16 & 2 & 4 & 8 & 16  \\
 %$N_c \rightarrow$ & 2 & 4 & 8 & 16 & 2 & 4 & 2 \\
 \hline
 \hline
 \multicolumn{5}{|c|}{Llama2-7B (FP32 Accuracy = 45.8\%)} & \multicolumn{4}{|c|}{Llama2-70B (FP32 Accuracy = 69.12\%)} \\ 
 \hline
 \hline
 64 & 43.9 & 43.4 & 43.9 & 44.9 & 68.07 & 68.27 & 68.17 & 68.75 \\
 \hline
 32 & 44.5 & 43.8 & 44.9 & 44.5 & 68.37 & 68.51 & 68.35 & 68.27  \\
 \hline
 16 & 43.9 & 42.7 & 44.9 & 45 & 68.12 & 68.77 & 68.31 & 68.59  \\
 \hline
 \hline
 \multicolumn{5}{|c|}{GPT3-22B (FP32 Accuracy = 38.75\%)} & \multicolumn{4}{|c|}{Nemotron4-15B (FP32 Accuracy = 64.3\%)} \\ 
 \hline
 \hline
 64 & 36.71 & 38.85 & 38.13 & 38.92 & 63.17 & 62.36 & 63.72 & 64.09 \\
 \hline
 32 & 37.95 & 38.69 & 39.45 & 38.34 & 64.05 & 62.30 & 63.8 & 64.33  \\
 \hline
 16 & 38.88 & 38.80 & 38.31 & 38.92 & 63.22 & 63.51 & 63.93 & 64.43  \\
 \hline
\end{tabular}
\caption{\label{tab:mmlu_abalation} Accuracy on MMLU dataset across GPT3-22B, Llama2-7B, 70B and Nemotron4-15B models.}
\end{table}


%\subsection{Perplexity achieved by various LO-BCQ configurations on LM evaluation harness}

\begin{table} \centering
\begin{tabular}{|c||c|c|c|c||c|c|c|c|} 
\hline
 $L_b \rightarrow$& \multicolumn{4}{c||}{8} & \multicolumn{4}{c||}{8}\\
 \hline
 \backslashbox{$L_A$\kern-1em}{\kern-1em$N_c$} & 2 & 4 & 8 & 16 & 2 & 4 & 8 & 16  \\
 %$N_c \rightarrow$ & 2 & 4 & 8 & 16 & 2 & 4 & 2 \\
 \hline
 \hline
 \multicolumn{5}{|c|}{Race (FP32 Accuracy = 37.51\%)} & \multicolumn{4}{|c|}{Boolq (FP32 Accuracy = 64.62\%)} \\ 
 \hline
 \hline
 64 & 36.94 & 37.13 & 36.27 & 37.13 & 63.73 & 62.26 & 63.49 & 63.36 \\
 \hline
 32 & 37.03 & 36.36 & 36.08 & 37.03 & 62.54 & 63.51 & 63.49 & 63.55  \\
 \hline
 16 & 37.03 & 37.03 & 36.46 & 37.03 & 61.1 & 63.79 & 63.58 & 63.33  \\
 \hline
 \hline
 \multicolumn{5}{|c|}{Winogrande (FP32 Accuracy = 58.01\%)} & \multicolumn{4}{|c|}{Piqa (FP32 Accuracy = 74.21\%)} \\ 
 \hline
 \hline
 64 & 58.17 & 57.22 & 57.85 & 58.33 & 73.01 & 73.07 & 73.07 & 72.80 \\
 \hline
 32 & 59.12 & 58.09 & 57.85 & 58.41 & 73.01 & 73.94 & 72.74 & 73.18  \\
 \hline
 16 & 57.93 & 58.88 & 57.93 & 58.56 & 73.94 & 72.80 & 73.01 & 73.94  \\
 \hline
\end{tabular}
\caption{\label{tab:mmlu_abalation} Accuracy on LM evaluation harness tasks on GPT3-1.3B model.}
\end{table}

\begin{table} \centering
\begin{tabular}{|c||c|c|c|c||c|c|c|c|} 
\hline
 $L_b \rightarrow$& \multicolumn{4}{c||}{8} & \multicolumn{4}{c||}{8}\\
 \hline
 \backslashbox{$L_A$\kern-1em}{\kern-1em$N_c$} & 2 & 4 & 8 & 16 & 2 & 4 & 8 & 16  \\
 %$N_c \rightarrow$ & 2 & 4 & 8 & 16 & 2 & 4 & 2 \\
 \hline
 \hline
 \multicolumn{5}{|c|}{Race (FP32 Accuracy = 41.34\%)} & \multicolumn{4}{|c|}{Boolq (FP32 Accuracy = 68.32\%)} \\ 
 \hline
 \hline
 64 & 40.48 & 40.10 & 39.43 & 39.90 & 69.20 & 68.41 & 69.45 & 68.56 \\
 \hline
 32 & 39.52 & 39.52 & 40.77 & 39.62 & 68.32 & 67.43 & 68.17 & 69.30  \\
 \hline
 16 & 39.81 & 39.71 & 39.90 & 40.38 & 68.10 & 66.33 & 69.51 & 69.42  \\
 \hline
 \hline
 \multicolumn{5}{|c|}{Winogrande (FP32 Accuracy = 67.88\%)} & \multicolumn{4}{|c|}{Piqa (FP32 Accuracy = 78.78\%)} \\ 
 \hline
 \hline
 64 & 66.85 & 66.61 & 67.72 & 67.88 & 77.31 & 77.42 & 77.75 & 77.64 \\
 \hline
 32 & 67.25 & 67.72 & 67.72 & 67.00 & 77.31 & 77.04 & 77.80 & 77.37  \\
 \hline
 16 & 68.11 & 68.90 & 67.88 & 67.48 & 77.37 & 78.13 & 78.13 & 77.69  \\
 \hline
\end{tabular}
\caption{\label{tab:mmlu_abalation} Accuracy on LM evaluation harness tasks on GPT3-8B model.}
\end{table}

\begin{table} \centering
\begin{tabular}{|c||c|c|c|c||c|c|c|c|} 
\hline
 $L_b \rightarrow$& \multicolumn{4}{c||}{8} & \multicolumn{4}{c||}{8}\\
 \hline
 \backslashbox{$L_A$\kern-1em}{\kern-1em$N_c$} & 2 & 4 & 8 & 16 & 2 & 4 & 8 & 16  \\
 %$N_c \rightarrow$ & 2 & 4 & 8 & 16 & 2 & 4 & 2 \\
 \hline
 \hline
 \multicolumn{5}{|c|}{Race (FP32 Accuracy = 40.67\%)} & \multicolumn{4}{|c|}{Boolq (FP32 Accuracy = 76.54\%)} \\ 
 \hline
 \hline
 64 & 40.48 & 40.10 & 39.43 & 39.90 & 75.41 & 75.11 & 77.09 & 75.66 \\
 \hline
 32 & 39.52 & 39.52 & 40.77 & 39.62 & 76.02 & 76.02 & 75.96 & 75.35  \\
 \hline
 16 & 39.81 & 39.71 & 39.90 & 40.38 & 75.05 & 73.82 & 75.72 & 76.09  \\
 \hline
 \hline
 \multicolumn{5}{|c|}{Winogrande (FP32 Accuracy = 70.64\%)} & \multicolumn{4}{|c|}{Piqa (FP32 Accuracy = 79.16\%)} \\ 
 \hline
 \hline
 64 & 69.14 & 70.17 & 70.17 & 70.56 & 78.24 & 79.00 & 78.62 & 78.73 \\
 \hline
 32 & 70.96 & 69.69 & 71.27 & 69.30 & 78.56 & 79.49 & 79.16 & 78.89  \\
 \hline
 16 & 71.03 & 69.53 & 69.69 & 70.40 & 78.13 & 79.16 & 79.00 & 79.00  \\
 \hline
\end{tabular}
\caption{\label{tab:mmlu_abalation} Accuracy on LM evaluation harness tasks on GPT3-22B model.}
\end{table}

\begin{table} \centering
\begin{tabular}{|c||c|c|c|c||c|c|c|c|} 
\hline
 $L_b \rightarrow$& \multicolumn{4}{c||}{8} & \multicolumn{4}{c||}{8}\\
 \hline
 \backslashbox{$L_A$\kern-1em}{\kern-1em$N_c$} & 2 & 4 & 8 & 16 & 2 & 4 & 8 & 16  \\
 %$N_c \rightarrow$ & 2 & 4 & 8 & 16 & 2 & 4 & 2 \\
 \hline
 \hline
 \multicolumn{5}{|c|}{Race (FP32 Accuracy = 44.4\%)} & \multicolumn{4}{|c|}{Boolq (FP32 Accuracy = 79.29\%)} \\ 
 \hline
 \hline
 64 & 42.49 & 42.51 & 42.58 & 43.45 & 77.58 & 77.37 & 77.43 & 78.1 \\
 \hline
 32 & 43.35 & 42.49 & 43.64 & 43.73 & 77.86 & 75.32 & 77.28 & 77.86  \\
 \hline
 16 & 44.21 & 44.21 & 43.64 & 42.97 & 78.65 & 77 & 76.94 & 77.98  \\
 \hline
 \hline
 \multicolumn{5}{|c|}{Winogrande (FP32 Accuracy = 69.38\%)} & \multicolumn{4}{|c|}{Piqa (FP32 Accuracy = 78.07\%)} \\ 
 \hline
 \hline
 64 & 68.9 & 68.43 & 69.77 & 68.19 & 77.09 & 76.82 & 77.09 & 77.86 \\
 \hline
 32 & 69.38 & 68.51 & 68.82 & 68.90 & 78.07 & 76.71 & 78.07 & 77.86  \\
 \hline
 16 & 69.53 & 67.09 & 69.38 & 68.90 & 77.37 & 77.8 & 77.91 & 77.69  \\
 \hline
\end{tabular}
\caption{\label{tab:mmlu_abalation} Accuracy on LM evaluation harness tasks on Llama2-7B model.}
\end{table}

\begin{table} \centering
\begin{tabular}{|c||c|c|c|c||c|c|c|c|} 
\hline
 $L_b \rightarrow$& \multicolumn{4}{c||}{8} & \multicolumn{4}{c||}{8}\\
 \hline
 \backslashbox{$L_A$\kern-1em}{\kern-1em$N_c$} & 2 & 4 & 8 & 16 & 2 & 4 & 8 & 16  \\
 %$N_c \rightarrow$ & 2 & 4 & 8 & 16 & 2 & 4 & 2 \\
 \hline
 \hline
 \multicolumn{5}{|c|}{Race (FP32 Accuracy = 48.8\%)} & \multicolumn{4}{|c|}{Boolq (FP32 Accuracy = 85.23\%)} \\ 
 \hline
 \hline
 64 & 49.00 & 49.00 & 49.28 & 48.71 & 82.82 & 84.28 & 84.03 & 84.25 \\
 \hline
 32 & 49.57 & 48.52 & 48.33 & 49.28 & 83.85 & 84.46 & 84.31 & 84.93  \\
 \hline
 16 & 49.85 & 49.09 & 49.28 & 48.99 & 85.11 & 84.46 & 84.61 & 83.94  \\
 \hline
 \hline
 \multicolumn{5}{|c|}{Winogrande (FP32 Accuracy = 79.95\%)} & \multicolumn{4}{|c|}{Piqa (FP32 Accuracy = 81.56\%)} \\ 
 \hline
 \hline
 64 & 78.77 & 78.45 & 78.37 & 79.16 & 81.45 & 80.69 & 81.45 & 81.5 \\
 \hline
 32 & 78.45 & 79.01 & 78.69 & 80.66 & 81.56 & 80.58 & 81.18 & 81.34  \\
 \hline
 16 & 79.95 & 79.56 & 79.79 & 79.72 & 81.28 & 81.66 & 81.28 & 80.96  \\
 \hline
\end{tabular}
\caption{\label{tab:mmlu_abalation} Accuracy on LM evaluation harness tasks on Llama2-70B model.}
\end{table}

%\section{MSE Studies}
%\textcolor{red}{TODO}


\subsection{Number Formats and Quantization Method}
\label{subsec:numFormats_quantMethod}
\subsubsection{Integer Format}
An $n$-bit signed integer (INT) is typically represented with a 2s-complement format \citep{yao2022zeroquant,xiao2023smoothquant,dai2021vsq}, where the most significant bit denotes the sign.

\subsubsection{Floating Point Format}
An $n$-bit signed floating point (FP) number $x$ comprises of a 1-bit sign ($x_{\mathrm{sign}}$), $B_m$-bit mantissa ($x_{\mathrm{mant}}$) and $B_e$-bit exponent ($x_{\mathrm{exp}}$) such that $B_m+B_e=n-1$. The associated constant exponent bias ($E_{\mathrm{bias}}$) is computed as $(2^{{B_e}-1}-1)$. We denote this format as $E_{B_e}M_{B_m}$.  

\subsubsection{Quantization Scheme}
\label{subsec:quant_method}
A quantization scheme dictates how a given unquantized tensor is converted to its quantized representation. We consider FP formats for the purpose of illustration. Given an unquantized tensor $\bm{X}$ and an FP format $E_{B_e}M_{B_m}$, we first, we compute the quantization scale factor $s_X$ that maps the maximum absolute value of $\bm{X}$ to the maximum quantization level of the $E_{B_e}M_{B_m}$ format as follows:
\begin{align}
\label{eq:sf}
    s_X = \frac{\mathrm{max}(|\bm{X}|)}{\mathrm{max}(E_{B_e}M_{B_m})}
\end{align}
In the above equation, $|\cdot|$ denotes the absolute value function.

Next, we scale $\bm{X}$ by $s_X$ and quantize it to $\hat{\bm{X}}$ by rounding it to the nearest quantization level of $E_{B_e}M_{B_m}$ as:

\begin{align}
\label{eq:tensor_quant}
    \hat{\bm{X}} = \text{round-to-nearest}\left(\frac{\bm{X}}{s_X}, E_{B_e}M_{B_m}\right)
\end{align}

We perform dynamic max-scaled quantization \citep{wu2020integer}, where the scale factor $s$ for activations is dynamically computed during runtime.

\subsection{Vector Scaled Quantization}
\begin{wrapfigure}{r}{0.35\linewidth}
  \centering
  \includegraphics[width=\linewidth]{sections/figures/vsquant.jpg}
  \caption{\small Vectorwise decomposition for per-vector scaled quantization (VSQ \citep{dai2021vsq}).}
  \label{fig:vsquant}
\end{wrapfigure}
During VSQ \citep{dai2021vsq}, the operand tensors are decomposed into 1D vectors in a hardware friendly manner as shown in Figure \ref{fig:vsquant}. Since the decomposed tensors are used as operands in matrix multiplications during inference, it is beneficial to perform this decomposition along the reduction dimension of the multiplication. The vectorwise quantization is performed similar to tensorwise quantization described in Equations \ref{eq:sf} and \ref{eq:tensor_quant}, where a scale factor $s_v$ is required for each vector $\bm{v}$ that maps the maximum absolute value of that vector to the maximum quantization level. While smaller vector lengths can lead to larger accuracy gains, the associated memory and computational overheads due to the per-vector scale factors increases. To alleviate these overheads, VSQ \citep{dai2021vsq} proposed a second level quantization of the per-vector scale factors to unsigned integers, while MX \citep{rouhani2023shared} quantizes them to integer powers of 2 (denoted as $2^{INT}$).

\subsubsection{MX Format}
The MX format proposed in \citep{rouhani2023microscaling} introduces the concept of sub-block shifting. For every two scalar elements of $b$-bits each, there is a shared exponent bit. The value of this exponent bit is determined through an empirical analysis that targets minimizing quantization MSE. We note that the FP format $E_{1}M_{b}$ is strictly better than MX from an accuracy perspective since it allocates a dedicated exponent bit to each scalar as opposed to sharing it across two scalars. Therefore, we conservatively bound the accuracy of a $b+2$-bit signed MX format with that of a $E_{1}M_{b}$ format in our comparisons. For instance, we use E1M2 format as a proxy for MX4.

\begin{figure}
    \centering
    \includegraphics[width=1\linewidth]{sections//figures/BlockFormats.pdf}
    \caption{\small Comparing LO-BCQ to MX format.}
    \label{fig:block_formats}
\end{figure}

Figure \ref{fig:block_formats} compares our $4$-bit LO-BCQ block format to MX \citep{rouhani2023microscaling}. As shown, both LO-BCQ and MX decompose a given operand tensor into block arrays and each block array into blocks. Similar to MX, we find that per-block quantization ($L_b < L_A$) leads to better accuracy due to increased flexibility. While MX achieves this through per-block $1$-bit micro-scales, we associate a dedicated codebook to each block through a per-block codebook selector. Further, MX quantizes the per-block array scale-factor to E8M0 format without per-tensor scaling. In contrast during LO-BCQ, we find that per-tensor scaling combined with quantization of per-block array scale-factor to E4M3 format results in superior inference accuracy across models. 


\end{document}
