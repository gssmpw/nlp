\section{Related Work}
Control barrier functions provide sufficient and necessary conditions for a system to guarantee safety, i.e., they satisfy a forward invariance property within a defined safe set~\cite{ames2019control}. Despite their reliable performance in tasks such as collision avoidance~\cite{abdi2023safe}, lane keeping, and adaptive cruise control~\cite{xu2017correctness}, they synthesize reactive control inputs considering only the current states. This property leads to short-sighted behaviors in complex tasks where planning is preferred. For example, the CBF-based controller leads to deadlocks in multi-robot navigation~\cite{wang2017safety}. In visual contact, reactive control inputs result in dramatic changes to robots's headings and positions and cause inefficient trajectories~\cite{catellani2023distributed} or even compromise goal-reaching capability. 
In this work, we combine trajectory planning with CBFs to overcome such short-sightedness.% and generate a trajectory and control certified by control barrier functions.

Some distributed multi-robot trajectory generation approaches, including~\cite{luis2020online, csenbacslar2023rlss, zhou2017fast}, utilize safety corridors for planning. Corridor planning is a decoupling method, as it decomposes the optimization in the joint configuration space into single-robot configuration space. Control barrier functions, compared to  decoupling methods, impose minimal conservative constraints in the optimization and significantly improve  feasibility. More importantly, incorporating control barrier functions into planning provides theoretical robustness guarantees. By robustness, we refer to the stabilization property of control Lyapunov-barrier functions~\cite{xiao2021hoclbf}, which drive the system towards the safe set, even if starting outside. We design a robust and resilient controller that tolerates temporary violation of constraints and actively stabilizes the system back to the safe set. Such a property is desired when uncertainty is present or when it is infeasible to satisfy all constraints. To the best of our knowledge, this work is the first of its kind to generate a continuous-time trajectory and controller concurrently, certified by control barrier functions utilizing piecewise splines.

There have been attempts to combine MPC with CBFs in a discrete-time formulation~\cite{liu2023iterative, zeng2021safety}. In continuous-time formulation, a multi-layer controller~\cite{sforni2024receding} solves a control sequence with CBF constraints. A spline-based trajectory generation method imposes CBF constraints by constraining robot state within polygonal cells~\cite{dickson2024spline}. However, it only solves trajectory and requires additional optimization steps for control synthesis. 
%, and it is hard to adapt a general-purpose control barrier function, such as visual contact in our scenario. 
Our continuous-time formulation solves a different problem, i.e., optimizing continuous-time piecewise spline as trajectory and obtaining control inputs concurrently. Compared to~\cite{sforni2024receding}, our approach provides additional smoothness and derivatives up to an arbitrary order, which is desired by agile robotic systems. Compared to~\cite{dickson2024spline}, our framework unifies the trajectory generation and control synthesis in one optimization problem. In addition, our framework can adapt to any control barrier functions without changing the solver. Adaptiveness to a general-purpose CBF is crucial for applications other than collision avoidance, such as visual contact in our scenario. Due to its simplicity, our approach can be applied to different robotic platforms and used as an online (real-time) navigation stack for different certification requirements.
%Our work generates continuous-time trajectory and control inputs and provides control derivatives up to an arbitrary order, which is desired by agile robotic systems, all in real time. 
Our approach has the following advantages. 1) It unifies  trajectory generation and control into one spline-based framework; control inputs can be computed directly from the trajectory. 2) It provides smooth control inputs up to an arbitrary order of derivative. 3) It adopts any general-purpose control barrier functions. 4) It delivers \emph{continuous-time} trajectory and control. Thus, we can evaluate control inputs at any time $t$ within the planning horizon. This gives us a higher granularity control of the system compared to applying a fixed input over a time step, such as in discrete MPC, thus responding better to control delays and stabilizing agile systems, such as quadrotors. 5) It does this all in real time. 

In the aforementioned multi-robot trajectory generation approaches, perfect sensing of neighbors is assumed~\cite{csenbacslar2023rlss, zhou2017fast}. Sensing uncertainty, however, is always present in the physical world. We adopt a sensing model with field-of-view constraints similar to~\cite{catellani2023distributed} and estimate neighbors' relative positions. Several solutions exist in the literature to achieve vision-based localization, from simple tag-based localization~\cite{malyuta2020long} to deep learning models~\cite{ge2022vision} or blinking UV markers~\cite{walter2019uvdar}. Due to hardware limitations, measurement uncertainty in these approaches should be carefully modeled.