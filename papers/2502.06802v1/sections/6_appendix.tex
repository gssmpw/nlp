\appendix
\section{LLM Prompt for Game Profile Generation}
\label{sec:game_profile_generation}

\begin{lstlisting}
Given a Roblox game, we have access only to the in-game text features.
These features are provided in the following list format: %s.
The game's language is: %s. If the language is specified as "NONE," please
analyze the text to determine the game's language.

Context:
1. In-game text features are the text elements displayed to users while they play the game. (WHAT)
2. These text features provide crucial information to help users understand and navigate the game. (WHY)
3. Understanding these text features is essential for users to play the game effectively. (PURPOSE)
4. The in-game text features can be noisy and may contain irrelevant information. Please focus only on the relevant information and omit any irrelevant details. (NOTE)

Task:
1. Generate a summary for the game. This summary is vital for the recommender
system to better understand the game.
2. The summary should be concise, informative, and a few sentences long. It
will help in understanding user preferences and recommending games accordingly.
3. The summary MUST be in JSON format, directly readable by json.loads(). The
JSON should have the following structure, where each key represents an
attribute of the game and the value is the corresponding attribute's value:
{
    "game_about": "Provide a concise and informative description of the Roblox
    game. Include the main theme or storyline, primary objectives, core gameplay
    mechanics, unique features, and target audience. This should give a clear
    overview of what the game is about and what players can expect.",
    "game_genre": "Specify the genre of the Roblox game. Examples include obby
    (obstacle course), tycoon, role-playing, simulator, adventure, etc. This helps
    categorize the game and gives an idea of the type of gameplay involved.",
    "suitable_for": "Indicate the target audience for the Roblox game. This
    could be based on age group (e.g., kids, teens, all ages) or other demographic
    factors (e.g., casual players, competitive players). This helps in
    understanding who the game is designed for.",
    "features": "List the key features of the Roblox game. These could include
    multiplayer modes, character customization, in-game purchases, special
    abilities, unique controls, etc. This highlights what makes the game
    interesting and engaging.",
    "includes": "Mention any additional content or elements included in the
    Roblox game. This could be special events, seasonal updates, exclusive items,
    etc. This provides information on the extra content available to players.",
    "game_language": "Specify the language of the Roblox game. If the language
    is 'NONE', analyze the in-game text to determine the language. This helps in
    understanding the linguistic accessibility of the game.",
    "game_scale": "Describe the scale of the Roblox game. This could refer to
    the size of the game world, the number of levels or stages, the length of the
    gameplay, etc. This gives an idea of the game's scope and depth."
}
\end{lstlisting}

\section{LLM Prompt for User Profile and Ranking Strategy Generation}
\label{sec:user_profile}
\begin{lstlisting}
Given the user play history in the past 7 days, please write a personalized ranking strategy for the user for future ranking usage, you can consider below attributes but not limit of them:
1. What type of games the user has played in the past 7 days?
2. What type of games the user played most frequently?
3. Analyze the user's preference based on the game genres.
4. In the ranking strategy, we do not need to mention the game ID that user has played, since game ID doesnot reflect any game features.

Below is the user play history in the past 7 days, each game is represnted by a unique id with the game profile information.
{user_play_history_str}
\end{lstlisting}

\section{LLM Prompt for Game Reranking}
\label{sec:game_reranking}
\begin{lstlisting}
Given the user's personalized ranking strategy, please rank the following games based on the user's preference.
You can use the following game profile information to rank the games.
The output format MUST be top {ranking_length} game_id list in the order of the ranking WITHOUT any other information.
Here is the user's personalized ranking strategy:
{user_profile}

Here is the game profile information for the games to be ranked:
<Candidate Game Info Start>
{ranking_results_str}
<Candidate Game Info End>
\end{lstlisting}