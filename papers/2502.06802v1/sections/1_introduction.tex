\section{Introduction}

Roblox is a popular online platform where users create and play games designed by other users, resulting in a vast and diverse collection of interactive experiences. To enhance user engagement, Roblox relies on a multi-stage recommendation system that ranks games based on deep neural network (DNN)-driven models, leveraging sparse ID features (such as User ID, Game ID) and dense features derived from user behavior and game statistics. However, these recommendations often fail to capture individual user preferences fully, as they lack content-based signals—particularly game-related text features such as genre and descriptions.

Recent advances in large language models (LLMs) offer new opportunities to improve recommendation systems by leveraging game text data. LLMs can deepen the understanding of game content, enabling not only enhanced personalization but also supporting essential tasks such as game content validation, scam and fraud detection, and age-appropriate recommendations. As illustrated in Figure~\ref{fig:roblox_logo}, in-game text understanding enables a broad range of applications, including explainable ranking, toxicity detection, and conversational recommendations, fostering user trust by providing transparent and engaging interactions. 

Re-ranking techniques refine recommendation lists to better align with user preferences, balancing factors like accuracy, diversity, and personalization~\cite{gao2024llm, carraro2024enhancing}. Recent advancements leverage LLMs, including Chain-of-Thought (CoT) reasoning~\cite{gao2024llm} and instruction-tuning frameworks like RecRanker~\cite{luo2023recranker}, which use enriched prompts to enhance personalization. Transformer-based models further improve quality by modeling item relationships and user preferences holistically~\cite{gao2024llmenhancedrerankingrecommendersystems}. While effective in structured environments, these methods are underexplored in dynamic, unstructured platforms like Roblox. Our work extends these approaches, adapting LLMs to handle Roblox’s noisy game text data, enabling personalized re-ranking in a user-generated ecosystem.

Unlike platforms like Steam~\cite{yang2022large,cheuque2019recommender,pathak2017generating}, where game text is structured and professionally crafted, Roblox faces unique challenges. With accessible tools like Roblox Studio, even young users can create games, resulting in inconsistent and unstructured game text, such as titles and descriptions. This variability, coupled with the rapid influx of new games, complicates the use of text-based features for recommendations. While LLMs excel with structured text, they struggle with Roblox’s unrefined descriptions. Generating high-quality, reliable text features without extensive human annotation is essential for delivering personalized recommendations.

This paper addresses two critical challenges in improving game recommendations on Roblox, where the platform’s user-generated content results in inconsistent and sparse game text features, such as titles and descriptions. This scenario creates a “chicken-and-egg” problem: effective game recommendations require high-quality text features, but without structured or professionally curated descriptions, LLMs struggle to interpret game content—particularly for new or rapidly changing games.

The first challenge is to \textbf{develop a method for generating high-quality, structured text features for Roblox games without extensive human annotation}, which is infeasible given Roblox’s scale and rapid content turnover. The second challenge is to \textbf{establish a framework to validate the quality of these generated text features} to ensure that they enhance recommendation accuracy. Addressing these challenges is essential for building a scalable, content-driven recommendation system that adapts to Roblox's unique dynamics.

To address the first challenge—generating high-quality game text features—we propose a method centered on extracting and understanding raw in-game text. Our approach leverages the fact that developers often guide players with in-game instructions to prevent drop-off. Using LLMs equipped with strong global knowledge, we analyze this in-game text to infer attributes such as game genre, content themes, and player objectives, ultimately constructing a high-quality, structured game profile. To tackle the second challenge—validating the quality of these generated profiles—we introduce an LLM-based re-ranking mechanism. This model integrates the generated text features to validate ranking performance and personalize game recommendations. By re-ranking based on text feature quality, we ensure that our generated profiles effectively improve both recommendation relevance and user satisfaction.
% Roblox is a popular online platform where users create and play games designed by other users, resulting in a vast and diverse collection of interactive experiences. To enhance user engagement, Roblox relies on a multi-stage recommendation system that ranks games based on deep neural network (DNN)-driven models, leveraging sparse ID features (such as User ID, Game ID) and dense features derived from user behavior and game statistics. However, these recommendations do not fully capture individual user preferences, as they lack content-based signals—particularly game-related text features such as genre and descriptions.

% Recent advances in large language models (LLMs) offer new opportunities to improve recommendation systems by leveraging game text data. LLMs can deepen the understanding of game content, enabling not only enhanced personalization but also supporting essential tasks such as game content validation, scam and fraud detection, and age-appropriate recommendations. As illustrated in Figure~\ref{fig:roblox_logo}, in-game text understanding enables a broad range of applications, including explainable ranking, toxicity detection, and conversational recommendations, fostering user trust by providing transparent and engaging interactions.

% However, unlike other gaming platforms like Steam, where game text is structured and professionally crafted, Roblox presents a unique challenge. On Roblox, every user can become a game developer using accessible tools like Roblox Studio, with some developers as young as nine years old. This low barrier to entry promotes creativity but also results in highly variable quality and consistency of game text data, such as titles and descriptions, which are often unstructured or incomplete. This variability, combined with the rapid influx of new games on the platform, poses significant challenges in utilizing text-based features for recommendations. While LLMs are highly effective with well-structured text, they struggle with Roblox’s inconsistent and unrefined game descriptions, particularly those created by younger or non-professional developers. For Roblox to deliver effective and personalized recommendations, generating high-quality, reliable text features without extensive human annotation is crucial.

% This paper addresses two critical challenges in improving game recommendations on Roblox, where the platform’s user-generated content results in inconsistent and sparse game text features, such as titles and descriptions. This scenario creates a “chicken-and-egg” problem: effective game recommendations require high-quality text features, but without structured or professionally curated descriptions, LLMs struggle to interpret game content—particularly for new or rapidly changing games.

% The first challenge is to \textbf{develop a method for generating high-quality, structured text features for Roblox games without extensive human annotation}, which is infeasible given Roblox’s scale and rapid content turnover. The second challenge is to \textbf{establish a framework to validate the quality of these generated text features} to ensure that they enhance recommendation accuracy. Addressing these challenges is essential for building a scalable, content-driven recommendation system that adapts to Roblox's unique dynamics.

% To address the first challenge—generating high-quality game text features—we propose a method centered on extracting and understanding raw in-game text. Our approach leverages the fact that developers often guide players with in-game instructions to prevent drop-off. Using LLMs equipped with strong global knowledge, we analyze this in-game text to infer attributes such as game genre, content themes, and player objectives, ultimately constructing a high-quality, structured game profile. To tackle the second challenge—validating the quality of these generated profiles—we introduce an LLM-based re-ranking mechanism. This model integrates the generated text features to validate ranking performance and personalize game recommendations. By re-ranking based on text feature quality, we ensure that our generated profiles effectively improve both recommendation relevance and user satisfaction.

In summary, the key contributions of this paper are as follows:
\begin{itemize}[leftmargin=*]
\item \textbf{In-Game Text Extraction and Understanding}: We propose a novel approach for generating high-quality game profiles on Roblox by extracting and interpreting raw in-game text, using LLMs to infer game attributes such as genre, content, and play style without human annotation.
\item \textbf{LLM-Based Re-Ranking for Quality Verification}: To validate the effectiveness of generated game profiles, we introduce an LLM-based re-ranking model that incorporates text features into the recommendation system, enhancing ranking personalization and relevance.
\item \textbf{Scalable Framework for Content-Driven Recommendations}: By addressing the challenges of text feature generation and quality validation, this work lays the foundation for scalable, content-based recommendation improvements on Roblox, adaptable to the platform’s dynamic, user-generated ecosystem.
\end{itemize}