\section{Related Works}

% \subsection{Content-Based Recommendation Systems}
Content-based recommendation systems enhance personalization by leveraging textual and content-derived features. Early methods relied on structured metadata, such as product descriptions and user reviews~\cite{degemmis2007content,aciar2007informed}, while recent advances in deep learning extract rich representations from unstructured text~\cite{liu2024once,bondevik2024systematic}. In gaming, studies have explored large-scale game recommendations~\cite{yang2022large}, user preferences in online games~\cite{POLITOWSKI2018103}, and action recommendations in text-based games~\cite{Recommend_Actions_in_Text_Games}. However, these approaches often assume clean, structured input data, which is unavailable on user-generated platforms like Roblox, where text is frequently noisy and inconsistent. Our work addresses this gap by extracting raw in-game text and generating structured profiles to capture genre, objectives, and gameplay mechanics, enabling effective recommendations in noisy environments.
% Content-based recommendation systems leverage textual and other content-derived features to enhance personalization. Early works relied on structured metadata, such as product descriptions and user reviews~\cite{degemmis2007content,aciar2007informed}, focusing on extracting explicit features for recommendations. Recent advancements in deep learning have enabled the extraction of rich representations from unstructured text, addressing more complex scenarios~\cite{liu2024once,bondevik2024systematic}. In the context of gaming, researchers have explored large-scale game recommendations~\cite{yang2022large}, analyzed user preferences in online games~\cite{POLITOWSKI2018103}, and developed methods to recommend actions in text-based games~\cite{Recommend_Actions_in_Text_Games}. These studies highlight the growing interest in using textual data to understand and predict user preferences in gaming environments. Despite these advancements, most content-based approaches assume clean and structured input data, which is often unavailable in user-generated platforms like Roblox. Game descriptions and in-game text on Roblox are frequently noisy, sparse, or inconsistent, making traditional methods less effective. Our work addresses this limitation by focusing on raw in-game text extraction and analysis, generating structured game profiles that capture genre, objectives, and gameplay mechanics.

% \subsection{Re-ranking Techniques for Personalization}
% Re-ranking refines recommendation lists to better align with user preferences and criteria such as diversity, fairness, and personalization. Traditional methods often focus on single objectives but struggle to scale or balance multiple factors effectively~\cite{gao2024llm, carraro2024enhancing}. Recent advancements incorporate LLMs for re-ranking, leveraging Chain-of-Thought reasoning~\cite{gao2024llm} and instruction-tuning techniques like RecRanker~\cite{luo2023recranker}, which enrich prompts with contextual information and hybrid ranking strategies to improve personalization. Transformer-based re-ranking models further optimize entire recommendation lists by modeling item relationships and user preferences~\cite{gao2024llmenhancedrerankingrecommendersystems}. While LLMs have shown potential for diversity and accuracy in re-ranking~\cite{carraro2024enhancing}, their application to noisy, unstructured environments like Roblox remains underexplored. Our work builds on these advancements by adapting LLMs to handle unstructured in-game text, enabling personalized re-ranking in a dynamic, user-generated ecosystem.