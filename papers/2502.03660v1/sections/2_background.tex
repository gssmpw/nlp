\section{Background \& Current Work} \label{sec:background}

Most atomistic ML methods are based on geometric deep learning architectures \citep{duval2023hitchhiker}, which provide a useful framework for encoding inductive biases, such as invariance, equivariance, and other symmetries. Here, we provide an overview of relevant quantum mechanical simulation methods for training data generation, which are graphically depicted in \Cref{fig:jacobsladder} with Jacob's ladder of modeling accuracy. \Cref{app:aimd} details how MLIP inference is coupled with MD simulation for materials modeling.

\begin{figure}[h]
    \vspace{-0.2cm}
    \centering
    \includegraphics[width=\linewidth]{figures/snakes-and-ladders.pdf}
    \vspace{-0.65cm}
    \caption{An artistic interpretation of Jacob's ladder \citep{perdew2005prescription}, extended to include wavefunction methods. Going up the ladders improves both accuracy and precision, but is generally proportional to increased time complexity. We stress that, while there are many nuances associated with method choice for atomistic systems, particularly with multireference methods like CASSCF \citep{roos1980complete}, these accuracy trends are generally observed.}
    \label{fig:jacobsladder}
    \vspace{-0.2cm}
\end{figure}


\textbf{Density Functional Theory (DFT):} DFT  is a widely used quantum mechanical framework to model the behavior of electrons in atomistic systems. Conventionally, DFT aims to solve the non-relativistic electronic Schr\"{o}dinger equation by modeling the electron density from which all other electronic properties can be derived (e.g. potential energy, multipole moments). A hypothetical functional $F$, of the electron density, is estimated by iteratively minimizing its corresponding electronic energy, thereby producing a variational solution. \citet{jonesDensityFunctionalTheory2015} provides a comprehensive review of DFT fundamentals, which primarily rely on the assumption that an exact functional can perfectly describe the electron density for all and any point cloud of atoms in 3D space.  This exact functional is unknown---some regard it as \emph{unknowable} \citep{schuch2009computational}---and in practice is approximated using an exchange-correlation term \citep{kohn1965self}. \footnote{The theoretically exact framework for DFT only holds true for a very small number of cases \citep{mayerConceptualProblemCalculating2017}. In alignment with the literature, we use DFT to refer to the approximation.} Two of the most commonly used functionals include the Perdew-Burke-Ernzerhof functional (PBE) \citep{perdew1996generalized} and Becke, 3-parameter, Lee-Yang-Parr functional (B3LYP) \citep{leeDevelopmentColleSalvettiCorrelationenergy1988,beckeDensityfunctionalExchangeenergyApproximation1988}. These functionals have been applied in many popular ML materials datasets, such as Materials Project \citep{jain2013commentary}, Alexandria \citep{schmidt2024improving}, and OpenCatalyst \citep{chanussot2021open}. Different approximations have subsequently led to families of density functionals for diverse cases. As such, the \emph{choice} of approximations and parameterizations ultimately affects the performance of the functional on specific elements of the periodic table and/or classes of molecules and materials \citep{apra2020nwchem, kuhne2020cp2k, giannozzi2009quantum}. While DFT, which scales $\mathcal{O}(N^3-N^5)$ with $N$ electrons, has been popular for training current MLIPs, it has known inaccuracies, inconsistent results, and other shortcomings as described in \Cref{sec:materials}.


\textbf{Coupled Cluster (CC) Theory:} CC methods \citep{cizekCorrelationProblemAtomic1966} employ highly accurate ``wavefunction'' or \textit{ab initio} quantum mechanical formulations that yield systematically improvable results. CCSD(T) \citep{purvisFullCoupledclusterSingles1982}, which applies single, double, and perturbative triple excitations is regarded as the ``gold standard'' of quantum chemical methods. The hallmarks of CC-based methods is the ability to truncate a Taylor series of excitation operators to systematically choose between computational cost and accuracy---this series converges to the exact solution to the non-relativistic Schr\"{o}dinger equation (\Cref{fig:jacobsladder}) for fixed nuclei, as opposed to DFT methods that involve semi-empirical tuning. Notably, MLIP evaluation today is often done with DFT references while new DFT functionals are benchmarked against CCSD(T), which in turn is usually compared against experimental observables. The computational cost of CCSD(T), which scales $\mathcal{O}(N^7)$ with $N$ electrons, has limited its application to small molecules until very recently \citep{tang2024approaching}, where hybrid methods with MLIPs have enabled large scale, finite-temperature simulations of periodic systems with CC quality \citep{herzogCoupledClusterFinite2024}.


