\section{Introduction}

Machine learning (ML) opens up the possibility to greatly accelerate materials discovery for a vast range of applications, including the development of energy technologies to mitigate climate change, ubiqutuous computing technologies, and sustainable agriculture and manufacturing. 
The complexity of materials systems in modern applications continues to make empirical testing of devices, and their underlying materials, more and more challenging. As such, insilico materials simulation methods have become increasingly important to understand the properties and behavior of an growing set of diverse materials systems, ranging from solid-state crystals, molecules, and protein structures to name a few \citep{zeni2025generative, miret2024perspective, zitnick2020introduction, lee2023towards, terwilliger2024alphafold}. To fully realize the potential of insilico evaluation, it is necessary to consider how complex materials, which are comprised of large numbers of atoms governed by quantum mechanical laws, behave in diverse application conditions. Concretely, material systems are often part of multi-material devices that contain defects and imperfections while operating under various temperatures, pressures and other exogenous conditions. Therefore, similarly to how biological applications motivate the building of a virtual cell \citep{bunne2024build}, materials applications require the building of virtual devices (e.g. transistors, battery cells, nano-sized air filters) to accurately capture atomic interactions with quantum mechanical accuracy in complex environments. 


\begin{figure*}[t]
    \centering
    \includegraphics[width=0.85\textwidth]{figures/MLIP-Position-Fig1-smaller.pdf}
    \vspace{-0.80cm}
    \caption{Overview of Machine Learning Interatomic Potentials (MLIP) requirements for device scale modeling. Current research focuses mainly on bulk structures in ideal conditions with regression-based training and error metric evaluation. To enable materials foundation models, we require higher quality training datasets that use more accurate simulation methods like Coupled Cluster Theory. MLIPs, in turn, should be evaluated as part of atomistic simulations of real-world materials in application-informed conditions (e.g., defects, standard temperature \& pressure, etc.). To reach MLIP-accelerated device scale modeling with quantum mechanical accuracy, we require new datasets and evaluation methods for complex materials systems in modern devices (e.g., 2D Material Transistors with multiple layers of distinct materials with specific functions), computational acceleration for proper inference, and comprehensive MLIP metrology.}
    \label{fig:mlip-req}
    \vspace{-0.50cm}
\end{figure*}


To date, the computational evaluation of materials systems with quantum accurate methods has mostly focused on applying approximations, such as Density Functional Theory (DFT), to idealized bulk structure materials \citep{jain2011high, yang2019predicting, saal2013materials, horton2019high, doerr2016htmd, miret2023the}. While these efforts have significantly improved the understanding of diverse materials, their applicability has been hindered by prohibitively high computational costs that scale exponentially with the number of atoms in the system. Modern ML methods for atomistic modeling, mainly based on geometric deep learning \citep{duval2023hitchhiker}, have emerged as a promising alternative to traditional computational chemistry pipelines. These ML methods boast the ability to model large material systems with quantum accuracy at constant compute cost following model training. This has led to the development of machine learning interatomic potentials (MLIPs), which are trained on large amounts of quantum simulation data. MLIPs aim to approximate the potential energy of diverse materials in a fast and accurate manner. In recent years, an increasing number of datasets \citep{chanussot2021open, tran2023open, barroso2024open, lee2023matsciml, schmidt2024improving, wang2024perovs, fuemmeler2024advancing} and models \citep{batatia2023foundation, neumann2024orb, yang2024mattersim, chen2022universal, deng2023chgnet} have been released, each claiming various degrees of generalizability for atomistic modeling of materials. While ongoing research has shown initial promise, significant work remains to realize the full potential of MLIPs in enabling physically accurate materials simulation at device scale.
 

We believe that for MLIPs to have real-world impact, current research priorities need to shift away from current benchmarks \citep{lee2023matsciml, riebesell2023matbench, choudhary2020joint, chanussot2021open} and place greater focus on enabling \emph{device scale} simulations.
Device scale simulation requires new algorithms beyond current training and inference methods that have mainly focused on predicting DFT trajectories for idealized bulk materials. Concretely, coupling large-scale MLIP inference with MD simulations can lead to faster simulations of larger, more complex materials in realistic application conditions. This in turn would allow practitioners to exctract more actionable insights from MLIP-driven simulations. To realize this interdisciplinary vision, we propose the following directions: 

\emph{1. Higher Accuracy MLIP Training Data Creation} 

As described in \Cref{sec:materials}, we require dedicated and collective efforts to create datasets and benchmarks that better represent the complexity of real-world materials applications. These new datasets should be both chemically and structurally diverse, and cover a wide range of physical conditions (e.g. temperatures, pressures, defects) and phase changes (e.g. solid-to-liquid, liquid-to-gas). 
Data generation methods should be reproducible and verifiable for both simulation and experimentally measured data. As we discuss in \Cref{sec:materials}, nuances with popular DFT approximations, methodologies, and implementations lend themselves to large variances in simulation results, thereby making MLIPs trained solely on DFT data less reliable. To improve training data quality, we recommend using higher accuracy simulation methods like Coupled Cluster Theory.

\emph{2. MLIP Metrology -- Interpretable and Materials Science Informed MLIP Evaluation}

As described in \Cref{sec:eval}, MLIP evaluation methods should capture relevant behavior for MLIPs applied to MD simulations that model experimentally measured properties. We also require MLIP metrology methods that enable better interpretability of MLIPs to highlight their capabilities and limitations. Taken together, these efforts will enable greater understanding of MLIP methods and utilities, while also fostering trust in the materials science community.

\emph{3. Computationally Efficient MLIP Inference Workflows for Device-Scale Materials Simulation}

As described in \Cref{sec:computation}, MLIPs require efficient inference when deployed in MD simulations to disrupt traditional simulation methods and unlock the promise of device scale modeling. Faster inference will enable MLIPs to model larger and more chemically complex systems compared to classical methods. However, this will require significant engineering advances, including: re-usable computation kernels, optimized architectures, and scalable inference methods deployed in simulation software stacks.
