\section{Related Work}
\subsection{Time-Series Modeling}
Models like \textit{LSTMs} and \textit{Transformers} with attention mechanisms have been successfully applied to time-series anomaly detection in fields such as mechanical enginered system, aerospace, and industrial monitoring \cite{lachekhab_lstm-autoencoder_2024,liu_deep_2023,wei_lstm-autoencoder-based_2023}. While these methods effectively capture long-range dependencies and irregular temporal patterns, they are computationally expensive. For example, common benchmark datasets require time steps ranging from 30 to 500 with hidden dimensions between 20 and 50, illustrating the complexity of modeling system dynamics efficiently \cite{mahmoud_ae-lstm_2022}. Their high memory requirements often exceed the constraints of typical MCU-level devices, making real-time deployment infeasible For example, a typical LSTM with 500 time steps and 50 hidden units already surpasses the available memory of common low-power devices, limiting its practical applicability. Additionally, the sequential nature of RNNs restricts parallelization, further increasing computational cost \cite{rezk_recurrent_2020}. These challenges highlight the need for alternative approaches that balance computational efficiency with robust anomaly detection, aligning with broader goals of sustainability and practical deployability. \par
\subsection{Physics-Informed Methods}
In many applications, incorporating domain knowledge can help reduce computational demands. Physics-informed neural networks (PINNs) embed physical laws directly into neural networks, enabling them to leverage known system dynamics \cite{cai_physics-informed_2021,raissi_physics-informed_2019,raissi_hidden_2020}. However, applying PINNs to complex dynamical physical systems remains challenging. These systems involve numerous interdependent physical principles, and explicitly modeling them within a neural network would require extensive computational resources and domain expertise, making large-scale applications impractical \cite{karniadakis_physics-informed_2021,wang_when_2022}.

\subsection{Benchmark Use Case: Turbofan Aeroengine Degradation}
We use the C-MAPSS dataset as a benchmark for studying complex dynamical systems due to its realistic representation of turbofan aeroengine degradation. Aeroengines are complex dynamical systems governed by nonlinear, time-dependent interactions of physical processes, making them an ideal test case for evaluating anomaly detection methods in real-world settings.\par

In the literature, anomaly detection for the C-MAPSS dataset generally follows two main approaches. The first focuses on fleet-level anomaly detection, where engines with shorter lifetimes are classified as abnormal based on their total life cycles \cite{jakubowski_anomaly_2022,yildirim_enhancing_2024}. This method aims to distinguish early failures from normal operating conditions at a system-wide level.

The second approach considers individual engine degradation, defining anomalies based on a 60/40 time-based split \cite{bataineh_autoencoder_2020,zhu_anomaly_2024}. In this setup, the first 60\% of an engine’s life is labeled as normal, while the final 40\% is considered abnormal. The dataset is then divided into train/test subsets, and models are evaluated based on their ability to classify each time step accordingly.

Since we aim to develop methods that capture system dynamics, we consider the second approach a more suitable benchmark, as it focuses on time-dependent degradation rather than static fleet-level classification.

Previous work on anomaly detection in this setting has explored various deep learning models. One study employed an LSTM-based approach \cite{zhu_anomaly_2024}, leveraging recurrent structures to model time dependencies. Another approach used a standard autoencoder (AE) without explicit temporal modeling \cite{bataineh_autoencoder_2020}. More recent research has investigated Transformer-based models, which excel at capturing long-range dependencies but introduce high computational costs and require extensive training data due to their large parameter space \cite{inproceedings,liu_deep_2023}. Moreover, both Transformer studies formulated the problem as a multiclass classification task, with one using a slightly different dataset and the other applying the method directly to C-MAPSS. However, defining well-separated fault categories in a complex dynamical system is challenging in practice, as non-trivial interactions between multiple physical components create highly unpredictable behaviors. As system dynamics grow more intricate, these interactions become even less predictable, further complicating precise fault categorization.
\subsection{Proposed Approach}

Instead of relying on overly complex models that attempt to directly differentiate between normal and abnormal behavior, we propose a system-theoretic approach inspired by \textbf{classical embedding theory} to study the dimensionality of the system's latent representation. Our approach introduces \textbf{physics-inspired consistency principles} that approximate the underlying causal mechanisms governing the system dynamics, without explicitly enforcing physical laws. We hypothesize that complex systems in a stable regime exhibit predictable behavior, allowing for well-approximated lower-dimensional embeddings. In contrast, anomalies introduce additional complexity, disrupting these stable relationships. We aim to leverage this property to detect anomalous states of the system when the learned embedding no longer adequately captures the system dynamics.