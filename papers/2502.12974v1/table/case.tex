\begin{table*}[t]
\centering
\small
\begin{tabular}{lp{12.5cm}} % 第二列宽度设置为 10cm
\hline
\textbf{Query} & In the \textcolor{red}{US}, is it a good idea to \textcolor{red}{hire a tax consultant} for doing \textcolor{red}{taxes}? \\
\hline
\textbf{MiniCPM} & This may not exactly answer your question but, as a small business owner, \textcolor{blue}{I would highly recommend} having a professional handle your \textcolor{blue}{taxes}. It is worth the money to have it done correctly rather than doing something wrong and getting audited or worse having penalties assessed and owing more than you thought would be possible. I would recommend this especially if this is how you make your primary income, you can always write it off as a business expense.\\
\hline
\textbf{\method{}} & Whether you do decide to go with a tax advisor or not, be sure to do some research on your own. When we moved to the \textcolor{orange}{US} about 5 years ago, I did find the taxes here pretty complicated and confusing \textbf{···} After all, they are also humans prone to mistakes and your taxes are your liability in the end.  \textcolor{orange}{My suggestion is to start with a good tool that supports tax filing for non-residents.} Most of them provide a step-by-step QA based tool. As you go through the steps, Google each question you don't understand. \textcolor{orange}{It may take more time than hiring a tax advisor directly but in the end it will all be worth it.} \\
\hline
\textbf{Query} &What is the \textcolor{red}{mechanism of inflammatory response} and \textcolor{red}{pathogenesis of COVID-19} cases?\\
\hline
\textbf{MiniCPM}&The novel coronavirus disease (COVID-19) pandemic is placing significant strains on health systems, scientific communities, essential public services, and economies all over the world. In this context, the worlds scientific biomedical establishment is unleashing an unprecedented response to the COVID-19 pandemic. This is a battle against time, considering the thousands of human lives are lost every day. In this commentary, based on a very recent research report, we intend to highlight \textcolor{blue}{how a new mechanism describing the RAGE transactivation produced by Ang II-mediated ATR1 activation can run continuously and thus, reinforcing a sustained inflammation in lungs, due to the SARS-Cov-2-mediated imbalance of the ACE/And II/ATR1 pathway.}\\
\hline
\textbf{\method{}}&The evidence on the pathophysiology of the novel coronavirus SARS-CoV-2 infection is rapidly growing. Understanding why some patients suffering from COVID-19 are getting so sick, while others are not, has become an informal imperative for researchers and clinicians around the globe. The answer to this question would allow rationalizing the fear surrounding this pandemic. \textcolor{orange}{Understanding of the pathophysiology of COVID-19 relies on an understanding of interplaying mechanisms, including SARS-CoV-2 \textcolor{orange}{virulence}, human immune response, and complex inflammatory reactions with coagulation playing a major role.} An interplay with bacterial co-infections, as well as the vascular system and microcirculation affected throughout the body should also be examined. More importantly, a comprehensive understanding of pathological mechanisms of COVID-19 will increase the efficacy of therapy and decrease mortality. Herewith, presented is the current state of knowledge on COVID-19: \textcolor{orange}{beginning from the virus, its transmission, and mechanisms of entry into the human body, through the pathological effects on the cellular level, up to immunological reaction, systemic and organ presentation.} Last but not least, currently available and possible future therapeutic and diagnostic options are briefly commented on.\\
\hline
\end{tabular}
\caption{Case Studies. We present two cases from FiQA and TREC-COVID, and show the top 1 passage retrieved from MiniCPM and \method{}, with the key contents in the passage highlighted.}
\label{tab:case_study}
\end{table*}