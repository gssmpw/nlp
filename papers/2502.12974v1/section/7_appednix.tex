\section{Appendix}
\subsection{Licenses}
The E5 dataset and BEIR benchmark are released under the Apache License 2.0.
The terms of use can be found on their Github pages.
This license allows users to freely use, modify, and distribute the data, permitting academic use of their dataset.

\subsection{More Details of Datasets Used in Experiments}
In this section, we provide a detailed description of the training dataset and the evaluation benchmark for our \method{}.

\textbf{E5 Dataset}.
We employ the E5 dataset~\cite{wang2023improving, springer2024repetition} for training, comprising a collection of publicly available datasets. The E5 dataset is carefully curated to encompass a wide range of retrieval scenarios and tasks. We present the statistics of the E5 dataset in Table~\ref{tab:train_dataset_details}. The instructions used in the E5 dataset are shown in Table~\ref{tab:dataset_descriptions}. The full dataset can be available at their website\footnote{\url{https://github.com/jakespringer/echo-embeddings}}.

\textbf{BEIR Benchmark}.
We evaluate our \method{} models using the BEIR benchmark, which includes 18 datasets from 9 heterogeneous retrieval tasks. We focus on the 14 publicly available datasets and use the standard instructions in MTEB~\cite{muennighoff2022mteb}. The statistics for these datasets are provided in Table~\ref{tab:dataset_stats}, and the instructions can be found in Table~\ref{tab:retrieval_dataset}.

\begin{table*}[t]
\centering
\small
\begin{tabular}{l|r|r} % 只保留两列
\hline
\textbf{Dataset} & \textbf{\#Samples} & \textbf{Proportion} \\ \hline
ELI5 \cite{fan2019eli5longformquestion} & 32,547&2.16\% \\
HotpotQA \cite{yang2018hotpotqadatasetdiverseexplainable} &90,447&5.99\% \\
FEVER \cite{thorne-etal-2018-fever} & 101,578&6.73\% \\
MIRACL \cite{zhang-etal-2023-miracl} & 32,561& 2.16\%\\
MSMARCO Passage Ranking \cite{bajaj2018msmarcohumangenerated} & 249,592&16.53\% \\
MSMARCO Document Ranking \cite{bajaj2018msmarcohumangenerated} & 73,400&4.86\% \\
NQ \cite{kwiatkowski-etal-2019-natural} & 100,231&6.64\% \\
NLI \cite{gao-etal-2021-simcse} & 277,230&18.36\% \\
SQuAD \cite{rajpurkar-etal-2016-squad} & 87,599&5.80\% \\
TriviaQA \cite{joshi2017triviaqa} & 73,346&4.86\% \\
Quora Duplicate Questions \cite{quora-question-pairs} & 101,762&6.74\% \\
Mr-TyDi \cite{zhang-etal-2021-mr} & 48,715&3.23\% \\
DuReader \cite{he-etal-2018-dureader} & 86,395 &5.72\% \\
T2Ranking \cite{10.1145/3539618.3591874} & 154,294&10.22\% \\ \hline
\end{tabular}
\caption{Data Statistics of E5 Dataset. We show the composition and distribution of E5 Dataset.}
\label{tab:train_dataset_details}
\end{table*}
\begin{table*}[ht]
    \footnotesize
    \centering
    \renewcommand{\arraystretch}{1.1} % Adjusts the row spacing
    \resizebox{16cm}{!} 
    { 
    \begin{tblr}{hline{1,2,Z} = 0.8pt, hline{3-Y} = 0.2pt,
                 colspec = {Q[l,m, 13em] Q[l,m, 6em] Q[c,m, 8em] Q[c,m, 5em] Q[l,m, 14em]},
                 colsep  = 4pt,
                 row{1}  = {0.4cm, font=\bfseries, bg=gray!30},
                 row{2-Z} = {0.2cm},
                 }
\textbf{Dataset}       & \textbf{Table Source} & \textbf{\# Tables / Statements} & \textbf{\# Words / Statement} & \textbf{Explicit Control}\\ 
\SetCell[c=5]{c} \textit{Single-sentence Table-to-Text}\\
ToTTo \cite{parikh2020tottocontrolledtabletotextgeneration}   & Wikipedia        & 83,141 / 83,141                  & 17.4                          & Table region      \\
LOGICNLG \cite{chen2020logicalnaturallanguagegeneration} & Wikipedia        & 7,392 / 36,960                  & 14.2                          & Table regions      \\ 
HiTab \cite{cheng-etal-2022-hitab}   & Statistics web   & 3,597 / 10,672                  & 16.4                          & Table regions \& reasoning operator \\ 
\SetCell[c=5]{c} \textit{Generic Table Summarization}\\
ROTOWIRE \cite{wiseman2017challengesdatatodocumentgeneration} & NBA games      & 4,953 / 4,953                   & 337.1                         & \textbf{\textit{X}}                   \\
SciGen \cite{moosavi2021scigen} & Sci-Paper      & 1,338 / 1,338                   & 116.0                         & \textbf{\textit{X}}                   \\
NumericNLG \cite{suadaa-etal-2021-towards} & Sci-Paper   & 1,355 / 1,355                   & 94.2                          & \textbf{\textit{X}}                    \\
\SetCell[c=5]{c} \textit{Table Question Answering}\\
FeTaQA \cite{nan2021fetaqafreeformtablequestion}     & Wikipedia      & 10,330 / 10,330                 & 18.9                          & Queries rewritten from ToTTo \\
\SetCell[c=5]{c} \textit{Query-Focused Table Summarization}\\
QTSumm \cite{zhao2023qtsummqueryfocusedsummarizationtabular}                        & Wikipedia      & 2,934 / 7,111                   & 68.0                          & Queries from real-world scenarios\\ 
\textbf{eC-Tab2Text} (\textit{ours})                           & e-Commerce products      & 1,452 / 3,354                   & 56.61                          & Queries from e-commerce products\\
    \end{tblr}
    }
\caption{Comparison between \textbf{eC-Tab2Text} (\textit{ours}) and existing table-to-text generation datasets. Statements and queries are used interchangeably. Our dataset specifically comprises tables from the e-commerce domain.}
\label{tab:datasets}
\end{table*}
\begin{table*}[t]
\centering
\small
\resizebox{\linewidth}{!}{
\begin{tabular}{ll}
\hline
\textbf{Datasets} & \textbf{Instructions} \\
\hline
DuReader & Given a Chinese search query, retrieve web passages that answer the question. \\
\hline
ELI5 & Provided a user question, retrieve the highest voted answers on the Reddit ELI5 forum. \\
\hline

FEVER & Given a claim, retrieve documents that support or refute the claim. \\
\hline

HotpotQA & Given a multi-hop question, retrieve documents that can help answer the question. \\
\hline

MIRACL & Given a question, retrieve Wikipedia passages that answer the question. \\
\hline

MrTyDi & Given a question, retrieve Wikipedia passages that answer the question. \\
\hline

MSMARCO Passage & Given a web search query, retrieve relevant passages that answer the query. \\
\hline
MSMARCO Document & Given a web search query, retrieve relevant documents that answer the query. \\
\hline
NQ & Given a question, retrieve Wikipedia passages that answer the question. \\
\hline
Squad & Retrieve Wikipedia passages that answer the question. \\
\hline
T2Ranking & Given a Chinese search query, retrieve web passages that answer the question. \\
\hline
TriviaQA & Retrieve Wikipedia passages that answer the question. \\
\hline
\multirow{2}{*}{QuoraDuplicates} & Given a question, retrieve questions that are semantically equivalent to the given question. \\
& Find questions that have the same meaning as the input question. \\
\hline
\multirow{2}{*}{NLI} & Given a premise, retrieve a hypothesis that is entailed by the premise. \\
& Retrieve semantically similar text. \\
\hline
\end{tabular}}
\caption{Instructions Used for E5 Dataset.}
\label{tab:dataset_descriptions}
\end{table*}

\begin{table*}[t]
\centering
\small
\begin{tabular}{l|p{11cm}}
\hline
\textbf{Task Name} & \textbf{Instruction} \\
\hline
Arguana & Given a claim, find documents that refute the claim \\
\hline
FiQA2018 & Given a financial question, retrieve user replies that best answer the question \\
\hline

NFCorpus & Given a question, retrieve relevant documents that best answer the question \\
\hline

Quora & Given a question, retrieve questions that are semantically equivalent to the given question \\
\hline

SCIDOCS & Given a scientific paper title, retrieve paper abstracts that are cited by the given paper \\
\hline

TREC-COVID & Given a query on COVID-19, retrieve documents that answer the query \\
\hline

Touche2020 & Given a question, retrieve detailed and persuasive arguments that answer the question \\
\hline
ClimateFEVER & Given a claim about climate change, retrieve documents that support or refute the claim \\
\hline
FEVER & Given a claim, retrieve documents that support or refute the claim \\
\hline
HotpotQA & Given a multi-hop question, retrieve documents that can help answer the question \\
\hline
DBPedia &  Given a query, retrieve relevant entity descriptions from DBPedia\\
\hline
NQ & Given a question, retrieve Wikipedia passages that answer the question \\
\hline
MS MARCO & Given a web search query, retrieve relevant passages that answer the query \\
\hline
\multirow{2}{*}{CQADupStack} & Given a question, retrieve detailed question descriptions from Stackexchange that are duplicates to the given question \\

\hline
\end{tabular}
\caption{Instructions Used for Evaluation on the BEIR Benchmark.}
\label{tab:retrieval_dataset}
\end{table*}


\label{app:dataset}


\subsection{Case Study}
\label{app:case_study}
\section{Case Study}\label{sec:appendix_casestudy}

We present cases across multiple coding datasets, comparing compressed and original code examples. For instance, as demonstrated in Figures 8, 9, and 10, \ourtool prioritizes discarding \textbf{Invocation} tokens first, followed by \textbf{Symbol} tokens.
\begin{figure}[!h]
\begin{tcolorbox}
\begin{lstlisting}[language=Java,frame=single,framerule=0pt]
### FOCAL_METHOD 
getProduction(java.lang.String) { 
 return productionsByName.get(name); }  
### UNIT_TEST  
testJustifications() { 
 runTest("testJustifications", 2); org.jsoar.kernel.Production j = agent.getProductions() .getProduction("justification-1"); "<AssertPlaceHolder>"; 
}    
\end{lstlisting}
\end{tcolorbox}
\caption{Original Code Examples of Assertion Generation (63 tokens)}
\label{fig:code-example}
\end{figure}

\begin{figure}[!h]
\begin{tcolorbox}
\begin{lstlisting}[language=Java,frame=single,framerule=0pt]
### FOCAL_METHOD 
getProduction(java.lang.String) { 
 return productionsByName; }  
### UNIT_TEST  
testJustifications() { 
 ; 
 org.jsoar.kernel.Production j = agent.getProductions() .getProduction("justification-1"); "<AssertPlaceHolder>"; 
}    
\end{lstlisting}
\end{tcolorbox}
\caption{Compressed Code Examples of Assertion Generation (55 tokens, $\tau_{code}$: 0.1)}
\label{fig:code-example}
\end{figure}

\begin{figure}[!h]
\begin{tcolorbox}
\begin{lstlisting}[language=Java,frame=single,framerule=0pt]
### FOCAL_METHOD 
getProduction(java.lang.String)  
 return productionsByName;     
### UNIT_TEST  
testJustifications()  
 ; 
 org.jsoar.kernel.Production j = agent;
  "<AssertPlaceHolder>"; 
\end{lstlisting}
\end{tcolorbox}
\caption{Compressed Code Examples of Assertion Generation (39 tokens, $\tau_{code}$: 0.4)}
\label{fig:code-example}
\end{figure}


\begin{figure}[!h]
\begin{tcolorbox}
\begin{lstlisting}[language=Java,frame=single,framerule=0pt]
### BUGGY_CODE 
public static TYPE_1 init(java.lang.String name, java.util.Date date) {
   TYPE_1 VAR_1 = new TYPE_1();
   VAR_1.METHOD_1(name);
   java.util.Calendar VAR_2 = java.util.Calendar.getInstance();
   VAR_2.METHOD_2(date);
   VAR_1.METHOD_3(VAR_2);
   return VAR_1;
}
### FIXED_CODE   
public static TYPE_1 init(java.lang.String name, java.util.Date date) {
   TYPE_1 VAR_1 = new TYPE_1();
   VAR_1.METHOD_1(name);
   java.util.Calendar VAR_2 = null;
   if (date != null) {
       VAR_2 = java.util.Calendar.getInstance();
       VAR_2.METHOD_2(date);
   } 
   VAR_1.METHOD_3(VAR_2);
   return VAR_1;
}
\end{lstlisting}
\end{tcolorbox}
\caption{Original Code Examples of Bugs2Fix (195 tokens)}
\label{fig:code-example}
\end{figure}

\begin{figure}[!h]
\begin{tcolorbox}
\begin{lstlisting}[language=Java,frame=single,framerule=0pt]
### BUGGY_CODE 
public static TYPE_1 init(java.lang.String name, java.util.Date date) {
    = new TYPE_1();
   ;
   java.util.Calendar = java.util.Calendar;
   .METHOD_2(date);
   .METHOD_3(VAR_2);
   return ;
}
### FIXED_CODE   
public static TYPE_1 init(java.lang.String name, java.util.Date date) {
    = new TYPE_1();
   ;
   java.util.Calendar = null;
   if (date != null) {
        = java.util.Calendar;
       .METHOD_2(date);
   } 
   .METHOD_3(VAR_2);
   return ;
}
\end{lstlisting}
\end{tcolorbox}
\caption{Compressed Code Examples of Bugs2Fix (136 tokens, $\tau_{code}$: 0.3)}
\label{fig:code-example}
\end{figure}

\begin{figure}[!h]
\begin{tcolorbox}
\begin{lstlisting}[language=Java,frame=single,framerule=0pt]
### METHOD_HEADER 
protected final void fastPathEmit ( U value , boolean delayError , Disposable dispose )
### WHOLE_METHOD  
protected final void fastPathEmit(U value, boolean delayError, Disposable dispose) {
   final Observer<? super V> s = actual;
   final SimplePlainQueue<U> q = queue;
   if (wip.get() == 0 && wip.compareAndSet(0, 1)) {
       accept(s, value);
       if (leave(-1) == 0) {
           return;
       }
   } else {
       q.offer(value);
       if (!enter()) {
           return;
       }
   }
   QueueDrainHelper.drainLoop(q, s, delayError, dispose, this);
}
\end{lstlisting}
\end{tcolorbox}
\caption{Original Code Examples  of \taskthree (157 tokens, $\tau_{code}$: 0.3)}
\label{fig:code-example}
\end{figure}


\begin{figure}[!h]
\begin{tcolorbox}
Original Code Examples (121 tokens, $\tau_{code}$: 0.3)
\begin{lstlisting}[language=Java,frame=single,framerule=0pt]
### METHOD_HEADER 
protected final void fastPathEmit ( U value , boolean delayError , Disposable dispose )
### WHOLE_METHOD  
   final Observer<? super V> = 
   final SimplePlainQueue<U> = 
   if (wip.get() == 0 && wip.compareAndSet(0, 1)) 
       ;
       if (leave(-1) == 0) 
           return;    
    else 
       .offer(value);
       if (!enter()) 
           return;
   .drainLoop(q, s, delayError, dispose, this);
\end{lstlisting}
\end{tcolorbox}
\caption{Compressed Code Examples  of \taskthree (121 tokens, $\tau_{code}$: 0.3)}
\end{figure}



%\begin{figure}[t]
    \centering
    \includegraphics[width=1\linewidth]{image/case_hot.png}
    \caption{An Example of Similarity Scores Between All Embeddings.}
    \label{fig:case_hot}
\end{figure}

In this subsection, we present two case studies to demonstrate the effectiveness of \method{}. Specifically, we show the top 1 retrieved document comes from both MiniCPM and \method{} in the FiQA and Trec-COVID datasets. The retrieval cases are shown in Table~\ref{tab:case_study}.
%Then, we plot a detailed similarity relationship for one case to explore how embeddings at various stages affect the final representation used for retrieval.
%\textbf{Retrieval Cases.}

For the query ``In the US, is it a good idea to hire a tax consultant for doing taxes?'', while the retrieval result of MiniCPM suggests hiring a professional, it fails to mention the specific context of ``In the US''. 
In contrast, the document retrieved by \method{} aligns more closely with the query, addressing all the relevant aspects. This highlights that \method{} effectively activates the reasoning capabilities of LLMs, enabling a more fine-grained document representation.

For the query ``What is the mechanism of inflammatory response and pathogenesis of COVID-19 cases?'', MiniCPM retrieves documents containing more pathological terms, such as ``RAGE transactivation'' and ``the ACE/Ang II/ATR1 pathway''. 
However, these retrieved documents cannot accurately address the inflammatory response and pathogenesis of COVID-19. This suggests that MiniCPM lacks a fine-grained understanding of document content, leading to suboptimal retrieval performance. In contrast, \method{} provides accurate retrieval results, demonstrating the importance of deliberate thinking before searching.

%\textbf{Embedding Analysis.} 
%We show a heatmap of detailed similarity relationships for one case in Figure~\ref{fig:case_hot}. The similarity relationships between neighboring thinking steps indicating stepwise learning behavior of Chain-of-Deliberation. 


\subsection{Comparison with Additional Baseline Retrievers}
\label{app:comparison}
In this subsection, we conduct additional comparisons with baseline retrievers not discussed in the main section, including several advanced retrievers from the BEIR benchmark.

Specifically, we use CPT~\cite{neelakantan2022text}, Udever~\cite{zhang2023languagemodelsuniversalembedders}, ULLME, Ada2, and Promptriever as our additional retrievers to compare with \method{}.
CPT refers to a series of large encoder models developed by OpenAI \cite{brown2020language}, which are pre-trained on large-scale and unsupervised data. We use the CPT-text-175B model as our baseline model.   
Udever contrastively trains the BLOOM~\cite{workshop2023bloom176bparameteropenaccessmultilingual} to build dense retrievers, enabling it to effectively generate aligned multilingual embeddings for both text and code. 
ULLME~\cite{man-etal-2024-ullme} enables bidirectional attention between LLMs and enhances them for text embedding using Generative Reinforcement Learning (GRL).
Ada2 is a versatile text embedding model introduced by OpenAI. The evaluation result for Ada2 comes from~\citet{kamalloo-etal-2023-evaluating-embedding}. 
Promptriever~\cite{weller2025promptriever} is trained on an instance-level instruction dataset from MS MARCO, enhancing its ability to follow instructions for retrieval tasks.

The performance comparison results on the BEIR benchmark are shown in Table~\ref{tab:other_baseline}. 
Compared with these baseline models, \method{} still exhibits competitive performance, demonstrating its effectiveness. Compared to Udever and ULLME, \method{} shows significant improvement on FEVER and Climate-FEVER, indicating its effectiveness across diverse fact-checking scenarios. 
When compared to Promptriever, \method{} demonstrates similar retrieval performance. However, unlike Promptriever, \method{} does not rely on synthetic data for training its retrieval models.

\begin{table*}[t]
\centering
\small
\begin{tabular}{l|c|c|c|c|c|cc}
    \hline
    \textbf{Method} ($\rightarrow$) &\textbf{CPT}& \textbf{Udever} &  \textbf{ULLME} & \textbf{Ada2} & \textbf{Promptriever} & \multicolumn{2}{c}{\textbf{\method{}}} \\
    \textbf{Model Size} ($\rightarrow$) & 175B& 7B & 7B & unk. & 7B & 2.4B & 4B \\
    \hline
    TREC-COVID$^{\dagger}$  &0.649 &0.838 &0.836 &0.813 &0.839 & 0.795 & 0.836 \\
    NFCorpus$^{\dagger}$  &0.407 &0.360 &0.394 &0.358 &0.365 & 0.378 & 0.399 \\
    NQ &/ &0.533 &0.614 &0.482&0.619 &0.560 & 0.561 \\
    HotpotQA$^{\dagger}$  &0.688 &0.567 &0.674 &0.654 & 0.692&0.678 & 0.678 \\
    FiQA$^{\dagger}$  &0.512 &0.367 &0.423 &0.411 &0.459 & 0.434 & 0.462 \\
    ArguAna$^{\dagger}$  &0.435 &0.522 &0.468 &0.567 &0.518  & 0.567 & 0.562 \\
    Touché-2020$^{\dagger}$  &0.291 &0.252 &0.271 &0.280 &0.314 & 0.211 & 0.250 \\
    Quora$^{\dagger}$  &0.638 &0.883 &0.878 &0.876&0.865  & 0.886 & 0.886 \\
    DBPedia$^{\dagger}$  &0.408 &0.376 &0.464&0.402 &0.450 & 0.430 & 0.432 \\
    SCIDOCS &/ &0.189 &0.211 &0.186 &0.173  & 0.197 & 0.212 \\
    FEVER$^{\dagger}$  &0.775 &0.740 &0.615 & 0.773&0.828 & 0.859 & 0.857 \\
    Climate-FEVER$^{\dagger}$  &0.223 &0.268 &0.222 &0.237 &0.276 & 0.303 & 0.294 \\
    SciFact$^{\dagger}$  &0.754 &0.693 &0.724 &0.736 &0.750  & 0.735 & 0.743 \\
   % MS MARCO &/ &0.420 &0.357 & /& / &0.367 & 0.391 \\
    CQADupStack &/ &0.393 &/ &0.391 & / & 0.431 & 0.428 \\
    \hline
    \textbf{Avg CPT sub$^{\dagger}$ } &0.525 &0.533 &0.543 &0.555 &0.571 &0.571&0.582\\
    \textbf{Avg} &/ &0.499 &/ &/ &/  &0.533 & 0.543 \\
    \hline
\end{tabular}
\caption{Retrieval Performances of \method{} and Other Baseline Models on the BEIR Benchmark.} 
\label{tab:other_baseline}
\end{table*}

