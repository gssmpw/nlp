\section{Appendix}
\subsection{Licenses}
The E5 dataset and BEIR benchmark are released under the Apache License 2.0.
The terms of use can be found on their Github pages.
This license allows users to freely use, modify, and distribute the data, permitting academic use of their dataset.

\subsection{More Details of Datasets Used in Experiments}
In this section, we provide a detailed description of the training dataset and the evaluation benchmark for our \method{}.

\textbf{E5 Dataset}.
We employ the E5 dataset~\cite{wang2023improving, springer2024repetition} for training, comprising a collection of publicly available datasets. The E5 dataset is carefully curated to encompass a wide range of retrieval scenarios and tasks. We present the statistics of the E5 dataset in Table~\ref{tab:train_dataset_details}. The instructions used in the E5 dataset are shown in Table~\ref{tab:dataset_descriptions}. The full dataset can be available at their website\footnote{\url{https://github.com/jakespringer/echo-embeddings}}.

\textbf{BEIR Benchmark}.
We evaluate our \method{} models using the BEIR benchmark, which includes 18 datasets from 9 heterogeneous retrieval tasks. We focus on the 14 publicly available datasets and use the standard instructions in MTEB~\cite{muennighoff2022mteb}. The statistics for these datasets are provided in Table~\ref{tab:dataset_stats}, and the instructions can be found in Table~\ref{tab:retrieval_dataset}.

\begin{table*}[t]
\centering
\small
\begin{tabular}{l|r|r} % 只保留两列
\hline
\textbf{Dataset} & \textbf{\#Samples} & \textbf{Proportion} \\ \hline
ELI5 \cite{fan2019eli5longformquestion} & 32,547&2.16\% \\
HotpotQA \cite{yang2018hotpotqadatasetdiverseexplainable} &90,447&5.99\% \\
FEVER \cite{thorne-etal-2018-fever} & 101,578&6.73\% \\
MIRACL \cite{zhang-etal-2023-miracl} & 32,561& 2.16\%\\
MSMARCO Passage Ranking \cite{bajaj2018msmarcohumangenerated} & 249,592&16.53\% \\
MSMARCO Document Ranking \cite{bajaj2018msmarcohumangenerated} & 73,400&4.86\% \\
NQ \cite{kwiatkowski-etal-2019-natural} & 100,231&6.64\% \\
NLI \cite{gao-etal-2021-simcse} & 277,230&18.36\% \\
SQuAD \cite{rajpurkar-etal-2016-squad} & 87,599&5.80\% \\
TriviaQA \cite{joshi2017triviaqa} & 73,346&4.86\% \\
Quora Duplicate Questions \cite{quora-question-pairs} & 101,762&6.74\% \\
Mr-TyDi \cite{zhang-etal-2021-mr} & 48,715&3.23\% \\
DuReader \cite{he-etal-2018-dureader} & 86,395 &5.72\% \\
T2Ranking \cite{10.1145/3539618.3591874} & 154,294&10.22\% \\ \hline
\end{tabular}
\caption{Data Statistics of E5 Dataset. We show the composition and distribution of E5 Dataset.}
\label{tab:train_dataset_details}
\end{table*}
\begin{table}[h] \footnotesize  \centering\resizebox{0.48\textwidth}{!}{\begin{tabular}{c|l|c|c|c}
\toprule
\textbf{Task} & \textbf{Dataset} & \textbf{N-shot} & \multirowcell{\textbf{Train texts} \\ \textbf{for STMD}} & \multirowcell{\textbf{Evaluation} \\ \textbf{texts}} \\
\midrule
\multirow{3}{*}{\multirowcell{Text \\ Summarization}} & CNN/DailyMail & 0 & 2,000 & 2,000 \\
& XSum & 0 & 2,000 & 2,000 \\
& SamSum & 0 & 2,000 & 819 \\
\midrule
\multirow{4}{*}{\multirowcell{QA \\ Long answer}} & PubMedQA & 0 & 2,000 & 2,000 \\
& MedQUAD & 5 & 2,000 & 2,000 \\
& TruthfulQA & 5 & 408 & 409 \\
& GSM8k & 5 & 2,000 & 1,319 \\
\midrule
\multirow{4}{*}{\multirowcell{QA \\ Short answer}} & SciQ & 0 & 5,000 & 1,000 \\
& CoQA & \multirowcell{all preceding \\ questions} & 5,000 & 2,000 \\
& TriviaQA & 5 & 5,000 & 2,000 \\
\midrule
\multirow{1}{*}{\multirowcell{MCQA}} & MMLU & 5 & 5,000 & 2,000 \\
\bottomrule
\end{tabular}
}\caption{\label{tab:dataset_stat} The statistics of the datasets used for evaluation.}
\end{table}
\begin{table*}[t]
\centering
\small
\resizebox{\linewidth}{!}{
\begin{tabular}{ll}
\hline
\textbf{Datasets} & \textbf{Instructions} \\
\hline
DuReader & Given a Chinese search query, retrieve web passages that answer the question. \\
\hline
ELI5 & Provided a user question, retrieve the highest voted answers on the Reddit ELI5 forum. \\
\hline

FEVER & Given a claim, retrieve documents that support or refute the claim. \\
\hline

HotpotQA & Given a multi-hop question, retrieve documents that can help answer the question. \\
\hline

MIRACL & Given a question, retrieve Wikipedia passages that answer the question. \\
\hline

MrTyDi & Given a question, retrieve Wikipedia passages that answer the question. \\
\hline

MSMARCO Passage & Given a web search query, retrieve relevant passages that answer the query. \\
\hline
MSMARCO Document & Given a web search query, retrieve relevant documents that answer the query. \\
\hline
NQ & Given a question, retrieve Wikipedia passages that answer the question. \\
\hline
Squad & Retrieve Wikipedia passages that answer the question. \\
\hline
T2Ranking & Given a Chinese search query, retrieve web passages that answer the question. \\
\hline
TriviaQA & Retrieve Wikipedia passages that answer the question. \\
\hline
\multirow{2}{*}{QuoraDuplicates} & Given a question, retrieve questions that are semantically equivalent to the given question. \\
& Find questions that have the same meaning as the input question. \\
\hline
\multirow{2}{*}{NLI} & Given a premise, retrieve a hypothesis that is entailed by the premise. \\
& Retrieve semantically similar text. \\
\hline
\end{tabular}}
\caption{Instructions Used for E5 Dataset.}
\label{tab:dataset_descriptions}
\end{table*}

\begin{table*}[t]
\centering
\small
\begin{tabular}{l|p{11cm}}
\hline
\textbf{Task Name} & \textbf{Instruction} \\
\hline
Arguana & Given a claim, find documents that refute the claim \\
\hline
FiQA2018 & Given a financial question, retrieve user replies that best answer the question \\
\hline

NFCorpus & Given a question, retrieve relevant documents that best answer the question \\
\hline

Quora & Given a question, retrieve questions that are semantically equivalent to the given question \\
\hline

SCIDOCS & Given a scientific paper title, retrieve paper abstracts that are cited by the given paper \\
\hline

TREC-COVID & Given a query on COVID-19, retrieve documents that answer the query \\
\hline

Touche2020 & Given a question, retrieve detailed and persuasive arguments that answer the question \\
\hline
ClimateFEVER & Given a claim about climate change, retrieve documents that support or refute the claim \\
\hline
FEVER & Given a claim, retrieve documents that support or refute the claim \\
\hline
HotpotQA & Given a multi-hop question, retrieve documents that can help answer the question \\
\hline
DBPedia &  Given a query, retrieve relevant entity descriptions from DBPedia\\
\hline
NQ & Given a question, retrieve Wikipedia passages that answer the question \\
\hline
MS MARCO & Given a web search query, retrieve relevant passages that answer the query \\
\hline
\multirow{2}{*}{CQADupStack} & Given a question, retrieve detailed question descriptions from Stackexchange that are duplicates to the given question \\

\hline
\end{tabular}
\caption{Instructions Used for Evaluation on the BEIR Benchmark.}
\label{tab:retrieval_dataset}
\end{table*}


\label{app:dataset}


\subsection{Case Study}
\label{app:case_study}
\begin{table*}[t]
\centering
\caption{A case study examining \model's response to a query about ``\emph{graders}'' in OpenAI's 12-day edu    cational series demonstrates its video retrieval capabilities and knowledge synthesis.}
\label{tab:case study}
\small
\begin{tabularx}{\textwidth}{X X}

\toprule
\textbf{Query:} Explain the purpose and functionality of 'graders' in the context of reinforcement fine-tuning.\\

\midrule
\textbf{Input Videos:} \href{https://www.youtube.com/playlist?list=PLOXw6I10VTv9lin5AzsHAHCTrC7BdVdEM}{12 Days of OpenAI} (A total of 12 videos) \\

\midrule

\textbf{\model\ Response:} 

\texttt{\#\#\# Understanding Gradation in Reinforcement Fine-Tuning}

\texttt{**Purpose of Graders**}

\texttt{Graders are integral components in the process of reinforcement fine-tuning, mainly utilized to evaluate and improve the performance of machine learning models...} (Omitted for brevity)

\texttt{**Functionality of Graders**}

\texttt{\colorbox{lightgreen}{The grading system operates by comparing a model's outputs against those of}}

\texttt{\colorbox{lightgreen}{verified correct responses. The mechanics of this assessment involve assigning}}

\texttt{\colorbox{lightgreen}{scores to the model's outputs based on their correctness. These scores range}}

\texttt{\colorbox{lightgreen}{from zero to one, where:}}


\texttt{\colorbox{lightblue}{- A score of **0** indicates that the model's answer is completely incorrect.}}

\texttt{\colorbox{lightblue}{- A score of **1** signals that the answer is fully correct.}}

\texttt{- Scores can also be fractional, allowing for partial credit. For instance, \colorbox{lightpink}{a score of **0.7** may denote that the model's suggestion was }}
\texttt{\colorbox{lightpink}{correct but not the highest in a ranked list of possible answers.}} 

\\

\texttt{The process includes:}

\texttt{1. **Input Comparison**: The grader takes both the model’s output and the correct answer, analyzing them to derive a score.}

\texttt{2. **Graded Outputs**: This interaction allows for a dynamic reflection of the model's performance, where the scores directly influence how the model is fine-tuned in subsequent training iterations.}

\texttt{3. **Feedback Loop**: The scores obtained from graders provide feedback that can be used to adjust learning parameters and improve the model’s future responses. This highlights the feedback mechanism that is crucial for reinforcement learning.}

\\

(The remaining content are omitted for brevity.)

\\

\midrule
\textbf{Retrieved Sources by \model:}

\model\ accurately retrieved four continuous video segments from the OpenAI's 12-day show, ranging from 10:00 to 12:00 in Day 2 of the video "\href{https://www.youtube.com/watch?v=yCIYS9fx56U}{Reinforcement Fine-Tuning}." Here, we highlight key moments relevant to the detailed content in the answer. From left to right, these are retrieved moments at timestamps \colorbox{lightgreen}{10:35}, \colorbox{lightblue}{10:39}, and \colorbox{lightpink}{11:10}, which provide informative insights that help \model\ give a comprehensive answer to the query.

\\

\begin{tabular}{ccc}
    {\includegraphics[width=0.3\textwidth]{figs/openai-1.png}} &
    {\includegraphics[width=0.3\textwidth]{figs/openai-2.png}} &
    {\includegraphics[width=0.3\textwidth]{figs/openai-3.png}} \\
\end{tabular}

\\

\bottomrule

\end{tabularx}
\vspace{-0.2in}
\end{table*}
%\begin{figure}[t]
    \centering
    \includegraphics[width=1\linewidth]{image/case_hot.png}
    \caption{An Example of Similarity Scores Between All Embeddings.}
    \label{fig:case_hot}
\end{figure}

In this subsection, we present two case studies to demonstrate the effectiveness of \method{}. Specifically, we show the top 1 retrieved document comes from both MiniCPM and \method{} in the FiQA and Trec-COVID datasets. The retrieval cases are shown in Table~\ref{tab:case_study}.
%Then, we plot a detailed similarity relationship for one case to explore how embeddings at various stages affect the final representation used for retrieval.
%\textbf{Retrieval Cases.}

For the query ``In the US, is it a good idea to hire a tax consultant for doing taxes?'', while the retrieval result of MiniCPM suggests hiring a professional, it fails to mention the specific context of ``In the US''. 
In contrast, the document retrieved by \method{} aligns more closely with the query, addressing all the relevant aspects. This highlights that \method{} effectively activates the reasoning capabilities of LLMs, enabling a more fine-grained document representation.

For the query ``What is the mechanism of inflammatory response and pathogenesis of COVID-19 cases?'', MiniCPM retrieves documents containing more pathological terms, such as ``RAGE transactivation'' and ``the ACE/Ang II/ATR1 pathway''. 
However, these retrieved documents cannot accurately address the inflammatory response and pathogenesis of COVID-19. This suggests that MiniCPM lacks a fine-grained understanding of document content, leading to suboptimal retrieval performance. In contrast, \method{} provides accurate retrieval results, demonstrating the importance of deliberate thinking before searching.

%\textbf{Embedding Analysis.} 
%We show a heatmap of detailed similarity relationships for one case in Figure~\ref{fig:case_hot}. The similarity relationships between neighboring thinking steps indicating stepwise learning behavior of Chain-of-Deliberation. 


\subsection{Comparison with Additional Baseline Retrievers}
\label{app:comparison}
In this subsection, we conduct additional comparisons with baseline retrievers not discussed in the main section, including several advanced retrievers from the BEIR benchmark.

Specifically, we use CPT~\cite{neelakantan2022text}, Udever~\cite{zhang2023languagemodelsuniversalembedders}, ULLME, Ada2, and Promptriever as our additional retrievers to compare with \method{}.
CPT refers to a series of large encoder models developed by OpenAI \cite{brown2020language}, which are pre-trained on large-scale and unsupervised data. We use the CPT-text-175B model as our baseline model.   
Udever contrastively trains the BLOOM~\cite{workshop2023bloom176bparameteropenaccessmultilingual} to build dense retrievers, enabling it to effectively generate aligned multilingual embeddings for both text and code. 
ULLME~\cite{man-etal-2024-ullme} enables bidirectional attention between LLMs and enhances them for text embedding using Generative Reinforcement Learning (GRL).
Ada2 is a versatile text embedding model introduced by OpenAI. The evaluation result for Ada2 comes from~\citet{kamalloo-etal-2023-evaluating-embedding}. 
Promptriever~\cite{weller2025promptriever} is trained on an instance-level instruction dataset from MS MARCO, enhancing its ability to follow instructions for retrieval tasks.

The performance comparison results on the BEIR benchmark are shown in Table~\ref{tab:other_baseline}. 
Compared with these baseline models, \method{} still exhibits competitive performance, demonstrating its effectiveness. Compared to Udever and ULLME, \method{} shows significant improvement on FEVER and Climate-FEVER, indicating its effectiveness across diverse fact-checking scenarios. 
When compared to Promptriever, \method{} demonstrates similar retrieval performance. However, unlike Promptriever, \method{} does not rely on synthetic data for training its retrieval models.

\begin{table*}[t]
\centering
\small
\begin{tabular}{l|c|c|c|c|c|cc}
    \hline
    \textbf{Method} ($\rightarrow$) &\textbf{CPT}& \textbf{Udever} &  \textbf{ULLME} & \textbf{Ada2} & \textbf{Promptriever} & \multicolumn{2}{c}{\textbf{\method{}}} \\
    \textbf{Model Size} ($\rightarrow$) & 175B& 7B & 7B & unk. & 7B & 2.4B & 4B \\
    \hline
    TREC-COVID$^{\dagger}$  &0.649 &0.838 &0.836 &0.813 &0.839 & 0.795 & 0.836 \\
    NFCorpus$^{\dagger}$  &0.407 &0.360 &0.394 &0.358 &0.365 & 0.378 & 0.399 \\
    NQ &/ &0.533 &0.614 &0.482&0.619 &0.560 & 0.561 \\
    HotpotQA$^{\dagger}$  &0.688 &0.567 &0.674 &0.654 & 0.692&0.678 & 0.678 \\
    FiQA$^{\dagger}$  &0.512 &0.367 &0.423 &0.411 &0.459 & 0.434 & 0.462 \\
    ArguAna$^{\dagger}$  &0.435 &0.522 &0.468 &0.567 &0.518  & 0.567 & 0.562 \\
    Touché-2020$^{\dagger}$  &0.291 &0.252 &0.271 &0.280 &0.314 & 0.211 & 0.250 \\
    Quora$^{\dagger}$  &0.638 &0.883 &0.878 &0.876&0.865  & 0.886 & 0.886 \\
    DBPedia$^{\dagger}$  &0.408 &0.376 &0.464&0.402 &0.450 & 0.430 & 0.432 \\
    SCIDOCS &/ &0.189 &0.211 &0.186 &0.173  & 0.197 & 0.212 \\
    FEVER$^{\dagger}$  &0.775 &0.740 &0.615 & 0.773&0.828 & 0.859 & 0.857 \\
    Climate-FEVER$^{\dagger}$  &0.223 &0.268 &0.222 &0.237 &0.276 & 0.303 & 0.294 \\
    SciFact$^{\dagger}$  &0.754 &0.693 &0.724 &0.736 &0.750  & 0.735 & 0.743 \\
   % MS MARCO &/ &0.420 &0.357 & /& / &0.367 & 0.391 \\
    CQADupStack &/ &0.393 &/ &0.391 & / & 0.431 & 0.428 \\
    \hline
    \textbf{Avg CPT sub$^{\dagger}$ } &0.525 &0.533 &0.543 &0.555 &0.571 &0.571&0.582\\
    \textbf{Avg} &/ &0.499 &/ &/ &/  &0.533 & 0.543 \\
    \hline
\end{tabular}
\caption{Retrieval Performances of \method{} and Other Baseline Models on the BEIR Benchmark.} 
\label{tab:other_baseline}
\end{table*}

