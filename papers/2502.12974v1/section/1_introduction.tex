\section{Introduction}
Dense retrieval models encode both queries and documents into a dense embedding space and measure their similarity to retrieve relevant documents~\cite{karpukhin2020dense, zhao2024dense, xiong2020approximate}, demonstrating strong effectiveness in various downstream NLP tasks, such as open-domain question answering~\cite{chen2020open}, fact verification~\cite{liu2019fine}, and web search~\cite{chen2024ms}. However, recent findings have shown that dense retrievers suffer from significant performance degradation when applied to new tasks or domains~\cite{su2022one}, raising concerns about their versatility~\cite{luo2024large, khramtsova2024leveraging}.

\section{Introduction}
\label{sec:intro}

\begin{figure*}[tb]
    \centering
    \includegraphics[width=0.848\linewidth]{figs/circuitnn.pdf} 
    \caption{Illustration of differentiable CircuitNN. CircuitNN is designed based on differentiable NAND gates. After DAS is guided by PI and PO pairs of the truth table, CircuitNN can get the precise circuit architecture logic equivalent to the truth table.}
    \label{fig:circuitnn}
\end{figure*}

% 1. Describe the importance of logic synthesis
% 2. Existing Problems
% (a) Neural Architecture Search: Unstable, Predefined Setting, etc.
% (b) Circuit Generation: Probabilistic Model, Logic Equivalence

With the rapid advancement of technology, the scale of integrated circuits (ICs) has expanded exponentially. 
This expansion has introduced significant challenges in chip manufacturing, particularly concerning power and area metrics.
A primary objective in IC design is achieving the same circuit function with fewer transistors, thereby reducing power usage and area occupancy.

Logic synthesis~\cite{hachtel2005logicsynth}, a critical step in electronic design automation (EDA), transforms behavioral-level circuit designs into optimized gate-level circuits, ultimately yielding the final IC layout. 
The primary goal of logic synthesis is to identify the physical implementation with the fewest gates for a given circuit function. 
This task constitutes a challenging NP-hard combinatorial optimization problem. 
Current logic synthesis tools~\cite{brayton2010abc, wolf2013yosys} rely on human-designed heuristics, often leading to sub-optimal outcomes.

Differentiable architecture search (DAS) techniques~\cite{liu2018darts, chu2020darts} offer novel perspectives on addressing challenges in this problem.
Circuit functions can be represented through truth tables, which map binary inputs to their corresponding outputs. 
Truth tables provide a precise representation of input-output relationships, ensuring the design of functionally equivalent circuits.
Inspired by this, researchers~\cite{deepmind2024ai4sys, wang2024tnet} have begun exploring the application of DAS to synthesize circuits directly from truth tables.
Specifically, \citet{deepmind2024ai4sys} proposed CircuitNN, a framework that learns differentiable connection structures with logic gates, enabling the automatic generation of logic circuits from truth tables.
This approach significantly reduces the complexity of traditional circuit generation. 
Building on this, \citet{wang2024tnet} introduced T-Net, a triangle-shaped variant of CircuitNN, incorporating regularization techniques to enhance the efficiency of DAS.

Despite these advancements, several challenges remain. 
The computational complexity of DAS grows quadratically with the number of gates, posing scalability issues.
Although triangle-shaped architecture~\cite{wang2024tnet} partially mitigates this problem, redundancy persists. 
%Additionally, DAS is susceptible to converging to local optima, limiting the ability to search architectures that satisfy the given truth tables~\cite{liu2018darts}. 
%Furthermore, hyperparameters (network depth and layer width) require extensive searches, introducing complexity and prolonging the synthesis process. 
Additionally, DAS is susceptible to converging to local optima~\cite{liu2018darts} and hyperparameters (network depth and layer width) require extensive searches. 
The challenges arise from the vast search space in DAS. 
% Even with predefined settings for CircuitNN, finding a configuration that meets the truth table requires extensive trial and error during the DAS process. 
Intuitively, limiting the search space through predefined parameters (network depth, gates per layer, and connection probabilities) can significantly reduce the complexity.

Recent advances~\cite{openai2023gpt4, abramson2024alphafold3, esser2024sd3, li2024mar} in conditional generative models have demonstrated remarkable performance across language, vision, and graph generation tasks. 
Motivated by these developments, we propose a novel approach to circuit generation that generates preliminary circuit structures to guide DAS in generating refined circuits matching specified truth tables. 
Firstly, we introduce CircuitVQ, a tokenizer with a discrete codebook for circuit tokenization. 
Built upon our Circuit AutoEncoder framework~\cite{hou2022graphmae,li2023maskgae,wu2025mgvga}, CircuitVQ is trained through a circuit reconstruction task. 
Specifically, the CircuitVQ encoder encodes input circuits into discrete tokens using a learnable codebook, while the decoder reconstructs the circuit adjacency matrix based on these tokens.
Subsequently, the CircuitVQ encoder serves as a circuit tokenizer for CircuitAR pretraining, which employs a masked autoregressive modeling paradigm~\cite{chang2022maskgit, li2023mage}. 
In this process, the discrete codes function as supervision signals. 
After training, CircuitAR can generate discrete tokens progressively, which can be decoded into initial circuit structures by the decoder of the CircuitVQ. 
These prior insights can guide DAS in producing refined circuits that match the target truth tables precisely.

Our key contributions can be summarized as follows:
\begin{itemize}
\item We introduce CircuitVQ, a circuit tokenizer that facilitates graph autoregressive modeling for circuit generation, based on our Circuit AutoEncoder framework;
\item Develop CircuitAR, a model trained using masked autoregressive modeling, which generates initial circuit structures conditioned on given truth tables;
\item Propose a refinement framework that integrates differentiable architecture search to produce functionally equivalent circuits guided by target truth tables;
\item Comprehensive experiments demonstrating the scalability and capability emergence of our CircuitAR and the superior performance of the proposed circuit generation approach.
\end{itemize}

% Motivation
% (a) Diffusion (Vision, Graph), Autoregressive (Language, Vision)
% (b) Circuit Generation for Predefined Setting
% (c) Neural Architecture Search for Strict Logic Equivalence

% Contribution
% (a) Circuit Tokenizer (new transformer arch, training strategy)
% (b) CircuitAR (train and gen strategies, post-ar strategy)
% (c) Extensive Evaluation including BitD (Bit Distance) for Scalability

Large Language Models (LLMs), such as ChatGPT~\cite{achiam2023gpt} and LLaMA~\cite{touvron2023llama}, have demonstrated extraordinary emergent capabilities~\cite{wei2022emergent, zhao2023survey}, inspiring researchers to leverage them to enhance the task and domain generalization of dense retrievers~\cite{zhu2023large, khramtsova2024leveraging}. In particular, existing work has focused on prompting LLMs to generate dense representations for retrieval~\cite{zhuang2024promptreps}. These methods typically use task-specific instructions or in-context demonstrations to guide LLMs in generating task- and domain-aware embeddings. To learn more tailored representations for dense retrieval, researchers further focus on optimizing LLM-based retrievers using relevance labels~\cite{ma2024fine, neelakantan2022text, li2025making}. These methods exploit the superior reasoning abilities of LLMs, achieving impressive performance across various retrieval tasks~\cite{wang2023improving, zhu2023large, luo2024large}. Recent studies suggest that LLMs pose strong reasoning capability, particularly implemented by their step-by-step thinking~\cite{kudo2024think,wei2022chain}. LLM-based retrievers typically rely on the hidden state of the end-of-sequence token as both query and document representations. Nevertheless, only relying on one embedding usually shows less effectiveness in representing documents from different views that can match queries~\cite{zhang2022multi,khattab2020colbert}. 

In this paper, we propose a \textbf{D}\textbf{e}li\textbf{b}er\textbf{a}te \textbf{T}hinking based Dens\textbf{e} \textbf{R}etriever (\method{}) model to learn more effective document representations through deliberately thinking step-by-step before retrieval. As shown in Figure~\ref{fig:intro}, our method stimulates LLMs to conduct the reasoning process, enabling them to generate more fine-grained document representations for retrieval. Specifically, \method{} introduces the Chain-of-Deliberation mechanism to encourage LLMs to conduct deliberate thinking by autograssively decoding the document representations. Then \method{} utilizes the Self Distillation mechanisms to gather all information from previous steps and compress them into the document embedding at the last step.

Our experiments show that \method{} achieves comparable or even better retrieval performance than the baseline methods implemented by larger-scale LLMs, highlighting its effectiveness. Our further analyses show that both Chain-of-Deliberation and Self Distillation play important roles in \method{} and appropriately increasing the thinking steps can benefit these LLM-based dense retrieval models. Thriving on autograssively decoding different document representations during thinking, the document representations can be gradually refined to be more effective. By incorporating our Self Distillation, LLMs show the ability to capture different key information at different thinking steps and gather all crucial semantics from different steps to the final document representations. 

