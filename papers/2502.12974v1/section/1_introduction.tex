\section{Introduction}
Dense retrieval models encode both queries and documents into a dense embedding space and measure their similarity to retrieve relevant documents~\cite{karpukhin2020dense, zhao2024dense, xiong2020approximate}, demonstrating strong effectiveness in various downstream NLP tasks, such as open-domain question answering~\cite{chen2020open}, fact verification~\cite{liu2019fine}, and web search~\cite{chen2024ms}. However, recent findings have shown that dense retrievers suffer from significant performance degradation when applied to new tasks or domains~\cite{su2022one}, raising concerns about their versatility~\cite{luo2024large, khramtsova2024leveraging}.

\section{Introduction}


\begin{figure}[t]
\centering
\includegraphics[width=0.6\columnwidth]{figures/evaluation_desiderata_V5.pdf}
\vspace{-0.5cm}
\caption{\systemName is a platform for conducting realistic evaluations of code LLMs, collecting human preferences of coding models with real users, real tasks, and in realistic environments, aimed at addressing the limitations of existing evaluations.
}
\label{fig:motivation}
\end{figure}

\begin{figure*}[t]
\centering
\includegraphics[width=\textwidth]{figures/system_design_v2.png}
\caption{We introduce \systemName, a VSCode extension to collect human preferences of code directly in a developer's IDE. \systemName enables developers to use code completions from various models. The system comprises a) the interface in the user's IDE which presents paired completions to users (left), b) a sampling strategy that picks model pairs to reduce latency (right, top), and c) a prompting scheme that allows diverse LLMs to perform code completions with high fidelity.
Users can select between the top completion (green box) using \texttt{tab} or the bottom completion (blue box) using \texttt{shift+tab}.}
\label{fig:overview}
\end{figure*}

As model capabilities improve, large language models (LLMs) are increasingly integrated into user environments and workflows.
For example, software developers code with AI in integrated developer environments (IDEs)~\citep{peng2023impact}, doctors rely on notes generated through ambient listening~\citep{oberst2024science}, and lawyers consider case evidence identified by electronic discovery systems~\citep{yang2024beyond}.
Increasing deployment of models in productivity tools demands evaluation that more closely reflects real-world circumstances~\citep{hutchinson2022evaluation, saxon2024benchmarks, kapoor2024ai}.
While newer benchmarks and live platforms incorporate human feedback to capture real-world usage, they almost exclusively focus on evaluating LLMs in chat conversations~\citep{zheng2023judging,dubois2023alpacafarm,chiang2024chatbot, kirk2024the}.
Model evaluation must move beyond chat-based interactions and into specialized user environments.



 

In this work, we focus on evaluating LLM-based coding assistants. 
Despite the popularity of these tools---millions of developers use Github Copilot~\citep{Copilot}---existing
evaluations of the coding capabilities of new models exhibit multiple limitations (Figure~\ref{fig:motivation}, bottom).
Traditional ML benchmarks evaluate LLM capabilities by measuring how well a model can complete static, interview-style coding tasks~\citep{chen2021evaluating,austin2021program,jain2024livecodebench, white2024livebench} and lack \emph{real users}. 
User studies recruit real users to evaluate the effectiveness of LLMs as coding assistants, but are often limited to simple programming tasks as opposed to \emph{real tasks}~\citep{vaithilingam2022expectation,ross2023programmer, mozannar2024realhumaneval}.
Recent efforts to collect human feedback such as Chatbot Arena~\citep{chiang2024chatbot} are still removed from a \emph{realistic environment}, resulting in users and data that deviate from typical software development processes.
We introduce \systemName to address these limitations (Figure~\ref{fig:motivation}, top), and we describe our three main contributions below.


\textbf{We deploy \systemName in-the-wild to collect human preferences on code.} 
\systemName is a Visual Studio Code extension, collecting preferences directly in a developer's IDE within their actual workflow (Figure~\ref{fig:overview}).
\systemName provides developers with code completions, akin to the type of support provided by Github Copilot~\citep{Copilot}. 
Over the past 3 months, \systemName has served over~\completions suggestions from 10 state-of-the-art LLMs, 
gathering \sampleCount~votes from \userCount~users.
To collect user preferences,
\systemName presents a novel interface that shows users paired code completions from two different LLMs, which are determined based on a sampling strategy that aims to 
mitigate latency while preserving coverage across model comparisons.
Additionally, we devise a prompting scheme that allows a diverse set of models to perform code completions with high fidelity.
See Section~\ref{sec:system} and Section~\ref{sec:deployment} for details about system design and deployment respectively.



\textbf{We construct a leaderboard of user preferences and find notable differences from existing static benchmarks and human preference leaderboards.}
In general, we observe that smaller models seem to overperform in static benchmarks compared to our leaderboard, while performance among larger models is mixed (Section~\ref{sec:leaderboard_calculation}).
We attribute these differences to the fact that \systemName is exposed to users and tasks that differ drastically from code evaluations in the past. 
Our data spans 103 programming languages and 24 natural languages as well as a variety of real-world applications and code structures, while static benchmarks tend to focus on a specific programming and natural language and task (e.g. coding competition problems).
Additionally, while all of \systemName interactions contain code contexts and the majority involve infilling tasks, a much smaller fraction of Chatbot Arena's coding tasks contain code context, with infilling tasks appearing even more rarely. 
We analyze our data in depth in Section~\ref{subsec:comparison}.



\textbf{We derive new insights into user preferences of code by analyzing \systemName's diverse and distinct data distribution.}
We compare user preferences across different stratifications of input data (e.g., common versus rare languages) and observe which affect observed preferences most (Section~\ref{sec:analysis}).
For example, while user preferences stay relatively consistent across various programming languages, they differ drastically between different task categories (e.g. frontend/backend versus algorithm design).
We also observe variations in user preference due to different features related to code structure 
(e.g., context length and completion patterns).
We open-source \systemName and release a curated subset of code contexts.
Altogether, our results highlight the necessity of model evaluation in realistic and domain-specific settings.





Large Language Models (LLMs), such as ChatGPT~\cite{achiam2023gpt} and LLaMA~\cite{touvron2023llama}, have demonstrated extraordinary emergent capabilities~\cite{wei2022emergent, zhao2023survey}, inspiring researchers to leverage them to enhance the task and domain generalization of dense retrievers~\cite{zhu2023large, khramtsova2024leveraging}. In particular, existing work has focused on prompting LLMs to generate dense representations for retrieval~\cite{zhuang2024promptreps}. These methods typically use task-specific instructions or in-context demonstrations to guide LLMs in generating task- and domain-aware embeddings. To learn more tailored representations for dense retrieval, researchers further focus on optimizing LLM-based retrievers using relevance labels~\cite{ma2024fine, neelakantan2022text, li2025making}. These methods exploit the superior reasoning abilities of LLMs, achieving impressive performance across various retrieval tasks~\cite{wang2023improving, zhu2023large, luo2024large}. Recent studies suggest that LLMs pose strong reasoning capability, particularly implemented by their step-by-step thinking~\cite{kudo2024think,wei2022chain}. LLM-based retrievers typically rely on the hidden state of the end-of-sequence token as both query and document representations. Nevertheless, only relying on one embedding usually shows less effectiveness in representing documents from different views that can match queries~\cite{zhang2022multi,khattab2020colbert}. 

In this paper, we propose a \textbf{D}\textbf{e}li\textbf{b}er\textbf{a}te \textbf{T}hinking based Dens\textbf{e} \textbf{R}etriever (\method{}) model to learn more effective document representations through deliberately thinking step-by-step before retrieval. As shown in Figure~\ref{fig:intro}, our method stimulates LLMs to conduct the reasoning process, enabling them to generate more fine-grained document representations for retrieval. Specifically, \method{} introduces the Chain-of-Deliberation mechanism to encourage LLMs to conduct deliberate thinking by autograssively decoding the document representations. Then \method{} utilizes the Self Distillation mechanisms to gather all information from previous steps and compress them into the document embedding at the last step.

Our experiments show that \method{} achieves comparable or even better retrieval performance than the baseline methods implemented by larger-scale LLMs, highlighting its effectiveness. Our further analyses show that both Chain-of-Deliberation and Self Distillation play important roles in \method{} and appropriately increasing the thinking steps can benefit these LLM-based dense retrieval models. Thriving on autograssively decoding different document representations during thinking, the document representations can be gradually refined to be more effective. By incorporating our Self Distillation, LLMs show the ability to capture different key information at different thinking steps and gather all crucial semantics from different steps to the final document representations. 

