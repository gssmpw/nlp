% \begin{figure}[htb]
%     \centering
%     \subfigure[TREC-COVID]{ \label{fig:ndcg_Treccovid}
%     \includegraphics[width=0.43\linewidth]{image/TC.png}}
%     \subfigure[FiQA-2018]{ \label{fig:ndcg_FiQA}
%     \includegraphics[width=0.43\linewidth]{image/FQ.png}}
    
%     \subfigure[NFCorpus]{ \label{fig:ndcg_nfcorpus}
%     \includegraphics[width=0.43\linewidth]{image/NF.png}}

%     \subfigure[SCIDOCS]{ \label{fig:ndcg_Scidocs}
%     \includegraphics[width=0.43\linewidth]{image/SciDocs.png}}
    
%     \caption{Retrieval Performance of \method{} Variations with Different Thinking Depth. We adjust the length of the CoD to train different \method{} models and evaluate them on the subset of BEIR.}
%     \label{fig:different_tokens}
% \end{figure}

\begin{figure}[t]
    \centering
    \subfigure[TREC-COVID.]{ \label{fig:ndcg_treccovid}
    \includegraphics[width=0.48\linewidth]{image/TC.png}}
    \subfigure[FiQA.]{ \label{fig:fiqa}
    \includegraphics[width=0.48\linewidth]{image/FQ.png}}
    \subfigure[NFCorpus.]{ \label{fig:ndcg_nfcorpus}
    \includegraphics[width=0.48\linewidth]{image/NF.png}}
     \subfigure[SCIDOCS.]{ \label{fig:ndcg_scidocs}
    \includegraphics[width=0.48\linewidth]{image/SciDocs.png}}
    \caption{Retrieval Performance of \method{} with Different Thinking Depths. We set the length of the CoD to train different \method{} models and evaluate them on different subsets of BEIR.}
    \label{fig:different_tokens}
\end{figure}