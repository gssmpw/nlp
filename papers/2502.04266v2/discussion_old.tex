We present a methodology for auditing search engine results using bots (web crawlers), incrementally designed to resemble more "human-like" different online profiles. Focusing on the Israel-Palestine conflict, we examined how location, browser language, and browsing history impact the URLs, content types, and leanings presented by three different search engines: Google, DuckDuckGo and Yahoo.

Our findings confirm that location plays a critical role in shaping search engine results, as previously noted by the works of X and Y and as openly acknowledged by search engines. Coupling features like browser language and browsing history with location further increased differences, particularly for conflict-related queries. There is no inherent reason to expect more variation in search results for conflict-related queries than for general queries, especially when the latter included topics like recipes, books, and movies, where personalization is typically high [REF]. To explore whether this difference could be due to a higher proportion of news content —possibly more influenced by local sources — for searches related with the conflict, we applied the difference metric, $D = 1 - RBO$ (Rank-Biased Overlap) [REF], comparing website categories (e.g. "News", "Political") rather than just URLs. The results confirmed that differences remain when we compare types of content, rather than just URLs. With these differences once again being particularly high for conflict-related queries and more pronounced when bots used different browser languages and browsing histories. This finding suggests that search engines are not merely displaying the same content in local variations, but are showing different types of content to users in different locations. We consider this category-based comparison to be a valuable methodological contribution to studies of recommendation algorithms across different geographic regions, offering an improvement over previous approaches [REFs].

Previous research has shown that browsing history typically does not induce significant personalization in search results, except for top news stories [REFs]. Contrary to this, our findings revealed that comparing bots at different locations with distinct browsing histories further increased the average value of search result differences. However, when location was not controlled for, differences between bots with varying browsing histories were only statistically significant in the case of Google. This raises the question of whether the key driver of these differences was the type of content visited by the bots, or simply the fact that they interacted with news in their respective local languages. Further work should be pursued in order to clarify these aspects. 

Lastly, we analyzed the political leanings of the news articles delivered to the bots for conflict-related queries. This task is complex due to the inherently negative nature of war-related content and the subjective interpretation of leaning (e.g., the headline "Israel attacked Rafah" could be viewed as pro-Palestinian due to the negative portrayal of Israel, or as pro-Israel depending on the perspective). We used both ChatGPT and MTurk workers to classify these articles. Although the differences in content leanings were not highly pronounced, there was evidence that users in different locations were exposed to content with varying leanings regarding the conflict, particularly on top 3 results.

Search engines are widely used in everyday life for a range of tasks, from simple queries to more complex research. This work contributes to the understanding of how the algorithms behind these search engines can amplify differences in the information landscape, particularly on politically sensitive and polarized topics. Our findings show that tailoring content based on central factors such as location, language settings, and browsing history (outside of the platform) can together lead to significant variations in search results. This suggests that users around the world are often exposed to different content, sometimes with varying perspectives on the same issue, such as the Israel-Palestine conflict.

Furthermore, our study reveals that search engines may not always behave according to their stated policies. Both Google and DuckDuckGo presented different results for simulated users with distinct browsing histories, despite their claims of minimizing personalization. Interestingly, Yahoo, which acknowledges more personalization in its algorithm, showed the least evidence of personalization among the three search engines tested.

Finally, we believe this work provides a valuable methodology for auditing the information landscape of search engines. It offers new insights into how we can study and evaluate the types of content these platforms deliver, particularly in the context of politically charged issues.