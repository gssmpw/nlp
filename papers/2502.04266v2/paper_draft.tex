%%
%% This is file `sample-sigconf-authordraft.tex',
%% generated with the docstrip utility.
%%
%% The original source files were:
%%
%% samples.dtx  (with options: `all,proceedings,bibtex,authordraft')
%% 
%% IMPORTANT NOTICE:
%% 
%% For the copyright see the source file.
%% 
%% Any modified versions of this file must be renamed
%% with new filenames distinct from sample-sigconf-authordraft.tex.
%% 
%% For distribution of the original source see the terms
%% for copying and modification in the file samples.dtx.
%% 
%% This generated file may be distributed as long as the
%% original source files, as listed above, are part of the
%% same distribution. (The sources need not necessarily be
%% in the same archive or directory.)
%%
%% Commands for TeXCount
%TC:macro \cite [option:text,text]
%TC:macro \citep [option:text,text]
%TC:macro \citet [option:text,text]
%TC:envir table 0 1
%TC:envir table* 0 1
%TC:envir tabular [ignore] word
%TC:envir displaymath 0 word
%TC:envir math 0 word
%TC:envir comment 0 0
%%
%%
%% The first command in your LaTeX source must be the \documentclass
%% command.
%%
%% For submission and review of your manuscript please change the
%% command to \documentclass[manuscript, screen, review]{acmart}.
%%
%% When submitting camera ready or to TAPS, please change the command or whichever template is required
%% for your publication.
%%
%%
% \documentclass[sigconf, anonymous, review]{acmart}
\documentclass[sigconf, anonymous=false, nonacm=true, pbalance=true]{acmart}
% \documentclass[sigconf]{acmart}
\usepackage{multirow}
% \usepackage[table]{xcolor}
\usepackage[most]{tcolorbox}
\usepackage{booktabs}
\usepackage{tabularx}
\usepackage{longtable}
\usepackage{array}
\usepackage{pifont}
\usepackage{makecell}
\usepackage{hyperref}

\newtcbtheorem{Finding}{\bfseries Finding}{enhanced,drop shadow={black!50!white},
  coltitle=black,
  top=0.3in,
  attach boxed title to top right=
  {xshift=0em,yshift=-\tcboxedtitleheight/2},
  boxed title style={size=small,colback=darkgray}
}{finding}

%% NS: Takeaway macros

\usepackage{tcolorbox}
\definecolor{verylightgrey}{gray}{0.9}
\newtcolorbox{takeaway}{
  colback=verylightgrey,
  colframe=verylightgrey,
  sharp corners,
  boxrule=0mm,
  boxsep=0mm,
  left=1mm,
  right=1mm,
  top=1mm,
  bottom=1mm
}

\newcounter{myfindingscounter}
\setcounter{myfindingscounter}{0}
\newcommand\takeawaytitle[1]{\textbf{Finding \refstepcounter{myfindingscounter}\themyfindingscounter\label{#1}:}}

% Usage:
% \begin{takeaway}
% Write your take away.
% \end{takeaway}

%%

\usepackage{xspace}

\newcommand{\mypara}[1]{\vspace{4pt}\noindent{\textbf{{#1}}\xspace}}

%%
%% \BibTeX command to typeset BibTeX logo in the docs
\AtBeginDocument{%
  \providecommand\BibTeX{{%
    Bib\TeX}}}

%% Rights management information.  This information is sent to you
%% when you complete the rights form.  These commands have SAMPLE
%% values in them; it is your responsibility as an author to replace
%% the commands and values with those provided to you when you
%% complete the rights form.
\setcopyright{acmlicensed}
\copyrightyear{2018}
\acmYear{2018}
\acmDOI{XXXXXXX.XXXXXXX}

%% These commands are for a PROCEEDINGS abstract or paper.
% \acmConference[WWW '25]{The Web Conference}{April 28--May 2,
%   2025}{Sydney, Australia}
%%
%%  Uncomment \acmBooktitle if the title of the proceedings is different
%%  from ``Proceedings of ...''!
%%
%%\acmBooktitle{Woodstock '18: ACM Symposium on Neural Gaze Detection,
%%  June 03--05, 2018, Woodstock, NY}
\acmISBN{978-1-4503-XXXX-X/18/06}


%%
%% Submission ID.
%% Use this when submitting an article to a sponsored event. You'll
%% receive a unique submission ID from the organizers
%% of the event, and this ID should be used as the parameter to this command.
\acmSubmissionID{2085}

%%
%% For managing citations, it is recommended to use bibliography
%% files in BibTeX format.
%%
%% You can then either use BibTeX with the ACM-Reference-Format style,
%% or BibLaTeX with the acmnumeric or acmauthoryear sytles, that include
%% support for advanced citation of software artefact from the
%% biblatex-software package, also separately available on CTAN.
%%
%% Look at the sample-*-biblatex.tex files for templates showcasing
%% the biblatex styles.
%%

%%
%% The majority of ACM publications use numbered citations and
%% references.  The command \citestyle{authoryear} switches to the
%% "author year" style.
%%
%% If you are preparing content for an event
%% sponsored by ACM SIGGRAPH, you must use the "author year" style of
%% citations and references.
%% Uncommenting
%% the next command will enable that style.
%%\citestyle{acmauthoryear}
\newcommand{\supplementarysection}{%
  \setcounter{figure}{0}% Reset figure counter
  \let\oldthefigure\thefigure% Capture figure numbering scheme
  \renewcommand{\thefigure}{S\oldthefigure}% Prefix figure number with S
  \setcounter{table}{0}% Reset figure counter
  \let\oldthetable\thetable%
  \renewcommand{\thetable}{S\oldthetable}%
}


%%
%% end of the preamble, start of the body of the document source.
\begin{document}

%%
%% The "title" command has an optional parameter,
%% allowing the author to define a "short title" to be used in page headers.
\title{Digital Gatekeeping: An Audit of Search Engine Results shows tailoring of queries on the Israel-Palestine Conflict}

%%
%% The "author" command and its associated commands are used to define
%% the authors and their affiliations.
%% Of note is the shared affiliation of the first two authors, and the
%% "authornote" and "authornotemark" commands
%% used to denote shared contribution to the research.
\author{Íris Damião}
% \authornote{Both authors contributed equally to this research.}
\orcid{0009-0005-4931-2376}
\affiliation{
  \institution{LIP - Laboratory for Instrumentation and Particle Physics \&  Instituto Superior Técnico - University of Lisbon}
  \city{Lisbon}
  \country{Portugal}}
\email{irisdamiao@lip.pt}

\author{José M. Reis}
\orcid{0000-0002-8055-0170}
\affiliation{
 \institution{LIP, Laboratory for Instrumentation and Particle Physics*}
 \city{Lisbon}
 \country{Portugal}}
 \thanks{* current affiliation: Portuguese National Cybersecurity Centre, Lisbon, Portugal.}
 
\author{Paulo Almeida}
\orcid{0000-0002-9279-2353}
\affiliation{
 \institution{LIP, Laboratory for Instrumentation and Particle Physics}
 \city{Lisbon}
 \country{Portugal}}

\author{Nuno Santos}
\orcid{0000-0001-9938-0653}
\affiliation{%
  \institution{INESC-ID \& Instituto Superior Técnico, University of Lisbon}
  \city{Lisbon}
  \country{Portugal}
}

\author{Joana Gonçalves-Sá}
\orcid{0000-0001-6654-2126}
\affiliation{
  \institution{LIP, Laboratory for Instrumentation and Particle Physics \& NOVA LINCS, FCT NOVA University}
  \city{Lisbon \& Caparica}
  \country{Portugal}}
\email{joanagsa@lip.pt}

%%
%% By default, the full list of authors will be used in the page
%% headers. Often, this list is too long, and will overlap
%% other information printed in the page headers. This command allows
%% the author to define a more concise list
%% of authors' names for this purpose.
\renewcommand{\shortauthors}{Damião et al.}

%%
%% The abstract is a short summary of the work to be presented in the
%% article.
\begin{abstract}

Search engines, often viewed as reliable gateways to information, tailor search results using customization algorithms based on user preferences, location, and more. While this can be useful for routine queries, it raises concerns when the topics are sensitive or contentious, possibly limiting exposure to diverse viewpoints and increasing polarization.

To examine the extent of this tailoring, we focused on the Israel-Palestine conflict and developed a privacy-protecting tool to audit the behavior of three search engines: DuckDuckGo, Google and Yahoo. Our study focused on two main questions: (1) How do search results for the same query about the conflict vary among different users? and (2) Are these results influenced by the user's location and browsing history?

Our findings revealed significant customization based on location and browsing preferences, unlike previous studies that found only mild personalization for general topics. Moreover, queries related to the conflict were more customized than unrelated queries, and the results were not neutral concerning the conflict's portrayal.

%Search engines have become integral tools in our daily lives, often recognized as reliable gateways to obtain the “truth”. However, it is well established that these algorithms apply personalization techniques to prioritize search results based on user preferences, location, and more. This tailored experience, whether useful in routine tasks, raises strong concerns regarding exposure to different viewpoints, particularly around contentious issues.

%Taking the ongoing Israel-Palestine conflict as a case - study, we built a privacy protecting search-engine audit tool to address search engine levels of personalization.
%, for the first time for such a popular, sensitive and timely subject. Specifically, as the world sought answers using such algorithms, 
%Specifically, we questioned: (1) how do search engine results vary among users seeking identical information about the conflict? and (2) are these results personalized to the user's location and past browsing history?. 
%personalization levels different for broader/non-specific topics? 
%We found significant levels of customization due to location and browsing settings, contrasting with previous research that suggested milder customization for more general topics. Additionally, our results show generally stronger levels of customization for queries related with the conflict than non-related queries and these results were not neutral in terms of conflict leaning.
%This comprehensive analysis, coupled with the contemporaneous nature of the topic, is key to understand the scope and potential implications of search engine personalization.

\end{abstract}

%% ATTENTION - CHANGE BEFORE SUBMISSION
%% The code below is generated by the tool at http://dl.acm.org/ccs.cfm.
%% Please copy and paste the code instead of the example below.
%%
% \begin{CCSXML}
% <ccs2012>
%  <concept>
%   <concept_id>00000000.0000000.0000000</concept_id>
%   <concept_desc>Do Not Use This Code, Generate the Correct Terms for Your Paper</concept_desc>
%   <concept_significance>500</concept_significance>
%  </concept>
%  <concept>
%   <concept_id>00000000.00000000.00000000</concept_id>
%   <concept_desc>Do Not Use This Code, Generate the Correct Terms for Your Paper</concept_desc>
%   <concept_significance>300</concept_significance>
%  </concept>
%  <concept>
%   <concept_id>00000000.00000000.00000000</concept_id>
%   <concept_desc>Do Not Use This Code, Generate the Correct Terms for Your Paper</concept_desc>
%   <concept_significance>100</concept_significance>
%  </concept>
%  <concept>
%   <concept_id>00000000.00000000.00000000</concept_id>
%   <concept_desc>Do Not Use This Code, Generate the Correct Terms for Your Paper</concept_desc>
%   <concept_significance>100</concept_significance>
%  </concept>
% </ccs2012>
% \end{CCSXML}

% \ccsdesc[500]{Do Not Use This Code~Generate the Correct Terms for Your Paper}
% \ccsdesc[300]{Do Not Use This Code~Generate the Correct Terms for Your Paper}
% \ccsdesc{Do Not Use This Code~Generate the Correct Terms for Your Paper}
% \ccsdesc[100]{Do Not Use This Code~Generate the Correct Terms for Your Paper}

%%
%% Keywords. The author(s) should pick words that accurately describe
%% the work being presented. Separate the keywords with commas.
\keywords{Personalization, Search engines, Audit, Filter Bubble Effect}
% %% A "teaser" image appears between the author and affiliation
% %% information and the body of the document, and typically spans the
% %% page.
% \begin{teaserfigure}
%   \includegraphics[width=\textwidth]{sampleteaser}
%   \caption{Seattle Mariners at Spring Training, 2010.}
%   \Description{Enjoying the baseball game from the third-base
%   seats. Ichiro Suzuki preparing to bat.}
%   \label{fig:teaser}
% \end{teaserfigure}

% \received{20 February 2007}
% \received[revised]{12 March 2009}
% \received[accepted]{5 June 2009}

%%
%% This command processes the author and affiliation and title
%% information and builds the first part of the formatted document.
\maketitle

\section{Introduction}
\section{Introduction}
\label{sec:introduction}
The business processes of organizations are experiencing ever-increasing complexity due to the large amount of data, high number of users, and high-tech devices involved \cite{martin2021pmopportunitieschallenges, beerepoot2023biggestbpmproblems}. This complexity may cause business processes to deviate from normal control flow due to unforeseen and disruptive anomalies \cite{adams2023proceddsriftdetection}. These control-flow anomalies manifest as unknown, skipped, and wrongly-ordered activities in the traces of event logs monitored from the execution of business processes \cite{ko2023adsystematicreview}. For the sake of clarity, let us consider an illustrative example of such anomalies. Figure \ref{FP_ANOMALIES} shows a so-called event log footprint, which captures the control flow relations of four activities of a hypothetical event log. In particular, this footprint captures the control-flow relations between activities \texttt{a}, \texttt{b}, \texttt{c} and \texttt{d}. These are the causal ($\rightarrow$) relation, concurrent ($\parallel$) relation, and other ($\#$) relations such as exclusivity or non-local dependency \cite{aalst2022pmhandbook}. In addition, on the right are six traces, of which five exhibit skipped, wrongly-ordered and unknown control-flow anomalies. For example, $\langle$\texttt{a b d}$\rangle$ has a skipped activity, which is \texttt{c}. Because of this skipped activity, the control-flow relation \texttt{b}$\,\#\,$\texttt{d} is violated, since \texttt{d} directly follows \texttt{b} in the anomalous trace.
\begin{figure}[!t]
\centering
\includegraphics[width=0.9\columnwidth]{images/FP_ANOMALIES.png}
\caption{An example event log footprint with six traces, of which five exhibit control-flow anomalies.}
\label{FP_ANOMALIES}
\end{figure}

\subsection{Control-flow anomaly detection}
Control-flow anomaly detection techniques aim to characterize the normal control flow from event logs and verify whether these deviations occur in new event logs \cite{ko2023adsystematicreview}. To develop control-flow anomaly detection techniques, \revision{process mining} has seen widespread adoption owing to process discovery and \revision{conformance checking}. On the one hand, process discovery is a set of algorithms that encode control-flow relations as a set of model elements and constraints according to a given modeling formalism \cite{aalst2022pmhandbook}; hereafter, we refer to the Petri net, a widespread modeling formalism. On the other hand, \revision{conformance checking} is an explainable set of algorithms that allows linking any deviations with the reference Petri net and providing the fitness measure, namely a measure of how much the Petri net fits the new event log \cite{aalst2022pmhandbook}. Many control-flow anomaly detection techniques based on \revision{conformance checking} (hereafter, \revision{conformance checking}-based techniques) use the fitness measure to determine whether an event log is anomalous \cite{bezerra2009pmad, bezerra2013adlogspais, myers2018icsadpm, pecchia2020applicationfailuresanalysispm}. 

The scientific literature also includes many \revision{conformance checking}-independent techniques for control-flow anomaly detection that combine specific types of trace encodings with machine/deep learning \cite{ko2023adsystematicreview, tavares2023pmtraceencoding}. Whereas these techniques are very effective, their explainability is challenging due to both the type of trace encoding employed and the machine/deep learning model used \cite{rawal2022trustworthyaiadvances,li2023explainablead}. Hence, in the following, we focus on the shortcomings of \revision{conformance checking}-based techniques to investigate whether it is possible to support the development of competitive control-flow anomaly detection techniques while maintaining the explainable nature of \revision{conformance checking}.
\begin{figure}[!t]
\centering
\includegraphics[width=\columnwidth]{images/HIGH_LEVEL_VIEW.png}
\caption{A high-level view of the proposed framework for combining \revision{process mining}-based feature extraction with dimensionality reduction for control-flow anomaly detection.}
\label{HIGH_LEVEL_VIEW}
\end{figure}

\subsection{Shortcomings of \revision{conformance checking}-based techniques}
Unfortunately, the detection effectiveness of \revision{conformance checking}-based techniques is affected by noisy data and low-quality Petri nets, which may be due to human errors in the modeling process or representational bias of process discovery algorithms \cite{bezerra2013adlogspais, pecchia2020applicationfailuresanalysispm, aalst2016pm}. Specifically, on the one hand, noisy data may introduce infrequent and deceptive control-flow relations that may result in inconsistent fitness measures, whereas, on the other hand, checking event logs against a low-quality Petri net could lead to an unreliable distribution of fitness measures. Nonetheless, such Petri nets can still be used as references to obtain insightful information for \revision{process mining}-based feature extraction, supporting the development of competitive and explainable \revision{conformance checking}-based techniques for control-flow anomaly detection despite the problems above. For example, a few works outline that token-based \revision{conformance checking} can be used for \revision{process mining}-based feature extraction to build tabular data and develop effective \revision{conformance checking}-based techniques for control-flow anomaly detection \cite{singh2022lapmsh, debenedictis2023dtadiiot}. However, to the best of our knowledge, the scientific literature lacks a structured proposal for \revision{process mining}-based feature extraction using the state-of-the-art \revision{conformance checking} variant, namely alignment-based \revision{conformance checking}.

\subsection{Contributions}
We propose a novel \revision{process mining}-based feature extraction approach with alignment-based \revision{conformance checking}. This variant aligns the deviating control flow with a reference Petri net; the resulting alignment can be inspected to extract additional statistics such as the number of times a given activity caused mismatches \cite{aalst2022pmhandbook}. We integrate this approach into a flexible and explainable framework for developing techniques for control-flow anomaly detection. The framework combines \revision{process mining}-based feature extraction and dimensionality reduction to handle high-dimensional feature sets, achieve detection effectiveness, and support explainability. Notably, in addition to our proposed \revision{process mining}-based feature extraction approach, the framework allows employing other approaches, enabling a fair comparison of multiple \revision{conformance checking}-based and \revision{conformance checking}-independent techniques for control-flow anomaly detection. Figure \ref{HIGH_LEVEL_VIEW} shows a high-level view of the framework. Business processes are monitored, and event logs obtained from the database of information systems. Subsequently, \revision{process mining}-based feature extraction is applied to these event logs and tabular data input to dimensionality reduction to identify control-flow anomalies. We apply several \revision{conformance checking}-based and \revision{conformance checking}-independent framework techniques to publicly available datasets, simulated data of a case study from railways, and real-world data of a case study from healthcare. We show that the framework techniques implementing our approach outperform the baseline \revision{conformance checking}-based techniques while maintaining the explainable nature of \revision{conformance checking}.

In summary, the contributions of this paper are as follows.
\begin{itemize}
    \item{
        A novel \revision{process mining}-based feature extraction approach to support the development of competitive and explainable \revision{conformance checking}-based techniques for control-flow anomaly detection.
    }
    \item{
        A flexible and explainable framework for developing techniques for control-flow anomaly detection using \revision{process mining}-based feature extraction and dimensionality reduction.
    }
    \item{
        Application to synthetic and real-world datasets of several \revision{conformance checking}-based and \revision{conformance checking}-independent framework techniques, evaluating their detection effectiveness and explainability.
    }
\end{itemize}

The rest of the paper is organized as follows.
\begin{itemize}
    \item Section \ref{sec:related_work} reviews the existing techniques for control-flow anomaly detection, categorizing them into \revision{conformance checking}-based and \revision{conformance checking}-independent techniques.
    \item Section \ref{sec:abccfe} provides the preliminaries of \revision{process mining} to establish the notation used throughout the paper, and delves into the details of the proposed \revision{process mining}-based feature extraction approach with alignment-based \revision{conformance checking}.
    \item Section \ref{sec:framework} describes the framework for developing \revision{conformance checking}-based and \revision{conformance checking}-independent techniques for control-flow anomaly detection that combine \revision{process mining}-based feature extraction and dimensionality reduction.
    \item Section \ref{sec:evaluation} presents the experiments conducted with multiple framework and baseline techniques using data from publicly available datasets and case studies.
    \item Section \ref{sec:conclusions} draws the conclusions and presents future work.
\end{itemize}

\vspace{-0.1cm}
\section{Background and Related Work}
\section{RELATED WORK}
\label{sec:relatedwork}
In this section, we describe the previous works related to our proposal, which are divided into two parts. In Section~\ref{sec:relatedwork_exoplanet}, we present a review of approaches based on machine learning techniques for the detection of planetary transit signals. Section~\ref{sec:relatedwork_attention} provides an account of the approaches based on attention mechanisms applied in Astronomy.\par

\subsection{Exoplanet detection}
\label{sec:relatedwork_exoplanet}
Machine learning methods have achieved great performance for the automatic selection of exoplanet transit signals. One of the earliest applications of machine learning is a model named Autovetter \citep{MCcauliff}, which is a random forest (RF) model based on characteristics derived from Kepler pipeline statistics to classify exoplanet and false positive signals. Then, other studies emerged that also used supervised learning. \cite{mislis2016sidra} also used a RF, but unlike the work by \citet{MCcauliff}, they used simulated light curves and a box least square \citep[BLS;][]{kovacs2002box}-based periodogram to search for transiting exoplanets. \citet{thompson2015machine} proposed a k-nearest neighbors model for Kepler data to determine if a given signal has similarity to known transits. Unsupervised learning techniques were also applied, such as self-organizing maps (SOM), proposed \citet{armstrong2016transit}; which implements an architecture to segment similar light curves. In the same way, \citet{armstrong2018automatic} developed a combination of supervised and unsupervised learning, including RF and SOM models. In general, these approaches require a previous phase of feature engineering for each light curve. \par

%DL is a modern data-driven technology that automatically extracts characteristics, and that has been successful in classification problems from a variety of application domains. The architecture relies on several layers of NNs of simple interconnected units and uses layers to build increasingly complex and useful features by means of linear and non-linear transformation. This family of models is capable of generating increasingly high-level representations \citep{lecun2015deep}.

The application of DL for exoplanetary signal detection has evolved rapidly in recent years and has become very popular in planetary science.  \citet{pearson2018} and \citet{zucker2018shallow} developed CNN-based algorithms that learn from synthetic data to search for exoplanets. Perhaps one of the most successful applications of the DL models in transit detection was that of \citet{Shallue_2018}; who, in collaboration with Google, proposed a CNN named AstroNet that recognizes exoplanet signals in real data from Kepler. AstroNet uses the training set of labelled TCEs from the Autovetter planet candidate catalog of Q1–Q17 data release 24 (DR24) of the Kepler mission \citep{catanzarite2015autovetter}. AstroNet analyses the data in two views: a ``global view'', and ``local view'' \citep{Shallue_2018}. \par


% The global view shows the characteristics of the light curve over an orbital period, and a local view shows the moment at occurring the transit in detail

%different = space-based

Based on AstroNet, researchers have modified the original AstroNet model to rank candidates from different surveys, specifically for Kepler and TESS missions. \citet{ansdell2018scientific} developed a CNN trained on Kepler data, and included for the first time the information on the centroids, showing that the model improves performance considerably. Then, \citet{osborn2020rapid} and \citet{yu2019identifying} also included the centroids information, but in addition, \citet{osborn2020rapid} included information of the stellar and transit parameters. Finally, \citet{rao2021nigraha} proposed a pipeline that includes a new ``half-phase'' view of the transit signal. This half-phase view represents a transit view with a different time and phase. The purpose of this view is to recover any possible secondary eclipse (the object hiding behind the disk of the primary star).


%last pipeline applies a procedure after the prediction of the model to obtain new candidates, this process is carried out through a series of steps that include the evaluation with Discovery and Validation of Exoplanets (DAVE) \citet{kostov2019discovery} that was adapted for the TESS telescope.\par
%



\subsection{Attention mechanisms in astronomy}
\label{sec:relatedwork_attention}
Despite the remarkable success of attention mechanisms in sequential data, few papers have exploited their advantages in astronomy. In particular, there are no models based on attention mechanisms for detecting planets. Below we present a summary of the main applications of this modeling approach to astronomy, based on two points of view; performance and interpretability of the model.\par
%Attention mechanisms have not yet been explored in all sub-areas of astronomy. However, recent works show a successful application of the mechanism.
%performance

The application of attention mechanisms has shown improvements in the performance of some regression and classification tasks compared to previous approaches. One of the first implementations of the attention mechanism was to find gravitational lenses proposed by \citet{thuruthipilly2021finding}. They designed 21 self-attention-based encoder models, where each model was trained separately with 18,000 simulated images, demonstrating that the model based on the Transformer has a better performance and uses fewer trainable parameters compared to CNN. A novel application was proposed by \citet{lin2021galaxy} for the morphological classification of galaxies, who used an architecture derived from the Transformer, named Vision Transformer (VIT) \citep{dosovitskiy2020image}. \citet{lin2021galaxy} demonstrated competitive results compared to CNNs. Another application with successful results was proposed by \citet{zerveas2021transformer}; which first proposed a transformer-based framework for learning unsupervised representations of multivariate time series. Their methodology takes advantage of unlabeled data to train an encoder and extract dense vector representations of time series. Subsequently, they evaluate the model for regression and classification tasks, demonstrating better performance than other state-of-the-art supervised methods, even with data sets with limited samples.

%interpretation
Regarding the interpretability of the model, a recent contribution that analyses the attention maps was presented by \citet{bowles20212}, which explored the use of group-equivariant self-attention for radio astronomy classification. Compared to other approaches, this model analysed the attention maps of the predictions and showed that the mechanism extracts the brightest spots and jets of the radio source more clearly. This indicates that attention maps for prediction interpretation could help experts see patterns that the human eye often misses. \par

In the field of variable stars, \citet{allam2021paying} employed the mechanism for classifying multivariate time series in variable stars. And additionally, \citet{allam2021paying} showed that the activation weights are accommodated according to the variation in brightness of the star, achieving a more interpretable model. And finally, related to the TESS telescope, \citet{morvan2022don} proposed a model that removes the noise from the light curves through the distribution of attention weights. \citet{morvan2022don} showed that the use of the attention mechanism is excellent for removing noise and outliers in time series datasets compared with other approaches. In addition, the use of attention maps allowed them to show the representations learned from the model. \par

Recent attention mechanism approaches in astronomy demonstrate comparable results with earlier approaches, such as CNNs. At the same time, they offer interpretability of their results, which allows a post-prediction analysis. \par



\section{Methodology}
\section{Research Methodology}~\label{sec:Methodology}

In this section, we discuss the process of conducting our systematic review, e.g., our search strategy for data extraction of relevant studies, based on the guidelines of Kitchenham et al.~\cite{kitchenham2022segress} to conduct SLRs and Petersen et al.~\cite{PETERSEN20151} to conduct systematic mapping studies (SMSs) in Software Engineering. In this systematic review, we divide our work into a four-stage procedure, including planning, conducting, building a taxonomy, and reporting the review, illustrated in Fig.~\ref{fig:search}. The four stages are as follows: (1) the \emph{planning} stage involved identifying research questions (RQs) and specifying the detailed research plan for the study; (2) the \emph{conducting} stage involved analyzing and synthesizing the existing primary studies to answer the research questions; (3) the \emph{taxonomy} stage was introduced to optimize the data extraction results and consolidate a taxonomy schema for REDAST methodology; (4) the \emph{reporting} stage involved the reviewing, concluding and reporting the final result of our study.

\begin{figure}[!t]
    \centering
    \includegraphics[width=1\linewidth]{fig/methodology/searching-process.drawio.pdf}
    \caption{Systematic Literature Review Process}
    \label{fig:search}
\end{figure}

\subsection{Research Questions}
In this study, we developed five research questions (RQs) to identify the input and output, analyze technologies, evaluate metrics, identify challenges, and identify potential opportunities. 

\textbf{RQ1. What are the input configurations, formats, and notations used in the requirements in requirements-driven
automated software testing?} In requirements-driven testing, the input is some form of requirements specification -- which can vary significantly. RQ1 maps the input for REDAST and reports on the comparison among different formats for requirements specification.

\textbf{RQ2. What are the frameworks, tools, processing methods, and transformation techniques used in requirements-driven automated software testing studies?} RQ2 explores the technical solutions from requirements to generated artifacts, e.g., rule-based transformation applying natural language processing (NLP) pipelines and deep learning (DL) techniques, where we additionally discuss the potential intermediate representation and additional input for the transformation process.

\textbf{RQ3. What are the test formats and coverage criteria used in the requirements-driven automated software
testing process?} RQ3 focuses on identifying the formulation of generated artifacts (i.e., the final output). We map the adopted test formats and analyze their characteristics in the REDAST process.

\textbf{RQ4. How do existing studies evaluate the generated test artifacts in the requirements-driven automated software testing process?} RQ4 identifies the evaluation datasets, metrics, and case study methodologies in the selected papers. This aims to understand how researchers assess the effectiveness, accuracy, and practical applicability of the generated test artifacts.

\textbf{RQ5. What are the limitations and challenges of existing requirements-driven automated software testing methods in the current era?} RQ5 addresses the limitations and challenges of existing studies while exploring future directions in the current era of technology development. %It particularly highlights the potential benefits of advanced LLMs and examines their capacity to meet the high expectations placed on these cutting-edge language modeling technologies. %\textcolor{blue}{CA: Do we really need to focus on LLMs? TBD.} \textcolor{orange}{FW: About LLMs, I removed the direct emphase in RQ5 but kept the discussion in RQ5 and the solution section. I think that would be more appropriate.}

\subsection{Searching Strategy}

The overview of the search process is exhibited in Fig. \ref{fig:papers}, which includes all the details of our search steps.
\begin{table}[!ht]
\caption{List of Search Terms}
\label{table:search_term}
\begin{tabularx}{\textwidth}{lX}
\hline
\textbf{Terms Group} & \textbf{Terms} \\ \hline
Test Group & test* \\
Requirement Group & requirement* OR use case* OR user stor* OR specification* \\
Software Group & software* OR system* \\
Method Group & generat* OR deriv* OR map* OR creat* OR extract* OR design* OR priorit* OR construct* OR transform* \\ \hline
\end{tabularx}
\end{table}

\begin{figure}
    \centering
    \includegraphics[width=1\linewidth]{fig/methodology/search-papers.drawio.pdf}
    \caption{Study Search Process}
    \label{fig:papers}
\end{figure}

\subsubsection{Search String Formulation}
Our research questions (RQs) guided the identification of the main search terms. We designed our search string with generic keywords to avoid missing out on any related papers, where four groups of search terms are included, namely ``test group'', ``requirement group'', ``software group'', and ``method group''. In order to capture all the expressions of the search terms, we use wildcards to match the appendix of the word, e.g., ``test*'' can capture ``testing'', ``tests'' and so on. The search terms are listed in Table~\ref{table:search_term}, decided after iterative discussion and refinement among all the authors. As a result, we finally formed the search string as follows:


\hangindent=1.5em
 \textbf{ON ABSTRACT} ((``test*'') \textbf{AND} (``requirement*'' \textbf{OR} ``use case*'' \textbf{OR} ``user stor*'' \textbf{OR} ``specifications'') \textbf{AND} (``software*'' \textbf{OR} ``system*'') \textbf{AND} (``generat*'' \textbf{OR} ``deriv*'' \textbf{OR} ``map*'' \textbf{OR} ``creat*'' \textbf{OR} ``extract*'' \textbf{OR} ``design*'' \textbf{OR} ``priorit*'' \textbf{OR} ``construct*'' \textbf{OR} ``transform*''))

The search process was conducted in September 2024, and therefore, the search results reflect studies available up to that date. We conducted the search process on six online databases: IEEE Xplore, ACM Digital Library, Wiley, Scopus, Web of Science, and Science Direct. However, some databases were incompatible with our default search string in the following situations: (1) unsupported for searching within abstract, such as Scopus, and (2) limited search terms, such as ScienceDirect. Here, for (1) situation, we searched within the title, keyword, and abstract, and for (2) situation, we separately executed the search and removed the duplicate papers in the merging process. 

\subsubsection{Automated Searching and Duplicate Removal}
We used advanced search to execute our search string within our selected databases, following our designed selection criteria in Table \ref{table:selection}. The first search returned 27,333 papers. Specifically for the duplicate removal, we used a Python script to remove (1) overlapped search results among multiple databases and (2) conference or workshop papers, also found with the same title and authors in the other journals. After duplicate removal, we obtained 21,652 papers for further filtering.

\begin{table*}[]
\caption{Selection Criteria}
\label{table:selection}
\begin{tabularx}{\textwidth}{lX}
\hline
\textbf{Criterion ID} & \textbf{Criterion Description} \\ \hline
S01          & Papers written in English. \\
S02-1        & Papers in the subjects of "Computer Science" or "Software Engineering". \\
S02-2        & Papers published on software testing-related issues. \\
S03          & Papers published from 1991 to the present. \\ 
S04          & Papers with accessible full text. \\ \hline
\end{tabularx}
\end{table*}

\begin{table*}[]
\small
\caption{Inclusion and Exclusion Criteria}
\label{table:criteria}
\begin{tabularx}{\textwidth}{lX}
\hline
\textbf{ID}  & \textbf{Description} \\ \hline
\multicolumn{2}{l}{\textbf{Inclusion Criteria}} \\ \hline
I01 & Papers about requirements-driven automated system testing or acceptance testing generation, or studies that generate system-testing-related artifacts. \\
I02 & Peer-reviewed studies that have been used in academia with references from literature. \\ \hline
\multicolumn{2}{l}{\textbf{Exclusion Criteria}} \\ \hline
E01 & Studies that only support automated code generation, but not test-artifact generation. \\
E02 & Studies that do not use requirements-related information as an input. \\
E03 & Papers with fewer than 5 pages (1-4 pages). \\
E04 & Non-primary studies (secondary or tertiary studies). \\
E05 & Vision papers and grey literature (unpublished work), books (chapters), posters, discussions, opinions, keynotes, magazine articles, experience, and comparison papers. \\ \hline
\end{tabularx}
\end{table*}

\subsubsection{Filtering Process}

In this step, we filtered a total of 21,652 papers using the inclusion and exclusion criteria outlined in Table \ref{table:criteria}. This process was primarily carried out by the first and second authors. Our criteria are structured at different levels, facilitating a multi-step filtering process. This approach involves applying various criteria in three distinct phases. We employed a cross-verification method involving (1) the first and second authors and (2) the other authors. Initially, the filtering was conducted separately by the first and second authors. After cross-verifying their results, the results were then reviewed and discussed further by the other authors for final decision-making. We widely adopted this verification strategy within the filtering stages. During the filtering process, we managed our paper list using a BibTeX file and categorized the papers with color-coding through BibTeX management software\footnote{\url{https://bibdesk.sourceforge.io/}}, i.e., “red” for irrelevant papers, “yellow” for potentially relevant papers, and “blue” for relevant papers. This color-coding system facilitated the organization and review of papers according to their relevance.

The screening process is shown below,
\begin{itemize}
    \item \textbf{1st-round Filtering} was based on the title and abstract, using the criteria I01 and E01. At this stage, the number of papers was reduced from 21,652 to 9,071.
    \item \textbf{2nd-round Filtering}. We attempted to include requirements-related papers based on E02 on the title and abstract level, which resulted from 9,071 to 4,071 papers. We excluded all the papers that did not focus on requirements-related information as an input or only mentioned the term ``requirements'' but did not refer to the requirements specification.
    \item \textbf{3rd-round Filtering}. We selectively reviewed the content of papers identified as potentially relevant to requirements-driven automated test generation. This process resulted in 162 papers for further analysis.
\end{itemize}
Note that, especially for third-round filtering, we aimed to include as many relevant papers as possible, even borderline cases, according to our criteria. The results were then discussed iteratively among all the authors to reach a consensus.

\subsubsection{Snowballing}

Snowballing is necessary for identifying papers that may have been missed during the automated search. Following the guidelines by Wohlin~\cite{wohlin2014guidelines}, we conducted both forward and backward snowballing. As a result, we identified 24 additional papers through this process.

\subsubsection{Data Extraction}

Based on the formulated research questions (RQs), we designed 38 data extraction questions\footnote{\url{https://drive.google.com/file/d/1yjy-59Juu9L3WHaOPu-XQo-j-HHGTbx_/view?usp=sharing}} and created a Google Form to collect the required information from the relevant papers. The questions included 30 short-answer questions, six checkbox questions, and two selection questions. The data extraction was organized into five sections: (1) basic information: fundamental details such as title, author, venue, etc.; (2) open information: insights on motivation, limitations, challenges, etc.; (3) requirements: requirements format, notation, and related aspects; (4) methodology: details, including immediate representation and technique support; (5) test-related information: test format(s), coverage, and related elements. Similar to the filtering process, the first and second authors conducted the data extraction and then forwarded the results to the other authors to initiate the review meeting.

\subsubsection{Quality Assessment}

During the data extraction process, we encountered papers with insufficient information. To address this, we conducted a quality assessment in parallel to ensure the relevance of the papers to our objectives. This approach, also adopted in previous secondary studies~\cite{shamsujjoha2021developing, naveed2024model}, involved designing a set of assessment questions based on guidelines by Kitchenham et al.~\cite{kitchenham2022segress}. The quality assessment questions in our study are shown below:
\begin{itemize}
    \item \textbf{QA1}. Does this study clearly state \emph{how} requirements drive automated test generation?
    \item \textbf{QA2}. Does this study clearly state the \emph{aim} of REDAST?
    \item \textbf{QA3}. Does this study enable \emph{automation} in test generation?
    \item \textbf{QA4}. Does this study demonstrate the usability of the method from the perspective of methodology explanation, discussion, case examples, and experiments?
\end{itemize}
QA4 originates from an open perspective in the review process, where we focused on evaluation, discussion, and explanation. Our review also examined the study’s overall structure, including the methodology description, case studies, experiments, and analyses. The detailed results of the quality assessment are provided in the Appendix. Following this assessment, the final data extraction was based on 156 papers.

% \begin{table}[]
% \begin{tabular}{ll}
% \hline
% QA ID & QA Questions                                             \\ \hline
% Q01   & Does this study clearly state its aims?                  \\
% Q02   & Does this study clearly describe its methodology?        \\
% Q03   & Does this study involve automated test generation?       \\
% Q04   & Does this study include a promising evaluation?          \\
% Q05   & Does this study demonstrate the usability of the method? \\ \hline
% \end{tabular}%
% \caption{Questions for Quality Assessment}
% \label{table:qa}
% \end{table}

% automated quality assessment

% \textcolor{blue}{CA: Our search strategy focused on identifying requirements types first. We covered several sources, e.g., ~\cite{Pohl:11,wagner2019status} to identify different formats and notations of specifying requirements. However, this came out to be a long list, e.g., free-form NL requirements, semi-formal UML models, free-from textual use case models, UML class diagrams, UML activity diagrams, and so on. In this paper, we attempted to primarily focus on requirements-related aspects and not design-level information. Hence, we generalised our search string to include generic keywords, e.g., requirement*, use case*, and user stor*. We did so to avoid missing out on any papers, bringing too restrictive in our search strategy, and not creating a too-generic search string with all the aforementioned formats to avoid getting results beyond our review's scope.}


%% Use \subsection commands to start a subsection.



%\subsection{Study Selection}

% In this step, we further looked into the content of searched papers using our search strategy and applied our inclusion and exclusion criteria. Our filtering strategy aimed to pinpoint studies focused on requirements-driven system-level testing. Recognizing the presence of irrelevant papers in our search results, we established detailed selection criteria for preliminary inclusion and exclusion, as shown in Table \ref{table: criteria}. Specifically, we further developed the taxonomy schema to exclude two types of studies that did not meet the requirements for system-level testing: (1) studies supporting specification-driven test generation, such as UML-driven test generation, rather than requirements-driven testing, and (2) studies focusing on code-based test generation, such as requirement-driven code generation for unit testing.





% \subsection{Ethics}
% This research focuses on analyzing the internal representations of numerical values in Large Language Models and does not involve human subjects, sensitive data, or direct societal impact.

\section{Results}

\begin{table*}[t]
\centering
\fontsize{11pt}{11pt}\selectfont
\begin{tabular}{lllllllllllll}
\toprule
\multicolumn{1}{c}{\textbf{task}} & \multicolumn{2}{c}{\textbf{Mir}} & \multicolumn{2}{c}{\textbf{Lai}} & \multicolumn{2}{c}{\textbf{Ziegen.}} & \multicolumn{2}{c}{\textbf{Cao}} & \multicolumn{2}{c}{\textbf{Alva-Man.}} & \multicolumn{1}{c}{\textbf{avg.}} & \textbf{\begin{tabular}[c]{@{}l@{}}avg.\\ rank\end{tabular}} \\
\multicolumn{1}{c}{\textbf{metrics}} & \multicolumn{1}{c}{\textbf{cor.}} & \multicolumn{1}{c}{\textbf{p-v.}} & \multicolumn{1}{c}{\textbf{cor.}} & \multicolumn{1}{c}{\textbf{p-v.}} & \multicolumn{1}{c}{\textbf{cor.}} & \multicolumn{1}{c}{\textbf{p-v.}} & \multicolumn{1}{c}{\textbf{cor.}} & \multicolumn{1}{c}{\textbf{p-v.}} & \multicolumn{1}{c}{\textbf{cor.}} & \multicolumn{1}{c}{\textbf{p-v.}} &  &  \\ \midrule
\textbf{S-Bleu} & 0.50 & 0.0 & 0.47 & 0.0 & 0.59 & 0.0 & 0.58 & 0.0 & 0.68 & 0.0 & 0.57 & 5.8 \\
\textbf{R-Bleu} & -- & -- & 0.27 & 0.0 & 0.30 & 0.0 & -- & -- & -- & -- & - &  \\
\textbf{S-Meteor} & 0.49 & 0.0 & 0.48 & 0.0 & 0.61 & 0.0 & 0.57 & 0.0 & 0.64 & 0.0 & 0.56 & 6.1 \\
\textbf{R-Meteor} & -- & -- & 0.34 & 0.0 & 0.26 & 0.0 & -- & -- & -- & -- & - &  \\
\textbf{S-Bertscore} & \textbf{0.53} & 0.0 & {\ul 0.80} & 0.0 & \textbf{0.70} & 0.0 & {\ul 0.66} & 0.0 & {\ul0.78} & 0.0 & \textbf{0.69} & \textbf{1.7} \\
\textbf{R-Bertscore} & -- & -- & 0.51 & 0.0 & 0.38 & 0.0 & -- & -- & -- & -- & - &  \\
\textbf{S-Bleurt} & {\ul 0.52} & 0.0 & {\ul 0.80} & 0.0 & 0.60 & 0.0 & \textbf{0.70} & 0.0 & \textbf{0.80} & 0.0 & {\ul 0.68} & {\ul 2.3} \\
\textbf{R-Bleurt} & -- & -- & 0.59 & 0.0 & -0.05 & 0.13 & -- & -- & -- & -- & - &  \\
\textbf{S-Cosine} & 0.51 & 0.0 & 0.69 & 0.0 & {\ul 0.62} & 0.0 & 0.61 & 0.0 & 0.65 & 0.0 & 0.62 & 4.4 \\
\textbf{R-Cosine} & -- & -- & 0.40 & 0.0 & 0.29 & 0.0 & -- & -- & -- & -- & - & \\ \midrule
\textbf{QuestEval} & 0.23 & 0.0 & 0.25 & 0.0 & 0.49 & 0.0 & 0.47 & 0.0 & 0.62 & 0.0 & 0.41 & 9.0 \\
\textbf{LLaMa3} & 0.36 & 0.0 & \textbf{0.84} & 0.0 & {\ul{0.62}} & 0.0 & 0.61 & 0.0 &  0.76 & 0.0 & 0.64 & 3.6 \\
\textbf{our (3b)} & 0.49 & 0.0 & 0.73 & 0.0 & 0.54 & 0.0 & 0.53 & 0.0 & 0.7 & 0.0 & 0.60 & 5.8 \\
\textbf{our (8b)} & 0.48 & 0.0 & 0.73 & 0.0 & 0.52 & 0.0 & 0.53 & 0.0 & 0.7 & 0.0 & 0.59 & 6.3 \\  \bottomrule
\end{tabular}
\caption{Pearson correlation on human evaluation on system output. `R-': reference-based. `S-': source-based.}
\label{tab:sys}
\end{table*}



\begin{table}%[]
\centering
\fontsize{11pt}{11pt}\selectfont
\begin{tabular}{llllll}
\toprule
\multicolumn{1}{c}{\textbf{task}} & \multicolumn{1}{c}{\textbf{Lai}} & \multicolumn{1}{c}{\textbf{Zei.}} & \multicolumn{1}{c}{\textbf{Scia.}} & \textbf{} & \textbf{} \\ 
\multicolumn{1}{c}{\textbf{metrics}} & \multicolumn{1}{c}{\textbf{cor.}} & \multicolumn{1}{c}{\textbf{cor.}} & \multicolumn{1}{c}{\textbf{cor.}} & \textbf{avg.} & \textbf{\begin{tabular}[c]{@{}l@{}}avg.\\ rank\end{tabular}} \\ \midrule
\textbf{S-Bleu} & 0.40 & 0.40 & 0.19* & 0.33 & 7.67 \\
\textbf{S-Meteor} & 0.41 & 0.42 & 0.16* & 0.33 & 7.33 \\
\textbf{S-BertS.} & {\ul0.58} & 0.47 & 0.31 & 0.45 & 3.67 \\
\textbf{S-Bleurt} & 0.45 & {\ul 0.54} & {\ul 0.37} & 0.45 & {\ul 3.33} \\
\textbf{S-Cosine} & 0.56 & 0.52 & 0.3 & {\ul 0.46} & {\ul 3.33} \\ \midrule
\textbf{QuestE.} & 0.27 & 0.35 & 0.06* & 0.23 & 9.00 \\
\textbf{LlaMA3} & \textbf{0.6} & \textbf{0.67} & \textbf{0.51} & \textbf{0.59} & \textbf{1.0} \\
\textbf{Our (3b)} & 0.51 & 0.49 & 0.23* & 0.39 & 4.83 \\
\textbf{Our (8b)} & 0.52 & 0.49 & 0.22* & 0.43 & 4.83 \\ \bottomrule
\end{tabular}
\caption{Pearson correlation on human ratings on reference output. *not significant; we cannot reject the null hypothesis of zero correlation}
\label{tab:ref}
\end{table}


\begin{table*}%[]
\centering
\fontsize{11pt}{11pt}\selectfont
\begin{tabular}{lllllllll}
\toprule
\textbf{task} & \multicolumn{1}{c}{\textbf{ALL}} & \multicolumn{1}{c}{\textbf{sentiment}} & \multicolumn{1}{c}{\textbf{detoxify}} & \multicolumn{1}{c}{\textbf{catchy}} & \multicolumn{1}{c}{\textbf{polite}} & \multicolumn{1}{c}{\textbf{persuasive}} & \multicolumn{1}{c}{\textbf{formal}} & \textbf{\begin{tabular}[c]{@{}l@{}}avg. \\ rank\end{tabular}} \\
\textbf{metrics} & \multicolumn{1}{c}{\textbf{cor.}} & \multicolumn{1}{c}{\textbf{cor.}} & \multicolumn{1}{c}{\textbf{cor.}} & \multicolumn{1}{c}{\textbf{cor.}} & \multicolumn{1}{c}{\textbf{cor.}} & \multicolumn{1}{c}{\textbf{cor.}} & \multicolumn{1}{c}{\textbf{cor.}} &  \\ \midrule
\textbf{S-Bleu} & -0.17 & -0.82 & -0.45 & -0.12* & -0.1* & -0.05 & -0.21 & 8.42 \\
\textbf{R-Bleu} & - & -0.5 & -0.45 &  &  &  &  &  \\
\textbf{S-Meteor} & -0.07* & -0.55 & -0.4 & -0.01* & 0.1* & -0.16 & -0.04* & 7.67 \\
\textbf{R-Meteor} & - & -0.17* & -0.39 & - & - & - & - & - \\
\textbf{S-BertScore} & 0.11 & -0.38 & -0.07* & -0.17* & 0.28 & 0.12 & 0.25 & 6.0 \\
\textbf{R-BertScore} & - & -0.02* & -0.21* & - & - & - & - & - \\
\textbf{S-Bleurt} & 0.29 & 0.05* & 0.45 & 0.06* & 0.29 & 0.23 & 0.46 & 4.2 \\
\textbf{R-Bleurt} & - &  0.21 & 0.38 & - & - & - & - & - \\
\textbf{S-Cosine} & 0.01* & -0.5 & -0.13* & -0.19* & 0.05* & -0.05* & 0.15* & 7.42 \\
\textbf{R-Cosine} & - & -0.11* & -0.16* & - & - & - & - & - \\ \midrule
\textbf{QuestEval} & 0.21 & {\ul{0.29}} & 0.23 & 0.37 & 0.19* & 0.35 & 0.14* & 4.67 \\
\textbf{LlaMA3} & \textbf{0.82} & \textbf{0.80} & \textbf{0.72} & \textbf{0.84} & \textbf{0.84} & \textbf{0.90} & \textbf{0.88} & \textbf{1.00} \\
\textbf{Our (3b)} & 0.47 & -0.11* & 0.37 & 0.61 & 0.53 & 0.54 & 0.66 & 3.5 \\
\textbf{Our (8b)} & {\ul{0.57}} & 0.09* & {\ul 0.49} & {\ul 0.72} & {\ul 0.64} & {\ul 0.62} & {\ul 0.67} & {\ul 2.17} \\ \bottomrule
\end{tabular}
\caption{Pearson correlation on human ratings on our constructed test set. 'R-': reference-based. 'S-': source-based. *not significant; we cannot reject the null hypothesis of zero correlation}
\label{tab:con}
\end{table*}

\section{Results}
We benchmark the different metrics on the different datasets using correlation to human judgement. For content preservation, we show results split on data with system output, reference output and our constructed test set: we show that the data source for evaluation leads to different conclusions on the metrics. In addition, we examine whether the metrics can rank style transfer systems similar to humans. On style strength, we likewise show correlations between human judgment and zero-shot evaluation approaches. When applicable, we summarize results by reporting the average correlation. And the average ranking of the metric per dataset (by ranking which metric obtains the highest correlation to human judgement per dataset). 

\subsection{Content preservation}
\paragraph{How do data sources affect the conclusion on best metric?}
The conclusions about the metrics' performance change radically depending on whether we use system output data, reference output, or our constructed test set. Ideally, a good metric correlates highly with humans on any data source. Ideally, for meta-evaluation, a metric should correlate consistently across all data sources, but the following shows that the correlations indicate different things, and the conclusion on the best metric should be drawn carefully.

Looking at the metrics correlations with humans on the data source with system output (Table~\ref{tab:sys}), we see a relatively high correlation for many of the metrics on many tasks. The overall best metrics are S-BertScore and S-BLEURT (avg+avg rank). We see no notable difference in our method of using the 3B or 8B model as the backbone.

Examining the average correlations based on data with reference output (Table~\ref{tab:ref}), now the zero-shoot prompting with LlaMA3 70B is the best-performing approach ($0.59$ avg). Tied for second place are source-based cosine embedding ($0.46$ avg), BLEURT ($0.45$ avg) and BertScore ($0.45$ avg). Our method follows on a 5. place: here, the 8b version (($0.43$ avg)) shows a bit stronger results than 3b ($0.39$ avg). The fact that the conclusions change, whether looking at reference or system output, confirms the observations made by \citet{scialom-etal-2021-questeval} on simplicity transfer.   

Now consider the results on our test set (Table~\ref{tab:con}): Several metrics show low or no correlation; we even see a significantly negative correlation for some metrics on ALL (BLEU) and for specific subparts of our test set for BLEU, Meteor, BertScore, Cosine. On the other end, LlaMA3 70B is again performing best, showing strong results ($0.82$ in ALL). The runner-up is now our 8B method, with a gap to the 3B version ($0.57$ vs $0.47$ in ALL). Note our method still shows zero correlation for the sentiment task. After, ranks BLEURT ($0.29$), QuestEval ($0.21$), BertScore ($0.11$), Cosine ($0.01$).  

On our test set, we find that some metrics that correlate relatively well on the other datasets, now exhibit low correlation. Hence, with our test set, we can now support the logical reasoning with data evidence: Evaluation of content preservation for style transfer needs to take the style shift into account. This conclusion could not be drawn using the existing data sources: We hypothesise that for the data with system-based output, successful output happens to be very similar to the source sentence and vice versa, and reference-based output might not contain server mistakes as they are gold references. Thus, none of the existing data sources tests the limits of the metrics.  


\paragraph{How do reference-based metrics compare to source-based ones?} Reference-based metrics show a lower correlation than the source-based counterpart for all metrics on both datasets with ratings on references (Table~\ref{tab:sys}). As discussed previously, reference-based metrics for style transfer have the drawback that many different good solutions on a rewrite might exist and not only one similar to a reference.


\paragraph{How well can the metrics rank the performance of style transfer methods?}
We compare the metrics' ability to judge the best style transfer methods w.r.t. the human annotations: Several of the data sources contain samples from different style transfer systems. In order to use metrics to assess the quality of the style transfer system, metrics should correctly find the best-performing system. Hence, we evaluate whether the metrics for content preservation provide the same system ranking as human evaluators. We take the mean of the score for every output on each system and the mean of the human annotations; we compare the systems using the Kendall's Tau correlation. 

We find only the evaluation using the dataset Mir, Lai, and Ziegen to result in significant correlations, probably because of sparsity in a number of system tests (App.~\ref{app:dataset}). Our method (8b) is the only metric providing a perfect ranking of the style transfer system on the Lai data, and Llama3 70B the only one on the Ziegen data. Results in App.~\ref{app:results}. 


\subsection{Style strength results}
%Evaluating style strengths is a challenging task. 
Llama3 70B shows better overall results than our method. However, our method scores higher than Llama3 70B on 2 out of 6 datasets, but it also exhibits zero correlation on one task (Table~\ref{tab:styleresults}).%More work i s needed on evaluating style strengths. 
 
\begin{table}%[]
\fontsize{11pt}{11pt}\selectfont
\begin{tabular}{lccc}
\toprule
\multicolumn{1}{c}{\textbf{}} & \textbf{LlaMA3} & \textbf{Our (3b)} & \textbf{Our (8b)} \\ \midrule
\textbf{Mir} & 0.46 & 0.54 & \textbf{0.57} \\
\textbf{Lai} & \textbf{0.57} & 0.18 & 0.19 \\
\textbf{Ziegen.} & 0.25 & 0.27 & \textbf{0.32} \\
\textbf{Alva-M.} & \textbf{0.59} & 0.03* & 0.02* \\
\textbf{Scialom} & \textbf{0.62} & 0.45 & 0.44 \\
\textbf{\begin{tabular}[c]{@{}l@{}}Our Test\end{tabular}} & \textbf{0.63} & 0.46 & 0.48 \\ \bottomrule
\end{tabular}
\caption{Style strength: Pearson correlation to human ratings. *not significant; we cannot reject the null hypothesis of zero corelation}
\label{tab:styleresults}
\end{table}

\subsection{Ablation}
We conduct several runs of the methods using LLMs with variations in instructions/prompts (App.~\ref{app:method}). We observe that the lower the correlation on a task, the higher the variation between the different runs. For our method, we only observe low variance between the runs.
None of the variations leads to different conclusions of the meta-evaluation. Results in App.~\ref{app:results}.

\section{Discussion}
\section{Discussion of Assumptions}\label{sec:discussion}
In this paper, we have made several assumptions for the sake of clarity and simplicity. In this section, we discuss the rationale behind these assumptions, the extent to which these assumptions hold in practice, and the consequences for our protocol when these assumptions hold.

\subsection{Assumptions on the Demand}

There are two simplifying assumptions we make about the demand. First, we assume the demand at any time is relatively small compared to the channel capacities. Second, we take the demand to be constant over time. We elaborate upon both these points below.

\paragraph{Small demands} The assumption that demands are small relative to channel capacities is made precise in \eqref{eq:large_capacity_assumption}. This assumption simplifies two major aspects of our protocol. First, it largely removes congestion from consideration. In \eqref{eq:primal_problem}, there is no constraint ensuring that total flow in both directions stays below capacity--this is always met. Consequently, there is no Lagrange multiplier for congestion and no congestion pricing; only imbalance penalties apply. In contrast, protocols in \cite{sivaraman2020high, varma2021throughput, wang2024fence} include congestion fees due to explicit congestion constraints. Second, the bound \eqref{eq:large_capacity_assumption} ensures that as long as channels remain balanced, the network can always meet demand, no matter how the demand is routed. Since channels can rebalance when necessary, they never drop transactions. This allows prices and flows to adjust as per the equations in \eqref{eq:algorithm}, which makes it easier to prove the protocol's convergence guarantees. This also preserves the key property that a channel's price remains proportional to net money flow through it.

In practice, payment channel networks are used most often for micro-payments, for which on-chain transactions are prohibitively expensive; large transactions typically take place directly on the blockchain. For example, according to \cite{river2023lightning}, the average channel capacity is roughly $0.1$ BTC ($5,000$ BTC distributed over $50,000$ channels), while the average transaction amount is less than $0.0004$ BTC ($44.7k$ satoshis). Thus, the small demand assumption is not too unrealistic. Additionally, the occasional large transaction can be treated as a sequence of smaller transactions by breaking it into packets and executing each packet serially (as done by \cite{sivaraman2020high}).
Lastly, a good path discovery process that favors large capacity channels over small capacity ones can help ensure that the bound in \eqref{eq:large_capacity_assumption} holds.

\paragraph{Constant demands} 
In this work, we assume that any transacting pair of nodes have a steady transaction demand between them (see Section \ref{sec:transaction_requests}). Making this assumption is necessary to obtain the kind of guarantees that we have presented in this paper. Unless the demand is steady, it is unreasonable to expect that the flows converge to a steady value. Weaker assumptions on the demand lead to weaker guarantees. For example, with the more general setting of stochastic, but i.i.d. demand between any two nodes, \cite{varma2021throughput} shows that the channel queue lengths are bounded in expectation. If the demand can be arbitrary, then it is very hard to get any meaningful performance guarantees; \cite{wang2024fence} shows that even for a single bidirectional channel, the competitive ratio is infinite. Indeed, because a PCN is a decentralized system and decisions must be made based on local information alone, it is difficult for the network to find the optimal detailed balance flow at every time step with a time-varying demand.  With a steady demand, the network can discover the optimal flows in a reasonably short time, as our work shows.

We view the constant demand assumption as an approximation for a more general demand process that could be piece-wise constant, stochastic, or both (see simulations in Figure \ref{fig:five_nodes_variable_demand}).
We believe it should be possible to merge ideas from our work and \cite{varma2021throughput} to provide guarantees in a setting with random demands with arbitrary means. We leave this for future work. In addition, our work suggests that a reasonable method of handling stochastic demands is to queue the transaction requests \textit{at the source node} itself. This queuing action should be viewed in conjunction with flow-control. Indeed, a temporarily high unidirectional demand would raise prices for the sender, incentivizing the sender to stop sending the transactions. If the sender queues the transactions, they can send them later when prices drop. This form of queuing does not require any overhaul of the basic PCN infrastructure and is therefore simpler to implement than per-channel queues as suggested by \cite{sivaraman2020high} and \cite{varma2021throughput}.

\subsection{The Incentive of Channels}
The actions of the channels as prescribed by the DEBT control protocol can be summarized as follows. Channels adjust their prices in proportion to the net flow through them. They rebalance themselves whenever necessary and execute any transaction request that has been made of them. We discuss both these aspects below.

\paragraph{On Prices}
In this work, the exclusive role of channel prices is to ensure that the flows through each channel remains balanced. In practice, it would be important to include other components in a channel's price/fee as well: a congestion price  and an incentive price. The congestion price, as suggested by \cite{varma2021throughput}, would depend on the total flow of transactions through the channel, and would incentivize nodes to balance the load over different paths. The incentive price, which is commonly used in practice \cite{river2023lightning}, is necessary to provide channels with an incentive to serve as an intermediary for different channels. In practice, we expect both these components to be smaller than the imbalance price. Consequently, we expect the behavior of our protocol to be similar to our theoretical results even with these additional prices.

A key aspect of our protocol is that channel fees are allowed to be negative. Although the original Lightning network whitepaper \cite{poon2016bitcoin} suggests that negative channel prices may be a good solution to promote rebalancing, the idea of negative prices in not very popular in the literature. To our knowledge, the only prior work with this feature is \cite{varma2021throughput}. Indeed, in papers such as \cite{van2021merchant} and \cite{wang2024fence}, the price function is explicitly modified such that the channel price is never negative. The results of our paper show the benefits of negative prices. For one, in steady state, equal flows in both directions ensure that a channel doesn't loose any money (the other price components mentioned above ensure that the channel will only gain money). More importantly, negative prices are important to ensure that the protocol selectively stifles acyclic flows while allowing circulations to flow. Indeed, in the example of Section \ref{sec:flow_control_example}, the flows between nodes $A$ and $C$ are left on only because the large positive price over one channel is canceled by the corresponding negative price over the other channel, leading to a net zero price.

Lastly, observe that in the DEBT control protocol, the price charged by a channel does not depend on its capacity. This is a natural consequence of the price being the Lagrange multiplier for the net-zero flow constraint, which also does not depend on the channel capacity. In contrast, in many other works, the imbalance price is normalized by the channel capacity \cite{ren2018optimal, lin2020funds, wang2024fence}; this is shown to work well in practice. The rationale for such a price structure is explained well in \cite{wang2024fence}, where this fee is derived with the aim of always maintaining some balance (liquidity) at each end of every channel. This is a reasonable aim if a channel is to never rebalance itself; the experiments of the aforementioned papers are conducted in such a regime. In this work, however, we allow the channels to rebalance themselves a few times in order to settle on a detailed balance flow. This is because our focus is on the long-term steady state performance of the protocol. This difference in perspective also shows up in how the price depends on the channel imbalance. \cite{lin2020funds} and \cite{wang2024fence} advocate for strictly convex prices whereas this work and \cite{varma2021throughput} propose linear prices.

\paragraph{On Rebalancing} 
Recall that the DEBT control protocol ensures that the flows in the network converge to a detailed balance flow, which can be sustained perpetually without any rebalancing. However, during the transient phase (before convergence), channels may have to perform on-chain rebalancing a few times. Since rebalancing is an expensive operation, it is worthwhile discussing methods by which channels can reduce the extent of rebalancing. One option for the channels to reduce the extent of rebalancing is to increase their capacity; however, this comes at the cost of locking in more capital. Each channel can decide for itself the optimum amount of capital to lock in. Another option, which we discuss in Section \ref{sec:five_node}, is for channels to increase the rate $\gamma$ at which they adjust prices. 

Ultimately, whether or not it is beneficial for a channel to rebalance depends on the time-horizon under consideration. Our protocol is based on the assumption that the demand remains steady for a long period of time. If this is indeed the case, it would be worthwhile for a channel to rebalance itself as it can make up this cost through the incentive fees gained from the flow of transactions through it in steady state. If a channel chooses not to rebalance itself, however, there is a risk of being trapped in a deadlock, which is suboptimal for not only the nodes but also the channel.

\section{Conclusion}
This work presents DEBT control: a protocol for payment channel networks that uses source routing and flow control based on channel prices. The protocol is derived by posing a network utility maximization problem and analyzing its dual minimization. It is shown that under steady demands, the protocol guides the network to an optimal, sustainable point. Simulations show its robustness to demand variations. The work demonstrates that simple protocols with strong theoretical guarantees are possible for PCNs and we hope it inspires further theoretical research in this direction.

\section{Conclusions and Future Work}
\vspace{-0.2cm}
\section{Impact: Why Free Scientific Knowledge?}
\vspace{-0.1cm}

Historically, making knowledge widely available has driven transformative progress. Gutenberg’s printing press broke medieval monopolies on information, increasing literacy and contributing to the Renaissance and Scientific Revolution. In today's world, open source projects such as GNU/Linux and Wikipedia show that freely accessible and modifiable knowledge fosters innovation while ensuring creators are credited through copyleft licenses. These examples highlight a key idea: \textit{access to essential knowledge supports overall advancement.} 

This aligns with the arguments made by Prabhakaran et al. \cite{humanrightsbasedapproachresponsible}, who specifically highlight the \textbf{ human right to participate in scientific advancement} as enshrined in the Universal Declaration of Human Rights. They emphasize that this right underscores the importance of \textit{ equal access to the benefits of scientific progress for all}, a principle directly supported by our proposal for Knowledge Units. The UN Special Rapporteur on Cultural Rights further reinforces this, advocating for the expansion of copyright exceptions to broaden access to scientific knowledge as a crucial component of the right to science and culture \cite{scienceright}. 

However, current intellectual property regimes often create ``patently unfair" barriers to this knowledge, preventing innovation and access, especially in areas critical to human rights, as Hale compellingly argues \cite{patentlyunfair}. Finding a solution requires carefully balancing the imperative of open access with the legitimate rights of authors. As Austin and Ginsburg remind us, authors' rights are also human rights, necessitating robust protection \cite{authorhumanrights}. Shareable knowledge entities like Knowledge Units offer a potential mechanism to achieve this delicate balance in the scientific domain, enabling wider dissemination of research findings while respecting authors' fundamental rights.

\vspace{-0.2cm}
\subsection{Impact Across Sectors}

\textbf{Researchers:} Collaboration across different fields becomes easier when knowledge is shared openly. For instance, combining machine learning with biology or applying quantum principles to cryptography can lead to important breakthroughs. Removing copyright restrictions allows researchers to freely use data and methods, speeding up discoveries while respecting original contributions.

\textbf{Practitioners:} Professionals, especially in healthcare, benefit from immediate access to the latest research. Quick access to newer insights on the effectiveness of drugs, and alternative treatments speeds up adoption and awareness, potentially saving lives. Additionally, open knowledge helps developing countries gain access to health innovations.

\textbf{Education:} Education becomes more accessible when teachers use the latest research to create up-to-date curricula without prohibitive costs. Students can access high-quality research materials and use LM assistance to better understand complex topics, enhancing their learning experience and making high-quality education more accessible.

\textbf{Public Trust:} When information is transparent and accessible, the public can better understand and trust decision-making processes. Open access to government policies and industry practices allows people to review and verify information, helping to reduce misinformation. This transparency encourages critical thinking and builds trust in scientific and governmental institutions.

Overall, making scientific knowledge accessible supports global fairness. By viewing knowledge as a common resource rather than a product to be sold, we can speed up innovation, encourage critical thinking, and empower communities to address important challenges.

\vspace{-0.2cm}
\section{Open Problems}
\vspace{-0.1cm}

Moving forward, we identify key research directions to further exploit the potential of converting original texts into shareable knowledge entities such as demonstrated by the conversion into Knowledge Units in this work:


\textbf{1. Enhancing Factual Accuracy and Reliability:}  Refining KUs through cross-referencing with source texts and incorporating community-driven correction mechanisms, similar to Wikipedia, can minimize hallucinations and ensure the long-term accuracy of knowledge-based datasets at scale.

\textbf{2. Developing Applications for Education and Research:}  Using KU-based conversion for datasets to be employed in practical tools, such as search interfaces and learning platforms, can ensure rapid dissemination of any new knowledge into shareable downstream resources, significantly improving the accessibility, spread, and impact of KUs.

\textbf{3. Establishing Standards for Knowledge Interoperability and Reuse:}  Future research should focus on defining standardized formats for entities like KU and knowledge graph layouts \citep{lenat1990cyc}. These standards are essential to unlock seamless interoperability, facilitate reuse across diverse platforms, and foster a vibrant ecosystem of open scientific knowledge. 

\textbf{4. Interconnecting Shareable Knowledge for Scientific Workflow Assistance and Automation:} There might be further potential in constructing a semantic web that interconnects publicly shared knowledge, together with mechanisms that continually update and validate all shareable knowledge units. This can be starting point for a platform that uses all collected knowledge to assist scientific workflows, for instance by feeding such a semantic web into recently developed reasoning models equipped with retrieval augmented generation. Such assistance could assemble knowledge across multiple scientific papers, guiding scientists more efficiently through vast research landscapes. Given further progress in model capabilities, validation, self-repair and evolving new knowledge from already existing vast collection in the semantic web can lead to automation of scientific discovery, assuming that knowledge data in the semantic web can be freely shared.

We open-source our code and encourage collaboration to improve extraction pipelines, enhance Knowledge Unit capabilities, and expand coverage to additional fields.

\vspace{-0.2cm}
\section{Conclusion}
\vspace{-0.1cm}

In this paper, we highlight the potential of systematically separating factual scientific knowledge from protected artistic or stylistic expression. By representing scientific insights as structured facts and relationships, prototypes like Knowledge Units (KUs) offer a pathway to broaden access to scientific knowledge without infringing copyright, aligning with legal principles like German \S 24(1) UrhG and U.S. fair use standards. Extensive testing across a range of domains and models shows evidence that Knowledge Units (KUs) can feasibly retain core information. These findings offer a promising way forward for openly disseminating scientific information while respecting copyright constraints.

\section*{Author Contributions}

Christoph conceived the project and led organization. Christoph and Gollam led all the experiments. Nick and Huu led the legal aspects. Tawsif led the data collection. Ameya and Andreas led the manuscript writing. Ludwig, Sören, Robert, Jenia and Matthias provided feedback. advice and scientific supervision throughout the project. 

\section*{Acknowledgements}

The authors would like to thank (in alphabetical order): Sebastian Dziadzio, Kristof Meding, Tea Mustać, Shantanu Prabhat for insightful feedback and suggestions. Special thanks to Andrej Radonjic for help in scaling up data collection. GR and SA acknowledge financial support by the German Research Foundation (DFG) for the NFDI4DataScience Initiative (project number 460234259). AP and MB acknowledge financial support by the Federal Ministry of Education and Research (BMBF), FKZ: 011524085B and Open Philanthropy Foundation funded by the Good Ventures Foundation. AH acknowledges financial support by the Federal Ministry of Education and Research (BMBF), FKZ: 01IS24079A and the Carl Zeiss Foundation through the project "Certification and Foundations of Safe ML Systems" as well as the support from the International Max Planck Research School for Intelligent Systems (IMPRS-IS). JJ acknowledges funding by the Federal Ministry of Education and Research of Germany (BMBF) under grant no. 01IS22094B (WestAI - AI Service Center West), under grant no. 01IS24085C (OPENHAFM) and under the grant DE002571 (MINERVA), as well as co-funding by EU from EuroHPC Joint Undertaking programm under grant no. 101182737 (MINERVA) and from Digital Europe Programme under grant no. 101195233 (openEuroLLM) 

%%
%% The acknowledgments section is defined using the "acks" environment
%% (and NOT an unnumbered section). This ensures the proper
%% identification of the section in the article metadata, and the
%% consistent spelling of the heading.
\begin{acks}
We thank members of the Social Physics and Complexity (SPAC) group at LIP for comments and critical reading of the manuscript. We thank HoneyComb for support in identifying meaningful topics for this study and BrightData for generously providing proxy services, free of charge. This research was partially funded by ERC Stg FARE (853566) and ERC PoC FARE\_Audit (101100653), both to JGS, and by FCT PhD fellowship (2022.12547.BD) to ID.
\end{acks}
\balance
%%
%% The next two lines define the bibliography style to be used, and
%% the bibliography file.
\bibliographystyle{ACM-Reference-Format}
\bibliography{refs}

\clearpage
%%
%% If your work has an appendix, this is the place to put it.

\appendix
\supplementarysection
\section{Supplementary Description of the Data collected}

\subsection{Data Collection Timeline}
In Figure \ref{timeline}, we show the moments when data collection began for each Type of webcrawlers used in the experiment.

\begin{figure}[ht]
    \centering \includegraphics[width=1\columnwidth]{Figures/timeline.pdf}
    \caption{Temporal alignment of deployment of Type 1, Type 2 and Type 3 bots, and the temporal control.}
    \Description{}
    \label{timeline}
\end{figure}

\subsection{Queries used in the experiment}

Table \ref{tab:queries_long_table} shows all queries used in this experiment. Notice that Type 3 bots performed a smaller number (10) of the original number of queries (gray background). 

\begin{table}[h]
    \centering
    \caption{Queries used in the experimental setup. All specific and general queries shown here were used in deployment of Type 1 bots (Location only) and Type 2 (Location + Browser Language). Type 3 (Location + browser language + browsing history) only did queries in bold (10 per category).}
    \label{tab:queries_long_table}
    \begin{tabular}{|| p{3.7cm} | p{3.7cm} ||}
    \hline
    \textbf{Specific Queries} & \textbf{General Queries} \\
    \hline\hline
    \small military complex Al-Shifa hospital & \small How to tie a tie \\ 
    \hline
    \small Israel banned Olympics & \small Popular books 2024 \\
    \hline
    \small Tiktok antisemitism & \small \textbf{Home workout routines} \\
    \hline
    \small Israeli babies beheaded & \small best movies ever \\
    \hline
    \small \textbf{Hamas rapes} & \small How to grow indoor plants \\
    \hline
    \small strike Al-Ahli hospital & \small Uncommon hobbies to try \\
    \hline
    \small \textbf{Gaza tunnels} & \small Rare and endangered plant species \\
    \hline
    \small Palestine banned Olympics & \small \textbf{Experimental music genres}\\
    \hline
    \small Saint Porphyrios Orthodox Church Gaza & \small organic gardening tips \\
    \hline
    \small \textbf{Hamas} & \small how to be smarter \\
    \hline
    \small Supernova festival attack &  \small\textbf{ Financial planning tips} \\
    \hline
    \small Israel destroyed an orthodox church in Gaza & 
    \small \textbf{Benefits of meditation for mental health} \\
    \hline
    \small \textbf{Erdogan threatened to intervene and support Palestinians?} & \small Nutritional value of quinoa \\
    \hline
    \small Ukraine provided weapons to Hamas & \small How to improve sleep quality naturally? \\
    \hline
    \small Yemen has declared war against Israel & \small Interesting facts about dolphins \\
    \hline
    \small \textbf{Gaza} & \small \textbf{Guide to composting at home} \\
    \hline
    \small Palestinian nurse claims Hamas steals food and medicine from al-Shifa Hospital & \small Popular science fiction books 2024 \\
    \hline
    \small \textbf{Israel} & \small How to set up a home office? \\
    \hline
    \small \textbf{American troops have landed in Israel to help Netanyahu's war efforts} & \small \textbf{Effects of climate change on wildlife} \\
    \hline
    \small Orthodox church Gaza destroyed by Israel & \small Tips for growing herbs indoors \\
    \hline
    \small \textbf{Houthis} & \small underrated movies you must see \\
    \hline
    \small \textbf{Erdogan threatens to support Palestinians} & \small \textbf{Current trends in sustainable fashion} \\
    \hline
    \small Ukraine provides weapons to Hamas &  \small \textbf{How do electric cars work?}\\
    \hline
    \small Yemen declares war on Israel & \small \textbf{Healthy lunch ideas for work} \\
    \hline
    \small Palestinian nurse alleges Hamas theft from al-Shifa Hospital & \small How to invest in stocks for beginners? \\
    \hline
    \small \textbf{American troops land in Israel to aid Netanyahu's war} & \small \textbf{DIY home decor on a budget} \\
    \hline
    \small Middle East conflict & \small What is virtual reality and how does it work? \\
    \hline
    \end{tabular}
\end{table}


\subsection{Data Description}
In the deployment of each bot's type the first page of results was collected for each combination of search engine, bot, and query. However, not every audit (i.e., the collection of results for a query) was successful. Failures were often due to IP issues or problems with the xpaths of certain search engine results. In the analysis, only successful audits were considered. A successful audit was defined as the collection of at least four URLs from at least three different IPs within the same location for the same search engine and query. Since the failure of some audits introduced imbalances in the search engine results database, we ensured that each analysis used the same number of specific and general queries per search engine, as well as the same number of ports per location.  

Table \ref{tab:number_queries_successful}  presents the number of successful queries in each category for every search engine and location pair considered in the analysis. It is important to note that while the number of successful queries per search engine may vary for the same experimental step, we ensured that the number of specific and general queries was consistent within each search engine analysis.

\begin{table}[hb]
    \centering
    \caption{Number of successful queries within each query category per search engine.}
    \label{tab:number_queries_successful}
    \begin{tabular}{|p{2cm}|p{2cm}|p{2.5cm}|}
    \hline
    \textbf{} & \textbf{Search Engine} & \textbf{Number Queries}\\
    \hline
    \multirow{4}{*}{Type 1} & DuckDuckGo & 26 \\
    \cline{2-3}
    {} & Google & 27 \\
    \cline{2-3}
    {} & Yahoo & 27 \\
    \hline
    \multirow{4}{*}{Type 2} & DuckDuckGo & 27 \\
    \cline{2-3}
    {} & Google & 19 \\
    \cline{2-3}
    {} & Yahoo & 27 \\
    \hline
    \multirow{4}{*}{Type 3} & DuckDuckGo & 10 \\
    \cline{2-3}
    {} & Google & 10 \\
    \cline{2-3}
    {} & Yahoo & 10 \\
    \hline
    \end{tabular}
\end{table}

Figure \ref{number_successful_ports} shows the number of successful IPs for each location and search engine. 10 ports of Type 1 and Type 2 were deployed initially. However, of Type 3, only 8 different IPs were used per location. Of these 8 IPs, 3 were associated with visits to news articles about the studied conflict (labeled as "C" in Figure \ref{number_successful_ports}, Type 3), 3 had a profile of regular news browsing history (labeled as "G"), and 2 were stateless bots ("S").

Table \ref{tab:number_results} contains a small description of the total number of URLs collected in each step of the experiment. 

\newpage
Some results were present to all bots Types. In the Venn diagram of Figure \ref{venn_unique_urls} we show the intersection of raw URLs across the types of bots.  In the Veen diagram of Figure \ref{venn_unique_domains} the intersection of websites domains across bot types. 


\begin{figure}[hb]
\includegraphics[width=1\columnwidth]{Figures/n_ports.pdf}
    \caption{Number of successful ports per location across search engines and bot type. The gray bars represent the number of IPs successful in performing general queries, while the colored bars indicate those successful in specific queries. For Type 3 bots, each bar is further divided into the number of IPs trained with visits to conflict news websites (C), general news websites (G), and stateless bots (S).}
    \Description{}
    \label{number_successful_ports}
\end{figure}



\begin{table}[t]
    \centering
    \caption{Description of results (URLs) data base considered for analysis. N results corresponds to the number of total results, N unique results to the number of different URLs collected for all queries and N unique domains to the total number of different websites domains.}
    \label{tab:number_results}
    \begin{tabular}{|p{0.8cm}|p{1cm}|p{1.5cm}|p{1.5cm}|p{1.5cm}|}
    \hline
    \textbf{} & \textbf{Search \newline Engine} & \textbf{Query \newline category} & \textbf{N unique results} & \textbf {N unique domains}\\
    \hline
    \multirow{4}{*}{Type 1} & \multirow{2}{*}{DuckGo} & General & 547 & 335\\
    \cline{3-5}
    {} & {} & Specific & 597 & 104 \\
    \cline{2-5}
    {} & \multirow{2}{*}{Google} & General & 328 & 224\\
    \cline{3-5}
    {} & {} & Specific & 268 & 98\\
    \cline{2-5}
    {} & \multirow{2}{*}{Yahoo} & General & 244 & 168 \\
    \cline{3-5}
    {} & {} & Specific & 245 & 57\\
    \hline
    \multirow{4}{*}{Type 2} & \multirow{2}{*}{DuckGo} & General & 621 & 380\\
    \cline{3-5}
    {} & {} & Specific & 746 & 125 \\
    \cline{2-5}
    {} & \multirow{2}{*}{Google} & General & 304 & 216\\
    \cline{3-5}
    {} & {} & Specific & 338 & 119\\
    \cline{2-5}
    {} & \multirow{2}{*}{Yahoo} & General & 207 & 154 \\
    \cline{3-5}
    {} & {} & Specific & 208 & 51\\
    \hline
    \multirow{4}{*}{Type 3} & \multirow{2}{*}{DuckGo} & General & 226 & 158\\
    \cline{3-5}
    {} & {} & Specific & 379 & 66 \\
    \cline{2-5}
    {} & \multirow{2}{*}{Google} & General & 216 & 164\\
    \cline{3-5}
    {} & {} & Specific & 339 & 111\\
    \cline{2-5}
    {} & \multirow{2}{*}{Yahoo} & General & 108 & 83 \\
    \cline{3-5}
    {} & {} & Specific & 152 & 41\\
    \hline
    \end{tabular}
\end{table}

\begin{figure}[H]
    \centering
    \includegraphics[width=0.8\columnwidth]{Figures/venn_unique_urls.png}
    \caption{Venn diagram of unique URLs present in Type 1 - Location only (green), Type 2 - Location + browser languages (yellow) and Type 3 - Location + browser languages + browsing history (blue) and its intersections.}
    \Description{}
    \label{venn_unique_urls}
\end{figure}

\begin{figure}[H]
    \centering
    \includegraphics[width=0.8\columnwidth]{Figures/venn_unique_domains.png}
    \caption{Venn diagram of unique domains present in Type 1 - Location only (green),  Type 2 - Location + browser languages (yellow) and Type 3 - Location + browser languages + browsing history (blue) and its intersections.}
    \Description{}
    \label{venn_unique_domains}
\end{figure}

\newpage
\section{Additional Results and Analysis}

\subsection{Statistical Results}
The statistical analysis values present in \ref{fig:main_plot} are described in the the following tables \ref{tab:p_values_adjusted_step_1}, \ref{tab:p_values_adjusted_step_2}, \ref{tab:p_values_adjusted_categories_step_3}, \ref{tab:p_values_comparison_different_steps} in more detail. 

\begin{table}[ht]
    \centering
    \caption{Pairwise comparisons of RBO per query type and location for Type 1 bots across search engines, showing the respective p-values and Bonferroni-adjusted p-values. The Mann-Whitney U test was used for the comparisons, and significant adjusted p-values ($<0.05$) are highlighted in bold. The adjustment accounts for 12 comparisons across the three search engines.}
    \Description{}
    \label{tab:p_values_adjusted_step_1}
    \begin{tabular}{|p{1cm}|p{3cm}|p{1cm}|p{1.5cm}|}
    \hline
    \textbf{Search\newline Engine} & \textbf{Groups \newline compared} & \textbf{p-value} & \textbf{Adjusted \newline p-value} \\
    \hline\hline
    \multirow{4}{*}{DuckGo} & Same Location vs. Diff Location - General Queries & \textbf{$<<0.001$} & \textbf{$<<0.001$} \\
    \cline{2-4}
    & Same location vs. Diff location - Specific Queries & \textbf{0.002} & \textbf{0.02} \\
    \cline{2-4}
    & General vs. Specific - Same Location & \textbf{0.05} & 0.56\\
    \cline{2-4}
    & General vs. Specific - Diff Location & 0.88 & 10.52\\
    \hline \hline
    \multirow{4}{*}{Google} & Same Location vs. Diff Location - General Queries & 0.33 & 3.98 \\
    \cline{2-4}
    & Same location vs. Diff location - Specific Queries & 0.14 & 1.69 \\
    \cline{2-4}
    & General vs. Specific - Same Location & 0.21 & 2.48 \\
    \cline{2-4}
    & General vs. Specific - Diff Location & 0.25 & 2.96\\
    \hline \hline
    \multirow{4}{*}{Yahoo} & Same Location vs. Diff Location - General Queries & 0.07 & 0.86 \\
    \cline{2-4}
    & Same location vs. Diff location - Specific Queries & 0.02 & 0.29 \\
    \cline{2-4}
    & General vs. Specific - Same Location & 0.20 & 2.40\\
    \cline{2-4}
    & General vs. Specific - Diff Location & 0.11 & 1.41\\
    \hline
    \end{tabular}
\end{table}
\vspace{1cm}
\begin{table}[hb]
    \centering
    \caption{Pairwise comparisons of RBO per query type and location for Type 2 bots across search engines, showing the respective p-values and Bonferroni-adjusted p-values. The Mann-Whitney U test was used for the comparisons, and significant adjusted p-values ($<0.05$) are highlighted in bold. The adjustment accounts for 12 comparisons across the three search engines.}
    \Description{}
    \label{tab:p_values_adjusted_step_2}
    \begin{tabular}{|p{1cm}|p{3cm}|p{1cm}|p{1.5cm}|}
    \hline
    \textbf{Search\newline Engine} & \textbf{Groups \newline compared} & \textbf{p-value} & \textbf{Adjusted \newline p-value} \\
    \hline\hline
    \multirow{4}{*}{DuckGo} & Same Location vs. Diff Location - General Queries & \textbf{$<<0.001$} & \textbf{$<<0.001$} \\
    \cline{2-4}
    & Same location vs. Diff location - Specific Queries & \textbf{<<0.001} & \textbf{<<0.001}\\
    \cline{2-4}
    & General vs. Specific - Same Location & \textbf{<<0.001} & \textbf{<<0.001}\\
    \cline{2-4}
    & General vs. Specific - Diff Location & \textbf{0.04} & 0.47\\
    \hline \hline
    \multirow{4}{*}{Google} & Same Location vs. Diff Location - General Queries & \textbf{$<<0.001$} & \textbf{$<<0.001$} \\
    \cline{2-4}
    & Same location vs. Diff location - Specific Queries & \textbf{<<0.001} & \textbf{<<0.001} \\
    \cline{2-4}
    & General vs. Specific - Same Location & 0.10 & 1.30 \\
    \cline{2-4}
    & General vs. Specific - Diff Location & 0.10 & 1.15\\
    \hline \hline
    \multirow{4}{*}{Yahoo} & Same Location vs. Diff Location - General Queries & 0.27 & 3.30 \\
    \cline{2-4}
    & Same location vs. Diff location - Specific Queries & \textbf{0.02} & 0.25 \\
    \cline{2-4}
    & General vs. Specific - Same Location & 0.60 & 7.17\\
    \cline{2-4}
    & General vs. Specific - Diff Location & 0.12 & 1.48\\
    \hline
    \end{tabular}
\end{table}
\clearpage
\begin{table*}[t]
    \centering
    \caption{Pairwise comparisons of RBO per query type and location for Type 3 bots across search engines, showing the respective p-values and Bonferroni-adjusted p-values. The Mann-Whitney U test was used for the comparisons, and significant adjusted p-values ($<0.05$) are highlighted in bold. The adjustment accounts for 12 comparisons across the three search engines.}
    \Description{}
    \label{tab:p_values_adjusted_step_3}
    \begin{tabular}{|p{2cm}|p{7cm}|p{1.5cm}|p{2.5cm}|}
    \hline
    \textbf{Search Engine} & \textbf{Groups compared} & \textbf{p-value} & \textbf{Adjusted p-value} \\
    \hline\hline
    \multirow{4}{*}{DuckGo} & Same Location vs. Diff Location - General Queries & \textbf{0.03} & 0.37 \\
    \cline{2-4}
    & Same location vs. Diff location - Specific Queries & \textbf{0.001} & \textbf{0.01}\\
    \cline{2-4}
    & General vs. Specific - Same Location & 0.34 & 4.14\\
    \cline{2-4}
    & General vs. Specific - Diff Location & \textbf{0.01} & 0.14 \\
    \hline \hline
    \multirow{4}{*}{Google} & Same Location vs. Diff Location - General Queries & \textbf{$<<0.001$} & \textbf{0.009} \\
    \cline{2-4}
    & Same location vs. Diff location - Specific Queries & \textbf{$<<0.001$} & \textbf{<<0.001} \\
    \cline{2-4}
    & General vs. Specific - Same Location & \textbf{0.02} & 0.33 \\
    \cline{2-4}
    & General vs. Specific - Diff Location & \textbf{0.02} & 0.25\\
    \hline \hline
    \multirow{4}{*}{Yahoo} & Same Location vs. Diff Location - General Queries & 0.57 & 6.85 \\
    \cline{2-4}
    & Same location vs. Diff location - Specific Queries & 0.52 & 6.24 \\
    \cline{2-4}
    & General vs. Specific - Same Location & 0.14 & 1.69\\
    \cline{2-4}
    & General vs. Specific - Diff Location & 0.14 & 1.69\\
    \hline
    \end{tabular}
\end{table*}
\clearpage
\begin{table*}[b]
    \centering
    \caption{Pairwise Comparison of results from different bots, across search engines and query classifications with p-values and respective Adjusted p-values. Adjusted p-values are calculated using the Bonferroni method and when significant ($<0.05$) are highlighted in bold. The adjustment accounts for 36 comparisons counting with number of tests (6), per search engine and query groups.}
    \label{tab:p_values_comparison_different_steps}
    \begin{tabular}{|p{3cm}|p{2cm}|p{5cm}|p{1.5cm}|p{3cm}|}
    \hline
    \textbf{Search Engine} & \textbf{Query Type} & \textbf{Groups compared} & \textbf{p-value} & \textbf{Adjusted p-value} \\
    \hline
    \multirow{6}{*}{DuckDuckGo} & \multirow{6}{*}{General} & Type 1 vs. Type 2 - Same Location & 0.97 & 34.91 \\
    & & Type 1 vs. Type 3 - Same Location & 0.27 & 9.83 \\
    & & Type 2 vs. Type 3 - Same Location & 0.38 & 13.85 \\
    & & Type 1 vs. Type 2 - Diff Location & 0.57 & 20.55 \\
    & & Type 1 vs. Type 3 - Diff Location & 0.73 & 26.41 \\
    & & Type 2 vs. Type 3 - Diff Location & 0.91 & 32.75 \\
    \hline
    \multirow{6}{*}{Google} & \multirow{6}{*}{General} & Type 1 vs. Type 2 - Same Location & 0.34 & 12.41 \\
    & & Type 1 vs. Type 3 - Same Location & 0.79 & 28.49 \\
    & & Type 2 vs. Type 3 - Same Location & 0.14 & 5.06 \\
    & & Type 1 vs. Type 2 - Diff Location & \textbf{0.02} & 0.76 \\
    & & Type 1 vs. Type 3 - Diff Location & \textbf{0.0036} & 0.13 \\
    & & Type 2 vs. Type 3 - Diff Location & 0.85 & 30.60 \\
    \hline
    \multirow{6}{*}{Yahoo} & \multirow{6}{*}{General} & Type 1 vs. Type 2 - Same Location & 0.52 & 18.74 \\
    & & Type 1 vs. Type 3 - Same Location & \textbf{0.0013} & \textbf{0.05} \\
    & & Type 2 vs. Type 3 - Same Location & \textbf{0.0036} & 0.13  \\
    & & Type 1 vs. Type 2 - Diff Location & 0.68 & 24.39 \\
    & & Type 1 vs. Type 3 - Diff Location & \textbf{0.02} & 0.62 \\
    & & Type 2 vs. Type 3 - Diff Location & \textbf{0.03} & 1.12 \\
    \hline \hline
    \multirow{6}{*}{DuckDuckGo} & \multirow{6}{*}{Specific} & Type 1 vs. Type 2 - Same Location & 0.34 & 12.41 \\
    & & Type 1 vs. Type 3 - Same Location & 0.43 & 15.38 \\
    & & Type 2 vs. Type 3 - Same Location & 0.85 & 30.60 \\
    & & Type 1 vs. Type 2 - Diff Location & \textbf{0.0073} & 0.26 \\
    & & Type 1 vs. Type 3 - Diff Location & \textbf{0.0091} & 0.33 \\
    & & Type 2 vs. Type 3 - Diff Location & 0.21 & 7.64 \\
    \hline
    \multirow{6}{*}{Google} & \multirow{6}{*}{Specific} & Type 1 vs. Type 2 - Same Location & 0.38 & 13.85 \\
    & & Type 1 vs. Type 3 - Same Location & \textbf{0.01} & 0.41 \\
    & & Type 2 vs. Type 3 - Same Location & 0.14 & 5.06 \\
    & & Type 1 vs. Type 2 - Diff Location & \textbf{0.0002} & \textbf{0.01} \\
    & & Type 1 vs. Type 3 - Diff Location & \textbf{0.0002} & \textbf{0.01} \\
    & & Type 2 vs. Type 3 - Diff Location & 0.91 & 32.75 \\
    \hline
    \multirow{6}{*}{Yahoo} & \multirow{6}{*}{Specific} & Type 1 vs. Type 2 - Same Location & 0.82 & 29.54 \\
    & & Type 1 vs. Type 3 - Same Location & \textbf{0.0036} & 0.13 \\
    & & Type 2 vs. Type 3 - Same Location & \textbf{0.0028} & 0.10 \\
    & & Type 1 vs. Type 2 - Diff Location & 0.79 & 28.49 \\
    & & Type 1 vs. Type 3 - Diff Location & \textbf{0.0091} & 0.33 \\
    & & Type 2 vs. Type 3 - Diff Location & \textbf{0.0017} & 0.06 \\
    \hline
    \end{tabular}
\end{table*}

\clearpage
\begin{table}[hb]
\caption{ANOVA statistical results of RBO values for different browsing histories comparisons with the stateless profiles, across search engines and query groups.}
\centering
\begin{tabular}{p{1cm} p{1cm} p{2cm} p{1.3cm} p{1.2cm}}
\hline
Search\newline Engine &  Query\newline Group & Brow.profiles\newline comparisons & Mean \newline D=1-RBO & ANOVA \newline Result \\
\hline
\multirow{6}{*}{Duckgo} 
& \multirow{3}{*}{\small{general}} & \small{conflict v. stat.} & 0.22 & F=1.792\\
& & general v. stat. & 0.288047 & p=0.170 \\
& & stateless v. stat. & 0.221987 &\\
\cline{2-5} 
& \multirow{3}{*}{\small{specific}} & \small{conflict v. stat.} & 0.30 & F=2.33 \\
& & \small{general v. stat.} & 0.37 & p=0.10\\
& & \small{stateless v. stat.} & 0.27&\\
\hline
\multirow{6}{*}{Google} 
& \multirow{3}{*}{\small{general}} & \small{conflict v. stat.} & 0.15  & F=3.25 \\
& & \small{general v. stat.} & 0.13& p=0.04\\
& & \small{stateless v. stat.} & 0.09 &\\
\cline{2-5}
& \multirow{3}{*}{\small{specific}} & conflict v. stat. & 0.244388 & F=8.078 \\
& & \small{general v. stat.} & 0.22 & p =0.0\\
& & \small{stateless v. stat.} & 0.12 &\\
\hline
\multirow{6}{*}{Yahoo} 
& \multirow{3}{*}{\small{general}} & \small{conflict v. stat.} & 0.29& F =0.04 \\
& & \small{general v. stat.} & 0.29 & p =0.96\\
& & \small{stateless v. stat.} & 0.28 &\\
\cline{2-5}
& \multirow{3}{*}{\small{specific}} & \small{conflict v. stat.} & 0.32 & F =0.09 \\
& & \small{general v. stat.} & 0.31 & p =0.91\\
& & \small{stateless v. stat.} & 0.33 &\\
\hline
\end{tabular}

\label{table:stat_results}
\end{table}
\clearpage

\subsection{Additional metrics of results differences}

The $D = 1 - RBO$ measurement used in the paper calculates a value between $[0, 1]$ according to the similarity of two lists and the rank at which two elements of the list match. To complement this metric we also calculated: (1) the average number of URLs that were not present in both lists of results of the two bots being compared (top row in Figures \ref{fig:other_metrics_step_1},\ref{fig:other_metrics_step_2}, \ref{fig:other_metrics_step_3}); (2) average number of URLs that were present at both top 3 results for both lists of results being compared (middle row in Figures \ref{fig:other_metrics_step_1},\ref{fig:other_metrics_step_2}, \ref{fig:other_metrics_step_3}) and (3) the average edit distance (bottom row in Figures \ref{fig:other_metrics_step_1},\ref{fig:other_metrics_step_2}, \ref{fig:other_metrics_step_3}). 

\begin{figure}[ht]
    \centering
    \includegraphics[width=1\columnwidth]{Figures/step_1_n_diff_results.png}
    \caption{Results for Type 1 bots. (Top Row) Average number of URLs that were not present in both lists of results in the 10 results. (Middle row) Average number of URLs that were not present in the top 3 results for both lists of results being compared. (Bottom Row) Edit distance between lists.  Error bars represent 95\% confidence intervals based on the bootstrapped distribution.}
    \Description{}
    \label{fig:other_metrics_step_1}
\end{figure}

\begin{figure}[hb]
    \centering
    \includegraphics[width=1\columnwidth]{Figures/step_2_n_diff_results.png}
    \caption{Results for Type 2 bots. (Top Row) Average number of URLs that were not present in both lists of results in the 10 results. (Middle row) Average number of URLs that were not present in the top 3 results for both lists of results being compared. (Bottom Row) Edit distance between lists.  Error bars represent 95\% confidence intervals based on the bootstrapped distribution.}
    \Description{}
    \label{fig:other_metrics_step_2}
\end{figure}

\newpage
\begin{figure}[h]
    \centering
    \includegraphics[width=1\columnwidth]{Figures/step_3_n_diff_results.png}
    \caption{Results for Type 3 bots. (Top Row) Average number of URLs that were not present in both lists of results in the 10 results. (Middle row) Average number of URLs that were not present in the top 3 results for both lists of results being compared. (Bottom Row) Edit distance between lists.  Error bars represent 95\% confidence intervals based on the bootstrapped distribution.}
    \Description{}
    \label{fig:other_metrics_step_3}
\end{figure}


\subsection{Controlling for query size}

As shown in Table \ref{tab:queries_long_table} the  categories of queries have very different sizes. Therefore, we tested whether the higher values of $D = 1 - RBO$ for queries of the specific group was a consequence of a higher number of words (bigger dimension). Figure \ref{results_by_query_size} shows the value of $D = 1 - RBO$ as function of the number of words in the query for both queries of the general category (black circle) and specific category (gray triangle) across search engines. As the plot shows there is a tendency for the value of $D = 1 - RBO$ to be higher for specific queries than general ones even when comparing queries of the exact same size from both categories. 

\begin{figure}[h]
    \centering
    \includegraphics[width=1\columnwidth]{Figures/plot_rbo_by_n_words_search_engine_step_1.png}
    \caption{Average value of $D = 1 - RBO$ (Type 1 bots) by query size in terms of number of words. In black plotted with circles we have the results for queries of the general category and in grey plotted with triangles the results of queries in the specific category.}
    \Description{}
    \label{results_by_query_size}
\end{figure}

Additionally, when we compare the value of $D = 1- RBO$ per location (different and same location) for an equal number of queries of both categories with comparable sizes (within 3 and 9 words) we observe (Figure \ref{rbo_queries_same_size} that the results continue to present the same pattern as in Figure \ref{fig:main_plot}. That is, higher values of $D = 1- RBO$ for specific queries than general queries and also higher values when comparing results of bots from different locations than the same location. 

\begin{figure}[h]
    \centering
    \includegraphics[width=1\columnwidth]{Figures/plot_rbo_filtered_words_step_1.png}
    \caption{Average value of $D = 1 - RBO$ for bots in the same location (grey) and in different locations (green) considering exclusively queries of the same length (between 3 and 8 words).}
    \Description{}
    \label{rbo_queries_same_size}
\end{figure}



\subsection{D = 1 - RBO values for the different experimental steps (for the 10 common queries across bot types)}


\begin{figure}[h]
    \centering
    \includegraphics[width=1\columnwidth]{Figures/comparing_rbo_1_2_3_controled_queries_3_legend.png}
    \caption{$D = 1 - RBO$ Results: Comparison of the experimental outcomes across search engines for the same location (left) and different locations (right). The top row illustrates the results for specific queries, the middle row shows results for general queries, and the bottom row highlights the difference between specific and general queries.}
    \label{fig:comparison_steps}
    \Description{}
\end{figure}


\subsection{Differences across categories of websites}

To control for the possibility that bots were retrieving the same content but in localized versions, leading to differences in URLs, we used ChatGPT-4o to classify the websites by category and type. The prompt was the one depicted in Figure \ref{prompt_domain_cat}.

\begin{figure}[hb]
    \includegraphics[width=1\columnwidth]{Figures/prompt_domain_categories.png}
    \caption{Prompt given to ChatGPT for the domains classification.}
    \Description{}
    \label{prompt_domain_cat}
\end{figure}


All categories per query type are in Table \ref{tab:categories_category}. Figure \ref{categories_websites_general}, shows the most common categories across search engines and location, for different Type of bots for general queries. Figure \ref{categories_websites_specific} refers to specific queries. 

\begin{table}[h]
    \centering
    \caption{Classifications of ChatGPT for category of results websites associated with General and Specific queries.}
    \label{tab:categories_category}
    \begin{tabular}{|p{3cm}|p{4cm}|}
    \hline
    \textbf{Query category} & \textbf{Websites classifications}\\
    \hline
    General & Reference, Entertainment, Education, Technology, News, Lifestyle, Business, Finance, Health, Government, Non-Profit, Social Media, Travel, E-Commerce, Art, Science, Fashion, Legal, Career, Retail, Automotive, Food \\
    \hline
    Specific & Reference, Education, Government, News, Fact-Checking, Social Media, Non-Profit, Entertainment, Finance, Religion, E-Commerce, Technology, Sports, Travel, Science \\
    \hline
    \end{tabular}
\end{table}

\begin{figure}[ht]
    \centering
    \includegraphics[width=1\columnwidth]{Figures/categories_websites_specific.pdf}
    \caption{Proportion of results per top 5 category of website for the specific group of queries, per search engine, country and bot type.}
    \label{categories_websites_specific}
    \Description{}
\end{figure}


\begin{figure}[hb]
    \centering
    \includegraphics[width=1\columnwidth]{Figures/categories_websites_general.pdf}
    \caption{Proportion of results per top 5 category of website for the general group of queries, per search engine, country and bot type.}
    \label{categories_websites_general}
    \Description{}
\end{figure}


Figure \ref{categories_websites_specific} and \ref{categories_websites_general}reveal noticeable differences in the distribution of categories between the different query groups. 

% \begin{table}[h!]
%     \centering
%     \caption{Average number of websites categories per experimental step and search engine for both categories of queries (General and Specific).}
%     \label{tab:number_categories}
%     \begin{tabular}{|p{0.8cm}|p{2cm}|p{1.5cm}|p{1.5cm}|}
%     \hline
%     \textbf{} & \textbf{Search Engine} & \textbf{General Queries} & \textbf{Specific Queries}\\
%     \hline
%     \multirow{4}{*}{Step 1} & Bing & 2.90 & 2.19 \\
%     \cline{2-4}
%     {} & DuckDuckGo & 3.09 & 2.19 \\
%     \cline{2-4}
%     {} & Google & 4.01 & 3.39 \\
%     \cline{2-4}
%     {} & Yahoo & 2.54 & 1.96  \\
%     \hline
%     \multirow{4}{*}{Step 2} & Bing & 2.90 & 2.49 \\
%     \cline{2-4}
%     {} & DuckDuckGo & 2.92 & 2.15 \\
%     \cline{2-4}
%     {} & Google & 3.97 & 3.42\\
%     \cline{2-4}
%     {} & Yahoo & 2.59 & 1.86 \\
%     \hline
%     \multirow{4}{*}{Step 3} & Bing & 2.67 &  2.29 \\
%     \cline{2-4}
%     {} & DuckDuckGo & 3.15 & 1.77 \\
%     \cline{2-4}
%     {} & Google & 4.16 & 3.20\\
%     \cline{2-4}
%     {} & Yahoo & 2.75 & 1.68\\
%     \hline
%     \end{tabular}
% \end{table}

After classifying all domains present in all experimental steps (Table \ref{tab:categories_category}) we studied how the results varied as function of the location and query type for all bots types. Additionally to the metric $D = 1 - RBO$ we also calculated (1) the average number of categories that were not present in both lists of results of the two bots being compared (top row in Figures \ref{fig:other_categories_step_1},\ref{fig:other_categories_step_2}, \ref{fig:other_categories_step_3}); (2) average number of URLs that were present at both top 3 results for both lists of results being compared (middle row in Figures \ref{fig:other_categories_step_1},\ref{fig:other_categories_step_2}, \ref{fig:other_categories_step_3})) and (3) the average edit distance (bottom row in Figures \ref{fig:other_categories_step_1},\ref{fig:other_categories_step_2}, \ref{fig:other_categories_step_3})). 

Additionally, we show in \ref{tab:p_values_adjusted_categories_step_1}, \ref{tab:p_values_adjusted_categories_step_2} and \ref{tab:p_values_adjusted_categories_step_3} more information about the statistical values plotted in \ref{fig:rbo_per_category_website}. 
\newpage

\begin{figure}[ht]
    \centering
    \includegraphics[width=1\columnwidth]{Figures/other_measurements_category_websites_step_1.png}
    \caption{Results for Type 1 bots. (Top Row) Average number of categories that were not present in both lists of results in the 10 results. (Middle row) Average number of categories that were not present in the top 3 results for both lists of results being compared. (Bottom Row) Edit distance between lists of categories.  Error bars represent 95\% confidence intervals based on the bootstrapped distribution.}
    \Description{}
    \label{fig:other_categories_step_1}
\end{figure}

\begin{figure}[hb]
    \centering
    \includegraphics[width=1\columnwidth]{Figures/other_measurements_category_websites_step_2.png}
    \caption{Results for Type 2 bots. (Top Row) Average number of categories that were not present in both lists of results in the 10 results. (Middle row) Average number of categories that were not present in the top 3 results for both lists of results being compared. (Bottom Row) Edit distance between lists of categories.  Error bars represent 95\% confidence intervals based on the bootstrapped distribution.}
    \Description{}
    \label{fig:other_categories_step_2}
\end{figure}

\begin{figure}[hb]
    \centering
    \includegraphics[width=1\columnwidth]{Figures/other_measurements_category_websites_step_3.png}
    \caption{Results for Type 3 bots. (Top Row) Average number of categories that were not present in both lists of results in the 10 results. (Middle row) Average number of categories that were not present in the top 3 results for both lists of results being compared. (Bottom Row) Edit distance between lists of categories.  Error bars represent 95\% confidence intervals based on the bootstrapped distribution.}
    \Description{}
    \label{fig:other_categories_step_3}
\end{figure}

\begin{table}[hb]
    \centering
    \caption{Pairwise comparisons of RBO values for categories of websites for Type 1 bots and each search engine, showing p-values and Bonferroni-adjusted p-values. The Mann-Whitney U test was used for the comparisons, and significant adjusted p-values ($<0.05$) are highlighted in bold. The adjustment accounts for 12 comparisons across the three search engines.}
    \Description{}
    \label{tab:p_values_adjusted_categories_step_1}
    \begin{tabular}{|p{1cm}|p{3cm}|p{1cm}|p{1.5cm}|}
    \hline
    \textbf{Search\newline Engine} & \textbf{Groups \newline compared} & \textbf{p-value} & \textbf{Adjusted \newline p-value} \\
    \hline\hline
    \multirow{4}{*}{DuckGo} & Same Location vs. Diff Location - General Queries & \textbf{0.0014} & \textbf{0.0014} \\
    \cline{2-4}
    & Same Location vs. Diff Location - Specific Queries & 0.0659 & 0.7905 \\
    \cline{2-4}
    & General vs. Specific - Same Location & 0.1959 & 2.3512 \\
    \cline{2-4}
    & General vs. Specific - Diff Location & 0.2866 & 3.4392 \\
    \hline \hline
    \multirow{4}{*}{Google} & Same Location vs. Diff Location - General Queries & 0.5375 & 6.4501 \\
    \cline{2-4}
    & Same Location vs. Diff Location - Specific Queries & 0.3560 & 4.2714 \\
    \cline{2-4}
    & General vs. Specific - Same Location & 0.6808 & 8.1692 \\
    \cline{2-4}
    & General vs. Specific - Diff Location & 0.6895 & 8.2741 \\
    \hline \hline
    \multirow{4}{*}{Yahoo} & Same Location vs. Diff Location - General Queries & 0.8202 & 9.8419 \\
    \cline{2-4}
    & Same Location vs. Diff Location - Specific Queries & 0.2639 & 3.1672 \\
    \cline{2-4}
    & General vs. Specific - Same Location & 0.7807 & 9.3680 \\
    \cline{2-4}
    & General vs. Specific - Diff Location & 0.5738 & 6.8860 \\
    \hline
    \end{tabular}
\end{table}
\clearpage
\begin{table}[hb]
    \centering
    \caption{Pairwise comparisons of RBO values for categories of websites for Type 2 bots and each search engine, showing p-values and Bonferroni-adjusted p-values. The Mann-Whitney U test was used for the comparisons, and significant adjusted p-values ($<0.05$) are highlighted in bold. The adjustment accounts for 12 comparisons across the three search engines.}
    \Description{}
    \label{tab:p_values_adjusted_categories_step_2}
    \begin{tabular}{|p{1cm}|p{3cm}|p{1cm}|p{1.5cm}|}
    \hline
    \textbf{Search\newline Engine} & \textbf{Groups \newline compared} & \textbf{p-value} & \textbf{Adjusted \newline p-value} \\
    \hline\hline
    \multirow{4}{*}{DuckGo} & Same Location vs. Diff Location - General Queries & \textbf{0.00} & \textbf{0.03} \\
    \cline{2-4}
    & Same Location vs. Diff Location - Specific Queries & \textbf{0.01} & \textbf{0.15} \\
    \cline{2-4}
    & General vs. Specific - Same Location & 0.81 & 9.76 \\
    \cline{2-4}
    & General vs. Specific - Diff Location & 0.73 & 8.72 \\
    \hline \hline
    \multirow{4}{*}{Google} & Same Location vs. Diff Location - General Queries & \textbf{0.00} & \textbf{0.01} \\
    \cline{2-4}
    & Same Location vs. Diff Location - Specific Queries & \textbf{0.00} & \textbf{0.00} \\
    \cline{2-4}
    & General vs. Specific - Same Location & 0.41 & 4.87 \\
    \cline{2-4}
    & General vs. Specific - Diff Location & 0.75 & 9.01 \\
    \hline \hline
    \multirow{4}{*}{Yahoo} & Same Location vs. Diff Location - General Queries & 0.66 & 7.96 \\
    \cline{2-4}
    & Same Location vs. Diff Location - Specific Queries & \textbf{0.34} & 4.09 \\
    \cline{2-4}
    & General vs. Specific - Same Location & 0.83 & 10.01 \\
    \cline{2-4}
    & General vs. Specific - Diff Location & 0.72 & 8.61 \\
    \hline
    \end{tabular}
\end{table}

\begin{table}[hb]
    \centering
    \caption{Pairwise comparisons of RBO values for categories of websites for Type 3 bots and each search engine, showing p-values and Bonferroni-adjusted p-values. The Mann-Whitney U test was used for the comparisons, and significant adjusted p-values ($<0.05$) are highlighted in bold. The adjustment accounts for 12 comparisons across the three search engines.}
    \Description{}
    \label{tab:p_values_adjusted_categories_step_3}
    \begin{tabular}{|p{1cm}|p{3cm}|p{1cm}|p{1.5cm}|}
    \hline
    \textbf{Search\newline Engine} & \textbf{Groups \newline compared} & \textbf{p-value} & \textbf{Adjusted \newline p-value} \\
    \hline\hline
    \multirow{4}{*}{DuckGo} & Same Location vs. Diff Location - General Queries & 0.5054 & 6.0643 \\
    \cline{2-4}
    & Same Location vs. Diff Location - Specific Queries & \textbf{0.0412} & 0.4941 \\
    \cline{2-4}
    & General vs. Specific - Same Location & 0.90 & 10.76 \\
    \cline{2-4}
    & General vs. Specific - Diff Location & 0.41 & 4.90 \\
    \hline \hline
    \multirow{4}{*}{Google} & Same Location vs. Diff Location - General Queries & 0.09 & 1.12 \\
    \cline{2-4}
    & Same Location vs. Diff Location - Specific Queries & \textbf{0.0004} & \textbf{0.005} \\
    \cline{2-4}
    & General vs. Specific - Same Location & \textbf{0.03} & 0.37 \\
    \cline{2-4}
    & General vs. Specific - Diff Location & \textbf{0.05} & 0.67 \\
    \hline \hline
    \multirow{4}{*}{Yahoo} & Same Location vs. Diff Location - General Queries & 0.9296 & 11.16 \\
    \cline{2-4}
    & Same Location vs. Diff Location - Specific Queries & 0.88 & 10.56 \\
    \cline{2-4}
    & General vs. Specific - Same Location & 0.87 & 10.44 \\
    \cline{2-4}
    & General vs. Specific - Diff Location & 0.71 & 8.55 \\
    \hline
    \end{tabular}
\end{table}


\clearpage
\subsection{Leaning Analysis}
To ask ChatGPT-4o to classify the leaning of news articles, we gave ChatGPT the prompt shown in Figure \ref{prompt_leaning}.

\begin{figure}[h]
    \includegraphics[width=1\columnwidth]{Figures/prompt_leaning_classification.png}
    \caption{Prompt given to ChatGPT for classification of news leaning in the perspecitive of the Israel-Palestine conflict.}
    \Description{}
    \label{prompt_leaning}
\end{figure}

\subsection{Mturkers Methodology}

All Mturkers classifying text received the instructions present in Figure \ref{mturker_instructions}.

\begin{figure}[h]
    \includegraphics[width=1\columnwidth]{Figures/mturker_instructions.png}
    \caption{Instructions given to each Mturker selected to classify texts.}
    \Description{}
    \label{mturker_instructions}
\end{figure}

\begin{figure}[h]
    \includegraphics[width=1\columnwidth]{Figures/heat_mat_chatgpt_mtturkers_no_neutral.png}
    \caption{Heatmap displaying the proportion of bias alignments between annotators and MTurk workers, comparing different bias categories: pro-Israel, slightly pro-Israel, pro-Palestine, and slightly pro-Palestine. We excluded neutral for clarity but it is account for in the proportions. The intensity of the color indicates the proportion of agreement or alignment across the different categories.}
    \Description{}
    \label{heat_map_no_neutral}
\end{figure}

\begin{figure}[h]
    \includegraphics[width=1\columnwidth]{Figures/heat_mat_chatgpt_mtturkers.png}
    \caption{Heatmap displaying the proportion of bias alignments between annotators and MTurk workers, comparing different bias categories: pro-Israel, slightly pro-Israel, neutral, pro-Palestine, and slightly pro-Palestine. The intensity of the color indicates the proportion of agreement or alignment across the different categories.}
    \Description{}
    \label{heat_map}
\end{figure}

\subsection{Time control}

To measure the average variation of search engines results over time, we performed a control by deploying Type 2 webcrawlers in different moments in time (April, May and October). The goal was to understand the overtime variation of the page contents and consequent differences in results. Even if we observe some variation due to the small number of ports per location, smaller number of locations (US and SA) and smaller number of queries being compared (8 per category), we observe no specific trend, except for Yahoo where our differences were never really significant. 

\begin{figure}[h]
    \includegraphics[width=1\columnwidth]{Figures/time_controls_specific_different_location.png}
    \caption{Average values of D = 1- RBO for the same type of bot (Type 2) with different locations (US vs. SA) across different moments in time (Time 1- April, Time 2-May and Time 3- October), for specific queries. Only 8 ports per location were considered.}
    \Description{}
    \label{time_control_1}
\end{figure}

\begin{figure}[h]
    \includegraphics[width=1\columnwidth]{Figures/time_controls_specific_same_location.png}
    \caption{Average values of D = 1- RBO for the same type of bot (Type 2) with same locations (US vs. US or SA vs. SA) across different moments in time (Time 1- April, Time 2-May and Time 3- October), for specific queries. Only 8 ports per location were considered.}
    \Description{}
    \label{time_control_2}
\end{figure}

\begin{figure}[h]
    \includegraphics[width=1\columnwidth]{Figures/time_controls_general_different_location.png}
    \caption{Average values of D = 1- RBO for the same type of bot (Type 2) with different locations (US vs. SA) across different moments in time (Time 1- April, Time 2-May and Time 3- October), for general queries. Only 8 ports per location were considered.}
    \Description{}
    \label{time_control_3}
\end{figure}

\begin{figure}[h]
    \includegraphics[width=1\columnwidth]{Figures/time_controls_general_same_location.png}
    \caption{Average values of D = 1- RBO for the same type of bot (Type 2) with same locations (US vs. US or SA vs. SA) across different moments in time (Time 1- April, Time 2-May and Time 3- October), for general queries. Only 8 ports per location were considered.}
    \Description{}
    \label{time_control_4}
\end{figure}


% \begin{figure}[h!]
%     \centering
%     \includegraphics[width=0.8\columnwidth]{Figures/venn_unique_news_controled_step_3.png}
%     \caption{Venn diagram of unique domains present in Step 1 - Location only (green), Step 2 - Location + browser languages (yellow) and Step 3 - Location + browser languages + browsing history (blue) and its intersections.}
%     \Description{}
%     \label{venn_unique_news}
% \end{figure}


% \begin{table}[h!]
%     \centering
%     \caption{Number of unique news domains per search engine.}
%     \label{tab:unique_news}
%     \begin{tabular}{|p{0.8cm}|p{2cm}|p{2cm}|}
%     \hline
%     {} & \textbf{Search Engine} & \textbf{Num Unique News Websites} \\
%     \hline
%     \multirow{4}{*}{Step 1} & Bing &  44 \\
%     \cline{2-3}
%     {} & Duckduckgo & 39 \\
%     \cline{2-3}
%     {} & Google &  26 \\
%     \cline{2-3}
%     {} & Yahoo & 20 \\
%     \hline
%     \multirow{4}{*}{Step 2} & Bing & 49 \\
%     \cline{2-3}
%     {} & Duckduckgo & 47 \\
%     \cline{2-3}
%     {} & Google & 34 \\
%     \cline{2-3}
%     {} & Yahoo &  21 \\
%     \hline
%     \multirow{4}{*}{Step 3} &  Bing & 55 \\
%     \cline{2-3}
%     {} & Duckduckgo & 51 \\
%     \cline{2-3}
%     {} & Google &  54 \\
%     \cline{2-3}
%     {} & Yahoo & 32 \\
%     \hline
%     \end{tabular}
% \end{table}
\end{document}

