%%
%% This is file `sample-sigconf-authordraft.tex',
%% generated with the docstrip utility.
%%
%% The original source files were:
%%
%% samples.dtx  (with options: `all,proceedings,bibtex,authordraft')
%% 
%% IMPORTANT NOTICE:
%% 
%% For the copyright see the source file.
%% 
%% Any modified versions of this file must be renamed
%% with new filenames distinct from sample-sigconf-authordraft.tex.
%% 
%% For distribution of the original source see the terms
%% for copying and modification in the file samples.dtx.
%% 
%% This generated file may be distributed as long as the
%% original source files, as listed above, are part of the
%% same distribution. (The sources need not necessarily be
%% in the same archive or directory.)
%%
%% Commands for TeXCount
%TC:macro \cite [option:text,text]
%TC:macro \citep [option:text,text]
%TC:macro \citet [option:text,text]
%TC:envir table 0 1
%TC:envir table* 0 1
%TC:envir tabular [ignore] word
%TC:envir displaymath 0 word
%TC:envir math 0 word
%TC:envir comment 0 0
%%
%%
%% The first command in your LaTeX source must be the \documentclass
%% command.
%%
%% For submission and review of your manuscript please change the
%% command to \documentclass[manuscript, screen, review]{acmart}.
%%
%% When submitting camera ready or to TAPS, please change the command or whichever template is required
%% for your publication.
%%
%%
% \documentclass[sigconf, anonymous, review]{acmart}
\documentclass[sigconf, anonymous=false, nonacm=true, pbalance=true]{acmart}
% \documentclass[sigconf]{acmart}
\usepackage{multirow}
% \usepackage[table]{xcolor}
\usepackage[most]{tcolorbox}
\usepackage{booktabs}
\usepackage{tabularx}
\usepackage{longtable}
\usepackage{array}
\usepackage{pifont}
\usepackage{makecell}
\usepackage{hyperref}

\newtcbtheorem{Finding}{\bfseries Finding}{enhanced,drop shadow={black!50!white},
  coltitle=black,
  top=0.3in,
  attach boxed title to top right=
  {xshift=0em,yshift=-\tcboxedtitleheight/2},
  boxed title style={size=small,colback=darkgray}
}{finding}

%% NS: Takeaway macros

\usepackage{tcolorbox}
\definecolor{verylightgrey}{gray}{0.9}
\newtcolorbox{takeaway}{
  colback=verylightgrey,
  colframe=verylightgrey,
  sharp corners,
  boxrule=0mm,
  boxsep=0mm,
  left=1mm,
  right=1mm,
  top=1mm,
  bottom=1mm
}

\newcounter{myfindingscounter}
\setcounter{myfindingscounter}{0}
\newcommand\takeawaytitle[1]{\textbf{Finding \refstepcounter{myfindingscounter}\themyfindingscounter\label{#1}:}}

% Usage:
% \begin{takeaway}
% Write your take away.
% \end{takeaway}

%%

\usepackage{xspace}

\newcommand{\mypara}[1]{\vspace{4pt}\noindent{\textbf{{#1}}\xspace}}

%%
%% \BibTeX command to typeset BibTeX logo in the docs
\AtBeginDocument{%
  \providecommand\BibTeX{{%
    Bib\TeX}}}

%% Rights management information.  This information is sent to you
%% when you complete the rights form.  These commands have SAMPLE
%% values in them; it is your responsibility as an author to replace
%% the commands and values with those provided to you when you
%% complete the rights form.
\setcopyright{acmlicensed}
\copyrightyear{2018}
\acmYear{2018}
\acmDOI{XXXXXXX.XXXXXXX}

%% These commands are for a PROCEEDINGS abstract or paper.
% \acmConference[WWW '25]{The Web Conference}{April 28--May 2,
%   2025}{Sydney, Australia}
%%
%%  Uncomment \acmBooktitle if the title of the proceedings is different
%%  from ``Proceedings of ...''!
%%
%%\acmBooktitle{Woodstock '18: ACM Symposium on Neural Gaze Detection,
%%  June 03--05, 2018, Woodstock, NY}
\acmISBN{978-1-4503-XXXX-X/18/06}


%%
%% Submission ID.
%% Use this when submitting an article to a sponsored event. You'll
%% receive a unique submission ID from the organizers
%% of the event, and this ID should be used as the parameter to this command.
\acmSubmissionID{2085}

%%
%% For managing citations, it is recommended to use bibliography
%% files in BibTeX format.
%%
%% You can then either use BibTeX with the ACM-Reference-Format style,
%% or BibLaTeX with the acmnumeric or acmauthoryear sytles, that include
%% support for advanced citation of software artefact from the
%% biblatex-software package, also separately available on CTAN.
%%
%% Look at the sample-*-biblatex.tex files for templates showcasing
%% the biblatex styles.
%%

%%
%% The majority of ACM publications use numbered citations and
%% references.  The command \citestyle{authoryear} switches to the
%% "author year" style.
%%
%% If you are preparing content for an event
%% sponsored by ACM SIGGRAPH, you must use the "author year" style of
%% citations and references.
%% Uncommenting
%% the next command will enable that style.
%%\citestyle{acmauthoryear}
\newcommand{\supplementarysection}{%
  \setcounter{figure}{0}% Reset figure counter
  \let\oldthefigure\thefigure% Capture figure numbering scheme
  \renewcommand{\thefigure}{S\oldthefigure}% Prefix figure number with S
  \setcounter{table}{0}% Reset figure counter
  \let\oldthetable\thetable%
  \renewcommand{\thetable}{S\oldthetable}%
}


%%
%% end of the preamble, start of the body of the document source.
\begin{document}

%%
%% The "title" command has an optional parameter,
%% allowing the author to define a "short title" to be used in page headers.
\title{Digital Gatekeeping: An Audit of Search Engine Results shows tailoring of queries on the Israel-Palestine Conflict}

%%
%% The "author" command and its associated commands are used to define
%% the authors and their affiliations.
%% Of note is the shared affiliation of the first two authors, and the
%% "authornote" and "authornotemark" commands
%% used to denote shared contribution to the research.
\author{Íris Damião}
% \authornote{Both authors contributed equally to this research.}
\orcid{0009-0005-4931-2376}
\affiliation{
  \institution{LIP - Laboratory for Instrumentation and Particle Physics \&  Instituto Superior Técnico - University of Lisbon}
  \city{Lisbon}
  \country{Portugal}}
\email{irisdamiao@lip.pt}

\author{José M. Reis}
\orcid{0000-0002-8055-0170}
\affiliation{
 \institution{LIP, Laboratory for Instrumentation and Particle Physics*}
 \city{Lisbon}
 \country{Portugal}}
 \thanks{* current affiliation: Portuguese National Cybersecurity Centre, Lisbon, Portugal.}
 
\author{Paulo Almeida}
\orcid{0000-0002-9279-2353}
\affiliation{
 \institution{LIP, Laboratory for Instrumentation and Particle Physics}
 \city{Lisbon}
 \country{Portugal}}

\author{Nuno Santos}
\orcid{0000-0001-9938-0653}
\affiliation{%
  \institution{INESC-ID \& Instituto Superior Técnico, University of Lisbon}
  \city{Lisbon}
  \country{Portugal}
}

\author{Joana Gonçalves-Sá}
\orcid{0000-0001-6654-2126}
\affiliation{
  \institution{LIP, Laboratory for Instrumentation and Particle Physics \& NOVA LINCS, FCT NOVA University}
  \city{Lisbon \& Caparica}
  \country{Portugal}}
\email{joanagsa@lip.pt}

%%
%% By default, the full list of authors will be used in the page
%% headers. Often, this list is too long, and will overlap
%% other information printed in the page headers. This command allows
%% the author to define a more concise list
%% of authors' names for this purpose.
\renewcommand{\shortauthors}{Damião et al.}

%%
%% The abstract is a short summary of the work to be presented in the
%% article.
\begin{abstract}

Search engines, often viewed as reliable gateways to information, tailor search results using customization algorithms based on user preferences, location, and more. While this can be useful for routine queries, it raises concerns when the topics are sensitive or contentious, possibly limiting exposure to diverse viewpoints and increasing polarization.

To examine the extent of this tailoring, we focused on the Israel-Palestine conflict and developed a privacy-protecting tool to audit the behavior of three search engines: DuckDuckGo, Google and Yahoo. Our study focused on two main questions: (1) How do search results for the same query about the conflict vary among different users? and (2) Are these results influenced by the user's location and browsing history?

Our findings revealed significant customization based on location and browsing preferences, unlike previous studies that found only mild personalization for general topics. Moreover, queries related to the conflict were more customized than unrelated queries, and the results were not neutral concerning the conflict's portrayal.

%Search engines have become integral tools in our daily lives, often recognized as reliable gateways to obtain the “truth”. However, it is well established that these algorithms apply personalization techniques to prioritize search results based on user preferences, location, and more. This tailored experience, whether useful in routine tasks, raises strong concerns regarding exposure to different viewpoints, particularly around contentious issues.

%Taking the ongoing Israel-Palestine conflict as a case - study, we built a privacy protecting search-engine audit tool to address search engine levels of personalization.
%, for the first time for such a popular, sensitive and timely subject. Specifically, as the world sought answers using such algorithms, 
%Specifically, we questioned: (1) how do search engine results vary among users seeking identical information about the conflict? and (2) are these results personalized to the user's location and past browsing history?. 
%personalization levels different for broader/non-specific topics? 
%We found significant levels of customization due to location and browsing settings, contrasting with previous research that suggested milder customization for more general topics. Additionally, our results show generally stronger levels of customization for queries related with the conflict than non-related queries and these results were not neutral in terms of conflict leaning.
%This comprehensive analysis, coupled with the contemporaneous nature of the topic, is key to understand the scope and potential implications of search engine personalization.

\end{abstract}

%% ATTENTION - CHANGE BEFORE SUBMISSION
%% The code below is generated by the tool at http://dl.acm.org/ccs.cfm.
%% Please copy and paste the code instead of the example below.
%%
% \begin{CCSXML}
% <ccs2012>
%  <concept>
%   <concept_id>00000000.0000000.0000000</concept_id>
%   <concept_desc>Do Not Use This Code, Generate the Correct Terms for Your Paper</concept_desc>
%   <concept_significance>500</concept_significance>
%  </concept>
%  <concept>
%   <concept_id>00000000.00000000.00000000</concept_id>
%   <concept_desc>Do Not Use This Code, Generate the Correct Terms for Your Paper</concept_desc>
%   <concept_significance>300</concept_significance>
%  </concept>
%  <concept>
%   <concept_id>00000000.00000000.00000000</concept_id>
%   <concept_desc>Do Not Use This Code, Generate the Correct Terms for Your Paper</concept_desc>
%   <concept_significance>100</concept_significance>
%  </concept>
%  <concept>
%   <concept_id>00000000.00000000.00000000</concept_id>
%   <concept_desc>Do Not Use This Code, Generate the Correct Terms for Your Paper</concept_desc>
%   <concept_significance>100</concept_significance>
%  </concept>
% </ccs2012>
% \end{CCSXML}

% \ccsdesc[500]{Do Not Use This Code~Generate the Correct Terms for Your Paper}
% \ccsdesc[300]{Do Not Use This Code~Generate the Correct Terms for Your Paper}
% \ccsdesc{Do Not Use This Code~Generate the Correct Terms for Your Paper}
% \ccsdesc[100]{Do Not Use This Code~Generate the Correct Terms for Your Paper}

%%
%% Keywords. The author(s) should pick words that accurately describe
%% the work being presented. Separate the keywords with commas.
\keywords{Personalization, Search engines, Audit, Filter Bubble Effect}
% %% A "teaser" image appears between the author and affiliation
% %% information and the body of the document, and typically spans the
% %% page.
% \begin{teaserfigure}
%   \includegraphics[width=\textwidth]{sampleteaser}
%   \caption{Seattle Mariners at Spring Training, 2010.}
%   \Description{Enjoying the baseball game from the third-base
%   seats. Ichiro Suzuki preparing to bat.}
%   \label{fig:teaser}
% \end{teaserfigure}

% \received{20 February 2007}
% \received[revised]{12 March 2009}
% \received[accepted]{5 June 2009}

%%
%% This command processes the author and affiliation and title
%% information and builds the first part of the formatted document.
\maketitle

\section{Introduction}
\documentclass[../main.tex]{subfiles}
\graphicspath{{../images/}}
\makeatletter
\def\input@path{{../images/}}
\makeatother
\begin{document}
\section{Introduction}
\begin{figure}
\centering
\begin{tikzpicture}
\node[inner sep=0pt] (ws) at (0, 0) {
\includegraphics[height=.4\textwidth, trim={10cm 0 10cm 0},clip]{world_space.png}};
\node[inner sep=0pt] (cs) at (6,0) {\includegraphics[height=.4\textwidth, trim={10cm 1cm 10cm 4cm},clip]{conf_space.png}};
\end{tikzpicture}
\vspace{-5pt}
\label{fig:pbrm_intro}
\caption{\textbf{Left}: Shows world space obstacles as grey spheres. Robots start and goal configuration is colored red and green, respectively. Configurations along the computed path are colored transparent blue. \textbf{Right:} Mapped world space scenario to configuration space. Obstacle region is the grey mesh. Red spheres are collision-free regions computed by the neural SCDF. The optimized shortest path in the convex corridor is the blue curve.}
\vspace{-25pt}
\end{figure}
Motion planning is the problem of finding a collision-free trajectory that connects a given start and goal configuration. The planning takes place in the configuration space of the robot. For single body robots, like mobile robots or drones, the configuration space and the world space are usually the same. This simplifies the planning, since explicit obstacle representations are available which enables geometrical tools like separating hyperplanes, smallest distance to obstacles etc., to be used when designing motion planning algorithms. For multi-body robots like manipulators, the situation is completely different. The world space obstacles are usually mapped to non-convex regions, and to make the problem even harder, the mapping is usually not known. Forming explicit representations of the obstacle region in the configuration space is usually too expensive or intractable. Despite all of this, sampling based planners are used with great success, which mainly is due to their use of implicit representations of the obstacle region. The basic idea is to construct a graph in the configuration space that covers and connects the collision-free region. From this graph, a path can be extracted that connects a given start and goal configuration. The approach is computationally expensive, since the graph is constructed with the smallest geometrical building block available, points, which represents a collision-check. Furthermore, the extracted paths from the graph are non-smooth and jagged due to the stochastic nature of the approach. This adds an additional post-processing step to the process, where the paths are shortcutted and smoothened, before the path can be used for tracking. Clearly a lot of time is invested to form this graph and produce smooth paths. Thus, if the obstacles start to move, then all of this work is done in no use, since all points that make up this graph need to be re-verified, which is simply too time consuming to be done in real time.
\\\\
In this work, we want to address the existing drawbacks of the sampling based planners. Our main contribution is an improved motion planner where each vertex in the graph covers a collision-free region in the form of a sphere instead of a point and where the edges are formed with neighboring intersecting spheres. This representation has the advantage of instead of returning piecewise linear paths, returning a sequence of overlapping spheres, i.e. a convex corridor, that connects a given start and goal configuration, illustrated in Figure \ref{fig:pbrm_intro}. This convex corridor allows us to use convex optimization to produce smooth trajectories, instead of computationally expensive post-processing methods. The representation further allows us to estimate the coverage of the collision-free space, which gives us awareness and feedback in the offline roadmap construction phase. Finally, our representation is simple to adapt to moving obstacles, simply requery for the new radii and recheck for intersections. 
\\\\
The spherical collision-free regions are formed using a signed distance function (SDF), which is a function that returns the smallest distance from an arbitrary point to the boundary of an obstacle. As the name implies, the distance is signed, thus if the point is inside the obstacle it is negative otherwise positive. If the distance is positive, a sphere with radius equal to the distance is guaranteed to cover a collision-free region. Using an SDF in motion planning is not new, but what is novel about our approach is that we express the distance in the configuration space instead of the world space and by doing so allows us to form these convex collision-free regions. We refer to the resulting SDF as a signed configuration distance function (SCDF). Computing an SCDF analytically is non-trivial, our approach is therefore to parameterize the SCDF with a deep neural network and learn the mapping by supervised learning. Our resulting neural SCDF can compute distances for different parameter values of obstacle shapes and we also show how multiple distances can be combined, thus making our approach flexible.
\section{Related work}
Motion planning algorithms can roughly be divided into three families, grid-based, sampling based and optimization based methods. Grid-based methods (GBM) discretize the planning space from which a graph is then compiled. A standard search method is A$^\star$ \citep{a_star}, which is classified as an \textit{informed} search method, since it employs a heuristic function to speed up the search. A$^\star$ guarantees to return an optimal path at the level of discretization used. GBMs usually discretize the planning space by a regular lattice and this limits the GBMs to problems with low dimensionality due to the curse of dimensionality. Thus, GBMs are usually limited to single-body robots where the degrees of freedom (DOF) are low. To overcome the inherent scaling problem with the GBMs, stochastic methods are usually used for multi-body robots. These methods are termed as sampling-based methods (SBM) and core members within this family are the rapidly-exploring random trees (RRT) \citep{rrt} and the probabilistic roadmap (PRM) \citep{prm}. RRT grows a tree from the start configuration and explores the collision-free region in a rapid way until it is able to connect to the goal region. RRT is usually improved by bi-directional planning \citep{rrt_connect}, i.e. an additional tree is grown from the goal configuration and the trees are tested for connection after any tree has been expanded. RRT is a single-query method, thus it searches for a path from scratch each time it is queried. Contrary to this, PRM is a multi-query method, which solves for multiple queries without starting from scratch. PRM does this by creating a roadmap (graph) that covers the collision-free space as an offline step. The graph is then used to solve for multiple queries. PRMs are used in cases where the environment does not change since the extra offline step is too computationally costly and needs to be re-done if the environment is changed. In our work, we address this inherent issue by using a different roadmap representation. Our vertices in the graph cover a collision-free region in the form of spheres and we form the edges by checking for intersecting spheres. If something in the environment changes, we recompute the spheres radii and recheck the intersections, without relying on collision detection. We use a trained neural network to compute the sphere radius, therefore querying for the radius can be done fast, hence our representation enables the PRM for dynamic environments.
\\\\
In the recent decades, optimization based methods (OBM) \citep{chomp, schulman, itomp, stomp} have been introduced as an alternative to SBM for multi-body robots. Like the SBM, the OBMs scale well to higher dimensional problems and produce smoother motion. It is common to use a SDF in the optimization since it is a smooth function, thus enabling gradient-based methods. However, the standard way of expressing the SDF is in world space. The distance therefore needs to be mapped to the configuration space by the forward kinematics. This mapping makes the optimization problem a non-linear program (NLP), which is computationally expensive to solve. Recently, a different approach has been proposed. In \cite{mp_gcs} motion planning is formulated as a convex optimization problem by using the graph of convex sets framework \citep{gcs}. The underlying idea is to decompose the collision-free space into intersecting convex sets from which a convex optimization problem is formulated. In cases where an explicit representation of the obstacles in the configuration space exists, like for single-body robots, creating collision-free convex regions can be done fast \citep{iris}. For multi-body robots, this is non-trivial. Existing work does this successfully \citep{iris_nlp, iris_c} by an optimization based approach, but the methods are still too time consuming to be used in the presence of moving obstacles. Our approach is instead to use deep learning to learn an SDF expressed in the configuration space. With this, we can query for shortest distances to the collision boundary, which allows us to expand spherical regions which are collision-free. Our approach is fast and therefore enables our suggested roadmap planner to be used in dynamic environments.
\\\\
Recent research has focused on learning collision detection \citep{fk_kernel_distance, diffco, graphdistnet} by predicting the signed distance between the robot links and the surrounding obstacles in the world space. The learned SDF is used in trajectory optimization but since the distance is expressed in the world space, the problem becomes an NLP and therefore takes a long time to solve. We take a novel approach and suggest to instead express the signed distance in the configuration space. This allows us to improve the PRM at the same time as it enables convex optimization for trajectory optimization, which runs faster and is more reliable than NLP solvers. In \cite{cspf} a learned signed distance function in the configuration space is proposed similar to our approach. However, their approach is restricted to point cloud representations, while we propose to represent the obstacles as parameterized geometric shapes, e.g. spheres. Furthermore, we also show how to use our learned SCDF to improve an existing roadmap planner.
\section{Problem formulation}
A robot is located in the world space, $\W \subset \R^3 $. The unique location of the robot is given by its configuration $\q \in \C$, where $\C$ is the configuration space. The set of points covered by the robots bodies at a certain configuration is expressed as $\B(\q) \subset \W$. The robot is surrounded by $\NrObst$ obstacles $\O = \bigcup_{i=1}^{\NrObst} \O_i$, where  $\O_i \subset \W$. The representation of the obstacle in the configuration space is the set $\C\O_i = \{\q \in \C \: |\: \B(\q) \cap \O_i \neq \emptyset \}$. The obstacle space is formed as $\Co = \bigcup_{i=1}^{\NrObst} \C \O_i$. The complement is referred to as the free space, $\Cf = \C \setminus \Co$. The path planning problem is a tuple, ($\Cf$, $\qStart$, $\qGoal$), where we want to connect a query pair, consisting of a start, $\qStart$, and goal configuration, $\qGoal$, with a geometric path, $\q(s): [0, 1] \mapsto \Cf$, such that $\q(0)=\qStart$ and $\q(1)=\qGoal$, or report correctly when such a path does not exist.
\end{document}


\vspace{-0.1cm}
\section{Background and Related Work}
\section{Related Work}
% \subsection{Vision Language Model}
% 시각장애인에서 상황을 설명할 DB가 없으니 만들었다. 그리고 이를 VLM에 튜닝했다.
\subsection{Technical approaches for assisting the visually-impaired}


\subsection{Datasets for visual instruction tuning}


\section{Methodology}

\section{\label{sec:method}Methodology}

Each SIEM system uses its own RDL to define threat detection rules, and each RDL has its own schema.
For example, the Splunk SIEM uses the SPL to define its threat detection rules.
The task of understanding threat detection rules and recommending relevant MITRE ATT\&CK techniques (or sub-techniques) requires complex reasoning skills.
In the case of LLMs, this can be achieved with a technique called prompt chaining in which each task is divided into multiple sub-tasks in order to understand the complex reasoning behind the task.
Therefore, we employ a multi-phase architecture based on prompt chaining that leverages the power of LLMs to take a SIEM rule defined in any RDL and map it to relevant MITRE ATT\&CK techniques using the power of LLMs.
Our approach is based on the following intuitions:
\begin{itemize}[nosep,leftmargin=*]
    \item \textit{LLMs' implicit knowledge}: LLMs possess deep understanding of diverse RDLs. This enables them to interpret any rule, regardless of the RDL it is defined in, and convert it into comprehensible natural language text.
    \item \textit{LLMs' similarity comparison capability}: LLMs are adept at analyzing and comparing textual descriptions. 
    They can intelligently assess the similarity between two textual inputs to establish a meaningful connection.
\end{itemize}

\methodName has two main phases: (1) the rule to text translation phase, and (2) the MITRE ATT\&CK techniques recommendation phase.
These two phases in the pipeline include six key steps to determine relevant TTPs, as illustrated in Figure~\ref{fig:r2t}.

Although LLMs excel at translating SIEM rules into natural language, they often lack critical domain-specific contextual information related to IoCs in the rules.
To overcome this limitation, the \textit{rule to text translation} phase includes three steps: IoC extraction, contextual information retrieval, and natural language translation.

The workflow begins with the extraction of IoCs from the rules (for example, processes, log source, event codes, and file names) that the rule searches for in the logs (step (1)).In the next sstep a web search agent performs the task of obtaining additional contextual information about the IoCs discovered ((step 2)).
By incorporating this additional domain-specific information, the pipeline enhances the language translation, resulting in a more accurate and meaningful interpretation of SIEM rules.
The rule itself and the IoCs' contextual additional information from the previous stage are then used to translate the rule from RDL to natural language (step (3)).

The \textit{MITRE ATT\&CK techniques} recommendation phase of the pipeline includes the following three steps.
The rule in processed in data source identification step in which the probable origin of the data is identified (step (4)).
The description of the rule is then used to determine probable MITRE ATT\&CK techniques based on the implicit knowledge of the LLM (step (5)).
Finally, using chain-of-thought~\cite{wei2022chain} prompting, the most relevant techniques are extracted from the list of probable techniques (step (6)).
Each step of our method is further described in detail below.


% [bb=0 0 1440 900,width=1.43\linewidth,height=0.9\textwidth]
\begin{figure*}[htbp]
   \includegraphics[width=\textwidth]{Images/stages.jpg}
    
   \caption{An illustration of the different steps in \methodName.}
   \label{fig:stages}
\end{figure*} 

\subsection{IoC Extraction}
The context associated with a SIEM detection rule is crucial for its accurate interpretation and effective application. 
Obtaining this contextual understanding requires comprehensive analysis of the embedded IoCs in the SIEM rule.
In the first step, \methodName systematically identifies and extracts all IoCs, identifying the types of IoCs and their corresponding values that form the foundational elements of the detection rules. 
Leveraging the LLM's inherent understanding of rule structures and IoCs, we employ a zero-shot prompting approach for this task. 
Zero-shot prompting enables the direct extraction of IoCs from the rules without requiring extensive pre-training on specific datasets.

\noindent The result of this stage is a dictionary structure, where:
\begin{itemize}[nosep,leftmargin=*]
    \item Keys represent types of IoC, such as processes, files, IP addresses, and log sources.
    \item Values are lists containing specific IoC details, such as process names, file names, IP addresses, and log source identifiers.
\end{itemize}

In the example depicted in Figure~\ref{fig:stages}(a), the pipeline processes a rule for which relevant MITRE ATT\&CK techniques need to be recommended. 
The IoC extractor LLM produces a dictionary structure as output, organizing the IoCs in a structured format to support subsequent stages in the analysis pipeline. 



\subsection{Contextual Information Retrieval}
In this step, an LLM agent is employed to retrieve relevant information pertaining to the IoCs extracted from the rule.
A REACT agent~\cite{react} was used in this case to generate both reasoning traces and task-specific actions in an interleaved manner.
REACT agents interact with external tools to retrieve additional information that leads to more factual and reliable responses.
The LLM agent conducts a systematic search across web resources to gather additional contextual information for each IoC value present in the rule. 
This step addresses LLMS' lack of up-to-date knowledge or specialized domain expertise (which is critical to understanding the role and significance of the IoCs in the rule), without the need for retraining or fine-tuning.
Figure~\ref{fig:stages}(b) presents an example in which the rule includes the process name \texttt{soaphound.exe} as an IoC.
As can be seen, the web search results indicate that \texttt{soaphound.exe} is being used for active directory (AD) enumeration, which is important for the understanding of the attack. 

\subsection{Natural Language Translation}

The translation of detection rules into natural language textual descriptions fulfills three key objectives:
\begin{enumerate}[nosep,leftmargin=*]
    \item \textbf{Ensures that \methodName is format-agnostic}: It converts rules defined in various RDL formats into a generic, unstructured text format, ensuring compatibility with different SIEM systems, regardless of the specific rule format.
    \item \textbf{Provides contextual explanation}: It includes all relevant contextual information to produce a concise and comprehensible explanation of the rule.
    \item \textbf{Enhances the comprehension for LLMs}: It enables LLMs to more effectively compare the translated rule with descriptions in the MITRE ATT\&CK framework by providing a unified textual representation.
\end{enumerate}
To achieve these objectives, a zero-shot prompting technique is employed. 
The input to the LLM comprises two components:
\begin{itemize}
    \item \textbf{Syntactical information}: The rule itself, providing the structural and operational details.
    \item \textbf{Contextual information}: Details of the IoCs extracted from the rule, providing semantic insights into the rule's intent and function.
\end{itemize}
The LLM utilizes these inputs to generate a natural language textual description of the rule. 
This transformation not only ensures a more interpretable representation but also facilitates further steps of analysis and comparison, particularly in aligning the rule with MITRE ATT\&CK techniques and sub-techniques.



\subsection{Data Source or Mitigation Identification}
Identifying the most relevant data component or mitigation associated with the rule description in this step is critical for filtering out irrelevant MITRE ATT\&CK techniques (or sub-techniques) in subsequent steps of the pipeline.
In the MITRE ATT\&CK framework, data sources represent various categories of information that can be gathered from sensors or logs. 
These data sources include \textit{data components}, which are specific attributes or properties within a data source that are directly relevant to detecting a particular technique or sub-technique~. 
For example, in the context of the rule described in Figure~\ref{fig:stages}(a), the term \texttt{Endpoint.Processes} indicates that the activity is happening on an endpoint. 
Presence of the terms such as, \texttt{soaphound.exe}, \texttt{--buildcache}, \texttt{--certdump} and etc. indicate that the rule searches for command line execution of an executable named \texttt{soaphound.exe} with specific parameters. 
Therefore, the appropriate data source in this example is \textit{Command}, with the corresponding data component being \textit{Command Execution}.
Additionally, \textit{mitigations} are defined as categories of technologies or strategies that can prevent or reduce the impact of specific techniques or sub-techniques. 
The MITRE ATT\&CK framework explicitly establishes relationships between data components, mitigations, and techniques (or sub-techniques), enabling a systematic approach for identifying relevant elements.

To identify the most relevant data component or mitigation associated with a given rule description, we utilize agentic retrieval augmented generation (RAG), which incorporates an AI Agent-based implementation of the RAG framework.
Data from the MITRE ATT\&CK framework, specifically related to data components and mitigations, is stored in a vector database (e.g., ChromaDB). 
The process begins with the rule description from the previous stage, which serves as the input to the AI Agent. 
The LLM-powered agent automatically generates a search query tailored to retrieve relevant information from the RAG database.

For each query, the system retrieves the five most similar documents from the database, each containing contextual information about data components or mitigations. 
These documents are then utilized by the LLM agent to contextualize the rule description. 
By comparing the content of these retrieved documents with the rule description, the LLM agent determines and outputs the most relevant data component or mitigation along with a chain-of-thought as to why the data component or mitigation is related to the rule.


\subsection{Probable Technique Recommendation}

In this step, an LM agent is utilized to propose probable MITRE ATT\&CK techniques (and sub-techniques) that may be relevant to the description of the provided rule.
We used a REACT agent in this step as well to utilize both implicit and explicit knowledge during reasoning.
For explicit knowledge, the agent searches the MITRE ATT\&CK framework to obtain the list of probable techniques (and sub-techniques).
The natural language description of the rule from the previous step serves as input to the LLM agent.
The output of this stage consists of a list of JSON objects, each containing the MITRE technique ID, technique name, and technique description as seen in Figure~\ref{fig:stages}(c).

Throughout our experiments, we observed that as the number of recommendations ($k$) increases, both the framework's average recall and precision initially improve, however beyond a certain threshold of $k$, the %average 
precision begins to decline.
Based on these observations(please refer Table~\ref{tab:results3}), we selected a $k$-value of 11 to ensure a high recall.



\subsection{Relevant Technique Extraction}
In this step, \methodName refines the set of probable MITRE ATT\&CK techniques identified in the previous stage by eliminating irrelevant entries.
This step in the pipeline serves two primary purposes: (1) to enhance precision while maintaining recall achieved in previous step, and (2) to provide a clear rationale for the selection of the labels, ensuring transparency and interpretability of the mapping process.
This refinement process is grounded in the assumption that LLMs are effective for text similarity matching tasks.

The process comprises two key steps:
\begin{itemize}
    \item \textit{Rule-technique comparison}: The description of each technique in the set of probable techniques is compared with the rule description. 
    A chain-of-thought technique is then applied to elucidate the reasoning behind the association of each technique with the rule.
    \item \textit{Confidence calculation}: The generated chain-of-thought rationale for each technique (or sub-technique) is compared with the rule description to compute a relevance (or confidence) score, as done in prior work~\cite{freitas2024ai}.
    % \item \textbf{Reasoning}: \new{Add here the reasoning that it provides - explaining in NLP why it was selected...}
\end{itemize}

Techniques with higher confidence scores are deemed more relevant to the rule. 
Conversely, techniques with scores falling below a predefined threshold are excluded.
The techniques retained after this filtering step represent the most relevant techniques corresponding to the given rule's description. 


The chain-of-thought (CoT) rationale generated during the comparison of each rule to its probable technique is also provided as an output in this step.
This rationale offers a detailed natural language explanation, articulating why a particular technique is relevant to the given rule. 
Such explanations are highly valuable for security analysts, as they provide clear and transparent reasoning behind the mapping, enabling analysts to better understand and validate the association between the rule and the technique.
Other classification models proposed in previous works within this domain also suffer from the limitation of being black-box models, which lack the ability to provide clear reasoning or explanations. 
Unlike \methodName, these models fail to generate transparent, CoT rationales that explain why a particular rule is mapped to a specific technique, making them less interpretable and less useful for security analysts.

% \subsection{Ethics}
% This research focuses on analyzing the internal representations of numerical values in Large Language Models and does not involve human subjects, sensitive data, or direct societal impact.

\section{Results}
\begin{table}[ht!]
\centering
\caption{\textbf{Super Resolution Performance Results.} Our proposed WGAN EEG Spatial Upsampling method significantly outperforms a baseline of Bicubic Interpolation commonly used in EEG upsampling pipelines.}
\label{tab:results}
\resizebox{0.8\linewidth}{!}{%
\begin{tabular}{@{}cccccc@{}}
\toprule
\multirow{2}{*}{\textbf{Dataset}} & \multirow{2}{*}{\textbf{Scale}} & \multicolumn{2}{c}{\textbf{Bicubic}} & \multicolumn{2}{c}{\textbf{WGAN}} \\ \cmidrule(l){3-6} 
                      &   & \textbf{MSE} & \textbf{MAE} & \textbf{MSE}    & \textbf{MAE}   \\
\toprule
\multirow{2}{*}{Val}  & 2 & 3.71E7       & 3.89E3       & \textbf{2.01E3} & \textbf{24.38} \\
                      & 4 & 7.23E7       & 6.42E3       & \textbf{8.53E3} & \textbf{63.83} \\
\midrule
\multirow{2}{*}{Test} & 2 & 3.75E7       & 3.91E3       & \textbf{2.06E3} & \textbf{24.66} \\
                      & 4 & 7.30E7       & 6.45E3       & \textbf{8.68E3} & \textbf{64.39} \\
\bottomrule
\end{tabular}%
}
\end{table}

\section{Discussion}
This work identifies signal collapse as a critical bottleneck in one-shot neural network pruning. Performance loss in pruned networks is due to \textbf{signal collapse} in addition to the removal of critical parameters. We propose \textbf{REFLOW} (\textbf{Re}storing \textbf{F}low of \textbf{Low}-variance signals), a simple yet effective method that mitigates signal collapse without computationally expensive weight updates. By focusing on signal preservation, REFLOW highlights the importance of mitigating signal collapse in sparse networks and enables magnitude pruning to match or surpass state-of-the-art one-shot pruning methods such as CHITA, CBS, and WF.

REFLOW consistently achieves state-of-the-art accuracy across diverse architectures, restoring ResNeXt-101 from under 4.1\% to 78.9\% top-1 accuracy at 80\% sparsity on ImageNet. Its lightweight design makes it a practical solution for both research and deployment, delivering high-quality sparse models without the overhead of traditional approaches. These findings challenge the traditional emphasis on weight selection strategies and underscore the critical role of signal propagation for achieving high-quality sparse networks in the context of one-shot pruning.




\section{Conclusions and Future Work}
\section{Conclusion Remarks}
This work proposes a RBG graph model for disease spreading via hubs. We study the joint effect of the agent density, hub density, and connection function. The existence of a critical hub density depends only on the boundedness of the support of the connection function, which relates to curbing the traveling distance of individuals. When it comes to dispersion, both the degree distribution and the percolation threshold suggest that increasing dispersion helps spread the disease. The percolation properties of RBG graphs relate to unipartite graphs with modified connection functions. 
An interesting question in this direction is if and when the properties of the RBG graphs can be well represented by unipartite graphs with some modified connection functions. Our conjecture is that for independent connections between different pairs of agents, such representation is unlikely due to the oblivion of the local dependence (present in the RBG models). 
 Another direction is to consider hybrid models where agents may get infected either through common hubs or direct interactions between agents. The former infection mechanism is more centralized than the latter. 

%%
%% The acknowledgments section is defined using the "acks" environment
%% (and NOT an unnumbered section). This ensures the proper
%% identification of the section in the article metadata, and the
%% consistent spelling of the heading.
\begin{acks}
We thank members of the Social Physics and Complexity (SPAC) group at LIP for comments and critical reading of the manuscript. We thank HoneyComb for support in identifying meaningful topics for this study and BrightData for generously providing proxy services, free of charge. This research was partially funded by ERC Stg FARE (853566) and ERC PoC FARE\_Audit (101100653), both to JGS, and by FCT PhD fellowship (2022.12547.BD) to ID.
\end{acks}
\balance
%%
%% The next two lines define the bibliography style to be used, and
%% the bibliography file.
\bibliographystyle{ACM-Reference-Format}
\bibliography{refs}

\clearpage
%%
%% If your work has an appendix, this is the place to put it.

\appendix
\supplementarysection
\section{Supplementary Description of the Data collected}

\subsection{Data Collection Timeline}
In Figure \ref{timeline}, we show the moments when data collection began for each Type of webcrawlers used in the experiment.

\begin{figure}[ht]
    \centering \includegraphics[width=1\columnwidth]{Figures/timeline.pdf}
    \caption{Temporal alignment of deployment of Type 1, Type 2 and Type 3 bots, and the temporal control.}
    \Description{}
    \label{timeline}
\end{figure}

\subsection{Queries used in the experiment}

Table \ref{tab:queries_long_table} shows all queries used in this experiment. Notice that Type 3 bots performed a smaller number (10) of the original number of queries (gray background). 

\begin{table}[h]
    \centering
    \caption{Queries used in the experimental setup. All specific and general queries shown here were used in deployment of Type 1 bots (Location only) and Type 2 (Location + Browser Language). Type 3 (Location + browser language + browsing history) only did queries in bold (10 per category).}
    \label{tab:queries_long_table}
    \begin{tabular}{|| p{3.7cm} | p{3.7cm} ||}
    \hline
    \textbf{Specific Queries} & \textbf{General Queries} \\
    \hline\hline
    \small military complex Al-Shifa hospital & \small How to tie a tie \\ 
    \hline
    \small Israel banned Olympics & \small Popular books 2024 \\
    \hline
    \small Tiktok antisemitism & \small \textbf{Home workout routines} \\
    \hline
    \small Israeli babies beheaded & \small best movies ever \\
    \hline
    \small \textbf{Hamas rapes} & \small How to grow indoor plants \\
    \hline
    \small strike Al-Ahli hospital & \small Uncommon hobbies to try \\
    \hline
    \small \textbf{Gaza tunnels} & \small Rare and endangered plant species \\
    \hline
    \small Palestine banned Olympics & \small \textbf{Experimental music genres}\\
    \hline
    \small Saint Porphyrios Orthodox Church Gaza & \small organic gardening tips \\
    \hline
    \small \textbf{Hamas} & \small how to be smarter \\
    \hline
    \small Supernova festival attack &  \small\textbf{ Financial planning tips} \\
    \hline
    \small Israel destroyed an orthodox church in Gaza & 
    \small \textbf{Benefits of meditation for mental health} \\
    \hline
    \small \textbf{Erdogan threatened to intervene and support Palestinians?} & \small Nutritional value of quinoa \\
    \hline
    \small Ukraine provided weapons to Hamas & \small How to improve sleep quality naturally? \\
    \hline
    \small Yemen has declared war against Israel & \small Interesting facts about dolphins \\
    \hline
    \small \textbf{Gaza} & \small \textbf{Guide to composting at home} \\
    \hline
    \small Palestinian nurse claims Hamas steals food and medicine from al-Shifa Hospital & \small Popular science fiction books 2024 \\
    \hline
    \small \textbf{Israel} & \small How to set up a home office? \\
    \hline
    \small \textbf{American troops have landed in Israel to help Netanyahu's war efforts} & \small \textbf{Effects of climate change on wildlife} \\
    \hline
    \small Orthodox church Gaza destroyed by Israel & \small Tips for growing herbs indoors \\
    \hline
    \small \textbf{Houthis} & \small underrated movies you must see \\
    \hline
    \small \textbf{Erdogan threatens to support Palestinians} & \small \textbf{Current trends in sustainable fashion} \\
    \hline
    \small Ukraine provides weapons to Hamas &  \small \textbf{How do electric cars work?}\\
    \hline
    \small Yemen declares war on Israel & \small \textbf{Healthy lunch ideas for work} \\
    \hline
    \small Palestinian nurse alleges Hamas theft from al-Shifa Hospital & \small How to invest in stocks for beginners? \\
    \hline
    \small \textbf{American troops land in Israel to aid Netanyahu's war} & \small \textbf{DIY home decor on a budget} \\
    \hline
    \small Middle East conflict & \small What is virtual reality and how does it work? \\
    \hline
    \end{tabular}
\end{table}


\subsection{Data Description}
In the deployment of each bot's type the first page of results was collected for each combination of search engine, bot, and query. However, not every audit (i.e., the collection of results for a query) was successful. Failures were often due to IP issues or problems with the xpaths of certain search engine results. In the analysis, only successful audits were considered. A successful audit was defined as the collection of at least four URLs from at least three different IPs within the same location for the same search engine and query. Since the failure of some audits introduced imbalances in the search engine results database, we ensured that each analysis used the same number of specific and general queries per search engine, as well as the same number of ports per location.  

Table \ref{tab:number_queries_successful}  presents the number of successful queries in each category for every search engine and location pair considered in the analysis. It is important to note that while the number of successful queries per search engine may vary for the same experimental step, we ensured that the number of specific and general queries was consistent within each search engine analysis.

\begin{table}[hb]
    \centering
    \caption{Number of successful queries within each query category per search engine.}
    \label{tab:number_queries_successful}
    \begin{tabular}{|p{2cm}|p{2cm}|p{2.5cm}|}
    \hline
    \textbf{} & \textbf{Search Engine} & \textbf{Number Queries}\\
    \hline
    \multirow{4}{*}{Type 1} & DuckDuckGo & 26 \\
    \cline{2-3}
    {} & Google & 27 \\
    \cline{2-3}
    {} & Yahoo & 27 \\
    \hline
    \multirow{4}{*}{Type 2} & DuckDuckGo & 27 \\
    \cline{2-3}
    {} & Google & 19 \\
    \cline{2-3}
    {} & Yahoo & 27 \\
    \hline
    \multirow{4}{*}{Type 3} & DuckDuckGo & 10 \\
    \cline{2-3}
    {} & Google & 10 \\
    \cline{2-3}
    {} & Yahoo & 10 \\
    \hline
    \end{tabular}
\end{table}

Figure \ref{number_successful_ports} shows the number of successful IPs for each location and search engine. 10 ports of Type 1 and Type 2 were deployed initially. However, of Type 3, only 8 different IPs were used per location. Of these 8 IPs, 3 were associated with visits to news articles about the studied conflict (labeled as "C" in Figure \ref{number_successful_ports}, Type 3), 3 had a profile of regular news browsing history (labeled as "G"), and 2 were stateless bots ("S").

Table \ref{tab:number_results} contains a small description of the total number of URLs collected in each step of the experiment. 

\newpage
Some results were present to all bots Types. In the Venn diagram of Figure \ref{venn_unique_urls} we show the intersection of raw URLs across the types of bots.  In the Veen diagram of Figure \ref{venn_unique_domains} the intersection of websites domains across bot types. 


\begin{figure}[hb]
\includegraphics[width=1\columnwidth]{Figures/n_ports.pdf}
    \caption{Number of successful ports per location across search engines and bot type. The gray bars represent the number of IPs successful in performing general queries, while the colored bars indicate those successful in specific queries. For Type 3 bots, each bar is further divided into the number of IPs trained with visits to conflict news websites (C), general news websites (G), and stateless bots (S).}
    \Description{}
    \label{number_successful_ports}
\end{figure}



\begin{table}[t]
    \centering
    \caption{Description of results (URLs) data base considered for analysis. N results corresponds to the number of total results, N unique results to the number of different URLs collected for all queries and N unique domains to the total number of different websites domains.}
    \label{tab:number_results}
    \begin{tabular}{|p{0.8cm}|p{1cm}|p{1.5cm}|p{1.5cm}|p{1.5cm}|}
    \hline
    \textbf{} & \textbf{Search \newline Engine} & \textbf{Query \newline category} & \textbf{N unique results} & \textbf {N unique domains}\\
    \hline
    \multirow{4}{*}{Type 1} & \multirow{2}{*}{DuckGo} & General & 547 & 335\\
    \cline{3-5}
    {} & {} & Specific & 597 & 104 \\
    \cline{2-5}
    {} & \multirow{2}{*}{Google} & General & 328 & 224\\
    \cline{3-5}
    {} & {} & Specific & 268 & 98\\
    \cline{2-5}
    {} & \multirow{2}{*}{Yahoo} & General & 244 & 168 \\
    \cline{3-5}
    {} & {} & Specific & 245 & 57\\
    \hline
    \multirow{4}{*}{Type 2} & \multirow{2}{*}{DuckGo} & General & 621 & 380\\
    \cline{3-5}
    {} & {} & Specific & 746 & 125 \\
    \cline{2-5}
    {} & \multirow{2}{*}{Google} & General & 304 & 216\\
    \cline{3-5}
    {} & {} & Specific & 338 & 119\\
    \cline{2-5}
    {} & \multirow{2}{*}{Yahoo} & General & 207 & 154 \\
    \cline{3-5}
    {} & {} & Specific & 208 & 51\\
    \hline
    \multirow{4}{*}{Type 3} & \multirow{2}{*}{DuckGo} & General & 226 & 158\\
    \cline{3-5}
    {} & {} & Specific & 379 & 66 \\
    \cline{2-5}
    {} & \multirow{2}{*}{Google} & General & 216 & 164\\
    \cline{3-5}
    {} & {} & Specific & 339 & 111\\
    \cline{2-5}
    {} & \multirow{2}{*}{Yahoo} & General & 108 & 83 \\
    \cline{3-5}
    {} & {} & Specific & 152 & 41\\
    \hline
    \end{tabular}
\end{table}

\begin{figure}[H]
    \centering
    \includegraphics[width=0.8\columnwidth]{Figures/venn_unique_urls.png}
    \caption{Venn diagram of unique URLs present in Type 1 - Location only (green), Type 2 - Location + browser languages (yellow) and Type 3 - Location + browser languages + browsing history (blue) and its intersections.}
    \Description{}
    \label{venn_unique_urls}
\end{figure}

\begin{figure}[H]
    \centering
    \includegraphics[width=0.8\columnwidth]{Figures/venn_unique_domains.png}
    \caption{Venn diagram of unique domains present in Type 1 - Location only (green),  Type 2 - Location + browser languages (yellow) and Type 3 - Location + browser languages + browsing history (blue) and its intersections.}
    \Description{}
    \label{venn_unique_domains}
\end{figure}

\newpage
\section{Additional Results and Analysis}

\subsection{Statistical Results}
The statistical analysis values present in \ref{fig:main_plot} are described in the the following tables \ref{tab:p_values_adjusted_step_1}, \ref{tab:p_values_adjusted_step_2}, \ref{tab:p_values_adjusted_categories_step_3}, \ref{tab:p_values_comparison_different_steps} in more detail. 

\begin{table}[ht]
    \centering
    \caption{Pairwise comparisons of RBO per query type and location for Type 1 bots across search engines, showing the respective p-values and Bonferroni-adjusted p-values. The Mann-Whitney U test was used for the comparisons, and significant adjusted p-values ($<0.05$) are highlighted in bold. The adjustment accounts for 12 comparisons across the three search engines.}
    \Description{}
    \label{tab:p_values_adjusted_step_1}
    \begin{tabular}{|p{1cm}|p{3cm}|p{1cm}|p{1.5cm}|}
    \hline
    \textbf{Search\newline Engine} & \textbf{Groups \newline compared} & \textbf{p-value} & \textbf{Adjusted \newline p-value} \\
    \hline\hline
    \multirow{4}{*}{DuckGo} & Same Location vs. Diff Location - General Queries & \textbf{$<<0.001$} & \textbf{$<<0.001$} \\
    \cline{2-4}
    & Same location vs. Diff location - Specific Queries & \textbf{0.002} & \textbf{0.02} \\
    \cline{2-4}
    & General vs. Specific - Same Location & \textbf{0.05} & 0.56\\
    \cline{2-4}
    & General vs. Specific - Diff Location & 0.88 & 10.52\\
    \hline \hline
    \multirow{4}{*}{Google} & Same Location vs. Diff Location - General Queries & 0.33 & 3.98 \\
    \cline{2-4}
    & Same location vs. Diff location - Specific Queries & 0.14 & 1.69 \\
    \cline{2-4}
    & General vs. Specific - Same Location & 0.21 & 2.48 \\
    \cline{2-4}
    & General vs. Specific - Diff Location & 0.25 & 2.96\\
    \hline \hline
    \multirow{4}{*}{Yahoo} & Same Location vs. Diff Location - General Queries & 0.07 & 0.86 \\
    \cline{2-4}
    & Same location vs. Diff location - Specific Queries & 0.02 & 0.29 \\
    \cline{2-4}
    & General vs. Specific - Same Location & 0.20 & 2.40\\
    \cline{2-4}
    & General vs. Specific - Diff Location & 0.11 & 1.41\\
    \hline
    \end{tabular}
\end{table}
\vspace{1cm}
\begin{table}[hb]
    \centering
    \caption{Pairwise comparisons of RBO per query type and location for Type 2 bots across search engines, showing the respective p-values and Bonferroni-adjusted p-values. The Mann-Whitney U test was used for the comparisons, and significant adjusted p-values ($<0.05$) are highlighted in bold. The adjustment accounts for 12 comparisons across the three search engines.}
    \Description{}
    \label{tab:p_values_adjusted_step_2}
    \begin{tabular}{|p{1cm}|p{3cm}|p{1cm}|p{1.5cm}|}
    \hline
    \textbf{Search\newline Engine} & \textbf{Groups \newline compared} & \textbf{p-value} & \textbf{Adjusted \newline p-value} \\
    \hline\hline
    \multirow{4}{*}{DuckGo} & Same Location vs. Diff Location - General Queries & \textbf{$<<0.001$} & \textbf{$<<0.001$} \\
    \cline{2-4}
    & Same location vs. Diff location - Specific Queries & \textbf{<<0.001} & \textbf{<<0.001}\\
    \cline{2-4}
    & General vs. Specific - Same Location & \textbf{<<0.001} & \textbf{<<0.001}\\
    \cline{2-4}
    & General vs. Specific - Diff Location & \textbf{0.04} & 0.47\\
    \hline \hline
    \multirow{4}{*}{Google} & Same Location vs. Diff Location - General Queries & \textbf{$<<0.001$} & \textbf{$<<0.001$} \\
    \cline{2-4}
    & Same location vs. Diff location - Specific Queries & \textbf{<<0.001} & \textbf{<<0.001} \\
    \cline{2-4}
    & General vs. Specific - Same Location & 0.10 & 1.30 \\
    \cline{2-4}
    & General vs. Specific - Diff Location & 0.10 & 1.15\\
    \hline \hline
    \multirow{4}{*}{Yahoo} & Same Location vs. Diff Location - General Queries & 0.27 & 3.30 \\
    \cline{2-4}
    & Same location vs. Diff location - Specific Queries & \textbf{0.02} & 0.25 \\
    \cline{2-4}
    & General vs. Specific - Same Location & 0.60 & 7.17\\
    \cline{2-4}
    & General vs. Specific - Diff Location & 0.12 & 1.48\\
    \hline
    \end{tabular}
\end{table}
\clearpage
\begin{table*}[t]
    \centering
    \caption{Pairwise comparisons of RBO per query type and location for Type 3 bots across search engines, showing the respective p-values and Bonferroni-adjusted p-values. The Mann-Whitney U test was used for the comparisons, and significant adjusted p-values ($<0.05$) are highlighted in bold. The adjustment accounts for 12 comparisons across the three search engines.}
    \Description{}
    \label{tab:p_values_adjusted_step_3}
    \begin{tabular}{|p{2cm}|p{7cm}|p{1.5cm}|p{2.5cm}|}
    \hline
    \textbf{Search Engine} & \textbf{Groups compared} & \textbf{p-value} & \textbf{Adjusted p-value} \\
    \hline\hline
    \multirow{4}{*}{DuckGo} & Same Location vs. Diff Location - General Queries & \textbf{0.03} & 0.37 \\
    \cline{2-4}
    & Same location vs. Diff location - Specific Queries & \textbf{0.001} & \textbf{0.01}\\
    \cline{2-4}
    & General vs. Specific - Same Location & 0.34 & 4.14\\
    \cline{2-4}
    & General vs. Specific - Diff Location & \textbf{0.01} & 0.14 \\
    \hline \hline
    \multirow{4}{*}{Google} & Same Location vs. Diff Location - General Queries & \textbf{$<<0.001$} & \textbf{0.009} \\
    \cline{2-4}
    & Same location vs. Diff location - Specific Queries & \textbf{$<<0.001$} & \textbf{<<0.001} \\
    \cline{2-4}
    & General vs. Specific - Same Location & \textbf{0.02} & 0.33 \\
    \cline{2-4}
    & General vs. Specific - Diff Location & \textbf{0.02} & 0.25\\
    \hline \hline
    \multirow{4}{*}{Yahoo} & Same Location vs. Diff Location - General Queries & 0.57 & 6.85 \\
    \cline{2-4}
    & Same location vs. Diff location - Specific Queries & 0.52 & 6.24 \\
    \cline{2-4}
    & General vs. Specific - Same Location & 0.14 & 1.69\\
    \cline{2-4}
    & General vs. Specific - Diff Location & 0.14 & 1.69\\
    \hline
    \end{tabular}
\end{table*}
\clearpage
\begin{table*}[b]
    \centering
    \caption{Pairwise Comparison of results from different bots, across search engines and query classifications with p-values and respective Adjusted p-values. Adjusted p-values are calculated using the Bonferroni method and when significant ($<0.05$) are highlighted in bold. The adjustment accounts for 36 comparisons counting with number of tests (6), per search engine and query groups.}
    \label{tab:p_values_comparison_different_steps}
    \begin{tabular}{|p{3cm}|p{2cm}|p{5cm}|p{1.5cm}|p{3cm}|}
    \hline
    \textbf{Search Engine} & \textbf{Query Type} & \textbf{Groups compared} & \textbf{p-value} & \textbf{Adjusted p-value} \\
    \hline
    \multirow{6}{*}{DuckDuckGo} & \multirow{6}{*}{General} & Type 1 vs. Type 2 - Same Location & 0.97 & 34.91 \\
    & & Type 1 vs. Type 3 - Same Location & 0.27 & 9.83 \\
    & & Type 2 vs. Type 3 - Same Location & 0.38 & 13.85 \\
    & & Type 1 vs. Type 2 - Diff Location & 0.57 & 20.55 \\
    & & Type 1 vs. Type 3 - Diff Location & 0.73 & 26.41 \\
    & & Type 2 vs. Type 3 - Diff Location & 0.91 & 32.75 \\
    \hline
    \multirow{6}{*}{Google} & \multirow{6}{*}{General} & Type 1 vs. Type 2 - Same Location & 0.34 & 12.41 \\
    & & Type 1 vs. Type 3 - Same Location & 0.79 & 28.49 \\
    & & Type 2 vs. Type 3 - Same Location & 0.14 & 5.06 \\
    & & Type 1 vs. Type 2 - Diff Location & \textbf{0.02} & 0.76 \\
    & & Type 1 vs. Type 3 - Diff Location & \textbf{0.0036} & 0.13 \\
    & & Type 2 vs. Type 3 - Diff Location & 0.85 & 30.60 \\
    \hline
    \multirow{6}{*}{Yahoo} & \multirow{6}{*}{General} & Type 1 vs. Type 2 - Same Location & 0.52 & 18.74 \\
    & & Type 1 vs. Type 3 - Same Location & \textbf{0.0013} & \textbf{0.05} \\
    & & Type 2 vs. Type 3 - Same Location & \textbf{0.0036} & 0.13  \\
    & & Type 1 vs. Type 2 - Diff Location & 0.68 & 24.39 \\
    & & Type 1 vs. Type 3 - Diff Location & \textbf{0.02} & 0.62 \\
    & & Type 2 vs. Type 3 - Diff Location & \textbf{0.03} & 1.12 \\
    \hline \hline
    \multirow{6}{*}{DuckDuckGo} & \multirow{6}{*}{Specific} & Type 1 vs. Type 2 - Same Location & 0.34 & 12.41 \\
    & & Type 1 vs. Type 3 - Same Location & 0.43 & 15.38 \\
    & & Type 2 vs. Type 3 - Same Location & 0.85 & 30.60 \\
    & & Type 1 vs. Type 2 - Diff Location & \textbf{0.0073} & 0.26 \\
    & & Type 1 vs. Type 3 - Diff Location & \textbf{0.0091} & 0.33 \\
    & & Type 2 vs. Type 3 - Diff Location & 0.21 & 7.64 \\
    \hline
    \multirow{6}{*}{Google} & \multirow{6}{*}{Specific} & Type 1 vs. Type 2 - Same Location & 0.38 & 13.85 \\
    & & Type 1 vs. Type 3 - Same Location & \textbf{0.01} & 0.41 \\
    & & Type 2 vs. Type 3 - Same Location & 0.14 & 5.06 \\
    & & Type 1 vs. Type 2 - Diff Location & \textbf{0.0002} & \textbf{0.01} \\
    & & Type 1 vs. Type 3 - Diff Location & \textbf{0.0002} & \textbf{0.01} \\
    & & Type 2 vs. Type 3 - Diff Location & 0.91 & 32.75 \\
    \hline
    \multirow{6}{*}{Yahoo} & \multirow{6}{*}{Specific} & Type 1 vs. Type 2 - Same Location & 0.82 & 29.54 \\
    & & Type 1 vs. Type 3 - Same Location & \textbf{0.0036} & 0.13 \\
    & & Type 2 vs. Type 3 - Same Location & \textbf{0.0028} & 0.10 \\
    & & Type 1 vs. Type 2 - Diff Location & 0.79 & 28.49 \\
    & & Type 1 vs. Type 3 - Diff Location & \textbf{0.0091} & 0.33 \\
    & & Type 2 vs. Type 3 - Diff Location & \textbf{0.0017} & 0.06 \\
    \hline
    \end{tabular}
\end{table*}

\clearpage
\begin{table}[hb]
\caption{ANOVA statistical results of RBO values for different browsing histories comparisons with the stateless profiles, across search engines and query groups.}
\centering
\begin{tabular}{p{1cm} p{1cm} p{2cm} p{1.3cm} p{1.2cm}}
\hline
Search\newline Engine &  Query\newline Group & Brow.profiles\newline comparisons & Mean \newline D=1-RBO & ANOVA \newline Result \\
\hline
\multirow{6}{*}{Duckgo} 
& \multirow{3}{*}{\small{general}} & \small{conflict v. stat.} & 0.22 & F=1.792\\
& & general v. stat. & 0.288047 & p=0.170 \\
& & stateless v. stat. & 0.221987 &\\
\cline{2-5} 
& \multirow{3}{*}{\small{specific}} & \small{conflict v. stat.} & 0.30 & F=2.33 \\
& & \small{general v. stat.} & 0.37 & p=0.10\\
& & \small{stateless v. stat.} & 0.27&\\
\hline
\multirow{6}{*}{Google} 
& \multirow{3}{*}{\small{general}} & \small{conflict v. stat.} & 0.15  & F=3.25 \\
& & \small{general v. stat.} & 0.13& p=0.04\\
& & \small{stateless v. stat.} & 0.09 &\\
\cline{2-5}
& \multirow{3}{*}{\small{specific}} & conflict v. stat. & 0.244388 & F=8.078 \\
& & \small{general v. stat.} & 0.22 & p =0.0\\
& & \small{stateless v. stat.} & 0.12 &\\
\hline
\multirow{6}{*}{Yahoo} 
& \multirow{3}{*}{\small{general}} & \small{conflict v. stat.} & 0.29& F =0.04 \\
& & \small{general v. stat.} & 0.29 & p =0.96\\
& & \small{stateless v. stat.} & 0.28 &\\
\cline{2-5}
& \multirow{3}{*}{\small{specific}} & \small{conflict v. stat.} & 0.32 & F =0.09 \\
& & \small{general v. stat.} & 0.31 & p =0.91\\
& & \small{stateless v. stat.} & 0.33 &\\
\hline
\end{tabular}

\label{table:stat_results}
\end{table}
\clearpage

\subsection{Additional metrics of results differences}

The $D = 1 - RBO$ measurement used in the paper calculates a value between $[0, 1]$ according to the similarity of two lists and the rank at which two elements of the list match. To complement this metric we also calculated: (1) the average number of URLs that were not present in both lists of results of the two bots being compared (top row in Figures \ref{fig:other_metrics_step_1},\ref{fig:other_metrics_step_2}, \ref{fig:other_metrics_step_3}); (2) average number of URLs that were present at both top 3 results for both lists of results being compared (middle row in Figures \ref{fig:other_metrics_step_1},\ref{fig:other_metrics_step_2}, \ref{fig:other_metrics_step_3}) and (3) the average edit distance (bottom row in Figures \ref{fig:other_metrics_step_1},\ref{fig:other_metrics_step_2}, \ref{fig:other_metrics_step_3}). 

\begin{figure}[ht]
    \centering
    \includegraphics[width=1\columnwidth]{Figures/step_1_n_diff_results.png}
    \caption{Results for Type 1 bots. (Top Row) Average number of URLs that were not present in both lists of results in the 10 results. (Middle row) Average number of URLs that were not present in the top 3 results for both lists of results being compared. (Bottom Row) Edit distance between lists.  Error bars represent 95\% confidence intervals based on the bootstrapped distribution.}
    \Description{}
    \label{fig:other_metrics_step_1}
\end{figure}

\begin{figure}[hb]
    \centering
    \includegraphics[width=1\columnwidth]{Figures/step_2_n_diff_results.png}
    \caption{Results for Type 2 bots. (Top Row) Average number of URLs that were not present in both lists of results in the 10 results. (Middle row) Average number of URLs that were not present in the top 3 results for both lists of results being compared. (Bottom Row) Edit distance between lists.  Error bars represent 95\% confidence intervals based on the bootstrapped distribution.}
    \Description{}
    \label{fig:other_metrics_step_2}
\end{figure}

\newpage
\begin{figure}[h]
    \centering
    \includegraphics[width=1\columnwidth]{Figures/step_3_n_diff_results.png}
    \caption{Results for Type 3 bots. (Top Row) Average number of URLs that were not present in both lists of results in the 10 results. (Middle row) Average number of URLs that were not present in the top 3 results for both lists of results being compared. (Bottom Row) Edit distance between lists.  Error bars represent 95\% confidence intervals based on the bootstrapped distribution.}
    \Description{}
    \label{fig:other_metrics_step_3}
\end{figure}


\subsection{Controlling for query size}

As shown in Table \ref{tab:queries_long_table} the  categories of queries have very different sizes. Therefore, we tested whether the higher values of $D = 1 - RBO$ for queries of the specific group was a consequence of a higher number of words (bigger dimension). Figure \ref{results_by_query_size} shows the value of $D = 1 - RBO$ as function of the number of words in the query for both queries of the general category (black circle) and specific category (gray triangle) across search engines. As the plot shows there is a tendency for the value of $D = 1 - RBO$ to be higher for specific queries than general ones even when comparing queries of the exact same size from both categories. 

\begin{figure}[h]
    \centering
    \includegraphics[width=1\columnwidth]{Figures/plot_rbo_by_n_words_search_engine_step_1.png}
    \caption{Average value of $D = 1 - RBO$ (Type 1 bots) by query size in terms of number of words. In black plotted with circles we have the results for queries of the general category and in grey plotted with triangles the results of queries in the specific category.}
    \Description{}
    \label{results_by_query_size}
\end{figure}

Additionally, when we compare the value of $D = 1- RBO$ per location (different and same location) for an equal number of queries of both categories with comparable sizes (within 3 and 9 words) we observe (Figure \ref{rbo_queries_same_size} that the results continue to present the same pattern as in Figure \ref{fig:main_plot}. That is, higher values of $D = 1- RBO$ for specific queries than general queries and also higher values when comparing results of bots from different locations than the same location. 

\begin{figure}[h]
    \centering
    \includegraphics[width=1\columnwidth]{Figures/plot_rbo_filtered_words_step_1.png}
    \caption{Average value of $D = 1 - RBO$ for bots in the same location (grey) and in different locations (green) considering exclusively queries of the same length (between 3 and 8 words).}
    \Description{}
    \label{rbo_queries_same_size}
\end{figure}



\subsection{D = 1 - RBO values for the different experimental steps (for the 10 common queries across bot types)}


\begin{figure}[h]
    \centering
    \includegraphics[width=1\columnwidth]{Figures/comparing_rbo_1_2_3_controled_queries_3_legend.png}
    \caption{$D = 1 - RBO$ Results: Comparison of the experimental outcomes across search engines for the same location (left) and different locations (right). The top row illustrates the results for specific queries, the middle row shows results for general queries, and the bottom row highlights the difference between specific and general queries.}
    \label{fig:comparison_steps}
    \Description{}
\end{figure}


\subsection{Differences across categories of websites}

To control for the possibility that bots were retrieving the same content but in localized versions, leading to differences in URLs, we used ChatGPT-4o to classify the websites by category and type. The prompt was the one depicted in Figure \ref{prompt_domain_cat}.

\begin{figure}[hb]
    \includegraphics[width=1\columnwidth]{Figures/prompt_domain_categories.png}
    \caption{Prompt given to ChatGPT for the domains classification.}
    \Description{}
    \label{prompt_domain_cat}
\end{figure}


All categories per query type are in Table \ref{tab:categories_category}. Figure \ref{categories_websites_general}, shows the most common categories across search engines and location, for different Type of bots for general queries. Figure \ref{categories_websites_specific} refers to specific queries. 

\begin{table}[h]
    \centering
    \caption{Classifications of ChatGPT for category of results websites associated with General and Specific queries.}
    \label{tab:categories_category}
    \begin{tabular}{|p{3cm}|p{4cm}|}
    \hline
    \textbf{Query category} & \textbf{Websites classifications}\\
    \hline
    General & Reference, Entertainment, Education, Technology, News, Lifestyle, Business, Finance, Health, Government, Non-Profit, Social Media, Travel, E-Commerce, Art, Science, Fashion, Legal, Career, Retail, Automotive, Food \\
    \hline
    Specific & Reference, Education, Government, News, Fact-Checking, Social Media, Non-Profit, Entertainment, Finance, Religion, E-Commerce, Technology, Sports, Travel, Science \\
    \hline
    \end{tabular}
\end{table}

\begin{figure}[ht]
    \centering
    \includegraphics[width=1\columnwidth]{Figures/categories_websites_specific.pdf}
    \caption{Proportion of results per top 5 category of website for the specific group of queries, per search engine, country and bot type.}
    \label{categories_websites_specific}
    \Description{}
\end{figure}


\begin{figure}[hb]
    \centering
    \includegraphics[width=1\columnwidth]{Figures/categories_websites_general.pdf}
    \caption{Proportion of results per top 5 category of website for the general group of queries, per search engine, country and bot type.}
    \label{categories_websites_general}
    \Description{}
\end{figure}


Figure \ref{categories_websites_specific} and \ref{categories_websites_general}reveal noticeable differences in the distribution of categories between the different query groups. 

% \begin{table}[h!]
%     \centering
%     \caption{Average number of websites categories per experimental step and search engine for both categories of queries (General and Specific).}
%     \label{tab:number_categories}
%     \begin{tabular}{|p{0.8cm}|p{2cm}|p{1.5cm}|p{1.5cm}|}
%     \hline
%     \textbf{} & \textbf{Search Engine} & \textbf{General Queries} & \textbf{Specific Queries}\\
%     \hline
%     \multirow{4}{*}{Step 1} & Bing & 2.90 & 2.19 \\
%     \cline{2-4}
%     {} & DuckDuckGo & 3.09 & 2.19 \\
%     \cline{2-4}
%     {} & Google & 4.01 & 3.39 \\
%     \cline{2-4}
%     {} & Yahoo & 2.54 & 1.96  \\
%     \hline
%     \multirow{4}{*}{Step 2} & Bing & 2.90 & 2.49 \\
%     \cline{2-4}
%     {} & DuckDuckGo & 2.92 & 2.15 \\
%     \cline{2-4}
%     {} & Google & 3.97 & 3.42\\
%     \cline{2-4}
%     {} & Yahoo & 2.59 & 1.86 \\
%     \hline
%     \multirow{4}{*}{Step 3} & Bing & 2.67 &  2.29 \\
%     \cline{2-4}
%     {} & DuckDuckGo & 3.15 & 1.77 \\
%     \cline{2-4}
%     {} & Google & 4.16 & 3.20\\
%     \cline{2-4}
%     {} & Yahoo & 2.75 & 1.68\\
%     \hline
%     \end{tabular}
% \end{table}

After classifying all domains present in all experimental steps (Table \ref{tab:categories_category}) we studied how the results varied as function of the location and query type for all bots types. Additionally to the metric $D = 1 - RBO$ we also calculated (1) the average number of categories that were not present in both lists of results of the two bots being compared (top row in Figures \ref{fig:other_categories_step_1},\ref{fig:other_categories_step_2}, \ref{fig:other_categories_step_3}); (2) average number of URLs that were present at both top 3 results for both lists of results being compared (middle row in Figures \ref{fig:other_categories_step_1},\ref{fig:other_categories_step_2}, \ref{fig:other_categories_step_3})) and (3) the average edit distance (bottom row in Figures \ref{fig:other_categories_step_1},\ref{fig:other_categories_step_2}, \ref{fig:other_categories_step_3})). 

Additionally, we show in \ref{tab:p_values_adjusted_categories_step_1}, \ref{tab:p_values_adjusted_categories_step_2} and \ref{tab:p_values_adjusted_categories_step_3} more information about the statistical values plotted in \ref{fig:rbo_per_category_website}. 
\newpage

\begin{figure}[ht]
    \centering
    \includegraphics[width=1\columnwidth]{Figures/other_measurements_category_websites_step_1.png}
    \caption{Results for Type 1 bots. (Top Row) Average number of categories that were not present in both lists of results in the 10 results. (Middle row) Average number of categories that were not present in the top 3 results for both lists of results being compared. (Bottom Row) Edit distance between lists of categories.  Error bars represent 95\% confidence intervals based on the bootstrapped distribution.}
    \Description{}
    \label{fig:other_categories_step_1}
\end{figure}

\begin{figure}[hb]
    \centering
    \includegraphics[width=1\columnwidth]{Figures/other_measurements_category_websites_step_2.png}
    \caption{Results for Type 2 bots. (Top Row) Average number of categories that were not present in both lists of results in the 10 results. (Middle row) Average number of categories that were not present in the top 3 results for both lists of results being compared. (Bottom Row) Edit distance between lists of categories.  Error bars represent 95\% confidence intervals based on the bootstrapped distribution.}
    \Description{}
    \label{fig:other_categories_step_2}
\end{figure}

\begin{figure}[hb]
    \centering
    \includegraphics[width=1\columnwidth]{Figures/other_measurements_category_websites_step_3.png}
    \caption{Results for Type 3 bots. (Top Row) Average number of categories that were not present in both lists of results in the 10 results. (Middle row) Average number of categories that were not present in the top 3 results for both lists of results being compared. (Bottom Row) Edit distance between lists of categories.  Error bars represent 95\% confidence intervals based on the bootstrapped distribution.}
    \Description{}
    \label{fig:other_categories_step_3}
\end{figure}

\begin{table}[hb]
    \centering
    \caption{Pairwise comparisons of RBO values for categories of websites for Type 1 bots and each search engine, showing p-values and Bonferroni-adjusted p-values. The Mann-Whitney U test was used for the comparisons, and significant adjusted p-values ($<0.05$) are highlighted in bold. The adjustment accounts for 12 comparisons across the three search engines.}
    \Description{}
    \label{tab:p_values_adjusted_categories_step_1}
    \begin{tabular}{|p{1cm}|p{3cm}|p{1cm}|p{1.5cm}|}
    \hline
    \textbf{Search\newline Engine} & \textbf{Groups \newline compared} & \textbf{p-value} & \textbf{Adjusted \newline p-value} \\
    \hline\hline
    \multirow{4}{*}{DuckGo} & Same Location vs. Diff Location - General Queries & \textbf{0.0014} & \textbf{0.0014} \\
    \cline{2-4}
    & Same Location vs. Diff Location - Specific Queries & 0.0659 & 0.7905 \\
    \cline{2-4}
    & General vs. Specific - Same Location & 0.1959 & 2.3512 \\
    \cline{2-4}
    & General vs. Specific - Diff Location & 0.2866 & 3.4392 \\
    \hline \hline
    \multirow{4}{*}{Google} & Same Location vs. Diff Location - General Queries & 0.5375 & 6.4501 \\
    \cline{2-4}
    & Same Location vs. Diff Location - Specific Queries & 0.3560 & 4.2714 \\
    \cline{2-4}
    & General vs. Specific - Same Location & 0.6808 & 8.1692 \\
    \cline{2-4}
    & General vs. Specific - Diff Location & 0.6895 & 8.2741 \\
    \hline \hline
    \multirow{4}{*}{Yahoo} & Same Location vs. Diff Location - General Queries & 0.8202 & 9.8419 \\
    \cline{2-4}
    & Same Location vs. Diff Location - Specific Queries & 0.2639 & 3.1672 \\
    \cline{2-4}
    & General vs. Specific - Same Location & 0.7807 & 9.3680 \\
    \cline{2-4}
    & General vs. Specific - Diff Location & 0.5738 & 6.8860 \\
    \hline
    \end{tabular}
\end{table}
\clearpage
\begin{table}[hb]
    \centering
    \caption{Pairwise comparisons of RBO values for categories of websites for Type 2 bots and each search engine, showing p-values and Bonferroni-adjusted p-values. The Mann-Whitney U test was used for the comparisons, and significant adjusted p-values ($<0.05$) are highlighted in bold. The adjustment accounts for 12 comparisons across the three search engines.}
    \Description{}
    \label{tab:p_values_adjusted_categories_step_2}
    \begin{tabular}{|p{1cm}|p{3cm}|p{1cm}|p{1.5cm}|}
    \hline
    \textbf{Search\newline Engine} & \textbf{Groups \newline compared} & \textbf{p-value} & \textbf{Adjusted \newline p-value} \\
    \hline\hline
    \multirow{4}{*}{DuckGo} & Same Location vs. Diff Location - General Queries & \textbf{0.00} & \textbf{0.03} \\
    \cline{2-4}
    & Same Location vs. Diff Location - Specific Queries & \textbf{0.01} & \textbf{0.15} \\
    \cline{2-4}
    & General vs. Specific - Same Location & 0.81 & 9.76 \\
    \cline{2-4}
    & General vs. Specific - Diff Location & 0.73 & 8.72 \\
    \hline \hline
    \multirow{4}{*}{Google} & Same Location vs. Diff Location - General Queries & \textbf{0.00} & \textbf{0.01} \\
    \cline{2-4}
    & Same Location vs. Diff Location - Specific Queries & \textbf{0.00} & \textbf{0.00} \\
    \cline{2-4}
    & General vs. Specific - Same Location & 0.41 & 4.87 \\
    \cline{2-4}
    & General vs. Specific - Diff Location & 0.75 & 9.01 \\
    \hline \hline
    \multirow{4}{*}{Yahoo} & Same Location vs. Diff Location - General Queries & 0.66 & 7.96 \\
    \cline{2-4}
    & Same Location vs. Diff Location - Specific Queries & \textbf{0.34} & 4.09 \\
    \cline{2-4}
    & General vs. Specific - Same Location & 0.83 & 10.01 \\
    \cline{2-4}
    & General vs. Specific - Diff Location & 0.72 & 8.61 \\
    \hline
    \end{tabular}
\end{table}

\begin{table}[hb]
    \centering
    \caption{Pairwise comparisons of RBO values for categories of websites for Type 3 bots and each search engine, showing p-values and Bonferroni-adjusted p-values. The Mann-Whitney U test was used for the comparisons, and significant adjusted p-values ($<0.05$) are highlighted in bold. The adjustment accounts for 12 comparisons across the three search engines.}
    \Description{}
    \label{tab:p_values_adjusted_categories_step_3}
    \begin{tabular}{|p{1cm}|p{3cm}|p{1cm}|p{1.5cm}|}
    \hline
    \textbf{Search\newline Engine} & \textbf{Groups \newline compared} & \textbf{p-value} & \textbf{Adjusted \newline p-value} \\
    \hline\hline
    \multirow{4}{*}{DuckGo} & Same Location vs. Diff Location - General Queries & 0.5054 & 6.0643 \\
    \cline{2-4}
    & Same Location vs. Diff Location - Specific Queries & \textbf{0.0412} & 0.4941 \\
    \cline{2-4}
    & General vs. Specific - Same Location & 0.90 & 10.76 \\
    \cline{2-4}
    & General vs. Specific - Diff Location & 0.41 & 4.90 \\
    \hline \hline
    \multirow{4}{*}{Google} & Same Location vs. Diff Location - General Queries & 0.09 & 1.12 \\
    \cline{2-4}
    & Same Location vs. Diff Location - Specific Queries & \textbf{0.0004} & \textbf{0.005} \\
    \cline{2-4}
    & General vs. Specific - Same Location & \textbf{0.03} & 0.37 \\
    \cline{2-4}
    & General vs. Specific - Diff Location & \textbf{0.05} & 0.67 \\
    \hline \hline
    \multirow{4}{*}{Yahoo} & Same Location vs. Diff Location - General Queries & 0.9296 & 11.16 \\
    \cline{2-4}
    & Same Location vs. Diff Location - Specific Queries & 0.88 & 10.56 \\
    \cline{2-4}
    & General vs. Specific - Same Location & 0.87 & 10.44 \\
    \cline{2-4}
    & General vs. Specific - Diff Location & 0.71 & 8.55 \\
    \hline
    \end{tabular}
\end{table}


\clearpage
\subsection{Leaning Analysis}
To ask ChatGPT-4o to classify the leaning of news articles, we gave ChatGPT the prompt shown in Figure \ref{prompt_leaning}.

\begin{figure}[h]
    \includegraphics[width=1\columnwidth]{Figures/prompt_leaning_classification.png}
    \caption{Prompt given to ChatGPT for classification of news leaning in the perspecitive of the Israel-Palestine conflict.}
    \Description{}
    \label{prompt_leaning}
\end{figure}

\subsection{Mturkers Methodology}

All Mturkers classifying text received the instructions present in Figure \ref{mturker_instructions}.

\begin{figure}[h]
    \includegraphics[width=1\columnwidth]{Figures/mturker_instructions.png}
    \caption{Instructions given to each Mturker selected to classify texts.}
    \Description{}
    \label{mturker_instructions}
\end{figure}

\begin{figure}[h]
    \includegraphics[width=1\columnwidth]{Figures/heat_mat_chatgpt_mtturkers_no_neutral.png}
    \caption{Heatmap displaying the proportion of bias alignments between annotators and MTurk workers, comparing different bias categories: pro-Israel, slightly pro-Israel, pro-Palestine, and slightly pro-Palestine. We excluded neutral for clarity but it is account for in the proportions. The intensity of the color indicates the proportion of agreement or alignment across the different categories.}
    \Description{}
    \label{heat_map_no_neutral}
\end{figure}

\begin{figure}[h]
    \includegraphics[width=1\columnwidth]{Figures/heat_mat_chatgpt_mtturkers.png}
    \caption{Heatmap displaying the proportion of bias alignments between annotators and MTurk workers, comparing different bias categories: pro-Israel, slightly pro-Israel, neutral, pro-Palestine, and slightly pro-Palestine. The intensity of the color indicates the proportion of agreement or alignment across the different categories.}
    \Description{}
    \label{heat_map}
\end{figure}

\subsection{Time control}

To measure the average variation of search engines results over time, we performed a control by deploying Type 2 webcrawlers in different moments in time (April, May and October). The goal was to understand the overtime variation of the page contents and consequent differences in results. Even if we observe some variation due to the small number of ports per location, smaller number of locations (US and SA) and smaller number of queries being compared (8 per category), we observe no specific trend, except for Yahoo where our differences were never really significant. 

\begin{figure}[h]
    \includegraphics[width=1\columnwidth]{Figures/time_controls_specific_different_location.png}
    \caption{Average values of D = 1- RBO for the same type of bot (Type 2) with different locations (US vs. SA) across different moments in time (Time 1- April, Time 2-May and Time 3- October), for specific queries. Only 8 ports per location were considered.}
    \Description{}
    \label{time_control_1}
\end{figure}

\begin{figure}[h]
    \includegraphics[width=1\columnwidth]{Figures/time_controls_specific_same_location.png}
    \caption{Average values of D = 1- RBO for the same type of bot (Type 2) with same locations (US vs. US or SA vs. SA) across different moments in time (Time 1- April, Time 2-May and Time 3- October), for specific queries. Only 8 ports per location were considered.}
    \Description{}
    \label{time_control_2}
\end{figure}

\begin{figure}[h]
    \includegraphics[width=1\columnwidth]{Figures/time_controls_general_different_location.png}
    \caption{Average values of D = 1- RBO for the same type of bot (Type 2) with different locations (US vs. SA) across different moments in time (Time 1- April, Time 2-May and Time 3- October), for general queries. Only 8 ports per location were considered.}
    \Description{}
    \label{time_control_3}
\end{figure}

\begin{figure}[h]
    \includegraphics[width=1\columnwidth]{Figures/time_controls_general_same_location.png}
    \caption{Average values of D = 1- RBO for the same type of bot (Type 2) with same locations (US vs. US or SA vs. SA) across different moments in time (Time 1- April, Time 2-May and Time 3- October), for general queries. Only 8 ports per location were considered.}
    \Description{}
    \label{time_control_4}
\end{figure}


% \begin{figure}[h!]
%     \centering
%     \includegraphics[width=0.8\columnwidth]{Figures/venn_unique_news_controled_step_3.png}
%     \caption{Venn diagram of unique domains present in Step 1 - Location only (green), Step 2 - Location + browser languages (yellow) and Step 3 - Location + browser languages + browsing history (blue) and its intersections.}
%     \Description{}
%     \label{venn_unique_news}
% \end{figure}


% \begin{table}[h!]
%     \centering
%     \caption{Number of unique news domains per search engine.}
%     \label{tab:unique_news}
%     \begin{tabular}{|p{0.8cm}|p{2cm}|p{2cm}|}
%     \hline
%     {} & \textbf{Search Engine} & \textbf{Num Unique News Websites} \\
%     \hline
%     \multirow{4}{*}{Step 1} & Bing &  44 \\
%     \cline{2-3}
%     {} & Duckduckgo & 39 \\
%     \cline{2-3}
%     {} & Google &  26 \\
%     \cline{2-3}
%     {} & Yahoo & 20 \\
%     \hline
%     \multirow{4}{*}{Step 2} & Bing & 49 \\
%     \cline{2-3}
%     {} & Duckduckgo & 47 \\
%     \cline{2-3}
%     {} & Google & 34 \\
%     \cline{2-3}
%     {} & Yahoo &  21 \\
%     \hline
%     \multirow{4}{*}{Step 3} &  Bing & 55 \\
%     \cline{2-3}
%     {} & Duckduckgo & 51 \\
%     \cline{2-3}
%     {} & Google &  54 \\
%     \cline{2-3}
%     {} & Yahoo & 32 \\
%     \hline
%     \end{tabular}
% \end{table}
\end{document}

