Search engines apply various criteria to determine the information presented online, such as keyword matching, site authority, context, location, and language \cite{google_how_search_works, moz_how_search_engines_operate}. These algorithms choices can help shape individual views on particular topics, and studies have shown that even small changes in rank biased towards a political candidate can shift the voting preferences of undecided voters by 20\% \cite{epstein2015search_engine_manipulation}. As the factors used to tailor results to a specific user, remain uncertain, despite their potentially large impacts \cite{eli-pariser}, it is fundamental to develop independent tools to explore these choices. In fact, it is often impossible to distinguish if
differences are due to personalization, here defined as \textit{search results curated particularly for a user in question, considering the user's tastes and specific profile} or to customization, which, may result from differences in less personal features, such as location or language.

This work focuses on how different features can impact the order, the content, and the leaning of search-engine results, especially when a user is looking for geographical and politically sensitive information, as is the case of the Israel-Palestine conflict. 
%Particularly, when small changes can have big impacts of users perception \cite{epstein2015search_engine_manipulation}. 

%(so that when searching for a restaurant called "Berlin" in California, we do not get directed to a restaurant in Germany).
%Therefore, thorough out the paper, we will refer to personalization only when search engine differences due to other factors than our bot direct profile
%In this section, we review search engines' policies regarding results personalization and customization, and some past efforts to audit their recommendation systems.

\subsection{Search Engine Customization}

%Previous research has examined various aspects of search engine personalization, such as location, user accounts, and browsing history \cite{personalization_web_search, inconsistent_search_results}, but there is ongoing debate about the extent of personalization in search engines \cite{krafft2019search_engine_manipulation, partisan_audience_bias, bhaim2017burst} and the factors that drive it \cite{political_personalization, auditing_personalization_and_composition}. While these studies offer valuable insights, they are often conducted in isolation (a single user behavioral feature being analyzed), focusing predominantly on Google \cite{inconsistent_search_results, what_did_you_see, partisan_audience_bias, i_vote_for, auditing_personalization_and_composition}, and typically examine either broad, consensual topics \cite{inconsistent_search_results, personalization_web_search} or exclusively political issues surrounding electoral events \cite{auditing_personalization_and_composition, political_personalization}.

In this section, we briefly review search engine policies regarding result personalization and customization, covering the search engines we study in this work. Table \ref{engines_policies} summarizes these policies.

\textbf{DuckDuckGo} is known for its privacy-concerned model, explicitly stating it does not log IP addresses, browsing history, or search history to determine results. However, while it does not use exact user locations, it provides geographical results through proxies~\cite{duckduckgo_privacy}.

\textbf{Google} is the most widely used search engine, and claims that results may differ from user to user based on ``time, context and personalization'' \cite{Google2023}. Time differences arise from data being updated at various speeds across data centers. Context includes aspects such as location, language, device type (screen, mobile or desktop), and related results, which may appear if a user engages with specific content on the results page. Personalized results - specifically tailored to an individual user profile — are claimed to be tied to a Google Account.

% By far the most used search engine worldwide and the first to introduce personalization into its algorithm, claims in its \textit{Personalization \& Google Search results page} [REF] that results may differ from user to user based on "time, context and personalization". Time differences arise from data being upload at different rates on different data centers, while context features include aspects such as location, language, device type (screen type, mobile or desktop) and related results, which may appear after a user engages with specific content on the results page. Personalized results — those specifically tailored to an individual user profile — take into account user activity, including search history, interactions with results or ads, and liked or disliked content, all of which are exclusively linked to and stored in a Google Account. 

% \textbf{Bing:} 
% Bing ranks search results using personal factors such as search history, location, language, and device type \cite{Microsoft2023}. However, unlike Google, Bing doesn't specify if these factors require a Microsoft account to influence results, leaving it unclear whether user activity depends on being logged in.
% In \textit{The Basics of Search, How Bing Ranks Search Results} [REF], Bing explains that it uses individual-specific information to rank results. This includes search history, location, language, device characteristics (such as operating system and browser), and whether the user is on a mobile device, tablet, or desktop. Notably, unlike Google, Bing does not explicitly state that these features are used exclusively in conjunction with a Microsoft account, leaving it unclear whether search history and user activity require being logged into a Microsoft account to influence results.


% DuckDuckGo is known for its privacy concerned model, explicitly stating, "We don’t save your IP address or any unique identifiers alongside your searches or visits to our websites. We also never log IP addresses or any unique identifiers to disk." According to DuckDuckGo's policy [REF], it never uses browsing history or past search queries to determine search results. However, while it does not use a user's precise location, it still provides localized search results by utilizing a proxy location instead of the exact location of the user conducting the search.

\textbf{Yahoo} offers a search engine, but its primary business focus lies on email and news services, with these being particularly popular \cite{similarweb}. It openly claims to use past browsing history and search queries across all its recommendation systems, including email and news \cite{Yahoo2024}. Additionally, Yahoo email users can receive personalized search content based on their activity whithin these cross-platform services. % Yahoo is an interesting case because, although it offers a search engine, its primary business focus lies in its email and news services, with these being particularly popular [REF]. According to Yahoo's \textit{Information Collection \& Use Practices page} [REF], information such as past browsing history and search queries is used across all of its recommendation systems. Notably, due to Yahoo's specific services, some users may also receive personalized results from other sources, such as Yahoo Mail. These private results, however, are only displayed when the user is logged in. 


\begin{table}[h]
\centering
\caption{Customization Features Disclosed by Search Engines' Terms of Service}
\begin{tabular}{>{\raggedright\arraybackslash}p{4cm}|>{\centering\arraybackslash}p{1.1cm}|>{\centering\arraybackslash}p{0.9cm}|>{\centering\arraybackslash}p{0.9cm}}
\textbf{Feature} & \textbf{DuckGo} & \textbf{Google} & \textbf{Yahoo} \\ \midrule
Location & \textasciitilde proxy & \ding{51} & \ding{51} \\ \midrule
Language & \ding{51} & \ding{51} & \ding{51} \\ \midrule
Device Type & \ding{109} & \ding{51} & \ding{51} \\ \midrule
Search History (\small{within platform}) & \ding{55} & \ding{108} & \ding{51} \\ \midrule
Browsing History & \ding{55} & \ding{55} & \ding{51} \\ \midrule
Previous Content Engagement & \ding{55} & \ding{108} & \ding{55} \\ \midrule
Private Data (\small{e.g., Email}) & \ding{55} & \ding{55} & \ding{108} \\ \bottomrule 
\end{tabular}
\caption*{\ding{51} - feature impacts search results\\
\ding{55} - feature does not impact search results\\
\ding{108} - feature impacts results when user is logged in\\
\ding{109} - policy is unclear on whether feature impacts results}
\label{engines_policies}
\end{table}


% \begin{table*}[ht]
% \centering
% \begin{tabular}{|>{\raggedright\arraybackslash}p{2cm}|>{\centering\arraybackslash}p{1.5cm}|>{\centering\arraybackslash}p{1.5cm}|>{\centering\arraybackslash}p{1.5cm}|>{\centering\arraybackslash}p{1.5cm}|>{\centering\arraybackslash}p{1.5cm}|>{\centering\arraybackslash}p{2.0cm}|>{\centering\arraybackslash}p{3cm}|}
% \hline
% \textbf{Search Engine} & \textbf{Location} & \textbf{Language} & \textbf{Device Type} & \textbf{Search History} & \textbf{Browsing History} & \textbf{Previous Content Engagement} & \textbf{Private Data (e.g., Email)} \\ \hline
% \textbf{Google} & \ding{51} & \ding{51} & \ding{51}  & \ding{108} (logged-in only)  & \ding{108} (logged-in only)  & \ding{108} (logged-in only)   & \ding{55}  \\ \hline
% \textbf{Bing}  & \ding{51} & \ding{51} & \ding{51}  & \ding{51} & \ding{109}  & \ding{109}  & \ding{55} \\ \hline
% \textbf{DuckDuckGo} & \ding{108} (proxy) & \ding{51}  & \ding{109}  & \ding{55} & \ding{55} & \ding{55} & \ding{55} \\ \hline
% \textbf{Yahoo}  & \ding{51}  & \ding{51} & \ding{51} & \ding{51} & \ding{51}  & \ding{55} & \ding{108} (logged-in only)  \\ \hline
% \end{tabular}
% \caption{Comparison of Personalization Features Across Search Engines}
% \label{engines_policies}
% \end{table*}
\subsection{Search Engine Audits}

% Personalized results can be removed -> This is why we should not being looking exclusively for this


% Search engines rank among the most utilized online services globally. With Google leading the market, followed by Bing, then Yahoo and DuckduckGo in 5th place [REF]. Since Google first announced the "Personalized Search" in 2004 and implemented it in 2005 \cite{google_personalization}, personalized search became a norm worldwide. In this section of the paper we start 

% For the interest in our study, in the following paragraphs we summarize the announced policies by the three search engines approached.

% Google 

% In 2004 Google first announced the "Personalized Search", subsequently integrating it into their search engine in 2005 \cite{google_personalization}. By 2009, personalized results became the norm for Google users worldwide. Following Google's lead, other major search engines, including Yahoo and Bing, have adopted personalized result features tailored to users' locations and browsing histories \cite{bing_personalization, yahoo_personalization}. In contrast, DuckDuckGo upholds a more private concerned model, known for asserting non-tracking of user actions \cite{duckduck_personalization}. In the following Table we summarize the different personalization policies announced by the three different search engines. 



% In this study we approach three different search engine algorithms and we ultimately aim to understand how extensively people get different results when doing the same search on these platforms, particularly, when this search is related to a sensitive and polarized topic. 

% However, recent research, despite its constraints and experimental stage, suggests indications to the contrary, that is, that possibly there is personalization in DuckDuckGo \cite{personalization_google_duckduckgo}. 
% In this study we 



% Furthermore, recent studies have shown that DuckDuckGo users are indeed tracked upon engaging with advertisements \cite{privacy_risks_search_engines}.



% In summary, despite the longstanding acknowledgment of search engines personalization, there remains uncertainty about the user features considered by these algorithms. In fact, research has presented different results on the extend and consequences of this personalization, as discussed in the following paragraphs. 

Given the proprietary nature of search engine algorithms, researchers have relied on external audits to evaluate the extent and possible negative consequences of search engine customization.
%Since Eli Pariser's publication of the book "The Filter Bubble: What the Internet is Hiding from You" \cite{eli-pariser} raised concerns about search engine personalization, the scientific community has approached the issue using various approaches and methodologies. 
In 2013, Hannak et al.~\cite{personalization_web_search} introduced a methodology based on comparing the results of a ``real'' person with those of a ``fresh'' profile for the same query. They showed that ``location'' and ``being logged-in'' were the most important factors to induce search result differences. This work only audited Google but was later expanded to include Bing, which, on average, showed even more personalized results~\cite{measuring_personalization_2}.

Due to its importance, most past studies focused on political customization in the context of electoral processes, and almost solely audit Google~\cite{political_personalization, auditing_personalization_and_composition, partisan_audience_bias, krafft2019search_engine_manipulation, i_vote_for}. These often examine the effects of different factors in isolation (typically either location or browsing history), and have produced mixed findings. While some earlier research found no significant personalization in search results based on users' political profiles \cite{krafft2019search_engine_manipulation, partisan_audience_bias, haim2017burst}, more recent studies suggest that a user's political leaning can influence the political bias of results, especially in top news listings~\cite{political_personalization, auditing_personalization_and_composition}. Moreover, location has been shown to have a stronger influence on local queries than on political ones~\cite{inconsistent_search_results}. Importantly, and unlike our study, this literature primarily compares URLs, and only to a smaller extent, it investigates differences in the content that these URLs direct to. Important exceptions are the work of Le et al.~\cite{political_personalization}, measuring the leaning of search engine results page domains, but not of individual URLs, of Robertson et al.\cite{auditing_personalization_and_composition}, that groups search results in categories, and Hu et al.~\cite{10.1145/3308558.3313654}, that scores partisan bias of results' snippets and the corresponding web pages.

In contrast, our work aims to explore not only the impact of individual features on a single search engine, but how the incremental combination of several significant factors — those highlighted by previous studies and publicly acknowledged by search engines (Table \ref{engines_policies}) — affects search outcomes across platforms. Even if these differences are not strictly due to personalization, if they create noticeable variations in ranking and content, they may shape users' perceptions, on a timely and polarizing topic.

\begin{figure*}[t]  \includegraphics[width=0.95\textwidth]{Figures/scheme_method.pdf}
  \caption{Left side: Methodological scheme. Three types of bots were deployed (1): Type 1 differ in location, Type 2 in location and language, Type 3 in location, language and browsing history, having collected cookies. These bots are directed to search engines and make general or conflict specific queries (2). The results are analysed according to URL and differences in content and leaning (3). Right side. Table summarizes the features of the three types of bots. Order of columns: first, location (IL - Israel, BR - Brazil, SA - Saudi Arabia, United States, New York City - US-NY); second, browser language (he - Hebrew, pt - Portuguese, ar - Arabic, en - English); third, browsing history (type of websites visited); fourth, number of bots deployed per type.}
  \Description{}
  \label{scheme_methodology}
\end{figure*}



% Their study aimed to test the impact of multiple factors on the level of personalization within Google search results. They began by comparing the results of real users with those of a simulated "user" with no browsing history (incognito window), revealing measurable differences. Subsequently, they evaluated the influence of different variables by deploying bots that altered only the specific feature under investigation. This phase encompassed tests on the effects of cookies, browser type, geographical location, attributes of Google accounts (such as gender and age), and browsing history associated with various demographic profiles. These tests spanned a wide array of query categories, including Tech, News, Lifestyle, Quirky, Humanities, and Science. While discrepancies emerged when comparing results from real users to those of a "fresh" online user account, significant personalization was primarily attributed to being logged in with a Google Account, particularly for political queries within the News category. One major limitation of this study was its exclusive focus on Google. Subsequently, the researchers expanded their investigation, finding similar conclusions for Bing and DuckDuckGo \cite{measuring_personalization_2}.

% However, since 2013 search engine recommendation algorithms most likely changed significantly. Additionally, it is well known that online profiling techniques have become increasingly sophisticated \cite{webbrowsing_fingerprinting, cookiless_monster, stop_tracking_me_bro, partisan_audience_bias, i_vote_for, }. In fact, recent research has disagreed from Aniko et al. conclusions, finding significant search engine personalization linked to users location \cite{location_location, inconsistent_search_results}. Particularly, location has been shown to have a greater impact on local queries (e.g. "KFC") than on political and controversial ones \cite{location_location}. This is unsurprising, as location should logically play a larger role in queries that explicitly seek localized and personalized results. However, as Eli Pariser raised in his book \cite{eli-pariser} the concerns about the Filter Bubble effect are particularly high when we talk about more sensitive topics personalization, such as politics. Indeed, research indicates that biased search rankings can sway the voting preferences of 20\% or more of undecided voters.  

% Addressing these particular concerns, the scientific community has been giving considerable attention to the topic of search engines political personalization \cite{search_engine_manipulation_effect, search_query_filter_bubble, what_did_you_see, burst_or_filter_bubble}. However, the results diverge with some works finding evidence of search engine personalization related to political matters [ref], particularly, when the focus relies on top stories [ref], and others claiming no significant personalization or Filter Bubble effect [ref]. It is important to underline that the majority of these studies draws its conclusions by using real web users data and performing comparisons of the results for the same query, with a particular set of these studies performing comparisons of the results that a real user results gets with the results of a "fresh" simulated account [ref], as in the pioneer work of Aniko et al. [ref]. We encounter some limitations of this type of approach. First, it becomes particularly difficult to address what user features are at the origin of the observed differences in search engines results. Second, particularly for the studies that compare the results of a real user with the results of an incognito window of their browser, we argue that with the sophistication of online tracking methods nowadays, such as fingerprinting techniques, it becomes hard to isolate the possibility that some algorithms may be aware that the HTTP request comes from the same browser and user, even excluding the effect of cookies. Lastly, even that these studies happen at the consent of the data donors, we do not know how possibly the searches they perform within the scope of the study may affect their own recommendation algorithms and perception of some matters. 

% With a more similar approach to the one proposed in this work, that is, based on the deployment of virtual agents, previous research has focused particularly on how the results are different among agents for the same political query [ref]. That is, commonly, these studies focused solely in identifying (1) the existence or not of search engines results differences as indicative of personalization (2) or potential search engine bias related to the quality, order or news ecosystem it presents for queries related to different candidates or associate with different political spectrum. However, few less works used the automated browser methodology to isolate the direct impact of user features (such as location and browsing history) on search engines political personalization. With the few exceptions, at least to our best knowledge, of the work of XXXX that analyzed the impact of political leaning solely on Google Top Stories, the old pioneer work of Anaka et al. and the exploratory work of XXX. 

% Therefore, we present in this work, what we consider to be a more holistic and ethical concerned methodology associated with a particular polarized and timely topic: 

% \begin{itemize}
%     \item Approaches different 
% \end{itemize}