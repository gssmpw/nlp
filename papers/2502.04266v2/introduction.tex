
Access to high-quality information is a central tenet of democratic regimes, allowing citizens to make informed decisions. In recent decades, the way people seek information has shifted from traditional to online media \cite{pew_news_2006, news_access, ReutersInstitute2024}, with search engines playing a crucial role as gate keepers. Google, for instance, processes around 8.5 billion daily queries \cite{google_queries, google_maximize_access}, highlighting its influence on global information consumption.

While this shift has greatly enhanced the speed and convenience of accessing information, it also raises concerns about how search results shape users' worldviews. Studies show that individuals searching for news place strong trust in search engine results \cite{misplaced_trust, google_trust, trust_survey_2012, trust_survey_2017, trust_survey_2020}, often seeing these results as the most relevant \cite{EuropeanCommission2016}. Additionally, search engines rely on customization algorithms, which tailor content results based on user preferences, location, and other factors \cite{duckduckgo_privacy, google_terms, Yahoo2024, inconsistent_search_results, personalization_web_search}. While this facilitates routine searches, such as finding a restaurant, it becomes problematic for sensitive topics, like political news, where customized results can introduce bias and reinforce ``filter bubbles'' \cite{eli-pariser, personalization_web_search}. Given the opaque nature of search engine algorithms, which tend to monetize clicks, it is difficult to determine how much of this tailoring affects controversial topics compared to less sensitive ones. If differences exist, greater transparency is needed to explain how and why search engines suggest user-specific results.

Prior research has examined driving factors of customization, such as location, browsing history, and user accounts~\cite{inconsistent_search_results, personalization_web_search, political_personalization}, but these studies are often conducted in isolation, analyzing a single feature at a time, and predominantly examining Google. Much of this research has also focused either on broad, consensual topics or on political events such as elections, leaving a gap in understanding how search engines handle deeply polarized and contentious issues.

In this paper, we aim to address this gap by presenting the first empirical study on the effects of search engine customization on global controversial topics. We focus on the contemporary Israel-Palestine conflict and analyse the search results offered by three popular search engines---DuckDuckGo, Google, and Yahoo. We chose to study this conflict not only because of the seriousness of this polarizing global affair but also because it involves stakeholders across different political, geographical, cultural, and linguistic boundaries. This diversity offers a particularly rich context to examine how factors like user location, language, and browser history influence search engine customization. We explore three main research questions: \textit{1)} Do search engine results show higher levels of customization for more polarizing queries compared to non-sensitive topics, and, if so, what factors drive this tailoring? \textit{2)} How different in type and content are the URLs provided by the search engines for conflict-related searches when compared to general, non-sensitive topics? \textit{3)} Do the suggested contents vary according to geographical location and, if so, do they carry particular conflict-related leanings?
%Do the personalization factors used by the search engines influence the leanings of the results concerning the conflict? 


Performing this study presented several non-trivial challenges. First, search engines operate mostly as black boxes, making it difficult to systematically examine how their algorithms function. Second, it is essential to mimick realistic user behavior across different regions, languages, and browsing habits without introducing bias. Third, it is neither trivial to distinguish between neutral and politically charged queries to draw meaningful conclusions, and define reliable metrics to measure not only the variations in search engine results. This paper addresses these challenges and makes the following key contributions:

\mypara{1. Crawling method with incremental profiling:} We developed an automated crawling approach to perform online searches using software bots. These bots were progressively made more ``human-like'' by expanding their profiles: \textit{Type 1} bots varied only by location, \textit{Type 2} bots included different browser languages, and \textit{Type 3} bots incorporated an initial online browsing period to collect cookies. We deployed these bots in various countries and directed them to the search engines to perform identical queries. These queries included both conflict-specific topics (e.g., ``Hamas'') and general, non-specific topics (e.g., ``how to tie a tie''). In this way, we maintained control over each feature's contribution to customization while avoiding the use of private or sensitive user data.

\mypara{2. Metrics for measuring customization:} After deploying the bots, we analyzed whether the results varied and whether these variations were reflected in differences in content or in conflict-related bias. To do this end, we adopted a metric system based on Rank-Biased Overlap (RBO)~\cite{rank_biased_precision_measurement}, which emphasizes top results, and employed additional methods to assess qualitative differences (details below). This framework enables us to study how individual features combine to influence search engine results and serves as a tool to evaluate accuracy and potential biases beyond the Israel-Palestine case study.

\mypara{3. Significant findings:} Our study reveals a higher level of search engine customization than previously reported, even for sensitive topics like electoral events \cite{haim2017burst, krafft2019search_engine_manipulation, partisan_audience_bias}. Results were particularly pronounced for conflict-related queries, and when comparing across locations. Moreover, and despite claims from some search engines regarding limited or no personalization based on user data, browsing history influences the results displayed. In particular, Google and DuckDuckGo, which claim to limit personalization based on browsing history, showed significant differences in the results delivered, while Yahoo, which openly acknowledges more personalization, showed the least.



\if0 %% NS: Comentado até \fi, i.e, ate ao final deste ficheiro

% Access to quality information is a central tenet of democratic regi\-mes, allowing citizens to make informed decisions. In recent decades, the way people seek information has shifted from traditional to online media \cite{pew_news_2006, news_access, ReutersInstitute2024}, with search engines becoming key tools: Google alone processes around 8.5 billion daily queries \cite{google_queries, google_maximize_access}. While this shift has greatly enhanced the speed and convenience of accessing information, making the vast amount of knowledge online readily available, it also raises important concerns. 

% First, individuals who look for news and information online, typically place strong trust in the results and curation systems of search engines \cite{misplaced_trust, google_trust, trust_survey_2012, trust_survey_2017, trust_survey_2020}, often believing these results to be the most relevant \cite{EuropeanCommission2016}. Second, search engines algorithms apply personalization techniques to tailor search results based on user preferences, location, and others \cite{duckduckgo_privacy, google_terms, Yahoo2024, inconsistent_search_results, personalization_web_search}. While this personalization can enhance user experience for routine tasks, like finding a restaurant, it can be troublesome when the searches are for sensitive or controversial topics, such as political news: tailored results can bias users in different directions and even contribute to the creation of "filter bubble" effects \cite{eli-pariser, personalization_web_search}. Third, search engines rely on complex, proprietary, and often opaque algorithms that monetize clicks, making it difficult to ensure the quality and accuracy of results \cite{seo_london_monetization}.

% %. Moreover, when the world is increasingly polarized \cite{affective_polarization}, raising concerns of 

% % which raising concerns and is possible that search engines contribute to “filter bubble” effects \cite{eli-pariser, personalization_web_search}, meaning that personalization may limit the exposure of users to different viewpoints, reinforcing their prior beliefs. 
% % Third, search engines mostly rely on complex, proprietary and often black-box algorithms, making them very difficult to audit.

% However, despite their crucial role in shaping users' information consumption and the high levels of trust placed in their results, search-engines' algorithms have been comparatively under-researched, especially when compared to social media and other more passive sources of information \cite{opposing_views_social_media, political_polarization_menczer, review_echo_chambers}: as they become increasingly sophisticated and personalized \cite{google_personalization}, it is crucial to develop tools to audit them and better understand the extension and possible consequences of their recommendation systems, particularly around highly controversial and sensitive topics.

% %Previous research has examined various aspects of search engine personalization, such as location, user accounts, and browsing history \cite{personalization_web_search, inconsistent_search_results}, but there is ongoing debate about the extent of personalization in search engines \cite{krafft2019search_engine_manipulation, partisan_audience_bias, burst_or_filter_bubble} and the factors that drive it \cite{political_personalization, auditing_personalization_and_composition}. While these studies offer valuable insights, they are often conducted in isolation (a single user behavioral feature being analyzed), focusing predominantly on Google \cite{inconsistent_search_results, what_did_you_see, partisan_audience_bias, i_vote_for, auditing_personalization_and_composition}, and typically examine either broad, consensual topics \cite{inconsistent_search_results, personalization_web_search} or exclusively political issues surrounding electoral events \cite{auditing_personalization_and_composition, political_personalization}.



% Here, we present an unbiased, privacy-protective, and scalable audit system across three search engines: DuckDuckGo, Google, and Yahoo. This system is based on web crawlers (that we will refer to as bots) trained to mimic human browsing behavior, and perform online searches.

% We deployed these bots to explore search results related to the highly polarized and contemporary Israel-Palestine conflict. We aim to ask three main questions: 1) How (if at all) do search engines tailor their results to different bot profiles? 2) To what extent is this tailoring different for conflict-related searches versus for general, non-sensitive topics? and 3) Is this tailoring associated with particular contents or even positional leanings? 

%  Our approach was to progressively make the bots more "human-like" by expanding their profiles: Type 1 bots only had different locations, Type 2 bots also used different browser languages, and Type 3 incorporated an initial online browsing period, during which they collected cookies. These bots were then directed to the mentioned search engines and made the exact same queries. These queries could either be specific to the Israel-Palestine conflict (e.g., "Hamas") or more general and non-specific (e.g., "how to tie a tie"). We analyzed whether the retrieved results were different and, if so, if this variability also translated into in differences in content or conflict-related leaning. 
%  %Our results indicate strong levels of personalization for conflict-related searches based on location and browser language, with milder effects due to browsing history. Notably, as the profiles of our bots became more detailed and closer to real user personas, the results grew increasingly divergent and biased, particularly for conflict-related searches.

% This paper presents the first large-scale tool to study the impact(s) of different features (location at multiple countries level, browser settings, and browsing history) on three different search engines (DuckDuckGo, Google and Yahoo), using a privacy protecting framework. Our study provides important insights into the workings of search engines, % and possible consequences of their algoritmic choices, 
% underscoring the following main contributions:

%     \textbf{1. Methodology with incremental level of profiling:} previous research has examined the impact of individual features on search engine personalization \cite{inconsistent_search_results, personalization_web_search, political_personalization}. In this study, we propose a comprehensive approach, gradually increasing the sophistication of bot profiles to more closely resemble real users'. This allows us to study how individual features combine to affect search engine results. Additionally, by avoiding real user data, we maintain control over each feature's contribution to overall customization, while not requiring private or sensitive data.

%    \textbf{2. Topic Landscape:} Using the Israel-Palestine conflict as a case study, we examine the degree of search engine customization for a highly sensitive and timely topic. We developed a comprehensive metric to measure not only the variations in search engine results but also to evaluate their qualitative differences. Therefore, this study offers a tool to understand the accuracy and possible biases in search engine recommendation systems, when dealing with contentious issues. 
%    %Utilizing the Israel-Palestine conflict as a case study enables us to investigate the extent of search engine personalization concerning a topic of high sensitivity and current relevance. Moreover, this provides us with the opportunity to gain insights into the accuracy and biases present in search engine recommendation systems regarding such a contentious issue. Therefore, in this study, we have developed a robust metric to not only quantify the variations in search engine results but also to assess their qualitative differences. 

%     \textbf{3. Findings:} Our results reveal a higher level of search engine personalization than previously reported (even for sensitive topics like electoral events \cite{haim2017burst, krafft2019search_engine_manipulation, partisan_audience_bias}) and that this tailoring is particularly high for conflict-related queries. 


% To the best of our knowledge, this paper presents the first large-scale tool to study the impact of different features (location, browser settings and browsing history) on three different search engines (Google, DuckDuckGo, and Yahoo), using a privacy protecting framework. By applying it to an extremely relevant, timely, and sensitive topic, we found that enhancing the profiling of bot profiles not only leads to an increase of search engines personalization, it also leads to different and potentially biased results. 

% In summary, the main 
% %We argue that this broad and comparative analysis surrounding such a central topic is key to fully understand the extent and impacts of search engine personalization.


% The process followed different steps. First, and owing to the specific geographic constraints surrounding the Israel-Palestine conflict, we asked whether just changing the geographical locations of the bots (using VPN proxies) influences the search-engine results for identical queries. More importantly, we asked whether these differences in search engine results are more prevalent for queries related to conflict (e.g. "Hamas rapes") or queries outside of its scope (e,g, "best movies ever"). We chose four locations (Israel, Saudi Arabia, the United States, and Brazil) and the bots were programmed to simultaneously search the same query to one of the three search engines (Google, DuckDuckGo, and Yahoo) and  collect search engine results and top stories observed. Results showed a strong effect of location in search engine results. This adds to previous research finding a more moderate levels of personalization due to location in the context of more general and consensual topics \cite{location_location}.


% Second, we replicated step 1, but changing  both the location and browser language settings to lead search engines to retrieve local results. That is, bots in each of the locations visited the search engines using a browser set to the main language of the region in question (Hebrew, Arabic, English and Portuguese, respectively). The idea was to test whether enhancing the profiling of bot profiles to better mimic real user behavior would lead to an increase of search engines personalization.  In fact, our results show that to be the case. 

% Third, we wanted to analyse the impacts of browsing history. The rational is that browsing profiles (e.g. cookies placed by  websites) associated with different online consumption may change the results displayed by search engines. To test this, the bots were instructed to visit 20 websites of news articles associated with the conflict or news articles of general topics (e.g. health, beauty, sports). Our results show that this recent browsing history is...

% At last, we wanted to understand what exactly these differences in search engine results meant. Therefore, we developed a stringent quantitative analysis of the ecosystem of results that the bots with the different profiles found. ...

% In summary, this work is a strong step towards better understand the extension of search engines personalization and its impact on the “filter bubble” effect leaving the following main contributions: 

% \begin{itemize}
%     \item \textbf{Methodology with incremental level of profiling:} previous research has studied the impact of individual features on search engine personalization \cite{location_location, personalization_web_search}. In this study, we propose a holistic approach where step by step we increase the sophistication of bot profile. Our methodology gradually builds up on the bot profile, making each step more closely resemble that of a real web user. The underlying idea is to comprehensively understand how each individual feature contributes to the cumulative impact of all features on search engine results. We argue that this approach offers a more realistic perspective, as the personalization one may observe on search engines results from the sum of specific configurations rather than isolated features. Additionally, we opted out of using real users data, as this allows us greater control over assessing the contribution of each feature to the overall level of observed personalization. 

%    \item \textbf{Topic Landscape:} Utilizing the Israel-Palestine conflict as a case study enables us to investigate the extent of search engine personalization concerning a topic of high sensitivity and current relevance. Moreover, this provides us with the opportunity to gain insights into the accuracy and biases present in search engine recommendation systems regarding such a contentious issue. Therefore, in this study, we have developed a robust metric to not only quantify the variations in search engine results but also to assess their qualitative differences. 
    
%     \item \textbf{Findings:} Our results identify a higher level of search engines personalization than previous research [refs]. It is important to underline that some works have approached sensitive topics such as electoral moments [refs]. 
% \end{itemize}

\fi