This paper presents a novel methodology for auditing search engine results using web crawlers designed to mimic different ``human-like'' online profiles. By focusing on the Israel-Palestine conflict and examining how factors such as location, browser language, and browsing history affect search results across Google, DuckDuckGo, and Yahoo, we demonstrated that there are significant differences in search engine results, as these features vary.
%personalization plays a significant role in shaping the information landscape for sensitive or contentious topics.


Our findings confirm that location is a critical factor in influencing search engine results, as has been shown in previous studies \cite{inconsistent_search_results, personalization_web_search}. Importantly, tailoring search results to the user's location is typically considered ``neutral" or not worrisome, but that is clearly not the case for geo-political conflicts: that individuals across political borders are shown different content (possibly with even different political leaning) could have serious consequences, difficult to anticipate and measure.

We also found that combining location with other factors like browser language and browsing history increases the variability in results, particularly for conflict-related queries. Importantly, this effect is not limited to differences in URLs but extends to the types of content displayed, such as news or political content. Contrary to some earlier studies, our results revealed that browsing history does induce notable differences, particularly in the case of Google, and this may further amplify the effects of information bubbles or echo chambers. Moreover, we observed that search engines may not fully adhere to their stated personalization policies. Google and DuckDuckGo, despite their claims not to personalize results based on "naive users" browsing history, displayed significant differences in the results, whereas Yahoo showed the least variation, despite claiming the opposite policy. Future work could explore the nuances of how browsing history impacts customization, particularly by disentangling how and which cookies lead to particular results.

Currently, the way forward in the online tracking space is uncertain, with Google leading a shift away from third-party cookies and then seemingly turning around \cite{google_privacy_sandbox}. It is important that regardless of the technologies employed,  audits such as this remain possible. In the European Union, the Digital Services Act specifically states that very large online search engines must provide access to data and/or code deemed necessary to monitor and guarantee compliance on areas such as privacy protection and personalization. This regulation also includes a mechanism for researcher access and academic auditing around systemic threats. Using such a mechanism, it should be possible to extend this methodology or to request access to relevant information on algorithmic recommendations. Focusing on other controversial topics could help further investigate the role of search engines in shaping public discourse across key political or cultural issues. Additionally, the development of more sophisticated tools for measuring the political leanings of search results could provide deeper insights into the biases search engines may introduce, particularly in global conflicts.