Search engines are widely used in everyday life for a range of tasks, from simple queries to more complex research, from uncontroversial topics to contentious and consequential issues. This work contributes to the understanding of how the algorithms behind these search engines can amplify differences in the information landscape, particularly on politically sensitive and polarized topics, at a global scale. Our findings show that tailoring content based on central factors such as location, language settings, and browsing history (outside of the platform) can together lead to significant variations in search results. Differences that are innocuous when looking for a restaurant or local show, carry a different weight when the subject is of consequence. Users around the world are often exposed to different content, sometimes with varying perspectives on the same issue, such as the Israel-Palestine conflict.

Furthermore, our study reveals that search engines may not always behave according to their stated policies. Both Google and DuckDuckGo presented different results for simulated users with distinct browsing histories, despite their claims of minimizing personalization. Interestingly, Yahoo, which acknowledges more personalization in its algorithm, showed the least evidence of personalization among the three search engines tested.

Finally, we believe this work provides a valuable methodology for auditing the information landscape of search engines. It offers new insights into how we can study and evaluate the types of content these platforms deliver, particularly in the context of politically charged issues the world over.