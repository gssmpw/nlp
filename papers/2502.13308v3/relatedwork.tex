\section{Related Work}
\noindent \textbf{Graph Fraud Detection (GFD)} aims to identify fraudulent activities from graph-structured real-world systems, including financial fraud~\cite{motie2023financial}, spamming~\cite{deng2022markov}, and fake reviews~\cite{yu2022graph}. Various techniques have been utilized to detect the fraudsters, including attention~\cite{liu2020alleviating}, sampling~\cite{liu2021pick, liu2021pick} and multi-view learning~\cite{zhong2020financial}. 
Although these methods have achieved promising results, they suffer from heterophily-caused problem~\cite{dou2020enhancing}, i.e., fraudster nodes tend to build heterophilic connections with benign user nodes to make them indistinguishable from the majority.
Recent studies have developed many strategies to mitigate the negative impact of heterophily with node labels.
For example, GHRN~\cite{gao2023addressing} reduce heterophily by pruning the graph, while BWGNN~\cite{tang2022rethinking} leverage spectral GNNs to better capture high-frequency features associated with heterophily. GDN~\cite{gao2023alleviating}, GAGA~\cite{wang2023label}, and PMP~\cite{zhuo2024partitioning} overcome the feature-smoothing effect by crafting advanced encoders that sharpen feature separation, while ConsisGAD~\cite{chen2024consistency} directly utilizes annotated labels to create more effective graph augmentation method. 
Despite their promising results, the reliance on labels in existing methods restricts their applicability in unsupervised scenarios. Therefore, in this work, we aim to address the pressing need for developing unsupervised GFD methods.




\noindent \textbf{Graph Anomaly Detection (GAD)} is a broader concept than GFD, aiming to identify not only fraudsters but also any rare and unusual patterns that significantly deviate from the majority in graph data~\cite{ding2019deep, zheng2021heterogeneous,wang2024unifying,liu2024arc,cai2024lgfgad,cai2022plad,zhang2024deep,liu2024towards}. Therefore, GAD techniques can be directly applied to GFD, especially in unsupervised learning scenarios~\cite{li2024noise,liu2024self}. Given the broad scope of GAD and the difficulty in obtaining real-world anomalies, many unsupervised GAD methods have been designed and evaluated on several datasets with artificially injected anomalies~\cite{ding2019deep, liu2021anomaly, jin2021anemone, zheng2021generative, duan2023graph, duan2023arise, pan2023prem}.
Despite their decent performances, these methods rely on the strong homophily of the datasets with injected anomalies, which limits their applications under graph heterophily~\cite{zheng2022graph, zheng2023finding}. Recent studies explore this issue and suggest using estimated anomaly scores as pseudo-labels to mitigate the negative impacts of heterophily, e.g., dropping edges~\cite{he2024ada, qiao2024truncated} and adjusted message passing~\cite{chen2024boosting}. 
However, the estimated anomaly score might not be an effective indicator for modeling the heterophily under the homophily-based message-passing GNNs~\cite{zhu2022does}, leading to suboptimal GFD performance~\cite{GLOD}.
Hence, we introduce a novel heterophily measurement with desired properties as effective guidance for a powerful unsupervised GFD model.

More detailed related works are available in Appendix A.