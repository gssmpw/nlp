%%%%%%%% ICML 2025 EXAMPLE LATEX SUBMISSION FILE %%%%%%%%%%%%%%%%%

\documentclass{article}

% Recommended, but optional, packages for figures and better typesetting:
\usepackage{microtype}
\usepackage{bbm}
\usepackage{float}
\usepackage{enumitem}
\usepackage{algcompatible}
\usepackage{graphicx}
\usepackage{soul}
\usepackage{subfigure}
\usepackage{booktabs} % for professional tables

% hyperref makes hyperlinks in the resulting PDF.
% If your build breaks (sometimes temporarily if a hyperlink spans a page)
% please comment out the following usepackage line and replace
% \usepackage{icml2025} with \usepackage[nohyperref]{icml2025} above.
\usepackage{hyperref}


% Attempt to make hyperref and algorithmic work together better:
\newcommand{\theHalgorithm}{\arabic{algorithm}}
\newcommand{\amit}[1]{\textcolor{blue}{\textbf{Amit: #1}}}

% Use the following line for the initial blind version submitted for review:
%\usepackage{misc/icml2025}

% If accepted, instead use the following line for the camera-ready submission:
\usepackage[accepted]{misc/icml2025}

% For theorems and such
\usepackage{amsmath}
\usepackage{amssymb}
\usepackage{mathtools}
\usepackage{amsthm}

\usepackage[framemethod=TikZ]{mdframed}
\mdfsetup{
    middlelinecolor =   none,
    middlelinewidth =   1pt,
    backgroundcolor =   blue!5,
    roundcorner     =   5pt,
}

% if you use cleveref..
\usepackage[capitalize,noabbrev]{cleveref}

%%%%%%%%%%%%%%%%%%%%%%%%%%%%%%%%
% THEOREMS
%%%%%%%%%%%%%%%%%%%%%%%%%%%%%%%%
\theoremstyle{plain}
\newtheorem{theorem}{Theorem}[section]
\newtheorem{proposition}[theorem]{Proposition}
\newtheorem{lemma}[theorem]{Lemma}
\newtheorem{corollary}[theorem]{Corollary}
\theoremstyle{definition}
\newtheorem{definition}[theorem]{Definition}
\newtheorem{assumption}[theorem]{Assumption}
\theoremstyle{remark}
\newtheorem{remark}[theorem]{Remark}

% Todonotes is useful during development; simply uncomment the next line
%    and comment out the line below the next line to turn off comments
%\usepackage[disable,textsize=tiny]{todonotes}
\usepackage[textsize=tiny]{todonotes}

\newcommand{\dripta}[1]{\textcolor{blue}{\textbf{DR: #1}}}
\newcommand{\algo}{\texttt{PrefVLM}}

% The \icmltitle you define below is probably too long as a header.
% Therefore, a short form for the running title is supplied here:
\icmltitlerunning{Minimizing Human Feedback in Reinforcement Learning using Vision-Language Models}

\begin{document}

\twocolumn[
\icmltitle{Preference VLM: Leveraging VLMs for Scalable Preference-Based Reinforcement Learning}

% It is OKAY to include author information, even for blind
% submissions: the style file will automatically remove it for you
% unless you've provided the [accepted] option to the icml2025
% package.

% List of affiliations: The first argument should be a (short)
% identifier you will use later to specify author affiliations
% Academic affiliations should list Department, University, City, Region, Country
% Industry affiliations should list Company, City, Region, Country

% You can specify symbols, otherwise they are numbered in order.
% Ideally, you should not use this facility. Affiliations will be numbered
% in order of appearance and this is the preferred way.
\icmlsetsymbol{equal}{*}

\begin{icmlauthorlist}
\icmlauthor{Udita Ghosh}{ucr}
\icmlauthor{Dripta S. Raychaudhuri}{aws}
\icmlauthor{Jiachen Li}{ucr}
\icmlauthor{Konstantinos Karydis}{ucr}
\icmlauthor{Amit Roy-Chowdhury}{ucr}
\end{icmlauthorlist}

\icmlaffiliation{ucr}{University of California, Riverside}
\icmlaffiliation{aws}{AWS AI Labs}

\icmlcorrespondingauthor{Udita Ghosh}{ughos002@ucr.edu}

% You may provide any keywords that you
% find helpful for describing your paper; these are used to populate
% the "keywords" metadata in the PDF but will not be shown in the document
\icmlkeywords{Vision-language, Reinforcement learning, Embodied AI}

\vskip 0.3in
]

% this must go after the closing bracket ] following \twocolumn[ ...

% This command actually creates the footnote in the first column
% listing the affiliations and the copyright notice.
% The command takes one argument, which is text to display at the start of the footnote.
% The \icmlEqualContribution command is standard text for equal contribution.
% Remove it (just {}) if you do not need this facility.

%\printAffiliationsAndNotice{}  % leave blank if no need to mention equal contribution
\printAffiliationsAndNotice{\icmlEqualContribution} % otherwise use the standard text.

\begin{abstract}

% Recent works to jointly reconstruct 3D human and object from a single RGB image, are mostly model-based, that fail to capture the fine details of the clothed human body and object surface. In this paper, we introduce ReCHOR, a novel, model-free, first-method to produce realistic clothed human-object reconstructions from a monocular view. This is extremely challenging due to human-object occlusions, diverse interactions and depth ambiguity, as it needs to infer both 3D spatial awareness and high resolution details. Our core idea is based on estimating neural implicit representations for human and object respectively by an attention-based neural implicit model that attends to pixel-aligned features from both the global human-object image for spatial awareness and  the local separate view of human and object images for high quality details. Additionally, the network is conditioned on semantic features from an initial estimated human-object pose prior and a generative diffusion model that inpaints occluded regions, thus enabling the retrieval of details from them.
% We also propose a synthetic dataset with rendered scenes of diverse, inter-occluded 3D human and object scans, to train our network. We evaluate our method on the synthetic and real world BEHAVE dataset. Our experiments show that our method outperforms the SOTA in achieving realistic clothed human-object reconstructions.
Recent approaches to jointly reconstruct 3D humans and objects from a single RGB image represent 3D shapes with template-based or coarse models, which fail to capture details of loose clothing on human bodies. In this paper, we introduce a novel implicit approach for jointly reconstructing realistic 3D clothed humans and objects from a monocular view. For the first time, we model both the human and the object with an implicit representation, allowing to capture more realistic details such as clothing. This task is extremely challenging due to human-object occlusions and the lack of 3D information in 2D images, often leading to poor detail reconstruction and depth ambiguity. To address these problems, we propose a novel attention-based neural implicit model that leverages image pixel alignment from both the input human-object image for a global understanding of the human-object scene and from local separate views of the human and object images to improve realism with, for example, clothing details. Additionally, the network is conditioned on semantic features derived from an estimated human-object pose prior, which provides 3D spatial information about the shared space of humans and objects. To handle human occlusion caused by objects, we use a generative diffusion model that inpaints the occluded regions, recovering otherwise lost details. For training and evaluation, we introduce a synthetic dataset featuring rendered scenes of inter-occluded 3D human scans and diverse objects. Extensive evaluation on both synthetic and real-world datasets demonstrates the superior quality of the proposed human-object reconstructions over competitive methods.
\end{abstract}
\section{Introduction}\label{sec:intro}

In computational finance, Monte Carlo simulations are used extensively to estimate the expected value of financial payoffs based on the solution of stochastic differential equations (SDEs) which model the evolution of stock prices, interest rates, exchange rates and other quantities \cite{glasserman04}.  Monte Carlo methods are very general and flexible, but for high accuracy it requires generating a large number of costly SDE path approximations, which has motivated research into a number of variance reduction or, equivalently, cost reduction techniques. One such method is
Multilevel Monte Carlo (MLMC), which was proposed in \cite{GILES2008} and was adapted for various applications that are summarised in \cite{Giles_overview17} and successfully combined with other methods such as quasi-Monte Carlo methods. The main idea of MLMC is to approximate the payoff using different time stepping resolutions when numerically solving the underlying SDE and to generate an optimal number of samples on each level, such that the overall computational cost is minimised subject to the desired bound on the variance. %, such that the total computational cost is minimised. 
The computational savings come from the fact that most samples are computed on the coarser levels and hence are less expensive while only a few samples from the finest levels are required \cite{GILES2008}.


Among the directions in which the computational cost 
of MLMC methods could further be reduced, an important avenue is the use of lower precision calculations, especially for the first Monte Carlo levels where the targeted accuracy is relatively low. 
 An overview of the research on mixed precision for the standard Monte Carlo (MC) framework is provided in \cite{ChowMixedPrecisionStandardMC} but only a few references study the potential of low precision computation in the MLMC framework \cite{Rounding_error_oliver}. To the best of our knowledge, the only MLMC framework with customised precision in the literature is \cite{brugger2014mixed}, but they use a uniform precision for all operations on each Monte Carlo level instead of optimising 
 the precision of each intermediary variable to reduce as much as possible the cost of path generation.
 
An important motivation for an MLMC framework with variable precision would be performing the low precision computations on reconfigurable hardware devices such as Field Programmable Gate Arrays (FPGAs). FPGAs contain customizable logic blocks and connectors that make it easy to adapt the digital circuit architecture for a specific application, leading to a highly parallel and optimised implementation. Therefore they are successfully exploited in applications that require high speed and have high computational workload, such as signal processing \cite{woods2008fpga}, and real time applications like high frequency trading \cite{HFT1,HFT2}. That is why a number of previous works in hardware architecture design implemented the MLMC algorithm to price financial options using FPGAs as accelerators, which resulted in improved speed and power efficiency compared to full CPU architectures \cite{Schryver2013AMM}. The paper \cite{lindsey2016domain} also proposed 
a Domain Specific Language to automate the configuration of FPGAs for this specific application. However, only \cite{brugger2014mixed} proposed a heuristic to reduce the precision in calculations.

In addition, all aforementioned works considered that the random number generation (RNG) is performed in single or double precision. Yet in most cases an important portion of the workload in the overall MLMC simulation comes from the RNG and in \cite{brugger2014mixed} this limited the total computational savings.
To reduce the cost of MLMC simulations in particular those based on the Geometric Brownian Motion (GBM), \cite{approximateICDF_Oliver, NestedOliver} have proposed to use approximate random numbers that are generated by applying an approximation of the inverse CDF to uniform random numbers. In \cite{NestedOliver}, the authors proposed a way to integrate these lower precision random variables into a \textit{nested} MLMC framework and completed a numerical analysis to bound the resulting error at each MC level by a product of the time step and the error in the random number approximation. The same authors show in \cite{approximateICDF_Oliver} that using approximate random variables reduces the cost of path generation by a factor 7.


In this paper we propose a nested MLMC framework that combines the use of approximate random normal variables and lower precision calculations to reduce the computational cost of MLMC even further than \cite{brugger2014mixed,NestedOliver}. We illustrate the efficiency of our framework in Matlab, after making several assumptions on the cost of operations and size of the errors that we carefully justify. We focus on the case of GBM and use the approximate RNG methods presented in \cite{approximateICDF_Oliver} as well as a new slightly modified method that combines CDF inversion and the central limit theorem. To choose the precision of the variables in the low precision path generation, we introduce a novel method to optimise the bit-widths. This optimisation is performed before the main path generation loop is executed and is based on a linear model of the payoff error  
due to rounding when computing in low precision. The error model relies on algorithmic differentiation in a similar manner to \cite{unifying-bwoptim,bitwidth-AD,ADAPT}. The bit-width optimisation procedure can be performed off-line, so this stage can be excluded from the on-line time complexity of our framework. The user specified desired accuracy is then enforced by calculating on-line the number of samples that need to be generated.

In terms of hardware design, we suggest implementing the low precision path generation on FPGAs and the full-precision ones on a CPU or GPU. 
The FPGA offers enough flexibility to define a separate bit-width for every variable in the low precision path generation, and can be reconfigured periodically to update the bit-widths when the market parameters have changed considerably. 


The paper is organized as follows : \Cref{sec:MLMC} introduces MLMC and nested MLMC to make clear the estimator that is implemented in our framework. Then in \Cref{sec:RNG} we detail the methods that could be used to obtain approximate random normally distributed numbers very cheaply for the low precision path generation. In \Cref{sec:error_model} and \Cref{sec:costModel} we propose an error model and a cost model (resp.) that we then use to formulate the optimisation problem that is solved to obtain the optimal bit-widths of fixed point variables in \Cref{sec:optimisation}. Finally we summarise our results and future directions in \Cref{sec:conclusion}.



\section{Related Work}

\paragraph{LLMs for Agent tasks.}

Our research is related to deploying large language models (LLMs) as agents for decision-making tasks in interactive environments~\citep{liu2023agentbench,zhou2023webarena,shridhar2020alfred,toyama2021androidenv}. Earlier works, such as~\citep{yao2023webshopscalablerealworldweb}, fine-tuned models like BERT~\citep{devlin2019bertpretrainingdeepbidirectional} for decision-making in simplified environments, such as online shopping or mobile phone manipulation. With the advent of large language models~\citep{brown2020languagemodelsfewshotlearners,openai2024gpt4technicalreport}, it became feasible to perform decision-making tasks through zero-shot or few-shot in-context learning. To better assess the capabilities of LLMs as agents, several models have been developed~\citep{deng2024mind2web,xiong2024watch,hong2023cogagent,yan2023gpt}. Most approaches~\citep{zheng2024seeact,deng2024mind2web} provide the agent with observation and action history, and the language model predicts the next action via in-context learning. Additionally, some methods~\citep{zhang2023building,li2023camel,song2024trial} attempt to distill trajectories from state-of-the-art language models to train more effective policy models. In contrast, our paper introduces a novel framework that automatically learns a reward model from LLM agent navigation, using it to guide the agents in making more effective plans.

\textbf{LLM Planning.} Our paper is also related to planning with large language models. Early researchers~\citep{brown2020languagemodelsfewshotlearners} often prompted large language models to directly perform agent tasks. Later, \citet{yao2022react} proposed ReAct, which combined LLMs for action prediction with chain-of-thought prompting~\citep{wei2022chain}. Several other works~\citep{yao2023treethoughtsdeliberateproblem,hao2023reasoning,zhao2023large,qiao2024agentplanningworldknowledge} have focused on enhancing multi-step reasoning capabilities by integrating LLMs with tree search methods. Our model differs from these previous studies in several significant ways. First, rather than solely focusing on text generation tasks, our pipeline addresses multi-step action planning tasks in interactive environments, where we must consider not only historical input but also multimodal feedback from the environment. Additionally, our pipeline involves automatic learning of the reward model from the environment without relying on human-annotated data, whereas previous works rely on prompting-based frameworks that require large commercial LLMs like GPT-4~\citep{openai2024gpt4technicalreport} to learn action prediction. Furthermore, \Model supports a variety of planning algorithms beyond tree search.

\textbf{Learning from AI Feedback.} In contrast to prior work on LLM planning, our approach also draws on recent advances in learning from AI feedback~\citep{bai2022constitutional,lee2023rlaif,yuan2024self,sharma2024critical,pan2024autonomous,koh2024tree}. These studies initially prompt state-of-the-art large language models to generate text responses that adhere to predefined principles and then potentially fine-tune the LLMs with reinforcement learning. Like previous studies, we also prompt large language models to generate synthetic data. However, unlike them, we focus not on fine-tuning a better generative model but on developing a classification model that evaluates how well action trajectories fulfill the intended instructions. This approach is simpler, requires no reliance on state-of-the-art LLMs, and is more efficient. We also demonstrate that our learned reward model can integrate with various LLMs and planning algorithms, consistently improving their performance.

\textbf{Inference-Time Scaling.} ~\citet{snell2024scaling} validates the efficacy of inference-time scaling for language models. Based on inference-time scaling, various methods have been proposed, such as random sampling~\citep{wang2022self} and tree-search methods~\citep{hao2023reasoning, zhang2024accessing, guan2025rstar}. Concurrently, several works have also leveraged inference-time scaling to improve the performance of agentic tasks. ~\citet{koh2024tree} adopts a training-free approach, employing MCTS to enhance policy model performance during inference and prompting the LLM to return the reward. ~\citet{gu2024your} introduces a novel speculative reasoning approach to bypass irreversible actions by leveraging LLMs or VLMs. It also employs tree search to improve performance and prompts an LLM to output rewards. ~\citet{yu2024exact} proposes Reflective-MCTS to perform tree search and fine-tune the GPT model, leading to improvements in ~\citet{koh2024visualwebarena}. ~\citet{putta2024agent} also utilizes MCTS to enhance performance on web-based tasks such as ~\citet{yao2023webshopscalablerealworldweb} and real-world booking environments. ~\cite{lin2025qlass} utilizes the stepwise reward to give effective intermediate guidance across different agentic tasks. Our work differs from previous efforts in two key aspects: (1) Broader Application Domain. Unlike prior studies that primarily focus on tasks from a single domain, our method demonstrates strong generalizability across web agents, mathematical reasoning, and scientific discovery domains, further proving its effectiveness. (2) Flexible and Effective Reward Modeling. Instead of simply prompting an LLM as a reward model, we finetune a small scale VLM~\citep{lin2023vila} to evaluate input trajectories. %Our reward scores range continuously between 0 and 1, in contrast to existing methods that rely on discrete scoring (e.g., 0 and 1, or 0, 0.5, and 1) through direct LLM prompting.

% Concurrently, several works have also leveraged inference-time scaling to improve the performance of agentic tasks. ~\citet{pan2024autonomous} demonstrates that LLMs and VLMs, such as the GPT series, can function as evaluators or reward models to provide guidance for fine-tuning or reflection, thereby enhancing digital agents. This lays the groundwork for subsequent studies that directly prompt LLMs as reward models. ~\citet{koh2024tree} adopts a training-free approach, employing MCTS to enhance policy model performance during inference. However, it is limited to web environments~\citep{koh2024visualwebarena}. Moreover, its value function relies on prompting an LLM, which is less effective than our proposed method. We validate our approach through ablation studies, demonstrating that our fine-tuned reward model is more effective. ~\citet{gu2024your} introduces a novel speculative reasoning approach to bypass irreversible actions, such as purchasing a product, by leveraging LLMs or VLMs. It also employs tree search to improve performance, but it remains restricted to the web domain~\citep{koh2024visualwebarena, deng2024mind2web}. Additionally, it lacks reward modeling and instead prompts an LLM to output rewards. ~\citet{yu2024exact} proposes Reflective-MCTS to perform tree search and fine-tune the GPT model, leading to improvements in ~\citep{koh2024visualwebarena}. However, this work focuses solely on a single web agent task, and its reward modeling is derived from multi-agent debate, differing from our more effective and efficient reward modeling approach. ~\citet{putta2024agent} also utilizes MCTS to enhance performance, but it is limited to web-based tasks such as ~\citep{yao2023webshopscalablerealworldweb} and real-world booking environments.


\section{Methodology}
\paragraph{Preliminaries.}
We primarily focus on the homologous model merging, in which $\boldsymbol{\theta}_i$ all come from the same base model $\boldsymbol{\theta}_{\rm{base}}$. Given $K$ tasks $\{T_1,T_2,\cdots,T_K\}$ and $K$ corresponding fine-tuned models with parameters $\{\boldsymbol{\theta}_1,\boldsymbol{\theta}_2,\cdots,\boldsymbol{\theta}_K\}$, model merging aims to combine $K$ fine-tuned models into one single model simultaneously performing on $\{T_1,T_2,\cdots,T_K\}$ without post-training~\cite{method_p1_1,method_p1_2}.
Task vector~\cite{ilharco2023editing,yang2024adamerging} is a key element in merging method which could enhances the base model‘s ability or enable the model to handle other tasks. Specifically, for task $T_i$, the task vector $\boldsymbol\tau_i\in \mathbb{R}^D$ is defined as the vector obtained by subtracting the SFT weights $\boldsymbol{\theta}_i$ from the base model weight
$\boldsymbol{\theta}_{\rm{base}}$, \emph{i.e.}, $\boldsymbol\tau_i=\boldsymbol{\theta}_i-\boldsymbol{\theta}_{\rm{base}}$. The merged model could be denoted as $\boldsymbol{\theta}_m=\boldsymbol{\theta}_{\rm{base}}+\sum_i \lambda_i\boldsymbol{\tau}_i$, which $\lambda_i$ is the scaling factor measuring the importance of task vector. For clarification, we also denote the neuron set in $\boldsymbol{\theta}_i$ as $\mathcal{N}_i$, the neuron set in $\boldsymbol{\tau}_i$ as $\mathcal{T}_i$.



\begin{algorithm}[!ht]
    \caption{LED-Merging}
    \label{alg1}
    \begin{algorithmic}[1]
        \REQUIRE  base model $\boldsymbol{\theta}_{\rm{base}}$, SFT models $\{\boldsymbol{\theta}_{i}\mid i\in [K]\}$, mask ratios \{$r_{i} \mid i\in [K]\}$, scaling factors $\{\lambda_i\mid i\in[K]\}$, location datasets $\{\mathcal{X}_{i}\mid i\in[K]\}$
        \ENSURE merged parameter $\boldsymbol{\theta}_{m}$
        \STATE $\mathcal{M}\leftarrow\phi$
        \STATE $\boldsymbol{\theta}_{m}\leftarrow \boldsymbol{\theta}_{\rm{base}}$
        \FOR{$i\in [K]$}
        \STATE $I(\boldsymbol{\theta}_i)=\mathbb{E}_{x\sim \mathcal{X}_i}|\boldsymbol{\theta}_{i}\odot \nabla_{\boldsymbol{\theta}_i}\mathcal{L}(x)|$
        \STATE $I(\boldsymbol{\theta}_{\rm{base}})=\mathbb{E}_{x\sim \mathcal{X}_i}|\boldsymbol{\theta}_{\rm{base}}\odot \nabla_{\boldsymbol{\theta}_{\rm{base}}}\mathcal{L}(x)|$
        
        \STATE calculate $\mathcal{T}^{r_i}_{i}$ following Equation \ref{vote}
        \STATE  $\mathcal{M}\leftarrow \mathcal{M}\cup\{\mathcal{T}^{r_i}_i\}$
       
        
   
        
        
        \ENDFOR  
        \FOR{$i\in [K]$}
        
        \STATE calculate $\text{Disjoint}(\mathcal{T}_i^{r_i})$ use Equation~\ref{disjoint_safety}
        \STATE $\boldsymbol{m}_i \leftarrow \boldsymbol{0}$
        \FOR{$d\in \mathcal{T}_i^{r_i}$}
        \STATE $\boldsymbol{m}_{i,d}=1$
        \ENDFOR
        \STATE $\boldsymbol{\theta}_{m}\leftarrow \boldsymbol{\theta}_{m}+\lambda_i \boldsymbol{\tau}_i\odot \boldsymbol{m}_{i}$
        \ENDFOR
    \end{algorithmic}
\end{algorithm}
    %\vspace{-5pt}
\begin{figure*}[h!]
    \centering
    \includegraphics[width=\linewidth]{figs/pipeline_v2.pdf}
    \vspace{-40mm}
    \caption{Overview of our two-stage training pipeline {\ours}.}
    \label{fig:pipeline}
\end{figure*}


\paragraph{LED-Merging: Location, Election, and Disjoint Merging}
To address the neuron misidentification and interference issues in existing model merging methods, we propose LED-Merging (Location, Election, and Disjoint Merging). Specifically, previous studies \cite{modelstock, ilharco2023editing, tiesmerging} fail to accurately identify safety-related neurons in task vectors with a single magnitude score, namely \textit{neuron misidentification}. Meanwhile, there exists an interference between safety-related and utility-related task vector neurons during the merging process, namely \textit{neuron interference}. To address neuron misidentification, we first locate important neurons both in the base and fine-tuned models and then elect neurons from the task vector considering these two scores together. Subsequently, to mitigate the interference, we introduce a disjoint step, isolating these important neurons so that they influence different base neurons. The whole process is illustrated in Figure~\ref{fig:method}. 




In the location and election step, we consider the importance score from base and fine-tuned models simultaneously to locate task-specific neurons. In this way, it is more accurate than relying on the magnitude score alone because task-specific neurons with high importance score in the fine-tuned model may not necessarily score high in the base model, and vice versa.

{\textbf{Location}}.  We first calculate importance scores for each neuron in a base/fine-tuned model. Given a location dataset $\mathcal{X}_i=\{(x,y)_k\}$, where $x$ is the question and $y$ is the answer, we calculate the importance scores for the weight $\boldsymbol{\theta}_i\in\mathbb{R}^D$ in any  layer as follows~\cite{snip,spareseGPT,sun2024a}:
\begin{equation}
    I(\boldsymbol{\theta}_i)=\mathbb{E}_{x\sim \mathcal{X}_i}[\boldsymbol{\theta}_i\odot \nabla _{\boldsymbol{\theta}_i}\mathcal{L}(x)],
    \label{location}
\end{equation}
which $\mathcal{L}(x)=-\log p(y\mid x)$ is the conditional negative log-likelihood loss. We choose the SNIP score~\cite{snip} because it balances computational efficiency and performance~\cite{cq}. Please refer to Sec.~\ref{sec:ablation} for the comparison between different location methods. After computing importance scores, we choose top-$r_i$ neurons as the important neuron subset $\mathcal{N}_{i}^{r_i}$ from $I(\boldsymbol{\theta}_i)$.
 
 % After computing locating scores, we select the neurons scoring both high in base and fine-tuned models as important neurons in task vectors. Then in the disjoint step,  with preventing  polysemantic neurons  from receiving gradient updates towards different directions,
 % we use set difference to isolate the safety   and utility-related neurons  and construct corresponding masks for merging process,

{\textbf{Election}}. A natural question is how to select important neurons in the task vector $\boldsymbol{\tau}_i$ based on $I(\boldsymbol{\theta}_{\rm{base}})$ and $I(\boldsymbol{\theta}_{i})$. The important neurons in the base model may be different from neurons in the fine-tuned model. Therefore, we introduce the following election strategy to select neurons with high scores in both base and fine-tuned models:
\begin{equation}
    \mathcal{T}_i^{r_i}=\mathcal{N}_i^{r_i}\cap \mathcal{N}_{\rm{base}}^{r_i}.
    \label{vote}
\end{equation}
\emph{Remark}. We compare different choosing methods, including scoring low or high in base or fine-tuned model in Section~\ref{sec:ablation} and find that Equation \ref{vote} achieves the best performance.





{\textbf{Disjoint}}. As important neurons from different task vectors may conflict with each other at the same position, we use the set difference to disjoint the neurons from others to prevent interference:
\begin{equation}
    \text{Disjoint}(\mathcal{T}^{r_i}_{i})=\mathcal{T}^{r_i}_{i}-\mathop{\cup}\limits_{{J}\subsetneqq [K],|J|\geq 2}\mathop{\cap}\limits_{j\in {J}}\mathcal{T}^{r_j}_{j}.
    \label{disjoint_safety}
\end{equation}

Next, we construct a mask $\boldsymbol{m}_i\in\mathbb{R}^D$ to implement disjoint in the merging process. Specifically, this mask $\boldsymbol{m}_i$ is used to select neurons from $\mathcal{T}_i$. The mask ratio is $r_i$, where $r\in(0,1]$. The mask $\boldsymbol{m}_i$ can be derived from:
\begin{equation}
    \boldsymbol{m}_{i,d}=\begin{aligned} &\left\{ \begin{array}{ll} 1, & \text{if } d\in \text{Disjoint}(\mathcal{T}_{i}^{r_i}), \\ 0, & \text{otherwise}. \end{array} \right. \end{aligned}
    \label{mask_safety}
\end{equation}


% \subsection{Merging Models with Masks}
{\textbf{Merging}}. The final
merged task vector $\boldsymbol{\tau}_m$ is as follows:
\begin{equation}
    \boldsymbol{\tau}_m= \sum_i \lambda_i\boldsymbol{\tau}_{i}\odot\boldsymbol{m}_i.
    \label{merged_task_vector}
\end{equation}
We summarize the workflow in Algorithm \ref{alg1}.



\section{Experiments}
\label{sec:experiments}

\begin{figure*}[t]
\vspace{-6mm}
    \centering
    \includegraphics[width=0.8\linewidth]{figs/compare.pdf}
    \vspace{-4mm}
    \caption{\textbf{Qualitative comparison} with the baseline for generating a sequence of novel view images.  
    The results demonstrate that our method synthesizes more consistent multi-view images compared to our baseline model (Zero123). In addition, compared to SyncDreamer, our method visually maintains better similarity to the conditioned image and appears more natural.}
    \label{fig:sota_compare}
\vspace{-5mm}
\end{figure*}

\subsection{Experimental Setups}
\textbf{Dataset.}
Following previous work~\cite{zero123, SyncDreamer}, we evaluate our work on the Google Scanned Object (GSO)~\cite{GSO} dataset to verify the zero-shot novel view image synthesis capability. 
We also provide results for additional datasets in the Supplementary Material.
Specifically, we randomly select 30 objects from the GSO dataset with various object categories. 
Unlike recent approaches~\cite{mvdream, SyncDreamer} that aim to enhance the consistency of novel view synthesis models by generating multiple fixed-view images, our method can generate images from any camera pose and any number of views. Therefore, we conduct experiments under different camera pose settings to validate our approach:
specifically, 
1) \textit{16-views with free camera pose}: for each object, we circularly render 16 views with the elevation angles ranging in $[-10\degree, 40\degree]$ and the azimuth angles are evenly distributed in $[0\degree, 360\degree]$. 
2) \textit{16-views with fixed camera pose}: We maintain a constant elevation angle of $30\degree$ and uniformly sample azimuth angles (same as SyncDreamer~\cite{SyncDreamer}).
3) \textit{32-views with free camera pose}: Similar to the first setting, but we sample 32 views.
It's important to note that our method does not require additional training or fine-tuning on any datasets.

\noindent\textbf{Metrics.}
To validate the effectiveness of our method, we mainly evaluate it based on three criteria:
1) \textit{Quality Score}. We evaluate the image quality of synthesized multi-view images by measuring their similarity with ground truth images. Following prior research~\cite{zero123, sparsefusion}, we report the similarity between the synthesized images and the ground truth images with standard metrics: PSNR, SSIM~\cite{ssim}, and LPIPS~\cite{lpips}.
2) \textit{Multi-view Consistency Score}. As the primary goal of our work is to improve the consistency of generated images, we also employ the 3D consistency score~\cite{3dim} to verify the consistency among the synthesized images. Specifically, we train an Instant-NGP~\cite{instant_ngp} with the input image and part of the synthesized novel view images of our model and evaluate the similarity between the remaining synthesized images and the rendered images of Instant-NGP. For the synthesized multi-view images of each object, we allocate $3/4$ for training and reserve the remaining $1/4$ for validation.
Intuitively, if the consistency of synthesized images is improved, the NeRF-like model will train a better object representation, and the re-rendered images will agree more with the validation images.
3) \textit{Input Consistency Score}. To assess the faithfulness of synthesized images in preserving the identity of the input condition image, we introduce the input consistency score. This score calculates the similarity of each synthesized image with the input condition image, utilizing the LPIPS metric.

In addition, we use synthesized multi-view images to train a neural 3D reconstruction model (NeuS~\cite{neus}) and report commonly used Chamfer Distances (CD) and Volume IoUs between the trained 3D model and the ground truth.

\noindent\textbf{Baselines.}
Given that our main goal is to improve the consistency of the trained baseline model without further fine-tuning, we mainly compare our approach with the used baseline model Zero123~\cite{zero123}. Additionally, we compare our method to the SOTA approaches such as PGD~\cite{tseng2023consistent} and SyncDreamer~\cite{SyncDreamer} using the same Zero123 base model.

\noindent\textbf{Implementation Details.}
We use the official checkpoint provided by Zero123~\cite{zero123}, which is trained on objaverse~\cite{objaverse} for 165,000 steps. We inject our epipolar attention layer after step $T=4$ and layer $L=10$ by default. We find that feature fusion weight $\alpha=0.5$, and the number of context views $M=2$ work better.

\begin{table}[t]
\centering
\caption{Comparison of multi-view consistency, image quality, and input consistency of synthesized multi-view images at the 16-view setting with free camera pose.}
\label{tab:view16_free_compare}
\vspace{-2mm}
\scalebox{0.6}{
\begin{tabular}{c ccc ccc c}
\toprule
              & \multicolumn{3}{c}{Multi-view Consistency} & \multicolumn{3}{c}{Quality Score} & \multicolumn{1}{c}{Input Consis.} \\
              \cmidrule(lr){2-4} \cmidrule(lr){5-7} \cmidrule(lr){8-8}
              & PSNR$\uparrow$  & SSIM$\uparrow$ & LPIPS$\downarrow$ 
              & PSNR$\uparrow$  & SSIM$\uparrow$ & LPIPS$\downarrow$ 
              & LPIPS$\downarrow$ 
              \\ \midrule

Zero123
& 15.225        & 0.645       & 0.408
& 14.255        & 0.747       &	0.208
& 0.303         
\\
SyncDreamer
& 14.830        & 0.626       & 0.434
& 12.650        & 0.713       &	0.254
& 0.317         
\\
Ours 
& \best{18.300}	& \best{0.734}	& \best{0.355}
& \best{14.947}	& \best{0.763}	& \best{0.191}
& \best{0.282}
\\

\bottomrule
\end{tabular}
}
\end{table}

\begin{table}[t]
\vspace{-1mm}
\centering
\caption{Comparison of multi-view consistency, image quality, and input consistency at the 16-view setting with fixed camera pose as SyncDreamer~\cite{SyncDreamer}.}
\label{tab:view16_fxied_compare}
\vspace{-3mm}
\scalebox{0.6}{
\begin{tabular}{c ccc ccc c}
\toprule
              & \multicolumn{3}{c}{Multi-view Consistency} & \multicolumn{3}{c}{Quality Score} & \multicolumn{1}{c}{Input Consis.} \\
              \cmidrule(lr){2-4} \cmidrule(lr){5-7} \cmidrule(lr){8-8}
              & PSNR$\uparrow$  & SSIM$\uparrow$ & LPIPS$\downarrow$ 
              & PSNR$\uparrow$  & SSIM$\uparrow$ & LPIPS$\downarrow$ 
              & LPIPS$\downarrow$ 
              \\ \midrule

Zero123
& 16.556        & 0.682       & 0.378
& 14.592        & 0.750       &	0.207
& 0.305         
\\
SyncDreamer
& \best{22.424}        & \best{0.812}       & \best{0.268}
& 15.269        & 0.749       &	0.196
& 0.300         
\\
Ours 
& 21.151	& 0.780	& 0.302
& \best{15.293}	& \best{0.764}	& \best{0.184}
& \best{0.287}
\\

\bottomrule
\end{tabular}
}
\vspace{-4mm}
\end{table}


\subsection{Comparison With Baseline Models}
The quantitative comparison on three settings are shown in Tab.~\ref{tab:view16_free_compare}, Tab.~\ref{tab:view16_fxied_compare}, and Tab.~\ref{tab:view32_free_compare}. The qualitative comparison is shown in Fig.~\ref{fig:sota_compare}.

\begin{table}[t]
\centering
\caption{Comparison of multi-view consistency and image quality scores of synthesized multi-view images at the 32-view setting with free camera pose.}
\vspace{-3mm}
\label{tab:view32_free_compare}
\scalebox{0.7}{
\begin{tabular}{c ccc ccc}
\toprule
              & \multicolumn{3}{c}{Multi-view Consistency} & \multicolumn{3}{c}{Quality Score} \\
              \cmidrule(lr){2-4} \cmidrule(lr){5-7}
              & PSNR$\uparrow$  & SSIM$\uparrow$ & LPIPS$\downarrow$ 
              & PSNR$\uparrow$  & SSIM$\uparrow$ & LPIPS$\downarrow$ 
              \\ \midrule

Zero123
& 16.515        & 0.694       & 0.378
& 15.142        & 0.733       &	0.211
\\
PGD~\cite{tseng2023consistent}
& 18.481        & 0.720       & 0.343
& 15.281        & 0.739       &	0.205
\\
Ours 
& \best{20.655}	& \best{0.792}	& \best{0.305}
& \best{15.268}	& \best{0.742}	& \best{0.203}
\\

\bottomrule
\end{tabular}
}
\vspace{-3mm}
\end{table}

\begin{table*}
  [t]
  \centering
  \resizebox{\textwidth}{!}{%
  \begin{tabular}{cccccccccccc}
    \toprule \multicolumn{2}{c}{Components}                                                             & \multicolumn{5}{c}{Re-executability Rate (\%)} & \multicolumn{5}{c}{Readability (\#)} \\
    \cmidrule(lr){1-2} \cmidrule(lr){3-7} \cmidrule(lr){8-12}        \hspace{8pt}\labelemoji\hspace{8pt}                                                                & \hspace{8pt}\toolemoji\hspace{8pt}                                      & O0                                 & O1             & O2             & O3             & AVG            & O0             & O1             & O2             & O3             & AVG            \\
    \hline
    \rowcolor[rgb]{0.93,0.93,0.93}\multicolumn{12}{c}{\textbf{Initialize with LLM4Decompile-End-6.7B~\citep{llm4decompile}}}   \\
    \xmark                                                                                              & \xmark                                    & 69.51                              & 46.95          & 50.61          & 46.34          & 53.35          & 3.98 & 3.41 & 3.44 & 3.38 & 3.55 \\
    \cmark                                                                                              & \xmark                                    & 75.61                              & 50.61          & 50.00          & 50.00          & 56.55          & 4.01 & 3.44 & 3.39 & \textbf{3.49} & 3.58 \\
    \xmark                                                                                              & \cmark                                    & 83.54                     & \textbf{56.10}          & 51.22          & 50.61 & 60.37 & 4.05 & 3.51 & 3.51 & 3.42 & 3.62 \\
    \cmark                                                                                              & \cmark                                    & \textbf{85.37}                            & \textbf{56.10}                     & \textbf{51.83} & \textbf{52.43}          & \textbf{61.43} & \textbf{4.13} & \textbf{3.60} & \textbf{3.54} & \textbf{3.49} & \textbf{3.69} \\

    \rowcolor[rgb]{0.93,0.93,0.93}\multicolumn{12}{c}{\textbf{Initialize with Deepseek-Coder-6.7B-base~\citep{deepseekcoder}}} \\
    \xmark                                                                                              & \xmark                                    & 59.15                              & 35.98          & 39.02          & 37.80          & 42.99          & 3.71 & 3.05 & 3.16 & 3.05 & 3.24 \\
    \cmark                                                                                              & \xmark                                    & 66.46                              & 41.46          & 38.41          & 36.59          & 45.73          & 3.76 & 3.17 & \textbf{3.21} & 3.08 & 3.31 \\
    \xmark                                                                                              & \cmark                                    & 70.73                              & 39.63          & 39.02          & 40.24          & 47.41          & 3.90 & 3.17 & 3.08 & 3.11 & 3.31 \\
    \cmark                                                                                              & \cmark                                    & \textbf{79.88}                     & \textbf{45.73} & \textbf{43.90} & \textbf{42.68} & \textbf{53.05} & \textbf{3.96} & \textbf{3.21} & 3.18 & \textbf{3.19} & \textbf{3.38} \\
    \bottomrule
  \end{tabular}%
  }
  \caption{The ablation study of different methods across four optimization levels
  (O0, O1, O2, O3), as well as their average scores (AVG). The results in bold represent the optimal performance. The ~\labelemoji~ and ~\toolemoji~ means Relabedling and Function Call. \textbf{Bold} denotes the best performance.}
  \label{tab:ablation}
\end{table*}



\begin{figure*}[ht]
    \centering
    \begin{minipage}{0.65\textwidth}
        \centering
        \includegraphics[width=0.95\linewidth]{figs/ablation.pdf}
        \vspace{-2mm}
        \captionof{figure}{Qualitative Comparison for different design choices. Our method, employing multi-view epipolar attention, demonstrates the best consistency.}
        \label{fig:ablation}
    \end{minipage}\hfill
    \begin{minipage}{0.33\textwidth}
        \centering
        \includegraphics[width=0.8\linewidth]{figs/neus_ver.pdf}
        \vspace{-3mm}
        \caption{Our method shows better direct 3D reconstruction~\cite{neus}.}
        \label{fig:neus}
    \end{minipage}
    \vspace{-5mm}
\end{figure*}

\noindent\textbf{Multi-view Consistency.}
Tab.~\ref{tab:view16_fxied_compare} presents the 3D consistency scores compared to our baseline model (Zero123) and SyncDreamer. The results indicate a significant improvement across all three metrics achieved by our method when compared with Zero123.
While our method exhibits a marginally lower numerical consistency score compared to SyncDreamer, it enables the synthesis of images with arbitrary camera poses.	
This capability is illustrated in Tab.~\ref{tab:view16_free_compare}, where our method consistently enhances consistency with changes in camera pose settings, whereas SyncDreamer fails to do so and exhibits inferior results compared to Zero123.
Furthermore, our method facilitates the synthesis of multi-view images with any number of camera views. This versatility is demonstrated in Tab.~\ref{tab:view32_free_compare}, where our method continues to achieve significant improvements in consistency scores, while SyncDreamer is unable to operate under such conditions.	

Meanwhile, Fig.~\ref{fig:sota_compare} provides a qualitative comparison with the baseline. While both our method and SyncDreamer enhance consistency, our method visually preserves better similarity to the input image, including color and texture details. The input consistency score further corroborates this.

\noindent\textbf{Image Quality.}
While our primary goal centers around enhancing the consistency of synthesized multi-view images, we also evaluate the image quality by comparing the similarity with the ground truth images. The results shown in Tab.~\ref{tab:view16_free_compare}, Tab.~\ref{tab:view16_fxied_compare}, and Tab.~\ref{tab:view32_free_compare} indicate that our method also enhances the image quality under different settings besides improving the consistency.
Moreover, our method shows better image quality compared with SyncDreamer even in the 16-view setting with fixed camera pose.

\noindent\textbf{Input Consistency.}
Input consistency terms whether the results align with the input image.
Fig.~\ref{fig:sota_compare} illustrates that both our method and SyncDreamer enhance multi-view consistency. However, the color and texture details of SyncDreamer's results diverge from the input image and appear visually unnatural.
This discrepancy is evident in the input consistency score presented in Tab.~\ref{tab:view16_fxied_compare}, indicating lower similarity with the condition image in the SyncDreamer results.	

\subsection{Ablation Study}
The overall quantitative results are shown in Tab.~\ref{tab:ablation}, and the qualitative comparisons are shown in Fig.~\ref{fig:ablation}.

\noindent \textbf{Full Attention \vs Epipolar Attention.}
The results presented in Tab.\ref{tab:ablation} and Fig.\ref{fig:ablation} demonstrate that our epipolar attention mechanism can synthesize more consistent multi-view images compared with full attention. Furthermore, our epipolar attention achieves a greater performance improvement compared to full attention when using multiple reference images. This could be attributed to the fact that our epipolar attention more effectively localizes target information, as depicted in Fig.~\ref{fig:full_attn_compare}, thereby reducing noise from the reference images. In the multi-view setting, where multiple reference images are utilized, this noise reduction becomes particularly crucial.
Moreover, it is noteworthy that the epipolar attention mechanism consumes less GPU memory compared to our baseline, as discussed in Sec.~\ref{sec:attn_analysis}.

\noindent \textbf{Attending Single-View \vs Multi-View.}
Applying the epipolar attention significantly improves the consistency between the input and target views. However, the consistency between different views in the unobserved regions of the input view is not well preserved.
After implementing our epipolar attention in the multi-view setting, the consistency across the generated multi-view images is further improved. The last row in Tab.~\ref{tab:ablation} shows that after applying our multi-view epipolar attention, the consistency score is further improved compared with the single-view setting. Besides, the qualitative result in Fig.~\ref{fig:ablation} also shows better consistency among different target views.



\begin{table}[t]
\centering
\vspace{-1mm}
\caption{Comparison of 3D reconstruction results. Our method significantly improves the reconstruction quality.}
\vspace{-3mm}
\label{tab:neus}
\scalebox{0.7}{
\begin{tabular}{c cc}
\toprule
              &  Chamfer Dist.$\downarrow$  & Volume IoU$\uparrow$
\\ \midrule

            Zero123         & 0.017         & 0.819    \\
            SyncDreamer     & \best{0.013}         & \best{0.847}    \\
            Ours            & 0.014	& 0.842 \\

\bottomrule
\end{tabular}
}
\vspace{-5mm}
\end{table}


\vspace{-2mm}
\subsection{Downstream Application}
\vspace{-2mm}
To demonstrate the effectiveness of our method, we also applied it to the downstream 3D reconstruction task. Specifically, we trained the NeuS model~\cite{neus} directly using images synthesized by our method, Zero123, and SyncDreamer, respectively.
The quantitative results in Tab.~\ref{tab:neus} show that the consistent multi-view images synthesized by our method can significantly improve the 3D reconstruction quality.
Additionally, our method exhibits similar performance to SyncDreamer which requires time-consuming re-training.
The qualitative results in Fig.~\ref{fig:neus} show that it is challenging to train the NeuS model directly due to the lack of consistency in the images generated by Zero123. In contrast, our method generates more consistent multi-view images and, therefore, better reconstructs the geometry and texture details.
We show improvements on other downstream applications such as image-to-3D in the Supplementary Material.


\section{Conclusion}

%In this paper, w
We propose a new PEFT method called DiffoRA, which enables efficient and adaptive LLM fine-tuning based on LoRA. 
Instead of adjusting every interior rank, 
%of the decomposition matrices 
%of all modules, 
we argue that adopting LoRA module-wisely is sufficient. 
To achieve this, we construct a DAM to select the modules that are most suitable and essential to fine-tune. We theoretically analyze how the DAM impacts the convergence rate and generalization capability.
%of the pre-trained model. 
Furthermore, we adopt continuous relaxation and discretization to establish DAM.
%for each task. 
To alleviate the issue of discretization discrepancy, we utilize the weight-sharing strategy for optimization. 
%We fully implement our method and t
The experimental results demonstrate that our DiffoRA works consistently better than the baselines across all benchmarks. 


\bibliography{references}
\bibliographystyle{misc/icml2025}


%%%%%%%%%%%%%%%%%%%%%%%%%%%%%%%%%%%%%%%%%%%%%%%%%%%%%%%%%%%%%%%%%%%%%%%%%%%%%%%
%%%%%%%%%%%%%%%%%%%%%%%%%%%%%%%%%%%%%%%%%%%%%%%%%%%%%%%%%%%%%%%%%%%%%%%%%%%%%%%
% APPENDIX
%%%%%%%%%%%%%%%%%%%%%%%%%%%%%%%%%%%%%%%%%%%%%%%%%%%%%%%%%%%%%%%%%%%%%%%%%%%%%%%
%%%%%%%%%%%%%%%%%%%%%%%%%%%%%%%%%%%%%%%%%%%%%%%%%%%%%%%%%%%%%%%%%%%%%%%%%%%%%%%
\newpage
\appendix
\onecolumn
\section*{Appendix Overview}
\begin{itemize}
    \item Section~\ref{appendix:related}: Related Work.
    \item Section~\ref{appendix:more_dataset}: More Dataset Details.
    \item Section~\ref{appendix:error_analysis}: Error Analysis.
    \item Section~\ref{appendix:more_qualitative}: More Qualitative Examples.
    \item Section~\ref{appendix:eval_setup}: Evaluation Prompts.
\end{itemize}


\section{Related Work}
\label{appendix:related}
\subsection{Large Multimodal Models}
The field of multimodal~\citep{Radford2021LearningTV, li2022blip, openai2023gpt4v, openai2024gpt4o} AI has experienced extraordinary growth, particularly through the development of Large Multimodal Models (LMMs)~\cite{liu2023llava,zhu2023minigpt,lin2023sphinx,Qwen2-VL}. These models build upon the achievements of Large Language Models (LLMs)~\citep{touvron2023llama,qwen2} and advanced vision models~\cite{Radford2021LearningTV}, expanding their capabilities to process multiple kinds of visual input~\cite{li2024llava,guo2023point,li2023videochat}.

Closed-source models, such as OpenAI's GPT-4o~\citep{openai2024gpt4o}, have demonstrated exceptional capabilities in visual understanding and reasoning. However, their closed-source nature creates barriers to widespread adoption and further development by the broader research community. In response, significant progress has been made in developing open-source alternatives. Early approaches like LLaVA~\cite{liu2023llava}, LLaMA-Adapter~\cite{zhang2024llamaadapter}, and MiniGPT-4~\cite{zhu2023minigpt} established a foundation by combining frozen CLIP models for image encoding with LLMs, enabling multimodal instruction tuning. Subsequent developments through projects such as InternVL2~\cite{chen2024far}, Qwen2-VL~\cite{Qwen2-VL}, SPHINX~\cite{gao2024sphinx,lin2023sphinx}, and MiniCPM-V~\cite{yao2024minicpm} have expanded these capabilities by incorporating more diverse visual instruction datasets and broadening application scenarios.

Recently, with the introduction of o1~\cite{o1}, the field of LMMs has also focused on enhancing the reasoning capability. \cite{wang2024enhancing} introduces mixed preference optimization with automatically constructed data. \cite{yao2024mulberry} proposes to leverage collective knowledge from multiple models to identify effective reasoning paths. Besides, several works~\cite{qvq-72b-preview,du2025virgo} have demonstrated the ability to replicate behaviors similar to o1 models, particularly regarding multi-step CoT reasoning with iterative self-reflection and verification processes.

\subsection{Reasoning Evaluation}
Several methods have been developed to evaluate reasoning in natural language processing, including ROSCOE~\cite{golovneva2022roscoe} and ReCEval~\cite{prasad2023receval}, which assess reasoning chains across multiple dimensions such as correctness and informativeness. However, these approaches are limited to text-only scenarios and do not address the unique challenges present in visual reasoning tasks. Furthermore, the emergence of long chain-of-thought (CoT) reasoning has introduced additional considerations, such as output efficiency and reflection quality, which existing evaluation methods do not adequately address.

On the other hand, various multimodal benchmarks have been developed to assess reasoning abilities across specific domains. Current exploration of visual reasoning predominantly focuses on the mathematics~\cite{zhang2024mavis,peng2024chimera} domains. 
MathVista~\cite{Lu2023MathVistaEM} provides a comprehensive collection of mathematical problems that assess mathematical and logical reasoning abilities. 
Building on this, MathVerse~\cite{zhang2024mathverse} introduces a new benchmark by eliminating redundant textual information to evaluate whether LMMs can accurately interpret graphical representations. 
OlympiadBench~\cite{he2024olympiadbench} further raises the complexity bar by incorporating challenging Olympiad-level mathematics and physics problems. Despite these advances in specialized domains, broader applications such as general-scene reasoning remain relatively unexplored.
Recent developments have begun to expand beyond purely scientific reasoning. For instance, M³CoT~\cite{chen-etal-2024-m3cot} and SciVerse~\cite{sciverse} incorporate commonsense tasks alongside scientific reasoning and knowledge-based assessment in the multimodal benchmark. However, most existing benchmarks focus solely on evaluating final answers while overlooking the intermediate steps, thus providing limited insights into the process through which models arrive at their conclusions.


\section{More Dataset Details}
\label{appendix:more_dataset}
\subsection{Data Source Distribution}
We visualize the data source distributions in our benchmark, which consists of 15 sets, including MathVerse~\cite{zhang2024mathverse}, MMMUPro~\cite{yue2024mmmuprorobustmultidisciplinemultimodal}, OlympiadBench~\cite{he2024olympiadbench}, MMT-Bench~\cite{ying2024mmt}, MuirBench~\cite{wang2024muirbench}, ml-rpm-bench~\cite{zhang2024far}, MMSearch~\cite{jiang2024mmsearch}, CharXiv~\cite{wang2024charxiv}, and SciVerse~\cite{sciverse}.

\begin{figure*}[!h]
\centering
\includegraphics[width=0.4\textwidth]{fig/pie_supp.pdf} 
\caption{\textbf{Data Source Distribution of MME-CoT.}}
\label{appendix:more_dataset-source}
\end{figure*}

\newpage

\subsection{Preliminary Categorization Result}
\label{appendix:preliminary_result}
\begin{table}[htbp]
    \centering
    \caption{\textbf{Accuracy of MMT-Bench for different subcategories}. ACT: Action Understanding; AUT: Attribute Similarity; CNT: Cartoon Understanding; CIM: Counting; DOC: Diagram Understanding; EMO: Difference Spotting; HAL: Geographic Understanding; IIT: Image-Text Matching; IRT: Ordering; IQT: Scene Understanding; MEM: Visual Grounding; MIA: Visual Retrieval; OCR: Object Recognition; PLP: Physical Layout Prediction; RRE: Relationship Extraction; TMP: Temporal Reasoning; VCP: Visual Comprehension; VCR: Visual Coherence Reasoning; VGR: Visual Generation; VIL: Visual Identification; VPU: Visual Prediction Understanding; VRE: Visual Reasoning Evaluation.}
    \label{tab:hit_ratio}
    \setlength{\tabcolsep}{4pt} 
    \renewcommand{\arraystretch}{1.2}
    \small 
    \begin{tabularx}{\textwidth}{l *{22}{X}}
        \toprule
        File Name & 
        \rotatebox{90}{ACT} & \rotatebox{90}{AUT} & \rotatebox{90}{CNT} & \rotatebox{90}{CIM} & 
        \rotatebox{90}{DOC} & \rotatebox{90}{EMO} & \rotatebox{90}{HAL} & \rotatebox{90}{IIT} & 
        \rotatebox{90}{IRT} & \rotatebox{90}{IQT} & \rotatebox{90}{MEM} & \rotatebox{90}{MIA} & 
        \rotatebox{90}{OCR} & \rotatebox{90}{PLP} & \rotatebox{90}{RRE} & \rotatebox{90}{TMP} & 
        \rotatebox{90}{VCP} & \rotatebox{90}{VCR} & \rotatebox{90}{VGR} & \rotatebox{90}{VIL} & 
        \rotatebox{90}{VPU} & \rotatebox{90}{VRE} \\
        \midrule
        GPT4o-cot & 0.60 & 0.60 & 0.44 & 0.67 & 0.79 & 0.30 & 0.71 & 0.50 & 0.63 & 0.10 & 0.85 & 0.60 & 0.77 & 0.36 & 0.76 & 0.48 & 0.86 & 0.80 & 0.49 & 0.48 & 0.82 & 0.85 \\
        GPT4-direct & 0.53 & 0.60 & 0.44 & 0.67 & 0.81 & 0.23 & 0.69 & 0.33 & 0.66 & 0.25 & 0.80 & 0.43 & 0.78 & 0.42 & 0.78 & 0.36 & 0.89 & 0.85 & 0.41 & 0.37 & 0.85 & 0.85 \\
        Qwen2-VL-7B-cot & 0.53 & 0.61 & 0.34 & 0.65 & 0.77 & 0.53 & 0.74 & 0.40 & 0.31 & 0.20 & 0.78 & 0.58 & 0.60 & 0.43 & 0.69 & 0.43 & 0.85 & 0.90 & 0.54 & 0.35 & 0.79 & 0.81 \\
        Qwen2-VL-7B-direct & 0.49 & 0.67 & 0.40 & 0.78 & 0.75 & 0.52 & 0.73 & 0.43 & 0.31 & 0.10 & 0.78 & 0.55 & 0.60 & 0.54 & 0.69 & 0.40 & 0.85 & 0.85 & 0.67 & 0.38 & 0.85 & 0.82 \\
        \bottomrule
    \end{tabularx}
\end{table}


\begin{table}[htbp]
    \centering
    \caption{\textbf{Accuracy of MUIRBench for different subcategories}. AU: Action Understanding; AS: Attribute Similarity; CU: Cartoon Understanding; CO: Counting; DU: Diagram Understanding; DS: Difference Spotting; GU: Geographic Understanding; ITM: Image-Text Matching; OR: Ordering; SU: Scene Understanding; VG: Visual Grounding; VR: Visual Retrieval.}

    \label{tab:hit_ratio}
    \setlength{\tabcolsep}{4pt} 
    \renewcommand{\arraystretch}{1.2} 
    \small 
    \begin{tabularx}{\textwidth}{l XXXX XXXX XXXX XXXX}
        \toprule
        File Name & AU & AS & CU & CO & DU & DS & GU & ITM & OR & SU & VG & VR \\
        \midrule
        GPT4o-cot & 0.48 & 0.57 & 0.55 & 0.75 & 0.82 & 0.64 & 0.59 & 0.82 & 0.38 & 0.88 & 0.56 & 0.70 \\
        GPT4o-direct & 0.45 & 0.62 & 0.59 & 0.50 & 0.88 & 0.62 & 0.55 & 0.86 & 0.33 & 0.74 & 0.38 & 0.77 \\
        Qwen2-VL-7B-cot & 0.38 & 0.51 & 0.42 & 0.43 & 0.43 & 0.27 & 0.21 & 0.55 & 0.13 & 0.69 & 0.37 & 0.28 \\
        Qwen2-VL-7B-direct & 0.39 & 0.47 & 0.44 & 0.41 & 0.40 & 0.33 & 0.25 & 0.51 & 0.13 & 0.67 & 0.31 & 0.20 \\
        \bottomrule
    \end{tabularx}
\end{table}



\begin{table}[htbp]
    \centering
    \caption{\textbf{Accuracy of OlympiadBench for the mathematics and physics subcategories}.}
    \label{tab:hit_ratio_oe}
    \small 
    \begin{tabular}{lcc}
        \toprule
        File Name & Mathematics & Physics\\
        \midrule
        GPT4o-cot & 0.25 & 0.04 \\
        GPT4o-direct & 0.07 & 0.03 \\
        Qwen2-VL-7B-cot & 0.05 & 0.01 \\
        Qwen2-VL-7B-direct & 0.07 & 0.01 \\
        \bottomrule
    \end{tabular}
\end{table}

\newpage

\section{Error Analysis}
\label{appendix:error_analysis}
We showcase the examples of the identified error types of reflection in Fig.~\ref{fig:ref_error_example}.
\begin{figure*}[!h]
\centering
\includegraphics[width=\textwidth]{fig/ref_error_example.pdf} 
\caption{\textbf{Examples of Reflection Error Types.}}
\label{fig:ref_error_example}
\end{figure*}


\newpage

\section{More Qualitative Examples}
\label{appendix:more_qualitative}
\begin{figure*}[!h]
\centering
\includegraphics[width=0.6\textwidth]{fig/precision_recall_example_GPT.pdf} 
\caption{\textbf{Examples of Precision and Recall Evaluation.}}
\label{fig:precision_recall_example_GPT}
\end{figure*}
\newpage

\begin{figure*}[!h]
\centering
\includegraphics[width=0.9\textwidth]{fig/precision_recall_example_Qwen.pdf} 
\caption{\textbf{Examples of Precision and Recall Evaluation.}}
\label{fig:precision_recall_example_Qwen}
\end{figure*}
\newpage

\begin{figure*}[!h]
\centering
\includegraphics[width=0.58\textwidth]{fig/precision_recall_example_QVQ.pdf}
\caption{\textbf{Examples of Precision and Recall Evaluation.}}
\label{fig:precision_recall_example_QVQ}
\end{figure*}
\newpage

\begin{figure*}[!h]
\centering
\includegraphics[width=\textwidth]{fig/precision_recall_example_QVQ2.pdf} 
\caption{\textbf{Examples of Precision and Recall Evaluation.}}
\label{fig:precision_recall_example_QVQ2}
\end{figure*}
\newpage

\begin{figure*}[!h]
\centering
\includegraphics[width=0.51\textwidth]{fig/precision_recall_example2_GPT.pdf} 
\caption{\textbf{Examples of Precision and Recall Evaluation.}}
\label{fig:precision_recall_example2_GPT}
\end{figure*}
\newpage

\begin{figure*}[!h]
\centering
\includegraphics[width=0.79\textwidth]{fig/precision_recall_example2_Qwen.pdf} 
\caption{\textbf{Examples of Precision and Recall Evaluation.}}
\label{fig:precision_recall_example2_Qwen}
\end{figure*}
\newpage

\begin{figure*}[!h]
\centering
\includegraphics[width=0.81\textwidth]{fig/precision_recall_example2_QVQ.pdf} 
\caption{\textbf{Examples of Precision and Recall Evaluation.}}
\label{fig:precision_recall_example2_QVQ}
\end{figure*}
\newpage

\begin{figure*}[!h]
\centering
\includegraphics[width=\textwidth]{fig/relevance_example_GPT.pdf} 
\caption{\textbf{Examples of Relevance Rate Evaluation.}}
% \vspace{-1cm}
\label{fig:relevance_example_GPT}
\end{figure*}
\newpage

\begin{figure*}[!h]
\centering
\includegraphics[width=\textwidth]{fig/relevance_example_Qwen.pdf} 
\caption{\textbf{Examples of Relevance Rate Evaluation.}}
% \vspace{-1cm}
\label{fig:relevance_example_Qwen}
\end{figure*}
\newpage

\begin{figure*}[!h]
\centering
\includegraphics[width=\textwidth]{fig/relevance_example_QVQ.pdf} 
\caption{\textbf{Examples of Relevance Rate Evaluation.}}
% \vspace{-1cm}
\label{fig:relevance_example_QVQ}
\end{figure*}
\newpage

\begin{figure*}[!h]
\centering
\includegraphics[width=\textwidth]{fig/ref_example_QVQ.pdf} 
\caption{\textbf{Examples of Reflection Quality Evaluation.}}
% \vspace{-1cm}
\label{fig:ref_example_QVQ}
\end{figure*}
\newpage


\section{Detailed Evaluation Setup}
\label{appendix:eval_setup}
\subsection{CoT Quality Evaluation Prompts}

\begin{tcolorbox}[breakable, colback=gray!5!white, colframe=gray!75!black, 
title=Recall Evaluation Prompt, boxrule=0.5mm, width=\textwidth, arc=3mm, auto outer arc]

You are an expert system to verify solutions to image-based problems. Your task is to match the ground truth middle steps with the provided solution.\\

INPUT FORMAT:\\
1. Problem: The original question/task\\
2. A Solution of a model\\
3. Ground Truth: Essential steps required for a correct answer\\

MATCHING PROCESS:\\

You need to match each ground truth middle step with the solution:\\

Match Criteria:\\
- The middle step should exactly match in the content or is directly entailed by a certain content in the solution\\
- All the details must be matched, including the specific value and content\\
- You should judge all the middle steps for whether there is a match in the solution\\

OUTPUT FORMAT:
\begin{verbatim}
[
  {
    "step_index": \textless integer\textgreater,
    "judgment": "Matched" | "Unmatched"
  }
]
\end{verbatim}

ADDITIONAL RULES:\\
1. Only output the JSON array with no additional information.\\
2. Judge each ground truth middle step in order without omitting any step.\\

Here are the problem, answer, solution, and ground truth middle steps:\\

[Problem]\\

\{question\}\\

[Answer]\\

\{answer\}\\

[Solution]\\

\{solution\}\\

[Ground Truth Information]\\

\{gt\_annotation\}

\end{tcolorbox}

\begin{tcolorbox}[breakable, colback=gray!5!white, colframe=gray!75!black, 
title=Precision Evaluation Prompt, boxrule=0.5mm, width=\textwidth, arc=3mm, auto outer arc]

\# Task Overview\\
Given a solution with multiple reasoning steps for an image-based problem, reformat it into well-structured steps and evaluate their correctness.\\

\# Step 1: Reformatting the Solution\\
Convert the unstructured solution into distinct reasoning steps while:\\
- Preserving all original content and order\\
- Not adding new interpretations\\
- Not omitting any steps\\

\#\# Step Types\\
1. Logical Inference Steps\\
   - Contains exactly one logical deduction\\
   - Must produce a new derived conclusion\\
   - Cannot be just a summary or observation\\
\\
2. Image Observation Steps\\
   - Pure visual observations\\
   - Only includes directly visible elements\\
   - No inferences or assumptions\\
\\
3. Background Information Steps\\
   - External knowledge or question context\\
   - No inference process involved\\

\#\# Step Requirements\\
- Each step must be atomic (one conclusion per step)\\
- No content duplication across steps\\
- Initial analysis counts as background information\\
- Final answer determination counts as logical inference\\

\# Step 2: Evaluating Correctness\\
Evaluate each step against:\\

\#\# Ground Truth Matching\\
For image observations:\\
- Key elements must match ground truth observations\\
\\
For logical inferences:\\
- Conclusion must EXACTLY match or be DIRECTLY entailed by ground truth\\

\#\# Reasonableness Check (if no direct match)\\
Step must:\\
- Premises must not contradict any ground truth or correct answer\\
- Logic is valid\\
- Conclusion must not contradict any ground truth \\
- Conclusion must support or be neutral to correct answer\\

\#\# Judgement Categories\\
- "Match": Aligns with ground truth\\
- "Reasonable": Valid but not in ground truth\\
- "Wrong": Invalid or contradictory\\
- "N/A": For background information steps\\

\# Output Requirements\\
1. The output format must be in valid JSON format without any other content.\\
2. For highly repetitive patterns, output it as a single step.\\
3. Output maximum 40 steps. Always include the final step that contains the answer.\\

Here is the json output format:\\
\#\# Output Format
\begin{verbatim}
[
  {
    "step_type": "image observation|logical inference|background information",
    "premise": "Evidence (only for logical inference)",
    "conclusion": "Step result",
    "judgment": "Match|Reasonable|Wrong|N/A"
  }
]
\end{verbatim}

Here is the problem, and the solution that needs to be reformatted to steps:\\

[Problem]\\

\{question\}\\

[Solution]\\

\{solution\}\\

[Correct Answer]\\

\{answer\}\\

[Ground Truth Information]\\

\{gt\_annotation\}

\end{tcolorbox}

\subsection{CoT Efficiency Prompt}
\begin{tcolorbox}[breakable, colback=gray!5!white, colframe=gray!75!black, 
title=Relevance Rate Evaluation Prompt, boxrule=0.5mm, width=\textwidth, arc=3mm, auto outer arc]
\# Task Overview
Given a solution with multiple reasoning steps for an image-based problem, evaluate the relevance to get a solution (ignore correct or wrong) of each step.\\

\# Step 1: Reformatting the Solution
Convert the unstructured solution into distinct reasoning steps while:\\
- Preserving all original content and order\\
- Not adding new interpretations\\
- Not omitting any steps\\

\#\# Step Types \\
1. Logical Inference Steps\\
  - Contains exactly one logical deduction\\
  - Must produce a new derived conclusion\\
  - Cannot be just a summary or observation

2. Image Description Steps\\
  - Pure visual observations\\
  - Only includes directly visible elements\\
  - No inferences or assumptions

3. Background Information Steps\\
  - External knowledge or question context\\
  - No inference process involved\\

\#\# Step Requirements
- Each step must be atomic (one conclusion per step)\\
- No content duplication across steps\\
- Initial analysis counts as background information\\
- Final answer determination counts as logical inference\\

\# Step 2: Evaluating Relevancy\\
A relevant step is considered as: 75\% content of the step must be related to trying to get a solution (ignore correct or wrong) to the question.\\

IMPORTANT NOTE:\\
Evaluate relevancy independent of correctness. As long as the step is trying to get to a solution, it is considered relevant. Logical fallacy, knowledge mistake, inconsistent with previous steps, or other mistakes do not affect relevance. A logically wrong step can be relevant if the reasoning attempts to address the question.\\

The following behaviour is considered as relevant:\\
i. The step is planning, summarizing, thinking, verifying, calculating, or confirming an intermediate/final conclusion helpful to get a solution.\\
ii. The step is summarizing or reflecting on previously reached conclusion relevant to get a solution.\\
iii. Repeating the information in the question or give the final answer.\\
iv. A relevant image depiction should be in one of following situation:\\
1. help to obtain a conclusion helpful to solve the question later;\\
2. help to identify certain patterns in the image later;\\
3. directly contributes to the answer\\
v. Depicting or analyzing the options of the question is also relevant.\\
vi. Repeating previous relevant steps are also considered relevant.\\

The following behaviour is considered as irrelevant:\\
i. Depicting image information that does not related to what is asking in the question. Example: The question asks how many cars are present in all the images. If the step focuses on other visual elements like the road or building, the step is considered as irrelevant.\\
ii. Self-thought not related to what the question is asking.\\
iii. Other information that is tangential for answering the question.\\

\# Output Format

\begin{verbatim}
[
  {
    "step_type": "image observation|logical inference|background information",
    "conclusion": "A brief summary of step result",
    "relevant": "Yes|No"
  }
]
\end{verbatim}\\

\# Output Rules\\
Direct JSON output without any other output\\
Output at most 40 steps\\

Here is the problem, and the solution that needs to be reformatted to steps:

[Problem]\\

\{question\}\\

[Solution]\\

\{solution\}
\end{tcolorbox}

\begin{tcolorbox}[breakable, colback=gray!5!white, colframe=gray!75!black, 
title=Reflection Quality Evaluation Prompt, boxrule=0.5mm, width=\textwidth, arc=3mm, auto outer arc]

Here\'s a refined prompt that improves clarity and structure:\\

\# Task\\
Evaluate reflection steps in image-based problem solutions, where reflections are self-corrections or reconsideration of previous statements.\\

\# Reflection Step Identification \\
Reflections typically begin with phrases like:\\
- "But xxx"\\
- "Alternatively, xxx" \\
- "Maybe I should"\\
- "Let me double-check"\\
- "Wait xxx"\\
- "Perhaps xxx"\\
It will throw a doubt of its previously reached conclusion or raise a new thought.\\

\# Evaluation Criteria\\
Correct reflections must:\\
1. Reach accurate conclusions aligned with ground truth\\
2. Use new insights to find the mistake of the previous conclusion or verify its correctness. \\

Invalid reflections include:\\
1. Repetition - Restating previous content or method without new insights\\
2. Wrong Conclusion - Reaching incorrect conclusions vs ground truth\\
3. Incompleteness - Proposing but not executing new analysis methods\\
4. Other - Additional error types\\

\# Input Format\\

[Problem]\\

\{question\}\\

[Solution]\\

\{solution\}\\

[Ground Truth]\\

\{gt\_annotation\}\\

\# Output Requirements\\
1. The output format must be in valid JSON format without any other content.\\
2. Output maximum 30 reflection steps.\\

Here is the json output format:\\
\#\# Output Format
\begin{verbatim}
[
  {
    "conclusion": "One-sentence summary of reflection outcome",
    "judgment": "Correct|Wrong",
    "error_type": "N/A|Repetition|Wrong Conclusion|Incompleteness|Other"
  }
]
\end{verbatim}

\# Rules\\
1. Preserve original content and order\\
2. No new interpretations\\
3. Include ALL reflection steps\\
4. Empty list if no reflections found\\
5. Direct JSON output without any other output

\end{tcolorbox}

\subsection{Direct Evaluation Prompt}
\begin{tcolorbox}[breakable, colback=gray!5!white, colframe=gray!75!black, 
title=Answer Extraction Prompt, boxrule=0.5mm, width=\textwidth, arc=3mm, auto outer arc]
You are an AI assistant who will help me to extract an answer of a question. You are provided with a question and a response, and you need to find the final answer of the question. \\

Extract Rule:

[Multiple choice question]

1. The answer could be answering the option letter or the value. You should directly output the choice letter of the answer.

2. You should output a single uppercase character in A, B, C, D, E, F, G, H, I (if they are valid options), and Z.

3. If the meaning of all options are significantly different from the final answer, output Z. \\

[Non Multiple choice question]

1. Output the final value of the answer. It could be hidden inside the last step of calculation or inference. Pay attention to what the question is asking for to extract the value of the answer.

2. The final answer could also be a short phrase or sentence.

3. If the response doesn't give a final answer, output Z.\\

Output Format: 
Directly output the extracted answer of the response. \\

\{In Context Examples\}\\

Question: \{question\}

Answer: \{response\}\\

Your output: 

\end{tcolorbox}

\begin{tcolorbox}[breakable, colback=gray!5!white, colframe=gray!75!black, 
title=Answer Scoring Prompt, boxrule=0.5mm, width=\textwidth, arc=3mm, auto outer arc]

You are an AI assistant who will help me to judge whether two answers are consistent.\\

Input Illustration:
[Standard Answer] is the standard answer to the question. 
[Model Answer] is the answer extracted from a model's output to this question. 

Task Illustration:
Determine whether [Standard Answer] and [Model Answer] are consistent.\\

Consistent Criteria:

[Multiple-Choice questions]

1. If the [Model Answer] is the option letter, then it must completely matches the [Standard Answer].

2. If the [Model Answer] is not an option letter, then the [Model Answer] must completely match the option content of [Standard Answer].

[Nan-Multiple-Choice questions]

1. The [Model Answer] and [Standard Answer] should exactly match.

2. If the meaning is expressed in the same way, it is also considered consistent, for example, 0.5m and 50cm.\\

Output Format: 
1. If they are consistent, output 1; if they are different, output 0.

2. DIRECTLY output 1 or 0 without any other content.

\{In Context Examples\}\\

Question: \{question\}

[Model Answer]: \{extract\_answer\}

[Standard Answer]: \{gt\_answer\}

Your output:

\end{tcolorbox}

\end{document}


% This document was modified from the file originally made available by
% Pat Langley and Andrea Danyluk for ICML-2K. This version was created
% by Iain Murray in 2018, and modified by Alexandre Bouchard in
% 2019 and 2021 and by Csaba Szepesvari, Gang Niu and Sivan Sabato in 2022.
% Modified again in 2023 and 2024 by Sivan Sabato and Jonathan Scarlett.
% Previous contributors include Dan Roy, Lise Getoor and Tobias
% Scheffer, which was slightly modified from the 2010 version by
% Thorsten Joachims & Johannes Fuernkranz, slightly modified from the
% 2009 version by Kiri Wagstaff and Sam Roweis's 2008 version, which is
% slightly modified from Prasad Tadepalli's 2007 version which is a
% lightly changed version of the previous year's version by Andrew
% Moore, which was in turn edited from those of Kristian Kersting and
% Codrina Lauth. Alex Smola contributed to the algorithmic style files.
