\section{Related Work} \label{sec:related}
AVs have been the subject of extensive research in recent years, leading to significant advancements in their capabilities. Early efforts in AV development focused on improving core functionalities such as perception, planning, and control~\cite{levinson2011towards, yurtsever2020survey}. These areas are critical for enabling AVs to navigate complex environments safely. However, the limitations of AVs compared to human drivers, particularly regarding adaptability and decision-making in unpredictable scenarios, have prompted further research into more intelligent systems.


Several approaches have been proposed to address challenges encountered by AVs at runtime. Rule-based systems have been used to ensure adherence to safety and traffic regulations. For example, runtime enforcement mechanisms monitor the vehicle's actions~\cite{Mauritz-et_al16a, d2005lola, Watanabe-et_al18a} to prevent collisions and other unsafe behaviours~\cite{Grieser-et_al20a, hong2020avguardian, Cheng-et_al21a, Shankar-et_al20a}. Similarly, gradient-based algorithms, such as those in REDriver~\cite{sun2024redriver}, offer real-time solutions for handling property specification violations. While these methods provide valuable safety nets, their utility is limited by their narrower focus on specific tasks.

Recognising the expertise of human drivers, researchers have explored various ways to model and replicate human driving behaviour in AVs. Imitation learning has emerged as a prominent technique for training AVs to mimic expert human drivers~\cite{sama2020extracting, wei2010learning, xu2020learning}. These methods aim to capture the nuanced decision-making processes of human drivers to improve AV performance. However, challenges such as limited training data and the complexity of human driving behaviour have hindered the generalisation of these approaches~\cite{le2022survey}. As a result, there is a growing interest in developing more advanced systems that can better bridge the gap between human intelligence and AV technology.

The advent of MLLMs has opened new avenues for enhancing AV intelligence. MLLMs, with their advanced text and image understanding capabilities, offer promising solutions for interpreting and replicating human driving behaviour. They can provide natural language explanations for their decisions, thereby enhancing transparency and trust~\cite{cui2024survey}. Existing research has explored the use of MLLMs in various components of AVs, including perception, planning, and control~\cite{chen2023driving, mao2023gpt, wen2024road}. For instance, LLM-Driver~\cite{chen2023driving} abstracts driving scenarios into 2D object-level vectors and directly applies the LLM output as control commands for the AV system. GPT-Driver~\cite{mao2023gpt} translates motion planner inputs and outputs into language tokens, utilising LLMs as motion planners for AVs. Wen et al.~\cite{wen2024road} evaluated the potential of ChatGPT-4 as an autonomous driving agent, demonstrating its advanced scene understanding and causal reasoning capabilities. 
These works primarily employ LLMs for object perception, motion planning, and actuation control within AV systems. Despite their potential, the inherent delays and uncertainties associated with generative models pose challenges for real-time AV operations. Additionally, the gap between natural language and control commands remains a significant hurdle.


\coolname, in contrast, is a framework that generates driving strategy repairs for AVs. By providing a general offline solution, it ensures that MLLMs can generate general driving suggestions that are directly applicable to AVs. Through comprehensive testing in various scenarios, \coolname has demonstrated its effectiveness in improving AV decision-making and adherence to property specifications, offering an advancement to the field of autonomous driving.


