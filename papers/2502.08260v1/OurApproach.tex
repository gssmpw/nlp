\section{Our Approach}
\label{sec:our_approach}
\coolname, our framework for obtaining general driving strategy repairs via MLLMs, comprises three main steps. First, \emph{problem localisation}, which identifies when the violation of the specified properties occurred and any near-miss situations. Second, \emph{prompt generation}, which automatically generates the necessary text prompts and visualisations of specific driving conditions. These specific conditions refer to the time steps when property violations and near-miss situations occurred. Finally, \emph{$\mu$Drive script generation}, which formats the MLLM's responses into syntactically valid $\mu$Drive scripts, ensuring the driving strategy repairs are executed by the ADS.



\subsection{Problem Localisation}
One possible way to allow a language model to comprehend a driving incident is to provide it with a complete record, i.e., a structured log file that captures all necessary data to recreate the driving scenario. This log would include detailed information on the driving environment perception, routing details, predictions of other vehicles, pedestrians, and traffic lights, as well as motion planning and control commands.
However, processing such comprehensive records is computationally intensive, costly, and time-consuming. Fortunately, in driving scenarios where certain properties are violated, there are always a few key moments that hold significant importance. Identifying these moments allows for a better understanding of the driving scenario, and can also be automated.
For example, ACAV~\cite{sun2024acav} reduces the length of driving records by 62.23\% based on a causality analysis, and correctly identifies causal events in 93.64\% of a set of generated accident records.


In this work, we develop a lightweight approach---based on a quantitative semantics---to identify two critical moments: the \emph{near-miss} and \emph{violation moments}. The violation moment is the point at which a specific property is violated. The near-miss moment occurs a few time steps before the violation, during which the violation is likely but has not yet happened. For example, if the property that the AV should follow is `avoid collisions with other objects', the violation moment is when the vehicle collides with another object, while the near-miss moment is when the vehicle fails to maintain a safe distance from other vehicles. The logic is straightforward: the violation moment represents the final outcome, whereas the near-miss moment could be the potential cause of the property violation.
We explain in the following how to identify them.



\noindent \emph{\textbf{Quantitative Semantics.}}
\label{sec:Quantitative_Semantics}
To locate critical moments, we need a method for quantitatively evaluating whether the current trace satisfies a given property. To achieve this, given a property specification $\varphi$ and a driving record for the ADS, \coolname first constructs a trace $\pi$ from the record by evaluating all variables relevant to $\varphi$ at each time point. 
An execution trace $\pi$ is a sequence of scenes, denoted as {\small $\pi=\langle \theta_0, \theta_1, \ldots, \theta_n \rangle$}. A scene $\theta$ is a tuple of the form {\small $\theta=( f_0, f_1, \ldots, f_x )$}, where each $f_i$ is the valuation of a variable.
These variables describe the status of the vehicle, traffic signal states, and traffic conditions. For example, variables such as `isOverTaking', `junctionAhead(n)', and `NPCAhead(n)' indicate whether the vehicle is changing lanes, whether there is a junction ahead within \emph{n} metres, and whether there is a vehicle ahead within \emph{n} metres, respectively.
For a detailed introduction to these variables, we refer the readers to \cite{Sun-Poskitt-et_al22a}.

Next, \coolname computes how `close' the ego vehicle will come to violating $\varphi$.
To measure how close a trace $\pi$ is to violating $\varphi$, we adopt a quantitative semantics~\cite{maler2004monitoring, deshmukh2017robust,nivckovic2020rtamt} that produces a numerical \emph{robustness} degree. 

\begin{definition}[Quantitative Semantics]\label{def:Quantitative_Semantics}
Given a trace $\pi$ and a formula $\varphi$, the quantitative semantics is defined as the robustness degree $\rho(\varphi, \pi,t)$, computed as follows.
Recall that propositions $\mu$ are of the form $f(x_0,x_1,\cdots,x_k) \sim 0$.
{\small\begin{equation*}
  \rho(\mu, \pi, t) =
    \begin{cases}
      -\pi_t(f(x_0,x_1,\cdots, x_k)) & \text{if $\sim$ is $\leq$ or $<$}\\
      \pi_t(f(x_0,x_1,\cdots, x_k)) & \text{if $\sim$ is $\geq$ or $>$}\\
      \mid \pi_t(f(x_0,x_1,\cdots, x_k)) \mid & \text{if $\sim$ is $\neq$}\\
      -\mid \pi_t(f(x_0,x_1,\cdots, x_k)) \mid & \text{if $\sim$ is $=$}
    \end{cases}       
\end{equation*}}

\noindent where $t$ is the time step and $\pi_t(e)$ is the valuation of expression $e$ at time $t$ in $\pi$.
{\small\begin{align*}
\rho(\neg\varphi,\pi,t) & = -\rho(\varphi,\pi,t) \\
\rho(\varphi_1 \land \varphi_2,\pi,t) & = \min\{\rho(\varphi_1,\pi,t),\rho(\varphi_2,\pi,t)\} \\
\rho(\varphi_1 \lor \varphi_2,\pi,t) & = \max\{\rho(\varphi_1,\pi,t),\rho(\varphi_2,\pi,t)\} \\
\rho(\varphi_1 \;\mathtt{U_I}\; \varphi_2,\pi,t) & = \sup_{t_1 \in t+\mathtt{I}} \min \{\rho(\varphi_2,\pi,t_1), \inf_{t_2 \in [t,t_1]} \rho(\varphi_1,\pi,t_2)\}\\
\rho(\Diamond_\mathtt{I}\varphi,\pi,t) & = \sup_{t'\in t+\mathtt{I}}\rho(\varphi,\pi,t') \\
\rho(\Box_\mathtt{I}\varphi,\pi,t) & = \inf_{t'\in t+\mathtt{I}}\rho(\varphi,\pi,t') \\
\rho(\bigcirc \varphi,\pi,t) & = \rho(\varphi,\pi,t+1)
\end{align*}}
where $t+I$ is the interval $[l+t,u+t]$ given $I=[l,u]$.
\qed
\end{definition}

Note that the smaller $\rho(\varphi,\pi,t)$ is, the closer $\pi$ is to violating $\varphi$.
If $\rho(\varphi,\pi,t) \leq 0$, $\varphi$ is violated.
We write $\rho(\varphi,\pi)$ to denote $\rho(\varphi,\pi,0)$; $\pi \vDash \varphi$ to denote $\rho(\varphi,\pi,t) > 0$; and $\pi \not \vDash \varphi$ to denote $\rho(\varphi,\pi,t) \leq 0$. Note that time is discrete in our setting. 

\begin{example}
\label{example:robustness calculation}
Let $\varphi = \Box (speed < 60)$, i.e.~the speed limit is $60$km/h.
Suppose $\pi$ is $\langle (speed \mapsto 0, \dots), (speed \mapsto 0.3, \dots), \cdots  (speed \mapsto 50, \dots) \rangle$ where the ego vehicle's max $speed$ is $50$km/h at the last time step.
We have $\rho(\varphi, \pi) = \rho(\varphi, \pi, 0) = min_{t \in [0, |\pi|]} ( 60 - \pi_t(speed) ) = 10$.
This means that trace $\pi$ satisfies $\varphi$, and the robustness value is 10. 
\qed
\end{example}

\noindent \emph{\textbf{Violation and Near-Miss Moments.}}
\label{sec:Violation and Near-Miss Moment}
With quantitative semantics, we can now introduce the method to locate the \emph{violation moment} and \emph{near-miss moment}.
Given a trace $\pi = \langle \pi_0, \cdots, \pi_n \rangle$, let $\pi^k$ denote the prefix $\langle  \pi_0, \cdots, \pi_k \rangle$, where $k \leq n$. Intuitively, $\pi^k$ represents the first $k$ time steps of the original trace $\pi$.
For the \emph{violation moment}, we identify the smallest $k$ that satisfies $\rho(\varphi, \pi^k) \leq 0$.
For the \emph{near-miss moment}, we adopt a user-customisable threshold $\delta$. We aim to identify a time step $k$ such that $\rho(\varphi, \pi^k) \leq \delta$ and there does not exist a time step $l$ such that $l < k$ and $\rho(\varphi, \pi^l) < \delta$. Note that $\delta$ is determined empirically  in our evaluation (see Section~\ref{sec:implementation_evaluation}).
Intuitively, $k$ is the earliest time step when the robustness value falls below the threshold $\delta$. We identify the time step $k$ using a sequential search, starting from $k=0$ and incrementing $k$ until we find a $k$ such that $\rho(\varphi, \pi^k) < \delta$.

\begin{example}
\label{example:timestep location}
Let $\varphi = \Box (speed < 60)$, i.e.~the speed limit is $60$km/h. Suppose the threshold $\delta = 5$
Suppose $\pi$ is $\langle \pi_0 = (speed \mapsto 0, \dots), \pi_1 = (speed \mapsto 1, \dots), \cdots \pi_{90} = (speed \mapsto 90, \dots) \rangle$ where the ego vehicle $speed$ is increasing over time steps and the ego vehicle's max $speed$ is $90$km/h at the last time step.
We have $\rho(\varphi, \pi) = \rho(\varphi, \pi, 0) = min_{t \in [0, |\pi|]} ( 60 - \pi_t(speed) ) = -30$. Hence, the specification is violated. 
The following are computed in sequence:
\begin{align*}
    & \rho(\varphi, \pi^{0}) = 60, \rho(\varphi, \pi^1)  = 59, \cdots,
     \rho(\varphi, \pi^{55}) = 5, \cdots, \\
    & \rho(\varphi, \pi^{59}) = 1, \rho(\varphi, \pi^{60}) = 0,  
    \rho(\varphi, \pi^{61}) = -1, \cdots
\end{align*}

To identify the violation moment, we find the smallest time step $k$ where $\rho(\varphi, \pi^k) \leq 0$. In this case, $k = 60$. Similarly, the smallest time step $k$ where $\rho(\varphi, \pi^k) \leq 5$ is $k = 55$.
Therefore, the violation moment occurs at time step 60, and the near-miss moment occurs at time step 55.
\qed
\end{example}

\subsection{Prompt Generation}
\tikzstyle{box} = [rectangle, draw, text width=1.8cm, text centered, rounded corners, minimum height=1.5cm]
\tikzstyle{arrow} = [thick,->,>=stealth]



The input for an MLLM can be in various formats, such as images, videos, audio, and text prompts. Given that MLLMs are trained on extensive datasets rich in knowledge, we anticipate they will `understand' the prompts we provide, much like an intelligent human. In this work, we utilise two types of prompts: visualisations of driving conditions and text descriptions to convey essential information not covered by the visualisations.

\begin{figure*}[htbp]
    \centering
    \includegraphics[width=\textwidth]{visualization.png}
    \caption{Visualisation of a scenario, which is provided to the MLLM along with an `overall prompt'} 
    \label{fig:visualisation} 
    \vspace{-0.6cm}
\end{figure*}

\noindent\emph{\textbf{Visualisations of Scenarios.}}
In the driving records of ADS trajectories, each time step contains extensive information such as the speed, acceleration, and steering angle of the AV, as well as the positions of other vehicles and pedestrians. While it is possible to describe this information in natural language, it does not provide a direct impression of the driving scenario. For example, given the positions of the AV and another background vehicle, it can be challenging to determine the exact direction of the background vehicle relative to the AV.

Fortunately, visualising the driving scenario can help alleviate this problem, and state-of-the-art ADSs, such as Apollo, offer this capability. Detailed information, including the positions of various objects, can be effectively conveyed through visualisation by displaying a grid map that shows the relative positions of each object.


An example of this visualisation is shown in Figure~\ref{fig:visualisation}. In this visualisation, the upper-left section, labelled `Vehicle Visualization', displays the current driving conditions of the AV (marked in blue), other vehicles (marked in green boxes), pedestrians (marked in yellow boxes), cyclists (marked in blue boxes), and unknown objects (marked in purple boxes). Each box includes numerical values indicating the distance to the AV and the current speed of the object. The predicted trajectory of each object is shown as a coloured line.
The lower-left section, labelled `Console', shows logs from the ADS, while the `Module Delay' section indicates the delay of each module.
The right section, labelled `Vehicle Dashboard', shows the current status of the AV and the detected status of traffic lights ahead. The `Pnc Monitor' section provides detailed information on the inner decisions of the planning and control modules.


This visualisation compactly encodes rich information in a human-friendly manner.
This is important for the transparency of the approach: it allows the MLLM's decisions to be based on the same high-level information that drivers work with, instead of (for example) low-level gradient-based discrepancies.



\noindent \emph{\textbf{Overall Prompt.}}
The overall prompt consists of two parts: the first part includes two images illustrating the visualisation of the \emph{violation moment} and the \emph{near-miss moment} as shown in Section~\ref{sec:Violation and Near-Miss Moment}, and the second part is a text prompt.

Our text prompts complement the ADS visualisation by providing necessary additional information. These prompts follow a specific workflow to enable automatic generation. 
First, we specify an identity for the MLLM:
\vspace{-0.15cm}\begin{promptbox}\small
\vspace{-0.15cm}
(identity) Suppose you are a driver.
\vspace{-0.15cm}
\end{promptbox}\vspace{-0.15cm}

Next, since the visualisations do not contain weather information, we provide information about the current weather conditions. For example, we state the following if there is no rain, fog,  or snow, and the visibility is more than 50 metres:
\vspace{-0.15cm}\begin{promptbox}\small
\vspace{-0.15cm}
(weather) There is nothing noteworthy about the weather.
\vspace{-0.15cm}
\end{promptbox}\vspace{-0.15cm}
To aid the MLLM's understanding of the provided image, we offer background details about the input image. This includes describing different aspects of the visualisation to help the MLLM to understand it:
\vspace{-0.15cm}\begin{promptbox}\small
\vspace{-0.15cm}
(background) In these pictures, the left side shows the visualisation of the driving record. 
The right side displays the status of the traffic light, vehicle speed, and steering angle. 
The green boxes indicate detected vehicles, yellow boxes indicate detected pedestrians, blue boxes indicate detected bicycles, and purple boxes indicate unknown objects.
\vspace{-0.15cm}
\end{promptbox}\vspace{-0.15cm}
Following this, we describe the property specification that the AV is supposed to satisfy, e.g.~the avoidance of collisions or adherence to traffic laws. 
Note that we use the original traffic laws from~\cite{China_traffic_law} as input when the property is an STL-based traffic law specification.
For example, if we are testing the property `no collisions', we would state:


\vspace{-0.15cm}\begin{promptbox}\small
\vspace{-0.15cm}
(rule) You are supposed to follow the following rule: Avoid collisions with other objects.
\vspace{-0.15cm}
\end{promptbox}\vspace{-0.15cm}
\vspace{-0.08cm}
Then, we specify the time gap between the \emph{violation moment} and the \emph{near-miss moment}, and clarify that the former image depicts the violation:
%. In this case, we clarify when the violation happens relative to the near-miss image:
%and the possible reason for the violation:
\vspace{-0.15cm}\begin{promptbox}\small
\vspace{-0.15cm}
(sequence) The second picture was taken 4 seconds later than the first picture, capturing the moment when the rule violation occurred.
\vspace{-0.15cm}
\end{promptbox}\vspace{-0.15cm}
Finally, we provide some initial settings of the current ADS to assist the MLLM in making decisions:
\vspace{-0.15cm}\begin{promptbox}\small
\vspace{-0.15cm}
(default) In the original ADS, the initial settings are: $max\ planning\ speed = 72 km/h$, ...
\vspace{-0.15cm}
\end{promptbox}\vspace{-0.15cm}







\subsection{$\mu$Drive Script Generation}
The driving strategy repairs generated by the MLLM must be in the correct format, i.e.~pertaining to the $\mu$Drive grammar in Section~\ref{sec:udrive grammar}.
This is challenging to achieve with a language model alone, as they have the potential to hallucinate and make mistakes.
MLLMs such as ChatGPT-4, however, have added support for function calling~\cite{gpt_function_calling}, which enables users to connect the models with external tools and design integrated workflows. Additionally, OpenAI's introduction of `Structured Outputs' ensures that the arguments generated by the model adhere precisely to a specified JSON Schema, as defined by the user in the function call.

In \coolname, we implemented function calling based on a structured JSON Schema that describes the syntax of $\mu$Drive programs. This schema guides the MLLM to produce outputs that are always structured as syntactically valid $\mu$Drive programs, including all parameters and constructs mandated by the full $\mu$Drive grammar. By leveraging this function, we achieve a reliable and structured generation of $\mu$Drive scripts that align with expected syntax. For an example of function calling with our JSON Schema, please see our repository~\cite{source_code}.







Our function is designed with several key principles in mind:
1) \emph{Structural Integrity:} we ensure that the structure of the output $\mu$Drive program always adheres to the syntax of $\mu$Drive. Specifically, each program must include one trigger, zero or more conditions, one or more actions, and at most one exit trigger. The sequence is strictly enforced, i.e.~trigger, conditions, actions, followed by the exit trigger if it exists.
2) \emph{Comprehensive Descriptions:} to help the MLLM fully understand the meaning and functionality of each element, we add detailed descriptions to all events, conditions, and actions in natural language. These descriptions are sourced from the official documents of ADS and $\mu$Drive to ensure accuracy and clarity.
3) \emph{Clear Parameter Definitions:} the unit parameters within events, conditions, and actions are clearly defined to avoid any potential misunderstandings. This precision helps the MLLM to generate accurate and contextually appropriate programs.
By adhering to these design principles, integrating $\mu$Drive with an MLLM facilitates the creation of actionable and precise programs, that can then be applied in subsequent deployments of the ADS to improve its driving strategy.
For more information on our prompts, function calling, and implementation details, please refer to the 
source code~\cite{source_code}.


