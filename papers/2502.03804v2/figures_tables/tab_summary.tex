\begin{table*}[t]
\caption{Research Questions and Key Findings}
\label{tab_summary}
\centering
% \resizebox{\textwidth}{!}{
\begin{tabular}{>{\raggedright\arraybackslash}p{0.08\linewidth}>{\raggedright\arraybackslash}p{0.28\linewidth}>{\raggedright\arraybackslash}p{0.28\linewidth}>{\raggedright\arraybackslash}p{0.28\linewidth}}
\hline
 & \textbf{RQ1: How does a QA-based response-writing support approach affect workers’ email-replying process?} & \textbf{RQ2: How does a QA-based response-writing support approach affect the quality of the email response?} & \textbf{RQ3: How does a QA-based response-writing support approach affect the perceived relationship between email sender and recipient?} \\ \hline
\textbf{Key Findings} & 1. QA-based approach \textbf{reduced workload} for email comprehension and prompt creation and \textbf{improved work efficiency}. (H1-a, supported; H1-b, supported, Sec.~\ref{sec:result1_efficiency},~\ref{sec:result1_prompt_character_counts},~\ref{sec:result1_cognitive_load},~\ref{sec:result1_difficulty_in_understanding},~\ref{sec:result1_interview_RQ1},~\ref{sec:result2_efficiency}) 

2. QA-based approach \textbf{reduced the difficulty} of initiating the email replying task. (H1-d, supported, Sec.~\ref{sec:result1_initiating},~\ref{sec:result1_interview_RQ1},~\ref{sec:result2_initiating_the_action})

3. QA-based approach \textbf{decreased the sense of agency and control}. (H1-e, supported, Sec.~\ref{sec:result1_agency},~\ref{sec:result1_interview_RQ1},~\ref{sec:result2_agency_control})

4. QA-based approach \textbf{improved satisfaction} with the emails they wrote and willingness to use ResQ in the future. (H1-c, supported, Sec.~\ref{sec:result1_satisfaction},~\ref{sec:result1_interview_RQ1},~\ref{sec:result2_efficiency})& Writing emails with QA-based approach and Prompt-based approach led to \textbf{increased email quality} than No-AI condition. (H2, partially supported, Sec.~\ref{sec:result1_quality},~\ref{sec:result1_interview_RQ2}~\ref{sec:result2_quality})& 1. Writing emails with QA-based approach \textbf{\blue{did not lead to improved perceived impression of users by their counterparts}}. (H3-a, not supported, Sec.~\ref{sec:result1_impression},~\ref{sec:result2_self-presentation})

2. Writing emails with QA-based approach led to \textbf{increased psychological distance} between users and their counterparts than No-AI condition. (H3-b, partially supported, Sec.~\ref{sec:result1_psychological_distance},~\ref{sec:result1_interview_RQ3},~\ref{sec:result2_psychological_distance})\\ \hline
\end{tabular}
% }
\Description{This table summarizes the three research questions (RQs) investigated in the study and highlights the key findings associated with each. RQ1: How does a QA-based response-writing support approach affect workers’ email-replying process? Key Findings: 1. The QA-based approach reduced workload for email comprehension and prompt creation, leading to improved work efficiency. This supports hypotheses H1-a and H1-b. 2. It reduced the difficulty of initiating the email replying task, supporting hypothesis H1-d. 3. The approach decreased the sense of agency and control among users, supporting hypothesis H1-e. 4. Users experienced improved satisfaction with the emails they wrote and showed a greater willingness to use ResQ in the future, supporting hypothesis H1-c. RQ2: How does a QA-based response-writing support approach affect the quality of the email response? Key Findings: Writing emails using both the QA-based and prompt-based approaches led to an increase in email quality compared to the No-AI condition. This partially supports hypothesis H2. RQ3: How does a QA-based response-writing support approach affect the perceived relationship between email sender and recipient? Key Findings: 1. Writing emails with QA-based approach did not lead to improved perceived impression of users by their counterparts, meaning hypothesis H3-a was not supported. 2. Writing emails with the QA-based approach led to an increase in psychological distance between users and their counterparts compared to the No-AI condition. This partially supports hypothesis H3-b.}
\end{table*}


% \textbf{Summary} & Workers evaluated the system’s benefits (improvements in efficiency, cognitive load, and satisfaction), accepted a certain reduction in the agency, and showed a willingness to use it in the future. & It became possible to create responses that appropriately addressed email requests while maintaining politeness. (H2, supported) & ResQ improved workers' impression. Perceived psychological distance from others depends on a balance between a sense of agency and communication satisfaction. \\ \hline