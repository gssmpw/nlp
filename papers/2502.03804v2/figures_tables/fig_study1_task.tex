% \begin{figure*}[t]
% \centering
% \includegraphics[width=\textwidth]{figure/study1_task.png}
% \caption{Interface used in study 1. At the top, a box displays the email metadata (received date, subject, sender) along with the scenario. The email content is shown in a separate box below this. The next box contains supplementary information, which participants can use to determine the direction of their reply. Further down, a reply box is provided for participants to compose their responses. At the very bottom, a "Send Reply" button allows participants to finalize and submit their replies.}
% \label{fig_task}
% \Description{This figure shows the interface used in Study 1, consisting of several key components that guide the user through responding to an email. At the top of the interface, a section labeled Email Metadata and Scenario displays important details about the email, including the received date, subject, and sender. Additionally, a brief scenario description is provided, which helps set the context for the user's reply (e.g., the sender is their manager, and they are being asked to take on a specific role). Beneath this, the Email Content section presents the body of the email, where the sender requests the user to take on the role of a fixed asset committee member. The email outlines the main task involved, making the user aware of their potential responsibilities. Following the email content, a box labeled Supplementary Information offers additional guidance for the user. In this instance, it suggests that the user is willing to accept the request but wants more details regarding the responsibilities, providing context to shape the reply. Further down, a Reply Box is available for the user to compose their response, taking into consideration both the email content and the supplementary information provided. The user can freely edit and tailor their message in this section. Finally, at the bottom, a Send Reply Button allows the user to finalize and submit their response once they are satisfied with the content of their reply. This layout ensures that users can easily review the relevant information, use context-specific suggestions, and compose their replies within the same interface for a streamlined response process.}
% \end{figure*}