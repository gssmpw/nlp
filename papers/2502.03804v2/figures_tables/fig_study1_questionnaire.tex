\begin{figure*}[t]
\centering
\includegraphics[width=\textwidth]{figure/study1_2.pdf}
\caption{Summary of Likert scale responses. \red{Measurements H2 and H3-a were assessed by third-party evaluators rather than the participants themselves.} The significant differences between conditions were from post-hoc analysis after one-way repeated measure ANOVA or the Friedman test (* and \red{**} indicate the significance found at levels of 0.05 and 0.01, respectively).}
\label{fig_study1_questionnaire}
\Description{The figure consists of box plots comparing three experimental conditions, No-AI, Prompt-based, and QA-based, across multiple subjective measures related to email task performance. The box plots represent the distribution of responses across these conditions, with p-values indicating statistically significant differences between them. H1-b: Difficulty in Understanding Email Content: No-AI: The median is 4, with a first quartile at 3.25 and a third quartile at 6.25, with a minimum of 1 and a maximum of 7. Prompt-based: The median is 4, with a first quartile at 4 and a third quartile at 5.25, with a minimum of 1 and a maximum of 7. QA-based: The median is 6, with a first quartile at 5 and a third quartile at 7, with a minimum of 4 and a maximum of 7. Significant differences exist between Prompt-based and QA-based (p < 0.01) and between No-AI and QA-based (p < 0.01). H1-c: Satisfaction with Completing Tasks: No-AI: The median is 2.5, with a first quartile at 1.875 and a third quartile at 3.625, with a minimum of 1 and a maximum of 5. Prompt-based: The median is 5.25, with a first quartile at 4.375 and a third quartile at 6.5, with a minimum of 3.5 and a maximum of 7. QA-based: The median is 6.5, with a first quartile at 6 and a third quartile at 6.625, with a minimum of 5 and a maximum of 7. Significant differences exist between No-AI and QA-based (p < 0.01), between Prompt-based and QA-based (p < 0.01), and between Prompt-based and QA-based (p < 0.05). H1-d: Difficulty for Task Initiation: No-AI: The median is 5, with a first quartile at 5 and a third quartile at 6, with a minimum of 1 and a maximum of 7. Prompt-based: The median is 3, with a first quartile at 1.75 and a third quartile at 4, with a minimum of 1 and a maximum of 5. QA-based: The median is 2, with a first quartile at 1 and a third quartile at 2, with a minimum of 1 and a maximum of 3. Significant differences exist between No-AI and Prompt-based (p < 0.01), between Prompt-based and QA-based (p < 0.01), and between No-AI and QA-based (p < 0.01). H1-e: Sense of Agency: No-AI: The median is 7, with a first quartile at 6.75 and a third quartile at 7, with a minimum of 6 and a maximum of 7. Prompt-based: The median is 4, with a first quartile at 4 and a third quartile at 4.25, with a minimum of 3 and a maximum of 5. QA-based: The median is 2.5, with a first quartile at 1.75 and a third quartile at 3.25, with a minimum of 1 and a maximum of 5. Significant differences exist between No-AI and Prompt-based (p < 0.01), between Prompt-based and QA-based (p < 0.01), and between No-AI and QA-based (p < 0.01). H1-e: Sense of Control: No-AI: The median is 7, with a first quartile at 7 and a third quartile at 7, with a minimum of 5 and a maximum of 7. Prompt-based: The median is 5, with a first quartile at 3.5 and a third quartile at 5, with a minimum of 1 and a maximum of 6. QA-based: The median is 3.5, with a first quartile at 2.75 and a third quartile at 4, with a minimum of 1 and a maximum of 6. Significant differences exist between No-AI and Prompt-based (p < 0.01), between Prompt-based and QA-based (p < 0.01), and between No-AI and QA-based (p < 0.01). H2: Perceived Quality of the Email by Evaluators: No-AI: The median is 5.22, with a first quartile at 4.43 and a third quartile at 5.42, with a minimum of 3.78 and a maximum of 5.83. Prompt-based: The median is 5.61, with a first quartile at 5.47 and a third quartile at 6.07, with a minimum of 4.56 and a maximum of 6.56. QA-based: The median is 5.81, with a first quartile at 5.43 and a third quartile at 5.99, with a minimum of 4.78 and a maximum of 6.50. Significant differences exist between No-AI and Prompt-based (p < 0.05) and between No-AI and QA-based (p < 0.01). H3-a: Perceived Impression of Participants by Evaluators: No-AI: The median is 4.17, with a first quartile at 3.31 and a third quartile at 4.63, with a minimum of 2.42 and a maximum of 5.67. Prompt-based: The median is 5.17, with a first quartile at 4.42 and a third quartile at 5.33, with a minimum of 3.92 and a maximum of 6.00. QA-based: The median is 4.92, with a first quartile at 4.60 and a third quartile at 5.42, with a minimum of 4.00 and a maximum of 5.58. Significant differences do not exist. H3-b: Psychological Distance: No-AI: The median overlap score is 5, with a first quartile at 2.75 and a third quartile at 6.25. The minimum score is 1, and the maximum score is 7. Prompt-based: The median overlap score is 4, with a first quartile at 2.75 and a third quartile at 4. The minimum score is 1, and the maximum score is 5. QA-based: The median overlap score is 1.75, with a first quartile at 3 and a third quartile at 3.5. The minimum score is 1, and the maximum score is 7. Significant differences exist between No-AI and QA-based (p < 0.05). Each box plot represents the spread of participant responses, with the whiskers showing the variability outside the upper and lower quartiles. The statistical differences (p-values) highlight significant findings between different experimental conditions.}
\end{figure*}