\begin{figure*}[t]
\centering
\includegraphics[width=\textwidth]{figure/study1_1.pdf}
\caption{Results of participants' efficiency and cognitive load of replying to emails. Left: Efficiency for replying to emails. Middle: Prompt character count. Right: Cognitive load for replying to emails. The significant differences between conditions were from post-hoc analysis after doing one-way repeated measure ANOVA.}
\label{fig_study1_efficiency_and_cognitiveLoad}
\Description{The figure consists of three box plots, each representing different comparisons of three experimental conditions: No-AI, Prompt-based, and QA-based. Each box plot compares a specific measure between the conditions. The p-values indicating statistical significance between different conditions are also labeled above the plots. Efficiency of Replying to Emails: Three conditions are compared: No-AI, Prompt-based, and QA-based. The No-AI group has a median of 0.65, with a first quartile at 0.51 and a third quartile at 1.05, with a minimum of 0.34 and a maximum of 1.55. The Prompt-based group has a median of 1.50, with the first quartile at 0.87 and the third quartile at 2.12, with a minimum of 0.70 and a maximum of 2.89. The QA-based group has a median of 1.89, with a first quartile at 1.35 and a third quartile at 2.12, with a minimum of 0.90 and a maximum of 3.73. P-values indicate significant differences between groups: between No-AI and Prompt-based (p = 0.013), between No-AI and QA-based (p = 0.002), and between Prompt-based and QA-based (p = 0.046). Prompt Character Count: Only two conditions are compared: Prompt-based and QA-based. The Prompt-based group has a median of 37.42 characters, with a first quartile at 28.75 and a third quartile at 53.50, with a minimum of 11.67 and a maximum of 64.50. The QA-based group has a median of 25.50 characters, with a first quartile at 19.00 and a third quartile at 31.67, with a minimum of 14.33 and a maximum of 47.83. The p-value indicates a significant difference between the two groups (p = 0.01). Raw TLX: This plot compares three conditions: No-AI, Prompt-based, and QA-based. The No-AI group has a median score of 3.70, with a first quartile of 2.93 and a third quartile at 4.35, with a minimum of 2.20 and a maximum of 4.90. The Prompt-based group shows a median of 2.40, with a first quartile at 2.03 and a third quartile at 2.70, with a minimum of 1.00 and a maximum of 4.80. The QA-based group has a median of 2.10, with a first quartile of 1.93 and a third quartile at 2.48, with a minimum of 0.70 and a maximum of 4.40. The p-values indicate significant differences between No-AI and Prompt-based (p = 0.017), between No-AI and QA-based (p = 0.008), and between Prompt-based and QA-based (p = 0.018). Each box plot represents the distribution of values for the respective metric, and the whiskers show the variability outside the upper and lower quartiles. The statistical differences (p-values) highlight where the comparisons between conditions are significant.}
\end{figure*}