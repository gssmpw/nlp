\begin{figure*}[t]
\centering
\includegraphics[width=\textwidth]{figure/interface.pdf}
\caption{Interface of ResQ. On the left, the content of the email is displayed, with an editor and a ``Reply'' button below for sending a reply. In the center, questions and options for users are shown, allowing the creation of custom options if needed. Additionally, the section of the email corresponding to the selected question is highlighted. On the right, fields are provided to customize the reply generated by the LLM, including options to specify the relationship with the counterpart and buttons to choose the formality, tone, and length of the email. A free-text input field and a "Generate Reply" button are also below.
}
\label{fig_interface}
\Description{This figure illustrates the interface of the ResQ system, which is divided into three main sections. On the left side (A, B), the content of the incoming email is displayed. The email includes important information, such as the sender, subject, and body text. Key sections of the email are highlighted based on the questions generated by the system, helping the user focus on relevant points. Below the email, there is an editor where the user can compose their reply, with a "Reply" button (H) available to send the response once it's ready. In the center section (C, D), the system displays questions generated by the AI, which are intended to assist the user in composing their reply. These questions correspond to specific parts of the email, and as the user answers them, the relevant section in the email is highlighted (A). Users can also customize responses by adding new options if needed (D). user can specify their relationship with the email recipient (e.g., professor or student), adjust the formality and tone of the response, and select the desired length of the reply. An additional free-text input field is available for further customization requests (E). Once all preferences are set, the user can click the "Generate Reply" button (F) to produce a draft response based on their inputs.}
\end{figure*}