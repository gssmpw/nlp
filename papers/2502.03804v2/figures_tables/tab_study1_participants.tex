\red{
\begin{table*}[t]
% \caption{\red{Backgrounds of participants in Study 1, including age, job roles, email experience, frequency of email sending and AI tool usage, and use of AI for email purposes.}}
\caption{Backgrounds of participants in Study 1, including age, job roles, email experience, frequency of email sending and AI tool usage, and use of AI for email purposes. \blue{Some fields are marked as - due to missing responses from participants.}}
\Description{The table provides an overview of participants in Study 1, detailing their demographic information, email usage habits, and AI tool usage patterns. The participants, ranging in age from 20 to 57, include university students, office workers, a part-time worker, and individuals categorized under "other" occupations. Both male and female participants are represented, reflecting a diverse group in terms of age, occupation, and technological engagement. Participant P1 is a 34-year-old male office worker with 20 years of email experience. He typically replies to 7 emails per week but rarely uses AI tools, and he never utilizes AI for email-related tasks. P2, a 22-year-old male university student, has 5 years of email experience and actively engages with emails, replying to more than 21 per week. He uses AI tools daily, with AI assisting in 50–80\% of his email-related tasks. Similarly, P3, a 22-year-old female university student, reports no specific email experience or weekly email activity but uses AI tools frequently, relying on AI for 50–80\% of her email tasks. P4, a 21-year-old female university student with 4 years of email experience, replies to 0–2 emails weekly. She uses AI tools frequently but for less than 20\% of her email-related activities. P5, a 20-year-old male university student, has no reported email experience or weekly email activity. He rarely uses AI tools and does not use them for email purposes. P6, a 38-year-old male office worker, has 3 years of email experience and replies to 0–2 emails weekly. He rarely engages with AI tools and never applies them to email tasks. P7, a 31-year-old female categorized as "unemployed," has 12 years of email experience and replies to 3–4 emails per week. She never uses AI tools, either for general purposes or for email tasks. In contrast, P8, a 25-year-old male university student with 4 years of email experience, replies to 0–2 emails per week. He uses AI tools frequently, with AI assisting in 50–80\% of his email-related activities. P9, a 39-year-old female office worker, has 20 years of email experience and replies to over 21 emails per week. She uses AI tools frequently but never applies them to email tasks. P10, a 24-year-old male university student, has no reported email experience or weekly activity. He uses AI tools daily, although only for less than 20\% of his email-related tasks. P11, a 22-year-old female university student with 4 years of email experience, replies to 0–2 emails weekly. She rarely engages with AI tools and does not use them for email tasks. Finally, P12, a 57-year-old female office worker with 20 years of email experience, replies to 0–2 emails weekly. She uses AI tools frequently but never for email-related purposes.}
\label{tab_study1_participants}
\red{
\begin{tabular}{cccccccc}
\hline
ID   & Gender & Age & Job                  &  Email Experience&Emails/Week & AI Tool Usage& AI for Email Usage    \\ \hline
P1   & M& 34  & Office Worker        &  20 years&7           & Rarely         & Never                 \\
P2   & M& 22  & Univ. Student        &  5 years&21+         & Daily              & 50–80\%\\
P3   & F& 22  & Univ. Student        &  -&-& Frequently& 50–80\%\\
P4   & F& 21  & Univ. Student        &  4 years&0–2         & Frequently& <20\%\\
P5   & M& 20  & Univ. Student        &  -&-           & Rarely         & <20\%\\
P6   & M& 38  & Office Worker        &  3 years&0–2         & Rarely         & Never                 \\
P7   & F& 31  & \blue{Unemployed}                &  12 years&3–4         & Never         & Never                 \\
P8   & M& 25  & Univ. Student        &  4 years&0–2         & Frequently& 50–80\%\\
P9   & F& 39  & Office Worker        &  20 years&21+         & Frequently& Never                 \\
P10  & M& 24  & Univ. Student        &  -&-           & Daily              & <20\%\\
P11  & F& 22  & \blue{Univ. Student}     &  4 years&0–2         & Rarely         & Never                 \\
P12  & F& 57  & Office Worker        &  20 years&0–2         & Frequently& Never                 \\ \hline
\end{tabular}
}
\end{table*}
}

% \begin{itemize}
%     \item Job categories such as ‘Office Worker’, ‘Part-time Worker’, and ‘Other’ were chosen based on predefined options provided to participants, as shown in the survey (e.g., office workers, contract workers, freelancers, etc.).
%     \item Fields marked as ‘-’ indicate missing responses, as participants did not provide the requested information.
%     \item ‘Part-time Worker’ refers to individuals employed on a part-time basis, while ‘Other’ includes occupations such as freelance or non-traditional job roles.
% \end{itemize}