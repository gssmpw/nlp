% todo
% Study 1において、参加者が無視した質問と答えた質問についての定性的な結果を追加
% Study 1において、non-structured/structuredなメールのそれぞれにおける質問の出力例を記載し、定性的な結果を追加
\begin{figure*}[t]
\centering
\includegraphics[width=\textwidth]{figure/study1_1.pdf}
\caption{Results of participants' efficiency and cognitive load of replying to emails. Left: Efficiency for replying to emails. Middle: Prompt character count. Right: Cognitive load for replying to emails. The significant differences between conditions were from post-hoc analysis after doing one-way repeated measure ANOVA.}
\label{fig_study1_efficiency_and_cognitiveLoad}
\Description{The figure consists of three box plots, each representing different comparisons of three experimental conditions: No-AI, Prompt-based, and QA-based. Each box plot compares a specific measure between the conditions. The p-values indicating statistical significance between different conditions are also labeled above the plots. Efficiency of Replying to Emails: Three conditions are compared: No-AI, Prompt-based, and QA-based. The No-AI group has a median of 0.65, with a first quartile at 0.51 and a third quartile at 1.05, with a minimum of 0.34 and a maximum of 1.55. The Prompt-based group has a median of 1.50, with the first quartile at 0.87 and the third quartile at 2.12, with a minimum of 0.70 and a maximum of 2.89. The QA-based group has a median of 1.89, with a first quartile at 1.35 and a third quartile at 2.12, with a minimum of 0.90 and a maximum of 3.73. P-values indicate significant differences between groups: between No-AI and Prompt-based (p = 0.013), between No-AI and QA-based (p = 0.002), and between Prompt-based and QA-based (p = 0.046). Prompt Character Count: Only two conditions are compared: Prompt-based and QA-based. The Prompt-based group has a median of 37.42 characters, with a first quartile at 28.75 and a third quartile at 53.50, with a minimum of 11.67 and a maximum of 64.50. The QA-based group has a median of 25.50 characters, with a first quartile at 19.00 and a third quartile at 31.67, with a minimum of 14.33 and a maximum of 47.83. The p-value indicates a significant difference between the two groups (p = 0.01). Raw TLX: This plot compares three conditions: No-AI, Prompt-based, and QA-based. The No-AI group has a median score of 3.70, with a first quartile of 2.93 and a third quartile at 4.35, with a minimum of 2.20 and a maximum of 4.90. The Prompt-based group shows a median of 2.40, with a first quartile at 2.03 and a third quartile at 2.70, with a minimum of 1.00 and a maximum of 4.80. The QA-based group has a median of 2.10, with a first quartile of 1.93 and a third quartile at 2.48, with a minimum of 0.70 and a maximum of 4.40. The p-values indicate significant differences between No-AI and Prompt-based (p = 0.017), between No-AI and QA-based (p = 0.008), and between Prompt-based and QA-based (p = 0.018). Each box plot represents the distribution of values for the respective metric, and the whiskers show the variability outside the upper and lower quartiles. The statistical differences (p-values) highlight where the comparisons between conditions are significant.}
\end{figure*}
\begin{figure*}[t]
\centering
\includegraphics[width=\textwidth]{figure/study1_2.pdf}
\caption{Summary of Likert scale responses. \red{Measurements H2 and H3-a were assessed by third-party evaluators rather than the participants themselves.} The significant differences between conditions were from post-hoc analysis after one-way repeated measure ANOVA or the Friedman test (* and \red{**} indicate the significance found at levels of 0.05 and 0.01, respectively).}
\label{fig_study1_questionnaire}
\Description{The figure consists of box plots comparing three experimental conditions, No-AI, Prompt-based, and QA-based, across multiple subjective measures related to email task performance. The box plots represent the distribution of responses across these conditions, with p-values indicating statistically significant differences between them. H1-b: Difficulty in Understanding Email Content: No-AI: The median is 4, with a first quartile at 3.25 and a third quartile at 6.25, with a minimum of 1 and a maximum of 7. Prompt-based: The median is 4, with a first quartile at 4 and a third quartile at 5.25, with a minimum of 1 and a maximum of 7. QA-based: The median is 6, with a first quartile at 5 and a third quartile at 7, with a minimum of 4 and a maximum of 7. Significant differences exist between Prompt-based and QA-based (p < 0.01) and between No-AI and QA-based (p < 0.01). H1-c: Satisfaction with Completing Tasks: No-AI: The median is 2.5, with a first quartile at 1.875 and a third quartile at 3.625, with a minimum of 1 and a maximum of 5. Prompt-based: The median is 5.25, with a first quartile at 4.375 and a third quartile at 6.5, with a minimum of 3.5 and a maximum of 7. QA-based: The median is 6.5, with a first quartile at 6 and a third quartile at 6.625, with a minimum of 5 and a maximum of 7. Significant differences exist between No-AI and QA-based (p < 0.01), between Prompt-based and QA-based (p < 0.01), and between Prompt-based and QA-based (p < 0.05). H1-d: Difficulty for Task Initiation: No-AI: The median is 5, with a first quartile at 5 and a third quartile at 6, with a minimum of 1 and a maximum of 7. Prompt-based: The median is 3, with a first quartile at 1.75 and a third quartile at 4, with a minimum of 1 and a maximum of 5. QA-based: The median is 2, with a first quartile at 1 and a third quartile at 2, with a minimum of 1 and a maximum of 3. Significant differences exist between No-AI and Prompt-based (p < 0.01), between Prompt-based and QA-based (p < 0.01), and between No-AI and QA-based (p < 0.01). H1-e: Sense of Agency: No-AI: The median is 7, with a first quartile at 6.75 and a third quartile at 7, with a minimum of 6 and a maximum of 7. Prompt-based: The median is 4, with a first quartile at 4 and a third quartile at 4.25, with a minimum of 3 and a maximum of 5. QA-based: The median is 2.5, with a first quartile at 1.75 and a third quartile at 3.25, with a minimum of 1 and a maximum of 5. Significant differences exist between No-AI and Prompt-based (p < 0.01), between Prompt-based and QA-based (p < 0.01), and between No-AI and QA-based (p < 0.01). H1-e: Sense of Control: No-AI: The median is 7, with a first quartile at 7 and a third quartile at 7, with a minimum of 5 and a maximum of 7. Prompt-based: The median is 5, with a first quartile at 3.5 and a third quartile at 5, with a minimum of 1 and a maximum of 6. QA-based: The median is 3.5, with a first quartile at 2.75 and a third quartile at 4, with a minimum of 1 and a maximum of 6. Significant differences exist between No-AI and Prompt-based (p < 0.01), between Prompt-based and QA-based (p < 0.01), and between No-AI and QA-based (p < 0.01). H2: Perceived Quality of the Email by Evaluators: No-AI: The median is 5.22, with a first quartile at 4.43 and a third quartile at 5.42, with a minimum of 3.78 and a maximum of 5.83. Prompt-based: The median is 5.61, with a first quartile at 5.47 and a third quartile at 6.07, with a minimum of 4.56 and a maximum of 6.56. QA-based: The median is 5.81, with a first quartile at 5.43 and a third quartile at 5.99, with a minimum of 4.78 and a maximum of 6.50. Significant differences exist between No-AI and Prompt-based (p < 0.05) and between No-AI and QA-based (p < 0.01). H3-a: Perceived Impression of Participants by Evaluators: No-AI: The median is 4.17, with a first quartile at 3.31 and a third quartile at 4.63, with a minimum of 2.42 and a maximum of 5.67. Prompt-based: The median is 5.17, with a first quartile at 4.42 and a third quartile at 5.33, with a minimum of 3.92 and a maximum of 6.00. QA-based: The median is 4.92, with a first quartile at 4.60 and a third quartile at 5.42, with a minimum of 4.00 and a maximum of 5.58. Significant differences do not exist. H3-b: Psychological Distance: No-AI: The median overlap score is 5, with a first quartile at 2.75 and a third quartile at 6.25. The minimum score is 1, and the maximum score is 7. Prompt-based: The median overlap score is 4, with a first quartile at 2.75 and a third quartile at 4. The minimum score is 1, and the maximum score is 5. QA-based: The median overlap score is 1.75, with a first quartile at 3 and a third quartile at 3.5. The minimum score is 1, and the maximum score is 7. Significant differences exist between No-AI and QA-based (p < 0.05). Each box plot represents the spread of participant responses, with the whiskers showing the variability outside the upper and lower quartiles. The statistical differences (p-values) highlight significant findings between different experimental conditions.}
\end{figure*}
% \begin{figure*}[t]
\centering
\includegraphics[width=\textwidth]{figure/study1_IOS.pdf}
\caption{Inclusion of Other in the Self (IOS). The diagram above the x-axis is an example of what participants were shown when responding to the questionnaire. The degree of overlap between the two circles represents the psychological distance between oneself and others.}
\label{fig_study1_IOS}
\Description{This figure represents the "Inclusion of Other in the Self (IOS)" scale, which is used to measure psychological closeness or relational intimacy. The diagram depicts two circles, labeled "Self" and "Other," with varying degrees of overlap. Participants were asked to choose the level of overlap that best represented their relationship with another person. On the far left (score 1), the circles are completely separate, indicating a significant psychological distance between the self and the other. In the middle (score 4), the circles partially overlap, suggesting a moderate level of psychological closeness. On the far right (score 7), the circles almost completely overlap, representing a very close and intimate relationship between the self and the other.}
\end{figure*}
\section{Results of Study 1}
\red{Here, we first present the quantitative results of Study 1 for each research question. 
Subsequently, we include comments provided by the participants.}
\subsection{Participants' Email-Replying Process (RQ1)}
\label{sec:result1_RQ1}
\subsubsection{Efficiency of Replying to Emails (H1-a)}
\label{sec:result1_efficiency}
First, we compared the efficiency of replying to emails across three conditions.
After checking the data normality assumption with the Shapiro-Wilk test, the result of one-way repeated measures ANOVA showed that there was a significant difference in participants' efficiency of replying to emails across three conditions ($F[2, 22]=14.8$, $p<0.001$\red{, $\eta_p^2=0.57$}). 
Post-hoc analysis with Holm correction revealed that participants' efficiency of replying to emails in the QA-based condition was significantly higher compared to both the No-AI $(t(11), p=0.002\red{, d=1.38})$ and the Prompt-based $(t(11), p=0.046\red{, d=0.65})$ conditions.
Thus, H1-a was supported.
The QA-based approach enhanced participants’ email replying efficiency.
% 次にPrompt Character Countsを計算した
% QA-based < Prompt-based

\subsubsection{Prompt Character Counts (H1-a)}
\label{sec:result1_prompt_character_counts}
In order to understand how participants wrote prompts differently, we calculated the prompt character counts.
After the Shapiro-Wilk test, the paired t-test revealed that participants in the QA-based condition typed significantly fewer characters in their prompts than those in the Prompt-based condition $(t(11), p=0.010\red{, d=0.90})$.

\subsubsection{Cognitive Load for Replying to Emails (H1-b)}
\label{sec:result1_cognitive_load}
% Raw-TLXの計算結果を図1に示す
The results of the Raw-TLX are shown in Fig.~\ref{fig_study1_efficiency_and_cognitiveLoad}.
According to the one-way repeated measures ANOVA with Greenhouse-Geisser correction, there was a significant difference in participants' cognitive load for replying to emails among the three conditions $(F[1.1, 12.1]=12.6, p=0.003\red{, \eta_p^2=0.53})$. 
Post-hoc analysis with Holm correction revealed that participants' cognitive load for replying to emails in the QA-based condition was significantly lower compared to both the No-AI $(t(11), p=0.008\red{, d=1.12})$ and Prompt-based $(t(11), p=0.018\red{, d=0.81})$ conditions.
% これらの結果は、H1bの妥当性について示唆している。
Therefore, H1-b was supported.
The QA-based approach reduced participants’ cognitive workload while replying to the emails.

\subsubsection{\red{Difficulty in Understanding Email Content (H1-b)}}
\label{sec:result1_difficulty_in_understanding}
% また、H1-bはアンケート調査の結果によっても裏付けられた
Additionally, H1-b was also supported by the questionnaire survey results (Fig.~\ref{fig_study1_questionnaire} H1-b).
% easily understand content of email: QA-based > Prompt-based, No-AI
The Friedman test revealed a significant difference among the three conditions in terms of understanding the sender's intent and requests $(\chi^2(2)=10.6, p=0.005\red{, W=0.44})$. 
Post-hoc analysis using the Durbin-Conover test with Holm correction showed that participants in the QA-based condition found it significantly easier to understand the sender's intent and requests compared to those in both No-AI $(p=0.003\red{, r=0.61})$ and Prompt-based $(p=0.005\red{, r=0.73})$ conditions.
% system helped decide response strategy: QA-based > Prompt-based
% Furthermore, the Wilcoxon signed-rank test indicated that participants felt significantly more supported in determining the direction of their responses in the QA-based condition than in the Prompt-based condition $(p=0.010)$.

\subsubsection{Satisfaction with Completing Task (H1-c)}
\label{sec:result1_satisfaction}
The results of the satisfaction with completing participants' tasks are shown in Fig.~\ref{fig_study1_questionnaire}, H1-c.
\red{The two items measuring satisfaction showed high internal consistency, with a Cronbach's Alpha of $0.889$.}
After checking the data normality assumption with the Shapiro-Wilk test, the result of one-way repeated measures ANOVA showed that there was a significant difference in participants' satisfaction with completing tasks across three conditions $(F[2, 22], p<0.001\red{, \eta_p^2=0.79})$. 
Post-hoc analysis with Holm correction revealed that participants' satisfaction with completing tasks in the QA-based condition was significantly higher compared to both the No-AI ($t(11)$, $p<0.001$\red{, $d=2.39$}) and the Prompt-based ($t(11)$, $p=0.029$\red{, $d=0.72$}) conditions.
% Additionally, the result of the Friedman test showed that there was a significant difference in participants' satisfaction with the quality of their email responses across three conditions $(\chi^2(2)=17.6, p<0.001)$. 
% Post-hoc analysis using the Durbin-Conover test with Holm correction showed that participants' satisfaction with the quality of their email responses in the QA-based condition was significantly higher compared to the No-AI $(p<0.001)$ condition.
Therefore, H1-c was supported.
The QA-based approach improved participants’ satisfaction with completing their tasks while replying to the emails.

% \subsubsection{\red{Future Preference (H1-c)}}
% \red{
% The questionnaire survey results about participants' future preferences are shown in Fig.~\ref{fig_study1_questionnaire}, in H1-c.
% According to the Friedman test, a significant difference in participants' future preferences was observed among the three conditions $(\chi^2(2)=8.8, p=0.012, W=0.37)$.
% Post-hoc analysis using the Durbin-Conover test with Holm correction revealed that participants would prefer responding in the QA-based condition compared to the No-AI condition $(p=0.012, r=0.67)$.
% However, no significant difference was found between the Prompt-based and QA-based conditions $(p=0.800, r=0.27)$.
% }

\subsubsection{Difficulty in Initiating the Action for Replying to Emails (H1-d)}
\label{sec:result1_initiating}
% felt high barrier: QA-based < Prompt-based, No-AI
The questionnaire survey results about participants' difficulty in initiating the action for replying to emails are shown in Fig.~\ref{fig_study1_questionnaire}, in H1-d.
According to the Friedman test, a significant difference in participants' difficulty in initiating the action for replying to emails was observed among the three conditions $(\chi^2(2)=19.8, p<0.001\red{, W=0.83})$.
Post-hoc analysis using the Durbin-Conover test with Holm correction revealed that participants in the QA-based condition perceived significantly higher barriers to initiating email response tasks than those in the No-AI $(p<0.001\red{, r=0.85})$ and Prompt-based $(p<0.001\red{, r=0.68})$ conditions.
% This result suggests the validity of H1-c, which ResQ lowers the barriers to initiating email response tasks.
Therefore, H1-d was supported.
The QA-based approach reduced participants’ difficulty in initiating the action to reply to emails.

\subsubsection{Sense of Agency and Control (H1-e)}
\label{sec:result1_agency}
% agency, control: QA-based < Prompt-based, No-AI
The questionnaire survey results about a sense of agency and control are shown in Fig.~\ref{fig_study1_questionnaire}, H1-e.
The Friedman test revealed a significant difference among the three conditions for both the sense of agency $(\chi^2(2)=22.8, p<0.001\red{, W=0.95})$ and the sense of control $(\chi^2(2)=21.3, p<0.001$, $\red{W=0.89})$. 
Post-hoc analysis using the Durbin-Conover test with Holm correction showed that participants in the QA-based condition found that it significantly reduced their sense of agency compared to both the No-AI $(p<0.001\red{, r=0.88})$ and the Prompt-based $(p<0.001\red{, r=0.77})$ conditions.
Additionally, post-hoc analysis using the Durbin-Conover test with Holm correction showed that participants in the QA-based condition experienced a significantly reduction in their sense of control compared to both the No-AI $(p<0.001\red{, r=0.88})$ and the Prompt-based $(p=0.006\red{, r=0.56})$ conditions.
Thus, H1-e was supported.
The QA-based approach reduced participants’ sense of agency and sense of control while replying to the emails.

\subsection{Quality of the Email Responses (RQ2)}
\label{sec:result1_RQ2}
\subsubsection{Perceived Quality of the Email by Evaluators (H2)}
\label{sec:result1_quality}
In Fig.~\ref{fig_study1_questionnaire}, H2 shows the results regarding the quality of the emails.
\red{The Cronbach's Alpha of three items measuring the perceived quality of the email is $0.846$.}
% \red{The three items measuring the perceived quality of the email demonstrated high internal consistency, with a Cronbach's Alpha of $0.846$.}
After checking the data normality assumption with the Shapiro-Wilk test, the result of one-way repeated measures ANOVA showed that there was a significant difference in the perceived quality of the email across three conditions $(F[2, 22]=9.1, p=0.001\red{, \eta_p^2=0.45})$. 
Post-hoc analysis with Holm correction revealed that the perceived quality of the emails participants wrote in the QA-based condition was significantly higher compared to the No-AI $(t(11), p=0.005\red{, d=1.21})$ condition.
Thus, H2 was partially supported.
The QA-based approach improved the quality of the email responses compared to the No-AI condition.

% 各評価項目の表を載せて、それを根拠にする
\begin{table*}[t]
\caption{\red{Details of perceived quality of the emails. ($Mean\pm SD$)}}
\Description{The table presents the comparative evaluation of three methods, No-AI, Prompt-based, and QA-based, in terms of three key attributes: Politeness, Readability, and Meeting Demands. The results are displayed as mean scores with standard deviations. For Politeness, the No-AI method received a mean score of 4.39 ± 1.00, indicating lower politeness levels compared to the AI-based methods. The Prompt-based approach showed a significant improvement, scoring 5.65 ± 0.56, slightly outperforming the QA-based method, which scored 5.49 ± 0.51. In terms of Readability, the No-AI method achieved a score of 5.24 ± 0.79, again falling behind the AI-based methods. The Prompt-based approach scored 5.65 ± 0.69, while the QA-based method scored the highest at 5.78 ± 0.49, reflecting the most consistently readable outputs among the three. Finally, for Meeting Demands, the No-AI method scored 5.19 ± 0.76, which is comparatively lower than the AI-enhanced methods. The Prompt-based approach performed better with a score of 5.68 ± 0.60, but the QA-based method emerged as the best performer in this category, scoring 5.88 ± 0.60.}
\label{tab_study1_quality_of_emails}
\red{
\begin{tabular}{cccc}
\hline
             & Politeness                 & Readability                & Meeting Demands               \\ \hline
No-AI        & $4.39\pm1.00$ & $5.24\pm0.79$ & $5.19\pm0.76$ \\
Prompt-based & $5.65\pm0.56$ & $5.65\pm0.69$ & $5.68\pm0.60$ \\
QA-based     & $5.49\pm0.51$ & $5.78\pm0.49$ & $5.88\pm0.60$ \\ \hline
\end{tabular}
}
\end{table*}
\red{Tab.~\ref{tab_study1_quality_of_emails} shows the detailed results regarding the perceived quality of the emails across three evaluation dimensions (politeness, readability, and meeting demands).
These results further supported the partial acceptance of H2, showing that the AI-assisted approach tended to improve the email quality.} 
% of email responses across all dimensions.}

%%%%%%%%%%%%%%%%%%%%%%%%%%%%%%%%%%%%
% According to the Friedman test, a significant difference was observed among the three conditions $(\chi^2(2)=17.6, p<0.001)$.
% As a post hoc test, the Durbin-Conover test with Holm correction was conducted.
% The results indicated that, compared to the No-AI condition, both the AI-assisted QA-based $(p=0.007)$ and Prompt-based $(p=0.012)$ conditions produced significantly higher-quality responses.
%%%%%%%%%%%%%%%%%%%%%%%%%%%%%%%%%%%%

\subsection{Relationship between Participants and Their Counterpart (RQ3)}
\label{sec:result1_RQ3}
\subsubsection{Perceived Impression of Participants by Evaluators (H3-a)}
\label{sec:result1_impression}
The results of the perceived impression of the participants rated by another group of evaluators are shown in Fig.~\ref{fig_study1_questionnaire} H3-a.
\red{The two items assessing participants' impression as email senders showed high internal consistency, with a Cronbach's Alpha of $0.946$.}
After checking the data normality assumption with the Shapiro-Wilk test, the result of one-way repeated measures ANOVA showed that there was a significant difference in impression of the participants as an email sender across three conditions $(F[2, 22]=5.9, p=0.009\red{, \eta_p^2=0.35})$. 
Post-hoc analysis with Holm correction revealed that participants' impression in the QA-based condition was not significantly higher compared to both the No-AI $(t(11), p=0.058\red{, d=0.79})$ and the Prompt-based $(t(11), p=0.939\red{, d=0.02})$ conditions.
% no-AIとQA-basedには差がありそうということを書くべきか?
Thus, H3-a was not supported.
The QA-based approach didn't improve the impression of participants as email senders.
% Although there was no statistically significant difference in participants' impressions between the QA-based and No-AI conditions $(p=0.058)$, the effect size was large $(d=0.79)$. 
% This suggests that while the difference did not reach statistical significance, there is still a meaningful difference in the impression of participants as email senders between these conditions.

\subsubsection{Psychological Distance between Participants and Their Counterpart (H3-b)}
\label{sec:result1_psychological_distance}
The IOS result is shown in \red{Fig.~\ref{fig_study1_questionnaire}}.
The Friedman test showed a significant difference in psychological distance among the three conditions $(\chi^2(2)=7.47, p=0.024\red{, W=0.31})$.
Post-hoc analysis using the Durbin-Conover test with Holm correction revealed that IOS in the No-AI condition was significantly higher than in the QA-based condition $(p=0.021\red{, r=0.51})$. 
% この結果はH3-bを部分的に支持するが、Prompt-based条件とQA-based条件の間には有意差がないことから、完全には支持されなかった
This result partially supports H3-b; however, because there was no significant difference between the Prompt-based and QA-based conditions \red{$(p=0.053, r=0.30)$}, H3-b was partially supported.


\subsection{\red{Qualitative Feedback}}
\label{sec:result1_interview}
\red{This section synthesizes qualitative feedback to provide further insights into participants' experiences across three conditions.
The interview comments were translated from Japanese into English.}
\subsubsection{\red{Participants' Email-Replying Process (RQ1)}}
\label{sec:result1_interview_RQ1}
\red{Feedback from participants confirmed that the QA-based condition functioned as expected, contributing to improvements in efficiency, a reduction in workload, and a lowering of barriers to task initiation compared to the other conditions.
\begin{enumerate}[]
    \item \textit{``In the QA-based condition, AI summarized key points through questions and highlighted relevant sections of the email body, which facilitated my understanding of the email and reduced my overall burden''} [P10].
    \item \textit{``In the QA-based condition, I could easily obtain the desired output even without the technical skills to create prompts''} [P6].
    \item \textit{``By saving the time needed to read the counterpart's text, the psychological barrier to starting the task was lowered''} [P5].
\end{enumerate}}

\red{On the other hand, we found that the QA-based condition led to a reduced sense of agency and control compared to the other conditions.
\begin{enumerate}[]
    \item \textit{``Since the AI prompted me with questions at the beginning, the mental effort required to start thinking about the task was eliminated, reducing the stress associated with initiating the work''} [P10].
    \item \textit{``By saving the time needed to read the counterpart's text, the psychological barrier to starting the task was lowered''} [P5].
\end{enumerate}}

\red{This aspect was also found to have the potential to negatively impact users' willingness to adopt the system in the future.
Participants noted that they preferred the QA-based condition \textit{``when time is limited or speed is important''} [P4] or \textit{when the email is of low importance''} [P5], but in other situations, they favored writing responses themselves.}

\subsubsection{\red{Quality of the Email Responses (RQ2)}}
\label{sec:result1_interview_RQ2}
% ほとんどの参加者は、AIを使うと、構造・丁寧さ・言葉遣いが改善され、全体的に良い文章を書けたと述べた
% また参加者は、「Prompt-based条件だと、相手の要求を見落としていたかもしれないが、QA-based条件では自信を持って返信を作成することができた」 [P2]と述べた
% さらにある参加者は、「QA-based条件では、回答してもしなくても良いこと「XXの件、承知しました、など。」にも丁寧に返答を書いてくれた」 [P9]と述べ、QA-based条件によってメールの丁寧さが向上したことを強調した
\red{All participants stated that using AI improved their writing structure, politeness, and choice of words, ultimately enabling them to produce better overall responses. 
Furthermore, participants remarked, \textit{``Under the prompt-based condition, I might have overlooked the recipient's requests, but under the QA-based condition, I was able to craft responses with confidence''} [P2]. 
Additionally, one participant emphasized that \textit{``Under the QA-based condition, the AI even provided polite responses to matters where a reply was optional, such as acknowledging something with phrases like 'I Understood regarding XX, etc.'''} [P9], highlighting how the QA-based condition scaffolded user to construct a polite email in a formal setting.}
% enhanced the politeness of email communication.}

\subsubsection{Relationship between Participants and Their Counterpart (RQ3)}
\label{sec:result1_interview_RQ3}
% 参加者は、``相手との間に知覚する心理的距離は労力に比例した''と報告し、PXXは``特にQA-based条件では選択肢を選ぶだけだった相手のことを考えることが少なかった''と報告した。
% 一方でPXXは、``自分で返信を考えるより、AIを使うと相手に良い印象を与えられるメッセージを作ることができたので、関係性を近く感じた''と報告した
\red{Participants shared differing views on how AI's involvement affected their psychological distance from their counterparts.
P2, P9, and P11 reported that the psychological distance they felt from the other person was directly related to the amount of effort they put in.
Furthermore, P6 noted that \textit{``especially under QA-based condition, I barely thought about the counterpart because I only selected options to create responses.''}
In contrast, P8 reported that \textit{``compared to composing replies myself, using AI allowed me to create messages that left a better impression on my counterpart, which made the relationship feel closer.''}
These results suggested that, on the one hand, AI's mediation can potentially increase the psychological distance between senders and receivers. 
On the other hand, it can also diminish the perceived distance from the sender due to effective impression management. Thus, we conducted a field study to further clarify the impact of AI on interpersonal relationships.}
% These results suggested that while a reduction in cognitive load may decrease participants' psychological engagement with their counterparts, the perceived improvement in communication quality could, conversely, foster a greater sense of closeness in the relationship.}

%%%%%%%%%%%%%%%%%%%%%%%%%%%%%%%%%%%%%%%%%%%%%%%%%%%%%%%%%%%%%%%%%%%%%%%%%%%%%%%%%%%%%%%%%%%%%%%%%%%%%%%%%%%%%%%%%%%%
% \subsection{Qualitative Feedback}
% % We added a new subsection, "User Comments," to present follow-up interview results, including participant feedback on useful questions (Sec.6.4). 
% \subsubsection{Participants' Email-Replying Process (RQ1)}
% \paragraph{Enhanced Efficiency, Reduced Cognitive Load, and Lowered Barriers to Initiating Email Replies (H1-a, H1-b, H1-d)}
% % 参加者は、QA-based conditionは期待通り機能し、参加者の効率向上や負担低減に貢献したことが確認できました。
% % 質問で要点をまとめてくれ、メール本文の対応箇所がハイライトされてたので、メールの理解の効率が上がり、負担が減りました [P10]
% % QA-based条件では、Prompt作成の技術がなくても、期待する出力を簡単に得ることができました。[P6]
% % Prompt-based条件では、結局自分で相手のメールを全て読み、回答すべきことを整理する必要がありました。[P4]
% % Prompt-based条件では、自分でAIに対する指示を一から考える必要があり、手動の条件と効率や負担に差を感じませんでした。一方でQA-based条件は、圧倒的に早く返信を作成することができました。[P5]
% \red{Participants' comments confirmed that the QA-based improved efficiency and reduced workload when replying to emails.
% P10 explained, \textit{``In the QA-based condition, AI summarized key points through questions and highlighted relevant sections of the email body, which facilitated my understanding of the email and reduced my overall burden''}.
% P6 shared, \textit{``In the QA-based condition, I could easily obtain the desired output even without the technical skills to create prompts''}.
% In contrast, the Prompt-based condition required extra effort. 
% P4 noted, \textit{``In the Prompt-based condition, I had to read the counterpart's email completely and decide what to respond to''}. 
% P5 elaborated, \textit{``In the Prompt-based condition, I had to think of instructions for the AI from scratch, making it feel no different from the No-AI condition in terms of efficiency and workload. On the other hand, the QA-based condition allowed me to compose responses faster''}.}

% \paragraph{Reduced Difficulty in Initiating the Action for Replying to Emails (H1-d)}
% % 参加者のコメントから、QA-based conditionでは、全体的な労力が下がるとともに、作業開始時のAIによるの質問が、タスク開始の障壁の低下に役立つことがわかりました。
% % 最初にAIが質問を投げかけてくれるので、作業開始時の思考の労力がなくなり、作業に取り掛かる際のストレスが減りました[P10]
% % 相手の文章を読む時間が省けたことで、作業開始の心理的障壁が下がりました。[P5]
% \red{Comments from participants indicated that the QA-based condition helped lower the barriers to task initiation through AI-generated questions at the start of the process. 
% P10 explained, \textit{``Since the AI prompted me with questions at the beginning, the mental effort required to start thinking about the task was eliminated, reducing the stress associated with initiating the work''}. 
% P5 noted, \textit{``By saving the time needed to read the counterpart's text, the psychological barrier to starting the task was lowered''}.}

% \paragraph{Decreased Sense of Agency and Control (H1-e)}
% % QA-based conditionでは、agencyやcontrolの感覚が他の条件に比べて低下したと回答した参加者は、次のようにコメントした。
% % 「agencyとcontrolの感覚は、プロンプトを自分で打った量に比例しました。」[P7, 8, 9, 10]
% % 「QA-based条件では、要点も絞ってくれたので、AIに任せようという思いが強くなりました。」[P4]
% % 一方で感覚が変化しなかったと回答した参加者は「AIに任せても、自分で確認と修正を行ったので、agencyやcontrolの感覚に変化はありませんでした」[P3]と説明した。
% \red{Participants reported a decrease in their sense of agency and control in the QA-based condition.
% \textit{``The sense of agency and control was proportional to the amount of text I typed myself''} [P7, P8, P9, P10]. 
% \textit{``Under the QA-based condition, since the AI helped narrow down the key points, I felt a stronger inclination to leave the task to the AI''} [P4].
% On the other hand, one participant who reported no change in their sense of agency or control explained, \textit{``Even though I relied on the AI, I reviewed and edited the output myself, so there was no change in my sense of agency or control''} [P3].}

% \paragraph{Future Preference (H1-c)}
% % 多くの参加者はメール返信の効率が上がる、負担が減る、質が高いメールを執筆できるという理由から、QA-based conditionで返信をしたいと回答しました。
% % 自分で返信を考えるより、AIを使った方が、質の高い返信を早く作ることができました。特にQA-based conditionではその効果が大きかったので、将来はQA-based conditionで返信したいと思いました。[P6]
% % 一方でsense of agencyの低下や、AIへの依存の危惧を理由に、QA-based conditionでのメール返信を忌避する参加者もいました。
% % 時間がない時や、スピードを重視したい時[P4]、重要度が低い時[P5]は、QA-based conditionで返信をしたいと思ったが、そうでない場合は自分で執筆したいと思いました。
% % 「Prompt-based条件では、QA-based条件より思考する必要が多く、それが楽しかったです」 [P7]
% % 返信作業が楽にはなりましたが、メールを細部まで読まなくなり、内容が頭に入っていない感じがしたので、将来使いたいとは思いませんでした。[P11]
% \red{Participants expressed a preference for using the QA-based condition for email responses, citing increased efficiency, reduced workload, and the ability to produce high-quality emails as the primary reasons. 
% One participant explained, \textit{``Using AI allowed me to compose high-quality responses faster than if I had written them myself. The effect was particularly significant in the QA-based condition, which is why I would prefer to use it in the future''} [P6].}
% \red{However, some participants were hesitant to adopt the QA-based condition due to concerns about a reduced sense of agency or over-reliance on AI. 
% Participants noted that they preferred the QA-based condition \textit{``when time is limited or speed is important''} [P4] or \textit{when the email is of low importance''} [P5], but in other situations, they favored writing responses themselves.
% One participant reflected, \textit{``In the prompt-based condition, I found that I needed to think more actively compared to the QA-based condition, and I enjoyed that process''} [P7].
% Another observed, \textit{``While the QA-based condition made responding easier, I felt that I was no longer fully reading and absorbing the content of emails, which made me hesitant to use it in the future''} [P11].}

% \paragraph{Quality of AI-generated Questions and Options}
% % 参加者はQA-based conditionにおいて生成された質問や選択肢について、有益であったものとそうでなかったものがあったとコメントした。
% % 参加者は有益でない質問の例として、メール送信者の意図を汲み取れていないもの [P2, P11]、自分と相手の立場を勘違いしているもの [P4, P8]を挙げた。
% % またある参加者は、「自分が言いたいことに関連する質問がない場合、質問機能自体が役に立たなかった」[P8]と回答した。
% % また参加者は、質問の数についても意見を述べた
% % 参加者は「用件ごとに質問を生成してくれたのが、メールを理解する上で役に立った」[P4, P5, P8, P9, P10, P11]と回答した一方で、「質問が多すぎると煩雑に感じることもあった。またそれに全て答えると、返信が冗長になってしまった。」[P7, P12]と回答する参加者もいた。
% % また、参加者は選択肢についても意見を述べた
% % 参加者は、「(日程調整のシチュエーションにおいて)選択肢の中に自分が選びたい日程がなかったので、自分で日程を入力する必要があった」[P2]と説明し「より多くの選択肢を生成して欲しかった」と説明した[P8]。
% % 一方である参加者は、「必要以上に多くの選択肢があったときに、煩わしさを感じた」[P4]と説明した。
% \red{Participants commented on the questions and options generated by the QA-based condition, noting that some were useful while others were not. 
% Examples of less useful questions included those that failed to accurately capture the sender's intent [P2, P11] or misinterpreted the relationship between the sender and recipient [P4, P8]. 
% Additionally, one participant remarked, \textit{``when there were no questions related to what I wanted to say, the question feature itself was not helpful''} [P8].}

% \red{Participants also shared mixed opinions on the number of questions generated. 
% Some participants noted that \textit{``generating questions for each topic helped understand the email''} [P4, P5, P8, P9, P10, P11].
% However, others felt \textit{``an excessive number of questions felt overwhelming, and responding to all of them made the reply unnecessarily lengthy''} [P7, P12].}

% \red{Feedback on the generated options was similarly divided. 
% For instance, in scheduling scenarios, one participant shared that \textit{``none of the suggested dates matched what I wanted, so I had to input the date myself''} [P2] and another participant \textit{``wished for more options or a more flexible input type''} [P8].
% In contrast, another participant stated that \textit{``having more options than necessary felt burdensome''} [P4].}

% \subsubsection{Quality of the Email Responses (RQ2)}
% % ほとんどの参加者は、AIを使うと、構造・丁寧さ・言葉遣いが改善され、全体的に良い文章を書けたと述べた
% % また参加者は、「Prompt-based条件だと、相手の要求を見落としていたかもしれないが、QA-based条件では自信を持って返信を作成することができた」 [P2, 4]と述べた
% % さらにある参加者は、「QA-based条件では、回答してもしなくても良いこと「XXの件、承知しました、など。」にも丁寧に返答を書いてくれた」 [P9]と述べ、QA-based条件によってメールの丁寧さが向上したことを強調した
% \paragraph{Scaffolding a structured response}
% \red{Most participants stated that using AI improved their writing structure, politeness, and choice of words, ultimately enabling them to produce better overall responses. 
% Furthermore, participants remarked, \textit{``Under the prompt-based condition, I might have overlooked the recipient's requests, but under the QA-based condition, I was able to craft responses with confidence''} [P2, P4]. 
% Additionally, one participant emphasized that \textit{``Under the QA-based condition, the AI even provided polite responses to matters where a reply was optional, such as acknowledging something with phrases like 'I Understood regarding XX, etc.'''} [P9], highlighting how the QA-based condition enhanced the politeness of email communication.}

% \subsubsection{Relationship between Participants and Their Counterpart (RQ3)}
% % 参加者は、``相手との間に知覚する心理的距離は労力に比例した''と報告し、PXXは``特にQA-based条件では選択肢を選ぶだけだった相手のことを考えることが少なかった''と報告した。
% % 一方でPXXは、``自分で返信を考えるより、AIを使うと相手に良い印象を与えられるメッセージを作ることができたので、関係性を近く感じた''と報告した
% \red{Participants shared differing views on how AI's involvement affected their psychological distance from their counterparts.
% Several participants reported that the psychological distance they felt from the other person was directly related to the amount of effort they put in [P2, P9, P11].
% % \textit{``the psychological distance I perceived from the counterpart was proportional to the effort exerted''} [P2, P9, P11].
% Furthermore, P6 noted that \textit{``especially under QA-based condition, I barely thought about the counterpart because I only selected options to create responses''}.
% In contrast, P8 reported that \textit{``compared to composing replies myself, using AI allowed me to create messages that left a better impression on my counterpart, which made the relationship feel closer''}.}