\section{Introduction}
Email is a common tool for users to share information~\cite{Nelson2011Mail2Tag, whittaker2006email} and manage tasks~\cite{bellotti2005quality}. 
It has been found that users spend an average of 28\% of their workweek reading and replying to emails~\cite{McKinseySocialEconomy}.
However, for many, checking and responding to emails is time-consuming and cognitively demanding~\cite{mark2016email}. 
Responding to emails, especially in cultures that value courteous email exchanges, requires users to understand the sender's requests and compose polite messages that reflect the sender's intentions. 
This involves considering various elements such as tone, style, diction, and structure~\cite{Chen2015Chinese}.
Putting effort into constructing courteous emails and responding promptly is important because the lack of these elements can result in a negative perception by recipients~\cite{Resendes2012Send, vignovic2010computer, yoram2011online}, potentially harming trust and damaging relationships in formal communication settings. 
% In this paper, we define formal email exchange as a type where the email structure is typically clear and organized, employing polite and respectful language with a specific focus, unlike casual greetings (\textit{e.g.,} office-related communication, research collaboration in academic institutions, and interactions with external organizations).
\blue{In this paper, we define formal email exchange as a type of communication where the email structure is typically clear and organized, employing polite and respectful language with a specific focus. 
Examples of formal email exchange include office-related communication, research collaboration in academic institutions, and interactions with external organizations.}

Various approaches have been proposed to reduce the workload for replying to emails.
These include tools that aid in deciding whether to respond~\cite{dredze2008intelligent, di2016youvegotmail}, how to respond~\cite{vaish2017crowdtone, Respondable}, what to respond~\cite{Kannan2016Smart, Al-Alwani2013novel, Moravapalle2017DejaVuAC, Naeem2018A}, or reminding to respond~\cite{Dugan2017RemindMe:, Won2009Designing}.
With the recent advancement of generative artificial intelligence (AI), particularly large language models (LLMs), an increasing number of AI-mediated communication (AIMC) tools~\cite{ChatGPT, Grammarly, MicrosoftCopilot, bastola2024llmbasedsmartreplylsr, Claude, Foodman2022LaMPost, Fu2023Comparing, Chen2019Gmail} have been proposed.
For example, by inputting the content of an email into an AI chatbot, like ChatGPT~\cite{ChatGPT} or Claude~\cite{Claude}, along with an instruction for the model (``prompt''), these tools can generate reply drafts. 
This prompt-based response-generation approach has been shown to reduce users' overall workload and improve productivity~\cite{bastola2024llmbasedsmartreplylsr}.
While AIMC tools offer advantages, users must carefully craft prompts to achieve the desired content, tone, and style in emails~\cite{Zhou2024GlassMail}.
If the expected output is not obtained, users need to create prompts repeatedly~\cite{fu2024text}, which adds extra workload.

\red{To address this issue, we propose a QA-based approach in which the system analyzes incoming emails and generates questions that invite users to respond to efficiently create the desired draft.
In this paper, we define and implement a QA-based approach by generating questions based on the text of the email. 
Then, the system creates an email draft based on the users' answers to these questions.  
This QA-based approach was motivated by prior studies suggesting that answering structured questions helps users articulate their needs more effectively~\cite{kim2024aineedsplanner, jannach2021survey, cao2023comprehensive}. 
This approach aims to lower the cognitive burden of drafting replies, help users quickly understand the sender’s requests, and reduce the need of cumbersome prompting by breaking down the prompt-generation process into smaller, more manageable QA steps.}

Thus, this paper aims to comprehensively investigate the effect and effectiveness of the QA-based approach using \red{our prototype system, \textit{ResQ}, which generates questions and options using LLMs (Fig.~\ref{fig_teaser}).}
We also think it is important to investigate the impact of our tool on users' psychological aspects.
Therefore, we also investigate the users' sense of agency, control over the text, and perceived psychological distance, as extensive AI mediation may negatively impact these aspects~\cite{Fu2023Comparing, mieczkowski2022examining, Draxler2024The, Buschek2021The}, potentially leading to underutilization of the system.

We conducted a controlled experiment (N=12) and a field study (N=\red{8}).
We found that compared to a conventional prompt-based approach where users must consider appropriate prompts to obtain email drafts, the efficiency of replying to emails improved, and the overall workload was reduced, all while maintaining the quality of the replies.
Additionally, though ResQ lessens users’ sense of agency and control, the interview results in the field study suggest it could reduce the psychological distance from their counterparts by promoting perceptions of enhanced communication quality and quantity.

The contributions of this research are twofold. 
First, the results of two studies show how the QA-based approach\footnote{The proposed system, ResQ, will be released as an open-source Chrome extension. The source code will be publicly available at the following link: \url{https://github.com/miulab7/ResQ}.} 
affects users’ writing processes, the quality of the composed emails, and their relationships with recipients. 
Second, based on the results of these studies, we provide insights regarding both the opportunities and challenges of introducing a QA-based approach in email communication.

% \red{The proposed system, ResQ, will be released as an open-source Chrome extension. 
% The source code will be publicly available at the following link: https://github.com/XXXX/YYYYY.}