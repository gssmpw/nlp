\section{Results of Study 2}
\red{Tab.~\ref{tab_study2_participants_usage} presents the number of email replies composed using ResQ, along with the contexts in which it was used over five days.
We did not analyze usage frequency because participants reported avoiding using ResQ for emails for which they had privacy concerns.
% We did not analyze usage frequency because participants reported avoiding ResQ for certain types of emails, such as those deemed important or involving security concerns. 
Additionally, some participants refrained from using ResQ due to its availability only on PCs, as they frequently replied to emails via smartphone. 
Email frequency also varied among participants depending on their personal schedules (\textit{e.g.,} holidays).}

\red{Eight participants primarily used ResQ in formal workplace settings, while one (P9) used it only for informal exchanges. 
Because this was a field study, we could not limit participants to using ResQ only in formal contexts, though we instructed them to use it to reply to formal emails at the beginning. 
As a result, two participants (P3, P4) used ResQ to reply to both formal and informal emails, and P9 only used it for informal email exchanges. 
Hence, we excluded P9's data and only focused on analyzing the experience of P3 and P4 when they replied to formal emails using ResQ.}
% Since this study focuses on formal usage, we excluded P9's results and presented only interview data on formal scenarios.}
% 特にP1があまり使わなかった理由について言及したい

% The number of email replies composed by participants using ResQ over five days and the contexts of usage are shown in Tab.~\ref{tab_study2_participants_usage}. 
% 今回はResQの使用頻度に関する分析は行わなかった。理由は次のとおり。
% Participants reported that they did not use ResQ to compose replies for emails that required short responses, those deemed particularly important, or when they had security-related concerns.
% Additionally, some participants noted that they occasionally composed replies without using ResQ because the tool is currently only available on PCs, whereas they often replied to emails using their smartphones.
% また参加者の都合(休日など、やり取りが少ない/多い週)のため、メールの頻度はばらつきがある。
% Regarding the system's usage contexts, eight out of nine participants primarily utilized ResQ in formal workplace settings, whereas one participant (P9) used it exclusively for informal exchanges. 
% Additionally, two participants (P3, P4) noted that while most of their usage occurred in formal contexts, they occasionally employed the system for informal communication.
% Pariticipantsの使用状況はコントロールはできなかったが、今回論文ではformalにfocusしているため、P9の結果は除去した。また、formalな状況における使用に関するインタビュー結果のみを掲載した。

\red{This section explains the results of interviews conducted with participants after they used the system, with the interview comments translated from Japanese into English.}
\begin{table*}[t]
\caption{Usage of the participants in Study 2. D1 through D5 represents the number of emails replied to using the system each day, from Day 1 to Day 5.}
\Description{The table illustrates the daily system usage of participants in Study 2, detailing the number of emails replied to each day (from Day 1 to Day 5) and the primary purposes for which the system was used. The participants represent a range of tasks, including work-related scheduling, academic communication, and informal interactions. Participant P1 primarily used the system for scheduling, task confirmations, and submissions related to work. His email activity was moderate, replying to 3 emails on Day 1, 1 email on Day 2, and 1 email on Day 5, with no emails replied to on Days 3 and 4. P2 focused on task management and communication related to research and university administration. She maintained consistent usage on Days 1 and 2, replying to 6 emails each day, did not use the system on Day 3, replied to 1 email on Day 4, and increased her activity to 5 emails on Day 5. Participant P3 engaged with the system for task management related to research with professors and informal contact with friends. Her usage was highest on Days 1 and 2, replying to 6 and 7 emails respectively. She replied to 2 emails on both Days 3 and 4 and did not use the system on Day 5. P4 used the system for scheduling related to club activities, informal contact with friends, and inquiries with a museum abroad. She consistently replied to 6 emails on both Days 1 and 2, 5 emails on Day 3, 4 emails on Day 4, and 6 emails on Day 5. Participant P5’s primary usage involved meeting planning and confirmations related to work. He replied to 2 emails on Day 1, 3 emails on Day 2, 5 emails on Day 3, and 1 email on both Days 4 and 5. P6 utilized the system for scheduling and confirmations related to work, friends, and event organizers. Her activity peaked on Days 2 and 3, replying to 6 emails each day, followed by 3 emails on both Days 1 and 4 and 5 emails on Day 5. Participant P7 focused on scheduling and progress management related to research and business trips. His usage was irregular, with 4 emails replied to on Day 1, none on Day 2, 6 emails on Day 3, none on Day 4, and 2 emails on Day 5. Lastly, P8 primarily used the system for progress management and administrative confirmations related to work. Her activity started with 5 emails on Day 1, decreasing to 2 emails on Day 2, 1 email on Day 3, and 5 emails on Day 4, with no usage recorded on Day 5.}
\label{tab_study2_participants_usage}
% \resizebox{\textwidth}{!}{
\begin{tabular}{ccccccl}
\hline
\multirow{2}{*}{} & \multicolumn{5}{c}{Daily System Usage} & \multicolumn{1}{c}{\multirow{2}{*}{Main Usage}}                                                      \\ \cline{2-6}
                  & D1     & D2    & D3    & D4    & D5    & \multicolumn{1}{c}{}                                                                                 \\ \hline
P1                & 3      & 1     & 0     & 0     & 1     & Scheduling, task confirmations, and submissions related to work                                      \\
P2                & 6      & 6     & 0     & 1     & 5     & Task management and communication related to research and university administration                  \\
P3                & 6      & 7     & 2     & 2     & 0     & Task management related to research with professors, informal contact with friends                   \\
P4                & 6      & 6     & 5     & 4     & 6     & Scheduling related to club activities, informal contact with friends, inquiries with a museum abroad \\
P5                & 2      & 3     & 5     & 1     & 1     & Meeting planning and confirmations related to work                                                   \\
P6                & 3      & 6     & 6     & 3     & 5     & Scheduling and confirmations related to work, friends, and event organizers                          \\
P7                & 4      & 0     & 6     & 0     & 2     & Scheduling and progress management related to research and business trips                            \\
P8                & 5      & 2     & 1     & 5     & 0     & Progress management and administrative confirmations related to work                                 \\ \hline
\end{tabular}
% }
% \Description{The table provides background information about the participants in Study 2, including their system usage over five days (D1 to D5), which represents the number of emails they replied to using the system each day, as well as their main usage purpose. Participants range in age from 20 to 39 years, with both students and employees included, and a mix of male and female participants. Participant P1 is a 28-year-old male employee who primarily used the system for scheduling, task confirmations, and submissions related to work. His daily system usage across the five days was 3 emails on day one, 1 email on day two, and 1 email on day five, with no emails replied to on days three and four. Participant P2 is a 24-year-old female student who used the system for task management and communication related to research and university administration. Her daily system usage was consistent on the first two days, replying to 6 emails each day. She did not use the system on day three, replied to 1 email on day four, and increased her usage again with 5 emails on day five. Participant P3 is a 20-year-old female student who focused her system usage on task management related to research with professors and informal contact with friends. She replied to 6 emails on day one, 7 emails on day two, 2 emails on both day three and day four, and did not reply to any emails on day five. Participant P4 is a 24-year-old female student who used the system for scheduling related to club activities, informal contact with friends, and inquiries with a museum abroad. Her system usage was steady, replying to 6 emails on both day one and day two, 5 emails on day three, 4 emails on day four, and 6 emails on day five. Participant P5 is a 31-year-old male employee who primarily used the system for meeting planning and confirmations related to work. He replied to 2 emails on day one, 3 emails on day two, 5 emails on day three, and 1 email on both day four and day five. Participant P6 is a 39-year-old female employee who used the system for scheduling and confirmations related to work, friends, and event organizers. Her system usage was moderate, replying to 3 emails on day one, 6 emails on day two, 6 emails on day three, 3 emails on day four, and 5 emails on day five. Participant P7 is a 25-year-old male student who used the system for scheduling and progress management related to research and business trips. His system usage was irregular, replying to 4 emails on day one, none on day two, 6 emails on day three, none on day four, and 2 emails on day five. Participant P8 is a 23-year-old female employee who used the system for progress management and administrative confirmations related to work. Her daily system usage was 5 emails on day one, 2 emails on day two, 1 email on day three, 5 emails on day four, and none on day five. Participant P9 is a 38-year-old male employee who used the system for daily informal contact with friends. He had the highest and most consistent email activity, replying to 13 emails on day one, 11 emails on day two, 12 emails on both day three and day four, and 10 emails on day five.}
\end{table*}
% \begin{table*}[t]
% \caption{Usage of the participants in Study 2. D1 through D5 represents the number of emails replied to using the system each day, from Day 1 to Day 5.}
% \centering
% \label{tab_study2_participants_usage}
% \resizebox{\textwidth}{!}{
% % \setlength{\tabcolsep}{6pt} % default 6pt
% {\tabcolsep=2pt
% \begin{tabular}{ccccccl}
% \hline
% \multirow{2}{*}{} & \multicolumn{5}{c}{\centering Daily System Usage} & \multirow{2}{*}{\centering Main Usage} \\ \cline{5-9} 
%                   & D1 & D2 & D3 & D4 & D5 & \multicolumn{1}{l}{}                                \\ \hline
% P1                & 3     & 1     & 0     & 0     & 1     & Scheduling, task confirmations, and submissions related to work         \\
% P2                & 6     & 6     & 0     & 1     & 5     & Task management and communication related to research and university administration \\
% P3                & 6     & 7     & 2     & 2     & 0     & Task management related to research with professors, informal contact with friends        \\
% P4                & 6     & 6     & 5     & 4     & 6     & Scheduling related to club activities, informal contact with friends, inquiries with a museum abroad   \\
% P5                & 2     & 3     & 5     & 1     & 1     & Meeting planning and confirmations related to work                  \\
% P6                & 3     & 6     & 6     & 3     & 5     & Scheduling and confirmations related to work, friends, and event organizers \\
% P7                & 4     & 0     & 6     & 0     & 2     & Scheduling and progress management related to research and business trips \\
% P8                & 5     & 2     & 1     & 5     & 0     & Progress management and administrative confirmations related to work                \\
% P9                & 13    & 11    & 12    & 12    & 10    & Daily informal contact with friends         \\ \hline
% \end{tabular}
% }
% \Description{The table provides background information about the participants in Study 2, including their system usage over five days (D1 to D5), which represents the number of emails they replied to using the system each day, as well as their main usage purpose. Participants range in age from 20 to 39 years, with both students and employees included, and a mix of male and female participants. Participant P1 is a 28-year-old male employee who primarily used the system for scheduling, task confirmations, and submissions related to work. His daily system usage across the five days was 3 emails on day one, 1 email on day two, and 1 email on day five, with no emails replied to on days three and four. Participant P2 is a 24-year-old female student who used the system for task management and communication related to research and university administration. Her daily system usage was consistent on the first two days, replying to 6 emails each day. She did not use the system on day three, replied to 1 email on day four, and increased her usage again with 5 emails on day five. Participant P3 is a 20-year-old female student who focused her system usage on task management related to research with professors and informal contact with friends. She replied to 6 emails on day one, 7 emails on day two, 2 emails on both day three and day four, and did not reply to any emails on day five. Participant P4 is a 24-year-old female student who used the system for scheduling related to club activities, informal contact with friends, and inquiries with a museum abroad. Her system usage was steady, replying to 6 emails on both day one and day two, 5 emails on day three, 4 emails on day four, and 6 emails on day five. Participant P5 is a 31-year-old male employee who primarily used the system for meeting planning and confirmations related to work. He replied to 2 emails on day one, 3 emails on day two, 5 emails on day three, and 1 email on both day four and day five. Participant P6 is a 39-year-old female employee who used the system for scheduling and confirmations related to work, friends, and event organizers. Her system usage was moderate, replying to 3 emails on day one, 6 emails on day two, 6 emails on day three, 3 emails on day four, and 5 emails on day five. Participant P7 is a 25-year-old male student who used the system for scheduling and progress management related to research and business trips. His system usage was irregular, replying to 4 emails on day one, none on day two, 6 emails on day three, none on day four, and 2 emails on day five. Participant P8 is a 23-year-old female employee who used the system for progress management and administrative confirmations related to work. Her daily system usage was 5 emails on day one, 2 emails on day two, 1 email on day three, 5 emails on day four, and none on day five. Participant P9 is a 38-year-old male employee who used the system for daily informal contact with friends. He had the highest and most consistent email activity, replying to 13 emails on day one, 11 emails on day two, 12 emails on both day three and day four, and 10 emails on day five.}
% \end{table*}
\subsection{Participants' Email-Replying Process (RQ1)}
\subsubsection{Improved Perception of Efficiency and Workload}
\label{sec:result2_efficiency}
Participants reported that their perception of workload and work efficiency improved due to the support from ResQ.
Specifically, participants noted that ResQ's support helped clarify the topics they needed to address in the email. 
Participants explained that \textit{``Normally, when writing, I need to process multiple tasks simultaneously to ensure my intentions are appropriately expressed. However, [With ResQ,] replying to emails was divided into two different sub-tasks, answering questions and polishing emails with diverse expressions. As a result, I felt that the cognitive load was reduced.''} [P7], and \textit{``it felt like creating an email was as simple as answering a survey''} [P6].
Additionally, particularly when the counterparts' message was long, participants reported that the listing of requests as questions allowed them to \textit{``easily understand the content of the email''} [P3], with another participant noting that \textit{``I can quickly make decisions on what to reply [with ResQ]''} [P6]. 
Furthermore, compared to other AI tools like ChatGPT, the QA-based approach enabled participants to communicate their intentions more efficiently without extensive typing.
As one participant described,\textit{``[Writing with ResQ] made it easier to reflect my intentions while replying to the email''} [P4], while another participant added that \textit{``I could create the expected reply without even having to type on the keyboard''} [P6].
% While ResQ was helpful in formal situations, some participants found it less suitable for informal communications, such as with friends.
% One participant remarked that \textit{``the expressions were too polite, and I didn't like it''} [P3], and another mentioned that they \textit{``thoroughly changed the phrasing, like replacing 'Looking forward to seeing you again!' with just 'See you'''} [P4].
% Additionally, eight participants (with the exception of P9, who only used the system in informal settings) expressed increased satisfaction with the quality of the responses they wrote (for more details, see Sec.~\ref{sec:result2_quality}) and responded positively to the question, \textit{``What is your overall impression of using ResQ?''} and expressed a desire to continue using the system in the future. 
\red{Additionally, all participants expressed increased satisfaction with the quality of the responses they wrote (for more details, see Sec.~\ref{sec:result2_quality}) and responded positively to the question, \textit{``What is your overall impression of using ResQ?''} and expressed a desire to continue using the system in the future.}
Participants also mentioned that being able to craft clearer messages more quickly than before resulted in \textit{``greater confidence in the reply process and a more positive perception of the task''} [P8]. 
Additionally, a different participant expressed, \textit{``I felt joy in meeting societal expectations competently''} [P2].
These increases in achievement and confidence led participants to report that their \textit{``perception of the reply task became more positive''} [P8], and they felt \textit{``more motivated to engage actively in email responses''} [P3]. 

\subsubsection{Reduced Difficulty in Initiating the Action for Replying to Emails}
\label{sec:result2_initiating_the_action}
Participants reported that ResQ's support lowered the barrier to starting tasks, reducing procrastination in replying to emails.
One participant shared that they previously \textit{``felt reluctant to engage in replying due to the burden of the task''}, but with ResQ, \textit{``I felt motivated because I can complete the task quickly''} [P3]. 
Another participant noted that \textit{``I became able to craft replies to any email easily, so I could respond even on days when I was tired or when I would typically postpone replying to long emails''} [P6].
Participants also reported that using AI to initiate the task motivated them to start replying to emails without procrastinating. 
One participant explained that \textit{``just pressing a button prompts the AI to ask questions''}[P4], which led them to \textit{``delegate the initial steps entirely to the system''} [P3]. 
This reduction in the burden of the initial stage was cited as a key factor in lowering the barrier to starting to reply to emails. 

\subsubsection{Reduced Sense of Agency and Control}
\label{sec:result2_agency_control}
% Four out of nine participants (P3, P5, P6, and P9) reported a decreased sense of agency and control while replying to emails with ResQ. 
\red{Three out of eight participants (P3, P5, and P6) reported a decreased sense of agency and control while replying to emails with ResQ.}
They attributed this to several factors: one participant mentioned that their perception shifted \textit{``from that of an author to that of an editor''} [P5], which reduced the workload of replying but made the process feel \textit{``like an assembly line''} [P3], while another expressed, \textit{``I ended up using words or expressions I normally wouldn't [use in the email]''} [P6].
In contrast, for those who reported no change in their sense of agency or control (five participants), they explained that this was because  \textit{``the email content was strongly related to me''} [P7], and they \textit{``checked the content carefully''} [P7] or \textit{``modified words that I wouldn't normally use to the ones I would use''} [P2], leading them to feel that their \textit{``active involvement [to reply to the email] was indispensable''} [P8]

\subsection{Increased Perceived Quality of the Email (RQ2)}
\label{sec:result2_quality}
Participants reported that they felt the quality of their emails had improved. 
Participants explained that, in the process of creating responses, they were most concerned with \textit{``politeness in language, such as expressions and greetings''} [P6], and mentioned that ResQ provides support in these areas.
Participants reported that \textit{``it was helpful to have phrases that would have taken time to come up with on their own, expressions of apology and gratitude, and additional words of consideration for the other person''} [P2], \textit{``there was no need to think about the opening and closing greetings''} [P8], and \textit{``there were no typos or omissions at all''} [P5].
Additionally, participants mentioned that ResQ helped reduce the likelihood of overlooking requests in the emails they received. 
One participant shared, \textit{``Previously, when a single email contained multiple requests, I sometimes missed responding to all of them, but the questions provided by ResQ helped improve this''} [P6].
Participants attributed this improvement to the fact that ResQ \textit{``secured time to focus on understanding the recipient's requests and responding to them''} [P1], and the questions generated by ResQ \textit{``helped me ensure that nothing was overlooked in the content''} [P7]. 
Furthermore, participants reported that responding to AI-generated questions encouraged them to include details they would normally omit, resulting in more polite and comprehensive responses. 
One participant described an email regarding event attendance and multiple confirmations, explaining that while they would usually reply with something like \textit{``I will attend, thank you''}, answering the AI's questions led to a response where \textit{``each of the recipient's requirements was addressed more carefully''} [P2], ultimately leading to a more courteous email.

\subsection{Relationship between Participants and Their Counterpart (RQ3)}
\subsubsection{Enabling a Positive Self-Presentation as an Email Sender}
\label{sec:result2_self-presentation}
The participants reported feeling they could make a good impression on others using ResQ. 
They attributed this to improvements in the quality of their writing, shorter response times, and increased frequency of replies. 
One participant mentioned, \textit{``I could answer the other person's questions clearly, and the writing became more polished, making it easier for them to read''} [P5]. 
The participant also mentioned that \textit{``I felt the individuality of the email reply had faded''} but added that \textit{``I never intended to express individuality in my emails to begin with, so even if it was lost, it wasn't an issue as long as it felt natural to the recipient''} [P5].
Another participant shared that when they met a professor with whom they had communicated via ResQ, the person remarked, \textit{``Your emails have become more polished.''} 
They further elaborated, \textit{``I was particularly complimented on how much more understandable the structure of my emails has become''} [P3].
Additionally, this participant noted, \textit{``Previously, I would often respond to long emails with just, 'I'll get back to you later,' because reading through and thinking about a proper reply was tedious. However, [with ResQ's support,] I've started responding immediately instead of postponing. As a result, I've been assigned more tasks than before.''}

\subsubsection{Psychological Distance between Participants and Their Counterpart}
\label{sec:result2_psychological_distance}
Participants had mixed opinions regarding the psychological distance they perceived from their counterparts.
Those who felt the decreased psychological distance between themselves and their counterparts attributed this to the positive impression they believed they made on their counterparts. 
One participant reported that sending well-crafted emails quickly led to \textit{``a stronger sense of reassurance in [formal] communication''} [P5], while another participant noted that \textit{``[When I asked the museum staff a question,] I noticed that when I replied immediately after receiving a message from the other person, they responded quickly in return. When we communicated with such a good rhythm, I felt a strong sense of closeness towards the counterpart''} [P4].
In contrast, participants who felt the increased psychological distance mentioned a strong awareness that their replies were mediated by a system and the use of words they would not usually choose. 
One participant gave an example of communication with their university professor, stating, \textit{``While I know the counterpart typed their emails manually, I felt that using AI made the conversation more superficial, which weakened our relationship''} [P3]. 
% Another participant mentioned that the awareness when communicating with friends that \textit{``knowing that the email wasn’t something I had crafted from scratch by myself made me feel more distant from the counterpart''} [P9].
Participants also shared that they tended to forget about the email exchange with their counterparts due to the increased psychological distance.
% their sense of psychological distance from the sender as being caused by their forgetting interactions with them. 
One participant mentioned, \textit{``I found the email content easy to understand while working on it [with ResQ], but I felt it was difficult to retain our email exchange in long-term memory. When that counterpart [who is my professor] asked me, 'What happened with that issue? [that had been mentioned in our email]' there were times I couldn’t remember, which made me feel anxious''} [P3].

\subsection{Perceived Risks}
\label{sec:result2_risks}
Participants expressed concerns about the potential risks that ResQ might pose in the future. 
They expressed concerns about potential declines in their abilities and the risk of becoming overly dependent on AI, which could lead to carelessness in responding to work-related emails.
One participant explained, \textit{``I worry that the skills I've developed from composing emails myself might deteriorate''} [P8]. 
Another participant voiced concerns that \textit{``the advancement and usage of AI [in this context] might erode our ability to overcome psychological barriers''} [P2], fearing a decline in their interpersonal communication skills.
Additionally, participants raised the issue of over-reliance on AI, with one participant noting, \textit{``Given my trust in AI, I might eventually stop reviewing the content of the emails I send or the emails I receive''} [P8]. 
This reflects their concern about the potential for becoming overly dependent on AI-generated text in the future.

% \subsection{\red{Valuable AI-generated Questions and Options}}
% % ResQが生成した質問と選択肢は、十分に実用的であることが実験結果から示唆されたが、その質にはさらなる改善の余地がある
% % まず参加者は、実験1のインタビュー結果と同様、\textit{``自分と相手の関係性をAIが勘違いして質問を作成していることがあった''} [P2]と、質問生成の精度の低さを指摘した。
% % さらにある参加者は、\textit{``質問の生成に時間がかかる時があり、その時間にできることがないので少しフラストレーションを感じた''} [P1]と回答し、質問生成の生成時間の長さを指摘した。
% % またある参加者は、\textit{``システムが過剰に質問をしたことでフラストレーションを感じたことがあった''} [P3]と質問の多さを指摘し、別の参加者は、\textit{``様々な状況を想定してより多くの質問をして欲しかった''} [P8]と質問の少なさを指摘した。
% \red{
% The results suggest that the questions and options generated by ResQ were sufficiently practical; however, there is still room for improvement in their quality. 
% First, as in the interview results from Study 1, participants pointed out inaccuracies in question generation. 
% One participant noted, \textit{``there were instances where the AI misunderstood the relationship between myself and the other person when generating questions''} [P2], highlighting the low accuracy of the generated questions.
% Additionally, one participant pointed out the long generation time, explaining, \textit{``there were times when generating questions took too long, and I felt frustrated because there was nothing I could do during that time''} [P1]. 
% Another participant mentioned feeling frustrated by the excessive number of questions, stating, \textit{``there were times when the system asked too many questions, which I found frustrating''} [P3]. 
% On the other hand, another participant expressed dissatisfaction with the limited number of questions, saying, \textit{``I wanted the system to generate more questions that accounted for a wider range of scenarios''} [P8].
% }
