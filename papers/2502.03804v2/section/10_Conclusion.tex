\section{Conclusion}
\red{In formal email communication, users are often required to read detailed (lengthy or complex) emails. 
Crafting appropriate responses to such emails is time-consuming and may lead to overlooked sender requests or delayed responses, causing communication issues.}
Thus, we propose \red{QA-based approach}, which leverages LLM-based question generation to help users create efficient and high-quality replies by generating multiple question-answer pairs related to the received email content.
\red{To examine the comprehensive impact of the QA-based approach on both email senders and recipients, we conducted controlled and field experiments using our prototype system, \textit{ResQ}.
Our findings demonstrate that structuring email content into question-answer pairs improves efficiency, reduces cognitive load, and lowers barriers to initiating responses. 
Additionally, this approach enhances email quality and may leave a better impression on recipients.
However, our findings also revealed challenges, including a potential reduction in user agency and an increased psychological distance in communication. 
These trade-offs emphasize the need for adaptive designs that balance efficiency with personalization and user control.
% The QA-based approach shows promise for applications beyond email communication in domains requiring structured response generation. 
Future research should investigate the long-term effects of such systems on user behavior, cross-cultural differences in adoption, and the effectiveness of the QA-based approach across varying email characteristics}
% Future research should explore its long-term impacts, cross-cultural applicability, and integration into diverse communication platforms to optimize both efficiency and authenticity in AI-mediated interactions.}

% original
% In workplace email communication, users are often required to read lengthy emails and craft appropriate responses, a time-consuming task that may lead them to overlook parts of the sender's request or delay their response, causing communication issues.
% Thus, we propose \textit{ResQ}, which leverages LLM-based question generation to help users create efficient and high-quality replies by generating multiple question-answer pairs related to the received email content.
% Our controlled and field experiments confirmed that compared to a prompt-based approach, ResQ significantly improved email replying efficiency, reduced cognitive load, and lowered the barriers to task initiation.
% Additionally, AI support was shown to improve the quality of emails and enhance the recipient's impression.
% However, we observed that ResQ decreased users' sense of agency and control and enlarged the psychological distance between email senders and receivers.
% We also discussed communication scenarios where QA-based approach might be effective for AI-mediated communication.