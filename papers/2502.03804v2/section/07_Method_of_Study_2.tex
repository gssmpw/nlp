\section{Method of Study 2}
% この実験の目的などを書く
% Study1は、ResQのdesignは、フォーマルな場面における返信作業において、作業者の作業効率を上げ、認知負荷を下げるのに有効であることを明らかにした
Study 1 has revealed that the ResQ design effectively enhances user efficiency and reduces cognitive load during response tasks in formal settings.
\blue{To further examine how our QA-based system influenced users' actual email replying practice, we conducted a field experiment for Study 2.} 
% to gain a qualitative understanding of how our QA-based method influenced the practice of email replies.}
% Next, we set out to examine how people used the QA-based approach in their actual email replying practice.
\subsection{Field study}
% Study 2 was designed to qualitatively assess how ResQ influenced the practice of email replies.

% 本実験では、chrome拡張機能の形でのプロトタイプを開発し、システムを用いてPC版のGmailで返信作業をしてもらうように依頼した
% 拡張機能では、返信作業の開始は、参加者がGmailの返信ボックス上の"Reply with AI"ボタンを押すことによって検出した
% 参加者の返信作業の開始が確認されると、メール内容がリモートサーバに送信されると同時に、新しい返信エディタが開き、数秒間の後に質問と選択肢が生成された
% 参加者は返信文章を執筆し、reply boxの下にあるReplyボタンを押すと、Gmail上のReply boxに執筆した文章がそのまま反映されるようにした
% なお、メール内容や参加者が執筆した内容は、プライバシー保護のため、実験者からはアクセスできず、サーバにも保存されないようにした
In this five-day field experiment, we developed a prototype as a Chrome extension and asked participants to use the system to reply to emails using Gmail on a PC.
The extension detected the initiation of the reply task when participants clicked the ``Reply with AI'' button in the Gmail reply box. 
Upon confirming the task, the email content was sent to a remote server, a new reply editor opened, and a question with options appeared after a few seconds. 
Participants composed their replies, and upon pressing the Reply button, their text was directly reflected in the Gmail reply box. 
To ensure privacy, neither the email content nor the participants' responses were accessible to the experimenters or stored on the server.
\red{The specific implementation details and user interface are provided in the appendix.}

% \red{In this five-day field experiment, we developed a prototype system consisting of a Chrome extension and a backend service to enable participants to reply to emails using Gmail on a PC. 
% The Chrome extension detected the initiation of the reply task when participants clicked the ``Reply with AI'' button in the Gmail reply box (see Fig.~\ref{fig_UI}).
% Upon clicking the button, the extension extracted the email content directly from Gmail’s DOM structure using JavaScript and sent it to a backend API endpoint implemented with FastAPI~\footnote{\url{https://fastapi.tiangolo.com}}.
% The backend, hosted on an AWS EC2 instance~\footnote{\url{https://aws.amazon.com/ec2/}}, received the email content and forwarded it to the OpenAI API~\footnote{\url{https://platform.openai.com/docs/}} to generate questions or reply suggestions. 
% These outputs were then returned to the Chrome extension and displayed to the participant in a new reply editor.
% Finally, participants revise the reply suggestions and submit them back to the Gmail reply box by clicking the ``Reply'' button.
% To ensure privacy, neither the email content nor the participants' responses were accessible to the experimenters or stored on the server.}
% \begin{figure*}[ht]
\centering
\includegraphics[width=\textwidth]{figure/UI.pdf}
\caption{UI of the Gmail Reply Box with the ``Reply with AI’’ Feature, used in Study 2. Pressing the ``Reply with AI’’ button opens the window shown in Fig.~\ref{fig_interface}}
\label{fig_UI}
\Description{This figure illustrates the user interface of the Gmail reply box as enhanced by the prototype system. The Reply with AI button, shown in blue on the right-hand side of the toolbar, allows users to activate the AI-assisted reply generation feature. When the button is clicked, the system extracts the email content and opens the window shown in Fig.~\ref{fig_interface}. The standard Gmail toolbar options, such as send, formatting, and attachment icons, remain.}
\end{figure*}

\subsection{Participants}
\begin{table*}[t]
\caption{\red{Backgrounds of participants in Study 2, including age, gender, job roles, frequency of AI tool usage, and use of AI for email purposes.}}
\Description{The table outlines the demographic information and AI usage patterns of nine participants in Study 2, including their age, gender, job roles, general AI tool usage, and the extent to which they utilize AI for email-related tasks. The participants include university students and office workers, with a mix of both male and female representatives, and their AI adoption varies from frequent to rare use. Participant P1 is a 28-year-old male office worker who uses AI tools daily, with 20–50\% of his email tasks supported by AI. P2, a 24-year-old female university student, frequently uses AI tools but does not employ them for email-related purposes. Similarly, P3, a 20-year-old female university student, frequently uses AI tools, with 20–50\% of her email tasks facilitated by AI. Participant P4, a 24-year-old female university student, engages in daily AI tool usage, relying on AI for 50–80\% of her email activities. P5, a 31-year-old male office worker, also uses AI tools daily, supporting 20–50\% of his email tasks. Likewise, P6, a 39-year-old female office worker, uses AI tools daily, with AI assisting in 20–50\% of her email-related tasks. In contrast, Participant P7, a 25-year-old male university student, rarely uses AI tools and does not employ them for email-related purposes. P8, a 23-year-old female office worker, frequently uses AI tools but applies them to less than 20\% of her email tasks. Lastly, P9, a 38-year-old male office worker, rarely engages with AI tools, with AI playing a role in less than 20\% of his email-related tasks.}
\label{tab_study2_participants_background}
\red{
\begin{tabular}{cccccc}
\hline
Participants & Age & Gender & Job       & AI Tool Usage&AI for Email Usage\\ \hline
P1           & 28  & M& Office Worker& Daily&20-50\%\\
P2           & 24  & F& Univ. Student& Frequently&Never\\
P3           & 20  & F& Univ. Student& Frequently&20-50\%\\
P4           & 24  & F& Univ. Student& Daily&50-80\%\\
P5           & 31  & M& Office Worker& Daily&20-50\%\\
P6           & 39  & F& Office Worker& Daily&20-50\%\\
P7           & 25  & M& Univ. Student& Rarely&Never\\
 P8           & 23  & F& Office Worker& Frequently&<20\%\\
P9& 38& M& Office Worker& Rarely&<20\%\\ \hline
\end{tabular}
}
\end{table*}
As shown in Tab.~\ref{tab_study2_participants_background}, nine participants (four males and five females, aged 20-39) were recruited via a local Japanese participant recruiting platform.
The average age of the participants was 28.0 (SD = 6.7)\blue{, and they reported engaging in more than three email communications per day on average.}
This study was approved by the ethical review board of the authors' institute.
The participants were paid approximately \$37 USD for participation.
% 参加者は、実験やシステムの説明を30分間受けた後、5日間システムを使用し、その後1時間のインタビューを受けた
The participants received a 30-minute explanation of the experiment and system, used the system for five days, and subsequently participated in a one-hour interview.

\subsection{Procedure}
Participants first read the study instructions and their right to participate, after which they consented to participate in the experiment. 
Next, they were provided with an explanation of the study's purpose and instructions on how to use the QA-based system. 
Following this, they installed the Chrome extension we developed and confirmed its functionality according to the provided instructions.
Participants were asked to use the system for five days, during which they were free to use it to reply to emails at any time. 
After the five-day period, a one-hour semi-structured interview was conducted. 
During the interview, participants were asked a series of questions, such as: \textit{``Can you tell us your overall impression of using the system?''} \textit{``How did your email replying practice change before and after using the system?''} \textit{``What changes did you notice in the emails you composed?''} and \textit{``How did your relationship with the communication counterpart change after using this system?''}
This study was conducted remotely with all participants.

\subsection{Data Analysis}
% インタビュー記録をコード化し、分析するために、我々はインタビューデータを録音し、インタビューデータの文字起こしを行った上で、bottom-up approach rooted in grounded theoryを使用した
% 具体的には、暫定的なラベルを特定するためのopen codingと、ラベル間の関係を見出すためのaxial codingを行った
% coding結果、email-replying process, quality of the email responses, relationship between sender and recipient, future use intentinoの4つの主要なテーマを特定した
To analyze the interview data, we transcribed the interview recordings. 
We followed the thematic analysis method~\cite{braun_2006_thematicanalysis} to analyze the open-ended responses. 
One of the authors open-coded all relevant concepts that were related to our research questions, assigned labels that featured the concepts, and grouped labels into different themes. 
Next, the authors discussed the quotes and themes repeatedly.
Finally, the developed themes were compared and adjusted among all participants until they thoroughly covered the data.
As a result of the coding process, we identified four main themes: the email-replying process, the quality of the email responses, the relationship between the sender and recipient, and perceived risk.