% 150words
% Replying to workplace emails that are typically long and require politeness is time-consuming and cognitively demanding.
% Replying to lengthy and polite workplace emails is often time-consuming and cognitively demanding.
% takes time to understand and reply
\red{Replying to formal emails is time-consuming and cognitively demanding, as it requires crafting polite phrasing and providing an adequate response to the sender's demands.}
% \red{Replying to formal emails, which often takes time to understand and require polite phrasing, is time-consuming and cognitively demanding.}
Although systems with Large Language Models (LLM) were designed to simplify the email replying process, users still need to provide detailed prompts to obtain the expected output.
Therefore, we proposed and evaluated an \red{LLM-powered question-and-answer (QA)-based approach} for users to reply to emails by answering a set of simple and short questions generated from the incoming email.
We developed a prototype system, \textit{ResQ}, and conducted controlled and field experiments with 12 and \red{8} participants.
Our results demonstrated that \red{the QA-based approach} improves the efficiency of replying to emails and reduces workload while maintaining email quality, compared to a conventional prompt-based approach that requires users to craft appropriate prompts to obtain email drafts.
We discuss how \red{the QA-based approach} influences the email reply process and interpersonal relationship dynamics, as well as the opportunities and challenges associated with using a QA-based approach in AI-mediated communication.

% original
% Replying to lengthy and polite workplace emails is often time-consuming and cognitively demanding.
% Although systems with Large Language Models were designed to simplify the email replying process, users still needed to provide detailed prompts to obtain the expected output.
% Therefore, we proposed and evaluated a question-and-answer-based approach for users to reply to emails by answering a set of simple and short questions generated from the incoming email.
% We developed a prototype system, \textit{ResQ}, and conducted both controlled and field experiments with 12 and 9 participants.
% Our results demonstrated that ResQ improves the efficiency of replying to emails and reduces workload while maintaining email quality compared to a conventional prompt-based approach that requires users to craft appropriate prompts to obtain email drafts.
% We discuss how ResQ influences the email reply process and interpersonal relationship dynamics, as well as the opportunities and challenges associated with using a QA-based approach in AI-mediated communication.