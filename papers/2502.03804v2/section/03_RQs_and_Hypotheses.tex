\section{Research Questions and Hypotheses}
This paper aims to explore the effectiveness and potential risks of a QA-based response-writing support method by addressing the following three research questions:
% \begin{enumerate}[RQ1:]
%     \item How does a QA-based response-writing support approach affect users’ email-replying process?
%     \item How does a QA-based response-writing support approach affect the quality of the email response?
%     \item How does a QA-based response-writing support approach affect the perceived relationship between email sender and recipient?
% \end{enumerate}

\begin{enumerate}[RQ1:]
    \item How does a QA-based response-writing support approach affect users’ email-replying process?
    \item How does a QA-based response-writing support approach affect the quality of the email response?
    \item How does a QA-based response-writing support approach affect the perceived relationship between email sender and recipient?
\end{enumerate}

To answer the three research questions, we formed three sets of hypotheses.
The first set of hypotheses investigates the impact of a QA-based system on users' email-replying process.
AI-powered text generation reduces user input, saves time, and enhances efficiency~\cite{bastola2024llmbasedsmartreplylsr}. 
It also helps users quickly grasp email content with less cognitive effort, particularly through text summarization and list formatting, which enhances productivity~\cite{tarnpradab2017toward, nandhini2013use, modaresi2017commercial, daniel1998influence}. 
This suggests that presenting questions in a list format could streamline email responses, reducing the need for detailed prompts. 
Based on these insights, we propose the following hypotheses:
% \begin{enumerate}[\textrm{H1-}a:]
%     \item QA-based system enhances users’ email replying efficiency.
%     \item QA-based system reduces users’ cognitive load while replying to email.
% \end{enumerate}

\begin{enumerate}[\textrm{H1-}a:]
    \item QA-based system enhances users’ email replying efficiency.
    \item QA-based system reduces users’ cognitive load while replying to email.
\end{enumerate}

As a result, we expect users’ perceived work efficiency to improve.
Furthermore, since the QA-based system suggests appropriate language and helps create responses that align with the recipient’s needs, we anticipate that users’ satisfaction with their email replies will increase.
Thus, we propose the following hypothesis:
\begin{enumerate}[\textrm{H1-}c:]
    \item QA-based system enhances users' satisfaction with completing email response tasks, thereby being favorably received by users.
\end{enumerate}

Moreover, as described above, reducing users’ burden and improving their satisfaction may enhance their confidence in their tasks, which could lower their hesitation to begin working~\cite{schouwenburg1992procrastinators}.
Additionally, AI outputs that engage users’ curiosity may help trigger task initiation~\cite{brandtzaeg2017why, ling2021factors}.
Thus, we propose the following hypothesis:
\begin{enumerate}[\textrm{H1-}d:]
    \item QA-based system lowers the barriers to initiating email response tasks.
\end{enumerate}

According to previous research, there is a trade-off between the degree of AI intervention and the sense of agency and control, with higher levels of AI involvement shown to diminish these perceptions~\cite{Fu2023Comparing, Draxler2024The}.
Given that our QA-based approach also involves AI intervention during the phase where users create prompts for the LLM, the following hypothesis can be derived:
\begin{enumerate}[\textrm{H1-}e:]
    \item QA-based system diminishes users’ sense of agency and reduces their sense of control of the content.
\end{enumerate}

The second hypothesis concerns the quality of email responses.
AI support can be helpful in ensuring appropriate language use and grammar~\cite{fu2024text}. 
Furthermore, the QA-based approach is expected to assist users in correctly understanding the intent and demands of received emails and in verifying whether their responses meet these requirements.
Based on this, we propose the following hypothesis:
% \begin{enumerate}[\textrm{H2}:]
%     \item QA-based system enhances the perceived quality of the email response.
% \end{enumerate}
\begin{enumerate}[\textrm{H2}:]
    \item QA-based system enhances the perceived quality of the email response.
\end{enumerate}

The third set of hypotheses investigates the perceived relationship between email sender and recipient.
When users use the QA-based approach, it is expected that their communication partners will receive high-quality messages more quickly. 
Thus, the following hypothesis is derived.
\begin{enumerate}[\textrm{H3-}a:]
    \item QA-based system makes a good impression on the user's communication partner.
\end{enumerate}

On the other hand, when users create messages using the AIMC tool, they may feel a sense of discomfort with the message and guilt for not having fully composed it themselves~\cite{fu2024text}. 
We hypothesized that a QA-based approach would further intensify this discomfort by reducing the user’s sense of agency and control more than previous approaches.
\begin{enumerate}[\textrm{H3-}b:]
    \item QA-based system enlarges the psychological distance that users perceive toward their communication partners.
\end{enumerate}