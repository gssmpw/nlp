\section{Discussion}
\red{Through a controlled experiment (Study 1) and a field study (Study 2), we investigated the impact of the LLM-powered QA-based approach on both senders and receivers.
In this section, we discuss the findings (Fig.~\ref{tab_summary}) of the research and the key considerations for designing QA-based systems.}
\begin{table*}[t]
\caption{Research Questions and Key Findings}
\label{tab_summary}
\centering
% \resizebox{\textwidth}{!}{
\begin{tabular}{>{\raggedright\arraybackslash}p{0.08\linewidth}>{\raggedright\arraybackslash}p{0.28\linewidth}>{\raggedright\arraybackslash}p{0.28\linewidth}>{\raggedright\arraybackslash}p{0.28\linewidth}}
\hline
 & \textbf{RQ1: How does a QA-based response-writing support approach affect workers’ email-replying process?} & \textbf{RQ2: How does a QA-based response-writing support approach affect the quality of the email response?} & \textbf{RQ3: How does a QA-based response-writing support approach affect the perceived relationship between email sender and recipient?} \\ \hline
\textbf{Key Findings} & 1. QA-based approach \textbf{reduced workload} for email comprehension and prompt creation and \textbf{improved work efficiency}. (H1-a, supported; H1-b, supported, Sec.~\ref{sec:result1_efficiency},~\ref{sec:result1_prompt_character_counts},~\ref{sec:result1_cognitive_load},~\ref{sec:result1_difficulty_in_understanding},~\ref{sec:result1_interview_RQ1},~\ref{sec:result2_efficiency}) 

2. QA-based approach \textbf{reduced the difficulty} of initiating the email replying task. (H1-d, supported, Sec.~\ref{sec:result1_initiating},~\ref{sec:result1_interview_RQ1},~\ref{sec:result2_initiating_the_action})

3. QA-based approach \textbf{decreased the sense of agency and control}. (H1-e, supported, Sec.~\ref{sec:result1_agency},~\ref{sec:result1_interview_RQ1},~\ref{sec:result2_agency_control})

4. QA-based approach \textbf{improved satisfaction} with the emails they wrote and willingness to use ResQ in the future. (H1-c, supported, Sec.~\ref{sec:result1_satisfaction},~\ref{sec:result1_interview_RQ1},~\ref{sec:result2_efficiency})& Writing emails with QA-based approach and Prompt-based approach led to \textbf{increased email quality} than No-AI condition. (H2, partially supported, Sec.~\ref{sec:result1_quality},~\ref{sec:result1_interview_RQ2}~\ref{sec:result2_quality})& 1. Writing emails with QA-based approach \textbf{\blue{did not lead to improved perceived impression of users by their counterparts}}. (H3-a, not supported, Sec.~\ref{sec:result1_impression},~\ref{sec:result2_self-presentation})

2. Writing emails with QA-based approach led to \textbf{increased psychological distance} between users and their counterparts than No-AI condition. (H3-b, partially supported, Sec.~\ref{sec:result1_psychological_distance},~\ref{sec:result1_interview_RQ3},~\ref{sec:result2_psychological_distance})\\ \hline
\end{tabular}
% }
\Description{This table summarizes the three research questions (RQs) investigated in the study and highlights the key findings associated with each. RQ1: How does a QA-based response-writing support approach affect workers’ email-replying process? Key Findings: 1. The QA-based approach reduced workload for email comprehension and prompt creation, leading to improved work efficiency. This supports hypotheses H1-a and H1-b. 2. It reduced the difficulty of initiating the email replying task, supporting hypothesis H1-d. 3. The approach decreased the sense of agency and control among users, supporting hypothesis H1-e. 4. Users experienced improved satisfaction with the emails they wrote and showed a greater willingness to use ResQ in the future, supporting hypothesis H1-c. RQ2: How does a QA-based response-writing support approach affect the quality of the email response? Key Findings: Writing emails using both the QA-based and prompt-based approaches led to an increase in email quality compared to the No-AI condition. This partially supports hypothesis H2. RQ3: How does a QA-based response-writing support approach affect the perceived relationship between email sender and recipient? Key Findings: 1. Writing emails with QA-based approach did not lead to improved perceived impression of users by their counterparts, meaning hypothesis H3-a was not supported. 2. Writing emails with the QA-based approach led to an increase in psychological distance between users and their counterparts compared to the No-AI condition. This partially supports hypothesis H3-b.}
\end{table*}


% \textbf{Summary} & Workers evaluated the system’s benefits (improvements in efficiency, cognitive load, and satisfaction), accepted a certain reduction in the agency, and showed a willingness to use it in the future. & It became possible to create responses that appropriately addressed email requests while maintaining politeness. (H2, supported) & ResQ improved workers' impression. Perceived psychological distance from others depends on a balance between a sense of agency and communication satisfaction. \\ \hline
\subsection{Impact of the QA-based Approach}
\subsubsection{Enhancing Efficiency and Reducing Cognitive Load}
% Our studies indicate that the QA-based approach improves efficiency and reduces cognitive load when composing email replies (Sec.~\ref{sec:result1_efficiency},~\ref{sec:result1_prompt_character_counts}, \ref{sec:result1_cognitive_load},~\ref{sec:result1_difficulty_in_understanding},~\ref{sec:result1_interview_RQ1},~\ref{sec:result2_efficiency}). 
Our studies indicate that the QA-based approach improves efficiency and \blue{suggests a reduction in} cognitive load when composing email replies (Sec.~\ref{sec:result1_efficiency},~\ref{sec:result1_prompt_character_counts}, \ref{sec:result1_cognitive_load},~\ref{sec:result1_difficulty_in_understanding},~\ref{sec:result1_interview_RQ1},~\ref{sec:result2_efficiency}). 
One possible explanation is that the QA-based approach helps users focus on the most relevant details, simplifying email comprehension compared to prompt-based methods.
% understand key information and organize their responses effectively.
% By emphasizing key information through AI-generated questions, this approach allows users to focus on the most relevant details, simplifying email comprehension compared to prompt-based methods.
% This finding aligns with cognitive load theory~\cite{sweller2011cognitive}, suggesting that reducing extraneous cognitive load enables users to perform tasks more efficiently.
Additionally, the QA-based approach reduces the burden of prompt creation by partially replacing the task of crafting prompts with the simpler task of answering questions.
\blue{Our finding suggests that future email systems could use this QA-based approach to mediate the email exchange process.}
% 更なる効率改善、負荷低減のためには、質問の量や質、提示する順序の最適化が効果的な可能性がある
% また選択肢も調整することができる
% 例えばシンプルなYes/Noの選択肢は有用でしたが、スケジュールツールや自由入力フィールドのような柔軟な入力も求められたので、カスタマイズ可能な入力形式(例: スケジューリング用のカレンダーセレクター)を導入することで、さらに使いやすさを向上させることができる可能性がある
% To further enhance efficiency and reduce cognitive load, optimizing the quantity, quality, and sequence of questions may be beneficial. 
% Additionally, refining response options could improve usability. 
% For instance, while simple Yes/No choices were effective, users also expressed a need for more flexible input options, such as scheduling tools or free-text fields. 
% Thus, introducing customizable input formats, such as a calendar selector for scheduling, could further enhance usability and streamline the email composition process.

\subsubsection{Potential Reduction in Sense of Agency and Control}
\red{While the QA-based approach enhanced users' efficiency, our studies also revealed a potential trade-off in users' sense of agency and control (Sec.~\ref{sec:result1_agency},~\ref{sec:result1_interview_RQ1},~\ref{sec:result2_agency_control}).
Some participants reported a decreased sense of authorship, feeling more like editors than creators of their emails. 
This reduction in agency may be due to the diminished amount of text input required from the user, as the AI takes a more active role in content generation.
% またこれらの感覚は、ユーザの好みに強く影響を与えることも明らかとなり、agencyの維持を望むような重要な場面などではこのアプローチは使用したくないと報告する参加者もいました。
Moreover, we found that the sense of agency influenced users' preferences for future usage.
% For example, some participants expressed a reluctance to use this approach in critical contexts where preserving their sense of agency was particularly important.
% However, this effect was not uniform across all users. 
Among those participants who still maintained their sense of agency, we found that they tended to actively review and modify the AI-generated content to reflect their personal style and intentions.
% これはQA-based approachのようにAIの介入が大きくとも、ユーザが積極的に内容を確認・編集することで、ユーザの主体性を維持できる可能性があるということを示唆している
\blue{This suggests that even when AI intervention is substantial, users can maintain a sense of authorship by actively engaging with and refining the AI's suggestions.}
% These insights emphasize that the importance of the agency may depend on individual user preferences, the context of the communication, and the extent to which users personalize the AI-generated output.
}
% ユーザにとって望ましいagencyのレベルに最適化するために、シチュエーションやユーザの好みに応じて質問や選択肢の数、提案のレベル(文レベル、メッセージレベルなど)を変えるなど、AIの介入方法や度合いを変化させるアプローチが効果的な可能性がある
\blue{To optimize users' level of agency, adapting the degree of AI intervention in the email construction process can be helpful.
% based on the situation and user preferences may be effective. 
% This could involve 
For instance, by adjusting the number and type of AI-generated questions or varying the levels of AI-generated suggestions~\cite{Fu2023Comparing}, ranging from word-level to message-level.} 
% (\textit{e.g.}, sentence-level vs. message-level recommendations).}

% \subsubsection{\red{Improvement in Email Quality and Users' Impressions of Communication Partners}}
% \red{The QA-based approach was found to enhance the perceived quality of email responses (Sec.~\ref{sec:result1_quality},~\ref{sec:result1_interview_RQ2}~\ref{sec:result2_quality}). 
% Our studies revealed that emails composed with AI assistance were more polite, well-structured, and addressed multiple requests without omissions. 
% Particularly, the QA-based approach can support users in creating emails that meet the sender's demands by organizing them through AI-generated questions. 
% Furthermore, by enhancing efficiency and lowering the barriers to initiating tasks, this approach facilitates quicker response times, which, in turn, positively influence users' impressions of their correspondents~\cite{yoram2011online, Resendes2012Send, vignovic2010computer}. 
% In summary, these findings indicate that the QA-based approach not only supports users in creating higher-quality email responses but also fosters more positive impressions.}

\subsubsection{Possibility of Improving Relationship between Email Sender and Recipient}
\red{Our studies yielded mixed results regarding the impact of the QA-based approach on the psychological distance between users and their counterparts (Sec.~\ref{sec:result1_psychological_distance},~\ref{sec:result1_interview_RQ3},~\ref{sec:result2_psychological_distance}). 
% Some参加者は素早く、高い質のメールを送信できたことや、またそれによって相手からの返信が早くなったことで、相手との間に感じる距離感を近く感じた。
% 一方で他の参加者は、作業の労力が減ったことや、自分が普段使わない言葉を使っていることに気がついたことで、felt a sense of increased distance.
Some participants reported that they were able to send emails more quickly and with high quality, which in turn led to faster responses from others and a reduced sense of distance in their interactions.
In contrast, other participants experienced an increased sense of distance, which has also been reported in the previous studies~\cite{Fu2023Comparing,arnold2020predictive}. 
They noted that the reduced communication effort and the use of unfamiliar language made interactions feel less personal or authentic.
% Some participants felt closer to their counterparts due to quicker response times and higher-quality emails.
% Others, however, felt a sense of increased distance, partly because the AI-mediated communication felt less personal or authentic.
% These findings align with previous studies~\cite{Fu2023Comparing,arnold2020predictive} indicating that while AI can contribute to maintaining a professional tone, it may also result in less authenticity when AI-generated language deviates from a user's usual style, thereby increasing psychological distance.
% These divergent users' feedback suggests that while the QA-based approach can enhance certain aspects of communication, it may also inadvertently introduce a sense of impersonality. 
The degree of the perceived distance may depend on factors such as the nature of the relationship (\textit{e.g.,} colleagues vs. friends), the user's reliance on AI-generated language, and individual preferences regarding AI-mediation in communication.}

\subsection{Opportunities and Challenges of Introducing QA-Based Approach}
% \subsubsection{Situations where QA-based is Useful in Email Communication}
Our results indicate that the QA-based approach 
% effectively streamlines the process of composing responses and enhances the quality of replies.
% Therefore, this approach 
is particularly useful in situations where speed and high-quality responses are prioritized over email personality or a strong sense of personal agency. 
Contexts such as business, customer service, and technical support can greatly benefit from the QA-based approach, as they often require efficient and structured communication.
% Additionally, the QA-based AI-assisted email replying mechanism proves highly effective in scenarios where maintaining a neutral tone and low emotional engagement is appropriate, such as initial contacts or interactions among weak ties. 
% In these cases, the system’s ability to generate standardized, professional language supports users in composing suitable replies promptly.
% In situations where maintaining a neutral affect or low emotional engagement is encouraged (\textit{e.g.}, in initial contacts and weak ties), a QA-based AI-assisted email replying mechanism can be effective.

However, for more delicate or personal email exchanges, users may prefer more tailored interventions. 
In such situations, users can adjust the level of involvement of AI intervention.
% or automatically tuning the intervention based on the email content or the user’s past behavior could better meet their needs.
% This customization can help maintain the authenticity and personal touch necessary for meaningful communication.
% Future research should explore how different levels of AI-mediated intervention affect users’ sense of agency and email construction behavior across various communication contexts.
Furthermore, there is a risk that users could become overly reliant on technology to mediate their interpersonal communication. 
Our interviews revealed that users might become accustomed to trusting AI-generated questions and drafts due to the efficient outcomes. 
Consequently, they may become less diligent in reading the emails they receive or in reviewing the responses they send carefully.
This over-reliance could lead to miscommunication or the omission of important details, thus undermining the primary goal of using AI to improve communication efficiency.
Future research should explore how different levels of AI-mediated intervention can be designed to influence users’ sense of agency and email construction behavior for various communication purposes.


% Beyond email communication, a QA-based approach utilizing LLMs can alleviate users' workload while preserving a degree of agency in areas that require structured information gathering and intent-driven interactions.
% % Beyond email communication, QA-based systems powered by LLMs have the potential to reduce users' workload while maintaining a certain level of agency in domains requiring structured information elicitation and intent-driven interactions. 
% A QA-based approach can streamline tasks such as drafting structured documents (\textit{e.g.}, academic rebuttals), clarifying user needs (\textit{e.g.}, in customer support or medical consultations), and facilitating team consensus by presenting key points as questions. 
% % Additionally, emerging techniques, such as generating follow-up questions during conversations~\cite{hu2024designing} or creating multiple-choice questions to assess comprehension~\cite{cheng2024treequestions}, highlight the versatility of LLM-powered questioning systems. 
% % Integrating these innovations into QA-based systems could unlock new applications across a wide range of fields.


% Beyond email communication, a QA-based approach utilizing LLMs can alleviate users' workload while preserving a degree of agency in areas that require structured information gathering and intent-driven interactions.
% First, QA-based systems can streamline tasks that require the creation of formal response documents (\textit{e.g.}, academic rebuttals) or complex online applications (\textit{e.g.}, Visa application) by presenting key points to understand and address as questions.  
% Also, in settings requiring consensus-building, such as team projects, QA-based systems may facilitate discussions by presenting questions aimed at identifying mutual goals, challenges, or uncertainties. The system can help clarify differing needs and provide feedback to streamline the decision-making process, ultimately improving collaboration.

% \subsubsection{\red{Applicability Beyond Email Replying}}
% \label{sec:discuss_applicability_beyond_email_replying}
% \red{Our findings suggest that LLMs can generate questions to elicit users' intentions and help them organize their thoughts, thereby enabling more efficient and effective outputs through user interaction. 
% Moreover, approaches such as generating follow-up questions based on user responses during a conversation~\cite{hu2024designing}, or even creating multiple-choice questions to assess user comprehension~\cite{cheng2024treequestions}, demonstrate the evolving versatility of LLM-powered questioning systems. 
% This suggests that QA-based systems leveraging LLMs hold significant potential in a variety of domains where structured information elicitation and intent communication are required, extending far beyond email composition.}

% % 1
% \red{Beyond email communication, a QA-based approach utilizing LLMs can alleviate users' workload while preserving a degree of agency in areas that require structured information gathering and intent-driven interactions.
% First, QA-based systems can streamline tasks that require the creation of formal response documents (\textit{e.g.}, academic rebuttals) or complex online applications (\textit{e.g.}, Visa application) by presenting key points to understand and address as questions.  
% Also, in settings requiring consensus-building, such as team projects, QA-based systems may facilitate discussions by presenting questions aimed at identifying mutual goals, challenges, or uncertainties. The system can help clarify differing needs and provide feedback to streamline the decision-making process, ultimately improving collaboration.}

% 我々のfindingsは、LLMにはすでにユーザの意図を引き出したり、考えをまとめたりするための質問を生成でき、ユーザとのインタラクションを通じてより良いアウトプットを効率的に出力できることを示唆している
% また今回のResQのように、受信メールに基づいて質問を一度だけ生成するのではなく、会話中のユーザーの回答に基づいてフォローアップ質問をさせたり~\cite{}、さらにはユーザの理解度を評価するための多肢選択問題を生成する~\cite{}アプローチもとられている
% したがってLLM を利用した QA ベースのシステムは、電子メールの作成にとどまらず、構造化された情報の引き出しと意図の伝達が求められるさまざまな分野で潜在的可能性を秘めていることを示唆している
% 例えば...
% まず、構造的な文書作成をする必要がある場面において、QA-based systemは作業を効率化できる可能性がある
% 例えば、drafting academic rebuttals, legal case summaries, or detailed project reportsの作成の際、QA-based systemが重要な論点を質問として提示し、ユーザはそれに回答することで考えを整理するとともに、それが反映されたドラフトを受け取ることができる
% また、他者のニーズを明確化する必要がある場面において、QA-based systemは役に立つ可能性がある
% 例えば、カスタマーサポートや医療問診の際、QA-based systemが、事前に顧客の問題を把握するための質問を提示し、顧客がそれに回答することで、顧客は自分のニーズを明確化するとともに、サポートする側は顧客のニーズを効率的に知ることができる
% さらに、合意形成が求められる場面において、複数のステークホルダーの意見を整理することで、合意形成を支援できる可能性がある
% 例えば、チームのプロジェクトにおいて、QA-based systemが、互いの目的や問題点、不明点等を洗い出すための質問を両者に提示することで、互いのニーズを明確化し、議論を円滑化できる可能性がある

% Designing the Conversational Agent: Asking Follow-up Questions for Information Elicitation
% 対話型エージェント(CAs)がインタビューや情報収集の場面で、事前に決まった質問だけでなく、会話中のユーザーの回答に基づいたフォローアップ質問を生成する能力を向上させる。
% ヒトのインタビュアーが用いるフォローアップ質問のテクニックを取り入れることで、CAsが有益な情報を引き出せるように設計する。

% TreeQuestion
% 1. 背景
% オープンエンド質問(自由回答形式)は、学生の理解を評価するために使われますが、AI(例えばChatGPT)の利用により、学生が容易に長文回答を生成できる時代において課題となっています。
% 教師は依然として回答を読む時間や学習成果を推測する負担を抱えています。
% 2. TreeQuestionシステム
% 目的: 教師が概念学習成果を評価するための多肢選択問題を効率的に作成する支援。
% 仕組み:
% 大規模言語モデルを利用して、与えられた概念を基に多肢選択問題を生成。
% 質問はツリー構造で整理され、異なる理解レベル(記憶、理解、応用、分析、評価、作成)に対応。
% 誤解を誘う選択肢(Distractors)も生成し、学生が正しい選択肢を選べるかどうかで学習成果を評価。
% 3. 特徴
% 人間とAIの協働:
% 教師はAIが生成した内容を検証・修正し、適切な質問を作成。
% 「探索(Explore)- 検証(Validate)- 生成(Generate)」という段階的プロセスを採用。
% 効率向上:
% オープンエンド質問と比較して、TreeQuestionによるMCQ作成と採点の時間は大幅に短縮される。

% 1. 相手のニーズを明らかにする必要がある場合(他者のニーズを明確化する)
% % For instance, in customer support and technical assistance, QA-based systems can guide users through troubleshooting processes by asking targeted questions and providing predefined response options. 
% % This approach can reduce cognitive load for both customers and support agents, improving the overall experience and efficiency of problem resolution.

% 2. 個人間、あるいは集団間において、合意形成を取る必要がある場合(両者のニーズを明確化する)。1の延長線上かもしれない。
% 複数のステークホルダーが関与する場において、QAベースシステムが議論の論点を動的に生成し、意思決定をサポートすることで、合意形成や問題解決を促進できる可能性がある。
% 例えば、チーム会議やプロジェクト計画において、互いのニーズを整理し、それを元にシステムが次のステップを提案することが可能かもしれない。

% 3. メールと同様に、特定の個人や集団に対して、考えや情報を整理し提供する必要がある場合(構造的な文書作成を効率的にする)
% % Moreover, content creation for structured documents, such as academic rebuttals, legal case summaries, or collaborative reports, can benefit from QA-based systems. 
% % By generating task-relevant questions based on the XXX, these systems could assist users in organizing their thoughts, ensuring that all critical aspects are addressed systematically.

% \subsection{\red{Design Implications for QA-based Systems in Broader Contexts}}
% To ensure the effectiveness of QA-based systems across diverse applications, we identify several design considerations from our findings.

% \subsubsection{\red{Optimizing User Experience in QA-based Systems}}
% \red{The results of our studies suggest that customizing the questions and options generated by QA-based systems according to the task, communication context, and user characteristics can enhance the user experience. }

% \paragraph{\red{\textbf{Content of Questions}}}
% % 何をすべきか、なぜそうすべきかの順に書く
% % personalizationとかはあまり言わない方がいい
% \red{First, the system should generate questions that are relevant, precise, and easy to answer, while accurately reflecting the sender's intent. 
% For example, our studies revealed Yes/No questions or specific scheduling-related questions were particularly useful.
% One potential solution is to provide the system with prompts containing relevant user information, which can improve the accuracy and relevance of the generated questions.}

% \paragraph{\red{\textbf{Quantity of Questions}}}
% \red{Next, the system should maintain an optimal balance in the number of questions generated.
% Our findings revealed mixed reactions to the number of questions provided: while some participants appreciated confirmation questions (\textit{e.g.,} ``Do you understand XX?'') for ensuring clarity, others found them redundant. 
% Furthermore, generating too many questions often led to verbose and unfocused replies.
% To address this, the system should dynamically adjust the number of questions based on the complexity of the task and the user’s preferences. 
% Additionally, refining prompt designs to produce concise and relevant text can mitigate the issue of excessive verbosity in responses.}

% \paragraph{\red{\textbf{Order of Questions}}}
% \red{The logical sequencing of questions also plays a vital role in enhancing usability. 
% The system used in this study generated questions in accordance with the flow of the email, which could help participants better understand the content. 
% However, there is potential to further enhance the system by prioritizing questions based on their importance or relevance.
% Future developments could consider adjusting the order of questions based on their importance or relevance. 
% In addition, UI enhancements, such as grouping questions by priority or task category, may help users navigate the interaction more intuitively.}

% \paragraph{\red{\textbf{Efficiency of Question Generation}}}
% \red{The time taken to generate questions emerged as a source of frustration for some participants. 
% This issue can potentially be addressed by improving the processing speed of LLMs or implementing pre-generation mechanisms. 
% For example, questions could be generated in advance, before the user opens the email, thereby reducing waiting times and improving overall efficiency.}

% \paragraph{\red{\textbf{Balance and Flexibility of Options}}}
% \red{Finally, the system should ensure a balance in the number and diversity of response options provided. 
% Participants reported a decline in user experience when options did not align with their intent or when irrelevant options were presented. 
% Simple Yes/No choices are useful, but participants also expressed a need for more flexible input methods, such as scheduling tools or free-text fields.
% To address this, introducing customizable input types based on the specific task (\textit{e.g.,} UI components like calendar selectors for scheduling) can further enhance usability.}

% \subsubsection{Strategies to Maintain User Agency and Authenticity}
% なんでそれが重要か、何ができるかをfindingsから書く?
% To mitigate potential reductions in agency, allowing users to adjust the level of AI intervention or providing options to customize the AI's contributions may be beneficial....

% \subsubsection{Mitigating Risks of Over-Reliance on AI}
% % We found one concern associated with using QA-based systems is the potential risk of users becoming overly reliant on the system. 
% While the QA-based system offers advantages, one concern is the potential risk of users becoming overly reliant on the technology. 
% Interviews revealed that users might become accustomed to trusting AI-generated questions and drafts due to their high quality and the desire to reduce workload. 
% Consequently, they may become less diligent in reviewing the emails they receive or the responses they send.
% This over-reliance could lead to miscommunication or the omission of important details, undermining the primary goal of using AI to improve communication efficiency.
% % Furthermore, our findings suggest that the QA-based approach may alter users’ perceptions of email correspondence, making it feel more like completing a survey than engaging in meaningful communication.
% % Given the increasing integration of AI systems into daily tasks, this shift indicates that email's purpose and role as a communication tool may change.
% Thus, further exploration is needed to understand the long-term impact of AIMC tools on email communication's changing roles and needs. 
% Moreover, it is important to investigate in which scenarios and how interventions by AI should be made to ensure that the essence of human interaction is not compromised.

% \subsubsection{\red{Design Implications}}
% \red{Our studies suggest that tailoring questions and options in QA-based systems based on the task, communication context, and user characteristics can improve user experience.
% % 以下では、QA-basedシステムの設計において、検討および調整することができるオプションについて説明する。
% }
% \begin{enumerate}[]
%     \item \red{\textbf{Content of Questions:}}
%     \red{The system should generate questions that are relevant, precise, and easy to answer, accurately reflecting the sender's intent.
%     For example, Yes/No questions or specific scheduling-related questions proved particularly useful.
%     Providing the system with prompts containing relevant user information can improve the accuracy and relevance of generated questions.}
%     \item \red{\textbf{Quantity of Questions:}}
%     \red{The system must balance the number of questions generated.
%     While some users valued confirmation questions (\textit{e.g.}, ``Do you understand XX?''), others found them redundant, and answering many questions led to verbose replies.
%     Dynamic adjustment based on task complexity and user preferences, along with concise prompt designs, can address this issue.}
%     \item \red{\textbf{Order of Questions:}}
%     \red{Logical sequencing enhances usability.
%     In this study, questions generated according to email flow helped participants understand the content.
%     Prioritizing questions by relevance or grouping them by priority or task category in the UI could further improve navigation.}
%     \item \red{\textbf{Efficiency of Question Generation:}}
%     \red{Delays in generating questions frustrated participants.
%     Improving LLM processing speed or implementing pre-generation mechanisms (\textit{e.g.}, generating questions before users open emails) can reduce waiting times and improve efficiency.}
%     \item \red{\textbf{Balance and Flexibility of Options:}}
%     \red{A balance in response options is crucial.
%     Participants found user experience declined when options were irrelevant or inflexible.
%     Simple Yes/No choices were helpful, but flexible inputs, such as scheduling tools or free-text fields, were also desired.
%     Customizable input types (\textit{e.g.}, calendar selectors for scheduling) can further enhance usability.}
% \end{enumerate}

\subsection{Limitations and Future Work}
While it is evident that the QA-based approach positively impacted users' workload, the quality of the emails they produced, and their relationship with recipients in formal email responses, this study had several limitations.
Though we tried to use a mixed-method study to triangulate the findings from the control experiment and field study, we acknowledged that the quantitative results could be limited.
Because of privacy concerns, we were unable to access participants' email content, and as a result, we could not gather users' behavioral data. 
This includes information such as how they edited the prompts, the amount of time they dedicated to responding to emails, or how ResQ influenced the language they used in their actual email communications.
We encourage researchers to explore alternative research methods for capturing users' behavioral data in email exchanges in the wild to enrich the understanding of QA-based approaches in AI-mediated communication.

% また、メールの特徴ごとのQA-based approachの有効性については、さらなる調査が可能だろう。
% Study 1では、フォーマルなシチュエーションにおける様々なトピックのメールを使用して実験を行い、QA-based approachの及ぼす影響について調査した。
% しかし、その特徴(例えば、状況のformalさ、メールの丁寧さ、重要性、受信者と送信者の関係性など)ごとに、QA-based approachの有効性は異なる可能性がある。
\red{Second, the effectiveness of the QA-based approach may vary depending on the specific characteristics of the emails. 
We conducted Study 1 using emails on a variety of topics within formal scenarios to examine the impact of the QA-based approach. 
However, its effectiveness may differ based on characteristics such as the formality of the situation, the politeness of the email, its importance, or the relationship between the sender and recipient.
Therefore, future research could explore how these specific email characteristics influence the effectiveness of QA-based approaches, potentially tailoring AI-mediated tools to different communication contexts.}
% First, the quantitative results were confined to a controlled environment. 
% This limitation arose because, in field studies, accessing participants' email content was not feasible due to privacy concerns, making it difficult to perform fair comparisons using quantitative evaluation metrics. 
% For example, the time required to compose an email and the necessity of a reply vary depending on the content and context. 
% Therefore, we designed the field study to assess how ResQ influenced the practice of email replies qualitatively.

\blue{Third, the study was conducted with participants from a single cultural background, which could limit the generalizability of our findings. 
Although we contributed to a new understanding for populations from non-Western countries~\cite{WEIRD_CHI21}, we acknowledge that the practice of email exchange differs across cultures~\cite{Robertson2021ICant}.
% The findings could not be generalized to other cultural contexts since the nature and role of emails differ across cultures~\cite{Robertson2021ICant}. 
% Although some participants used ResQ in communications with individuals from different cultural backgrounds and observed a degree of effectiveness, 
Further studies are encouraged to examine whether similar results would be obtained among users from diverse cultural backgrounds or in cross-cultural email exchanges.}

% Also, the system has the potential to be applied to devices other than PCs and adapted for communication tools beyond email. 
% For example, integrating ResQ into business chat platforms like Slack or Microsoft Teams seems highly compatible, as these environments require efficient, goal-oriented communication. 
% Moreover, this QA-based approach could be extended to other domains where structured document creation is necessary. 
% One possible application is drafting rebuttal letters, where a structured format and the ability to address specific points are necessary.

% \red{Fourth, while Study 1 evaluates the impact of the LLM-powered QA-based approach under three conditions, it does not include a condition that replicates the QA-based approach without relying on LLMs. 
% This omission limits the ability to isolate the specific contribution of LLMs to the system's overall performance. 
% Incorporating a QA-based condition in future research (\textit{e.g.}, employing a rule-based approach or using manually created questions and options) could offer a more comprehensive understanding of the unique value LLMs provide compared to rule-based or manual methods.}

\red{Fourth, while this study focused on a QA-based approach driven by LLMs, future research could explore alternative methods of question generation to deepen our understanding of QA-based AI assistance. 
For instance, comparing the LLM-powered system with approaches utilizing rule-based question generation or manually prepared questions and options may help disentangle the effects of algorithmic sophistication from the inherent benefits of structuring communication as QA. 
This may potentially clarify whether the AI placebo or nocebo effect~\cite{kloft2024aiplacebo} exists in AI-mediated communication.
Examining these different methods could offer further insights into when and why the QA-based approach excels and guide the design of more tailored systems that accommodate a wide range of communication tasks and user needs.}

\red{Fifth, while this study demonstrated the effectiveness of the QA-based approach with initial design considerations (Sec.~\ref{sec:Proposed_Approach}), future research could explore tailoring these questions to specific communication goals or contexts. 
For example, designers or instructors could adjust factors such as the number of questions, their difficulty level, or their thematic focus to improve the user's understanding of challenging content. 
By iterating on the design to explore how different dimensions of question can affect communication outcomes, future work can better guide the QA-based approach.}

% \red{Fifth, future research could design the questions in our approach to be tailored for each user, such as question difficulty, quantity, and thematic focus. 
% % Tailored question sets may help users understand challenging content, expedite task completion, or improve communication efficiency in specialized domains. 
% By iterating on the design, researchers can develop more refined QA-based systems that better meet user needs.}