\documentclass[sigconf]{acmart}

%%
%% \BibTeX command to typeset BibTeX logo in the docs
\AtBeginDocument{%
  \providecommand\BibTeX{{%
    Bib\TeX}}}

\setcopyright{acmlicensed}
\copyrightyear{2025}
\acmYear{2025}
\setcopyright{cc}
\setcctype{by}
\acmConference[CHI '25]{CHI Conference on Human Factors in Computing Systems}{April 26-May 1, 2025}{Yokohama, Japan}
\acmBooktitle{CHI Conference on Human Factors in Computing Systems (CHI '25), April 26-May 1, 2025, Yokohama, Japan}\acmDOI{10.1145/3706598.3714016}
\acmISBN{979-8-4007-1394-1/25/04}

\usepackage{enumerate}
\usepackage{multirow}
\usepackage{array} % required for text wrapping in tables
\usepackage{listings}
\usepackage{graphicx}

% \newcommand{\red}{\textcolor[rgb]{0.757,0.153,0.212}}
\newcommand{\red}{\textcolor[rgb]{0,0,0}}
% \newcommand{\blue}{\textcolor[rgb]{0.082, 0, 1}}
\newcommand{\blue}{\textcolor[rgb]{0,0,0}}
\newcommand{\green}{\textcolor[rgb]{0.180, 0.518, 0.349}}

\begin{document}
\title{\red{Understanding and Supporting Formal Email Exchange\\by Answering AI-Generated Questions}}

\author{Yusuke Miura}
\email{miura.yusuke@toki.waseda.jp}
\orcid{0000-0003-1204-6623}
\affiliation{%
  \institution{Waseda University}
  \city{Tokyo}
  % \state{Tokyo}
  \country{Japan}
}

\author{Chi-Lan Yang}
\email{chilan.yang@cyber.t.u-tokyo.ac.jp}
\orcid{0000-0003-0603-2807}
\affiliation{%
  \institution{The University of Tokyo}
  \city{Tokyo}
  % \state{Tokyo}
  \country{Japan}
}

\author{Masaki Kuribayashi}
\email{rugbykuribayashi@waseda.jp}
\orcid{0000-0001-8412-223X}
\affiliation{%
  \institution{Waseda University}
  \city{Tokyo}
  % \state{Tokyo}
  \country{Japan}
}

\author{Keigo Matsumoto}
\email{matsumoto@cyber.t.u-tokyo.ac.jp}
\orcid{0000-0002-0038-0678}
\affiliation{%
  \institution{The University of Tokyo}
  \city{Tokyo}
  \country{Japan}
}

\author{Hideaki Kuzuoka}
\email{kuzuoka@cyber.t.u-tokyo.ac.jp}
\orcid{0000-0003-1252-7814}
\affiliation{%
  \institution{The University of Tokyo}
  \city{Tokyo}
  % \state{Tokyo}
  \country{Japan}
}

\author{Shigeo Morishima}
\email{shigeo@waseda.jp}
\orcid{0000-0001-8859-6539}
\affiliation{%
  \institution{Waseda Research Institute for Science and Engineering}
  \city{Tokyo}
  % \state{Tokyo}
  \country{Japan}
}

\renewcommand{\shortauthors}{Miura et al.}

\begin{abstract}
% \begin{abstract}\label{00_Abstract}
Research in the field of automated vehicles, or more generally cognitive cyber-physical systems that operate in the real world, is leading to increasingly complex systems. Among other things, artificial intelligence enables an ever-increasing degree of autonomy. In this context, the V-model, which has served for decades as a process reference model of the system development lifecycle is reaching its limits. To the contrary, innovative processes and frameworks have been developed that take into account the characteristics of emerging autonomous systems. To bridge the gap and merge the different methodologies, we present an extension of the V-model for iterative data-based development processes that harmonizes and formalizes the existing methods towards a generic framework. The iterative approach allows for seamless integration of continuous system refinement. While the data-based approach constitutes the consideration of data-based development processes and formalizes the use of synthetic and real world data. In this way, formalizing the process of development, verification, validation, and continuous integration contributes to ensuring the safety of emerging complex systems that incorporate AI. 
\end{abstract}


\begin{IEEEkeywords}
	Process Reference Model, V-Model, Continuous Integration, AI Systems, Autonomy Technology, Safety Assurance
\end{IEEEkeywords}

\red{Replying to formal emails is time-consuming and cognitively demanding, as it requires crafting polite phrasing and providing an adequate response to the sender's demands.}
% \red{Replying to formal emails, which often takes time to understand and require polite phrasing, is time-consuming and cognitively demanding.}
Although systems with Large Language Models (LLMs) were designed to simplify the email replying process, users still need to provide detailed prompts to obtain the expected output.
Therefore, we proposed and evaluated an \red{LLM-powered question-and-answer (QA)-based approach} for users to reply to emails by answering a set of simple and short questions generated from the incoming email.
We developed a prototype system, \textit{ResQ}, and conducted controlled and field experiments with 12 and \red{8} participants.
Our results demonstrated that \red{the QA-based approach} improves the efficiency of replying to emails and reduces workload while maintaining email quality, compared to a conventional prompt-based approach that requires users to craft appropriate prompts to obtain email drafts.
We discuss how \red{the QA-based approach} influences the email reply process and interpersonal relationship dynamics, as well as the opportunities and challenges associated with using a QA-based approach in AI-mediated communication.
\end{abstract}

\begin{CCSXML}
<ccs2012>
   <concept>
       <concept_id>10003120.10003130.10011762</concept_id>
       <concept_desc>Human-centered computing~Empirical studies in collaborative and social computing</concept_desc>
       <concept_significance>500</concept_significance>
       </concept>
 </ccs2012>
\end{CCSXML}

\ccsdesc[500]{Human-centered computing~Empirical studies in collaborative and social computing}

\keywords{AI-Mediated Communication, Large Language Models, Email}

\begin{teaserfigure}
  \includegraphics[width=\textwidth]{teaser.pdf}
  \caption{In our system, (1) users receive an email, (2) communicate their intentions by answering AI-generated questions, (3) receive an AI-generated draft, (4) make any necessary revisions, and finally (5) send the reply.  This process allows users to craft responses efficiently, reducing their overall workload.}
  \Description{This figure illustrates the five stages of using our proposed system. First, the user receives an email on their computer, marking the beginning of the workflow. In the second stage, the user answers a set of predefined questions related to the email, which can include options like "YES" or "NO" as well as multiple-choice selections (A, B, C). Based on the user's responses, the AI then generates a draft of the email reply in the third stage. In the fourth stage, the user reviews the AI-generated draft, checking for errors or making revisions as needed. Finally, in the fifth stage, the user sends the revised email reply, completing the process.}
  \label{fig_teaser}
\end{teaserfigure}

\maketitle
\section{Introduction}

% \textcolor{red}{Still on working}

% \textcolor{red}{add label for each section}


Robot learning relies on diverse and high-quality data to learn complex behaviors \cite{aldaco2024aloha, wang2024dexcap}.
Recent studies highlight that models trained on datasets with greater complexity and variation in the domain tend to generalize more effectively across broader scenarios \cite{mann2020language, radford2021learning, gao2024efficient}.
% However, creating such diverse datasets in the real world presents significant challenges.
% Modifying physical environments and adjusting robot hardware settings require considerable time, effort, and financial resources.
% In contrast, simulation environments offer a flexible and efficient alternative.
% Simulations allow for the creation and modification of digital environments with a wide range of object shapes, weights, materials, lighting, textures, friction coefficients, and so on to incorporate domain randomization,
% which helps improve the robustness of models when deployed in real-world conditions.
% These environments can be easily adjusted and reset, enabling faster iterations and data collection.
% Additionally, simulations provide the ability to consistently reproduce scenarios, which is essential for benchmarking and model evaluation.
% Another advantage of simulations is their flexibility in sensor integration. Sensors such as cameras, LiDARs, and tactile sensors can be added or repositioned without the physical limitations present in real-world setups. Simulations also eliminate the risk of damaging expensive hardware during edge-case experiments, making them an ideal platform for testing rare or dangerous scenarios that are impractical to explore in real life.
By leveraging immersive perspectives and interactions, Extended Reality\footnote{Extended Reality is an umbrella term to refer to Augmented Reality, Mixed Reality, and Virtual Reality \cite{wikipediaExtendedReality}}
(XR)
is a promising candidate for efficient and intuitive large scale data collection \cite{jiang2024comprehensive, arcade}
% With the demand for collecting data, XR provides a promising approach for humans to teach robots by offering users an immersive experience.
in simulation \cite{jiang2024comprehensive, arcade, dexhub-park} and real-world scenarios \cite{openteach, opentelevision}.
However, reusing and reproducing current XR approaches for robot data collection for new settings and scenarios is complicated and requires significant effort.
% are difficult to reuse and reproduce system makes it hard to reuse and reproduce in another data collection pipeline.
This bottleneck arises from three main limitations of current XR data collection and interaction frameworks: \textit{asset limitation}, \textit{simulator limitation}, and \textit{device limitation}.
% \textcolor{red}{ASSIGN THESE CITATION PROPERLY:}
% \textcolor{red}{list them by time order???}
% of collecting data by using XR have three main limitations.
Current approaches suffering from \textit{asset limitation} \cite{arclfd, jiang2024comprehensive, arcade, george2025openvr, vicarios}
% Firstly, recent works \cite{jiang2024comprehensive, arcade, dexhub-park}
can only use predefined robot models and task scenes. Configuring new tasks requires significant effort, since each new object or model must be specifically integrated into the XR application.
% and it takes too much effort to configure new tasks in their systems since they cannot spawn arbitrary models in the XR application.
The vast majority of application are developed for specific simulators or real-world scenarios. This \textit{simulator limitation} \cite{mosbach2022accelerating, lipton2017baxter, dexhub-park, arcade}
% Secondly, existing systems are limited to a single simulation platform or real-world scenarios.
significantly reduces reusability and makes adaptation to new simulation platforms challenging.
Additionally, most current XR frameworks are designed for a specific version of a single XR headset, leading to a \textit{device limitation} 
\cite{lipton2017baxter, armada, openteach, meng2023virtual}.
% and there is no work working on the extendability of transferring to a new headsets as far as we know.
To the best of our knowledge, no existing work has explored the extensibility or transferability of their framework to different headsets.
These limitations hamper reproducibility and broader contributions of XR based data collection and interaction to the research community.
% as each research group typically has its own data collection pipeline.
% In addition to these main limitations, existing XR systems are not well suited for managing multiple robot systems,
% as they are often designed for single-operator use.

In addition to these main limitations, existing XR systems are often designed for single-operator use, prohibiting collaborative data collection.
At the same time, controlling multiple robots at once can be very difficult for a single operator,
making data collection in multi-robot scenarios particularly challenging \cite{orun2019effect}.
Although there are some works using collaborative data collection in the context of tele-operation \cite{tung2021learning, Qin2023AnyTeleopAG},
there is no XR-based data collection system supporting collaborative data collection.
This limitation highlights the need for more advanced XR solutions that can better support multi-robot and multi-user scenarios.
% \textcolor{red}{more papers about collaborative data collection}

To address all of these issues, we propose \textbf{IRIS},
an \textbf{I}mmersive \textbf{R}obot \textbf{I}nteraction \textbf{S}ystem.
This general system supports various simulators, benchmarks and real-world scenarios.
It is easily extensible to new simulators and XR headsets.
IRIS achieves generalization across six dimensions:
% \begin{itemize}
%     \item \textit{Cross-scene} : diverse object models;
%     \item \textit{Cross-embodiment}: diverse robot models;
%     \item \textit{Cross-simulator}: 
%     \item \textit{Cross-reality}: fd
%     \item \textit{Cross-platform}: fd
%     \item \textit{Cross-users}: fd
% \end{itemize}
\textbf{Cross-Scene}, \textbf{Cross-Embodiment}, \textbf{Cross-Simulator}, \textbf{Cross-Reality}, \textbf{Cross-Platform}, and \textbf{Cross-User}.

\textbf{Cross-Scene} and \textbf{Cross-Embodiment} allow the system to handle arbitrary objects and robots in the simulation,
eliminating restrictions about predefined models in XR applications.
IRIS achieves these generalizations by introducing a unified scene specification, representing all objects,
including robots, as data structures with meshes, materials, and textures.
The unified scene specification is transmitted to the XR application to create and visualize an identical scene.
By treating robots as standard objects, the system simplifies XR integration,
allowing researchers to work with various robots without special robot-specific configurations.
\textbf{Cross-Simulator} ensures compatibility with various simulation engines.
IRIS simplifies adaptation by parsing simulated scenes into the unified scene specification, eliminating the need for XR application modifications when switching simulators.
New simulators can be integrated by creating a parser to convert their scenes into the unified format.
This flexibility is demonstrated by IRIS’ support for Mujoco \cite{todorov2012mujoco}, IsaacSim \cite{mittal2023orbit}, CoppeliaSim \cite{coppeliaSim}, and even the recent Genesis \cite{Genesis} simulator.
\textbf{Cross-Reality} enables the system to function seamlessly in both virtual simulations and real-world applications.
IRIS enables real-world data collection through camera-based point cloud visualization.
\textbf{Cross-Platform} allows for compatibility across various XR devices.
Since XR device APIs differ significantly, making a single codebase impractical, IRIS XR application decouples its modules to maximize code reuse.
This application, developed by Unity \cite{unity3dUnityManual}, separates scene visualization and interaction, allowing developers to integrate new headsets by reusing the visualization code and only implementing input handling for hand, head, and motion controller tracking.
IRIS provides an implementation of the XR application in the Unity framework, allowing for a straightforward deployment to any device that supports Unity. 
So far, IRIS was successfully deployed to the Meta Quest 3 and HoloLens 2.
Finally, the \textbf{Cross-User} ability allows multiple users to interact within a shared scene.
IRIS achieves this ability by introducing a protocol to establish the communication between multiple XR headsets and the simulation or real-world scenarios.
Additionally, IRIS leverages spatial anchors to support the alignment of virtual scenes from all deployed XR headsets.
% To make an seamless user experience for robot learning data collection,
% IRIS also tested in three different robot control interface
% Furthermore, to demonstrate the extensibility of our approach, we have implemented a robot-world pipeline for real robot data collection, ensuring that the system can be used in both simulated and real-world environments.
The Immersive Robot Interaction System makes the following contributions\\
\textbf{(1) A unified scene specification} that is compatible with multiple robot simulators. It enables various XR headsets to visualize and interact with simulated objects and robots, providing an immersive experience while ensuring straightforward reusability and reproducibility.\\
\textbf{(2) A collaborative data collection framework} designed for XR environments. The framework facilitates enhanced robot data acquisition.\\
\textbf{(3) A user study} demonstrating that IRIS significantly improves data collection efficiency and intuitiveness compared to the LIBERO baseline.

% \begin{table*}[t]
%     \centering
%     \begin{tabular}{lccccccc}
%         \toprule
%         & \makecell{Physical\\Interaction}
%         & \makecell{XR\\Enabled}
%         & \makecell{Free\\View}
%         & \makecell{Multiple\\Robots}
%         & \makecell{Robot\\Control}
%         % Force Feedback???
%         & \makecell{Soft Object\\Supported}
%         & \makecell{Collaborative\\Data} \\
%         \midrule
%         ARC-LfD \cite{arclfd}                              & Real        & \cmark & \xmark & \xmark & Joint              & \xmark & \xmark \\
%         DART \cite{dexhub-park}                            & Sim         & \cmark & \cmark & \cmark & Cartesian          & \xmark & \xmark \\
%         \citet{jiang2024comprehensive}                     & Sim         & \cmark & \xmark & \xmark & Joint \& Cartesian & \xmark & \xmark \\
%         \citet{mosbach2022accelerating}                    & Sim         & \cmark & \cmark & \xmark & Cartesian          & \xmark & \xmark \\
%         ARCADE \cite{arcade}                               & Real        & \cmark & \cmark & \xmark & Cartesian          & \xmark & \xmark \\
%         Holo-Dex \cite{holodex}                            & Real        & \cmark & \xmark & \cmark & Cartesian          & \cmark & \xmark \\
%         ARMADA \cite{armada}                               & Real        & \cmark & \xmark & \cmark & Cartesian          & \cmark & \xmark \\
%         Open-TeleVision \cite{opentelevision}              & Real        & \cmark & \cmark & \cmark & Cartesian          & \cmark & \xmark \\
%         OPEN TEACH \cite{openteach}                        & Real        & \cmark & \xmark & \cmark & Cartesian          & \cmark & \cmark \\
%         GELLO \cite{wu2023gello}                           & Real        & \xmark & \cmark & \cmark & Joint              & \cmark & \xmark \\
%         DexCap \cite{wang2024dexcap}                       & Real        & \xmark & \cmark & \xmark & Cartesian          & \cmark & \xmark \\
%         AnyTeleop \cite{Qin2023AnyTeleopAG}                & Real        & \xmark & \xmark & \cmark & Cartesian          & \cmark & \cmark \\
%         Vicarios \cite{vicarios}                           & Real        & \cmark & \xmark & \xmark & Cartesian          & \cmark & \xmark \\     
%         Augmented Visual Cues \cite{augmentedvisualcues}   & Real        & \cmark & \cmark & \xmark & Cartesian          & \xmark & \xmark \\ 
%         \citet{wang2024robotic}                            & Real        & \cmark & \cmark & \xmark & Cartesian          & \cmark & \xmark \\
%         Bunny-VisionPro \cite{bunnyvisionpro}              & Real        & \cmark & \cmark & \cmark & Cartesian          & \cmark & \xmark \\
%         IMMERTWIN \cite{immertwin}                         & Real        & \cmark & \cmark & \cmark & Cartesian          & \xmark & \xmark \\
%         \citet{meng2023virtual}                            & Sim \& Real & \cmark & \cmark & \xmark & Cartesian          & \xmark & \xmark \\
%         Shared Control Framework \cite{sharedctlframework} & Real        & \cmark & \cmark & \cmark & Cartesian          & \xmark & \xmark \\
%         OpenVR \cite{openvr}                               & Real        & \cmark & \cmark & \xmark & Cartesian          & \xmark & \xmark \\
%         \citet{digitaltwinmr}                              & Real        & \cmark & \cmark & \xmark & Cartesian          & \cmark & \xmark \\
        
%         \midrule
%         \textbf{Ours} & Sim \& Real & \cmark & \cmark & \cmark & Joint \& Cartesian  & \cmark & \cmark \\
%         \bottomrule
%     \end{tabular}
%     \caption{This is a cross-column table with automatic line breaking.}
%     \label{tab:cross-column}
% \end{table*}

% \begin{table*}[t]
%     \centering
%     \begin{tabular}{lccccccc}
%         \toprule
%         & \makecell{Cross-Embodiment}
%         & \makecell{Cross-Scene}
%         & \makecell{Cross-Simulator}
%         & \makecell{Cross-Reality}
%         & \makecell{Cross-Platform}
%         & \makecell{Cross-User} \\
%         \midrule
%         ARC-LfD \cite{arclfd}                              & \xmark & \xmark & \xmark & \xmark & \xmark & \xmark \\
%         DART \cite{dexhub-park}                            & \cmark & \cmark & \xmark & \xmark & \xmark & \xmark \\
%         \citet{jiang2024comprehensive}                     & \xmark & \cmark & \xmark & \xmark & \xmark & \xmark \\
%         \citet{mosbach2022accelerating}                    & \xmark & \cmark & \xmark & \xmark & \xmark & \xmark \\
%         ARCADE \cite{arcade}                               & \xmark & \xmark & \xmark & \xmark & \xmark & \xmark \\
%         Holo-Dex \cite{holodex}                            & \cmark & \xmark & \xmark & \xmark & \xmark & \xmark \\
%         ARMADA \cite{armada}                               & \cmark & \xmark & \xmark & \xmark & \xmark & \xmark \\
%         Open-TeleVision \cite{opentelevision}              & \cmark & \xmark & \xmark & \xmark & \cmark & \xmark \\
%         OPEN TEACH \cite{openteach}                        & \cmark & \xmark & \xmark & \xmark & \xmark & \cmark \\
%         GELLO \cite{wu2023gello}                           & \cmark & \xmark & \xmark & \xmark & \xmark & \xmark \\
%         DexCap \cite{wang2024dexcap}                       & \xmark & \xmark & \xmark & \xmark & \xmark & \xmark \\
%         AnyTeleop \cite{Qin2023AnyTeleopAG}                & \cmark & \cmark & \cmark & \cmark & \xmark & \cmark \\
%         Vicarios \cite{vicarios}                           & \xmark & \xmark & \xmark & \xmark & \xmark & \xmark \\     
%         Augmented Visual Cues \cite{augmentedvisualcues}   & \xmark & \xmark & \xmark & \xmark & \xmark & \xmark \\ 
%         \citet{wang2024robotic}                            & \xmark & \xmark & \xmark & \xmark & \xmark & \xmark \\
%         Bunny-VisionPro \cite{bunnyvisionpro}              & \cmark & \xmark & \xmark & \xmark & \xmark & \xmark \\
%         IMMERTWIN \cite{immertwin}                         & \cmark & \xmark & \xmark & \xmark & \xmark & \xmark \\
%         \citet{meng2023virtual}                            & \xmark & \cmark & \xmark & \cmark & \xmark & \xmark \\
%         \citet{sharedctlframework}                         & \cmark & \xmark & \xmark & \xmark & \xmark & \xmark \\
%         OpenVR \cite{george2025openvr}                               & \xmark & \xmark & \xmark & \xmark & \xmark & \xmark \\
%         \citet{digitaltwinmr}                              & \xmark & \xmark & \xmark & \xmark & \xmark & \xmark \\
        
%         \midrule
%         \textbf{Ours} & \cmark & \cmark & \cmark & \cmark & \cmark & \cmark \\
%         \bottomrule
%     \end{tabular}
%     \caption{This is a cross-column table with automatic line breaking.}
% \end{table*}

% \begin{table*}[t]
%     \centering
%     \begin{tabular}{lccccccc}
%         \toprule
%         & \makecell{Cross-Scene}
%         & \makecell{Cross-Embodiment}
%         & \makecell{Cross-Simulator}
%         & \makecell{Cross-Reality}
%         & \makecell{Cross-Platform}
%         & \makecell{Cross-User}
%         & \makecell{Control Space} \\
%         \midrule
%         % Vicarios \cite{vicarios}                           & \xmark & \xmark & \xmark & \xmark & \xmark & \xmark \\     
%         % Augmented Visual Cues \cite{augmentedvisualcues}   & \xmark & \xmark & \xmark & \xmark & \xmark & \xmark \\ 
%         % OpenVR \cite{george2025openvr}                     & \xmark & \xmark & \xmark & \xmark & \xmark & \xmark \\
%         \citet{digitaltwinmr}                              & \xmark & \xmark & \xmark & \xmark & \xmark & \xmark &  \\
%         ARC-LfD \cite{arclfd}                              & \xmark & \xmark & \xmark & \xmark & \xmark & \xmark &  \\
%         \citet{sharedctlframework}                         & \cmark & \xmark & \xmark & \xmark & \xmark & \xmark &  \\
%         \citet{jiang2024comprehensive}                     & \cmark & \xmark & \xmark & \xmark & \xmark & \xmark &  \\
%         \citet{mosbach2022accelerating}                    & \cmark & \xmark & \xmark & \xmark & \xmark & \xmark & \\
%         Holo-Dex \cite{holodex}                            & \cmark & \xmark & \xmark & \xmark & \xmark & \xmark & \\
%         ARCADE \cite{arcade}                               & \cmark & \cmark & \xmark & \xmark & \xmark & \xmark & \\
%         DART \cite{dexhub-park}                            & Limited & Limited & Mujoco & Sim & Vision Pro & \xmark &  Cartesian\\
%         ARMADA \cite{armada}                               & \cmark & \cmark & \xmark & \xmark & \xmark & \xmark & \\
%         \citet{meng2023virtual}                            & \cmark & \cmark & \xmark & \cmark & \xmark & \xmark & \\
%         % GELLO \cite{wu2023gello}                           & \cmark & \xmark & \xmark & \xmark & \xmark & \xmark \\
%         % DexCap \cite{wang2024dexcap}                       & \xmark & \xmark & \xmark & \xmark & \xmark & \xmark \\
%         % AnyTeleop \cite{Qin2023AnyTeleopAG}                & \cmark & \cmark & \cmark & \cmark & \xmark & \cmark \\
%         % \citet{wang2024robotic}                            & \xmark & \xmark & \xmark & \xmark & \xmark & \xmark \\
%         Bunny-VisionPro \cite{bunnyvisionpro}              & \cmark & \cmark & \xmark & \xmark & \xmark & \xmark & \\
%         IMMERTWIN \cite{immertwin}                         & \cmark & \cmark & \xmark & \xmark & \xmark & \xmark & \\
%         Open-TeleVision \cite{opentelevision}              & \cmark & \cmark & \xmark & \xmark & \cmark & \xmark & \\
%         \citet{szczurek2023multimodal}                     & \xmark & \xmark & \xmark & Real & \xmark & \cmark & \\
%         OPEN TEACH \cite{openteach}                        & \cmark & \cmark & \xmark & \xmark & \xmark & \cmark & \\
%         \midrule
%         \textbf{Ours} & \cmark & \cmark & \cmark & \cmark & \cmark & \cmark \\
%         \bottomrule
%     \end{tabular}
%     \caption{TODO, Bruce: this table can be further optimized.}
% \end{table*}

\definecolor{goodgreen}{HTML}{228833}
\definecolor{goodred}{HTML}{EE6677}
\definecolor{goodgray}{HTML}{BBBBBB}

\begin{table*}[t]
    \centering
    \begin{adjustbox}{max width=\textwidth}
    \renewcommand{\arraystretch}{1.2}    
    \begin{tabular}{lccccccc}
        \toprule
        & \makecell{Cross-Scene}
        & \makecell{Cross-Embodiment}
        & \makecell{Cross-Simulator}
        & \makecell{Cross-Reality}
        & \makecell{Cross-Platform}
        & \makecell{Cross-User}
        & \makecell{Control Space} \\
        \midrule
        % Vicarios \cite{vicarios}                           & \xmark & \xmark & \xmark & \xmark & \xmark & \xmark \\     
        % Augmented Visual Cues \cite{augmentedvisualcues}   & \xmark & \xmark & \xmark & \xmark & \xmark & \xmark \\ 
        % OpenVR \cite{george2025openvr}                     & \xmark & \xmark & \xmark & \xmark & \xmark & \xmark \\
        \citet{digitaltwinmr}                              & \textcolor{goodred}{Limited}     & \textcolor{goodred}{Single Robot} & \textcolor{goodred}{Unity}    & \textcolor{goodred}{Real}          & \textcolor{goodred}{Meta Quest 2} & \textcolor{goodgray}{N/A} & \textcolor{goodred}{Cartesian} \\
        ARC-LfD \cite{arclfd}                              & \textcolor{goodgray}{N/A}        & \textcolor{goodred}{Single Robot} & \textcolor{goodgray}{N/A}     & \textcolor{goodred}{Real}          & \textcolor{goodred}{HoloLens}     & \textcolor{goodgray}{N/A} & \textcolor{goodred}{Cartesian} \\
        \citet{sharedctlframework}                         & \textcolor{goodred}{Limited}     & \textcolor{goodred}{Single Robot} & \textcolor{goodgray}{N/A}     & \textcolor{goodred}{Real}          & \textcolor{goodred}{HTC Vive Pro} & \textcolor{goodgray}{N/A} & \textcolor{goodred}{Cartesian} \\
        \citet{jiang2024comprehensive}                     & \textcolor{goodred}{Limited}     & \textcolor{goodred}{Single Robot} & \textcolor{goodgray}{N/A}     & \textcolor{goodred}{Real}          & \textcolor{goodred}{HoloLens 2}   & \textcolor{goodgray}{N/A} & \textcolor{goodgreen}{Joint \& Cartesian} \\
        \citet{mosbach2022accelerating}                    & \textcolor{goodgreen}{Available} & \textcolor{goodred}{Single Robot} & \textcolor{goodred}{IsaacGym} & \textcolor{goodred}{Sim}           & \textcolor{goodred}{Vive}         & \textcolor{goodgray}{N/A} & \textcolor{goodgreen}{Joint \& Cartesian} \\
        Holo-Dex \cite{holodex}                            & \textcolor{goodgray}{N/A}        & \textcolor{goodred}{Single Robot} & \textcolor{goodgray}{N/A}     & \textcolor{goodred}{Real}          & \textcolor{goodred}{Meta Quest 2} & \textcolor{goodgray}{N/A} & \textcolor{goodred}{Joint} \\
        ARCADE \cite{arcade}                               & \textcolor{goodgray}{N/A}        & \textcolor{goodred}{Single Robot} & \textcolor{goodgray}{N/A}     & \textcolor{goodred}{Real}          & \textcolor{goodred}{HoloLens 2}   & \textcolor{goodgray}{N/A} & \textcolor{goodred}{Cartesian} \\
        DART \cite{dexhub-park}                            & \textcolor{goodred}{Limited}     & \textcolor{goodred}{Limited}      & \textcolor{goodred}{Mujoco}   & \textcolor{goodred}{Sim}           & \textcolor{goodred}{Vision Pro}   & \textcolor{goodgray}{N/A} & \textcolor{goodred}{Cartesian} \\
        ARMADA \cite{armada}                               & \textcolor{goodgray}{N/A}        & \textcolor{goodred}{Limited}      & \textcolor{goodgray}{N/A}     & \textcolor{goodred}{Real}          & \textcolor{goodred}{Vision Pro}   & \textcolor{goodgray}{N/A} & \textcolor{goodred}{Cartesian} \\
        \citet{meng2023virtual}                            & \textcolor{goodred}{Limited}     & \textcolor{goodred}{Single Robot} & \textcolor{goodred}{PhysX}   & \textcolor{goodgreen}{Sim \& Real} & \textcolor{goodred}{HoloLens 2}   & \textcolor{goodgray}{N/A} & \textcolor{goodred}{Cartesian} \\
        % GELLO \cite{wu2023gello}                           & \cmark & \xmark & \xmark & \xmark & \xmark & \xmark \\
        % DexCap \cite{wang2024dexcap}                       & \xmark & \xmark & \xmark & \xmark & \xmark & \xmark \\
        % AnyTeleop \cite{Qin2023AnyTeleopAG}                & \cmark & \cmark & \cmark & \cmark & \xmark & \cmark \\
        % \citet{wang2024robotic}                            & \xmark & \xmark & \xmark & \xmark & \xmark & \xmark \\
        Bunny-VisionPro \cite{bunnyvisionpro}              & \textcolor{goodgray}{N/A}        & \textcolor{goodred}{Single Robot} & \textcolor{goodgray}{N/A}     & \textcolor{goodred}{Real}          & \textcolor{goodred}{Vision Pro}   & \textcolor{goodgray}{N/A} & \textcolor{goodred}{Cartesian} \\
        IMMERTWIN \cite{immertwin}                         & \textcolor{goodgray}{N/A}        & \textcolor{goodred}{Limited}      & \textcolor{goodgray}{N/A}     & \textcolor{goodred}{Real}          & \textcolor{goodred}{HTC Vive}     & \textcolor{goodgray}{N/A} & \textcolor{goodred}{Cartesian} \\
        Open-TeleVision \cite{opentelevision}              & \textcolor{goodgray}{N/A}        & \textcolor{goodred}{Limited}      & \textcolor{goodgray}{N/A}     & \textcolor{goodred}{Real}          & \textcolor{goodgreen}{Meta Quest, Vision Pro} & \textcolor{goodgray}{N/A} & \textcolor{goodred}{Cartesian} \\
        \citet{szczurek2023multimodal}                     & \textcolor{goodgray}{N/A}        & \textcolor{goodred}{Limited}      & \textcolor{goodgray}{N/A}     & \textcolor{goodred}{Real}          & \textcolor{goodred}{HoloLens 2}   & \textcolor{goodgreen}{Available} & \textcolor{goodred}{Joint \& Cartesian} \\
        OPEN TEACH \cite{openteach}                        & \textcolor{goodgray}{N/A}        & \textcolor{goodgreen}{Available}  & \textcolor{goodgray}{N/A}     & \textcolor{goodred}{Real}          & \textcolor{goodred}{Meta Quest 3} & \textcolor{goodred}{N/A} & \textcolor{goodgreen}{Joint \& Cartesian} \\
        \midrule
        \textbf{Ours}                                      & \textcolor{goodgreen}{Available} & \textcolor{goodgreen}{Available}  & \textcolor{goodgreen}{Mujoco, CoppeliaSim, IsaacSim} & \textcolor{goodgreen}{Sim \& Real} & \textcolor{goodgreen}{Meta Quest 3, HoloLens 2} & \textcolor{goodgreen}{Available} & \textcolor{goodgreen}{Joint \& Cartesian} \\
        \bottomrule
        \end{tabular}
    \end{adjustbox}
    \caption{Comparison of XR-based system for robots. IRIS is compared with related works in different dimensions.}
\end{table*}


\section{Related Work}

In this section, we review research related to the importance and barriers to parental involvement; parental use of learning technologies; and the use of generative AI and robot in educational and parenting scenarios.

\subsection{Importance and Barriers to Parental Involvement}\label{sec-rw-2.1}

% 79 words
Early childhood is a critical period for predicting future success and well-being, with early education investments resulting in higher returns than later interventions \cite{duncan2007school, doyle2009investing}. Effective parental involvement fosters cognitive and social skills, especially in younger children \cite{blevins2016early, peck1992parent}. Parents are encouraged to prioritize home-based involvement to maximize their influence \cite{ma2016meta}, as their involvement has a greater impact on children's learning outcomes \cite{hoffner2002parents, fehrmann1987home, hill2004parent} within the family setting than partnerships with schools or communities \cite{ma2016meta, harris2008parents, fantuzzo2004multiple, sui1996effects}.

However, parents' involvement in their children's education is often constrained by practical challenges related to parents' \textit{skills}, \textit{time}, and \textit{energy}. The Hoover-Dempsey and Sandler (HDS) framework \cite{green2007parents} and the CAM framework \cite{ho2024s} both highlight these factors-- parents' perceived \textit{skills and knowledge} (capability), \textit{time} (availability), and \textit{motivation} (energy)--influence the extent of their engagement. For instance, a parent confident in math may choose to engage more in math-related tasks, while those facing inflexible schedules may participate less \cite{green2007parents}. Unlike teachers, parents often lack formal pedagogical training and may underestimate their role in supporting children's learning, particularly as young children struggle to articulate their needs \cite{hara1998parent}. The CAM framework similarly suggests parents delegate tasks to a robot when they feel less capable, have limited time, or are unmotivated. These factors reflect parents' life contexts, shaped by demographic backgrounds, occupations, and parenting responsibilities \cite{grolnick1997predictors}, highlighting the need to help parents overcome barriers to effective involvement in early education within their life contexts.

\subsection{Parental Use of Learning Technologies}\label{sec-rw-2.2}
% 207 words
Technology encourages parental involvement by facilitating parent-child engagement in learning activities while introducing risks that require active parental mediation \cite{gonzalez2022parental}. On the positive side, technology offers novel opportunities for parental engagement and enhances children's learning outcomes. For example, e-books promote interactive behaviors between parents and children better than print books \cite{korat2010new}. In addition, having access to computers at home significantly boost academic achievement of young children when parents actively mediate their use \cite{hofferth2010home, espinosa2006technology}. However, the effectiveness of these tools often depends on parents' familiarity with and attitude toward technology. Mobile applications, for instance, can improve learning outcomes but require parents to possess sufficient technology efficacy to guide their use \cite{papadakis2019parental}.

On the negative side, technology introduces risks such as excessive screen time, exposure to inappropriate content, and misinformation, which necessitate parental intervention \cite{oswald2020psychological, howard2021digital}. According to parental mediation theory, parents mitigate these risks through restrictive mediation (e.g., setting limits), active mediation (e.g., discussing content), and co-use (e.g., shared use of technology) \cite{valkenburg1999developing}. Modern technologies like video games, location-based games (\textit{e.g.,} Pokemon Go), and conversational agents (\textit{e.g.,} Alexa) also require parents to adapt their mediation strategies to ensure responsible use \cite{valkenburg1999developing, nikken2006parental, sobel2017wasn, beneteau2020parenting, yu2024parent}. Overall, parents seek to leverage technology to support their children's learning due to ite effectivenss but are also mindful of its risks. Their involvement is therefore driven by both opportunities and concerns, highlighting the need to design tools that effectively involve parents to balance benefits and risks.

\subsection{Generative AI and Companion Robots for Parenting and Education}
Generative AI and companion robots offer human-like affordances, with AI simulating human intelligence and robots providing physical human-like features. Compared to conventional models (\textit{e.g.,} machine learning) and devices (\textit{e.g.,} laptops), these emerging technologies enable natural and social interactions, creating opportunities for novel paradigms to enhance parental involvement and children's learning while introducing their unique challenges.

\subsubsection{Generative AI}
GAI offers promising support for parents by enhancing their ability to educate and engage with their children. Prior work suggested that AI-driven systems can support parenting education \cite{petsolari2024socio} and provide evidence-based advice through applications and chatbots, delivering micro-interventions such as teaching parents how to praise their children effectively \cite{davis2017parent, entenberg2023user} or offering strategies to teach complex concepts \cite{mogavi2024chatgpt, su2023unlocking}. Many parents also prefer using GAI to create educational materials tailored to their children's needs, rather than granting children direct access to these tools \cite{han2023design}. Beyond educational support, AI-based storytelling tools address practical challenges (\textit{e.g.,} time constraints) by alleviating physical labor while fostering parent-child interactions \cite{sun2024exploring}. Furthermore, GAI offers advantages to children's learning directly. It can help create personalized learning experiences by providing timely feedback and tailoring content \cite{su2023unlocking, mogavi2024chatgpt, han2024teachers}, enhancing positive learning experiences \cite{jauhiainen2023generative}. For example, a LLM-driven conversational system can teach children mathematical concepts through co-creative storytelling, achieving learning outcomes similar to human-led instructions \cite{zhang2024mathemyths}.

Despite these benefits, several concerns persist regarding the use of GAI in education. Prior work highlighted the limitations of GAI, such as its limited effectiveness in more complex learning tasks,the limited quality of the training data, and its inability to offer comprehensive educational support \cite{su2023unlocking}. There is also a significant risk of GAI producing inaccurate or biased information, discouraging independent thought among children, and threatening user privacy \cite{su2023unlocking, han2023design, han2024teachers}. Many parents are skeptical about the role of AI in their children's academic processes, concerned about the accuracy of AI-generated content, and worry that over-reliance on AI could stifle independent thinking \cite{han2023design}.

%\todo{might need to add some structural transition here}
\subsubsection{Social companion robots}
Social companion robots have proven potential to assist parents in home education settings through studies in \textit{parent-child-robot} interactions. \citet{gvirsman2020patricc} showed that the robotic system, \texttt{Patricc}, fostered more triadic interaction between parents and toddlers than a tablet, and \citet{gvirsman2024effect} found that, in a parent-toddler-robot interaction, parents tend to decrease their scaffolding affectively when the robot increases its scaffolding behavior. Similarly, \citet{chen2022designing} found that social robots enhanced parent-child co-reading activities, while \citet{chan2017wakey} demonstrated that the WAKEY robot improved morning routines and reduced parental frustration. Beyond educational support, \citet{ho2024s} uncovered that parents envisioned robots as their \textit{collaborators} to support their children's learning at home and that their collaboration patterns can be determined by the parents' capability, availability, and motivation. Although parents generally have positive attitudes toward incorporating robots into their children's learning, they remain concerned about the risk of disrupting school-based learning and potential teacher replacement \cite{tolksdorf2020parents, lee2008elementary, louie2021desire}.

In addition to parental support, social companion robots also support children in education directly through \textit{child-robot interactions}. Physically embodied robots provide adaptive assistance and verbal interaction similar to virtual or conversational agents \cite{ramachandran2019personalized, leyzberg2014personalizing, schodde2017adaptive, brown2014positive}, yet they foster greater engagement with the physical environment and encourage more advanced social behaviors during learning \cite{belpaeme2018social}, leading to improved learning outcomes \cite{leyzberg2012physical}. Prior work demonstrated that companion robots can effectively support both school-based learning (\textit{e.g.,} math \cite{lopez2018robotic}, literacy \cite{kennedy2016social, gordon2016affective}, and science \cite{davison2020working}) and home-based learning activities (\textit{e.g.,} reading \cite{michaelis2018reading, michaelis2019supporting}, number board games \cite{ho2021robomath}, and math-oriented conversations with parents \cite{ho2023designing}). For example, \citet{kennedy2016social} suggested that children can learn elements of a second language from a robot in short-term interactions, and \citet{tanaka2009use} found that children who took on the role of teaching the robot gained confidence and improved learning outcomes.

%\todo{may need to make this a separate section and explain why we propose AI-assisted robots}

% \subsubsection{Research Gap}
Parental involvement in early education is crucial and AI-assisted robots can offer promising support by helping parents overcome practical barriers (\textit{i.e.,} time, energy, and skills) and addressing concerns about technological risks. Yet, limited research has examined how technology design can simultaneously alleviate these barriers and concerns. Though \citet{zhang2022storybuddy} emphasized the importance of flexible parental involvement during reading through a system called \textit{Storybuddy}, yet they focused on a virtual chatbot rather than a physical robot, and how the flexible modes may be used in different scenarios remain unknown. Similarly, \textit{ContextQ} \cite{dietz2024contextq} presented auto-generated dialogic questions to caregivers for dialogic reading, but primarily considered situations where parents are actively involved, not scenarios where parents cannot participate fully.

In this work, we address these gaps by exploring parental involvement contexts, understanding parents' perceptions of AI-generated content, and examining how parents collaborate with AI and robots under different scenarios. In the following sections, we describe our development of  \texttt{SET}, a card-based activity, to understand parental involvement contexts (Section~\ref{sec-card}), the design of the \texttt{PAiREd} system to enable parents to co-create learning activities with an LLM (Section~\ref{sec-system}), and user study aimed to discover use patterns and understand user perceptions of the system (Section~\ref{sec-study}).
\section{Research Questions and Hypotheses}
This paper aims to explore the effectiveness and potential risks of a QA-based response-writing support method by addressing the following three research questions:
% \begin{enumerate}[RQ1:]
%     \item How does a QA-based response-writing support approach affect users’ email-replying process?
%     \item How does a QA-based response-writing support approach affect the quality of the email response?
%     \item How does a QA-based response-writing support approach affect the perceived relationship between email sender and recipient?
% \end{enumerate}

\begin{enumerate}[RQ1:]
    \item How does a QA-based response-writing support approach affect users’ email-replying process?
    \item How does a QA-based response-writing support approach affect the quality of the email response?
    \item How does a QA-based response-writing support approach affect the perceived relationship between email sender and recipient?
\end{enumerate}

To answer the three research questions, we formed three sets of hypotheses.
The first set of hypotheses investigates the impact of a QA-based system on users' email-replying process.
AI-powered text generation reduces user input, saves time, and enhances efficiency~\cite{bastola2024llmbasedsmartreplylsr}. 
It also helps users quickly grasp email content with less cognitive effort, particularly through text summarization and list formatting, which enhances productivity~\cite{tarnpradab2017toward, nandhini2013use, modaresi2017commercial, daniel1998influence}. 
This suggests that presenting questions in a list format could streamline email responses, reducing the need for detailed prompts. 
Based on these insights, we propose the following hypotheses:
% \begin{enumerate}[\textrm{H1-}a:]
%     \item QA-based system enhances users’ email replying efficiency.
%     \item QA-based system reduces users’ cognitive load while replying to email.
% \end{enumerate}

\begin{enumerate}[\textrm{H1-}a:]
    \item QA-based system enhances users’ email replying efficiency.
    \item QA-based system reduces users’ cognitive load while replying to email.
\end{enumerate}

As a result, we expect users’ perceived work efficiency to improve.
Furthermore, since the QA-based system suggests appropriate language and helps create responses that align with the recipient’s needs, we anticipate that users’ satisfaction with their email replies will increase.
Thus, we propose the following hypothesis:
\begin{enumerate}[\textrm{H1-}c:]
    \item QA-based system enhances users' satisfaction with completing email response tasks, thereby being favorably received by users.
\end{enumerate}

Moreover, as described above, reducing users’ burden and improving their satisfaction may enhance their confidence in their tasks, which could lower their hesitation to begin working~\cite{schouwenburg1992procrastinators}.
Additionally, AI outputs that engage users’ curiosity may help trigger task initiation~\cite{brandtzaeg2017why, ling2021factors}.
Thus, we propose the following hypothesis:
\begin{enumerate}[\textrm{H1-}d:]
    \item QA-based system lowers the barriers to initiating email response tasks.
\end{enumerate}

According to previous research, there is a trade-off between the degree of AI intervention and the sense of agency and control, with higher levels of AI involvement shown to diminish these perceptions~\cite{Fu2023Comparing, Draxler2024The}.
Given that our QA-based approach also involves AI intervention during the phase where users create prompts for the LLM, the following hypothesis can be derived:
\begin{enumerate}[\textrm{H1-}e:]
    \item QA-based system diminishes users’ sense of agency and reduces their sense of control of the content.
\end{enumerate}

The second hypothesis concerns the quality of email responses.
AI support can be helpful in ensuring appropriate language use and grammar~\cite{fu2024text}. 
Furthermore, the QA-based approach is expected to assist users in correctly understanding the intent and demands of received emails and in verifying whether their responses meet these requirements.
Based on this, we propose the following hypothesis:
% \begin{enumerate}[\textrm{H2}:]
%     \item QA-based system enhances the perceived quality of the email response.
% \end{enumerate}
\begin{enumerate}[\textrm{H2}:]
    \item QA-based system enhances the perceived quality of the email response.
\end{enumerate}

The third set of hypotheses investigates the perceived relationship between email sender and recipient.
When users use the QA-based approach, it is expected that their communication partners will receive high-quality messages more quickly. 
Thus, the following hypothesis is derived.
\begin{enumerate}[\textrm{H3-}a:]
    \item QA-based system makes a good impression on the user's communication partner.
\end{enumerate}

On the other hand, when users create messages using the AIMC tool, they may feel a sense of discomfort with the message and guilt for not having fully composed it themselves~\cite{fu2024text}. 
We hypothesized that a QA-based approach would further intensify this discomfort by reducing the user’s sense of agency and control more than previous approaches.
\begin{enumerate}[\textrm{H3-}b:]
    \item QA-based system enlarges the psychological distance that users perceive toward their communication partners.
\end{enumerate}
% 1. how many questions did we specify and how? Now it said appropriate number of questions, but it's unclear what is "appropriate"
% 2. How did the question generated? based on what criteria? for instance, did we ask gpt to generate random questions? if not, what exact did gpt do? would be better to show one example of prompt in this part.

\section{Proposed LLM-Powered QA-Based Approach: ResQ}
\label{sec:Proposed_Approach}
% 本セクションでは、電子メール対応タスクをサポートするために提案されたアプローチ「ResQ」について説明する
This section describes the proposed approach, ResQ, for supporting email response tasks. 
\begin{figure*}[t]
\centering
\includegraphics[width=\textwidth]{figure/overview_of_process.pdf}
\caption{The overview of the process of creating a reply message using ResQ. A) The LLM first generates multiple-choice questions in JSON format. B) Users select their desired responses to their counterparts. C) The LLM then generates a reply draft in JSON format based on the users' selections. D) Finally, users review and edit the LLM-generated draft before sending the reply.}
\label{fig_overview_of_process}
\Description{This figure illustrates the process of generating an email reply using a large language model (LLM) across four stages. In Stage A (Generate Questions), the system takes the email data and a prompt, which are then sent to the LLM server. The server processes this information and generates a set of questions to clarify the content of the reply. These questions, along with possible answer choices and context, are returned in JSON format. In Stage B (Answer Questions), the user is presented with the questions generated by the LLM. These questions may include simple "Yes" or "No" options or multiple-choice selections. The user answers the questions by choosing the appropriate options or providing custom responses. In Stage C (Generate Reply Draft), the user’s answers, along with the original email data and prompt, are sent back to the LLM server. Based on this input, the server generates a draft of the email reply, which is also returned in JSON format. In Stage D (Check Reply Draft), the user reviews the draft generated by the LLM. After checking the content and making any necessary revisions, the user finalizes and sends the email reply.}
\end{figure*}
\begin{figure*}[t]
\centering
\includegraphics[width=\textwidth]{figure/interface.pdf}
\caption{Interface of ResQ. On the left, the content of the email is displayed, with an editor and a ``Reply'' button below for sending a reply. In the center, questions and options for users are shown, allowing the creation of custom options if needed. Additionally, the section of the email corresponding to the selected question is highlighted. On the right, fields are provided to customize the reply generated by the LLM, including options to specify the relationship with the counterpart and buttons to choose the formality, tone, and length of the email. A free-text input field and a "Generate Reply" button are also below.
}
\label{fig_interface}
\Description{This figure illustrates the interface of the ResQ system, which is divided into three main sections. On the left side (A, B), the content of the incoming email is displayed. The email includes important information, such as the sender, subject, and body text. Key sections of the email are highlighted based on the questions generated by the system, helping the user focus on relevant points. Below the email, there is an editor where the user can compose their reply, with a "Reply" button (H) available to send the response once it's ready. In the center section (C, D), the system displays questions generated by the AI, which are intended to assist the user in composing their reply. These questions correspond to specific parts of the email, and as the user answers them, the relevant section in the email is highlighted (A). Users can also customize responses by adding new options if needed (D). user can specify their relationship with the email recipient (e.g., professor or student), adjust the formality and tone of the response, and select the desired length of the reply. An additional free-text input field is available for further customization requests (E). Once all preferences are set, the user can click the "Generate Reply" button (F) to produce a draft response based on their inputs.}
\end{figure*}
Fig.~\ref{fig_overview_of_process} illustrates the overview of how a reply message is created using ResQ.
% (A) After the system detects users' initiation of the reply task, it generates multiple-choice questions using an LLM (in this study, GPT-4o~\cite{GPT4o}). 
% (B) Then, users communicate their reply strategy to the AI by responding to these questions.
% (C) When the system detects that users have pressed the ``Generate Reply'' button, it presents a draft of the reply to users.
% (D) Finally, users review and revise the reply draft generated by the LLM and send the response.
Fig.~\ref{fig_interface} shows the actual interface of ResQ.
The following sections describe the specific functions involved in each step of this process.

\paragraph{\textbf{A: Generate Questions}}
\label{sec:generate_questions}
% ResQは返信の必要性を検知すると、大規模言語モデル (本研究ではGPT-4o) を使用して、多肢選択式の質問を生成する
% また我々は、ユーザが質問をクリックすると、その質問が受信メールのどの部分に対応しているかがハイライトされるようにした
% 我々はこれらの機能を実現するために、まずモデルに対して、モデルの役割(ユーザに対する質問と有益そうな選択肢のペアを複数生成すること)と、作成する質問の目的(メールに含まれる全ての要求を抽出し、送信者がそれぞれに対してどのように返答したいかを明確にすること)をプロンプトとして与えた
% さらに、モデルに受信メールの文章、タイトル、送信者の情報(名前、メールアドレス)、受信メールの過去のやり取りの文章、ユーザの情報(名前、メールアドレス)を提供した
When a user first activates ResQ, the system uses an LLM (in this study, GPT-4o~\cite{GPT4o}) to generate multiple-choice questions (Fig.~\ref{fig_interface}-C). 
The LLM extracts all parts of the email that require a reply, generates corresponding questions and presents possible response options.
Additionally, if a user clicks on any generated question, the relevant part of the email is highlighted (Fig.~\ref{fig_interface}-A).

% To implement these features, we first provided the LLM with the email's text, subject, sender information, text from past email interactions, and the user's information (name and email address).
% プロンプトは~\cite{relatedwork}を参考に作成し、文脈を踏まえており、返信を作成する上で役に立ち、適切な数の(メールのすべての要件に対しては漏れなく)質問と、それに役立つ選択肢を生成するように指示した。
% またプロンプトには質問と選択肢の生成例を含めることで、生成の質を上げた。
\red{Following the approach described in~\cite{bsharat2023principled}, we designed a structured prompt that guides the LLM to determine how many questions are necessary and sufficient to cover all requirements in the incoming email without omission. 
Instead of pre-specifying a fixed number of questions, the prompt instructs the model to produce an ``appropriate'' number of questions, where ``appropriate'' is defined as the minimal set of questions needed to address all points raised by the sender while avoiding redundant or irrelevant inquiries.
To ensure that the questions were generated systematically rather than randomly, we provided explicit criteria within the prompt. 
These criteria included referencing the sender's intent, quoting relevant portions of the original email verbatim, and offering multiple-choice options where applicable. 
We also provided concrete examples within the prompt to illustrate the desired format and style of the generated questions and corresponding answer choices. 
By doing so, we ensured that the LLM's output was both well-grounded and easy for the recipient to answer.
The detailed prompt used to guide the LLM in this process is shown in the appendix.}

% We designed the prompt with reference to~\cite{bsharat2023principled}, incorporating contextual considerations to ensure it effectively supports reply composition. 
% We instructed the LLM to generate an appropriate number of questions that comprehensively cover all the email's requirements without omissions, along with relevant response options. 
% To further guide the model and improve output accuracy, we included examples of question and option generation within the prompt.

\paragraph{\textbf{B: Answer Questions}}
% 次にユーザは受信メールと生成された質問、選択肢を同時に見ながら、質問に回答する
% 我々は有益な選択肢がない場合を想定して、ユーザ自身が選択肢を追加できるようにした
% また、LLMにメールのcontextを伝えることができるように、送信者と受信者の関係性を記入できるboxを設置した
% さらに以前の研究に基づき~\cite{fu2024text}、ユーザがAIメールの文章をカスタマイズできるように、ユーザが期待する返信のトーンやスタイル、長さを調整するための選択肢を提供した
% また、ユーザがそれ以外のリクエストをAIに対してできるように、AIに対する自由記述欄を設置した
% ユーザはこれらの作業が完了するとGenerate Replyボタンを押す
Next, users view the incoming email (Fig.~\ref{fig_interface}-B) alongside the generated questions (Fig.~\ref{fig_interface}-C) and options and proceed to answer them. 
In anticipation of situations where none of the provided options are useful, we enabled users to add their own options (Fig.~\ref{fig_interface}-D). 
Additionally, to help the LLM better understand the context of the email, we introduced a box where users can specify the relationship between the sender and the recipient (Fig.~\ref{fig_interface}-E, top). 
Furthermore, following previous research~\cite{fu2024text}, we provided users with controls to adjust the tone, style, and length of the reply to match their preferences better, thereby giving them more flexibility in customizing the AI-generated response (Fig.~\ref{fig_interface}-E, middle). 
A free-text field was also included to allow users to make other specific AI requests (Fig.~\ref{fig_interface}-E, bottom). 
After completing these steps, users can click the ``Generate Reply'' button (Fig.~\ref{fig_interface}-F).

\paragraph{\textbf{C: Generate Reply Draft}}
% ResQはユーザが"Generate Reply"ボタンを押したことを検知すると、大規模言語モデルを使用して、返信のドラフトを作成する
% ユーザの期待するような返信のドラフトが出力されるように、我々はモデルに対して、受信メールとその関連情報、AIの質問とそれに対応するユーザの回答、ユーザが返信案に期待する他の要素(トーン、スタイル、長さ、その他の要望)、ユーザの情報を提供した
When the user clicks the ``Generate Reply'' button, ResQ detects the action and uses the LLM to generate a reply draft.
% To ensure that the draft aligns with users' expectations, we first provided the LLM with the information provided when Sec.~\ref{sec:generate_questions}, the generated questions, corresponding users' answers, and users' preferences (\textit{e.g.}, tone, style, length, and any additional requests).
% Then, the LLM generates a draft of the reply.
The prompt used for this function is shown in the appendix.

\paragraph{\textbf{D: Review Reply Draft}}
Once the draft reply is generated, users can review the draft in detail (Fig.~\ref{fig_interface}-G).
Moreover, if users find that extensive revisions are needed or if they want to explore alternative phrasing, they have the option to request the AI to regenerate a new draft based on updated input or preferences.
After completing these steps, users can click the ``Reply'' button (Fig.~\ref{fig_interface}-H).
\section{Method of Study 1}
\red{To test our hypotheses and answer three research questions, we first conducted a controlled experiment, focusing on gaining a quantitative understanding.}
% original
% We conducted a controlled experiment (Study 1) and a field study (Study 2) to test our hypotheses and answer three research questions. 
% Study 1 was conducted in a controlled environment, focusing on gaining quantitative understanding. 
% To further examine the actual usage of our QA-based system, we conducted a field experiment for Study 2 to gain a qualitative understanding of how our QA-based method influenced the practice of email replies. 

\subsection{Experiment Design}
Study 1 was designed to quantitatively assess how ResQ influences the writing process (RQ1), the quality of email replies (RQ2), and the perceived relationships with others (RQ3) compared to scenarios without AI intervention and when using traditional AIMC tools. 
% 実験は日本人を対象に、日本語でやったことを明記する
\red{The experiment targeted Japanese participants and was conducted entirely in Japanese.}
% to ensure the tasks reflected natural communication in formal settings such as workplaces and schools.
This experimental context was designed as communication in formal settings, such as \red{office-related communication, research collaboration in academic institutions, and interactions with external organizations}. 
\red{It focused on time-consuming emails that were characterized as lengthy, containing multiple requests, or requiring detailed and polite responses. 
Simple and straightforward emails, such as those that can be answered with a single word or phrase (\textit{e.g.}, ``Understood''), were excluded from the scope.}

Participants were assigned the role of message recipients and required to craft replies based on the scenarios and supplementary information provided. 
\red{The messages used in the experiment were collected from 10 volunteers who provided real emails they had received in formal communication contexts.}
\red{These volunteers included office workers, graduate students, and teaching staff, all of whom were Japanese and engaged in email communication regularly (at least once per month).}
To ensure anonymity, identifying details were removed during the preparation process. 
\red{Based on the design principles of ResQ, we excluded extremely short emails and emails that could be replied to with a single word from the selection process.}
Each scenario included details about the sender, the recipient, and the context in which the message was received. 
Multiple scenarios were included in the experiment to minimize the influence of any single scenario and increase the variety. 
Additionally, supplementary information, such as the recipient’s schedule and potential questions, was provided to prevent excessive variability in the responses among participants. 

In total, we created twenty types of \red{emails}, with two assigned to the practice session and eighteen to the test session.
The length of the \red{emails} used in the test session averaged 404 \red{Japanese} characters, with the shortest being 135 characters and the longest being 925 characters.
\red{The scenarios covered a wide range of formal communication situations, including responding to a request for data submission in the workplace, answering a survey from a professor, asking questions based on guidance from a language school’s customer support team, and addressing a request for schedule adjustments as a part-time worker.}
\red{Additionally, these emails varied in structure, ranging from structured formats with bullet points to more non-structured, free-text formats.}
\red{The specific emails and examples of ResQ-generated questions and options used in the experiment can be referred to in the supplementary materials.}
% \red{The specific emails used in the experiment, including full message content, sender information, scenarios, and supplementary information, are provided as supplementary materials.}

\subsection{Experimental Conditions}
We employed a within-subject design with three conditions: QA-based, Prompt-based, and No-AI.
\red{This design was chosen to control for individual differences among participants, such as varying levels of language proficiency or familiarity with AI systems, ensuring a fair comparison across conditions.}
To illustrate each condition, consider the scenario of a participant who, as an employee of a company, is asked by their superiors to assume the role of a fixed asset committee member.

% QA-based method.
In the QA-based condition, participants created replies using the QA-based AI.
The system detected when participants navigated to the next email screen, inferred that they were initiating a reply, and then generated relevant questions.
For example, the system might ask, ``Would you be willing to take on the role of the fixed asset manager?'' ``Is there any issue with handling the annual inventory check?'' or ``Please let us know if you have any questions or concerns about the tasks.''
Participants could respond by selecting from provided options (\textit{e.g.}, ``yes,'' ``no''), adding their own options, or ignoring the questions entirely. 
After responding, they would press the ``Generate Reply'' button, which would produce an AI-generated draft in the reply box. 
Participants could then regenerate or modify the draft as needed to finalize their response.

% Prompt-based condition
In the Prompt-based condition, participants created replies using a prompt-based AI without the QA feature of the QA-based method.
Participants wrote prompts for the AI to generate a draft email response, which they then edited to create their replies.
For example, a participant might input a prompt such as, ``I want to convey my acceptance of the fixed asset committee role. ...''
Afterward, similar to the QA-based system, participants would press the ``Generate Reply'' button and, if necessary, either regenerate the draft or revise its content.

% No-AI condition
In the No-AI condition, participants created email replies manually without using AI assistance.

\subsection{Participants}
\red{
\begin{table*}[t]
% \caption{\red{Backgrounds of participants in Study 1, including age, job roles, email experience, frequency of email sending and AI tool usage, and use of AI for email purposes.}}
\caption{Backgrounds of participants in Study 1, including age, job roles, email experience, frequency of email sending and AI tool usage, and use of AI for email purposes. \blue{Some fields are marked as - due to missing responses from participants.}}
\Description{The table provides an overview of participants in Study 1, detailing their demographic information, email usage habits, and AI tool usage patterns. The participants, ranging in age from 20 to 57, include university students, office workers, a part-time worker, and individuals categorized under "other" occupations. Both male and female participants are represented, reflecting a diverse group in terms of age, occupation, and technological engagement. Participant P1 is a 34-year-old male office worker with 20 years of email experience. He typically replies to 7 emails per week but rarely uses AI tools, and he never utilizes AI for email-related tasks. P2, a 22-year-old male university student, has 5 years of email experience and actively engages with emails, replying to more than 21 per week. He uses AI tools daily, with AI assisting in 50–80\% of his email-related tasks. Similarly, P3, a 22-year-old female university student, reports no specific email experience or weekly email activity but uses AI tools frequently, relying on AI for 50–80\% of her email tasks. P4, a 21-year-old female university student with 4 years of email experience, replies to 0–2 emails weekly. She uses AI tools frequently but for less than 20\% of her email-related activities. P5, a 20-year-old male university student, has no reported email experience or weekly email activity. He rarely uses AI tools and does not use them for email purposes. P6, a 38-year-old male office worker, has 3 years of email experience and replies to 0–2 emails weekly. He rarely engages with AI tools and never applies them to email tasks. P7, a 31-year-old female categorized as "unemployed," has 12 years of email experience and replies to 3–4 emails per week. She never uses AI tools, either for general purposes or for email tasks. In contrast, P8, a 25-year-old male university student with 4 years of email experience, replies to 0–2 emails per week. He uses AI tools frequently, with AI assisting in 50–80\% of his email-related activities. P9, a 39-year-old female office worker, has 20 years of email experience and replies to over 21 emails per week. She uses AI tools frequently but never applies them to email tasks. P10, a 24-year-old male university student, has no reported email experience or weekly activity. He uses AI tools daily, although only for less than 20\% of his email-related tasks. P11, a 22-year-old female university student with 4 years of email experience, replies to 0–2 emails weekly. She rarely engages with AI tools and does not use them for email tasks. Finally, P12, a 57-year-old female office worker with 20 years of email experience, replies to 0–2 emails weekly. She uses AI tools frequently but never for email-related purposes.}
\label{tab_study1_participants}
\red{
\begin{tabular}{cccccccc}
\hline
ID   & Gender & Age & Job                  &  Email Experience&Emails/Week & AI Tool Usage& AI for Email Usage    \\ \hline
P1   & M& 34  & Office Worker        &  20 years&7           & Rarely         & Never                 \\
P2   & M& 22  & Univ. Student        &  5 years&21+         & Daily              & 50–80\%\\
P3   & F& 22  & Univ. Student        &  -&-& Frequently& 50–80\%\\
P4   & F& 21  & Univ. Student        &  4 years&0–2         & Frequently& <20\%\\
P5   & M& 20  & Univ. Student        &  -&-           & Rarely         & <20\%\\
P6   & M& 38  & Office Worker        &  3 years&0–2         & Rarely         & Never                 \\
P7   & F& 31  & \blue{Unemployed}                &  12 years&3–4         & Never         & Never                 \\
P8   & M& 25  & Univ. Student        &  4 years&0–2         & Frequently& 50–80\%\\
P9   & F& 39  & Office Worker        &  20 years&21+         & Frequently& Never                 \\
P10  & M& 24  & Univ. Student        &  -&-           & Daily              & <20\%\\
P11  & F& 22  & \blue{Univ. Student}     &  4 years&0–2         & Rarely         & Never                 \\
P12  & F& 57  & Office Worker        &  20 years&0–2         & Frequently& Never                 \\ \hline
\end{tabular}
}
\end{table*}
}

% \begin{itemize}
%     \item Job categories such as ‘Office Worker’, ‘Part-time Worker’, and ‘Other’ were chosen based on predefined options provided to participants, as shown in the survey (e.g., office workers, contract workers, freelancers, etc.).
%     \item Fields marked as ‘-’ indicate missing responses, as participants did not provide the requested information.
%     \item ‘Part-time Worker’ refers to individuals employed on a part-time basis, while ‘Other’ includes occupations such as freelance or non-traditional job roles.
% \end{itemize}
Twelve participants (six males and six females, aged 20-57) were recruited via a local Japanese participant recruiting platform (see Tab.~\ref{tab_study1_participants}).
The average age of the participants was 29.6 (SD = 11.0). 
\red{The sample size $n=12$ was determined based on an a priori power analysis (effect size $f=0.4$, significance level $p=0.05$, power = 0.8, correlation among repeated measures = 0.5) as well as the previous study~\cite{Mu2024Whispering}.}
The participants were paid approximately \$21 USD for participation, and the experiment lasted around two hours.
This study was approved by the institute's ethical review board.

\subsection{Procedure}
The participants first read the study instructions and the right to participate and then consented to participate in the experiment. 
Next, they were given an explanation of the purpose of the experiment and the use of the AI systems (Prompt-based and QA-based systems). 
% \red{To avoid the AI nocebo/placebo effect~\cite{}, the explanations were carefully designed to use neutral and standardized language, avoiding any statements that could imply one system was superior or inferior to another. 
% This ensured that participants’ perceptions of the systems were not biased before engaging in the tasks.}
% Subsequently, participants were randomly assigned to reply to six emails per condition based on the Latin square design.
\red{Participants were then randomly assigned to reply to six emails per condition using a Latin square design, which counterbalanced the order of conditions and mitigated potential order effects\blue{~\footnote{\blue{We conducted analyses to examine the potential order effect. The results of this analysis are provided in the appendix.}}}.}
In each condition, participants first engaged in a practice session where they read and replied to two emails \red{to familiarize themselves with the system.}
Then, they read and replied to six emails, which were presented in a randomly assigned order \red{to further reduce any sequence-related biases.}
After replying to six emails for each condition, participants were asked to complete a questionnaire regarding their experience with the task. 
% \blue{To ensure participants could manage their workload during the study, they were allowed to take break between conditions. }
\blue{To ensure participants could manage their workload during the study, they were allowed to take a short break after completing tasks in each condition. 
% Additionally, participants had the flexibility to take breaks at their discretion during non-task periods within each condition, such as after the practice session or before the questionnaire.
}
After completing all conditions, they were asked to fill out a comparative questionnaire evaluating the three conditions. 
\red{In addition, follow-up interviews were conducted to gather deeper insights into their experiences and preferences.}
This study was conducted remotely for all participants and lasted approximately two \red{and a half} hours in total.

\subsection{\red{Evaluation Session}}
After completing the main experiment, we conducted an additional evaluation session to assess the quality of the email responses created by participants and the impressions of participants as email senders.
This session involved a group of eighteen Japanese evaluators (ten males and eight females, aged 20-57) recruited via a local participant recruiting platform~\footnote{Participants were recruited from Lancers.jp, an online freelancing platform.}. 
The average age of the evaluators was 40.6 (SD = 8.3).
% evaluatorsは、メールを使用したコミュニケーションを最低5年以上、平均18.7年経験していた。
% また1人を除いて、evaluatorsは月に一回以上、メールを使用したコミュニケーションを行なっていた。
\blue{The evaluators had a minimum of five years and an average of 18.7 years of experience in email-based communication. 
Furthermore, with the exception of one individual, the evaluators engaged in email-based communication at least once a month.}
Each evaluator assessed email replies written by twelve different participants for a specific scenario. 
The evaluators were paid approximately \$2.5 USD for their participation, and the evaluation session lasted around fifteen minutes.

\subsection{Measurements}
% We used multiple measurements to test our hypotheses.
% 参加者が返信タスクに取り組んでいる際の行動から、Efficiency, Prompt Character Countsを算出した。
% また参加者の実験後のアンケートの回答から、Cognitive Load, Difficulty in Understanding Email Content, Satisfaction with Completing Task, Future Preference, Difficulty in Initiating the Action for Replying to Emails, Sense of Agency, Sense of Control, Psychological Distance between Participants and Their Counterpartを算出した。
% さらに、Evaluation Sessionにおけるevaluatorのアンケートの回答から、Perceived Quality of the Email, Impression of Participants as Email Sendersを算出した。
\red{We used multiple measurements to test our hypotheses. 
From participants' behavior during the email reply task, we calculated two measures: efficiency and prompt character count. 
From their post-experiment questionnaire responses, we evaluated cognitive load, difficulty in understanding email content, satisfaction with completing the task, difficulty in initiating the action for replying to emails, sense of agency, sense of control, and psychological distance between participants and their counterparts. 
Additionally, from evaluators’ questionnaire responses during the evaluation session, we assessed the perceived quality of the email and the impression of participants as email senders.}
\subsubsection{Efficiency of Replying to Emails (H1-a)}
We calculated the efficiency of replying to emails using task completion time and total character count.
The efficiency of replying to emails is defined as the amount of text contributing to the final output that can be typed per second, where a higher score indicates better task efficiency.
For task completion time, we recorded the time participants took to reply to an email, starting from when the email appeared on the screen to when the participant pressed the send button.
For total character count, we considered the text in the reply box when the participant pressed the Reply button as the final response and counted its characters.

\subsubsection{Prompt Character Counts (H1-a)}
We also calculated the average number of characters typed by participants to have the AI generate email drafts as the prompt character counts in each condition.
Under the Prompt-based condition, we measured the number of characters participants typed in the free-text field for the AI. 
Under the QA-based condition, the prompt character counts included this number plus any additional characters typed by the participants when they added their own options.

\subsubsection{Cognitive Load for Replying to Emails (H1-b)}
We used the NASA-TLX~\cite{hart1988development} questionnaire to measure cognitive load across six subscales: mental demand, physical demand, temporal demand, performance, effort, and frustration and calculated the Raw-TLX~\cite{byers1989traditional}.
Participants answered the above items using a 10-point Likert scale.
The Raw-TLX score is calculated as the simple average of six scales, where higher scores indicate a greater cognitive load.

% Additionally, to assess cognitive load specifically related to understanding received emails, we conducted a survey using a 7-point Likert scale, where 1 indicated strongly disagree, 4 indicated neutral, and 7 indicated strongly agree. 
% Participants were asked to rate the perceived load of understanding emails under all conditions.
\subsubsection{\red{Difficulty in Understanding Email Content (H1-b)}}
\red{Additionally, to assess cognitive load specifically related to understanding received emails, we used a 7-point Likert scale. 
Participants rated their agreement with the statement, ``I found it difficult to understand the sender’s intentions or requests in the email,’’ where 1 indicates strongly disagree, 4 indicates neutral, and 7 indicates strongly agree.}
% Specifically, for all conditions, we asked about the perceived load of understanding the received emails.

% Q. I was able to understand the sender's intent and request easily.

\subsubsection{Satisfaction with Completing Task (H1-c)}
We evaluated participants' satisfaction with completing their task using a 7-point Likert scale, where 1 indicates strongly disagree, 4 indicates neutral, and 7 indicates strongly agree.
Specifically, the satisfaction of completing their task was evaluated based on their satisfaction with efficiency and their satisfaction with the quality of the email they created.
We asked the following questions: 
(1) I felt that I was able to create a high-quality response. 
(2) I felt that I was able to complete the response efficiently.
We averaged the scores from two items and treated them as an index of the satisfaction with completing their task.
% The Cronbach's Alpha for the two items is $0.889$.

% \subsubsection{\red{Future Preference (H1-c)}}
% \red{We evaluated participants' preferences for future use across all conditions using a 7-point Likert scale.
% Participants rated their agreement with the statement, ``I would prefer to use this approach for replying to emails in the future,'' where 1 indicates strongly disagree, 4 indicates neutral, and 7 indicates strongly agree.
% }

\subsubsection{Difficulty in Initiating the Action for Replying to Emails (H1-d)}
We tested H1-d using a survey with a 7-point Likert scale (1 = strongly disagree, 4 = neutral, and 7 = strongly agree) to evaluate perceived barriers to task initiation. 
Specifically, we asked the following question: I felt a high barrier to initiating email response tasks.

\subsubsection{Sense of Agency and Control (H1-e)}
We evaluated participants' perceived sense of agency and control using a 7-point Likert scale, where 1 indicates strongly disagree, 4 indicates neutral, and 7 indicates strongly agree.
Specifically, drawing on previous research~\cite{Fu2023Comparing, Draxler2024The}, the sense of agency was evaluated by assessing whether participants felt they were the ones who wrote the responses, while the sense of control was evaluated by whether they felt they had control over the content of the responses. 

\subsubsection{Perceived Quality of the Email by Evaluators (H2)}
\label{sec:method2_H2}
The quality of each email reply was evaluated \red{by evaluators} using a 7-point Likert scale, where 1 indicates strongly disagree, 4 indicates neutral, and 7 indicates strongly agree.
It was evaluated on three aspects: politeness (whether it was politely written), readability (whether it had an easy-to-understand structure), and meeting demands (whether it appropriately addressed the recipient's demands). 
We averaged the scores from three items and treated it as an index of the perceived quality of the email.
% The Cronbach's Alpha for the three items is $0.846$.

\subsubsection{\blue{Perceived Impression of Participants by Evaluators (H3-a)}}
% \red{To evaluate the impression of the email senders (participants) as perceived by others, we recruited the same group of eighteen Japanese evaluators described in Sec.~\ref{sec:method2_H2}.
\red{Following a previous study~\cite{rau2009effects}, we asked the evaluators to read the email and assess their impressions of the senders (participants) based on two aspects: whether the participants were perceived as likable and whether they were perceived as kind, using a 7-point Likert scale, where 1 indicates strongly disagree, 4 indicates neutral, and 7 indicates strongly agree. 
We averaged the scores from these two items to create an index of the impression of the email sender.}

% The same eighteen evaluators were also asked to assess their impression of the email sender who were participants. 
% Following a previous study~\cite{rau2009effects}, we asked the eighteen evaluators to read the email and rate participants' impression toward the email senders (participants) in two aspects: whether they were perceived as likable or kind, using a 7-point Likert scale, where 1 indicates strongly disagree, 4 indicates neutral, and 7 indicates strongly agree.
% We averaged the scores from two items and treated it as an index of the impression of the email sender.
% Cronbach’s $\alpha$ for this index was as follows: $0.955$ for the No-AI condition, $0.950$ for the Prompt-based condition, and $0.917$ for the QA-based condition.
% The Cronbach's Alpha for the two items is $0.946$.

\subsubsection{Psychological Distance between Participants and Their Counterpart (H3-b)}
\begin{figure*}[t]
\centering
\includegraphics[width=\textwidth]{figure/study1_IOS.pdf}
\caption{Inclusion of Other in the Self (IOS). The diagram above the x-axis is an example of what participants were shown when responding to the questionnaire. The degree of overlap between the two circles represents the psychological distance between oneself and others.}
\label{fig_study1_IOS}
\Description{This figure represents the "Inclusion of Other in the Self (IOS)" scale, which is used to measure psychological closeness or relational intimacy. The diagram depicts two circles, labeled "Self" and "Other," with varying degrees of overlap. Participants were asked to choose the level of overlap that best represented their relationship with another person. On the far left (score 1), the circles are completely separate, indicating a significant psychological distance between the self and the other. In the middle (score 4), the circles partially overlap, suggesting a moderate level of psychological closeness. On the far right (score 7), the circles almost completely overlap, representing a very close and intimate relationship between the self and the other.}
\end{figure*}
We evaluated the perceived psychological distance using the Inclusion of Other in the Self (IOS) scale~\cite{aron1992inclusion}.
Participants choose a pair of circles from seven with different degrees of overlap (\red{see Fig.~\ref{fig_study1_IOS}}). 
1 = no overlap; 2 = little overlap; 3 = some overlap; 4 = equal overlap; 5 = strong overlap; 6 = very strong overlap; 7 = most overlap. 
The number chosen is the participants’ score.
The higher the score was, the closer participants felt they were with the email sender.
% todo
% Study 1において、参加者が無視した質問と答えた質問についての定性的な結果を追加
% Study 1において、non-structured/structuredなメールのそれぞれにおける質問の出力例を記載し、定性的な結果を追加
\begin{figure*}[t]
\centering
\includegraphics[width=\textwidth]{figure/study1_1.pdf}
\caption{Results of participants' efficiency and cognitive load of replying to emails. Left: Efficiency for replying to emails. Middle: Prompt character count. Right: Cognitive load for replying to emails. The significant differences between conditions were from post-hoc analysis after doing one-way repeated measure ANOVA.}
\label{fig_study1_efficiency_and_cognitiveLoad}
\Description{The figure consists of three box plots, each representing different comparisons of three experimental conditions: No-AI, Prompt-based, and QA-based. Each box plot compares a specific measure between the conditions. The p-values indicating statistical significance between different conditions are also labeled above the plots. Efficiency of Replying to Emails: Three conditions are compared: No-AI, Prompt-based, and QA-based. The No-AI group has a median of 0.65, with a first quartile at 0.51 and a third quartile at 1.05, with a minimum of 0.34 and a maximum of 1.55. The Prompt-based group has a median of 1.50, with the first quartile at 0.87 and the third quartile at 2.12, with a minimum of 0.70 and a maximum of 2.89. The QA-based group has a median of 1.89, with a first quartile at 1.35 and a third quartile at 2.12, with a minimum of 0.90 and a maximum of 3.73. P-values indicate significant differences between groups: between No-AI and Prompt-based (p = 0.013), between No-AI and QA-based (p = 0.002), and between Prompt-based and QA-based (p = 0.046). Prompt Character Count: Only two conditions are compared: Prompt-based and QA-based. The Prompt-based group has a median of 37.42 characters, with a first quartile at 28.75 and a third quartile at 53.50, with a minimum of 11.67 and a maximum of 64.50. The QA-based group has a median of 25.50 characters, with a first quartile at 19.00 and a third quartile at 31.67, with a minimum of 14.33 and a maximum of 47.83. The p-value indicates a significant difference between the two groups (p = 0.01). Raw TLX: This plot compares three conditions: No-AI, Prompt-based, and QA-based. The No-AI group has a median score of 3.70, with a first quartile of 2.93 and a third quartile at 4.35, with a minimum of 2.20 and a maximum of 4.90. The Prompt-based group shows a median of 2.40, with a first quartile at 2.03 and a third quartile at 2.70, with a minimum of 1.00 and a maximum of 4.80. The QA-based group has a median of 2.10, with a first quartile of 1.93 and a third quartile at 2.48, with a minimum of 0.70 and a maximum of 4.40. The p-values indicate significant differences between No-AI and Prompt-based (p = 0.017), between No-AI and QA-based (p = 0.008), and between Prompt-based and QA-based (p = 0.018). Each box plot represents the distribution of values for the respective metric, and the whiskers show the variability outside the upper and lower quartiles. The statistical differences (p-values) highlight where the comparisons between conditions are significant.}
\end{figure*}
\begin{figure*}[t]
\centering
\includegraphics[width=\textwidth]{figure/study1_2.pdf}
\caption{Summary of Likert scale responses. \red{Measurements H2 and H3-a were assessed by third-party evaluators rather than the participants themselves.} The significant differences between conditions were from post-hoc analysis after one-way repeated measure ANOVA or the Friedman test (* and \red{**} indicate the significance found at levels of 0.05 and 0.01, respectively).}
\label{fig_study1_questionnaire}
\Description{The figure consists of box plots comparing three experimental conditions, No-AI, Prompt-based, and QA-based, across multiple subjective measures related to email task performance. The box plots represent the distribution of responses across these conditions, with p-values indicating statistically significant differences between them. H1-b: Difficulty in Understanding Email Content: No-AI: The median is 4, with a first quartile at 3.25 and a third quartile at 6.25, with a minimum of 1 and a maximum of 7. Prompt-based: The median is 4, with a first quartile at 4 and a third quartile at 5.25, with a minimum of 1 and a maximum of 7. QA-based: The median is 6, with a first quartile at 5 and a third quartile at 7, with a minimum of 4 and a maximum of 7. Significant differences exist between Prompt-based and QA-based (p < 0.01) and between No-AI and QA-based (p < 0.01). H1-c: Satisfaction with Completing Tasks: No-AI: The median is 2.5, with a first quartile at 1.875 and a third quartile at 3.625, with a minimum of 1 and a maximum of 5. Prompt-based: The median is 5.25, with a first quartile at 4.375 and a third quartile at 6.5, with a minimum of 3.5 and a maximum of 7. QA-based: The median is 6.5, with a first quartile at 6 and a third quartile at 6.625, with a minimum of 5 and a maximum of 7. Significant differences exist between No-AI and QA-based (p < 0.01), between Prompt-based and QA-based (p < 0.01), and between Prompt-based and QA-based (p < 0.05). H1-d: Difficulty for Task Initiation: No-AI: The median is 5, with a first quartile at 5 and a third quartile at 6, with a minimum of 1 and a maximum of 7. Prompt-based: The median is 3, with a first quartile at 1.75 and a third quartile at 4, with a minimum of 1 and a maximum of 5. QA-based: The median is 2, with a first quartile at 1 and a third quartile at 2, with a minimum of 1 and a maximum of 3. Significant differences exist between No-AI and Prompt-based (p < 0.01), between Prompt-based and QA-based (p < 0.01), and between No-AI and QA-based (p < 0.01). H1-e: Sense of Agency: No-AI: The median is 7, with a first quartile at 6.75 and a third quartile at 7, with a minimum of 6 and a maximum of 7. Prompt-based: The median is 4, with a first quartile at 4 and a third quartile at 4.25, with a minimum of 3 and a maximum of 5. QA-based: The median is 2.5, with a first quartile at 1.75 and a third quartile at 3.25, with a minimum of 1 and a maximum of 5. Significant differences exist between No-AI and Prompt-based (p < 0.01), between Prompt-based and QA-based (p < 0.01), and between No-AI and QA-based (p < 0.01). H1-e: Sense of Control: No-AI: The median is 7, with a first quartile at 7 and a third quartile at 7, with a minimum of 5 and a maximum of 7. Prompt-based: The median is 5, with a first quartile at 3.5 and a third quartile at 5, with a minimum of 1 and a maximum of 6. QA-based: The median is 3.5, with a first quartile at 2.75 and a third quartile at 4, with a minimum of 1 and a maximum of 6. Significant differences exist between No-AI and Prompt-based (p < 0.01), between Prompt-based and QA-based (p < 0.01), and between No-AI and QA-based (p < 0.01). H2: Perceived Quality of the Email by Evaluators: No-AI: The median is 5.22, with a first quartile at 4.43 and a third quartile at 5.42, with a minimum of 3.78 and a maximum of 5.83. Prompt-based: The median is 5.61, with a first quartile at 5.47 and a third quartile at 6.07, with a minimum of 4.56 and a maximum of 6.56. QA-based: The median is 5.81, with a first quartile at 5.43 and a third quartile at 5.99, with a minimum of 4.78 and a maximum of 6.50. Significant differences exist between No-AI and Prompt-based (p < 0.05) and between No-AI and QA-based (p < 0.01). H3-a: Perceived Impression of Participants by Evaluators: No-AI: The median is 4.17, with a first quartile at 3.31 and a third quartile at 4.63, with a minimum of 2.42 and a maximum of 5.67. Prompt-based: The median is 5.17, with a first quartile at 4.42 and a third quartile at 5.33, with a minimum of 3.92 and a maximum of 6.00. QA-based: The median is 4.92, with a first quartile at 4.60 and a third quartile at 5.42, with a minimum of 4.00 and a maximum of 5.58. Significant differences do not exist. H3-b: Psychological Distance: No-AI: The median overlap score is 5, with a first quartile at 2.75 and a third quartile at 6.25. The minimum score is 1, and the maximum score is 7. Prompt-based: The median overlap score is 4, with a first quartile at 2.75 and a third quartile at 4. The minimum score is 1, and the maximum score is 5. QA-based: The median overlap score is 1.75, with a first quartile at 3 and a third quartile at 3.5. The minimum score is 1, and the maximum score is 7. Significant differences exist between No-AI and QA-based (p < 0.05). Each box plot represents the spread of participant responses, with the whiskers showing the variability outside the upper and lower quartiles. The statistical differences (p-values) highlight significant findings between different experimental conditions.}
\end{figure*}
% \begin{figure*}[t]
\centering
\includegraphics[width=\textwidth]{figure/study1_IOS.pdf}
\caption{Inclusion of Other in the Self (IOS). The diagram above the x-axis is an example of what participants were shown when responding to the questionnaire. The degree of overlap between the two circles represents the psychological distance between oneself and others.}
\label{fig_study1_IOS}
\Description{This figure represents the "Inclusion of Other in the Self (IOS)" scale, which is used to measure psychological closeness or relational intimacy. The diagram depicts two circles, labeled "Self" and "Other," with varying degrees of overlap. Participants were asked to choose the level of overlap that best represented their relationship with another person. On the far left (score 1), the circles are completely separate, indicating a significant psychological distance between the self and the other. In the middle (score 4), the circles partially overlap, suggesting a moderate level of psychological closeness. On the far right (score 7), the circles almost completely overlap, representing a very close and intimate relationship between the self and the other.}
\end{figure*}
\section{Results of Study 1}
\red{Here, we first present the quantitative results of Study 1 for each research question. 
Subsequently, we include comments provided by the participants.}
\subsection{Participants' Email-Replying Process (RQ1)}
\label{sec:result1_RQ1}
\subsubsection{Efficiency of Replying to Emails (H1-a)}
\label{sec:result1_efficiency}
First, we compared the efficiency of replying to emails across three conditions.
After checking the data normality assumption with the Shapiro-Wilk test, the result of one-way repeated measures ANOVA showed that there was a significant difference in participants' efficiency of replying to emails across three conditions ($F[2, 22]=14.8$, $p<0.001$\red{, $\eta_p^2=0.57$}). 
Post-hoc analysis with Holm correction revealed that participants' efficiency of replying to emails in the QA-based condition was significantly higher compared to both the No-AI $(t(11), p=0.002\red{, d=1.38})$ and the Prompt-based $(t(11), p=0.046\red{, d=0.65})$ conditions.
Thus, H1-a was supported.
The QA-based approach enhanced participants’ email replying efficiency.
% 次にPrompt Character Countsを計算した
% QA-based < Prompt-based

\subsubsection{Prompt Character Counts (H1-a)}
\label{sec:result1_prompt_character_counts}
In order to understand how participants wrote prompts differently, we calculated the prompt character counts.
After the Shapiro-Wilk test, the paired t-test revealed that participants in the QA-based condition typed significantly fewer characters in their prompts than those in the Prompt-based condition $(t(11), p=0.010\red{, d=0.90})$.

\subsubsection{Cognitive Load for Replying to Emails (H1-b)}
\label{sec:result1_cognitive_load}
% Raw-TLXの計算結果を図1に示す
The results of the Raw-TLX are shown in Fig.~\ref{fig_study1_efficiency_and_cognitiveLoad}.
According to the one-way repeated measures ANOVA with Greenhouse-Geisser correction, there was a significant difference in participants' cognitive load for replying to emails among the three conditions $(F[1.1, 12.1]=12.6, p=0.003\red{, \eta_p^2=0.53})$. 
Post-hoc analysis with Holm correction revealed that participants' cognitive load for replying to emails in the QA-based condition was significantly lower compared to both the No-AI $(t(11), p=0.008\red{, d=1.12})$ and Prompt-based $(t(11), p=0.018\red{, d=0.81})$ conditions.
% これらの結果は、H1bの妥当性について示唆している。
Therefore, H1-b was supported.
The QA-based approach reduced participants’ cognitive workload while replying to the emails.

\subsubsection{\red{Difficulty in Understanding Email Content (H1-b)}}
\label{sec:result1_difficulty_in_understanding}
% また、H1-bはアンケート調査の結果によっても裏付けられた
Additionally, H1-b was also supported by the questionnaire survey results (Fig.~\ref{fig_study1_questionnaire} H1-b).
% easily understand content of email: QA-based > Prompt-based, No-AI
The Friedman test revealed a significant difference among the three conditions in terms of understanding the sender's intent and requests $(\chi^2(2)=10.6, p=0.005\red{, W=0.44})$. 
Post-hoc analysis using the Durbin-Conover test with Holm correction showed that participants in the QA-based condition found it significantly easier to understand the sender's intent and requests compared to those in both No-AI $(p=0.003\red{, r=0.61})$ and Prompt-based $(p=0.005\red{, r=0.73})$ conditions.
% system helped decide response strategy: QA-based > Prompt-based
% Furthermore, the Wilcoxon signed-rank test indicated that participants felt significantly more supported in determining the direction of their responses in the QA-based condition than in the Prompt-based condition $(p=0.010)$.

\subsubsection{Satisfaction with Completing Task (H1-c)}
\label{sec:result1_satisfaction}
The results of the satisfaction with completing participants' tasks are shown in Fig.~\ref{fig_study1_questionnaire}, H1-c.
\red{The two items measuring satisfaction showed high internal consistency, with a Cronbach's Alpha of $0.889$.}
After checking the data normality assumption with the Shapiro-Wilk test, the result of one-way repeated measures ANOVA showed that there was a significant difference in participants' satisfaction with completing tasks across three conditions $(F[2, 22], p<0.001\red{, \eta_p^2=0.79})$. 
Post-hoc analysis with Holm correction revealed that participants' satisfaction with completing tasks in the QA-based condition was significantly higher compared to both the No-AI ($t(11)$, $p<0.001$\red{, $d=2.39$}) and the Prompt-based ($t(11)$, $p=0.029$\red{, $d=0.72$}) conditions.
% Additionally, the result of the Friedman test showed that there was a significant difference in participants' satisfaction with the quality of their email responses across three conditions $(\chi^2(2)=17.6, p<0.001)$. 
% Post-hoc analysis using the Durbin-Conover test with Holm correction showed that participants' satisfaction with the quality of their email responses in the QA-based condition was significantly higher compared to the No-AI $(p<0.001)$ condition.
Therefore, H1-c was supported.
The QA-based approach improved participants’ satisfaction with completing their tasks while replying to the emails.

% \subsubsection{\red{Future Preference (H1-c)}}
% \red{
% The questionnaire survey results about participants' future preferences are shown in Fig.~\ref{fig_study1_questionnaire}, in H1-c.
% According to the Friedman test, a significant difference in participants' future preferences was observed among the three conditions $(\chi^2(2)=8.8, p=0.012, W=0.37)$.
% Post-hoc analysis using the Durbin-Conover test with Holm correction revealed that participants would prefer responding in the QA-based condition compared to the No-AI condition $(p=0.012, r=0.67)$.
% However, no significant difference was found between the Prompt-based and QA-based conditions $(p=0.800, r=0.27)$.
% }

\subsubsection{Difficulty in Initiating the Action for Replying to Emails (H1-d)}
\label{sec:result1_initiating}
% felt high barrier: QA-based < Prompt-based, No-AI
The questionnaire survey results about participants' difficulty in initiating the action for replying to emails are shown in Fig.~\ref{fig_study1_questionnaire}, in H1-d.
According to the Friedman test, a significant difference in participants' difficulty in initiating the action for replying to emails was observed among the three conditions $(\chi^2(2)=19.8, p<0.001\red{, W=0.83})$.
Post-hoc analysis using the Durbin-Conover test with Holm correction revealed that participants in the QA-based condition perceived significantly higher barriers to initiating email response tasks than those in the No-AI $(p<0.001\red{, r=0.85})$ and Prompt-based $(p<0.001\red{, r=0.68})$ conditions.
% This result suggests the validity of H1-c, which ResQ lowers the barriers to initiating email response tasks.
Therefore, H1-d was supported.
The QA-based approach reduced participants’ difficulty in initiating the action to reply to emails.

\subsubsection{Sense of Agency and Control (H1-e)}
\label{sec:result1_agency}
% agency, control: QA-based < Prompt-based, No-AI
The questionnaire survey results about a sense of agency and control are shown in Fig.~\ref{fig_study1_questionnaire}, H1-e.
The Friedman test revealed a significant difference among the three conditions for both the sense of agency $(\chi^2(2)=22.8, p<0.001\red{, W=0.95})$ and the sense of control $(\chi^2(2)=21.3, p<0.001$, $\red{W=0.89})$. 
Post-hoc analysis using the Durbin-Conover test with Holm correction showed that participants in the QA-based condition found that it significantly reduced their sense of agency compared to both the No-AI $(p<0.001\red{, r=0.88})$ and the Prompt-based $(p<0.001\red{, r=0.77})$ conditions.
Additionally, post-hoc analysis using the Durbin-Conover test with Holm correction showed that participants in the QA-based condition experienced a significantly reduction in their sense of control compared to both the No-AI $(p<0.001\red{, r=0.88})$ and the Prompt-based $(p=0.006\red{, r=0.56})$ conditions.
Thus, H1-e was supported.
The QA-based approach reduced participants’ sense of agency and sense of control while replying to the emails.

\subsection{Quality of the Email Responses (RQ2)}
\label{sec:result1_RQ2}
\subsubsection{Perceived Quality of the Email by Evaluators (H2)}
\label{sec:result1_quality}
In Fig.~\ref{fig_study1_questionnaire}, H2 shows the results regarding the quality of the emails.
\red{The Cronbach's Alpha of three items measuring the perceived quality of the email is $0.846$.}
% \red{The three items measuring the perceived quality of the email demonstrated high internal consistency, with a Cronbach's Alpha of $0.846$.}
After checking the data normality assumption with the Shapiro-Wilk test, the result of one-way repeated measures ANOVA showed that there was a significant difference in the perceived quality of the email across three conditions $(F[2, 22]=9.1, p=0.001\red{, \eta_p^2=0.45})$. 
Post-hoc analysis with Holm correction revealed that the perceived quality of the emails participants wrote in the QA-based condition was significantly higher compared to the No-AI $(t(11), p=0.005\red{, d=1.21})$ condition.
Thus, H2 was partially supported.
The QA-based approach improved the quality of the email responses compared to the No-AI condition.

% 各評価項目の表を載せて、それを根拠にする
\begin{table*}[t]
\caption{\red{Details of perceived quality of the emails. ($Mean\pm SD$)}}
\Description{The table presents the comparative evaluation of three methods, No-AI, Prompt-based, and QA-based, in terms of three key attributes: Politeness, Readability, and Meeting Demands. The results are displayed as mean scores with standard deviations. For Politeness, the No-AI method received a mean score of 4.39 ± 1.00, indicating lower politeness levels compared to the AI-based methods. The Prompt-based approach showed a significant improvement, scoring 5.65 ± 0.56, slightly outperforming the QA-based method, which scored 5.49 ± 0.51. In terms of Readability, the No-AI method achieved a score of 5.24 ± 0.79, again falling behind the AI-based methods. The Prompt-based approach scored 5.65 ± 0.69, while the QA-based method scored the highest at 5.78 ± 0.49, reflecting the most consistently readable outputs among the three. Finally, for Meeting Demands, the No-AI method scored 5.19 ± 0.76, which is comparatively lower than the AI-enhanced methods. The Prompt-based approach performed better with a score of 5.68 ± 0.60, but the QA-based method emerged as the best performer in this category, scoring 5.88 ± 0.60.}
\label{tab_study1_quality_of_emails}
\red{
\begin{tabular}{cccc}
\hline
             & Politeness                 & Readability                & Meeting Demands               \\ \hline
No-AI        & $4.39\pm1.00$ & $5.24\pm0.79$ & $5.19\pm0.76$ \\
Prompt-based & $5.65\pm0.56$ & $5.65\pm0.69$ & $5.68\pm0.60$ \\
QA-based     & $5.49\pm0.51$ & $5.78\pm0.49$ & $5.88\pm0.60$ \\ \hline
\end{tabular}
}
\end{table*}
\red{Tab.~\ref{tab_study1_quality_of_emails} shows the detailed results regarding the perceived quality of the emails across three evaluation dimensions (politeness, readability, and meeting demands).
These results further supported the partial acceptance of H2, showing that the AI-assisted approach tended to improve the email quality.} 
% of email responses across all dimensions.}

%%%%%%%%%%%%%%%%%%%%%%%%%%%%%%%%%%%%
% According to the Friedman test, a significant difference was observed among the three conditions $(\chi^2(2)=17.6, p<0.001)$.
% As a post hoc test, the Durbin-Conover test with Holm correction was conducted.
% The results indicated that, compared to the No-AI condition, both the AI-assisted QA-based $(p=0.007)$ and Prompt-based $(p=0.012)$ conditions produced significantly higher-quality responses.
%%%%%%%%%%%%%%%%%%%%%%%%%%%%%%%%%%%%

\subsection{Relationship between Participants and Their Counterpart (RQ3)}
\label{sec:result1_RQ3}
\subsubsection{Perceived Impression of Participants by Evaluators (H3-a)}
\label{sec:result1_impression}
The results of the perceived impression of the participants rated by another group of evaluators are shown in Fig.~\ref{fig_study1_questionnaire} H3-a.
\red{The two items assessing participants' impression as email senders showed high internal consistency, with a Cronbach's Alpha of $0.946$.}
After checking the data normality assumption with the Shapiro-Wilk test, the result of one-way repeated measures ANOVA showed that there was a significant difference in impression of the participants as an email sender across three conditions $(F[2, 22]=5.9, p=0.009\red{, \eta_p^2=0.35})$. 
Post-hoc analysis with Holm correction revealed that participants' impression in the QA-based condition was not significantly higher compared to both the No-AI $(t(11), p=0.058\red{, d=0.79})$ and the Prompt-based $(t(11), p=0.939\red{, d=0.02})$ conditions.
% no-AIとQA-basedには差がありそうということを書くべきか?
Thus, H3-a was not supported.
The QA-based approach didn't improve the impression of participants as email senders.
% Although there was no statistically significant difference in participants' impressions between the QA-based and No-AI conditions $(p=0.058)$, the effect size was large $(d=0.79)$. 
% This suggests that while the difference did not reach statistical significance, there is still a meaningful difference in the impression of participants as email senders between these conditions.

\subsubsection{Psychological Distance between Participants and Their Counterpart (H3-b)}
\label{sec:result1_psychological_distance}
The IOS result is shown in \red{Fig.~\ref{fig_study1_questionnaire}}.
The Friedman test showed a significant difference in psychological distance among the three conditions $(\chi^2(2)=7.47, p=0.024\red{, W=0.31})$.
Post-hoc analysis using the Durbin-Conover test with Holm correction revealed that IOS in the No-AI condition was significantly higher than in the QA-based condition $(p=0.021\red{, r=0.51})$. 
% この結果はH3-bを部分的に支持するが、Prompt-based条件とQA-based条件の間には有意差がないことから、完全には支持されなかった
This result partially supports H3-b; however, because there was no significant difference between the Prompt-based and QA-based conditions \red{$(p=0.053, r=0.30)$}, H3-b was partially supported.


\subsection{\red{Qualitative Feedback}}
\label{sec:result1_interview}
\red{This section synthesizes qualitative feedback to provide further insights into participants' experiences across three conditions.
The interview comments were translated from Japanese into English.}
\subsubsection{\red{Participants' Email-Replying Process (RQ1)}}
\label{sec:result1_interview_RQ1}
\red{Feedback from participants confirmed that the QA-based condition functioned as expected, contributing to improvements in efficiency, a reduction in workload, and a lowering of barriers to task initiation compared to the other conditions.
\begin{enumerate}[]
    \item \textit{``In the QA-based condition, AI summarized key points through questions and highlighted relevant sections of the email body, which facilitated my understanding of the email and reduced my overall burden''} [P10].
    \item \textit{``In the QA-based condition, I could easily obtain the desired output even without the technical skills to create prompts''} [P6].
    \item \textit{``By saving the time needed to read the counterpart's text, the psychological barrier to starting the task was lowered''} [P5].
\end{enumerate}}

\red{On the other hand, we found that the QA-based condition led to a reduced sense of agency and control compared to the other conditions.
\begin{enumerate}[]
    \item \textit{``Since the AI prompted me with questions at the beginning, the mental effort required to start thinking about the task was eliminated, reducing the stress associated with initiating the work''} [P10].
    \item \textit{``By saving the time needed to read the counterpart's text, the psychological barrier to starting the task was lowered''} [P5].
\end{enumerate}}

\red{This aspect was also found to have the potential to negatively impact users' willingness to adopt the system in the future.
Participants noted that they preferred the QA-based condition \textit{``when time is limited or speed is important''} [P4] or \textit{when the email is of low importance''} [P5], but in other situations, they favored writing responses themselves.}

\subsubsection{\red{Quality of the Email Responses (RQ2)}}
\label{sec:result1_interview_RQ2}
% ほとんどの参加者は、AIを使うと、構造・丁寧さ・言葉遣いが改善され、全体的に良い文章を書けたと述べた
% また参加者は、「Prompt-based条件だと、相手の要求を見落としていたかもしれないが、QA-based条件では自信を持って返信を作成することができた」 [P2]と述べた
% さらにある参加者は、「QA-based条件では、回答してもしなくても良いこと「XXの件、承知しました、など。」にも丁寧に返答を書いてくれた」 [P9]と述べ、QA-based条件によってメールの丁寧さが向上したことを強調した
\red{All participants stated that using AI improved their writing structure, politeness, and choice of words, ultimately enabling them to produce better overall responses. 
Furthermore, participants remarked, \textit{``Under the prompt-based condition, I might have overlooked the recipient's requests, but under the QA-based condition, I was able to craft responses with confidence''} [P2]. 
Additionally, one participant emphasized that \textit{``Under the QA-based condition, the AI even provided polite responses to matters where a reply was optional, such as acknowledging something with phrases like 'I Understood regarding XX, etc.'''} [P9], highlighting how the QA-based condition scaffolded user to construct a polite email in a formal setting.}
% enhanced the politeness of email communication.}

\subsubsection{Relationship between Participants and Their Counterpart (RQ3)}
\label{sec:result1_interview_RQ3}
% 参加者は、``相手との間に知覚する心理的距離は労力に比例した''と報告し、PXXは``特にQA-based条件では選択肢を選ぶだけだった相手のことを考えることが少なかった''と報告した。
% 一方でPXXは、``自分で返信を考えるより、AIを使うと相手に良い印象を与えられるメッセージを作ることができたので、関係性を近く感じた''と報告した
\red{Participants shared differing views on how AI's involvement affected their psychological distance from their counterparts.
P2, P9, and P11 reported that the psychological distance they felt from the other person was directly related to the amount of effort they put in.
Furthermore, P6 noted that \textit{``especially under QA-based condition, I barely thought about the counterpart because I only selected options to create responses.''}
In contrast, P8 reported that \textit{``compared to composing replies myself, using AI allowed me to create messages that left a better impression on my counterpart, which made the relationship feel closer.''}
These results suggested that, on the one hand, AI's mediation can potentially increase the psychological distance between senders and receivers. 
On the other hand, it can also diminish the perceived distance from the sender due to effective impression management. Thus, we conducted a field study to further clarify the impact of AI on interpersonal relationships.}
% These results suggested that while a reduction in cognitive load may decrease participants' psychological engagement with their counterparts, the perceived improvement in communication quality could, conversely, foster a greater sense of closeness in the relationship.}

%%%%%%%%%%%%%%%%%%%%%%%%%%%%%%%%%%%%%%%%%%%%%%%%%%%%%%%%%%%%%%%%%%%%%%%%%%%%%%%%%%%%%%%%%%%%%%%%%%%%%%%%%%%%%%%%%%%%
% \subsection{Qualitative Feedback}
% % We added a new subsection, "User Comments," to present follow-up interview results, including participant feedback on useful questions (Sec.6.4). 
% \subsubsection{Participants' Email-Replying Process (RQ1)}
% \paragraph{Enhanced Efficiency, Reduced Cognitive Load, and Lowered Barriers to Initiating Email Replies (H1-a, H1-b, H1-d)}
% % 参加者は、QA-based conditionは期待通り機能し、参加者の効率向上や負担低減に貢献したことが確認できました。
% % 質問で要点をまとめてくれ、メール本文の対応箇所がハイライトされてたので、メールの理解の効率が上がり、負担が減りました [P10]
% % QA-based条件では、Prompt作成の技術がなくても、期待する出力を簡単に得ることができました。[P6]
% % Prompt-based条件では、結局自分で相手のメールを全て読み、回答すべきことを整理する必要がありました。[P4]
% % Prompt-based条件では、自分でAIに対する指示を一から考える必要があり、手動の条件と効率や負担に差を感じませんでした。一方でQA-based条件は、圧倒的に早く返信を作成することができました。[P5]
% \red{Participants' comments confirmed that the QA-based improved efficiency and reduced workload when replying to emails.
% P10 explained, \textit{``In the QA-based condition, AI summarized key points through questions and highlighted relevant sections of the email body, which facilitated my understanding of the email and reduced my overall burden''}.
% P6 shared, \textit{``In the QA-based condition, I could easily obtain the desired output even without the technical skills to create prompts''}.
% In contrast, the Prompt-based condition required extra effort. 
% P4 noted, \textit{``In the Prompt-based condition, I had to read the counterpart's email completely and decide what to respond to''}. 
% P5 elaborated, \textit{``In the Prompt-based condition, I had to think of instructions for the AI from scratch, making it feel no different from the No-AI condition in terms of efficiency and workload. On the other hand, the QA-based condition allowed me to compose responses faster''}.}

% \paragraph{Reduced Difficulty in Initiating the Action for Replying to Emails (H1-d)}
% % 参加者のコメントから、QA-based conditionでは、全体的な労力が下がるとともに、作業開始時のAIによるの質問が、タスク開始の障壁の低下に役立つことがわかりました。
% % 最初にAIが質問を投げかけてくれるので、作業開始時の思考の労力がなくなり、作業に取り掛かる際のストレスが減りました[P10]
% % 相手の文章を読む時間が省けたことで、作業開始の心理的障壁が下がりました。[P5]
% \red{Comments from participants indicated that the QA-based condition helped lower the barriers to task initiation through AI-generated questions at the start of the process. 
% P10 explained, \textit{``Since the AI prompted me with questions at the beginning, the mental effort required to start thinking about the task was eliminated, reducing the stress associated with initiating the work''}. 
% P5 noted, \textit{``By saving the time needed to read the counterpart's text, the psychological barrier to starting the task was lowered''}.}

% \paragraph{Decreased Sense of Agency and Control (H1-e)}
% % QA-based conditionでは、agencyやcontrolの感覚が他の条件に比べて低下したと回答した参加者は、次のようにコメントした。
% % 「agencyとcontrolの感覚は、プロンプトを自分で打った量に比例しました。」[P7, 8, 9, 10]
% % 「QA-based条件では、要点も絞ってくれたので、AIに任せようという思いが強くなりました。」[P4]
% % 一方で感覚が変化しなかったと回答した参加者は「AIに任せても、自分で確認と修正を行ったので、agencyやcontrolの感覚に変化はありませんでした」[P3]と説明した。
% \red{Participants reported a decrease in their sense of agency and control in the QA-based condition.
% \textit{``The sense of agency and control was proportional to the amount of text I typed myself''} [P7, P8, P9, P10]. 
% \textit{``Under the QA-based condition, since the AI helped narrow down the key points, I felt a stronger inclination to leave the task to the AI''} [P4].
% On the other hand, one participant who reported no change in their sense of agency or control explained, \textit{``Even though I relied on the AI, I reviewed and edited the output myself, so there was no change in my sense of agency or control''} [P3].}

% \paragraph{Future Preference (H1-c)}
% % 多くの参加者はメール返信の効率が上がる、負担が減る、質が高いメールを執筆できるという理由から、QA-based conditionで返信をしたいと回答しました。
% % 自分で返信を考えるより、AIを使った方が、質の高い返信を早く作ることができました。特にQA-based conditionではその効果が大きかったので、将来はQA-based conditionで返信したいと思いました。[P6]
% % 一方でsense of agencyの低下や、AIへの依存の危惧を理由に、QA-based conditionでのメール返信を忌避する参加者もいました。
% % 時間がない時や、スピードを重視したい時[P4]、重要度が低い時[P5]は、QA-based conditionで返信をしたいと思ったが、そうでない場合は自分で執筆したいと思いました。
% % 「Prompt-based条件では、QA-based条件より思考する必要が多く、それが楽しかったです」 [P7]
% % 返信作業が楽にはなりましたが、メールを細部まで読まなくなり、内容が頭に入っていない感じがしたので、将来使いたいとは思いませんでした。[P11]
% \red{Participants expressed a preference for using the QA-based condition for email responses, citing increased efficiency, reduced workload, and the ability to produce high-quality emails as the primary reasons. 
% One participant explained, \textit{``Using AI allowed me to compose high-quality responses faster than if I had written them myself. The effect was particularly significant in the QA-based condition, which is why I would prefer to use it in the future''} [P6].}
% \red{However, some participants were hesitant to adopt the QA-based condition due to concerns about a reduced sense of agency or over-reliance on AI. 
% Participants noted that they preferred the QA-based condition \textit{``when time is limited or speed is important''} [P4] or \textit{when the email is of low importance''} [P5], but in other situations, they favored writing responses themselves.
% One participant reflected, \textit{``In the prompt-based condition, I found that I needed to think more actively compared to the QA-based condition, and I enjoyed that process''} [P7].
% Another observed, \textit{``While the QA-based condition made responding easier, I felt that I was no longer fully reading and absorbing the content of emails, which made me hesitant to use it in the future''} [P11].}

% \paragraph{Quality of AI-generated Questions and Options}
% % 参加者はQA-based conditionにおいて生成された質問や選択肢について、有益であったものとそうでなかったものがあったとコメントした。
% % 参加者は有益でない質問の例として、メール送信者の意図を汲み取れていないもの [P2, P11]、自分と相手の立場を勘違いしているもの [P4, P8]を挙げた。
% % またある参加者は、「自分が言いたいことに関連する質問がない場合、質問機能自体が役に立たなかった」[P8]と回答した。
% % また参加者は、質問の数についても意見を述べた
% % 参加者は「用件ごとに質問を生成してくれたのが、メールを理解する上で役に立った」[P4, P5, P8, P9, P10, P11]と回答した一方で、「質問が多すぎると煩雑に感じることもあった。またそれに全て答えると、返信が冗長になってしまった。」[P7, P12]と回答する参加者もいた。
% % また、参加者は選択肢についても意見を述べた
% % 参加者は、「(日程調整のシチュエーションにおいて)選択肢の中に自分が選びたい日程がなかったので、自分で日程を入力する必要があった」[P2]と説明し「より多くの選択肢を生成して欲しかった」と説明した[P8]。
% % 一方である参加者は、「必要以上に多くの選択肢があったときに、煩わしさを感じた」[P4]と説明した。
% \red{Participants commented on the questions and options generated by the QA-based condition, noting that some were useful while others were not. 
% Examples of less useful questions included those that failed to accurately capture the sender's intent [P2, P11] or misinterpreted the relationship between the sender and recipient [P4, P8]. 
% Additionally, one participant remarked, \textit{``when there were no questions related to what I wanted to say, the question feature itself was not helpful''} [P8].}

% \red{Participants also shared mixed opinions on the number of questions generated. 
% Some participants noted that \textit{``generating questions for each topic helped understand the email''} [P4, P5, P8, P9, P10, P11].
% However, others felt \textit{``an excessive number of questions felt overwhelming, and responding to all of them made the reply unnecessarily lengthy''} [P7, P12].}

% \red{Feedback on the generated options was similarly divided. 
% For instance, in scheduling scenarios, one participant shared that \textit{``none of the suggested dates matched what I wanted, so I had to input the date myself''} [P2] and another participant \textit{``wished for more options or a more flexible input type''} [P8].
% In contrast, another participant stated that \textit{``having more options than necessary felt burdensome''} [P4].}

% \subsubsection{Quality of the Email Responses (RQ2)}
% % ほとんどの参加者は、AIを使うと、構造・丁寧さ・言葉遣いが改善され、全体的に良い文章を書けたと述べた
% % また参加者は、「Prompt-based条件だと、相手の要求を見落としていたかもしれないが、QA-based条件では自信を持って返信を作成することができた」 [P2, 4]と述べた
% % さらにある参加者は、「QA-based条件では、回答してもしなくても良いこと「XXの件、承知しました、など。」にも丁寧に返答を書いてくれた」 [P9]と述べ、QA-based条件によってメールの丁寧さが向上したことを強調した
% \paragraph{Scaffolding a structured response}
% \red{Most participants stated that using AI improved their writing structure, politeness, and choice of words, ultimately enabling them to produce better overall responses. 
% Furthermore, participants remarked, \textit{``Under the prompt-based condition, I might have overlooked the recipient's requests, but under the QA-based condition, I was able to craft responses with confidence''} [P2, P4]. 
% Additionally, one participant emphasized that \textit{``Under the QA-based condition, the AI even provided polite responses to matters where a reply was optional, such as acknowledging something with phrases like 'I Understood regarding XX, etc.'''} [P9], highlighting how the QA-based condition enhanced the politeness of email communication.}

% \subsubsection{Relationship between Participants and Their Counterpart (RQ3)}
% % 参加者は、``相手との間に知覚する心理的距離は労力に比例した''と報告し、PXXは``特にQA-based条件では選択肢を選ぶだけだった相手のことを考えることが少なかった''と報告した。
% % 一方でPXXは、``自分で返信を考えるより、AIを使うと相手に良い印象を与えられるメッセージを作ることができたので、関係性を近く感じた''と報告した
% \red{Participants shared differing views on how AI's involvement affected their psychological distance from their counterparts.
% Several participants reported that the psychological distance they felt from the other person was directly related to the amount of effort they put in [P2, P9, P11].
% % \textit{``the psychological distance I perceived from the counterpart was proportional to the effort exerted''} [P2, P9, P11].
% Furthermore, P6 noted that \textit{``especially under QA-based condition, I barely thought about the counterpart because I only selected options to create responses''}.
% In contrast, P8 reported that \textit{``compared to composing replies myself, using AI allowed me to create messages that left a better impression on my counterpart, which made the relationship feel closer''}.}
\section{Method of Study 2}
% この実験の目的などを書く
% Study1は、ResQのdesignは、フォーマルな場面における返信作業において、作業者の作業効率を上げ、認知負荷を下げるのに有効であることを明らかにした
Study 1 has revealed that the ResQ design effectively enhances user efficiency and reduces cognitive load during response tasks in formal settings.
\blue{To further examine how our QA-based system influenced users' actual email replying practice, we conducted a field experiment for Study 2.} 
% to gain a qualitative understanding of how our QA-based method influenced the practice of email replies.}
% Next, we set out to examine how people used the QA-based approach in their actual email replying practice.
\subsection{Field study}
% Study 2 was designed to qualitatively assess how ResQ influenced the practice of email replies.

% 本実験では、chrome拡張機能の形でのプロトタイプを開発し、システムを用いてPC版のGmailで返信作業をしてもらうように依頼した
% 拡張機能では、返信作業の開始は、参加者がGmailの返信ボックス上の"Reply with AI"ボタンを押すことによって検出した
% 参加者の返信作業の開始が確認されると、メール内容がリモートサーバに送信されると同時に、新しい返信エディタが開き、数秒間の後に質問と選択肢が生成された
% 参加者は返信文章を執筆し、reply boxの下にあるReplyボタンを押すと、Gmail上のReply boxに執筆した文章がそのまま反映されるようにした
% なお、メール内容や参加者が執筆した内容は、プライバシー保護のため、実験者からはアクセスできず、サーバにも保存されないようにした
In this five-day field experiment, we developed a prototype as a Chrome extension and asked participants to use the system to reply to emails using Gmail on a PC.
The extension detected the initiation of the reply task when participants clicked the ``Reply with AI'' button in the Gmail reply box. 
Upon confirming the task, the email content was sent to a remote server, a new reply editor opened, and a question with options appeared after a few seconds. 
Participants composed their replies, and upon pressing the Reply button, their text was directly reflected in the Gmail reply box. 
To ensure privacy, neither the email content nor the participants' responses were accessible to the experimenters or stored on the server.
\red{The specific implementation details and user interface are provided in the appendix.}

% \red{In this five-day field experiment, we developed a prototype system consisting of a Chrome extension and a backend service to enable participants to reply to emails using Gmail on a PC. 
% The Chrome extension detected the initiation of the reply task when participants clicked the ``Reply with AI'' button in the Gmail reply box (see Fig.~\ref{fig_UI}).
% Upon clicking the button, the extension extracted the email content directly from Gmail’s DOM structure using JavaScript and sent it to a backend API endpoint implemented with FastAPI~\footnote{\url{https://fastapi.tiangolo.com}}.
% The backend, hosted on an AWS EC2 instance~\footnote{\url{https://aws.amazon.com/ec2/}}, received the email content and forwarded it to the OpenAI API~\footnote{\url{https://platform.openai.com/docs/}} to generate questions or reply suggestions. 
% These outputs were then returned to the Chrome extension and displayed to the participant in a new reply editor.
% Finally, participants revise the reply suggestions and submit them back to the Gmail reply box by clicking the ``Reply'' button.
% To ensure privacy, neither the email content nor the participants' responses were accessible to the experimenters or stored on the server.}
% \begin{figure*}[ht]
\centering
\includegraphics[width=\textwidth]{figure/UI.pdf}
\caption{UI of the Gmail Reply Box with the ``Reply with AI’’ Feature, used in Study 2. Pressing the ``Reply with AI’’ button opens the window shown in Fig.~\ref{fig_interface}}
\label{fig_UI}
\Description{This figure illustrates the user interface of the Gmail reply box as enhanced by the prototype system. The Reply with AI button, shown in blue on the right-hand side of the toolbar, allows users to activate the AI-assisted reply generation feature. When the button is clicked, the system extracts the email content and opens the window shown in Fig.~\ref{fig_interface}. The standard Gmail toolbar options, such as send, formatting, and attachment icons, remain.}
\end{figure*}

\subsection{Participants}
\begin{table*}[t]
\caption{\red{Backgrounds of participants in Study 2, including age, gender, job roles, frequency of AI tool usage, and use of AI for email purposes.}}
\Description{The table outlines the demographic information and AI usage patterns of nine participants in Study 2, including their age, gender, job roles, general AI tool usage, and the extent to which they utilize AI for email-related tasks. The participants include university students and office workers, with a mix of both male and female representatives, and their AI adoption varies from frequent to rare use. Participant P1 is a 28-year-old male office worker who uses AI tools daily, with 20–50\% of his email tasks supported by AI. P2, a 24-year-old female university student, frequently uses AI tools but does not employ them for email-related purposes. Similarly, P3, a 20-year-old female university student, frequently uses AI tools, with 20–50\% of her email tasks facilitated by AI. Participant P4, a 24-year-old female university student, engages in daily AI tool usage, relying on AI for 50–80\% of her email activities. P5, a 31-year-old male office worker, also uses AI tools daily, supporting 20–50\% of his email tasks. Likewise, P6, a 39-year-old female office worker, uses AI tools daily, with AI assisting in 20–50\% of her email-related tasks. In contrast, Participant P7, a 25-year-old male university student, rarely uses AI tools and does not employ them for email-related purposes. P8, a 23-year-old female office worker, frequently uses AI tools but applies them to less than 20\% of her email tasks. Lastly, P9, a 38-year-old male office worker, rarely engages with AI tools, with AI playing a role in less than 20\% of his email-related tasks.}
\label{tab_study2_participants_background}
\red{
\begin{tabular}{cccccc}
\hline
Participants & Age & Gender & Job       & AI Tool Usage&AI for Email Usage\\ \hline
P1           & 28  & M& Office Worker& Daily&20-50\%\\
P2           & 24  & F& Univ. Student& Frequently&Never\\
P3           & 20  & F& Univ. Student& Frequently&20-50\%\\
P4           & 24  & F& Univ. Student& Daily&50-80\%\\
P5           & 31  & M& Office Worker& Daily&20-50\%\\
P6           & 39  & F& Office Worker& Daily&20-50\%\\
P7           & 25  & M& Univ. Student& Rarely&Never\\
 P8           & 23  & F& Office Worker& Frequently&<20\%\\
P9& 38& M& Office Worker& Rarely&<20\%\\ \hline
\end{tabular}
}
\end{table*}
As shown in Tab.~\ref{tab_study2_participants_background}, nine participants (four males and five females, aged 20-39) were recruited via a local Japanese participant recruiting platform.
The average age of the participants was 28.0 (SD = 6.7)\blue{, and they reported engaging in more than three email communications per day on average.}
This study was approved by the ethical review board of the authors' institute.
The participants were paid approximately \$37 USD for participation.
% 参加者は、実験やシステムの説明を30分間受けた後、5日間システムを使用し、その後1時間のインタビューを受けた
The participants received a 30-minute explanation of the experiment and system, used the system for five days, and subsequently participated in a one-hour interview.

\subsection{Procedure}
Participants first read the study instructions and their right to participate, after which they consented to participate in the experiment. 
Next, they were provided with an explanation of the study's purpose and instructions on how to use the QA-based system. 
Following this, they installed the Chrome extension we developed and confirmed its functionality according to the provided instructions.
Participants were asked to use the system for five days, during which they were free to use it to reply to emails at any time. 
After the five-day period, a one-hour semi-structured interview was conducted. 
During the interview, participants were asked a series of questions, such as: \textit{``Can you tell us your overall impression of using the system?''} \textit{``How did your email replying practice change before and after using the system?''} \textit{``What changes did you notice in the emails you composed?''} and \textit{``How did your relationship with the communication counterpart change after using this system?''}
This study was conducted remotely with all participants.

\subsection{Data Analysis}
% インタビュー記録をコード化し、分析するために、我々はインタビューデータを録音し、インタビューデータの文字起こしを行った上で、bottom-up approach rooted in grounded theoryを使用した
% 具体的には、暫定的なラベルを特定するためのopen codingと、ラベル間の関係を見出すためのaxial codingを行った
% coding結果、email-replying process, quality of the email responses, relationship between sender and recipient, future use intentinoの4つの主要なテーマを特定した
To analyze the interview data, we transcribed the interview recordings. 
We followed the thematic analysis method~\cite{braun_2006_thematicanalysis} to analyze the open-ended responses. 
One of the authors open-coded all relevant concepts that were related to our research questions, assigned labels that featured the concepts, and grouped labels into different themes. 
Next, the authors discussed the quotes and themes repeatedly.
Finally, the developed themes were compared and adjusted among all participants until they thoroughly covered the data.
As a result of the coding process, we identified four main themes: the email-replying process, the quality of the email responses, the relationship between the sender and recipient, and perceived risk.
\section{Results of Study 2}
\red{Tab.~\ref{tab_study2_participants_usage} presents the number of email replies composed using ResQ, along with the contexts in which it was used over five days.
We did not analyze usage frequency because participants reported avoiding using ResQ for emails for which they had privacy concerns.
% We did not analyze usage frequency because participants reported avoiding ResQ for certain types of emails, such as those deemed important or involving security concerns. 
Additionally, some participants refrained from using ResQ due to its availability only on PCs, as they frequently replied to emails via smartphone. 
Email frequency also varied among participants depending on their personal schedules (\textit{e.g.,} holidays).}

\red{Eight participants primarily used ResQ in formal workplace settings, while one (P9) used it only for informal exchanges. 
Because this was a field study, we could not limit participants to using ResQ only in formal contexts, though we instructed them to use it to reply to formal emails at the beginning. 
As a result, two participants (P3, P4) used ResQ to reply to both formal and informal emails, and P9 only used it for informal email exchanges. 
Hence, we excluded P9's data and only focused on analyzing the experience of P3 and P4 when they replied to formal emails using ResQ.}
% Since this study focuses on formal usage, we excluded P9's results and presented only interview data on formal scenarios.}
% 特にP1があまり使わなかった理由について言及したい

% The number of email replies composed by participants using ResQ over five days and the contexts of usage are shown in Tab.~\ref{tab_study2_participants_usage}. 
% 今回はResQの使用頻度に関する分析は行わなかった。理由は次のとおり。
% Participants reported that they did not use ResQ to compose replies for emails that required short responses, those deemed particularly important, or when they had security-related concerns.
% Additionally, some participants noted that they occasionally composed replies without using ResQ because the tool is currently only available on PCs, whereas they often replied to emails using their smartphones.
% また参加者の都合(休日など、やり取りが少ない/多い週)のため、メールの頻度はばらつきがある。
% Regarding the system's usage contexts, eight out of nine participants primarily utilized ResQ in formal workplace settings, whereas one participant (P9) used it exclusively for informal exchanges. 
% Additionally, two participants (P3, P4) noted that while most of their usage occurred in formal contexts, they occasionally employed the system for informal communication.
% Pariticipantsの使用状況はコントロールはできなかったが、今回論文ではformalにfocusしているため、P9の結果は除去した。また、formalな状況における使用に関するインタビュー結果のみを掲載した。

\red{This section explains the results of interviews conducted with participants after they used the system, with the interview comments translated from Japanese into English.}
\begin{table*}[t]
\caption{Usage of the participants in Study 2. D1 through D5 represents the number of emails replied to using the system each day, from Day 1 to Day 5.}
\Description{The table illustrates the daily system usage of participants in Study 2, detailing the number of emails replied to each day (from Day 1 to Day 5) and the primary purposes for which the system was used. The participants represent a range of tasks, including work-related scheduling, academic communication, and informal interactions. Participant P1 primarily used the system for scheduling, task confirmations, and submissions related to work. His email activity was moderate, replying to 3 emails on Day 1, 1 email on Day 2, and 1 email on Day 5, with no emails replied to on Days 3 and 4. P2 focused on task management and communication related to research and university administration. She maintained consistent usage on Days 1 and 2, replying to 6 emails each day, did not use the system on Day 3, replied to 1 email on Day 4, and increased her activity to 5 emails on Day 5. Participant P3 engaged with the system for task management related to research with professors and informal contact with friends. Her usage was highest on Days 1 and 2, replying to 6 and 7 emails respectively. She replied to 2 emails on both Days 3 and 4 and did not use the system on Day 5. P4 used the system for scheduling related to club activities, informal contact with friends, and inquiries with a museum abroad. She consistently replied to 6 emails on both Days 1 and 2, 5 emails on Day 3, 4 emails on Day 4, and 6 emails on Day 5. Participant P5’s primary usage involved meeting planning and confirmations related to work. He replied to 2 emails on Day 1, 3 emails on Day 2, 5 emails on Day 3, and 1 email on both Days 4 and 5. P6 utilized the system for scheduling and confirmations related to work, friends, and event organizers. Her activity peaked on Days 2 and 3, replying to 6 emails each day, followed by 3 emails on both Days 1 and 4 and 5 emails on Day 5. Participant P7 focused on scheduling and progress management related to research and business trips. His usage was irregular, with 4 emails replied to on Day 1, none on Day 2, 6 emails on Day 3, none on Day 4, and 2 emails on Day 5. Lastly, P8 primarily used the system for progress management and administrative confirmations related to work. Her activity started with 5 emails on Day 1, decreasing to 2 emails on Day 2, 1 email on Day 3, and 5 emails on Day 4, with no usage recorded on Day 5.}
\label{tab_study2_participants_usage}
% \resizebox{\textwidth}{!}{
\begin{tabular}{ccccccl}
\hline
\multirow{2}{*}{} & \multicolumn{5}{c}{Daily System Usage} & \multicolumn{1}{c}{\multirow{2}{*}{Main Usage}}                                                      \\ \cline{2-6}
                  & D1     & D2    & D3    & D4    & D5    & \multicolumn{1}{c}{}                                                                                 \\ \hline
P1                & 3      & 1     & 0     & 0     & 1     & Scheduling, task confirmations, and submissions related to work                                      \\
P2                & 6      & 6     & 0     & 1     & 5     & Task management and communication related to research and university administration                  \\
P3                & 6      & 7     & 2     & 2     & 0     & Task management related to research with professors, informal contact with friends                   \\
P4                & 6      & 6     & 5     & 4     & 6     & Scheduling related to club activities, informal contact with friends, inquiries with a museum abroad \\
P5                & 2      & 3     & 5     & 1     & 1     & Meeting planning and confirmations related to work                                                   \\
P6                & 3      & 6     & 6     & 3     & 5     & Scheduling and confirmations related to work, friends, and event organizers                          \\
P7                & 4      & 0     & 6     & 0     & 2     & Scheduling and progress management related to research and business trips                            \\
P8                & 5      & 2     & 1     & 5     & 0     & Progress management and administrative confirmations related to work                                 \\ \hline
\end{tabular}
% }
% \Description{The table provides background information about the participants in Study 2, including their system usage over five days (D1 to D5), which represents the number of emails they replied to using the system each day, as well as their main usage purpose. Participants range in age from 20 to 39 years, with both students and employees included, and a mix of male and female participants. Participant P1 is a 28-year-old male employee who primarily used the system for scheduling, task confirmations, and submissions related to work. His daily system usage across the five days was 3 emails on day one, 1 email on day two, and 1 email on day five, with no emails replied to on days three and four. Participant P2 is a 24-year-old female student who used the system for task management and communication related to research and university administration. Her daily system usage was consistent on the first two days, replying to 6 emails each day. She did not use the system on day three, replied to 1 email on day four, and increased her usage again with 5 emails on day five. Participant P3 is a 20-year-old female student who focused her system usage on task management related to research with professors and informal contact with friends. She replied to 6 emails on day one, 7 emails on day two, 2 emails on both day three and day four, and did not reply to any emails on day five. Participant P4 is a 24-year-old female student who used the system for scheduling related to club activities, informal contact with friends, and inquiries with a museum abroad. Her system usage was steady, replying to 6 emails on both day one and day two, 5 emails on day three, 4 emails on day four, and 6 emails on day five. Participant P5 is a 31-year-old male employee who primarily used the system for meeting planning and confirmations related to work. He replied to 2 emails on day one, 3 emails on day two, 5 emails on day three, and 1 email on both day four and day five. Participant P6 is a 39-year-old female employee who used the system for scheduling and confirmations related to work, friends, and event organizers. Her system usage was moderate, replying to 3 emails on day one, 6 emails on day two, 6 emails on day three, 3 emails on day four, and 5 emails on day five. Participant P7 is a 25-year-old male student who used the system for scheduling and progress management related to research and business trips. His system usage was irregular, replying to 4 emails on day one, none on day two, 6 emails on day three, none on day four, and 2 emails on day five. Participant P8 is a 23-year-old female employee who used the system for progress management and administrative confirmations related to work. Her daily system usage was 5 emails on day one, 2 emails on day two, 1 email on day three, 5 emails on day four, and none on day five. Participant P9 is a 38-year-old male employee who used the system for daily informal contact with friends. He had the highest and most consistent email activity, replying to 13 emails on day one, 11 emails on day two, 12 emails on both day three and day four, and 10 emails on day five.}
\end{table*}
% \begin{table*}[t]
% \caption{Usage of the participants in Study 2. D1 through D5 represents the number of emails replied to using the system each day, from Day 1 to Day 5.}
% \centering
% \label{tab_study2_participants_usage}
% \resizebox{\textwidth}{!}{
% % \setlength{\tabcolsep}{6pt} % default 6pt
% {\tabcolsep=2pt
% \begin{tabular}{ccccccl}
% \hline
% \multirow{2}{*}{} & \multicolumn{5}{c}{\centering Daily System Usage} & \multirow{2}{*}{\centering Main Usage} \\ \cline{5-9} 
%                   & D1 & D2 & D3 & D4 & D5 & \multicolumn{1}{l}{}                                \\ \hline
% P1                & 3     & 1     & 0     & 0     & 1     & Scheduling, task confirmations, and submissions related to work         \\
% P2                & 6     & 6     & 0     & 1     & 5     & Task management and communication related to research and university administration \\
% P3                & 6     & 7     & 2     & 2     & 0     & Task management related to research with professors, informal contact with friends        \\
% P4                & 6     & 6     & 5     & 4     & 6     & Scheduling related to club activities, informal contact with friends, inquiries with a museum abroad   \\
% P5                & 2     & 3     & 5     & 1     & 1     & Meeting planning and confirmations related to work                  \\
% P6                & 3     & 6     & 6     & 3     & 5     & Scheduling and confirmations related to work, friends, and event organizers \\
% P7                & 4     & 0     & 6     & 0     & 2     & Scheduling and progress management related to research and business trips \\
% P8                & 5     & 2     & 1     & 5     & 0     & Progress management and administrative confirmations related to work                \\
% P9                & 13    & 11    & 12    & 12    & 10    & Daily informal contact with friends         \\ \hline
% \end{tabular}
% }
% \Description{The table provides background information about the participants in Study 2, including their system usage over five days (D1 to D5), which represents the number of emails they replied to using the system each day, as well as their main usage purpose. Participants range in age from 20 to 39 years, with both students and employees included, and a mix of male and female participants. Participant P1 is a 28-year-old male employee who primarily used the system for scheduling, task confirmations, and submissions related to work. His daily system usage across the five days was 3 emails on day one, 1 email on day two, and 1 email on day five, with no emails replied to on days three and four. Participant P2 is a 24-year-old female student who used the system for task management and communication related to research and university administration. Her daily system usage was consistent on the first two days, replying to 6 emails each day. She did not use the system on day three, replied to 1 email on day four, and increased her usage again with 5 emails on day five. Participant P3 is a 20-year-old female student who focused her system usage on task management related to research with professors and informal contact with friends. She replied to 6 emails on day one, 7 emails on day two, 2 emails on both day three and day four, and did not reply to any emails on day five. Participant P4 is a 24-year-old female student who used the system for scheduling related to club activities, informal contact with friends, and inquiries with a museum abroad. Her system usage was steady, replying to 6 emails on both day one and day two, 5 emails on day three, 4 emails on day four, and 6 emails on day five. Participant P5 is a 31-year-old male employee who primarily used the system for meeting planning and confirmations related to work. He replied to 2 emails on day one, 3 emails on day two, 5 emails on day three, and 1 email on both day four and day five. Participant P6 is a 39-year-old female employee who used the system for scheduling and confirmations related to work, friends, and event organizers. Her system usage was moderate, replying to 3 emails on day one, 6 emails on day two, 6 emails on day three, 3 emails on day four, and 5 emails on day five. Participant P7 is a 25-year-old male student who used the system for scheduling and progress management related to research and business trips. His system usage was irregular, replying to 4 emails on day one, none on day two, 6 emails on day three, none on day four, and 2 emails on day five. Participant P8 is a 23-year-old female employee who used the system for progress management and administrative confirmations related to work. Her daily system usage was 5 emails on day one, 2 emails on day two, 1 email on day three, 5 emails on day four, and none on day five. Participant P9 is a 38-year-old male employee who used the system for daily informal contact with friends. He had the highest and most consistent email activity, replying to 13 emails on day one, 11 emails on day two, 12 emails on both day three and day four, and 10 emails on day five.}
% \end{table*}
\subsection{Participants' Email-Replying Process (RQ1)}
\subsubsection{Improved Perception of Efficiency and Workload}
\label{sec:result2_efficiency}
Participants reported that their perception of workload and work efficiency improved due to the support from ResQ.
Specifically, participants noted that ResQ's support helped clarify the topics they needed to address in the email. 
Participants explained that \textit{``Normally, when writing, I need to process multiple tasks simultaneously to ensure my intentions are appropriately expressed. However, [With ResQ,] replying to emails was divided into two different sub-tasks, answering questions and polishing emails with diverse expressions. As a result, I felt that the cognitive load was reduced.''} [P7], and \textit{``it felt like creating an email was as simple as answering a survey''} [P6].
Additionally, particularly when the counterparts' message was long, participants reported that the listing of requests as questions allowed them to \textit{``easily understand the content of the email''} [P3], with another participant noting that \textit{``I can quickly make decisions on what to reply [with ResQ]''} [P6]. 
Furthermore, compared to other AI tools like ChatGPT, the QA-based approach enabled participants to communicate their intentions more efficiently without extensive typing.
As one participant described,\textit{``[Writing with ResQ] made it easier to reflect my intentions while replying to the email''} [P4], while another participant added that \textit{``I could create the expected reply without even having to type on the keyboard''} [P6].
% While ResQ was helpful in formal situations, some participants found it less suitable for informal communications, such as with friends.
% One participant remarked that \textit{``the expressions were too polite, and I didn't like it''} [P3], and another mentioned that they \textit{``thoroughly changed the phrasing, like replacing 'Looking forward to seeing you again!' with just 'See you'''} [P4].
% Additionally, eight participants (with the exception of P9, who only used the system in informal settings) expressed increased satisfaction with the quality of the responses they wrote (for more details, see Sec.~\ref{sec:result2_quality}) and responded positively to the question, \textit{``What is your overall impression of using ResQ?''} and expressed a desire to continue using the system in the future. 
\red{Additionally, all participants expressed increased satisfaction with the quality of the responses they wrote (for more details, see Sec.~\ref{sec:result2_quality}) and responded positively to the question, \textit{``What is your overall impression of using ResQ?''} and expressed a desire to continue using the system in the future.}
Participants also mentioned that being able to craft clearer messages more quickly than before resulted in \textit{``greater confidence in the reply process and a more positive perception of the task''} [P8]. 
Additionally, a different participant expressed, \textit{``I felt joy in meeting societal expectations competently''} [P2].
These increases in achievement and confidence led participants to report that their \textit{``perception of the reply task became more positive''} [P8], and they felt \textit{``more motivated to engage actively in email responses''} [P3]. 

\subsubsection{Reduced Difficulty in Initiating the Action for Replying to Emails}
\label{sec:result2_initiating_the_action}
Participants reported that ResQ's support lowered the barrier to starting tasks, reducing procrastination in replying to emails.
One participant shared that they previously \textit{``felt reluctant to engage in replying due to the burden of the task''}, but with ResQ, \textit{``I felt motivated because I can complete the task quickly''} [P3]. 
Another participant noted that \textit{``I became able to craft replies to any email easily, so I could respond even on days when I was tired or when I would typically postpone replying to long emails''} [P6].
Participants also reported that using AI to initiate the task motivated them to start replying to emails without procrastinating. 
One participant explained that \textit{``just pressing a button prompts the AI to ask questions''}[P4], which led them to \textit{``delegate the initial steps entirely to the system''} [P3]. 
This reduction in the burden of the initial stage was cited as a key factor in lowering the barrier to starting to reply to emails. 

\subsubsection{Reduced Sense of Agency and Control}
\label{sec:result2_agency_control}
% Four out of nine participants (P3, P5, P6, and P9) reported a decreased sense of agency and control while replying to emails with ResQ. 
\red{Three out of eight participants (P3, P5, and P6) reported a decreased sense of agency and control while replying to emails with ResQ.}
They attributed this to several factors: one participant mentioned that their perception shifted \textit{``from that of an author to that of an editor''} [P5], which reduced the workload of replying but made the process feel \textit{``like an assembly line''} [P3], while another expressed, \textit{``I ended up using words or expressions I normally wouldn't [use in the email]''} [P6].
In contrast, for those who reported no change in their sense of agency or control (five participants), they explained that this was because  \textit{``the email content was strongly related to me''} [P7], and they \textit{``checked the content carefully''} [P7] or \textit{``modified words that I wouldn't normally use to the ones I would use''} [P2], leading them to feel that their \textit{``active involvement [to reply to the email] was indispensable''} [P8]

\subsection{Increased Perceived Quality of the Email (RQ2)}
\label{sec:result2_quality}
Participants reported that they felt the quality of their emails had improved. 
Participants explained that, in the process of creating responses, they were most concerned with \textit{``politeness in language, such as expressions and greetings''} [P6], and mentioned that ResQ provides support in these areas.
Participants reported that \textit{``it was helpful to have phrases that would have taken time to come up with on their own, expressions of apology and gratitude, and additional words of consideration for the other person''} [P2], \textit{``there was no need to think about the opening and closing greetings''} [P8], and \textit{``there were no typos or omissions at all''} [P5].
Additionally, participants mentioned that ResQ helped reduce the likelihood of overlooking requests in the emails they received. 
One participant shared, \textit{``Previously, when a single email contained multiple requests, I sometimes missed responding to all of them, but the questions provided by ResQ helped improve this''} [P6].
Participants attributed this improvement to the fact that ResQ \textit{``secured time to focus on understanding the recipient's requests and responding to them''} [P1], and the questions generated by ResQ \textit{``helped me ensure that nothing was overlooked in the content''} [P7]. 
Furthermore, participants reported that responding to AI-generated questions encouraged them to include details they would normally omit, resulting in more polite and comprehensive responses. 
One participant described an email regarding event attendance and multiple confirmations, explaining that while they would usually reply with something like \textit{``I will attend, thank you''}, answering the AI's questions led to a response where \textit{``each of the recipient's requirements was addressed more carefully''} [P2], ultimately leading to a more courteous email.

\subsection{Relationship between Participants and Their Counterpart (RQ3)}
\subsubsection{Enabling a Positive Self-Presentation as an Email Sender}
\label{sec:result2_self-presentation}
The participants reported feeling they could make a good impression on others using ResQ. 
They attributed this to improvements in the quality of their writing, shorter response times, and increased frequency of replies. 
One participant mentioned, \textit{``I could answer the other person's questions clearly, and the writing became more polished, making it easier for them to read''} [P5]. 
The participant also mentioned that \textit{``I felt the individuality of the email reply had faded''} but added that \textit{``I never intended to express individuality in my emails to begin with, so even if it was lost, it wasn't an issue as long as it felt natural to the recipient''} [P5].
Another participant shared that when they met a professor with whom they had communicated via ResQ, the person remarked, \textit{``Your emails have become more polished.''} 
They further elaborated, \textit{``I was particularly complimented on how much more understandable the structure of my emails has become''} [P3].
Additionally, this participant noted, \textit{``Previously, I would often respond to long emails with just, 'I'll get back to you later,' because reading through and thinking about a proper reply was tedious. However, [with ResQ's support,] I've started responding immediately instead of postponing. As a result, I've been assigned more tasks than before.''}

\subsubsection{Psychological Distance between Participants and Their Counterpart}
\label{sec:result2_psychological_distance}
Participants had mixed opinions regarding the psychological distance they perceived from their counterparts.
Those who felt the decreased psychological distance between themselves and their counterparts attributed this to the positive impression they believed they made on their counterparts. 
One participant reported that sending well-crafted emails quickly led to \textit{``a stronger sense of reassurance in [formal] communication''} [P5], while another participant noted that \textit{``[When I asked the museum staff a question,] I noticed that when I replied immediately after receiving a message from the other person, they responded quickly in return. When we communicated with such a good rhythm, I felt a strong sense of closeness towards the counterpart''} [P4].
In contrast, participants who felt the increased psychological distance mentioned a strong awareness that their replies were mediated by a system and the use of words they would not usually choose. 
One participant gave an example of communication with their university professor, stating, \textit{``While I know the counterpart typed their emails manually, I felt that using AI made the conversation more superficial, which weakened our relationship''} [P3]. 
% Another participant mentioned that the awareness when communicating with friends that \textit{``knowing that the email wasn’t something I had crafted from scratch by myself made me feel more distant from the counterpart''} [P9].
Participants also shared that they tended to forget about the email exchange with their counterparts due to the increased psychological distance.
% their sense of psychological distance from the sender as being caused by their forgetting interactions with them. 
One participant mentioned, \textit{``I found the email content easy to understand while working on it [with ResQ], but I felt it was difficult to retain our email exchange in long-term memory. When that counterpart [who is my professor] asked me, 'What happened with that issue? [that had been mentioned in our email]' there were times I couldn’t remember, which made me feel anxious''} [P3].

\subsection{Perceived Risks}
\label{sec:result2_risks}
Participants expressed concerns about the potential risks that ResQ might pose in the future. 
They expressed concerns about potential declines in their abilities and the risk of becoming overly dependent on AI, which could lead to carelessness in responding to work-related emails.
One participant explained, \textit{``I worry that the skills I've developed from composing emails myself might deteriorate''} [P8]. 
Another participant voiced concerns that \textit{``the advancement and usage of AI [in this context] might erode our ability to overcome psychological barriers''} [P2], fearing a decline in their interpersonal communication skills.
Additionally, participants raised the issue of over-reliance on AI, with one participant noting, \textit{``Given my trust in AI, I might eventually stop reviewing the content of the emails I send or the emails I receive''} [P8]. 
This reflects their concern about the potential for becoming overly dependent on AI-generated text in the future.

% \subsection{\red{Valuable AI-generated Questions and Options}}
% % ResQが生成した質問と選択肢は、十分に実用的であることが実験結果から示唆されたが、その質にはさらなる改善の余地がある
% % まず参加者は、実験1のインタビュー結果と同様、\textit{``自分と相手の関係性をAIが勘違いして質問を作成していることがあった''} [P2]と、質問生成の精度の低さを指摘した。
% % さらにある参加者は、\textit{``質問の生成に時間がかかる時があり、その時間にできることがないので少しフラストレーションを感じた''} [P1]と回答し、質問生成の生成時間の長さを指摘した。
% % またある参加者は、\textit{``システムが過剰に質問をしたことでフラストレーションを感じたことがあった''} [P3]と質問の多さを指摘し、別の参加者は、\textit{``様々な状況を想定してより多くの質問をして欲しかった''} [P8]と質問の少なさを指摘した。
% \red{
% The results suggest that the questions and options generated by ResQ were sufficiently practical; however, there is still room for improvement in their quality. 
% First, as in the interview results from Study 1, participants pointed out inaccuracies in question generation. 
% One participant noted, \textit{``there were instances where the AI misunderstood the relationship between myself and the other person when generating questions''} [P2], highlighting the low accuracy of the generated questions.
% Additionally, one participant pointed out the long generation time, explaining, \textit{``there were times when generating questions took too long, and I felt frustrated because there was nothing I could do during that time''} [P1]. 
% Another participant mentioned feeling frustrated by the excessive number of questions, stating, \textit{``there were times when the system asked too many questions, which I found frustrating''} [P3]. 
% On the other hand, another participant expressed dissatisfaction with the limited number of questions, saying, \textit{``I wanted the system to generate more questions that accounted for a wider range of scenarios''} [P8].
% }

\section{Discussion}
\red{Through a controlled experiment (Study 1) and a field study (Study 2), we investigated the impact of the LLM-powered QA-based approach on both senders and receivers.
In this section, we discuss the findings (Fig.~\ref{tab_summary}) of the research and the key considerations for designing QA-based systems.}
\begin{table*}[t]
\caption{Research Questions and Key Findings}
\label{tab_summary}
\centering
% \resizebox{\textwidth}{!}{
\begin{tabular}{>{\raggedright\arraybackslash}p{0.08\linewidth}>{\raggedright\arraybackslash}p{0.28\linewidth}>{\raggedright\arraybackslash}p{0.28\linewidth}>{\raggedright\arraybackslash}p{0.28\linewidth}}
\hline
 & \textbf{RQ1: How does a QA-based response-writing support approach affect workers’ email-replying process?} & \textbf{RQ2: How does a QA-based response-writing support approach affect the quality of the email response?} & \textbf{RQ3: How does a QA-based response-writing support approach affect the perceived relationship between email sender and recipient?} \\ \hline
\textbf{Key Findings} & 1. QA-based approach \textbf{reduced workload} for email comprehension and prompt creation and \textbf{improved work efficiency}. (H1-a, supported; H1-b, supported, Sec.~\ref{sec:result1_efficiency},~\ref{sec:result1_prompt_character_counts},~\ref{sec:result1_cognitive_load},~\ref{sec:result1_difficulty_in_understanding},~\ref{sec:result1_interview_RQ1},~\ref{sec:result2_efficiency}) 

2. QA-based approach \textbf{reduced the difficulty} of initiating the email replying task. (H1-d, supported, Sec.~\ref{sec:result1_initiating},~\ref{sec:result1_interview_RQ1},~\ref{sec:result2_initiating_the_action})

3. QA-based approach \textbf{decreased the sense of agency and control}. (H1-e, supported, Sec.~\ref{sec:result1_agency},~\ref{sec:result1_interview_RQ1},~\ref{sec:result2_agency_control})

4. QA-based approach \textbf{improved satisfaction} with the emails they wrote and willingness to use ResQ in the future. (H1-c, supported, Sec.~\ref{sec:result1_satisfaction},~\ref{sec:result1_interview_RQ1},~\ref{sec:result2_efficiency})& Writing emails with QA-based approach and Prompt-based approach led to \textbf{increased email quality} than No-AI condition. (H2, partially supported, Sec.~\ref{sec:result1_quality},~\ref{sec:result1_interview_RQ2}~\ref{sec:result2_quality})& 1. Writing emails with QA-based approach \textbf{\blue{did not lead to improved perceived impression of users by their counterparts}}. (H3-a, not supported, Sec.~\ref{sec:result1_impression},~\ref{sec:result2_self-presentation})

2. Writing emails with QA-based approach led to \textbf{increased psychological distance} between users and their counterparts than No-AI condition. (H3-b, partially supported, Sec.~\ref{sec:result1_psychological_distance},~\ref{sec:result1_interview_RQ3},~\ref{sec:result2_psychological_distance})\\ \hline
\end{tabular}
% }
\Description{This table summarizes the three research questions (RQs) investigated in the study and highlights the key findings associated with each. RQ1: How does a QA-based response-writing support approach affect workers’ email-replying process? Key Findings: 1. The QA-based approach reduced workload for email comprehension and prompt creation, leading to improved work efficiency. This supports hypotheses H1-a and H1-b. 2. It reduced the difficulty of initiating the email replying task, supporting hypothesis H1-d. 3. The approach decreased the sense of agency and control among users, supporting hypothesis H1-e. 4. Users experienced improved satisfaction with the emails they wrote and showed a greater willingness to use ResQ in the future, supporting hypothesis H1-c. RQ2: How does a QA-based response-writing support approach affect the quality of the email response? Key Findings: Writing emails using both the QA-based and prompt-based approaches led to an increase in email quality compared to the No-AI condition. This partially supports hypothesis H2. RQ3: How does a QA-based response-writing support approach affect the perceived relationship between email sender and recipient? Key Findings: 1. Writing emails with QA-based approach did not lead to improved perceived impression of users by their counterparts, meaning hypothesis H3-a was not supported. 2. Writing emails with the QA-based approach led to an increase in psychological distance between users and their counterparts compared to the No-AI condition. This partially supports hypothesis H3-b.}
\end{table*}


% \textbf{Summary} & Workers evaluated the system’s benefits (improvements in efficiency, cognitive load, and satisfaction), accepted a certain reduction in the agency, and showed a willingness to use it in the future. & It became possible to create responses that appropriately addressed email requests while maintaining politeness. (H2, supported) & ResQ improved workers' impression. Perceived psychological distance from others depends on a balance between a sense of agency and communication satisfaction. \\ \hline
\subsection{Impact of the QA-based Approach}
\subsubsection{Enhancing Efficiency and Reducing Cognitive Load}
% Our studies indicate that the QA-based approach improves efficiency and reduces cognitive load when composing email replies (Sec.~\ref{sec:result1_efficiency},~\ref{sec:result1_prompt_character_counts}, \ref{sec:result1_cognitive_load},~\ref{sec:result1_difficulty_in_understanding},~\ref{sec:result1_interview_RQ1},~\ref{sec:result2_efficiency}). 
Our studies indicate that the QA-based approach improves efficiency and \blue{suggests a reduction in} cognitive load when composing email replies (Sec.~\ref{sec:result1_efficiency},~\ref{sec:result1_prompt_character_counts}, \ref{sec:result1_cognitive_load},~\ref{sec:result1_difficulty_in_understanding},~\ref{sec:result1_interview_RQ1},~\ref{sec:result2_efficiency}). 
One possible explanation is that the QA-based approach helps users focus on the most relevant details, simplifying email comprehension compared to prompt-based methods.
% understand key information and organize their responses effectively.
% By emphasizing key information through AI-generated questions, this approach allows users to focus on the most relevant details, simplifying email comprehension compared to prompt-based methods.
% This finding aligns with cognitive load theory~\cite{sweller2011cognitive}, suggesting that reducing extraneous cognitive load enables users to perform tasks more efficiently.
Additionally, the QA-based approach reduces the burden of prompt creation by partially replacing the task of crafting prompts with the simpler task of answering questions.
\blue{Our finding suggests that future email systems could use this QA-based approach to mediate the email exchange process.}
% 更なる効率改善、負荷低減のためには、質問の量や質、提示する順序の最適化が効果的な可能性がある
% また選択肢も調整することができる
% 例えばシンプルなYes/Noの選択肢は有用でしたが、スケジュールツールや自由入力フィールドのような柔軟な入力も求められたので、カスタマイズ可能な入力形式(例: スケジューリング用のカレンダーセレクター)を導入することで、さらに使いやすさを向上させることができる可能性がある
% To further enhance efficiency and reduce cognitive load, optimizing the quantity, quality, and sequence of questions may be beneficial. 
% Additionally, refining response options could improve usability. 
% For instance, while simple Yes/No choices were effective, users also expressed a need for more flexible input options, such as scheduling tools or free-text fields. 
% Thus, introducing customizable input formats, such as a calendar selector for scheduling, could further enhance usability and streamline the email composition process.

\subsubsection{Potential Reduction in Sense of Agency and Control}
\red{While the QA-based approach enhanced users' efficiency, our studies also revealed a potential trade-off in users' sense of agency and control (Sec.~\ref{sec:result1_agency},~\ref{sec:result1_interview_RQ1},~\ref{sec:result2_agency_control}).
Some participants reported a decreased sense of authorship, feeling more like editors than creators of their emails. 
This reduction in agency may be due to the diminished amount of text input required from the user, as the AI takes a more active role in content generation.
% またこれらの感覚は、ユーザの好みに強く影響を与えることも明らかとなり、agencyの維持を望むような重要な場面などではこのアプローチは使用したくないと報告する参加者もいました。
Moreover, we found that the sense of agency influenced users' preferences for future usage.
% For example, some participants expressed a reluctance to use this approach in critical contexts where preserving their sense of agency was particularly important.
% However, this effect was not uniform across all users. 
Among those participants who still maintained their sense of agency, we found that they tended to actively review and modify the AI-generated content to reflect their personal style and intentions.
% これはQA-based approachのようにAIの介入が大きくとも、ユーザが積極的に内容を確認・編集することで、ユーザの主体性を維持できる可能性があるということを示唆している
\blue{This suggests that even when AI intervention is substantial, users can maintain a sense of authorship by actively engaging with and refining the AI's suggestions.}
% These insights emphasize that the importance of the agency may depend on individual user preferences, the context of the communication, and the extent to which users personalize the AI-generated output.
}
% ユーザにとって望ましいagencyのレベルに最適化するために、シチュエーションやユーザの好みに応じて質問や選択肢の数、提案のレベル(文レベル、メッセージレベルなど)を変えるなど、AIの介入方法や度合いを変化させるアプローチが効果的な可能性がある
\blue{To optimize users' level of agency, adapting the degree of AI intervention in the email construction process can be helpful.
% based on the situation and user preferences may be effective. 
% This could involve 
For instance, by adjusting the number and type of AI-generated questions or varying the levels of AI-generated suggestions~\cite{Fu2023Comparing}, ranging from word-level to message-level.} 
% (\textit{e.g.}, sentence-level vs. message-level recommendations).}

% \subsubsection{\red{Improvement in Email Quality and Users' Impressions of Communication Partners}}
% \red{The QA-based approach was found to enhance the perceived quality of email responses (Sec.~\ref{sec:result1_quality},~\ref{sec:result1_interview_RQ2}~\ref{sec:result2_quality}). 
% Our studies revealed that emails composed with AI assistance were more polite, well-structured, and addressed multiple requests without omissions. 
% Particularly, the QA-based approach can support users in creating emails that meet the sender's demands by organizing them through AI-generated questions. 
% Furthermore, by enhancing efficiency and lowering the barriers to initiating tasks, this approach facilitates quicker response times, which, in turn, positively influence users' impressions of their correspondents~\cite{yoram2011online, Resendes2012Send, vignovic2010computer}. 
% In summary, these findings indicate that the QA-based approach not only supports users in creating higher-quality email responses but also fosters more positive impressions.}

\subsubsection{Possibility of Improving Relationship between Email Sender and Recipient}
\red{Our studies yielded mixed results regarding the impact of the QA-based approach on the psychological distance between users and their counterparts (Sec.~\ref{sec:result1_psychological_distance},~\ref{sec:result1_interview_RQ3},~\ref{sec:result2_psychological_distance}). 
% Some参加者は素早く、高い質のメールを送信できたことや、またそれによって相手からの返信が早くなったことで、相手との間に感じる距離感を近く感じた。
% 一方で他の参加者は、作業の労力が減ったことや、自分が普段使わない言葉を使っていることに気がついたことで、felt a sense of increased distance.
Some participants reported that they were able to send emails more quickly and with high quality, which in turn led to faster responses from others and a reduced sense of distance in their interactions.
In contrast, other participants experienced an increased sense of distance, which has also been reported in the previous studies~\cite{Fu2023Comparing,arnold2020predictive}. 
They noted that the reduced communication effort and the use of unfamiliar language made interactions feel less personal or authentic.
% Some participants felt closer to their counterparts due to quicker response times and higher-quality emails.
% Others, however, felt a sense of increased distance, partly because the AI-mediated communication felt less personal or authentic.
% These findings align with previous studies~\cite{Fu2023Comparing,arnold2020predictive} indicating that while AI can contribute to maintaining a professional tone, it may also result in less authenticity when AI-generated language deviates from a user's usual style, thereby increasing psychological distance.
% These divergent users' feedback suggests that while the QA-based approach can enhance certain aspects of communication, it may also inadvertently introduce a sense of impersonality. 
The degree of the perceived distance may depend on factors such as the nature of the relationship (\textit{e.g.,} colleagues vs. friends), the user's reliance on AI-generated language, and individual preferences regarding AI-mediation in communication.}

\subsection{Opportunities and Challenges of Introducing QA-Based Approach}
% \subsubsection{Situations where QA-based is Useful in Email Communication}
Our results indicate that the QA-based approach 
% effectively streamlines the process of composing responses and enhances the quality of replies.
% Therefore, this approach 
is particularly useful in situations where speed and high-quality responses are prioritized over email personality or a strong sense of personal agency. 
Contexts such as business, customer service, and technical support can greatly benefit from the QA-based approach, as they often require efficient and structured communication.
% Additionally, the QA-based AI-assisted email replying mechanism proves highly effective in scenarios where maintaining a neutral tone and low emotional engagement is appropriate, such as initial contacts or interactions among weak ties. 
% In these cases, the system’s ability to generate standardized, professional language supports users in composing suitable replies promptly.
% In situations where maintaining a neutral affect or low emotional engagement is encouraged (\textit{e.g.}, in initial contacts and weak ties), a QA-based AI-assisted email replying mechanism can be effective.

However, for more delicate or personal email exchanges, users may prefer more tailored interventions. 
In such situations, users can adjust the level of involvement of AI intervention.
% or automatically tuning the intervention based on the email content or the user’s past behavior could better meet their needs.
% This customization can help maintain the authenticity and personal touch necessary for meaningful communication.
% Future research should explore how different levels of AI-mediated intervention affect users’ sense of agency and email construction behavior across various communication contexts.
Furthermore, there is a risk that users could become overly reliant on technology to mediate their interpersonal communication. 
Our interviews revealed that users might become accustomed to trusting AI-generated questions and drafts due to the efficient outcomes. 
Consequently, they may become less diligent in reading the emails they receive or in reviewing the responses they send carefully.
This over-reliance could lead to miscommunication or the omission of important details, thus undermining the primary goal of using AI to improve communication efficiency.
Future research should explore how different levels of AI-mediated intervention can be designed to influence users’ sense of agency and email construction behavior for various communication purposes.


% Beyond email communication, a QA-based approach utilizing LLMs can alleviate users' workload while preserving a degree of agency in areas that require structured information gathering and intent-driven interactions.
% % Beyond email communication, QA-based systems powered by LLMs have the potential to reduce users' workload while maintaining a certain level of agency in domains requiring structured information elicitation and intent-driven interactions. 
% A QA-based approach can streamline tasks such as drafting structured documents (\textit{e.g.}, academic rebuttals), clarifying user needs (\textit{e.g.}, in customer support or medical consultations), and facilitating team consensus by presenting key points as questions. 
% % Additionally, emerging techniques, such as generating follow-up questions during conversations~\cite{hu2024designing} or creating multiple-choice questions to assess comprehension~\cite{cheng2024treequestions}, highlight the versatility of LLM-powered questioning systems. 
% % Integrating these innovations into QA-based systems could unlock new applications across a wide range of fields.


% Beyond email communication, a QA-based approach utilizing LLMs can alleviate users' workload while preserving a degree of agency in areas that require structured information gathering and intent-driven interactions.
% First, QA-based systems can streamline tasks that require the creation of formal response documents (\textit{e.g.}, academic rebuttals) or complex online applications (\textit{e.g.}, Visa application) by presenting key points to understand and address as questions.  
% Also, in settings requiring consensus-building, such as team projects, QA-based systems may facilitate discussions by presenting questions aimed at identifying mutual goals, challenges, or uncertainties. The system can help clarify differing needs and provide feedback to streamline the decision-making process, ultimately improving collaboration.

% \subsubsection{\red{Applicability Beyond Email Replying}}
% \label{sec:discuss_applicability_beyond_email_replying}
% \red{Our findings suggest that LLMs can generate questions to elicit users' intentions and help them organize their thoughts, thereby enabling more efficient and effective outputs through user interaction. 
% Moreover, approaches such as generating follow-up questions based on user responses during a conversation~\cite{hu2024designing}, or even creating multiple-choice questions to assess user comprehension~\cite{cheng2024treequestions}, demonstrate the evolving versatility of LLM-powered questioning systems. 
% This suggests that QA-based systems leveraging LLMs hold significant potential in a variety of domains where structured information elicitation and intent communication are required, extending far beyond email composition.}

% % 1
% \red{Beyond email communication, a QA-based approach utilizing LLMs can alleviate users' workload while preserving a degree of agency in areas that require structured information gathering and intent-driven interactions.
% First, QA-based systems can streamline tasks that require the creation of formal response documents (\textit{e.g.}, academic rebuttals) or complex online applications (\textit{e.g.}, Visa application) by presenting key points to understand and address as questions.  
% Also, in settings requiring consensus-building, such as team projects, QA-based systems may facilitate discussions by presenting questions aimed at identifying mutual goals, challenges, or uncertainties. The system can help clarify differing needs and provide feedback to streamline the decision-making process, ultimately improving collaboration.}

% 我々のfindingsは、LLMにはすでにユーザの意図を引き出したり、考えをまとめたりするための質問を生成でき、ユーザとのインタラクションを通じてより良いアウトプットを効率的に出力できることを示唆している
% また今回のResQのように、受信メールに基づいて質問を一度だけ生成するのではなく、会話中のユーザーの回答に基づいてフォローアップ質問をさせたり~\cite{}、さらにはユーザの理解度を評価するための多肢選択問題を生成する~\cite{}アプローチもとられている
% したがってLLM を利用した QA ベースのシステムは、電子メールの作成にとどまらず、構造化された情報の引き出しと意図の伝達が求められるさまざまな分野で潜在的可能性を秘めていることを示唆している
% 例えば...
% まず、構造的な文書作成をする必要がある場面において、QA-based systemは作業を効率化できる可能性がある
% 例えば、drafting academic rebuttals, legal case summaries, or detailed project reportsの作成の際、QA-based systemが重要な論点を質問として提示し、ユーザはそれに回答することで考えを整理するとともに、それが反映されたドラフトを受け取ることができる
% また、他者のニーズを明確化する必要がある場面において、QA-based systemは役に立つ可能性がある
% 例えば、カスタマーサポートや医療問診の際、QA-based systemが、事前に顧客の問題を把握するための質問を提示し、顧客がそれに回答することで、顧客は自分のニーズを明確化するとともに、サポートする側は顧客のニーズを効率的に知ることができる
% さらに、合意形成が求められる場面において、複数のステークホルダーの意見を整理することで、合意形成を支援できる可能性がある
% 例えば、チームのプロジェクトにおいて、QA-based systemが、互いの目的や問題点、不明点等を洗い出すための質問を両者に提示することで、互いのニーズを明確化し、議論を円滑化できる可能性がある

% Designing the Conversational Agent: Asking Follow-up Questions for Information Elicitation
% 対話型エージェント(CAs)がインタビューや情報収集の場面で、事前に決まった質問だけでなく、会話中のユーザーの回答に基づいたフォローアップ質問を生成する能力を向上させる。
% ヒトのインタビュアーが用いるフォローアップ質問のテクニックを取り入れることで、CAsが有益な情報を引き出せるように設計する。

% TreeQuestion
% 1. 背景
% オープンエンド質問(自由回答形式)は、学生の理解を評価するために使われますが、AI(例えばChatGPT)の利用により、学生が容易に長文回答を生成できる時代において課題となっています。
% 教師は依然として回答を読む時間や学習成果を推測する負担を抱えています。
% 2. TreeQuestionシステム
% 目的: 教師が概念学習成果を評価するための多肢選択問題を効率的に作成する支援。
% 仕組み:
% 大規模言語モデルを利用して、与えられた概念を基に多肢選択問題を生成。
% 質問はツリー構造で整理され、異なる理解レベル(記憶、理解、応用、分析、評価、作成)に対応。
% 誤解を誘う選択肢(Distractors)も生成し、学生が正しい選択肢を選べるかどうかで学習成果を評価。
% 3. 特徴
% 人間とAIの協働:
% 教師はAIが生成した内容を検証・修正し、適切な質問を作成。
% 「探索(Explore)- 検証(Validate)- 生成(Generate)」という段階的プロセスを採用。
% 効率向上:
% オープンエンド質問と比較して、TreeQuestionによるMCQ作成と採点の時間は大幅に短縮される。

% 1. 相手のニーズを明らかにする必要がある場合(他者のニーズを明確化する)
% % For instance, in customer support and technical assistance, QA-based systems can guide users through troubleshooting processes by asking targeted questions and providing predefined response options. 
% % This approach can reduce cognitive load for both customers and support agents, improving the overall experience and efficiency of problem resolution.

% 2. 個人間、あるいは集団間において、合意形成を取る必要がある場合(両者のニーズを明確化する)。1の延長線上かもしれない。
% 複数のステークホルダーが関与する場において、QAベースシステムが議論の論点を動的に生成し、意思決定をサポートすることで、合意形成や問題解決を促進できる可能性がある。
% 例えば、チーム会議やプロジェクト計画において、互いのニーズを整理し、それを元にシステムが次のステップを提案することが可能かもしれない。

% 3. メールと同様に、特定の個人や集団に対して、考えや情報を整理し提供する必要がある場合(構造的な文書作成を効率的にする)
% % Moreover, content creation for structured documents, such as academic rebuttals, legal case summaries, or collaborative reports, can benefit from QA-based systems. 
% % By generating task-relevant questions based on the XXX, these systems could assist users in organizing their thoughts, ensuring that all critical aspects are addressed systematically.

% \subsection{\red{Design Implications for QA-based Systems in Broader Contexts}}
% To ensure the effectiveness of QA-based systems across diverse applications, we identify several design considerations from our findings.

% \subsubsection{\red{Optimizing User Experience in QA-based Systems}}
% \red{The results of our studies suggest that customizing the questions and options generated by QA-based systems according to the task, communication context, and user characteristics can enhance the user experience. }

% \paragraph{\red{\textbf{Content of Questions}}}
% % 何をすべきか、なぜそうすべきかの順に書く
% % personalizationとかはあまり言わない方がいい
% \red{First, the system should generate questions that are relevant, precise, and easy to answer, while accurately reflecting the sender's intent. 
% For example, our studies revealed Yes/No questions or specific scheduling-related questions were particularly useful.
% One potential solution is to provide the system with prompts containing relevant user information, which can improve the accuracy and relevance of the generated questions.}

% \paragraph{\red{\textbf{Quantity of Questions}}}
% \red{Next, the system should maintain an optimal balance in the number of questions generated.
% Our findings revealed mixed reactions to the number of questions provided: while some participants appreciated confirmation questions (\textit{e.g.,} ``Do you understand XX?'') for ensuring clarity, others found them redundant. 
% Furthermore, generating too many questions often led to verbose and unfocused replies.
% To address this, the system should dynamically adjust the number of questions based on the complexity of the task and the user’s preferences. 
% Additionally, refining prompt designs to produce concise and relevant text can mitigate the issue of excessive verbosity in responses.}

% \paragraph{\red{\textbf{Order of Questions}}}
% \red{The logical sequencing of questions also plays a vital role in enhancing usability. 
% The system used in this study generated questions in accordance with the flow of the email, which could help participants better understand the content. 
% However, there is potential to further enhance the system by prioritizing questions based on their importance or relevance.
% Future developments could consider adjusting the order of questions based on their importance or relevance. 
% In addition, UI enhancements, such as grouping questions by priority or task category, may help users navigate the interaction more intuitively.}

% \paragraph{\red{\textbf{Efficiency of Question Generation}}}
% \red{The time taken to generate questions emerged as a source of frustration for some participants. 
% This issue can potentially be addressed by improving the processing speed of LLMs or implementing pre-generation mechanisms. 
% For example, questions could be generated in advance, before the user opens the email, thereby reducing waiting times and improving overall efficiency.}

% \paragraph{\red{\textbf{Balance and Flexibility of Options}}}
% \red{Finally, the system should ensure a balance in the number and diversity of response options provided. 
% Participants reported a decline in user experience when options did not align with their intent or when irrelevant options were presented. 
% Simple Yes/No choices are useful, but participants also expressed a need for more flexible input methods, such as scheduling tools or free-text fields.
% To address this, introducing customizable input types based on the specific task (\textit{e.g.,} UI components like calendar selectors for scheduling) can further enhance usability.}

% \subsubsection{Strategies to Maintain User Agency and Authenticity}
% なんでそれが重要か、何ができるかをfindingsから書く?
% To mitigate potential reductions in agency, allowing users to adjust the level of AI intervention or providing options to customize the AI's contributions may be beneficial....

% \subsubsection{Mitigating Risks of Over-Reliance on AI}
% % We found one concern associated with using QA-based systems is the potential risk of users becoming overly reliant on the system. 
% While the QA-based system offers advantages, one concern is the potential risk of users becoming overly reliant on the technology. 
% Interviews revealed that users might become accustomed to trusting AI-generated questions and drafts due to their high quality and the desire to reduce workload. 
% Consequently, they may become less diligent in reviewing the emails they receive or the responses they send.
% This over-reliance could lead to miscommunication or the omission of important details, undermining the primary goal of using AI to improve communication efficiency.
% % Furthermore, our findings suggest that the QA-based approach may alter users’ perceptions of email correspondence, making it feel more like completing a survey than engaging in meaningful communication.
% % Given the increasing integration of AI systems into daily tasks, this shift indicates that email's purpose and role as a communication tool may change.
% Thus, further exploration is needed to understand the long-term impact of AIMC tools on email communication's changing roles and needs. 
% Moreover, it is important to investigate in which scenarios and how interventions by AI should be made to ensure that the essence of human interaction is not compromised.

% \subsubsection{\red{Design Implications}}
% \red{Our studies suggest that tailoring questions and options in QA-based systems based on the task, communication context, and user characteristics can improve user experience.
% % 以下では、QA-basedシステムの設計において、検討および調整することができるオプションについて説明する。
% }
% \begin{enumerate}[]
%     \item \red{\textbf{Content of Questions:}}
%     \red{The system should generate questions that are relevant, precise, and easy to answer, accurately reflecting the sender's intent.
%     For example, Yes/No questions or specific scheduling-related questions proved particularly useful.
%     Providing the system with prompts containing relevant user information can improve the accuracy and relevance of generated questions.}
%     \item \red{\textbf{Quantity of Questions:}}
%     \red{The system must balance the number of questions generated.
%     While some users valued confirmation questions (\textit{e.g.}, ``Do you understand XX?''), others found them redundant, and answering many questions led to verbose replies.
%     Dynamic adjustment based on task complexity and user preferences, along with concise prompt designs, can address this issue.}
%     \item \red{\textbf{Order of Questions:}}
%     \red{Logical sequencing enhances usability.
%     In this study, questions generated according to email flow helped participants understand the content.
%     Prioritizing questions by relevance or grouping them by priority or task category in the UI could further improve navigation.}
%     \item \red{\textbf{Efficiency of Question Generation:}}
%     \red{Delays in generating questions frustrated participants.
%     Improving LLM processing speed or implementing pre-generation mechanisms (\textit{e.g.}, generating questions before users open emails) can reduce waiting times and improve efficiency.}
%     \item \red{\textbf{Balance and Flexibility of Options:}}
%     \red{A balance in response options is crucial.
%     Participants found user experience declined when options were irrelevant or inflexible.
%     Simple Yes/No choices were helpful, but flexible inputs, such as scheduling tools or free-text fields, were also desired.
%     Customizable input types (\textit{e.g.}, calendar selectors for scheduling) can further enhance usability.}
% \end{enumerate}

\subsection{Limitations and Future Work}
While it is evident that the QA-based approach positively impacted users' workload, the quality of the emails they produced, and their relationship with recipients in formal email responses, this study had several limitations.
Though we tried to use a mixed-method study to triangulate the findings from the control experiment and field study, we acknowledged that the quantitative results could be limited.
Because of privacy concerns, we were unable to access participants' email content, and as a result, we could not gather users' behavioral data. 
This includes information such as how they edited the prompts, the amount of time they dedicated to responding to emails, or how ResQ influenced the language they used in their actual email communications.
We encourage researchers to explore alternative research methods for capturing users' behavioral data in email exchanges in the wild to enrich the understanding of QA-based approaches in AI-mediated communication.

% また、メールの特徴ごとのQA-based approachの有効性については、さらなる調査が可能だろう。
% Study 1では、フォーマルなシチュエーションにおける様々なトピックのメールを使用して実験を行い、QA-based approachの及ぼす影響について調査した。
% しかし、その特徴(例えば、状況のformalさ、メールの丁寧さ、重要性、受信者と送信者の関係性など)ごとに、QA-based approachの有効性は異なる可能性がある。
\red{Second, the effectiveness of the QA-based approach may vary depending on the specific characteristics of the emails. 
We conducted Study 1 using emails on a variety of topics within formal scenarios to examine the impact of the QA-based approach. 
However, its effectiveness may differ based on characteristics such as the formality of the situation, the politeness of the email, its importance, or the relationship between the sender and recipient.
Therefore, future research could explore how these specific email characteristics influence the effectiveness of QA-based approaches, potentially tailoring AI-mediated tools to different communication contexts.}
% First, the quantitative results were confined to a controlled environment. 
% This limitation arose because, in field studies, accessing participants' email content was not feasible due to privacy concerns, making it difficult to perform fair comparisons using quantitative evaluation metrics. 
% For example, the time required to compose an email and the necessity of a reply vary depending on the content and context. 
% Therefore, we designed the field study to assess how ResQ influenced the practice of email replies qualitatively.

\blue{Third, the study was conducted with participants from a single cultural background, which could limit the generalizability of our findings. 
Although we contributed to a new understanding for populations from non-Western countries~\cite{WEIRD_CHI21}, we acknowledge that the practice of email exchange differs across cultures~\cite{Robertson2021ICant}.
% The findings could not be generalized to other cultural contexts since the nature and role of emails differ across cultures~\cite{Robertson2021ICant}. 
% Although some participants used ResQ in communications with individuals from different cultural backgrounds and observed a degree of effectiveness, 
Further studies are encouraged to examine whether similar results would be obtained among users from diverse cultural backgrounds or in cross-cultural email exchanges.}

% Also, the system has the potential to be applied to devices other than PCs and adapted for communication tools beyond email. 
% For example, integrating ResQ into business chat platforms like Slack or Microsoft Teams seems highly compatible, as these environments require efficient, goal-oriented communication. 
% Moreover, this QA-based approach could be extended to other domains where structured document creation is necessary. 
% One possible application is drafting rebuttal letters, where a structured format and the ability to address specific points are necessary.

% \red{Fourth, while Study 1 evaluates the impact of the LLM-powered QA-based approach under three conditions, it does not include a condition that replicates the QA-based approach without relying on LLMs. 
% This omission limits the ability to isolate the specific contribution of LLMs to the system's overall performance. 
% Incorporating a QA-based condition in future research (\textit{e.g.}, employing a rule-based approach or using manually created questions and options) could offer a more comprehensive understanding of the unique value LLMs provide compared to rule-based or manual methods.}

\red{Fourth, while this study focused on a QA-based approach driven by LLMs, future research could explore alternative methods of question generation to deepen our understanding of QA-based AI assistance. 
For instance, comparing the LLM-powered system with approaches utilizing rule-based question generation or manually prepared questions and options may help disentangle the effects of algorithmic sophistication from the inherent benefits of structuring communication as QA. 
This may potentially clarify whether the AI placebo or nocebo effect~\cite{kloft2024aiplacebo} exists in AI-mediated communication.
Examining these different methods could offer further insights into when and why the QA-based approach excels and guide the design of more tailored systems that accommodate a wide range of communication tasks and user needs.}

\red{Fifth, while this study demonstrated the effectiveness of the QA-based approach with initial design considerations (Sec.~\ref{sec:Proposed_Approach}), future research could explore tailoring these questions to specific communication goals or contexts. 
For example, designers or instructors could adjust factors such as the number of questions, their difficulty level, or their thematic focus to improve the user's understanding of challenging content. 
By iterating on the design to explore how different dimensions of question can affect communication outcomes, future work can better guide the QA-based approach.}

% \red{Fifth, future research could design the questions in our approach to be tailored for each user, such as question difficulty, quantity, and thematic focus. 
% % Tailored question sets may help users understand challenging content, expedite task completion, or improve communication efficiency in specialized domains. 
% By iterating on the design, researchers can develop more refined QA-based systems that better meet user needs.}
% \section{Conclusion}

The question addressed in this paper is whether it is possible to develop a \emph{general framework} for point-and-click AUIs that does not depend on task-specific heuristics or data to generate policies offline. To this end, we have introduced \marlui, a multi-agent reinforcement learning approach. Our method features a \useragent and an \interfaceagent. The \useragent aims to achieve a task-dependent goal as quickly as possible, while the \interfaceagent learns the underlying task structure by observing the interactions between the \useragent and the UI. Since the \useragent is RL-based and thus learns through trial-and-error interactions with the interface, it does not require real user data. We have evaluated our approach in simulation and with participants, by replacing the \useragent with real users, across five different interfaces and various underlying task structures. The tasks ranged from assigning items to a toolbar, handing out-of-reach objects to the user, selecting the best-performing interface, providing the correct object to the user, and enabling more efficient interaction with a hierarchical menu. 
Results show that our framework enables the development of AUIs with minimal adjustments while being able to assist real users in their task.
We believe that \marlui, and a multi-agent perspective in general, is a promising step towards tools for developing adaptive interfaces, thereby reducing the overhead of developing adaptive strategies on an interface- and task-specific basis.

\begin{acks}
This work was supported by JSPS KAKENHI (JP24H00742 and JP24H00748).
We thank all the participants for their interest and involvement in this study.
We also appreciate the reviewers for their constructive feedback, which helped us refine this work.
\end{acks}

% \bibliographystyle{ACM-Reference-Format}
% \bibliography{reference}
% This must be in the first 5 lines to tell arXiv to use pdfLaTeX, which is strongly recommended.
\pdfoutput=1
% In particular, the hyperref package requires pdfLaTeX in order to break URLs across lines.

\documentclass[11pt]{article}

% Change "review" to "final" to generate the final (sometimes called camera-ready) version.
% Change to "preprint" to generate a non-anonymous version with page numbers.
\usepackage{acl}

% Standard package includes
\usepackage{times}
\usepackage{latexsym}

% Draw tables
\usepackage{booktabs}
\usepackage{multirow}
\usepackage{xcolor}
\usepackage{colortbl}
\usepackage{array} 
\usepackage{amsmath}

\newcolumntype{C}{>{\centering\arraybackslash}p{0.07\textwidth}}
% For proper rendering and hyphenation of words containing Latin characters (including in bib files)
\usepackage[T1]{fontenc}
% For Vietnamese characters
% \usepackage[T5]{fontenc}
% See https://www.latex-project.org/help/documentation/encguide.pdf for other character sets
% This assumes your files are encoded as UTF8
\usepackage[utf8]{inputenc}

% This is not strictly necessary, and may be commented out,
% but it will improve the layout of the manuscript,
% and will typically save some space.
\usepackage{microtype}
\DeclareMathOperator*{\argmax}{arg\,max}
% This is also not strictly necessary, and may be commented out.
% However, it will improve the aesthetics of text in
% the typewriter font.
\usepackage{inconsolata}

%Including images in your LaTeX document requires adding
%additional package(s)
\usepackage{graphicx}
% If the title and author information does not fit in the area allocated, uncomment the following
%
%\setlength\titlebox{<dim>}
%
% and set <dim> to something 5cm or larger.

\title{Wi-Chat: Large Language Model Powered Wi-Fi Sensing}

% Author information can be set in various styles:
% For several authors from the same institution:
% \author{Author 1 \and ... \and Author n \\
%         Address line \\ ... \\ Address line}
% if the names do not fit well on one line use
%         Author 1 \\ {\bf Author 2} \\ ... \\ {\bf Author n} \\
% For authors from different institutions:
% \author{Author 1 \\ Address line \\  ... \\ Address line
%         \And  ... \And
%         Author n \\ Address line \\ ... \\ Address line}
% To start a separate ``row'' of authors use \AND, as in
% \author{Author 1 \\ Address line \\  ... \\ Address line
%         \AND
%         Author 2 \\ Address line \\ ... \\ Address line \And
%         Author 3 \\ Address line \\ ... \\ Address line}

% \author{First Author \\
%   Affiliation / Address line 1 \\
%   Affiliation / Address line 2 \\
%   Affiliation / Address line 3 \\
%   \texttt{email@domain} \\\And
%   Second Author \\
%   Affiliation / Address line 1 \\
%   Affiliation / Address line 2 \\
%   Affiliation / Address line 3 \\
%   \texttt{email@domain} \\}
% \author{Haohan Yuan \qquad Haopeng Zhang\thanks{corresponding author} \\ 
%   ALOHA Lab, University of Hawaii at Manoa \\
%   % Affiliation / Address line 2 \\
%   % Affiliation / Address line 3 \\
%   \texttt{\{haohany,haopengz\}@hawaii.edu}}
  
\author{
{Haopeng Zhang$\dag$\thanks{These authors contributed equally to this work.}, Yili Ren$\ddagger$\footnotemark[1], Haohan Yuan$\dag$, Jingzhe Zhang$\ddagger$, Yitong Shen$\ddagger$} \\
ALOHA Lab, University of Hawaii at Manoa$\dag$, University of South Florida$\ddagger$ \\
\{haopengz, haohany\}@hawaii.edu\\
\{yiliren, jingzhe, shen202\}@usf.edu\\}



  
%\author{
%  \textbf{First Author\textsuperscript{1}},
%  \textbf{Second Author\textsuperscript{1,2}},
%  \textbf{Third T. Author\textsuperscript{1}},
%  \textbf{Fourth Author\textsuperscript{1}},
%\\
%  \textbf{Fifth Author\textsuperscript{1,2}},
%  \textbf{Sixth Author\textsuperscript{1}},
%  \textbf{Seventh Author\textsuperscript{1}},
%  \textbf{Eighth Author \textsuperscript{1,2,3,4}},
%\\
%  \textbf{Ninth Author\textsuperscript{1}},
%  \textbf{Tenth Author\textsuperscript{1}},
%  \textbf{Eleventh E. Author\textsuperscript{1,2,3,4,5}},
%  \textbf{Twelfth Author\textsuperscript{1}},
%\\
%  \textbf{Thirteenth Author\textsuperscript{3}},
%  \textbf{Fourteenth F. Author\textsuperscript{2,4}},
%  \textbf{Fifteenth Author\textsuperscript{1}},
%  \textbf{Sixteenth Author\textsuperscript{1}},
%\\
%  \textbf{Seventeenth S. Author\textsuperscript{4,5}},
%  \textbf{Eighteenth Author\textsuperscript{3,4}},
%  \textbf{Nineteenth N. Author\textsuperscript{2,5}},
%  \textbf{Twentieth Author\textsuperscript{1}}
%\\
%\\
%  \textsuperscript{1}Affiliation 1,
%  \textsuperscript{2}Affiliation 2,
%  \textsuperscript{3}Affiliation 3,
%  \textsuperscript{4}Affiliation 4,
%  \textsuperscript{5}Affiliation 5
%\\
%  \small{
%    \textbf{Correspondence:} \href{mailto:email@domain}{email@domain}
%  }
%}

\begin{document}
\maketitle
\begin{abstract}
Recent advancements in Large Language Models (LLMs) have demonstrated remarkable capabilities across diverse tasks. However, their potential to integrate physical model knowledge for real-world signal interpretation remains largely unexplored. In this work, we introduce Wi-Chat, the first LLM-powered Wi-Fi-based human activity recognition system. We demonstrate that LLMs can process raw Wi-Fi signals and infer human activities by incorporating Wi-Fi sensing principles into prompts. Our approach leverages physical model insights to guide LLMs in interpreting Channel State Information (CSI) data without traditional signal processing techniques. Through experiments on real-world Wi-Fi datasets, we show that LLMs exhibit strong reasoning capabilities, achieving zero-shot activity recognition. These findings highlight a new paradigm for Wi-Fi sensing, expanding LLM applications beyond conventional language tasks and enhancing the accessibility of wireless sensing for real-world deployments.
\end{abstract}

\section{Introduction}

In today’s rapidly evolving digital landscape, the transformative power of web technologies has redefined not only how services are delivered but also how complex tasks are approached. Web-based systems have become increasingly prevalent in risk control across various domains. This widespread adoption is due their accessibility, scalability, and ability to remotely connect various types of users. For example, these systems are used for process safety management in industry~\cite{kannan2016web}, safety risk early warning in urban construction~\cite{ding2013development}, and safe monitoring of infrastructural systems~\cite{repetto2018web}. Within these web-based risk management systems, the source search problem presents a huge challenge. Source search refers to the task of identifying the origin of a risky event, such as a gas leak and the emission point of toxic substances. This source search capability is crucial for effective risk management and decision-making.

Traditional approaches to implementing source search capabilities into the web systems often rely on solely algorithmic solutions~\cite{ristic2016study}. These methods, while relatively straightforward to implement, often struggle to achieve acceptable performances due to algorithmic local optima and complex unknown environments~\cite{zhao2020searching}. More recently, web crowdsourcing has emerged as a promising alternative for tackling the source search problem by incorporating human efforts in these web systems on-the-fly~\cite{zhao2024user}. This approach outsources the task of addressing issues encountered during the source search process to human workers, leveraging their capabilities to enhance system performance.

These solutions often employ a human-AI collaborative way~\cite{zhao2023leveraging} where algorithms handle exploration-exploitation and report the encountered problems while human workers resolve complex decision-making bottlenecks to help the algorithms getting rid of local deadlocks~\cite{zhao2022crowd}. Although effective, this paradigm suffers from two inherent limitations: increased operational costs from continuous human intervention, and slow response times of human workers due to sequential decision-making. These challenges motivate our investigation into developing autonomous systems that preserve human-like reasoning capabilities while reducing dependency on massive crowdsourced labor.

Furthermore, recent advancements in large language models (LLMs)~\cite{chang2024survey} and multi-modal LLMs (MLLMs)~\cite{huang2023chatgpt} have unveiled promising avenues for addressing these challenges. One clear opportunity involves the seamless integration of visual understanding and linguistic reasoning for robust decision-making in search tasks. However, whether large models-assisted source search is really effective and efficient for improving the current source search algorithms~\cite{ji2022source} remains unknown. \textit{To address the research gap, we are particularly interested in answering the following two research questions in this work:}

\textbf{\textit{RQ1: }}How can source search capabilities be integrated into web-based systems to support decision-making in time-sensitive risk management scenarios? 
% \sq{I mention ``time-sensitive'' here because I feel like we shall say something about the response time -- LLM has to be faster than humans}

\textbf{\textit{RQ2: }}How can MLLMs and LLMs enhance the effectiveness and efficiency of existing source search algorithms? 

% \textit{\textbf{RQ2:}} To what extent does the performance of large models-assisted search align with or approach the effectiveness of human-AI collaborative search? 

To answer the research questions, we propose a novel framework called Auto-\
S$^2$earch (\textbf{Auto}nomous \textbf{S}ource \textbf{Search}) and implement a prototype system that leverages advanced web technologies to simulate real-world conditions for zero-shot source search. Unlike traditional methods that rely on pre-defined heuristics or extensive human intervention, AutoS$^2$earch employs a carefully designed prompt that encapsulates human rationales, thereby guiding the MLLM to generate coherent and accurate scene descriptions from visual inputs about four directional choices. Based on these language-based descriptions, the LLM is enabled to determine the optimal directional choice through chain-of-thought (CoT) reasoning. Comprehensive empirical validation demonstrates that AutoS$^2$-\ 
earch achieves a success rate of 95–98\%, closely approaching the performance of human-AI collaborative search across 20 benchmark scenarios~\cite{zhao2023leveraging}. 

Our work indicates that the role of humans in future web crowdsourcing tasks may evolve from executors to validators or supervisors. Furthermore, incorporating explanations of LLM decisions into web-based system interfaces has the potential to help humans enhance task performance in risk control.






\section{Related Work}
\label{sec:relatedworks}

% \begin{table*}[t]
% \centering 
% \renewcommand\arraystretch{0.98}
% \fontsize{8}{10}\selectfont \setlength{\tabcolsep}{0.4em}
% \begin{tabular}{@{}lc|cc|cc|cc@{}}
% \toprule
% \textbf{Methods}           & \begin{tabular}[c]{@{}c@{}}\textbf{Training}\\ \textbf{Paradigm}\end{tabular} & \begin{tabular}[c]{@{}c@{}}\textbf{$\#$ PT Data}\\ \textbf{(Tokens)}\end{tabular} & \begin{tabular}[c]{@{}c@{}}\textbf{$\#$ IFT Data}\\ \textbf{(Samples)}\end{tabular} & \textbf{Code}  & \begin{tabular}[c]{@{}c@{}}\textbf{Natural}\\ \textbf{Language}\end{tabular} & \begin{tabular}[c]{@{}c@{}}\textbf{Action}\\ \textbf{Trajectories}\end{tabular} & \begin{tabular}[c]{@{}c@{}}\textbf{API}\\ \textbf{Documentation}\end{tabular}\\ \midrule 
% NexusRaven~\citep{srinivasan2023nexusraven} & IFT & - & - & \textcolor{green}{\CheckmarkBold} & \textcolor{green}{\CheckmarkBold} &\textcolor{red}{\XSolidBrush}&\textcolor{red}{\XSolidBrush}\\
% AgentInstruct~\citep{zeng2023agenttuning} & IFT & - & 2k & \textcolor{green}{\CheckmarkBold} & \textcolor{green}{\CheckmarkBold} &\textcolor{red}{\XSolidBrush}&\textcolor{red}{\XSolidBrush} \\
% AgentEvol~\citep{xi2024agentgym} & IFT & - & 14.5k & \textcolor{green}{\CheckmarkBold} & \textcolor{green}{\CheckmarkBold} &\textcolor{green}{\CheckmarkBold}&\textcolor{red}{\XSolidBrush} \\
% Gorilla~\citep{patil2023gorilla}& IFT & - & 16k & \textcolor{green}{\CheckmarkBold} & \textcolor{green}{\CheckmarkBold} &\textcolor{red}{\XSolidBrush}&\textcolor{green}{\CheckmarkBold}\\
% OpenFunctions-v2~\citep{patil2023gorilla} & IFT & - & 65k & \textcolor{green}{\CheckmarkBold} & \textcolor{green}{\CheckmarkBold} &\textcolor{red}{\XSolidBrush}&\textcolor{green}{\CheckmarkBold}\\
% LAM~\citep{zhang2024agentohana} & IFT & - & 42.6k & \textcolor{green}{\CheckmarkBold} & \textcolor{green}{\CheckmarkBold} &\textcolor{green}{\CheckmarkBold}&\textcolor{red}{\XSolidBrush} \\
% xLAM~\citep{liu2024apigen} & IFT & - & 60k & \textcolor{green}{\CheckmarkBold} & \textcolor{green}{\CheckmarkBold} &\textcolor{green}{\CheckmarkBold}&\textcolor{red}{\XSolidBrush} \\\midrule
% LEMUR~\citep{xu2024lemur} & PT & 90B & 300k & \textcolor{green}{\CheckmarkBold} & \textcolor{green}{\CheckmarkBold} &\textcolor{green}{\CheckmarkBold}&\textcolor{red}{\XSolidBrush}\\
% \rowcolor{teal!12} \method & PT & 103B & 95k & \textcolor{green}{\CheckmarkBold} & \textcolor{green}{\CheckmarkBold} & \textcolor{green}{\CheckmarkBold} & \textcolor{green}{\CheckmarkBold} \\
% \bottomrule
% \end{tabular}
% \caption{Summary of existing tuning- and pretraining-based LLM agents with their training sample sizes. "PT" and "IFT" denote "Pre-Training" and "Instruction Fine-Tuning", respectively. }
% \label{tab:related}
% \end{table*}

\begin{table*}[ht]
\begin{threeparttable}
\centering 
\renewcommand\arraystretch{0.98}
\fontsize{7}{9}\selectfont \setlength{\tabcolsep}{0.2em}
\begin{tabular}{@{}l|c|c|ccc|cc|cc|cccc@{}}
\toprule
\textbf{Methods} & \textbf{Datasets}           & \begin{tabular}[c]{@{}c@{}}\textbf{Training}\\ \textbf{Paradigm}\end{tabular} & \begin{tabular}[c]{@{}c@{}}\textbf{\# PT Data}\\ \textbf{(Tokens)}\end{tabular} & \begin{tabular}[c]{@{}c@{}}\textbf{\# IFT Data}\\ \textbf{(Samples)}\end{tabular} & \textbf{\# APIs} & \textbf{Code}  & \begin{tabular}[c]{@{}c@{}}\textbf{Nat.}\\ \textbf{Lang.}\end{tabular} & \begin{tabular}[c]{@{}c@{}}\textbf{Action}\\ \textbf{Traj.}\end{tabular} & \begin{tabular}[c]{@{}c@{}}\textbf{API}\\ \textbf{Doc.}\end{tabular} & \begin{tabular}[c]{@{}c@{}}\textbf{Func.}\\ \textbf{Call}\end{tabular} & \begin{tabular}[c]{@{}c@{}}\textbf{Multi.}\\ \textbf{Step}\end{tabular}  & \begin{tabular}[c]{@{}c@{}}\textbf{Plan}\\ \textbf{Refine}\end{tabular}  & \begin{tabular}[c]{@{}c@{}}\textbf{Multi.}\\ \textbf{Turn}\end{tabular}\\ \midrule 
\multicolumn{13}{l}{\emph{Instruction Finetuning-based LLM Agents for Intrinsic Reasoning}}  \\ \midrule
FireAct~\cite{chen2023fireact} & FireAct & IFT & - & 2.1K & 10 & \textcolor{red}{\XSolidBrush} &\textcolor{green}{\CheckmarkBold} &\textcolor{green}{\CheckmarkBold}  & \textcolor{red}{\XSolidBrush} &\textcolor{green}{\CheckmarkBold} & \textcolor{red}{\XSolidBrush} &\textcolor{green}{\CheckmarkBold} & \textcolor{red}{\XSolidBrush} \\
ToolAlpaca~\cite{tang2023toolalpaca} & ToolAlpaca & IFT & - & 4.0K & 400 & \textcolor{red}{\XSolidBrush} &\textcolor{green}{\CheckmarkBold} &\textcolor{green}{\CheckmarkBold} & \textcolor{red}{\XSolidBrush} &\textcolor{green}{\CheckmarkBold} & \textcolor{red}{\XSolidBrush}  &\textcolor{green}{\CheckmarkBold} & \textcolor{red}{\XSolidBrush}  \\
ToolLLaMA~\cite{qin2023toolllm} & ToolBench & IFT & - & 12.7K & 16,464 & \textcolor{red}{\XSolidBrush} &\textcolor{green}{\CheckmarkBold} &\textcolor{green}{\CheckmarkBold} &\textcolor{red}{\XSolidBrush} &\textcolor{green}{\CheckmarkBold}&\textcolor{green}{\CheckmarkBold}&\textcolor{green}{\CheckmarkBold} &\textcolor{green}{\CheckmarkBold}\\
AgentEvol~\citep{xi2024agentgym} & AgentTraj-L & IFT & - & 14.5K & 24 &\textcolor{red}{\XSolidBrush} & \textcolor{green}{\CheckmarkBold} &\textcolor{green}{\CheckmarkBold}&\textcolor{red}{\XSolidBrush} &\textcolor{green}{\CheckmarkBold}&\textcolor{red}{\XSolidBrush} &\textcolor{red}{\XSolidBrush} &\textcolor{green}{\CheckmarkBold}\\
Lumos~\cite{yin2024agent} & Lumos & IFT  & - & 20.0K & 16 &\textcolor{red}{\XSolidBrush} & \textcolor{green}{\CheckmarkBold} & \textcolor{green}{\CheckmarkBold} &\textcolor{red}{\XSolidBrush} & \textcolor{green}{\CheckmarkBold} & \textcolor{green}{\CheckmarkBold} &\textcolor{red}{\XSolidBrush} & \textcolor{green}{\CheckmarkBold}\\
Agent-FLAN~\cite{chen2024agent} & Agent-FLAN & IFT & - & 24.7K & 20 &\textcolor{red}{\XSolidBrush} & \textcolor{green}{\CheckmarkBold} & \textcolor{green}{\CheckmarkBold} &\textcolor{red}{\XSolidBrush} & \textcolor{green}{\CheckmarkBold}& \textcolor{green}{\CheckmarkBold}&\textcolor{red}{\XSolidBrush} & \textcolor{green}{\CheckmarkBold}\\
AgentTuning~\citep{zeng2023agenttuning} & AgentInstruct & IFT & - & 35.0K & - &\textcolor{red}{\XSolidBrush} & \textcolor{green}{\CheckmarkBold} & \textcolor{green}{\CheckmarkBold} &\textcolor{red}{\XSolidBrush} & \textcolor{green}{\CheckmarkBold} &\textcolor{red}{\XSolidBrush} &\textcolor{red}{\XSolidBrush} & \textcolor{green}{\CheckmarkBold}\\\midrule
\multicolumn{13}{l}{\emph{Instruction Finetuning-based LLM Agents for Function Calling}} \\\midrule
NexusRaven~\citep{srinivasan2023nexusraven} & NexusRaven & IFT & - & - & 116 & \textcolor{green}{\CheckmarkBold} & \textcolor{green}{\CheckmarkBold}  & \textcolor{green}{\CheckmarkBold} &\textcolor{red}{\XSolidBrush} & \textcolor{green}{\CheckmarkBold} &\textcolor{red}{\XSolidBrush} &\textcolor{red}{\XSolidBrush}&\textcolor{red}{\XSolidBrush}\\
Gorilla~\citep{patil2023gorilla} & Gorilla & IFT & - & 16.0K & 1,645 & \textcolor{green}{\CheckmarkBold} &\textcolor{red}{\XSolidBrush} &\textcolor{red}{\XSolidBrush}&\textcolor{green}{\CheckmarkBold} &\textcolor{green}{\CheckmarkBold} &\textcolor{red}{\XSolidBrush} &\textcolor{red}{\XSolidBrush} &\textcolor{red}{\XSolidBrush}\\
OpenFunctions-v2~\citep{patil2023gorilla} & OpenFunctions-v2 & IFT & - & 65.0K & - & \textcolor{green}{\CheckmarkBold} & \textcolor{green}{\CheckmarkBold} &\textcolor{red}{\XSolidBrush} &\textcolor{green}{\CheckmarkBold} &\textcolor{green}{\CheckmarkBold} &\textcolor{red}{\XSolidBrush} &\textcolor{red}{\XSolidBrush} &\textcolor{red}{\XSolidBrush}\\
API Pack~\cite{guo2024api} & API Pack & IFT & - & 1.1M & 11,213 &\textcolor{green}{\CheckmarkBold} &\textcolor{red}{\XSolidBrush} &\textcolor{green}{\CheckmarkBold} &\textcolor{red}{\XSolidBrush} &\textcolor{green}{\CheckmarkBold} &\textcolor{red}{\XSolidBrush}&\textcolor{red}{\XSolidBrush}&\textcolor{red}{\XSolidBrush}\\ 
LAM~\citep{zhang2024agentohana} & AgentOhana & IFT & - & 42.6K & - & \textcolor{green}{\CheckmarkBold} & \textcolor{green}{\CheckmarkBold} &\textcolor{green}{\CheckmarkBold}&\textcolor{red}{\XSolidBrush} &\textcolor{green}{\CheckmarkBold}&\textcolor{red}{\XSolidBrush}&\textcolor{green}{\CheckmarkBold}&\textcolor{green}{\CheckmarkBold}\\
xLAM~\citep{liu2024apigen} & APIGen & IFT & - & 60.0K & 3,673 & \textcolor{green}{\CheckmarkBold} & \textcolor{green}{\CheckmarkBold} &\textcolor{green}{\CheckmarkBold}&\textcolor{red}{\XSolidBrush} &\textcolor{green}{\CheckmarkBold}&\textcolor{red}{\XSolidBrush}&\textcolor{green}{\CheckmarkBold}&\textcolor{green}{\CheckmarkBold}\\\midrule
\multicolumn{13}{l}{\emph{Pretraining-based LLM Agents}}  \\\midrule
% LEMUR~\citep{xu2024lemur} & PT & 90B & 300.0K & - & \textcolor{green}{\CheckmarkBold} & \textcolor{green}{\CheckmarkBold} &\textcolor{green}{\CheckmarkBold}&\textcolor{red}{\XSolidBrush} & \textcolor{red}{\XSolidBrush} &\textcolor{green}{\CheckmarkBold} &\textcolor{red}{\XSolidBrush}&\textcolor{red}{\XSolidBrush}\\
\rowcolor{teal!12} \method & \dataset & PT & 103B & 95.0K  & 76,537  & \textcolor{green}{\CheckmarkBold} & \textcolor{green}{\CheckmarkBold} & \textcolor{green}{\CheckmarkBold} & \textcolor{green}{\CheckmarkBold} & \textcolor{green}{\CheckmarkBold} & \textcolor{green}{\CheckmarkBold} & \textcolor{green}{\CheckmarkBold} & \textcolor{green}{\CheckmarkBold}\\
\bottomrule
\end{tabular}
% \begin{tablenotes}
%     \item $^*$ In addition, the StarCoder-API can offer 4.77M more APIs.
% \end{tablenotes}
\caption{Summary of existing instruction finetuning-based LLM agents for intrinsic reasoning and function calling, along with their training resources and sample sizes. "PT" and "IFT" denote "Pre-Training" and "Instruction Fine-Tuning", respectively.}
\vspace{-2ex}
\label{tab:related}
\end{threeparttable}
\end{table*}

\noindent \textbf{Prompting-based LLM Agents.} Due to the lack of agent-specific pre-training corpus, existing LLM agents rely on either prompt engineering~\cite{hsieh2023tool,lu2024chameleon,yao2022react,wang2023voyager} or instruction fine-tuning~\cite{chen2023fireact,zeng2023agenttuning} to understand human instructions, decompose high-level tasks, generate grounded plans, and execute multi-step actions. 
However, prompting-based methods mainly depend on the capabilities of backbone LLMs (usually commercial LLMs), failing to introduce new knowledge and struggling to generalize to unseen tasks~\cite{sun2024adaplanner,zhuang2023toolchain}. 

\noindent \textbf{Instruction Finetuning-based LLM Agents.} Considering the extensive diversity of APIs and the complexity of multi-tool instructions, tool learning inherently presents greater challenges than natural language tasks, such as text generation~\cite{qin2023toolllm}.
Post-training techniques focus more on instruction following and aligning output with specific formats~\cite{patil2023gorilla,hao2024toolkengpt,qin2023toolllm,schick2024toolformer}, rather than fundamentally improving model knowledge or capabilities. 
Moreover, heavy fine-tuning can hinder generalization or even degrade performance in non-agent use cases, potentially suppressing the original base model capabilities~\cite{ghosh2024a}.

\noindent \textbf{Pretraining-based LLM Agents.} While pre-training serves as an essential alternative, prior works~\cite{nijkamp2023codegen,roziere2023code,xu2024lemur,patil2023gorilla} have primarily focused on improving task-specific capabilities (\eg, code generation) instead of general-domain LLM agents, due to single-source, uni-type, small-scale, and poor-quality pre-training data. 
Existing tool documentation data for agent training either lacks diverse real-world APIs~\cite{patil2023gorilla, tang2023toolalpaca} or is constrained to single-tool or single-round tool execution. 
Furthermore, trajectory data mostly imitate expert behavior or follow function-calling rules with inferior planning and reasoning, failing to fully elicit LLMs' capabilities and handle complex instructions~\cite{qin2023toolllm}. 
Given a wide range of candidate API functions, each comprising various function names and parameters available at every planning step, identifying globally optimal solutions and generalizing across tasks remains highly challenging.



\section{Preliminaries}
\label{Preliminaries}
\begin{figure*}[t]
    \centering
    \includegraphics[width=0.95\linewidth]{fig/HealthGPT_Framework.png}
    \caption{The \ourmethod{} architecture integrates hierarchical visual perception and H-LoRA, employing a task-specific hard router to select visual features and H-LoRA plugins, ultimately generating outputs with an autoregressive manner.}
    \label{fig:architecture}
\end{figure*}
\noindent\textbf{Large Vision-Language Models.} 
The input to a LVLM typically consists of an image $x^{\text{img}}$ and a discrete text sequence $x^{\text{txt}}$. The visual encoder $\mathcal{E}^{\text{img}}$ converts the input image $x^{\text{img}}$ into a sequence of visual tokens $\mathcal{V} = [v_i]_{i=1}^{N_v}$, while the text sequence $x^{\text{txt}}$ is mapped into a sequence of text tokens $\mathcal{T} = [t_i]_{i=1}^{N_t}$ using an embedding function $\mathcal{E}^{\text{txt}}$. The LLM $\mathcal{M_\text{LLM}}(\cdot|\theta)$ models the joint probability of the token sequence $\mathcal{U} = \{\mathcal{V},\mathcal{T}\}$, which is expressed as:
\begin{equation}
    P_\theta(R | \mathcal{U}) = \prod_{i=1}^{N_r} P_\theta(r_i | \{\mathcal{U}, r_{<i}\}),
\end{equation}
where $R = [r_i]_{i=1}^{N_r}$ is the text response sequence. The LVLM iteratively generates the next token $r_i$ based on $r_{<i}$. The optimization objective is to minimize the cross-entropy loss of the response $\mathcal{R}$.
% \begin{equation}
%     \mathcal{L}_{\text{VLM}} = \mathbb{E}_{R|\mathcal{U}}\left[-\log P_\theta(R | \mathcal{U})\right]
% \end{equation}
It is worth noting that most LVLMs adopt a design paradigm based on ViT, alignment adapters, and pre-trained LLMs\cite{liu2023llava,liu2024improved}, enabling quick adaptation to downstream tasks.


\noindent\textbf{VQGAN.}
VQGAN~\cite{esser2021taming} employs latent space compression and indexing mechanisms to effectively learn a complete discrete representation of images. VQGAN first maps the input image $x^{\text{img}}$ to a latent representation $z = \mathcal{E}(x)$ through a encoder $\mathcal{E}$. Then, the latent representation is quantized using a codebook $\mathcal{Z} = \{z_k\}_{k=1}^K$, generating a discrete index sequence $\mathcal{I} = [i_m]_{m=1}^N$, where $i_m \in \mathcal{Z}$ represents the quantized code index:
\begin{equation}
    \mathcal{I} = \text{Quantize}(z|\mathcal{Z}) = \arg\min_{z_k \in \mathcal{Z}} \| z - z_k \|_2.
\end{equation}
In our approach, the discrete index sequence $\mathcal{I}$ serves as a supervisory signal for the generation task, enabling the model to predict the index sequence $\hat{\mathcal{I}}$ from input conditions such as text or other modality signals.  
Finally, the predicted index sequence $\hat{\mathcal{I}}$ is upsampled by the VQGAN decoder $G$, generating the high-quality image $\hat{x}^\text{img} = G(\hat{\mathcal{I}})$.



\noindent\textbf{Low Rank Adaptation.} 
LoRA\cite{hu2021lora} effectively captures the characteristics of downstream tasks by introducing low-rank adapters. The core idea is to decompose the bypass weight matrix $\Delta W\in\mathbb{R}^{d^{\text{in}} \times d^{\text{out}}}$ into two low-rank matrices $ \{A \in \mathbb{R}^{d^{\text{in}} \times r}, B \in \mathbb{R}^{r \times d^{\text{out}}} \}$, where $ r \ll \min\{d^{\text{in}}, d^{\text{out}}\} $, significantly reducing learnable parameters. The output with the LoRA adapter for the input $x$ is then given by:
\begin{equation}
    h = x W_0 + \alpha x \Delta W/r = x W_0 + \alpha xAB/r,
\end{equation}
where matrix $ A $ is initialized with a Gaussian distribution, while the matrix $ B $ is initialized as a zero matrix. The scaling factor $ \alpha/r $ controls the impact of $ \Delta W $ on the model.

\section{HealthGPT}
\label{Method}


\subsection{Unified Autoregressive Generation.}  
% As shown in Figure~\ref{fig:architecture}, 
\ourmethod{} (Figure~\ref{fig:architecture}) utilizes a discrete token representation that covers both text and visual outputs, unifying visual comprehension and generation as an autoregressive task. 
For comprehension, $\mathcal{M}_\text{llm}$ receives the input joint sequence $\mathcal{U}$ and outputs a series of text token $\mathcal{R} = [r_1, r_2, \dots, r_{N_r}]$, where $r_i \in \mathcal{V}_{\text{txt}}$, and $\mathcal{V}_{\text{txt}}$ represents the LLM's vocabulary:
\begin{equation}
    P_\theta(\mathcal{R} \mid \mathcal{U}) = \prod_{i=1}^{N_r} P_\theta(r_i \mid \mathcal{U}, r_{<i}).
\end{equation}
For generation, $\mathcal{M}_\text{llm}$ first receives a special start token $\langle \text{START\_IMG} \rangle$, then generates a series of tokens corresponding to the VQGAN indices $\mathcal{I} = [i_1, i_2, \dots, i_{N_i}]$, where $i_j \in \mathcal{V}_{\text{vq}}$, and $\mathcal{V}_{\text{vq}}$ represents the index range of VQGAN. Upon completion of generation, the LLM outputs an end token $\langle \text{END\_IMG} \rangle$:
\begin{equation}
    P_\theta(\mathcal{I} \mid \mathcal{U}) = \prod_{j=1}^{N_i} P_\theta(i_j \mid \mathcal{U}, i_{<j}).
\end{equation}
Finally, the generated index sequence $\mathcal{I}$ is fed into the decoder $G$, which reconstructs the target image $\hat{x}^{\text{img}} = G(\mathcal{I})$.

\subsection{Hierarchical Visual Perception}  
Given the differences in visual perception between comprehension and generation tasks—where the former focuses on abstract semantics and the latter emphasizes complete semantics—we employ ViT to compress the image into discrete visual tokens at multiple hierarchical levels.
Specifically, the image is converted into a series of features $\{f_1, f_2, \dots, f_L\}$ as it passes through $L$ ViT blocks.

To address the needs of various tasks, the hidden states are divided into two types: (i) \textit{Concrete-grained features} $\mathcal{F}^{\text{Con}} = \{f_1, f_2, \dots, f_k\}, k < L$, derived from the shallower layers of ViT, containing sufficient global features, suitable for generation tasks; 
(ii) \textit{Abstract-grained features} $\mathcal{F}^{\text{Abs}} = \{f_{k+1}, f_{k+2}, \dots, f_L\}$, derived from the deeper layers of ViT, which contain abstract semantic information closer to the text space, suitable for comprehension tasks.

The task type $T$ (comprehension or generation) determines which set of features is selected as the input for the downstream large language model:
\begin{equation}
    \mathcal{F}^{\text{img}}_T =
    \begin{cases}
        \mathcal{F}^{\text{Con}}, & \text{if } T = \text{generation task} \\
        \mathcal{F}^{\text{Abs}}, & \text{if } T = \text{comprehension task}
    \end{cases}
\end{equation}
We integrate the image features $\mathcal{F}^{\text{img}}_T$ and text features $\mathcal{T}$ into a joint sequence through simple concatenation, which is then fed into the LLM $\mathcal{M}_{\text{llm}}$ for autoregressive generation.
% :
% \begin{equation}
%     \mathcal{R} = \mathcal{M}_{\text{llm}}(\mathcal{U}|\theta), \quad \mathcal{U} = [\mathcal{F}^{\text{img}}_T; \mathcal{T}]
% \end{equation}
\subsection{Heterogeneous Knowledge Adaptation}
We devise H-LoRA, which stores heterogeneous knowledge from comprehension and generation tasks in separate modules and dynamically routes to extract task-relevant knowledge from these modules. 
At the task level, for each task type $ T $, we dynamically assign a dedicated H-LoRA submodule $ \theta^T $, which is expressed as:
\begin{equation}
    \mathcal{R} = \mathcal{M}_\text{LLM}(\mathcal{U}|\theta, \theta^T), \quad \theta^T = \{A^T, B^T, \mathcal{R}^T_\text{outer}\}.
\end{equation}
At the feature level for a single task, H-LoRA integrates the idea of Mixture of Experts (MoE)~\cite{masoudnia2014mixture} and designs an efficient matrix merging and routing weight allocation mechanism, thus avoiding the significant computational delay introduced by matrix splitting in existing MoELoRA~\cite{luo2024moelora}. Specifically, we first merge the low-rank matrices (rank = r) of $ k $ LoRA experts into a unified matrix:
\begin{equation}
    \mathbf{A}^{\text{merged}}, \mathbf{B}^{\text{merged}} = \text{Concat}(\{A_i\}_1^k), \text{Concat}(\{B_i\}_1^k),
\end{equation}
where $ \mathbf{A}^{\text{merged}} \in \mathbb{R}^{d^\text{in} \times rk} $ and $ \mathbf{B}^{\text{merged}} \in \mathbb{R}^{rk \times d^\text{out}} $. The $k$-dimension routing layer generates expert weights $ \mathcal{W} \in \mathbb{R}^{\text{token\_num} \times k} $ based on the input hidden state $ x $, and these are expanded to $ \mathbb{R}^{\text{token\_num} \times rk} $ as follows:
\begin{equation}
    \mathcal{W}^\text{expanded} = \alpha k \mathcal{W} / r \otimes \mathbf{1}_r,
\end{equation}
where $ \otimes $ denotes the replication operation.
The overall output of H-LoRA is computed as:
\begin{equation}
    \mathcal{O}^\text{H-LoRA} = (x \mathbf{A}^{\text{merged}} \odot \mathcal{W}^\text{expanded}) \mathbf{B}^{\text{merged}},
\end{equation}
where $ \odot $ represents element-wise multiplication. Finally, the output of H-LoRA is added to the frozen pre-trained weights to produce the final output:
\begin{equation}
    \mathcal{O} = x W_0 + \mathcal{O}^\text{H-LoRA}.
\end{equation}
% In summary, H-LoRA is a task-based dynamic PEFT method that achieves high efficiency in single-task fine-tuning.

\subsection{Training Pipeline}

\begin{figure}[t]
    \centering
    \hspace{-4mm}
    \includegraphics[width=0.94\linewidth]{fig/data.pdf}
    \caption{Data statistics of \texttt{VL-Health}. }
    \label{fig:data}
\end{figure}
\noindent \textbf{1st Stage: Multi-modal Alignment.} 
In the first stage, we design separate visual adapters and H-LoRA submodules for medical unified tasks. For the medical comprehension task, we train abstract-grained visual adapters using high-quality image-text pairs to align visual embeddings with textual embeddings, thereby enabling the model to accurately describe medical visual content. During this process, the pre-trained LLM and its corresponding H-LoRA submodules remain frozen. In contrast, the medical generation task requires training concrete-grained adapters and H-LoRA submodules while keeping the LLM frozen. Meanwhile, we extend the textual vocabulary to include multimodal tokens, enabling the support of additional VQGAN vector quantization indices. The model trains on image-VQ pairs, endowing the pre-trained LLM with the capability for image reconstruction. This design ensures pixel-level consistency of pre- and post-LVLM. The processes establish the initial alignment between the LLM’s outputs and the visual inputs.

\noindent \textbf{2nd Stage: Heterogeneous H-LoRA Plugin Adaptation.}  
The submodules of H-LoRA share the word embedding layer and output head but may encounter issues such as bias and scale inconsistencies during training across different tasks. To ensure that the multiple H-LoRA plugins seamlessly interface with the LLMs and form a unified base, we fine-tune the word embedding layer and output head using a small amount of mixed data to maintain consistency in the model weights. Specifically, during this stage, all H-LoRA submodules for different tasks are kept frozen, with only the word embedding layer and output head being optimized. Through this stage, the model accumulates foundational knowledge for unified tasks by adapting H-LoRA plugins.

\begin{table*}[!t]
\centering
\caption{Comparison of \ourmethod{} with other LVLMs and unified multi-modal models on medical visual comprehension tasks. \textbf{Bold} and \underline{underlined} text indicates the best performance and second-best performance, respectively.}
\resizebox{\textwidth}{!}{
\begin{tabular}{c|lcc|cccccccc|c}
\toprule
\rowcolor[HTML]{E9F3FE} &  &  &  & \multicolumn{2}{c}{\textbf{VQA-RAD \textuparrow}} & \multicolumn{2}{c}{\textbf{SLAKE \textuparrow}} & \multicolumn{2}{c}{\textbf{PathVQA \textuparrow}} &  &  &  \\ 
\cline{5-10}
\rowcolor[HTML]{E9F3FE}\multirow{-2}{*}{\textbf{Type}} & \multirow{-2}{*}{\textbf{Model}} & \multirow{-2}{*}{\textbf{\# Params}} & \multirow{-2}{*}{\makecell{\textbf{Medical} \\ \textbf{LVLM}}} & \textbf{close} & \textbf{all} & \textbf{close} & \textbf{all} & \textbf{close} & \textbf{all} & \multirow{-2}{*}{\makecell{\textbf{MMMU} \\ \textbf{-Med}}\textuparrow} & \multirow{-2}{*}{\textbf{OMVQA}\textuparrow} & \multirow{-2}{*}{\textbf{Avg. \textuparrow}} \\ 
\midrule \midrule
\multirow{9}{*}{\textbf{Comp. Only}} 
& Med-Flamingo & 8.3B & \Large \ding{51} & 58.6 & 43.0 & 47.0 & 25.5 & 61.9 & 31.3 & 28.7 & 34.9 & 41.4 \\
& LLaVA-Med & 7B & \Large \ding{51} & 60.2 & 48.1 & 58.4 & 44.8 & 62.3 & 35.7 & 30.0 & 41.3 & 47.6 \\
& HuatuoGPT-Vision & 7B & \Large \ding{51} & 66.9 & 53.0 & 59.8 & 49.1 & 52.9 & 32.0 & 42.0 & 50.0 & 50.7 \\
& BLIP-2 & 6.7B & \Large \ding{55} & 43.4 & 36.8 & 41.6 & 35.3 & 48.5 & 28.8 & 27.3 & 26.9 & 36.1 \\
& LLaVA-v1.5 & 7B & \Large \ding{55} & 51.8 & 42.8 & 37.1 & 37.7 & 53.5 & 31.4 & 32.7 & 44.7 & 41.5 \\
& InstructBLIP & 7B & \Large \ding{55} & 61.0 & 44.8 & 66.8 & 43.3 & 56.0 & 32.3 & 25.3 & 29.0 & 44.8 \\
& Yi-VL & 6B & \Large \ding{55} & 52.6 & 42.1 & 52.4 & 38.4 & 54.9 & 30.9 & 38.0 & 50.2 & 44.9 \\
& InternVL2 & 8B & \Large \ding{55} & 64.9 & 49.0 & 66.6 & 50.1 & 60.0 & 31.9 & \underline{43.3} & 54.5 & 52.5\\
& Llama-3.2 & 11B & \Large \ding{55} & 68.9 & 45.5 & 72.4 & 52.1 & 62.8 & 33.6 & 39.3 & 63.2 & 54.7 \\
\midrule
\multirow{5}{*}{\textbf{Comp. \& Gen.}} 
& Show-o & 1.3B & \Large \ding{55} & 50.6 & 33.9 & 31.5 & 17.9 & 52.9 & 28.2 & 22.7 & 45.7 & 42.6 \\
& Unified-IO 2 & 7B & \Large \ding{55} & 46.2 & 32.6 & 35.9 & 21.9 & 52.5 & 27.0 & 25.3 & 33.0 & 33.8 \\
& Janus & 1.3B & \Large \ding{55} & 70.9 & 52.8 & 34.7 & 26.9 & 51.9 & 27.9 & 30.0 & 26.8 & 33.5 \\
& \cellcolor[HTML]{DAE0FB}HealthGPT-M3 & \cellcolor[HTML]{DAE0FB}3.8B & \cellcolor[HTML]{DAE0FB}\Large \ding{51} & \cellcolor[HTML]{DAE0FB}\underline{73.7} & \cellcolor[HTML]{DAE0FB}\underline{55.9} & \cellcolor[HTML]{DAE0FB}\underline{74.6} & \cellcolor[HTML]{DAE0FB}\underline{56.4} & \cellcolor[HTML]{DAE0FB}\underline{78.7} & \cellcolor[HTML]{DAE0FB}\underline{39.7} & \cellcolor[HTML]{DAE0FB}\underline{43.3} & \cellcolor[HTML]{DAE0FB}\underline{68.5} & \cellcolor[HTML]{DAE0FB}\underline{61.3} \\
& \cellcolor[HTML]{DAE0FB}HealthGPT-L14 & \cellcolor[HTML]{DAE0FB}14B & \cellcolor[HTML]{DAE0FB}\Large \ding{51} & \cellcolor[HTML]{DAE0FB}\textbf{77.7} & \cellcolor[HTML]{DAE0FB}\textbf{58.3} & \cellcolor[HTML]{DAE0FB}\textbf{76.4} & \cellcolor[HTML]{DAE0FB}\textbf{64.5} & \cellcolor[HTML]{DAE0FB}\textbf{85.9} & \cellcolor[HTML]{DAE0FB}\textbf{44.4} & \cellcolor[HTML]{DAE0FB}\textbf{49.2} & \cellcolor[HTML]{DAE0FB}\textbf{74.4} & \cellcolor[HTML]{DAE0FB}\textbf{66.4} \\
\bottomrule
\end{tabular}
}
\label{tab:results}
\end{table*}
\begin{table*}[ht]
    \centering
    \caption{The experimental results for the four modality conversion tasks.}
    \resizebox{\textwidth}{!}{
    \begin{tabular}{l|ccc|ccc|ccc|ccc}
        \toprule
        \rowcolor[HTML]{E9F3FE} & \multicolumn{3}{c}{\textbf{CT to MRI (Brain)}} & \multicolumn{3}{c}{\textbf{CT to MRI (Pelvis)}} & \multicolumn{3}{c}{\textbf{MRI to CT (Brain)}} & \multicolumn{3}{c}{\textbf{MRI to CT (Pelvis)}} \\
        \cline{2-13}
        \rowcolor[HTML]{E9F3FE}\multirow{-2}{*}{\textbf{Model}}& \textbf{SSIM $\uparrow$} & \textbf{PSNR $\uparrow$} & \textbf{MSE $\downarrow$} & \textbf{SSIM $\uparrow$} & \textbf{PSNR $\uparrow$} & \textbf{MSE $\downarrow$} & \textbf{SSIM $\uparrow$} & \textbf{PSNR $\uparrow$} & \textbf{MSE $\downarrow$} & \textbf{SSIM $\uparrow$} & \textbf{PSNR $\uparrow$} & \textbf{MSE $\downarrow$} \\
        \midrule \midrule
        pix2pix & 71.09 & 32.65 & 36.85 & 59.17 & 31.02 & 51.91 & 78.79 & 33.85 & 28.33 & 72.31 & 32.98 & 36.19 \\
        CycleGAN & 54.76 & 32.23 & 40.56 & 54.54 & 30.77 & 55.00 & 63.75 & 31.02 & 52.78 & 50.54 & 29.89 & 67.78 \\
        BBDM & {71.69} & {32.91} & {34.44} & 57.37 & 31.37 & 48.06 & \textbf{86.40} & 34.12 & 26.61 & {79.26} & 33.15 & 33.60 \\
        Vmanba & 69.54 & 32.67 & 36.42 & {63.01} & {31.47} & {46.99} & 79.63 & 34.12 & 26.49 & 77.45 & 33.53 & 31.85 \\
        DiffMa & 71.47 & 32.74 & 35.77 & 62.56 & 31.43 & 47.38 & 79.00 & {34.13} & {26.45} & 78.53 & {33.68} & {30.51} \\
        \rowcolor[HTML]{DAE0FB}HealthGPT-M3 & \underline{79.38} & \underline{33.03} & \underline{33.48} & \underline{71.81} & \underline{31.83} & \underline{43.45} & {85.06} & \textbf{34.40} & \textbf{25.49} & \underline{84.23} & \textbf{34.29} & \textbf{27.99} \\
        \rowcolor[HTML]{DAE0FB}HealthGPT-L14 & \textbf{79.73} & \textbf{33.10} & \textbf{32.96} & \textbf{71.92} & \textbf{31.87} & \textbf{43.09} & \underline{85.31} & \underline{34.29} & \underline{26.20} & \textbf{84.96} & \underline{34.14} & \underline{28.13} \\
        \bottomrule
    \end{tabular}
    }
    \label{tab:conversion}
\end{table*}

\noindent \textbf{3rd Stage: Visual Instruction Fine-Tuning.}  
In the third stage, we introduce additional task-specific data to further optimize the model and enhance its adaptability to downstream tasks such as medical visual comprehension (e.g., medical QA, medical dialogues, and report generation) or generation tasks (e.g., super-resolution, denoising, and modality conversion). Notably, by this stage, the word embedding layer and output head have been fine-tuned, only the H-LoRA modules and adapter modules need to be trained. This strategy significantly improves the model's adaptability and flexibility across different tasks.


\section{Experiment}
\label{s:experiment}

\subsection{Data Description}
We evaluate our method on FI~\cite{you2016building}, Twitter\_LDL~\cite{yang2017learning} and Artphoto~\cite{machajdik2010affective}.
FI is a public dataset built from Flickr and Instagram, with 23,308 images and eight emotion categories, namely \textit{amusement}, \textit{anger}, \textit{awe},  \textit{contentment}, \textit{disgust}, \textit{excitement},  \textit{fear}, and \textit{sadness}. 
% Since images in FI are all copyrighted by law, some images are corrupted now, so we remove these samples and retain 21,828 images.
% T4SA contains images from Twitter, which are classified into three categories: \textit{positive}, \textit{neutral}, and \textit{negative}. In this paper, we adopt the base version of B-T4SA, which contains 470,586 images and provides text descriptions of the corresponding tweets.
Twitter\_LDL contains 10,045 images from Twitter, with the same eight categories as the FI dataset.
% 。
For these two datasets, they are randomly split into 80\%
training and 20\% testing set.
Artphoto contains 806 artistic photos from the DeviantArt website, which we use to further evaluate the zero-shot capability of our model.
% on the small-scale dataset.
% We construct and publicly release the first image sentiment analysis dataset containing metadata.
% 。

% Based on these datasets, we are the first to construct and publicly release metadata-enhanced image sentiment analysis datasets. These datasets include scenes, tags, descriptions, and corresponding confidence scores, and are available at this link for future research purposes.


% 
\begin{table}[t]
\centering
% \begin{center}
\caption{Overall performance of different models on FI and Twitter\_LDL datasets.}
\label{tab:cap1}
% \resizebox{\linewidth}{!}
{
\begin{tabular}{l|c|c|c|c}
\hline
\multirow{2}{*}{\textbf{Model}} & \multicolumn{2}{c|}{\textbf{FI}}  & \multicolumn{2}{c}{\textbf{Twitter\_LDL}} \\ \cline{2-5} 
  & \textbf{Accuracy} & \textbf{F1} & \textbf{Accuracy} & \textbf{F1}  \\ \hline
% (\rownumber)~AlexNet~\cite{krizhevsky2017imagenet}  & 58.13\% & 56.35\%  & 56.24\%& 55.02\%  \\ 
% (\rownumber)~VGG16~\cite{simonyan2014very}  & 63.75\%& 63.08\%  & 59.34\%& 59.02\%  \\ 
(\rownumber)~ResNet101~\cite{he2016deep} & 66.16\%& 65.56\%  & 62.02\% & 61.34\%  \\ 
(\rownumber)~CDA~\cite{han2023boosting} & 66.71\%& 65.37\%  & 64.14\% & 62.85\%  \\ 
(\rownumber)~CECCN~\cite{ruan2024color} & 67.96\%& 66.74\%  & 64.59\%& 64.72\% \\ 
(\rownumber)~EmoVIT~\cite{xie2024emovit} & 68.09\%& 67.45\%  & 63.12\% & 61.97\%  \\ 
(\rownumber)~ComLDL~\cite{zhang2022compound} & 68.83\%& 67.28\%  & 65.29\% & 63.12\%  \\ 
(\rownumber)~WSDEN~\cite{li2023weakly} & 69.78\%& 69.61\%  & 67.04\% & 65.49\% \\ 
(\rownumber)~ECWA~\cite{deng2021emotion} & 70.87\%& 69.08\%  & 67.81\% & 66.87\%  \\ 
(\rownumber)~EECon~\cite{yang2023exploiting} & 71.13\%& 68.34\%  & 64.27\%& 63.16\%  \\ 
(\rownumber)~MAM~\cite{zhang2024affective} & 71.44\%  & 70.83\% & 67.18\%  & 65.01\%\\ 
(\rownumber)~TGCA-PVT~\cite{chen2024tgca}   & 73.05\%  & 71.46\% & 69.87\%  & 68.32\% \\ 
(\rownumber)~OEAN~\cite{zhang2024object}   & 73.40\%  & 72.63\% & 70.52\%  & 69.47\% \\ \hline
(\rownumber)~\shortname  & \textbf{79.48\%} & \textbf{79.22\%} & \textbf{74.12\%} & \textbf{73.09\%} \\ \hline
\end{tabular}
}
\vspace{-6mm}
% \end{center}
\end{table}
% 

\subsection{Experiment Setting}
% \subsubsection{Model Setting.}
% 
\textbf{Model Setting:}
For feature representation, we set $k=10$ to select object tags, and adopt clip-vit-base-patch32 as the pre-trained model for unified feature representation.
Moreover, we empirically set $(d_e, d_h, d_k, d_s) = (512, 128, 16, 64)$, and set the classification class $L$ to 8.

% 

\textbf{Training Setting:}
To initialize the model, we set all weights such as $\boldsymbol{W}$ following the truncated normal distribution, and use AdamW optimizer with the learning rate of $1 \times 10^{-4}$.
% warmup scheduler of cosine, warmup steps of 2000.
Furthermore, we set the batch size to 32 and the epoch of the training process to 200.
During the implementation, we utilize \textit{PyTorch} to build our entire model.
% , and our project codes are publicly available at https://github.com/zzmyrep/MESN.
% Our project codes as well as data are all publicly available on GitHub\footnote{https://github.com/zzmyrep/KBCEN}.
% Code is available at \href{https://github.com/zzmyrep/KBCEN}{https://github.com/zzmyrep/KBCEN}.

\textbf{Evaluation Metrics:}
Following~\cite{zhang2024affective, chen2024tgca, zhang2024object}, we adopt \textit{accuracy} and \textit{F1} as our evaluation metrics to measure the performance of different methods for image sentiment analysis. 



\subsection{Experiment Result}
% We compare our model against the following baselines: AlexNet~\cite{krizhevsky2017imagenet}, VGG16~\cite{simonyan2014very}, ResNet101~\cite{he2016deep}, CECCN~\cite{ruan2024color}, EmoVIT~\cite{xie2024emovit}, WSCNet~\cite{yang2018weakly}, ECWA~\cite{deng2021emotion}, EECon~\cite{yang2023exploiting}, MAM~\cite{zhang2024affective} and TGCA-PVT~\cite{chen2024tgca}, and the overall results are summarized in Table~\ref{tab:cap1}.
We compare our model against several baselines, and the overall results are summarized in Table~\ref{tab:cap1}.
We observe that our model achieves the best performance in both accuracy and F1 metrics, significantly outperforming the previous models. 
This superior performance is mainly attributed to our effective utilization of metadata to enhance image sentiment analysis, as well as the exceptional capability of the unified sentiment transformer framework we developed. These results strongly demonstrate that our proposed method can bring encouraging performance for image sentiment analysis.

\setcounter{magicrownumbers}{0} 
\begin{table}[t]
\begin{center}
\caption{Ablation study of~\shortname~on FI dataset.} 
% \vspace{1mm}
\label{tab:cap2}
\resizebox{.9\linewidth}{!}
{
\begin{tabular}{lcc}
  \hline
  \textbf{Model} & \textbf{Accuracy} & \textbf{F1} \\
  \hline
  (\rownumber)~Ours (w/o vision) & 65.72\% & 64.54\% \\
  (\rownumber)~Ours (w/o text description) & 74.05\% & 72.58\% \\
  (\rownumber)~Ours (w/o object tag) & 77.45\% & 76.84\% \\
  (\rownumber)~Ours (w/o scene tag) & 78.47\% & 78.21\% \\
  \hline
  (\rownumber)~Ours (w/o unified embedding) & 76.41\% & 76.23\% \\
  (\rownumber)~Ours (w/o adaptive learning) & 76.83\% & 76.56\% \\
  (\rownumber)~Ours (w/o cross-modal fusion) & 76.85\% & 76.49\% \\
  \hline
  (\rownumber)~Ours  & \textbf{79.48\%} & \textbf{79.22\%} \\
  \hline
\end{tabular}
}
\end{center}
\vspace{-5mm}
\end{table}


\begin{figure}[t]
\centering
% \vspace{-2mm}
\includegraphics[width=0.42\textwidth]{fig/2dvisual-linux4-paper2.pdf}
\caption{Visualization of feature distribution on eight categories before (left) and after (right) model processing.}
% 
\label{fig:visualization}
\vspace{-5mm}
\end{figure}

\subsection{Ablation Performance}
In this subsection, we conduct an ablation study to examine which component is really important for performance improvement. The results are reported in Table~\ref{tab:cap2}.

For information utilization, we observe a significant decline in model performance when visual features are removed. Additionally, the performance of \shortname~decreases when different metadata are removed separately, which means that text description, object tag, and scene tag are all critical for image sentiment analysis.
Recalling the model architecture, we separately remove transformer layers of the unified representation module, the adaptive learning module, and the cross-modal fusion module, replacing them with MLPs of the same parameter scale.
In this way, we can observe varying degrees of decline in model performance, indicating that these modules are indispensable for our model to achieve better performance.

\subsection{Visualization}
% 


% % 开始使用minipage进行左右排列
% \begin{minipage}[t]{0.45\textwidth}  % 子图1宽度为45%
%     \centering
%     \includegraphics[width=\textwidth]{2dvisual.pdf}  % 插入图片
%     \captionof{figure}{Visualization of feature distribution.}  % 使用captionof添加图片标题
%     \label{fig:visualization}
% \end{minipage}


% \begin{figure}[t]
% \centering
% \vspace{-2mm}
% \includegraphics[width=0.45\textwidth]{fig/2dvisual.pdf}
% \caption{Visualization of feature distribution.}
% \label{fig:visualization}
% % \vspace{-4mm}
% \end{figure}

% \begin{figure}[t]
% \centering
% \vspace{-2mm}
% \includegraphics[width=0.45\textwidth]{fig/2dvisual-linux3-paper.pdf}
% \caption{Visualization of feature distribution.}
% \label{fig:visualization}
% % \vspace{-4mm}
% \end{figure}



\begin{figure}[tbp]   
\vspace{-4mm}
  \centering            
  \subfloat[Depth of adaptive learning layers]   
  {
    \label{fig:subfig1}\includegraphics[width=0.22\textwidth]{fig/fig_sensitivity-a5}
  }
  \subfloat[Depth of fusion layers]
  {
    % \label{fig:subfig2}\includegraphics[width=0.22\textwidth]{fig/fig_sensitivity-b2}
    \label{fig:subfig2}\includegraphics[width=0.22\textwidth]{fig/fig_sensitivity-b2-num.pdf}
  }
  \caption{Sensitivity study of \shortname~on different depth. }   
  \label{fig:fig_sensitivity}  
\vspace{-2mm}
\end{figure}

% \begin{figure}[htbp]
% \centerline{\includegraphics{2dvisual.pdf}}
% \caption{Visualization of feature distribution.}
% \label{fig:visualization}
% \end{figure}

% In Fig.~\ref{fig:visualization}, we use t-SNE~\cite{van2008visualizing} to reduce the dimension of data features for visualization, Figure in left represents the metadata features before model processing, the features are obtained by embedding through the CLIP model, and figure in right shows the features of the data after model processing, it can be observed that after the model processing, the data with different label categories fall in different regions in the space, therefore, we can conclude that the Therefore, we can conclude that the model can effectively utilize the information contained in the metadata and use it to guide the model for classification.

In Fig.~\ref{fig:visualization}, we use t-SNE~\cite{van2008visualizing} to reduce the dimension of data features for visualization.
The left figure shows metadata features before being processed by our model (\textit{i.e.}, embedded by CLIP), while the right shows the distribution of features after being processed by our model.
We can observe that after the model processing, data with the same label are closer to each other, while others are farther away.
Therefore, it shows that the model can effectively utilize the information contained in the metadata and use it to guide the classification process.

\subsection{Sensitivity Analysis}
% 
In this subsection, we conduct a sensitivity analysis to figure out the effect of different depth settings of adaptive learning layers and fusion layers. 
% In this subsection, we conduct a sensitivity analysis to figure out the effect of different depth settings on the model. 
% Fig.~\ref{fig:fig_sensitivity} presents the effect of different depth settings of adaptive learning layers and fusion layers. 
Taking Fig.~\ref{fig:fig_sensitivity} (a) as an example, the model performance improves with increasing depth, reaching the best performance at a depth of 4.
% Taking Fig.~\ref{fig:fig_sensitivity} (a) as an example, the performance of \shortname~improves with the increase of depth at first, reaching the best performance at a depth of 4.
When the depth continues to increase, the accuracy decreases to varying degrees.
Similar results can be observed in Fig.~\ref{fig:fig_sensitivity} (b).
Therefore, we set their depths to 4 and 6 respectively to achieve the best results.

% Through our experiments, we can observe that the effect of modifying these hyperparameters on the results of the experiments is very weak, and the surface model is not sensitive to the hyperparameters.


\subsection{Zero-shot Capability}
% 

% (1)~GCH~\cite{2010Analyzing} & 21.78\% & (5)~RA-DLNet~\cite{2020A} & 34.01\% \\ \hline
% (2)~WSCNet~\cite{2019WSCNet}  & 30.25\% & (6)~CECCN~\cite{ruan2024color} & 43.83\% \\ \hline
% (3)~PCNN~\cite{2015Robust} & 31.68\%  & (7)~EmoVIT~\cite{xie2024emovit} & 44.90\% \\ \hline
% (4)~AR~\cite{2018Visual} & 32.67\% & (8)~Ours (Zero-shot) & 47.83\% \\ \hline


\begin{table}[t]
\centering
\caption{Zero-shot capability of \shortname.}
\label{tab:cap3}
\resizebox{1\linewidth}{!}
{
\begin{tabular}{lc|lc}
\hline
\textbf{Model} & \textbf{Accuracy} & \textbf{Model} & \textbf{Accuracy} \\ \hline
(1)~WSCNet~\cite{2019WSCNet}  & 30.25\% & (5)~MAM~\cite{zhang2024affective} & 39.56\%  \\ \hline
(2)~AR~\cite{2018Visual} & 32.67\% & (6)~CECCN~\cite{ruan2024color} & 43.83\% \\ \hline
(3)~RA-DLNet~\cite{2020A} & 34.01\%  & (7)~EmoVIT~\cite{xie2024emovit} & 44.90\% \\ \hline
(4)~CDA~\cite{han2023boosting} & 38.64\% & (8)~Ours (Zero-shot) & 47.83\% \\ \hline
\end{tabular}
}
\vspace{-5mm}
\end{table}

% We use the model trained on the FI dataset to test on the artphoto dataset to verify the model's generalization ability as well as robustness to other distributed datasets.
% We can observe that the MESN model shows strong competitiveness in terms of accuracy when compared to other trained models, which suggests that the model has a good generalization ability in the OOD task.

To validate the model's generalization ability and robustness to other distributed datasets, we directly test the model trained on the FI dataset, without training on Artphoto. 
% As observed in Table 3, compared to other models trained on Artphoto, we achieve highly competitive zero-shot performance, indicating that the model has good generalization ability in out-of-distribution tasks.
From Table~\ref{tab:cap3}, we can observe that compared with other models trained on Artphoto, we achieve competitive zero-shot performance, which shows that the model has good generalization ability in out-of-distribution tasks.


%%%%%%%%%%%%
%  E2E     %
%%%%%%%%%%%%


\section{Conclusion}
In this paper, we introduced Wi-Chat, the first LLM-powered Wi-Fi-based human activity recognition system that integrates the reasoning capabilities of large language models with the sensing potential of wireless signals. Our experimental results on a self-collected Wi-Fi CSI dataset demonstrate the promising potential of LLMs in enabling zero-shot Wi-Fi sensing. These findings suggest a new paradigm for human activity recognition that does not rely on extensive labeled data. We hope future research will build upon this direction, further exploring the applications of LLMs in signal processing domains such as IoT, mobile sensing, and radar-based systems.

\section*{Limitations}
While our work represents the first attempt to leverage LLMs for processing Wi-Fi signals, it is a preliminary study focused on a relatively simple task: Wi-Fi-based human activity recognition. This choice allows us to explore the feasibility of LLMs in wireless sensing but also comes with certain limitations.

Our approach primarily evaluates zero-shot performance, which, while promising, may still lag behind traditional supervised learning methods in highly complex or fine-grained recognition tasks. Besides, our study is limited to a controlled environment with a self-collected dataset, and the generalizability of LLMs to diverse real-world scenarios with varying Wi-Fi conditions, environmental interference, and device heterogeneity remains an open question.

Additionally, we have yet to explore the full potential of LLMs in more advanced Wi-Fi sensing applications, such as fine-grained gesture recognition, occupancy detection, and passive health monitoring. Future work should investigate the scalability of LLM-based approaches, their robustness to domain shifts, and their integration with multimodal sensing techniques in broader IoT applications.


% Bibliography entries for the entire Anthology, followed by custom entries
%\bibliography{anthology,custom}
% Custom bibliography entries only
\bibliography{main}
\newpage
\appendix

\section{Experiment prompts}
\label{sec:prompt}
The prompts used in the LLM experiments are shown in the following Table~\ref{tab:prompts}.

\definecolor{titlecolor}{rgb}{0.9, 0.5, 0.1}
\definecolor{anscolor}{rgb}{0.2, 0.5, 0.8}
\definecolor{labelcolor}{HTML}{48a07e}
\begin{table*}[h]
	\centering
	
 % \vspace{-0.2cm}
	
	\begin{center}
		\begin{tikzpicture}[
				chatbox_inner/.style={rectangle, rounded corners, opacity=0, text opacity=1, font=\sffamily\scriptsize, text width=5in, text height=9pt, inner xsep=6pt, inner ysep=6pt},
				chatbox_prompt_inner/.style={chatbox_inner, align=flush left, xshift=0pt, text height=11pt},
				chatbox_user_inner/.style={chatbox_inner, align=flush left, xshift=0pt},
				chatbox_gpt_inner/.style={chatbox_inner, align=flush left, xshift=0pt},
				chatbox/.style={chatbox_inner, draw=black!25, fill=gray!7, opacity=1, text opacity=0},
				chatbox_prompt/.style={chatbox, align=flush left, fill=gray!1.5, draw=black!30, text height=10pt},
				chatbox_user/.style={chatbox, align=flush left},
				chatbox_gpt/.style={chatbox, align=flush left},
				chatbox2/.style={chatbox_gpt, fill=green!25},
				chatbox3/.style={chatbox_gpt, fill=red!20, draw=black!20},
				chatbox4/.style={chatbox_gpt, fill=yellow!30},
				labelbox/.style={rectangle, rounded corners, draw=black!50, font=\sffamily\scriptsize\bfseries, fill=gray!5, inner sep=3pt},
			]
											
			\node[chatbox_user] (q1) {
				\textbf{System prompt}
				\newline
				\newline
				You are a helpful and precise assistant for segmenting and labeling sentences. We would like to request your help on curating a dataset for entity-level hallucination detection.
				\newline \newline
                We will give you a machine generated biography and a list of checked facts about the biography. Each fact consists of a sentence and a label (True/False). Please do the following process. First, breaking down the biography into words. Second, by referring to the provided list of facts, merging some broken down words in the previous step to form meaningful entities. For example, ``strategic thinking'' should be one entity instead of two. Third, according to the labels in the list of facts, labeling each entity as True or False. Specifically, for facts that share a similar sentence structure (\eg, \textit{``He was born on Mach 9, 1941.''} (\texttt{True}) and \textit{``He was born in Ramos Mejia.''} (\texttt{False})), please first assign labels to entities that differ across atomic facts. For example, first labeling ``Mach 9, 1941'' (\texttt{True}) and ``Ramos Mejia'' (\texttt{False}) in the above case. For those entities that are the same across atomic facts (\eg, ``was born'') or are neutral (\eg, ``he,'' ``in,'' and ``on''), please label them as \texttt{True}. For the cases that there is no atomic fact that shares the same sentence structure, please identify the most informative entities in the sentence and label them with the same label as the atomic fact while treating the rest of the entities as \texttt{True}. In the end, output the entities and labels in the following format:
                \begin{itemize}[nosep]
                    \item Entity 1 (Label 1)
                    \item Entity 2 (Label 2)
                    \item ...
                    \item Entity N (Label N)
                \end{itemize}
                % \newline \newline
                Here are two examples:
                \newline\newline
                \textbf{[Example 1]}
                \newline
                [The start of the biography]
                \newline
                \textcolor{titlecolor}{Marianne McAndrew is an American actress and singer, born on November 21, 1942, in Cleveland, Ohio. She began her acting career in the late 1960s, appearing in various television shows and films.}
                \newline
                [The end of the biography]
                \newline \newline
                [The start of the list of checked facts]
                \newline
                \textcolor{anscolor}{[Marianne McAndrew is an American. (False); Marianne McAndrew is an actress. (True); Marianne McAndrew is a singer. (False); Marianne McAndrew was born on November 21, 1942. (False); Marianne McAndrew was born in Cleveland, Ohio. (False); She began her acting career in the late 1960s. (True); She has appeared in various television shows. (True); She has appeared in various films. (True)]}
                \newline
                [The end of the list of checked facts]
                \newline \newline
                [The start of the ideal output]
                \newline
                \textcolor{labelcolor}{[Marianne McAndrew (True); is (True); an (True); American (False); actress (True); and (True); singer (False); , (True); born (True); on (True); November 21, 1942 (False); , (True); in (True); Cleveland, Ohio (False); . (True); She (True); began (True); her (True); acting career (True); in (True); the late 1960s (True); , (True); appearing (True); in (True); various (True); television shows (True); and (True); films (True); . (True)]}
                \newline
                [The end of the ideal output]
				\newline \newline
                \textbf{[Example 2]}
                \newline
                [The start of the biography]
                \newline
                \textcolor{titlecolor}{Doug Sheehan is an American actor who was born on April 27, 1949, in Santa Monica, California. He is best known for his roles in soap operas, including his portrayal of Joe Kelly on ``General Hospital'' and Ben Gibson on ``Knots Landing.''}
                \newline
                [The end of the biography]
                \newline \newline
                [The start of the list of checked facts]
                \newline
                \textcolor{anscolor}{[Doug Sheehan is an American. (True); Doug Sheehan is an actor. (True); Doug Sheehan was born on April 27, 1949. (True); Doug Sheehan was born in Santa Monica, California. (False); He is best known for his roles in soap operas. (True); He portrayed Joe Kelly. (True); Joe Kelly was in General Hospital. (True); General Hospital is a soap opera. (True); He portrayed Ben Gibson. (True); Ben Gibson was in Knots Landing. (True); Knots Landing is a soap opera. (True)]}
                \newline
                [The end of the list of checked facts]
                \newline \newline
                [The start of the ideal output]
                \newline
                \textcolor{labelcolor}{[Doug Sheehan (True); is (True); an (True); American (True); actor (True); who (True); was born (True); on (True); April 27, 1949 (True); in (True); Santa Monica, California (False); . (True); He (True); is (True); best known (True); for (True); his roles in soap operas (True); , (True); including (True); in (True); his portrayal (True); of (True); Joe Kelly (True); on (True); ``General Hospital'' (True); and (True); Ben Gibson (True); on (True); ``Knots Landing.'' (True)]}
                \newline
                [The end of the ideal output]
				\newline \newline
				\textbf{User prompt}
				\newline
				\newline
				[The start of the biography]
				\newline
				\textcolor{magenta}{\texttt{\{BIOGRAPHY\}}}
				\newline
				[The ebd of the biography]
				\newline \newline
				[The start of the list of checked facts]
				\newline
				\textcolor{magenta}{\texttt{\{LIST OF CHECKED FACTS\}}}
				\newline
				[The end of the list of checked facts]
			};
			\node[chatbox_user_inner] (q1_text) at (q1) {
				\textbf{System prompt}
				\newline
				\newline
				You are a helpful and precise assistant for segmenting and labeling sentences. We would like to request your help on curating a dataset for entity-level hallucination detection.
				\newline \newline
                We will give you a machine generated biography and a list of checked facts about the biography. Each fact consists of a sentence and a label (True/False). Please do the following process. First, breaking down the biography into words. Second, by referring to the provided list of facts, merging some broken down words in the previous step to form meaningful entities. For example, ``strategic thinking'' should be one entity instead of two. Third, according to the labels in the list of facts, labeling each entity as True or False. Specifically, for facts that share a similar sentence structure (\eg, \textit{``He was born on Mach 9, 1941.''} (\texttt{True}) and \textit{``He was born in Ramos Mejia.''} (\texttt{False})), please first assign labels to entities that differ across atomic facts. For example, first labeling ``Mach 9, 1941'' (\texttt{True}) and ``Ramos Mejia'' (\texttt{False}) in the above case. For those entities that are the same across atomic facts (\eg, ``was born'') or are neutral (\eg, ``he,'' ``in,'' and ``on''), please label them as \texttt{True}. For the cases that there is no atomic fact that shares the same sentence structure, please identify the most informative entities in the sentence and label them with the same label as the atomic fact while treating the rest of the entities as \texttt{True}. In the end, output the entities and labels in the following format:
                \begin{itemize}[nosep]
                    \item Entity 1 (Label 1)
                    \item Entity 2 (Label 2)
                    \item ...
                    \item Entity N (Label N)
                \end{itemize}
                % \newline \newline
                Here are two examples:
                \newline\newline
                \textbf{[Example 1]}
                \newline
                [The start of the biography]
                \newline
                \textcolor{titlecolor}{Marianne McAndrew is an American actress and singer, born on November 21, 1942, in Cleveland, Ohio. She began her acting career in the late 1960s, appearing in various television shows and films.}
                \newline
                [The end of the biography]
                \newline \newline
                [The start of the list of checked facts]
                \newline
                \textcolor{anscolor}{[Marianne McAndrew is an American. (False); Marianne McAndrew is an actress. (True); Marianne McAndrew is a singer. (False); Marianne McAndrew was born on November 21, 1942. (False); Marianne McAndrew was born in Cleveland, Ohio. (False); She began her acting career in the late 1960s. (True); She has appeared in various television shows. (True); She has appeared in various films. (True)]}
                \newline
                [The end of the list of checked facts]
                \newline \newline
                [The start of the ideal output]
                \newline
                \textcolor{labelcolor}{[Marianne McAndrew (True); is (True); an (True); American (False); actress (True); and (True); singer (False); , (True); born (True); on (True); November 21, 1942 (False); , (True); in (True); Cleveland, Ohio (False); . (True); She (True); began (True); her (True); acting career (True); in (True); the late 1960s (True); , (True); appearing (True); in (True); various (True); television shows (True); and (True); films (True); . (True)]}
                \newline
                [The end of the ideal output]
				\newline \newline
                \textbf{[Example 2]}
                \newline
                [The start of the biography]
                \newline
                \textcolor{titlecolor}{Doug Sheehan is an American actor who was born on April 27, 1949, in Santa Monica, California. He is best known for his roles in soap operas, including his portrayal of Joe Kelly on ``General Hospital'' and Ben Gibson on ``Knots Landing.''}
                \newline
                [The end of the biography]
                \newline \newline
                [The start of the list of checked facts]
                \newline
                \textcolor{anscolor}{[Doug Sheehan is an American. (True); Doug Sheehan is an actor. (True); Doug Sheehan was born on April 27, 1949. (True); Doug Sheehan was born in Santa Monica, California. (False); He is best known for his roles in soap operas. (True); He portrayed Joe Kelly. (True); Joe Kelly was in General Hospital. (True); General Hospital is a soap opera. (True); He portrayed Ben Gibson. (True); Ben Gibson was in Knots Landing. (True); Knots Landing is a soap opera. (True)]}
                \newline
                [The end of the list of checked facts]
                \newline \newline
                [The start of the ideal output]
                \newline
                \textcolor{labelcolor}{[Doug Sheehan (True); is (True); an (True); American (True); actor (True); who (True); was born (True); on (True); April 27, 1949 (True); in (True); Santa Monica, California (False); . (True); He (True); is (True); best known (True); for (True); his roles in soap operas (True); , (True); including (True); in (True); his portrayal (True); of (True); Joe Kelly (True); on (True); ``General Hospital'' (True); and (True); Ben Gibson (True); on (True); ``Knots Landing.'' (True)]}
                \newline
                [The end of the ideal output]
				\newline \newline
				\textbf{User prompt}
				\newline
				\newline
				[The start of the biography]
				\newline
				\textcolor{magenta}{\texttt{\{BIOGRAPHY\}}}
				\newline
				[The ebd of the biography]
				\newline \newline
				[The start of the list of checked facts]
				\newline
				\textcolor{magenta}{\texttt{\{LIST OF CHECKED FACTS\}}}
				\newline
				[The end of the list of checked facts]
			};
		\end{tikzpicture}
        \caption{GPT-4o prompt for labeling hallucinated entities.}\label{tb:gpt-4-prompt}
	\end{center}
\vspace{-0cm}
\end{table*}
% \section{Full Experiment Results}
% \begin{table*}[th]
    \centering
    \small
    \caption{Classification Results}
    \begin{tabular}{lcccc}
        \toprule
        \textbf{Method} & \textbf{Accuracy} & \textbf{Precision} & \textbf{Recall} & \textbf{F1-score} \\
        \midrule
        \multicolumn{5}{c}{\textbf{Zero Shot}} \\
                Zero-shot E-eyes & 0.26 & 0.26 & 0.27 & 0.26 \\
        Zero-shot CARM & 0.24 & 0.24 & 0.24 & 0.24 \\
                Zero-shot SVM & 0.27 & 0.28 & 0.28 & 0.27 \\
        Zero-shot CNN & 0.23 & 0.24 & 0.23 & 0.23 \\
        Zero-shot RNN & 0.26 & 0.26 & 0.26 & 0.26 \\
DeepSeek-0shot & 0.54 & 0.61 & 0.54 & 0.52 \\
DeepSeek-0shot-COT & 0.33 & 0.24 & 0.33 & 0.23 \\
DeepSeek-0shot-Knowledge & 0.45 & 0.46 & 0.45 & 0.44 \\
Gemma2-0shot & 0.35 & 0.22 & 0.38 & 0.27 \\
Gemma2-0shot-COT & 0.36 & 0.22 & 0.36 & 0.27 \\
Gemma2-0shot-Knowledge & 0.32 & 0.18 & 0.34 & 0.20 \\
GPT-4o-mini-0shot & 0.48 & 0.53 & 0.48 & 0.41 \\
GPT-4o-mini-0shot-COT & 0.33 & 0.50 & 0.33 & 0.38 \\
GPT-4o-mini-0shot-Knowledge & 0.49 & 0.31 & 0.49 & 0.36 \\
GPT-4o-0shot & 0.62 & 0.62 & 0.47 & 0.42 \\
GPT-4o-0shot-COT & 0.29 & 0.45 & 0.29 & 0.21 \\
GPT-4o-0shot-Knowledge & 0.44 & 0.52 & 0.44 & 0.39 \\
LLaMA-0shot & 0.32 & 0.25 & 0.32 & 0.24 \\
LLaMA-0shot-COT & 0.12 & 0.25 & 0.12 & 0.09 \\
LLaMA-0shot-Knowledge & 0.32 & 0.25 & 0.32 & 0.28 \\
Mistral-0shot & 0.19 & 0.23 & 0.19 & 0.10 \\
Mistral-0shot-Knowledge & 0.21 & 0.40 & 0.21 & 0.11 \\
        \midrule
        \multicolumn{5}{c}{\textbf{4 Shot}} \\
GPT-4o-mini-4shot & 0.58 & 0.59 & 0.58 & 0.53 \\
GPT-4o-mini-4shot-COT & 0.57 & 0.53 & 0.57 & 0.50 \\
GPT-4o-mini-4shot-Knowledge & 0.56 & 0.51 & 0.56 & 0.47 \\
GPT-4o-4shot & 0.77 & 0.84 & 0.77 & 0.73 \\
GPT-4o-4shot-COT & 0.63 & 0.76 & 0.63 & 0.53 \\
GPT-4o-4shot-Knowledge & 0.72 & 0.82 & 0.71 & 0.66 \\
LLaMA-4shot & 0.29 & 0.24 & 0.29 & 0.21 \\
LLaMA-4shot-COT & 0.20 & 0.30 & 0.20 & 0.13 \\
LLaMA-4shot-Knowledge & 0.15 & 0.23 & 0.13 & 0.13 \\
Mistral-4shot & 0.02 & 0.02 & 0.02 & 0.02 \\
Mistral-4shot-Knowledge & 0.21 & 0.27 & 0.21 & 0.20 \\
        \midrule
        
        \multicolumn{5}{c}{\textbf{Suprevised}} \\
        SVM & 0.94 & 0.92 & 0.91 & 0.91 \\
        CNN & 0.98 & 0.98 & 0.97 & 0.97 \\
        RNN & 0.99 & 0.99 & 0.99 & 0.99 \\
        % \midrule
        % \multicolumn{5}{c}{\textbf{Conventional Wi-Fi-based Human Activity Recognition Systems}} \\
        E-eyes & 1.00 & 1.00 & 1.00 & 1.00 \\
        CARM & 0.98 & 0.98 & 0.98 & 0.98 \\
\midrule
 \multicolumn{5}{c}{\textbf{Vision Models}} \\
           Zero-shot SVM & 0.26 & 0.25 & 0.25 & 0.25 \\
        Zero-shot CNN & 0.26 & 0.25 & 0.26 & 0.26 \\
        Zero-shot RNN & 0.28 & 0.28 & 0.29 & 0.28 \\
        SVM & 0.99 & 0.99 & 0.99 & 0.99 \\
        CNN & 0.98 & 0.99 & 0.98 & 0.98 \\
        RNN & 0.98 & 0.99 & 0.98 & 0.98 \\
GPT-4o-mini-Vision & 0.84 & 0.85 & 0.84 & 0.84 \\
GPT-4o-mini-Vision-COT & 0.90 & 0.91 & 0.90 & 0.90 \\
GPT-4o-Vision & 0.74 & 0.82 & 0.74 & 0.73 \\
GPT-4o-Vision-COT & 0.70 & 0.83 & 0.70 & 0.68 \\
LLaMA-Vision & 0.20 & 0.23 & 0.20 & 0.09 \\
LLaMA-Vision-Knowledge & 0.22 & 0.05 & 0.22 & 0.08 \\

        \bottomrule
    \end{tabular}
    \label{full}
\end{table*}




\end{document}


% \appendix
\section{Dataset Statistics}\label{apppendix:statistic}
The FoREST dataset statistic is provided in the Table~\ref{tab:data_statistic}.

\begin{table*}[t!]
    \centering
    \small
    \begin{tabular}{|c|c|c||c|c|c|}
        \hline
         Case & A-Split & A-Split for T2I & FoR class & C-Spilt & C-split for T2I\\
         \hline
         Cow Case & 792 & 3168 & External Relative & 1528 & 4288\\
         Box Case & 120 & 120 & External Intrinsic & 920 & 3680\\
         Car Case & 128 & 512 & Internal Intrinsic & 128 & 0\\
         Pen Case & 488 & 488 & Internal Relative & 248 & 0\\
         \hline
         Total & 1528 & 4288 & Total & 2824 & 7968 \\
         \hline
    \end{tabular}
    \caption{Dataset Statistic of FoREST dataset. }
    \label{tab:data_statistic}
\end{table*}


\section{Details Creation of FoREST dataset}\label{appendix:dataset_creation}
We define the nine categories of objects selected in our dataset as indicated below in Table~\ref{tab:selected object}. We select sets of locatum and relatum based on the properties of each class to cover four cases of frame of reference defined in Section~\ref{sec:FoR_Relatum_scenario}. Notice that we also consider the appropriateness of the container; for example, the car should not contain the bus.

Based on the selected locatum and relatum. To create an A-split spatial expression, we substitute the actual locatum and relatum objects in the Spatial Relation template. After obtaining the A-split contexts, we create their counterparts using the perspective/topology clauses to make the counterparts in C-spilt. Then, we obtain the I-A and I-C split by applying the directional template to the first occurrence of relatum when it has intrinsic directions. The directional templates are "that is facing towards," "that is facing backward," "that is facing to the left," and "that is facing to the right." All the templates are in the Table~\ref{tab:templates}.
We then construct the scene configuration from each modified spatial expression and send it to the simulator developed using Unity3D. 
Eventually, the simulator produces four visualization images for each scene configuration. 

\begin{table*}[t]
    \centering
    \small
    \begin{tabular}{|c|c|c|c|}
    \hline
        Category &  Object & Intrinsic Direction & Container \\
        \hline
        small object without intrinsic directions & umbrella, bag, suitcase, fire hydrant & \xmark & \xmark \\
        \hline
        bog object with intrinsic directions & bench, chair & \checkmark &  \xmark \\ 
        \hline
        big object without intrinsic direction & water tank & \xmark & \xmark \\
        \hline
        container & box, container & \xmark & \checkmark \\
        \hline
        small animal & chicken, dog, cat & \checkmark & \xmark \\
        \hline
        big animal & deer, horse, cow, sheep & \checkmark & \xmark \\
        \hline
        small vehicle & bicycle & \checkmark & \xmark \\
        \hline
        big vehicle & bus, car & \checkmark & \checkmark \\
        \hline
        tree & tree & \xmark & \xmark \\
        \hline
    \end{tabular}
    \caption{All selected objects with two properties: intrinsic direction, affordance of being container}
    \label{tab:selected object}
\end{table*}

\subsection{Simulation Details}
The simulation starts with randomly placing the relatum into the scene with the orientation based on the given scene configuration. 
We randomly select the orientation by given scene configuration, [-40, 40] for front, [40, 140] for left, [140, 220] for back, and [220, 320] for right. Then, we create the locatum from the relatum position and move it in the spatial relation provided. If the frame of reference is relative, we move the locatum based on the camera's orientation. Otherwise, we move it from the relatum's orientation. Then, we check the camera's visibility of both objects. If one of them is not visible, we repeat the process of generating the relatum until the correct placement is achieved. After getting the proper placement, we randomly choose the background from 6 backgrounds. Eventually, we repeat the procedures four times for one configuration. 

\subsection{Object Models and Background}
For the object models and background, we find it from the unity assert store\footnote{https://assetstore.unity.com}. All of them are free and available for download. All of the 3D models used are shown in Figure~\ref{fig:3D_model}.

\begin{figure*}[t]
    \centering
    \includegraphics[width=\linewidth]{Figures/all_3d_model.png}
    \caption{All 3d models used to generate visualizations for FoREST.}
    \label{fig:3D_model}
\end{figure*}

\begin{table*}[t]
    \centering
    \tiny
    \begin{tabular}{|c|c|} 
    \hline
        &   \{locatum\} is in front of \{relatum\} \\
         &  \{locatum\} is on the left of \{relatum\} \\
           &  \{locatum\} is to the left of \{relatum\}\\ 
        Spatial Relation Templates &  \{locatum\} is behind of \{relatum\} \\
           &  \{locatum\} is back of \{relatum\} \\
          &  \{locatum\} is on the right of \{relatum\} \\
          &   \{locatum\} is to the right of \{relatum\} \\
         \hline
         &  within \{relatum\} \\ 
         Topology Templates &  and inside \{relatum\} \\ 
         &  and outside of \{relatum\} \\ 
         \hline
         & from \{relatum\}'s view \\ 
         & relative to \{relatum\} \\ 
         Perspective Templates & from \{relatum\}'s perspective\\ 
         & from my perspective \\ 
          & from my point of view \\ 
           & relative to observer \\ 
         \hline
          &  \{relatum\} facing toward that camera\\ 
         Orientation Templates &  \{relatum\}is facing away from the camera. \\ 
          &  \{relatum\} facing left relative to the camera\\ 
           &  \{relatum\} facing right relative to the camera \\ 
         \hline
         & In the camera view, how is \{locatum\} positioned in relation to \{relatum\}? \\
        &  Based on the camera perspective, where is the \{locatum\} from the \{relatum\}'s position? \\ 
         Question Templates &  From the camera perspective, what is the relation of the\{locatum\} to the \{relatum\}?\\ 
          &  Looking through the camera perspective, how does \{locatum\} appear to be oriented relative to \{relatum\}'s position?\\ 
           &  Based on the camera angle, where is \{locatum\} located with respect to \{relatum\}'s location? \\ 
         \hline
    \end{tabular}
    \caption{All templates used to create FoREST dataset.}
    \label{tab:templates}
\end{table*}


\subsection{Textual templates}\label{appendix:textual_template}
All the templates used to create FoREST are given in Table~\ref{tab:templates}.



\section{VISOR {uncond} Score}\label{appendix:Visor_uncond}

VISOR$_{uncond}$ provides the overall spatial relation score, including images with object generation errors. Since it is less focused on evaluating spatial interpretation than VISOR$_{cond}$, which assesses explicitly the text-to-image model's spatial reasoning, we report VISOR$_{uncond}$ results here in the Table~\ref{tab:VISOR_uncode} rather than in the main paper. The results are similar to the pattern observed in VISOR$_{uncond}$ that the based models(SD-1.5 and SD-2.1) perform better in the relative frame of reference, while the layout-to-image models, i.e., GLIGEN, are better in the intrinsic frame of reference.

\begin{table*}[t!]
    \centering
    \small
    % \begin{adjustbox}{width=\columnwidth -0mm, center}
    \begin{tabular}{|l | c c | c | c c | c |}
    \hline
         & \multicolumn{6}{c|}{VISOR(\%)} \\ \cline{2-7}
        Model & uncond (I) & uncond (R) & uncond (avg) & uncond (I) & uncond (R) & uncond (avg) \\
        \hline
         & \multicolumn{3}{|c|}{ A-Split } & \multicolumn{3}{|c|}{ C-Split }  \\ \hline
        SD-1.5 & $ 45.43$  & $33.22$ & $ 43.51$ &  $ 35.06$  & $ 35.68$ & $ 35.40$ \\
        SD-2.1 & $\mathbf{62.87}$  & $ 43.90$ & $\mathbf{59.89}$ & $\mathbf{45.98}$  & $ 46.59$ & $\mathbf{46.31}$ \\
        \hline
        Llama3-8B + GLIGEN & $ 46.74$  & $ 38.16$ & $ 45.39$ & $ 33.98$  & $ 39.36$ & $ 36.89$ \\
        Llama3-70B + GLIGEN & $ 54.33$  & $ 46.89$ & $ 53.17$ & $ 38.04$  & $ 46.04$ & $ 42.37$ \\
        Llama3-8B + SG + GLIGEN (Our) & $ 51.83$  & $ 43.24$ & $ 50.48$ & $ 36.28$  & $ 44.43$ & $ 40.70$ \\
        Llama3-70B + SG + GLIGEN (Our) & $ 58.92$  & $\mathbf{47.44}$ & $ 57.12$ & $ 38.23$  & $\mathbf{48.62}$ & $ 43.86$ \\
        \hline
    \end{tabular}
    \caption{VISOR$_{uncond}$ score on the A-Split and C-Split where $I$ refer to the Cow Case and Car Case where relatum has intrinsic directions, and $R$ refer to the Box Case and Pen case where relatum lacks intrinsic directions, $avg$  is mirco-average of $I$ and $R$. cond and uncond are explained in Section~\ref{sec:evaluation_setting}.}
    \label{tab:VISOR_uncode}
\end{table*}


\section{Analyze the improvements in SG-prompting for T2I.} \label{appedix:anaylize_SG_improment_t2i}
\begin{table}[t]
    \small
    \centering
    \begin{tabular}{|l|c|c|c|}
         \hline
         Model & Layout & Layout$_{cond}$  \\
         \hline
          Llama3-8B  & $85.26$ & $88.84$\\
          Llama3-8B + SG  & $85.04$ & $88.86$  \\
          Llama3-70B  & $88.47$ & $93.16$ \\
          Llama3-70B + SG & $91.95$ & $95.45$ \\
          \hline
    \end{tabular}
    \caption{Layout accuracy where spatial relations are left or right relative to the camera. Layout is evaluated for all generated layouts in I-C split while Layout$_{cond}$ uses the same testing examples as VISOR$_{cond}$.}
    \label{tab:layout_results}
\end{table}


To further explain improvements of SG-prompting in T2I task, we assess the generated bounding boxes
in the I-C split for left and right relations relative to the camera since these can be evaluated using only bounding boxes without depth information.
As seen is Table~\ref{tab:layout_results}, our SG prompting improved 
Llama3-70B's by $3.48\%$, while Llama3-8B saw a slight decrease of $0.22\%$. 
This evaluation was conducted on all generated layouts from the I-C split, which differs from the evaluation subset of images used for VISOR$_{cond}$ in Table~\ref{tab:I_split}. 
We report the layout$_{cond}$ score for a consistent evaluation in the same table. Layout$_{cond}$ shows that Llama3-8B improves within the same evaluation subset with VISOR$_{cond}$.
Overall, by incorporating FoR information through SG layout diffusion, Llama3 generates better spatial configurations, enhancing image generation performance.



% \section{\iclr{Qualitative Examples of Failure Cases of CoT and SG}}\label{appendix:false_examples}


% \subsection{CoT Incorrect but Resolved by using SG}

% \begin{itemize}
% \item \textbf{Context} : A cat is inside and on the left of a car from the car's perspective

% \textbf{CoT} : Explanation: Cat is inside of the car, and the car has the intrinsic direction. However, the left of the car is referring to the car's perspective rather than the intrinsic direction of the car. Answer: internal relative.

% \textbf{SG} : Explanation: Topological: cat is inside of the car. Distance: cat should be in the same location with the car from the context. Direction: The car which is the relative object has the direction, and cat is on the left side of the car. Answer: internal intrinsic.

% \textbf{label} : internal intrinsic

% \item \textbf{Context} : A container is on the right of a dog from the dog's perspective

% \textbf{CoT} : Explanation: Based on the context, dog does not contain the container. The dog has the direction; however, the right of a dog referring to the dog's perspective rather than intrinsic direction of the dog. Answer: external relative.

% \textbf{SG} : Explanation: Topological: container is not inside of the dog. Distance: container may has some distance from dog according to the context. Direction: Dog has the direction and the context right relation is referred from dog's perspective that a container is on the right side of the dog's facing orientation. Answer: external intrinsic.

% \textbf{label} : external intrinsic

% \item \textbf{Context} : A water tank is on the left of a cow from the cow's perspective

% \textbf{CoT} : Explanation: Based on the context, cow does not contain the water tank. The cow has the direction; however, the left of a water tank referring to the cow's perspective rather than cow's intrinsic direction. Answer: external relative.

% \textbf{SG} : Explanation: Topological: water tank is not inside of the cow. Distance: water tank may has some distance from cow according to the context. Direction: The cow which is the relative object has the direction, and water tank is on the left side of the cow. Answer: external intrinsic.

% \textbf{label} : external intrinsic

% \item \textbf{Context} : A bicycle is in front of a container inside the container

% \textbf{CoT} : Explanation: The bicycle is contained within the container, fulfilling the first condition for internal frames of reference. However, since the bicycle is also described as being 'in front' of another object (another container), this implies a spatial relationship that is dependent on the container's intrinsic direction. But due to thembiguity of having two containers mentioned, and assuming the second 'container' is a typo or misstatement, we interpret it as a single container scenario. As such, the spatial relation is intrinsic to the container itself, not relative to an external observer or another entity. Answer: internal intrinsic. 

% However, the context provided contains a potential error or ambiguity ("inside the container"), which could alter the interpretation if clarified. Assuming there was meant to be only one container, thebove categorization holds. If there were indeed two distinct containers, further clarification would be needed to accurately categorize the spatial relation.

% \textbf{SG} : Explanation: Topological: bicycle is inside of the container. Distance: bicycle should be in the same location with the container from the context. Direction: Container doesn't have the direction and context refere from observer perspective for bicycle is in the front of the container. Answer: internal relative. However, there seems to be a contradiction in the statement as it says the bicycle is both inside and in front of the container. For the purpose of categorization, we will assume that the intended meaning was that the bicycle is inside the container and positioned towards the front end of it.

% \textbf{label} : internal relative

% \item \textbf{Context} : A tree is on the left of a box inside the box

% \textbf{CoT} : Explanation: Based on the context, box does not contain the tree. However, it seems there might be a mistake in the sentence as it mentions 'inside the box' twice. Assuming the intention was to describe the tree's position relative to the box without specifying a container for the tree, we can interpret this as the tree being positioned relative to the box without being contained within it. Since the box does not have an intrinsic direction, the reference to 'left' is relative to an external observer's perspective. Answer: external relative.

% \textbf{SG} : This context seems to have a logical inconsistency as a tree cannot be inside a box. However, assuming this was intended to describe two separate spatial relations, we can interpret it as follows:

% Explanation: Topological: tree is not inside of the box. Distance: tree may have some distance from the box according to the context. Direction: the box does not have a facing orientation, so the tree being on the left is referred from the observer's perspective. Answer: external relative for the relation between the tree and the box.

% However, the second part of the sentence "inside the box" does not provide enough information to determine a spatial relation between entities. It appears to be incomplete or incorrectly phrased. If you meant to say "A tree is on the left of a box inside a room," then the relation between the tree and the box would still be external relative, while the relation of the box being inside the room would be internal relative.

% \textbf{label} : internal relative

% \item \textbf{Context} : A deer is outside and on the left of a bus relative to the bus

% \textbf{CoT} : Explanation: The bus has the front/back direction. However, the left of the bus is not referring to the bus's intrinsic direction, but rather to an observer's perspective. Answer: external relative.

% \textbf{SG} : Explanation: Topological: deer is outside of the bus. Distance: deer may be little bit far from the left of the bus from the context. Direction: The bus which is the relative object has the direction and deer is actually on the left side of the bus. Answer: external intrinsic.

% \textbf{label} : external intrinsic

% \item \textbf{Context} : A bicycle is behind a fire hydrant

% \textbf{CoT} : Explanation: Based on the context, fire hydrant does not contain the bicycle. Fire hydrant does not have intrinsic direction, so the behind relation is based on the observer's perspective. However, since there is no explicit mention of an observer or reference point, we can consider it as an external intrinsic relation where the observer's perspective is implied. Answer: external intrinsic.

% \textbf{SG} : Explanation: Topological: bicycle is not inside of the fire hydrant. Distance: bicycle may have some distance from the fire hydrant according to the context. Direction: Fire hydrant does not have a facing orientation, so the spatial relation is not relative to its intrinsic direction. Instead, it is described from an external perspective. Answer: external relative.

% \textbf{label} : external relative

% \item \textbf{Context} : A box is behind a chicken relative to the chicken

% \textbf{CoT} : Explanation: The chicken has no intrinsic direction, so the 'behind' relation is based on the perspective of the chicken itself, not its intrinsic orientation. Answer: external relative. Note that this could potentially be seen as external intrinsic if we assume that the chicken has a front-back orientation, but since chickens do not typically have a fixed or understood front-back orientation like humans or cars do, it is more likely to be considered external relative.

% \textbf{SG} : Explanation: Topological: box is not inside of the chicken. Distance: box may be little bit far from the chicken from the context. Direction: The chicken which is the relative object has the direction and box is actually behind the chicken. Answer: external intrinsic.

% \textbf{label} : external intrinsic

% \item \textbf{Context} : A car is in front of a deer from the deer's perspective

% \textbf{CoT} : Explanation: Based on the context, deer does not contain the car. The deer has the direction; however, the front of a deer referring to the deer's perspective rather than intrinsic direction of deer. Answer: external relative.

% \textbf{SG} : Explanation: Topological: car is not inside of the deer. Distance: car may has some distance from deer according to the context. Direction: The deer which is the relative object has the direction, and car is in front of the deer. Answer: external intrinsic.

% \textbf{label} : external intrinsic

% \item \textbf{Context} : An umbrella is on the left of a chicken from the chicken's perspective

% \textbf{CoT} : Explanation: Based on the context, chicken does not contain the umbrella. The chicken has the direction; however, the left of a chicken referring to the chicken's perspective rather than intrinsic direction of the chicken. Answer: external intrinsic. Note that this sentence is ambiguous because it can be interpreted as external relative if the left is referring to the observer's perspective. However, given the context provided, it is more likely to be interpreted as external intrinsic.

% \textbf{SG} : Explanation: Topological: umbrella is not inside of the chicken. Distance: umbrella may has some distance from chicken according to the context. Direction: The context left relation is referred from the chicken's perspective that an umbrella is on the left side of the chicken's facing orientation. Answer: external intrinsic.

% \textbf{label} : external intrinsic

% \end{itemize}

% \subsubsection{Failure Case of SG}

% \begin{itemize}
%     \item \textbf{Context} : A bus is on the right of a chair from the chair's perspective

% \textbf{SG} : Explanation: Topological: bus is not inside of the chair. Distance: bus may has some distance from chair according to the context. Direction: Even though the chair does not have a facing orientation, the perspective is given as the chair's, making this a case of an object being described relative to another object's perspective. Answer: external intrinsic. However, usually chairs do not have a defined facing direction, so typically this would fall under external relative. But given the instruction to consider the chair's perspective, it can be seen as an exception to the rule, aligning more closely with external intrinsic in this specific context.

% \textbf{label} : external intrinsic

% \item \textbf{Context} : A tree is on the left of a box inside the box

% \textbf{SG} : This context seems to have a logical inconsistency as a tree cannot be inside a box. However, assuming this was intended to mean 'A tree is on the left of a box inside a room', the categorization would be:

% Explanation: Topological: tree is not inside of the box. Distance: tree may have some distance from the box according to the context. Direction: The context left relation is referred from observer's perspective that a tree is on the left side of the box location. Answer: external relative. 

% If the original context was correct, it would fall under none of these categories due to the physical impossibility of the scenario described.

% \textbf{label} : internal relative

% \item \textbf{Context} : A cow is in front of a bench relative to the bench

% \textbf{SG} : Explanation: Topological: cow is not inside of the bench. Distance: cow may has some distance from bench according to the context. Direction: Bench doesn't have the direction and context refere from observer perspective for cow is in the front of the bench. Answer: external relative. But if we consider that 'in front of' refers to the bench's intrinsic direction, then it would fall under external intrinsic. However, benches typically do not have a defined facing direction, so this is less common.

% \textbf{label} : external intrinsic

% \item \textbf{Context} : A dog is on the left of a box within the box

% \textbf{SG} : Explanation: Topological: dog is not inside of the box. Distance: dog may has some distance from box according to the context. Direction: Even if box has direction, the context left relation is referred from observer's perspective that a dog is on the left side of the box location. However, since the dog is not contained by the box, this does not fall under internal intrinsic or internal relative. Answer: external relative. Note that the phrase "within the box" seems redundant and does not change the frame of reference.

% \textbf{label} : internal relative

% \end{itemize}

\section{Frame of Reference Identification}\label{appendix:FoRIdentification}

We evaluate the LLMs' performance in recognizing the FoR classes from given spatial expressions. 
The LLMs receive spatial expression, denoted as $T$, and output one FoR class, $FoR$, from the valid set of FoR classes, $For \in $ \{external relative, external intrinsic, internal intrinsic, internal relative\}. All in-context learning examples are in the Appendix~\ref{appendix:in-context}.

\subsection{Experimental Setting}
\noindent\textbf{Zero-shot model.} We follow the regular setting of \textit{zero-shot} prompting. 
We only provide instruction to LLM with spatial context. 
The instruction prompt briefly explains each class of the FoR and candidate answers for the LLM. We called the LLM with the instruction prompt and $T$ to find $F$.

\noindent\textbf{Few-shot model.} We manually craft four spatial expressions for each FoR class. 
To avoid creating bias, each spatial expression is ensured to fit in only one FoR class. These expressions serve as examples of our \textit{few-shot}setting.
We provide these examples in addition to the instruction as a part of the prompt, followed by $T$ and query $F$ from the LLM.

\noindent\textbf{Chain-of-Thought (CoT) model.}
To create CoT~\citep{wei2023chainofthoughtpromptingelicitsreasoning} examples, we modify the prompt to require reasoning before answering.
Then, we manually crafted reasoning explanations with the necessary information for each example used in few-shot.
Finally, we call the LLMs, adding modified instructions to updated examples, followed by $T$ and query $F$. 

\noindent\textbf{Spatial-Guided Prompting (SG) model.}
We hypothesize that the general spatial relation types defined in Section~\ref{sec:primitives} can provide meaningful information for recognizing FoR classes. For instance, a topological relation, such as ``inside," is intuitively associated with an internal FoR.
Therefore, we propose Spatial-Guided Prompting to direct the model in identifying the type of relations before querying $F$. 
We revise the prompting instruction to guide the model in considering these three aspects. 
Then, we manually explain these three aspects.
We specify the relation's origin from the context for direction relations, such as "the left direction is relative to the observer."
We hypothesize that this information helps the model distinguish between intrinsic and relative FoR.
Next, we specify whether the locatum is inside or outside the relatum for topological relations. 
This information should help distinguish between internal and external FoR classes.
Lastly, we provide the potential quantitative distance, e.g., far. This quantitative distance further encourages identifying the correct topological and directional relations. 
Eventually, we insert these new explanations in examples and call the model with the updated instructions followed by $T$ to query $F$.

\subsection{Evaluation Metrics}
We report the accuracy of the model on the multi-class classification task. Note that the expressions in A-split can have multiple correct answers. Therefore, we consider the prediction correct when it is in one of the valid FoR classes for the given spatial expression. 

\begin{table}[t]
    \tiny
    \centering
    \begin{tabular}{|l|c | c|c|c|}
    \hline
    \textbf{Model} & \multicolumn{2}{c|}{inherently clear} & \multicolumn{2}{c|}{require template} \\ \cline{2-5}
     & CoT & SG & {CoT} & {SG} \\ 
    \hline
    Llama3-70B  & 19.84 & 44.64 \improve{24.80} & 76.72 & 87.39 \improve{10.67}\\
    Qwen2-72B & 58.20 & 84.22 \improve{26.02} & 88.36 & 93.86 \improve{10.67} \\
    GPT-4o & 12.50 & 29.17 \improve{16.67} & 87.73 & 90.74 \improve{3.01}  \\
    \hline
    \end{tabular}
    \caption{The comparison between CoT and SG prompting in C-split separated by inherently clear / required template to be clear.}
    \label{tab:model_performance}
\end{table}



\begin{table*}[t]
    \tiny
    \centering
    \begin{tabular}{| l | c | c c c c | c|}
        \hline
        & A-split & \multicolumn{5}{|c|}{C-Split} \\ \cline{3-7}
         Model &  & ER-C-Split & EC-Split & IC-Split & IR-C-Split & Avg. \\
         \hline
         Gemma2-9B (0-shot) & $94.17$ & $\mathbf{94.24}$ & $35.98$ & $53.91$ & $57.66$  & $60.45$\\
          Gemma2-9B (4-shot) & $59.58$  & $55.89$\worse{38.34} & $72.61$\improve{36.63} & $74.22$\improve{20.31} & $54.44$\worse{3.23} & $64.29$\improve{3.84}\\
         Gemma2-9B (CoT) & $60.49$  & $60.49$\worse{33.74} & $60.54$\improve{24.57} & $87.50$\improve{33.59} & $54.03$\worse{3.63} & $65.64$\improve{5.20}\\
          Gemma2-9B (SG)(Our) & $72.67$ & $65.87$\worse{28.37} & $65.54$\improve{29.57} & $53.12$\worse{0.78} & $\mathbf{95.97}$\improve{38.31} & $70.13$\improve{9.68}\\
         \hline
         llama3-8B (0-shot) & $60.21$ & $32.20$ & $90.11$ & $75.78$ & $0.00$ & $49.52$\\
         llama3-8B (4-shot) & $60.14$ & $47.77$\improve{15.58} & $54.35$\worse{35.76} & $100.00$\improve{24.22} & $41.13$\improve{41.13} & $60.81$\improve{11.29}\\
         llama3-8B (CoT) & $61.32$ & $61.06$\improve{28.86} & $97.28$\improve{7.17} & $100.00$\improve{24.22} & $36.29$\improve{36.29} & $73.66$\improve{24.14}\\
         llama3-8B (SG) (Our) & $62.95$ & $63.29$\improve{31.09} & $94.57$\improve{4.46} & $100.00$\improve{24.22} & $43.55$\improve{43.55} & $75.35$\improve{25.83}\\

         \hline
         llama3-70B (0-shot) & $84.23$ & $74.08$ & $9.57$ & $92.19$ & $68.55$ & $61.10$\\
         llama3-70B (4-shot) & $78.47$ & $81.81$\improve{7.72} & $64.89$\improve{55.33} & $100.00$\improve{7.81} & $75.81$\improve{7.26} & $80.63$\improve{19.53}\\
         llama3-70B (CoT) & $69.11$ & $72.05$\worse{2.03} & $97.07$\improve{87.50} & $100.00$\improve{7.81} & $79.44$\improve{10.89} & $87.14$\improve{26.04}\\
         llama3-70B (SG) (Our) & $76.50$ & $78.21$\improve{4.12} & $97.61$\improve{88.04} & $100.00$\improve{7.81} & $72.18$\improve{3.63} & $87.00$\improve{25.90}\\
        % llama3-70B (0-shot) & $77.33$  & $35.04$ & $32.39$ & $57.81$ & $53.23$ & $44.62$\\
        %  llama3-70B (4-shot) & $59.78$ & $59.78$\improve{24.74} & $66.52$\improve{34.13} & $77.34$\improve{19.53} & $51.61$\worse{1.61} & $63.81$\improve{19.20}\\
        %  llama3-70B (CoT) & $66.00$  & $68.01$\improve{32.97} & $65.65$\improve{33.26} & $91.41$\improve{33.59} & $58.47$\improve{5.24} & $70.88$\improve{26.27}\\
        %  llama3-70B (SG) (Our) & $74.94$  & $78.17$\improve{43.13} & $70.87$\improve{38.48} & $100.00$\improve{42.19} & $84.27$\improve{31.05} & $83.33$\improve{38.71}\\
         \hline
         Qwen2-7B (0-shot) & $83.64$ & $79.97$ & $59.24$ & $77.34$ & $40.73$ & $64.32$\\
        Qwen2-7B (4-shot) & $61.12$ & $50.52$\worse{29.45} & $65.76$\improve{6.52} & $93.75$\improve{16.41} & $56.05$\improve{15.32} & $66.52$\improve{2.20}\\
        Qwen2-7B (CoT) & $72.12$ & $70.81$\worse{9.16} & $63.80$\improve{4.57} & $99.22$\improve{21.88} & $51.61$\improve{10.89} & $71.36$\improve{7.04}\\
        Qwen2-7B (SG) & $70.61$ & $68.00$\worse{11.98} & $71.20$\improve{11.96} & $88.28$\improve{10.94} & $57.26$\improve{16.53} & $71.18$\improve{6.86}\\
        \hline
        Qwen2-72B (0-shot)& $64.46$ & $62.70$ & $100.00$ & $100.00$ & $39.11$ & $75.45$\\
        Qwen2-72B (4-shot)& $79.12$ & $78.73$\improve{16.03} & $99.35$\worse{0.65} & $87.50$\worse{12.50} & $87.10$\improve{47.98} & $88.17$\improve{12.72}\\
        Qwen2-72B (CoT)& $88.54$ & $88.87$\improve{26.18} & $89.57$\worse{10.43} & $93.75$\worse{6.25} & $83.47$\improve{44.35} & $88.91$\improve{13.46}\\
        Qwen2-72B (SG)& $90.51$ & $90.18$\improve{27.49} & $93.26$\worse{6.74} & $98.44$\worse{1.56} & $85.08$\improve{45.97} & $91.74$\improve{16.29}\\
        \hline
         GPT3.5 (0-shot) & $83.11$ & $88.15$ & $17.50$ & $70.31$ & $41.13$ & $54.27$\\
         GPT3.5 (4-shot) & $61.25$  & $48.95$\worse{39.20} & $62.72$\improve{45.22} & $100.00$\improve{29.69} & $28.63$\worse{12.50} & $60.07$\improve{5.80}\\

         GPT3.5 (CoT) & $66.55$ & $66.62$\worse{21.53} & $96.85$\improve{79.35} & $100.00$\improve{29.69} & $50.81$\improve{9.68} & $78.57$\improve{24.30}\\
         GPT3.5 (SG) (Our) & $70.61$  & $73.30$\worse{14.86} & $92.93$\improve{75.43} & $99.22$\improve{28.91} & $49.19$\improve{8.06} & $78.66$\improve{24.39}\\
         \hline
         GPT4o (0-shot) & $73.82$  & $71.27$ & $98.80$ & $100.00$ & $70.56$ & $85.16$\\
         GPT4o (4-shot) & $66.23$  & $67.87$\worse{3.40} & $98.70$\worse{0.11} & $100.00$\improve{0.00} & $78.63$\improve{8.06} & $86.30$\improve{1.14}\\
         GPT4o (CoT) & $72.44$  & $72.77$\improve{1.51} & $100.00$\improve{1.20} & $100.00$\improve{0.00} & $73.79$\improve{3.23} & $86.64$\improve{1.48}\\
         GPT4o (SG) (Our) & $76.44$ & $74.67$\improve{3.40} & $97.72$\worse{1.09} & $100.00$\improve{0.00} & $68.55$\worse{2.02} & $85.23$\improve{0.08}\\
         \hline
    \end{tabular}
    \caption{Accuracy results report from FoR Identification with LLMs. The correct prediction is one of the valid FoR classes for the given spatial expression. All FoR classes are external relative (ER), external intrinsic (EI), internal intrinsic (II), and internal relative (IR).}
    \label{tab:text_experiment}
\end{table*}




\begin{figure*}[t]
    \centering
    \begin{subfigure}[ht]{0.4\textwidth}
        \centering
        \includegraphics[width=0.8\textwidth, trim={0 0 0 2cm}]{Figures/A_cow_case.png}
        \caption{Results of Cow Case in A-Split. 
        % Valid predictions are external intrinsic and external relative.
        }
    \end{subfigure}%
    ~ 
    \begin{subfigure}[ht]{0.4\textwidth}
        \centering
        \includegraphics[width=0.8\textwidth, trim={0 0 0 2cm}]{Figures/A_car_case.png}
        \caption{Results of Car Case in A-Split. 
        % All FoRs are valid predictions of this case.
        }
    \end{subfigure}
    
    \vskip\baselineskip
    
    \begin{subfigure}[ht]{0.4\textwidth}   
        \centering 
        \includegraphics[width=0.8\textwidth, trim={0 0 0 1cm}]{Figures/A_box_case.png}
        \caption{Results of Box Case in A-Split. 
        % The correct predictions are external relative and internal relative.
        }    
    \end{subfigure}
        ~
    \begin{subfigure}[ht]{0.4\textwidth}   
        \centering 
        \includegraphics[width=0.8\textwidth, trim={0 0 0 1cm}]{Figures/A_pen_case.png}
        \caption{Results of Pen Case in A-Split. 
        % The only applicable FoR is external relative.
        }  
    \end{subfigure}
    \caption{Red shows the wrong FoR identifications, and green shows the correct ones. The dark color is for relative FoRs, while the light color is for intrinsic FoRs. The round shape is for the external FoRs, while the square is for internal FoRs. The depth of the plots shows the four FoRs, i.e., \textit{external relative, external intrinsic, internal intrinsic, and internal relative}, \textbf{from front to back}.}
    \label{fig:cow_car_case}
\end{figure*}


\begin{figure*}[t]
    \centering
    \begin{subfigure}[t]{0.4\textwidth}
        \centering
        \includegraphics[width=0.88\textwidth, trim={0 0 0 0}]{Figures/A_cow_case_s.png}
        \caption{Results of Cow Case in A-Split. 
        % Valid predictions are external intrinsic and external relative.
        }
    \end{subfigure}%
    ~ 
    \begin{subfigure}[t]{0.4\textwidth}
        \centering
        \includegraphics[width=0.88\textwidth, trim={0 0 0 0}]{Figures/A_car_case_s.png}
        \caption{Results of Car Case in A-Split. 
        % All FoRs are valid predictions of this case.
        }
    \end{subfigure}
    
    \vskip\baselineskip
    
    \begin{subfigure}[t]{0.4\textwidth}   
        \centering 
        \includegraphics[width=0.88\textwidth, trim={0 0 0 0}]{Figures/A_box_case_s.png}
        \caption{Results of Box Case in A-Split. 
        % The correct predictions are external relative and internal relative.
        }    
    \end{subfigure}
        ~
    \begin{subfigure}[t]{0.4\textwidth}   
        \centering 
        \includegraphics[width=0.88\textwidth, trim={0 0 0 0}]{Figures/A_pen_case_s.png}
        \caption{Results of Pen Case in A-Split. 
        % The only applicable FoR is external relative.
        }  
    \end{subfigure}
    \caption{Red shows the wrong FoR identifications, and green shows the correct ones. The dark color is for relative FoRs, while the light color is for intrinsic FoRs. The round shape is for the external FoRs, while the square is for internal FoRs. The depth of the plots shows the four FoRs, i.e., external relative, external intrinsic, internal intrinsic, and internal relative, from front to back. This plot is the result of the rest of LLMs.}
    \label{fig:cow_car_case2}
\end{figure*}

\subsection{Results}

\subsubsection{FoR Inherently Bias in LLMs} 
\noindent\textbf{C-spilt.}
The \textit{zero-shot} setting reflects the LLMs' inherent bias in identifying FoR.
Table~\ref{tab:text_experiment} presents the accuracy for each FoR class in C-split, where sentences explicitly include information about topology and perspectives.
We found that some models strongly prefer specific FoR classes.
Notably, Gemme2-9B achieves a near-perfect accuracy on external relative FoR but performs poorly on other classes, especially external intrinsic, indicating a notable bias towards external relative. 
In contrast, GPT4o and Qwen2-72B perform exceptionally in both intrinsic FoR classes. However, they perform poorly in the relative FoRs.

\noindent\textbf{A-spilt.}
We examine the FoR bias in the A-split.
Based on the results in Table~\ref{tab:text_experiment}, we plotted the top-3 models' results (Gemma2-9B, Llama3-70B, and GPT4o) for a more precise analysis in Figures~\ref{fig:cow_car_case}. 
The plots show the frequencies of each FoR category. 
According to the plot, Gemma and GPT have strong biases toward external relative and external intrinsic, respectively. 
This bias helps Gemma2 perform well in the A-split since all spatial expressions can be interpreted as external relative. 
However, GPT4o's bias leads to errors when intrinsic FoRs aren't valid, as in the Box and Pen cases (see plots (c) and (d)).
Llama3 exhibits different behavior, showing a bias based on the relatum’s properties, specifically the relatum's affordance as a container.
In cases where relatum cannot serve as containers, i.e., Cow and Pen cases, Llama3 favors external relative. 
Conversely, Llama3 tends to favor external intrinsic when the relatum has the potential to be a container.

\subsubsection{Behavior with ICL variations}\label{sec:result_A_ICL}

\noindent\textbf{C-spilt.}
We evaluate the models’ behavior under various in-context learning (ICL) methods.
As observed in Table~\ref{tab:text_experiment}, the \textit{few-shot} method improves the performance of the \textit{zero-shot} method across multiple LLMs by reducing their original bias toward specific classes. 
Reducing the bias, however, lowers the performance in some cases, such as the performance of Gemma 2 in ER class.
One noteworthy observation is that while the \textit{CoT} prompting generally improves performance in larger LLMs, it is counterproductive in smaller models for some FoR classes. 
This suggests that the smaller models have difficulty inferring FoR from the longer context. 
This negative effect also appears in SG prompting, which uses longer explanations.
Despite performance degradation in particular classes of small models, SG prompting performs exceptionally well across various models and achieves outstanding performance with Qwen2-72B. 
We further investigate the performance of CoT and SG prompting. 
As shown in Table~\ref{tab:model_performance}, CoT exhibits a substantial difference in performance between contexts with inherently clear FoR and contexts requiring the template to clarify FoR ambiguity.
This implies that CoT heavily relies on the specific template to identify FoR classes. 
In contrast, SG prompting demonstrates a smaller gap between these two scenarios and significantly enhances performance over CoT in inherently clear FoR contexts.  
Therefore, guiding the model to provide characteristics regarding topological, distance, and directional types of relations improves FoR comprehension.

\noindent\textbf{A-spilt.}
We use the same Figure~\ref{fig:cow_car_case} to observe the behavior when applying ICL. 
The A-split shows minimal improvement with ICL variations, though some notable changes are observed.
With \textit{few-shot}, all models show a strong bias toward external intrinsic FoR, even when the relatum lacks intrinsic directions, i.e., Box and Pen cases. 
This bias appears even in Gemma2-9B, which usually behaves differently. 
This suggests that the models pick up biases from the examples despite efforts to avoid such patterns.
However, \textit{CoT} reduces some bias, leading LLMs to revisit relative, which is generally valid across scenarios. 
In Gemma2, the model predicts relative FoR where the relatum has intrinsic directions, i.e., Cow and Car cases.
Llama3 behaves similarly in cases where the relatum cannot act as a container, i.e., Cow and Pen cases.
GPT4o, however, does not depend on the relatum's properties and shows slight improvements across all cases.
Unlike \textit{CoT}, our SG prompting is effective in all scenarios.
It significantly reduces biases while following a similar pattern to \textit{CoT}. 
Specifically, SG prompting increases external relative predictions for Car and Cow in Gemma2-9B, and for Cow and Pen in Llama3-70B.
Nevertheless, GPT4o shows only a slight bias reduction.
However, Our proposed method improves the overall performance of most models, as shown in Table~\ref{tab:text_experiment}. 
The Llama3-70B behaviors are also seen in LLama3-8B and GPT3.5. 
The plots for LLama3-8B and GPT3.5 are in Figure~\ref{fig:cow_car_case2}.

\subsubsection{Experiment with different temperatures}
We conducted additional experiments to further investigate the impact of temperature on the biased interpretation of the model in the A-split of our dataset.
As presented in Table~\ref{tab:temp_table}, comparing distinct temperatures (0 and 1) revealed a shift in the distribution. The frequencies of the classes experienced a change of up to 10\%.
However, the magnitude of this change is relatively minor, and the relative preferences for most categories remained unchanged.
Specifically, the model exhibited the highest frequency responses for the cow, car, and pen cases, even with higher frequencies in certain settings. Consequently, a high temperature does not substantially alter the diversity of LLMs’ responses to this task, which is an intriguing finding.

\begin{table*}[t]
    \tiny
    \centering
    \begin{tabular}{|l|c c |c c| c c | c c |}
    \hline
    Model & \multicolumn{2}{|c|}{ER} & \multicolumn{2}{|c|}{EI} & \multicolumn{2}{|c|}{II} & \multicolumn{2}{|c|}{IR} \\
    & temp-0 & temp-1 & temp-0 & temp-1 & temp-0 & temp-1 & temp-0 & temp-1 \\
    \hline
    \multicolumn{9}{| l |}{Cow Case} \\
    \hline
     0-shot   & 75.38 &  87.12 & 23.86 & 12.50 & 0.76 & 0.13 & 0.00 & 0.25\\ 
     4-shot   & 0.00 &  15.66 & 100.00 & 84.34 & 0.00 & 0.00 & 0.00 & 0.00\\
     CoT & 31.82 & 49.87 & 68.18 & 49.87 & 0.00 & 0.13 & 0.00 & 0.13 \\
     SG & 51.39 & 70.45 & 48.61 & 29.42 & 0.00 & 0.00 & 0.00 & 0.13\\
     \hline
     \multicolumn{9}{| l |}{Box Case} \\
     \hline
     0-shot   & 22.50 &  41.67 & 77.50 & 58.33 & 0.00 & 0.13 & 0.00 & 0.25\\ 
     4-shot   & 0.00 &  0.00 & 100.00 & 100.00 & 0.00 & 0.00 & 0.00 & 0.00\\
     CoT & 0.00 &  5.83 & 100.00 & 94.17 & 0.00 & 0.00 & 0.00 & 0.00\\
     SG & 11.67 &  33.33 & 88.33 & 66.67 & 0.00 & 0.00 & 0.00 & 0.00\\
     \hline
     \multicolumn{9}{| l |}{Car Case} \\
     \hline
     0-shot   & 55.20 & 68.24 & 49.01 & 31.15 & 0.79 & 0.61 & 0.00 & 0.00\\ 
     4-shot   & 0.60 &  5.94 & 99.40 & 94.06 & 0.00 & 0.00 & 0.00 & 0.00\\
     CoT & 19.64 &  38.52 & 80.16 & 61.27 & 0.20 & 0.20 & 0.00 & 0.00\\
     SG & 44.25 &  56.97 & 55.75 & 43.03 & 0.00 & 0.00 & 0.00 & 0.00\\
     \hline
     \multicolumn{9}{| l |}{Pen Case} \\
     \hline
     0-shot   & 90.62 & 96.88 & 9.38 & 3.12 & 0.00 & 0.61 & 0.00 & 0.00\\ 
     4-shot   & 0.00  &  7.03 & 100.00 & 92.97 & 0.00 & 0.00 & 0.00 & 0.00\\
     CoT & 17.19 &  28.91 & 82.81 & 71.09 & 0.20 & 0.20 & 0.00 & 0.00\\
     SG & 48.31 &  57.81 & 54.69 & 42.19 & 0.00 & 0.00 & 0.00 & 0.00\\
     \hline
    \end{tabular}
    \caption{The results between two different temperatures of Llam3-70B on the A-spilt of FoREST. The number shows the percentage frequency of responses from the model.}
    \label{tab:temp_table}
\end{table*}

\section{In-context learning}\label{appendix:in-context}
\subsection{FoR Identification}
We provide the prompting for each in-context learning. The prompting for \textit{zero-shot} and \textit{few-shot} is provided in Listing~\ref{lst:base_instruction}. The instruction answer for these two in-context learning is ``Answer only the category without any explanation. The answer should be in the form of \{Answer: Category.\}"

For the Chain of Thought (CoT), we only modified the instruction answer to ``Answer only the category with an explanation. The answer should be in the form of \{Explanation: Explanation Answer: Category.\}" 
Similarly to CoT, we only modified the instruction answer to ``Answer only the category with an explanation regarding topological, distance, and direction aspects. The answer should be in the form of \{Explanation: Explanation Answer: Category.\}", respectively. The example responses are provided in Listing~\ref{lst:example_answer} for Spatial Guided prompting.

\begin{lstlisting}[caption={Prompt for finding the frame of reference class of given context.}, label={lst:base_instruction}]
# Instruction to find frame of reference class of given context
"""
Instruction: 
You specialize in language and spatial relations, specifically in the frame of context (multiple perspectives in the spatial relation). Identify the frame of reference category given the following context. There are four classes of the frame of reference (external intrinsic, internal intrinsic, external relative, internal relative). Note that the intrinsic direction refers to whether the model has the front/back by itself. (Example: a bird, human. Counter Example: a ball, a box). "

External intrinsic. The spatial description of an entity A relative to another entity B, where (1) A is not contained by B, (2) the spatial relation is based on B's facing orientation (intrinsic direction) if B has one.

Internal intrinsic. The spatial description of an entity A relative to another entity B, where (1) A is contained by B, (2) the spatial relation is based on B's facing orientation (intrinsic direction) if B has one.

External relative. The spatial description of an entity A relative to another entity B, where (1) A is not contained by B, (2) The spatial relation is relative to entity C, where C is not involved in the spatial relation and only provides the perspective. An example of C is the camera.

Internal relative. The spatial description of an entity A relative to another entity B, where (1) A is contained by B, (2) The spatial relation is relative to entity C, where C is not involved in the spatial relation and only provides the perspective. An example of C is the camera.

{Instruction answer}
# Normal Instruction answer: Find the frame of reference category without any explanations. The answer should be in the form of Answer: Category.
# COT Instruction answer: Find the frame of reference category with an explanation. The answer should be in the form of Explanation: Explanation. Answer: Category.
# SG Instruction answer: Find the frame of reference category with an explanation. Additionally, include three more explanations relating to topology, distance, and direction from the given context. The answer should be in the form of \"Explanation: Explanation. Topology: Explanation. Distance: Explanation. Direction: Explanation. Answer: Category.

Context: {spatial exprssion}
"""

\end{lstlisting}

\begin{lstlisting}[caption={Spatial expression examples using for few-shots, Chain-of-Thought (CoT), and Spatial Guide (SG) prompting}, label={lst:example_answerQA}]
## These are four examples used in few-shot, CoT, and SG
# Example 1: The bird is outside the car and in front of the car relative to the car.
# Few-shot response: Answer: external intrinsic
# CoT response: Explanation: The car does not contain the bird, and the spatial relation is relative to the car's facing orientation. The intrinsic direction of the car is relevant in this context, as the description \"in front of\" implies a direction relative to the sheep's front or back. Answer: external intrinsic.
# SG response: Explanation: The car does not contain the bird, and the spatial relation is relative to the car's facing orientation. The intrinsic direction of the car is relevant in this context, as the description \"in front of\" implies a direction relative to the sheep's front or back. Topology: The bird is outside the car. Distance: There is little distance between the bird and the car. Direction: The bird is in the front direction of the car (entity B). Answer: external intrinsic.

# Example 2: The bird is inside of the car and left of the car from the car's perspective.
# Few-shot response: Answer: internal intrinsic
# CoT response: Explanation: In this scenario, the bird is contained within the car, meaning it is an internal relationship. Additionally, the description specifies the bird's position relative to the car, indicating that the spatial relation is relative to the car itself, considering its intrinsic direction. Answer: internal intrinsic.
# SG response:  Explanation: In this scenario, the bird is contained within the car, meaning it is an internal relationship. Additionally, the description specifies the bird's position relative to the car, indicating that the spatial relation is relative to the car itself, considering its intrinsic direction. Topological: The bird is inside the car. Distance: The bird is in the same location as the car. Direction: The bird is in the left direction of the car (entity B). Answer: internal intrinsic.

# Example 3: The box is inside and at the back of the room.
# Few-shot response: Answer: internal relative
# CoT response: Explanation: In this context, the box (entity A) is described as inside the room (entity B), fulfilling the first condition for an internal frame of reference. However, the spatial relation given (\" at the back\") cannot originate from the room since the room does not have intrinsic direction, implying it is relative to an external observer or reference point (C), such as a camera or a person viewing the scene. Answer: internal relative
# SG response:  Explanation: In this context, the box (entity A) is described as inside the room (entity B), fulfilling the first condition for an internal frame of reference. However, the spatial relation given (\" at the back\") cannot originate from the room since the room does not have intrinsic direction, implying it is relative to an external observer or reference point (C), such as a camera or a person viewing the scene. Topological: The box is inside the room. Distance: The box is in the same location as the room. Direction: From the observer's perspective, the box is located at the back of the room. Answer: internal relative

# Example 4: A phone is on the left of a tablet from my perspective.
# Few-shot response: Answer: external relative
# CoT response: Explanation: In this context, the spatial relation is described from the perspective of an observer (C) who is not involved in the spatial relation. The phone (A) is not contained by the tablet (B), and the spatial relation \"left of\" is relative to the observer's perspective, not the tablet's facing orientation. Answer: external relative.
# SG response:  Explanation: In this context, the spatial relation is described from the perspective of an observer (C) who is not involved in the spatial relation. The phone (A) is not contained by the tablet (B), and the spatial relation \"left of\" is relative to the observer's perspective, not the tablet's facing orientation. Topological: The phone is not contained by the tablet. Distance: There is some distance between the phone and the tablet. Direction: From the observer's perspective, the phone is located to the left of the tablet. Answer: external relative.
\end{lstlisting}

\subsection{Question Answering Example}\label{appendix:QA_example}


\begin{lstlisting}[caption={Spatial expression examples using for few-shots, Chain-of-Thought (CoT), and Spatial Guide (SG) prompting for question-answering.}, label={lst:example_answerQA}]
'''
#Context: The bird is outside the car and in front of the car relative to the car. The car is facing toward the camera. 
#Question: Based on the camera's perspective, where is the bird from the car position in the scene? 
#SG information (used in SG + CoT): Frame of reference Explanation: The car does not contain the bird, and the spatial relation is relative to the car's facing orientation. The intrinsic direction of the car is relevant in this context, as the description \"in front of\" implies a direction relative to the sheep's front or back. Topology: The bird is outside the car. Distance: There is little distance between the bird and the car. Direction: The bird is in the front direction of the car (entity B). Frame of reference: external intrinsic.
#Normal response: Answer: front
#CoT / (SG + COT) response:Explanation: Based on the context, the bird's position is in the front direction of the car. The car is facing toward the camera. Then, the car's front direction is the camera's front direction. Therefore, the bird's position is in front of the car's position from the camera's perspective. Answer: front

#Context: The bird is inside the car and left of the car from the car's perspective. The car is facing to the right relative to the camera.  #Question: Based on the camera's perspective, where is the bird from the car's position?
#SG information (used in SG + CoT):  Frame of reference Explanation: In this scenario, the bird is contained within the car, meaning it is an internal relationship. Additionally, the description specifies the bird's position relative to the car, indicating that the spatial relation is relative to the car itself, considering its intrinsic direction. Topological: The bird is inside the car. Distance: The bird is in the same location as the car. Direction: The bird is in the left direction of the car (entity B). Frame of reference: internal intrinsic.
#Normal response: Answer: back
#CoT / (SG + COT) response: Explanation: Based on the context, the bird's position is in the left direction of the car. The car is facing to the right. Then, the car's left direction is the camera's back direction. Therefore, the bird's position is to the back of the car's position from the camera's perspective. Answer: back

Context: The box is inside and to the right of the room from the observer's perspective. 
Question: From the observer's perspective, what is the spatial relation of the box to the room? 
#SG information (used in SG + CoT): Frame of reference Explanation: In this context, the box (entity A) is described as inside the room (entity B), fulfilling the first condition for an internal frame of reference. However, the spatial relation given (\" at the back\") cannot originate from the room since the room does not have intrinsic direction, implying it is relative to an external observer or reference point (C), such as a camera or a person viewing the scene. Topological: The box is inside the room. Distance: The box is in the same location as the room. Direction: From the observer's perspective, the box is located at the back of the room. Frame of reference: internal relative.
#Normal response: Answer: right
#CoT / (SG + COT) response: Explanation: Based on the context, the box is to the right of the room from the camera's direction. Therefore, the box's position is to the right of the room's position from the observer's perspective. Answer: right

Context: A phone is to the left of a tablet from my perspective. The tablet is facing to the right. Question: From my perspective, what is the spatial relation of the phone to the tablet?
#SG information (used in SG + CoT): Frame of Reference Explanation: In this context, the spatial relation is described from the perspective of an observer (C) who is not involved in the spatial relation. The phone (A) is not contained by the tablet (B), and the spatial relation \"left of\" is relative to the observer's perspective, not the tablet's facing orientation. Topological: The phone is not contained by the tablet. Distance: There is some distance between the phone and the tablet. Direction: From the observer's perspective, the phone is located to the left of the tablet. Frame of Reference: external relative.
#Normal response: Answer: left
#CoT / (SG + COT) response: Explanation: Based on the context, the phone is to the left of the tablet from my perspective. The direction of the tablet is not relevant in this context since the left relation is from my perspective. Therefore, from my perspective, the phone is to the left of the tablet. Answer: left
'''
\end{lstlisting}

\subsection{Text to Layout}
\begin{lstlisting}[caption={Prompt for generating bounding coordinates to use as the layout for layout-to-image models.}, label={lst:example_answer}]
    # Instruction for generating bounding box
"""
Your task is to generate the bounding boxes of objects mentioned in the caption.
The image is size 512x512. The bounding box should be in the format of (x, y, width, height). Please considering the frame of reference of caption and direction of reference object if possible. If needed, you can make the reasonable guess.
"""
\end{lstlisting}







 

\end{document}