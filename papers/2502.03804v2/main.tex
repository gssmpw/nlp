\documentclass[sigconf]{acmart}

%%
%% \BibTeX command to typeset BibTeX logo in the docs
\AtBeginDocument{%
  \providecommand\BibTeX{{%
    Bib\TeX}}}

\setcopyright{acmlicensed}
\copyrightyear{2025}
\acmYear{2025}
\setcopyright{cc}
\setcctype{by}
\acmConference[CHI '25]{CHI Conference on Human Factors in Computing Systems}{April 26-May 1, 2025}{Yokohama, Japan}
\acmBooktitle{CHI Conference on Human Factors in Computing Systems (CHI '25), April 26-May 1, 2025, Yokohama, Japan}\acmDOI{10.1145/3706598.3714016}
\acmISBN{979-8-4007-1394-1/25/04}

\usepackage{enumerate}
\usepackage{multirow}
\usepackage{array} % required for text wrapping in tables
\usepackage{listings}
\usepackage{graphicx}

% \newcommand{\red}{\textcolor[rgb]{0.757,0.153,0.212}}
\newcommand{\red}{\textcolor[rgb]{0,0,0}}
% \newcommand{\blue}{\textcolor[rgb]{0.082, 0, 1}}
\newcommand{\blue}{\textcolor[rgb]{0,0,0}}
\newcommand{\green}{\textcolor[rgb]{0.180, 0.518, 0.349}}

\begin{document}
\title{\red{Understanding and Supporting Formal Email Exchange\\by Answering AI-Generated Questions}}

\author{Yusuke Miura}
\email{miura.yusuke@toki.waseda.jp}
\orcid{0000-0003-1204-6623}
\affiliation{%
  \institution{Waseda University}
  \city{Tokyo}
  % \state{Tokyo}
  \country{Japan}
}

\author{Chi-Lan Yang}
\email{chilan.yang@cyber.t.u-tokyo.ac.jp}
\orcid{0000-0003-0603-2807}
\affiliation{%
  \institution{The University of Tokyo}
  \city{Tokyo}
  % \state{Tokyo}
  \country{Japan}
}

\author{Masaki Kuribayashi}
\email{rugbykuribayashi@waseda.jp}
\orcid{0000-0001-8412-223X}
\affiliation{%
  \institution{Waseda University}
  \city{Tokyo}
  % \state{Tokyo}
  \country{Japan}
}

\author{Keigo Matsumoto}
\email{matsumoto@cyber.t.u-tokyo.ac.jp}
\orcid{0000-0002-0038-0678}
\affiliation{%
  \institution{The University of Tokyo}
  \city{Tokyo}
  \country{Japan}
}

\author{Hideaki Kuzuoka}
\email{kuzuoka@cyber.t.u-tokyo.ac.jp}
\orcid{0000-0003-1252-7814}
\affiliation{%
  \institution{The University of Tokyo}
  \city{Tokyo}
  % \state{Tokyo}
  \country{Japan}
}

\author{Shigeo Morishima}
\email{shigeo@waseda.jp}
\orcid{0000-0001-8859-6539}
\affiliation{%
  \institution{Waseda Research Institute for Science and Engineering}
  \city{Tokyo}
  % \state{Tokyo}
  \country{Japan}
}

\renewcommand{\shortauthors}{Miura et al.}

\begin{abstract}
% % 150words
% Replying to workplace emails that are typically long and require politeness is time-consuming and cognitively demanding.
% Replying to lengthy and polite workplace emails is often time-consuming and cognitively demanding.
% takes time to understand and reply
\red{Replying to formal emails is time-consuming and cognitively demanding, as it requires crafting polite phrasing and providing an adequate response to the sender's demands.}
% \red{Replying to formal emails, which often takes time to understand and require polite phrasing, is time-consuming and cognitively demanding.}
Although systems with Large Language Models (LLM) were designed to simplify the email replying process, users still need to provide detailed prompts to obtain the expected output.
Therefore, we proposed and evaluated an \red{LLM-powered question-and-answer (QA)-based approach} for users to reply to emails by answering a set of simple and short questions generated from the incoming email.
We developed a prototype system, \textit{ResQ}, and conducted controlled and field experiments with 12 and \red{8} participants.
Our results demonstrated that \red{the QA-based approach} improves the efficiency of replying to emails and reduces workload while maintaining email quality, compared to a conventional prompt-based approach that requires users to craft appropriate prompts to obtain email drafts.
We discuss how \red{the QA-based approach} influences the email reply process and interpersonal relationship dynamics, as well as the opportunities and challenges associated with using a QA-based approach in AI-mediated communication.

% original
% Replying to lengthy and polite workplace emails is often time-consuming and cognitively demanding.
% Although systems with Large Language Models were designed to simplify the email replying process, users still needed to provide detailed prompts to obtain the expected output.
% Therefore, we proposed and evaluated a question-and-answer-based approach for users to reply to emails by answering a set of simple and short questions generated from the incoming email.
% We developed a prototype system, \textit{ResQ}, and conducted both controlled and field experiments with 12 and 9 participants.
% Our results demonstrated that ResQ improves the efficiency of replying to emails and reduces workload while maintaining email quality compared to a conventional prompt-based approach that requires users to craft appropriate prompts to obtain email drafts.
% We discuss how ResQ influences the email reply process and interpersonal relationship dynamics, as well as the opportunities and challenges associated with using a QA-based approach in AI-mediated communication.
\red{Replying to formal emails is time-consuming and cognitively demanding, as it requires crafting polite phrasing and providing an adequate response to the sender's demands.}
% \red{Replying to formal emails, which often takes time to understand and require polite phrasing, is time-consuming and cognitively demanding.}
Although systems with Large Language Models (LLMs) were designed to simplify the email replying process, users still need to provide detailed prompts to obtain the expected output.
Therefore, we proposed and evaluated an \red{LLM-powered question-and-answer (QA)-based approach} for users to reply to emails by answering a set of simple and short questions generated from the incoming email.
We developed a prototype system, \textit{ResQ}, and conducted controlled and field experiments with 12 and \red{8} participants.
Our results demonstrated that \red{the QA-based approach} improves the efficiency of replying to emails and reduces workload while maintaining email quality, compared to a conventional prompt-based approach that requires users to craft appropriate prompts to obtain email drafts.
We discuss how \red{the QA-based approach} influences the email reply process and interpersonal relationship dynamics, as well as the opportunities and challenges associated with using a QA-based approach in AI-mediated communication.
\end{abstract}

\begin{CCSXML}
<ccs2012>
   <concept>
       <concept_id>10003120.10003130.10011762</concept_id>
       <concept_desc>Human-centered computing~Empirical studies in collaborative and social computing</concept_desc>
       <concept_significance>500</concept_significance>
       </concept>
 </ccs2012>
\end{CCSXML}

\ccsdesc[500]{Human-centered computing~Empirical studies in collaborative and social computing}

\keywords{AI-Mediated Communication, Large Language Models, Email}

\begin{teaserfigure}
  \includegraphics[width=\textwidth]{teaser.pdf}
  \caption{In our system, (1) users receive an email, (2) communicate their intentions by answering AI-generated questions, (3) receive an AI-generated draft, (4) make any necessary revisions, and finally (5) send the reply.  This process allows users to craft responses efficiently, reducing their overall workload.}
  \Description{This figure illustrates the five stages of using our proposed system. First, the user receives an email on their computer, marking the beginning of the workflow. In the second stage, the user answers a set of predefined questions related to the email, which can include options like "YES" or "NO" as well as multiple-choice selections (A, B, C). Based on the user's responses, the AI then generates a draft of the email reply in the third stage. In the fourth stage, the user reviews the AI-generated draft, checking for errors or making revisions as needed. Finally, in the fifth stage, the user sends the revised email reply, completing the process.}
  \label{fig_teaser}
\end{teaserfigure}

\maketitle

\section{Introduction}

\begin{figure*}
    \centering
    \includegraphics[width=\textwidth]{figures/Introduction.pdf}
    \caption{Showing the novel problem statement applied to traffic prediction use case. Multiple unstructured observations from the past are used to reconstruct a hidden traffic state from which a full traffic state is forecast with a set of query locations. }
    \label{fig:intro}
\end{figure*}

% Was sagen denn die anderen warum Traffic Prediction gut ist? 
Forecasting the traffic in the near future is an important task for city management.
Data from the near past is used to predict future traffic states with spatio-temporal Graph Neural Networks \cite{bui22}.
Accurate prediction provides the opportunity to optimize traffic flow, reduce traffic jams and increase air quality \cite{Po19}.

% Wieso ist Sparsity in allen Dimensionen wichtig.
While traffic prediction relies on the availability of data from traffic sensors, there exists a plethora of reasons why sensors may stop working temporarily, such as simple errors, energy saving, or overloaded communication systems.
Considering small- or medium-sized cities, the coverage of sensors may be low because the sensors are too expensive or not available.
Also, the sensors are typically static and do not adapt to changes in the traffic flow (e.g. caused by a construction site), which motivates moving sensors that for example could be mounted on cars. 
However, both missing and moving sensors introduce sparsity, since measurements may not be available for all locations at all times.
This sparsity must be explicitly addressed in traffic prediction for a realistic application scenario, which is illustrated in figure \ref{fig:intro}.
From one hour of data on Sunday morning, only few observations of the traffic state are available at each timestep.
The number of observations may differ throughout the observed time and the observation itself can be distributed arbitrarily in the city. 
We assume a relatively low number of sensors to account for resource saving and sensor failure in our proposed framework SUSTeR.
The task is to predict the dense traffic state one timestep after the observations at all possible sensor locations.
We study this problem on the traffic dataset Metr-LA and PEMS-BAY to test our assumption that only a fraction of the sensor values would be enough for good predictions.
By modifying an existing traffic dataset, we are able to compare our results from very sparse observations to the bottom line with all information available.
A successful study will provide insights in how sensors in new cities can be reduced before installing them and further mobile sensors would save more resources and are able to adapt to new traffic situations.
We argue that in order to be adaptable to other cities and changes in traffic flows, prior information like the road network should be neglected and just the sparse observations considered.
This comes with the added benefit of making our solution applicable in regions where no openly available road network is maintained or pathways change frequently (e.g. flood areas, animal observations). 


The aforementioned problem is novel and more challenging than the commonly considered traffic prediction problem, since there exist very few observations in each input sample.
Current works for the traffic prediction problem do not consider any missing values. \cite{Li2021, Shao22}
A common method among state of the art approaches is the usage of Graph Neural Networks on graphs that model the sensor network \cite{bui22}.
The values of a sensor are applied to the same graph node for each timestep which prohibits any non-stationary sensors . 
With fixed sensor locations, the resulting sensor network is highly correlated with the road network.
Streets connecting two intersections with sensors should be also an interesting point for correlations in the sensor network.
However, variable observations and high temporal sparsity rates can not be modeled adequately in a static network.
We show in our experiments that the road network has only a small influence on the traffic predictions.

Besides the traffic prediction for future timesteps, some works explore the field of traffic speed imputation \cite{Cini22, Cuza22} where missing sensor values are predicted.
But the amount of missing values is assumed to be at most 80\%, which on average are still over 40 given sensors in each timestep in the Metr-LA dataset with a total of 207 sensors.
We consider up to 99.9\% missing values which are on average 2.4 observations in each timestep that are used as input.
Such high sparsity rates drastically decrease the chance that multiple values are present in one input sample from the same sensor location, which makes it challenging to recognize and learn temporal correlations for each location on its own.

High sparsity rates (>95\%) result in few sensor values, but if a reconstruction of the traffic state would be possible, we question if spatio-temporal graphs require nodes for each sensor.
In SUSTeR we utilize only a small amount of graph nodes for the encoding of information and do not relate such nodes to the sensor network.
We call this the hidden graph (see figure \ref{fig:intro}), which is still able to reconstruct the complete traffic state.
Due to the reduced number of nodes SUSTeR achieves faster runtimes, as shown in the experiments.
This hidden graph is not embedded directly in the spatial domain, which is why the assignment of observations, as well as the querying of the future traffic, is done with an encoder and a decoder, implemented as neural networks.
The decoding from the hidden graph to future values depends on a set of query locations.
Figure \ref{fig:intro} shows the query locations as given from outside and in combination with the reconstructed traffic state the future values are predicted.

To construct the hidden graph we encode observations from each timestep into from multiple graphs, one for each timestep. 
The graphs are created in a residual style and information is added to the node embeddings from the previous timesteps.
We choose this method to incorporate all timesteps equally into the hidden state because the redundant information along the past is non-existing for high sparsity rates.
From the sequence of graphs where our framework inserted the observations step by step we apply STGCN \cite{Yu18}, an algorithm for traffic prediction to find and learn the spatio-temporal correlations on our small number of graph nodes.
The first future timestep of the STGCN is our hidden graph in which the traffic state is reconstructed. 

% Recent work has an implicit embedding of the graph nodes into the spatial domain as the assignment from the sensor to graph node is fixed one by one.
% Because the graph has the same structure as the road network spatio-temporal correlations can be learned between those sensors.
% We reduce the number of nodes and use a non-linear assignment learned data-driven from the observations.

We find in the experiments that SUSTeR outperforms the plain STGCN and modern traffic prediction frameworks like D2STGNN for high sparsity rates $(\geq 99\%)$.
This is equivalent to only $0.2$ to $2.4$ observation for each timestep on average.
SUSTeR uses fewer parameters than the baselines and can train faster and with less training data.
Our main contributions can be summarized as follows:
\begin{itemize}
    \item We introduce a sparse and unstructured variant of the traffic prediction problem with sparsity in all dimensions. The sensors report only a fraction of their values and are arbitrarily distributed in the spatial domain.
    \item We propose SUSTeR, a framework around the STGCN architecture, which maps sparse observations onto a dense hidden graph to reconstruct the complete traffic state.
    Our code is available at github.\footnote{https://github.com/ywoelker/SUSTeR}
    \item We conducts experiments that show that SUSTeR outperforms the baselines in very sparse situations ($\geq 95\%$) and has a competitive performance in low sparsity rates.
    % \item SUSTeR trains a third faster than the next competitor.
\end{itemize}

\section{Related Work}
\label{sec:related_work}

\subsection{Robustness of Audio-Visual Speech Recognition} 

The robustness of AVSR systems has significantly advanced by integrating auditory and visual cues to improve speech recognition, especially in noisy environments. Conventional ASR methods have evolved from relying solely on audio signals \cite{schneider2019wav2vec, gulati2020conformer, baevski2020wav2vec, hsu2021hubert, chen2022wavlm, chiu2022self, radford2023robust} to incorporating visual data from speech videos \citep{makino2019recurrent}.
The multimodal AVSR methods \citep{pan2022leveraging, shi2022learning, seo2023avformer, ma2023auto} have enhanced robustness under audio-corrupted conditions, leveraging visual details like speaker's face or lip movements as well as acoustic features of speech. These advancements have been driven by various approaches, including end-to-end learning frameworks \citep{dupont2000audio, ma2021end, hong2022visual, burchi2023audio} and self-supervised pretraining \citep{ma2021lira, qu2022lipsound2, seo2023avformer, zhu2023vatlm, kim2025multitask}, which focus on audio-visual alignment and the joint training of modalities~\citep{zhang2023self, lian2023av, haliassos2022jointly, haliassos2024braven}.


Furthermore, recent advancements in AVSR highlight the importance of visual understanding alongside audio \citep{dai2024study, kim2024learning}. While initial research primarily targeted audio disturbances \citep{shi2022robust, hu2023hearing, hu2023cross, chen2023leveraging}, latest studies increasingly focus on the visual robustness to address challenges such as real-world audio-visual corruptions~\citep{hong2023watch, wang2024restoring, kim2025multitask} or modality asynchrony~\citep{zhang2024visual, fu2024boosting, li2024unified}. These efforts remark a shift towards a more balanced use of audio and visual modalities. Yet, there has been limited exploration in scaling model capacity or introducing innovative architectural designs, leaving room for further developments in AVSR system that can meticulously balance audio and visual modalities.



\subsection{MoE for Language, Vision, and Speech Models}

Mixture-of-Experts (MoE), first introduced by \citet{jacobs1991adaptive}, is a hybrid structure incorporating multiple sub-models, \ie experts, within a unified framework. The essence of sparsely-gated MoE \cite{shazeer2017outrageously, lepikhin2021gshard, dai2022stablemoe} lies in its routing mechanism where a learned router activates only a subset of experts for processing each token, significantly enhancing computational efficiency. Initially applied within LLMs using Transformer blocks, this structure has enabled unprecedented scalability \cite{fedus2022switch, zoph2022st, jiang2024mixtral, guo2025deepseek} and has been progressively adopted in multimodal models, especially in large vision-language models (LVLMs) \cite{mustafa2022multimodal, lin2024moellava, mckinzie2025mm1}.
Among these multimodal MoEs, \citet{zhu2022uni, shen2023scaling, li2023pace, li2024uni} and \citet{lee2025moai} share the similar philosophy to ours, assigning specific roles to each expert and decoupling them based on distinct modalities or tasks. These models design an expert to focus on specialized segments of input and enhance the targeted processing.

Beyond its applications in LLMs and LVLMs, the MoE framework has also been applied for speech processing \cite{you2021speechmoe, you2022speechmoe2, hu2023mixture, wang2023language}, where it has shown remarkable effectiveness in multilingual and code-switching ASR tasks. In addition, MoE has been employed in audio-visual models \cite{cheng2024mixtures, wu2024robust}, although they primarily focus on general video processing and not specifically on human speech videos. These approaches leverage MoE to model interactions between audio and visual tokens without directly processing multimodal tokens.
Our research advances the application of the MoE framework to AVSR by designing a modality-aware hierarchical gating mechanism, which categorizes experts into audio and visual groups and effectively dispatches multimodal tokens to each expert group. 
This tailored design enhances the adaptability in managing audio-visual speech inputs, which often vary in complexity due to diverse noise conditions.

\section{Research Questions and Hypotheses}
This paper aims to explore the effectiveness and potential risks of a QA-based response-writing support method by addressing the following three research questions:
% \begin{enumerate}[RQ1:]
%     \item How does a QA-based response-writing support approach affect users’ email-replying process?
%     \item How does a QA-based response-writing support approach affect the quality of the email response?
%     \item How does a QA-based response-writing support approach affect the perceived relationship between email sender and recipient?
% \end{enumerate}

\begin{enumerate}[RQ1:]
    \item How does a QA-based response-writing support approach affect users’ email-replying process?
    \item How does a QA-based response-writing support approach affect the quality of the email response?
    \item How does a QA-based response-writing support approach affect the perceived relationship between email sender and recipient?
\end{enumerate}

To answer the three research questions, we formed three sets of hypotheses.
The first set of hypotheses investigates the impact of a QA-based system on users' email-replying process.
AI-powered text generation reduces user input, saves time, and enhances efficiency~\cite{bastola2024llmbasedsmartreplylsr}. 
It also helps users quickly grasp email content with less cognitive effort, particularly through text summarization and list formatting, which enhances productivity~\cite{tarnpradab2017toward, nandhini2013use, modaresi2017commercial, daniel1998influence}. 
This suggests that presenting questions in a list format could streamline email responses, reducing the need for detailed prompts. 
Based on these insights, we propose the following hypotheses:
% \begin{enumerate}[\textrm{H1-}a:]
%     \item QA-based system enhances users’ email replying efficiency.
%     \item QA-based system reduces users’ cognitive load while replying to email.
% \end{enumerate}

\begin{enumerate}[\textrm{H1-}a:]
    \item QA-based system enhances users’ email replying efficiency.
    \item QA-based system reduces users’ cognitive load while replying to email.
\end{enumerate}

As a result, we expect users’ perceived work efficiency to improve.
Furthermore, since the QA-based system suggests appropriate language and helps create responses that align with the recipient’s needs, we anticipate that users’ satisfaction with their email replies will increase.
Thus, we propose the following hypothesis:
\begin{enumerate}[\textrm{H1-}c:]
    \item QA-based system enhances users' satisfaction with completing email response tasks, thereby being favorably received by users.
\end{enumerate}

Moreover, as described above, reducing users’ burden and improving their satisfaction may enhance their confidence in their tasks, which could lower their hesitation to begin working~\cite{schouwenburg1992procrastinators}.
Additionally, AI outputs that engage users’ curiosity may help trigger task initiation~\cite{brandtzaeg2017why, ling2021factors}.
Thus, we propose the following hypothesis:
\begin{enumerate}[\textrm{H1-}d:]
    \item QA-based system lowers the barriers to initiating email response tasks.
\end{enumerate}

According to previous research, there is a trade-off between the degree of AI intervention and the sense of agency and control, with higher levels of AI involvement shown to diminish these perceptions~\cite{Fu2023Comparing, Draxler2024The}.
Given that our QA-based approach also involves AI intervention during the phase where users create prompts for the LLM, the following hypothesis can be derived:
\begin{enumerate}[\textrm{H1-}e:]
    \item QA-based system diminishes users’ sense of agency and reduces their sense of control of the content.
\end{enumerate}

The second hypothesis concerns the quality of email responses.
AI support can be helpful in ensuring appropriate language use and grammar~\cite{fu2024text}. 
Furthermore, the QA-based approach is expected to assist users in correctly understanding the intent and demands of received emails and in verifying whether their responses meet these requirements.
Based on this, we propose the following hypothesis:
% \begin{enumerate}[\textrm{H2}:]
%     \item QA-based system enhances the perceived quality of the email response.
% \end{enumerate}
\begin{enumerate}[\textrm{H2}:]
    \item QA-based system enhances the perceived quality of the email response.
\end{enumerate}

The third set of hypotheses investigates the perceived relationship between email sender and recipient.
When users use the QA-based approach, it is expected that their communication partners will receive high-quality messages more quickly. 
Thus, the following hypothesis is derived.
\begin{enumerate}[\textrm{H3-}a:]
    \item QA-based system makes a good impression on the user's communication partner.
\end{enumerate}

On the other hand, when users create messages using the AIMC tool, they may feel a sense of discomfort with the message and guilt for not having fully composed it themselves~\cite{fu2024text}. 
We hypothesized that a QA-based approach would further intensify this discomfort by reducing the user’s sense of agency and control more than previous approaches.
\begin{enumerate}[\textrm{H3-}b:]
    \item QA-based system enlarges the psychological distance that users perceive toward their communication partners.
\end{enumerate}
% 1. how many questions did we specify and how? Now it said appropriate number of questions, but it's unclear what is "appropriate"
% 2. How did the question generated? based on what criteria? for instance, did we ask gpt to generate random questions? if not, what exact did gpt do? would be better to show one example of prompt in this part.

\section{Proposed LLM-Powered QA-Based Approach: ResQ}
\label{sec:Proposed_Approach}
% 本セクションでは、電子メール対応タスクをサポートするために提案されたアプローチ「ResQ」について説明する
This section describes the proposed approach, ResQ, for supporting email response tasks. 
\begin{figure*}[t]
\centering
\includegraphics[width=\textwidth]{figure/overview_of_process.pdf}
\caption{The overview of the process of creating a reply message using ResQ. A) The LLM first generates multiple-choice questions in JSON format. B) Users select their desired responses to their counterparts. C) The LLM then generates a reply draft in JSON format based on the users' selections. D) Finally, users review and edit the LLM-generated draft before sending the reply.}
\label{fig_overview_of_process}
\Description{This figure illustrates the process of generating an email reply using a large language model (LLM) across four stages. In Stage A (Generate Questions), the system takes the email data and a prompt, which are then sent to the LLM server. The server processes this information and generates a set of questions to clarify the content of the reply. These questions, along with possible answer choices and context, are returned in JSON format. In Stage B (Answer Questions), the user is presented with the questions generated by the LLM. These questions may include simple "Yes" or "No" options or multiple-choice selections. The user answers the questions by choosing the appropriate options or providing custom responses. In Stage C (Generate Reply Draft), the user’s answers, along with the original email data and prompt, are sent back to the LLM server. Based on this input, the server generates a draft of the email reply, which is also returned in JSON format. In Stage D (Check Reply Draft), the user reviews the draft generated by the LLM. After checking the content and making any necessary revisions, the user finalizes and sends the email reply.}
\end{figure*}
\begin{figure*}[t]
\centering
\includegraphics[width=\textwidth]{figure/interface.pdf}
\caption{Interface of ResQ. On the left, the content of the email is displayed, with an editor and a ``Reply'' button below for sending a reply. In the center, questions and options for users are shown, allowing the creation of custom options if needed. Additionally, the section of the email corresponding to the selected question is highlighted. On the right, fields are provided to customize the reply generated by the LLM, including options to specify the relationship with the counterpart and buttons to choose the formality, tone, and length of the email. A free-text input field and a "Generate Reply" button are also below.
}
\label{fig_interface}
\Description{This figure illustrates the interface of the ResQ system, which is divided into three main sections. On the left side (A, B), the content of the incoming email is displayed. The email includes important information, such as the sender, subject, and body text. Key sections of the email are highlighted based on the questions generated by the system, helping the user focus on relevant points. Below the email, there is an editor where the user can compose their reply, with a "Reply" button (H) available to send the response once it's ready. In the center section (C, D), the system displays questions generated by the AI, which are intended to assist the user in composing their reply. These questions correspond to specific parts of the email, and as the user answers them, the relevant section in the email is highlighted (A). Users can also customize responses by adding new options if needed (D). user can specify their relationship with the email recipient (e.g., professor or student), adjust the formality and tone of the response, and select the desired length of the reply. An additional free-text input field is available for further customization requests (E). Once all preferences are set, the user can click the "Generate Reply" button (F) to produce a draft response based on their inputs.}
\end{figure*}
Fig.~\ref{fig_overview_of_process} illustrates the overview of how a reply message is created using ResQ.
% (A) After the system detects users' initiation of the reply task, it generates multiple-choice questions using an LLM (in this study, GPT-4o~\cite{GPT4o}). 
% (B) Then, users communicate their reply strategy to the AI by responding to these questions.
% (C) When the system detects that users have pressed the ``Generate Reply'' button, it presents a draft of the reply to users.
% (D) Finally, users review and revise the reply draft generated by the LLM and send the response.
Fig.~\ref{fig_interface} shows the actual interface of ResQ.
The following sections describe the specific functions involved in each step of this process.

\paragraph{\textbf{A: Generate Questions}}
\label{sec:generate_questions}
% ResQは返信の必要性を検知すると、大規模言語モデル (本研究ではGPT-4o) を使用して、多肢選択式の質問を生成する
% また我々は、ユーザが質問をクリックすると、その質問が受信メールのどの部分に対応しているかがハイライトされるようにした
% 我々はこれらの機能を実現するために、まずモデルに対して、モデルの役割(ユーザに対する質問と有益そうな選択肢のペアを複数生成すること)と、作成する質問の目的(メールに含まれる全ての要求を抽出し、送信者がそれぞれに対してどのように返答したいかを明確にすること)をプロンプトとして与えた
% さらに、モデルに受信メールの文章、タイトル、送信者の情報(名前、メールアドレス)、受信メールの過去のやり取りの文章、ユーザの情報(名前、メールアドレス)を提供した
When a user first activates ResQ, the system uses an LLM (in this study, GPT-4o~\cite{GPT4o}) to generate multiple-choice questions (Fig.~\ref{fig_interface}-C). 
The LLM extracts all parts of the email that require a reply, generates corresponding questions and presents possible response options.
Additionally, if a user clicks on any generated question, the relevant part of the email is highlighted (Fig.~\ref{fig_interface}-A).

% To implement these features, we first provided the LLM with the email's text, subject, sender information, text from past email interactions, and the user's information (name and email address).
% プロンプトは~\cite{relatedwork}を参考に作成し、文脈を踏まえており、返信を作成する上で役に立ち、適切な数の(メールのすべての要件に対しては漏れなく)質問と、それに役立つ選択肢を生成するように指示した。
% またプロンプトには質問と選択肢の生成例を含めることで、生成の質を上げた。
\red{Following the approach described in~\cite{bsharat2023principled}, we designed a structured prompt that guides the LLM to determine how many questions are necessary and sufficient to cover all requirements in the incoming email without omission. 
Instead of pre-specifying a fixed number of questions, the prompt instructs the model to produce an ``appropriate'' number of questions, where ``appropriate'' is defined as the minimal set of questions needed to address all points raised by the sender while avoiding redundant or irrelevant inquiries.
To ensure that the questions were generated systematically rather than randomly, we provided explicit criteria within the prompt. 
These criteria included referencing the sender's intent, quoting relevant portions of the original email verbatim, and offering multiple-choice options where applicable. 
We also provided concrete examples within the prompt to illustrate the desired format and style of the generated questions and corresponding answer choices. 
By doing so, we ensured that the LLM's output was both well-grounded and easy for the recipient to answer.
The detailed prompt used to guide the LLM in this process is shown in the appendix.}

% We designed the prompt with reference to~\cite{bsharat2023principled}, incorporating contextual considerations to ensure it effectively supports reply composition. 
% We instructed the LLM to generate an appropriate number of questions that comprehensively cover all the email's requirements without omissions, along with relevant response options. 
% To further guide the model and improve output accuracy, we included examples of question and option generation within the prompt.

\paragraph{\textbf{B: Answer Questions}}
% 次にユーザは受信メールと生成された質問、選択肢を同時に見ながら、質問に回答する
% 我々は有益な選択肢がない場合を想定して、ユーザ自身が選択肢を追加できるようにした
% また、LLMにメールのcontextを伝えることができるように、送信者と受信者の関係性を記入できるboxを設置した
% さらに以前の研究に基づき~\cite{fu2024text}、ユーザがAIメールの文章をカスタマイズできるように、ユーザが期待する返信のトーンやスタイル、長さを調整するための選択肢を提供した
% また、ユーザがそれ以外のリクエストをAIに対してできるように、AIに対する自由記述欄を設置した
% ユーザはこれらの作業が完了するとGenerate Replyボタンを押す
Next, users view the incoming email (Fig.~\ref{fig_interface}-B) alongside the generated questions (Fig.~\ref{fig_interface}-C) and options and proceed to answer them. 
In anticipation of situations where none of the provided options are useful, we enabled users to add their own options (Fig.~\ref{fig_interface}-D). 
Additionally, to help the LLM better understand the context of the email, we introduced a box where users can specify the relationship between the sender and the recipient (Fig.~\ref{fig_interface}-E, top). 
Furthermore, following previous research~\cite{fu2024text}, we provided users with controls to adjust the tone, style, and length of the reply to match their preferences better, thereby giving them more flexibility in customizing the AI-generated response (Fig.~\ref{fig_interface}-E, middle). 
A free-text field was also included to allow users to make other specific AI requests (Fig.~\ref{fig_interface}-E, bottom). 
After completing these steps, users can click the ``Generate Reply'' button (Fig.~\ref{fig_interface}-F).

\paragraph{\textbf{C: Generate Reply Draft}}
% ResQはユーザが"Generate Reply"ボタンを押したことを検知すると、大規模言語モデルを使用して、返信のドラフトを作成する
% ユーザの期待するような返信のドラフトが出力されるように、我々はモデルに対して、受信メールとその関連情報、AIの質問とそれに対応するユーザの回答、ユーザが返信案に期待する他の要素(トーン、スタイル、長さ、その他の要望)、ユーザの情報を提供した
When the user clicks the ``Generate Reply'' button, ResQ detects the action and uses the LLM to generate a reply draft.
% To ensure that the draft aligns with users' expectations, we first provided the LLM with the information provided when Sec.~\ref{sec:generate_questions}, the generated questions, corresponding users' answers, and users' preferences (\textit{e.g.}, tone, style, length, and any additional requests).
% Then, the LLM generates a draft of the reply.
The prompt used for this function is shown in the appendix.

\paragraph{\textbf{D: Review Reply Draft}}
Once the draft reply is generated, users can review the draft in detail (Fig.~\ref{fig_interface}-G).
Moreover, if users find that extensive revisions are needed or if they want to explore alternative phrasing, they have the option to request the AI to regenerate a new draft based on updated input or preferences.
After completing these steps, users can click the ``Reply'' button (Fig.~\ref{fig_interface}-H).
\section{Method of Study 1}
\red{To test our hypotheses and answer three research questions, we first conducted a controlled experiment, focusing on gaining a quantitative understanding.}
% original
% We conducted a controlled experiment (Study 1) and a field study (Study 2) to test our hypotheses and answer three research questions. 
% Study 1 was conducted in a controlled environment, focusing on gaining quantitative understanding. 
% To further examine the actual usage of our QA-based system, we conducted a field experiment for Study 2 to gain a qualitative understanding of how our QA-based method influenced the practice of email replies. 

\subsection{Experiment Design}
Study 1 was designed to quantitatively assess how ResQ influences the writing process (RQ1), the quality of email replies (RQ2), and the perceived relationships with others (RQ3) compared to scenarios without AI intervention and when using traditional AIMC tools. 
% 実験は日本人を対象に、日本語でやったことを明記する
\red{The experiment targeted Japanese participants and was conducted entirely in Japanese.}
% to ensure the tasks reflected natural communication in formal settings such as workplaces and schools.
This experimental context was designed as communication in formal settings, such as \red{office-related communication, research collaboration in academic institutions, and interactions with external organizations}. 
\red{It focused on time-consuming emails that were characterized as lengthy, containing multiple requests, or requiring detailed and polite responses. 
Simple and straightforward emails, such as those that can be answered with a single word or phrase (\textit{e.g.}, ``Understood''), were excluded from the scope.}

Participants were assigned the role of message recipients and required to craft replies based on the scenarios and supplementary information provided. 
\red{The messages used in the experiment were collected from 10 volunteers who provided real emails they had received in formal communication contexts.}
\red{These volunteers included office workers, graduate students, and teaching staff, all of whom were Japanese and engaged in email communication regularly (at least once per month).}
To ensure anonymity, identifying details were removed during the preparation process. 
\red{Based on the design principles of ResQ, we excluded extremely short emails and emails that could be replied to with a single word from the selection process.}
Each scenario included details about the sender, the recipient, and the context in which the message was received. 
Multiple scenarios were included in the experiment to minimize the influence of any single scenario and increase the variety. 
Additionally, supplementary information, such as the recipient’s schedule and potential questions, was provided to prevent excessive variability in the responses among participants. 

In total, we created twenty types of \red{emails}, with two assigned to the practice session and eighteen to the test session.
The length of the \red{emails} used in the test session averaged 404 \red{Japanese} characters, with the shortest being 135 characters and the longest being 925 characters.
\red{The scenarios covered a wide range of formal communication situations, including responding to a request for data submission in the workplace, answering a survey from a professor, asking questions based on guidance from a language school’s customer support team, and addressing a request for schedule adjustments as a part-time worker.}
\red{Additionally, these emails varied in structure, ranging from structured formats with bullet points to more non-structured, free-text formats.}
\red{The specific emails and examples of ResQ-generated questions and options used in the experiment can be referred to in the supplementary materials.}
% \red{The specific emails used in the experiment, including full message content, sender information, scenarios, and supplementary information, are provided as supplementary materials.}

\subsection{Experimental Conditions}
We employed a within-subject design with three conditions: QA-based, Prompt-based, and No-AI.
\red{This design was chosen to control for individual differences among participants, such as varying levels of language proficiency or familiarity with AI systems, ensuring a fair comparison across conditions.}
To illustrate each condition, consider the scenario of a participant who, as an employee of a company, is asked by their superiors to assume the role of a fixed asset committee member.

% QA-based method.
In the QA-based condition, participants created replies using the QA-based AI.
The system detected when participants navigated to the next email screen, inferred that they were initiating a reply, and then generated relevant questions.
For example, the system might ask, ``Would you be willing to take on the role of the fixed asset manager?'' ``Is there any issue with handling the annual inventory check?'' or ``Please let us know if you have any questions or concerns about the tasks.''
Participants could respond by selecting from provided options (\textit{e.g.}, ``yes,'' ``no''), adding their own options, or ignoring the questions entirely. 
After responding, they would press the ``Generate Reply'' button, which would produce an AI-generated draft in the reply box. 
Participants could then regenerate or modify the draft as needed to finalize their response.

% Prompt-based condition
In the Prompt-based condition, participants created replies using a prompt-based AI without the QA feature of the QA-based method.
Participants wrote prompts for the AI to generate a draft email response, which they then edited to create their replies.
For example, a participant might input a prompt such as, ``I want to convey my acceptance of the fixed asset committee role. ...''
Afterward, similar to the QA-based system, participants would press the ``Generate Reply'' button and, if necessary, either regenerate the draft or revise its content.

% No-AI condition
In the No-AI condition, participants created email replies manually without using AI assistance.

\subsection{Participants}
\red{
\begin{table*}[t]
% \caption{\red{Backgrounds of participants in Study 1, including age, job roles, email experience, frequency of email sending and AI tool usage, and use of AI for email purposes.}}
\caption{Backgrounds of participants in Study 1, including age, job roles, email experience, frequency of email sending and AI tool usage, and use of AI for email purposes. \blue{Some fields are marked as - due to missing responses from participants.}}
\Description{The table provides an overview of participants in Study 1, detailing their demographic information, email usage habits, and AI tool usage patterns. The participants, ranging in age from 20 to 57, include university students, office workers, a part-time worker, and individuals categorized under "other" occupations. Both male and female participants are represented, reflecting a diverse group in terms of age, occupation, and technological engagement. Participant P1 is a 34-year-old male office worker with 20 years of email experience. He typically replies to 7 emails per week but rarely uses AI tools, and he never utilizes AI for email-related tasks. P2, a 22-year-old male university student, has 5 years of email experience and actively engages with emails, replying to more than 21 per week. He uses AI tools daily, with AI assisting in 50–80\% of his email-related tasks. Similarly, P3, a 22-year-old female university student, reports no specific email experience or weekly email activity but uses AI tools frequently, relying on AI for 50–80\% of her email tasks. P4, a 21-year-old female university student with 4 years of email experience, replies to 0–2 emails weekly. She uses AI tools frequently but for less than 20\% of her email-related activities. P5, a 20-year-old male university student, has no reported email experience or weekly email activity. He rarely uses AI tools and does not use them for email purposes. P6, a 38-year-old male office worker, has 3 years of email experience and replies to 0–2 emails weekly. He rarely engages with AI tools and never applies them to email tasks. P7, a 31-year-old female categorized as "unemployed," has 12 years of email experience and replies to 3–4 emails per week. She never uses AI tools, either for general purposes or for email tasks. In contrast, P8, a 25-year-old male university student with 4 years of email experience, replies to 0–2 emails per week. He uses AI tools frequently, with AI assisting in 50–80\% of his email-related activities. P9, a 39-year-old female office worker, has 20 years of email experience and replies to over 21 emails per week. She uses AI tools frequently but never applies them to email tasks. P10, a 24-year-old male university student, has no reported email experience or weekly activity. He uses AI tools daily, although only for less than 20\% of his email-related tasks. P11, a 22-year-old female university student with 4 years of email experience, replies to 0–2 emails weekly. She rarely engages with AI tools and does not use them for email tasks. Finally, P12, a 57-year-old female office worker with 20 years of email experience, replies to 0–2 emails weekly. She uses AI tools frequently but never for email-related purposes.}
\label{tab_study1_participants}
\red{
\begin{tabular}{cccccccc}
\hline
ID   & Gender & Age & Job                  &  Email Experience&Emails/Week & AI Tool Usage& AI for Email Usage    \\ \hline
P1   & M& 34  & Office Worker        &  20 years&7           & Rarely         & Never                 \\
P2   & M& 22  & Univ. Student        &  5 years&21+         & Daily              & 50–80\%\\
P3   & F& 22  & Univ. Student        &  -&-& Frequently& 50–80\%\\
P4   & F& 21  & Univ. Student        &  4 years&0–2         & Frequently& <20\%\\
P5   & M& 20  & Univ. Student        &  -&-           & Rarely         & <20\%\\
P6   & M& 38  & Office Worker        &  3 years&0–2         & Rarely         & Never                 \\
P7   & F& 31  & \blue{Unemployed}                &  12 years&3–4         & Never         & Never                 \\
P8   & M& 25  & Univ. Student        &  4 years&0–2         & Frequently& 50–80\%\\
P9   & F& 39  & Office Worker        &  20 years&21+         & Frequently& Never                 \\
P10  & M& 24  & Univ. Student        &  -&-           & Daily              & <20\%\\
P11  & F& 22  & \blue{Univ. Student}     &  4 years&0–2         & Rarely         & Never                 \\
P12  & F& 57  & Office Worker        &  20 years&0–2         & Frequently& Never                 \\ \hline
\end{tabular}
}
\end{table*}
}

% \begin{itemize}
%     \item Job categories such as ‘Office Worker’, ‘Part-time Worker’, and ‘Other’ were chosen based on predefined options provided to participants, as shown in the survey (e.g., office workers, contract workers, freelancers, etc.).
%     \item Fields marked as ‘-’ indicate missing responses, as participants did not provide the requested information.
%     \item ‘Part-time Worker’ refers to individuals employed on a part-time basis, while ‘Other’ includes occupations such as freelance or non-traditional job roles.
% \end{itemize}
Twelve participants (six males and six females, aged 20-57) were recruited via a local Japanese participant recruiting platform (see Tab.~\ref{tab_study1_participants}).
The average age of the participants was 29.6 (SD = 11.0). 
\red{The sample size $n=12$ was determined based on an a priori power analysis (effect size $f=0.4$, significance level $p=0.05$, power = 0.8, correlation among repeated measures = 0.5) as well as the previous study~\cite{Mu2024Whispering}.}
The participants were paid approximately \$21 USD for participation, and the experiment lasted around two hours.
This study was approved by the institute's ethical review board.

\subsection{Procedure}
The participants first read the study instructions and the right to participate and then consented to participate in the experiment. 
Next, they were given an explanation of the purpose of the experiment and the use of the AI systems (Prompt-based and QA-based systems). 
% \red{To avoid the AI nocebo/placebo effect~\cite{}, the explanations were carefully designed to use neutral and standardized language, avoiding any statements that could imply one system was superior or inferior to another. 
% This ensured that participants’ perceptions of the systems were not biased before engaging in the tasks.}
% Subsequently, participants were randomly assigned to reply to six emails per condition based on the Latin square design.
\red{Participants were then randomly assigned to reply to six emails per condition using a Latin square design, which counterbalanced the order of conditions and mitigated potential order effects\blue{~\footnote{\blue{We conducted analyses to examine the potential order effect. The results of this analysis are provided in the appendix.}}}.}
In each condition, participants first engaged in a practice session where they read and replied to two emails \red{to familiarize themselves with the system.}
Then, they read and replied to six emails, which were presented in a randomly assigned order \red{to further reduce any sequence-related biases.}
After replying to six emails for each condition, participants were asked to complete a questionnaire regarding their experience with the task. 
% \blue{To ensure participants could manage their workload during the study, they were allowed to take break between conditions. }
\blue{To ensure participants could manage their workload during the study, they were allowed to take a short break after completing tasks in each condition. 
% Additionally, participants had the flexibility to take breaks at their discretion during non-task periods within each condition, such as after the practice session or before the questionnaire.
}
After completing all conditions, they were asked to fill out a comparative questionnaire evaluating the three conditions. 
\red{In addition, follow-up interviews were conducted to gather deeper insights into their experiences and preferences.}
This study was conducted remotely for all participants and lasted approximately two \red{and a half} hours in total.

\subsection{\red{Evaluation Session}}
After completing the main experiment, we conducted an additional evaluation session to assess the quality of the email responses created by participants and the impressions of participants as email senders.
This session involved a group of eighteen Japanese evaluators (ten males and eight females, aged 20-57) recruited via a local participant recruiting platform~\footnote{Participants were recruited from Lancers.jp, an online freelancing platform.}. 
The average age of the evaluators was 40.6 (SD = 8.3).
% evaluatorsは、メールを使用したコミュニケーションを最低5年以上、平均18.7年経験していた。
% また1人を除いて、evaluatorsは月に一回以上、メールを使用したコミュニケーションを行なっていた。
\blue{The evaluators had a minimum of five years and an average of 18.7 years of experience in email-based communication. 
Furthermore, with the exception of one individual, the evaluators engaged in email-based communication at least once a month.}
Each evaluator assessed email replies written by twelve different participants for a specific scenario. 
The evaluators were paid approximately \$2.5 USD for their participation, and the evaluation session lasted around fifteen minutes.

\subsection{Measurements}
% We used multiple measurements to test our hypotheses.
% 参加者が返信タスクに取り組んでいる際の行動から、Efficiency, Prompt Character Countsを算出した。
% また参加者の実験後のアンケートの回答から、Cognitive Load, Difficulty in Understanding Email Content, Satisfaction with Completing Task, Future Preference, Difficulty in Initiating the Action for Replying to Emails, Sense of Agency, Sense of Control, Psychological Distance between Participants and Their Counterpartを算出した。
% さらに、Evaluation Sessionにおけるevaluatorのアンケートの回答から、Perceived Quality of the Email, Impression of Participants as Email Sendersを算出した。
\red{We used multiple measurements to test our hypotheses. 
From participants' behavior during the email reply task, we calculated two measures: efficiency and prompt character count. 
From their post-experiment questionnaire responses, we evaluated cognitive load, difficulty in understanding email content, satisfaction with completing the task, difficulty in initiating the action for replying to emails, sense of agency, sense of control, and psychological distance between participants and their counterparts. 
Additionally, from evaluators’ questionnaire responses during the evaluation session, we assessed the perceived quality of the email and the impression of participants as email senders.}
\subsubsection{Efficiency of Replying to Emails (H1-a)}
We calculated the efficiency of replying to emails using task completion time and total character count.
The efficiency of replying to emails is defined as the amount of text contributing to the final output that can be typed per second, where a higher score indicates better task efficiency.
For task completion time, we recorded the time participants took to reply to an email, starting from when the email appeared on the screen to when the participant pressed the send button.
For total character count, we considered the text in the reply box when the participant pressed the Reply button as the final response and counted its characters.

\subsubsection{Prompt Character Counts (H1-a)}
We also calculated the average number of characters typed by participants to have the AI generate email drafts as the prompt character counts in each condition.
Under the Prompt-based condition, we measured the number of characters participants typed in the free-text field for the AI. 
Under the QA-based condition, the prompt character counts included this number plus any additional characters typed by the participants when they added their own options.

\subsubsection{Cognitive Load for Replying to Emails (H1-b)}
We used the NASA-TLX~\cite{hart1988development} questionnaire to measure cognitive load across six subscales: mental demand, physical demand, temporal demand, performance, effort, and frustration and calculated the Raw-TLX~\cite{byers1989traditional}.
Participants answered the above items using a 10-point Likert scale.
The Raw-TLX score is calculated as the simple average of six scales, where higher scores indicate a greater cognitive load.

% Additionally, to assess cognitive load specifically related to understanding received emails, we conducted a survey using a 7-point Likert scale, where 1 indicated strongly disagree, 4 indicated neutral, and 7 indicated strongly agree. 
% Participants were asked to rate the perceived load of understanding emails under all conditions.
\subsubsection{\red{Difficulty in Understanding Email Content (H1-b)}}
\red{Additionally, to assess cognitive load specifically related to understanding received emails, we used a 7-point Likert scale. 
Participants rated their agreement with the statement, ``I found it difficult to understand the sender’s intentions or requests in the email,’’ where 1 indicates strongly disagree, 4 indicates neutral, and 7 indicates strongly agree.}
% Specifically, for all conditions, we asked about the perceived load of understanding the received emails.

% Q. I was able to understand the sender's intent and request easily.

\subsubsection{Satisfaction with Completing Task (H1-c)}
We evaluated participants' satisfaction with completing their task using a 7-point Likert scale, where 1 indicates strongly disagree, 4 indicates neutral, and 7 indicates strongly agree.
Specifically, the satisfaction of completing their task was evaluated based on their satisfaction with efficiency and their satisfaction with the quality of the email they created.
We asked the following questions: 
(1) I felt that I was able to create a high-quality response. 
(2) I felt that I was able to complete the response efficiently.
We averaged the scores from two items and treated them as an index of the satisfaction with completing their task.
% The Cronbach's Alpha for the two items is $0.889$.

% \subsubsection{\red{Future Preference (H1-c)}}
% \red{We evaluated participants' preferences for future use across all conditions using a 7-point Likert scale.
% Participants rated their agreement with the statement, ``I would prefer to use this approach for replying to emails in the future,'' where 1 indicates strongly disagree, 4 indicates neutral, and 7 indicates strongly agree.
% }

\subsubsection{Difficulty in Initiating the Action for Replying to Emails (H1-d)}
We tested H1-d using a survey with a 7-point Likert scale (1 = strongly disagree, 4 = neutral, and 7 = strongly agree) to evaluate perceived barriers to task initiation. 
Specifically, we asked the following question: I felt a high barrier to initiating email response tasks.

\subsubsection{Sense of Agency and Control (H1-e)}
We evaluated participants' perceived sense of agency and control using a 7-point Likert scale, where 1 indicates strongly disagree, 4 indicates neutral, and 7 indicates strongly agree.
Specifically, drawing on previous research~\cite{Fu2023Comparing, Draxler2024The}, the sense of agency was evaluated by assessing whether participants felt they were the ones who wrote the responses, while the sense of control was evaluated by whether they felt they had control over the content of the responses. 

\subsubsection{Perceived Quality of the Email by Evaluators (H2)}
\label{sec:method2_H2}
The quality of each email reply was evaluated \red{by evaluators} using a 7-point Likert scale, where 1 indicates strongly disagree, 4 indicates neutral, and 7 indicates strongly agree.
It was evaluated on three aspects: politeness (whether it was politely written), readability (whether it had an easy-to-understand structure), and meeting demands (whether it appropriately addressed the recipient's demands). 
We averaged the scores from three items and treated it as an index of the perceived quality of the email.
% The Cronbach's Alpha for the three items is $0.846$.

\subsubsection{\blue{Perceived Impression of Participants by Evaluators (H3-a)}}
% \red{To evaluate the impression of the email senders (participants) as perceived by others, we recruited the same group of eighteen Japanese evaluators described in Sec.~\ref{sec:method2_H2}.
\red{Following a previous study~\cite{rau2009effects}, we asked the evaluators to read the email and assess their impressions of the senders (participants) based on two aspects: whether the participants were perceived as likable and whether they were perceived as kind, using a 7-point Likert scale, where 1 indicates strongly disagree, 4 indicates neutral, and 7 indicates strongly agree. 
We averaged the scores from these two items to create an index of the impression of the email sender.}

% The same eighteen evaluators were also asked to assess their impression of the email sender who were participants. 
% Following a previous study~\cite{rau2009effects}, we asked the eighteen evaluators to read the email and rate participants' impression toward the email senders (participants) in two aspects: whether they were perceived as likable or kind, using a 7-point Likert scale, where 1 indicates strongly disagree, 4 indicates neutral, and 7 indicates strongly agree.
% We averaged the scores from two items and treated it as an index of the impression of the email sender.
% Cronbach’s $\alpha$ for this index was as follows: $0.955$ for the No-AI condition, $0.950$ for the Prompt-based condition, and $0.917$ for the QA-based condition.
% The Cronbach's Alpha for the two items is $0.946$.

\subsubsection{Psychological Distance between Participants and Their Counterpart (H3-b)}
\begin{figure*}[t]
\centering
\includegraphics[width=\textwidth]{figure/study1_IOS.pdf}
\caption{Inclusion of Other in the Self (IOS). The diagram above the x-axis is an example of what participants were shown when responding to the questionnaire. The degree of overlap between the two circles represents the psychological distance between oneself and others.}
\label{fig_study1_IOS}
\Description{This figure represents the "Inclusion of Other in the Self (IOS)" scale, which is used to measure psychological closeness or relational intimacy. The diagram depicts two circles, labeled "Self" and "Other," with varying degrees of overlap. Participants were asked to choose the level of overlap that best represented their relationship with another person. On the far left (score 1), the circles are completely separate, indicating a significant psychological distance between the self and the other. In the middle (score 4), the circles partially overlap, suggesting a moderate level of psychological closeness. On the far right (score 7), the circles almost completely overlap, representing a very close and intimate relationship between the self and the other.}
\end{figure*}
We evaluated the perceived psychological distance using the Inclusion of Other in the Self (IOS) scale~\cite{aron1992inclusion}.
Participants choose a pair of circles from seven with different degrees of overlap (\red{see Fig.~\ref{fig_study1_IOS}}). 
1 = no overlap; 2 = little overlap; 3 = some overlap; 4 = equal overlap; 5 = strong overlap; 6 = very strong overlap; 7 = most overlap. 
The number chosen is the participants’ score.
The higher the score was, the closer participants felt they were with the email sender.
% todo
% Study 1において、参加者が無視した質問と答えた質問についての定性的な結果を追加
% Study 1において、non-structured/structuredなメールのそれぞれにおける質問の出力例を記載し、定性的な結果を追加
\begin{figure*}[t]
\centering
\includegraphics[width=\textwidth]{figure/study1_1.pdf}
\caption{Results of participants' efficiency and cognitive load of replying to emails. Left: Efficiency for replying to emails. Middle: Prompt character count. Right: Cognitive load for replying to emails. The significant differences between conditions were from post-hoc analysis after doing one-way repeated measure ANOVA.}
\label{fig_study1_efficiency_and_cognitiveLoad}
\Description{The figure consists of three box plots, each representing different comparisons of three experimental conditions: No-AI, Prompt-based, and QA-based. Each box plot compares a specific measure between the conditions. The p-values indicating statistical significance between different conditions are also labeled above the plots. Efficiency of Replying to Emails: Three conditions are compared: No-AI, Prompt-based, and QA-based. The No-AI group has a median of 0.65, with a first quartile at 0.51 and a third quartile at 1.05, with a minimum of 0.34 and a maximum of 1.55. The Prompt-based group has a median of 1.50, with the first quartile at 0.87 and the third quartile at 2.12, with a minimum of 0.70 and a maximum of 2.89. The QA-based group has a median of 1.89, with a first quartile at 1.35 and a third quartile at 2.12, with a minimum of 0.90 and a maximum of 3.73. P-values indicate significant differences between groups: between No-AI and Prompt-based (p = 0.013), between No-AI and QA-based (p = 0.002), and between Prompt-based and QA-based (p = 0.046). Prompt Character Count: Only two conditions are compared: Prompt-based and QA-based. The Prompt-based group has a median of 37.42 characters, with a first quartile at 28.75 and a third quartile at 53.50, with a minimum of 11.67 and a maximum of 64.50. The QA-based group has a median of 25.50 characters, with a first quartile at 19.00 and a third quartile at 31.67, with a minimum of 14.33 and a maximum of 47.83. The p-value indicates a significant difference between the two groups (p = 0.01). Raw TLX: This plot compares three conditions: No-AI, Prompt-based, and QA-based. The No-AI group has a median score of 3.70, with a first quartile of 2.93 and a third quartile at 4.35, with a minimum of 2.20 and a maximum of 4.90. The Prompt-based group shows a median of 2.40, with a first quartile at 2.03 and a third quartile at 2.70, with a minimum of 1.00 and a maximum of 4.80. The QA-based group has a median of 2.10, with a first quartile of 1.93 and a third quartile at 2.48, with a minimum of 0.70 and a maximum of 4.40. The p-values indicate significant differences between No-AI and Prompt-based (p = 0.017), between No-AI and QA-based (p = 0.008), and between Prompt-based and QA-based (p = 0.018). Each box plot represents the distribution of values for the respective metric, and the whiskers show the variability outside the upper and lower quartiles. The statistical differences (p-values) highlight where the comparisons between conditions are significant.}
\end{figure*}
\begin{figure*}[t]
\centering
\includegraphics[width=\textwidth]{figure/study1_2.pdf}
\caption{Summary of Likert scale responses. \red{Measurements H2 and H3-a were assessed by third-party evaluators rather than the participants themselves.} The significant differences between conditions were from post-hoc analysis after one-way repeated measure ANOVA or the Friedman test (* and \red{**} indicate the significance found at levels of 0.05 and 0.01, respectively).}
\label{fig_study1_questionnaire}
\Description{The figure consists of box plots comparing three experimental conditions, No-AI, Prompt-based, and QA-based, across multiple subjective measures related to email task performance. The box plots represent the distribution of responses across these conditions, with p-values indicating statistically significant differences between them. H1-b: Difficulty in Understanding Email Content: No-AI: The median is 4, with a first quartile at 3.25 and a third quartile at 6.25, with a minimum of 1 and a maximum of 7. Prompt-based: The median is 4, with a first quartile at 4 and a third quartile at 5.25, with a minimum of 1 and a maximum of 7. QA-based: The median is 6, with a first quartile at 5 and a third quartile at 7, with a minimum of 4 and a maximum of 7. Significant differences exist between Prompt-based and QA-based (p < 0.01) and between No-AI and QA-based (p < 0.01). H1-c: Satisfaction with Completing Tasks: No-AI: The median is 2.5, with a first quartile at 1.875 and a third quartile at 3.625, with a minimum of 1 and a maximum of 5. Prompt-based: The median is 5.25, with a first quartile at 4.375 and a third quartile at 6.5, with a minimum of 3.5 and a maximum of 7. QA-based: The median is 6.5, with a first quartile at 6 and a third quartile at 6.625, with a minimum of 5 and a maximum of 7. Significant differences exist between No-AI and QA-based (p < 0.01), between Prompt-based and QA-based (p < 0.01), and between Prompt-based and QA-based (p < 0.05). H1-d: Difficulty for Task Initiation: No-AI: The median is 5, with a first quartile at 5 and a third quartile at 6, with a minimum of 1 and a maximum of 7. Prompt-based: The median is 3, with a first quartile at 1.75 and a third quartile at 4, with a minimum of 1 and a maximum of 5. QA-based: The median is 2, with a first quartile at 1 and a third quartile at 2, with a minimum of 1 and a maximum of 3. Significant differences exist between No-AI and Prompt-based (p < 0.01), between Prompt-based and QA-based (p < 0.01), and between No-AI and QA-based (p < 0.01). H1-e: Sense of Agency: No-AI: The median is 7, with a first quartile at 6.75 and a third quartile at 7, with a minimum of 6 and a maximum of 7. Prompt-based: The median is 4, with a first quartile at 4 and a third quartile at 4.25, with a minimum of 3 and a maximum of 5. QA-based: The median is 2.5, with a first quartile at 1.75 and a third quartile at 3.25, with a minimum of 1 and a maximum of 5. Significant differences exist between No-AI and Prompt-based (p < 0.01), between Prompt-based and QA-based (p < 0.01), and between No-AI and QA-based (p < 0.01). H1-e: Sense of Control: No-AI: The median is 7, with a first quartile at 7 and a third quartile at 7, with a minimum of 5 and a maximum of 7. Prompt-based: The median is 5, with a first quartile at 3.5 and a third quartile at 5, with a minimum of 1 and a maximum of 6. QA-based: The median is 3.5, with a first quartile at 2.75 and a third quartile at 4, with a minimum of 1 and a maximum of 6. Significant differences exist between No-AI and Prompt-based (p < 0.01), between Prompt-based and QA-based (p < 0.01), and between No-AI and QA-based (p < 0.01). H2: Perceived Quality of the Email by Evaluators: No-AI: The median is 5.22, with a first quartile at 4.43 and a third quartile at 5.42, with a minimum of 3.78 and a maximum of 5.83. Prompt-based: The median is 5.61, with a first quartile at 5.47 and a third quartile at 6.07, with a minimum of 4.56 and a maximum of 6.56. QA-based: The median is 5.81, with a first quartile at 5.43 and a third quartile at 5.99, with a minimum of 4.78 and a maximum of 6.50. Significant differences exist between No-AI and Prompt-based (p < 0.05) and between No-AI and QA-based (p < 0.01). H3-a: Perceived Impression of Participants by Evaluators: No-AI: The median is 4.17, with a first quartile at 3.31 and a third quartile at 4.63, with a minimum of 2.42 and a maximum of 5.67. Prompt-based: The median is 5.17, with a first quartile at 4.42 and a third quartile at 5.33, with a minimum of 3.92 and a maximum of 6.00. QA-based: The median is 4.92, with a first quartile at 4.60 and a third quartile at 5.42, with a minimum of 4.00 and a maximum of 5.58. Significant differences do not exist. H3-b: Psychological Distance: No-AI: The median overlap score is 5, with a first quartile at 2.75 and a third quartile at 6.25. The minimum score is 1, and the maximum score is 7. Prompt-based: The median overlap score is 4, with a first quartile at 2.75 and a third quartile at 4. The minimum score is 1, and the maximum score is 5. QA-based: The median overlap score is 1.75, with a first quartile at 3 and a third quartile at 3.5. The minimum score is 1, and the maximum score is 7. Significant differences exist between No-AI and QA-based (p < 0.05). Each box plot represents the spread of participant responses, with the whiskers showing the variability outside the upper and lower quartiles. The statistical differences (p-values) highlight significant findings between different experimental conditions.}
\end{figure*}
% \begin{figure*}[t]
\centering
\includegraphics[width=\textwidth]{figure/study1_IOS.pdf}
\caption{Inclusion of Other in the Self (IOS). The diagram above the x-axis is an example of what participants were shown when responding to the questionnaire. The degree of overlap between the two circles represents the psychological distance between oneself and others.}
\label{fig_study1_IOS}
\Description{This figure represents the "Inclusion of Other in the Self (IOS)" scale, which is used to measure psychological closeness or relational intimacy. The diagram depicts two circles, labeled "Self" and "Other," with varying degrees of overlap. Participants were asked to choose the level of overlap that best represented their relationship with another person. On the far left (score 1), the circles are completely separate, indicating a significant psychological distance between the self and the other. In the middle (score 4), the circles partially overlap, suggesting a moderate level of psychological closeness. On the far right (score 7), the circles almost completely overlap, representing a very close and intimate relationship between the self and the other.}
\end{figure*}
\section{Results of Study 1}
\red{Here, we first present the quantitative results of Study 1 for each research question. 
Subsequently, we include comments provided by the participants.}
\subsection{Participants' Email-Replying Process (RQ1)}
\label{sec:result1_RQ1}
\subsubsection{Efficiency of Replying to Emails (H1-a)}
\label{sec:result1_efficiency}
First, we compared the efficiency of replying to emails across three conditions.
After checking the data normality assumption with the Shapiro-Wilk test, the result of one-way repeated measures ANOVA showed that there was a significant difference in participants' efficiency of replying to emails across three conditions ($F[2, 22]=14.8$, $p<0.001$\red{, $\eta_p^2=0.57$}). 
Post-hoc analysis with Holm correction revealed that participants' efficiency of replying to emails in the QA-based condition was significantly higher compared to both the No-AI $(t(11), p=0.002\red{, d=1.38})$ and the Prompt-based $(t(11), p=0.046\red{, d=0.65})$ conditions.
Thus, H1-a was supported.
The QA-based approach enhanced participants’ email replying efficiency.
% 次にPrompt Character Countsを計算した
% QA-based < Prompt-based

\subsubsection{Prompt Character Counts (H1-a)}
\label{sec:result1_prompt_character_counts}
In order to understand how participants wrote prompts differently, we calculated the prompt character counts.
After the Shapiro-Wilk test, the paired t-test revealed that participants in the QA-based condition typed significantly fewer characters in their prompts than those in the Prompt-based condition $(t(11), p=0.010\red{, d=0.90})$.

\subsubsection{Cognitive Load for Replying to Emails (H1-b)}
\label{sec:result1_cognitive_load}
% Raw-TLXの計算結果を図1に示す
The results of the Raw-TLX are shown in Fig.~\ref{fig_study1_efficiency_and_cognitiveLoad}.
According to the one-way repeated measures ANOVA with Greenhouse-Geisser correction, there was a significant difference in participants' cognitive load for replying to emails among the three conditions $(F[1.1, 12.1]=12.6, p=0.003\red{, \eta_p^2=0.53})$. 
Post-hoc analysis with Holm correction revealed that participants' cognitive load for replying to emails in the QA-based condition was significantly lower compared to both the No-AI $(t(11), p=0.008\red{, d=1.12})$ and Prompt-based $(t(11), p=0.018\red{, d=0.81})$ conditions.
% これらの結果は、H1bの妥当性について示唆している。
Therefore, H1-b was supported.
The QA-based approach reduced participants’ cognitive workload while replying to the emails.

\subsubsection{\red{Difficulty in Understanding Email Content (H1-b)}}
\label{sec:result1_difficulty_in_understanding}
% また、H1-bはアンケート調査の結果によっても裏付けられた
Additionally, H1-b was also supported by the questionnaire survey results (Fig.~\ref{fig_study1_questionnaire} H1-b).
% easily understand content of email: QA-based > Prompt-based, No-AI
The Friedman test revealed a significant difference among the three conditions in terms of understanding the sender's intent and requests $(\chi^2(2)=10.6, p=0.005\red{, W=0.44})$. 
Post-hoc analysis using the Durbin-Conover test with Holm correction showed that participants in the QA-based condition found it significantly easier to understand the sender's intent and requests compared to those in both No-AI $(p=0.003\red{, r=0.61})$ and Prompt-based $(p=0.005\red{, r=0.73})$ conditions.
% system helped decide response strategy: QA-based > Prompt-based
% Furthermore, the Wilcoxon signed-rank test indicated that participants felt significantly more supported in determining the direction of their responses in the QA-based condition than in the Prompt-based condition $(p=0.010)$.

\subsubsection{Satisfaction with Completing Task (H1-c)}
\label{sec:result1_satisfaction}
The results of the satisfaction with completing participants' tasks are shown in Fig.~\ref{fig_study1_questionnaire}, H1-c.
\red{The two items measuring satisfaction showed high internal consistency, with a Cronbach's Alpha of $0.889$.}
After checking the data normality assumption with the Shapiro-Wilk test, the result of one-way repeated measures ANOVA showed that there was a significant difference in participants' satisfaction with completing tasks across three conditions $(F[2, 22], p<0.001\red{, \eta_p^2=0.79})$. 
Post-hoc analysis with Holm correction revealed that participants' satisfaction with completing tasks in the QA-based condition was significantly higher compared to both the No-AI ($t(11)$, $p<0.001$\red{, $d=2.39$}) and the Prompt-based ($t(11)$, $p=0.029$\red{, $d=0.72$}) conditions.
% Additionally, the result of the Friedman test showed that there was a significant difference in participants' satisfaction with the quality of their email responses across three conditions $(\chi^2(2)=17.6, p<0.001)$. 
% Post-hoc analysis using the Durbin-Conover test with Holm correction showed that participants' satisfaction with the quality of their email responses in the QA-based condition was significantly higher compared to the No-AI $(p<0.001)$ condition.
Therefore, H1-c was supported.
The QA-based approach improved participants’ satisfaction with completing their tasks while replying to the emails.

% \subsubsection{\red{Future Preference (H1-c)}}
% \red{
% The questionnaire survey results about participants' future preferences are shown in Fig.~\ref{fig_study1_questionnaire}, in H1-c.
% According to the Friedman test, a significant difference in participants' future preferences was observed among the three conditions $(\chi^2(2)=8.8, p=0.012, W=0.37)$.
% Post-hoc analysis using the Durbin-Conover test with Holm correction revealed that participants would prefer responding in the QA-based condition compared to the No-AI condition $(p=0.012, r=0.67)$.
% However, no significant difference was found between the Prompt-based and QA-based conditions $(p=0.800, r=0.27)$.
% }

\subsubsection{Difficulty in Initiating the Action for Replying to Emails (H1-d)}
\label{sec:result1_initiating}
% felt high barrier: QA-based < Prompt-based, No-AI
The questionnaire survey results about participants' difficulty in initiating the action for replying to emails are shown in Fig.~\ref{fig_study1_questionnaire}, in H1-d.
According to the Friedman test, a significant difference in participants' difficulty in initiating the action for replying to emails was observed among the three conditions $(\chi^2(2)=19.8, p<0.001\red{, W=0.83})$.
Post-hoc analysis using the Durbin-Conover test with Holm correction revealed that participants in the QA-based condition perceived significantly higher barriers to initiating email response tasks than those in the No-AI $(p<0.001\red{, r=0.85})$ and Prompt-based $(p<0.001\red{, r=0.68})$ conditions.
% This result suggests the validity of H1-c, which ResQ lowers the barriers to initiating email response tasks.
Therefore, H1-d was supported.
The QA-based approach reduced participants’ difficulty in initiating the action to reply to emails.

\subsubsection{Sense of Agency and Control (H1-e)}
\label{sec:result1_agency}
% agency, control: QA-based < Prompt-based, No-AI
The questionnaire survey results about a sense of agency and control are shown in Fig.~\ref{fig_study1_questionnaire}, H1-e.
The Friedman test revealed a significant difference among the three conditions for both the sense of agency $(\chi^2(2)=22.8, p<0.001\red{, W=0.95})$ and the sense of control $(\chi^2(2)=21.3, p<0.001$, $\red{W=0.89})$. 
Post-hoc analysis using the Durbin-Conover test with Holm correction showed that participants in the QA-based condition found that it significantly reduced their sense of agency compared to both the No-AI $(p<0.001\red{, r=0.88})$ and the Prompt-based $(p<0.001\red{, r=0.77})$ conditions.
Additionally, post-hoc analysis using the Durbin-Conover test with Holm correction showed that participants in the QA-based condition experienced a significantly reduction in their sense of control compared to both the No-AI $(p<0.001\red{, r=0.88})$ and the Prompt-based $(p=0.006\red{, r=0.56})$ conditions.
Thus, H1-e was supported.
The QA-based approach reduced participants’ sense of agency and sense of control while replying to the emails.

\subsection{Quality of the Email Responses (RQ2)}
\label{sec:result1_RQ2}
\subsubsection{Perceived Quality of the Email by Evaluators (H2)}
\label{sec:result1_quality}
In Fig.~\ref{fig_study1_questionnaire}, H2 shows the results regarding the quality of the emails.
\red{The Cronbach's Alpha of three items measuring the perceived quality of the email is $0.846$.}
% \red{The three items measuring the perceived quality of the email demonstrated high internal consistency, with a Cronbach's Alpha of $0.846$.}
After checking the data normality assumption with the Shapiro-Wilk test, the result of one-way repeated measures ANOVA showed that there was a significant difference in the perceived quality of the email across three conditions $(F[2, 22]=9.1, p=0.001\red{, \eta_p^2=0.45})$. 
Post-hoc analysis with Holm correction revealed that the perceived quality of the emails participants wrote in the QA-based condition was significantly higher compared to the No-AI $(t(11), p=0.005\red{, d=1.21})$ condition.
Thus, H2 was partially supported.
The QA-based approach improved the quality of the email responses compared to the No-AI condition.

% 各評価項目の表を載せて、それを根拠にする
\begin{table*}[t]
\caption{\red{Details of perceived quality of the emails. ($Mean\pm SD$)}}
\Description{The table presents the comparative evaluation of three methods, No-AI, Prompt-based, and QA-based, in terms of three key attributes: Politeness, Readability, and Meeting Demands. The results are displayed as mean scores with standard deviations. For Politeness, the No-AI method received a mean score of 4.39 ± 1.00, indicating lower politeness levels compared to the AI-based methods. The Prompt-based approach showed a significant improvement, scoring 5.65 ± 0.56, slightly outperforming the QA-based method, which scored 5.49 ± 0.51. In terms of Readability, the No-AI method achieved a score of 5.24 ± 0.79, again falling behind the AI-based methods. The Prompt-based approach scored 5.65 ± 0.69, while the QA-based method scored the highest at 5.78 ± 0.49, reflecting the most consistently readable outputs among the three. Finally, for Meeting Demands, the No-AI method scored 5.19 ± 0.76, which is comparatively lower than the AI-enhanced methods. The Prompt-based approach performed better with a score of 5.68 ± 0.60, but the QA-based method emerged as the best performer in this category, scoring 5.88 ± 0.60.}
\label{tab_study1_quality_of_emails}
\red{
\begin{tabular}{cccc}
\hline
             & Politeness                 & Readability                & Meeting Demands               \\ \hline
No-AI        & $4.39\pm1.00$ & $5.24\pm0.79$ & $5.19\pm0.76$ \\
Prompt-based & $5.65\pm0.56$ & $5.65\pm0.69$ & $5.68\pm0.60$ \\
QA-based     & $5.49\pm0.51$ & $5.78\pm0.49$ & $5.88\pm0.60$ \\ \hline
\end{tabular}
}
\end{table*}
\red{Tab.~\ref{tab_study1_quality_of_emails} shows the detailed results regarding the perceived quality of the emails across three evaluation dimensions (politeness, readability, and meeting demands).
These results further supported the partial acceptance of H2, showing that the AI-assisted approach tended to improve the email quality.} 
% of email responses across all dimensions.}

%%%%%%%%%%%%%%%%%%%%%%%%%%%%%%%%%%%%
% According to the Friedman test, a significant difference was observed among the three conditions $(\chi^2(2)=17.6, p<0.001)$.
% As a post hoc test, the Durbin-Conover test with Holm correction was conducted.
% The results indicated that, compared to the No-AI condition, both the AI-assisted QA-based $(p=0.007)$ and Prompt-based $(p=0.012)$ conditions produced significantly higher-quality responses.
%%%%%%%%%%%%%%%%%%%%%%%%%%%%%%%%%%%%

\subsection{Relationship between Participants and Their Counterpart (RQ3)}
\label{sec:result1_RQ3}
\subsubsection{Perceived Impression of Participants by Evaluators (H3-a)}
\label{sec:result1_impression}
The results of the perceived impression of the participants rated by another group of evaluators are shown in Fig.~\ref{fig_study1_questionnaire} H3-a.
\red{The two items assessing participants' impression as email senders showed high internal consistency, with a Cronbach's Alpha of $0.946$.}
After checking the data normality assumption with the Shapiro-Wilk test, the result of one-way repeated measures ANOVA showed that there was a significant difference in impression of the participants as an email sender across three conditions $(F[2, 22]=5.9, p=0.009\red{, \eta_p^2=0.35})$. 
Post-hoc analysis with Holm correction revealed that participants' impression in the QA-based condition was not significantly higher compared to both the No-AI $(t(11), p=0.058\red{, d=0.79})$ and the Prompt-based $(t(11), p=0.939\red{, d=0.02})$ conditions.
% no-AIとQA-basedには差がありそうということを書くべきか?
Thus, H3-a was not supported.
The QA-based approach didn't improve the impression of participants as email senders.
% Although there was no statistically significant difference in participants' impressions between the QA-based and No-AI conditions $(p=0.058)$, the effect size was large $(d=0.79)$. 
% This suggests that while the difference did not reach statistical significance, there is still a meaningful difference in the impression of participants as email senders between these conditions.

\subsubsection{Psychological Distance between Participants and Their Counterpart (H3-b)}
\label{sec:result1_psychological_distance}
The IOS result is shown in \red{Fig.~\ref{fig_study1_questionnaire}}.
The Friedman test showed a significant difference in psychological distance among the three conditions $(\chi^2(2)=7.47, p=0.024\red{, W=0.31})$.
Post-hoc analysis using the Durbin-Conover test with Holm correction revealed that IOS in the No-AI condition was significantly higher than in the QA-based condition $(p=0.021\red{, r=0.51})$. 
% この結果はH3-bを部分的に支持するが、Prompt-based条件とQA-based条件の間には有意差がないことから、完全には支持されなかった
This result partially supports H3-b; however, because there was no significant difference between the Prompt-based and QA-based conditions \red{$(p=0.053, r=0.30)$}, H3-b was partially supported.


\subsection{\red{Qualitative Feedback}}
\label{sec:result1_interview}
\red{This section synthesizes qualitative feedback to provide further insights into participants' experiences across three conditions.
The interview comments were translated from Japanese into English.}
\subsubsection{\red{Participants' Email-Replying Process (RQ1)}}
\label{sec:result1_interview_RQ1}
\red{Feedback from participants confirmed that the QA-based condition functioned as expected, contributing to improvements in efficiency, a reduction in workload, and a lowering of barriers to task initiation compared to the other conditions.
\begin{enumerate}[]
    \item \textit{``In the QA-based condition, AI summarized key points through questions and highlighted relevant sections of the email body, which facilitated my understanding of the email and reduced my overall burden''} [P10].
    \item \textit{``In the QA-based condition, I could easily obtain the desired output even without the technical skills to create prompts''} [P6].
    \item \textit{``By saving the time needed to read the counterpart's text, the psychological barrier to starting the task was lowered''} [P5].
\end{enumerate}}

\red{On the other hand, we found that the QA-based condition led to a reduced sense of agency and control compared to the other conditions.
\begin{enumerate}[]
    \item \textit{``Since the AI prompted me with questions at the beginning, the mental effort required to start thinking about the task was eliminated, reducing the stress associated with initiating the work''} [P10].
    \item \textit{``By saving the time needed to read the counterpart's text, the psychological barrier to starting the task was lowered''} [P5].
\end{enumerate}}

\red{This aspect was also found to have the potential to negatively impact users' willingness to adopt the system in the future.
Participants noted that they preferred the QA-based condition \textit{``when time is limited or speed is important''} [P4] or \textit{when the email is of low importance''} [P5], but in other situations, they favored writing responses themselves.}

\subsubsection{\red{Quality of the Email Responses (RQ2)}}
\label{sec:result1_interview_RQ2}
% ほとんどの参加者は、AIを使うと、構造・丁寧さ・言葉遣いが改善され、全体的に良い文章を書けたと述べた
% また参加者は、「Prompt-based条件だと、相手の要求を見落としていたかもしれないが、QA-based条件では自信を持って返信を作成することができた」 [P2]と述べた
% さらにある参加者は、「QA-based条件では、回答してもしなくても良いこと「XXの件、承知しました、など。」にも丁寧に返答を書いてくれた」 [P9]と述べ、QA-based条件によってメールの丁寧さが向上したことを強調した
\red{All participants stated that using AI improved their writing structure, politeness, and choice of words, ultimately enabling them to produce better overall responses. 
Furthermore, participants remarked, \textit{``Under the prompt-based condition, I might have overlooked the recipient's requests, but under the QA-based condition, I was able to craft responses with confidence''} [P2]. 
Additionally, one participant emphasized that \textit{``Under the QA-based condition, the AI even provided polite responses to matters where a reply was optional, such as acknowledging something with phrases like 'I Understood regarding XX, etc.'''} [P9], highlighting how the QA-based condition scaffolded user to construct a polite email in a formal setting.}
% enhanced the politeness of email communication.}

\subsubsection{Relationship between Participants and Their Counterpart (RQ3)}
\label{sec:result1_interview_RQ3}
% 参加者は、``相手との間に知覚する心理的距離は労力に比例した''と報告し、PXXは``特にQA-based条件では選択肢を選ぶだけだった相手のことを考えることが少なかった''と報告した。
% 一方でPXXは、``自分で返信を考えるより、AIを使うと相手に良い印象を与えられるメッセージを作ることができたので、関係性を近く感じた''と報告した
\red{Participants shared differing views on how AI's involvement affected their psychological distance from their counterparts.
P2, P9, and P11 reported that the psychological distance they felt from the other person was directly related to the amount of effort they put in.
Furthermore, P6 noted that \textit{``especially under QA-based condition, I barely thought about the counterpart because I only selected options to create responses.''}
In contrast, P8 reported that \textit{``compared to composing replies myself, using AI allowed me to create messages that left a better impression on my counterpart, which made the relationship feel closer.''}
These results suggested that, on the one hand, AI's mediation can potentially increase the psychological distance between senders and receivers. 
On the other hand, it can also diminish the perceived distance from the sender due to effective impression management. Thus, we conducted a field study to further clarify the impact of AI on interpersonal relationships.}
% These results suggested that while a reduction in cognitive load may decrease participants' psychological engagement with their counterparts, the perceived improvement in communication quality could, conversely, foster a greater sense of closeness in the relationship.}

%%%%%%%%%%%%%%%%%%%%%%%%%%%%%%%%%%%%%%%%%%%%%%%%%%%%%%%%%%%%%%%%%%%%%%%%%%%%%%%%%%%%%%%%%%%%%%%%%%%%%%%%%%%%%%%%%%%%
% \subsection{Qualitative Feedback}
% % We added a new subsection, "User Comments," to present follow-up interview results, including participant feedback on useful questions (Sec.6.4). 
% \subsubsection{Participants' Email-Replying Process (RQ1)}
% \paragraph{Enhanced Efficiency, Reduced Cognitive Load, and Lowered Barriers to Initiating Email Replies (H1-a, H1-b, H1-d)}
% % 参加者は、QA-based conditionは期待通り機能し、参加者の効率向上や負担低減に貢献したことが確認できました。
% % 質問で要点をまとめてくれ、メール本文の対応箇所がハイライトされてたので、メールの理解の効率が上がり、負担が減りました [P10]
% % QA-based条件では、Prompt作成の技術がなくても、期待する出力を簡単に得ることができました。[P6]
% % Prompt-based条件では、結局自分で相手のメールを全て読み、回答すべきことを整理する必要がありました。[P4]
% % Prompt-based条件では、自分でAIに対する指示を一から考える必要があり、手動の条件と効率や負担に差を感じませんでした。一方でQA-based条件は、圧倒的に早く返信を作成することができました。[P5]
% \red{Participants' comments confirmed that the QA-based improved efficiency and reduced workload when replying to emails.
% P10 explained, \textit{``In the QA-based condition, AI summarized key points through questions and highlighted relevant sections of the email body, which facilitated my understanding of the email and reduced my overall burden''}.
% P6 shared, \textit{``In the QA-based condition, I could easily obtain the desired output even without the technical skills to create prompts''}.
% In contrast, the Prompt-based condition required extra effort. 
% P4 noted, \textit{``In the Prompt-based condition, I had to read the counterpart's email completely and decide what to respond to''}. 
% P5 elaborated, \textit{``In the Prompt-based condition, I had to think of instructions for the AI from scratch, making it feel no different from the No-AI condition in terms of efficiency and workload. On the other hand, the QA-based condition allowed me to compose responses faster''}.}

% \paragraph{Reduced Difficulty in Initiating the Action for Replying to Emails (H1-d)}
% % 参加者のコメントから、QA-based conditionでは、全体的な労力が下がるとともに、作業開始時のAIによるの質問が、タスク開始の障壁の低下に役立つことがわかりました。
% % 最初にAIが質問を投げかけてくれるので、作業開始時の思考の労力がなくなり、作業に取り掛かる際のストレスが減りました[P10]
% % 相手の文章を読む時間が省けたことで、作業開始の心理的障壁が下がりました。[P5]
% \red{Comments from participants indicated that the QA-based condition helped lower the barriers to task initiation through AI-generated questions at the start of the process. 
% P10 explained, \textit{``Since the AI prompted me with questions at the beginning, the mental effort required to start thinking about the task was eliminated, reducing the stress associated with initiating the work''}. 
% P5 noted, \textit{``By saving the time needed to read the counterpart's text, the psychological barrier to starting the task was lowered''}.}

% \paragraph{Decreased Sense of Agency and Control (H1-e)}
% % QA-based conditionでは、agencyやcontrolの感覚が他の条件に比べて低下したと回答した参加者は、次のようにコメントした。
% % 「agencyとcontrolの感覚は、プロンプトを自分で打った量に比例しました。」[P7, 8, 9, 10]
% % 「QA-based条件では、要点も絞ってくれたので、AIに任せようという思いが強くなりました。」[P4]
% % 一方で感覚が変化しなかったと回答した参加者は「AIに任せても、自分で確認と修正を行ったので、agencyやcontrolの感覚に変化はありませんでした」[P3]と説明した。
% \red{Participants reported a decrease in their sense of agency and control in the QA-based condition.
% \textit{``The sense of agency and control was proportional to the amount of text I typed myself''} [P7, P8, P9, P10]. 
% \textit{``Under the QA-based condition, since the AI helped narrow down the key points, I felt a stronger inclination to leave the task to the AI''} [P4].
% On the other hand, one participant who reported no change in their sense of agency or control explained, \textit{``Even though I relied on the AI, I reviewed and edited the output myself, so there was no change in my sense of agency or control''} [P3].}

% \paragraph{Future Preference (H1-c)}
% % 多くの参加者はメール返信の効率が上がる、負担が減る、質が高いメールを執筆できるという理由から、QA-based conditionで返信をしたいと回答しました。
% % 自分で返信を考えるより、AIを使った方が、質の高い返信を早く作ることができました。特にQA-based conditionではその効果が大きかったので、将来はQA-based conditionで返信したいと思いました。[P6]
% % 一方でsense of agencyの低下や、AIへの依存の危惧を理由に、QA-based conditionでのメール返信を忌避する参加者もいました。
% % 時間がない時や、スピードを重視したい時[P4]、重要度が低い時[P5]は、QA-based conditionで返信をしたいと思ったが、そうでない場合は自分で執筆したいと思いました。
% % 「Prompt-based条件では、QA-based条件より思考する必要が多く、それが楽しかったです」 [P7]
% % 返信作業が楽にはなりましたが、メールを細部まで読まなくなり、内容が頭に入っていない感じがしたので、将来使いたいとは思いませんでした。[P11]
% \red{Participants expressed a preference for using the QA-based condition for email responses, citing increased efficiency, reduced workload, and the ability to produce high-quality emails as the primary reasons. 
% One participant explained, \textit{``Using AI allowed me to compose high-quality responses faster than if I had written them myself. The effect was particularly significant in the QA-based condition, which is why I would prefer to use it in the future''} [P6].}
% \red{However, some participants were hesitant to adopt the QA-based condition due to concerns about a reduced sense of agency or over-reliance on AI. 
% Participants noted that they preferred the QA-based condition \textit{``when time is limited or speed is important''} [P4] or \textit{when the email is of low importance''} [P5], but in other situations, they favored writing responses themselves.
% One participant reflected, \textit{``In the prompt-based condition, I found that I needed to think more actively compared to the QA-based condition, and I enjoyed that process''} [P7].
% Another observed, \textit{``While the QA-based condition made responding easier, I felt that I was no longer fully reading and absorbing the content of emails, which made me hesitant to use it in the future''} [P11].}

% \paragraph{Quality of AI-generated Questions and Options}
% % 参加者はQA-based conditionにおいて生成された質問や選択肢について、有益であったものとそうでなかったものがあったとコメントした。
% % 参加者は有益でない質問の例として、メール送信者の意図を汲み取れていないもの [P2, P11]、自分と相手の立場を勘違いしているもの [P4, P8]を挙げた。
% % またある参加者は、「自分が言いたいことに関連する質問がない場合、質問機能自体が役に立たなかった」[P8]と回答した。
% % また参加者は、質問の数についても意見を述べた
% % 参加者は「用件ごとに質問を生成してくれたのが、メールを理解する上で役に立った」[P4, P5, P8, P9, P10, P11]と回答した一方で、「質問が多すぎると煩雑に感じることもあった。またそれに全て答えると、返信が冗長になってしまった。」[P7, P12]と回答する参加者もいた。
% % また、参加者は選択肢についても意見を述べた
% % 参加者は、「(日程調整のシチュエーションにおいて)選択肢の中に自分が選びたい日程がなかったので、自分で日程を入力する必要があった」[P2]と説明し「より多くの選択肢を生成して欲しかった」と説明した[P8]。
% % 一方である参加者は、「必要以上に多くの選択肢があったときに、煩わしさを感じた」[P4]と説明した。
% \red{Participants commented on the questions and options generated by the QA-based condition, noting that some were useful while others were not. 
% Examples of less useful questions included those that failed to accurately capture the sender's intent [P2, P11] or misinterpreted the relationship between the sender and recipient [P4, P8]. 
% Additionally, one participant remarked, \textit{``when there were no questions related to what I wanted to say, the question feature itself was not helpful''} [P8].}

% \red{Participants also shared mixed opinions on the number of questions generated. 
% Some participants noted that \textit{``generating questions for each topic helped understand the email''} [P4, P5, P8, P9, P10, P11].
% However, others felt \textit{``an excessive number of questions felt overwhelming, and responding to all of them made the reply unnecessarily lengthy''} [P7, P12].}

% \red{Feedback on the generated options was similarly divided. 
% For instance, in scheduling scenarios, one participant shared that \textit{``none of the suggested dates matched what I wanted, so I had to input the date myself''} [P2] and another participant \textit{``wished for more options or a more flexible input type''} [P8].
% In contrast, another participant stated that \textit{``having more options than necessary felt burdensome''} [P4].}

% \subsubsection{Quality of the Email Responses (RQ2)}
% % ほとんどの参加者は、AIを使うと、構造・丁寧さ・言葉遣いが改善され、全体的に良い文章を書けたと述べた
% % また参加者は、「Prompt-based条件だと、相手の要求を見落としていたかもしれないが、QA-based条件では自信を持って返信を作成することができた」 [P2, 4]と述べた
% % さらにある参加者は、「QA-based条件では、回答してもしなくても良いこと「XXの件、承知しました、など。」にも丁寧に返答を書いてくれた」 [P9]と述べ、QA-based条件によってメールの丁寧さが向上したことを強調した
% \paragraph{Scaffolding a structured response}
% \red{Most participants stated that using AI improved their writing structure, politeness, and choice of words, ultimately enabling them to produce better overall responses. 
% Furthermore, participants remarked, \textit{``Under the prompt-based condition, I might have overlooked the recipient's requests, but under the QA-based condition, I was able to craft responses with confidence''} [P2, P4]. 
% Additionally, one participant emphasized that \textit{``Under the QA-based condition, the AI even provided polite responses to matters where a reply was optional, such as acknowledging something with phrases like 'I Understood regarding XX, etc.'''} [P9], highlighting how the QA-based condition enhanced the politeness of email communication.}

% \subsubsection{Relationship between Participants and Their Counterpart (RQ3)}
% % 参加者は、``相手との間に知覚する心理的距離は労力に比例した''と報告し、PXXは``特にQA-based条件では選択肢を選ぶだけだった相手のことを考えることが少なかった''と報告した。
% % 一方でPXXは、``自分で返信を考えるより、AIを使うと相手に良い印象を与えられるメッセージを作ることができたので、関係性を近く感じた''と報告した
% \red{Participants shared differing views on how AI's involvement affected their psychological distance from their counterparts.
% Several participants reported that the psychological distance they felt from the other person was directly related to the amount of effort they put in [P2, P9, P11].
% % \textit{``the psychological distance I perceived from the counterpart was proportional to the effort exerted''} [P2, P9, P11].
% Furthermore, P6 noted that \textit{``especially under QA-based condition, I barely thought about the counterpart because I only selected options to create responses''}.
% In contrast, P8 reported that \textit{``compared to composing replies myself, using AI allowed me to create messages that left a better impression on my counterpart, which made the relationship feel closer''}.}
\section{Method of Study 2}
% この実験の目的などを書く
% Study1は、ResQのdesignは、フォーマルな場面における返信作業において、作業者の作業効率を上げ、認知負荷を下げるのに有効であることを明らかにした
Study 1 has revealed that the ResQ design effectively enhances user efficiency and reduces cognitive load during response tasks in formal settings.
\blue{To further examine how our QA-based system influenced users' actual email replying practice, we conducted a field experiment for Study 2.} 
% to gain a qualitative understanding of how our QA-based method influenced the practice of email replies.}
% Next, we set out to examine how people used the QA-based approach in their actual email replying practice.
\subsection{Field study}
% Study 2 was designed to qualitatively assess how ResQ influenced the practice of email replies.

% 本実験では、chrome拡張機能の形でのプロトタイプを開発し、システムを用いてPC版のGmailで返信作業をしてもらうように依頼した
% 拡張機能では、返信作業の開始は、参加者がGmailの返信ボックス上の"Reply with AI"ボタンを押すことによって検出した
% 参加者の返信作業の開始が確認されると、メール内容がリモートサーバに送信されると同時に、新しい返信エディタが開き、数秒間の後に質問と選択肢が生成された
% 参加者は返信文章を執筆し、reply boxの下にあるReplyボタンを押すと、Gmail上のReply boxに執筆した文章がそのまま反映されるようにした
% なお、メール内容や参加者が執筆した内容は、プライバシー保護のため、実験者からはアクセスできず、サーバにも保存されないようにした
In this five-day field experiment, we developed a prototype as a Chrome extension and asked participants to use the system to reply to emails using Gmail on a PC.
The extension detected the initiation of the reply task when participants clicked the ``Reply with AI'' button in the Gmail reply box. 
Upon confirming the task, the email content was sent to a remote server, a new reply editor opened, and a question with options appeared after a few seconds. 
Participants composed their replies, and upon pressing the Reply button, their text was directly reflected in the Gmail reply box. 
To ensure privacy, neither the email content nor the participants' responses were accessible to the experimenters or stored on the server.
\red{The specific implementation details and user interface are provided in the appendix.}

% \red{In this five-day field experiment, we developed a prototype system consisting of a Chrome extension and a backend service to enable participants to reply to emails using Gmail on a PC. 
% The Chrome extension detected the initiation of the reply task when participants clicked the ``Reply with AI'' button in the Gmail reply box (see Fig.~\ref{fig_UI}).
% Upon clicking the button, the extension extracted the email content directly from Gmail’s DOM structure using JavaScript and sent it to a backend API endpoint implemented with FastAPI~\footnote{\url{https://fastapi.tiangolo.com}}.
% The backend, hosted on an AWS EC2 instance~\footnote{\url{https://aws.amazon.com/ec2/}}, received the email content and forwarded it to the OpenAI API~\footnote{\url{https://platform.openai.com/docs/}} to generate questions or reply suggestions. 
% These outputs were then returned to the Chrome extension and displayed to the participant in a new reply editor.
% Finally, participants revise the reply suggestions and submit them back to the Gmail reply box by clicking the ``Reply'' button.
% To ensure privacy, neither the email content nor the participants' responses were accessible to the experimenters or stored on the server.}
% \begin{figure*}[ht]
\centering
\includegraphics[width=\textwidth]{figure/UI.pdf}
\caption{UI of the Gmail Reply Box with the ``Reply with AI’’ Feature, used in Study 2. Pressing the ``Reply with AI’’ button opens the window shown in Fig.~\ref{fig_interface}}
\label{fig_UI}
\Description{This figure illustrates the user interface of the Gmail reply box as enhanced by the prototype system. The Reply with AI button, shown in blue on the right-hand side of the toolbar, allows users to activate the AI-assisted reply generation feature. When the button is clicked, the system extracts the email content and opens the window shown in Fig.~\ref{fig_interface}. The standard Gmail toolbar options, such as send, formatting, and attachment icons, remain.}
\end{figure*}

\subsection{Participants}
\begin{table*}[t]
\caption{\red{Backgrounds of participants in Study 2, including age, gender, job roles, frequency of AI tool usage, and use of AI for email purposes.}}
\Description{The table outlines the demographic information and AI usage patterns of nine participants in Study 2, including their age, gender, job roles, general AI tool usage, and the extent to which they utilize AI for email-related tasks. The participants include university students and office workers, with a mix of both male and female representatives, and their AI adoption varies from frequent to rare use. Participant P1 is a 28-year-old male office worker who uses AI tools daily, with 20–50\% of his email tasks supported by AI. P2, a 24-year-old female university student, frequently uses AI tools but does not employ them for email-related purposes. Similarly, P3, a 20-year-old female university student, frequently uses AI tools, with 20–50\% of her email tasks facilitated by AI. Participant P4, a 24-year-old female university student, engages in daily AI tool usage, relying on AI for 50–80\% of her email activities. P5, a 31-year-old male office worker, also uses AI tools daily, supporting 20–50\% of his email tasks. Likewise, P6, a 39-year-old female office worker, uses AI tools daily, with AI assisting in 20–50\% of her email-related tasks. In contrast, Participant P7, a 25-year-old male university student, rarely uses AI tools and does not employ them for email-related purposes. P8, a 23-year-old female office worker, frequently uses AI tools but applies them to less than 20\% of her email tasks. Lastly, P9, a 38-year-old male office worker, rarely engages with AI tools, with AI playing a role in less than 20\% of his email-related tasks.}
\label{tab_study2_participants_background}
\red{
\begin{tabular}{cccccc}
\hline
Participants & Age & Gender & Job       & AI Tool Usage&AI for Email Usage\\ \hline
P1           & 28  & M& Office Worker& Daily&20-50\%\\
P2           & 24  & F& Univ. Student& Frequently&Never\\
P3           & 20  & F& Univ. Student& Frequently&20-50\%\\
P4           & 24  & F& Univ. Student& Daily&50-80\%\\
P5           & 31  & M& Office Worker& Daily&20-50\%\\
P6           & 39  & F& Office Worker& Daily&20-50\%\\
P7           & 25  & M& Univ. Student& Rarely&Never\\
 P8           & 23  & F& Office Worker& Frequently&<20\%\\
P9& 38& M& Office Worker& Rarely&<20\%\\ \hline
\end{tabular}
}
\end{table*}
As shown in Tab.~\ref{tab_study2_participants_background}, nine participants (four males and five females, aged 20-39) were recruited via a local Japanese participant recruiting platform.
The average age of the participants was 28.0 (SD = 6.7)\blue{, and they reported engaging in more than three email communications per day on average.}
This study was approved by the ethical review board of the authors' institute.
The participants were paid approximately \$37 USD for participation.
% 参加者は、実験やシステムの説明を30分間受けた後、5日間システムを使用し、その後1時間のインタビューを受けた
The participants received a 30-minute explanation of the experiment and system, used the system for five days, and subsequently participated in a one-hour interview.

\subsection{Procedure}
Participants first read the study instructions and their right to participate, after which they consented to participate in the experiment. 
Next, they were provided with an explanation of the study's purpose and instructions on how to use the QA-based system. 
Following this, they installed the Chrome extension we developed and confirmed its functionality according to the provided instructions.
Participants were asked to use the system for five days, during which they were free to use it to reply to emails at any time. 
After the five-day period, a one-hour semi-structured interview was conducted. 
During the interview, participants were asked a series of questions, such as: \textit{``Can you tell us your overall impression of using the system?''} \textit{``How did your email replying practice change before and after using the system?''} \textit{``What changes did you notice in the emails you composed?''} and \textit{``How did your relationship with the communication counterpart change after using this system?''}
This study was conducted remotely with all participants.

\subsection{Data Analysis}
% インタビュー記録をコード化し、分析するために、我々はインタビューデータを録音し、インタビューデータの文字起こしを行った上で、bottom-up approach rooted in grounded theoryを使用した
% 具体的には、暫定的なラベルを特定するためのopen codingと、ラベル間の関係を見出すためのaxial codingを行った
% coding結果、email-replying process, quality of the email responses, relationship between sender and recipient, future use intentinoの4つの主要なテーマを特定した
To analyze the interview data, we transcribed the interview recordings. 
We followed the thematic analysis method~\cite{braun_2006_thematicanalysis} to analyze the open-ended responses. 
One of the authors open-coded all relevant concepts that were related to our research questions, assigned labels that featured the concepts, and grouped labels into different themes. 
Next, the authors discussed the quotes and themes repeatedly.
Finally, the developed themes were compared and adjusted among all participants until they thoroughly covered the data.
As a result of the coding process, we identified four main themes: the email-replying process, the quality of the email responses, the relationship between the sender and recipient, and perceived risk.
\section{Results of Study 2}
\red{Tab.~\ref{tab_study2_participants_usage} presents the number of email replies composed using ResQ, along with the contexts in which it was used over five days.
We did not analyze usage frequency because participants reported avoiding using ResQ for emails for which they had privacy concerns.
% We did not analyze usage frequency because participants reported avoiding ResQ for certain types of emails, such as those deemed important or involving security concerns. 
Additionally, some participants refrained from using ResQ due to its availability only on PCs, as they frequently replied to emails via smartphone. 
Email frequency also varied among participants depending on their personal schedules (\textit{e.g.,} holidays).}

\red{Eight participants primarily used ResQ in formal workplace settings, while one (P9) used it only for informal exchanges. 
Because this was a field study, we could not limit participants to using ResQ only in formal contexts, though we instructed them to use it to reply to formal emails at the beginning. 
As a result, two participants (P3, P4) used ResQ to reply to both formal and informal emails, and P9 only used it for informal email exchanges. 
Hence, we excluded P9's data and only focused on analyzing the experience of P3 and P4 when they replied to formal emails using ResQ.}
% Since this study focuses on formal usage, we excluded P9's results and presented only interview data on formal scenarios.}
% 特にP1があまり使わなかった理由について言及したい

% The number of email replies composed by participants using ResQ over five days and the contexts of usage are shown in Tab.~\ref{tab_study2_participants_usage}. 
% 今回はResQの使用頻度に関する分析は行わなかった。理由は次のとおり。
% Participants reported that they did not use ResQ to compose replies for emails that required short responses, those deemed particularly important, or when they had security-related concerns.
% Additionally, some participants noted that they occasionally composed replies without using ResQ because the tool is currently only available on PCs, whereas they often replied to emails using their smartphones.
% また参加者の都合(休日など、やり取りが少ない/多い週)のため、メールの頻度はばらつきがある。
% Regarding the system's usage contexts, eight out of nine participants primarily utilized ResQ in formal workplace settings, whereas one participant (P9) used it exclusively for informal exchanges. 
% Additionally, two participants (P3, P4) noted that while most of their usage occurred in formal contexts, they occasionally employed the system for informal communication.
% Pariticipantsの使用状況はコントロールはできなかったが、今回論文ではformalにfocusしているため、P9の結果は除去した。また、formalな状況における使用に関するインタビュー結果のみを掲載した。

\red{This section explains the results of interviews conducted with participants after they used the system, with the interview comments translated from Japanese into English.}
\begin{table*}[t]
\caption{Usage of the participants in Study 2. D1 through D5 represents the number of emails replied to using the system each day, from Day 1 to Day 5.}
\Description{The table illustrates the daily system usage of participants in Study 2, detailing the number of emails replied to each day (from Day 1 to Day 5) and the primary purposes for which the system was used. The participants represent a range of tasks, including work-related scheduling, academic communication, and informal interactions. Participant P1 primarily used the system for scheduling, task confirmations, and submissions related to work. His email activity was moderate, replying to 3 emails on Day 1, 1 email on Day 2, and 1 email on Day 5, with no emails replied to on Days 3 and 4. P2 focused on task management and communication related to research and university administration. She maintained consistent usage on Days 1 and 2, replying to 6 emails each day, did not use the system on Day 3, replied to 1 email on Day 4, and increased her activity to 5 emails on Day 5. Participant P3 engaged with the system for task management related to research with professors and informal contact with friends. Her usage was highest on Days 1 and 2, replying to 6 and 7 emails respectively. She replied to 2 emails on both Days 3 and 4 and did not use the system on Day 5. P4 used the system for scheduling related to club activities, informal contact with friends, and inquiries with a museum abroad. She consistently replied to 6 emails on both Days 1 and 2, 5 emails on Day 3, 4 emails on Day 4, and 6 emails on Day 5. Participant P5’s primary usage involved meeting planning and confirmations related to work. He replied to 2 emails on Day 1, 3 emails on Day 2, 5 emails on Day 3, and 1 email on both Days 4 and 5. P6 utilized the system for scheduling and confirmations related to work, friends, and event organizers. Her activity peaked on Days 2 and 3, replying to 6 emails each day, followed by 3 emails on both Days 1 and 4 and 5 emails on Day 5. Participant P7 focused on scheduling and progress management related to research and business trips. His usage was irregular, with 4 emails replied to on Day 1, none on Day 2, 6 emails on Day 3, none on Day 4, and 2 emails on Day 5. Lastly, P8 primarily used the system for progress management and administrative confirmations related to work. Her activity started with 5 emails on Day 1, decreasing to 2 emails on Day 2, 1 email on Day 3, and 5 emails on Day 4, with no usage recorded on Day 5.}
\label{tab_study2_participants_usage}
% \resizebox{\textwidth}{!}{
\begin{tabular}{ccccccl}
\hline
\multirow{2}{*}{} & \multicolumn{5}{c}{Daily System Usage} & \multicolumn{1}{c}{\multirow{2}{*}{Main Usage}}                                                      \\ \cline{2-6}
                  & D1     & D2    & D3    & D4    & D5    & \multicolumn{1}{c}{}                                                                                 \\ \hline
P1                & 3      & 1     & 0     & 0     & 1     & Scheduling, task confirmations, and submissions related to work                                      \\
P2                & 6      & 6     & 0     & 1     & 5     & Task management and communication related to research and university administration                  \\
P3                & 6      & 7     & 2     & 2     & 0     & Task management related to research with professors, informal contact with friends                   \\
P4                & 6      & 6     & 5     & 4     & 6     & Scheduling related to club activities, informal contact with friends, inquiries with a museum abroad \\
P5                & 2      & 3     & 5     & 1     & 1     & Meeting planning and confirmations related to work                                                   \\
P6                & 3      & 6     & 6     & 3     & 5     & Scheduling and confirmations related to work, friends, and event organizers                          \\
P7                & 4      & 0     & 6     & 0     & 2     & Scheduling and progress management related to research and business trips                            \\
P8                & 5      & 2     & 1     & 5     & 0     & Progress management and administrative confirmations related to work                                 \\ \hline
\end{tabular}
% }
% \Description{The table provides background information about the participants in Study 2, including their system usage over five days (D1 to D5), which represents the number of emails they replied to using the system each day, as well as their main usage purpose. Participants range in age from 20 to 39 years, with both students and employees included, and a mix of male and female participants. Participant P1 is a 28-year-old male employee who primarily used the system for scheduling, task confirmations, and submissions related to work. His daily system usage across the five days was 3 emails on day one, 1 email on day two, and 1 email on day five, with no emails replied to on days three and four. Participant P2 is a 24-year-old female student who used the system for task management and communication related to research and university administration. Her daily system usage was consistent on the first two days, replying to 6 emails each day. She did not use the system on day three, replied to 1 email on day four, and increased her usage again with 5 emails on day five. Participant P3 is a 20-year-old female student who focused her system usage on task management related to research with professors and informal contact with friends. She replied to 6 emails on day one, 7 emails on day two, 2 emails on both day three and day four, and did not reply to any emails on day five. Participant P4 is a 24-year-old female student who used the system for scheduling related to club activities, informal contact with friends, and inquiries with a museum abroad. Her system usage was steady, replying to 6 emails on both day one and day two, 5 emails on day three, 4 emails on day four, and 6 emails on day five. Participant P5 is a 31-year-old male employee who primarily used the system for meeting planning and confirmations related to work. He replied to 2 emails on day one, 3 emails on day two, 5 emails on day three, and 1 email on both day four and day five. Participant P6 is a 39-year-old female employee who used the system for scheduling and confirmations related to work, friends, and event organizers. Her system usage was moderate, replying to 3 emails on day one, 6 emails on day two, 6 emails on day three, 3 emails on day four, and 5 emails on day five. Participant P7 is a 25-year-old male student who used the system for scheduling and progress management related to research and business trips. His system usage was irregular, replying to 4 emails on day one, none on day two, 6 emails on day three, none on day four, and 2 emails on day five. Participant P8 is a 23-year-old female employee who used the system for progress management and administrative confirmations related to work. Her daily system usage was 5 emails on day one, 2 emails on day two, 1 email on day three, 5 emails on day four, and none on day five. Participant P9 is a 38-year-old male employee who used the system for daily informal contact with friends. He had the highest and most consistent email activity, replying to 13 emails on day one, 11 emails on day two, 12 emails on both day three and day four, and 10 emails on day five.}
\end{table*}
% \begin{table*}[t]
% \caption{Usage of the participants in Study 2. D1 through D5 represents the number of emails replied to using the system each day, from Day 1 to Day 5.}
% \centering
% \label{tab_study2_participants_usage}
% \resizebox{\textwidth}{!}{
% % \setlength{\tabcolsep}{6pt} % default 6pt
% {\tabcolsep=2pt
% \begin{tabular}{ccccccl}
% \hline
% \multirow{2}{*}{} & \multicolumn{5}{c}{\centering Daily System Usage} & \multirow{2}{*}{\centering Main Usage} \\ \cline{5-9} 
%                   & D1 & D2 & D3 & D4 & D5 & \multicolumn{1}{l}{}                                \\ \hline
% P1                & 3     & 1     & 0     & 0     & 1     & Scheduling, task confirmations, and submissions related to work         \\
% P2                & 6     & 6     & 0     & 1     & 5     & Task management and communication related to research and university administration \\
% P3                & 6     & 7     & 2     & 2     & 0     & Task management related to research with professors, informal contact with friends        \\
% P4                & 6     & 6     & 5     & 4     & 6     & Scheduling related to club activities, informal contact with friends, inquiries with a museum abroad   \\
% P5                & 2     & 3     & 5     & 1     & 1     & Meeting planning and confirmations related to work                  \\
% P6                & 3     & 6     & 6     & 3     & 5     & Scheduling and confirmations related to work, friends, and event organizers \\
% P7                & 4     & 0     & 6     & 0     & 2     & Scheduling and progress management related to research and business trips \\
% P8                & 5     & 2     & 1     & 5     & 0     & Progress management and administrative confirmations related to work                \\
% P9                & 13    & 11    & 12    & 12    & 10    & Daily informal contact with friends         \\ \hline
% \end{tabular}
% }
% \Description{The table provides background information about the participants in Study 2, including their system usage over five days (D1 to D5), which represents the number of emails they replied to using the system each day, as well as their main usage purpose. Participants range in age from 20 to 39 years, with both students and employees included, and a mix of male and female participants. Participant P1 is a 28-year-old male employee who primarily used the system for scheduling, task confirmations, and submissions related to work. His daily system usage across the five days was 3 emails on day one, 1 email on day two, and 1 email on day five, with no emails replied to on days three and four. Participant P2 is a 24-year-old female student who used the system for task management and communication related to research and university administration. Her daily system usage was consistent on the first two days, replying to 6 emails each day. She did not use the system on day three, replied to 1 email on day four, and increased her usage again with 5 emails on day five. Participant P3 is a 20-year-old female student who focused her system usage on task management related to research with professors and informal contact with friends. She replied to 6 emails on day one, 7 emails on day two, 2 emails on both day three and day four, and did not reply to any emails on day five. Participant P4 is a 24-year-old female student who used the system for scheduling related to club activities, informal contact with friends, and inquiries with a museum abroad. Her system usage was steady, replying to 6 emails on both day one and day two, 5 emails on day three, 4 emails on day four, and 6 emails on day five. Participant P5 is a 31-year-old male employee who primarily used the system for meeting planning and confirmations related to work. He replied to 2 emails on day one, 3 emails on day two, 5 emails on day three, and 1 email on both day four and day five. Participant P6 is a 39-year-old female employee who used the system for scheduling and confirmations related to work, friends, and event organizers. Her system usage was moderate, replying to 3 emails on day one, 6 emails on day two, 6 emails on day three, 3 emails on day four, and 5 emails on day five. Participant P7 is a 25-year-old male student who used the system for scheduling and progress management related to research and business trips. His system usage was irregular, replying to 4 emails on day one, none on day two, 6 emails on day three, none on day four, and 2 emails on day five. Participant P8 is a 23-year-old female employee who used the system for progress management and administrative confirmations related to work. Her daily system usage was 5 emails on day one, 2 emails on day two, 1 email on day three, 5 emails on day four, and none on day five. Participant P9 is a 38-year-old male employee who used the system for daily informal contact with friends. He had the highest and most consistent email activity, replying to 13 emails on day one, 11 emails on day two, 12 emails on both day three and day four, and 10 emails on day five.}
% \end{table*}
\subsection{Participants' Email-Replying Process (RQ1)}
\subsubsection{Improved Perception of Efficiency and Workload}
\label{sec:result2_efficiency}
Participants reported that their perception of workload and work efficiency improved due to the support from ResQ.
Specifically, participants noted that ResQ's support helped clarify the topics they needed to address in the email. 
Participants explained that \textit{``Normally, when writing, I need to process multiple tasks simultaneously to ensure my intentions are appropriately expressed. However, [With ResQ,] replying to emails was divided into two different sub-tasks, answering questions and polishing emails with diverse expressions. As a result, I felt that the cognitive load was reduced.''} [P7], and \textit{``it felt like creating an email was as simple as answering a survey''} [P6].
Additionally, particularly when the counterparts' message was long, participants reported that the listing of requests as questions allowed them to \textit{``easily understand the content of the email''} [P3], with another participant noting that \textit{``I can quickly make decisions on what to reply [with ResQ]''} [P6]. 
Furthermore, compared to other AI tools like ChatGPT, the QA-based approach enabled participants to communicate their intentions more efficiently without extensive typing.
As one participant described,\textit{``[Writing with ResQ] made it easier to reflect my intentions while replying to the email''} [P4], while another participant added that \textit{``I could create the expected reply without even having to type on the keyboard''} [P6].
% While ResQ was helpful in formal situations, some participants found it less suitable for informal communications, such as with friends.
% One participant remarked that \textit{``the expressions were too polite, and I didn't like it''} [P3], and another mentioned that they \textit{``thoroughly changed the phrasing, like replacing 'Looking forward to seeing you again!' with just 'See you'''} [P4].
% Additionally, eight participants (with the exception of P9, who only used the system in informal settings) expressed increased satisfaction with the quality of the responses they wrote (for more details, see Sec.~\ref{sec:result2_quality}) and responded positively to the question, \textit{``What is your overall impression of using ResQ?''} and expressed a desire to continue using the system in the future. 
\red{Additionally, all participants expressed increased satisfaction with the quality of the responses they wrote (for more details, see Sec.~\ref{sec:result2_quality}) and responded positively to the question, \textit{``What is your overall impression of using ResQ?''} and expressed a desire to continue using the system in the future.}
Participants also mentioned that being able to craft clearer messages more quickly than before resulted in \textit{``greater confidence in the reply process and a more positive perception of the task''} [P8]. 
Additionally, a different participant expressed, \textit{``I felt joy in meeting societal expectations competently''} [P2].
These increases in achievement and confidence led participants to report that their \textit{``perception of the reply task became more positive''} [P8], and they felt \textit{``more motivated to engage actively in email responses''} [P3]. 

\subsubsection{Reduced Difficulty in Initiating the Action for Replying to Emails}
\label{sec:result2_initiating_the_action}
Participants reported that ResQ's support lowered the barrier to starting tasks, reducing procrastination in replying to emails.
One participant shared that they previously \textit{``felt reluctant to engage in replying due to the burden of the task''}, but with ResQ, \textit{``I felt motivated because I can complete the task quickly''} [P3]. 
Another participant noted that \textit{``I became able to craft replies to any email easily, so I could respond even on days when I was tired or when I would typically postpone replying to long emails''} [P6].
Participants also reported that using AI to initiate the task motivated them to start replying to emails without procrastinating. 
One participant explained that \textit{``just pressing a button prompts the AI to ask questions''}[P4], which led them to \textit{``delegate the initial steps entirely to the system''} [P3]. 
This reduction in the burden of the initial stage was cited as a key factor in lowering the barrier to starting to reply to emails. 

\subsubsection{Reduced Sense of Agency and Control}
\label{sec:result2_agency_control}
% Four out of nine participants (P3, P5, P6, and P9) reported a decreased sense of agency and control while replying to emails with ResQ. 
\red{Three out of eight participants (P3, P5, and P6) reported a decreased sense of agency and control while replying to emails with ResQ.}
They attributed this to several factors: one participant mentioned that their perception shifted \textit{``from that of an author to that of an editor''} [P5], which reduced the workload of replying but made the process feel \textit{``like an assembly line''} [P3], while another expressed, \textit{``I ended up using words or expressions I normally wouldn't [use in the email]''} [P6].
In contrast, for those who reported no change in their sense of agency or control (five participants), they explained that this was because  \textit{``the email content was strongly related to me''} [P7], and they \textit{``checked the content carefully''} [P7] or \textit{``modified words that I wouldn't normally use to the ones I would use''} [P2], leading them to feel that their \textit{``active involvement [to reply to the email] was indispensable''} [P8]

\subsection{Increased Perceived Quality of the Email (RQ2)}
\label{sec:result2_quality}
Participants reported that they felt the quality of their emails had improved. 
Participants explained that, in the process of creating responses, they were most concerned with \textit{``politeness in language, such as expressions and greetings''} [P6], and mentioned that ResQ provides support in these areas.
Participants reported that \textit{``it was helpful to have phrases that would have taken time to come up with on their own, expressions of apology and gratitude, and additional words of consideration for the other person''} [P2], \textit{``there was no need to think about the opening and closing greetings''} [P8], and \textit{``there were no typos or omissions at all''} [P5].
Additionally, participants mentioned that ResQ helped reduce the likelihood of overlooking requests in the emails they received. 
One participant shared, \textit{``Previously, when a single email contained multiple requests, I sometimes missed responding to all of them, but the questions provided by ResQ helped improve this''} [P6].
Participants attributed this improvement to the fact that ResQ \textit{``secured time to focus on understanding the recipient's requests and responding to them''} [P1], and the questions generated by ResQ \textit{``helped me ensure that nothing was overlooked in the content''} [P7]. 
Furthermore, participants reported that responding to AI-generated questions encouraged them to include details they would normally omit, resulting in more polite and comprehensive responses. 
One participant described an email regarding event attendance and multiple confirmations, explaining that while they would usually reply with something like \textit{``I will attend, thank you''}, answering the AI's questions led to a response where \textit{``each of the recipient's requirements was addressed more carefully''} [P2], ultimately leading to a more courteous email.

\subsection{Relationship between Participants and Their Counterpart (RQ3)}
\subsubsection{Enabling a Positive Self-Presentation as an Email Sender}
\label{sec:result2_self-presentation}
The participants reported feeling they could make a good impression on others using ResQ. 
They attributed this to improvements in the quality of their writing, shorter response times, and increased frequency of replies. 
One participant mentioned, \textit{``I could answer the other person's questions clearly, and the writing became more polished, making it easier for them to read''} [P5]. 
The participant also mentioned that \textit{``I felt the individuality of the email reply had faded''} but added that \textit{``I never intended to express individuality in my emails to begin with, so even if it was lost, it wasn't an issue as long as it felt natural to the recipient''} [P5].
Another participant shared that when they met a professor with whom they had communicated via ResQ, the person remarked, \textit{``Your emails have become more polished.''} 
They further elaborated, \textit{``I was particularly complimented on how much more understandable the structure of my emails has become''} [P3].
Additionally, this participant noted, \textit{``Previously, I would often respond to long emails with just, 'I'll get back to you later,' because reading through and thinking about a proper reply was tedious. However, [with ResQ's support,] I've started responding immediately instead of postponing. As a result, I've been assigned more tasks than before.''}

\subsubsection{Psychological Distance between Participants and Their Counterpart}
\label{sec:result2_psychological_distance}
Participants had mixed opinions regarding the psychological distance they perceived from their counterparts.
Those who felt the decreased psychological distance between themselves and their counterparts attributed this to the positive impression they believed they made on their counterparts. 
One participant reported that sending well-crafted emails quickly led to \textit{``a stronger sense of reassurance in [formal] communication''} [P5], while another participant noted that \textit{``[When I asked the museum staff a question,] I noticed that when I replied immediately after receiving a message from the other person, they responded quickly in return. When we communicated with such a good rhythm, I felt a strong sense of closeness towards the counterpart''} [P4].
In contrast, participants who felt the increased psychological distance mentioned a strong awareness that their replies were mediated by a system and the use of words they would not usually choose. 
One participant gave an example of communication with their university professor, stating, \textit{``While I know the counterpart typed their emails manually, I felt that using AI made the conversation more superficial, which weakened our relationship''} [P3]. 
% Another participant mentioned that the awareness when communicating with friends that \textit{``knowing that the email wasn’t something I had crafted from scratch by myself made me feel more distant from the counterpart''} [P9].
Participants also shared that they tended to forget about the email exchange with their counterparts due to the increased psychological distance.
% their sense of psychological distance from the sender as being caused by their forgetting interactions with them. 
One participant mentioned, \textit{``I found the email content easy to understand while working on it [with ResQ], but I felt it was difficult to retain our email exchange in long-term memory. When that counterpart [who is my professor] asked me, 'What happened with that issue? [that had been mentioned in our email]' there were times I couldn’t remember, which made me feel anxious''} [P3].

\subsection{Perceived Risks}
\label{sec:result2_risks}
Participants expressed concerns about the potential risks that ResQ might pose in the future. 
They expressed concerns about potential declines in their abilities and the risk of becoming overly dependent on AI, which could lead to carelessness in responding to work-related emails.
One participant explained, \textit{``I worry that the skills I've developed from composing emails myself might deteriorate''} [P8]. 
Another participant voiced concerns that \textit{``the advancement and usage of AI [in this context] might erode our ability to overcome psychological barriers''} [P2], fearing a decline in their interpersonal communication skills.
Additionally, participants raised the issue of over-reliance on AI, with one participant noting, \textit{``Given my trust in AI, I might eventually stop reviewing the content of the emails I send or the emails I receive''} [P8]. 
This reflects their concern about the potential for becoming overly dependent on AI-generated text in the future.

% \subsection{\red{Valuable AI-generated Questions and Options}}
% % ResQが生成した質問と選択肢は、十分に実用的であることが実験結果から示唆されたが、その質にはさらなる改善の余地がある
% % まず参加者は、実験1のインタビュー結果と同様、\textit{``自分と相手の関係性をAIが勘違いして質問を作成していることがあった''} [P2]と、質問生成の精度の低さを指摘した。
% % さらにある参加者は、\textit{``質問の生成に時間がかかる時があり、その時間にできることがないので少しフラストレーションを感じた''} [P1]と回答し、質問生成の生成時間の長さを指摘した。
% % またある参加者は、\textit{``システムが過剰に質問をしたことでフラストレーションを感じたことがあった''} [P3]と質問の多さを指摘し、別の参加者は、\textit{``様々な状況を想定してより多くの質問をして欲しかった''} [P8]と質問の少なさを指摘した。
% \red{
% The results suggest that the questions and options generated by ResQ were sufficiently practical; however, there is still room for improvement in their quality. 
% First, as in the interview results from Study 1, participants pointed out inaccuracies in question generation. 
% One participant noted, \textit{``there were instances where the AI misunderstood the relationship between myself and the other person when generating questions''} [P2], highlighting the low accuracy of the generated questions.
% Additionally, one participant pointed out the long generation time, explaining, \textit{``there were times when generating questions took too long, and I felt frustrated because there was nothing I could do during that time''} [P1]. 
% Another participant mentioned feeling frustrated by the excessive number of questions, stating, \textit{``there were times when the system asked too many questions, which I found frustrating''} [P3]. 
% On the other hand, another participant expressed dissatisfaction with the limited number of questions, saying, \textit{``I wanted the system to generate more questions that accounted for a wider range of scenarios''} [P8].
% }

\section{Discussion}
\red{Through a controlled experiment (Study 1) and a field study (Study 2), we investigated the impact of the LLM-powered QA-based approach on both senders and receivers.
In this section, we discuss the findings (Fig.~\ref{tab_summary}) of the research and the key considerations for designing QA-based systems.}
\begin{table*}[t]
\caption{Research Questions and Key Findings}
\label{tab_summary}
\centering
% \resizebox{\textwidth}{!}{
\begin{tabular}{>{\raggedright\arraybackslash}p{0.08\linewidth}>{\raggedright\arraybackslash}p{0.28\linewidth}>{\raggedright\arraybackslash}p{0.28\linewidth}>{\raggedright\arraybackslash}p{0.28\linewidth}}
\hline
 & \textbf{RQ1: How does a QA-based response-writing support approach affect workers’ email-replying process?} & \textbf{RQ2: How does a QA-based response-writing support approach affect the quality of the email response?} & \textbf{RQ3: How does a QA-based response-writing support approach affect the perceived relationship between email sender and recipient?} \\ \hline
\textbf{Key Findings} & 1. QA-based approach \textbf{reduced workload} for email comprehension and prompt creation and \textbf{improved work efficiency}. (H1-a, supported; H1-b, supported, Sec.~\ref{sec:result1_efficiency},~\ref{sec:result1_prompt_character_counts},~\ref{sec:result1_cognitive_load},~\ref{sec:result1_difficulty_in_understanding},~\ref{sec:result1_interview_RQ1},~\ref{sec:result2_efficiency}) 

2. QA-based approach \textbf{reduced the difficulty} of initiating the email replying task. (H1-d, supported, Sec.~\ref{sec:result1_initiating},~\ref{sec:result1_interview_RQ1},~\ref{sec:result2_initiating_the_action})

3. QA-based approach \textbf{decreased the sense of agency and control}. (H1-e, supported, Sec.~\ref{sec:result1_agency},~\ref{sec:result1_interview_RQ1},~\ref{sec:result2_agency_control})

4. QA-based approach \textbf{improved satisfaction} with the emails they wrote and willingness to use ResQ in the future. (H1-c, supported, Sec.~\ref{sec:result1_satisfaction},~\ref{sec:result1_interview_RQ1},~\ref{sec:result2_efficiency})& Writing emails with QA-based approach and Prompt-based approach led to \textbf{increased email quality} than No-AI condition. (H2, partially supported, Sec.~\ref{sec:result1_quality},~\ref{sec:result1_interview_RQ2}~\ref{sec:result2_quality})& 1. Writing emails with QA-based approach \textbf{\blue{did not lead to improved perceived impression of users by their counterparts}}. (H3-a, not supported, Sec.~\ref{sec:result1_impression},~\ref{sec:result2_self-presentation})

2. Writing emails with QA-based approach led to \textbf{increased psychological distance} between users and their counterparts than No-AI condition. (H3-b, partially supported, Sec.~\ref{sec:result1_psychological_distance},~\ref{sec:result1_interview_RQ3},~\ref{sec:result2_psychological_distance})\\ \hline
\end{tabular}
% }
\Description{This table summarizes the three research questions (RQs) investigated in the study and highlights the key findings associated with each. RQ1: How does a QA-based response-writing support approach affect workers’ email-replying process? Key Findings: 1. The QA-based approach reduced workload for email comprehension and prompt creation, leading to improved work efficiency. This supports hypotheses H1-a and H1-b. 2. It reduced the difficulty of initiating the email replying task, supporting hypothesis H1-d. 3. The approach decreased the sense of agency and control among users, supporting hypothesis H1-e. 4. Users experienced improved satisfaction with the emails they wrote and showed a greater willingness to use ResQ in the future, supporting hypothesis H1-c. RQ2: How does a QA-based response-writing support approach affect the quality of the email response? Key Findings: Writing emails using both the QA-based and prompt-based approaches led to an increase in email quality compared to the No-AI condition. This partially supports hypothesis H2. RQ3: How does a QA-based response-writing support approach affect the perceived relationship between email sender and recipient? Key Findings: 1. Writing emails with QA-based approach did not lead to improved perceived impression of users by their counterparts, meaning hypothesis H3-a was not supported. 2. Writing emails with the QA-based approach led to an increase in psychological distance between users and their counterparts compared to the No-AI condition. This partially supports hypothesis H3-b.}
\end{table*}


% \textbf{Summary} & Workers evaluated the system’s benefits (improvements in efficiency, cognitive load, and satisfaction), accepted a certain reduction in the agency, and showed a willingness to use it in the future. & It became possible to create responses that appropriately addressed email requests while maintaining politeness. (H2, supported) & ResQ improved workers' impression. Perceived psychological distance from others depends on a balance between a sense of agency and communication satisfaction. \\ \hline
\subsection{Impact of the QA-based Approach}
\subsubsection{Enhancing Efficiency and Reducing Cognitive Load}
% Our studies indicate that the QA-based approach improves efficiency and reduces cognitive load when composing email replies (Sec.~\ref{sec:result1_efficiency},~\ref{sec:result1_prompt_character_counts}, \ref{sec:result1_cognitive_load},~\ref{sec:result1_difficulty_in_understanding},~\ref{sec:result1_interview_RQ1},~\ref{sec:result2_efficiency}). 
Our studies indicate that the QA-based approach improves efficiency and \blue{suggests a reduction in} cognitive load when composing email replies (Sec.~\ref{sec:result1_efficiency},~\ref{sec:result1_prompt_character_counts}, \ref{sec:result1_cognitive_load},~\ref{sec:result1_difficulty_in_understanding},~\ref{sec:result1_interview_RQ1},~\ref{sec:result2_efficiency}). 
One possible explanation is that the QA-based approach helps users focus on the most relevant details, simplifying email comprehension compared to prompt-based methods.
% understand key information and organize their responses effectively.
% By emphasizing key information through AI-generated questions, this approach allows users to focus on the most relevant details, simplifying email comprehension compared to prompt-based methods.
% This finding aligns with cognitive load theory~\cite{sweller2011cognitive}, suggesting that reducing extraneous cognitive load enables users to perform tasks more efficiently.
Additionally, the QA-based approach reduces the burden of prompt creation by partially replacing the task of crafting prompts with the simpler task of answering questions.
\blue{Our finding suggests that future email systems could use this QA-based approach to mediate the email exchange process.}
% 更なる効率改善、負荷低減のためには、質問の量や質、提示する順序の最適化が効果的な可能性がある
% また選択肢も調整することができる
% 例えばシンプルなYes/Noの選択肢は有用でしたが、スケジュールツールや自由入力フィールドのような柔軟な入力も求められたので、カスタマイズ可能な入力形式(例: スケジューリング用のカレンダーセレクター)を導入することで、さらに使いやすさを向上させることができる可能性がある
% To further enhance efficiency and reduce cognitive load, optimizing the quantity, quality, and sequence of questions may be beneficial. 
% Additionally, refining response options could improve usability. 
% For instance, while simple Yes/No choices were effective, users also expressed a need for more flexible input options, such as scheduling tools or free-text fields. 
% Thus, introducing customizable input formats, such as a calendar selector for scheduling, could further enhance usability and streamline the email composition process.

\subsubsection{Potential Reduction in Sense of Agency and Control}
\red{While the QA-based approach enhanced users' efficiency, our studies also revealed a potential trade-off in users' sense of agency and control (Sec.~\ref{sec:result1_agency},~\ref{sec:result1_interview_RQ1},~\ref{sec:result2_agency_control}).
Some participants reported a decreased sense of authorship, feeling more like editors than creators of their emails. 
This reduction in agency may be due to the diminished amount of text input required from the user, as the AI takes a more active role in content generation.
% またこれらの感覚は、ユーザの好みに強く影響を与えることも明らかとなり、agencyの維持を望むような重要な場面などではこのアプローチは使用したくないと報告する参加者もいました。
Moreover, we found that the sense of agency influenced users' preferences for future usage.
% For example, some participants expressed a reluctance to use this approach in critical contexts where preserving their sense of agency was particularly important.
% However, this effect was not uniform across all users. 
Among those participants who still maintained their sense of agency, we found that they tended to actively review and modify the AI-generated content to reflect their personal style and intentions.
% これはQA-based approachのようにAIの介入が大きくとも、ユーザが積極的に内容を確認・編集することで、ユーザの主体性を維持できる可能性があるということを示唆している
\blue{This suggests that even when AI intervention is substantial, users can maintain a sense of authorship by actively engaging with and refining the AI's suggestions.}
% These insights emphasize that the importance of the agency may depend on individual user preferences, the context of the communication, and the extent to which users personalize the AI-generated output.
}
% ユーザにとって望ましいagencyのレベルに最適化するために、シチュエーションやユーザの好みに応じて質問や選択肢の数、提案のレベル(文レベル、メッセージレベルなど)を変えるなど、AIの介入方法や度合いを変化させるアプローチが効果的な可能性がある
\blue{To optimize users' level of agency, adapting the degree of AI intervention in the email construction process can be helpful.
% based on the situation and user preferences may be effective. 
% This could involve 
For instance, by adjusting the number and type of AI-generated questions or varying the levels of AI-generated suggestions~\cite{Fu2023Comparing}, ranging from word-level to message-level.} 
% (\textit{e.g.}, sentence-level vs. message-level recommendations).}

% \subsubsection{\red{Improvement in Email Quality and Users' Impressions of Communication Partners}}
% \red{The QA-based approach was found to enhance the perceived quality of email responses (Sec.~\ref{sec:result1_quality},~\ref{sec:result1_interview_RQ2}~\ref{sec:result2_quality}). 
% Our studies revealed that emails composed with AI assistance were more polite, well-structured, and addressed multiple requests without omissions. 
% Particularly, the QA-based approach can support users in creating emails that meet the sender's demands by organizing them through AI-generated questions. 
% Furthermore, by enhancing efficiency and lowering the barriers to initiating tasks, this approach facilitates quicker response times, which, in turn, positively influence users' impressions of their correspondents~\cite{yoram2011online, Resendes2012Send, vignovic2010computer}. 
% In summary, these findings indicate that the QA-based approach not only supports users in creating higher-quality email responses but also fosters more positive impressions.}

\subsubsection{Possibility of Improving Relationship between Email Sender and Recipient}
\red{Our studies yielded mixed results regarding the impact of the QA-based approach on the psychological distance between users and their counterparts (Sec.~\ref{sec:result1_psychological_distance},~\ref{sec:result1_interview_RQ3},~\ref{sec:result2_psychological_distance}). 
% Some参加者は素早く、高い質のメールを送信できたことや、またそれによって相手からの返信が早くなったことで、相手との間に感じる距離感を近く感じた。
% 一方で他の参加者は、作業の労力が減ったことや、自分が普段使わない言葉を使っていることに気がついたことで、felt a sense of increased distance.
Some participants reported that they were able to send emails more quickly and with high quality, which in turn led to faster responses from others and a reduced sense of distance in their interactions.
In contrast, other participants experienced an increased sense of distance, which has also been reported in the previous studies~\cite{Fu2023Comparing,arnold2020predictive}. 
They noted that the reduced communication effort and the use of unfamiliar language made interactions feel less personal or authentic.
% Some participants felt closer to their counterparts due to quicker response times and higher-quality emails.
% Others, however, felt a sense of increased distance, partly because the AI-mediated communication felt less personal or authentic.
% These findings align with previous studies~\cite{Fu2023Comparing,arnold2020predictive} indicating that while AI can contribute to maintaining a professional tone, it may also result in less authenticity when AI-generated language deviates from a user's usual style, thereby increasing psychological distance.
% These divergent users' feedback suggests that while the QA-based approach can enhance certain aspects of communication, it may also inadvertently introduce a sense of impersonality. 
The degree of the perceived distance may depend on factors such as the nature of the relationship (\textit{e.g.,} colleagues vs. friends), the user's reliance on AI-generated language, and individual preferences regarding AI-mediation in communication.}

\subsection{Opportunities and Challenges of Introducing QA-Based Approach}
% \subsubsection{Situations where QA-based is Useful in Email Communication}
Our results indicate that the QA-based approach 
% effectively streamlines the process of composing responses and enhances the quality of replies.
% Therefore, this approach 
is particularly useful in situations where speed and high-quality responses are prioritized over email personality or a strong sense of personal agency. 
Contexts such as business, customer service, and technical support can greatly benefit from the QA-based approach, as they often require efficient and structured communication.
% Additionally, the QA-based AI-assisted email replying mechanism proves highly effective in scenarios where maintaining a neutral tone and low emotional engagement is appropriate, such as initial contacts or interactions among weak ties. 
% In these cases, the system’s ability to generate standardized, professional language supports users in composing suitable replies promptly.
% In situations where maintaining a neutral affect or low emotional engagement is encouraged (\textit{e.g.}, in initial contacts and weak ties), a QA-based AI-assisted email replying mechanism can be effective.

However, for more delicate or personal email exchanges, users may prefer more tailored interventions. 
In such situations, users can adjust the level of involvement of AI intervention.
% or automatically tuning the intervention based on the email content or the user’s past behavior could better meet their needs.
% This customization can help maintain the authenticity and personal touch necessary for meaningful communication.
% Future research should explore how different levels of AI-mediated intervention affect users’ sense of agency and email construction behavior across various communication contexts.
Furthermore, there is a risk that users could become overly reliant on technology to mediate their interpersonal communication. 
Our interviews revealed that users might become accustomed to trusting AI-generated questions and drafts due to the efficient outcomes. 
Consequently, they may become less diligent in reading the emails they receive or in reviewing the responses they send carefully.
This over-reliance could lead to miscommunication or the omission of important details, thus undermining the primary goal of using AI to improve communication efficiency.
Future research should explore how different levels of AI-mediated intervention can be designed to influence users’ sense of agency and email construction behavior for various communication purposes.


% Beyond email communication, a QA-based approach utilizing LLMs can alleviate users' workload while preserving a degree of agency in areas that require structured information gathering and intent-driven interactions.
% % Beyond email communication, QA-based systems powered by LLMs have the potential to reduce users' workload while maintaining a certain level of agency in domains requiring structured information elicitation and intent-driven interactions. 
% A QA-based approach can streamline tasks such as drafting structured documents (\textit{e.g.}, academic rebuttals), clarifying user needs (\textit{e.g.}, in customer support or medical consultations), and facilitating team consensus by presenting key points as questions. 
% % Additionally, emerging techniques, such as generating follow-up questions during conversations~\cite{hu2024designing} or creating multiple-choice questions to assess comprehension~\cite{cheng2024treequestions}, highlight the versatility of LLM-powered questioning systems. 
% % Integrating these innovations into QA-based systems could unlock new applications across a wide range of fields.


% Beyond email communication, a QA-based approach utilizing LLMs can alleviate users' workload while preserving a degree of agency in areas that require structured information gathering and intent-driven interactions.
% First, QA-based systems can streamline tasks that require the creation of formal response documents (\textit{e.g.}, academic rebuttals) or complex online applications (\textit{e.g.}, Visa application) by presenting key points to understand and address as questions.  
% Also, in settings requiring consensus-building, such as team projects, QA-based systems may facilitate discussions by presenting questions aimed at identifying mutual goals, challenges, or uncertainties. The system can help clarify differing needs and provide feedback to streamline the decision-making process, ultimately improving collaboration.

% \subsubsection{\red{Applicability Beyond Email Replying}}
% \label{sec:discuss_applicability_beyond_email_replying}
% \red{Our findings suggest that LLMs can generate questions to elicit users' intentions and help them organize their thoughts, thereby enabling more efficient and effective outputs through user interaction. 
% Moreover, approaches such as generating follow-up questions based on user responses during a conversation~\cite{hu2024designing}, or even creating multiple-choice questions to assess user comprehension~\cite{cheng2024treequestions}, demonstrate the evolving versatility of LLM-powered questioning systems. 
% This suggests that QA-based systems leveraging LLMs hold significant potential in a variety of domains where structured information elicitation and intent communication are required, extending far beyond email composition.}

% % 1
% \red{Beyond email communication, a QA-based approach utilizing LLMs can alleviate users' workload while preserving a degree of agency in areas that require structured information gathering and intent-driven interactions.
% First, QA-based systems can streamline tasks that require the creation of formal response documents (\textit{e.g.}, academic rebuttals) or complex online applications (\textit{e.g.}, Visa application) by presenting key points to understand and address as questions.  
% Also, in settings requiring consensus-building, such as team projects, QA-based systems may facilitate discussions by presenting questions aimed at identifying mutual goals, challenges, or uncertainties. The system can help clarify differing needs and provide feedback to streamline the decision-making process, ultimately improving collaboration.}

% 我々のfindingsは、LLMにはすでにユーザの意図を引き出したり、考えをまとめたりするための質問を生成でき、ユーザとのインタラクションを通じてより良いアウトプットを効率的に出力できることを示唆している
% また今回のResQのように、受信メールに基づいて質問を一度だけ生成するのではなく、会話中のユーザーの回答に基づいてフォローアップ質問をさせたり~\cite{}、さらにはユーザの理解度を評価するための多肢選択問題を生成する~\cite{}アプローチもとられている
% したがってLLM を利用した QA ベースのシステムは、電子メールの作成にとどまらず、構造化された情報の引き出しと意図の伝達が求められるさまざまな分野で潜在的可能性を秘めていることを示唆している
% 例えば...
% まず、構造的な文書作成をする必要がある場面において、QA-based systemは作業を効率化できる可能性がある
% 例えば、drafting academic rebuttals, legal case summaries, or detailed project reportsの作成の際、QA-based systemが重要な論点を質問として提示し、ユーザはそれに回答することで考えを整理するとともに、それが反映されたドラフトを受け取ることができる
% また、他者のニーズを明確化する必要がある場面において、QA-based systemは役に立つ可能性がある
% 例えば、カスタマーサポートや医療問診の際、QA-based systemが、事前に顧客の問題を把握するための質問を提示し、顧客がそれに回答することで、顧客は自分のニーズを明確化するとともに、サポートする側は顧客のニーズを効率的に知ることができる
% さらに、合意形成が求められる場面において、複数のステークホルダーの意見を整理することで、合意形成を支援できる可能性がある
% 例えば、チームのプロジェクトにおいて、QA-based systemが、互いの目的や問題点、不明点等を洗い出すための質問を両者に提示することで、互いのニーズを明確化し、議論を円滑化できる可能性がある

% Designing the Conversational Agent: Asking Follow-up Questions for Information Elicitation
% 対話型エージェント(CAs)がインタビューや情報収集の場面で、事前に決まった質問だけでなく、会話中のユーザーの回答に基づいたフォローアップ質問を生成する能力を向上させる。
% ヒトのインタビュアーが用いるフォローアップ質問のテクニックを取り入れることで、CAsが有益な情報を引き出せるように設計する。

% TreeQuestion
% 1. 背景
% オープンエンド質問(自由回答形式)は、学生の理解を評価するために使われますが、AI(例えばChatGPT)の利用により、学生が容易に長文回答を生成できる時代において課題となっています。
% 教師は依然として回答を読む時間や学習成果を推測する負担を抱えています。
% 2. TreeQuestionシステム
% 目的: 教師が概念学習成果を評価するための多肢選択問題を効率的に作成する支援。
% 仕組み:
% 大規模言語モデルを利用して、与えられた概念を基に多肢選択問題を生成。
% 質問はツリー構造で整理され、異なる理解レベル(記憶、理解、応用、分析、評価、作成)に対応。
% 誤解を誘う選択肢(Distractors)も生成し、学生が正しい選択肢を選べるかどうかで学習成果を評価。
% 3. 特徴
% 人間とAIの協働:
% 教師はAIが生成した内容を検証・修正し、適切な質問を作成。
% 「探索(Explore)- 検証(Validate)- 生成(Generate)」という段階的プロセスを採用。
% 効率向上:
% オープンエンド質問と比較して、TreeQuestionによるMCQ作成と採点の時間は大幅に短縮される。

% 1. 相手のニーズを明らかにする必要がある場合(他者のニーズを明確化する)
% % For instance, in customer support and technical assistance, QA-based systems can guide users through troubleshooting processes by asking targeted questions and providing predefined response options. 
% % This approach can reduce cognitive load for both customers and support agents, improving the overall experience and efficiency of problem resolution.

% 2. 個人間、あるいは集団間において、合意形成を取る必要がある場合(両者のニーズを明確化する)。1の延長線上かもしれない。
% 複数のステークホルダーが関与する場において、QAベースシステムが議論の論点を動的に生成し、意思決定をサポートすることで、合意形成や問題解決を促進できる可能性がある。
% 例えば、チーム会議やプロジェクト計画において、互いのニーズを整理し、それを元にシステムが次のステップを提案することが可能かもしれない。

% 3. メールと同様に、特定の個人や集団に対して、考えや情報を整理し提供する必要がある場合(構造的な文書作成を効率的にする)
% % Moreover, content creation for structured documents, such as academic rebuttals, legal case summaries, or collaborative reports, can benefit from QA-based systems. 
% % By generating task-relevant questions based on the XXX, these systems could assist users in organizing their thoughts, ensuring that all critical aspects are addressed systematically.

% \subsection{\red{Design Implications for QA-based Systems in Broader Contexts}}
% To ensure the effectiveness of QA-based systems across diverse applications, we identify several design considerations from our findings.

% \subsubsection{\red{Optimizing User Experience in QA-based Systems}}
% \red{The results of our studies suggest that customizing the questions and options generated by QA-based systems according to the task, communication context, and user characteristics can enhance the user experience. }

% \paragraph{\red{\textbf{Content of Questions}}}
% % 何をすべきか、なぜそうすべきかの順に書く
% % personalizationとかはあまり言わない方がいい
% \red{First, the system should generate questions that are relevant, precise, and easy to answer, while accurately reflecting the sender's intent. 
% For example, our studies revealed Yes/No questions or specific scheduling-related questions were particularly useful.
% One potential solution is to provide the system with prompts containing relevant user information, which can improve the accuracy and relevance of the generated questions.}

% \paragraph{\red{\textbf{Quantity of Questions}}}
% \red{Next, the system should maintain an optimal balance in the number of questions generated.
% Our findings revealed mixed reactions to the number of questions provided: while some participants appreciated confirmation questions (\textit{e.g.,} ``Do you understand XX?'') for ensuring clarity, others found them redundant. 
% Furthermore, generating too many questions often led to verbose and unfocused replies.
% To address this, the system should dynamically adjust the number of questions based on the complexity of the task and the user’s preferences. 
% Additionally, refining prompt designs to produce concise and relevant text can mitigate the issue of excessive verbosity in responses.}

% \paragraph{\red{\textbf{Order of Questions}}}
% \red{The logical sequencing of questions also plays a vital role in enhancing usability. 
% The system used in this study generated questions in accordance with the flow of the email, which could help participants better understand the content. 
% However, there is potential to further enhance the system by prioritizing questions based on their importance or relevance.
% Future developments could consider adjusting the order of questions based on their importance or relevance. 
% In addition, UI enhancements, such as grouping questions by priority or task category, may help users navigate the interaction more intuitively.}

% \paragraph{\red{\textbf{Efficiency of Question Generation}}}
% \red{The time taken to generate questions emerged as a source of frustration for some participants. 
% This issue can potentially be addressed by improving the processing speed of LLMs or implementing pre-generation mechanisms. 
% For example, questions could be generated in advance, before the user opens the email, thereby reducing waiting times and improving overall efficiency.}

% \paragraph{\red{\textbf{Balance and Flexibility of Options}}}
% \red{Finally, the system should ensure a balance in the number and diversity of response options provided. 
% Participants reported a decline in user experience when options did not align with their intent or when irrelevant options were presented. 
% Simple Yes/No choices are useful, but participants also expressed a need for more flexible input methods, such as scheduling tools or free-text fields.
% To address this, introducing customizable input types based on the specific task (\textit{e.g.,} UI components like calendar selectors for scheduling) can further enhance usability.}

% \subsubsection{Strategies to Maintain User Agency and Authenticity}
% なんでそれが重要か、何ができるかをfindingsから書く?
% To mitigate potential reductions in agency, allowing users to adjust the level of AI intervention or providing options to customize the AI's contributions may be beneficial....

% \subsubsection{Mitigating Risks of Over-Reliance on AI}
% % We found one concern associated with using QA-based systems is the potential risk of users becoming overly reliant on the system. 
% While the QA-based system offers advantages, one concern is the potential risk of users becoming overly reliant on the technology. 
% Interviews revealed that users might become accustomed to trusting AI-generated questions and drafts due to their high quality and the desire to reduce workload. 
% Consequently, they may become less diligent in reviewing the emails they receive or the responses they send.
% This over-reliance could lead to miscommunication or the omission of important details, undermining the primary goal of using AI to improve communication efficiency.
% % Furthermore, our findings suggest that the QA-based approach may alter users’ perceptions of email correspondence, making it feel more like completing a survey than engaging in meaningful communication.
% % Given the increasing integration of AI systems into daily tasks, this shift indicates that email's purpose and role as a communication tool may change.
% Thus, further exploration is needed to understand the long-term impact of AIMC tools on email communication's changing roles and needs. 
% Moreover, it is important to investigate in which scenarios and how interventions by AI should be made to ensure that the essence of human interaction is not compromised.

% \subsubsection{\red{Design Implications}}
% \red{Our studies suggest that tailoring questions and options in QA-based systems based on the task, communication context, and user characteristics can improve user experience.
% % 以下では、QA-basedシステムの設計において、検討および調整することができるオプションについて説明する。
% }
% \begin{enumerate}[]
%     \item \red{\textbf{Content of Questions:}}
%     \red{The system should generate questions that are relevant, precise, and easy to answer, accurately reflecting the sender's intent.
%     For example, Yes/No questions or specific scheduling-related questions proved particularly useful.
%     Providing the system with prompts containing relevant user information can improve the accuracy and relevance of generated questions.}
%     \item \red{\textbf{Quantity of Questions:}}
%     \red{The system must balance the number of questions generated.
%     While some users valued confirmation questions (\textit{e.g.}, ``Do you understand XX?''), others found them redundant, and answering many questions led to verbose replies.
%     Dynamic adjustment based on task complexity and user preferences, along with concise prompt designs, can address this issue.}
%     \item \red{\textbf{Order of Questions:}}
%     \red{Logical sequencing enhances usability.
%     In this study, questions generated according to email flow helped participants understand the content.
%     Prioritizing questions by relevance or grouping them by priority or task category in the UI could further improve navigation.}
%     \item \red{\textbf{Efficiency of Question Generation:}}
%     \red{Delays in generating questions frustrated participants.
%     Improving LLM processing speed or implementing pre-generation mechanisms (\textit{e.g.}, generating questions before users open emails) can reduce waiting times and improve efficiency.}
%     \item \red{\textbf{Balance and Flexibility of Options:}}
%     \red{A balance in response options is crucial.
%     Participants found user experience declined when options were irrelevant or inflexible.
%     Simple Yes/No choices were helpful, but flexible inputs, such as scheduling tools or free-text fields, were also desired.
%     Customizable input types (\textit{e.g.}, calendar selectors for scheduling) can further enhance usability.}
% \end{enumerate}

\subsection{Limitations and Future Work}
While it is evident that the QA-based approach positively impacted users' workload, the quality of the emails they produced, and their relationship with recipients in formal email responses, this study had several limitations.
Though we tried to use a mixed-method study to triangulate the findings from the control experiment and field study, we acknowledged that the quantitative results could be limited.
Because of privacy concerns, we were unable to access participants' email content, and as a result, we could not gather users' behavioral data. 
This includes information such as how they edited the prompts, the amount of time they dedicated to responding to emails, or how ResQ influenced the language they used in their actual email communications.
We encourage researchers to explore alternative research methods for capturing users' behavioral data in email exchanges in the wild to enrich the understanding of QA-based approaches in AI-mediated communication.

% また、メールの特徴ごとのQA-based approachの有効性については、さらなる調査が可能だろう。
% Study 1では、フォーマルなシチュエーションにおける様々なトピックのメールを使用して実験を行い、QA-based approachの及ぼす影響について調査した。
% しかし、その特徴(例えば、状況のformalさ、メールの丁寧さ、重要性、受信者と送信者の関係性など)ごとに、QA-based approachの有効性は異なる可能性がある。
\red{Second, the effectiveness of the QA-based approach may vary depending on the specific characteristics of the emails. 
We conducted Study 1 using emails on a variety of topics within formal scenarios to examine the impact of the QA-based approach. 
However, its effectiveness may differ based on characteristics such as the formality of the situation, the politeness of the email, its importance, or the relationship between the sender and recipient.
Therefore, future research could explore how these specific email characteristics influence the effectiveness of QA-based approaches, potentially tailoring AI-mediated tools to different communication contexts.}
% First, the quantitative results were confined to a controlled environment. 
% This limitation arose because, in field studies, accessing participants' email content was not feasible due to privacy concerns, making it difficult to perform fair comparisons using quantitative evaluation metrics. 
% For example, the time required to compose an email and the necessity of a reply vary depending on the content and context. 
% Therefore, we designed the field study to assess how ResQ influenced the practice of email replies qualitatively.

\blue{Third, the study was conducted with participants from a single cultural background, which could limit the generalizability of our findings. 
Although we contributed to a new understanding for populations from non-Western countries~\cite{WEIRD_CHI21}, we acknowledge that the practice of email exchange differs across cultures~\cite{Robertson2021ICant}.
% The findings could not be generalized to other cultural contexts since the nature and role of emails differ across cultures~\cite{Robertson2021ICant}. 
% Although some participants used ResQ in communications with individuals from different cultural backgrounds and observed a degree of effectiveness, 
Further studies are encouraged to examine whether similar results would be obtained among users from diverse cultural backgrounds or in cross-cultural email exchanges.}

% Also, the system has the potential to be applied to devices other than PCs and adapted for communication tools beyond email. 
% For example, integrating ResQ into business chat platforms like Slack or Microsoft Teams seems highly compatible, as these environments require efficient, goal-oriented communication. 
% Moreover, this QA-based approach could be extended to other domains where structured document creation is necessary. 
% One possible application is drafting rebuttal letters, where a structured format and the ability to address specific points are necessary.

% \red{Fourth, while Study 1 evaluates the impact of the LLM-powered QA-based approach under three conditions, it does not include a condition that replicates the QA-based approach without relying on LLMs. 
% This omission limits the ability to isolate the specific contribution of LLMs to the system's overall performance. 
% Incorporating a QA-based condition in future research (\textit{e.g.}, employing a rule-based approach or using manually created questions and options) could offer a more comprehensive understanding of the unique value LLMs provide compared to rule-based or manual methods.}

\red{Fourth, while this study focused on a QA-based approach driven by LLMs, future research could explore alternative methods of question generation to deepen our understanding of QA-based AI assistance. 
For instance, comparing the LLM-powered system with approaches utilizing rule-based question generation or manually prepared questions and options may help disentangle the effects of algorithmic sophistication from the inherent benefits of structuring communication as QA. 
This may potentially clarify whether the AI placebo or nocebo effect~\cite{kloft2024aiplacebo} exists in AI-mediated communication.
Examining these different methods could offer further insights into when and why the QA-based approach excels and guide the design of more tailored systems that accommodate a wide range of communication tasks and user needs.}

\red{Fifth, while this study demonstrated the effectiveness of the QA-based approach with initial design considerations (Sec.~\ref{sec:Proposed_Approach}), future research could explore tailoring these questions to specific communication goals or contexts. 
For example, designers or instructors could adjust factors such as the number of questions, their difficulty level, or their thematic focus to improve the user's understanding of challenging content. 
By iterating on the design to explore how different dimensions of question can affect communication outcomes, future work can better guide the QA-based approach.}

% \red{Fifth, future research could design the questions in our approach to be tailored for each user, such as question difficulty, quantity, and thematic focus. 
% % Tailored question sets may help users understand challenging content, expedite task completion, or improve communication efficiency in specialized domains. 
% By iterating on the design, researchers can develop more refined QA-based systems that better meet user needs.}
\section{Conclusion}
\red{In formal email communication, users are often required to read detailed (lengthy or complex) emails. 
Crafting appropriate responses to such emails is time-consuming and may lead to overlooked sender requests or delayed responses, causing communication issues.}
Thus, we propose \red{QA-based approach}, which leverages LLM-based question generation to help users create efficient and high-quality replies by generating multiple question-answer pairs related to the received email content.
\red{To examine the comprehensive impact of the QA-based approach on both email senders and recipients, we conducted controlled and field experiments using our prototype system, \textit{ResQ}.
Our findings demonstrate that structuring email content into question-answer pairs improves efficiency, reduces cognitive load, and lowers barriers to initiating responses. 
Additionally, this approach enhances email quality and may leave a better impression on recipients.
However, our findings also revealed challenges, including a potential reduction in user agency and an increased psychological distance in communication. 
These trade-offs emphasize the need for adaptive designs that balance efficiency with personalization and user control.
% The QA-based approach shows promise for applications beyond email communication in domains requiring structured response generation. 
Future research should investigate the long-term effects of such systems on user behavior, cross-cultural differences in adoption, and the effectiveness of the QA-based approach across varying email characteristics}
% Future research should explore its long-term impacts, cross-cultural applicability, and integration into diverse communication platforms to optimize both efficiency and authenticity in AI-mediated interactions.}

% original
% In workplace email communication, users are often required to read lengthy emails and craft appropriate responses, a time-consuming task that may lead them to overlook parts of the sender's request or delay their response, causing communication issues.
% Thus, we propose \textit{ResQ}, which leverages LLM-based question generation to help users create efficient and high-quality replies by generating multiple question-answer pairs related to the received email content.
% Our controlled and field experiments confirmed that compared to a prompt-based approach, ResQ significantly improved email replying efficiency, reduced cognitive load, and lowered the barriers to task initiation.
% Additionally, AI support was shown to improve the quality of emails and enhance the recipient's impression.
% However, we observed that ResQ decreased users' sense of agency and control and enlarged the psychological distance between email senders and receivers.
% We also discussed communication scenarios where QA-based approach might be effective for AI-mediated communication.
\begin{acks}
This work was supported by JSPS KAKENHI (JP24H00742 and JP24H00748).
We thank all the participants for their interest and involvement in this study.
We also appreciate the reviewers for their constructive feedback, which helped us refine this work.
\end{acks}

% \bibliographystyle{ACM-Reference-Format}
% \bibliography{reference}
\documentclass{MITstyle}

%\usepackage[table]{xcolor}
\usepackage{chngcntr}
\usepackage{hyperref}
\usepackage{microtype}

\title{A Lightweight and Extensible Cell Segmentation and Classification Model for Whole Slide Images}

\author{Nikita Shvetsov~$^{1, }$\footnote{Correspondence e-mail: nikita.shvetsov@uit.no}, Thomas K. Kilvaer~$^{2, 3}$, Masoud Tafavvoghi~$^{4}$, Anders Sildnes~$^{1}$, \\ Kajsa Møllersen~$^{4}$, Lill-Tove Rasmussen Busund~$^{5, 6}$, Lars Ailo Bongo~$^{1}$ \\
%
\vspace{1em} % Space between authors and afilliations
%
\normalfont{\small $^{1}$Department of Computer Science, UiT The Arctic University of Norway}\\
\normalfont{\small $^{2}$Department of Oncology, University Hospital of North Norway}\\
\normalfont{\small $^{3}$Department of Clinical Medicine, UiT The Arctic University of Norway}\\
\normalfont{\small $^{4}$Department of Community Medicine, UiT The Arctic University of Norway}\\
\normalfont{\small $^{5}$Department of Medical Biology, UiT The Arctic University of Norway} \\
\normalfont{\small $^{6}$Department of Clinical Pathology, University Hospital of North Norway} %\vspace{2em}
}

\begin{document}
\maketitle

\section*{Abstract}

% \begin{abstract}
% Developing clinically useful cell-level analysis tools in digital pathology remains challenging due to limitations in dataset granularity, inconsistent annotations, computational demands of advanced models, and difficulties in integrating new technologies into clinical workflows. To address these challenges, we propose a multi-faceted solution that enhances data quality, model performance, and usability to create a lightweight and extensible cell segmentation and classification model.

% First, we update data labels by employing a cross-relabeling process that refines the labels of two existing datasets, PanNuke and MoNuSAC, to create a new unified dataset with enhanced granularity, encompassing seven distinct cell types. Second, we leverage the H-Optimus foundation model as a fixed encoder to improve feature representation for simultaneous cell segmentation and classification tasks. Third, to address the computational demands of foundation models, we employ knowledge distillation to reduce model size and complexity while maintaining comparable performance. Finally, to facilitate integration into clinical workflows, we integrate the distilled model into the QuPath software, a widely used open-source platform in digital pathology.

% Our results demonstrate improvements in cell segmentation and classification performance using the H‑Optimus-based model compared to a CNN-based model. Specifically, the average $R^2$ improved from 0.575 to 0.871, and the average $PQ$ score improved from 0.450 to 0.492, indicating better alignment with actual cell counts and enhanced segmentation and classification quality. Furthermore, the distilled student model maintains performance comparable to the larger foundation model while reducing the parameter count by a factor of 48.
% Overall, by reducing computational complexity and integrating it into existing workflows, the proposed approach may significantly impact diagnostic processes, reduce the workload of pathologists, and contribute to improved patient outcomes. Though our approach shows potential enhancements in efficiency and usability of cell segmentation and classification models in digital pathology, extensive validation is needed to deploy these models in clinical practice.
% \end{abstract}

%%% shortened abstract
\begin{abstract}
Developing clinically useful cell-level analysis tools in digital pathology remains challenging due to limitations in dataset granularity, inconsistent annotations, high computational demands, and difficulties integrating new technologies into workflows. To address these issues, we propose a solution that enhances data quality, model performance, and usability by creating a lightweight, extensible cell segmentation and classification model. 

First, we update data labels through cross-relabeling to refine annotations of PanNuke and MoNuSAC, producing a unified dataset with seven distinct cell types. Second, we leverage the H-Optimus foundation model as a fixed encoder to improve feature representation for simultaneous segmentation and classification tasks. Third, to address foundation models' computational demands, we distill knowledge to reduce model size and complexity while maintaining comparable performance. Finally, we integrate the distilled model into QuPath, a widely used open-source digital pathology platform. 

Results demonstrate improved segmentation and classification performance using the H-Optimus-based model compared to a CNN-based model. Specifically, average $R^2$ improved from 0.575 to 0.871, and average $PQ$ score improved from 0.450 to 0.492, indicating better alignment with actual cell counts and enhanced segmentation quality. The distilled model maintains comparable performance while reducing parameter count by a factor of 48. By reducing computational complexity and integrating into workflows, this approach may significantly impact diagnostics, reduce pathologist workload, and improve outcomes. Although the method shows promise, extensive validation is necessary prior to clinical deployment.
\end{abstract}
\clearpage

\section{Introduction}
In digital pathology, accurate segmentation and classification of cells are crucial for many diagnostic, prognostic, and predictive analyses \cite{Jaber_Beziaeva_etal._2019,Lin_Pan_etal._2022,Park_Ock_etal._2022,Shen_Choi_etal._2024}. Nowadays, developments in computational pathology offer multiple solutions \cite{H._Qu_P._Wu_etal._2020,Javed_Mahmood_etal._2020} to utilize cell-level datasets to train machine learning models that solve these problems. The quality and specificity of training datasets are critical for robust and accurate models. Adhering to the principle of "garbage in, garbage out", it is essential to ensure that these datasets are extensively and accurately labeled with distinct classes that reflect the diverse biological characteristics of different cell types. Unfortunately, the number of open-source datasets comprising such high-quality annotations is limited. Existing cell segmentation datasets \cite{Gamper_Koohbanani_etal._2019,Graham_Vu_etal._2019,Verma_Kumar_etal._2021} may offer extensive annotations for certain cell types while providing more general labels for others. For example, in PanNuke, which is one of the largest open-source datasets comprising labeled cells, various types of morphologically and functionally different inflammatory cells like macrophages and lymphocytes are clustered in a broad "inflammatory" class. Consequently, these classes are frequently omitted from analyses or aggregated into broader meta-classes \cite{Gamper_Koohbanani_etal._2020} and likely interfere with other cell classes included in the dataset. This and similar inconsistencies in annotation granularity limit the ability of machine learning models to learn the comprehensive and nuanced features necessary for accurate cell segmentation and classification. To address these challenges, methods for refining and standardizing dataset annotations are essential to enhance the quality of training data.

A complementary approach to mitigate the absence of high-quality training data is the use of foundation models. Foundation models as encoders are defined as large-scale, versatile networks pre-trained on vast, diverse datasets using self-supervised learning, contrasting with convolutional neural network (CNN) pre-trained encoders that rely on supervised learning with labeled data. In practice, foundation models leverage enormous amounts of weakly or unlabeled data from millions of whole slide images (WSIs) and employ self-attention mechanisms to capture long-range dependencies and global context \cite{Chen_Ding_etal._2024,Saillard_Jenatton_etal._2024,Vorontsov_Bozkurt_etal._2024,Xu_Usuyama_etal._2024}. As a consequence, foundation models are able to produce transferable feature representations across different cell types and tissue environments. The feature representations can be leveraged by decoder networks to produce segmentation masks and pixel-level classifications. Because foundation models have comprehensive feature representations, they can be effectively fine-tuned using much smaller amounts of cell-level data compared to the large datasets needed to train models from scratch. Furthermore, foundation models incorporate adversarial training elements or contrastive learning \cite{Chen_Ding_etal._2024,Xu_Usuyama_etal._2024}, enhancing their resilience and adaptability by exposing them to challenging and varied scenarios during training. This may result in more generalizable models, often making them well-suited for diverse and complex tasks in digital pathology.

Despite the inherent advantages of foundation models, their deployment for practical use faces its own obstacles. In particular, they require substantial computational power, financial investments and rigorous testing to ensure reliability and efficacy for a given task \cite{Akkus_Dangott_etal._2022,Dragomir_Cocuz_etal._2022,Go_2022,Jafri_Farooqui_etal._2024}. Moreover, while foundation models enhance feature representation and performance, they depend on the quality of available annotations for decoder fine-tuning and, like any other model, cannot resolve existing inconsistencies or ambiguities in data labels. Therefore, there remains a critical need for solutions that address both data quality and practical deployment considerations.
Further, integrating new technologies into existing clinical workflows often encounters resistance, as it necessitates adjustments to established diagnostic processes. So, there is a need to develop solutions that could be integrated into current practices, minimizing the burden on medical professionals to adopt new tools \cite{King_Williams_etal._2023}.

Existing solutions \cite{Goldsborough_Philps_etal._2024,Hörst_Rempe_etal._2024}, while addressing some aspects of these challenges, fall short in providing a comprehensive approach. To address the data quality and clinical deployment issues, we propose a multi-faceted solution that encompasses data refinement, model optimization, and integration with existing pathology tools (\hyperref[fig:fig1]{Figure 1}). The outcome is a lightweight cell segmentation and classification model that can be integrated into digital pathology workflows for practical clinical use.

\begin{figure}[h!]
    \centering
    \includegraphics[width=\textwidth, height=0.82\textheight, keepaspectratio]{images/Figure_1.pdf}
    \caption{Overview of the proposed solution, including 1) Data refinement using cross-relabeling, 2) Teacher model development and fine tuning, 3) Student model optimization with knowledge distillation and 4) Student model and QuPath integration}
    \label{fig:fig1}
\end{figure}
\clearpage

Our approach begins with preparing the data for the fine-tuning and training of the machine learning models. We create a refined dataset, acquired via cross-relabeling two cell-level datasets, enhancing annotation specificity and consistency of the labeled data. Subsequently, we create a cell segmentation and classification model based on the foundation model. We leverage the foundation model as a fixed encoder and fine-tune a decoder using the refined dataset to improve generalization across diverse tissue- and cell types.
To ensure that the model remains lightweight and deployable in a possibly resource-constrained environment, we employ knowledge distillation to approximate the functionality of the foundation model. Finally, to facilitate the practical application of our model in digital pathology workflows, we integrate it with the QuPath \cite{Bankhead_Loughrey_etal._2017} application. Each methodological component contributes to the overarching goal of enhancing model performance, generalizability, and usability in clinical settings.

The primary contributions of this paper are:
\begin{enumerate}
    \item \textit{Data labels refinement through cross-relabeling:}
    
    We propose a new method for refining labels of cell-level datasets through cross-relabeling. This method employs classification models to re-label broad and ambiguous instances, resulting in a more diverse dataset. Our evaluation demonstrates that these classification models achieve high accuracy on test subsets, indicating the reliability of the method for label refinement.

    \item \textit{Enhanced model performance via foundation models:}
    
    We employ a foundation model as a feature extractor for the cell segmentation and classification task. In comparison with training a CNN model from scratch, the foundation model backbone only needs fine-tuning, which significantly reduces training time, computational resources and data requirements. We show that using a foundation model encoder leads to better performance in cell segmentation and classification networks than using a CNN-based encoder. This improvement may enable the model to generalize more effectively across various tissue types and imaging methods.
    
    \item \textit{Model optimization through knowledge distillation:}
    
    We show that a smaller student model trained using knowledge distillation on the refined dataset obtained via our cross-relabeling approach from a foundation model achieves comparable performance in cell segmentation and quantification tasks. As a result, this model is more suitable for deployment in environments without high-performance computing resources.
    
    \item \textit{Integration with QuPath:}
    
    We integrate the distilled cell segmentation and classification model into QuPath, a widely used open-source digital pathology platform, to accelerate clinical adaptation by enabling pathologists to more easily incorporate advanced computational tools into their existing workflows.
\end{enumerate}

Through these methodological steps, we aim to bridge the gap between advanced machine learning techniques and practical clinical applications, making accurate and efficient digital pathology accessible in a broader range of healthcare settings.

\section{Refining Existing Datasets Using Cross-Relabeling}
To address the limitations of sparse and ambiguous labeling of cell-level datasets, we propose a generalizable cross-relabeling strategy that can be applied to any dataset containing broadly categorized or imprecisely labeled cell types. This approach involves training and subsequently leveraging classification models to refine broad categories into more specific or biologically relevant classes.
When applied to cell-level data, the methodology includes extracting individual cell images from the dataset patches, preprocessing these images to standardize the size and accommodate partial cells, and then training deep learning classifiers capable of distinguishing between the finer cell subtypes within the coarser categories. 
To illustrate our approach, we focus on the PanNuke \cite{Gamper_Koohbanani_etal._2020, Gamper_Koohbanani_etal._2019} and MoNuSAC \cite{Verma_Kumar_etal._2021} datasets that we have used to train models for cell quantification in our previous works \cite{Shvetsov_Grønnesby_etal._2022,Shvetsov_Sildnes_etal._2024}. We find that for better cell differentiation we have to introduce more granular labels. PanNuke includes a broad classification of "inflammatory" cells, encompassing lymphocytes, macrophages, and neutrophils. Each cell type differs significantly in structure, function, and clinical relevance. Conversely, MoNuSAC uses the label "epithelial" for a class that comprises both benign epithelial cells and malignant neoplastic cells. This practice makes it challenging to differentiate between benign and malignant epithelial cells in the dataset, which is a critical distinction when identifying tumor areas within tissue samples. To address these issues, we implement a cross-relabeling strategy as shown in \hyperref[fig:fig2]{Figure 2}. The key components are two classification models: one is trained on singular cell images from PanNuke data to classify the epithelial meta-class into epithelial and neoplastic classes. The other is trained on MoNuSAC to refine the inflammatory class into lymphocytes, neutrophils, and macrophages.

\begin{figure}[h!]
    \centering
    \includegraphics[width=\textwidth]{images/Figure_2.pdf}
    \caption{Refined dataset generation via cross relabeling}
    \label{fig:fig2}
\end{figure}

The refining approach consists of three consecutive steps. The first is the preprocessing step, in which we extract individual cells from both datasets (\hyperref[fig:fig3]{Figure 3}). The specifics of PanNuke and MoNuSAC patch preparation before cell preprocessing are provided in \hyperref[chap:S1]{Appendix S1}.

\begin{figure}[h!]
    \centering
    \includegraphics[width=\textwidth]{images/Figure_3.pdf}
    \caption{Cell instances preprocessing including (1) cell map extraction, (2) bounding box delineation, (3) adjusting cell boxes and (4) cropping and resizing of cell images}
    \label{fig:fig3}
\end{figure}

During preprocessing, we extract cell type maps from the ground truth label mask and calculate bounding boxes around each cell instance. To accommodate partial cells at patch borders, a common issue in cropped patch images, we employ mirror padding and extend the field of view of the cell label by 15 pixels to capture adjacent cells. We then crop and resize the identified regions to $64 \times 64$ pixels using bicubic interpolation.

The preprocessed PanNuke dataset comprises 68,031 neoplastic and 23,207 epithelial cell images, while MoNuSAC comprises  33,104 lymphocytes, 1,252 neutrophils, and 1,695 macrophages, which we subsequently use in training cell classification models and classifying the cell image data \hyperref[fig:S2]{Appendix Figure S2 (1)}. 

The next step is to train two distinct ResNet50-based classifiers tailored to address the specific labeling challenges inherent in each dataset. We use ResNet50 for classification models due to its proven effectiveness for image classification tasks in histopathology \cite{pan2022reviewmachinelearningapproaches}, and its compatibility with small images. For the PanNuke dataset, we design the classifier, trained on MoNuSAC data, to disaggregate the heterogeneous "inflammatory" cell category into distinct subtypes: lymphocytes, macrophages, and neutrophils. Similarly, for the MoNuSAC dataset, the classifier is trained on PanNuke data and distinguishes between benign and malignant epithelial cells within the overarching "epithelial" label. By applying these targeted classifiers to their respective datasets, we assign more specific labels to individual cell instances, thus enabling us to create a unified dataset.
To ensure a balanced representation of classes, we train both models on datasets that had been equalized to match the size of the least represented class. Thus, we obtain datasets comprising 23,207 samples per class for PanNuke and 1,252 samples per class for MoNuSAC data. Next, we partition both of them into training (70\%), validation (20\%), and testing (10\%) subsets. To mitigate the risk of overfitting, we use a single dropout layer with a rate of p=0.5 in both models and data augmentation using randomized color perturbations, rotation, and horizontal and vertical flipping. We employ AdamW optimizer and the cross-entropy loss function for the training criterion.

To evaluate the two trained models, we measure the classification accuracy on the respective test subsets. The accuracies on the test subset for both classifiers are presented in \hyperref[tab:1]{Table 1}. The PanNuke model achieves an average accuracy of 93.57\%, with higher accuracy for neoplastic cells (96.06\%) compared to epithelial cells (86.26\%). The confusion matrix in Figure A3.1 shows that the model predominantly distinguishes accurately between epithelial and neoplastic tissues, with a substantial number of correct classifications and relatively few misclassifications. The MoNuSAC model demonstrates an average accuracy of 98.92\%, excelling in classifying lymphocytes (99.67\%) and macrophages (94.12\%), with lower performance for neutrophils (85.71\%). The confusion matrix in Figure A3.2 shows that the model identifies lymphocytes and performs reasonably well with macrophages and neutrophils.

\begin{table}[h!]
\renewcommand{\arraystretch}{1.5}
  \centering
  \caption{Cell classification results for PanNuke and MoNuSAC trained models (CI 95\%).}
  \label{tab:1}
  \begin{tabular}{|l|c|c|}
   \hline
   %\rowcolor{gray!30}
    Accuracy               & PanNuke model              & MoNuSAC model              \\
    \hline
    Average      & 0.936 (0.931--0.941)         & 0.989 (0.986--0.993)        \\
    \hline
    Neoplastic   & 0.961 (0.956--0.965)         & -                          \\
    \hline
    Epithelial   & 0.863 (0.849--0.877)         & -                          \\
    \hline
    Lymphocytes  & -                          & 0.997 (0.995--0.999)        \\
    \hline
    Neutrophils  & -                          & 0.857 (0.796--0.918)        \\
    \hline
    Macrophages  & -                          & 0.941 (0.906--0.976)        \\
    \hline
  \end{tabular}
\end{table}

Finally, during the last step, we use the model trained on PanNuke data for epithelial cells in MoNuSAC and the model trained on MoNuSAC for the inflammatory cells class in PanNuke. Specifically, we use classifier models to relabel epithelial cells in MoNuSAC and inflammatory cells in PanNuke data. Then we combine cells with refined labels and the rest of the cells in both datasets to create a refined dataset (\hyperref[fig:S2]{Appendix Figure S2 (2)}). The process of relabeling cells and visualizing them on a patch is shown in \hyperref[fig:fig4]{Figure 4}. The cell counts in the refined dataset are provided in \hyperref[tab:S4]{Appendix Table S4}.

\begin{figure}[h!]
    \centering
    \includegraphics[width=\textwidth, height=0.42\textheight, keepaspectratio]{images/Figure_4.pdf}
    \caption{Cell relabeling procedure for epithelial and inflammatory cell classes}
    \label{fig:fig4}
\end{figure}

%\hfill

Relabeling and combining datasets have been explored in a prior study \cite{Parulekar_Kanwat_etal._2023}, where consecutive fine-tuning on multiple datasets was employed to account for hierarchical class label structures. While the method presented in \cite{Parulekar_Kanwat_etal._2023} is intuitive, it often lacks consistency and requires multiple fine-tuning runs, which can be cumbersome and time-consuming. 
In contrast, cross-relabeling simplifies this process by using specialized classification models tailored to each dataset's specific labeling challenges. This approach provides better transparency and produces a unified dataset encompassing seven distinct cell types across multiple tissue samples, enhancing data diversity for further model training or fine-tuning.

Despite these improvements, cross-relabeling does not entirely resolve issues related to poor labeling quality or the amount of labeled data. Specifically, our results show lower accuracies persist for underrepresented classes, such as macrophages, which may stem from a limited sample availability and intrinsic challenges in distinguishing these cells based solely on H\&E staining. Furthermore, while our method enhances label specificity, it relies on the initial quality of the broad labels; thus, any fundamental inaccuracies in the original annotations can propagate through the relabeling process. Addressing the overall problem of limited data labels may require integrating additional data sources or utilizing complementary immunohistochemical staining methods.
Although the reported performance metrics are obtained from evaluations on the native test sets of each dataset, it is important to note that the primary application of these classifiers is to perform cross-relabeling, where a model trained on one dataset (e.g., PanNuke) is applied to another (e.g., MoNuSAC) and vice versa. We acknowledge that a more systematic evaluation of cross-dataset generalization is needed and could be performed in future work.

Overall, the refined dataset produced by our approach can enhance the supervised training or fine-tuning of cell segmentation and classification models, especially those that utilize pre-trained foundation models to improve feature extraction robustness. In addition, these models can detect nuanced classes that enable researchers to conduct more detailed analyses of biological processes in computational pathology.

\section{Foundation models for robust cell segmentation and classification}

Accurate cell segmentation and classification in digital pathology are hindered by limited labeled data and the fact that conventional CNNs are unable to capture global contextual information due to their local receptive field constraints \cite{Gheflati_Rivaz_2022,Yang_Marcus_etal.}. Traditional approaches in cell quantification have predominantly relied on CNN encoders, such as ResNet50, given their proven effectiveness in semantic segmentation tasks \cite{Deshmane_2023,Graham_Vu_etal._2019,Mukasheva_Koishiyeva_etal._2024,Stringer_Wang_etal._2021}. However, approaches that include fine-tuning of pretrained CNNs, data augmentation, and stain normalization to partially increase data variability and address staining differences often fail to achieve the necessary generalization and robustness across diverse tissue types and staining conditions \cite{G._Wang_W._Li_etal._2018,Gao_Bagci_etal._2018,Karim_El_Khoury_Martin_Fockedey_etal._2021}.

To overcome these challenges, we leverage an encoder-decoder network that uses a foundation model as the encoder and a CNN upsampling decoder (\hyperref[fig:fig5]{Figure 5}) for simultaneous cell segmentation and classification in 2D patches extracted from WSIs. Foundation models with transformer-based architectures are viable alternatives to CNN-based encoders \cite{Shamshad_Khan_etal._2023,Sourget_2023}. They enable the creation of more advanced architectures that can decode or transform learned features more effectively \cite{Chen_Duan_etal._2023,Cheng_Misra_etal._2022,Xie_Wang_etal._2021}.

\begin{figure}[h!]
    \centering
    \includegraphics[width=\textwidth]{images/Figure_5.pdf}
    \caption{UNETR-like model with foundational model as backbone}
    \label{fig:fig5}
\end{figure}

By utilizing a transformer-based encoder, we incorporate global contextual information into the feature extraction process, which is a key advantage of such architectures \cite{Chen_Lu_etal._2021}. This foundation model integration facilitates accurate pixel-wise segmentation and classification without the need for extensive encoder training, thereby potentially improving generalization across varied cellular structures and tissue types.
In our implementation, we employ a modified UNETR \cite{Hatamizadeh_Tang_etal._2021} architecture that combines a vision transformer (ViT) \cite{Dosovitskiy_Beyer_etal._2021} encoder with a CNN-based decoder. The encoder utilizes the pretrained H-Optimus foundation model, which contains 1.1 billion parameters and is trained on over 500,000 H\&E stained WSIs \cite{Saillard_Jenatton_etal._2024}. We extract outputs from four evenly spaced transformer blocks $Z_i$, where $i \in [1, 14, 26, 38]$, to serve as residual connections for the CNN decoder. We select these blocks based on our observation that features from non-adjacent levels of the encoder lead to better overall performance on the test subset.

The CNN decoder upsamples the feature representations, acquired from the transformer blocks, to generate an intermediate vector that is handled by two task-specific layers that generate cell segmentation and classification masks. The first task-specific layer is the ‘Cellpose head’,  which is used to delineate cell instances. The layer generates horizontal and vertical gradient maps to form vector fields that are refined through gradient tracking in a post-processing step using the Cellpose algorithm \cite{Stringer_Wang_etal._2021}, known for its efficacy in cell segmentation tasks and generalizability across multiple domains \cite{Pachitariu_Stringer_2022,Stringer_Pachitariu_2024}. The second task-specific layer is the "Cell type head", which assigns labels to individual pixels. In the post-processing step, we determine the output classification label of each segmented cell instance by majority voting over the labeled pixels that comprise the cell in the segmentation map.

To evaluate model performance and measure the impact of adding a foundation model as backbone, we compare it to a ResNet50-based model. ResNet50 is a widely used solution for encoders in segmentation architectures in the medical domain \cite{Deshmane_2023,Graham_Vu_etal._2019,Mukasheva_Koishiyeva_etal._2024,Stringer_Wang_etal._2021}. For the H-Optimus-based model, we utilize frozen weights for the encoder and only fine-tune the decoder to take advantage of the extensive pre-training of the foundation model. For the ResNet50-based model we start with ImageNet \cite{Deng_Dong_etal.} weights and train both encoder and decoder parts. Hyperparameters for the training step are set to be identical, where possible, for comparable evaluation. 
For this evaluation, we deliberately use the PanNuke dataset to provide a standardized and controlled comparison between the H‑Optimus and ResNet50-based models (\hyperref[fig:S2]{Appendix Figure S2 (3)}). Specifically, we use two of the default PanNuke dataset splits (66\%) for training and validation, and reserve the third split (33\%) for testing.

To address the challenge of cell class imbalance in the PanNuke dataset, which is a common characteristic in most cell-level H\&E patch datasets, both models’ training processes employ a weighted loss function comprising cross-entropy and focal loss \cite{Lin_Goyal_etal._2018}. The focal loss component is adjusted with coefficients derived from each cell class' instance frequency, emphasizing learning from underrepresented classes and enhancing the model's sensitivity to rare but significant cellular patterns. The cross-entropy loss is augmented with spectral decoupling regularization \cite{Pezeshki_Kaba_etal._2021,Pohjonen_Stürenberg_etal._2022} and spatially varying label smoothing \cite{Islam_Glocker_2021}, which potentially stabilizes training and improves generalization in case of complex tissue morphologies. For optimization, we employ the \textit{AdamW} \cite{Loshchilov_Hutter_2019} to counter unbalanced class scenarios, with cosine annealing learning rate scheduler.

We utilize the scikit-learn library \cite{Van_der_Walt_Schönberger_etal._2014} and HoVer-Net \cite{Graham_Vu_etal._2019} implementations of $R^2$ (the coefficient of determination) and $PQ$ (panoptic quality) to evaluate our experiments. Complete mathematical formulations and detailed explanations of these metrics are provided in \hyperref[chap:S5]{Appendix S5}. To compute confidence intervals, we use nonparametric bootstrapping, where after calculating the metric on the full sample, we generated 1000 bootstrap replicates by resampling with replacement and then determined the 95\% confidence intervals as the 2.5th and 97.5th percentiles of the resulting empirical distribution.

%\hfill

The model comparisons are summarized in \hyperref[tab:2]{Table 2}. The H‑Optimus-based model achieves higher $R^2$ across all cell classes compared to the ResNet50-based model, which means that its predictions are more closely aligned with the PanNuke cell counts, indicating a stronger correlation with the observed data. Notably, the improvement of $R^2_{dead}$ may be an indicator of better global contextual representations provided by the foundation model backbone. In terms of segmentation and classification quality combined, measured by the PQ score, the H‑Optimus-based model demonstrates notable improvements across most cell classes. Overall, the average $R^2$ improved from 0.575 to 0.871, while the average $PQ$ score improved from 0.450 to 0.492, demonstrating better performance of the H-Optimus-based model.

\begin{table}[h!]
\renewcommand{\arraystretch}{1.5}
  \centering
  \caption{Cell quantification metrics for baseline and proposed models (CI 95\%).}
  \label{tab:2}
  \begin{tabular}{|l|c|c|}
    \hline
    %\rowcolor{gray!30}
    Metric             & Resnet50-based            & H-optimus-based              \\
    \hline
    $R^2_{neoplastic}$    & 0.681 (0.576--0.769)       & \textbf{0.941 (0.917--0.960)} \\
    \hline
    $R^2_{inflammatory}$  & 0.863 (0.778--0.903)       & \textbf{0.949 (0.918--0.966)} \\
    \hline
    $R^2_{connective}$    & 0.600 (0.488--0.698)       & 0.609 (0.436--0.772)          \\
    \hline
    $R^2_{dead}$          & 0.097 (-11.389--0.669)     & 0.925 (0.404--0.982)          \\
    \hline
    $R^2_{epithelial}$    & 0.635 (0.490--0.747)       & \textbf{0.930 (0.886--0.964)} \\
    \hline
    $PQ_{neoplastic}$       & 0.517 (0.499--0.535)       & \textbf{0.589 (0.575--0.604)} \\
    \hline
    $PQ_{inflammatory}$     & 0.455 (0.429--0.482)       & \textbf{0.528 (0.507--0.549)} \\
    \hline
    $PQ_{connective}$       & 0.416 (0.400--0.431)       & \textbf{0.451 (0.436--0.465)} \\
    \hline
    $PQ_{dead}$             & 0.374 (0.342--0.408)       & 0.292 (0.209--0.365)          \\
    \hline
    $PQ_{epithelial}$       & 0.488 (0.460--0.519)       & \textbf{0.599 (0.579--0.618)} \\
    \hline
  \end{tabular}
\end{table}

Our results  show that integrating the H‑Optimus foundation model within the UNETR architecture enhances the model's ability to segment and classify cells across diverse tissues from PanNuke data. The pretrained transformer encoder provides robust feature representations, resulting in higher average $R^2$ and $PQ$ scores compared to the CNN-based model. This leads to more reliable cell quantification and more accurate downstream analysis. Additionally, the streamlined fine-tuning process reduces computational overhead and training time, making the model more adaptable for new data.

Despite these advancements, the foundation model-based approach does not fully resolve all challenges related to cell segmentation and classification. We observe lower metric scores for underrepresented classes in the training data. Furthermore, foundation models typically encompass billions of parameters, resulting in substantial computational and memory requirements. It therefore poses challenges for deployment in resource-constrained environments, limiting their practical applicability in certain clinical settings.

\section{Model optimization via Knowledge Distillation}

To address the limitations posed by the extensive size of foundation models, we implement knowledge distillation — a model compression technique that leverages the teacher-student paradigm \cite{Hinton_Vinyals_etal._2015}. By training a smaller, more efficient student model to replicate the output of a larger, pre-trained teacher model, we retain performance while significantly reducing the model's complexity and resource requirements (\hyperref[fig:fig6]{Figure 6}).

\begin{figure}[h!]
    \centering
    \includegraphics[width=\textwidth, height=0.45\textheight, keepaspectratio]{images/Figure_6.pdf}
    \caption{Knowledge distillation framework for training a student model using a pre-trained teacher}
    \label{fig:fig6}
\end{figure}

We employ knowledge distillation to compress the H‑Optimus-based teacher model into a more efficient student model. The teacher model is the modified UNETR architecture with the H‑Optimus foundation model described in the previous chapter. The student model is based on a UNet architecture augmented with residual connections and incorporates a smaller ViT encoder with 9 million parameters \cite{Steiner_Kolesnikov_etal._2022,Wightman_2019}. 

First, we fine-tune the teacher model using the refined dataset from the cross-relabeling procedure (Section 2). Initially we train the decoder of the teacher model while keeping the encoder weights frozen. We split the refined dataset into train (70\%), validation (20\%) and test (10\%) subsets (\hyperref[fig:S2]{Appendix Figure S2 (4)}). During fine-tuning, we use the train and validation subsets, while leaving the test subset for model evaluation. We set the training procedure and model hyperparameters to be identical to those that were used to demonstrate the utility of foundation models for the simultaneous cell segmentation and classification task.

Next, we perform knowledge distillation from teacher to student using the refined dataset used to fine-tune the teacher model. The student model is trained to replicate the teacher model's outputs. We utilize a specialized loss function that aligns the student's predicted probability distribution with the teacher's, incorporating the teacher's class probability distribution derived from the output. Following the methodology of Hinton et al. \cite{Hinton_Vinyals_etal._2015}, we experiment with various hyperparameter settings for the temperature ($T$) and the balancing coefficients ($\alpha$ and $\beta$) in the loss function. We vary $T$ from 1 to 20 and adjust $\alpha$ and $\beta$ to balance the distillation and student losses. Through iterative tuning and evaluation, we identify that setting $T=14$, $\alpha=0.3$, and $\beta=0.7$ yields a configuration that converges and closely approximates the teacher model's performance during training.

Finally, we assess the performance of both models using the $R^2$ and $PQ$ (defined in \hyperref[chap:S5]{Appendix S5}) on the test set of the refined dataset (\hyperref[tab:3]{Table 3}). We observe that the 95\% confidence intervals overlap for most cell types, so we cannot claim statistically significant performance differences between the teacher and student models. One exception appears in the neoplastic class. The teacher model produces an $R^2$ of 0.919, while the student model shows an $R^2$ of 0.852. In addition, the student model achieves higher $PQ$ values for the neoplastic and connective classes, though the confidence intervals show overlap.

\begin{table}[h!]
\renewcommand{\arraystretch}{1.5}
  \centering
  \caption{Cell quantification metrics for teacher and distilled student models (CI 95\%).}
  \label{tab:3}
  \begin{tabular}{|l|c|c|}
    \hline
    %\rowcolor{gray!30}
    Metric & Teacher & Student \\
    \hline
    $R^2_{neoplastic}$    & \textbf{0.919} (0.898--0.939) & 0.852 (0.800--0.891) \\
    \hline
    $R^2_{lymphocyte}$    & 0.969 (0.956--0.977)         & 0.969 (0.956--0.978) \\
    \hline
    $R^2_{connective}$    & 0.694 (0.548--0.809)         & 0.618 (0.469--0.741) \\
    \hline
    $R^2_{dead}$          & 0.755 (0.400--0.908)         & 0.424 (0.100--0.731) \\
    \hline
    $R^2_{epithelial}$    & 0.922 (0.870--0.958)         & 0.843 (0.738--0.917) \\
    \hline
    $R^2_{macrophage}$    & 0.384 (-0.369--0.724)        & 0.704 (0.352--0.859) \\
    \hline
    $R^2_{neutrofil}$     & 0.854 (0.578--0.929)         & 0.833 (0.502--0.925) \\
    \hline
    $PQ_{neoplastic}$       & 0.581 (0.569--0.593)         & 0.601 (0.588--0.613) \\
    \hline
    $PQ_{lymphocyte}$       & 0.536 (0.520--0.553)         & 0.563 (0.544--0.579) \\
    \hline
    $PQ_{connective}$       & 0.436 (0.421--0.451)         & 0.457 (0.441--0.474) \\
    \hline
    $PQ_{dead}$             & 0.272 (0.235--0.315)         & 0.279 (0.201--0.369) \\
    \hline
    $PQ_{epithelial}$       & 0.522 (0.500--0.545)         & 0.530 (0.506--0.555) \\
    \hline
    $PQ_{macrophage}$       & 0.524 (0.459--0.588)         & 0.474 (0.405--0.543) \\
    \hline
    $PQ_{neutrofil}$        & 0.541 (0.490--0.592)         & 0.565 (0.522--0.607) \\
    \hline
  \end{tabular}
\end{table}


We further decompose the $PQ$ metric into its $SQ$ and $DQ$ components (\hyperref[tab:S6]{Appendix Table S6}). Both models produce nearly identical $SQ$ values, which indicates that they predict instance boundaries with similar precision. Although the student model shows some improvement in $DQ$ scores for certain classes, the confidence intervals overlap and do not confirm a statistically significant difference.

We observe that the student and teacher models yield comparable detection performance despite the student model using a much smaller and simpler architecture. A model with fewer parameters reduces the risk of overfitting when training data are scarce relative to the model’s complexity \cite{Farias_Ludermir_etal._2022}. The knowledge distillation process also encourages the student model to focus on the most generalizable detection features learned from the teacher. These factors enable the student model to achieve similar detection performance across different cell types.

Additionally, considering the model sizes reported in \hyperref[tab:4]{Table 4}, the distilled model achieves a significant reduction compared to the teacher model, with a 48-fold decrease in parameter count and a 5.5-fold reduction in on-disk size. In inference mode, the teacher model requires 16 GB of VRAM for a batch size of 32, while the distilled model only needs 3 GB of VRAM for the same batch size. These reductions make the distilled model significantly more practical for fine-tuning and deployment in resource-constrained environments.

\begin{table}[h!]
\renewcommand{\arraystretch}{1.5}
  \centering
  \caption{Parameter counts and size of teacher and distilled model}
  \label{tab:4}
  \adjustbox{max width=\textwidth}{%
  \begin{tabular}{|l|c|c|c|}
    \hline
    %\rowcolor{gray!30}
    Metric & H-optimus-based (Teacher) & mobileViT-based (Student) & Magnitude of difference \\
    \hline
    Parameters count       & 1,158,917,906   & \textbf{24,093,393}   & \textbf{48x}  \\
    \hline
    Estimated Total Size (MB) & 87,912       & \textbf{15,935}    & \textbf{5.5x} \\
    \hline
  \end{tabular}%
}
\end{table}

%\hfill

With recent advancements in complex network architectures and the use of pretrained encoders to achieve state-of-the-art performance \cite{Baumann_Dislich_etal._2024,Hörst_Rempe_etal._2024} in cell segmentation and classification tasks, model size, computational complexity, and processing times have increased. This limits the scalability and accessibility of these models. As we demonstrate, this may be mitigated using knowledge distillation. Studies in the field of natural language processing have demonstrated the efficacy of knowledge distillation in retaining the capabilities of the teacher model while achieving significant reductions in size and complexity \cite{Huangpu_Gao_2024,Sun_Yu_etal.}. 

We demonstrate the feasibility of knowledge distillation in digital pathology, specifically for cell segmentation and classification tasks. Moreover, we achieve this performance while also significantly reducing the parameter count. In addressing the challenge of knowledge transfer, we found that distillation from a transformer-based model to a smaller transformer is more straightforward than attempting to map transformer features to CNN blocks. In our experiments, using a CNN-based network as a student results in worse cell quantification performance due to the structural constraints of CNN feature space dimensions. 

Although our primary approach relies on a transformer-based student model that performs well, it can be further optimized to incorporate advantages from CNN architectures. For example, employing alternative techniques such as using ViT adapters \cite{Chen_Duan_etal._2023} or $1 \times 1$ convolutions to adjust feature map sizes may be beneficial for harnessing CNN advantages like enhanced local feature extraction. Moreover, if additional performance improvements are desired, the process can be further enhanced by applying supplementary knowledge distillation techniques, such as self-distillation \cite{Zhang_Song_etal._2019} or online distillation \cite{Houyon_Cioppa_etal._2023}.

Despite these promising results, further validation on independent datasets is necessary to fully understand the model's limitations. Underrepresented classes may pose challenges when addressing complex cases. Pathologists need to validate these models to adopt them in clinical settings. While the distilled models are smaller and more deployable, a technological gap persists because pathologists traditionally rely on established methods for inspecting WSIs and diagnosing diseases. Addressing the complexities involved in deploying models for inference and supporting pathologists in adopting new tools is essential for integrating these models into clinical workflows.

\section{Model integration with QuPath}
Digital pathology tools with graphical user interfaces are essential for visualizing and analyzing WSIs. To make our student model useful in clinical pathology workflows, it needs to be integrated into a tool that enables inspecting regions, creating annotations, and providing quantitative analyses of biomarkers. Therefore, we integrate the trained student model from the previous chapter into the QuPath open‑source platform \cite{Bankhead_Loughrey_etal._2017}. QuPath provides the required annotation, visualization, and analysis tools to interpret complex histological data, including workflows for cell segmentation, classification, and quantification (\hyperref[fig:fig7]{Figure 7}). 

\begin{figure}[h!]
    \centering
    \includegraphics[width=\textwidth]{images/Figure_7.pdf}
    \caption{Visualization of model-generated cell quantification annotations (left) and the corresponding unannotated slide (right) in QuPath}
    \label{fig:fig7}
\end{figure}

To identify the regions in a WSI critical for prognosticating tumor development, such as specific tumor areas or border regions without overlapping healthy tissue, the pathologist uses QuPath to outline these regions. Then, the pathologist initiates a cell segmentation and classification script through the QuPath interface for the selected regions. The resulting annotations and quantified cell information are then directly overlaid onto the WSI in the QuPath interface. Additional design and implementation details are in \hyperref[chap:S7]{Appendix S7}. 

Two common approaches for integrating deep learning models into QuPath are Java‑based native QuPath extensions \cite{Goldsborough_Philps_etal._2024} and the execution of RESTful API requests to a model server coupled with handling the response via an extension, as demonstrated in the application of cell segmentation models applied to immunofluorescence images \cite{Sugawara_2023}. While the community is actively working on these integration strategies, there is currently no universal solution that fully addresses all integration and performance requirements.

Extensions may offer better integration with QuPath, allowing slightly improved performance and more widespread usage of the built-in QuPath models, but they lack the flexibility to customize models and modify their behavior. For example, the newest version of QuPath includes models such as StarDist \cite{Weigert_Schmidt} and InstanSeg \cite{Goldsborough_Philps_etal._2024} that can perform cell segmentation. Both models pose limitations when applied to simultaneous cell segmentation and classification. StarDist performs well only on convex, round shapes by design, whereas some neoplastic, inflammatory, and connective cells exhibit complex and non-convex shapes. InstanSeg provides only semantic segmentation without assigning classes to the segmented cells.

%\hfill

In contrast, our approach offers an alternative integration strategy. It utilizes the paquo library to directly interact with QuPath’s internal application programming interface from within Python. This enables data exchange and processing without the need for intermediate conversion steps and provides greater control over model customization, retraining, and the incorporation of custom processing steps.

The integration of our custom model with QuPath underscores its potential to significantly enhance the diagnostic process by reducing the time burden on pathologists and enabling them to focus on more complex interpretative tasks using familiar software. Leveraging a tool that is already well-established among pathologists increases the likelihood of its adoption into daily clinical workflows. The quantitative data generated through the automated workflow is critical for both clinical decision-making and research, facilitating more accurate biomarker analysis, enabling robust statistical evaluations, and supporting hypothesis generation and testing. Additionally, by streamlining cell segmentation and classification, the tool enhances the scalability and reproducibility of pathological assessments, ultimately contributing to improved diagnostic accuracy and patient outcomes.

\section{Conclusion and future work}

In this study, we address critical challenges in digital pathology and tackle the usability and deployment issues of the developed models in standard computing environments without the need for high-performance computing systems. Our multi-faceted approach encompasses data refinement through cross-relabeling, leveraging foundation models for robust cell segmentation and classification, optimizing model performance via knowledge distillation, and integrating the optimized model into the QuPath software for practical application. This approach is used to construct a capable, versatile, and adjustable model for cell segmentation and classification, with enhanced performance and usability.

\begin{sloppypar}
While our approach shows potential in the field of computational pathology, certain limitations persist. 
For example, our implementation currently exhibits lower performance in detecting macrophages. 
This serves as an instance of the broader challenge of accurately identifying complex cell types. In order to address this issue, extending our approach to incorporate additional data sources, exploring alternative modeling approaches, and integrating other imaging modalities such as immunohistochemical staining may help improve detection accuracy. Moreover, although the distilled model reduces computational demands, integrating advanced deep learning models into clinical practice requires addressing technological gaps and potential resistance to adopting new tools within established diagnostic processes.
\end{sloppypar}

Future work could focus on several key areas to refine the proposed approach and facilitate its adoption in clinical environments. Enhancing the cell-relabeling process with additional datasets \cite{Graham_Jahanifar_etal._2021} could improve the representation of underrepresented cell types and enhance overall model performance. Also, incorporating additional data sources, such as multi-modal imaging or complementary staining methods, may address limitations related to cell type differentiation and class imbalance. Exploring other foundation models \cite{Vorontsov_Bozkurt_etal._2024,Zimmermann_Vorontsov_etal._2024} or introducing additional modalities \cite{Ding_Wagner_etal._2024,Vaidya_Zhang_etal._2025} may provide alternative architectures better suited to specific tasks or offer improved efficiency. Implementing more complex knowledge distillation techniques \cite{Houyon_Cioppa_etal._2023,Zhang_Song_etal._2019} could further optimize the model's performance and adaptability. Additionally, deeper integration with QuPath or other digital pathology software could provide pathologists more control over cell quantification analysis directly within the QuPath interface, thereby increasing accessibility and usability. Such enhancements would not only refine model performance but also ensure greater adaptability and scalability within various clinical environments. Finally, extensive validation of the model by pathologists and benchmarking against independent datasets are essential steps toward establishing the model's reliability and fostering confidence in its clinical utility.

\section*{Acknowledgments} 
This work was funded in part by the Research Council of Norway grant no. 309439 SFI Visual Intelligence, and the North Norwegian Health Authority grant no. HNF1521-20.

\bibliographystyle{IEEEtran}
\begin{sloppypar}
\begin{thebibliography}{99}

\bibitem{chaplot2020neural} Chaplot, Devendra Singh, et al. "Neural topological slam for visual navigation." Proceedings of the IEEE/CVF conference on computer vision and pattern recognition. 2020.

\bibitem{maksymets2021thda} Maksymets, Oleksandr, et al. "Thda: Treasure hunt data augmentation for semantic navigation." Proceedings of the IEEE/CVF International Conference on Computer Vision. 2021.

\bibitem{mezghan2022memory} Mezghan, Lina, et al. "Memory-augmented reinforcement learning for image-goal navigation." 2022 IEEE/RSJ International Conference on Intelligent Robots and Systems (IROS). IEEE, 2022.

\bibitem{al2022zero} Al-Halah, Ziad, Santhosh Kumar Ramakrishnan, and Kristen Grauman. "Zero experience required: Plug \& play modular transfer learning for semantic visual navigation." Proceedings of the IEEE/CVF Conference on Computer Vision and Pattern Recognition. 2022.

\bibitem{ye2021auxiliary} Ye, Joel, et al. "Auxiliary tasks and exploration enable objectgoal navigation." Proceedings of the IEEE/CVF international conference on computer vision. 2021.

\bibitem{chaplot2020object} Chaplot, Devendra Singh, et al. "Object goal navigation using goal-oriented semantic exploration." Advances in Neural Information Processing Systems 33 (2020)

\bibitem{ramakrishnan2022poni} Ramakrishnan, Santhosh Kumar, et al. "Poni: Potential functions for objectgoal navigation with interaction-free learning." Proceedings of the IEEE/CVF Conference on Computer Vision and Pattern Recognition. 2022.

\bibitem{ramrakhya2022habitat} Ramrakhya, Ram, et al. "Habitat-web: Learning embodied object-search strategies from human demonstrations at scale." Proceedings of the IEEE/CVF Conference on Computer Vision and Pattern Recognition. 2022.

\bibitem{mousavian2019visual} Mousavian, Arsalan, et al. "Visual representations for semantic target driven navigation." 2019 International Conference on Robotics and Automation (ICRA). IEEE, 2019.

\bibitem{dhariwal2021diffusion} Dhariwal, Prafulla, and Alexander Nichol. "Diffusion models beat gans on image synthesis." Advances in neural information processing systems 34 (2021)

\bibitem{ho2022classifier} Ho, Jonathan, and Tim Salimans. "Classifier-free diffusion guidance." arXiv preprint arXiv:2207.12598 (2022).

\bibitem{nichol2021glide} Nichol, Alex, et al. "Glide: Towards photorealistic image generation and editing with text-guided diffusion models." arXiv preprint arXiv:2112.10741 (2021)

\bibitem{brooks2023instructpix2pix} Brooks, Tim, Aleksander Holynski, and Alexei A. Efros. "Instructpix2pix: Learning to follow image editing instructions." Proceedings of the IEEE/CVF Conference on Computer Vision and Pattern Recognition. 2023.

\bibitem{fu2023guiding} Fu, Tsu-Jui, et al. "Guiding instruction-based image editing via multimodal large language models." arXiv preprint arXiv:2309.17102 (2023).

\bibitem{geng2024instructdiffusion} Geng, Zigang, et al. "Instructdiffusion: A generalist modeling interface for vision tasks." Proceedings of the IEEE/CVF Conference on Computer Vision and Pattern Recognition. 2024.

\bibitem{zhou2024minedreamer} Zhou, Enshen, et al. "Minedreamer: Learning to follow instructions via chain-of-imagination for simulated-world control." arXiv preprint arXiv:2403.12037 (2024).

\bibitem{zhou2023esc} Zhou, Kaiwen, et al. "Esc: Exploration with soft commonsense constraints for zero-shot object navigation." International Conference on Machine Learning. PMLR, 2023.

\bibitem{yu2023l3mvn} Yu, Bangguo, Hamidreza Kasaei, and Ming Cao. "L3mvn: Leveraging large language models for visual target navigation." 2023 IEEE/RSJ International Conference on Intelligent Robots and Systems (IROS). IEEE, 2023.

\bibitem{gadre2023cows} Gadre, Samir Yitzhak, et al. "Cows on pasture: Baselines and benchmarks for language-driven zero-shot object navigation." Proceedings of the IEEE/CVF Conference on Computer Vision and Pattern Recognition. 2023.

\bibitem{shah2023navigation} Shah, Dhruv, et al. "Navigation with large language models: Semantic guesswork as a heuristic for planning." Conference on Robot Learning. PMLR, 2023.

\bibitem{cai2024bridging} Cai, Wenzhe, et al. "Bridging zero-shot object navigation and foundation models through pixel-guided navigation skill." 2024 IEEE International Conference on Robotics and Automation (ICRA). IEEE, 2024.

\bibitem{yu2023co} Yu, Bangguo, Hamidreza Kasaei, and Ming Cao. "Co-NavGPT: Multi-robot cooperative visual semantic navigation using large language models." arXiv preprint arXiv:2310.07937 (2023).

\bibitem{wu2024voronav} Wu, Pengying, et al. "Voronav: Voronoi-based zero-shot object navigation with large language model." arXiv preprint arXiv:2401.02695 (2024).

\bibitem{qin2023mp5} Qin, Yiran, et al. "Mp5: A multi-modal open-ended embodied system in minecraft via active perception." arXiv preprint arXiv:2312.07472 (2023).

\bibitem{du2024learning} Du, Yilun, et al. "Learning universal policies via text-guided video generation." Advances in Neural Information Processing Systems 36 (2024).

\bibitem{ajay2024compositional} Ajay, Anurag, et al. "Compositional foundation models for hierarchical planning." Advances in Neural Information Processing Systems 36 (2024).

\bibitem{liang2024skilldiffuser} Liang, Zhixuan, et al. "Skilldiffuser: Interpretable hierarchical planning via skill abstractions in diffusion-based task execution." Proceedings of the IEEE/CVF Conference on Computer Vision and Pattern Recognition. 2024.

\bibitem{heusel2017gans} Heusel, Martin, et al. "Gans trained by a two time-scale update rule converge to a local nash equilibrium." Advances in neural information processing systems 30 (2017).

\bibitem{zhang2018unreasonable} Zhang, Richard, et al. "The unreasonable effectiveness of deep features as a perceptual metric." Proceedings of the IEEE conference on computer vision and pattern recognition. 2018.

\bibitem{brown2020language} Brown, Tom B. "Language models are few-shot learners." arXiv preprint arXiv:2005.14165 (2020).

\bibitem{podell2023sdxl} Podell, Dustin, et al. "Sdxl: Improving latent diffusion models for high-resolution image synthesis." arXiv preprint arXiv:2307.01952 (2023).

\bibitem{brohan2022rt} Brohan, Anthony, et al. "Rt-1: Robotics transformer for real-world control at scale." arXiv preprint arXiv:2212.06817 (2022).

\bibitem{brohan2023rt} Brohan, Anthony, et al. "Rt-2: Vision-language-action models transfer web knowledge to robotic control." arXiv preprint arXiv:2307.15818 (2023).

\bibitem{li2024manipllm} Li, Xiaoqi, et al. "Manipllm: Embodied multimodal large language model for object-centric robotic manipulation." Proceedings of the IEEE/CVF Conference on Computer Vision and Pattern Recognition. 2024.

\bibitem{shah2023vint} Shah, Dhruv, et al. "ViNT: A foundation model for visual navigation." arXiv preprint arXiv:2306.14846 (2023).

\bibitem{liu2024visual} Liu, Haotian, et al. "Visual instruction tuning." Advances in neural information processing systems 36 (2024).

\bibitem{hu2021lora} Hu, Edward J., et al. "Lora: Low-rank adaptation of large language models." arXiv preprint arXiv:2106.09685 (2021).

\bibitem{qin2023supfusion} Qin, Yiran, et al. "SupFusion: Supervised LiDAR-camera fusion for 3D object detection." Proceedings of the IEEE/CVF International Conference on Computer Vision. 2023.

\bibitem{qin2024worldsimbench} Qin, Yiran, et al. "Worldsimbench: Towards video generation models as world simulators." arXiv preprint arXiv:2410.18072 (2024).

\bibitem{yu2025gamefactory} Yu, Jiwen, et al. "GameFactory: Creating New Games with Generative Interactive Videos." arXiv preprint arXiv:2501.08325 (2025).

\bibitem{zhou2024code} Zhou, Enshen, et al. "Code-as-Monitor: Constraint-aware Visual Programming for Reactive and Proactive Robotic Failure Detection." arXiv preprint arXiv:2412.04455 (2024).

\bibitem{zhang2024ad} Zhang, Zaibin, et al. "AD-H: Autonomous Driving with Hierarchical Agents." arXiv preprint arXiv:2406.03474 (2024).

\bibitem{wang2024toward} Wang, Chaoqun, et al. "Toward Accurate Camera-based 3D Object Detection via Cascade Depth Estimation and Calibration." arXiv preprint arXiv:2402.04883 (2024).

\bibitem{huang2024story3d} Huang, Yuzhou, et al. "Story3d-agent: Exploring 3d storytelling visualization with large language models." arXiv preprint arXiv:2408.11801 (2024).

\bibitem{savinov2018semi} Savinov, Nikolay, Alexey Dosovitskiy, and Vladlen Koltun. "Semi-parametric topological memory for navigation." arXiv preprint arXiv:1803.00653 (2018).

\bibitem{majumdar2022zson} Majumdar, Arjun, et al. "Zson: Zero-shot object-goal navigation using multimodal goal embeddings." Advances in Neural Information Processing Systems 35 (2022): 32340-32352.

\bibitem{yadav2023offline} Yadav, Karmesh, et al. "Offline visual representation learning for embodied navigation." Workshop on Reincarnating Reinforcement Learning at ICLR 2023. 2023.

\bibitem{yadav2023ovrl} Yadav, Karmesh, et al. "Ovrl-v2: A simple state-of-art baseline for imagenav and objectnav." arXiv preprint arXiv:2303.07798 (2023).

\bibitem{sun2024fgprompt} Sun, Xinyu, et al. "FGPrompt: fine-grained goal prompting for image-goal navigation." Advances in Neural Information Processing Systems 36 (2024).

\bibitem{zhu2017target} Zhu, Yuke, et al. "Target-driven visual navigation in indoor scenes using deep reinforcement learning." 2017 IEEE international conference on robotics and automation (ICRA). IEEE, 2017.

\bibitem{koh2024generating} Koh, Jing Yu, Daniel Fried, and Russ R. Salakhutdinov. "Generating images with multimodal language models." Advances in Neural Information Processing Systems 36 (2024).

\bibitem{krantz2022instance} Krantz, Jacob, et al. "Instance-specific image goal navigation: Training embodied agents to find object instances." arXiv preprint arXiv:2211.15876 (2022).

\bibitem{schulman2017proximal} Schulman, John, et al. "Proximal policy optimization algorithms." arXiv preprint arXiv:1707.06347 (2017).

\bibitem{anderson2018evaluation} Anderson, Peter, et al. "On evaluation of embodied navigation agents." arXiv preprint arXiv:1807.06757 (2018).

\bibitem{lin2024navcot} Lin, Bingqian, et al. "NavCoT: Boosting LLM-Based Vision-and-Language Navigation via Learning Disentangled Reasoning." arXiv preprint arXiv:2403.07376 (2024).

\bibitem{NavGPT} Zhou, Gengze, Yicong Hong, and Qi Wu. "Navgpt: Explicit reasoning in vision-and-language navigation with large language models." Proceedings of the AAAI Conference on Artificial Intelligence.

\bibitem{hahn2021no} Hahn, Meera, et al. "No rl, no simulation: Learning to navigate without navigating." Advances in Neural Information Processing Systems 34 (2021): 26661-26673.

\bibitem{li2025t2isafety} Li, Lijun, et al. "T2ISafety: Benchmark for Assessing Fairness, Toxicity, and Privacy in Image Generation." arXiv preprint arXiv:2501.12612 (2025).

\bibitem{an2024agfsync} An, Jingkun, et al. "AGFSync: Leveraging AI-Generated Feedback for Preference Optimization in Text-to-Image Generation." arXiv preprint arXiv:2403.13352 (2024).


\end{thebibliography}
\end{sloppypar}

\clearpage
\beginsupplement
\section*{Appendix}
\renewcommand{\thesubsection}{S\arabic{subsection}}

\subsection{\label{chap:S1}PanNuke and MoNuSAC preprocessing}
The PanNuke dataset comprises a set of 7,901 RGB patches, each with dimensions of $256 \times 256$ pixels, which we set as the standard patch size for our analysis. In contrast, the MoNuSAC dataset encompasses 294 images of heterogeneous dimensions. To standardize the MoNuSAC images with our experiments, we implement a standardization protocol. Specifically, for images exceeding the dimensions of $256 \times 256$ pixels, we segment them into equal-sized patches and apply mirror padding to the remaining portions to avoid information loss at the peripherals. Patches with dimensions less than $128 \times 128$ pixels are excluded from the dataset due to the insufficient resolution to capture relevant cellular details. For patches where either dimension falls between 128 and 256 pixels, we employ upsampling to achieve the standard patch size. As a result, we obtain a total of 2,823 RGB patches derived from the MoNuSAC dataset for subsequent analysis. For additional details on the MoNuSAC data preparation process, refer to the source code \cite{Shvetsov_2025a}.
\clearpage

\subsection{\label{chap:S2}Data usage for the methodology}

\counterwithin{figure}{subsection}
\renewcommand{\thefigure}{S\arabic{subsection}}

\begin{figure}[h!]
    \centering
    \includegraphics[width=\textwidth, height=0.85\textheight, keepaspectratio]{images/A2.pdf}
    \caption{Overview of the methodology for cross-labeling, dataset refinement, and model comparison. (1) Cross-relabeling - training and testing cell classification models, (2) Cross-relabeling - using cell classification models to create refined dataset, (3) Fine-tuning and training models for comparison, (4) Student knowledge distillation with refined dataset}
    \label{fig:S2}
\end{figure}
\clearpage

\subsection{\label{chap:S3}Confusion matrices for classification models}
\counterwithin{figure}{subsection}
\renewcommand{\thefigure}{S\arabic{subsection}.\arabic{figure}}

\begin{figure}[h!]
    \centering
    \includegraphics[width=\textwidth, height=0.4\textheight, keepaspectratio]{images/A3_1.pdf}
    \caption{Confusion matrix for PanNuke trained model}
    \label{fig:S3.1}
\end{figure}

\begin{figure}[h!]
    \centering
    \includegraphics[width=\textwidth, height=0.4\textheight, keepaspectratio]{images/A3_2.pdf}
    \caption{Confusion matrix for MoNuSAC trained model}
    \label{fig:S3.2}
\end{figure}

\clearpage

\subsection{\label{chap:S4}Datasets cell counts}

\counterwithin{table}{subsection}
\renewcommand{\thetable}{S\arabic{subsection}}

\begin{table}[h!]
\renewcommand{\arraystretch}{2.0}
\centering
\caption{\label{tab:S4}Cell counts for PanNuke, MoNuSAC and refined datasets. Numbers in parentheses indicate preprocessed cell counts for cell classifier models training and testing.}
%\adjustbox{max width=\textwidth}{%
\begin{tabular}{|l|c|c|c|}
\hline
%\rowcolor{gray!30}
Cell type & PanNuke & MoNuSAC & Refined \\
\hline
Neoplastic & 77,403 (68,031) & - & 105,451 \\
\hline
Epithelial & 26,572 (23,207) & - & 29,926 \\
\hline
Epithelial (benign and malignant) & - & 31,402 & - \\
\hline
Inflammatory & 32,276 & - & - \\
\hline
Lymphocytes & - & 37,045 (33,104) & 65,275 \\
\hline
Neutrophils & - & 1,355 (1,252) & 3,833 \\
\hline
Macrophage & - & 1,842 (1,695) & 3,410 \\
\hline
Dead & 2,908 & - & 2,908 \\
\hline
Connective & 50,585 & - & 50,585 \\
\hline
\end{tabular}
%
%}
\end{table}



\clearpage

\subsection{\label{chap:S5}Definition of validation metrics}
\counterwithin{equation}{subsection}
\renewcommand{\theequation}{\arabic{equation}}

\subsubsection{\label{chap:S5.1}R\textsuperscript{2}}
The coefficient of determination, denoted as $R^2$, is a statistical measure that represents the proportion of variance in the dependent variable that is predictable from the independent variables. In the context of cell quantification in pathology, $R^2$ is used to assess how well the predicted quantities of different cell types in a patch align with the actual quantities observed in the ground truth data, with higher values representing more accurate quantification. $R^2$ is defined as
\begin{equation*}
R^2 = 1 - \frac{\sum_{i=1}^n (y_i - \hat{y}_i)^2}{\sum_{i=1}^n (y_i - \bar{y})^2},
\end{equation*}
where $y_i$ represents the actual number of cells of a specific type in the $i$-th image, $\hat{y}_i$ represents the predicted number of cells of that type in the $i$-th image, $\bar{y}$ is the mean of the actual numbers across all images, and $n$ is the total number of images in the dataset.

The $R^2$ metric has a range of $(-\infty, 1]$. An $R^2$ of 1 indicates perfect prediction, where all predicted values exactly match the actual values. An $R^2$ of 0 suggests that the model explains none of the variability of the response data around its mean. If $R^2$ is negative, it indicates that the model performs worse than a model that simply predicts the mean of the actual values for all observations.

\subsubsection{\label{chap:S5.2}PQ}
Panoptic Quality ($PQ$) is a comprehensive metric used to evaluate the performance of segmentation models in tasks that require both instance segmentation and classification. $PQ$ provides a single score that encapsulates both the detection accuracy (i.e., how many objects were correctly identified) and the segmentation quality (i.e., how accurately the objects' boundaries were delineated). This metric is particularly useful in multiclass scenarios where each pixel is classified into distinct categories, such as different cell types in pathology images.

$PQ$ is calculated as the product of two terms: Detection Quality ($DQ$) and Segmentation Quality ($SQ$). It can be expressed as
\begin{equation*}
PQ = DQ \cdot SQ,
\end{equation*}
where
\begin{equation*}
DQ = \frac{TP}{TP + 0.5\, FP + 0.5\, FN},
\end{equation*}
\begin{equation*}
SQ = \frac{\sum_{(p, g) \in \mathcal{M}} IoU(p, g)}{TP}.
\end{equation*}
In these formulas, $TP$ denotes the number of correctly matched instances between ground truth and prediction, $FP$ denotes the predicted instances that have no corresponding ground truth, $FN$ denotes the ground truth instances that were not detected, $IoU(p, g)$ is the Intersection over Union for a pair of matched instances $p$ (prediction) and $g$ (ground truth), and $\mathcal{M}$ is the set of matched pairs.

The $PQ$ metric is calculated for each class and is averaged across classes to provide a global performance measure.

The $PQ$ score has a range of $[0, 1.0]$, where a higher score indicates better performance in both detecting and segmenting the instances correctly. A $PQ$ of 1 signifies perfect identification and segmentation of all instances, whereas a $PQ$ of 0 indicates that no instances were correctly identified and segmented.

\clearpage

\subsection{\label{chap:S6}Segmentation and Detection quality metrics for teacher and student models}

\begin{table}[h!]
\renewcommand{\arraystretch}{2.0}
\centering
\caption{Segmentation and detection quality for student and teacher models (CI 95\%)}
\label{tab:S6}
%\adjustbox{max width=\textwidth}{%
\begin{tabular}{|l|c|c|}
\hline
%\rowcolor{gray!30}
Metric & Teacher & Student \\
\hline
$SQ_{neoplastic}$ & 0.819 (0.815--0.823) & 0.824 (0.819--0.828) \\
\hline
$SQ_{lymphocyte}$ & 0.795 (0.788--0.802) & 0.790 (0.783--0.796) \\
\hline
$SQ_{connective}$ & 0.770 (0.762--0.776) & 0.780 (0.772--0.786) \\
\hline
$SQ_{dead}$ & 0.659 (0.623--0.688) & 0.657 (0.624--0.695) \\
\hline
$SQ_{epithelial}$ & 0.780 (0.770--0.790) & 0.788 (0.779--0.797) \\
\hline
$SQ_{macrophage}$ & 0.788 (0.760--0.810) & 0.757 (0.730--0.783) \\
\hline
$SQ_{neutrofil}$ & 0.782 (0.761--0.801) & 0.775 (0.759--0.792) \\
\hline
$DQ_{neoplastic}$ & 0.706 (0.692--0.719) & 0.727 (0.712--0.741) \\
\hline
$DQ_{lymphocyte}$ & 0.675 (0.656--0.698) & 0.713 (0.691--0.734) \\
\hline
$DQ_{connective}$ & 0.566 (0.546--0.584) & 0.583 (0.565--0.602) \\
\hline
$DQ_{dead}$ & 0.410 (0.361--0.465) & 0.435 (0.306--0.561) \\
\hline
$DQ_{epithelial}$ & 0.668 (0.639--0.694) & 0.673 (0.644--0.702) \\
\hline
$DQ_{macrophage}$ & 0.657 (0.583--0.727) & 0.615 (0.531--0.703) \\
\hline
$DQ_{neutrofil}$ & 0.691 (0.625--0.753) & 0.729 (0.679--0.778) \\
\hline
\end{tabular}
%
%}
\end{table}

\clearpage

\subsection{\label{chap:S7}QuPath integration method}
We adopt an integration strategy leveraging the paquo \cite{Bayer_AG} library, a Python package that enables direct interaction with QuPath’s internal API, thereby facilitating seamless data exchange without intermediate conversion steps. The data processing pipeline (\hyperref[fig:S7]{Appendix Figure S7}) begins with the acquisition of WSIs and their associated annotations from QuPath, which are represented as Shapely \cite{Gillies_Wel_etal._2024} polygons. Utilizing paquo, we directly read, create, and modify these annotations and detections within a QuPath project in the Python environment. Images are then cropped using these polygons and processed by cell segmentation and classification models employing standard vision processing toolkits such as OpenCV, pyvips, and PyTorch. Additionally, QuPath employs Groovy scripts to initiate a Python process that starts the entire pipeline from QuPath graphical interface: fetching polygons, extracting images from them, and running deep learning model inference on the cropped images. 
The results are returned to QuPath, leveraging paquo's Python bindings to manipulate QuPath data while minimizing the computational overhead typically associated with cross-environment communication.

\counterwithin{figure}{subsection}
\renewcommand{\thefigure}{S\arabic{subsection}}

\begin{figure}[h!]
    \centering
    \includegraphics[width=\textwidth]{images/A7.pdf}
    \caption{QuPath integration workflow using Python environment}
    \label{fig:S7}
\end{figure}

Compared to traditional workflows that involve exporting annotations as GeoJSON, classifying them in Python, and reimporting them into QuPath, our approach offers several advantages. We eliminate the need to switch between programming languages, providing a cohesive and streamlined development process entirely within QuPath software and removing the necessity to use other tools. Meanwhile, we avoid storing annotations as intermediate JSON files unless required for external use or archiving. By conducting the entire inference and post-processing workflow within the Python environment, we leverage the power and flexibility of Python libraries for image processing and machine learning. This approach also enables adjustments to any set of labels and models, thereby improving its applicability.

%\hfill

The distilled model and QuPath integration code are packaged into a Docker container, enabling streamlined execution with the Docker engine. Detailed integration code and deployment instructions can be found in the GitHub repository \cite{Shvetsov_2025b}.

Despite these benefits, we acknowledge that the paquo library is a proof‑of‑concept project in its early development stage and has not been tested across all versions of QuPath.

\clearpage

\subsection{\label{chap:S8}Data and code availability statement}
All datasets, models, and code used in this study are publicly available and can be obtained from the repositories listed below. 
The PanNuke \cite{Gamper_Koohbanani_etal._2019} and MoNuSAC \cite{Verma_Kumar_etal._2021} datasets are publicly accessible, and download information along with detailed descriptions can be found in their respective articles. Preprocessing scripts for PanNuke and MoNuSAC data, as well as individual cell extraction scripts, are available on GitHub \cite{Shvetsov_2025a}. The H-Optimus foundation model used in our experiments can be downloaded from the HuggingFace repository \cite{hoptimus2024}, and model information is available on GitHub \cite{Saillard_Jenatton_etal._2024}. In addition, the integration code for QuPath and the distilled model packaged in a Docker container are provided in the repository \cite{Shvetsov_2025b}, and paquo Python library is available from the authors GitHub repository \cite{Bayer_AG}.
\clearpage

\end{document}


% \appendix
\appendix

\lstset{
  backgroundcolor=\color{gray!20}, % 20% gray background
  basicstyle=\ttfamily\footnotesize, % Monospaced font with smaller size
  breaklines=true,                 % Automatically break long lines
  frame=single,                    % Frame the listing
  framerule=0pt,                   % No frame border
  xleftmargin=5pt, xrightmargin=5pt % Add some margin around the text
}


\section{Prompt}
In this section, we list the full prompts given to the LLM, which were used in this paper.
\subsection{Prompt to Generate Questions}
\begin{lstlisting}
###Instruction###
You are assisting the audience who has received an email and needs to respond.
You're like a secretary for your audience, asking them questions and creating email responses on their behalf based on their answers.
Your goal is to make it as clear as possible what and how your audience wants to answer in response to all requirements of the email.
Therefore, you must create as specific questions as possible.
Specifically, you will assist your audience in composing emails by following 3 steps:

Step 1: Create Questions:  
    To achieve your goal, you must create well-thought-out questions without omission by considering the sender's intent and requirements. 
    The number and content of the questions must be determined with this in mind.
Step 2: Receive Answers:  
    Ask your audience the questions you created and collect their responses. 
    These will guide the crafting of the reply.
Step 3: Propose a Reply:  
    Based on the answers received, suggest a reply that your audience can edit and send.
    From now on, you will perform step 1.  

You must consider the following 7 matters in generating your response.

1. You must create questions with choices for your audience and output the results in JSON format.
2. The questions must be created in the native language of your audience.
3. If necessary, your audience can write any free answers to your questions, so you will be penalized if you create an "other" option.
4. In 'corresponding_part', you must quote a part of the provided 'Incoming Mail' verbatim. That is, output corresponding_part = IncomingMail_HTML[x:x+h]. 
5. You must quote spaces, `<br>`, periods, and commas exactly as in the provided 'Incoming Mail'. 
6. You will be penalized if you edit or combine multiple parts of the 'Incoming Mail' for your questions.
7. You will be penalized if you create unhelpful questions to compose a reply. You must keep the number of questions to a minimum.

I'm going to tip $100 for a better solution!
Ensure that your output is unbiased and avoids relying on stereotypes.

###Output JSON Format###
{
    "questions": [
        {
            "id": "1",
            "question": "Will you participate in the event on October 24th?", 
            "choices": ["Yes", "No"], 
            "corresponding_part": "We will hold an event on October 24th."
        },
        {
            "id": "2",
            "question": "Please select the available dates (multiple selections possible).", 
            "choices": ["July 10th", "July 11th", "July 12th", "July 13th", "July 14th", "July 15th", "July 16th"], 
            "corresponding_part": "Please let us know your available dates within a week."
        }
    ]
}
\end{lstlisting}

\subsection{Prompt to Generate Reply Draft}
Below is the prompt used to generate a balanced-length description.
\begin{lstlisting}
Please provide a draft reply to the sender of this email on behalf of the user.
\end{lstlisting}

% \section{Proposed LLM-Powered QA-Based Approach: ResQ}
% % 本セクションでは、電子メール対応タスクをサポートするために提案されたアプローチ「ResQ」について説明する
% This section describes the proposed approach, ResQ, for supporting email response tasks. 
% % \begin{figure*}[t]
\centering
\includegraphics[width=\textwidth]{figure/overview_of_process.pdf}
\caption{The overview of the process of creating a reply message using ResQ. A) The LLM first generates multiple-choice questions in JSON format. B) Users select their desired responses to their counterparts. C) The LLM then generates a reply draft in JSON format based on the users' selections. D) Finally, users review and edit the LLM-generated draft before sending the reply.}
\label{fig_overview_of_process}
\Description{This figure illustrates the process of generating an email reply using a large language model (LLM) across four stages. In Stage A (Generate Questions), the system takes the email data and a prompt, which are then sent to the LLM server. The server processes this information and generates a set of questions to clarify the content of the reply. These questions, along with possible answer choices and context, are returned in JSON format. In Stage B (Answer Questions), the user is presented with the questions generated by the LLM. These questions may include simple "Yes" or "No" options or multiple-choice selections. The user answers the questions by choosing the appropriate options or providing custom responses. In Stage C (Generate Reply Draft), the user’s answers, along with the original email data and prompt, are sent back to the LLM server. Based on this input, the server generates a draft of the email reply, which is also returned in JSON format. In Stage D (Check Reply Draft), the user reviews the draft generated by the LLM. After checking the content and making any necessary revisions, the user finalizes and sends the email reply.}
\end{figure*}
% % \begin{figure*}[t]
\centering
\includegraphics[width=\textwidth]{figure/interface.pdf}
\caption{Interface of ResQ. On the left, the content of the email is displayed, with an editor and a ``Reply'' button below for sending a reply. In the center, questions and options for users are shown, allowing the creation of custom options if needed. Additionally, the section of the email corresponding to the selected question is highlighted. On the right, fields are provided to customize the reply generated by the LLM, including options to specify the relationship with the counterpart and buttons to choose the formality, tone, and length of the email. A free-text input field and a "Generate Reply" button are also below.
}
\label{fig_interface}
\Description{This figure illustrates the interface of the ResQ system, which is divided into three main sections. On the left side (A, B), the content of the incoming email is displayed. The email includes important information, such as the sender, subject, and body text. Key sections of the email are highlighted based on the questions generated by the system, helping the user focus on relevant points. Below the email, there is an editor where the user can compose their reply, with a "Reply" button (H) available to send the response once it's ready. In the center section (C, D), the system displays questions generated by the AI, which are intended to assist the user in composing their reply. These questions correspond to specific parts of the email, and as the user answers them, the relevant section in the email is highlighted (A). Users can also customize responses by adding new options if needed (D). user can specify their relationship with the email recipient (e.g., professor or student), adjust the formality and tone of the response, and select the desired length of the reply. An additional free-text input field is available for further customization requests (E). Once all preferences are set, the user can click the "Generate Reply" button (F) to produce a draft response based on their inputs.}
\end{figure*}
% % 図は、ResQのアプローチの全体的な流れについて説明している
% % (A) ユーザが返信作業を開始したことを検知すると、大規模言語モデル (本研究ではGPT-4o) を使用して、多肢選択式の質問を生成する。
% % ユーザは、この質問に対して回答を行うことで、自身の返信方針をAIに伝える
% % (B) 最後に、ユーザが"Generate Reply'' buttonを押したことを検知したら、返信のドラフトをユーザに提示する
% % 図は、実際のResQのinterfaceである
% % 以下のセクションでは、それぞれの手順における具体的な機能について説明する
% Fig.~\ref{fig_overview_of_process} illustrates the overview of the process of a reply message using ResQ.
% % (A) After the system detects users' initiation of the reply task, it generates multiple-choice questions using an LLM (in this study, GPT-4o~\cite{GPT4o}). 
% % (B) Then, users communicate their reply strategy to the AI by responding to these questions.
% % (C) When the system detects that users have pressed the ``Generate Reply'' button, it presents a draft of the reply to users.
% % (D) Finally, users review and revise the reply draft generated by the LLM and send the response.
% Fig.~\ref{fig_interface} shows the actual interface of ResQ.
% The following sections describe the specific functions involved in each step of this process.

% \paragraph{\textbf{A: Generate Questions}}
% \label{sec:generate_questions}
% % ResQは返信の必要性を検知すると、大規模言語モデル (本研究ではGPT-4o) を使用して、多肢選択式の質問を生成する
% % また我々は、ユーザが質問をクリックすると、その質問が受信メールのどの部分に対応しているかがハイライトされるようにした
% % 我々はこれらの機能を実現するために、まずモデルに対して、モデルの役割(ユーザに対する質問と有益そうな選択肢のペアを複数生成すること)と、作成する質問の目的(メールに含まれる全ての要求を抽出し、送信者がそれぞれに対してどのように返答したいかを明確にすること)をプロンプトとして与えた
% % さらに、モデルに受信メールの文章、タイトル、送信者の情報(名前、メールアドレス)、受信メールの過去のやり取りの文章、ユーザの情報(名前、メールアドレス)を提供した
% When the first activates ResQ to reply to an email, the system uses an LLM (in this study, GPT-4o~\cite{GPT4o}) to generate multiple-choice questions (Fig.~\ref{fig_interface}-C). 
% Additionally, when users click on the generated question, the relevant part of the email is highlighted (Fig.~\ref{fig_interface}-A).
% To implement these features, we first provided the LLM with the email's text, subject, sender information, text from past email interactions, and the user's information (name and email address).
% The LLM then extracts all parts that require a user's reply, presents the possible response options, and generates a question for the user. 
% % プロンプトは~\cite{relatedwork}を参考に作成し、文脈を踏まえており、返信を作成する上で役に立ち、適切な数の(メールのすべての要件に対しては漏れなく)質問と、それに役立つ選択肢を生成するように指示した。
% % またプロンプトには質問と選択肢の生成例を含めることで、XXXした。
% \red{We designed the prompt with reference to~\cite{bsharat2023principled}, incorporating contextual considerations to ensure it effectively supports reply composition. 
% We instructed the LLM to generate an appropriate number of questions that comprehensively cover all the email's requirements without omissions, along with relevant response options. 
% To further guide the model and improve output accuracy, we included examples of question and option generation within the prompt.}
% The detailed prompt used for this function is shown in the appendix. 
% The prompt used for this function is shown in the appendix. 

% \paragraph{\textbf{B: Answer Questions}}
% % 次にユーザは受信メールと生成された質問、選択肢を同時に見ながら、質問に回答する
% % 我々は有益な選択肢がない場合を想定して、ユーザ自身が選択肢を追加できるようにした
% % また、LLMにメールのcontextを伝えることができるように、送信者と受信者の関係性を記入できるboxを設置した
% % さらに以前の研究に基づき~\cite{fu2024text}、ユーザがAIメールの文章をカスタマイズできるように、ユーザが期待する返信のトーンやスタイル、長さを調整するための選択肢を提供した
% % また、ユーザがそれ以外のリクエストをAIに対してできるように、AIに対する自由記述欄を設置した
% % ユーザはこれらの作業が完了するとGenerate Replyボタンを押す
% Next, users view the incoming email (Fig.~\ref{fig_interface}-B) alongside the generated questions (Fig.~\ref{fig_interface}-C) and options and proceed to answer them. 
% In anticipation of situations where none of the provided options are useful, we enabled users to add their own options (Fig.~\ref{fig_interface}-D). 
% Additionally, to help the LLM better understand the context of the email, we introduced a box where users can specify the relationship between the sender and the recipient (Fig.~\ref{fig_interface}-E, top). 
% Furthermore, following previous research~\cite{fu2024text}, we provided users with controls to adjust the tone, style, and length of the reply to match their preferences better, thereby giving them more flexibility in customizing the AI-generated response (Fig.~\ref{fig_interface}-E, middle). 
% A free-text field was also included to allow users to make other specific AI requests (Fig.~\ref{fig_interface}-E, bottom). 
% After completing these steps, users can click the ``Generate Reply'' button (Fig.~\ref{fig_interface}-F).

% \paragraph{\textbf{C: Generate Reply Draft}}
% % ResQはユーザが"Generate Reply"ボタンを押したことを検知すると、大規模言語モデルを使用して、返信のドラフトを作成する
% % ユーザの期待するような返信のドラフトが出力されるように、我々はモデルに対して、受信メールとその関連情報、AIの質問とそれに対応するユーザの回答、ユーザが返信案に期待する他の要素(トーン、スタイル、長さ、その他の要望)、ユーザの情報を提供した
% When the user clicks the ``Generate Reply'' button, ResQ detects the action and uses the LLM to generate a reply draft.
% To ensure that the draft aligns with users' expectations, we first provided the LLM with the information provided when Sec.~\ref{sec:generate_questions}, the generated questions, corresponding users' answers, and users' preferences (\textit{e.g.}, tone, style, length, and any additional requests).
% Then, the LLM generates a draft of the reply.
% The prompt used for this function is shown in the appendix.

% \paragraph{\textbf{D: Review Reply Draft}}
% Once the draft reply is generated, users can review the draft in detail (Fig.~\ref{fig_interface}-G).
% Moreover, if users find that extensive revisions are needed or if they want to explore alternative phrasing, they have the option to request the AI to regenerate a new draft based on updated input or preferences.
% After completing these steps, users can click the ``Reply'' button (Fig.~\ref{fig_interface}-H).

% \section{User Comments in Study 1}
% % We added a new subsection, "User Comments," to present follow-up interview results, including participant feedback on useful questions (Sec.6.4). 
% \subsection{Participants' Email-Replying Process (RQ1)}
% \paragraph{Enhanced Efficiency and Reduced Cognitive Load when Replying to Emails (H1-a, H1-b)}
% % 参加者のコメントから、QA-based conditionは期待通り機能し、参加者の効率向上や負担低減に貢献したことが確認できます。
% % 質問で要点をまとめてくれ、メール本文の対応箇所がハイライトされてたので、メールの理解の効率が上がり、負担が減りました [P10]
% % QA-based条件では、Prompt作成の技術がなくても、期待する出力を簡単に得ることができました。[P6]
% % Prompt-based条件では、結局自分で相手のメールを全て読み、回答すべきことを整理する必要がありました [P4]
% % Prompt-based条件では、自分でAIに対する指示を一から考える必要があり、手動の条件と効率や負担に差を感じませんでした。一方でQA-based条件は、圧倒的に早く返信を作成することができました。[P5]
% \red{
% Participants' comments confirmed that the QA-based improved efficiency and reduced workload when replying to emails. 
% P10 explained, \textit{``In the QA-based condition, AI summarized key points through questions and highlighted relevant sections of the email body, which facilitated my understanding of the email and reduced my overall burden''}.
% P6 shared, \textit{``In the QA-based condition, I could easily obtain the desired output even without the technical skills to create prompts''}.
% In contrast, the Prompt-based condition required extra effort. 
% P4 noted, \textit{``In the Prompt-based condition, I had to read the counterpart's email completely and decide what to respond to''}. 
% P5 elaborated, \textit{``In the Prompt-based condition, I had to think of instructions for the AI from scratch, making it feel no different from the No-AI condition in terms of efficiency and workload. On the other hand, the QA-based condition allowed me to compose responses faster''}.
% }

% \paragraph{Reduced Difficulty in Initiating the Action for Replying to Emails (H1-d)}
% % 参加者のコメントから、QA-based conditionでは、全体的な労力が下がるとともに、作業開始時のAIによるの質問が、タスク開始の障壁の低下に役立つことがわかりました。
% % 最初にAIが質問を投げかけてくれるので、作業開始時の思考の労力がなくなり、作業に取り掛かる際のストレスが減りました[P10]
% % 相手の文章を読む時間が省けたことで、作業開始の心理的障壁が下がりました。[P5]
% \red{
% Comments from participants indicated that the QA-based condition helped lower the barriers to task initiation through AI-generated questions at the start of the process. 
% P10 explained, \textit{``Since the AI prompted me with questions at the beginning, the mental effort required to start thinking about the task was eliminated, reducing the stress associated with initiating the work''}. 
% P5 noted, \textit{``By saving the time needed to read the counterpart's text, the psychological barrier to starting the task was lowered''}.
% }

% \paragraph{Decreased Sense of Agency and Control (H1-e)}
% % QA-based conditionでは、agencyやcontrolの感覚が他の条件に比べて低下したと回答した参加者は、次のようにコメントした。
% % 「agencyとcontrolの感覚は、プロンプトを自分で打った量に比例しました。」[P7, 8, 9, 10]
% % 「QA-based条件では、要点も絞ってくれたので、AIに任せようという思いが強くなりました。」[P4]
% % 一方で感覚が変化しなかったと回答した参加者は「AIに任せても、自分で確認と修正を行ったので、agencyやcontrolの感覚に変化はありませんでした」[P3]と説明した。
% \red{
% Participants reported a decrease in their sense of agency and control in the QA-based condition.
% \textit{``The sense of agency and control was proportional to the amount of text I typed myself''} [P7, P8, P9, P10]. 
% \textit{``Under the QA-based condition, since the AI helped narrow down the key points, I felt a stronger inclination to leave the task to the AI''} [P4].
% On the other hand, one participant who reported no change in their sense of agency or control explained, \textit{``Even though I relied on the AI, I reviewed and edited the output myself, so there was no change in my sense of agency or control''} [P3].
% }

% \paragraph{Future Preference (H1-c)}
% % 多くの参加者はメール返信の効率が上がる、負担が減る、質が高いメールを執筆できるという理由から、QA-based conditionで返信をしたいと回答しました。
% % 自分で返信を考えるより、AIを使った方が、質の高い返信を早く作ることができました。特にQA-based conditionではその効果が大きかったので、将来はQA-based conditionで返信したいと思いました。[P6]
% % 一方でsense of agencyの低下や、AIへの依存の危惧を理由に、QA-based conditionでのメール返信を忌避する参加者もいました。
% % 時間がない時や、スピードを重視したい時[P4]、重要度が低い時[P5]は、QA-based conditionで返信をしたいと思ったが、そうでない場合は自分で執筆したいと思いました。
% % 「Prompt-based条件では、QA-based条件より思考する必要が多く、それが楽しかったです」 [P7]
% % 返信作業が楽にはなりましたが、メールを細部まで読まなくなり、内容が頭に入っていない感じがしたので、将来使いたいとは思いませんでした。[P11]
% \red{
% Participants expressed a preference for using the QA-based condition for email responses, citing increased efficiency, reduced workload, and the ability to produce high-quality emails as the primary reasons. 
% One participant explained, \textit{``Using AI allowed me to compose high-quality responses faster than if I had written them myself. The effect was particularly significant in the QA-based condition, which is why I would prefer to use it in the future''} [P6].
% }
% \red{However, some participants were hesitant to adopt the QA-based condition due to concerns about a reduced sense of agency or over-reliance on AI. 
% Participants noted that they preferred the QA-based condition \textit{``when time is limited or speed is important''} [P4] or \textit{when the email is of low importance''} [P5], but in other situations, they favored writing responses themselves.
% One participant reflected, \textit{``In the prompt-based condition, I found that I needed to think more actively compared to the QA-based condition, and I enjoyed that process''} [P7].
% Another observed, \textit{``While the QA-based condition made responding easier, I felt that I was no longer fully reading and absorbing the content of emails, which made me hesitant to use it in the future''} [P11].
% }

% \paragraph{Quality of AI-generated Questions and Options}
% % 参加者はQA-based conditionにおいて生成された質問や選択肢について、有益であったものとそうでなかったものがあったとコメントした。
% % 参加者は有益でない質問の例として、メール送信者の意図を汲み取れていないもの [P2, P11]、自分と相手の立場を勘違いしているもの [P4, P8]を挙げた。
% % またある参加者は、「自分が言いたいことに関連する質問がない場合、質問機能自体が役に立たなかった」[P8]と回答した。
% % また参加者は、質問の数についても意見を述べた
% % 参加者は「用件ごとに質問を生成してくれたのが、メールを理解する上で役に立った」[P4, P5, P8, P9, P10, P11]と回答した一方で、「質問が多すぎると煩雑に感じることもあった。またそれに全て答えると、返信が冗長になってしまった。」[P7, P12]と回答する参加者もいた。
% % また、参加者は選択肢についても意見を述べた
% % 参加者は、「(日程調整のシチュエーションにおいて)選択肢の中に自分が選びたい日程がなかったので、自分で日程を入力する必要があった」[P2]と説明し「より多くの選択肢を生成して欲しかった」と説明した[P8]。
% % 一方である参加者は、「必要以上に多くの選択肢があったときに、煩わしさを感じた」[P4]と説明した。
% \red{
% Participants commented on the questions and options generated by the QA-based condition, noting that some were useful while others were not. 
% Examples of less useful questions included those that failed to accurately capture the sender's intent [P2, P11] or misinterpreted the relationship between the sender and recipient [P4, P8]. 
% Additionally, one participant remarked, \textit{``when there were no questions related to what I wanted to say, the question feature itself was not helpful''} [P8].
% }

% \red{
% Participants also shared mixed opinions on the number of questions generated. 
% Some participants noted that \textit{``generating questions for each topic helped understand the email''} [P4, P5, P8, P9, P10, P11].
% However, others felt \textit{``an excessive number of questions felt overwhelming, and responding to all of them made the reply unnecessarily lengthy''} [P7, P12].
% }

% \red{
% Feedback on the generated options was similarly divided. 
% For instance, in scheduling scenarios, one participant shared that \textit{``none of the suggested dates matched what I wanted, so I had to input the date myself''} [P2] and another participant \textit{``wished for more options or a more flexible input type''} [P8].
% In contrast, another participant stated that \textit{``having more options than necessary felt burdensome''} [P4].
% }

% \subsection{Quality of the Email Responses (RQ2)}
% % ほとんどの参加者は、AIを使うと、構造・丁寧さ・言葉遣いが改善され、全体的に良い文章を書けたと述べた
% % また参加者は、「Prompt-based条件だと、相手の要求を見落としていたかもしれないが、QA-based条件では自信を持って返信を作成することができた」 [P2, 4]と述べた
% % さらにある参加者は、「QA-based条件では、回答してもしなくても良いこと「XXの件、承知しました、など。」にも丁寧に返答を書いてくれた」 [P9]と述べ、QA-based条件によってメールの丁寧さが向上したことを強調した
% \paragraph{Scaffolding a structured response}
% \red{
% Most participants stated that using AI improved their writing structure, politeness, and choice of words, ultimately enabling them to produce better overall responses. 
% Furthermore, participants remarked, \textit{``Under the prompt-based condition, I might have overlooked the recipient's requests, but under the QA-based condition, I was able to craft responses with confidence''} [P2, P4]. 
% Additionally, one participant emphasized that \textit{``Under the QA-based condition, the AI even provided polite responses to matters where a reply was optional, such as acknowledging something with phrases like 'I Understood regarding XX, etc.'''} [P9], highlighting how the QA-based condition enhanced the politeness of email communication.
% }

% \subsection{Relationship between Participants and Their Counterpart (RQ3)}
% % 参加者は、``相手との間に知覚する心理的距離は労力に比例した''と報告し、PXXは``特にQA-based条件では選択肢を選ぶだけだった相手のことを考えることが少なかった''と報告した。
% % 一方でPXXは、``自分で返信を考えるより、AIを使うと相手に良い印象を与えられるメッセージを作ることができたので、関係性を近く感じた''と報告した
% \red{
% Participants shared differing views on how AI's involvement affected their psychological distance from their counterparts.
% Several participants reported that the psychological distance they felt from the other person was directly related to the amount of effort they put in [P2, P9, P11].
% % \textit{``the psychological distance I perceived from the counterpart was proportional to the effort exerted''} [P2, P9, P11].
% Furthermore, P6 noted that \textit{``especially under QA-based condition, I barely thought about the counterpart because I only selected options to create responses''}.
% In contrast, P8 reported that \textit{``compared to composing replies myself, using AI allowed me to create messages that left a better impression on my counterpart, which made the relationship feel closer''}.
% }
% % \subsubsection{Psychological Distance between Participants and Their Counterpart (H3-b)}

\section{\blue{Order Effect in Study 1}}
\blue{We conducted analyses to examine whether the order of conditions influenced various dependent variables. 
Table~\ref{tab_ordereffects} summarizes the results of the order effect and its interaction with the condition. 
Depending on the nature of the data, we employed either a Mixed-Design ANOVA or the Aligned Rank Transform (ART) method.}

\begin{table*}[t]
\caption{Order effects in Study 1 (* indicates significance at the 0.05 level).}
\Description{The table presents the order effects observed in Study 1, examining whether the order of conditions influenced various measurements. It reports p-values for the main effect of Order and the interaction effect between Condition and Order, with asterisks indicating statistical significance at the 0.05 level. The statistical method used for each measurement is also specified, including Mixed-Design ANOVA and Mixed-Design ANOVA with Aligned Rank Transform (ART). For Efficiency of Replying to Emails, the Order effect has a p-value of 0.643, and the Condition × Order interaction has a p-value of 0.454, analyzed using Mixed-Design ANOVA. Prompt Character Count shows an Order effect p-value of 0.052, close to significance, while the Condition × Order interaction has a p-value of 0.186, also analyzed using Mixed-Design ANOVA. Raw TLX, which measures subjective workload, has a significant Order effect (p < 0.05), indicating that workload perception is influenced by order, whereas its Condition × Order interaction is non-significant (p = 0.978), analyzed using Mixed-Design ANOVA. For Difficulty in Understanding Email Content, the Order effect p-value is 0.188, and the Condition × Order interaction p-value is 0.232, analyzed using Mixed-Design ANOVA with ART. Satisfaction with Completing Tasks shows non-significant effects, with an Order effect p-value of 0.348 and a Condition × Order interaction p-value of 0.175, analyzed using Mixed-Design ANOVA. Similarly, Difficulty for Task Initiation has an Order effect p-value of 0.131 and a Condition × Order interaction p-value of 0.515, analyzed using Mixed-Design ANOVA with ART. Psychological perceptions are also analyzed. Sense of Agency shows non-significant results, with an Order effect p-value of 0.982 and a Condition × Order interaction p-value of 0.911, analyzed using Mixed-Design ANOVA with ART. Sense of Control similarly has an Order effect p-value of 0.825 and a Condition × Order interaction p-value of 0.871, analyzed using Mixed-Design ANOVA with ART. Regarding Perceived Quality of the Email, the Order effect is 0.930, and the Condition × Order interaction is 0.433, analyzed using Mixed-Design ANOVA. Perceived Impression of Participants also shows non-significant effects, with an Order effect p-value of 0.963 and a Condition × Order interaction p-value of 0.481, analyzed using Mixed-Design ANOVA. Finally, Psychological Distance presents an Order effect p-value of 1.000, indicating no effect of order, but its Condition × Order interaction is significant (p < 0.05), suggesting that condition effects on psychological distance are influenced by order. This measurement is analyzed using Mixed-Design ANOVA with ART.}
\label{tab_ordereffects}
\blue{
\begin{tabular}{cccc}
\hline
Measurements                              & Order (p-value) & Condition $\times$ Order (p-value) & Statistical Method          \\ \hline
Efficiency of Replying to Emails          & 0.643            & 0.454                        & Mixed-Design ANOVA          \\
Prompt Character Count                    & 0.052           & 0.186                       & Mixed-Design ANOVA          \\
Raw TLX                                   & $<0.05$*         & 0.978                       & Mixed-Design ANOVA          \\
Difficulty in Understanding Email Content & 0.188           & 0.232                       & Mixed-Design ANOVA with ART \\
Satisfaction with Completing Tasks        & 0.348           & 0.175                       & Mixed-Design ANOVA          \\
Difficulty for Task Initiation            & 0.131           & 0.515                       & Mixed-Design ANOVA with ART \\
Sense of Agency                           & 0.982           & 0.911                       & Mixed-Design ANOVA with ART \\
Sense of Control                          & 0.825           & 0.871                       & Mixed-Design ANOVA with ART \\
Perceived Quality of the Email            & 0.93            & 0.433                       & Mixed-Design ANOVA          \\
Perceived Impression of Participants      & 0.963           & 0.481                       & Mixed-Design ANOVA          \\
Psychological Distance                    & 1.000           & $<0.05$*                     & Mixed-Design ANOVA with ART \\ \hline
\end{tabular}
}
\end{table*}

\blue{The results indicate that the order effect was not significant for most dependent variables. 
However, a significant effect was observed for Raw TLX ($p = 0.041$), suggesting that task load perception may have been influenced by the presentation order. 
Additionally, a significant interaction effect between condition and order was found for IOS ($p = 0.043$), indicating that the order of presentation might have impacted this specific measure.}

\section{Future Preference in Study 1}
\begin{figure*}[ht]
\centering
\includegraphics[width=\textwidth]{figure/study1_pref.pdf}
\caption{Participants' future preferences in Study 1. Significant differences between conditions were identified through post-hoc analysis following the Friedman test (* indicates significance at the 0.05 level).}
\label{fig_study1_preference}
\Description{The figure displays participants' future preferences using box plots for three experimental conditions: No-AI, Prompt-based, and QA-based. For the No-AI condition, the median overlap score is 3.5, with the first quartile at 1.75 and the third quartile at 4.25. The scores range from a minimum of 1 to a maximum of 7. For the Prompt-based condition, the median overlap score is 6, with the first quartile at 3 and the third quartile at 7. The scores range from 2 to 7. For the QA-based condition, the median overlap score is 6, with the first quartile at 6 and the third quartile at 7. The scores range from 3 to 7. Significant differences were observed between the No-AI condition and both the Prompt-based and QA-based conditions (p < 0.05).}
\end{figure*}
\red{We evaluated participants' preferences for future use across all conditions using a 7-point Likert scale.
Participants rated their agreement with the statement, ``I would prefer to use this approach for replying to emails in the future,'' where 1 indicates strongly disagree, 4 indicates neutral, and 7 indicates strongly agree.}

\red{The questionnaire survey results about participants' future preferences are shown in Fig.~\ref{fig_study1_preference}.
According to the Friedman test, a significant difference in participants' future preferences was observed among the three conditions $(\chi^2(2)=8.8, p=0.012, W=0.37)$.
Post-hoc analysis using the Durbin-Conover test with Holm correction revealed that participants would prefer responding in the QA-based condition compared to the No-AI condition $(p=0.012, r=0.67)$.
However, no significant difference was found between the Prompt-based and QA-based conditions $(p=0.800, r=0.27)$.}

\section{Technical Details of ResQ}
\begin{figure*}[ht]
\centering
\includegraphics[width=\textwidth]{figure/UI.pdf}
\caption{UI of the Gmail Reply Box with the ``Reply with AI’’ Feature, used in Study 2. Pressing the ``Reply with AI’’ button opens the window shown in Fig.~\ref{fig_interface}}
\label{fig_UI}
\Description{This figure illustrates the user interface of the Gmail reply box as enhanced by the prototype system. The Reply with AI button, shown in blue on the right-hand side of the toolbar, allows users to activate the AI-assisted reply generation feature. When the button is clicked, the system extracts the email content and opens the window shown in Fig.~\ref{fig_interface}. The standard Gmail toolbar options, such as send, formatting, and attachment icons, remain.}
\end{figure*}
\red{In this section, we provide the specific implementation details and user interface used in Study 2.
We developed a prototype system consisting of a Chrome extension and a backend service to enable participants to reply to emails using Gmail on a PC.
The Chrome extension detected the initiation of the reply task when participants clicked the ``Reply with AI'' button in the Gmail reply box (see Fig.~\ref{fig_UI}).
Upon clicking the button, the extension extracted the email content directly from Gmail’s DOM structure using JavaScript and sent it to a backend API endpoint implemented with FastAPI~\footnote{\url{https://fastapi.tiangolo.com}}.
The backend, hosted on an AWS EC2 instance~\footnote{\url{https://aws.amazon.com/ec2/}}, received the email content and forwarded it to the OpenAI API~\footnote{\url{https://platform.openai.com/docs/}} to generate questions or reply suggestions. 
These outputs were then returned to the Chrome extension and displayed to participant in a new reply editor.
Finally, participants revise the reply suggestions and submit them back to the Gmail reply box by clicking the ``Reply'' button.
To ensure privacy, neither the email content nor the participants' responses were accessible to the experimenters or stored on the server.}

\red{Additionally, to implement ResQ's features for generating questions and options, we provided the LLM with various contextual inputs, including the email text, subject, sender information, text from prior email interactions, and the user's details (such as name and email address). 
Furthermore, to ensure that the generated draft aligned with user expectations, the LLM was further given information outlined in Sec.~\ref{sec:generate_questions}, including the generated questions, corresponding user answers, and user preferences (\textit{e.g.}, tone, style, length, and any specific requests). 
Based on this input, the LLM produced a draft of the email reply.}

\end{document}