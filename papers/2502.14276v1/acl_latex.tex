% This must be in the first 5 lines to tell arXiv to use pdfLaTeX, which is strongly recommended.
\pdfoutput=1
% In particular, the hyperref package requires pdfLaTeX in order to break URLs across lines.

\documentclass[11pt]{article}

% Change "review" to "final" to generate the final (sometimes called camera-ready) version.
% Change to "preprint" to generate a non-anonymous version with page numbers.
\usepackage[preprint]{acl}

% Standard package includes
\usepackage{times}
\usepackage{latexsym}

% For proper rendering and hyphenation of words containing Latin characters (including in bib files)
\usepackage[T1]{fontenc}
% For Vietnamese characters
% \usepackage[T5]{fontenc}
% See https://www.latex-project.org/help/documentation/encguide.pdf for other character sets

% This assumes your files are encoded as UTF8
\usepackage[utf8]{inputenc}

% This is not strictly necessary, and may be commented out,
% but it will improve the layout of the manuscript,
% and will typically save some space.
\usepackage{microtype}

% This is also not strictly necessary, and may be commented out.
% However, it will improve the aesthetics of text in
% the typewriter font.
\usepackage{inconsolata}

%Including images in your LaTeX document requires adding
%additional package(s)
\usepackage{booktabs}
\usepackage{multirow}
\usepackage{multicol}
\usepackage{makecell}
\usepackage{enumitem}
\usepackage{balance}
\usepackage{graphicx}
\usepackage{subfigure}
\usepackage{amsmath}
\usepackage{amsthm}
\usepackage{amssymb}
\usepackage{amsfonts}
\usepackage{hyperref}
\usepackage{enumitem}
\usepackage{xparse}
\usepackage[most]{tcolorbox}
\usepackage{xspace}

\usepackage{soul}    % 提供 \hl 和 \sethlcolor
\usepackage{xcolor}  % 提供颜色支持
\usepackage{colortbl}
\usepackage{booktabs}

\definecolor{mintleaf}{RGB}{0, 184, 148}
\definecolor{dm-blue-500}{RGB}{0, 69, 177}
\definecolor{dm-purple-500}{RGB}{105,50,230}
\definecolor{mysilver}{RGB}{128,129,128}
\definecolor{my_green}{RGB}{0, 176, 80}
\definecolor{my_yellow}{RGB}{255,165,0}
\definecolor{my_red}{RGB}{255, 0, 0}
\definecolor{my_purple}{RGB}{126, 100, 158}
\definecolor{my_blue}{RGB}{49, 133, 155}
\definecolor{case_purple}{RGB}{160, 43, 147}
\definecolor{case_blue}{RGB}{15, 158, 213}

% Define a command to bold the first letter of a word
\newcommand{\boldfirst}[1]{%
    \textbf{\MakeUppercase{#1}}%
    \unskip
}

% Define our method
\newcommand{\model}{STeCa\xspace}

% for comments
\newcommand{\wj}[1]{\textcolor{red}{[#1 --WJ]}} 

% Define a custom deep green color
\definecolor{deepgreen}{RGB}{0,100,0}
% Define a custom deep red color
\definecolor{deepred}{RGB}{139,0,0}

% If the title and author information does not fit in the area allocated, uncomment the following
%
%\setlength\titlebox{<dim>}
%
% and set <dim> to something 5cm or larger.


% 
\title{STeCa: Step-level Trajectory Calibration for LLM Agent Learning}




\author{
Hanlin Wang, ~~Jian Wang$^{\dagger}$, ~~Chak Tou Leong, ~~Wenjie Li \\
Department of Computing, The Hong Kong Polytechnic University \\
\texttt{\{hanlin-henry.wang,chak-tou.leong\}@connect.polyu.hk} \\
\texttt{jian51.wang@polyu.edu.hk} ~~
\texttt{cswjli@comp.polyu.edu.hk}
}


\begin{document}
\maketitle

% self-define the footnote symbol
\renewcommand{\thefootnote}{$\dagger$}
\footnotetext[1]{Corresponding author.}
% reset footnote counter
\setcounter{footnote}{0}
\renewcommand{\thefootnote}{\arabic{footnote}}

\begin{abstract}

Large language model (LLM)-based agents have shown promise in tackling complex tasks by interacting dynamically with the environment. 
Existing work primarily focuses on behavior cloning from expert demonstrations and preference learning through exploratory trajectory sampling. However, these methods often struggle in long-horizon tasks, where suboptimal actions accumulate step by step, causing agents to deviate from correct task trajectories.
To address this, we highlight the importance of \textit{timely calibration} and the need to automatically construct calibration trajectories for training agents. We propose \textbf{S}tep-Level \textbf{T}raj\textbf{e}ctory \textbf{Ca}libration (\textbf{\model}), a novel framework for LLM agent learning. 
Specifically, \model identifies suboptimal actions through a step-level reward comparison during exploration. It constructs calibrated trajectories using LLM-driven reflection, enabling agents to learn from improved decision-making processes. These calibrated trajectories, together with successful trajectory data, are utilized for reinforced training.
Extensive experiments demonstrate that \model significantly outperforms existing methods. Further analysis highlights that step-level calibration enables agents to complete tasks with greater robustness. 
Our code and data are available at \url{https://github.com/WangHanLinHenry/STeCa}.

\end{abstract}



\section{Introduction}

Despite the remarkable capabilities of large language models (LLMs)~\cite{DBLP:conf/emnlp/QinZ0CYY23,DBLP:journals/corr/abs-2307-09288}, they often inevitably exhibit hallucinations due to incorrect or outdated knowledge embedded in their parameters~\cite{DBLP:journals/corr/abs-2309-01219, DBLP:journals/corr/abs-2302-12813, DBLP:journals/csur/JiLFYSXIBMF23}.
Given the significant time and expense required to retrain LLMs, there has been growing interest in \emph{model editing} (a.k.a., \emph{knowledge editing})~\cite{DBLP:conf/iclr/SinitsinPPPB20, DBLP:journals/corr/abs-2012-00363, DBLP:conf/acl/DaiDHSCW22, DBLP:conf/icml/MitchellLBMF22, DBLP:conf/nips/MengBAB22, DBLP:conf/iclr/MengSABB23, DBLP:conf/emnlp/YaoWT0LDC023, DBLP:conf/emnlp/ZhongWMPC23, DBLP:conf/icml/MaL0G24, DBLP:journals/corr/abs-2401-04700}, 
which aims to update the knowledge of LLMs cost-effectively.
Some existing methods of model editing achieve this by modifying model parameters, which can be generally divided into two categories~\cite{DBLP:journals/corr/abs-2308-07269, DBLP:conf/emnlp/YaoWT0LDC023}.
Specifically, one type is based on \emph{Meta-Learning}~\cite{DBLP:conf/emnlp/CaoAT21, DBLP:conf/acl/DaiDHSCW22}, while the other is based on \emph{Locate-then-Edit}~\cite{DBLP:conf/acl/DaiDHSCW22, DBLP:conf/nips/MengBAB22, DBLP:conf/iclr/MengSABB23}. This paper primarily focuses on the latter.

\begin{figure}[t]
  \centering
  \includegraphics[width=0.48\textwidth]{figures/demonstration.pdf}
  \vspace{-4mm}
  \caption{(a) Comparison of regular model editing and EAC. EAC compresses the editing information into the dimensions where the editing anchors are located. Here, we utilize the gradients generated during training and the magnitude of the updated knowledge vector to identify anchors. (b) Comparison of general downstream task performance before editing, after regular editing, and after constrained editing by EAC.}
  \vspace{-3mm}
  \label{demo}
\end{figure}

\emph{Sequential} model editing~\cite{DBLP:conf/emnlp/YaoWT0LDC023} can expedite the continual learning of LLMs where a series of consecutive edits are conducted.
This is very important in real-world scenarios because new knowledge continually appears, requiring the model to retain previous knowledge while conducting new edits. 
Some studies have experimentally revealed that in sequential editing, existing methods lead to a decrease in the general abilities of the model across downstream tasks~\cite{DBLP:journals/corr/abs-2401-04700, DBLP:conf/acl/GuptaRA24, DBLP:conf/acl/Yang0MLYC24, DBLP:conf/acl/HuC00024}. 
Besides, \citet{ma2024perturbation} have performed a theoretical analysis to elucidate the bottleneck of the general abilities during sequential editing.
However, previous work has not introduced an effective method that maintains editing performance while preserving general abilities in sequential editing.
This impacts model scalability and presents major challenges for continuous learning in LLMs.

In this paper, a statistical analysis is first conducted to help understand how the model is affected during sequential editing using two popular editing methods, including ROME~\cite{DBLP:conf/nips/MengBAB22} and MEMIT~\cite{DBLP:conf/iclr/MengSABB23}.
Matrix norms, particularly the L1 norm, have been shown to be effective indicators of matrix properties such as sparsity, stability, and conditioning, as evidenced by several theoretical works~\cite{kahan2013tutorial}. In our analysis of matrix norms, we observe significant deviations in the parameter matrix after sequential editing.
Besides, the semantic differences between the facts before and after editing are also visualized, and we find that the differences become larger as the deviation of the parameter matrix after editing increases.
Therefore, we assume that each edit during sequential editing not only updates the editing fact as expected but also unintentionally introduces non-trivial noise that can cause the edited model to deviate from its original semantics space.
Furthermore, the accumulation of non-trivial noise can amplify the negative impact on the general abilities of LLMs.

Inspired by these findings, a framework termed \textbf{E}diting \textbf{A}nchor \textbf{C}ompression (EAC) is proposed to constrain the deviation of the parameter matrix during sequential editing by reducing the norm of the update matrix at each step. 
As shown in Figure~\ref{demo}, EAC first selects a subset of dimension with a high product of gradient and magnitude values, namely editing anchors, that are considered crucial for encoding the new relation through a weighted gradient saliency map.
Retraining is then performed on the dimensions where these important editing anchors are located, effectively compressing the editing information.
By compressing information only in certain dimensions and leaving other dimensions unmodified, the deviation of the parameter matrix after editing is constrained. 
To further regulate changes in the L1 norm of the edited matrix to constrain the deviation, we incorporate a scored elastic net ~\cite{zou2005regularization} into the retraining process, optimizing the previously selected editing anchors.

To validate the effectiveness of the proposed EAC, experiments of applying EAC to \textbf{two popular editing methods} including ROME and MEMIT are conducted.
In addition, \textbf{three LLMs of varying sizes} including GPT2-XL~\cite{radford2019language}, LLaMA-3 (8B)~\cite{llama3} and LLaMA-2 (13B)~\cite{DBLP:journals/corr/abs-2307-09288} and \textbf{four representative tasks} including 
natural language inference~\cite{DBLP:conf/mlcw/DaganGM05}, 
summarization~\cite{gliwa-etal-2019-samsum},
open-domain question-answering~\cite{DBLP:journals/tacl/KwiatkowskiPRCP19},  
and sentiment analysis~\cite{DBLP:conf/emnlp/SocherPWCMNP13} are selected to extensively demonstrate the impact of model editing on the general abilities of LLMs. 
Experimental results demonstrate that in sequential editing, EAC can effectively preserve over 70\% of the general abilities of the model across downstream tasks and better retain the edited knowledge.

In summary, our contributions to this paper are three-fold:
(1) This paper statistically elucidates how deviations in the parameter matrix after editing are responsible for the decreased general abilities of the model across downstream tasks after sequential editing.
(2) A framework termed EAC is proposed, which ultimately aims to constrain the deviation of the parameter matrix after editing by compressing the editing information into editing anchors. 
(3) It is discovered that on models like GPT2-XL and LLaMA-3 (8B), EAC significantly preserves over 70\% of the general abilities across downstream tasks and retains the edited knowledge better.
%!TEX root = gcn.tex
\section{Preliminary}

\paragraph{Notations.}
Table~\ref{tab:notation} summarizes the main notations.
$\Abb$ denotes an Abelian group.
$\Sbb_n$ denotes a permutation group of $n$ elements.
$\Mbb_{m,n}(\Rcal)$ denotes a matrix ring, which defines a set of $m\times n$ matrices with entries in a ring $\Rcal$, forming a ring under matrix addition and
%matrix 
multiplication.
$\Msf_{m \times n}=(\Msf[i,j])_{i,j=1}^{m,n}$ denotes an $m\times n$ matrix\footnote{
For simplicity, we omit the subscript of 
$\Msf_{m \times n}$ when the values of $m$ and $n$ are clear from the context.
Also, we write 
$\Msf = (\Msf[i, j])_{i, j = 1}^{n}$ if $m = n$.
} 
where row indices are $\{1, 2, \ldots, m\}$ and column indices are $\{1, 2, \ldots, n\}$, and $\Msf[i,j]$ is the value at the $i$-th row and $j$-th column.
$\Pi(\ ;\ )$ denotes a protocol execution between two parties, $\pp_0$ and $\pp_1$, where $\pp_0$'s inputs are the left part of `;' and $\pp_1$'s inputs are the right part of `;'.

\paragraph{Secret Sharing.} 
We use $2$-out-of-$2$ additive secret sharing over a ring, where
the floating-point
%input 
values are encoded to be fixed-point numbers $x\in \Zbb/2^f\Zbb$, with $L = 64$ bits representing decimals and $f = 18$ bits representing the fractional part~\cite{sp/MohasselZ17}.
Specifically, one party $\pp_0$ holds the share $\l x\r _0\in\Zbb $, while the other party $\pp_1$ holds the share $\l x\r_1\in\Zbb$ such that $x\cdot 2^f = \l x\r_0 + \l x\r_1$.
The shares can be arithmetic or binary.


%$e$ &The degree of the permutation group
%$\Abb$& Abelian group

\paragraph{Graph Convolutional Networks.}
GCN~\cite{iclr/KipfW17} has been proposed for training over graph data, using graph structure and node features as input.
Like most neural networks, GCN consists of multiple linear and non-linear layers.
Compared to CNN, GCN replaces convolutional layers with graph convolution layers
%to learn the graph topology 
(more details on GCN architecture and its training/inference are in Section~\ref{sec::secgcn}).
Graph convolution can be computed by SMM, often yielding many $0$-value results.

Let $\adjmat \in \Mbb_{\numnode,\numnode}(\Rcal)$ be a (normalized) adjacency matrix of a graph with $\numnode$ nodes and $\Xsf \in \Mbb_{\numnode, d}(\Rcal)$ be the feature matrix (with dimensionality $d$) of the nodes.
The graph convolution layer (with output dimensionality $k$) is defined as $\mathsf{Y} = \adjmat \feamat \weimat$,
where $\mathsf{Y} \in \Mbb_{\numnode, k}(\Rcal)$ is the output and $\weimat \in \Mbb_{d, k}(\Rcal)$ is a trainable parameter.
As matrix multiplication costs increase linearly with input size and $k \ll \numnode$ in practice, the challenge of secure GCN
%training 
lies in the
%secure 
SMM of $\adjmat \feamat$.
Multiplying (dense) $\weimat$ can be done using Beaver's approach.

\begin{table}[!t]
\centering
\caption{Notation and Definition}
\label{tab:notation}
\setlength\tabcolsep{2pt}
\begin{tabular}{l|l}
\hline
%\textbf{Notation} & \textbf{Definition}
%\\\hline
$\pp_i, \l \ \r_i$& Party $i$ and its share ($i \in \{0, 1\})$
\\\hline
$\bitlen$ & The bit-length of data
\\\hline
$\pi,b,\uvec,\l u \r$ & Pre-computed randomnesses
%$\pi,b,\uvec,u$ & \multirow{2}{*}{Offline-generated randomness (\eg., vectors, values)}
\\\hline
$\delta_x$ & Masked version of value/vector $x$
\\\hline
$\Mbb_{m, n}{(\Rcal)}$ & A set of ${m\times n}$ matrices with entries in a ring $\Rcal$
\\\hline
$\Msf_{m,n}$ & A matrix $\Msf$ of size $m\times n$
\\\hline
$\Msf[i, j]$ & The value of $\Msf$ at the $i$-th row and $j$-th column
\\\hline
$\sigma\xvec$& Permutation operation $\sigma$ over a matrix/vector $\xvec$
\\\hline
$\Sbb_n$ & Permutation group of $n$ elements
\\\hline
\end{tabular}
\end{table}

\vspace{-5pt}
\section{Method}
\label{sec:method}
\section{Overview}

\revision{In this section, we first explain the foundational concept of Hausdorff distance-based penetration depth algorithms, which are essential for understanding our method (Sec.~\ref{sec:preliminary}).
We then provide a brief overview of our proposed RT-based penetration depth algorithm (Sec.~\ref{subsec:algo_overview}).}



\section{Preliminaries }
\label{sec:Preliminaries}

% Before we introduce our method, we first overview the important basics of 3D dynamic human modeling with Gaussian splatting. Then, we discuss the diffusion-based 3d generation techniques, and how they can be applied to human modeling.
% \ZY{I stopp here. TBC.}
% \subsection{Dynamic human modeling with Gaussian splatting}
\subsection{3D Gaussian Splatting}
3D Gaussian splatting~\cite{kerbl3Dgaussians} is an explicit scene representation that allows high-quality real-time rendering. The given scene is represented by a set of static 3D Gaussians, which are parameterized as follows: Gaussian center $x\in {\mathbb{R}^3}$, color $c\in {\mathbb{R}^3}$, opacity $\alpha\in {\mathbb{R}}$, spatial rotation in the form of quaternion $q\in {\mathbb{R}^4}$, and scaling factor $s\in {\mathbb{R}^3}$. Given these properties, the rendering process is represented as:
\begin{equation}
  I = Splatting(x, c, s, \alpha, q, r),
  \label{eq:splattingGA}
\end{equation}
where $I$ is the rendered image, $r$ is a set of query rays crossing the scene, and $Splatting(\cdot)$ is a differentiable rendering process. We refer readers to Kerbl et al.'s paper~\cite{kerbl3Dgaussians} for the details of Gaussian splatting. 



% \ZY{I would suggest move this part to the method part.}
% GaissianAvatar is a dynamic human generation model based on Gaussian splitting. Given a sequence of RGB images, this method utilizes fitted SMPLs and sampled points on its surface to obtain a pose-dependent feature map by a pose encoder. The pose-dependent features and a geometry feature are fed in a Gaussian decoder, which is employed to establish a functional mapping from the underlying geometry of the human form to diverse attributes of 3D Gaussians on the canonical surfaces. The parameter prediction process is articulated as follows:
% \begin{equation}
%   (\Delta x,c,s)=G_{\theta}(S+P),
%   \label{eq:gaussiandecoder}
% \end{equation}
%  where $G_{\theta}$ represents the Gaussian decoder, and $(S+P)$ is the multiplication of geometry feature S and pose feature P. Instead of optimizing all attributes of Gaussian, this decoder predicts 3D positional offset $\Delta{x} \in {\mathbb{R}^3}$, color $c\in\mathbb{R}^3$, and 3D scaling factor $ s\in\mathbb{R}^3$. To enhance geometry reconstruction accuracy, the opacity $\alpha$ and 3D rotation $q$ are set to fixed values of $1$ and $(1,0,0,0)$ respectively.
 
%  To render the canonical avatar in observation space, we seamlessly combine the Linear Blend Skinning function with the Gaussian Splatting~\cite{kerbl3Dgaussians} rendering process: 
% \begin{equation}
%   I_{\theta}=Splatting(x_o,Q,d),
%   \label{eq:splatting}
% \end{equation}
% \begin{equation}
%   x_o = T_{lbs}(x_c,p,w),
%   \label{eq:LBS}
% \end{equation}
% where $I_{\theta}$ represents the final rendered image, and the canonical Gaussian position $x_c$ is the sum of the initial position $x$ and the predicted offset $\Delta x$. The LBS function $T_{lbs}$ applies the SMPL skeleton pose $p$ and blending weights $w$ to deform $x_c$ into observation space as $x_o$. $Q$ denotes the remaining attributes of the Gaussians. With the rendering process, they can now reposition these canonical 3D Gaussians into the observation space.



\subsection{Score Distillation Sampling}
Score Distillation Sampling (SDS)~\cite{poole2022dreamfusion} builds a bridge between diffusion models and 3D representations. In SDS, the noised input is denoised in one time-step, and the difference between added noise and predicted noise is considered SDS loss, expressed as:

% \begin{equation}
%   \mathcal{L}_{SDS}(I_{\Phi}) \triangleq E_{t,\epsilon}[w(t)(\epsilon_{\phi}(z_t,y,t)-\epsilon)\frac{\partial I_{\Phi}}{\partial\Phi}],
%   \label{eq:SDSObserv}
% \end{equation}
\begin{equation}
    \mathcal{L}_{\text{SDS}}(I_{\Phi}) \triangleq \mathbb{E}_{t,\epsilon} \left[ w(t) \left( \epsilon_{\phi}(z_t, y, t) - \epsilon \right) \frac{\partial I_{\Phi}}{\partial \Phi} \right],
  \label{eq:SDSObservGA}
\end{equation}
where the input $I_{\Phi}$ represents a rendered image from a 3D representation, such as 3D Gaussians, with optimizable parameters $\Phi$. $\epsilon_{\phi}$ corresponds to the predicted noise of diffusion networks, which is produced by incorporating the noise image $z_t$ as input and conditioning it with a text or image $y$ at timestep $t$. The noise image $z_t$ is derived by introducing noise $\epsilon$ into $I_{\Phi}$ at timestep $t$. The loss is weighted by the diffusion scheduler $w(t)$. 
% \vspace{-3mm}

\subsection{Overview of the RTPD Algorithm}\label{subsec:algo_overview}
Fig.~\ref{fig:Overview} presents an overview of our RTPD algorithm.
It is grounded in the Hausdorff distance-based penetration depth calculation method (Sec.~\ref{sec:preliminary}).
%, similar to that of Tang et al.~\shortcite{SIG09HIST}.
The process consists of two primary phases: penetration surface extraction and Hausdorff distance calculation.
We leverage the RTX platform's capabilities to accelerate both of these steps.

\begin{figure*}[t]
    \centering
    \includegraphics[width=0.8\textwidth]{Image/overview.pdf}
    \caption{The overview of RT-based penetration depth calculation algorithm overview}
    \label{fig:Overview}
\end{figure*}

The penetration surface extraction phase focuses on identifying the overlapped region between two objects.
\revision{The penetration surface is defined as a set of polygons from one object, where at least one of its vertices lies within the other object. 
Note that in our work, we focus on triangles rather than general polygons, as they are processed most efficiently on the RTX platform.}
To facilitate this extraction, we introduce a ray-tracing-based \revision{Point-in-Polyhedron} test (RT-PIP), significantly accelerated through the use of RT cores (Sec.~\ref{sec:RT-PIP}).
This test capitalizes on the ray-surface intersection capabilities of the RTX platform.
%
Initially, a Geometry Acceleration Structure (GAS) is generated for each object, as required by the RTX platform.
The RT-PIP module takes the GAS of one object (e.g., $GAS_{A}$) and the point set of the other object (e.g., $P_{B}$).
It outputs a set of points (e.g., $P_{\partial B}$) representing the penetration region, indicating their location inside the opposing object.
Subsequently, a penetration surface (e.g., $\partial B$) is constructed using this point set (e.g., $P_{\partial B}$) (Sec.~\ref{subsec:surfaceGen}).
%
The generated penetration surfaces (e.g., $\partial A$ and $\partial B$) are then forwarded to the next step. 

The Hausdorff distance calculation phase utilizes the ray-surface intersection test of the RTX platform (Sec.~\ref{sec:RT-Hausdorff}) to compute the Hausdorff distance between two objects.
We introduce a novel Ray-Tracing-based Hausdorff DISTance algorithm, RT-HDIST.
It begins by generating GAS for the two penetration surfaces, $P_{\partial A}$ and $P_{\partial B}$, derived from the preceding step.
RT-HDIST processes the GAS of a penetration surface (e.g., $GAS_{\partial A}$) alongside the point set of the other penetration surface (e.g., $P_{\partial B}$) to compute the penetration depth between them.
The algorithm operates bidirectionally, considering both directions ($\partial A \to \partial B$ and $\partial B \to \partial A$).
The final penetration depth between the two objects, A and B, is determined by selecting the larger value from these two directional computations.

%In the Hausdorff distance calculation step, we compute the Hausdorff distance between given two objects using a ray-surface-intersection test. (Sec.~\ref{sec:RT-Hausdorff}) Initially, we construct the GAS for both $\partial A$ and $\partial B$ to utilize the RT-core effectively. The RT-based Hausdorff distance algorithms then determine the Hausdorff distance by processing the GAS of one object (e.g. $GAS_{\partial A}$) and set of the vertices of the other (e.g. $P_{\partial B}$). Following the Hausdorff distance definition (Eq.~\ref{equation:hausdorff_definition}), we compute the Hausdorff distance to both directions ($\partial A \to \partial B$) and ($\partial B \to \partial A$). As a result, the bigger one is the final Hausdorff distance, and also it is the penetration depth between input object $A$ and $B$.


%the proposed RT-based penetration depth calculation pipeline.
%Our proposed methods adopt Tang's Hausdorff-based penetration depth methods~\cite{SIG09HIST}. The pipeline is divided into the penetration surface extraction step and the Hausdorff distance calculation between the penetration surface steps. However, since Tang's approach is not suitable for the RT platform in detail, we modified and applied it with appropriate methods.

%The penetration surface extraction step is extracting overlapped surfaces on other objects. To utilize the RT core, we use the ray-intersection-based PIP(Point-In-Polygon) algorithms instead of collision detection between two objects which Tang et al.~\cite{SIG09HIST} used. (Sec.~\ref{sec:RT-PIP})
%RT core-based PIP test uses a ray-surface intersection test. For purpose this, we generate the GAS(Geometry Acceleration Structure) for each object. RT core-based PIP test takes the GAS of one object (e.g. $GAS_{A}$) and a set of vertex of another one (e.g. $P_{B}$). Then this computes the penetrated vertex set of another one (e.g. $P_{\partial B}$). To calculate the Hausdorff distance, these vertex sets change to objects constructed by penetrated surface (e.g. $\partial B$). Finally, the two generated overlapped surface objects $\partial A$ and $\partial B$ are used in the Hausdorff distance calculation step.

Our goal is to increase the robustness of T2I models, particularly with rare or unseen concepts, which they struggle to generate. To do so, we investigate a retrieval-augmented generation approach, through which we dynamically select images that can provide the model with missing visual cues. Importantly, we focus on models that were not trained for RAG, and show that existing image conditioning tools can be leveraged to support RAG post-hoc.
As depicted in \cref{fig:overview}, given a text prompt and a T2I generative model, we start by generating an image with the given prompt. Then, we query a VLM with the image, and ask it to decide if the image matches the prompt. If it does not, we aim to retrieve images representing the concepts that are missing from the image, and provide them as additional context to the model to guide it toward better alignment with the prompt.
In the following sections, we describe our method by answering key questions:
(1) How do we know which images to retrieve? 
(2) How can we retrieve the required images? 
and (3) How can we use the retrieved images for unknown concept generation?
By answering these questions, we achieve our goal of generating new concepts that the model struggles to generate on its own.

\vspace{-3pt}
\subsection{Which images to retrieve?}
The amount of images we can pass to a model is limited, hence we need to decide which images to pass as references to guide the generation of a base model. As T2I models are already capable of generating many concepts successfully, an efficient strategy would be passing only concepts they struggle to generate as references, and not all the concepts in a prompt.
To find the challenging concepts,
we utilize a VLM and apply a step-by-step method, as depicted in the bottom part of \cref{fig:overview}. First, we generate an initial image with a T2I model. Then, we provide the VLM with the initial prompt and image, and ask it if they match. If not, we ask the VLM to identify missing concepts and
focus on content and style, since these are easy to convey through visual cues.
As demonstrated in \cref{tab:ablations}, empirical experiments show that image retrieval from detailed image captions yields better results than retrieval from brief, generic concept descriptions.
Therefore, after identifying the missing concepts, we ask the VLM to suggest detailed image captions for images that describe each of the concepts. 

\vspace{-4pt}
\subsubsection{Error Handling}
\label{subsec:err_hand}

The VLM may sometimes fail to identify the missing concepts in an image, and will respond that it is ``unable to respond''. In these rare cases, we allow up to 3 query repetitions, while increasing the query temperature in each repetition. Increasing the temperature allows for more diverse responses by encouraging the model to sample less probable words.
In most cases, using our suggested step-by-step method yields better results than retrieving images directly from the given prompt (see 
\cref{subsec:ablations}).
However, if the VLM still fails to identify the missing concepts after multiple attempts, we fall back to retrieving images directly from the prompt, as it usually means the VLM does not know what is the meaning of the prompt.

The used prompts can be found in \cref{app:prompts}.
Next, we turn to retrieve images based on the acquired image captions.

\vspace{-3pt}
\subsection{How to retrieve the required images?}

Given $n$ image captions, our goal is to retrieve the images that are most similar to these captions from a dataset. 
To retrieve images matching a given image caption, we compare the caption to all the images in the dataset using a text-image similarity metric and retrieve the top $k$ most similar images.
Text-to-image retrieval is an active research field~\cite{radford2021learning, zhai2023sigmoid, ray2024cola, vendrowinquire}, where no single method is perfect.
Retrieval is especially hard when the dataset does not contain an exact match to the query \cite{biswas2024efficient} or when the task is fine-grained retrieval, that depends on subtle details~\cite{wei2022fine}.
Hence, a common retrieval workflow is to first retrieve image candidates using pre-computed embeddings, and then re-rank the retrieved candidates using a different, often more expensive but accurate, method \cite{vendrowinquire}.
Following this workflow, we experimented with cosine similarity over different embeddings, and with multiple re-ranking methods of reference candidates.
Although re-ranking sometimes yields better results compared to simply using cosine similarity between CLIP~\cite{radford2021learning} embeddings, the difference was not significant in most of our experiments. Therefore, for simplicity, we use cosine similarity between CLIP embeddings as our similarity metric (see \cref{tab:sim_metrics}, \cref{subsec:ablations} for more details about our experiments with different similarity metrics).

\vspace{-3pt}
\subsection{How to use the retrieved images?}
Putting it all together, after retrieving relevant images, all that is left to do is to use them as context so they are beneficial for the model.
We experimented with two types of models; models that are trained to receive images as input in addition to text and have ICL capabilities (e.g., OmniGen~\cite{xiao2024omnigen}), and T2I models augmented with an image encoder in post-training (e.g., SDXL~\cite{podellsdxl} with IP-adapter~\cite{ye2023ip}).
As the first model type has ICL capabilities, we can supply the retrieved images as examples that it can learn from, by adjusting the original prompt.
Although the second model type lacks true ICL capabilities, it offers image-based control functionalities, which we can leverage for applying RAG over it with our method.
Hence, for both model types, we augment the input prompt to contain a reference of the retrieved images as examples.
Formally, given a prompt $p$, $n$ concepts, and $k$ compatible images for each concept, we use the following template to create a new prompt:
``According to these examples of 
$\mathord{<}c_1\mathord{>:<}img_{1,1}\mathord{>}, ... , \mathord{<}img_{1,k}\mathord{>}, ... , \mathord{<}c_n\mathord{>:<}img_{n,1}\mathord{>}, ... , $
$\mathord{<}img_{n,k}\mathord{>}$,
generate $\mathord{<}p\mathord{>}$'', 
where $c_i$ for $i\in{[1,n]}$ is a compatible image caption of the image $\mathord{<}img_{i,j}\mathord{>},  j\in{[1,k]}$. 

This prompt allows models to learn missing concepts from the images, guiding them to generate the required result. 

\textbf{Personalized Generation}: 
For models that support multiple input images, we can apply our method for personalized generation as well, to generate rare concept combinations with personal concepts. In this case, we use one image for personal content, and 1+ other reference images for missing concepts. For example, given an image of a specific cat, we can generate diverse images of it, ranging from a mug featuring the cat to a lego of it or atypical situations like the cat writing code or teaching a classroom of dogs (\cref{fig:personalization}).
\vspace{-2pt}
\begin{figure}[htp]
  \centering
   \includegraphics[width=\linewidth]{Assets/personalization.pdf}
   \caption{\textbf{Personalized generation example.}
   \emph{ImageRAG} can work in parallel with personalization methods and enhance their capabilities. For example, although OmniGen can generate images of a subject based on an image, it struggles to generate some concepts. Using references retrieved by our method, it can generate the required result.
}
   \label{fig:personalization}\vspace{-10pt}
\end{figure}
\section{Simulations and Experiment}
In this section, we conduct comprehensive experiments in both simulation and the real-world robot to address the following questions:
\begin{itemize}[leftmargin=*]
    \item \textbf{Q1(Sim)}: How does the \our policy perform in tracking across different commands?
    \item  \textbf{Q2(Sim)}: How to reasonably combine various commands in the general command space? % Command Analysis
    \item \textbf{Q3(Sim)}: How does large-scale noise intervention training help in policy robustness? % Ablation Study
    \item \textbf{Q4(Real)}: How does \our behave in the real world? % Real World Demo
\end{itemize}

\noindent\textbf{Robot and Simulator.} 
Our main experiments in this paper are conducted on the Unitree H1 robot, which has 19 Degrees of Freedom (DOF) in total, including 
two 3-DOF shoulder joints, two elbow joints, one waist joint, two 3-DOF hip joints, two knee joints, and two ankle joints.
The simulation training is based on the NVIDIA IsaacGym simulator~\citep{makoviychuk2021isaac}. It takes 16 hours on a single RTX 4090 GPU to train one policy.

\noindent\textbf{Command analysis principle and metric.}
One of the main contributions of this paper is an extended and general command space for humanoid robots. Therefore, we pay much attention to command analysis (regarding Q1 and Q2). This includes analysis of single command tracking errors, along with the combination of different commands under different gaits.
% we categorize the commands into three groups: \emph{movement}, \emph{foot}, and \emph{posture}. The \emph{movement} commands include the linear velocity and angular velocity, forming the foundational locomotion commands and are considered the most critical aspect of the tasks. The \emph{foot} commands include the gait frequency and foot swing height, representing the mode of leg movement. The \emph{posture} commands include body height, body pitch and waist yaw, which determine the desired body posture.
For analysis, we evaluate the averaged episodic command tracking error (denoted as $E_\text{cmd}$), which measures the discrepancy between the actual robot states and the command space using $L_1$ norm.
% The tracking error is measured in units of $m/s$, $rad/s$, $Hz$, $m$, and $rad$, corresponding to linear velocity, angular velocity, frequency, position, and rotation, respectively.
All commands are uniformly sampled within a pre-defined command range, as shown in \tb{tab:commands}\footnote{Note that the hopping gait keeps a different command range, due to its asymmetric type of motion. More details can be referred to \ap{ap:Hopping}.}.

%%%%%%%%%%%---SETME-----%%%%%%%%%%%%%
%replace @@ with the submission number submission site.
\newcommand{\thiswork}{INF$^2$\xspace}
%%%%%%%%%%%%%%%%%%%%%%%%%%%%%%%%%%%%


%\newcommand{\rev}[1]{{\color{olivegreen}#1}}
\newcommand{\rev}[1]{{#1}}


\newcommand{\JL}[1]{{\color{cyan}[\textbf{\sc JLee}: \textit{#1}]}}
\newcommand{\JW}[1]{{\color{orange}[\textbf{\sc JJung}: \textit{#1}]}}
\newcommand{\JY}[1]{{\color{blue(ncs)}[\textbf{\sc JSong}: \textit{#1}]}}
\newcommand{\HS}[1]{{\color{magenta}[\textbf{\sc HJang}: \textit{#1}]}}
\newcommand{\CS}[1]{{\color{navy}[\textbf{\sc CShin}: \textit{#1}]}}
\newcommand{\SN}[1]{{\color{olive}[\textbf{\sc SNoh}: \textit{#1}]}}

%\def\final{}   % uncomment this for the submission version
\ifdefined\final
\renewcommand{\JL}[1]{}
\renewcommand{\JW}[1]{}
\renewcommand{\JY}[1]{}
\renewcommand{\HS}[1]{}
\renewcommand{\CS}[1]{}
\renewcommand{\SN}[1]{}
\fi

%%% Notion for baseline approaches %%% 
\newcommand{\baseline}{offloading-based batched inference\xspace}
\newcommand{\Baseline}{Offloading-based batched inference\xspace}


\newcommand{\ans}{attention-near storage\xspace}
\newcommand{\Ans}{Attention-near storage\xspace}
\newcommand{\ANS}{Attention-Near Storage\xspace}

\newcommand{\wb}{delayed KV cache writeback\xspace}
\newcommand{\Wb}{Delayed KV cache writeback\xspace}
\newcommand{\WB}{Delayed KV Cache Writeback\xspace}

\newcommand{\xcache}{X-cache\xspace}
\newcommand{\XCACHE}{X-Cache\xspace}


%%% Notions for our methods %%%
\newcommand{\schemea}{\textbf{Expanding supported maximum sequence length with optimized performance}\xspace}
\newcommand{\Schemea}{\textbf{Expanding supported maximum sequence length with optimized performance}\xspace}

\newcommand{\schemeb}{\textbf{Optimizing the storage device performance}\xspace}
\newcommand{\Schemeb}{\textbf{Optimizing the storage device performance}\xspace}

\newcommand{\schemec}{\textbf{Orthogonally supporting Compression Techniques}\xspace}
\newcommand{\Schemec}{\textbf{Orthogonally supporting Compression Techniques}\xspace}



% Circular numbers
\usepackage{tikz}
\newcommand*\circled[1]{\tikz[baseline=(char.base)]{
            \node[shape=circle,draw,inner sep=0.4pt] (char) {#1};}}

\newcommand*\bcircled[1]{\tikz[baseline=(char.base)]{
            \node[shape=circle,draw,inner sep=0.4pt, fill=black, text=white] (char) {#1};}}

\subsection{Single Command Tracking}
We first analyze each command separately while keeping all other commands held at their default values. The results are shown in \tb{tab:Single commands}.
It is easily observed that the tracking errors in the walking and standing gaits are significantly lower than those in the jumping and hopping, with hopping exhibiting the largest tracking errors.
For hopping gaits, the robot may fall during the tracking of specific commands, like high-speed tracking, body pitch, and waist-yaw control.
This can be attributed to the fact that hopping requires rather high stability. Moreover, the complex postures and motions further exacerbate the risk of instability. Consequently, the policy prioritizes learning to maintain the balance, which, to some extent, compromises the accuracy of command tracking.

We conclude that the tracking accuracy of each gait aligns with the training difficulty of that gait in simulation. For example, the walking and standing patterns can be learned first during training, while the jumping and hopping gaits appear later and require an extended training period for the robot to acquire proficiency.
Similarly, the tracking accuracy of robots under low velocity is significantly better than those under high velocity, since 1) the locomotion skills under low velocity are much easier to master, and 2) the dynamic stability of the robot decreases at high speeds, leading to a trade-off with tracking accuracy.

We also found that the tracking accuracy for longitudinal velocity commands $v_x$ surpasses that of horizontal velocity commands $v_y$, which is due to the limitation of the hardware configuration of the selected Unitree H1 robots. In addition, the {foot swing height} $l$ is the least accurately tracked.
Furthermore, the tracking reward related to foot placement outperforms the tracking performance associated with posture control, since adjusting posture introduces greater challenges to stability. In response, the policy adopts more conservative actions to mitigate balance-threatening postural changes.
% In contrast, the influence of foot placement on stability is comparatively less pronounced, allowing for more precise tracking.

\begin{table}[t]
\setlength{\abovecaptionskip}{0.cm}
\setlength{\belowcaptionskip}{-0.cm}
\centering
\caption{\small \textbf{Single command tracking error.} The tracking errors for foot commands are calculated over a complete gait cycle, and the remaining ones are over one environmental step. For standing gait, we only tested the body height, body pitch, and waist yaw tracking error. $E^\text{high}$ and $E^\text{low}$ represents high-speed ($v_x > 1m/s$) and low-speed ($v_x \le 1m/s$) modes categorized by the linear velocity $v$. 
The tracking error is computed by sampling each command in a predefined range (\tb{tab:commands}) while keeping all other commands held at their default values.}
\label{tab:Single commands}
\resizebox{\columnwidth}{!}{
\begin{tabular}{@{}c|cccc|cc|ccc@{}}
\toprule
\multirow{3}{*}{Gait} & \multicolumn{4}{c|}{Movement} & \multicolumn{2}{c|}{Foot} & \multicolumn{3}{c}{Posture} \\
\cmidrule(l){2-5} \cmidrule{6-7} \cmidrule{8-10} 
& \multirow{2}{*}{\makecell{$E_{v_x}^\text{low}$\\($m/s$)}} & \multirow{2}{*}{\makecell{$E_{v_x}^\text{high}$\\($m/s$)}} & \multirow{2}{*}{\makecell{$E_{v_y}$\\($m/s$)}} & \multirow{2}{*}{\makecell{$E_{\omega}$\\$rad/s$}} & \multirow{2}{*}{\makecell{$E_{f}$\\($HZ$)}} & \multirow{2}{*}{\makecell{$E_{l}$\\($m$)}} & \multirow{2}{*}{\makecell{$E_{h}$\\($m$)}}  & \multirow{2}{*}{\makecell{$E_{p}$\\($rad$)}} & \multirow{2}{*}{\makecell{$E_{w}$\\($rad$)}}   \\ 
&  &  &  &  &  &  &  &  &    \\ 
\midrule
Standing  & - & - & - & - & - & - & 0.035 & 0.047 & 0.022  \\
Walking   & 0.030 & 0.216 & 0.085 & 0.054 & 0.028 & 0.011 & 0.064 & 0.038 & 0.075  \\
Jumping  & 0.090 & 0.532 & 0.069 & 0.077 & 0.027 & 0.012 & 0.058 & 0.048 & 0.022 \\
Hopping   & 0.033 & - & 0.046 & 0.078 & - & - & 0.103 & - & - \\
\bottomrule
\end{tabular}}
\end{table}



\begin{table*}[t]
\setlength{\abovecaptionskip}{0.cm}
\setlength{\belowcaptionskip}{-0.cm}
\centering
\caption{\small \textbf{Tracking errors with different intervention strategies under the walking gait}. We evaluate three upper-body intervention training strategies: Noise (\our), the AMASS dataset, and no intervention at all. The tracking errors across various task and behavior commands reflect the intervention tolerance, \textit{i.e.}, the ability of precise locomotion control under external intervention.}
\label{tab:Intervetion Tracking Error}
\begin{tabular}{c|c|ccc|cc|ccc}
\toprule
\multirow{3}{*}{Training Strategy} & \multirow{3}{*}{Intervention Task} & \multicolumn{3}{c|}{Task Commands}                        & \multicolumn{5}{c}{Behavior Commands}\\ \cmidrule{3-10}
 & & \multicolumn{3}{c|}{Movement}                        & \multicolumn{2}{c|}{Foot}          & \multicolumn{3}{c}{Posture}                         \\ \cmidrule{3-10}
                                      &                                      &$E_{v_x}$ ($m/s$)     & $E_{v_y}$ ($m/s$)   & $E_{\omega}$ ($rad/s$)    & $E_{f}$ ($Hz$)         & $E_{l}$ ($m$)         & $E_{h}$ ($m$)        & $E_{p}$ ($rad$)     & $E_{w}$ ($rad$)         \\ \midrule
\multirow{3}{*}{\makecell{Noise Curriculum\\(\our)}}        & Noise                        & \textbf{0.0483} & \textbf{0.0962} & \textbf{0.1879} & \textbf{0.0471} & \textbf{0.0542} & \textbf{0.0402} & \textbf{0.0432} & \textbf{0.0552} \\
                                      & AMASS                                & \textbf{0.0391} & \textbf{0.0920} & \textbf{0.1039} & \textbf{0.0464} & \textbf{0.0543} & \textbf{0.0387} & \textbf{0.0364} & \textbf{0.0540} \\
                                      & None                                 & \textbf{0.0264} & \textbf{0.0863} & \textbf{0.0543} & \textbf{0.0447} & \textbf{0.0522} & 0.0372          & 0.0375          & 0.0475          \\ \cmidrule{1-10}
\multirow{3}{*}{AMASS}                & Noise                        & 0.1697          & 0.1055          & 0.2156          & 0.0621          & 0.0542          & 0.0620          & 0.0812          & 0.0694          \\
                                      & AMASS                                & 0.0567          & 0.0965          & 0.1593          & 0.0466          & 0.0555          & 0.0579          & 0.0458          & 0.0554          \\
                                      & None                                 & 0.0645          & 0.0916          & 0.0802          & 0.0460          & 0.0531          & 0.0577          & 0.0455          & 0.0568          \\ \cmidrule{1-10}
\multirow{3}{*}{No Intervention}                 & Noise                        & 0.8658          & 0.7511          & 0.9116          & 0.1930          & 0.1913          & 0.1658          & 0.3622          & 0.2241          \\
                                      & AMASS                                & 0.6299          & 0.4026          & 0.5758          & 0.2245          & 0.2527          & 0.1305          & 0.2367          & 0.1112          \\
                                      & None                                 & 0.0755          & 0.1076          & 0.1151          & 0.0450          & 0.0678          & \textbf{0.0255} & \textbf{0.0211} & \textbf{0.0380} \\ \bottomrule
\end{tabular}
\end{table*}



\begin{table}[t]
\setlength{\abovecaptionskip}{0.cm}
\setlength{\belowcaptionskip}{-0.cm}
\centering
\caption{ \small
\textbf{Averaged foot displacement under intervention}. We compare foot displacement $D_\text{cmd}$ of different training strategies under various intervention tasks, which computes the total movement of both feet in one episode with sampled posture behavior commands.
}
\label{tab:Intervention Mean Foot Movement}
\resizebox{\linewidth}{!}{
\begin{tabular}{ccccc}
\toprule
Training Strategy                 & Intervention Task     & $D_{h}$ ($m/s$)                  & $D_{p}$ ($m/s$)      & $D_{w}$ ($m/s$)       \\ \midrule
\multirow{3}{*}{\makecell{Noise Curriculum\\(\our)}}  & Noise & \textbf{0.0339}             & \textbf{0.0892} & \textbf{0.0199} \\
                       & AMASS         & \textbf{0.0454}             & \textbf{0.0728} & \textbf{0.0196} \\
                       & None          & \textbf{0.0003}             & \textbf{0.0016} & \textbf{0.0007} \\ \midrule
\multirow{3}{*}{AMASS only} & Noise         & 2.0815                      & 2.8978          & 3.2630          \\
                       & AMASS         & 0.0536                      & 0.1743          & 0.0396          \\
                       & None          & 0.0139                      & 0.0160          & 0.0013          \\ \midrule
\multirow{3}{*}{No Intervention}  & Noise         & 17.5358                     & 17.9732         & 25.7132         \\
                       & AMASS         & 25.3802 & 26.3496         & 21.3078         \\
                       & None          & 0.0159  & 1.7065          & 1.7152          \\ \bottomrule
\end{tabular}}
\end{table}

\subsection{Command Combination Analysis}
To provide an in-depth analysis of the command space and to 
reveal the underlying interaction of various commands under different gaits.
Here, we aim to analyze the \emph{orthogonality} of commands based on the interference or conflict between the tracking errors of these commands across their reasonable ranges. For instance, when we say that a set of commands are \emph{orthogonal}, each command does not significantly affect the tracking performance of each other in its range. To this end, we plot the tracking error $E_\text{cmd}$ as heat maps, generated by systematically scanning the command values for each pair of parameters, revealing the correlation of each command.
We leave the full heat maps at \ap{ap:heatmaps}, and conclude our main observation for all gaits.

\noindent\textbf{Walking.} Walking is the most basic gait, which preserves the best performance of the robot hardware.
\begin{itemize}[leftmargin=*]
    \item The {linear velocity} $v_x$, the {angular velocity yaw} $\omega$, the {body height} $h$, and the {waist yaw} $w$ are orthogonal during walking.
    \item When the {linear velocity} $v_x$ exceeds $1.5m/s$, the orthogonality between $v_x$ and other commands decreases due to reduced dynamic stability and the robot's need to maintain body stability over tracking accuracy.
    \item The {gait frequency} $f$ shows discrete orthogonality, with optimal tracking performance at frequencies of 1.5 or 2. High-frequency gait conditions reduce tracking accuracy.
    \item The {linear velocity} $v_y$, the {foot swing height} $l$, and the {body pitch} $p$ are orthogonal to other commands only within a narrow range.
\end{itemize}

\noindent\textbf{Jumping.} The command orthogonality in jumping is similar to walking, but the overall orthogonal range is smaller, due to the increased challenge of the jumping gait, especially in high-speed movement modes.
During each gait cycle, the robot must leap forward significantly to maintain its speed. To execute this complex jumping action continuously, the robot must adopt an optimal posture at the beginning of each cycle. Both legs exert substantial torque to propel the body forward. Upon landing, the robot must quickly readjust its posture to maintain stability and repeat the actions. Consequently, during movement, the robot can only execute other commands within a relatively narrow range.

\noindent\textbf{Hopping.}
The hopping gait introduces more instability, and the robot's control system must focus more on maintaining balance, making it difficult to simultaneously handle complex, multi-dimensional commands.
\begin{itemize}[leftmargin=*]
    \item Hopping gait commands lack clear orthogonal relationships.
    \item Effective tracking is limited to the x-axis {linear velocity} $v_x$, the y-axis {linear velocity} $v_y$, the {angular velocity yaw} $\omega$, and the {body height} $h$.
    \item Adjustments to $h$ can be understood that a lower body height improves dynamic stability, therefore, it plays a positive role in maintaining the target body posture.
    % enhancing the robot's hopping performance.
\end{itemize}

\noindent\textbf{Standing.} As for the standing gait, we tested the tracking errors of commands related to posture. The results showed that the tracking errors were similar to those observed during walking with zero velocity.

\begin{itemize}[leftmargin=*]
    \item The {waist yaw} $w$ command is almost orthogonal to the other two commands.
    \item As the range of commands increases, orthogonality between the {body height} $h$ and the {body pitch} $p$ decreases. This is because the H1 robot has only one degree of freedom at the waist, limiting posture adjustments to the hip pitch joint.
    \item A 0.3 m decrease of the body height relative to the default height reduces the range of motion of the hip pitch joint to almost zero, hindering precise tracking of body pitch.
\end{itemize}

Furthermore, we conclude that {gait frequency} $f$ highly affects the tracking accuracy of \emph{movement} commands when it is excessively high and low; the \emph{posture} commands can significantly impact the tracking errors of other commands, especially when they are near the range limits.
% We categorize the commands into three groups: \emph{movement}, \emph{foot}, and \emph{posture}. 1) The \emph{movement} commands include the linear velocity $v_x, v_y$ and angular velocity $\omega$, forming the foundational locomotion commands, and are considered the most critical aspect of the tasks. 2) The \emph{foot} commands include the {foot swing height} $l$, which is the least accurately tracked; and the {gait frequency} $f$, which can affect the tracking accuracy of \emph{movement} commands when it is excessively high and low. 3) The \emph{posture} commands, which include body height $h$, the body pitch $p$, and waist yaw $w$, determine the desired body posture, and can significantly impact the tracking errors of other commands, especially when the command is challenging. 
For different gaits, the orthogonality range between commands is greatest in the walking gait and smallest in the hopping gait.

\subsection{Ablation on Intervention Training Strategy}
\label{sec:InterventionExp}
% The three policies use the same random seeds and training time.
To validate the effectiveness of the intervention training strategy on the policy robustness when external upper-body intervention is involved, we compare the policies trained with different strategies, including noise curriculum (\our), filtered AMASS data~\citep{he2024omnih2o}, and no intervention. We test the tracking errors under two different intervention tasks, \textit{i.e.}, uniform noise, AAMAS dataset, along with a no-intervention setup. The results under the walking gait are shown in \tb{tab:Intervetion Tracking Error}, and we leave other gaits in \ap{ap:SingleCommandsTracking-REMAIN}. 
It is obvious that the noise curriculum strategy of \our achieved the best performance under almost all test cases, except the posture-related tracking with no intervention. 
In particular, \our showed less of a decrease in tracking accuracy with various interventions, indicating our noise curriculum intervention strategy enables the control policy to handle a large range of arm movements, making it very useful and supportive for loco-manipulation tasks.
In comparison, the policy trained with AMASS data shows a significant decrease in the tracking accuracy when intervening with uniform noise, due to the limited motion in the training data. The policy trained without any intervention only performs well without external upper-body control.

It is worth noting that when intervention training is involved, the tracking error related to the movement and foot is also better than those of the policy trained without intervention, and \our provides the most accurate tracking. This shows that intervention training also contributes to the robustness of the policy. During our real robot experiments, we further observed that the robot behaves with a harder force when in contact with the floor, indicating a possible trade-off between motion regularization and tracking accuracy when involving intervention.

\noindent\textbf{Stability under standing gait.}
Adjusting posture in the standing state introduces additional requirements for stability, since the robot pacing to maintain balance may increase the difficulty of achieving manipulation tasks that require stand still. To investigate the necessity of noise curriculum for manipulation, we further measured the averaged foot displacement (in meters) under the standing gait, which computes the total movement of both feet in one episode (20 seconds) while tracking the posture behavior commands. Results in \tb{tab:Intervention Mean Foot Movement} show that \our exhibits minimal foot displacement. On the contrary, the strategy trained on AMASS data requires frequent small steps to adjust the posture and maintain stability for noise interventions. 
Without intervention training, the policy tends to tip over when involving intervention, leading to failure of the entire task.

%  鲁棒性测试的结果分析
\begin{figure}[t]
    \centering
    \includegraphics[width=\linewidth]{imgs/radar_chart_V2.pdf}
    \vspace{-13pt}
    \caption{\small \textbf{External disturbance tolerance}. Left: A constant and continuous force is applied to the robot. Right: A one-second force is exerted on the robot. The experiment is conducted under a standing gait with default commands. If the robot's survival ratio exceeds $98\%$, it is deemed capable of tolerating such external disturbance. 
    The survival ratio computes the trajectory ratio of non-termination (ends of timeout) during 4096 rollouts.}
    \label{fig:Robust}
    \vspace{-12pt}
\end{figure}
\noindent\textbf{Robustness for external disturbance.}
Finally, we test the contribution of intervention training and noise curriculum to the robustness of external disturbance. In particular, we evaluated the robot's maximum tolerance to external disturbance forces in eight directions and compared the policy trained without intervention. Results illustrated in \fig{fig:Robust} demonstrate that \our preserves greater tolerance for external disturbances in both pushing and loading scenarios across most of the directions. The reason behind this is that the intervention brings the robot exposed to various disturbances originating from its upper body, and thereby enhances the overall stability by dynamically adjusting leg strength.

% \our has a significantly higher tolerance for external disturbance forces in almost all directions compared to the strategy without intervention training.
% This is attributed to the fact that, during large-scale noise intervention training, the robot effectively explored a wide range of extreme scenarios and learned to enhance body stability by adjusting leg movements.

\subsection{Real-World Experiments}
We deploy \our on a real-world robot to verify its effectiveness. In \fig{fig:teaser}, we illustrate the humanoid capabilities supported by \our, showing the versatile behavior of the Unitree H1 robot. In particular, we demonstrate the intriguing potential of the comprehensive task range that \our is able to achieve, with a flexible combination of commands in high dynamics. To qualitatively analyze the performance of \our, we estimate the tracking error of two pose parameters (body pitch $p$ and waist rotation $w$ from the motor readings) on real robots, since other commands are hard to measure without a highly accurate motion capture system. The results are shown in \tb{tb:track-real}, where $E^{\text{real}}_{\text{cmd}}$ illustrates the tracking error of the posture command.
We observe that the tracking error in real-world experiments is slightly higher than in simulation environments, primarily due to sensor noise and the wear of the robot's hardware. Among different gaits, the tracking error for the waist rotation $w$ is smaller compared to that for the body pitch $p$, as waist control has less impact on the robot’s overall stability. In both error tests, the jumping gait exhibited the smallest $E_{cmd}$, while the walking gait showed slightly higher errors, consistent with the findings observed in the simulation environment.

\begin{table}[t]
\centering
\caption{\small \textbf{Tracking error in real world.} We conducted five tests to measure the tracking error for each command under three gaits. The tracking error for each command was calculated during each control step. The tested commands gradually increased from the minimum to the maximum values within a predefined range, while the remaining commands were kept at their default values.} % To account for the impact of communication delays on the actual tracking error, we introduced a 0.1-second delay in the command execution.
\label{tb:track-real}
\begin{tabular}{c|cc} \toprule
Gait     & $E_p^{\text{real}}$ & $E_w^{\text{real}}$ \\ \midrule
Standing & 0.0712 $\pm$ 0.0425 & 0.0718 $\pm$ 0.0614 \\
Walking  & 0.1006 $\pm$ 0.0581  & 0.0571 $\pm$ 0.0489 \\
Jumping  & 0.0674 $\pm$ 0.0569  & 0.0552 $\pm$ 0.0469 \\ \bottomrule
\end{tabular}
\end{table}

\section{Discussion}


\section{Related Work}

\begin{figure}[t!]
  \includegraphics[width=1.0\linewidth]{figure/deviation_success_rate.pdf}
  \caption{Correlation between the deviation distance and success rate (measured by average final reward).
  }
  \label{fig:deviation_success_rate}
\end{figure}

\paragraph{LLM Agent Learning.}
LLM agents are widely used for tackling complex real-world tasks~\citep{wang2023voyager,hu2024dawn,wang-etal-2024-e2cl}, relying on iterative interactions with their environment guided by task objectives and constraints. However, in long-horizon planning, excessive interactions make them prone to suboptimal actions, increasing the risk of failure. While closed-source LLMs demonstrate strong intelligence, open-source counterparts still lag behind~\citep{liu2023agentbench,wang2023mint}.
To address this gap, some studies focus on improving task success rates by increasing the likelihood of generating optimal actions~\citep{chen2023fireact, yuan2023scaling}. Alternatively, another line of research seeks to mitigate suboptimal actions by collecting them and applying preference learning methods to reduce their occurrence~\citep{song2024trial,xiong2024watch}.
Recently, researchers have explored the capacity of LLM agents to self-correct errors, enhancing their ability to ensure successful task completion~\citep{wang-etal-2024-e2cl, qu2024recursive}.
However, these methods primarily focus on self-correction after errors have already occurred, lacking the ability to detect suboptimal actions in advance and calibrate subsequent planning accordingly.

\paragraph{Process Supervision.}
 
Process supervision provides fine-grained guidance, making it a promising approach for addressing long-horizon problems~\citep{uesato2022solving}. Early studies have explored obtaining step-level rewards and using them to optimize intermediate processes through reinforcement learning~\citep{lightman2023let, deng2024novice, wang-etal-2024-math}. Others have focused on constructing step-level positive and negative data pairs and applying preference learning techniques to achieve more precise optimization~\citep{xiong2024watch, jiao2024learning}.
However, existing studies have yet to address the construction of step-level reflection data. Such data could empower LLM agents to detect suboptimal actions, analyze the reasons for their suboptimality, and determine how to calibrate them to ensure successful task completion.

\section{Conclusion}

In this paper, we introduce STeCa, a novel agent learning framework designed to enhance the performance of LLM agents in long-horizon tasks. 
STeCa identifies deviated actions through step-level reward comparisons and constructs calibration trajectories via reflection. 
These trajectories serve as critical data for reinforced training. Extensive experiments demonstrate that STeCa significantly outperforms baseline methods, with additional analyses underscoring its robust calibration capabilities.

% \newpage

\section*{Limitations and Ethical Considerations}

\noindent\textbf{Limitations.} The primary limitation of our work is that it extends only the dataset provided by MUSE and employs DeepSeek-v3 for question generation. 
To mitigate this generalization risk, we have released our code and the generated audit suite, allowing researchers to utilize our framework to create additional audit datasets and evaluate their quality. Meanwhile, this is also our future work to extend our framework to other benchmarks.

\noindent\textbf{Ethical Considerations.} Machine unlearning can be employed to mitigate risks associated with LLMs in terms of privacy, security, bias, and copyright. Our work is dedicated to providing a comprehensive evaluation framework to help researchers better understand the unlearning effectiveness of LLMs, which we believe will have a positive impact on society.

% \section*{Acknowledgments}

% Bibliography entries for the entire Anthology, followed by custom entries
% \bibliography{anthology,custom}
% Custom bibliography entries only
\bibliography{custom}

\appendix
\newpage

\subsection{Lloyd-Max Algorithm}
\label{subsec:Lloyd-Max}
For a given quantization bitwidth $B$ and an operand $\bm{X}$, the Lloyd-Max algorithm finds $2^B$ quantization levels $\{\hat{x}_i\}_{i=1}^{2^B}$ such that quantizing $\bm{X}$ by rounding each scalar in $\bm{X}$ to the nearest quantization level minimizes the quantization MSE. 

The algorithm starts with an initial guess of quantization levels and then iteratively computes quantization thresholds $\{\tau_i\}_{i=1}^{2^B-1}$ and updates quantization levels $\{\hat{x}_i\}_{i=1}^{2^B}$. Specifically, at iteration $n$, thresholds are set to the midpoints of the previous iteration's levels:
\begin{align*}
    \tau_i^{(n)}=\frac{\hat{x}_i^{(n-1)}+\hat{x}_{i+1}^{(n-1)}}2 \text{ for } i=1\ldots 2^B-1
\end{align*}
Subsequently, the quantization levels are re-computed as conditional means of the data regions defined by the new thresholds:
\begin{align*}
    \hat{x}_i^{(n)}=\mathbb{E}\left[ \bm{X} \big| \bm{X}\in [\tau_{i-1}^{(n)},\tau_i^{(n)}] \right] \text{ for } i=1\ldots 2^B
\end{align*}
where to satisfy boundary conditions we have $\tau_0=-\infty$ and $\tau_{2^B}=\infty$. The algorithm iterates the above steps until convergence.

Figure \ref{fig:lm_quant} compares the quantization levels of a $7$-bit floating point (E3M3) quantizer (left) to a $7$-bit Lloyd-Max quantizer (right) when quantizing a layer of weights from the GPT3-126M model at a per-tensor granularity. As shown, the Lloyd-Max quantizer achieves substantially lower quantization MSE. Further, Table \ref{tab:FP7_vs_LM7} shows the superior perplexity achieved by Lloyd-Max quantizers for bitwidths of $7$, $6$ and $5$. The difference between the quantizers is clear at 5 bits, where per-tensor FP quantization incurs a drastic and unacceptable increase in perplexity, while Lloyd-Max quantization incurs a much smaller increase. Nevertheless, we note that even the optimal Lloyd-Max quantizer incurs a notable ($\sim 1.5$) increase in perplexity due to the coarse granularity of quantization. 

\begin{figure}[h]
  \centering
  \includegraphics[width=0.7\linewidth]{sections/figures/LM7_FP7.pdf}
  \caption{\small Quantization levels and the corresponding quantization MSE of Floating Point (left) vs Lloyd-Max (right) Quantizers for a layer of weights in the GPT3-126M model.}
  \label{fig:lm_quant}
\end{figure}

\begin{table}[h]\scriptsize
\begin{center}
\caption{\label{tab:FP7_vs_LM7} \small Comparing perplexity (lower is better) achieved by floating point quantizers and Lloyd-Max quantizers on a GPT3-126M model for the Wikitext-103 dataset.}
\begin{tabular}{c|cc|c}
\hline
 \multirow{2}{*}{\textbf{Bitwidth}} & \multicolumn{2}{|c|}{\textbf{Floating-Point Quantizer}} & \textbf{Lloyd-Max Quantizer} \\
 & Best Format & Wikitext-103 Perplexity & Wikitext-103 Perplexity \\
\hline
7 & E3M3 & 18.32 & 18.27 \\
6 & E3M2 & 19.07 & 18.51 \\
5 & E4M0 & 43.89 & 19.71 \\
\hline
\end{tabular}
\end{center}
\end{table}

\subsection{Proof of Local Optimality of LO-BCQ}
\label{subsec:lobcq_opt_proof}
For a given block $\bm{b}_j$, the quantization MSE during LO-BCQ can be empirically evaluated as $\frac{1}{L_b}\lVert \bm{b}_j- \bm{\hat{b}}_j\rVert^2_2$ where $\bm{\hat{b}}_j$ is computed from equation (\ref{eq:clustered_quantization_definition}) as $C_{f(\bm{b}_j)}(\bm{b}_j)$. Further, for a given block cluster $\mathcal{B}_i$, we compute the quantization MSE as $\frac{1}{|\mathcal{B}_{i}|}\sum_{\bm{b} \in \mathcal{B}_{i}} \frac{1}{L_b}\lVert \bm{b}- C_i^{(n)}(\bm{b})\rVert^2_2$. Therefore, at the end of iteration $n$, we evaluate the overall quantization MSE $J^{(n)}$ for a given operand $\bm{X}$ composed of $N_c$ block clusters as:
\begin{align*}
    \label{eq:mse_iter_n}
    J^{(n)} = \frac{1}{N_c} \sum_{i=1}^{N_c} \frac{1}{|\mathcal{B}_{i}^{(n)}|}\sum_{\bm{v} \in \mathcal{B}_{i}^{(n)}} \frac{1}{L_b}\lVert \bm{b}- B_i^{(n)}(\bm{b})\rVert^2_2
\end{align*}

At the end of iteration $n$, the codebooks are updated from $\mathcal{C}^{(n-1)}$ to $\mathcal{C}^{(n)}$. However, the mapping of a given vector $\bm{b}_j$ to quantizers $\mathcal{C}^{(n)}$ remains as  $f^{(n)}(\bm{b}_j)$. At the next iteration, during the vector clustering step, $f^{(n+1)}(\bm{b}_j)$ finds new mapping of $\bm{b}_j$ to updated codebooks $\mathcal{C}^{(n)}$ such that the quantization MSE over the candidate codebooks is minimized. Therefore, we obtain the following result for $\bm{b}_j$:
\begin{align*}
\frac{1}{L_b}\lVert \bm{b}_j - C_{f^{(n+1)}(\bm{b}_j)}^{(n)}(\bm{b}_j)\rVert^2_2 \le \frac{1}{L_b}\lVert \bm{b}_j - C_{f^{(n)}(\bm{b}_j)}^{(n)}(\bm{b}_j)\rVert^2_2
\end{align*}

That is, quantizing $\bm{b}_j$ at the end of the block clustering step of iteration $n+1$ results in lower quantization MSE compared to quantizing at the end of iteration $n$. Since this is true for all $\bm{b} \in \bm{X}$, we assert the following:
\begin{equation}
\begin{split}
\label{eq:mse_ineq_1}
    \tilde{J}^{(n+1)} &= \frac{1}{N_c} \sum_{i=1}^{N_c} \frac{1}{|\mathcal{B}_{i}^{(n+1)}|}\sum_{\bm{b} \in \mathcal{B}_{i}^{(n+1)}} \frac{1}{L_b}\lVert \bm{b} - C_i^{(n)}(b)\rVert^2_2 \le J^{(n)}
\end{split}
\end{equation}
where $\tilde{J}^{(n+1)}$ is the the quantization MSE after the vector clustering step at iteration $n+1$.

Next, during the codebook update step (\ref{eq:quantizers_update}) at iteration $n+1$, the per-cluster codebooks $\mathcal{C}^{(n)}$ are updated to $\mathcal{C}^{(n+1)}$ by invoking the Lloyd-Max algorithm \citep{Lloyd}. We know that for any given value distribution, the Lloyd-Max algorithm minimizes the quantization MSE. Therefore, for a given vector cluster $\mathcal{B}_i$ we obtain the following result:

\begin{equation}
    \frac{1}{|\mathcal{B}_{i}^{(n+1)}|}\sum_{\bm{b} \in \mathcal{B}_{i}^{(n+1)}} \frac{1}{L_b}\lVert \bm{b}- C_i^{(n+1)}(\bm{b})\rVert^2_2 \le \frac{1}{|\mathcal{B}_{i}^{(n+1)}|}\sum_{\bm{b} \in \mathcal{B}_{i}^{(n+1)}} \frac{1}{L_b}\lVert \bm{b}- C_i^{(n)}(\bm{b})\rVert^2_2
\end{equation}

The above equation states that quantizing the given block cluster $\mathcal{B}_i$ after updating the associated codebook from $C_i^{(n)}$ to $C_i^{(n+1)}$ results in lower quantization MSE. Since this is true for all the block clusters, we derive the following result: 
\begin{equation}
\begin{split}
\label{eq:mse_ineq_2}
     J^{(n+1)} &= \frac{1}{N_c} \sum_{i=1}^{N_c} \frac{1}{|\mathcal{B}_{i}^{(n+1)}|}\sum_{\bm{b} \in \mathcal{B}_{i}^{(n+1)}} \frac{1}{L_b}\lVert \bm{b}- C_i^{(n+1)}(\bm{b})\rVert^2_2  \le \tilde{J}^{(n+1)}   
\end{split}
\end{equation}

Following (\ref{eq:mse_ineq_1}) and (\ref{eq:mse_ineq_2}), we find that the quantization MSE is non-increasing for each iteration, that is, $J^{(1)} \ge J^{(2)} \ge J^{(3)} \ge \ldots \ge J^{(M)}$ where $M$ is the maximum number of iterations. 
%Therefore, we can say that if the algorithm converges, then it must be that it has converged to a local minimum. 
\hfill $\blacksquare$


\begin{figure}
    \begin{center}
    \includegraphics[width=0.5\textwidth]{sections//figures/mse_vs_iter.pdf}
    \end{center}
    \caption{\small NMSE vs iterations during LO-BCQ compared to other block quantization proposals}
    \label{fig:nmse_vs_iter}
\end{figure}

Figure \ref{fig:nmse_vs_iter} shows the empirical convergence of LO-BCQ across several block lengths and number of codebooks. Also, the MSE achieved by LO-BCQ is compared to baselines such as MXFP and VSQ. As shown, LO-BCQ converges to a lower MSE than the baselines. Further, we achieve better convergence for larger number of codebooks ($N_c$) and for a smaller block length ($L_b$), both of which increase the bitwidth of BCQ (see Eq \ref{eq:bitwidth_bcq}).


\subsection{Additional Accuracy Results}
%Table \ref{tab:lobcq_config} lists the various LOBCQ configurations and their corresponding bitwidths.
\begin{table}
\setlength{\tabcolsep}{4.75pt}
\begin{center}
\caption{\label{tab:lobcq_config} Various LO-BCQ configurations and their bitwidths.}
\begin{tabular}{|c||c|c|c|c||c|c||c|} 
\hline
 & \multicolumn{4}{|c||}{$L_b=8$} & \multicolumn{2}{|c||}{$L_b=4$} & $L_b=2$ \\
 \hline
 \backslashbox{$L_A$\kern-1em}{\kern-1em$N_c$} & 2 & 4 & 8 & 16 & 2 & 4 & 2 \\
 \hline
 64 & 4.25 & 4.375 & 4.5 & 4.625 & 4.375 & 4.625 & 4.625\\
 \hline
 32 & 4.375 & 4.5 & 4.625& 4.75 & 4.5 & 4.75 & 4.75 \\
 \hline
 16 & 4.625 & 4.75& 4.875 & 5 & 4.75 & 5 & 5 \\
 \hline
\end{tabular}
\end{center}
\end{table}

%\subsection{Perplexity achieved by various LO-BCQ configurations on Wikitext-103 dataset}

\begin{table} \centering
\begin{tabular}{|c||c|c|c|c||c|c||c|} 
\hline
 $L_b \rightarrow$& \multicolumn{4}{c||}{8} & \multicolumn{2}{c||}{4} & 2\\
 \hline
 \backslashbox{$L_A$\kern-1em}{\kern-1em$N_c$} & 2 & 4 & 8 & 16 & 2 & 4 & 2  \\
 %$N_c \rightarrow$ & 2 & 4 & 8 & 16 & 2 & 4 & 2 \\
 \hline
 \hline
 \multicolumn{8}{c}{GPT3-1.3B (FP32 PPL = 9.98)} \\ 
 \hline
 \hline
 64 & 10.40 & 10.23 & 10.17 & 10.15 &  10.28 & 10.18 & 10.19 \\
 \hline
 32 & 10.25 & 10.20 & 10.15 & 10.12 &  10.23 & 10.17 & 10.17 \\
 \hline
 16 & 10.22 & 10.16 & 10.10 & 10.09 &  10.21 & 10.14 & 10.16 \\
 \hline
  \hline
 \multicolumn{8}{c}{GPT3-8B (FP32 PPL = 7.38)} \\ 
 \hline
 \hline
 64 & 7.61 & 7.52 & 7.48 &  7.47 &  7.55 &  7.49 & 7.50 \\
 \hline
 32 & 7.52 & 7.50 & 7.46 &  7.45 &  7.52 &  7.48 & 7.48  \\
 \hline
 16 & 7.51 & 7.48 & 7.44 &  7.44 &  7.51 &  7.49 & 7.47  \\
 \hline
\end{tabular}
\caption{\label{tab:ppl_gpt3_abalation} Wikitext-103 perplexity across GPT3-1.3B and 8B models.}
\end{table}

\begin{table} \centering
\begin{tabular}{|c||c|c|c|c||} 
\hline
 $L_b \rightarrow$& \multicolumn{4}{c||}{8}\\
 \hline
 \backslashbox{$L_A$\kern-1em}{\kern-1em$N_c$} & 2 & 4 & 8 & 16 \\
 %$N_c \rightarrow$ & 2 & 4 & 8 & 16 & 2 & 4 & 2 \\
 \hline
 \hline
 \multicolumn{5}{|c|}{Llama2-7B (FP32 PPL = 5.06)} \\ 
 \hline
 \hline
 64 & 5.31 & 5.26 & 5.19 & 5.18  \\
 \hline
 32 & 5.23 & 5.25 & 5.18 & 5.15  \\
 \hline
 16 & 5.23 & 5.19 & 5.16 & 5.14  \\
 \hline
 \multicolumn{5}{|c|}{Nemotron4-15B (FP32 PPL = 5.87)} \\ 
 \hline
 \hline
 64  & 6.3 & 6.20 & 6.13 & 6.08  \\
 \hline
 32  & 6.24 & 6.12 & 6.07 & 6.03  \\
 \hline
 16  & 6.12 & 6.14 & 6.04 & 6.02  \\
 \hline
 \multicolumn{5}{|c|}{Nemotron4-340B (FP32 PPL = 3.48)} \\ 
 \hline
 \hline
 64 & 3.67 & 3.62 & 3.60 & 3.59 \\
 \hline
 32 & 3.63 & 3.61 & 3.59 & 3.56 \\
 \hline
 16 & 3.61 & 3.58 & 3.57 & 3.55 \\
 \hline
\end{tabular}
\caption{\label{tab:ppl_llama7B_nemo15B} Wikitext-103 perplexity compared to FP32 baseline in Llama2-7B and Nemotron4-15B, 340B models}
\end{table}

%\subsection{Perplexity achieved by various LO-BCQ configurations on MMLU dataset}


\begin{table} \centering
\begin{tabular}{|c||c|c|c|c||c|c|c|c|} 
\hline
 $L_b \rightarrow$& \multicolumn{4}{c||}{8} & \multicolumn{4}{c||}{8}\\
 \hline
 \backslashbox{$L_A$\kern-1em}{\kern-1em$N_c$} & 2 & 4 & 8 & 16 & 2 & 4 & 8 & 16  \\
 %$N_c \rightarrow$ & 2 & 4 & 8 & 16 & 2 & 4 & 2 \\
 \hline
 \hline
 \multicolumn{5}{|c|}{Llama2-7B (FP32 Accuracy = 45.8\%)} & \multicolumn{4}{|c|}{Llama2-70B (FP32 Accuracy = 69.12\%)} \\ 
 \hline
 \hline
 64 & 43.9 & 43.4 & 43.9 & 44.9 & 68.07 & 68.27 & 68.17 & 68.75 \\
 \hline
 32 & 44.5 & 43.8 & 44.9 & 44.5 & 68.37 & 68.51 & 68.35 & 68.27  \\
 \hline
 16 & 43.9 & 42.7 & 44.9 & 45 & 68.12 & 68.77 & 68.31 & 68.59  \\
 \hline
 \hline
 \multicolumn{5}{|c|}{GPT3-22B (FP32 Accuracy = 38.75\%)} & \multicolumn{4}{|c|}{Nemotron4-15B (FP32 Accuracy = 64.3\%)} \\ 
 \hline
 \hline
 64 & 36.71 & 38.85 & 38.13 & 38.92 & 63.17 & 62.36 & 63.72 & 64.09 \\
 \hline
 32 & 37.95 & 38.69 & 39.45 & 38.34 & 64.05 & 62.30 & 63.8 & 64.33  \\
 \hline
 16 & 38.88 & 38.80 & 38.31 & 38.92 & 63.22 & 63.51 & 63.93 & 64.43  \\
 \hline
\end{tabular}
\caption{\label{tab:mmlu_abalation} Accuracy on MMLU dataset across GPT3-22B, Llama2-7B, 70B and Nemotron4-15B models.}
\end{table}


%\subsection{Perplexity achieved by various LO-BCQ configurations on LM evaluation harness}

\begin{table} \centering
\begin{tabular}{|c||c|c|c|c||c|c|c|c|} 
\hline
 $L_b \rightarrow$& \multicolumn{4}{c||}{8} & \multicolumn{4}{c||}{8}\\
 \hline
 \backslashbox{$L_A$\kern-1em}{\kern-1em$N_c$} & 2 & 4 & 8 & 16 & 2 & 4 & 8 & 16  \\
 %$N_c \rightarrow$ & 2 & 4 & 8 & 16 & 2 & 4 & 2 \\
 \hline
 \hline
 \multicolumn{5}{|c|}{Race (FP32 Accuracy = 37.51\%)} & \multicolumn{4}{|c|}{Boolq (FP32 Accuracy = 64.62\%)} \\ 
 \hline
 \hline
 64 & 36.94 & 37.13 & 36.27 & 37.13 & 63.73 & 62.26 & 63.49 & 63.36 \\
 \hline
 32 & 37.03 & 36.36 & 36.08 & 37.03 & 62.54 & 63.51 & 63.49 & 63.55  \\
 \hline
 16 & 37.03 & 37.03 & 36.46 & 37.03 & 61.1 & 63.79 & 63.58 & 63.33  \\
 \hline
 \hline
 \multicolumn{5}{|c|}{Winogrande (FP32 Accuracy = 58.01\%)} & \multicolumn{4}{|c|}{Piqa (FP32 Accuracy = 74.21\%)} \\ 
 \hline
 \hline
 64 & 58.17 & 57.22 & 57.85 & 58.33 & 73.01 & 73.07 & 73.07 & 72.80 \\
 \hline
 32 & 59.12 & 58.09 & 57.85 & 58.41 & 73.01 & 73.94 & 72.74 & 73.18  \\
 \hline
 16 & 57.93 & 58.88 & 57.93 & 58.56 & 73.94 & 72.80 & 73.01 & 73.94  \\
 \hline
\end{tabular}
\caption{\label{tab:mmlu_abalation} Accuracy on LM evaluation harness tasks on GPT3-1.3B model.}
\end{table}

\begin{table} \centering
\begin{tabular}{|c||c|c|c|c||c|c|c|c|} 
\hline
 $L_b \rightarrow$& \multicolumn{4}{c||}{8} & \multicolumn{4}{c||}{8}\\
 \hline
 \backslashbox{$L_A$\kern-1em}{\kern-1em$N_c$} & 2 & 4 & 8 & 16 & 2 & 4 & 8 & 16  \\
 %$N_c \rightarrow$ & 2 & 4 & 8 & 16 & 2 & 4 & 2 \\
 \hline
 \hline
 \multicolumn{5}{|c|}{Race (FP32 Accuracy = 41.34\%)} & \multicolumn{4}{|c|}{Boolq (FP32 Accuracy = 68.32\%)} \\ 
 \hline
 \hline
 64 & 40.48 & 40.10 & 39.43 & 39.90 & 69.20 & 68.41 & 69.45 & 68.56 \\
 \hline
 32 & 39.52 & 39.52 & 40.77 & 39.62 & 68.32 & 67.43 & 68.17 & 69.30  \\
 \hline
 16 & 39.81 & 39.71 & 39.90 & 40.38 & 68.10 & 66.33 & 69.51 & 69.42  \\
 \hline
 \hline
 \multicolumn{5}{|c|}{Winogrande (FP32 Accuracy = 67.88\%)} & \multicolumn{4}{|c|}{Piqa (FP32 Accuracy = 78.78\%)} \\ 
 \hline
 \hline
 64 & 66.85 & 66.61 & 67.72 & 67.88 & 77.31 & 77.42 & 77.75 & 77.64 \\
 \hline
 32 & 67.25 & 67.72 & 67.72 & 67.00 & 77.31 & 77.04 & 77.80 & 77.37  \\
 \hline
 16 & 68.11 & 68.90 & 67.88 & 67.48 & 77.37 & 78.13 & 78.13 & 77.69  \\
 \hline
\end{tabular}
\caption{\label{tab:mmlu_abalation} Accuracy on LM evaluation harness tasks on GPT3-8B model.}
\end{table}

\begin{table} \centering
\begin{tabular}{|c||c|c|c|c||c|c|c|c|} 
\hline
 $L_b \rightarrow$& \multicolumn{4}{c||}{8} & \multicolumn{4}{c||}{8}\\
 \hline
 \backslashbox{$L_A$\kern-1em}{\kern-1em$N_c$} & 2 & 4 & 8 & 16 & 2 & 4 & 8 & 16  \\
 %$N_c \rightarrow$ & 2 & 4 & 8 & 16 & 2 & 4 & 2 \\
 \hline
 \hline
 \multicolumn{5}{|c|}{Race (FP32 Accuracy = 40.67\%)} & \multicolumn{4}{|c|}{Boolq (FP32 Accuracy = 76.54\%)} \\ 
 \hline
 \hline
 64 & 40.48 & 40.10 & 39.43 & 39.90 & 75.41 & 75.11 & 77.09 & 75.66 \\
 \hline
 32 & 39.52 & 39.52 & 40.77 & 39.62 & 76.02 & 76.02 & 75.96 & 75.35  \\
 \hline
 16 & 39.81 & 39.71 & 39.90 & 40.38 & 75.05 & 73.82 & 75.72 & 76.09  \\
 \hline
 \hline
 \multicolumn{5}{|c|}{Winogrande (FP32 Accuracy = 70.64\%)} & \multicolumn{4}{|c|}{Piqa (FP32 Accuracy = 79.16\%)} \\ 
 \hline
 \hline
 64 & 69.14 & 70.17 & 70.17 & 70.56 & 78.24 & 79.00 & 78.62 & 78.73 \\
 \hline
 32 & 70.96 & 69.69 & 71.27 & 69.30 & 78.56 & 79.49 & 79.16 & 78.89  \\
 \hline
 16 & 71.03 & 69.53 & 69.69 & 70.40 & 78.13 & 79.16 & 79.00 & 79.00  \\
 \hline
\end{tabular}
\caption{\label{tab:mmlu_abalation} Accuracy on LM evaluation harness tasks on GPT3-22B model.}
\end{table}

\begin{table} \centering
\begin{tabular}{|c||c|c|c|c||c|c|c|c|} 
\hline
 $L_b \rightarrow$& \multicolumn{4}{c||}{8} & \multicolumn{4}{c||}{8}\\
 \hline
 \backslashbox{$L_A$\kern-1em}{\kern-1em$N_c$} & 2 & 4 & 8 & 16 & 2 & 4 & 8 & 16  \\
 %$N_c \rightarrow$ & 2 & 4 & 8 & 16 & 2 & 4 & 2 \\
 \hline
 \hline
 \multicolumn{5}{|c|}{Race (FP32 Accuracy = 44.4\%)} & \multicolumn{4}{|c|}{Boolq (FP32 Accuracy = 79.29\%)} \\ 
 \hline
 \hline
 64 & 42.49 & 42.51 & 42.58 & 43.45 & 77.58 & 77.37 & 77.43 & 78.1 \\
 \hline
 32 & 43.35 & 42.49 & 43.64 & 43.73 & 77.86 & 75.32 & 77.28 & 77.86  \\
 \hline
 16 & 44.21 & 44.21 & 43.64 & 42.97 & 78.65 & 77 & 76.94 & 77.98  \\
 \hline
 \hline
 \multicolumn{5}{|c|}{Winogrande (FP32 Accuracy = 69.38\%)} & \multicolumn{4}{|c|}{Piqa (FP32 Accuracy = 78.07\%)} \\ 
 \hline
 \hline
 64 & 68.9 & 68.43 & 69.77 & 68.19 & 77.09 & 76.82 & 77.09 & 77.86 \\
 \hline
 32 & 69.38 & 68.51 & 68.82 & 68.90 & 78.07 & 76.71 & 78.07 & 77.86  \\
 \hline
 16 & 69.53 & 67.09 & 69.38 & 68.90 & 77.37 & 77.8 & 77.91 & 77.69  \\
 \hline
\end{tabular}
\caption{\label{tab:mmlu_abalation} Accuracy on LM evaluation harness tasks on Llama2-7B model.}
\end{table}

\begin{table} \centering
\begin{tabular}{|c||c|c|c|c||c|c|c|c|} 
\hline
 $L_b \rightarrow$& \multicolumn{4}{c||}{8} & \multicolumn{4}{c||}{8}\\
 \hline
 \backslashbox{$L_A$\kern-1em}{\kern-1em$N_c$} & 2 & 4 & 8 & 16 & 2 & 4 & 8 & 16  \\
 %$N_c \rightarrow$ & 2 & 4 & 8 & 16 & 2 & 4 & 2 \\
 \hline
 \hline
 \multicolumn{5}{|c|}{Race (FP32 Accuracy = 48.8\%)} & \multicolumn{4}{|c|}{Boolq (FP32 Accuracy = 85.23\%)} \\ 
 \hline
 \hline
 64 & 49.00 & 49.00 & 49.28 & 48.71 & 82.82 & 84.28 & 84.03 & 84.25 \\
 \hline
 32 & 49.57 & 48.52 & 48.33 & 49.28 & 83.85 & 84.46 & 84.31 & 84.93  \\
 \hline
 16 & 49.85 & 49.09 & 49.28 & 48.99 & 85.11 & 84.46 & 84.61 & 83.94  \\
 \hline
 \hline
 \multicolumn{5}{|c|}{Winogrande (FP32 Accuracy = 79.95\%)} & \multicolumn{4}{|c|}{Piqa (FP32 Accuracy = 81.56\%)} \\ 
 \hline
 \hline
 64 & 78.77 & 78.45 & 78.37 & 79.16 & 81.45 & 80.69 & 81.45 & 81.5 \\
 \hline
 32 & 78.45 & 79.01 & 78.69 & 80.66 & 81.56 & 80.58 & 81.18 & 81.34  \\
 \hline
 16 & 79.95 & 79.56 & 79.79 & 79.72 & 81.28 & 81.66 & 81.28 & 80.96  \\
 \hline
\end{tabular}
\caption{\label{tab:mmlu_abalation} Accuracy on LM evaluation harness tasks on Llama2-70B model.}
\end{table}

%\section{MSE Studies}
%\textcolor{red}{TODO}


\subsection{Number Formats and Quantization Method}
\label{subsec:numFormats_quantMethod}
\subsubsection{Integer Format}
An $n$-bit signed integer (INT) is typically represented with a 2s-complement format \citep{yao2022zeroquant,xiao2023smoothquant,dai2021vsq}, where the most significant bit denotes the sign.

\subsubsection{Floating Point Format}
An $n$-bit signed floating point (FP) number $x$ comprises of a 1-bit sign ($x_{\mathrm{sign}}$), $B_m$-bit mantissa ($x_{\mathrm{mant}}$) and $B_e$-bit exponent ($x_{\mathrm{exp}}$) such that $B_m+B_e=n-1$. The associated constant exponent bias ($E_{\mathrm{bias}}$) is computed as $(2^{{B_e}-1}-1)$. We denote this format as $E_{B_e}M_{B_m}$.  

\subsubsection{Quantization Scheme}
\label{subsec:quant_method}
A quantization scheme dictates how a given unquantized tensor is converted to its quantized representation. We consider FP formats for the purpose of illustration. Given an unquantized tensor $\bm{X}$ and an FP format $E_{B_e}M_{B_m}$, we first, we compute the quantization scale factor $s_X$ that maps the maximum absolute value of $\bm{X}$ to the maximum quantization level of the $E_{B_e}M_{B_m}$ format as follows:
\begin{align}
\label{eq:sf}
    s_X = \frac{\mathrm{max}(|\bm{X}|)}{\mathrm{max}(E_{B_e}M_{B_m})}
\end{align}
In the above equation, $|\cdot|$ denotes the absolute value function.

Next, we scale $\bm{X}$ by $s_X$ and quantize it to $\hat{\bm{X}}$ by rounding it to the nearest quantization level of $E_{B_e}M_{B_m}$ as:

\begin{align}
\label{eq:tensor_quant}
    \hat{\bm{X}} = \text{round-to-nearest}\left(\frac{\bm{X}}{s_X}, E_{B_e}M_{B_m}\right)
\end{align}

We perform dynamic max-scaled quantization \citep{wu2020integer}, where the scale factor $s$ for activations is dynamically computed during runtime.

\subsection{Vector Scaled Quantization}
\begin{wrapfigure}{r}{0.35\linewidth}
  \centering
  \includegraphics[width=\linewidth]{sections/figures/vsquant.jpg}
  \caption{\small Vectorwise decomposition for per-vector scaled quantization (VSQ \citep{dai2021vsq}).}
  \label{fig:vsquant}
\end{wrapfigure}
During VSQ \citep{dai2021vsq}, the operand tensors are decomposed into 1D vectors in a hardware friendly manner as shown in Figure \ref{fig:vsquant}. Since the decomposed tensors are used as operands in matrix multiplications during inference, it is beneficial to perform this decomposition along the reduction dimension of the multiplication. The vectorwise quantization is performed similar to tensorwise quantization described in Equations \ref{eq:sf} and \ref{eq:tensor_quant}, where a scale factor $s_v$ is required for each vector $\bm{v}$ that maps the maximum absolute value of that vector to the maximum quantization level. While smaller vector lengths can lead to larger accuracy gains, the associated memory and computational overheads due to the per-vector scale factors increases. To alleviate these overheads, VSQ \citep{dai2021vsq} proposed a second level quantization of the per-vector scale factors to unsigned integers, while MX \citep{rouhani2023shared} quantizes them to integer powers of 2 (denoted as $2^{INT}$).

\subsubsection{MX Format}
The MX format proposed in \citep{rouhani2023microscaling} introduces the concept of sub-block shifting. For every two scalar elements of $b$-bits each, there is a shared exponent bit. The value of this exponent bit is determined through an empirical analysis that targets minimizing quantization MSE. We note that the FP format $E_{1}M_{b}$ is strictly better than MX from an accuracy perspective since it allocates a dedicated exponent bit to each scalar as opposed to sharing it across two scalars. Therefore, we conservatively bound the accuracy of a $b+2$-bit signed MX format with that of a $E_{1}M_{b}$ format in our comparisons. For instance, we use E1M2 format as a proxy for MX4.

\begin{figure}
    \centering
    \includegraphics[width=1\linewidth]{sections//figures/BlockFormats.pdf}
    \caption{\small Comparing LO-BCQ to MX format.}
    \label{fig:block_formats}
\end{figure}

Figure \ref{fig:block_formats} compares our $4$-bit LO-BCQ block format to MX \citep{rouhani2023microscaling}. As shown, both LO-BCQ and MX decompose a given operand tensor into block arrays and each block array into blocks. Similar to MX, we find that per-block quantization ($L_b < L_A$) leads to better accuracy due to increased flexibility. While MX achieves this through per-block $1$-bit micro-scales, we associate a dedicated codebook to each block through a per-block codebook selector. Further, MX quantizes the per-block array scale-factor to E8M0 format without per-tensor scaling. In contrast during LO-BCQ, we find that per-tensor scaling combined with quantization of per-block array scale-factor to E4M3 format results in superior inference accuracy across models. 



\end{document}
