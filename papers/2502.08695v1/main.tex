\documentclass[11pt]{article}


\usepackage[final]{cvpr}
\usepackage{times}
\usepackage{epsfig}
\usepackage[utf8]{inputenc}
\usepackage{color}
\usepackage{bm}
\usepackage{amsfonts,mathtools,amsmath, amsthm}
\usepackage{graphicx}
%\usepackage{subcaption}
\usepackage{mwe}
\usepackage{makecell}
\usepackage{booktabs}
\usepackage{enumitem}
\usepackage{float} %Nate added this
\usepackage{graphicx} %Nate added this
\graphicspath{{images/}} %Nate added this

\usepackage[ruled,vlined]{algorithm2e} 
\SetKwRepeat{Do}{do}{while}%Nate added this

\definecolor{cvprblue}{rgb}{0.21,0.49,0.74}
\usepackage[pagebackref,breaklinks,colorlinks,citecolor=cvprblue]{hyperref}

\usepackage{flushend}
\usepackage{setspace}
% \usepackage[table]{xcolor}
%\usepackage[usenames,dvipsnames]{xcolor}
\usepackage{comment}
% \usepackage{todonotes}
\usepackage[capitalize]{cleveref}
\usepackage{multirow}
\providecommand\algorithmname{algorithm}
% \usepackage[thmmarks, amsmath, thref, amsthm]{ntheorem}

%\iccvfinalcopy
\def\paperID{1183} % *** Enter the Paper ID here
\def\confName{CVPR}
\def\confYear{2025}

% \def\iccvPaperID{824} % *** Enter the ICCV Paper ID here
\def\httilde{\mbox{\tt\raisebox{-.5ex}{\symbol{126}}}}

%%%%%%%%% TITLE
% \title{A Hierarchical Block QR Decomposition for Flag Recovery}
\title{A Flag Decomposition for Hierarchical Data Sets}

\author{Nathan Mankovich\\
University of Valencia
\and
Ignacio Santamaria\\
University of Cantabria
\and
Gustau Camps-Valls\\
University of Valencia
% For a paper whose authors are all at the same institution,
% omit the following lines up until the closing ``}''.
% Additional authors and addresses can be added with ``\and'',
% just like the second author.
% To save space, use either the email address or home page, not both
\and
Tolga Birdal\\
Imperial College London
}



\title{A Bayesian Nonparametric Perspective on Mahalanobis Distance for Out of Distribution Detection}
\author{
  Randolph W.~Linderman$^{1}$, Yiran Chen$^{1}$, and Scott W.~Linderman$^{2}$
}

\begin{document}
\maketitle

% Author block
\begin{figure}[!b]
\begin{minipage}[l]{\textwidth}
\footnotesize
$^{1}$ Electrical and Computer Engineering Department, Duke University, Durham, NC, USA\\[-0.5ex]
$^{2}$ Statistics Department and The Wu Tsai Neurosciences Institute, Stanford University, Stanford, CA, USA\\[-0.5ex]
Correspondence should be addressed to randolph.linderman@duke.edu and scott.linderman@stanford.edu.
\end{minipage}
\end{figure}
% End author block

\begin{abstract}
  In this work, we present a novel technique for GPU-accelerated Boolean satisfiability (SAT) sampling. Unlike conventional sampling algorithms that directly operate on conjunctive normal form (CNF), our method transforms the logical constraints of SAT problems by factoring their CNF representations into simplified multi-level, multi-output Boolean functions. It then leverages gradient-based optimization to guide the search for a diverse set of valid solutions. Our method operates directly on the circuit structure of refactored SAT instances, reinterpreting the SAT problem as a supervised multi-output regression task. This differentiable technique enables independent bit-wise operations on each tensor element, allowing parallel execution of learning processes. As a result, we achieve GPU-accelerated sampling with significant runtime improvements ranging from $33.6\times$ to $523.6\times$ over state-of-the-art heuristic samplers. We demonstrate the superior performance of our sampling method through an extensive evaluation on $60$ instances from a public domain benchmark suite utilized in previous studies. 


  
  % Generating a wide range of diverse solutions to logical constraints is crucial in software and hardware testing, verification, and synthesis. These solutions can serve as inputs to test specific functionalities of a software program or as random stimuli in hardware modules. In software verification, techniques like fuzz testing and symbolic execution use this approach to identify bugs and vulnerabilities. In hardware verification, stimulus generation is particularly vital, forming the basis of constrained-random verification. While generating multiple solutions improves coverage and increases the chances of finding bugs, high-throughput sampling remains challenging, especially with complex constraints and refined coverage criteria. In this work, we present a novel technique that enables GPU-accelerated sampling, resulting in high-throughput generation of satisfying solutions to Boolean satisfiability (SAT) problems. Unlike conventional sampling algorithms that directly operate on conjunctive normal form (CNF), our method refines the logical constraints of SAT problems by transforming their CNF into simplified multi-level Boolean expressions. It then leverages gradient-based optimization to guide the search for a diverse set of valid solutions.
  % Our method specifically takes advantage of the circuit structure of refined SAT instances by using GD to learn valid solutions, reinterpreting the SAT problem as a supervised multi-output regression task. This differentiable technique enables independent bit-wise operations on each tensor element, allowing parallel execution of learning processes. As a result, we achieve GPU-accelerated sampling with significant runtime improvements ranging from $10\times$ to $1000\times$ over state-of-the-art heuristic samplers. Specifically, we demonstrate the superior performance of our sampling method through an extensive evaluation on $60$ instances from a public domain benchmark suite utilized in previous studies.

\end{abstract}

\begin{IEEEkeywords}
Boolean Satisfiability, Gradient Descent, Multi-level Circuits, Verification, and Testing.
\end{IEEEkeywords}
\section{Introduction}\label{sec:Intro} 


Novel view synthesis offers a fundamental approach to visualizing complex scenes by generating new perspectives from existing imagery. 
This has many potential applications, including virtual reality, movie production and architectural visualization \cite{Tewari2022NeuRendSTAR}. 
An emerging alternative to the common RGB sensors are event cameras, which are  
 bio-inspired visual sensors recording events, i.e.~asynchronous per-pixel signals of changes in brightness or color intensity. 

Event streams have very high temporal resolution and are inherently sparse, as they only happen when changes in the scene are observed. 
Due to their working principle, event cameras bring several advantages, especially in challenging cases: they excel at handling high-speed motions 
and have a substantially higher dynamic range of the supported signal measurements than conventional RGB cameras. 
Moreover, they have lower power consumption and require varied storage volumes for captured data that are often smaller than those required for synchronous RGB cameras \cite{Millerdurai_3DV2024, Gallego2022}. 

The ability to handle high-speed motions is crucial in static scenes as well,  particularly with handheld moving cameras, as it helps avoid the common problem of motion blur. It is, therefore, not surprising that event-based novel view synthesis has gained attention, although color values are not directly observed.
Notably, because of the substantial difference between the formats, RGB- and event-based approaches require fundamentally different design choices. %

The first solutions to event-based novel view synthesis introduced in the literature demonstrate promising results \cite{eventnerf, enerf} and outperform non-event-based alternatives for novel view synthesis in many challenging scenarios. 
Among them, EventNeRF \cite{eventnerf} enables novel-view synthesis in the RGB space by assuming events associated with three color channels as inputs. 
Due to its NeRF-based architecture \cite{nerf}, it can handle single objects with complete observations from roughly equal distances to the camera. 
It furthermore has limitations in training and rendering speed: 
the MLP used to represent the scene requires long training time and can only handle very limited scene extents or otherwise rendering quality will deteriorate. 
Hence, the quality of synthesized novel views will degrade for larger scenes. %

We present Event-3DGS (E-3DGS), i.e.,~a new method for novel-view synthesis from event streams using 3D Gaussians~\cite{3dgs} 
demonstrating fast reconstruction and rendering as well as handling of unbounded scenes. 
The technical contributions of this paper are as follows: 
\begin{itemize}
\item With E-3DGS, we introduce the first approach for novel view synthesis from a color event camera that combines 3D Gaussians with event-based supervision. 
\item We present frustum-based initialization, adaptive event windows, isotropic 3D Gaussian regularization and 3D camera pose refinement, and demonstrate that high-quality results can be obtained. %

\item Finally, we introduce new synthetic and real event datasets for large scenes to the community to study novel view synthesis in this new problem setting. 
\end{itemize}
Our experiments demonstrate systematically superior results compared to EventNeRF \cite{eventnerf} and other baselines. 
The source code and dataset of E-3DGS are released\footnote{\url{https://4dqv.mpi-inf.mpg.de/E3DGS/}}. 





\section{Related Work}
\label{sec:relatedwork}
The OOD detection task has been widely studied and many solutions have been proposed.
For example, some approaches alter the architecture or objective of a classifier~\citep{TackJ20,HuangR21,wei2022mitigating,linderman23}, and others exploit auxiliary outlier datasets~\citep{HendrycksD19,zhang2021mixture}.
Our approach is related to a class of post-hoc methods including max softmax probability~\citep[MSP;][]{HendrycksD17}, temperature-scaled MSP~\citep{guo17tempscale}, ODIN~\citep{LiangS18}, energy-based OOD~\cite{LiuW20}, the Mahalanobis distance score~\citep[MDS;][]{lee18mds}, and the Relative MDS~\citep{ren21rmds}, which derive OOD scores from embeddings or activations of a pre-trained network.

Recently, \citet{zhang23openood15} proposed a set of Near and Far OOD benchmarks, as well as a leaderboard named OpenOOD to facilitate comparison across methods. The OpenOOD benchmarks found (1) that post-hoc methods are more scalable to large datasets, (2) there is no method that is best on all datasets, and (3) methods are sensitive to which model was used for embedding.
The best performing OpenOOD methods for vision transformer (ViT) feature embeddings are the MDS and RMDS.
The relative Mahalanobis distance score was inspired by earlier work by~\citet{ren2019likelihood} that addressed the poor performance of the OOD performance with density estimation methods.
\citet{sun2022out} propose to relax some of the assumptions of Mahalanobis distance methods by using the negative $k$-th nearest neighbor distances instead.
We will show that the relative Mahalanobis distance score (RMDS) is similar to scores derived from Bayesian nonparametric mixture models in \Cref{sec:theory}.

Bayesian nonparametric methods have previously been proposed for outlier detection and used in several applications. \citet{shotwell2011} proposed to detect outliers within datasets by partitioning data via a DPMM and identifying clusters containing a small number of samples as outliers.
\citet{varadarajan17} developed a method for detecting anomalous activity in video snippets by modeling object motion with DPMMs. Another line of work explored Dirichlet prior networks~\citep[DPN;][]{malinin2018predictive,malinin2019reverse} that explicitly model distributional uncertainty arising from dataset shift as a Dirichlet distribution over the categorical class probabilities.
More recently, \citet{Kim2024UnsupervisedOD} performed unsupervised anomaly detection through an ensemble of Gaussian DPMMs fit to random projections of a subset of datapoints.
Our work focuses on connecting DPMMs to post-hoc confidence scores and developing \textit{hierarchical} Gaussian DPMMs that share statistical strength across classes in order to estimate their high-dimensional covariance matrices.

\section{Background}
\label{sec:background}

To present our work systematically, we formulate it as a two-level hierarchical solution of Markov Decision Processes (MDPs), corresponding to the two-phase pipeline. In contrast to some hierarchical methods targeting action levels \citep{mcgovern1998hierarchical, hauskrecht2013hierarchical}, we focus on the level of reward functions.

\textbf{Low-level MDP of Controlling Problem}
Specifically, the scene reconstruction phase is to construct the MDP $\mathcal{G} = \langle \mathcal{S}, \mathcal{A}, T, \mathcal{R}_{0|1} \rangle$ from videos, where $\mathcal{S} \in \mathbb{R}^m$ represents the states of the environment, $\mathcal{A}$ is the action space of the agent, and $T$ is the transition probability function. 
$\mathcal{R}_{0|1}$ is the 0-1 reward function that distinguish whether the trajectory is successful. $\mathcal{R}_{0|1}$ can be a scalar evaluation function. 
To solve this MDP, we will train a policy $\pi$ through reinforcement learning in the Isaac Gym simulation. To achieve high performance, we leverage the LLM to sample various reward functions to learn policies in a high-level manner, based on the evaluation results from $\mathcal{R}_{0|1}$ and some CoT \citep{wei2022chain} instructions.

\textbf{High-level MDP of Reward Designs} The aim of reward design is to create a shaped reward function that simplifies the optimization of a challenging given reward function, such as the sparse 0-1 reward function $\mathcal{R}_{0|1}$. Following the definition of Reward Design Problem (RDP) from previous works \citep{singh2009rewards, ma2023eureka}, we consider a high-level MDP $\hat{\mathcal{G}} = \langle \hat{\mathcal{S}}, \hat{\mathcal{A}}, \hat{T}, F \rangle$. Here, $\hat{\mathcal{A}}$ is the space of reward functions. Each time we choose an action $\hat{R} \in \mathcal{A}$ in the MDP $\hat{\mathcal{G}}$, we will train a policy $\pi$ by RL for the low-level MDP $\mathcal{G} = \langle \mathcal{S}, \mathcal{A}, T, \hat{\mathcal{R}} + \mathcal{R}_{0|1} \rangle$. The horizon of the MDP $\hat{\mathcal{G}}$ is the iteration number in the second phase.
$\hat{\mathcal{S}}$ includes the training and evaluation information during RL and the policy model $\pi$. $\hat{T}$ is the state transition function, and $F$ is the reward function that produces a scalar evaluation of any policy $\pi$. Specifically, $F$ is equal to $\mathcal{R}_{0|1}$. Thus, the high-level MDP's goal is to find a reward function $\hat{\mathcal{R}} \in \hat{A}$ to maximize the success rates of the low-level policies.


\section{A Linear Multi-View Acyclic Model} 
\label{sec:theory}
Next, we propose a new multi-view model for causal discovery. It is a generalization of LiNGAM, but we generalize its identifiability to the Gaussian case.

\paragraph{Definition of a multi-view model for causal discovery}
We can extend the model in \eqref{eq:causal_model} to the setup where the observations are collected from different views $i \in \llbracket 1, m \rrbracket$.
In such a multi-view context, one can impose different conditions to exploit the group structure: for example, the causal order $\mP$ could be common over the views, or there might be some similarities between the disturbances.

We choose to model the disturbances as a sum of \textit{common disturbances} $\vs$ and \textit{view-specific noises} $\vn^i$, inspired by multi-view ICA in \eqref{eq:multiview_ica_model}.
We thus define a generalization of the model in~\eqref{eq:causal_model} as
\begin{align}
    \label{eq:multiview_causal_model}
    \vx^i = \mB^i \vx^i + \vs + \vn^i
\end{align}
where each $\mB^i$ is a DAG, and we make the following statistical assumptions: 1) The common disturbances $s_j$ (i.e.\ the entries of $\vs$) are all mutually independent, but no assumption is made on their marginal distributions, such as their non-Gaussianity. 2) The view-specific noises $n_j^i$ are assumed to depend on the view $i$, in the sense of satisfying $\vn^i \sim \cN(\vzero, \mSigma^i)$, where the $\mSigma^i$ are diagonal matrices. 3) The vectors $\vs$ and $\vn^i,i=1\ldots,m$ are assumed mutually independent.

\paragraph{Two scenarios for causal ordering}
We consider two different scenarios for the $\mB^i$. 
On the one hand, we can assume that the $\mB^i$ are independent of each other. That means that each $\mB^i$ can be decomposed with its own causal ordering matrix $\mP^i$ and lower diagonal matrix $\mT^i$ as
\begin{equation}
    \label{eq:decomposition_multiple_causal_orderings}
    \mB^i = (\mP^i)^\top \mT^i \mP^i
    \enspace.
\end{equation}
This we call the model with \textit{multiple causal orderings}.
On the other hand, we consider a constrained case where the causal orders are shared by all views.
This means that each $\mB^i$ is decomposed as
\begin{equation}
    \label{eq:def_B_tilde_i}
    \mB^i = \mP^\top \mT^i \mP
\end{equation}
where the causal ordering $\mP$ of the DAGs is the same across views. This latter case we call the model with \textit{shared causal ordering}.
Such a shared $\mP$ or causal order contains information about all views and can be very useful when estimating view-specific causalities is too inaccurate. If the amount of data per view is very limited, such a shared causal ordering may be the only quantity that can be reliably estimated from the data.

This completes the definition of our Linear Multi-View Acyclic Model (LiMVAM). It has two variants depending on whether the causal orderings are constrained to be shared or not.

\paragraph{Relation to Multi-view ICA}
Our model can be rewritten as a multi-view ICA model as in \eqref{eq:multiview_ica_model}, just like in the case of LiNGAM.
Thus, the methods developed for multi-view ICA can be used to estimate the matrices $\mA^i$, as well as to analyze its identifiability.
Again, it is important to note that the view-dependent mixing matrix $\mA^i$ is structured: it is specifically equal to 
$\mA^i = (\mI - \mB^i)^{-1}$; in the shared causal ordering case, even further structure is imposed.
Importantly, the theory of Shared ICA \textit{does not require the common disturbances $\vs$ to be non-Gaussian}~\citep{richard2021sharedica,anderson2013multiviewidentifiability}, as soon as there are at least 3 views and there is sufficient diversity in the view-specific noises; this condition will be made rigorous in the next section. This makes our multi-view model more widely applicable as well: we can prove its identifiability regardless of the Gaussianity of the common disturbances $\vs$.

\section{Identifiability theory}
Now we proceed to show the identifiability of the proposed model, in its different variants.
As in~\citet{richard2021sharedica}, the following assumption is central.
\begin{assumption}[Noise diversity]
    \label{assump:noise_diversity}
    Let $j$ and $j'$, $j \neq j'$, be the indices of two Gaussian common disturbances in $\vs$.
    Then, the sequences $(\mSigma^i_{jj})_{i=1, \dots, m}$ and $(\mSigma^i_{j'j'})_{i=1, \dots, m}$ are different.
\end{assumption}

The following theorem gives a fundamental identifiability result, ensuring the identifiability of the causal matrices ${\mB}^i$ under Assumption~\ref{assump:noise_diversity}.

\begin{theorem}(Identifiability of the causal matrices)          
\label{theorem:identifiability_B_tilde_i}
     Under Assumption~\ref{assump:noise_diversity}, the LiMVAM model (whether with multiple or shared causal orderings)
    is identifiable in the sense that the parameters $\mB^i$ and $\mSigma^i$ are identifiable.    
\end{theorem}
We reiterate that no assumption of non-Gaussianity is made here, only that shared Gaussian disturbances have enough noise diversity according to Assumption~\ref{assump:noise_diversity}.

Next, we proceed to the question of the identifiability of the causal ordering.
As mentioned above, the identifiability of the $\mB^i$ does not necessarily mean that their decomposition in~\eqref{eq:def_B_tilde_i} with $\mT^i$ and $\mP$ (or in~\eqref{eq:decomposition_multiple_causal_orderings} with $\mT^i$ and $\mP^i$) is unique.
Here, with possible abuse of terminology, we say that the $\mP$ is ``identifiable" if it can be uniquely defined based on the $\mB^i$ (see footnote~\ref{footnote:1} above for a discussion on this point), and we thus consider $\mP$ as if it were a parameter of the model.

It turns out that in the view-specific $\mP^i$ case, identifiability is obtained by assuming that the directed acyclic graph $\mB^i$ is dense enough in each view, as formalized in the following assumption.
However, since the $\mB^i$ can be permuted to strictly lower triangular matrices, they cannot have more than $\frac{p(p-1)}{2}$ non-zero entries.
\begin{assumption}[Dense connectivity in each view]
\label{assump:total_order_different_Pi}
    For each view $i$, the matrix $\mB^i$ has exactly $\frac{p(p-1)}{2}$ non-zero entries. 
\end{assumption}

Using Assumption~\ref{assump:total_order_different_Pi}, the following theorem states that, in addition to identifying $\mB^i$ and $\mSigma^i$, one can also identify $\mT^i$ and $\mP^i$.

\begin{theorem}(Identifiability of multiple causal orderings)
    \label{theorem:identifiability_multiview_different_Pi}
    Consider the LiMVAM model with multiple causal orderings. Consider the quantities $(\mSigma^1, \dots, \mSigma^m, \mP^1, \dots, \mP^m, \mT^1, \dots, \mT^m)$ as the set of parameters to be estimated. 
    Under Assumptions~\ref{assump:noise_diversity} and~\ref{assump:total_order_different_Pi}, all these parameters are identifiable.
\end{theorem}
However, Assumption~\ref{assump:total_order_different_Pi} may be considered rather severe because it prevents the strictly lower triangular part of $\mT^i$ matrices from being sparse.
Moreover, in the case of view-specific orderings $\mP^i$, the main quantity of interest is probably $\mB^i$ rather than $\mT^i$ and $\mP^i$, which limits the interest of Theorem~\ref{theorem:identifiability_multiview_different_Pi}.

On the other hand, the scenario of a shared $\mP$ is very interesting and specific to the multi-view model. 
In fact, the identifiability in this scenario can be obtained with weaker conditions. Consider the following assumption.

\begin{assumption}[Dense connectivity when pooled across views]
\label{assump:total_order_shared_P}
    There exists a permutation $\bar{\mP}$ such that the matrices $\bar{\mT}^i = \bar{\mP} \mB^i \bar{\mP}^\top$ are strictly lower triangular, and for each entry $(j, k)$ with $j > k$, there is at least one view $i$ such that $\bar{\mT}^i_{j, k} \neq 0$.
\end{assumption}

Under the DAG constraint with shared causal ordering, Assumption~\ref{assump:total_order_different_Pi} implies Assumption~\ref{assump:total_order_shared_P}.
So, Assumption~\ref{assump:total_order_shared_P} is much weaker than Assumption~\ref{assump:total_order_different_Pi}, as it allows the $\mT^i$ to be very sparse. This is especially the case when the number of views is large, since only one non-zero entry is required over all views.
The following theorem states the identifiability of our model in the case of a shared $\mP$ and under Assumptions~\ref{assump:noise_diversity} and~\ref{assump:total_order_shared_P}.

\begin{theorem}(Identifiability of a shared causal ordering)
\label{theorem:identifiability_multiview_shared_P}
    Consider the LiMVAM model with shared causal ordering.
    Consider the quantities $(\mSigma^1, \dots, \mSigma^m, \mP, \mT^1, \dots, \mT^m)$ as the set of parameters to be estimated.
    Under Assumptions~\ref{assump:noise_diversity} and~\ref{assump:total_order_shared_P},
    all these parameters are identifiable.
\end{theorem}
In other words, the \textit{common} causal ordering $\mP$ is now perfectly identifiable, and under weaker conditions than above. We point out that this last part of our identifiability theory is based on a very novel approach specific to the multi-view case.
The proofs of the theorems are given in Appendix~\ref{app:sec:multiview_lingam}.

\section{Hierarchical Gaussian DPMMs}
\label{sec:models}


RMDS has proven to be a highly effective outlier detection method, but it assumes that all clusters share the same covariance.
This assumption helps avoid overfitting the covariance matrices for each class~\citep{ren21rmds}, but it is not always warranted.
\Cref{fig:classwise-forstner} shows a histogram of differences between empirical covariance matrices $\hat{\Sigma}_k$ and $\hat{\Sigma}_{k'}$ for all pairs of classes $(k,k')$ in the Imagenet-1K dataset, as measured by the F\"orstner-Moonen distance~\citep{forstner_metric_2003}.

\begin{wrapfigure}[18]{r}{2.75in}
    \begin{center}
    \includegraphics[width=2.75in]{forstner-dists-vs-null.pdf}
    \caption{F\"orstner-Moonen distance between all pairs of covariance matrices from the 1000 classes of the Imagenet-1k ViT-B-16 feature space (Data) and between 1000 samples of the Wishart null distribution, $\mathrm{W}(\overline{N}, \hat{\Sigma}/\overline{N})$.
    See~\Cref{app:exploratory_details} for complete details.
    This discrepancy motivates the hierarchical models below.
    }
    \label{fig:classwise-forstner}
    \end{center}
\end{wrapfigure}

These pairwise distances are systematically larger than what we would expect under a null distribution where the true covariance matrices are the same for all classes, and the empirical estimates differ solely due to sampling variability.
Complete details of this analysis are provided in~\Cref{app:exploratory_details}.
This analysis suggests that the covariance matrices are significantly different across classes and motivates a more flexible approach.

The connection between RMDS and Gaussian DPMMs established above suggests a natural way of relaxing the tied-covariance assumption without sacrificing statistical power:
Instead of estimating covariance matrices independently, we could infer them jointly under a hierarchical Bayesian model~\citep{gelman1995bayesian}.
With such a model, we can estimate separate covariance matrices for each cluster, while sharing information via a prior.
By tuning the strength of the prior, we can obtain the tied covariance model in one limit and a fully independent model in the other.
Finally, we can estimate these hierarchical prior parameters using a simple expectation-maximization algorithm that runs in a matter of minutes, even with large, high-dimensional datasets.

\subsection{Full Covariance Model}
\label{sec:hierarchical-cov}

First, we propose a hierarchical Gaussian DPMM with full covariance matrices and a conjugate prior.
The cluster parameters, $\theta_k = (\mu_k, \Sigma_k)$, are drawn from a conjugate, normal-inverse Wishart (NIW) prior,
\begin{equation}
    p(\theta_k) = \mathrm{IW}\big(\Sigma_k \mid \nu_0, (\nu_0 - D - 1) \Sigma_0 \big)
    \times \cN\big(\mu_k \mid \mu_0, \kappa_0^{-1} \Sigma_k \big),
\end{equation}
where $\mathrm{IW}$ denotes the inverse Wishart density. Under this parameterization, $\E[\Sigma_k] = \Sigma_0$ for $\nu_0 > D + 1$.
The hyperparameters of the prior are~$\eta = (\nu_0, \kappa_0, \mu_0, \Sigma_0)$.

The most important hyperparameters are $\nu_0$ and $\Sigma_0$, as they specify the prior on covariance matrices.
As $\nu_0 \to \infty$, the prior concentrates around its mean and we recover a tied covariance model. For small values of $\nu_0$, the hierarchical model shares little strength across clusters, and the covariance estimates are effectively independent.

We propose a simple approach to estimate these hyperparameters in~\Cref{app:em-hdpmm}. Briefly, we use empirical Bayes estimates for the prior mean and covariance, setting $\mu_0 = \hat{\mu}_0$ and $\Sigma_0 = \hat{\Sigma}$. We derive an expectation-maximization~(EM) algorithm to optimize $\nu_0$ and $\kappa_0$. Thanks to the conjugacy of the model, the E-step and the M-step for $\kappa_0$ can be computed in closed form. We leverage a generalized Newton method~\citep{minka2000beyond} to update the concentration hyperparameter, $\nu_0$, effectively learning the strength of the prior to maximize the marginal likelihood of the data.

Finally, the prior and posterior predictive distributions are  multivariate Student's t distributions with closed-form densities.
The log density ratios derived from these predictive distributions form the basis of the DPMM scores, $\widetilde{C}(x)$.

\subsection{Diagonal Covariance Model}

Even with the hierarchical prior, we find that the full covariance model can still overfit to high-dimensional embeddings.
Thus, we also consider a simplified version of the hierarchical Gaussian DPMM with diagonal covariance matrices.
Here, the cluser parameters are $\theta_k = \{\mu_{k,d}, \sigma_{k,d}^2\}_{d=1}^D$, and the conjugate prior is,
\begin{equation}
    p(\theta_k) = \prod_{d=1}^D \chi^{-2}(\sigma_{k,d}^2 \mid \nu_{0,d}, \sigma_{0,d}^2)
    \times \cN(\mu_{k,d} \mid \mu_{0,d}, \kappa_{0,d}^{-1} \sigma_{k,d}^2)
\end{equation}
where $\chi^{-2}$ is the scaled inverse chi-squared density.

In addition to having fewer parameters per cluster, another advantage of this model is that it allows for
different concentration hyperparameters for each dimension, $\nu_{0,d}$. We estimate the hyperparameters using a
procedure that closely parallels the full covariance model. Likewise, the prior
and posterior predictive densities reduce to products of scalar Student's t
densities, which are even more efficient to compute.
Complete details are in~\Cref{app:em-diag-hdpmm}.

\begin{figure}[!t]
    \begin{center}
    \includegraphics[width=6.5in]{empirical-cov-diag-combined-wide.pdf}
    \vspace{-.25in}
    \caption{\textbf{A:} Diagonal of empirical covariance matrices,~$\mathrm{diag}(\hat{\Sigma}_k)$ for five randomly chosen clusters (colored lines) over dimensions. Compared to the diagonal of the average covariance matrix,~$\mathrm{diag}(\hat{\Sigma})$, individual clusters tend to have systematically larger or smaller variances than average.
    \textbf{B:} The correlation between dimensions of the deviation from the mean, $\hat{\Sigma}_k - \hat{\Sigma}$, of the diagonal components. The strong positive correlations between all but the first few dimensions indicates that the relationship observed in \textbf{A} is consistent across all clusters.}
    \label{fig:cov-analysis}
    \end{center}
\end{figure}

\subsection{Coupled Diagonal Covariance Model}

The diagonal covariance model dramatically reduces the number of parameters per cluster, but it also makes a strong assumption about the per-class covariance matrices.
Specifically, it assumes the variances, $\sigma_{k,d}^2$, are conditionally independent across dimensions.
\Cref{fig:cov-analysis} suggests that this is not the case: the diagonals of the empirical covariance matrices, $\hat{\Sigma}_k$, tend to be systematically larger
or smaller than those of the average covariance matrix, $\hat{\Sigma}$.
This analysis suggests that $\sigma_{k,d}^2$ are not independent; rather, if $\sigma_{k,d}^2$ is larger than average, then $\sigma_{k,d'}^2$ is likely to be larger as well.

We propose a novel, \emph{coupled} diagonal covariance model to capture these effects. Based on the analysis above, we introduce a scale factor $\gamma_k \in \bbR_+$ that shrinks or amplifies the variances for class~$k$ compared to the average.
In this model, the cluster parameters are $\theta_k = (\gamma_k, \{\mu_{k,d}, \sigma_{k,d}^2\}_{d=1}^D)$, and the prior is,
\begin{equation}
    p(\theta_k) =
    \chi^2(\gamma_k \mid \alpha_0)
    \prod_{d=1}^D \bigg[\chi^{-2}(\sigma_{k,d}^2 \mid \nu_{0,d}, \gamma_k \sigma_{0,d}^2)
    \times \cN(\mu_{k,d} \mid \mu_{0,d}, \kappa_{0,d}^{-1} \sigma_{k,d}^2) \bigg]
\end{equation}
where $\gamma_k$ scales the means of $\sigma_{k,d}^2$ for all dimensions $d$ in order to capture the correlations seen above.

\def\synthwidth{6.5 in}

\begin{figure*}[t]
    \captionsetup{aboveskip=.5em}
    \begin{center}
        \includegraphics[width=\synthwidth]{synthetic.pdf}
    \end{center}

    \caption{Synthetic experiments panel. Example sampled 2D dataset from DPMM with params $\nu_0=4$ (\textbf{A}) and $16$ (\textbf{B}). Each data set has $K=10$ clusters with $N_k=20$ training data points each (colored dots). We evaluate performance on classifying outliers (gray dots) drawn from the prior predictive distribution. \textbf{C:} Performance of DPMM models vs. RMDS when sweeping over $\nu_0$ with $N_k=20$ shows that DPMMs outperform when $\nu_0$ is small and there is greater variation in the $\Sigma_k$'s. \textbf{D:} Independent RMDS performance vs. DPMMs as a function of $N_k$ with $\nu_0=4$. Independent RMDS only performs well when there are adequate numbers of samples per class.}
    \label{fig:synthpanel}
\end{figure*}


Our procedure for hyperparameter estimation and computing DPMM scores is very similar to those described above.
The only complication is that with the $\gamma_k$, the posterior distribution no longer has a simple closed form.
However, for any fixed value of $\gamma_k$, the coupled model is a straightforward generalization of the diagonal model above.
Since $\gamma_k$ is a one-dimensional variable, we can use numerical quadrature to integrate over its possible values.
Likewise, we can estimate the hyperparameter $\alpha_0$ using a generalized Newton method, just like for the concentration parameter $\nu_0$.
See \Cref{app:em-coupled} for complete details of this model.

\section{Proof of Concept Experiments}
\label{sec:experiments}

%\begin{itemize}
%    \item joint exploration non e' spesso un opzione
%    \item specificare che le policy sono decentralizzate a differenza di tutti i casi precedenti
%    \item decentralizzata con feedback decentralizzato non si coordina e il problema e' abbastanza semplice da portare a policy quasi deterministiche
%\end{itemize}



%\mirco{questo primo paragrafo è un po' convoluto. Prova a ristruttura la sezione in questo modo: quali sono le domande a cui cerchiamo risposta? Quali sono i domini sperimentali? Quali sono gli algoritmi che compariamo? Quali sono i take away? Per l'ultimo potresti anche evidenziare qualche frase in grassetto o emph con le principali conclusioni empiriche}

In this section, we provide some empirical validations of the findings discussed so far. Especially, we aim to answer the following questions: (\textbf{a}) Is Algorithm~\ref{alg:trpe} actually capable of optimizing finite-trials objectives? (\textbf{b}) Do different objectives enforce different behaviors, as expected from Section~\ref{sec:problem_formulation}? (\textbf{c}) Does the \emph{clustering} behavior of mixture objectives play a crucial role? If yes, when and why?\\
Throughout the experiments, we will compare the result of optimizing finite-trial objectives, either joint, disjoint, mixture ones, through Algorithm~\ref{alg:trpe} via fully decentralized policies. The experiments will be performed with different values of the exploration horizon $T$, so as to test their capabilities in different exploration efficiency regimes.\footnote{The exploration horizon $T$, rather than being a given trajectory length, has to be seen as a parameter of the exploration phase which allows to tradeoff exploration quality with exploration efficiency.} The full implementation details are reported in Appendix~\ref{apx:exp}.
\vspace{-6pt}
\paragraph*{Experimental Domains.}~The experiments were performed on two domains. The first is a notoriously difficult multi-agent exploration task called \emph{secret room}~\citep[MPE,][]{pmlr-v139-liu21j},\footnote{We highlight that all previous efforts in this task employed centralized policies. We are interested on the role of the entropic feedback in fostering coordination rather than full-state conditioning, then maintaining fully decentralized policies instead.} referred to as  Env.~(\textbf{i}). In such task, two agents are required to reach a target while navigating over two rooms divided by a door. In order to keep the door open, at least one agent have to remain on a switch. Two switches are located at the corners of the two rooms. The hardness of the task then comes from the need of coordinated exploration, where one agent allows for the exploration of the other. The second is a simpler exploration task yet over a high dimensional state-space, namely a 2-agent instantiation of \emph{Reacher}~\citep[MaMuJoCo,][]{peng2021facmac}, referred to as Env.~(\textbf{ii}). Each agent corresponds to one joint and equipped with decentralized policies conditioned on her own states. In order to allow for the use of plug-in estimator of the entropy~\citep{paninski2003}, each state dimension was discretized over 10 bins.


\begin{figure*}[!]
    \centering
    \begin{tikzpicture}
    % Draw rounded box for the legend
    \node[draw=black, rounded corners, inner sep=2pt, fill=white] (legend) at (0,0) {
        \begin{tikzpicture}[scale=0.8]
            % Mixture
            \draw[thick, color={rgb,255:red,230; green,159; blue,0}, opacity=0.8] (0,0) -- (1,0);
            \fill[color={rgb,255:red,230; green,159; blue,0}, opacity=0.2] (0,-0.1) rectangle (1,0.1);
            \node[anchor=west, font=\scriptsize] at (1.2,0) {Mixture};
            
            % Joint
            \draw[thick, dashed, color={rgb,255:red,86; green,180; blue,233}, opacity=0.8] (2.5,0) -- (3.5,0);
            \fill[color={rgb,255:red,86; green,180; blue,233}, opacity=0.2] (2.5,-0.1) rectangle (3.5,0.1);
            \node[anchor=west, font=\scriptsize] at (3.7,0) {Joint};
            
            
            % Disjoint
            \draw[thick, dotted, color={rgb,255:red,204; green,121; blue,167}, opacity=0.8] (4.7,0) -- (5.7,0);
            \fill[color={rgb,255:red,204; green,121; blue,167}, opacity=0.2] (4.7,-0.1) rectangle (5.7,0.1);
            \node[anchor=west, font=\scriptsize] at (5.9,0) {Disjoint};
            
            % Uniform
            \draw[thick, color={rgb,255:red,153; green,153; blue,153}, opacity=0.8] (7.2,0) -- (8.2,0);
            \fill[color={rgb,255:red,153; green,153; blue,153}, opacity=0.2] (7.2,-0.1) rectangle (8.2,0.1);
            \node[anchor=west, font=\scriptsize] at (8.4,0) {Random Initialization};
        \end{tikzpicture}
    };
\end{tikzpicture}

    %\hfill
    \vfill
    %vspace{-0.2cm}
    \begin{subfigure}[b]{0.3\textwidth}
        \includegraphics[width=\textwidth]{figures/room_150_AverageReturnnokl.pdf}
        %\vspace{-0.8cm}
        \caption{\centering MA-TRPO with TRPE Pre-Training (Env.~(\textbf{i}), $T=150$).}
        \label{subfig:image9}
    \end{subfigure}
    \hfill
    \begin{subfigure}[b]{0.3\textwidth}
        \includegraphics[width=\textwidth]{figures/room_50_AverageReturnnokl.pdf}
        %\vspace{-0.8cm}
        \caption{\centering MA-TRPO with TRPE Pre-Training (Env.~(\textbf{i}), $T=50$).}
        \label{subfig:image10}
    \end{subfigure}
    \hfill
    \begin{subfigure}[b]{0.3\textwidth}
        \centering
        \includegraphics[width=0.8\textwidth]{figures/hand_100_AverageReturn.pdf}
        %\vspace{-0.8cm}
        \caption{\centering MA-TRPO with TRPE Pre-Training (Env.~(\textbf{ii}), $T=100$).}
        \label{subfig:image11}
    \end{subfigure}
\caption{\centering Effect of pre-training in sparse-reward settings.(\emph{left}) Policies initialized with either Uniform or TRPE pre-trained policies over 4 runs over a worst-case goal. (\emph{rigth}) Policies initialized with either Zero-Mean or TRPE pre-trained policies over 4 runs over 3 possible goal state. We report the average and 95\% c.i.}
\label{fig:pretraining}
\end{figure*}
\vspace{-10pt}
\paragraph*{Task-Agnostic Exploration.}~Algorithm~\ref{alg:trpe} was first tested in her ability to address task-agnostic exploration \emph{per se}. This was done by considering the well-know hard-exploration task of Env.~(\textbf{i}). The results are reported in Figure~\ref{fig:room} for a short exploration horizon $(T=50)$. Interestingly, at this efficiency regime, when looking at the joint entropy in Figure~\ref{subfig:image2}, joint and disjoint objectives perform rather well compared to mixture ones in terms of induced joint entropy, while they fail to address mixture entropy explicitly, as seen in Figure~\ref{subfig:image3}. On the other hand mixture-based objectives result in optimizing both mixture \emph{and} joint entropy effectively, as one would expect by the bounds in Th.~\ref{lem:entropymismatch}. By looking at the actual state visitation induced by the trained policies, the difference between the objectives is apparent. While optimizing joint objectives, agents exploit the high-dimensionality of the joint space to induce highly entropic distributions even without exploring the space uniformly via coordination (Fig.~\ref{subfig:image5}); the same outcome happens in disjoint objectives, with which agents focus on over-optimizing over a restricted space loosing any incentive for coordinated exploration (Fig.\ref{subfig:image6}). On the other hand, mixture objectives enforce a clustering behavior (Fig.\ref{subfig:image6}) and result in a better efficient exploration. 

\paragraph*{Policy Pre-Training via Task-Agnostic Exploration.}~More interestingly, we tested the effect of pre-training policies via different objectives as a way to alleviate the well-known hardness of sparse-reward settings, either throught faster learning or zero-short generalization. In order to do so, we employed a multi-agent counterpart of the TRPO algorithm~\cite{schulman2017trustregionpolicyoptimization} with different pre-trained policies. First, we investigated the effect on the learning curve in the hard-exploration task of Env.~(\textbf{i}) under long horizons ($T=150$), with a worst-case goal set on the the opposite corner of the closed room. Pre-training via mixture objectives still lead to a faster learning compared to initializing the policy with a uniform distribution. On the other hand, joint objective pre-training did not lead to substantial improvements over standard initializations. More interestingly, when extremely short horizons were taken into account ($T=50$) the difference became appalling, as shown in Fig.~\ref{subfig:image9}: pre-training via mixture-based objectives leaded to faster learning and higher performances, while pre-training via disjoint objectives turned out to be even \emph{harmful} (Fig.~\ref{subfig:image10}). This was motivated by the fact that the disjoint objective overfitted the task over the states reachable without coordinated exploration, resulting in almost deterministic policies, as shown in Fig~\ref{fig:333} in Appendix~\ref{apx:exp}. Finally, we tested the zero-shot capabilities of policy pre-training on the simpler but high dimensional exploration task of Env.~(\textbf{ii}), where the goal was sampled randomly between worst-case positions at the boundaries of the region reachable by the arm. As shown in Fig.~\ref{subfig:image11}, both joint and mixture were able to guarantee zero-shot performances via pre-training compatible with MA-TRPO after learning over $2$e$4$ samples, while disjoint objectives were not. On the other hand, pre-training with joint objectives showed an extremely high-variance, leading to worst-case performances not better than the ones of random initialization. Mixture objectives on the other hand showed higher stability in guaranteeing compelling zero-shot performance.
\vspace{-6pt}
\paragraph*{Take-Aways.}~Overall, the proposed proof of concepts experiments managed to answer to all of the experimental questions: (\textbf{a}) Algorithm~\ref{alg:trpe} is indeed able to explicitly optimize for finite-trial entropic objectives. Additionally, (\textbf{b}) \textbf{mixture distributions enforce diverse yet coordinated exploration}, that helps when high efficiency is required. Joint or disjoint objectives on the other hand may fail to lead to relevant solutions because of under or over optimization. Finally, (\textbf{c}) \textbf{efficient exploration} enforced by mixture distributions was shown to be a \textbf{crucial factor} not only for the sake of task-agnostic exploration per se, but also for the ability of \textbf{pre-training via task-agnostic exploration} to lead to \textbf{faster and better training} and even \textbf{zero-shot generalization}.
\section{Discussion}
\omniUIST is capable of tracking a passive tool with an accuracy of roughly 6.9 mm and, at the same time, deliver a maximum force of up to 2 N to the tool. This is enabled by our novel gradient-based approach in 3D position reconstruction that accounts for the force exerted by the electromagnet. 

Over extended periods of time, \omniUIST can comfortably produce a force of 0.615 N without the risk of overheating. In our applications, we show that \omniUIST has the potential for a wide range of usage scenarios, specifically to enrich AR and VR interactions.

\omniUIST is, however, not limited to spatial applications. We believe that \omniUIST can be a valuable addition to desktop interfaces, e.g., navigating through video editing tools or gaming. We plan to broaden \omniUIST's usage scenarios in the future.

The overall tracking performance of \omniUIST suffices for interactive applications such as the ones shown in this paper. The accuracy could be improved by adding more Hall sensors, or optimizing their placement further (e.g., placing them on the outer hull of the device).
Furthermore, a spherical tip on the passive tool that more closely resembles the dipole in our magnetic model could further improve \omniUIST's accuracy. We believe, however, that the design of \omniUIST represents a good balance of cost and complexity of manufacturing, and accuracy.

Our current implementation of \omniUIST and the accompanying tracking and actuation algorithms assumes the presence of a single passive tool. Our method, however, potentially generalizes to tracking multiple passive tools by accounting for the presence of multiple permanent magnets. This poses another interesting challenge: the magnets of multiple tools will interact with each other, i.e., attract and repel each other.The electromagnet will also jointly interact with those tools, leading to challenges in terms of computation and convergence. We believe that our gradient-based optimization can account for such interactions and plan to investigate this in the future.

In developing and testing our applications, we found that \omniUIST's current frame rate of 40 Hz suffices for many interactive scenarios. The frame rate is a trade-off between speed and accuracy. In our tests, decreasing the desired accuracy in our optimization doubled the frame rate, while resulting in errors in the 3D position estimation of more than 1 cm, however. Finding the sweet spot for this trade-off depends on the application. While our applications worked well with 40 Hz and the current accuracy, more intricate actions such as high-precision sculpting might benefit from higher frame rates \textit{and} precision.
Reducing the latency of several system components (e.g., sensor latency, convergence time) is another interesting direction of future research. 

Furthermore, the control strategy we used was fairly naïve, as it only takes the current tool position into account. A model predictive strategy could account for future states, user intent, and optimize to reduce heating. We will explore in the next chapter how model predictive approaches can be used for haptic systems.

Overall, the main benefits of \omniUIST lie in the high accuracy and large force it can produce. It does so without mechanically moving parts, which would be subject to wear.
Such wear is not the case for our device, because it is exclusively based on electromagnetic force. We believe that different form factors of \omniUIST (e.g., body-mounted, larger size) can present interesting directions of future research. \add{A body-mounted version could be interesting for VR applications in which the user moves in 3D space. The larger size could result in more discernible points.}

Additionally, the influence of strength on user perception and factors such as just-noticeable-difference will allow us to characterize the benefits and challenges of \omniUIST, and electromagnetic haptic devices in general.
We believe that \omniUIST opens interesting directions for future research in terms of novel devices, and magnetic actuation and tracking.

\subsection*{Acknowledgements}
We thank Noah Cowan for his helpful feedback on this manuscript. We also thank Jingyang Zhang for his feedback and detailed knowledge of the OpenOOD codebase.
RWL was supported in part by the National Science Foundation (NSF) under Grant No.
2112562, the U.S. Army Research Laboratory and the U.S. Army Research Office (ARL/ARO)
under grant number ARO-W911NF-23-2-0224, and the Department of Defense (DoD) through the National Defense Science \& Engineering Graduate (NDSEG) Fellowship Program.
SWL was supported in part by fellowships from the Simons Collaboration on the Global Brain, the Alfred P. Sloan Foundation, and the McKnight Foundation.

Any opinions, findings, and conclusions or recommendations expressed in this
material are those of the authors and do not necessarily reflect the views of
the NSF, the ARL/ARO, or the DoD.

\bibliography{main}
\bibliographystyle{unsrtnat}

\clearpage
\appendix
% \section{Proposed model}
% \label{section:app:model}
% Table \ref{table:define} lists the symbols and their definitions used in this paper. \par
% % \vspace{-1.5em}
% % \TSK{
% % \begin{table}[t]
\vspace{-0.5em}
\centering
\small
% \footnotesize
\caption{Symbols and definitions.}
\label{table:define}
\vspace{-1.2em}
\begin{tabular}{l|l}
\toprule
Symbol & Definition \\
\midrule
$d$ & Number of dimensions \\
$t_c$ & Current time point \\
% $N$ & Current window length \\
$\mX$ & Co-evolving multivariate data stream (semi-infinite) \\
$\mX^c$ & Current window, i.e., $\mX^c = \mX[t_m:t_c]\in\R^{d\times N}$ \\
\midrule
$h$ & Embedding dimension \\
% $\mH$ & Hankel matrix, i.e., $\begin{bmatrix}
%             \embed{\vx_1} & \embed{\vx_2} & \cdots & \embed{\vx_{n-h+1}}
%         \end{bmatrix}$ \\
$\embed{\cdot}$ & Observable for time-delay embedding, i.e., $g\colon\R \rightarrow \R^{h}$ \\
% $\mH$ & Hankel matrix of $\mX$, i.e., $\mH = [\embed{\vx_1}~\embed{\vx_2} ~ ... ~ \embed{\vx_{n-h+1}}]$ \\
$\mH$ & Hankel matrix \\
$\nmodes$ & Number of modes \\
% $\imode$ & Modes of the system for $i$-th dimension of $\mX$, i.e., $\imode \in \R^{h \times r_i}$ \\
$\modes$ & Modes of the system, i.e., $\modes \in \R^{h\times\nmodes}$ \\
% $\ieig$ & Eigenvalues of the system for $i$-th dimension of $\mX$, $i.e., \ieig \in \R^{r_i \times r_i}$ \\
$\eigs$ & Eigenvalues of the system, i.e., $\eigs \in \R^{\nmodes\times\nmodes}$ \\
$\demixing$ & Demixing matrix, i.e., $\mW = [\rowvect{w}_1, ..., \rowvect{w}_d]^\top \in \R^{d \times d}$ \\
$\mB$ & Causal adjacency matrix, i.e., $\mB \in \R^{d \times d}$ \\
\midrule
% $\vs(t)$ & Latent variables at time point $t$, i.e., $\vs(t) = \{ \vs_1(t), ..., \vs_d(t) \} $ \\
$\ind(t)$ & Inherent signal at time point $t$, i.e., $\ind(t) = \{ \ith{e}(t) \}_{i=1}^d$ \\
$\mat{S}(t)$ & Latent vectors at time point $t$, i.e., $\mat{S}(t) = \{ 
\ith{\vs}(t) \}_{i=1}^d$ \\
$\vvec(t)$ & Estimated vector at time point $t$, i.e., $\vvec(t) = \{ \ith{v}(t) \}_{i=1}^d$ \\
\midrule
$\mathcal{D}$ & Self-dynamics factor set, i.e., $\mathcal{D} = \{\modes, \eigs\}$\\
$\regime$ & Regime parameter set, i.e., $\regime = \{ \mW, \mathcal{D}_{(1)}, ..., \mathcal{D}_{(d)} \}$\\
\midrule
$R$ & Number of regimes \\
$\regimeset$ & Regime set, i.e., $\regimeset 
 = \{ \regime^1, ..., \regime^R \}$\\
 $\mathcal{B}$ & \Relation, i.e., $\mathcal{B} = \{\mB^1, ..., \mB^R\}$\\
$\updateset$ & Update parameter set,  i.e., $\updateset 
 = \{ \update^1, ..., \update^R \}$ \\
\midrule
 $\modelparam$ & Full parameter set,  i.e., $\modelparam 
 = \{ \regimeset, \updateset \}$ \\

\bottomrule
% \midrule
\end{tabular}
\normalsize
% \vspace{-2.0em}
\vspace{1.0em}
\end{table}

% % }
% \vspace{0.6em}

\section{Optimization Algorithm}
% Algorithm \ref{alg:model} is the overall procedure of \method. Algorithm \ref{alg:estimator}, namely, \modelestimator continuously updates the full parameter set $\modelparam$ and the model candidate $\candparam$, which describes the current window $\mX^c$.
\TSK{
\begin{figure}[!h]
\vspace{-5.0ex}
\begin{algorithm}[H]
    \normalsize
    \caption{\method($\vx(t_c), \modelparam, \candparam$)}
    \label{alg:model}
    \begin{algorithmic}[1]
        \STATE {\bf Input:}
        \hspace{0mm}    (a) New value $\vx(t_c)$ at time point $t_c$ \\
        \hspace{9.5mm} (b) Full parameter set $\modelparam = \{\regimeset, \updateset\}$ \\
        \hspace{9.68mm} (c) Model candidate $\candparam = \{\regime^c, \update^c, \bm{s}^c_{en}\}$
        \STATE {\bf Output:}
        \hspace{0mm}    (a) Updated full parameter set $\modelparam'$ \\
        \hspace{11.8mm} (b) Updated model candidate $\candparam'$ \\
        \hspace{11.9mm} (c) $l_s$-steps-ahead future value $\vect{v}(t_c+l_s)$ \\
        \hspace{11.8mm} (d) Causal adjacency matrix $\mB$
        \STATE /* Update current window $\mX^c$ */
        \STATE $\mX^c \leftarrow \mX[t_m : t_c]$
        \STATE /* Estimate optimal regime $\regime$ */
        \STATE $\{\modelparam', \candparam'\} \leftarrow$ \modelestimator($\mX^c, \modelparam$, $\candparam$)
        \STATE /* Forecast future value and discover causal relationship */
        \STATE $\{\vect{v}(t_c+l_s),~\mB\} \leftarrow$ \modelgenerator($\candparam'$)
        \STATE /* Update regime $\regime$ */
        \IF{NOT create new regime}
            \STATE $\candparam' \leftarrow \regimeupdate(\mX^c, \candparam')$
        \ENDIF
    \RETURN $\{\modelparam', \candparam', \vect{v}(t_c+l_s), \mB\}$
    \end{algorithmic}
\end{algorithm}
\vspace{-4.5em}
\end{figure}
}\par
\TSK{
\begin{figure}[!h]
\vspace{-0.0ex}
\begin{algorithm}[H]
    \normalsize
    \caption{\modelestimator($\mX^c, \modelparam, \candparam$)}
    \label{alg:estimator}
    \begin{algorithmic}[1]
        \STATE {\bf Input:}
        \hspace{0.0mm}  (a) Current window $\mX^c$ \\
        \hspace{9.5mm} (b) Full parameter set $\modelparam$ \\
        \hspace{9.68mm} (c) Model candidate $\candparam$
        \STATE {\bf Output:}
        \hspace{0.0mm}  (a) Updated full parameter set $\modelparam'$ \\
        \hspace{11.8mm} (b) Updated model candidate $\candparam'$
        \STATE /* Calculate optimal initial conditions */
        \STATE $\mat{S}_{0}^c \leftarrow \argmin_{\mat{S}_{0}^c} f(\mX^c; \mat{S}_{0}^c, \regime^c)$
        \IF{$f(\mX^c; \mat{S}_{0}^c, \regime^c) > \tau$}
            \STATE /* Find better regime in $\bm{\Theta}$ */
            \STATE $\{ \mat{S}_{0}^c, \regime^c \} \leftarrow \argmin_{\mat{S}_{0}^c, \regime \in \regimeset} \,f(\mX^c; \mat{S}_{0}^c, \regime^c)$
            \IF{$f(\mX^c; \mat{S}_{0}^c, \regime^c) > \tau$}
                \STATE /* Create new regime */
                \STATE $\{ \regime^c, \update^c \} \leftarrow \textsc{RegimeCreation}(\mX^c)$
                \STATE $\regimeset \leftarrow \regimeset \cup \regime^c$; $\updateset \leftarrow \updateset \cup \update^c$
            \ENDIF
        \ENDIF
        \STATE $\modelparam' \leftarrow \{\regimeset, \updateset\}$; $\candparam' \leftarrow \{\regime^c, \update^c, \mat{S}_{en}^c\}$
        \RETURN $\modelparam', \candparam'$
    \end{algorithmic}
\end{algorithm}
\vspace{-3.3em}
\end{figure}
}
% Algorithm \ref{alg:model} shows the overall procedure for \method,
% including \modelestimator (Algorithm \ref{alg:estimator}).
% \modelestimator continuously updates the full parameter set $\modelparam$ and
% the model candidate $\candparam$, which describes the current window $\mX^c$.
% \par
\label{section:app:algorithm}
% \myparaitemize{Details in Eq. \eqref{eq:update_trans}}
\subsection{Details of Eq. (\ref{eq:update_trans})}
% \myparaitemize{Details of Eq. (\ref{eq:update_trans})}
Here, we introduce the recurrence relation of transition matrix $\ith{\trans}$.
As mentioned earlier, we use the following cost function (below, index $i$ denoting $i$-th dimension is omitted for the sake of simplicity, e.g., we write $\ith{\trans}$ as $\trans$):
\begin{align*}
    \mathcal{E} &= \sum_{t'=t_m+h}^{t_c}\forgetting^{t_c-t'}||\embed{e(t')} - \trans\embed{e(t'-1)}||_2^2 \\
    &= \sum_{l=1}^h (\mat{L}(l, :) - \trans(l, :)\mat{R})\Forgetting(\mat{L}(l, :) - \trans(l, :)\mat{R})^\top
\end{align*}
where,
% $\Forgetting = diag(\forgetting^{N-2}, ..., \forgetting^0) \in \R^{(N-1) \times (N-1)}$ and
$\Forgetting, \mat{L}$ and $\mat{R}$ are synonymous with the definition in Section \ref{section:alg:creation}.
Because we want to obtain $\trans$ that minimizes this cost function $\mathcal{E}$, we differentiate it with respect to $\trans$.
\begin{align*}
    \dfrac{\partial}{\partial\trans(l, :)}\mathcal{E}
    &= -2(\mat{L}(l, :) - \trans(l, :)\mat{R}) \Forgetting \mat{R}^\top
\end{align*}
Solving the equation $\partial\mathcal{E}/\partial\trans(l, :) = 0$ for each $l$, $1 \leq l \leq h$,
the optimal solution for $\trans$ is given by
% the optimal solution is $ \trans = (\mat{L}\Forgetting\mat{R}^\top)(\mat{R}\Forgetting\mat{R}^\top)^{-1} $
$$ \trans = (\mat{L}\Forgetting\mat{R}^\top)(\mat{R}\Forgetting\mat{R}^\top)^{-1} $$
where we define
\begin{align*}
    \mat{Q} = \mat{L}\Forgetting\mat{R}^\top,\quad
    \mat{P} = (\mat{R}\Forgetting\mat{R}^\top)^{-1}
\end{align*}
The recurrence relations of $\mat{Q}$ can be written as
\begin{align*}
    \mat{Q} &= \sum_{t'=t_m+h}^{t_c}\forgetting^{t_c-t'}\embed{e(t')}\embed{e(t'-1)}^\top \\
    &= \forgetting\sum_{t'=t_m+h}^{t_c-1}\forgetting^{t_c-t'-1}\embed{e(t')}\embed{e(t'-1)}^\top + \embed{e(t_c)}\embed{e(t_c-1)}^\top
\end{align*}
\begin{align}
    \label{eq:Q}
    \therefore \mat{Q}^{new} = \forgetting\mat{Q}^{prev} + \embed{e(t_c-1)}\embed{e(t_c)}^\top
\end{align}
and similarly
\begin{align}
    \label{eq:bP}
    \mat{P}^{new} &= (\forgetting{(\mat{P}^{prev})}^{-1} + \embed{e(t_c)}\embed{e(t_c)}^\top)^{-1}
\end{align}Here, we apply the Sherman-Morrison formula~\cite{sherman1950adjustment} to the RHS of Eq. \eqref{eq:bP}.
Note that $\embed{e(t_c)}^\top\mat{P}^{prev}\embed{e(t_c)} > 0$
because $\mat{P}^{-1} = \mat{R}\Forgetting\mat{R}^\top$ is positive definite by definition.
\begin{align}
    \label{eq:P}
    \therefore \mat{P}^{new} = \frac{1}{\forgetting}(\mat{P}^{prev} - \frac{\mat{P}^{prev}\embed{e(t_c-1)}\embed{e(t_c-1)}^\top\mat{P}^{prev}}{\forgetting + \embed{e(t_c-1)}^\top\mat{P}^{prev}\embed{e(t_c-1)}})
\end{align}
Finally, combining Eq. \eqref{eq:Q} and Eq. \eqref{eq:P} gives the recurrence relations of $\trans$ for Eq. \eqref{eq:update_trans}.
\begin{align*}
    \begin{split}
        \trans^{new} &= \trans^{prev} + (\embed{e(t_c)} - \trans^{prev}\embed{e(t_c-1)}\boldsymbol\gamma \\
        \boldsymbol\gamma &= \frac{\embed{e(t_c-1)}^\top\mat{P}^{prev}}{\forgetting + \embed{e(t_c-1)}^\top\mat{P}^{prev}\embed{e(t_c-1)}}
    \end{split}
\end{align*}
    
% \end{enumerate}
\par
% \TSK{
% \begin{figure*}[t]
    \begin{tabular}{cccc}
      \hspace{-1.5em}
      \begin{minipage}[c]{0.24\linewidth}
        \centering
        % \vspace{1em}
        \includegraphics[width=\linewidth]{results/web/original1_ver1.0.pdf}
        \vspace{-2em} \\
        \hspace{1.5em}
        % (a-i) Snapshot
        % (a-i) $l_s$-steps-ahead future value forecasting
        (a-i) Original data $\mX^c$
        \label{fig:web:forecast}
      \end{minipage} &
      \hspace{-1.5em}
      \begin{minipage}[c]{0.24\linewidth}
        \centering
        \includegraphics[width=\linewidth]{results/web/latent1_ver1.2.pdf}
        \vspace{-2em} \\
        \hspace{1.5em}
        (a-ii) Inherent signals $\mE$
      \end{minipage} &
      \hspace{-1.5em}
      \begin{minipage}[c]{0.24\linewidth}
        \centering
        \includegraphics[width=0.8\linewidth]{results/web/causal1_ver1.0.pdf}
        % \vspace{-2em}
        \\
        % \hspace{2.0em}
        (a-iii) Causal relationship $\mB$
      \end{minipage} &
      \hspace{-1.5em}
      \begin{minipage}[c]{0.24\linewidth}
        \centering
        \includegraphics[width=0.95\linewidth]{results/web/mode1_ver1.0.pdf}
        % \vspace{-2em}
        \\
        \hspace{-0.7em}
        (a-iv) Latent dynamics $\eigs$
      \end{minipage} \vspace{0.5em} \\
      % \caption{(a) Snapshot at the current time point $t_c = 208$}
      \multicolumn{4}{c}{\textbf{(a) Snapshots at current time point $t_c=208$.}}
      \vspace{0.5em} \\
      \hspace{-1.5em}
      \begin{minipage}[c]{0.24\linewidth}
        \centering
        % \vspace{1em}
        \includegraphics[width=\linewidth]{results/web/original2_ver1.0.pdf}
        \vspace{-2em} \\
        \hspace{1.5em}
        % (a-i) Snapshot
        % (b-i) $l_s$-steps-ahead future value forecasting
        (b-i) Original data $\mX^c$
      \end{minipage} &
      \hspace{-1.5em}
      \begin{minipage}[c]{0.24\linewidth}
        \centering
        \includegraphics[width=\linewidth]{results/web/latent2_ver1.2.pdf}
        \vspace{-2em} \\
        \hspace{1.5em}
        (b-ii) Inherent signals $\mE$
      \end{minipage} &
      \hspace{-1.5em}
      \begin{minipage}[c]{0.24\linewidth}
        \centering
        \includegraphics[width=0.8\linewidth]{results/web/causal2_ver1.0.pdf}
        % \vspace{-2em}
        \\
        % \hspace{2.0em}
        (b-iii) Causal relationship $\mB$
      \end{minipage} &
      \hspace{-1.5em}
      \begin{minipage}[c]{0.24\linewidth}
        \centering
        \includegraphics[width=0.95\linewidth]{results/web/mode2_ver1.0.pdf}
        % \vspace{-2em} 
        \\
        \hspace{-0.7em}
        (b-iv) Latent dynamics $\eigs$
      \end{minipage} \vspace{0.5em} \\
      \multicolumn{4}{c}{\textbf{(b) Snapshots at current time point $t_c=443$.}}
    \end{tabular}
    \vspace{-1.0em}
    \caption{\method modeling for a web-click activity stream related to beer query sets (i.e., \googletrend).
      Two sets of
      % invaluable knowledge
      snapshots taken
      at two different time points
      % (i.e., $t_c = 208, 443$, respectively)
      % on December 27, 2007 (top) and July 14, 2011 (bottom)
      show:
      (a/b-i) the current window of the original data stream,
      % where, the blue right vertical and red axes represent the current and $l_s$-steps-ahead time points, (i.e., $t_c, t_c+l_s$), respectively;
      % where, the blue vertical line to the right represents the current time point $t_c$;
      % where, the blue right vertical axis represents the current time point $t_c$;
      (a/b-ii) independent signals $\mE$ specific to each observation;
      % (c) time-evolving relationships with each other based on variables generating processes (i.e., \relations) and
      (a/b-iii) causal relationships $\mB\in\mathcal{B}$ and
      (a/b-iv) interpretable latent dynamics $\eigs$,
      where the argument and the absolute value of each point correspond to
      the temporal frequency and decay rate of modes, respectively.
      }
    \label{fig:web}
    \vspace{-1.2em}
\end{figure*}
% }
\setcounter{lemma}{1}
\subsection{Proof of Lemma \ref{lemma:create_time}}
\begin{proof}
The dominant steps in \textsc{RegimeCreation} are I, IV, and VI.
The decomposition $\mX$ into $\demixing^{-1}$ and $\mE$ using ICA requires $O(d^2N)$.
For each observation,
the SVD of $\ith{\mat{R}}\mat{M}$ requires $O(h^2N)$, and the eigendecomposition of $\ith{\tilde{\trans}}$ takes $O(k_i^3)$.
The straightforward way to
process IV and VI
is to perform the calculation $d$ times sequentially, i.e., they require $O(dh^2N+\sum_ik_i^3)$ in total.
However, since these operations do not interfere with each other,
they are simultaneously computed by parallel processing.
Therefore, the time complexity of \textsc{RegimeCreation} is $O(N(d^2+h^2)+k^3)$, where $k=\max_i(k_i)$.
\end{proof}
\subsection{Proof of Lemma \ref{lemma:causal}}
\begin{proof}
First, we need to formulate the causal structure.
Here, we utilize the structural equation model~\cite{pearl2009causality}, denoted by $\mX_{\text{sem}} = \mB_{\text{sem}}\mX_{\text{sem}} + \mE_{\text{sem}}$.
Because this model is known as the general formulation of causality, if $\mB_{\text{sem}}$ in this model is identified, then it can be said that we discover causality.
In other words, we need to prove that our proposed algorithm can find the causal adjacency matrix $\mB$ aligning with this model.
Solving the structural equation model for $\mX_{\text{sem}}$, we obtain 
$\mX_{\text{sem}} = \demixing^{-1}_{\text{sem}}\mE_{\text{sem}}$
where $\demixing_{\text{sem}} = \mat{I} - \mB_{\text{sem}}$.
It is shown that we can identify $\demixing_{\text{sem}}$ in the above equation by ICA,
except for the order and scaling of the independent components, if the observed data is a linear, invertible mixture of non-Gaussian independent components~\cite{comon1994independent}.
Thus, demonstrating that \modelgenerator precisely resolves the two indeterminacies of a mixing matrix $\mW^{-1}$ (i.e., the inverse of $\demixing \in \regime^c$) suffices to complete the proof because $\demixing$ is computed by ICA in \textsc{RegimeCreation}. \par
First, we reveal that our algorithm can resolve the order indeterminacy.
We can permutate the causal adjacency matrix $\mB$ to strict lower triangularity thanks to the acyclicity assumption~\cite{bollen1989structural}.
%, which is without loss of generality.
Thus, correctly permuted and scaled $\mW$
is a lower triangular matrix with all ones on the diagonal.
It is also said that there would only be one way to permutate $\mW$, which meets the above condition~\cite{shimizu2006linear}.
Thus, \modelgenerator can identify the order of a mixing matrix by the process in step I (i.e., finding the permutation of rows of a mixing matrix that yields a matrix without any zeros on the main diagonal).
Next, with regard to the scale of indeterminacy,
it is apparent that we only need to focus on the diagonal element,
remembering that the permuted and scaled $\mW$ has all ones on the diagonal.
Therefore, we prove that \modelgenerator can resolve the order and scaling of the indeterminacies of a mixing matrix $\demixing^{-1}$.
\end{proof}
% \myparaitemize{Proof of Lemma \ref{lemma:time}} \par
\subsection{Proof of Lemma \ref{lemma:stream_time}}
\begin{proof}
For each time point, \method first runs \modelestimator,
which estimates the optimal full parameter set $\modelparam$ and the model candidate $\candparam$ for the current window $\mX^c$.
If the current regime $\regime^c$ fits well,
it takes $O(N\sum_i k_i)$ time.
Otherwise, it takes $O(RN\sum_i k_i)$ time to find a better regime in $\regimeset\in\modelparam$.
Furthermore, if \method encounters an unknown pattern,
it runs \textsc{RegimeCreation}, which takes $O(N(d^2+h^2)+k^3)$ time.
Subsequently, it runs \modelgenerator to identify the causal adjacency matrix and forecast an $l_s$-steps-ahead future value,
which takes $O(d^2)$ and $O(l_s)$ time, respectively.
Note that $l_s$ is negligible because of the small constant value.
Finally, when \method does not create a new regime,
it executes \regimeupdate, which needs $O(dh^2)$ time.
Thus, the total time complexity is at least $O(N\sum_ik_i+dh^2)$ time and at most $O(RN\sum_i k_i+N(d^2+h^2)+k^3)$ time per process.
\end{proof}

% \input{components/table_acc_app_forecast}
% \begin{table*}[t]
    % \small
    \centering
    \caption{Ablation study results with forecasting steps $l_s\in\{5, 10, 15\}$ for both synthetic and real-world datasets.}
    \vspace{-1.0em}
    \begin{tabular}{c|c|cc|cc|cc|cc|cc}
    \toprule
    % \:Datasets\:
    \multicolumn{2}{c|}{Datasets}
    % & \#0 & \#1 & \#2 & \#3 & \#4 \\
    & \multicolumn{2}{c|}{\synthetic} & \multicolumn{2}{c|}{\covid} & \multicolumn{2}{c|}{\googletrend} & \multicolumn{2}{c|}{\chickendance} & \multicolumn{2}{c}{\exercise} \\
    \midrule
    \multicolumn{2}{c|}{Metrics}
    % \:Metrics\:
    & \:RMSE & MAE\:\,
    & \:RMSE & MAE\:\,
    & \:RMSE & MAE\:\,
    & \:RMSE & MAE\:
    & \:RMSE & MAE\:\: \\
    \midrule
    \multirow[t]{3}{*}{\:\:\method (full)\:\:}
    % \:\:\method (full)\:\:
    & 5 & \:0.722 & 0.528\:\, & \:0.588 & 0.268\:\, & \:0.573 & 0.442\:\, & \:0.353 & 0.221\: & \:0.309 & 0.177\:\, \\
    & 10 & \:0.829 & 0.607\:\, & \:0.740 & 0.361\:\, & \:0.620 & 0.481\:\, & \:0.511 & 0.325\: & \:0.501 & 0.309\:\, \\
    & 15 & \:0.923 & 0.686\:\, & \:0.932 & 0.461\:\, & \:0.646 & 0.505\:\, & \:0.653 & 0.419\: & \:0.687 & 0.433\:\, \\
    \midrule
    \multirow[t]{3}{*}{\:\:w/o causality\:\:}
    & 5 & \:0.759 & 0.563\:\, & \:0.758 & 0.374\:\, & \:0.575 & 0.437\:\, & \:0.391 & 0.262\: & \:0.375 & 0.218\:\, \\
    & 10 & \:0.925 & 0.696\:\, & \:0.848 & 0.466\:\, & \:0.666 & 0.511\:\, & \:0.590 & 0.398\: & \:0.707 & 0.433\:\, \\
    & 15 & \:1.001 & 0.760\:\, & \:1.144 & 0.583\:\, & \:0.708 & 0.545\:\, & \:0.821 & 0.537\: & \:0.856 & 0.533\:\, \\
    \bottomrule
    \end{tabular}
    \label{table:ablation}
    \vspace{-0.75em}
\end{table*}

\section{Experimental Setup}
% \label{section:app:experiments}
\label{section:app:experiments:setting}
In this section, we describe the experimental setting in detail.
% \subsection{Experimental Setting}
% \myparaitemize{Experimental Setting}
We conducted all our experiments on
% \unclear{<server spec>}.
an Intel Xeon Platinum 8268 2.9GHz quad core CPU
with 512GB of memory and running Linux.
We normalized the values of each dataset based on their mean and variance (z-normalization).
The length of the current window $N$ was $50$ steps in all experiments.
\par
\myparaitemize{Generating the Datasets}
We randomly generated synthetic multivariate data streams containing multiple clusters, each of which exhibited a certain causal relationship.
For each cluster, the causal adjacency matrix $\mB$ was generated from a well-known random graph model, namely Erdös-Rényi (ER)~\cite{erdos1960evolution} with edge density $0.5$ and the number of observed variables $d$ was set at 5.
The data generation process was modeled as a structural equation model~\cite{pearl2009causality},
where each value of the causal adjacency matrix $\mB$ was sampled from a uniform distribution $\mathcal{U}(-2, -0.5)\cup(0.5, 2)$.
In addition, to demonstrate the time-changing nature of exogenous variables, 
we allowed the inherent signals variance $\sigma^2_{i, t}$ (i.e., $\ith{e}(t)\sim\text{Laplace}(0, \sigma_{i, t}^2)$)
to change over time.
Specifically, we introduced $h_{i, t}=\text{log}(\sigma^2_{i, t})$, which evolves according to an autoregressive model, where the coefficient and noise variance of the autoregressive model were sampled from $\mathcal{U}(0.8, 0.998)$ and $\mathcal{U}(0.01, 0.1)$, respectively.
% however, 

The overall data stream was then generated by constructing a temporal sequence of cluster segments and each segment had $500$ observations (e.g., ``$1,2,1$'' consists of three segments containing two types of causal relationships and its total sample size is $1,500$). We ran our experiments on five different temporal sequences: ``$1,2,1$'', ``$1,2,3$'', ``$1,2,2,1$'', ``$1,2,3,4$'', and ``$1,2,3,2,1$'' to encompass various types of real-world scenarios.
\par
\myparaitemize{Baselines}
The details of the baselines we used throughout our extensive experiments are summarized as follows:
\par\noindent
(1) Causal discovering methods
{\setlength{\leftmargini}{11pt}
\vspace{-0.3ex}
\begin{itemize}
    \item CASPER~\cite{liu2023discovering}: is a state-of-the-art method for causal discovery, integrating the graph structure into the score function and reflecting the causal distance between estimated and ground truth causal structure. We tuned the parameters by following the original paper setting.
    \item DARING~\cite{he2021daring}: introduces an adversarial learning strategy to impose an explicit residual independence constraint for causal discovery. We searched for three types of regularization penalties $\{\alpha, \beta, \gamma\}\in\{0.001, 0.01, 0.1, 1.0, 10\}$.
    % aiming to improve the learning of acyclic graphs.
    \item NoCurl~\cite{yu2021dag}: uses a two-step procedure: initialize a cyclic solution first and then employ the Hodge decomposition of graphs. We set the optimal parameter presented in the original paper.
    % and learn a DAG structure by projecting the cyclic graph to the gradient of a potential function.
    \item NOTEARS-MLP~\cite{zheng2020learning}: is an extension of NOTEARS~\cite{zheng2018dags} (mentioned below) for nonlinear settings, which aims to approximate the generative structural equation model by MLP.
    We used the default parameters provided in authors' codes\footnote[2]{\url{https://github.com/xunzheng/notears} \label{fot:notears}}.
    \item NOTEARS~\cite{zheng2018dags}:
    % is specifically designed for linear settings and
    is a differentiable optimization method with an acyclic regularization term to estimate a causal adjacency matrix.
    We used the default parameters provided in authors' codes\footref{fot:notears}.
    % estimates the true causal graph by minimizing the fixed reconstruction loss with the continuous acyclicity constraint.
    \item LiNGAM~\cite{shimizu2006linear}:
    exploits the non-Gaussianity of data to determine the direction of causal relationships. It has no parameters to set.
    % and we used the authors source codes\footnote{https://github.com/cdt15/lingam}.
    \item GES~\cite{chickering2002optimal}: is a traditional score-based bayesian algorithm that discovers causal relationships in a greedy manner.
    It has no parameters to set.
    We employed BIC as the score function and utilized the open-source in~\cite{kalainathan2020causal}.
\end{itemize}
\vspace{-0.5ex}}
\par\noindent
(2) Time series forecasting methods
{\setlength{\leftmargini}{11pt}
\vspace{-0.3ex}
\begin{itemize}
    \item TimesNet/PatchTST~\cite{wu2023timesnet, Yuqietal-2023-PatchTST}: are state-of-the-art TCN-based and Transformer-based methods, respectively.
    The past sequence length was set as 16 (to match the current window length).
    % Other parameters follow the parameter settings suggested in the original paper.
    Other parameters followed the original paper setting.
    % \item PatchTST~\cite{Yuqietal-2023-PatchTST}: is a state-of-the-art Transformer-based method for time series forecasting. The past sequence length is set as 16 for the same reason as above.
    \item DeepAR~\cite{salinas2020deepar}: models probabilistic distribution in the future, based on RNN. We built the model with 2-layer 64-unit RNNs. We used Adam optimization~\cite{adam} with a learning rate of 0.01 and let it learn for 20 epochs with early stopping.
    % to choose the best model.
    \item OrbitMap~\cite{matsubara2019dynamic}:
    % is a stream forecasting algorithm that finds important time-evolving patterns with multiple discrete non-linear dynamical systems.
    finds important time-evolving patterns for stream forecasting.
    We determined the optimal transition strength $\rho$ to minimize the forecasting error in training.
    \item ARIMA~\cite{box1976arima}: is one of the traditional time series forecasting approaches based on linear
    equations. We determined the optimal parameter set using AIC.
\end{itemize}}


\end{document}
