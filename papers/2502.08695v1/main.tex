\documentclass[11pt]{article}

\input{preamble_arxiv}

\title{A Bayesian Nonparametric Perspective on Mahalanobis Distance for Out of Distribution Detection}
\author{
  Randolph W.~Linderman$^{1}$, Yiran Chen$^{1}$, and Scott W.~Linderman$^{2}$
}

\begin{document}
\maketitle

% Author block
\begin{figure}[!b]
\begin{minipage}[l]{\textwidth}
\footnotesize
$^{1}$ Electrical and Computer Engineering Department, Duke University, Durham, NC, USA\\[-0.5ex]
$^{2}$ Statistics Department and The Wu Tsai Neurosciences Institute, Stanford University, Stanford, CA, USA\\[-0.5ex]
Correspondence should be addressed to randolph.linderman@duke.edu and scott.linderman@stanford.edu.
\end{minipage}
\end{figure}
% End author block

End-to-end imitation learning offers a promising approach for training robot policies. However, generalizing to new settings—such as unseen scenes, tasks, and object instances—remains a significant challenge. Although large-scale robot demonstration datasets have shown potential for inducing generalization, they are resource-intensive to scale. In contrast, human video data is abundant and diverse, presenting an attractive alternative. Yet, these human-video datasets lack action labels, complicating their use in imitation learning. Existing methods attempt to extract grounded action representations (e.g., hand poses), but resulting policies struggle to bridge the embodiment gap between human and robot actions.
% our approach
We propose an alternative approach: leveraging language-based reasoning from human videos - essential for guiding robot actions - to train generalizable robot policies. Building on recent advances in reasoning-based policy architectures, we introduce Reasoning through Action-free Data (RAD). RAD learns from both robot demonstration data (with reasoning and action labels) and action-free human video data (with only reasoning labels). The robot data teaches the model to map reasoning to low-level actions, while the action-free data enhances reasoning capabilities. Additionally, we will release a new dataset of 3,377 human-hand demonstrations compatible with the Bridge V2 benchmark. This dataset includes chain-of-thought reasoning annotations and hand-tracking data to help facilitate future work on reasoning-driven robot learning.
% experiments
Our experiments demonstrate that RAD enables effective transfer across the embodiment gap, allowing robots to perform tasks seen only in action-free data. Furthermore, scaling up action-free reasoning data significantly improves policy performance and generalization to novel tasks. These results highlight the promise of reasoning-driven learning from action-free datasets for advancing generalizable robot control. 
% releasing dataset
Website: \href{https://rad-generalization.github.io}{here}.

\section{Introduction}
\label{sec:intro}


\ps{Challenges of technology scaling}

The growing demand for computing performance has always been met by increasing the number of transistors per chip, which is only possible due to CMOS technology scaling.
However, as we keep pushing the boundaries of technology scaling, we encounter multiple challenges.
Firstly, whenever we transition to a more advanced technology node, the non-recurring cost due to physical design, verification, software, mask sets, and prototyping almost doubles \cite{cost-tech-node}.
As a result, designing a chip in an advanced technology node is only economically viable if the chip is manufactured in vast quantities.
Secondly, many chip components such as I/O drivers, analog circuits, or \gls{srams} have reached their scaling limit.
This means that we cannot shrink these components further, even if we use a more advanced technology with a smaller feature size.
Thirdly, advanced technology nodes suffer from high defect rates, diminishing the yield and inflating the recurring cost.
To tackle these challenges, new chip-design paradigms have been developed.

\ps{Why 2.5D integration?}

One of these new paradigms is 2.5D integration, where multiple silicon dies called chiplets are integrated into the same package.
Once designed, a single chiplet can be reused in multiple 2.5D stacked chips, which increases the ratio of production volume to non-recurring cost.
Another advantage is that multiple chiplets - fabricated in different technologies - can be integrated into the same package.
This means that only components that can take full advantage of technology scaling are built in bleeding-edge technologies.
Components that have reached their scaling limit are fabricated in more mature and hence less costly technology nodes.
Furthermore, chiplets are smaller than monolithic chips.
Therefore, manufacturing chiplets results in less silicon area loss due to fabrication defects and hence a higher yield.
Due to these economic advantages, chip vendors such as AMD \cite{amd-chiplet} and NVIDIA \cite{chiplet-book} have adopted the 2.5D integration paradigm.  

\ps{Challenges of 2.5D integration}

An important challenge when designing 2.5D stacked chips is the construction of a low-latency and high-throughput \gls{ici}. 
To build an \gls{ici}, we connect different chiplets using \gls{d2d} links.
These links are fabricated in an organic package substrate, silicon bridge, or silicon interposer, and they are connected to the chiplets using \gls{c4} bumps or microbumps.
The number of bumps per chiplet is limited, and so is the bandwidth of \gls{d2d} links.
In addition to having lower bandwidth than links in monolithic chips, \gls{d2d} links also have higher latency.
This latency is caused by wire delay and by \gls{phys} that are necessary in both the sending and the receiving chiplet.
\gls{phys} are needed to convert between protocols, voltage levels, and frequencies, which are usually different between on-chiplet links and \gls{d2d} links.
Due to these limitations, the \gls{ici} can quickly become a bottleneck.

\ps{How we solve these challenges differently than the related work does.}

Existing approaches to maximize the performance of the \gls{ici} either optimize the placement of chiplets (with potentially heterogeneous shapes) for a predetermined \gls{ici} topology 
\cite{ho,liu,seemuth,eris,osmolovskyi,tap25d,chiou}, select one topology out of a set of candidates \cite{coskun-1, coskun-2}, or they optimize the \gls{ici} topology for a 2D grid of homogeneously shaped chiplets on an active interposer \cite{butterdonut, cluscross, kite}.
To the best of our knowledge, there is no prior work on \gls{ici} topologies for chips with heterogeneously shaped chiplets or with passive silicon interposers or silicon bridges.
To fill this gap, we propose \name, a novel optimization methodology to jointly optimize the chiplet placement and \gls{ici} topology of such architectures.
\ifnb
\else
\newpage
\fi

\ps{Details on \name~and the key idea}

The key idea is as follows: 
We optimize the chiplet placement without a predetermined topology.
For each placement generated by an optimization algorithm, we infer a placement-based \gls{ici} topology by connecting chiplets that are in close proximity in that specific placement.
We then compute the latency and throughput of this combination of placement and topology for different traffic types.
These latencies and throughputs together with the total chip area are used to compute a user-defined quality-score of the placement, which is returned to the optimization algorithm.
Based on this quality score, the algorithm can further optimize the placement.
By following this iterative process, we jointly optimize the chiplet placement and the \gls{ici} topology.

\ps{Short evaluation-summary}

We provide our open-source framework implementing the proposed placement and topology co-optimization methodology, which we evaluate using both synthetic traffic and traffic traces.
A 2D grid of chiplets with a mesh topology is used as a baseline since many proposals for 2.5D stacked chips \cite{dataflow_accel_dnn, cifher, simba, hecaton, dojo} use such an architecture.
We reduce the latency of synthetic L1-to-L2 and L2-to-memory traffic, the two most important traffic types for cache coherency traffic, by up to 28\% and 62\% respectively.
For real traffic traces, we reduce the average packet latency for almost all traces and architectures considered (reduced by an 8\% or 18\% on average depending on the configuration of \gls{phys} within a chiplet).

\section{Related Work}
\label{sec:relatedwork}
The OOD detection task has been widely studied and many solutions have been proposed.
For example, some approaches alter the architecture or objective of a classifier~\citep{TackJ20,HuangR21,wei2022mitigating,linderman23}, and others exploit auxiliary outlier datasets~\citep{HendrycksD19,zhang2021mixture}.
Our approach is related to a class of post-hoc methods including max softmax probability~\citep[MSP;][]{HendrycksD17}, temperature-scaled MSP~\citep{guo17tempscale}, ODIN~\citep{LiangS18}, energy-based OOD~\cite{LiuW20}, the Mahalanobis distance score~\citep[MDS;][]{lee18mds}, and the Relative MDS~\citep{ren21rmds}, which derive OOD scores from embeddings or activations of a pre-trained network.

Recently, \citet{zhang23openood15} proposed a set of Near and Far OOD benchmarks, as well as a leaderboard named OpenOOD to facilitate comparison across methods. The OpenOOD benchmarks found (1) that post-hoc methods are more scalable to large datasets, (2) there is no method that is best on all datasets, and (3) methods are sensitive to which model was used for embedding.
The best performing OpenOOD methods for vision transformer (ViT) feature embeddings are the MDS and RMDS.
The relative Mahalanobis distance score was inspired by earlier work by~\citet{ren2019likelihood} that addressed the poor performance of the OOD performance with density estimation methods.
\citet{sun2022out} propose to relax some of the assumptions of Mahalanobis distance methods by using the negative $k$-th nearest neighbor distances instead.
We will show that the relative Mahalanobis distance score (RMDS) is similar to scores derived from Bayesian nonparametric mixture models in \Cref{sec:theory}.

Bayesian nonparametric methods have previously been proposed for outlier detection and used in several applications. \citet{shotwell2011} proposed to detect outliers within datasets by partitioning data via a DPMM and identifying clusters containing a small number of samples as outliers.
\citet{varadarajan17} developed a method for detecting anomalous activity in video snippets by modeling object motion with DPMMs. Another line of work explored Dirichlet prior networks~\citep[DPN;][]{malinin2018predictive,malinin2019reverse} that explicitly model distributional uncertainty arising from dataset shift as a Dirichlet distribution over the categorical class probabilities.
More recently, \citet{Kim2024UnsupervisedOD} performed unsupervised anomaly detection through an ensemble of Gaussian DPMMs fit to random projections of a subset of datapoints.
Our work focuses on connecting DPMMs to post-hoc confidence scores and developing \textit{hierarchical} Gaussian DPMMs that share statistical strength across classes in order to estimate their high-dimensional covariance matrices.


\section{Background}
\label{sec:background}

\paragraph{Causal ordering}
In the following, we consider a matrix $\mB \in \R^{p \times p}$ that encodes the structure of a directed acyclic graph (DAG), which means that the non-zero entries of $\mB$ represent the edges of the graph.
It follows that there exist a strictly lower triangular matrix $\mT$ and a permutation matrix $\mP$, referred to as the \emph{causal ordering}, such that $\mB = \mP^\top \mT \mP$.
In general, such an ordering is not unique.

\paragraph{LiNGAM, a model for causal discovery}
Let $\vx \in \R^p$ be a random vector of observations and let $\mB \in \R^{p \times p}$ represent a DAG.
We consider a structural equation model, also known as a functional causal model, where the data follows
\begin{align}
    \label{eq:causal_model_original}
    \vx = \mB \vx + \vs
\end{align}
where the entries $s_1,\ldots,s_p$ of the vector $\vs \in \R^p$ are independent noise terms, called disturbances.
Causal discovery consists in inferring the parameters $\mB$ of the model, from observations of $\vx$.
Yet, the identifiability of such a model and therefore the uniqueness of the inferred $\mB$ is not straightforward.
In fact, it is well-known that Gaussian noises make the model unidentifiable, in general~\citep{richardson2002identifiability,genin2021identifiability}.
A major advance was that \citet{shimizu2006linear} assumed the noise variables to be
\textit{non-Gaussian} leading to their Linear Non-Gaussian Acyclic Model (LiNGAM), which leads to identifiability, as discussed later.
In the model, we can then interpret $\mP$ as representing a reordering of the observations, such that the permuted entries $\mP \vx$ satisfy
\begin{align}
    \label{eq:causal_model}
    \mP \vx = \mT \mP \vx + \mP \vs
\end{align}
where the causal matrix between the $\mP\vx$ is now $\mT$ and thus strictly lower triangular.
This means any entry $(\mP \vx)_j$ is a weighted sum of previous entries $(\mP \vx)_{<j}$ and noise $(\mP \vs)_{j}$.
Because $(\mP \vx)_j$ does not depend on future entries $(\mP \vx)_{>j}$, we say that $\vx$ follows a causal ordering given by $\mP$.

\paragraph{Relation of LiNGAM to ICA}
The LiNGAM model, or any similar model based on~\eqref{eq:causal_model_original}, can be rewritten as a latent variable model, in particular an Independent Component Analysis (ICA) model \citep{Hyvabook} as
\begin{align}
    \label{eq:ica_model}
    \vx = \mA \vs
\end{align}
where, again, the entries in $\vs$ are independent and non-Gaussian, and the ``mixing" matrix $\mA$ expresses how the data is generated from the latents, and is given by $\mA = (\mI - \mB)^{-1}$, where $\mB$ is a DAG. Now, many methods developed for ICA can be used to estimate the matrix $\mA$, but it is important to take this special structure into account~\citep{shimizu2006linear}.
In particular, any ICA algorithm does not directly return the correct matrix $\mA$ but rather a related matrix where the columns of $\mA$ may appear in an arbitrary order.

\paragraph{Identifiability of LiNGAM} 
\citet{shimizu2006linear} showed that the LiNGAM model is identifiable in terms of the matrix $\mB$, with no indeterminacies unlike in basic ICA. A rigorous re-statement of this result is given in the following theorem which we prove for completeness in Appendix~\ref{app:ssec:lingam}.
\begin{theorem}[Identifiability of LiNGAM]
\label{theorem:lingam}
In the statistical model defined by~\eqref{eq:causal_model_original}, the parameter $\mB$ is identifiable, provided that the entries in $\vs$ are mutually independent, that at most one of them is Gaussian, and that $\mB$ is a DAG.
\end{theorem}
A further question that has received less attention is whether the model is identifiable in terms of the causal ordering $\mP$. In fact, it is not in general: specifically, there may exist many permutation matrices $\mP$ and strictly lower triangular matrices $\mT$ such that the generated data has the same distribution and the generating permutation cannot be identified.
For instance, in the degenerate case $\mB = \mT = \vzero$, any permutation matrix $\mP$ is equally valid and gives the same data distribution. As is well-known, a DAG in general defines only a partial order in the sense that for some pairs of variables, we cannot necessarily say which is ``earlier" and which is ``later". Thus, to make the causal ordering well-defined, we need further assumptions, as will be considered below.\footnote{One would argue that in some cases, $\mP$ is not even well-defined, and thus it cannot be identifiable, and it is not appropriate to use the concept of identiability. We prefer here to confound the concepts in the sense that we talk about identifiability of $\mP$ if it is uniquely defined and can be uniquely recovered from the data.
A more rigorous justification could be developed by assuming a hierarchical data generation process, where the $\mP$ and $\mT$ are generated first, and the $\mB$ is generated based on them. In that case, if the decomposition of $\mB$ is unique, and $\mB$ is identifiable, also $\mP$ is identifiable in the conventional statistical sense.\label{footnote:1}}


\paragraph{Multi-view ICA}
A multi-view version of ICA is of great practical interest.  One might obtain a number of views of the same data that might be, for example, different subjects in a biomedical context, different users in more technological applications, or different measurement systems of the same physical phenomenon. 
A multi-view extension of ICA
can then be defined in various ways. Here, we consider the case where the components are, at least partly, shared over views, while the mixing matrices (as well as optional noise terms) are view-dependent. 

This leads to the definition of Shared ICA \citep{richard2021sharedica,anderson2013multiviewidentifiability}:
\begin{align}
    \label{eq:multiview_ica_model}
    \vx^i = \mA^i (\vs + \vn^i)
\end{align}
where $\vx^i$ are the different views indexed by the view index $i$.
Recently, identifiability conditions have been explored for such multi-view ICA models. 
On the one hand, it is obvious that if the components are non-Gaussian, the Shared ICA model reduces to a standard ICA model when stacking the different views in a single vector, and hence identifiable. But \citet{richard2021sharedica,anderson2013multiviewidentifiability} showed the surprising result that even if more than one component is Gaussian, the model can still be identifiable if the variances of the noises $\vn^i$ are sufficiently diverse.

\section{Theory: Connecting Relative Mahalanobis Distance and DPMMs}
\label{sec:theory}
Here we show that a widely used outlier detection method called the relative Mahalanobis distance score~\citep[RMDS;][]{ren21rmds} is closely related to the outlier probabilities obtained using a Gaussian DPMM with tied covariances. RMDS outputs a score, $C(x)$, where smaller values indicate that a data point $x$ is more likely to be an outlier.
The RMDS score of a new point $x$ is defined as follows,\footnote{We flip the sign of $\mathrm{RMD}_k(x)$ compared to~\citet{ren21rmds}, but account for it in the definition of $C(x)$ so that the resulting score is unchanged. Our presentation is in keeping with the definition of the MDS score~\citep{lee18mds}.}
\begin{align}
    \nonumber
    \mathrm{MD}_0(x) &= (x - \hat{\mu}_0)^\top \hat{\Sigma}_0^{-1} (x - \hat{\mu}_0) \\
    \nonumber
    \mathrm{MD}_k(x) &= (x - \hat{\mu}_k)^\top \hat{\Sigma}^{-1} (x - \hat{\mu}_k) \\
    \nonumber
    \mathrm{RMD}_k(x) &= \mathrm{MD}_0(x) - \mathrm{MD}_k(x)\\
    C(x) &= \max_k \; \mathrm{RMD}_k(x),
\end{align}
where $\mathrm{MD}_0(x)$ and $\mathrm{MD}_k(x)$ are squared Mahalanobis distances, $\hat{\mu}_0$ and $\hat{\Sigma}_0$ are the sample mean and covariance of the data, $\hat{\mu}_k$ is the sample mean of cluster $k$, and $\hat{\Sigma} = \tfrac{1}{N} \sum_{n=1}^N (x_n - \hat{\mu}_{y_n})(x_n - \hat{\mu}_{y_n})^\top$ is the sample within-class covariance.

\citet{ren21rmds} motivated this score in terms of log density ratios between a Gaussian distribution for each cluster and a Gaussian ``background'' model.
Specifically,
\begin{align}
    \label{eq:rmdk}
    \mathrm{RMD}_k(x) &= 2 \log \frac{\cN(x \mid \hat{\mu}_k, \hat{\Sigma})}{\cN(x \mid \hat{\mu}_0, \hat{\Sigma}_0)} + d,
\end{align}
where $d=\log |\hat{\Sigma}| - \log|\hat{\Sigma}_0|$ does not depend on $x$ or $k$.
Larger values of $\mathrm{RMD}_k(x)$ indicate that $x$ is more likely under cluster $k$ than under the background model.

The procedure for mapping $\mathrm{RMD}_k(x)$ values to the score~$C(x)$ is inherited from the Mahalanobis distance score~\citep[MDS;][]{lee18mds}. If the log density ratio is negative for all $k$, then the background model is more likely than \textit{all} of the existing clusters, and hence $x$ is likely to be an outlier.
Propositions~\ref{prop:rmds_dpmm} and~\ref{prop:rmds_tied_dpmm} show that a very similar computation is at work in the calculation of outlier probabilities for DPMMs.

First, we show that the \emph{inlier} probabilities under a general DPMM (not necessarily Gaussian) can be expressed in terms of a quantity analogous to $C(x)$.
\begin{proposition}
\label{prop:rmds_dpmm}
The \emph{inlier} probability of a general DPMM with concentration $\alpha$ can be expressed as follows,
\begin{align}
    p(y \in [K] \mid x, \cD)
    &= \sigma(\widetilde{C}(x) - \log \nicefrac{\alpha}{\overline{N}})
\end{align}
where $\sigma(u) = (1 + e^{-u})^{-1}$ is the logistic (sigmoid) function, ${\overline{N}=\tfrac{1}{K} \sum_k N_k}$ is the average cluster size,
and
\begin{align}
    \nonumber \widetilde{C}(x) &= \log \sum_{k=1}^K e^{\lambda_k + \log \nicefrac{N_k}{\overline{N}}} \\
    \label{eq:Ctilde}
    &= \mathrm{softmax}_k \left\{ \lambda_k + \log \nicefrac{N_k}{\overline{N}} \right\}, \\
    \lambda_k &= \log \frac{p(x \mid \cD_k)}{p(x)}.
\end{align}
Here, $\lambda_k$ is the log density ratio of the posterior predictive and prior predictive distributions from eq.~\eqref{eq:outlier_prob}.
\end{proposition}

\begin{proof}
The inlier probability is one minus the outlier probability.
Normalizing the outlier probability in eq.~\eqref{eq:outlier_prob} and rearranging, we can write the inlier probability as,
\begin{align}
    \nonumber p(y \in [K] \mid x, \cD) &=
    1 - \frac{\alpha p(x)}{\alpha p(x) + \sum_{k=1}^K N_k p(x \mid \cD_k)} \\
    \nonumber &=
    1 - \left( 1 +  \sum_{k=1}^K \frac{\nicefrac{N_k}{\overline{N}} \, p(x \mid \cD_k)}{\nicefrac{\alpha}{\overline{N}} \, p(x)} \right)^{-1} \\
    \nonumber &= 1 - \left(1 + e^{\widetilde{C}(x) - \log \nicefrac{\alpha}{\overline{N}}}\right)^{-1} \\
    &= \sigma(\widetilde{C}(x) - \log \nicefrac{\alpha}{\overline{N}})
\end{align}
where $\widetilde{C}(x)$ is defined in eq.~\eqref{eq:Ctilde} and the last line follows from the fact that $1 - \sigma(-u) = \sigma(u)$.
\end{proof}

This proposition says that the log-odds of data point $x$ belonging to an existing cluster is the difference of a \textit{DPMM score},~$\widetilde{C}(x)$, which is analogous to the relative Mahalanobis score, and a \textit{threshold},~$\log \nicefrac{\alpha}{\overline{N}}$, which is tuned by the hyperparameter $\alpha$.

Next, we show that in certain regimes, the DPMM score from a Gaussian DPMM with tied covariance is almost perfectly correlated with the RMDS.

\begin{proposition}
\label{prop:rmds_tied_dpmm}
If the average cluster covariance $\hat{\Sigma}$ is much smaller than the population covariance $\hat{\Sigma}_0$ and all clusters are of size $N_k = \overline{N} \gg 1$, then a Gaussan DPMM with tied covariance (c.f. Sec.~\ref{sec:tied-cov}) and hyperparameters $\eta = (\hat{\mu}_0, \hat{\Sigma}_0, \hat{\Sigma})$ will have,
\begin{align}
    \widetilde{C}(x) \approx \tfrac{1}{2} C(x) - d,
\end{align}
where $C(x)$ is the RMDS and $d$ is an additive constant, as defined in eq.~\eqref{eq:rmdk}.
\end{proposition}

\begin{proof}
In the regime where~${\hat{\Sigma} \ll \hat{\Sigma}_0}$, the posterior mean is approximately $\mu_k' \approx \hat{\mu}_k$, the
posterior variance is approximately $\Sigma_k' \approx N_k^{-1} \hat{\Sigma}$, and the prior predictive variance is approximately $\hat{\Sigma}_0 + \hat{\Sigma} \approx \hat{\Sigma}_0$. Furthermore, since $N_k = \overline{N} \gg 1$, it follows that $\Sigma_k' + \hat{\Sigma} \approx \tfrac{\overline{N} + 1}{\overline{N}} \hat{\Sigma} \approx \hat{\Sigma}$. Thus,
\begin{align}
    \nonumber
    \lambda_k
    &= \log \frac{\cN(x \mid \mu_k', \Sigma_k' + \hat{\Sigma})}{\cN(x \mid \hat{\mu}_0, \hat{\Sigma}_0 + \hat{\Sigma})} \\
    \nonumber
    &\approx \log \frac{\cN(x \mid \hat{\mu}_k, \hat{\Sigma})}{\cN(x \mid \hat{\mu}_0, \hat{\Sigma}_0)} \\
    &= \tfrac{1}{2}\mathrm{RMD}_k(x) - d,
\end{align}
Finally, since $N_k = \overline{N}$, and since the softmax is approximately equal to the maximum,
\begin{align}
    \nonumber
    \widetilde{C}(x)
    &= \mathrm{softmax}_k \{\lambda_k\} \\
    \nonumber
    & \approx \mathrm{softmax}_k \left\{\tfrac{1}{2} \mathrm{RMD}_k(x)  - d \right\} \\
    \nonumber
    & \approx \max_k \left\{\tfrac{1}{2} \mathrm{RMD}_k(x) -d \right\} \\
    &= \tfrac{1}{2} C(x) - d,
\end{align}
which completes the proof.
\end{proof}

This proposition establishes the close correspondence between the relative Mahalanobis distance score and the log-odds that a point is an inlier under a Gaussian DPMM with tied covariance.
We show that this correspondence holds in practice in the experiments below and in \Cref{app:rmds_dpmm_corr}.
This correspondence provides further support for using RMDS for outlier detection, beyond the original motivation in terms of log likelihood ratios.
However, from this perspective, we also recognize several natural generalizations of RMDS that could improve outlier detection through richer DPMMs.
We present three such generalizations below.

\section{Hierarchical Gaussian DPMMs}
\label{sec:models}


RMDS has proven to be a highly effective outlier detection method, but it assumes that all clusters share the same covariance.
This assumption helps avoid overfitting the covariance matrices for each class~\citep{ren21rmds}, but it is not always warranted.
\Cref{fig:classwise-forstner} shows a histogram of differences between empirical covariance matrices $\hat{\Sigma}_k$ and $\hat{\Sigma}_{k'}$ for all pairs of classes $(k,k')$ in the Imagenet-1K dataset, as measured by the F\"orstner-Moonen distance~\citep{forstner_metric_2003}.

\begin{wrapfigure}[18]{r}{2.75in}
    \begin{center}
    \includegraphics[width=2.75in]{forstner-dists-vs-null.pdf}
    \caption{F\"orstner-Moonen distance between all pairs of covariance matrices from the 1000 classes of the Imagenet-1k ViT-B-16 feature space (Data) and between 1000 samples of the Wishart null distribution, $\mathrm{W}(\overline{N}, \hat{\Sigma}/\overline{N})$.
    See~\Cref{app:exploratory_details} for complete details.
    This discrepancy motivates the hierarchical models below.
    }
    \label{fig:classwise-forstner}
    \end{center}
\end{wrapfigure}

These pairwise distances are systematically larger than what we would expect under a null distribution where the true covariance matrices are the same for all classes, and the empirical estimates differ solely due to sampling variability.
Complete details of this analysis are provided in~\Cref{app:exploratory_details}.
This analysis suggests that the covariance matrices are significantly different across classes and motivates a more flexible approach.

The connection between RMDS and Gaussian DPMMs established above suggests a natural way of relaxing the tied-covariance assumption without sacrificing statistical power:
Instead of estimating covariance matrices independently, we could infer them jointly under a hierarchical Bayesian model~\citep{gelman1995bayesian}.
With such a model, we can estimate separate covariance matrices for each cluster, while sharing information via a prior.
By tuning the strength of the prior, we can obtain the tied covariance model in one limit and a fully independent model in the other.
Finally, we can estimate these hierarchical prior parameters using a simple expectation-maximization algorithm that runs in a matter of minutes, even with large, high-dimensional datasets.

\subsection{Full Covariance Model}
\label{sec:hierarchical-cov}

First, we propose a hierarchical Gaussian DPMM with full covariance matrices and a conjugate prior.
The cluster parameters, $\theta_k = (\mu_k, \Sigma_k)$, are drawn from a conjugate, normal-inverse Wishart (NIW) prior,
\begin{equation}
    p(\theta_k) = \mathrm{IW}\big(\Sigma_k \mid \nu_0, (\nu_0 - D - 1) \Sigma_0 \big)
    \times \cN\big(\mu_k \mid \mu_0, \kappa_0^{-1} \Sigma_k \big),
\end{equation}
where $\mathrm{IW}$ denotes the inverse Wishart density. Under this parameterization, $\E[\Sigma_k] = \Sigma_0$ for $\nu_0 > D + 1$.
The hyperparameters of the prior are~$\eta = (\nu_0, \kappa_0, \mu_0, \Sigma_0)$.

The most important hyperparameters are $\nu_0$ and $\Sigma_0$, as they specify the prior on covariance matrices.
As $\nu_0 \to \infty$, the prior concentrates around its mean and we recover a tied covariance model. For small values of $\nu_0$, the hierarchical model shares little strength across clusters, and the covariance estimates are effectively independent.

We propose a simple approach to estimate these hyperparameters in~\Cref{app:em-hdpmm}. Briefly, we use empirical Bayes estimates for the prior mean and covariance, setting $\mu_0 = \hat{\mu}_0$ and $\Sigma_0 = \hat{\Sigma}$. We derive an expectation-maximization~(EM) algorithm to optimize $\nu_0$ and $\kappa_0$. Thanks to the conjugacy of the model, the E-step and the M-step for $\kappa_0$ can be computed in closed form. We leverage a generalized Newton method~\citep{minka2000beyond} to update the concentration hyperparameter, $\nu_0$, effectively learning the strength of the prior to maximize the marginal likelihood of the data.

Finally, the prior and posterior predictive distributions are  multivariate Student's t distributions with closed-form densities.
The log density ratios derived from these predictive distributions form the basis of the DPMM scores, $\widetilde{C}(x)$.

\subsection{Diagonal Covariance Model}

Even with the hierarchical prior, we find that the full covariance model can still overfit to high-dimensional embeddings.
Thus, we also consider a simplified version of the hierarchical Gaussian DPMM with diagonal covariance matrices.
Here, the cluser parameters are $\theta_k = \{\mu_{k,d}, \sigma_{k,d}^2\}_{d=1}^D$, and the conjugate prior is,
\begin{equation}
    p(\theta_k) = \prod_{d=1}^D \chi^{-2}(\sigma_{k,d}^2 \mid \nu_{0,d}, \sigma_{0,d}^2)
    \times \cN(\mu_{k,d} \mid \mu_{0,d}, \kappa_{0,d}^{-1} \sigma_{k,d}^2)
\end{equation}
where $\chi^{-2}$ is the scaled inverse chi-squared density.

In addition to having fewer parameters per cluster, another advantage of this model is that it allows for
different concentration hyperparameters for each dimension, $\nu_{0,d}$. We estimate the hyperparameters using a
procedure that closely parallels the full covariance model. Likewise, the prior
and posterior predictive densities reduce to products of scalar Student's t
densities, which are even more efficient to compute.
Complete details are in~\Cref{app:em-diag-hdpmm}.

\begin{figure}[!t]
    \begin{center}
    \includegraphics[width=6.5in]{empirical-cov-diag-combined-wide.pdf}
    \vspace{-.25in}
    \caption{\textbf{A:} Diagonal of empirical covariance matrices,~$\mathrm{diag}(\hat{\Sigma}_k)$ for five randomly chosen clusters (colored lines) over dimensions. Compared to the diagonal of the average covariance matrix,~$\mathrm{diag}(\hat{\Sigma})$, individual clusters tend to have systematically larger or smaller variances than average.
    \textbf{B:} The correlation between dimensions of the deviation from the mean, $\hat{\Sigma}_k - \hat{\Sigma}$, of the diagonal components. The strong positive correlations between all but the first few dimensions indicates that the relationship observed in \textbf{A} is consistent across all clusters.}
    \label{fig:cov-analysis}
    \end{center}
\end{figure}

\subsection{Coupled Diagonal Covariance Model}

The diagonal covariance model dramatically reduces the number of parameters per cluster, but it also makes a strong assumption about the per-class covariance matrices.
Specifically, it assumes the variances, $\sigma_{k,d}^2$, are conditionally independent across dimensions.
\Cref{fig:cov-analysis} suggests that this is not the case: the diagonals of the empirical covariance matrices, $\hat{\Sigma}_k$, tend to be systematically larger
or smaller than those of the average covariance matrix, $\hat{\Sigma}$.
This analysis suggests that $\sigma_{k,d}^2$ are not independent; rather, if $\sigma_{k,d}^2$ is larger than average, then $\sigma_{k,d'}^2$ is likely to be larger as well.

We propose a novel, \emph{coupled} diagonal covariance model to capture these effects. Based on the analysis above, we introduce a scale factor $\gamma_k \in \bbR_+$ that shrinks or amplifies the variances for class~$k$ compared to the average.
In this model, the cluster parameters are $\theta_k = (\gamma_k, \{\mu_{k,d}, \sigma_{k,d}^2\}_{d=1}^D)$, and the prior is,
\begin{equation}
    p(\theta_k) =
    \chi^2(\gamma_k \mid \alpha_0)
    \prod_{d=1}^D \bigg[\chi^{-2}(\sigma_{k,d}^2 \mid \nu_{0,d}, \gamma_k \sigma_{0,d}^2)
    \times \cN(\mu_{k,d} \mid \mu_{0,d}, \kappa_{0,d}^{-1} \sigma_{k,d}^2) \bigg]
\end{equation}
where $\gamma_k$ scales the means of $\sigma_{k,d}^2$ for all dimensions $d$ in order to capture the correlations seen above.

\def\synthwidth{6.5 in}

\begin{figure*}[t]
    \captionsetup{aboveskip=.5em}
    \begin{center}
        \includegraphics[width=\synthwidth]{synthetic.pdf}
    \end{center}

    \caption{Synthetic experiments panel. Example sampled 2D dataset from DPMM with params $\nu_0=4$ (\textbf{A}) and $16$ (\textbf{B}). Each data set has $K=10$ clusters with $N_k=20$ training data points each (colored dots). We evaluate performance on classifying outliers (gray dots) drawn from the prior predictive distribution. \textbf{C:} Performance of DPMM models vs. RMDS when sweeping over $\nu_0$ with $N_k=20$ shows that DPMMs outperform when $\nu_0$ is small and there is greater variation in the $\Sigma_k$'s. \textbf{D:} Independent RMDS performance vs. DPMMs as a function of $N_k$ with $\nu_0=4$. Independent RMDS only performs well when there are adequate numbers of samples per class.}
    \label{fig:synthpanel}
\end{figure*}


Our procedure for hyperparameter estimation and computing DPMM scores is very similar to those described above.
The only complication is that with the $\gamma_k$, the posterior distribution no longer has a simple closed form.
However, for any fixed value of $\gamma_k$, the coupled model is a straightforward generalization of the diagonal model above.
Since $\gamma_k$ is a one-dimensional variable, we can use numerical quadrature to integrate over its possible values.
Likewise, we can estimate the hyperparameter $\alpha_0$ using a generalized Newton method, just like for the concentration parameter $\nu_0$.
See \Cref{app:em-coupled} for complete details of this model.

\section{Proof of Concept Experiments}
\label{sec:experiments}

%\begin{itemize}
%    \item joint exploration non e' spesso un opzione
%    \item specificare che le policy sono decentralizzate a differenza di tutti i casi precedenti
%    \item decentralizzata con feedback decentralizzato non si coordina e il problema e' abbastanza semplice da portare a policy quasi deterministiche
%\end{itemize}



%\mirco{questo primo paragrafo è un po' convoluto. Prova a ristruttura la sezione in questo modo: quali sono le domande a cui cerchiamo risposta? Quali sono i domini sperimentali? Quali sono gli algoritmi che compariamo? Quali sono i take away? Per l'ultimo potresti anche evidenziare qualche frase in grassetto o emph con le principali conclusioni empiriche}

In this section, we provide some empirical validations of the findings discussed so far. Especially, we aim to answer the following questions: (\textbf{a}) Is Algorithm~\ref{alg:trpe} actually capable of optimizing finite-trials objectives? (\textbf{b}) Do different objectives enforce different behaviors, as expected from Section~\ref{sec:problem_formulation}? (\textbf{c}) Does the \emph{clustering} behavior of mixture objectives play a crucial role? If yes, when and why?\\
Throughout the experiments, we will compare the result of optimizing finite-trial objectives, either joint, disjoint, mixture ones, through Algorithm~\ref{alg:trpe} via fully decentralized policies. The experiments will be performed with different values of the exploration horizon $T$, so as to test their capabilities in different exploration efficiency regimes.\footnote{The exploration horizon $T$, rather than being a given trajectory length, has to be seen as a parameter of the exploration phase which allows to tradeoff exploration quality with exploration efficiency.} The full implementation details are reported in Appendix~\ref{apx:exp}.
\vspace{-6pt}
\paragraph*{Experimental Domains.}~The experiments were performed on two domains. The first is a notoriously difficult multi-agent exploration task called \emph{secret room}~\citep[MPE,][]{pmlr-v139-liu21j},\footnote{We highlight that all previous efforts in this task employed centralized policies. We are interested on the role of the entropic feedback in fostering coordination rather than full-state conditioning, then maintaining fully decentralized policies instead.} referred to as  Env.~(\textbf{i}). In such task, two agents are required to reach a target while navigating over two rooms divided by a door. In order to keep the door open, at least one agent have to remain on a switch. Two switches are located at the corners of the two rooms. The hardness of the task then comes from the need of coordinated exploration, where one agent allows for the exploration of the other. The second is a simpler exploration task yet over a high dimensional state-space, namely a 2-agent instantiation of \emph{Reacher}~\citep[MaMuJoCo,][]{peng2021facmac}, referred to as Env.~(\textbf{ii}). Each agent corresponds to one joint and equipped with decentralized policies conditioned on her own states. In order to allow for the use of plug-in estimator of the entropy~\citep{paninski2003}, each state dimension was discretized over 10 bins.


\begin{figure*}[!]
    \centering
    \input{figures/pretraining_legend.tex}
    %\hfill
    \vfill
    %vspace{-0.2cm}
    \begin{subfigure}[b]{0.3\textwidth}
        \includegraphics[width=\textwidth]{figures/room_150_AverageReturnnokl.pdf}
        %\vspace{-0.8cm}
        \caption{\centering MA-TRPO with TRPE Pre-Training (Env.~(\textbf{i}), $T=150$).}
        \label{subfig:image9}
    \end{subfigure}
    \hfill
    \begin{subfigure}[b]{0.3\textwidth}
        \includegraphics[width=\textwidth]{figures/room_50_AverageReturnnokl.pdf}
        %\vspace{-0.8cm}
        \caption{\centering MA-TRPO with TRPE Pre-Training (Env.~(\textbf{i}), $T=50$).}
        \label{subfig:image10}
    \end{subfigure}
    \hfill
    \begin{subfigure}[b]{0.3\textwidth}
        \centering
        \includegraphics[width=0.8\textwidth]{figures/hand_100_AverageReturn.pdf}
        %\vspace{-0.8cm}
        \caption{\centering MA-TRPO with TRPE Pre-Training (Env.~(\textbf{ii}), $T=100$).}
        \label{subfig:image11}
    \end{subfigure}
\caption{\centering Effect of pre-training in sparse-reward settings.(\emph{left}) Policies initialized with either Uniform or TRPE pre-trained policies over 4 runs over a worst-case goal. (\emph{rigth}) Policies initialized with either Zero-Mean or TRPE pre-trained policies over 4 runs over 3 possible goal state. We report the average and 95\% c.i.}
\label{fig:pretraining}
\end{figure*}
\vspace{-10pt}
\paragraph*{Task-Agnostic Exploration.}~Algorithm~\ref{alg:trpe} was first tested in her ability to address task-agnostic exploration \emph{per se}. This was done by considering the well-know hard-exploration task of Env.~(\textbf{i}). The results are reported in Figure~\ref{fig:room} for a short exploration horizon $(T=50)$. Interestingly, at this efficiency regime, when looking at the joint entropy in Figure~\ref{subfig:image2}, joint and disjoint objectives perform rather well compared to mixture ones in terms of induced joint entropy, while they fail to address mixture entropy explicitly, as seen in Figure~\ref{subfig:image3}. On the other hand mixture-based objectives result in optimizing both mixture \emph{and} joint entropy effectively, as one would expect by the bounds in Th.~\ref{lem:entropymismatch}. By looking at the actual state visitation induced by the trained policies, the difference between the objectives is apparent. While optimizing joint objectives, agents exploit the high-dimensionality of the joint space to induce highly entropic distributions even without exploring the space uniformly via coordination (Fig.~\ref{subfig:image5}); the same outcome happens in disjoint objectives, with which agents focus on over-optimizing over a restricted space loosing any incentive for coordinated exploration (Fig.\ref{subfig:image6}). On the other hand, mixture objectives enforce a clustering behavior (Fig.\ref{subfig:image6}) and result in a better efficient exploration. 

\paragraph*{Policy Pre-Training via Task-Agnostic Exploration.}~More interestingly, we tested the effect of pre-training policies via different objectives as a way to alleviate the well-known hardness of sparse-reward settings, either throught faster learning or zero-short generalization. In order to do so, we employed a multi-agent counterpart of the TRPO algorithm~\cite{schulman2017trustregionpolicyoptimization} with different pre-trained policies. First, we investigated the effect on the learning curve in the hard-exploration task of Env.~(\textbf{i}) under long horizons ($T=150$), with a worst-case goal set on the the opposite corner of the closed room. Pre-training via mixture objectives still lead to a faster learning compared to initializing the policy with a uniform distribution. On the other hand, joint objective pre-training did not lead to substantial improvements over standard initializations. More interestingly, when extremely short horizons were taken into account ($T=50$) the difference became appalling, as shown in Fig.~\ref{subfig:image9}: pre-training via mixture-based objectives leaded to faster learning and higher performances, while pre-training via disjoint objectives turned out to be even \emph{harmful} (Fig.~\ref{subfig:image10}). This was motivated by the fact that the disjoint objective overfitted the task over the states reachable without coordinated exploration, resulting in almost deterministic policies, as shown in Fig~\ref{fig:333} in Appendix~\ref{apx:exp}. Finally, we tested the zero-shot capabilities of policy pre-training on the simpler but high dimensional exploration task of Env.~(\textbf{ii}), where the goal was sampled randomly between worst-case positions at the boundaries of the region reachable by the arm. As shown in Fig.~\ref{subfig:image11}, both joint and mixture were able to guarantee zero-shot performances via pre-training compatible with MA-TRPO after learning over $2$e$4$ samples, while disjoint objectives were not. On the other hand, pre-training with joint objectives showed an extremely high-variance, leading to worst-case performances not better than the ones of random initialization. Mixture objectives on the other hand showed higher stability in guaranteeing compelling zero-shot performance.
\vspace{-6pt}
\paragraph*{Take-Aways.}~Overall, the proposed proof of concepts experiments managed to answer to all of the experimental questions: (\textbf{a}) Algorithm~\ref{alg:trpe} is indeed able to explicitly optimize for finite-trial entropic objectives. Additionally, (\textbf{b}) \textbf{mixture distributions enforce diverse yet coordinated exploration}, that helps when high efficiency is required. Joint or disjoint objectives on the other hand may fail to lead to relevant solutions because of under or over optimization. Finally, (\textbf{c}) \textbf{efficient exploration} enforced by mixture distributions was shown to be a \textbf{crucial factor} not only for the sake of task-agnostic exploration per se, but also for the ability of \textbf{pre-training via task-agnostic exploration} to lead to \textbf{faster and better training} and even \textbf{zero-shot generalization}.

\section{Discussion}


In this paper, we adopted a learner-centered design approach, beginning with a formative study to identify students' challenges with existing tools. Based on these insights, we developed DBox, a tool that scaffolds students in breaking problems into smaller parts and provides personalized, adaptive support. Our user study demonstrated that DBox improved learners' performance on similar algorithmic problems, increased perceived learning gains, and fostered greater cognitive engagement, achievement, and satisfaction. In this section, we discuss design implications and generalizability based on our key findings.


\ms{
\subsection{Chaining Learners' Thoughts with Visualized Structured UI Components}

Decomposition requires students to effectively organize their thoughts. While visual elements are known to promote structured thinking and support mental model construction \cite{mcdougall2001effects, liu2010mental}, our formative and user studies revealed shortcomings in existing tools like LeetCode and ChatGPT, which rely on textual representations without adequately supporting structured mental models. In contrast, DBox uses an interactive step tree to visually organize learners' thoughts. This feature was praised by 22 of 24 participants for enhancing algorithmic thinking, serving as a progress tracker, and providing value even without AI assistance.

DBox's interactive step tree and tree-based scaffolding demonstrate the broader potential of intelligent tutoring systems (ITS) to promote active learning and self-regulated problem-solving in fields requiring problem decomposition. Similar principles could benefit STEM education, such as physics or engineering, by externalizing abstract concepts and facilitating multi-step problem-solving. Additionally, progress-tracking visual components may inspire designs for professional training tools in areas like medical diagnostics or software engineering.

\subsection{Promoting Independent Thinking and Active Decomposition Learning}

\subsubsection{\textbf{Transforming Learners from Passive Readers to Active Thinkers}}

Many coding tools provide direct answers or solutions \cite{kazemitabaar2023novices, phung2023generating}, which, while efficient, often bypass opportunities to develop critical problem-solving skills. In contrast, DBox cultivates students' decomposition abilities through structured scaffolding, fostering critical thinking and self-regulated learning in line with learning by doing \cite{anzai1979theory} and constructivist principles \cite{tobias2009constructivist}.

To strengthen decomposition skills, DBox first encourages students to develop their own decomposition strategies by coding or building a step tree from scratch. While DBox can generate parts of a step tree from a student's existing code, these steps are derived from the learner's own reasoning, with DBox acting solely as a modality converter. Besides, DBox provides feedback on tree node statuses, identifying potential errors or missing steps without directly showing the correct answer, challenging students to critically evaluate and refine their decomposition plans.


DBox's scaffolded hint system further supports decomposition skill development by providing adaptive guidance tailored to the student’s progress without overwhelming them. All hints are based on the learner's current decomposition skeleton, with the most detailed hint—``reveal substep''—triggered only after repeated attempts and struggles. Notably, even the most detailed hints prompt only one substep, requiring students to complete the rest independently. As shown in Sec \ref{hintusage}, only 19\% of hints are this detailed, with students primarily relying on simpler, thought-provoking question hints. This scaffolded support system balances guidance and independent thinking, keeping students engaged during challenges without compromising their ability to independently decompose problems \cite{kinnunen2006students}.

Based on these findings, we recommend fostering active problem-solving by shifting students from passive content consumption to active solution creation. Designers could adopt layered scaffolding, starting with minimal guidance and increasing support as needed, to help students progressively master decomposition skills while maintaining confidence and avoiding frustration. Additionally, adaptive learning techniques, such as real-time feedback and progress tracking, can further tailor the support to individual decomposition barriers, encouraging deeper engagement with decomposition tasks. Moreover, designers could integrate metacognitive strategies, such as encouraging students to articulate or reflect on their decomposition approaches, to further enhance critical thinking and foster habits of independent thinking.




\subsubsection{\textbf{Choice of Scaffolding: Balancing Independent Problem-Solving and Efforts}}

Scaffolding involves providing tailored support to help learners accomplish tasks they cannot yet complete independently \cite{kim2011scaffolding, tobias2009constructivist}. Broadly, scaffolding strategies fall into two categories \cite{van2010scaffolding}: (1) gradually reducing assistance as learners gain proficiency, and (2) encouraging independent problem-solving while offering incremental support to address challenges. DBox adopts the second approach, emphasizing independent thinking and encouraging learners to actively decompose problems \cite{zimmerman2013theories}. While our scaffolding strategies successfully enhanced critical thinking, satisfaction, and perceived usefulness, they also led to increased cognitive effort (Sec. \ref{Effects_on_UX}). This tradeoff underscores the importance of carefully balancing cognitive effort with the promotion of independent thinking.

Future designs could incorporate adaptive scaffolding that adjusts support dynamically based on learner proficiency, reducing unnecessary effort in areas where students have demonstrated competence. Additionally, while incremental scaffolding was effective for algorithmic problem-solving, tailoring strategies to different educational contexts could enhance their applicability in diverse domains. Such adaptive, context-specific approaches could further optimize the balance between support and independence in learning environments.


\subsection{Supporting Personalized Algorithmic Programming Learning}

\subsubsection{\textbf{Prioritizing Learners' Own Solutions Over Optimality}}

Algorithmic problems often have multiple solutions with varying time and space complexities. DBox prioritizes independent exploration by supporting learners' strategies rather than steering them toward a single ``optimal'' solution. Using LLM-driven prompts, it evaluates and guides each step based on the learner's reasoning, preserving their step decomposition and respecting their input—even when errors occur. While some solutions may not be the most efficient, this approach fosters autonomy by aligning feedback with learners’ thought processes instead of enforcing rigid standards.

Our user study showed that this approach improves learning outcomes and is well-received by students. We recommend designing systems that respect personalized problem-solving strategies by aligning feedback with learners' reasoning while allowing for diverse approaches. Designers should balance flexibility and rigor, using prompts and interfaces that support varied strategies while gently guiding learners toward effective solutions.


\subsubsection{\textbf{Catering to Individual Learning Styles and Contextual Needs}}

DBox accommodates diverse problem-solving approaches with two input modes: coding and natural language descriptions. Each mode offers distinct advantages tailored to different learners, stages, and situations. Learners can switch seamlessly between modes, with progress automatically synced across the interface. Features such as verifying code-step alignment ensure strong integration between modes.

Our findings reveal that this flexibility enhances user experience. Participant interaction logs and interviews revealed three usage patterns, highlighting that each mode fits different needs: code mode works well for students with a clear and detailed problem-solving plan already, while the step tree with natural language descriptions helps less experienced students with only a basic idea who are not ready to write code directly, boosting their confidence.


We argue there is no universal “best” mode for programming education—each has unique benefits depending on the learner habits, expertise, and context. Future tools should provide flexibility, like DBox, or use adaptive algorithms to recommend modes based on user needs and context. This flexibility highlights the importance of designing educational tools that accommodate varying levels of expertise and problem-solving styles, which can be generalized to other domains requiring personalized learning \cite{bernacki2021systematic}.

\subsection{Appropriate Usage of LLMs for Supporting Algorithmic Programming Learning}

\subsubsection{\textbf{Caution About LLM Errors}}

Although LLMs have shown strong performance in coding tasks \cite{finnie2023my, leinonen2023using}, they remain prone to errors. Our technical evaluation and user study revealed that even with comprehensive context—such as problem statements, user code, and natural language steps—LLM sometimes misinterprets user descriptions. These errors likely arise from discrepancies between the natural language used by students and the formal, precise language the LLM was trained on, which is primarily sourced from web-based code and comments \cite{liu2023wants}.

Such misinterpretations can hinder learning by causing confusion or frustration. While future improvements to training data and GPT versions may mitigate these issues, design strategies can help address them. \textbf{First}, LLMs should avoid giving direct solutions and instead focus on fostering active problem-solving through explanations and hints. \textbf{Second}, feedback could be paired with interactive features, like a ``Run Code'' option, allowing students to validate their reasoning. \textbf{Third}, simple tutorials could teach users how to phrase their descriptions more clearly, improving LLM's understanding. Additionally, future tools could integrate a ``Language Enhancement'' feature to suggest improvements or assess the clarity of descriptions, aiding LLM in accurately capturing user intent. Most importantly, we recommend designers prioritize technical feasibility, such as conducting rigorous evaluations like ours, before fully integrating LLMs into programming learning tools.
}



\subsubsection{\textbf{Learner-LLM Co-Decomposition of Solutions: Learner as Leader, LLM as Aid}}

A central feature of DBox is the construction of a step tree, where students break solutions into steps and sub-steps. The LLM supports this by mapping code to step descriptions, evaluating them, and offering hints. However, students maintain full control, deciding how to decompose problems and define each step, fostering independent thinking. The LLM acts solely as an aid, using a scaffolding approach to support the development of learners' Zone of Proximal Development (ZPD) \cite{chaiklin2003zone}. Unlike tools like ChatGPT or Copilot that dominate problem-solving, DBox fosters deeper cognitive engagement. Students reported greater accomplishment and found this approach more effective for learning.

This contrasts with existing human-AI collaboration paradigms in non-educational scenarios where AI usually suggest options, leaving final decisions to users \cite{dang2023choice, gao2024collabcoder, gebreegziabher2023patat, ma2019smarteye, ma2022glancee}, such as in human-AI decision-making \cite{ma2023should, ma2024towards, ma2024you}. Some educational tools, like Jin et al. \cite{jin2024teach}, use LLMs to generate solutions for students to evaluate, which aids in syntax learning but such ``LLM-generate then learner-evaluate'' approach is less effective for algorithmic problem-solving, where constructing solutions is key. Just evaluating LLM-generated contents can place a cognitive anchor on learners \cite{furnham2011literature}, limiting independent thinking and creativity. Thus, task allocation between humans and AI should align with the educational context (e.g., whether it is basic knowledge/concept learning or higher-level creative thinking). Future LLM-based educational tools should carefully define the division of roles between LLMs and learners, tailoring it to specific learning contexts and goals.




% \subsubsection{Human-LLM Co-Decomposition of Solution: AI Should Judge Instead of Recommending}

% A core interaction in DBox is the construction of a step tree, where the entire solution is broken down into a series of steps and sub-steps. We refer to this as the human-LLM co-decomposition process. In this process, the LLM behind DBox plays three roles: First, it maps the student's written code into step descriptions. Second, it evaluates the status of each step and sub-step (whether they are correct, incorrect, missing, or need further decomposition). Third, it provides hints for incorrect or missing steps or sub-steps. However, the actual construction of the step tree—such as dividing the solution into steps and sub-steps and determining the content of each node—remains primarily the student's responsibility.

% This division of labor maximizes student engagement in independent thinking and problem-solving. The LLM does not provide any suggestions for decomposition nor directly recommend content for specific steps, aligning with the scaffolding educational approach, where guidance is provided appropriately, but the main task of forming the solution is left to the students.

% In contrast, when students directly seek help from an LLM, such as asking questions in ChatGPT or using Copilot for code completion, the LLM takes too much initiative by directly offering ideas or code. In our co-decomposition design, however, students demonstrated higher cognitive engagement and more active critical thinking. Furthermore, students reported that constructing solutions in this way gave them a greater sense of achievement and made them feel the process was more beneficial for learning, leading to higher satisfaction with the experience.

% Related work has proposed similar approaches. For instance, XXX, in the context of problem-solving, uses the "learning by teaching" concept, where students take on the tasks of judging and teaching, while the LLM generates most of the solutions. Compared to our approach, their division of labor between the student and the LLM is reversed. This method works well in introductory programming, where the focus is on mastering syntax. Having students guide the LLM to generate code or evaluate potentially incorrect code produced by the LLM is an effective way to quiz them. However, in our work, which focuses on algorithmic programming, the key step is constructing a solution from scratch. If the LLM builds the solution, leaving students only to judge it, it hampers their independent thinking.

% Thus, when designing LLM-based educational tools in the future, it is crucial to consider the specific context to effectively allocate tasks between the student and the LLM, ensuring that students derive the maximum benefit from the co-decomposition process.


% \subsection{Future Design Opportunities}

% \emph{Providing Appropriate Generative Assistance:} While DBox promotes independent problem-solving, some users showed interest in features like auto-completion for trivial coding tasks. Future versions could balance promoting independence with targeted assistance by enabling adjustable difficulty levels and offering contextual suggestions when appropriate.

% \emph{Covering All Stages of Algorithmic Programming:} DBox currently lacks a focus on foundational algorithm instruction and problem comprehension. Future iterations could include features like generating distractor solutions, input-output tests, and step-by-step rephrasing to help students grasp key concepts and understand the coding problem.

% \emph{Combining Step Trees with Dialogue:} Users can currently describe their thought processes but cannot ask questions. Adding a dialogue system to the step tree would allow students to share challenges and ask follow-up questions. GPT could then provide guided feedback without giving direct answers, supporting independent problem-solving.





% \emph{Other Important Features.} DBox could offer more control by allowing users to select specific parts of their code for targeted evaluation and guidance. A ``review'' feature could also help students reflect on key stumbling points, understand where their thought process went wrong, and how they eventually solved the problem.


% \subsection{Future Design Opportunities}

% \emph{Providing Appropriate Generative Assistance.} Our tool primarily focuses on encouraging users to create the step tree and write the code independently, with the system mainly serving as a judge. However, users expressed a desire for some intelligent completion features, particularly for repetitive or simple code, allowing them to focus their efforts on learning the key parts. Future improvements should strike a balance between fostering independent thinking and providing appropriate assistance. One approach could be designing basic rules where the tool offers intelligent suggestions and completions for parts unrelated to the core logic, while maintaining the current level of independence for key learning areas. Additionally, the system could offer different modes, allowing users to choose the level of assistance, from basic judgment-only feedback to a combination of judgment, guidance, necessary completions, and even on-demand suggestions.

% \emph{Covering All Stages of Algorithmic Programming.} Currently, our system does not cover the basic teaching of algorithms or the problem comprehension stage. In the future, to address the diversity and uncertainty in solutions and help students grasp multiple approaches, we could expand assistance during the idea formation phase. For example, GPT could generate multiple potential solutions with distractors, prompting students to identify the one that meets the problem's complexity requirements. We could also introduce specialized algorithm training, where students select a specific algorithm, and the system’s guidance focuses solely on that algorithm. To assist with problem comprehension, we could incorporate input-output tests to check students' understanding of the problem and step-by-step rephrasing to help them grasp more complex problems.

% \emph{Combining Interactive Step Trees with Dialogue Boxes.} Sometimes users want to describe their difficulties, and currently, we ask them to outline their thought processes. Additionally, users may want to ask follow-up questions. In the future, we could combine the structured step tree with a small dialogue box. The primary goal would still be to construct the step tree, but users could engage in a conversation with GPT in the context of the current step tree or a specific step. Importantly, GPT should guide the user without revealing direct answers.

% \emph{Other Important Features.} First, DBox could offer learners more control, such as allowing users to select specific parts of the code for targeted evaluation and guidance. We could also introduce a summary feature for key stumbling points, helping students reflect on the challenges they faced, where their thought process went wrong, and how they eventually overcame the problem.




\subsection{Limitations and Future Work}

This study has several limitations. \emph{First}, we tested DBox's effectiveness on only two problem types; future work should examine a broader range of algorithms. \emph{Second}, participants engaged in just one learning session per condition due to time constraints, whereas mastering algorithmic problems typically requires extended practice. Longitudinal studies should explore how DBox supports skill development over time, including changes in mental models and skill retention. \emph{Third}, we assessed learning gains based on correctness in a test session using similar learning and test problems. Future research should evaluate knowledge transfer to less similar problems. Due to time constraints, we conducted a single post-test rather than a pre-post comparison. While pre-test expertise filtering and randomization minimized prior familiarity effects, a more rigorous pre-post design would yield more accurate learning gain measurements. Looking ahead, we plan to release DBox as a Chrome plugin for integration with existing coding platforms, enabling large-scale field studies. This will allow for the collection of long-term usage data and periodic surveys to identify usage patterns and learning experiences over time.



% This study has several limitations. First, in our within-subject design, we selected two types of algorithm problems—Greedy and Binary Search—and randomly assigned them to two conditions (DBox and baseline). However, selection bias may still exist, as some participants might naturally excel at one type of algorithm. Although we addressed this by filtering participants' proficiency through a pre-test and using a Latin Square design, further validation across a broader range of algorithms is needed in future work.

% Second, students experienced only one learning session per condition before the test session. While this allowed for a fair comparison, mastering algorithmic problems typically requires extended practice. Future work should explore how DBox supports students' long-term improvement in algorithmic skills. Longitudinal studies could provide insights into changes in learners' mental models, allowing students more time to deepen their understanding and refine their decomposition methods. Additionally, retention tests could assess whether students can still apply learned problem-solving methods after a time gap.

% We measured learning gains through correctness scores in the test session, with relatively similar learning and test problems. Future work should explore students' ability to transfer their knowledge to problems with lower similarity. Due to time constraints, we opted for a single post-test rather than a pre-post comparison. While we minimized prior familiarity effects by filtering participants and randomizing problem assignments, future studies could adopt a more rigorous pre-post test design for better measurement of learning gains.

% Looking ahead, we plan to release DBox as a Chrome plugin for integration with existing online coding platforms and large-scale real-world testing. In such settings, where students may be more motivated (e.g., preparing for algorithm interviews), we can gather long-term usage data while ensuring privacy. We also plan to conduct periodic surveys to track changes in students' usage patterns and learning experiences over time.



% \subsection{Limitations and Future Work}

% This study has several limitations. First, in our within-subjects study, we selected two types of algorithm problems, Greedy and Binary Search, and randomly assigned them to two conditions, DBox and the baseline. However, there may still be selection bias, where some participants were naturally better at one type of algorithm. While we mitigated this issue to a large extent by filtering participants' proficiency through a pre-test and employing a Latin Square design to randomize the problem-condition assignment, there is still room for improvement. Future work should validate DBox's effectiveness across a broader range of problem types.

% Second, in our experiment, students only experienced one learning session in each condition before moving on to the test session. Although this comparison was fair (as both conditions had only one learning session), mastering an algorithmic problem often requires extended practice. Future work should explore how DBox can help students gradually improve their algorithmic programming skills over time. Longitudinal studies may reveal significant changes in learners' mental models, providing more time for them to understand a specific algorithm and enhance their decomposition methods. Additionally, future studies could include retention tests to measure whether students can still effectively apply previously learned problem-solving methods after a period of time.

% Furthermore, when objectively measuring students' learning gains, we calculated their correctness score in the test session. On the one hand, the learning session and test session problems had a relatively high degree of similarity. Future work should investigate whether students can transfer what they have learned to solve problems of the same algorithm type with lower similarity. On the other hand, due to time constraints, we did not include a pre-post test comparison, opting for a single post-test instead. This result might be influenced by students' pre-existing familiarity with the problems. Although we mitigated this issue by filtering for familiarity (ensuring participants were not too familiar with the problems) and randomizing the problem assignments, future work could include a more rigorous pre-post test design to better calculate students' learning gains.

% Moreover, DBox is currently only applied in algorithmic programming, specifically solving algorithm problems. However, this decomposition-based computational thinking approach could be extended to other learning scenarios, such as project-based learning. Future work could explore how to adapt DBox to broader educational contexts outside of algorithmic programming.

% Looking forward, we aim to deploy DBox in real-world algorithm courses. Since algorithms are a core required subject in undergraduate computer science curricula, we hope to investigate how students who have just learned algorithm concepts use DBox to develop their problem-solving skills. Additionally, we plan to convert DBox into a Chrome plugin and release it in the Chrome Web Store for real-world testing. This would allow DBox to seamlessly integrate with existing online coding platforms, enabling large-scale experiments. In such settings, students' motivation may be stronger (e.g., a graduate preparing for an algorithm interview), leading to more realistic usage patterns. Students could use DBox to tackle a wide variety of algorithm problems. We hope to collect long-term (e.g., six-month) usage data from real-world users while ensuring privacy, and use periodic surveys to capture changes in students' usage patterns and learning experiences over time.





\section{Conclusion}
% In this paper, we introduced Decomposition Box (DBox), a novel tool designed to scaffold learners in decomposing problems during algorithmic programming learning. Based on insights from a formative study, we identified key design goals to address the limitations of existing tools in algorithmic programming education. DBox supports two critical stages of the programming process: idea formation and idea implementation. By offering two modes (code mode and language mode), it encourages users to independently develop their solution strategies. The interactive, visual step tree helps students break down problems and build a structured mental model. DBox provides fine-grained, step-level feedback, enabling students to quickly identify issues, while its multi-level guidance offers targeted support without undermining independent thinking.

% Our user study demonstrated that DBox led to significantly higher learning gains, cognitive engagement, and critical thinking. Students reported a stronger sense of achievement and found the assistance both appropriate and effective for their learning. We identified three main usage patterns, underscoring the importance of respecting students' problem-solving habits and offering them autonomy. The learner-LLM co-decomposition model we designed promotes independent thinking while allowing the LLM to contribute meaningfully, even with occasional imperfections. 

% We hope the formative study, design goals, features, technical evaluation, and key findings from this work will inspire future research on developing educational tools for broader programming learning.
In this paper, we introduced DBox, an interactive tool designed to help learners decompose algorithmic programming problems by supporting both solution formation and implementation. Featuring an intuitive tree-like box widget, DBox accepts input in both code and natural language, fostering independent problem-solving while its step tree structure helps learners develop structured mental models. It provides step-level feedback and layered guidance without compromising learner autonomy.
Our user study showed that DBox significantly improved learning outcomes, cognitive engagement, and critical thinking, with students reporting a greater sense of achievement and finding the support highly effective. Additionally, we identified three key usage patterns, highlighting the importance of accommodating individual problem-solving styles. Moreover, our findings suggest that the learner-LLM co-decomposition approach fosters independent thinking while providing meaningful guidance, even with occasional imperfections.
We hope the insights from our system design will inspire future research on integrating LLMs into educational tools for programming learning.

\subsection*{Acknowledgements}
We thank Noah Cowan for his helpful feedback on this manuscript. We also thank Jingyang Zhang for his feedback and detailed knowledge of the OpenOOD codebase.
RWL was supported in part by the National Science Foundation (NSF) under Grant No.
2112562, the U.S. Army Research Laboratory and the U.S. Army Research Office (ARL/ARO)
under grant number ARO-W911NF-23-2-0224, and the Department of Defense (DoD) through the National Defense Science \& Engineering Graduate (NDSEG) Fellowship Program.
SWL was supported in part by fellowships from the Simons Collaboration on the Global Brain, the Alfred P. Sloan Foundation, and the McKnight Foundation.

Any opinions, findings, and conclusions or recommendations expressed in this
material are those of the authors and do not necessarily reflect the views of
the NSF, the ARL/ARO, or the DoD.

\bibliography{main}
\bibliographystyle{unsrtnat}

\clearpage
\appendix

\section{How is MENTAT Different from Medical Exam Questions?}
\label{app:medqa_to_mentat}

For years, medical AI benchmarks have focused on fact-based assessments. Most medical evaluations for LMs rely on board exams and medical student tests, primarily measuring knowledge recall rather than real-world clinical decision-making. These exams have little correlation with actual clinical practice, as passing them does not equate to the ability to manage patients effectively even in humans \cite{Saguil2015}.

\begin{figure}[ht]
    \begin{framed}
    A 32-year-old woman with type 1 diabetes mellitus has had progressive renal failure during the past 2 years. 
    She has not yet started dialysis. Examination shows no abnormalities. Her hemoglobin concentration is 9 g/dL, 
    hematocrit is 28\%, and mean corpuscular volume is 94 $\mu$m\textsuperscript{3}. 
    A blood smear shows normochromic, normocytic cells. 
    Which of the following is the most likely cause?
    
    (A) Acute blood loss \\
    (B) Chronic lymphocytic leukemia\\
    (C) Erythrocyte enzyme deficiency\\
    (D) Erythropoietin deficiency\\
    (E) Immunohemolysis\\
    (F) Microangiopathic hemolysis\\
    (G) Polycythemia vera \\
    (H) Sickle cell disease \\
    (I) Sideroblastic anemia \\
    (J) $\beta$-Thalassemia trait\\
    \textbf{(Answer: D)}
    \end{framed}
    \caption{USMLE board exam question example }
    \label{fig:usmle_example_q}
\end{figure}




For example, \Cref{fig:usmle_example_q} presents a classic USMLE board exam question \cite{USMLE2021}, which tests an AI model’s ability to recall factual knowledge rather than apply practical decision-making skills. The question may assess the recognition of a laboratory abnormality in diabetes, but it does not evaluate whether the model can adjust insulin regimens, recognize psychosocial factors, or determine hospitalization needs—key components of real-world patient care. As highlighted in previous research, medical licensing exams do not strongly correlate with clinical competency, reinforcing the need for benchmarks that evaluate accurate decision-making skills rather than memorization.

\begin{table}[h]
    \centering
    \begin{tabular}{llp{10cm}}
        \toprule
        \textbf{Question type} & \textbf{Attribute type} & \textbf{Example template question} \\
        \midrule
        \multirow{6}{*}{Single-Verify} 
        & SCP Code & Does this ECG show symptoms of \textbf{non-specific ST changes}? \\
        & Noise & Does this ECG show \textbf{baseline drift in lead I}? \\
        & Stage of infarction & Does this ECG show \textbf{early stage of myocardial infarction}? \\
        & Extra systole & Does this ECG show \textbf{ventricular extrasystoles}? \\
        & Heart axis & Does this ECG show \textbf{left axis deviation}? \\
        & Numeric feature & Does the \textbf{RR interval} of this ECG fall \textbf{within the normal range}? \\
        \bottomrule
    \end{tabular}
    \caption{Example template questions for different ECG attributes.}
    \label{tab:ecg_questions}
\end{table}
\begin{table}[h]
    \centering
    \begin{tabular}{lp{3.cm}p{3.5cm}p{1.5cm}p{3.5cm}}
        \toprule
        \textbf{Category} & \textbf{Task} & \textbf{Prompt} & \textbf{Result} & \textbf{AI Response} \\
        \midrule
        \multirow{2}{*}{Sequence alignment} 
        & DNA sequence alignment to human genome 
        & Align the DNA sequence to the human genome: \texttt{TGGGCTCA AGTGATCATA……} 
        & chr7 
        & As a language model AI, I do not have the capability to align a DNA sequence to the human genome…… 
        \\
        \midrule
        & DNA sequence alignment to multiple species 
        & Which organism does the DNA sequence come from: \texttt{CGTACACC ATTGGTGC……} 
        & yeast 
        & The organism from which the DNA sequence comes cannot be determined based solely on the DNA sequence…… 
         \\
        \bottomrule
    \end{tabular}
    \caption{DNA Sequence Alignment Tasks and AI Responses}
    \label{tab:sequence_alignment}
\end{table}


\Cref{tab:ecg_questions} and \Cref{tab:sequence_alignment} illustrate additional examples of widely used AI benchmarks, such as ECG-QA \cite{Oh2024} and GeneTuring \cite{Hou2023}, which focus on highly structured, fact-based medical knowledge. These datasets and others like MedQA \cite{Jin2021} have been leveraged by major AI companies, including Google’s Gemini initiative \cite{Saab2024}, to highlight model performance. While these benchmarks evaluate text-based and multimodal AI capabilities, they focus heavily on fact memorization rather than applied clinical reasoning.

Unlike traditional medical AI benchmarks, MENTAT is designed by practicing psychiatrists to reflect real-world clinical scenarios. The dataset also includes ambiguous, multi-choice decision-making tasks rather than a single correct answer, simulating the complex nature of psychiatric practice. Furthermore, MENTAT aims to reduce bias by empowering a diverse group of clinicians in its development from the start, making it less likely to reinforce harmful racial, gender, or sexuality-based biases in mental healthcare. In summary, MENTAT differs from medical exam questions by moving beyond fact recall to assess practical clinical decision-making in mental healthcare. While traditional benchmarks test AI models on medical knowledge, MENTAT evaluates whether AI can handle real-world psychiatric tasks, manage patient uncertainty, and make informed decisions in complex clinical environments.




% For years, benchmark evaluations have been utilized in medical AI to track the progress of new and updated models. However, they have largely focused on genetics, radiology, cardiology, and electronic medical record data processing\cite{Hou2023, Zambrano2023, Oh2024}. Little work has thus far been invested in the creation of benchmark evaluations and datasets for mental healthcare. Most medical evaluations of language and multi-modal models have also only focused on specialty board exams and exams intended for medical students. Both categories of exams assess knowledge but have been noted to have relatively little correlation to the real-world practice of medicine\cite{Saguil2015}. Every medical specialty would benefit from a creation of a dataset of question-answer pairs tailored specifically to clinical practice as opposed to the fact-based assessment that is common in licensing and medical student exams. As an example, Figure 1 demonstrates a classic board exam question directly from the USMLE website \cite{USMLE2021}, a medical licensing exam all medical students must take. It does not assess knowledge about pragmatic clinical management of diabetes; rather, it focuses on fact-based knowledge. Just as Saguil et al highlighted a lack of correlation between medical licensing exams and clinical skills, a model’s ability to answer Figure 1 correctly does not correlate to its ability to care for an individual with diabetes. Construction of datasets that test practical medical knowledge are necessary to robustly evaluate language models’ appropriateness in clinical settings. Furthermore, clinicians have recently called for moving beyond a medical exam benchmark, stating, "it is essential to move beyond medical exams and adopt more grounded, task-specific approaches for evaluation" \cite{Raji2025}. A related study, published in 2025, identified 11 high-level clinical tasks and created a benchmark (MedS-bench) meant to "address this gap" as current benchmarks and evaluation datasets "fail to adequately reflect the practical utility of LLMs in real-world clinical scenarios". Our work is different, but complementary, as our dataset applies this concept (testing for skills required to practice as a clinician as opposed to esoteric medical facts) to mental healthcare. In the realm of mental healthcare, researchers have expanded evaluations of LLM's into the realm of psychotherapy, which is complementary to our approach of evaluating LLM performance in the related field of psychiatry.


% As previously mentioned, most investigations into the evaluation of AI models in the healthcare setting have focused on pre-existing fact-based datasets such as the USMLE exams and specialty-specific board exams. In practice, these knowledge-based tests (e.g., USMLE Step 1) are designed to assess whether human trainees have acquired sufficient baseline knowledge to enter post-graduate training in a selected medical specialty. However, passing these tests alone is not considered sufficient for practicing as a physician in the United States. Residency training is required, during which trainees apply fundamental medical knowledge to real-world clinical cases \cite{Mowery2015}. 
% %
% Similarly, AI models must progress beyond simple recall of medical facts. They should be trained and evaluated on real-world clinical tasks that require the application of baseline medical knowledge. To maximize external validity, datasets should be created and vetted by actively practicing clinicians. Some of the most widely used benchmarks in medical AI are frequently leveraged by major companies to showcase the clinical capabilities of their fine-tuned models. For instance, in 2024, Google published *Capabilities of Gemini Models in Medicine*, incorporating several prominent medical benchmarks \cite{Saab2024}. The authors highlight the novelty of their work as “the most comprehensive benchmarking of multimodal medical models to date” based on their use of 14 different medical benchmarks \cite{Saab2024}. These benchmarks include ECG-QA \cite{Oh2024}, MedQA \cite{Jin2021}, GeneTuring \cite{Hou2023}, MMMU (health medicine) \cite{Yue2023}, NEJM Image Challenges \cite{NEJM2024}, Path-VQA \cite{He2020}, and others. Additionally, an effort was made to enhance the clinical relevance of these findings by evaluating the models’ ability to summarize complex medical information and generate referral letters for specialists \cite{Saab2024}. 
% %
% Examples of questions from these datasets are provided in Figures 3 and 4, which illustrate excerpts from ECG-QA and GeneTuring, respectively. MedQA, on the other hand, is best represented by the example in Figure 1. In the domain of mental health datasets and summarization, Adhikary et al. introduced a dataset comprising 191 counseling sessions with associated summaries \cite{Adhikary2024}. 
% %
% Despite the breadth of existing benchmarks, there is still no robust, clinician-led, and clinician-vetted mental healthcare benchmark for AI models. Current medical benchmarks remain overly narrow and fact-based, limiting their external validity and clinical relevance. As noted earlier, a high score on the MedQA benchmark does not equate to excellence in clinical care. Our dataset shifts the paradigm by moving beyond fact-based assessments (e.g., USMLE exams) and introducing a comprehensive evaluation of clinician-level decision-making in mental healthcare. 
% %
% Our benchmark uniquely assesses an AI model’s ability to **triage, diagnose, treat, and monitor mental health conditions**, establishing a new category of medical benchmarks that we hope other specialties will adopt. Furthermore, because our dataset is developed and overseen by a diverse group of practicing clinicians, it is significantly less likely to perpetuate harmful racial, gender, or sexuality-based biases in mental healthcare. It has been validated by [insert number] practicing psychiatrists.

\newpage

\section{Further Annotation Processing Results}
\label{app:annotation_details}

% \begin{figure}[ht]
%     \vskip 0.2in
%     \begin{center}
%     \centerline{\includegraphics[width=0.5\columnwidth]{figures/raw_annotation_krippendorf.pdf}}
%     \caption{Test.}
%     \label{fig:raw_annotation_krippendorf}
%     \end{center}
%     \vskip -0.2in
% \end{figure}

\begin{figure}[ht!]
    \centering
    \begin{minipage}[b]{0.49\textwidth}
        \centering
        \includegraphics[width=\linewidth]{figures/annotator_scores_hbt_pars.pdf}
        \caption{(Top) We show the average raw annotation score with with bootstrapped (95\% CL) uncertainties for each annotator. All of them deviate from 50 with statistical significance (the random baseline). 
        (Bottom) Fitted individual annotator parameters from the hierarchical Bradley-Terry model.
        Besides regularization in the log-likelihood objective, we bound the individual annotator parameters ($\gamma_a \in [-3.0, 3.0]$, $\alpha_a \in [0.5, 2.0]$) during the optimization to balance the goal of slightly de-noising the resulting preference dataset while keeping the majority of differences between individual annotator preferences.
        These bounds prevent the model from fixing contradictory data by pushing a parameter to an extreme.
        The fact that all annotators have a positive offset $\gamma_a$ indicates that they all tend to choose one answer option to prefer over all others in a single annotation of one question.}
        \label{fig:annotator_scores_hbt_pars}
    \end{minipage}%
    \hfill
    \begin{minipage}[b]{0.49\textwidth}
        \centering
        \includegraphics[width=\linewidth]{figures/raw_annotation_krippendorf.pdf}
        \caption{
        We show the distribution of  Krippendorff's $\alpha$ for raw triage and documentation question annotations.
        We verify that the expert annotators do not converge on one answer option and that there is sufficient inter-annotator disagreement.
        Given our design choices, we expect $\alpha$ to be naturally low as our goal is not to measure the presence of a single ground truth and low $\alpha$ values ($\alpha \leq 0.5$) will not tell us how useful a set of annotations is—only that experts statistically disagree. 
        }
        \label{fig:raw_annotation_krippendorf}
    \end{minipage}
\end{figure}
% \begin{figure}[ht]
%     \vskip 0.2in
%     \begin{center}
%     \centerline{\includegraphics[width=0.5\columnwidth]{figures/annotator_scores_hbt_pars.pdf}}
%     \caption{Test.}
%     \label{fig:annotator_scores_hbt_pars}
%     \end{center}
%     \vskip -0.2in
% \end{figure}
% \begin{figure}[ht]
%     \vskip 0.2in
%     \begin{center}
%     \centerline{\includegraphics[width=0.5\columnwidth]{figures/frac_ct in topk_bt_vs_hbt.pdf}}
%     \caption{Test.}
%     \label{fig:frac_ct in topk_bt_vs_hbt}
%     \end{center}
%     \vskip -0.2in
% \end{figure}

\newpage

\section{Language Model Prompts}
\label{app:prompting}

\begin{figure}[ht]
    \centering
    \begin{minipage}[b]{0.38\textwidth}
        \begin{framed}
        \texttt{
        f"Question: \{q\}\textbackslash n\textbackslash n"\\
        f"A: \{answer\_list[0]\}\textbackslash n"\\
        f"B: \{answer\_list[1]\}\textbackslash n"\\
        f"C: \{answer\_list[2]\}\textbackslash n"\\
        f"D: \{answer\_list[3]\}\textbackslash n"\\
        f"E: \{answer\_list[4]\}\textbackslash n\textbackslash n"\\
        "Answer (single letter): "
        }
        \end{framed}
    \end{minipage}%
    \hfill
    \begin{minipage}[b]{0.62\textwidth}
        % \centering
        \begin{framed}
        \texttt{
        f"Question: \{q\}\textbackslash n\textbackslash n"\\
        f"A: \{answer\_list[0]\}\textbackslash n"\\
        f"B: \{answer\_list[1]\}\textbackslash n"\\
        f"C: \{answer\_list[2]\}\textbackslash n"\\
        f"D: \{answer\_list[3]\}\textbackslash n"\\
        f"E: \{answer\_list[4]\}\textbackslash n\textbackslash n"\\
        "Answer (only reply with a single letter!): "
        }
        \end{framed}
    \end{minipage}
    \caption{(Left) Prompt text MCQA variation A (as used for \textit{gpt-4o-mini-2024-07-18}, \textit{gpt-4o-2024-08-06}, \textit{o1-2024-12-17}, and \textit{o1-mini-2024-09-12}).
    (Right) Prompt text MCQA variation B (all other models).
By looking at the responses from models evaluated with variation A, we verified that the recorded accuracy difference caused by using different promtps was $\leq 1$\%.
The only exception was \textit{o1-mini-2024-09-12}, for which we corrected the evaluation.}
    \label{fig:eval_prompts_mcqa}
\end{figure}

\begin{figure}[ht]
    \vskip 0.2in
    \begin{framed}
        \texttt{
        f"Question: \{q\}\textbackslash n\textbackslash n"\\
        "Answer (write your reply in only one short sentence!): "
        }
        \end{framed}
        \caption{Prompt text free-form (as used for the models evaluated in \Cref{sec:4_4_consistency}).}
    \vskip -0.2in
\end{figure}

\newpage

\section{Annotator Interface}
\label{app:annotator_interface}

\begin{figure}[ht]
    \centering
    \begin{minipage}[b]{0.49\textwidth}
        \centering
        \includegraphics[width=\linewidth]{figures/mentat_q36_question.png}
    \end{minipage}%
    \hfill
    \begin{minipage}[b]{0.49\textwidth}
        \centering
        \includegraphics[width=\linewidth]{figures/mentat_q36_answers.png}
    \end{minipage}
    \caption{Example of the online annotation interface using the \textit{jsPsych} library \citep{de_Leeuw2023} (MIT license). There is also a comment box below the sliders for feedback/comments, that is not shown.}
    \label{fig:mentat_q36_combined}
\end{figure}

\newpage

\section{Further Evaluation Results}
\label{app:more_experiment_results}

\begin{figure}[ht]
    \centering
    \begin{minipage}[b]{0.49\textwidth}
        \centering
        \includegraphics[width=\linewidth]{figures/final_eval_results_by_gender.pdf}
        \caption{Using the $\mathcal{D}_\text{G}$ dataset, we evaluate eleven off-the-shelf instruction-tuned and three (mental) healthcare fine-tuned models for overall accuracy and how it is impacted by different patient genders.}
        \label{fig:final_eval_results_by_gender}
    \end{minipage}%
    \hfill
    \begin{minipage}[b]{0.49\textwidth}
        \centering
        \includegraphics[width=\linewidth]{figures/final_eval_results_by_age.pdf}
        \caption{Using the $\mathcal{D}_\text{A}$ dataset, we evaluate eleven off-the-shelf instruction-tuned and three (mental) healthcare fine-tuned models for overall accuracy and how it is impacted by different patient ages.}
        \label{fig:final_eval_results_by_age}
    \end{minipage}
\end{figure}

% \begin{figure}[ht]
%     \vskip 0.2in
%     \begin{center}
%     \centerline{\includegraphics[width=0.5\columnwidth]{figures/final_eval_results_by_gender.pdf}}
%     \caption{Test.}
%     \label{fig:final_eval_results_by_gender}
%     \end{center}
%     \vskip -0.2in
% \end{figure}
% \begin{figure}[ht]
%     \vskip 0.2in
%     \begin{center}
%     \centerline{\includegraphics[width=0.5\columnwidth]{figures/final_eval_results_by_age.pdf}}
%     \caption{Test.}
%     \label{fig:final_eval_results_by_age}
%     \end{center}
%     \vskip -0.2in
% \end{figure}

\begin{figure}[ht!]
    \vskip 0.2in
    \begin{center}
    \centerline{\includegraphics[width=0.47\columnwidth]{figures/final_eval_results_by_nat.pdf}}
    \caption{Using the $\mathcal{D}_\text{N}$ dataset, we evaluate eleven off-the-shelf instruction-tuned and three (mental) healthcare fine-tuned models for overall accuracy and how it is impacted by different patient ethnicities.}
    \label{fig:final_eval_results_by_nat}
    \end{center}
    \vskip -0.2in
\end{figure}


\end{document}
